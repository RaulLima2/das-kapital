Розділ десятий

Вирівнення загальної норми зиску через
Конкуренцію. Ринкові ціни і ринкові вартості.
Надзиск

Частина сфер виробництва має середній або пересічний склад
застосовуваного в них капіталу, тобто склад капіталу, який цілком чи приблизно збігається з складом
пересічного суспільного
капіталу.

Ціна виробництва товарів, вироблюваних у цих сферах виробництва, цілком чи приблизно збігається з їх
вартістю, вираженою
в грошах. І коли б ніяким іншим способом не можна було досягти математичної границі, то цього можна
було б досягти цим
способом. Конкуренція так розподіляє суспільний капітал між
різними сферами виробництва, що ціни виробництва в кожній
сфері утворюються на зразок цін виробництва в цих сферах
середнього складу, тобто = k + kp' (витрати виробництва плюс
добуток пересічної норми зиску і витрат виробництва). Але
ця пересічна норма зиску є не що інше, як обчислений в процентах зиск у сфері виробництва середнього
складу, де, отже,
зиск збігається з додатковою вартістю. Отже, норма зиску в усіх
сферах виробництва є одна й та ж, а саме вирівнена до норми
зиску цих середніх сфер виробництва, в яких панує пересічний
склад капіталу. Тому сума зисків усіх різних сфер виробництва
мусить дорівнювати сумі додаткових вартостей і сума цін виробництва сукупного суспільного продукту
мусить дорівнювати
сумі його вартостей. Але ясно, що це вирівнювання між сферами виробництва з різним складом завжди
мусить прагнути
урівняти ці сфери з сферами середнього складу, однаково, чи
ці останні точно чи тільки приблизно відповідають пересічному
суспільному складові. У сферах виробництва, які більш чи менш
наближаються до середньої, знову таки має місце тенденція
до вирівнення, яка прагне до ідеального, тобто в дійсності не
наявного середнього рівня, тобто тенденція до вирівнення
навколо нього, як норми. Таким чином у цьому відношенні
необхідно панує тенденція зробити ціни виробництва просто
перетвореними формами вартості, або перетворити зиски в
прості частини додаткової вартості, які, однак, розподіляються
не пропорційно до додаткової вартості, створеної в кожній
окремій сфері виробництва, а пропорційно до маси капіталу,
застосовуваного в кожній сфері виробництва, так що на рівновеликі маси капіталу, хоч би який був їх
склад, припадають відповідно рівновеликі частини сукупної додаткової вартості, створеної сукупним
суспільним капіталом.

Отже, для капіталів середнього чи приблизно середнього
складу ціна виробництва збігається цілком або приблизно з вар-
