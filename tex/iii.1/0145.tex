ляції були для нього незрівняно важливіші, ніж мускульна сила,
звиклий до подвійної, іноді до потрійної плати, порівняно з тією,
яку він може одержати тепер, — такий робітник, згоджуючись
на пропоновану йому роботу, виявляє стільки самозречення
і розсудливості, що це робить йому найвищу честь. У Блекберні
ці люди випробувані майже чи не на всякій роботі під голим
небом: на копанні твердого, важкого глинистого грунту на
значній глибині, на осушуванні грунту, на розбиванні каменю,
на прокладанні шляхів, на копанні вуличних каналів глибиною
в 14, 16, а іноді й 20 футів. Часто їм доводиться при цьому
стояти в грязі й воді на глибині в 10—12 дюймів і вони завжди
підпадають при цьому впливові вогкого й холодного клімату,
гіршого чи навіть рівного якому взагалі не знайдеш ні в одній
окрузі Англії“ (стор. 91, 92). — „Поведінка робітників була майже
бездоганна... їх готовість працювати під голим небом і цим
перебиватися“ (стор. 69).

1864 рік. Квітень. „В різних округах час від часу чути нарікання
на недостачу робітників, головним чином у певних галузях
промисловості, наприклад, у ткацькій... але ці нарікання в однаковій
мірі є результатом як тієї незначної плати, яку можуть заробити
робітники в наслідок застосовування поганих сортів пряжі,
так і деякої дійсної недостачі самих робітників в цій особливій
галузі. Минулого місяця відбулося багато сутичок між деякими
фабрикантами і їх робітниками з приводу заробітної плати.
Я шкодую, що страйки відбувалися занадто часто... Діяння
Public Works Act’a фабриканти сприймають як конкуренцію,
і в наслідок цього місцевий комітет у Bacup’i припинив свою
діяльність, бо, хоч працюють не всі ще фабрики, а, проте, виявилась
недостача робітників“ („Rep. of Insp. of Fact., April 1864“,
стор. 9, 10). І справді, для панів фабрикантів це був крайній
час, коли вони повинні були діяти. В наслідок Public Works
Act’a попит настільки зріс, що в каменоломнях Bacup’а деякі
фабричні робітники заробляли тепер 4—5 шилінгів на день.
І тому помалу були припинені громадські роботи, це нове видання
ateliers nationaux [національних майстерень] 1848 року,
але організованих цього разу в інтересах буржуазії.

Experimente in corpore vili\footnote*{
[Експерименти над нічого не вартим тілом].
}

„Хоч я навів тут дуже знижену заробітну плату (робітників,
що працюють повний час), дійсний заробіток робітників на
різних фабриках, проте з цього ніяк не випливає, що вони
з тижня на тиждень заробляють ту саму суму. Заробіток робітників
зазнає тут великих коливань в наслідок постійного експериментування
фабрикантів з різними сортами й пропорціями бавовни
і відпадів на тій самій фабриці; ці „мішанки“, як їх звуть,