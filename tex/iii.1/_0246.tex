\parcont{}  %% абзац починається на попередній сторінці
\index{iii1}{0246}  %% посилання на сторінку оригінального видання
які можуть бути перетворені в капітал, незалежно від їх мінової
вартості; речей, які можуть служити для того, щоб вбирати добавну
працю, отже й добавну додаткову працю, і таким чином
утворити додатковий капітал. Маса праці, якою може розпоряджатись
капітал, залежить не від його вартості, а від маси
сировинних і допоміжних матеріалів, машин і елементів основного
капіталу, засобів існування, тобто всього того, з чого складається
капітал, яка б не була його вартість. В той час, як зростає
таким чином маса вживаної праці, отже й додаткової праці, зростає
також вартість репродукованого капіталу і новододана до
неї додаткова вартість.

Але ці обидва моменти, які включає в собі процес нагромадження,
не можна розглядати тільки в тому стані спокійного співіснування
їх одного поряд одного, в якому їх досліджує Рікардо;
вони містять у собі суперечність, яка виявляється в суперечних
тенденціях і явищах. Антагоністичні фактори діють одночасно
один проти одного.

Одночасно з спонуканням до дійсного збільшення робітничого
населення, що випливає із збільшення частини сукупного
суспільного продукту, яка функціонує як капітал, діють фактори,
які створюють тільки відносне перенаселення.

Одночасно з падінням норми зиску зростає маса капіталів,
і рука в руку з цим відбувається знецінення наявного капіталу,
яке затримує це падіння і спонукає до прискореного нагромадження
капітальної вартості.

Одночасно з розвитком продуктивної сили розвивається
вищий склад капіталу, відносне зменшення змінної частини порівняно
з сталою.

Ці різні впливи виявляються то більше один поряд з одним
у просторі, то більше один по одному в часі; конфлікт антагоністичних
факторів періодично розв’язується (macht sich Luft)
в кризах. Кризи завжди є тільки тимчасові насильні розв’язання
наявних суперечностей, насильні вибухи, які на мить відновлюють
порушену рівновагу.

Суперечність, в її найзагальнішому виразі, полягає в тому,
що капіталістичний спосіб виробництва має тенденцію до абсолютного
розвитку продуктивних сил, незалежно від вартості і
вміщеної в ній додаткової вартості, а також незалежно від
суспільних відносин, при яких відбувається капіталістичне виробництво;
тимчасом як, з другого боку, він має своєю метою
збереження існуючої капітальної вартості і збільшення її в якнайвищій
мірі (тобто постійно прискорюване зростання цієї вартості).
Специфічна особливість капіталістичного способу виробництва
полягає в тому, щоб наявну капітальну вартість використати,
як засіб для якомога дужчого збільшення цієї вартості. Методи,
якими він цього досягає, ведуть до зменшення норми зиску,
знецінення наявного капіталу і розвитку продуктивних сил праці
за рахунок вироблених уже продуктивних сил.
