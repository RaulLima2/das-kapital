ньому відділі ми бачили, що зміна величини вартості змінного
капіталу іноді виражає не що інше, як більшу або меншу ціну
тієї самої маси праці; але тут, де норма додаткової вартості
і робочий день розглядаються як незмінні, а заробітна плата
за певний робочий час як величина дана, це відпадає. Навпаки,
ріжниця у величині сталого капіталу може, правда, бути також
показником зміни маси засобів виробництва, приведених в рух
певною кількістю робочої сили; але вона може також походити
з ріжниці у вартості засобів виробництва, приведених в рух
у певній сфері виробництва, порівняно з іншими сферами. Тим
то тут треба взяти до уваги обидві ці точки зору.

Нарешті, треба зробити ще таке істотне зауваження:

Припустім, що 100 фунтів стерлінгів становлять тижневу
заробітну плату 100 робітників. Припустім, що тижневий робочий час дорівнює 60 годинам. Припустімо,
далі, що норма
додаткової вартості = 100\%. В цьому випадку робітники 30 годин з 60 працюють на себе самих, а 30
даром на капіталіста.
В 100 фунтах стерлінгів заробітної плати в дійсності втілено
тільки 30 робочих годин 100 робітників, або разом 3000 робочих годин, тимчасом як інші 3000 годин,
які вони працюють,
втілені в 100 фунтах стерлінгів додаткової вартості, відповідно — зиску, що його забирає собі
капіталіст. Тому, хоч заробітна плата в 100 фунтів стерлінгів не виражає тієї вартості,
в якій упредметнюється тижнева праця 100 робітників, вона
все ж показує (бо довжина робочого дня і норма додаткової
вартості є дані), що цим капіталом приведено в рух 100 робітників на протязі загалом 6000 робочих
годин. Капітал в 100 фунтів стерлінгів показує це, тому що він, по-перше, показує
число приведених в рух робітників, бо 1 фунт стерлінгів = 1 робітникові за тиждень, отже 100 фунтів
стерлінгів = 100 робітникам; і, по-друге, тому що кожний приведений
в рух робітник, при даній нормі додаткової вартості в 100\%,
виконує вдвоє більше праці, ніж міститься в його заробітній
платі, отже, 1 фунт стерлінгів, його заробітна плата, що є виразом півтижневої праці, приводить в
рух працю цілого тижня,
і так само 100 фунтів стерлінгів, хоч вони містять в собі тільки 50
тижнів праці, приводять в рух працю 100 робочих тижнів. Отже,
тут треба мати на увазі дуже істотну ріжницю між змінним капіталом, витраченим на заробітну плату,
оскільки його вартість,
сума заробітних плат, представляє певну кількість упредметненої праці, і цим капіталом, оскільки
його вартість є простий показник маси живої праці, яку він приводить в рух. Ця
остання завжди більша, ніж кількість праці, яка міститься
в змінному капіталі, і тому вона виражається також у вартості
більшій, ніж вартість змінного капіталу — у вартості, яка визначається, з одного боку, числом
приведених в рух змінним капіталом робітників, а з другого боку, кількістю виконуваної ними
додаткової праці.
