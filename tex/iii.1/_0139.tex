\index{iii1}{0139}  %% посилання на сторінку оригінального видання
1861—1864 рр. Американська громадянська війна. Cotton Famine [бавовняний
голод]. Найбільший приклад перерви в процесі виробництва в наслідок
недостачі й дорожнечі сировинного матеріалу

1860 рік. Квітень. „Щодо стану справ, то я радий можливості
повідомити вас, що, не зважаючи на високу ціну сировинних
матеріалів, всі галузі текстильної промисловості, за винятком
шовку, працювали протягом останнього півроку дуже добре...
В деяких бавовняних округах робітників шукали шляхом оголошень,
і робітники йшли туди з Норфолька та інших землеробських
графств... Як видно, в усіх галузях промисловості панує велика недостача
сировинного матеріалу. Тільки... ця недостача тримає нас
у певних межах. В бавовняній промисловості число новозбудованих
фабрик, розширення наявних фабрик і попит на робітників,
мабуть, ніколи ще не досягали такого високого рівня, як
тепер. Скрізь і всюди шукають сировинного матеріалу“ („Rep.
of Insp. of Fact., April 1860“ [стор. 57]).

1860 рік. Жовтень. „Стан справ у бавовняних, шерстяних
і льонопрядільних округах був добрий; в Ірландії він, як кажуть,
вже більше року навіть „дуже добрий“, і був би ще кращий,
коли б не висока ціна на сировинний матеріал. Прядільники
льону, здається, з більшим нетерпінням, ніж будьколи,
чекають відкриття індійських джерел постачання за допомогою
залізниць і відповідного розвитку індійського землеробства, щоб,
нарешті... добитися відповідного їх потребам подання льону“
(„Rep. of Insp. of Fact., Oct. 1860“, стор. 37).

1861 рік. Квітень. „Стан справ у даний момент пригнічений...
деякі бавовняні фабрики працюють неповний час і багато шовкових
фабрик працюють тільки частково. Сировинний матеріал
дорогий. Майже в усіх галузях текстильної промисловості
ціна його вища, ніж та, при якій він міг би бути перероблений
для маси споживачів“ („Rep. of Insp. of Fact., April 1861“, стор. 33).

Тепер виявилось, що в 1860 році в бавовняній промисловості
була перепродукція; наслідки цього давалися взнаки ще протягом
ближчих років. „Потрібно було від двох до трьох років,
поки світовий ринок поглинув перепродукцію 1860 року“ („Rep.
of Insp. of Fact., October 1863“, стор. 127). „Пригнічений стан
ринків бавовняних фабрикатів у Східній Азії, на початку 1860 року,
справив відповідний зворотний вплив на стан справ у Блекберні,
де пересічно 30 000 механічних ткацьких верстатів майже виключно
заняті у виробництві тканин для цього ринку. В наслідок
цього попит на працю був уже тут обмеженим багато місяців
перед тим, як став відчутним вплив бавовняної блокади...
На щастя, це уберегло багатьох фабрикантів від краху. Запаси,
поки їх тримали на складах, підвищились у своїй вартості, і таким
чином уникнуто було того жахливого знецінення, яке
інакше при такій кризі було б неминучим“ („Rep. of Insp. of
Fact., Oct. 1862“, стор. 28, 29 [30]).
