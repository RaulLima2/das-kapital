ють у функції монети, хоч вони постійно заміщають капітал.
Певна частина грошей у країні завжди присвячена цій функції,
хоч ця частина складається з окремих монет, які постійно міняються. Навпаки, оскільки гроші
опосереднюють передачу капіталу, чи то як засіб купівлі (засіб циркуляції), чи як засіб
платежу, вони є капітал. Отже, не функція грошей як засобу
купівлі і не функція їх як засобу платежу відрізняє їх від
монети, бо гроші можуть функціонувати як засіб купівлі навіть
між торговцем і торговцем, оскільки вони купують один у одного
за готівку, і вони можуть також функціонувати як засіб платежу
між торговцем і споживачем, оскільки дається кредит, і дохід спочатку споживається, а потім
сплачується. Отже, ріжниця полягає в тому, що в другому випадку ці гроші не тільки заміщають капітал
для однієї сторони, продавця, але й витрачаються,
авансуються як капітал другою стороною, покупцем. Отже,
в дійсності це відмінність грошової форма доходу від грошової
форми капіталу, а не відмінність засобів циркуляції від капіталу,
бо як посередник між торговцями, цілком так само як і посередник між споживачами і торговцями,
циркулює певна за своєю
кількістю частина грошей, і в наслідок цього це в обох функціях однаково циркуляція. Але в погляди
Тука тут домішується
різного роду плутанина:

1) внаслідок змішання функціональних визначень;

2) внаслідок приплутання питання про кількість циркулюючих грошей, узятих разом в обох функціях;

3) внаслідок приплутання питання про відносні пропорції
кількостей засобів циркуляції, що циркулюють в обох функціях
і тому в обох сферах процесу репродукції.

До пункту 1) про змішання функціональних визначень грошей, тобто що гроші в одній формі є засіб
циркуляції (currency),
а в другій формі — капітал. Оскільки гроші служать в тій або
другій функції, чи то для реалізації доходу, чи для передачі
капіталу, вони функціонують у купівлі й продажу або в платежу,
як засіб купівлі або засіб платежу, і в дальшому значенні
слова як засіб циркуляції. Дальше визначення грошей, яке вони
мають в рахунку того, хто їх витрачає або одержує, — чи представляють вони для нього капітал чи
дохід, — тут абсолютно нічого не змінює, і це виявляється в двоякій формі. Хоч грошові
знаки, які циркулюють в обох сферах, є різні, проте, той самий
грошовий знак, наприклад, п’ятифунтова банкнота, переходить
з однієї сфери в другу і навпереміну виконує обидві функції; це
вже тому є неминуче, що роздрібний торговець може дати
своєму капіталові грошову форму тільки в формі монети, яку
він одержує від своїх покупців. Можна прийняти, що власне
розмінна монета має центр ваги своєї циркуляції в сфері
роздрібної торгівлі; роздрібному торговцеві вона постійно потрібна для розміну, і він постійно
одержує її назад у платежах
від своїх покупців. Але він одержує також гроші, тобто монету
