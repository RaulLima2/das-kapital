чається не тим часом, протягом якого вони просто існують,
а загальною тривалістю процесу праці, на протязі якого вони
функціонують і використовуються. Якщо робітники мусять працювати
18 годин замість 12, то це становить за тиждень на три
дні більше, тиждень перетворюється в півтора тижня, два
роки — в три. Отже, якщо надурочний час не оплачується, то
робітники, крім нормального часу додаткової праці, дають задарма
на кожні два тижні третій, на кожні два роки третій.
І таким чином репродукція вартості машин прискорюється на 50\%
і закінчується за 2/3 часу, необхідного при звичайних умовах.

У цьому дослідженні, так само як і в дослідженні коливань
ціни сировинного матеріалу (в розд. VI), ми, щоб уникнути
зайвих ускладнень, виходимо з припущення, що масу і норму
додаткової вартості дано.

Як уже зазначено при розгляді кооперації, поділу праці
і ролі машин, економія в умовах виробництва, яка характеризує
виробництво у великому масштабі, в істотному виникає з того,
що ці умови функціонують як умови суспільної, суспільно-комбінованої
праці, отже, як суспільні умови праці. Вони
споживаються у процесі виробництва спільно, колективним робітником,
замість споживатись у роздрібненій формі масою
незв’язаних між собою робітників або в кращому разі робітниками,
в незначній мірі безпосередньо зв’язаними відносинами співробітництва.
На великій фабриці з одним або двома центральними
двигунами витрати на ці двигуни зростають не в тій самій пропорції,
в якій зростає кількість їх кінських сил, і отже можлива сфера
їх діяння; витрати на передатні механізми зростають не в тій самій
пропорції, в якій зростає маса робочих машин, яким вони передають
рух; самий корпус робочої машини дорожчає не в тій
пропорції, в якій збільшується число знарядь, якими вона діє
як своїми органами, і т. д. Далі, концентрація засобів виробництва
дає заощадження на будівлях усякого роду, не тільки
на власне майстернях, але й на складських приміщеннях і т. д.
Так само стоїть справа з видатками на опалення, освітлення
і т. д. Інші умови виробництва лишаються ті самі, все одно,
багато чи мало людей використовує їх.

Але вся ця економія, яка виникає з концентрації засобів виробництва
та їх масового застосування, передбачає, як істотну
умову, скупчення й спільну діяльність робітників, тобто суспільну
комбінацію праці. Отже, вона виникає з суспільного характеру
праці цілком так само, як додаткова вартість виникає
з додаткової праці кожного окремого робітника, розглядуваного
ізольовано. Навіть постійні поліпшення, які тут можливі й потрібні,
виникають виключно з суспільних дослідів і спостережень,
що їх дає і уможливлює виробництво комбінованого у
великому масштабі колективного робітника.

Те саме стосується і другої великої галузі економії в умовах
виробництва. Ми маємо на увазі зворотне перетворення
