йонів. Поки не настане загальне раптове, вимагання повернути вклади (a run
on the banks [штурм банків]), ті самі 1000 фунтів стерлінгів, мандруючи
назад, можуть з такою самою легкістю покрити таку ж невизначну суму.
Через те що ті самі 1000 фунтів стерлінгів, якими я сьогодні покриваю свій
борг якомусь комерсантові, завтра можуть покрити його борг іншому купцеві,
а позавтра борг цього останнього банкові, і так далі до безконечності, то ті ж
самі 1000 фунтів стерлінгів можуть переходити з рук у руки, від банку до
банку, і покрити яку завгодно суму вкладів“.\footnote*{
Цей абзац перенесений сюди Енгельсом з іншої частини рукопису Маркса і в першому німецькому
виданні позначений нумером 7. Примітка ред. нім. вид. ІМЕЛ.
}

[Ми бачили, що Gilbart вже в 1834 році розумів, що „все,
що полегшує справи, полегшує і спекуляцію, що те і друге
в багатьох випадках так тісно зв’язане між собою, що важко
сказати, де кінчаються справи і починається спекуляція“. Чим
легше можна одержати позики під непродані товари, тим більше
беруться такі позики, тим більша спокуса виробляти товари або
вироблені вже товари кидати на віддалені ринки, тільки для того,
щоб спершу одержати під них грошову позику. Як весь торговельний світ країни може бути охоплений
такою спекуляцією
і чим вона закінчується, — яскравий приклад цього дає нам історія
англійської торгівлі 1845—1847 рр. Тут ми бачимо, що може зробити кредит. Для пояснення дальших
прикладів слід спочатку
зробити лиш декілька коротких зауважень.

Наприкінці 1842 року пригнічення, яке тяжило над англійською промисловістю майже безперервно з 1837
року, почало
слабшати. Протягом двох наступних років закордонний попит
на продукти англійської промисловості підвищився ще більше;
роки 1845—1846 були періодом найвищого розквіту. В 1843 році
війна за довіз опію відкрила для англійської торгівлі Китай. Новий
ринок дав новий поштовх розширенню, особливо бавовняної
промисловості, яка вже досягла повного розвитку. „Як можемо
ми виробляти занадто багато? Адже нам треба одягти 300 мільйонів чоловіка“, — казав тоді авторові
цих рядків один манчестерський фабрикант. Але всіх цих новозбудованих фабричних
будівель, парових і прядільних машин і ткацьких верстатів було
недосить для поглинення додаткової вартості Ланкашіра, яка
припливала великою масою. З такою самою пристрастю, з якою
збільшували виробництво, кинулись на будівництво залізниць;
тут насамперед знайшла собі задоволення жадоба фабрикантів і
купців до спекуляції, і саме вже з літа 1844 року. Підписувались на акції, скільки могли, тобто
оскільки вистачало грошей для покриття перших внесків; для дальших внесків — засоби знайдуться! А
коли настав строк дальших внесків, — згідно
з запитанням 1059, „Commercial Distress“ 1848/57, капітал, вкладений у 1846/47 рр. в залізниці,
становив 75 мільйонів фунтів
стерлінгів, — довелося вдатися до кредиту, і при цьому власні
справи фірми здебільшого теж мусили потерпіти.