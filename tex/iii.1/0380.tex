ховішої зовнішньої видимості речей. Потім цей фальшивий і тривіальний
зміст має бути „піднесений“ і опоетизований містифікуючим
способом викладу.

Процес нагромадження капіталу може бути зображений як
нагромадження процентів на проценти лиш остільки, оскільки
можна назвати процентом ту частину зиску (додаткової вартості),
яка зворотно перетворюється в капітал, тобто служить для висмоктування
нової додаткової праці. Але:

1) Залишаючи осторонь усі випадкові порушення, більша частина
наявного капіталу протягом процесу репродукції постійно
більш чи менш знецінюється, бо вартість товарів визначається
не тим робочим часом, якого первісно коштує їх виробництво,
а тим робочим часом, якого коштує їх репродукція, а цей останній
в наслідок розвитку суспільної продуктивної сили праці постійно
зменшується. Тому на вищому ступені розвитку суспільної
продуктивності весь наявний капітал виступає не як результат
довгого процесу нагромадження капіталу, а як результат
порівняно дуже короткого часу репродукції.82

2) Як показано у відділі III цієї книги, норма зиску зменшується
в міру зростаючого нагромадження капіталу і відповідного
йому зростання продуктивної сили суспільної праці, яке
виражається якраз у зростаючому відносному зменшенні змінної
частини капіталу порівняно з сталою. Для того, щоб дати ту
саму норму зиску, коли сталий капітал, що його приводить в рух
один робітник, збільшується вдесятеро, додатковий робочий час
мусив би збільшитися вдесятеро, і скоро для цього не вистачило
б усього робочого часу, всіх 24 годин на добу, навіть
коли б цей час цілком привласнювався капіталом. Уявлення ж, що
норма зиску не зменшується, лежить в основі прогресії Прайса
і взагалі в основі „all-engrossing capital, with compound interest“
(всезахоплюючого капіталу з складними процентами) 83.

Тотожність додаткової вартості і додаткової праці ставить
якісну межу нагромадженню капіталу: весь робочий день, даний
у кожний момент розвиток продуктивних сил і населення, який

82    Див. Мілля і Кері, а також хибний коментарій Рошера до цього. [Пор.
J. St. Mill: „Principles of Political Economy“. 2 вид, Лондон, 1849, стор. 92. H. G.
Carey: „Principles of Social Science“, т. III. Philadelphia 1860, стор. 71 і далі.
W. Roscher: „Die Grundlagen der Nationalökonomie“. 2 вид, Штутгарт і Аугсбург
1857, стор. 70 і далі].

83 „It is clear, that no labour, no productive power, no ingenuity, and no art,
can answer the everwhelming demands of compound interest. But all saving is
made from the revenue of the capitalist, so that actually the e demands are constantly
made and as constantly the productive power of labour refuses to satisfy
them. A sort of balance is, therefore, constant y struck“. („Ясно, що ніяка праця,
ніяка продуктивна сила, ніякий талант і ніяка вправність не можуть задовольнити
всепоглинаючих вимог складного процента. Але всяке заощадження робиться з
доходу капіталіста, так що в дійсності ці вимоги ставляться постійно і продуктивна сила праці так
само постійно відмовляється задовольняти їх. Тому
постійно порушується певний рід рівноваги“. („Labour defended against the Claims
of Capital“, стор. 23. — Hodgskin).
