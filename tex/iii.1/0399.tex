товарами, вищеописаний примусовий вивіз до Індії та примусовий
довіз з Індії з однією тільки метою здобути бронзові векселі.
Всі ці обставини, перепродукція в промисловості, так само як
і недопродукція в землеробстві, отже цілком різні причини, викликали підвищення попиту на грошовий
капітал, тобто на кредит і гроші. Підвищений попит на грошовий капітал мав свої
причини в ході самого процесу виробництва. Але, яка б не була
причина, саме попит на грошовий капітал викликав підвищення
розміру процента, вартості грошового капіталу. Якщо Оверстон хоче сказати, що вартість грошового
капіталу підвищилась, тому що вона підвищилась, то це — тавтологія. Якщо ж
він під „вартістю капіталу“ розуміє тут підвищення норми зиску
як причину підвищення розміру процента, то помилковість цього
зразу ж виявиться. Попит на грошовий капітал, а тому „вартість капіталу“, можуть підвищитись, хоч
зиск знижується; як
тільки відносне подання грошового капіталу знижується, „вартість“ його підвищується. Оверстон хоче
довести, що криза 1847 року і висока норма процента, яка супроводила її, ні трохи
не залежали від „кількості наявних грошей“, тобто від постанов інспірованого ним банкового акта 1844
року; хоча в дійсності вони залежали від неї, оскільки страх перед вичерпанням
банкового резерву — витвір Оверстона — долучив до кризи 1847—1848 рр. грошову паніку. Але тепер
питання не в цьому.
В наявності була нужда в грошовому капіталі, яка була спричинена надмірними розмірами операцій,
порівняно з наявними засобами, і яка вибухла через порушення процесу репродукції в наслідок
неврожаю, надмірного капіталовкладення в залізниці, перепродукції, особливо бавовняних товарів,
індійської і китайської
шахрайської торгівлі, спекуляції, надмірного довозу цукру і т. д.
Тим, хто купив хліб, коли він коштував 120 шилінгів за квартер, бракувало тоді, коли він впав до 60
шилінгів, саме тих
60 шилінгів, які вони переплатили, і відповідного кредиту під
заставу цього хліба. Зовсім не недостача в банкнотах заважала
їм обернути свій хліб в гроші по старій ціні в 120 шилінгів.
Так само стояла справа і з тими, хто довіз занадто багато
цукру, який потім майже не можна було продати. Так само й
у тих панів, які міцно вклали свій обіговий капітал (floating capital)
у залізниці і щодо заміщення його в своєму „законному“ підприємстві поклалися на кредит. Все це для
Оверстона виражається в „моральній свідомості підвищеної вартості його грошей“ („а moral sense of
the enhanced value of his money“). Але цій
підвищеній вартості грошового капіталу безпосередньо відповідала на другому боці знижена грошова
вартість реального капіталу
(товарного капіталу і продуктивного капіталу). Вартість капіталу
в одній формі підвищилась, тому що вартість капіталу в другій
формі знизилась. А Оверстон намагається обидві ці вартості різних
родів капіталу ототожнити в єдиній вартості капіталу взагалі, і
при тому таким способом, що протиставляє обидві ці вартості не-
