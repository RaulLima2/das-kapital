Хоч і яке велике значення має вивчення таких тертів для кожної
спеціальної праці про заробітну плату, все ж при загальному
дослідженні капіталістичного виробництва їх треба залишити
осторонь як випадкові і неістотні. При такому загальному
дослідженні взагалі завжди припускається, що дійсні відносини
відповідають своєму поняттю, або, що є те саме, дійсні відносини
зображаються лиш остільки, оскільки вони виражають свій
власний загальний тип.

Ріжниця норм додаткової вартості в різних країнах, отже,
національні ріжниці в ступенях експлуатації праці, для даного
дослідження зовсім не мають значення. Адже в цьому відділі ми
саме хочемо показати, яким чином в межах даної країни утворюється
певна загальна норма зиску. Однак, ясно, що при порівнянні
різних національних норм зиску треба тільки зіставити розвинуте
нами раніше з тим, що ми маємо розвинути тут. Спочатку
треба розглянути ріжницю в національних нормах додаткової
вартості, а потім, на основі цих даних норм додаткової вартості,
порівняти ріжницю національних норм зиску. Оскільки ріжниця
цих останніх не є результатом ріжниці національних норм додаткової
вартості, вона мусить виникати з обставин, при яких,
як і в нашому дослідженні в цьому розділі, додаткова вартість
припускається повсюди однаковою, постійною.

В попередньому розділі було показано, що, коли норму додаткової
вартості припустити незмінною, норма зиску, яку дає певний
капітал, може підвищуватись чи падати в наслідок обставин,
які підвищують або знижують вартість тієї чи іншої частини
сталого капіталу і тому взагалі впливають на відношення між
сталою і змінною складовими частинами капіталу. Далі було
відзначено, що обставини, які здовжують або скорочують час
обороту капіталу, можуть справляти аналогічний вплив на
норму зиску. Через те що маса зиску тотожна з масою додаткової
вартості, з самою додатковою вартістю, то виявилось також,
що маса зиску — відмінно від норми зиску — не зачіпається
щойно згаданими коливаннями вартості. Вони модифікують тільки
норму, в якій виражається дана додаткова вартість, отже й зиск
даної величини, тобто модифікують його відносну величину,
його величину порівняно з величиною авансованого капіталу.
Оскільки в наслідок таких коливань вартості відбувається зв’язування
або звільнення капіталу, таким посереднім шляхом може
бути зачеплена не тільки норма зиску, але й самий зиск. Однак,
це завжди має силу тільки для капіталу, уже вкладеного, а не
для нових капіталовкладень; і, крім того, збільшення або зменшення
самого зиску завжди залежить від того, наскільки більше
чи менше праці в наслідок таких коливань вартості може бути
приведено в рух тим самим капіталом, отже, від того, наскільки
більшу чи меншу масу додаткової вартості може — при незмінній
нормі додаткової вартості — виробити той самий капітал.
Аж ніяк не суперечачи загальному законові і не становлячи ви-
