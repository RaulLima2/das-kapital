\parcont{}  %% абзац починається на попередній сторінці
\index{iii1}{0257}  %% посилання на сторінку оригінального видання
полягає саме в тому, що частка живої праці зменшується,
а частка минулої праці збільшується, але так, що загальна сума
вміщеної в товарі праці зменшується; отже, так, що жива праця
зменшується дужче, ніж збільшується минула. Минула праця,
втілена у вартості товару — стала частина капіталу — складається
почасти із зношування основного, почасти з обігового
сталого капіталу, який цілком входить у товар, — сировинного
й допоміжного матеріалу. Та частина вартості, що походить
з сировинного й допоміжного матеріалу, мусить з розвитком
продуктивності праці зменшуватись, бо ця продуктивність відносно
зазначених матеріалів виявляється саме в тому, що їх вартість
знижується. Навпаки, найхарактернішим для зростаючої
продуктивної сили праці є саме те, що основна частина сталого
капіталу дуже значно збільшується, а разом з тим так само
збільшується і та частина його вартості, яка в наслідок зношування
переноситься на товари. Для того, щоб новий метод
виробництва виявився як справжнє підвищення продуктивності,
він мусить переносити на окремий товар меншу додаткову частину
вартості, відповідну зношуванню основного капіталу,
ніж та частина вартості, яка віднімається, заощаджується в наслідок
зменшення живої праці, — одним словом він мусить зменшувати
вартість товару. Само собою зрозуміло, що він мусить
зменшувати її навіть і тоді, коли — як це буває в окремих випадках
— в утворення вартості товару входить, крім додатково зношуваної
частини основного капіталу, додаткова частина вартості
відповідно до більшої кількості або до дорожчих сировинних і
допоміжних матеріалів. Всі надбавки до вартості мусять бути
більше ніж урівноважені зменшенням вартості, яке випливає із
зменшення живої праці.

Тому це зменшення загальної кількості праці, яка входить
у товар, здавалося б, мало бути істотною ознакою підвищеної
продуктивної сили праці, незалежно від того, при яких суспільних
умовах відбувається виробництво. В суспільстві, в якому
виробники регулюють своє виробництво за складеним заздалегідь
планом, і навіть при простому товарному виробництві продуктивність
праці безумовно вимірювалась би цим масштабом.
Але як стоїть справа при капіталістичному виробництві?

Припустім, що певна капіталістична галузь виробництва
виробляє нормальну штуку свого товару при таких умовах: зношування
основного капіталу становить на штуку 1/2 шилінга або
марки; сировинного й допоміжного матеріалу входить у кожну
штуку на 17 1/2 шилінгів; заробітної плати 2 шилінги, і при нормі
додаткової вартості в 100\% додаткова вартість становить 2 шилінги.
Вся вартість = 22 шилінгам або маркам. Для спрощення
ми припускаємо, що в цій галузі виробництва капітал має пересічний
склад суспільного капіталу, що, отже, ціна виробництва
товару збігається з його вартістю, а зиск капіталіста збігається
з виробленою додатковою вартістю. В такому разі витрати
\parbreak{}  %% абзац продовжується на наступній сторінці
