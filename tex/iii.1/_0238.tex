\parcont{}  %% абзац починається на попередній сторінці
\index{iii1}{0238}  %% посилання на сторінку оригінального видання
І до спожитих на їх виробництво засобів праці дедалі знижується; отже, та обставина, що в цих
товарах упредметнюється дедалі менша кількість додаваної живої праці, бо з розвитком суспільної
продуктивної сили потрібно менше праці для їх виробництва — ця обставина не стосується до того
відношення, в якому уміщена в товарі жива праця ділиться на оплачену і неоплачену. Навпаки. Хоча
загальна кількість уміщеної в товарі додаваної живої праці зменшується, неоплачена частина зростає
порівняно з оплаченою в наслідок або абсолютного, або відносного зниження оплаченої частини; бо той
самий спосіб виробництва, який зменшує загальну масу додаваної живої праці в кожному окремому
товарі, супроводиться зростанням абсолютної і відносної додаткової вартості. Тенденція норми зиску
до зниження зв’язана з тенденцією до підвищення норми додаткової вартості, отже й ступеня
експлуатації праці. Тому немає нічого більш безглуздого, як поясняти зниження норми зиску
підвищенням норми заробітної плати, хоч винятково і це може мати місце. Тільки зрозумівши відносини,
при яких утворюється норма зиску, статистика стає спроможною взятись до дійсного аналізу норми
заробітної плати в різні епохи і в різних країнах. Норма зиску падає не тому, що праця стає менш
продуктивною, а тому, що вона стає більш продуктивною. І те і друге, підвищення норми додаткової
вартості і падіння норми зиску, є тільки особливі форми, в яких капіталістично виражається зростаюча
продуктивність праці.

VI. Збільшення акційного капіталу

До вищенаведених п’яти пунктів можна додати ще один пункт, на якому ми, однак, покищо не можемо
детальніше спинятись. З прогресом капіталістичного виробництва, який іде рука в руку з прискореним
нагромадженням, частина капіталу враховується і застосовується тільки як капітал, що дає процент. Не
в тому розумінні, в якому кожний капіталіст, що позичає комусь капітал, задовольняється процентами,
тимчасом як промисловий капіталіст одержує підприємницький зиск. Це не стосується до висоти
загальної норми зиску, бо для неї зиск = процентові + зиск усякого роду + земельна рента, при чому
розподіл на ці особливі категорії для неї не має значення. А в тому
розумінні, що ці капітали, хоч і вкладені у великі продуктивні підприємства, дають після
відрахування всіх витрат тільки великі або малі проценти, так звані дивіденди; наприклад, у
залізничній справі. Отже, вони не беруть участі у вирівнюванні загальної норми зиску, бо вони дають
меншу норму зиску, ніж пересічна норма. Коли б вони брали участь у вирівнюванні, то ця остання упала
б значно нижче. Розглядаючи справу теоретично, їх можна врахувати, і тоді одержимо меншу норму
зиску, ніж та, яка, видимо, існує і дійсно є визначальною для капіталістів, — одержимо меншу норму
зиску, бо саме в цих підприємствах сталий капітал є найбільший порівняно з змінним.
