\index{iii1}{0282}  %% посилання на сторінку оригінального видання
Якого б роду не були ці витрати циркуляції, — чи виникають
вони з чисто купецького підприємства як такого, отже, належать
до специфічних витрат циркуляції купця, чи представляють
затрати, які виникають з додаткових процесів виробництва, що
долучаються під час процесу циркуляції, як відправка, транспортування,
зберігання і т. д., — на боці купця вони завжди передбачають,
крім авансованого на купівлю товарів грошового капіталу,
додатковий капітал, авансований на купівлю і оплату цих
засобів циркуляції. Оскільки цей елемент витрат складається
з обігового капіталу, він як додатковий елемент входить цілком
у продажну ціну товарів; оскільки ж він складається з основного
капіталу, він входить у неї як додатковий елемент в міру
свого зношування; але він входить у неї як елемент, який утворює
номінальну вартість, навіть якщо він, як от чисто купецькі
витрати циркуляції, не становить ніякого дійсного додатку до
вартості товару. Але весь цей додатковий капітал, обіговий
чи основний, бере участь в утворенні загальної норми зиску.

Чисто купецькі витрати циркуляції (отже, за винятком витрат
відправки, транспортування, зберігання і т. д.) зводяться до тих
витрат, які потрібні для того, щоб реалізувати вартість товару,
перетворити її з товару в гроші або з грошей у товар, опосереднити
обмін між ними. При цьому цілком залишаються осторонь
можливі процеси виробництва, які продовжуються протягом
акту циркуляції і від яких торговельне підприємство може
існувати цілком відокремлено; подібно до того, як в дійсності,
наприклад, власне транспортна промисловість та відправка можуть
бути і є галузі промисловості, цілком відмінні від торгівлі,
так само й товари, які мають бути куплені й продані,
можуть лежати в доках та інших громадських приміщеннях, при
чому витрати, які виникають з цього, оскільки купцеві доводиться
їх авансувати, нараховуються на нього третіми особами.
Все це має місце у власне гуртовій торгівлі, де купецький
капітал виступає в найчистішому вигляді і найменше переплітається
з іншими функціями. Підприємець-перевізник, директор
залізниці, судновласник — не „купці“. Витрати, які ми тут розглядаємо,
це витрати купівлі й продажу. Ми вже раніш відзначили,
що вони зводяться до обрахунків, ведення книг, ринкових
видатків, кореспонденції і т. д. Потрібний для цього сталий
капітал складається з контори, паперів, поштових знаків
і т. д. Інші витрати зводяться до змінного капіталу, який авансується
на вживання торговельних найманих робітників. (Витрати
відправки, витрати транспортування, авансування на оплату
мита і т. д. почасти можна розглядати таким чином, ніби купець
авансує їх на закупівлі товарів, і що тому вони для нього входять
у купівельну ціну.)

Всі ці витрати робляться не при виробництві споживної вартості
товарів, а при реалізації їх вартості; це — чисті витрати
циркуляції. Вони входять не в безпосередній' процес виробництва,
\index{iii1}{0283}  %% посилання на сторінку оригінального видання
а в процес циркуляції, а тому в сукупний процес репродукції.

Єдина частина цих витрат, яка нас тут цікавить, це частина,
витрачена на змінний капітал. (Крім того, слід було б дослідити:
поперше, яким чином зберігає в процесі циркуляції своє
значення закон, згідно з яким у вартість товару входить тільки
необхідна праця? Подруге, в чому виявляється нагромадження
при купецькому капіталі? Потрете, як функціонує купецький капітал
у дійсному сукупному суспільному процесі репродукції?)

Ці витрати зумовлюються економічною формою продукту як
товару.

Якщо робочий час, який втрачають самі промислові капіталісти
на те, щоб продавати свої товари безпосередньо один
одному, — отже, об’єктивно кажучи, час обігу товарів, — зовсім
не додає до цих товарів ніякої вартості, то ясно, що цей робочий
час не набуває іншого характеру від того, що його доводиться
втрачати не промисловому капіталістові, а купцеві. Перетворення
товару (продукту) в гроші і грошей у товар (у засоби
виробництва) є необхідна функція промислового капіталу і, отже,
необхідна операція капіталіста, який в дійсності є тільки персоніфікований
капітал, обдарований власною свідомістю і волею.
Але ці функції не збільшують вартості і не створюють додаткової
вартості. Виконуючи ці операції або опосереднюючи далі
функції капіталу в сфері циркуляції, після того як продуктивний
капіталіст перестав це робити, купець тільки займає місце
промислового капіталіста. Робочий час, що його коштують ці
операції, вживається на необхідні операції в процесі репродукції
капіталу, але він не додає ніякої вартості. Коли б купець
не виконував цих операцій (отже, і не витрачав би потрібного на
них робочого часу), то він не застосовував би свого капіталу
як агент циркуляції промислового капіталу, він не продовжував би
перерваної функції промислового капіталіста, і тому не брав би
участі як капіталіст pro rata [пропорціонально до] свого авансованого
капіталу в одержанні маси зиску, вироблюваної класом
промислових капіталістів. Тому, щоб брати участь в одержанні
маси додаткової вартості, щоб авансована ним сума зростала
у своїй вартості як капітал, торговельному капіталістові немає
потреби вживати найманих робітників. Якщо його підприємство
і його капітал невеликі, то він сам може бути тим єдиним робітником,
якого він уживає. Те, чим він оплачується, є частина
зиску, яка виникає для нього з ріжниці між купівельною ціною
товарів і їх дійсною ціною виробництва.

Але, з другого боку, при незначному розмірі авансовуваного
купцем капіталу, зиск, який він реалізує, може бути ні трохи
не більший і навіть менший, ніж заробітна плата краще оплачуваних
вправних найманих робітників. Справді, поряд з ним функціонують
безпосередні торговельні агенти продуктивного капіталіста,
закупники, продавці, комівояжери, які одержують стільки ж
\parbreak{}  %% абзац продовжується на наступній сторінці
