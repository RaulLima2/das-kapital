Безглузде уявлення, ніби витрати виробництва товару становлять
його дійсну вартість, а додаткова вартість виникає
з продажу товару вище його вартості, що, отже, товари продаються
по їх вартостях, якщо їх продажна ціна дорівнює витратам
їх виробництва, тобто дорівнює ціні спожитих на них
засобів виробництва плюс заробітна плата, — це уявлення Прудон
з звичним шахрайством, яке чваниться вченістю, просурмив
як нововідкриту таємницю соціалізму. Це зведення вартості
товарів до витрат їх виробництва становить по суті
основу його народного банку. Раніше ми з’ясували, що різні складові
частини вартості продукту можна представити в пропорціональних
частинах самого продукту. Якщо, наприклад (книга І,
розд. VIІ, 2, стор. 229 \footnote*{
Стор. 153—154 рос. вид. 1935 р. Ред. укр. перекладу.
}), вартість 20 фунтів пряжі становить
30 шилінгів — а саме 24 шилінги засоби виробництва, 3 шилінги
робоча сила і 3 шилінги додаткова вартість, — то цю додаткову
вартість можна представити як 1/10 продукту = 2 фунтам пряжі.
Тепер, якщо ці 20 фунтів пряжі продаються по витратах їх
виробництва, за 27 шилінгів, то покупець дістає даром 2 фунти
пряжі, або товар продано на 1/10 нижче його вартості; але робітник
так само, як і раніш, дав свою додаткову працю — тільки
для покупця пряжі, а не для капіталістичного виробника пряжі.
Було б цілком помилково припускати, що коли б усі товари
продавались по витратах їх виробництва, то результат фактично
був би той самий, як коли б усі товари продавались вище витрат
їх виробництва, але по їх вартостях. Бо навіть коли припустити,
що вартість робочої сили, довжина робочого дня
і ступінь експлуатації праці повсюди однакові, то все ж маси
додаткової вартості, які містяться у вартостях різних видів
товару, аж ніяк не рівні, залежно від різного органічного складу
капіталів, авансованих на їх виробництво\footnote{
„Вироблювані різними капіталами маси вартості і додаткової вартості, при
даній вартості і однаковому ступені експлуатації робочої сили, прямо пропорціональні
до величин змінних складових частин цих капіталів, тобто їх складових
частин, перетворених у живу робочу силу“ (книга 1, розд. ІХ, стор. 321
[стор. 227 рос. вид. 1935 р.]).
}.

Розділ другий
Норма зиску

Загальна формула капіталу є Г — Т — Г'; тобто певна сума
вартості кидається в циркуляцію для того, щоб витягти з неї
більшу суму вартості. Процес, який породжує цю більшу суму
вартості, є капіталістичне виробництво; процес, який реалізує
її, є циркуляція капіталу. Капіталіст виробляє товар не ради
самого товару, не ради його споживної вартості або свого особистого
споживання. Продукт, який в дійсності цікавить капіта-