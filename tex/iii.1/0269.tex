ляції у сферу споживання, Т — Г, остаточне вивершення її першої метаморфози. Отже, чи купує купець у
промислового капіталіста чи продає йому, його Г — Т — Г, кругобіг купецького капіталу, завжди
виражає тільки те, що відносно самого товарного капіталу, як перехідної форми промислового капіталу,
який
репродукує себе, є просто Т — Г, просто здійснення його першої метаморфози. Г — Т купецького
капіталу є разом з тим Т — Г
тільки для промислового капіталіста, але не для вироблюваного
ним товарного капіталу: це тільки перехід товарного капіталу
з рук промисловця в руки агента циркуляції; тільки Т — Г купецького капіталу є остаточне Т — Г
функціонуючого товарного
капіталу. Г — Т — Г є тільки два Т — Г того самого товарного капіталу, два послідовні продажі його,
які тільки й опосереднюють його останній і остаточний продаж.

Отже, товарний капітал набирає у товарно-торговельному
капіталі вигляду самостійного роду капіталу в наслідок того, що
купець авансує грошовий капітал, який зростає в своїй вартості
як капітал, функціонує як капітал лиш остільки, оскільки він
уживається виключно для того, щоб опосереднювати метаморфозу товарного капіталу, його функцію як
товарного капіталу,
тобто його перетворення в гроші, і він робить це за допомогою постійної купівлі й продажу товарів.
Це є його виключна
операція; ця діяльність, яка опосереднює процес циркуляції
промислового капіталу, є виключна функція грошового капіталу,
яким оперує купець. Завдяки цій функції він перетворює свої
гроші в грошовий капітал, надає своєму Г вигляду Г — Т — Г'
і за допомогою того самого процесу перетворює товарний капітал у товарно-торговельний капітал.

Товарно-торговельний капітал, оскільки і поки він існує в формі
товарного капіталу, — з точки зору процесу репродукції сукупного суспільного капіталу, — є,
очевидно, не що інше, як
частина промислового капіталу, яка перебуває ще на ринку,
пророблює процес своєї метаморфози і тепер існує та функціонує як товарний капітал. Отже, грошовий
капітал, який ми
повинні тепер розглядати у відношенні до сукупного процесу
репродукції капіталу, є тільки авансований купцем грошовий
капітал, який призначений виключно для купівлі й продажу
і який через це ніколи не набирає іншої форми, крім форми товарного капіталу і грошового капіталу,
ніколи не набирає форми
продуктивного капіталу і завжди лишається замкненим у сфері
циркуляції капіталу.

Як тільки виробник, фабрикант полотна, продасть свої 30000
метрів купцеві за 3000 фунтів стерлінгів, він купує на виручені таким чином гроші потрібні засоби
виробництва, і його капітал
знову вступає у процес виробництва; його процес виробництва
триває далі, іде безперервно. Перетворення його товару в гроші
для нього відбулось. Але для самого полотна це перетворення
як ми бачили, ще не відбулось. Полотно ще не остаточно пере-
