Цей розвиток продуктивної сили у кінцевому рахунку завжди
зводиться до суспільного характеру праці, приведеної в діяльність;
до поділу праці всередині суспільства; до розвитку інтелектуальної
праці, особливо природознавства. Капіталіст використовує
тут вигоди всієї системи суспільного поділу праці.
Вартість застосовуваного капіталістом сталого капіталу тут
відносно знижується, отже й норма зиску підвищується, в наслідок
розвитку продуктивної сили праці в галузі, яка лежить
поза межами даної галузі промисловості, в галузі, яка постачає
капіталістові засоби виробництва.

Підвищення норми зиску виникає ще й іншим шляхом, а саме
не з економії на тій праці, яка виробляє сталий капітал, а з економії
в застосуванні самого сталого капіталу. З одного боку,
сталий капітал заощаджується в наслідок концентрації робітників
та їх кооперації у великому масштабі. Ті самі будівлі, пристрої
для опалення й освітлення тощо коштують відносно
менше при виробництві у великому масштабі, ніж при виробництві
у невеликому масштабі. Те саме можна сказати і щодо
рушійних і робочих машин.. Хоч вартість їх абсолютно підвищується,
але відносно, порівняно з зростаючим розширенням
виробництва і величиною змінного капіталу або масою робочої
сили, яка приводиться в рух, вона падає. Та економія, яку певний
капітал реалізує у своїй власній галузі виробництва, полягає
насамперед і безпосередньо в економії на праці, тобто в скороченні
оплачуваної праці його власних робітників; навпаки, згадана
раніш економія полягає в тому, щоб це якомога більше привласнювання
чужої неоплаченої праці здійснювати якнайбільш економним
способом, тобто з якнайменшими при даному масштабі
виробництва витратами. Оскільки ця економія основана не
на згаданій вже експлуатації продуктивності суспільної праці,
вживаної у виробництві сталого капіталу, а на економії в застосуванні
самого сталого капіталу, вона виникає або безпосередньо
з кооперації і суспільної форми праці в самій даній галузі виробництва,
абож з виробництва машин і т. д. в такому масштабі,
при якому їх вартість зростає не в такій мірі, як їх споживна
вартість.

Тут треба мати на увазі дві обставини: коли б вартість с = 0,
то р' було б = m' і норма зиску досягла б свого максимуму.
Але подруге: для безпосередньої експлуатації самої праці важлива
ні в якому разі не вартість застосовуваних засобів експлуатації,
чи то основного капіталу, чи сировинних і допоміжних
матеріалів. Оскільки вони служать вбирачами праці, засобами
(Media), в яких і за допомогою яких упредметнюється праця,
а тому й додаткова праця, мінова вартість машин, будівель,
сировинних матеріалів і т. д. зовсім не має значення. Єдине, що
тут має значення, це, з одного боку, їх маса, технічно потрібна
для сполучення з певною кількістю живої праці, з другого боку,
їх доцільність, отже, не тільки добрі машини, але й добрі сиро-
