шується продуктивна сила праці, отже, коли та сама маса праці
створює більше продукту і тому приводить в рух більше сталого капіталу. Те саме при певних
обставинах може мати місце
тоді, коли продуктивна сила зменшується, як, наприклад, у землеробстві, так що та сама кількість
праці потребує для створення
того самого продукту більше засобів виробництва, наприклад,
більше насіння або добрива, дренування і т. д. Без знецінення
сталий капітал може звільнятись тоді, коли в наслідок удосконалень, застосовування сил природи і т.
ін. сталий капітал меншої вартості стає спроможним технічно виконувати ту саму службу, яку раніше
виконував капітал вищої вартості.

У книзі II ми бачили, що після того як товари перетворені
в гроші, продані, певна частина цих грошей знову мусить бути
перетворена в речові елементи сталого капіталу і саме в тих
пропорціях, яких вимагає певний технічний характер кожної
даної сфери виробництва. Щодо цього в усіх галузях — залишаючи осторонь заробітну плату, отже,
змінний капітал — найважливішим елементом є сировинний матеріал, включаючи й допоміжні матеріали,
які особливо важать у тих галузях виробництва, в які не входить сировинний матеріал у власному
значенні, як от у копальнях і добувній промисловості взагалі. Та частина ціни, яка мусить замістити
зношування машин, поки
машини ще взагалі здатні функціонувати, входить в обрахунок
більше ідеально; не має особливого значення, коли саме ця
частина буде оплачена й заміщена грішми, сьогодні чи завтра,
чи в якийсь інший період часу обороту капіталу. Інакше стоїть
справа з сировинним матеріалом. Якщо ціна сировинного матеріалу підвищується, то може стати
неможливим, після відрахування заробітної плати, цілком замістити ціну його з вартості товару. Тому
сильні коливання цін викликають перерви, великі
колізії і навіть катастрофи в процесі репродукції. Продукти
землеробства у власному розумінні слова, сировинні матеріали,
які походять з органічної природи, особливо підпадають таким
коливанням вартості в наслідок мінливих врожаїв і т. д. — кредитну систему ми тут ще цілком
залишаємо осторонь. Та сама
кількість праці може тут в наслідок непіддатних контролеві природних умов, сприятливості чи
несприятливості діб року та ін.,
виражатися в дуже різних кількостях споживних вартостей,
і тому певна кількість цих споживних вартостей матиме дуже
різну ціну. Якщо вартість x представлена в 100 фунтах товару a, то ціна одного фунта a = x/100; якщо
ж вона представлена в 1000 фунтах а, то ціна одного фунта a = x/1000 і т. д. Такий, отже, є один
елемент цих коливань ціни сировинного матеріалу.
Другий елемент, про який тут згадується тільки для повноти, — бо конкуренція, як і кредитна система,
лежить тут поки що поза
межами нашого розгляду, — є такий: з самої природи речей
рослинні й тваринні речовини, ріст і виробництво яких підлягають
