хліба, яку він послідовно може купити й продати протягом даного
часу, наприклад, протягом року, тимчасом як оборот капіталу
фермера, незалежно від часу обігу, обмежений часом
виробництва, що триває рік.

Але оборот одного й того ж купецького капіталу може з таким
самим успіхом опосереднювати обороти капіталів у різних
галузях виробництва.

Оскільки один і той самий купецький капітал в різних оборотах
служить для того, щоб послідовно перетворювати різні
товарні капітали в гроші, отже, по черзі купує іі продає їх, він
як грошовий капітал виконує відносно товарного капіталу ту
саму функцію, яку взагалі гроші числом своїх обігів за даний
період виконують відносно товарів.

Оборот купецького капіталу не є тотожний з оборотом або
з одноразовою репродукцією рівновеликого промислового капіталу;
навпаки, він дорівнює сумі оборотів певного числа таких
капіталів, чи в тій самій, чи в різних сферах виробництва. Чим
швидше обертається купецький капітал, тим менша є та частина
всього грошового капіталу, яка фігурує як купецький капітал;
чим повільніше він обертається, тим більша ця частина. Чим
менше розвинене виробництво, тим більша є сума купецького
капіталу порівняно з сумою товарів, що їх взагалі кидають у циркуляцію;
але тим менша вона абсолютно або порівняно з більш
розвиненим станом виробництва. І навпаки. Тому при такому
нерозвиненому стані виробництва більша частина власне грошового
капіталу перебуває в руках купців, майно яких у відміну
від майна інших становить таким чином грошове майно.

Швидкість циркуляції авансовуваного купцем грошового
капіталу залежить: 1) від швидкості, з якою відновлюється
процес виробництва і переплітаються між собою різні процеси
виробництва; 2) від швидкості споживання.

Для того, щоб купецький капітал проробив тільки розглянутий
вище оборот, немає потреби спочатку купити товарів на
всю величину його вартості, а потім продавати їх. Купець одночасно
проробляє обидва ці рухи. Його капітал поділяється тоді
на дві частини. Одна складається з товарного капіталу, друга —
з грошового капіталу. Він купує в одному місці і перетворює
цим свої гроші у товар. Він продає в другому місці і перетворює
цим другу частину товарного капіталу в гроші. З одного
боку, до нього повертається його капітал як грошовий капітал,
тимчасом як, з другого боку, до нього припливає товарний капітал.
Чим більша частина, яка існує в одній формі, тим менша
частина, яка існує в другій формі. Ці частини міняються одна
з другою і урівноважують одна другу. Коли з уживанням грошей
як засобу циркуляції сполучається вживання їх як платіжного
засобу і кредитна система, яка виростає на цьому грунті, то
грошова частина купецького капіталу ще більше зменшується порівняно
з розмірами операцій, виконуваних цим купецьким капіта-
