Збільшення норми зиску завжди походить від того, що додаткова
вартість відносно або абсолютно збільшується порівняно
з витратами її виробництва, тобто порівняно з сукупним авансованим
капіталом, інакше кажучи, від того, що ріжниця між
нормою зиску і нормою додаткової вартості зменшується.

Коливання в нормі зиску, незалежно від зміни в органічних
складових частинах капіталу чи від абсолютної величини капіталу,
можливі через те, що вартість авансованого капіталу,
в якій би формі — основній чи обіговій — він не існував,
підвищується або падає в наслідок незалежного від уже наявного
капіталу збільшення чи зменшення робочого часу, потрібного
для його репродукції. Вартість всякого товару — отже
й тих товарів, з яких складається капітал, — визначається не
тим необхідним робочим часом, який міститься в ньому самому,
а тим суспільно-необхідним робочим часом, який потрібен для
його репродукції. Ця репродукція може відбутися при обтяжливіших
або полегшених обставинах, відмінних від умов первісного
виробництва. Якщо при змінених обставинах потрібно,
загалом кажучи, вдвоє більше або, навпаки, вдвоє менше часу,
щоб репродукувати той самий речовий капітал, то при незміненій
вартості грошей капітал, який раніше був вартий 100 фунтів
стерлінгів, тепер буде вартий 200 фунтів стерлінгів, відповідно
50 фунтів стерлінгів. Якби це підвищення вартості або
зниження вартості зачіпало всі частини капіталу в однаковій
мірі, то й зиск відповідно до цього виразився б у подвійній
або вдвоє меншій грошовій сумі. Якщо ж воно включає і зміну
в органічному складі капіталу, якщо воно підвищує або знижує
відношення змінної частини капіталу до сталої, то, при інших
однакових умовах, норма зиску зростатиме при відносному
зростанні і падатиме при відносному зменшенні змінного капіталу.
Якщо ж підвищується або падає тільки грошова вартість
(в наслідок зміни вартості грошей) авансованого капіталу, то
в тому самому відношенні підвищується або падає грошовий вираз
додаткової вартості. Норма зиску лишається незмінною.
