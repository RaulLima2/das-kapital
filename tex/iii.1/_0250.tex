\index{iii1}{0250}  %% посилання на сторінку оригінального видання
Частина старого капіталу мусила б лишитись без діла при
всяких обставинах, лишитись без діла щодо своєї властивості як
капіталу — функціонувати і зростати в своїй вартості. Яка саме
частина лишилася б без діла, це вирішила б конкурентна боротьба.
Поки все йде добре, конкуренція, як це виявилось при
вирівненні загальної норми зиску, діє, як практичний братерський
союз класу капіталістів, так що вони спільно ділять між собою загальну
здобич пропорціонально до величини частки, вкладеної
кожним з них. Але як тільки справа йде вже не про розподіл
зиску, а про розподіл збитку, то кожний з них намагається
якомога зменшити свою участь в ньому і перекласти його на
шию іншому. Для всього класу збиток є неминучий. Але скільки
з нього припаде на кожного окремого капіталіста, наскільки
взагалі кожний з них повинен взяти участь в ньому, це стає
тоді питанням сили й хитрості, і конкуренція перетворюється
тоді в боротьбу ворогуючих братів. Протилежність між інтересами
кожного окремого капіталіста і інтересами класу капіталістів виявляється
при цьому цілком так само, як перед тим за допомогою
конкуренції проявлялась на практиці тотожність цих інтересів.

Яким же чином міг би бути знов усунений цей конфлікт
і як могли б відновитись відносини, відповідні „здоровому“
рухові капіталістичного виробництва? Спосіб усунення міститься
вже в простому констатуванні конфлікту, про усунення якого
йде мова. Він полягає в залишенні без діла і навіть частковому
знищенні капіталу, рівного своєю вартістю всьому додатковому
капіталові ΔК або принаймні частині його. Хоча — як
це вже випливає з викладу конфлікту — розподіл цього збитку
ні в якому разі не поширюється рівномірно на поодинокі окремі
капітали, а вирішується в конкурентній боротьбі, в якій збиток
розподіляється дуже нерівно і в дуже різних формах, залежно
від особливих переваг або особливих завойованих уже позицій,
так що один капітал лишається лежати без діла, другий знищується,
третій має тільки відносний збиток або зазнає тільки
тимчасового знецінення і т. д.

Але при всяких обставинах рівновага відновилась би в наслідок
бездіяльності і навіть знищення капіталу в більшому чи
меншому розмірі. Це почасти поширилося б на матеріальну
субстанцію капіталу; тобто частина засобів виробництва, основний
і обіговий капітал, не функціонувала б, не діяла б як капітал;
частина підприємств, що вже почали функціонувати, припинила
б роботу. Хоча в цьому відношенні час робить своє
і погіршує всі засоби виробництва (за винятком землі), але тут
в наслідок припинення функціонування мало б місце значно
сильніше справжнє руйнування засобів виробництва. Головний
результат у цьому відношенні був би, однак, у тому, що ці засоби
виробництва перестали б діяти як засоби виробництва, — в зруйнуванні
їх функції як засобів виробництва на коротший чи довший
час.
