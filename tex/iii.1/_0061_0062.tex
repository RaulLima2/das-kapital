\parcont{}  %% абзац починається на попередній сторінці
\index{iii1}{0061}  %% посилання на сторінку оригінального видання
частини упредметненої в ньому праці, яку він оплатив. Додаткова
праця, що міститься в товарі, нічого не коштує капіталістові,
хоч робітникові вона цілком так само коштує праці, як
і оплачена, і хоч вона цілком так само, як і оплачена, створює
вартість і входить у товар як вартостетворчий елемент. Зиск
капіталіста постає з того, що він має для продажу щось, чого
він не оплатив. Додаткова вартість, відповідно зиск, складається
саме з надлишку товарної вартості понад витрати її виробництва,
тобто з надлишку всієї суми праці, вміщеної в товарі, понад
вміщену в ньому оплачену суму праці. Таким чином додаткова
вартість, звідки б вона не виникала, є надлишок понад увесь
авансований капітал. Отже, цей надлишок стоїть у такому відношенні
до всього капіталу, яке виражається дробом \frac{m}{K}, де
$К$ означає весь капітал. Таким чином одержуємо \emph{норму зиску}
\frac{m}{K} = \frac{m}{c+v}, у відміну від норми додаткової вартості \frac{m}{v}.

Величина додаткової вартості у її відношенні до змінного
капіталу зветься нормою додаткової вартості; величина додаткової
вартості у її відношенні до всього капіталу зветься нормою зиску.
Це два різні виміри тієї самої величини, які в наслідок ріжниці в
масштабах виражають одночасно різні пропорції або відношення
однієї і тої самої величини.

З перетворення норми додаткової вартості в норму зиску
слід виводити перетворення додаткової вартості в зиск, а не
навпаки. І справді, вихідним пунктом історично була норма зиску.
Додаткова вартість і норма додаткової вартості є, відносно, те
невидиме і суттьове, що треба розкрити, тимчасом як норма
зиску, а тому й така форма додаткової вартості як зиск виявляються
на поверхні явищ.

Щодо окремого капіталіста, то ясно, що єдине, що його
інтересує, це відношення додаткової вартості або надлишку вартості,
ради якого він продає свої товари, до всього капіталу,
авансованого на виробництво товару; тимчасом як певне відношення
цього надлишку до окремих складових частин капіталу
і його внутрішній зв’язок з цими складовими частинами не тільки
не інтересує його, але він ще й заінтересований в тому, щоб
оповити туманом це певне відношення і цей внутрішній зв’язок.

Хоча надлишок вартості товару понад витрати його виробництва
виникає в безпосередньому процесі виробництва, але реалізується
він тільки в процесі циркуляції, — і він тим легше набуває видимості
виникнення з процесу циркуляції, що в дійсності, серед
конкуренції, на дійсному ринку, від ринкових відносин залежить,
чи реалізується цей надлишок, чи ні, і в якому розмірі. Тут
немає потреби пояснювати, що коли товар продається вище
або нижче його вартості, то має місце. тільки інший розподіл
додаткової вартості, і що цей інший розподіл, змінене
відношення, в якому різні особи ділять між собою додаткову вартість,
\index{iii1}{0062}  %% посилання на сторінку оригінального видання
нічого не змінює ні в величині, ні в природі додаткової
вартості. В дійсному процесі циркуляції не тільки відбуваються
перетворення, які ми розглянули в книзі II, але вони збігаються
з дійсною конкуренцією, з купівлею і продажем товарів вище
або нижче їх вартості, так що для окремого капіталіста реалізована
ним самим додаткова вартість залежить так само від
взаємного ошукування, як і від безпосередньої експлуатації
праці.

В процесі циркуляції поряд робочого часу починає діяти час
циркуляції, який цим самим обмежує масу додаткової вартості,
яку можна реалізувати за певний період. На безпосередній
процес виробництва впливають визначально ще й інші моменти,
які виникають з циркуляції. І те і друге, безпосередній
процес виробництва і процес циркуляції, постійно переходять
один в один, пронизують один одного, і тим самим постійно
перекручують свої характерні відмінні ознаки. Виробництво додаткової
вартості, як і вартості взагалі, набуває в процесі циркуляції,
як показано вище, нових визначень; капітал перебігає
круг своїх перетворень; нарешті, він вступає з свого, так би
мовити, внутрішнього органічного життя в зовнішні життьові
відносини, у відносини, де один одному протистоять не капітал
і праця, а з одного боку капітал і капітал, з другого боку
індивіди знов таки просто як покупці і продавці; час циркуляції
і робочий час перехрещуються на своєму шляху, і таким
чином здається, ніби вони обидва в однаковій мірі визначають
додаткову вартість; та первісна форма, в якій протистоять один
одному капітал і наймана праця, замасковується в наслідок втручання
відносин, які, як здається, незалежні від неї; сама додаткова
вартість здається не продуктом привласнення робочого часу,
а надлишком продажної ціни товарів понад витрати їх виробництва,
в наслідок чого витрати виробництва легко можуть здаватися
дійсною вартістю (vaLeur intrinsèque) товарів, так що зиск
здається надлишком продажної ціни товарів понад їх імманентну
вартість.

Правда, під час безпосереднього процесу виробництва природа
додаткової вартості постійно доходить до свідомості капіталіста,
як це вже при розгляді додаткової вартості показала
нам його жадоба до чужого робочого часу і т. д. Але: 1) сам
безпосередній процес виробництва є тільки минущий момент,
який постійно переходить у процес циркуляції, як і цей останній
переходить у нього, так що ясніше чи туманніше проблискуюча
в процесі виробництва догадка про джерело здобутого у ньому
баришу, тобто про природу додаткової вартості, щонайбільше
виступає як момент рівноправний з тим уявленням, ніби реалізований
надлишок походить від руху, який не залежить від
процесу виробництва і виникає з самої циркуляції, отже, руху,
який належить капіталові незалежно від його відношення до
праці. Адже навіть сучасними економістами, як от Рамсей,
\parbreak{}  %% абзац продовжується на наступній сторінці
