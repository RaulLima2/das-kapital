як ці різні складові частини авансованого капіталу щодо обороту
набувають форм основного і обігового капіталу.

Отже, авансований капітал = 1680 фунтам стерлінгів: основний
капітал = 1200 фунтам стерлінгів плюс обіговий капітал =
480 фунтам стерлінгів (= 380 фунтам стерлінгів у матеріалах
виробництва плюс 100 фунтів стерлінгів у заробітній платі).

Витрати виробництва товару, навпаки, = тільки 500 фунтам
стерлінгів (20 фунтів стерлінгів на зношування основного капіталу,
480 фунтів стерлінгів на обіговий капітал).

Ця ріжниця між витратами виробництва товару і авансованим
капіталом підтверджує, однак, тільки те, що витрати
виробництва товару утворюються виключно капіталом, дійсно
витраченим на виробництво товару.

Для виробництва товару застосовуються засоби праці вартістю
в 1200 фунтів стерлінгів, але з цієї авансованої капітальної
вартості у виробництві втрачаються тільки 20 фунтів
стерлінгів. Тому застосований основний капітал лиш почасти
входить у витрати виробництва товару, бо він лиш почасти
витрачається на виробництво товару. Застосований обіговий
капітал цілком входить у витрати виробництва товару, бо він
цілком витрачається на виробництво товару. Але що ж це доводить,
як не те, що спожиті основні і обігові частини капіталу,
pro rata [пропорціонально] величині їх вартості, рівномірно
входять у витрати виробництва даного товару і що ця складова
частина вартості товару взагалі виникає тільки з капіталу,
витраченого на його виробництво? Коли б це було не так, то
не можна було б зрозуміти, чому авансований основний капітал
у 1200 фунтів стерлінгів не додає до вартості продукту крім
20 фунтів стерлінгів, які він утрачає в процесі виробництва,
також і тих 1180 фунтів стерлінгів, які він не втрачає в цьому
процесі.

Отже, ця ріжниця між основним і обіговим капіталом щодо
обчислення витрат виробництва тільки підтверджує позірне
виникнення витрат виробництва з витраченої капітальної вартості
або з тієї ціни, якої коштують самому капіталістові витрачені
елементи виробництва, включаючи й працю. З другого
боку, щодо утворення вартості, змінна, витрачена на робочу
силу частина капіталу прямо ототожнюється тут із сталим капіталом
(частиною капіталу, яка існує в матеріалах виробництва)
під рубрикою обігового капіталу, і таким чином вивершується
містифікація процесу зростання вартості капіталу. 1

Досі ми розглядали тільки один елемент товарної вартостівитрати
виробництва. Тепер ми мусимо поглянути і на другу
складову частину товарної вартості, на лишок понад витрати
виробництва, або на додаткову вартість. Отже, насамперед до-

1    Яка плутанина може постати через це в голові економіста, показано
в книзі І, розд. VII, 3, на прикладі Н. В. Сеніора.
