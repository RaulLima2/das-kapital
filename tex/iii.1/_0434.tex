\parcont{}  %% абзац починається на попередній сторінці
\index{iii1}{0434}  %% посилання на сторінку оригінального видання
відбувається не за допомогою простої кредитної операції без
будьякої участі грошей; що, отже, при великому попиті на грошові
позики може відбуватися величезна кількість цих операцій,
не збільшуючи при цьому циркуляції. Але самий той факт,
що циркуляція Англійського банку лишається незмінною або
навіть зменшується одночасно із значним збільшенням грошових
позик, які він видає, зовсім не доводить prima facie,
як це вважають Фуллартон, Тук та інші (в наслідок їх
помилкової думки, що грошова позика є те саме, що й одержання
capital on loan [позикового капіталу], додаткового капіталу), що
циркуляція грошей (банкнот) в їх функції як засобу платежу
не збільшується й не розширюється. Тому що циркуляція банкнот
як засобу купівлі в періоди застою у справах, коли потрібні такі
великі позики, скорочується, то їх циркуляція як засобу платежу
може збільшуватись, а загальна сума циркуляції, сума банкнот,
що функціонують як засоби купівлі й платежу, все ж може
лишатись незмінною або навіть зменшуватись. Циркуляція, як
засобу платежу, банкнот, які відразу ж припливають назад до
банку, що їх видав, в очах згаданих економістів зовсім не є
циркуляція.

Коли б циркуляція грошей як засобу платежу збільшилась
у вищій мірі, ніж вона зменшилась би як циркуляція засобів
купівлі, то вся циркуляція зросла б, хоч би маса грошей,
що функціонують як засіб купівлі, значно зменшилась. І це
дійсно настає в певні моменти кризи, а саме при цілковитому
краху кредиту, коли стає неможливим не тільки продаж товарів
і цінних паперів, але й дисконт векселів і коли вже ніщо не
дійсне, крім платежу готівкою, або, як каже купець: каса. Тому
що Фуллартон та інші не розуміють, що циркуляція банкнот як
засобу платежу є характерна для таких часів грошової скрути,
то вони розглядають це явище як випадкове. „With respect again
to those examples of eager competition for the possession of banknotes,
which characterise seasons of panic and which may sometimes,
as at the close of 1825, lead to a sudden, though only temporary,
enlargement of the issues, even while the efflux of bullion
is still going on, these, I apprehend, are not to be regarded as
among the natural or necessary concomitants of a low exchange; the
demand in such cases is not for circulation (слід було б сказати:
не на циркуляцію як на засоби купівлі) but for hoarding, a demand
on the part of alarmed bankers and capitalists which arises
generally in the last act of the crisis (отже, попит на резерв засобів
платежу) after a long continuation of the drain, and is the
precursor of its termination“ [„Щодо прикладів тієї завзятої конкуренції
за оволодіння банкнотами, яка характеризує часи
паніки і яка іноді, як от наприкінці 1825 року, може привести
до раптового, хоч би тільки тимчасового збільшення
емісії банкнот, навіть тоді, коли відплив золота все ще триває,
то, на мою думку, на них не можна дивитись як на природних
\parbreak{}  %% абзац продовжується на наступній сторінці
