\parcont{}  %% абзац починається на попередній сторінці
\index{iii1}{0357}  %% посилання на сторінку оригінального видання
тобто треба виходити з того припущення, що грошовий капіталіст
і продуктивний капіталіст дійсно протистоять один одному
не тільки як юридично відмінні особи, але й як особи, що грають
цілком різні ролі в процесі репродукції, або як особи, в руках
яких один і той самий капітал дійсно пророблює двоякий і цілком
різний рух. Один тільки віддає капітал у позику, другий
продуктивно застосовує його.

Для продуктивного капіталіста, який працює за допомогою взятого
у позику капіталу, гуртовий зиск розпадається на дві частини
— на процент, який він повинен заплатити позикодавцеві, і на
той надлишок понад процент, який становить його власну частину
в зиску. Якщо загальна норма зиску є дана, то ця остання частина
визначається розміром процента; якщо розмір процента є даний,
то вона визначається загальною нормою зиску. І далі: як би гуртовий
зиск, дійсна величина вартості всього зиску, не відхилявся
в кожному окремому випадку від пересічного зиску, та частина
його, яка належить функціонуючому капіталістові, визначається
процентом, бо цей останній фіксується загальним розміром процента
(залишаючи осторонь особливі юридичні договори) і припускається
як наперед дана величина, раніше ніж починаються
процес виробництва, отже, раніше ніж буде одержаний його результат,
гуртовий зиск. Ми бачили, що особливий специфічний продукт
капіталу є додаткова вартість, точніше кажучи, зиск. Але для
капіталіста, який працює за допомогою взятого в позику капіталу,
цей продукт — не зиск, а зиск мінус процент, частина зиску, яка
лишається йому після сплати процента. Отже, ця частина зиску
необхідно здається йому продуктом капіталу, оскільки цей капітал
функціонує; і для нього це дійсно так і є, бо він представник
капіталу тільки як функціонуючого капіталу. Він — персоніфікація
капіталу, оскільки капітал функціонує, а функціонує він
остільки, оскільки він вкладений у промисловість або торгівлю
таким чином, що дає зиск, і оскільки капіталіст, що застосовує
його, робить з ним ті операції, яких вимагає дана галузь підприємств.
Протилежно до процента, який він повинен сплачувати
з гуртового зиску позикодавцеві, решта зиску, яка
припадає йому, необхідно набирає таким чином форми промислового
чи торговельного зиску, або, охоплюючи одним висловом
і той і другий зиск, форми підприємницького доходу. Якщо
гуртовий зиск дорівнює пересічному зискові, то величина цього
підприємницького доходу визначається виключно розміром процента.
Якщо гуртовий зиск відхиляється від пересічного зиску,
то ріжниця між ним і пересічним зиском (після відрахування
процента з того й другого) визначається всілякими коньюнктурами,
які спричиняють тимчасове відхилення, чи то норми зиску
в якійсь окремій сфері виробництва від загальної норми зиску,
чи зиску, що його одержує окремий капіталіст у певній сфері, —
від пересічного зиску цієї окремої сфери. Алеж ми бачили, що
норма зиску в самому процесі виробництва залежить не тільки
\parbreak{}  %% абзац продовжується на наступній сторінці
