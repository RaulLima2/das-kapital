(Звичайно, здешевлення сталого капіталу в усіх цих галузях
може підвищити норму зиску при незмінній експлуатації робітника.)
Як тільки новий метод виробництва починає поширюватись,
і цим фактично дається доказ того, що ці товари можуть
вироблятись дешевше, то капіталісти, які працюють при старих
умовах виробництва, мусять продавати свій продукт нижче
його повної ціни виробництва, бо вартість цього товару впала,
робочий час, потрібний їм для виробництва цього товару, стоїть
вище суспільного. Одним словом, — і це виявляється як наслідок
конкуренції, — вони так само мусять запровадити новий метод
виробництва, при якому відношення змінного капіталу до
сталого є менше.

Всі обставини, які спричиняють те, що застосування машин
здешевлює ціну товарів, вироблюваних цими машинами,
завжди зводяться до зменшення тієї кількості праці, яку вбирає
одиниця товару; а подруге, вони зводяться до зменшення
зношуваної частини машин, вартість якої входить в одиницю
товару. Чим повільнішим є зношування машин, тим на більшу
кількість товарів воно розподіляється, тим більше живої праці
заміщають вони до строку їх репродукції. В обох випадках
збільшується кількість і вартість основного сталого капіталу
порівняно із змінним.

„All other things being equal, the power of a nation to save from
its profits varies with the rate of profits, is great when they are high,
less, when low; but as the rate of profit declines, all other things do
not remain equal... A low rate of profit is ordinarily acompanied by
a rapid rate of accumulation, relatively to the numbers of the people,
as in England... a high rate of profit by a slower rate of accumulation,
relatively to the numbers of the people“. [„При всіх інших
однакових умовах спроможність нації робити заощадження з своїх
зисків змінюється із зміною норми зиску; вона більша, коли
норма зиску — висока, менша, коли вона — низька; але якщо
норма зиску знижується, всі інші умови не лишаються однаковими...
Низька норма зиску звичайно супроводиться швидким
темпом нагромадження порівняно з чисельністю населення,
як в Англії... висока норма зиску — повільнішим темпом нагромадження
порівняно з чисельністю населення“.] Приклади: Польща,
Росія, Індія і т. д. (Richard Jones: „An Introductory Lecture on Political
Economy“, London 1833, стор. 50 і далі). Джонс правильно відзначає,
що, не зважаючи на падаючу норму зиску, inducements
and faculties to accumulate [спонуки до нагромадження і можливості
нагромаджувати] збільшуються. Поперше, в наслідок зростаючого
відносного перенаселення. Подруге, тому, що з зростанням
продуктивності праці збільшується маса споживних вартостей,
представлених тією самою міновою вартістю, отже, збільшується
маса речових елементів капіталу. Потрете, тому що
збільшується різноманітність галузей виробництва. Почетверте,
в наслідок розвитку кредитної системи, акційних товариств і т. д.
