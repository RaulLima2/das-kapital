В І за нововироблену вартість у 40 сплачується 20 v, в II
за нововироблену вартість у 48 сплачується 28 v, в III за нововироблену
вартість у 36 сплачується 16 v. Як нововироблена
вартість, так і заробітна плата змінились; але зміна нововиробленої
вартості означає зміну кількості витраченої праці, отже,
або числа робітників, або тривалості праці, або інтенсивності
праці, або кількох з цих трьох факторів.

с) Зміни m' і v відбуваються в тому самому напрямі; тоді
одна підсилює вплив другої.

90 с + 10 v + 10 m; m' = 100\%, p' = 10\%
80 с + 20 v + 30 m; m' = 150\%, p' = 30\%
92 с + 8 v + 6 m; m' = 75\%, p' = 6\%

І тут усі три нововироблені вартості різні, а саме 20, 50 і 14;
і ця ріжниця в величині витрачуваної в кожному випадку кількості
праці знову зводиться до ріжниці числа робітників, тривалості
праці, інтенсивності праці, або кількох, а то й усіх цих факторів.

III.* m', v і К змінюються

Цей випадок не дає нових точок зору і розв’язується за допомогою
загальної формули, даної в рубриці: II. m' змінюється.

Отже, вплив зміни величини норми додаткової вартості на
норму зиску дає такі випадки:

1. р' збільшується або зменшується в тій самій пропорції, як
i m', якщо v/K  лишається незмінним.

80 с + 20 v + 20 m; m' = 100\%, p' = 20\%
80 с + 20 v + 10 m; m' = 50\%, p' = 10\%
100\% : 50\% = 20\% : 10\%.

2. р' підвищується або падає в більшій пропорції, ніж m',
якщо v/K рухається в тому самому напрямі, що й m', тобто
збільшується чи зменшується, коли збільшується чи зменшується
m'.

80 с + 20 v + 10 m; m' = 50\%, p' = 10\%
70 с + 30 v + 20 m; m' = 66 2/3\%, p' = 20\%
50\% : 66 2/3\% < 10\% : 20\%. **

* В першому німецькому виданні: 3. Примітка ред. нім. вид. ІМЕЛ.

** Знак < означає тут, що збільшення з 50 до 66 2/3 є порівняно менше, ніж
збільшення з 10 до 20. Знак > у дальшій формулі означає зворотне. Примітка
ред. нім. вид. ІМЕЛ.
