цінами виробництва і в кінцевому рахунку визначають їх. Навпаки,
конкуренція показує: 1) пересічні зиски, які є незалежні від
органічного складу капіталу в різних сферах виробництва, отже,
і від маси живої праці, привласненої даним капіталом у даній сфері
експлуатації; 2) підвищення і падіння цін виробництва в наслідок
зміни висоти заробітної плати — явище, яке на перший
погляд цілком суперечить вартісному відношенню товарів;

3) коливання ринкових цін, які за даний період часу зводять
пересічну ринкову ціну товарів не до ринкової вартості, а до
ринкової ціни виробництва, яка відхиляється від цієї ринкової
вартості, дуже відмінна від неї. Всі ці явища, як здається, в такій
самій мірі суперечать визначенню вартості робочим часом,
як і природі додаткової вартості, яка складається з неоплаченої
додаткової праці. Отже, в конкуренції все з’являється у перекрученому
вигляді. Економічні відносини в готовому вигляді, як
вони виявляються на поверхні, в їх реальному існуванні, отже,
і в тих уявленнях, за допомогою яких носії та агенти цих
відносин намагаються їх собі з’ясувати, дуже відрізняються
від їх внутрішньої, істотної, але скритої суті (Kerngestalt) та відповідного
цій суті поняття і в дійсності перекручені та протилежні
цій суті та відповідному їй поняттю.

Далі. Коли капіталістичне виробництво досягає певного ступеня
розвитку, вирівнення різних норм зиску окремих сфер виробництва
в одну загальну норму зиску зовсім не відбувається
тільки через гру притягання і відштовхування, за допомогою
якої ринкові ціни притягають або відштовхують капітал. Після
того, як за певний період часу встановились пересічні ціни
і відповідні їм ринкові ціни, до свідомості окремих капіталістів
доходить, що в цьому процесі вирівнення вирівнюються певні
ріжниці, так що вони відразу включають їх у свої взаємні розрахунки.
В уявленні капіталістів ці ріжниці живуть і включаються
ними в обрахунки як підстави для компенсації.

Основне уявлення при цьому є сам пересічний зиск, —
уявлення, що рівновеликі капітали за однакові періоди часу мусять
давати рівновеликі зиски. В основі цього уявлення знов
таки лежить уявлення, що капітал кожної сфери виробництва
повинен pro rata [пропорціонально] своїй величині брати участь
в сукупній додатковій вартості, видушеній з робітників сукупним
суспільним капіталом; або що кожний окремий капітал треба
розглядати тільки як частину сукупного капіталу, а кожного
капіталіста в дійсності — як акціонера спільного підприємства,
який бере участь в сукупному зиску pro rata величині своєї частини
капіталу.

На цьому уявленні базується потім обрахунок капіталіста.
Так, наприклад, якщо капітал обертається повільніше — або тому,
що товар довше затримується в процесі виробництва, або тому,
що він мусить бути проданий на віддалених ринках, — то зиск,
який в наслідок цього вислизає з рук капіталіста, він все ж на-
