\parcont{}  %% абзац починається на попередній сторінці
\index{iii1}{0241}  %% посилання на сторінку оригінального видання
в 500, а в другому випадку капітал в 4000 дасть зиск в 1000.
m або p подвоїлось, але p' упало наполовину. І якщо раніш
з 50\% 20 припадало на зиск, 10 на процент і 20 на ренту, то p1/K =
20\%, z/K = 10\%, r/K = 20\%. Якщо при перетворенні в 25\%
відношення лишаються попередні, то p1/K = 10\%, z/K = 5\% і r/K =
10\%. Якщо ж, навпаки, p1 знизиться до 8\% і z/K до 4\%, то r/K
підвищиться до 13\%. Відносна величина r підвищилася б порівняно
з p1 і z, але все ж p' лишилося б попереднє. При обох припущеннях
сума p1, z і r підвищилася б, бо вона виробляється
за допомогою вчетверо більшого капіталу. А втім, припущення
Рікардо, що первісно промисловий зиск (плюс процент) містить
у собі всю додаткову вартість, історично й логічно невірне.
Навпаки, тільки прогрес капіталістичного виробництва приводить
до того, що 1) весь зиск спочатку потрапляє в руки промислових
і торговельних капіталістів для дальшого розподілу
і що 2) рента зводиться до надлишку понад зиск. Потім на цій
капіталістичній базі знову розвивається рента, яка є частина зиску
(тобто додаткової вартості, розглядуваної як продукт сукупного
капіталу), але не та специфічна частина продукту, яку забирає
собі капіталіст.

При припущенні наявності потрібних засобів виробництва,
тобто достатнього нагромадження капіталу, творення додаткової
вартості не має ніякої іншої межі, крім робітничого населення,
якщо норма додаткової вартості, отже й ступінь експлуатації
праці, є дана; і ніякої іншої межі, крім ступеня експлуатації
праці, якщо дане робітниче населення. І капіталістичний
процес виробництва по суті полягає у виробництві додаткової
вартості, представленої додатковим продуктом або відповідною
частиною вироблених товарів, в яких упредметнена неоплачена
праця. Ніколи не слід забувати, що виробництво цієї додаткової
вартості, — а зворотне перетворення деякої частини її в капітал,
або нагромадження, становить інтегральну частину цього виробництва
додаткової вартості, — є безпосередня мета і визначальний
мотив капіталістичного виробництва. Тому ніколи не можна зображати
його тим, чим воно не є, саме таким виробництвом, що має
своєю безпосередньою метою споживання або створення засобів
споживання для капіталістів. При цьому був би цілком
залишений осторонь його специфічний характер, який виражається
в усій його внутрішній суті (Kerngestalt).

Добування цієї додаткової вартості становить безпосередній
процес виробництва, який, як уже сказано, не має ніяких інших
меж, крім зазначених вище. Як тільки та кількість додаткової
\parbreak{}  %% абзац продовжується на наступній сторінці
