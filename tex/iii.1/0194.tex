і, з другого боку, тенденцію до знищення цих відхилень, тобто
до знищення впливу відношення між попитом і поданням. (Виняткові
товари, що мають ціни, не маючи вартості, тут не розглядаються.)
Попит і подання можуть в дуже різній формі
знищувати вплив, що його викликає їх нерівність. Наприклад,
якщо падає попит, а тому й ринкова ціна, то це може привести
до того, що капітал відтягатиметься і таким чином подання
зменшиться. Але це, може привести й до того, що сама ринкова
вартість завдяки винаходам, які скорочують необхідний робочий
час, знизиться і через це зрівняється з ринковою ціною. Навпаки:
якщо попит підвищується і, отже, ринкова ціна підвищується
понад ринкову вартість, то це може привести до того,
що до цієї галузі виробництва припливе занадто багато капіталу
і виробництво зросте настільки, що ринкова ціна впаде
навіть нижче ринкової вартості; або, з другого боку, це може
привести до такого підвищення цін, яке скоротить самий попит.
В деяких галузях виробництва це може привести також до
того, що на більш-менш довгий період часу підвищиться сама
ринкова вартість, бо протягом цього часу частина продуктів,
на які є попит, мусить вироблятися при гірших умовах.

Якщо попит і подання визначають ринкову ціну, то, з другого
боку, ринкова ціна і, при дальшому аналізі, ринкова вартість
визначає попит і подання. Щодо попиту це очевидно, бо
попит рухається в напрямі, протилежному до цін, підвищується,
коли ціни падають, і навпаки. Але те саме стосується й до
подання. Бо ціни засобів виробництва, що входять у товар, який
подається на ринок, визначають попит на ці засоби виробництва,
отже й подання тих товарів, подання яких включає в собі
попит на ці засоби виробництва. Ціни на бавовну мають визначальний
вплив на подання бавовняних матерій.

До цієї плутанини — визначення цін попитом і поданням і,
поруч з цим, визначення попиту й подання цінами — долучається
ще й те, що подання визначається попитом і, навпаки,
попит визначається поданням, ринок визначається виробництвом,
а виробництво — ринком.31

31 Великим тупоумством є оця „дотепність“: „Where the quantity of wages,
capital, and land, required to produce an article, have become different from what
they were, that which Adam Smith calls the natural price of it, is also different,
and that price which was previously its natural price, becomes, with reference to
this alteration, its market-price; because, though neither the supply, nor the quantity
wanted may have changed (і те і друге змінюється тут якраз тому, що
ринкова вартість або — про що йде мова в А. Сміта — ціна виробництва змінюється
в наслідок зміни вартості) that supply is not now exactly enough for
those persons who are able and willing to pay what is now the cost of production,
but is either greater or less than that; so that the proportion between the supply,
and what is, with reference to the new cost of production, the effectual demand,
is different from what is was. An alteration in the rate of supply will then take
place if there is no obstacle іn the way of it, and at last bring the commodity
to its new natural price. It may then seem good to some persons to say that, as
the commodity gets to its natural price by an alteration in its supply, the natural
