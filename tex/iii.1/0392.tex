роджено на світ векселів на 600 000 фунтів стерлінгів“ [стор. 61]. — „№ 971.
Тепер, якщо яканебудь фірма в Калькутті купує корабельний вантаж“ (для Англії)
„і оплачує його своїми власними траттами на свого лондонського кореспондента,
і накладні відсилаються сюди, то ці накладні відразу використовуються для одержання позик на Lombard
Street; отже, вона має вісім місяців, протягом яких
може користуватись цими грішми, раніше ніж її кореспонденти муситимуть
оплатити ці векселі“.

IV. В 1848 році засідала таємна комісія палати лордів для дослідження причин
кризи 1847 року. Свідчення перед цією комісією були, однак, опубліковані
лиш в 1857 році (Minutes of Evidence, taken before the Secret Committee of the
House of Lords appointed to inquire into the Causes of Distress etc.“ 1857; цитовано як: „Commercial
Distress“ 1848—1857). В цій комісії пан Lister, управитель
Union Bank of Liverpool, сказав між іншим таке:

„2444. Весною 1847 року кредит нечувано розширився... бо ділові люди
перенесли свій капітал з підприємств у залізниці і все ж хотіли й далі провадити свої підприємства в
попередніх розмірах. Кожен, мабуть, спочатку думав,
що зможе продати залізничні акції з зиском і таким чином вернути гроші в підприємство. Побачивши,
мабуть, що це неможливо, кожний почав брати для свого
підприємства в кредит там, де раніше платив готівкою. Звідси виникло розширення кредиту“.

„2500. Чи ці векселі, на яких банки, що їх прийняли, потерпіли збитки, — чи
були ці векселі видані головним чином під хліб, чи під бавовну?... Це були
векселі під продукти всякого роду, хліб, бавовну й цукор і іноземні продукти всякого роду. Тоді не
було майже жодного продукту, за винятком хіба
олії, який не впав би в ціні“. — „2506. Маклер, який акцептує вексель, не акцептує його без
достатнього покриття, включаючи й можливість падіння ціни того
товару, що служить покриттям“.

„2512. Під продукти видаються двоякого роду векселі. До першого роду належить первісний вексель,
який виставляється за кордоном на імпортера... Векселям,
які таким чином видаються під продукти, часто настає строк раніше, ніж прибувають продукти. Тому
купець, якщо прибуде товар і в нього немає достатнього
капіталу, мусить заставити його в маклера, поки не зможе продати його. Тоді
ліверпульським купцем негайно виставляється вексель другого роду на маклера,
під забезпечення цього товару... це вже тоді справа банкіра довідатись у маклера, чи є в нього товар
і скільки він дав під нього позики. Він мусить переконатися, що маклер має покриття, щоб підправити
свої справи в разі збитків“.

„2516. Ми одержуємо також векселі зза кордону... Хто-небудь купує за кордоном вексель на Англію і
надсилає його якійсь англійській фірмі; з цього векселя ми не можемо бачити, чи виданий він розумно
чи нерозумно, чи репрезентує він товар, чи вітер“.

„2533. Ви сказали, що закордонні продукти майже всякого роду були продані з великими збитками. Чи
думаєте ви, що це було наслідком неоправдуваної
спекуляції цими продуктами? — Збитки виникли від дуже великого довозу, тимчасом як не було
відповідного споживання для поглинення його. Як видно з
усього, споживання дуже впало“. — „2534. У жовтні... продуктів майже не можна
було продати“.

Як під час вищої точки краху лунає загальне sauve qui peut [рятуйся, хто
може], про це говорить в тому самому звіті першорядний знавець, вельмишановний бувалий квакер Samuel
Gurney з фірми Overend Gurney and С°: „1262. Коли
панує паніка, то ділова людина не запитує себе, по якій ціні вона може вмістити свої банкноти, або
чи втратить вона 1 чи 2% при продажу своїх державних або трипроцентних цінних паперів. Раз вона
перебуває під впливом страху, її вже не цікавить ні зиск, ні збиток; вона забезпечує саму себе, всі
інші
можуть робити, що хочуть“.

V. Про взаємне переповнення двох ринків пан Alexander, купець, що веде
торгівлю з Ост-Індією, показав перед комісією нижньої палати в справі банкового акту 1857 року
(цитується як „Bank Committee“ 1857) таке: „4330. В даний
момент, якщо я витрачаю в Манчестері 6 шилінгів, то в Індії одержую назад
5 шилінгів. Якщо я витрачаю в Індії 6 шилінгів, то в Лондоні одержую назад
