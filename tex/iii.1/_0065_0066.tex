\parcont{}  %% абзац починається на попередній сторінці
\index{iii1}{0065}  %% посилання на сторінку оригінального видання
те, чим вона в дійсності є: інший вимір додаткової вартості,
її вимір вартістю всього капіталу, а не вартістю тієї частини
капіталу, з якої через її обмін на працю вона безпосередньо
виникає. Але в дійсності (тобто у світі явищ) справа стоїть навпаки.
Додаткова вартість дана, але вона дана як надлишок продажної
ціни товару понад витрати його виробництва; при чому
лишається таємницею, звідки походить цей надлишок, — з експлуатації
праці в процесі виробництва, з ошукування покупців
у процесі циркуляції, чи з того і другого. Далі, дано відношення
цього надлишку до вартості всього капіталу, або
норма зиску. Обчислення цього надлишку продажної ціни понад
витрати виробництва відносно вартості всього авансованого капіталу
дуже важливе й природне, бо з допомогою цього обчислення
дійсно можна знайти те числове відношення, в якому
зростає вартість всього капіталу, або ступінь зростання його
вартості. Отже, якщо виходити з цієї норми зиску, то немає
ніякої можливості вивести звідси специфічне відношення між
надлишком і тією частиною капіталу, яка витрачена на заробітну
плату. В одному з пізніших розділів ми побачимо, які кумедні
промахи робить Мальтус, коли він цим шляхом намагається
проникнути в таємницю додаткової вартості і специфічного відношення
її до змінної частини капіталу. На що вказує норма
зиску як така, так це скоріше на однакове відношення надлишку
до рівновеликих частин капіталу, який з цієї точки зору взагалі
не виявляє ніяких внутрішніх ріжниць, крім ріжниці між основним
і обіговим капіталом. Та й цю ріжницю він виявляє тільки
тому, що надлишок обчислюється двояким способом. А саме,
поперше, як проста величина: надлишок понад витрати виробництва.
В цій першій формі надлишку весь обіговий капітал входить
у витрати виробництва, тимчасом як з основного капіталу
в них входить тільки зношування. Далі, по-друге: відношення
цього надлишку вартості до всієї вартості авансованого капіталу.
Тут в обчислення входить вартість всього основного капіталу
так само, як і вартість обігового капіталу. Отже, обіговий
капітал в обох випадках входить однаковим способом, тимчасом
як основний капітал в одному випадку входить іншим
способом, а в другому випадку таким самим способом, як обіговий
капітал. Таким чином ріжниця між обіговим і основним капіталом
нав’язується тут як єдина ріжниця.

Отже, надлишок, якщо він, висловлюючись за Гегелем, з норми
зиску відбивається назад в собі, або, інакше, надлишок, який
ближче характеризується нормою зиску, здається надлишком,
що його щорічно або за певний період циркуляції створює капітал
понад свою власну вартість.

Тому, хоч чисельно норма зиску відрізняється від норми додаткової
вартості, тимчасом як додаткова вартість і зиск в
дійсності є те саме і рівні також чисельно, проте зиск є перетворена
форма додаткової вартості, форма, в якій її походження
\index{iii1}{0066}  %% посилання на сторінку оригінального видання
і таємниця її існування завуальовані і стерті. Справді,
зиск є форма виявлення додаткової вартості, і ця остання тільки
за допомогою аналізу може бути вилущена з першої. В додатковій
вартості відношення між капіталом і працею оголене; у
відношенні капіталу й зиску, — тобто капіталу і додаткової вартості,
якою вона виступає, з одного боку, як реалізований у
процесі циркуляції надлишок понад витрати виробництва товару,
а з другого боку, як надлишок, ближче визначений його
відношенням до всього капіталу, — \emph{капітал} виступає \emph{як відношення
до себе самого}, як відношення, в якому він як первісна
сума вартості відрізняється від нової вартості, створеної ним
самим. Що він створює цю нову вартість під час свого
руху через процес виробництва і процес циркуляції, — це є в свідомості.
Але як це стається, це тепер містифіковано і, як
здається, походить від таємних властивостей, належних самому
капіталові.

Чим далі ми стежимо за процесом зростання вартості капіталу,
тим більше містифікується капіталістичне відношення і тим
менше розкривається таємниця його внутрішнього організму.

В цьому відділі норма зиску чисельно відрізняється від
норми додаткової вартості; навпаки, зиск і додаткова вартість
розглядаються як одна й та сама числова величина, тільки
в різній формі. В дальшому відділі ми побачимо, як відчужування
йде далі і як зиск і чисельно виражається як величина,
відмінна від додаткової вартості.

\section{Відношення норми зиску до норми
додаткової вартості}

Як відзначено наприкінці попереднього розділу, ми припускаємо
тут, як і взагалі в усьому цьому першому відділі, що
сума зиску, яка припадає на даний капітал, дорівнює всій сумі
додаткової вартості, виробленої за допомогою цього капіталу
протягом даного періоду циркуляції. Отже, покищо ми залишаємо
осторонь те, що, з одного боку, ця додаткова вартість
розпадається на різні підвиди (Unterformen): процент на капітал,
земельна рента, податки і т. д., і що вона, з другого боку,
в більшості випадків зовсім не збігається з зиском, як він привласнюється
в силу загальної пересічної норми зиску, про яку
буде мова в другому відділі.

Оскільки зиск припускається кількісно рівним додатковій
вартості, його величина і величина норми зиску визначається відношеннями
простих числових величин, які в кожному окремому
випадку дані або можуть бути визначені. Отже, дослідження
рухається спочатку в чисто математичній галузі.
