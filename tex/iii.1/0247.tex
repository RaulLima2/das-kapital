Періодичне знецінення наявного капіталу, що є імманентним
капіталістичному способові виробництва засобом затримувати
падіння норми зиску і прискорювати нагромадження капітальної
вартості шляхом утворення нового капіталу, порушує дані відносини,
в яких відбувається процес циркуляції і репродукції
капіталу, і тому воно супроводиться раптовими зупиненнями й
кризами процесу виробництва.

Відносне зменшення змінного капіталу порівняно з сталим,
яке йде рука в руку з розвитком продуктивних сил, дає стимул
зростанню робітничого населення і в той же час постійно створює
штучне перенаселення. Нагромадження капіталу, розглядуване
щодо вартості, уповільнюється, в наслідок падіння норми зиску, і
разом з тим ще більше прискорюється нагромадження споживних
вартостей, тимчасом як це останнє знов таки приводить до прискореного
ходу нагромадження, розглядуваного щодо вартості.

Капіталістичне виробництво постійно намагається перебороти
ці імманентні йому межі, але воно переборює їх тільки такими
засобами, які ставлять перед ним ці межі знову і в ще колосальнішому
масштабі.

Справжня межа капіталістичного виробництва є сам капітал,
є те, що капітал і самозростання його вартості являє собою
вихідний і кінцевий пункт, мотив і мету виробництва; що виробництво
є тільки виробництво для капіталу, а не навпаки:
засоби виробництва не є просто засобами для все ширшого й
ширшого розвитку життьового процесу суспільства виробників.
Межі, в яких тільки й може рухатись збереження і зростання
капітальної вартості, яке грунтується на експропріації і зубоженні
широких мас виробників, — ці межі постійно приходять
через це в суперечність з методами виробництва, які капітал
мусить застосовувати для своєї мети і які спрямовані до необмеженого
збільшення виробництва, до виробництва як самоцілі, до безумовного
розвитку суспільних продуктивних сил праці. Засіб —
безумовний розвиток суспільних продуктивних сил — вступає
в постійний конфлікт з обмеженою метою — збільшенням вартості
наявного капіталу. Тому, якщо капіталістичний спосіб виробництва
є історичний засіб для розвитку матеріальної продуктивної
сили і для створення відповідного їй світового ринку, то
він разом з тим становить постійну суперечність між цим його
історичним завданням і відповідними йому суспільними відносинами
виробництва.

III. Надмір капіталу при надмірі населення

З падінням норми зиску зростає той мінімум капіталу, який
потрібен окремому капіталістові для продуктивного вживання
праці, — потрібен як для експлуатації праці взагалі, так і для
того, щоб вживаний робочий час був робочим часом, необхідним
для виробництва товарів, щоб він не перевищував
