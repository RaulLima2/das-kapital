\parcont{}  %% абзац починається на попередній сторінці
\index{iii1}{0194}  %% посилання на сторінку оригінального видання
і, з другого боку, тенденцію до знищення цих відхилень, тобто
до знищення впливу відношення між попитом і поданням. (Виняткові
товари, що мають ціни, не маючи вартості, тут не розглядаються.)
Попит і подання можуть в дуже різній формі
знищувати вплив, що його викликає їх нерівність. Наприклад,
якщо падає попит, а тому й ринкова ціна, то це може привести
до того, що капітал відтягатиметься і таким чином подання
зменшиться. Але це, може привести й до того, що сама ринкова
вартість завдяки винаходам, які скорочують необхідний робочий
час, знизиться і через це зрівняється з ринковою ціною. Навпаки:
якщо попит підвищується і, отже, ринкова ціна підвищується
понад ринкову вартість, то це може привести до того,
що до цієї галузі виробництва припливе занадто багато капіталу
і виробництво зросте настільки, що ринкова ціна впаде
навіть нижче ринкової вартості; або, з другого боку, це може
привести до такого підвищення цін, яке скоротить самий попит.
В деяких галузях виробництва це може привести також до
того, що на більш-менш довгий період часу підвищиться сама
ринкова вартість, бо протягом цього часу частина продуктів,
на які є попит, мусить вироблятися при гірших умовах.

Якщо попит і подання визначають ринкову ціну, то, з другого
боку, ринкова ціна і, при дальшому аналізі, ринкова вартість
визначає попит і подання. Щодо попиту це очевидно, бо
попит рухається в напрямі, протилежному до цін, підвищується,
коли ціни падають, і навпаки. Але те саме стосується й до
подання. Бо ціни засобів виробництва, що входять у товар, який
подається на ринок, визначають попит на ці засоби виробництва,
отже й подання тих товарів, подання яких включає в собі
попит на ці засоби виробництва. Ціни на бавовну мають визначальний
вплив на подання бавовняних матерій.

До цієї плутанини — визначення цін попитом і поданням і,
поруч з цим, визначення попиту й подання цінами — долучається
ще й те, що подання визначається попитом і, навпаки,
попит визначається поданням, ринок визначається виробництвом,
а виробництво — ринком.\footnote{
Великим тупоумством є оця „дотепність“: „Where the quantity of wages,
capital, and land, required to produce an article, have become different from what
they were, that which Adam Smith calls the natural price of it, is also different,
and that price which was previously its natural price, becomes, with reference to
this alteration, its market-price; because, though neither the supply, nor the quantity
wanted may have changed (і те і друге змінюється тут якраз тому, що
ринкова вартість або — про що йде мова в А. Сміта — ціна виробництва змінюється
в наслідок зміни вартості) that supply is not now exactly enough for
those persons who are able and willing to pay what is now the cost of production,
but is either greater or less than that; so that the proportion between the supply,
and what is, with reference to the new cost of production, the effectual demand,
is different from what is was. An alteration in the rate of supply will then take
place if there is no obstacle іn the way of it, and at last bring the commodity
to its new natural price. It may then seem good to some persons to say that, as
the commodity gets to its natural price by an alteration in its supply, the natural
price is as much owing to one proportion between the demand and the supply, as
the market-price is to another; and consequently, that the natural price, just as
much as the market-price, depends on the proportion that demand and supply
bear to each other". („The great principle of demand and supply is called into
action to to determine wat A. Smith calls natural prices as well as market-prices". —
Malthus.) [„Якщо кількість заробітної плати, капіталу й землі, потрібна для
виготовлення якогось товару, змінюється порівняно з попередньою, то змінюється
й те, що Адам Сміт називає його природною ціною, і та ціна, яка
первісно була його природною ціною, стає у відношенні до цієї зміни його
ринковою ціною; бо, хоч ні подання, ні кількість товару, на яку є попит,
може, не змінилися“ (і те і друге змінюється тут якраз тому, що ринкова
вартість або — про що йде мова в А. Сміта — ціна виробництва змінюється
в наслідок зміни вартості), „проте, подання це тепер не цілком точно відповідає
попитові тих осіб, які спроможні і хочуть заплатити те, що становить тепер
витрати виробництва; воно або більше, або менше; так що відношення між
поданням і тим, що становить тепер при нових витратах виробництва дійсний
попит, відрізняється від попереднього. Отже, якщо не буде перешкод,
настане зміна в розмірі подання і це кінець-кінцем приведе товар до його
нової природної ціни. Дехто, може, вважатиме тоді можливим сказати, що —
через те що товар досягає своєї природної ціни в наслідок зміни розміру
його подання — природна ціна так само завдячує своє існування одному відношенню
між попитом і поданням, як ринкова ціна — другому; і що, отже,
природна ціна, цілком так само як і ринкова ціна, залежить від того відношення,
в якому стоять одне до одного попит і подання“. („Великий принцип
попиту й подання покликається до дії для того, щоб так само визначити
те, що А. Сміт називає природною ціною, як і те, що він називає ринковою
ціною". — Мальтус)]. („Observations on certain verbal disputes etc. ", Лондон 1821,
стор. 60, 61). Наш мудрець не розуміє, що в даному випадку саме зміна в cost
of production [витратах виробництва], отже і в вартості, викликала зміну в попиті,
отже й у відношенні між попитом і поданням, і що ця зміна в попиті
може викликати зміну в поданні; а це доводить якраз протилежне тому, що
хоче довести наш мислитель; а саме, це доводить, що зміна у витратах виробництва
аж ніяк не регулюється відношенням між попитом і поданням, а, навпаки,
сама регулює це відношення.}

\index{iii1}{0195}  %% посилання на сторінку оригінального видання
Навіть ординарний економіст (див. виноску) розуміє, що
й без викликаної зовнішніми обставинами зміни подання чи потреби
відношення між попитом і поданням може змінитися
в наслідок зміни у ринковій вартості товарів. Навіть він мусить
згодитись, що, яка б не була ринкова вартість, попит і подання
мусять урівноважитись, щоб вона реалізувалась. Це значить,
що не відношення між попитом і поданням пояснює ринкову
вартість, а, навпаки, ця остання пояснює коливання попиту
й подання. Автор „Observations“ після місця, цитованого у виносці,
каже далі: „This proportion (між попитом і поданням),
however if we still mean by „demand“ and „natural price“, what
we meant just now, when referring to Adam Smith, must always
be a proportion of equality; for it is only when the supply is equal
to effectual demand, that is, to that demand, which will pay neither
more nor less than the natural price, that the natural price is in
fact paid; consequently, there may be two very different natural
prices, at different times, for the same commodity, and yet the
proportion which the supply bears to the demand, be in both cases
the same, namely the proportion of equality“. [„Однак, це відношення
(між попитом і поданням), якщо ми й далі розумітимем
\parbreak{}  %% абзац продовжується на наступній сторінці
