а також тому, що більшість тих, хто витрачає дохід, робітники,
порівняно мало можуть купити в кредит; тимчасом як в оборотах торговельного світу, де засіб
циркуляції є грошовою
формою капіталу, гроші почасти в наслідок концентрації, почасти в наслідок переважання кредитної
системи функціонують
головним чином як засіб платежу. Але ріжниця між грішми як засобом платежу і грішми як засобом
купівлі (засобом циркуляції)
є ріжниця, належна самим грошам, а не ріжниця між грішми
і капіталом. З тієї причини, що в роздрібній торгівлі циркулює
більше міді й срібла, а в гуртовій — більше золота, ріжниця
між сріблом і міддю, з одного боку, і золотом, з другого боку,
не є ріжниця між засобом циркуляції і капіталом.

До пункту 2) про приплутання питання про кількість грошей, що циркулюють в обох функціях разом:
оскільки гроші
циркулюють як засіб купівлі чи як засіб платежу, — однаково,
в якій з обох сфер і незалежно від їх функції реалізації доходу
або капіталу, — закони, розвинені раніше при дослідженні простої товарної циркуляції (книга І, розд.
III, 2, b), зберігають свою
силу для кількості циркулюючих грошей. Ступінь швидкості
циркуляції, тобто число повторень тією самою монетою за даний
період часу тієї самої функції засобу купівлі й засобу платежу,
кількість одночасних купівель і продажів або платежів, сума
цін товарів, що циркулюють, нарешті, платіжні баланси, які
треба одночасно покрити, в обох випадках визначають масу циркулюючих грошей, currency. Чи
представляють гроші, що функціонують таким чином, для платіжника або одержувача капітал
чи дохід, це байдуже, це абсолютно нічого не міняє в стані
справи. Маса грошей визначається просто їх функцією як засобу купівлі й засобу платежу.

До пункту 3), до питання про відносні пропорції кількостей
засобів циркуляції, що циркулюють в обох функціях і тому
в обох сферах процесу репродукції. Обидві сфери циркуляції
стоять у внутрішньому зв’язку між собою, оскільки, з одного
боку, кількість доходів, що мають витрачатись, виражає розмір
споживання, а з другого боку, величина мас капіталу, що циркулюють у виробництві й торгівлі, виражає
розмір і швидкість
процесу репродукції. Не зважаючи на це, ці обставини впливають різно, і навіть у протилежному
напрямі, на кількості грошових мас, що циркулюють в обох функціях чи сферах, або,
як кажуть англійці банковою мовою, на кількості циркуляції.
І це знову дає Тукові привід до безглуздого розрізнення циркуляції і капіталу. Та обставина, що
панове прихильники теорії
currency переплутують дві різні речі, зовсім не є підставою
для того, щоб зображати їх як різні поняття.

В періоди процвітання, великого розширення, прискорення
і енергії процесу репродукції, робітники зайняті повністю. Здебільшого настає також підвищення
заробітної плати і дещо вирівнює її падіння нижче пересічного рівня в інші періоди ко-
