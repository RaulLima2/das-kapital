талістові довелося б зробити окреме обчислення, приклад якого
ми хочемо тут навести.

Для цього ми візьмемо бавовнопрядільну фабрику на 10 000
мюльних веретен, описану в книзі І, стор. 227 *, і припустимо
при цьому, що дані, взяті для одного тижня квітня 1871 року,
зберігають своє значення для цілого року. Основний капітал, вміщений
у машинах, становив 10 000 фунтів стерлінгів. Обіговий
капітал не був указаний; припустімо, що він становив 2 500 фунтів
стерлінгів, — досить висока сума, що виправдується, однак, тим
припущенням, яке ми тут весь час мусимо робити, а саме, що
не відбувається ніяких кредитних операцій, отже, що немає
тривалого чи тимчасового користування чужим капіталом. Тижневий
продукт щодо своєї вартості складався з 20 фунтів стерлінгів
на зношування машин, 358 фунтів стерлінгів авансованого
обігового сталого капіталу (плата за найом — 6 фунтів стерлінгів,
бавовна — 342 фунти стерлінгів, вугілля, газ, мастило — 10 фунтів
стерлінгів), 52 фунтів стерлінгів витраченого на заробітну
плату змінного капіталу і 80 фунтів стерлінгів додаткової вартості,
отже:

20 с (зношування) + 358 с + 52 v + 80 m = 510.

Отже, щотижневе авансування обігового капіталу становило
358 с + 52 v = 410, і його процентний склад = 87,3 с + 12,7 v. При
обчисленні на весь обіговий капітал у 2500 фунтів стерлінгів
це дає 2182 фунти стерлінгів сталого і 318 фунтів стерлінгів
змінного капіталу. Через те, що вся витрата на заробітну плату
становила на рік 52 рази по 52 фунти стерлінгів, отже, 2704 фунти
стерлінгів, то виходить, що змінний капітал у 318 фунтів стерлінгів
обернувся за рік майже точно 8 1/2 разів. Норма додаткової
вартості була  80/52—153 11/13\%. За цими елементами ми обчисляємо
норму зиску, підставивши в формулу р' = m'n (v/K) значення:
m' = 153 11/13\%, n = 8 1/2, v = 318, K = 12500; отже:

р' = 153 11/13 × 8 1/2 × 318/12 500 = 33,27\%

Для перевірки цього ми скористуємось простою формулою
р' = m/K. Вся додаткова вартість, або зиск, становить за рік
52 фунти стерлінгів × 80 = 4160 фунтів стерлінгів; поділене на
весь капітал у 12 500 фунтів стерлінгів, це дає майже стільки ж, як
вище, 33,28\%, ненормально високу норму зиску, яка пояснюється
тільки надзвичайно сприятливими умовами даного моменту (дуже
дешеві ціни на бавовну поряд з дуже високими цінами на пряжу)
і в дійсності існувала, без сумніву, не на протязі всього року.

* Стор. 152 рос. вид. 1935 р. Ред. укр. перекладу.
