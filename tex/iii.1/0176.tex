індивідуальних ріжниць, які й без того зникають, бо в обох
випадках ми маємо перед собою тільки пересічний склад всієї
сфери виробництва. Одиничний капіталіст (або й сукупність капіталістів у кожній окремій сфері
виробництва), погляд якого
обмежений, справедливо гадає, що його зиск походить не тільки
з праці, вживаної ним або в його галузі промисловості. Це цілком правильно, оскільки справа йде про
його пересічний зиск.
В якій мірі цей зиск є результат загальної експлуатації праці
сукупним капіталом, тобто всіма його товаришами-капіталістами, — це є для нього цілковитою
містерією, тим більше, що самі теоретики буржуазії, політико-економи, досі ще не розкрили цієї
містерії. Заощадження на праці — не тільки на праці, необхідній
для вироблення певного продукту, але й на числі зайнятих робітників — і збільшене вживання мертвої
праці (сталого капіталу)
з економічного погляду виступає як цілком правильна операція і здається, що це ніяк не впливає на
загальну норму
зиску і пересічний зиск. Але ж яким чином жива праця може
бути виключним джерелом зиску, якщо зменшення кількості
праці, потрібної для виробництва, здається, не тільки не впливає на зиск, а, навпаки, при певних
обставинах є найближчим
джерелом збільшення зиску, принаймні для окремого капіталіста?

Якщо в даній сфері виробництва підвищується або скорочується та частина витрат виробництва, яка
репрезентує вартість
сталого капіталу, то ця частина виходить з циркуляції і входить
у процес виробництва товару вже з самого початку збільшеною або зменшеною. Якщо, з другого боку,
дане число зайнятих робітників за той самий час виробляє більше чи менше продукту, отже, якщо при
незмінному числі робітників змінюється кількість праці, потрібна для виробництва певної кількості
товарів, то та частина витрат виробництва, яка репрезентує вартість змінного капіталу, може лишитись
незмінною,
тобто в тих самих розмірах увійти в витрати виробництва сукупного продукту. Але на кожний окремий
товар з тих товарів, сума яких становить сукупний продукт, припадає більше чи менше праці
(оплаченої, а тому й неоплаченої), отже й більше
чи менше витрат на цю працю, більша чи менша кількість заробітної плати. Вся виплачена капіталістом
заробітна плата лишається та сама, але вона змінюється, якщо обчислювати її на кожну штуку товару.
Отже, тут відбувається зміна в цій частині
витрат виробництва товару. Чи підвищуються чи зменшуються витрати виробництва окремого товару в
наслідок таких змін вартості
або його самого, або товарів, які є його складовими елементами
(або ж підвищуються чи зменшуються витрати виробництва суми
товарів, вироблених капіталом даної величини), — в усякому
разі пересічний зиск, якщо він був, наприклад, 10\%, так і лишається 10\%; хоч 10\%, розглядувані щодо
окремого товару,
являють собою дуже різну величину, залежно від тієї зміни
