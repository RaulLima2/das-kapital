тягів з парламентських звітів про всякі можливі речі, зачеплені
в цьому відділі, перемішана з довшими чи коротшими увагами
автора. Наприкінці витяги та коментарії все більше й більше
концентруються коло руху грошових металів та вексельного
курсу і знов закінчуються всякого роду додатками. Навпаки, розділ
(36) „Передкапіталістичні відносини“ був цілком оброблений.

З усього цього матеріалу, починаючи з „Плутанини“, і оскільки
його не було вже вміщено в попередніх місцях, я склав
розділи 33—35. Звичайно, тут не обійшлося без значних вставок
з мого боку для встановлення зв’язку. Оскільки ці вставки
не чисто формального характеру, вони прямо позначені як мої.
Таким способом мені, нарешті, вдалося умістити в тексті всі так
чи інакше належні до справи судження автора; нічого не випущено,
крім незначної частини витягів, які або тільки повторювали
наведене вже в іншому місці, абож торкалися пунктів, яких
рукопис докладно не розглядав.

Відділ про земельну ренту був далеко повніше оброблений,
хоч і зовсім не впорядкований, як це видно вже з того, що
в 43 розділі (в рукопису кінець відділу про ренту) Маркс
вважав за потрібне коротко повторити план всього відділу. І це
було тим більш бажаним для видання, що рукопис починається
розділом 37, після якого йдуть розділи 45—47, і тільки після
цього розділи 38—44. Найбільше праці потребували таблиці
при диференціальній ренті II і те відкриття, що в 43 розділі
зовсім не був досліджений третій випадок цього роду ренти,
який треба було тут розглянути.

Для цього відділу про земельну ренту Маркс у семидесятих
роках взявся до цілком нових спеціальних досліджень. Протягом
ряду років він вивчав в оригіналах статистичні досліди
та інші видання про землеволодіння, які стали неминучими
в Росії після „реформи“ 1861 року і які йому постачали в бажаній
повноті його російські друзі, робив з них виписки і намірявся
їх використати при новому обробленні цього відділу. При
різноманітності форм як землеволодіння, так і експлуатації
землеробських виробників у Росії, у відділі про земельну ренту
Росія мала відігравати таку саму роль, як в першій книзі, при
розгляді промислової найманої праці, Англія. На жаль, Марксу
не вдалося здійснити цей план.

Нарешті, сьомий відділ був цілком закінчений у рукопису,
але тільки як перший нарис, безконечно заплутані періоди якого
спочатку треба було розчленувати, щоб зробити їх придатними
до друку. Від останнього розділу існує тільки початок. Тут
малося розглянути відповідні трьом головним формам доходу:
земельна рента, зиск, заробітна плата — три великі класи розвиненого
капіталістичного суспільства: землевласники, капіталісти,
наймані робітники — і неминуче дану з їх існуванням класову
боротьбу як фактично наявний результат капіталістичного
періоду. Подібні кінцеві резюме Маркс мав звичай відкладати
