екскрементів виробництва, так званих відпадів його, в нові елементи
виробництва чи тієї самої, чи якоїсь іншої, галузі промисловості,
процеси, за допомогою яких ці так звані екскременти
кидаються знову в кругобіг виробництва, а тому й споживання —
продуктивного чи особистого. І ця галузь заощаджень, на якій
ми пізніше спинимось трохи ближче, є результат суспільної
праці у великому масштабі. Саме відповідна цій останній масовість
цих відпадів робить з них самих знову предмет торгівлі,
отже й нові елементи виробництва. Тільки як відпади колективного
виробництва і, отже, виробництва у великому масштабі, набирають
вони цього значення для процесу виробництва, лишаються
носіями мінової вартості. Ці відпади — незалежно від тієї служби,
яку вони виконують як нові елементи виробництва, — зменшують
у тій мірі, в якій вони знову можуть бути продані, витрати на
сировинний матеріал, до яких завжди зараховуються нормальні
відпади цього матеріалу, а саме, та кількість їх, яка пересічно
мусить бути втрачена при його обробленні. Зменшення витрат
на цю частину сталого капіталу підвищує pro tanto [відповідно
до цього] норму зиску при даній величині змінного капіталу
і даній нормі додаткової вартості.

Якщо додаткова вартість дана, норма зиску може бути
збільшена тільки шляхом зменшення вартості сталого капіталу,
потрібного для виробництва товару. Оскільки сталий капітал
входить у виробництво товарів, остільки значення має не його
мінова вартість, а виключно його споживна вартість. Скільки
праці може ввібрати в себе льон на якійсь прядільні, залежить
не від його вартості, а від його кількості, якщо дано рівень
продуктивності праці, тобто рівень технічного розвитку. Так
само та допомога, яку машина дає, наприклад, трьом робітникам,
залежить не від її вартості, а від її споживної вартості як машини.
На одному ступені технічного розвитку погана машина може бути
дорогою, на другому — добра машина може бути дешевою.

Підвищений зиск, який капіталіст одержує в наслідок того,
що, наприклад, подешевшали бавовна й прядільні машини, є результат
підвищеної продуктивності праці, правда, не в прядільному
виробництві, а у виробництві машин і бавовни. Для того,
щоб упредметнити дану кількість праці, отже, привласнити дану
кількість додаткової праці, тепер потрібно менше видатків на
умови праці. Зменшуються витрати, потрібні для того, щоб привласнити
певну кількість додаткової праці.

Ми вже казали про те заощадження, яке постає в процесі
виробництва в наслідок спільного застосування засобів виробництва
колективним робітником — суспільно-комбінованим робітником.
Дальші заощадження на видатках сталого капіталу, що
виникають із скорочення часу циркуляції (де розвиток засобів
сполучення є істотний матеріальний момент), ми розглянемо
нижче. Але вже тут треба ще згадати про ту економію, яка
походить з безперервного поліпшення машин, а саме: 1) з поліп-
