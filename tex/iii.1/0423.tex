дитові зворотний приплив у грошовій формі стає незалежним
від часу дійсного зворотного припливу як для промислового капіталіста, так і для купця. Кожний з них
продає в кредит; отже,
товар їх відчужується раніше, ніж він зворотно перетвориться
для них у гроші, тобто повернеться до них самих назад у грошовій формі. З другого боку, кожен з них
купує в кредит, і таким
чином вартість їх товару зворотно перетворюється для них
в продуктивний капітал чи в товарний капітал раніше, ніж ця вартість дійсно перетвориться в гроші,
раніше ніж настане строк виплати ціни товару і вона буде виплачена. В такі періоди процвітання
зворотний приплив відбувається легко й гладко. Роздрібний
торговець з цілковитою певністю платить гуртовому торговцеві,
цей останній — фабрикантові, фабрикант — імпортерові сировинного матеріалу і т. д. Видимість швидких
і певних зворотних припливів капіталу завжди тримається ще довгий час після того, як у
дійсності цього вже немає — тримається завдяки налагодженому
кредитові, бо зворотні припливи в формі кредиту заступають
дійсні зворотні припливи. Банки починають передчувати біду,
як тільки їх клієнти починають платити більше векселями, ніж
грішми. Дивись вищенаведене свідчення ліверпульського директора банку, стор. 392 і далі.

Тут треба ще вставити те, про що я згадував раніше: „В періоди,
коли кредит процвітає, швидкість обігу грошей зростає скоріше,
ніж зростають ціни товарів; тимчасом як при скороченні кредиту ціни товарів падають повільніше, ніж
швидкість циркуляції“
(„Zur Kritik der politischen Oekonomie“, 1859, стор. 83, 84 [„До
критики політичної економії“, укр. вид. 1935 р., стор. 133]).

В періоди криз справа стоїть навпаки. Циркуляція № І скорочується, ціни падають, так само й
заробітні плати; число зайнятих робітників скорочується, маса оборотів зменшується. Навпаки, в
циркуляції № II з скороченням кредиту зростає потреба
в грошових позиках — обставина, яку ми зараз розглянемо докладніше.

Не підлягає ніякому сумніву, що при скороченні кредиту,
яке збігається з застоєм у процесі репродукції, маса засобів
циркуляції, потрібна для № І, витрачання доходів, зменшується,
тимчасом як маса засобів циркуляції, потрібна для № II, передачі капіталів, збільшується. Але треба
дослідити, наскільки
це положення тотожне з положенням, виставленим Фуллартоном
та іншими: „Попит на позиковий капітал і попит на додаткові
засоби циркуляції — зовсім різні речі і не часто зустрічаються
разом“.\footnote{
„А demand for capital on loan and a demand for additional circulation are
quite distinct things, and not often found associated“ (Fullarton: „On the Regulation of
Currencies“, 2 вид., Лондон 1845, стор. 82, заголовок до розд. 5). — „It is
a great error, indeed, to imagine that the demand for pecuniary accommodation (i. e.
for the loan of capital) is identical with a demand for additional means of circulation,
or even that the two are frequently associated. Each demand originates in circumstances peculiarly
affecting itself, and very distinct from each other. It is when
}