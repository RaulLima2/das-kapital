частини упредметненої в ньому праці, яку він оплатив. Додаткова
праця, що міститься в товарі, нічого не коштує капіталістові,
хоч робітникові вона цілком так само коштує праці, як
і оплачена, і хоч вона цілком так само, як і оплачена, створює
вартість і входить у товар як вартостетворчий елемент. Зиск
капіталіста постає з того, що він має для продажу щось, чого
він не оплатив. Додаткова вартість, відповідно зиск, складається
саме з надлишку товарної вартості понад витрати її виробництва,
тобто з надлишку всієї суми праці, вміщеної в товарі, понад
вміщену в ньому оплачену суму праці. Таким чином додаткова
вартість, звідки б вона не виникала, є надлишок понад увесь
авансований капітал. Отже, цей надлишок стоїть у такому відношенні
до всього капіталу, яке виражається дробом m: К, де
К означає весь капітал. Таким чином одержуємо норму зиску
m: К = m: (с + v), у відміну від норми додаткової вартості m: v.

Величина додаткової вартості у її відношенні до змінного
капіталу зветься нормою додаткової вартості; величина додаткової
вартості у її відношенні до всього капіталу зветься нормою зиску.
Це два різні виміри тієї самої величини, які в наслідок ріжниці в
масштабах виражають одночасно різні пропорції або відношення
однієї і тої самої величини.

З перетворення норми додаткової вартості в норму зиску
слід виводити перетворення додаткової вартості в зиск, а не
навпаки. І справді, вихідним пунктом історично була норма зиску.
Додаткова вартість і норма додаткової вартості є, відносно, те
невидиме і суттьове, що треба розкрити, тимчасом як норма
зиску, а тому й така форма додаткової вартості як зиск виявляються
на поверхні явищ.

Щодо окремого капіталіста, то ясно, що єдине, що його
інтересує, це відношення додаткової вартості або надлишку вартості,
ради якого він продає свої товари, до всього капіталу,
авансованого на виробництво товару; тимчасом як певне відношення
цього надлишку до окремих складових частин капіталу
і його внутрішній зв’язок з цими складовими частинами не тільки
не інтересує його, але він ще й заінтересований в тому, щоб
оповити туманом це певне відношення і цей внутрішній зв’язок.

Хоча надлишок вартості товару понад витрати його виробництва
виникає в безпосередньому процесі виробництва, але реалізується
він тільки в процесі циркуляції, — і він тим легше набуває видимості
виникнення з процесу циркуляції, що в дійсності, серед
конкуренції, на дійсному ринку, від ринкових відносин залежить,
чи реалізується цей надлишок, чи ні, і в якому розмірі. Тут
немає потреби пояснювати, що коли товар продається вище
або нижче його вартості, то має місце. тільки інший розподіл
додаткової вартості, і що цей інший розподіл, змінене
відношення, в якому різні особи ділять між собою додаткову вар-
