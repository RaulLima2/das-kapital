\parcont{}  %% абзац починається на попередній сторінці
\index{iii1}{0429}  %% посилання на сторінку оригінального видання
переповнених ринків, припинення закордонного попиту на наші
продукти, уповільнення зворотних припливів і, як неминучий
наслідок усього цього, комерційного недовір’я, закриття мануфактур,
голодування робітників і загального застою промисловості
та підприємств“] (стор. 129). Разом з тим це є, звичайно,
найкраще спростування того твердження представників теорії
currency, ніби a full circulation drives out bullion and a low circulation
attracts it [переповнена циркуляція викликає відплив благородного
металу, а низька циркуляція притягає його]. Навпаки,
хоч в Англійському банку в періоди процвітання здебільшого є
значний золотий запас, але утворюється цей запас завжди в часи
тиші й застою, які настають услід за бурею.

Отже, вся мудрість щодо відпливу золота зводиться до того,
що попит на \emph{міжнародні} засоби циркуляції і платежу є відмінний
від попиту на \emph{внутрішні} засоби циркуляції і платежу (з чого
само собою випливає і те, що „the existence of a drain does not
necessarily imply any diminution of the internal demand for circulation“
[„існування відпливу не передбачає неодмінно зменшення
внутрішнього попиту на засоби циркуляції“], як каже Фуллартон,
стор. 112), і що відправка благородних металів з країни,
викидання їх у міжнародну циркуляцію, не тотожне з викиданням
банкнот або монет у внутрішню циркуляцію. Зрештою,
я вже раніше показав, що рух скарбу, сконцентрованого як запасний
фонд для міжнародних платежів, сам по собі не має
нічого спільного з рухом грошей як засобу циркуляції. Звичайно,
при цьому виникає ускладнення в наслідок того, що різні функції
скарбу, які я вивів з природи грошей: його функція як резервного
фонду для засобів платежу, для платежів всередині
країни, яким надходить строк; як резервного фонду засобів циркуляції;
нарешті, як резервного фонду світових грошей — всі ці
функції покладаються на один-єдиний резервний фонд; з чого випливає
також, що при певних обставинах відплив золота з банку
всередину країни може комбінуватися з його відпливом за кордон.
Далі виникає ще ускладнення в наслідок того, що на цей скарб
цілком самовільно покладається ще одна функція — служити гарантійним
фондом розмінності банкнот у країнах з розвиненою
кредитного системою і кредитними грішми. Потім до всього
цього, кінець-кінцем, долучається: 1) концентрація національного
резервного фонду в єдиному головному банку, 2) його зведення
до якнайменшого мінімуму. Звідси й нарікання Фуллартона
(стор. 143): „One cannot contemplate the perfect silence and facility
with which variations of the exchange usually pass off in continental
countries, compared with the state of feverish disquiet and
alarm always produced in England whenever the treasure in the
bank seems to be at all approaching to exhaustion, without being
struck with the great advantage in this respect which a metallic currency
possesses“ [„He можна споглядати того цілковитого спокою
і легкості, з якими звичайно відбуваються зміни вексельного
\parbreak{}  %% абзац продовжується на наступній сторінці
