ють Фуллартон та інші. Це є a question of money в тій формі, в якій
гроші є міжнародний засіб платежу. „Whether that capital (купівельна
ціна мільйонів квартерів закордонної пшениці після
неврожаю в країні) is transmitted in merchandize or in specie, is
a point which in no way affects the nature of the transaction“ [„Чи
передається цей капітал“ (купівельна ціна мільйонів квартерів
закордонної пшениці після неврожаю в країні) „товарами, чи
грішми готівкою, — ця обставина ні трохи не впливає на характер
операції“] (Фуллартон, там же, стор. 131). Але це має велику
вагу для питання, чи відбувається відплив золота, чи ні. Капітал
передається в формі благородного металу, тому що в формі
товарів він зовсім не може бути переданий або може бути
переданий тільки з дуже великими втратами. Страх сучасної
банкової системи перед відпливом золота перевищує все, що
будьколи марилося монетарній системі, для якої благородний
метал є єдине справжнє багатство. Візьмімо, наприклад, таке
свідчення управителя Англійського банку, Морріса, перед парламентським
комітетом про кризу 1847—1848 рр.: „3846. (Запитання:)
Коли я кажу про знецінювання запасів (stocks) і основного
капіталу, то чи не відомо вам, що весь капітал, вкладений у запаси
і продукти всякого роду, так само знецінився; що бавовна-сирець,
шовк-сирець, сировинна вовна відправлялися на континент
по таких самих бросових цінах і що цукор, кофе й чай продавалися
з великими жертвами, як при продажах з молотка? — Країна
неминуче повинна була принести значні жертви для того, щоб
протидіяти відпливові золота, який відбувся в наслідок величезного
довозу харчових продуктів“. — „3848. Чи не думаєте
ви, що було б краще зачепити ті 8 мільйонів фунтів стерлінгів, які
лежали в сховищах банку, ніж намагатися одержати назад золото
з такими жертвами? — Ні, я цього не думаю“. — Золото вважається
тут за єдине справжнє багатство.

Цитоване Фуллартоном відкриття Тука, що „with only one
or two exceptions, and those admitting of satisfactory explanation,
every remarkable fall of the exchange, followed by a drain of gold,
that has occurred during the last half century, has been coincident
throughout with a comparatively low state of the circulating medium,
and vice versa“ [„за одним або двома винятками, яким
можна дати достатнє пояснення, всяке помітне падіння вексельного
курсу, супроводжуване відпливом золота, яке відбувалось
за останні півстоліття, завжди збігалося з порівняно низьким
рівнем засобів циркуляції, і навпаки“] (Fullarton, стор. 121), —
доводить, що ці відпливи золота настають здебільшого після періоду
пожвавлення і спекуляції, як „а signal of a collapse already
commenced... an indication of overstocked markets, of a cessation
of the foreign demand for our productions, of delayed returns, and,
as the necessary sequel of all these, of commercial discredit, manufactories
shut up, artisans starving, and a general stagnation of
industry and enterprise“ [„сигнал краху, що вже почався... ознака
