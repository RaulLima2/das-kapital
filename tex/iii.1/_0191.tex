\parcont{}  %% абзац починається на попередній сторінці
\index{iii1}{0191}  %% посилання на сторінку оригінального видання
а існує тільки випадковий зв’язок між сукупною кількістю суспільної
праці, вжитою на певний суспільний продукт, тобто між
тією відповідною частиною сукупної робочої сили суспільства,
яку воно вживає на виробництво цього продукту, отже, між
розміром, який займає виробництво цього продукту в сукупному
виробництві, з одного боку, і, з другого боку, тим розміром, в якому
суспільство вимагає покриття тієї потреби, яку задовольняє цей
певний продукт. Хоч кожний окремий продукт або кожна дана кількість
певного сорту товару може містити в собі тільки суспільну
працю, потрібну для його виробництва, і розглядувана з цього
боку ринкова вартість усієї кількості товарів цього сорту представляє
тільки необхідну працю, — все ж, якщо певний товар
вироблено в такій мірі, яка перевищує суспільну потребу, то
частина суспільного робочого часу витрачена марно і вироблена
маса товарів репрезентує тоді на ринку значно меншу
кількість суспільної праці, ніж у ній дійсно міститься. (Тільки
там, де виробництво відбувається за заздалегідь визначеним
планом і під дійсним контролем суспільства, суспільство створює
зв’язок між кількістю суспільного робочого часу, вживаного на виробництво
певних продуктів, і розмірами суспільної потреби, яку
належить задовольнити за допомогою цих продуктів.) Тому ці товари
мусять бути продані нижче їх ринкової вартості, а частина їх
може зовсім залишитись непроданою. — Протилежне матимем тоді,
коли кількість суспільної праці, вжитої на виробництво певного
сорту товарів, занадто мала порівняно з розмірами певної суспільної
потреби, яку належить задовольнити за допомогою цього продукту.
— Якщо ж кількість суспільної праці, вжитої на виробництво
певного продукту, відповідає розмірам суспільної потреби, яку
належить задовольнити, так що вироблена маса продукту відповідає
звичайному масштабові репродукції при незмінному
попиті, то товар продається по його ринковій вартості. Обмін
або продаж товарів по їх вартості є раціональна основа, природний
закон їх рівноваги; відхилення слід поясняти, виходячи
з цього закону, а не навпаки — з відхилень поясняти самий закон.

Розгляньмо тепер другий бік справи — попит.

Товари купуються як засоби виробництва або як засоби
існування, — при чому справа ніяк не змінюється від того, що
деякі сорти товарів можуть служити обом цілям, — для того,
щоб увійти в продуктивне або особисте споживання. Отже,
попит на них пред’являється з боку виробників (в даному випадку
капіталістів, бо припускається, що засоби виробництва
перетворені в капітал) і з боку споживачів. І те і друге насамперед,
очевидно, передбачає на стороні попиту певні розміри
суспільної потреби, якій на другій стороні відповідають певні
розміри суспільного виробництва в різних галузях виробництва.
Для того, щоб бавовняна промисловість могла і далі провадити
свою річну репродукцію в даному масштабі, потрібна звичайна
кількість бавовни, а якщо взяти до уваги щорічне розширення
\parbreak{}  %% абзац продовжується на наступній сторінці
