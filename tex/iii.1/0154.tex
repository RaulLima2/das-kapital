вість, хоч при певних обставинах можуть спричиняти однаковий
вплив.

Перше відношення грунтується на технічній базі, і на певному
ступені розвитку продуктивної сили його треба розглядати
як дане. Потрібна певна маса робочої сили, представлена
певним числом робітників, щоб виробити певну масу продукту,
наприклад, протягом одного дня, і, отже — що при цьому само
собою розуміється — привести в рух, продуктивно спожити
певну масу засобів виробництва, машин, сировинних матеріалів
і т. д. Певне число робітників припадає на певну кількість засобів
виробництва, отже певна кількість живої праці припадає
на певну кількість праці, вже упредметненої в засобах виробництва.
Це відношення дуже різне в різних сферах виробництва,
часто в різних галузях однієї й тієї ж промисловості, хоч, з другого
боку, випадково воно може бути цілком або приблизно
однаковим в дуже віддалених одна від одної галузях промисловості.
Це відношення становить технічний склад капіталу і є дійсна
основа його органічного складу.

Але можливо також, що це відношення однакове в різних
галузях промисловості, оскільки змінний капітал є простий показник
робочої сили, а сталий капітал — простий показник маси
засобів виробництва, приведеної в рух цією робочою силою.
Наприклад, певні роботи з міддю й залізом можуть вимагати
однакового відношення між робочою силою і масою засобів
виробництва. Але через те що мідь дорожча, ніж залізо, то вартісне
відношення між змінним і сталим капіталом в обох випадках
буде різне і разом з тим буде різний і вартісний склад
обох цілих капіталів. Ріжниця між технічним складом і вартісним
складом виявляється в кожній галузі промисловості
в тому, що при незмінному технічному складі вартісне відношення
обох частин капіталу може змінюватись, а при зміні
технічного складу вартісне відношення може лишатись незмінним;
останнє має місце, звичайно, тільки тоді, коли зміна
відношення між застосованою масою засобів виробництва і масою
робочої сили вирівнюється протилежною зміною їх вартостей.
Вартісний склад капіталу, оскільки він визначається його
технічним складом і відображає цей останній, ми звемо органічним
складом капіталу.20

Отже, щодо змінного капіталу ми припускаємо, що він є показник
певної кількості робочої сили, певного числа робітників
або певної маси приводжуваної в рух живої праці. В поперед-

20 Вищевикладене коротко було розвинуте уже в третьому виданні першої
книги, стор. 628 [стор. 485 рос. вид. 1935 р.]. на початку розділу XXIII. Через
те що в перших двох виданнях немає цього місця, повторення його тут мало
тим більше підстав. — Ф. Е.
