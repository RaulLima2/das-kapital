певним органічним законам, зв’язаним з певними природними
строками, не можуть бути раптом збільшені в такій мірі, як,
наприклад, машини та інший основний капітал, вугілля, руда
та ін., збільшення яких, якщо припустити незмінність природних
умов, може відбуватися в промислово розвиненій країні
в найкоротші строки. Тим то можливо, а при розвиненому капіталістичному
виробництві навіть неминуче, що виробництво і
збільшення тієї частини сталого капіталу, яка складається з
основного капіталу, машин і т. д., значно випереджає виробництво
і збільшення тієї його частини, яка складається з органічних
сировинних матеріалів, так що попит на ці сировинні матеріали
зростає швидше, ніж подання їх, в наслідок чого ціна їх підвищується.
Це підвищення ціни приводить в дійсності до того:
1) що ці сировинні матеріали починають довозитися з дальших
місцевостей, бо підвищена ціна покриває збільшені витрати транспорту;
2) що виробництво цих сировинних матеріалів збільшується
— обставина, яка, однак, залежно від природних умов,
можливо, тільки через рік зможе дійсно збільшити масу продукту,
і 3) що використовуються всякі раніш невикористовувані
сурогати і починають економніше поводитися з відпадами. Коли
підвищення цін починає дуже помітно впливати на розширення
виробництва й подання, то здебільшого вже настав поворотний
пункт, після якого в наслідок триваючого збільшення кількості
сировинного матеріалу і всіх тих товарів, в які він входить як
елемент, попит падає, і тому настає також реакція в русі ціни
сировинного матеріалу. Незалежно від тих конвульсій, які викликає
ця реакція в наслідок знецінення капіталу в різних формах,
сюди долучаються ще й інші обставини, про які треба
зараз згадати.

Але, насамперед, уже з досі сказаного ясно таке: чим розвиненіше
капіталістичне виробництво і чим більше тому засобів
для раптового й безперервного збільшення тієї частини
сталого капіталу, яка складається з машин і т. д., чим швидше нагромадження
(як, особливо, в часи процвітання), тим більша відносна
перепродукція машин та іншого основного капіталу, і тим
частіша відносна недопродукція рослинних і тваринних сировинних
матеріалів, тим чіткіше вищеописане підвищення їх ціни і відповідна
цьому останньому реакція. Отже, тим частіші ті потрясіння
(Revulsionen), які виникають в наслідок цього сильного коливання
ціни одного з головних елементів процесу репродукції.

Але якщо настає крах цих високих цін, через те що їх підвищення
викликало почасти зменшення попиту, а почасти розширення
виробництва і подання з віддалених і досі мало або
й зовсім невикористовуваних виробничих місцевостей, при чому
обидві ці обставини приводять до перевищення подання сировинних
матеріалів над попитом на них — перевищення саме при старих
високих цінах, — то результат цього слід розглядати з різних
точок зору. Раптовий крах ціни сировинного продукту гальмує
