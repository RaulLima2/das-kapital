\parcont{}  %% абзац починається на попередній сторінці
\index{iii1}{0104}  %% посилання на сторінку оригінального видання
Кінкред, оповідає про одну фірму з Глазго, що вона на своїй
фабриці зробила з старого заліза захисні пристрої коло всіх
своїх машин, що їй коштувало 9 фунтів стерлінгів 1 шилінг.
Коли б вона приєдналась до згаданої асоціації, то вона мусила б
заплатити за свої 110 кінських сил 11 фунтів стерлінгів внесків,
тобто більше, ніж їй коштували всі захисні пристрої. І все ж
в 1854 році була заснована національна асоціація з явною метою
чинити опір законові, який приписував такі захисні пристрої.
Протягом всього часу з 1844 до 1854 року фабриканти не звертали
ні найменшої уваги на цей закон. З наказу Пальмерстона
фабричні інспектори повідомили фабрикантів, що віднині закон
застосовуватиметься з усією серйозністю. Фабриканти відразу
заснували свою асоціацію, серед найвидатніших членів якої було
навіть багато мирових суддів, які самі повинні були, як такі, застосовувати цей закон. Коли в квітні
1855 року новий міністр
внутрішніх справ, сер Джордж Грей, зробив компромісну пропозицію, за якою уряд згідний був
задовольнитись майже тільки
номінальними захисними пристроями, то асоціація з обуренням
відкинула і цю пропозицію. Відомий інженер Томас Ферберн
рискував своєю репутацією, виступаючи в різних процесах як
експерт на захист економії і порушеної свободи капіталу. Голову
фабричної інспекції, Леонарда Горнера, фабриканти переслідували і обмовляли всякими способами.

Однак, фабриканти не заспокоїлись, поки не добились вироку
Court of Queens Bench [Суду королівської лави], за тлумаченням
якого закон 1844 року не приписував ніяких захисних пристроїв коло горизонтальних валів, поставлених
вище 7 футів над
підлогою; нарешті, в 1856 році їм удалося з допомогою святоші
Вільсона Паттена, — одного з тих побожних людей, які виставляють
свою релігійність напоказ і завжди готові виконувати найбруднішу
роботу, щоб догодити рицарям гаманця, — провести в парламенті
закон, яким вони при тодішніх обставинах могли бути задоволені.
Цей закон фактично позбавив робітників усякого спеціального захисту і відсилав їх шукати
відшкодування при нещасних випадках,
спричинених машинами, до звичайних судів (чистий глум при
англійських судових витратах), тимчасом як, з другого боку,
за допомогою дуже хитромудрих приписів щодо переведення
експертизи він зробив майже неможливим, щоб фабрикант програв процес. Результатом цього було швидке
збільшення нещасних випадків. За півріччя, з травня до жовтня 1858 року,
інспектор Бекер констатував збільшення нещасних випадків на
21\% порівняно тільки з минулим півріччям. 36,7\% усіх нещасних
випадків, на його думку, можна було б уникнути. Правда,
в 1858 і 1859 рр. число нещасних випадків значно зменшилось
проти 1845 і 1846 рр., а саме на 29\%, при збільшенні числа робітників на 20\% у галузях
промисловості, що підлягали інспекції.
Але в чому причина цього? Оскільки це питання, що викликало
боротьбу, до цього часу (1865 р.) розв’язане, воно розв’язане
\parbreak{}  %% абзац продовжується на наступній сторінці
