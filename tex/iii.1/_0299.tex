\parcont{}  %% абзац починається на попередній сторінці
\index{iii1}{0299}  %% посилання на сторінку оригінального видання
По-перше: явища конкуренції; але вони стосуються тільки до
розподілу торговельного зиску між окремими купцями, володільцями
тієї чи іншої частини сукупного купецького капіталу; якщо,
наприклад, один продає дешевше, щоб вибити з позиції своїх
противників.

По-друге: економіст такого калібру, як професор Рошер, все
ще може уявляти собі в Лейпцігу, що зміна в продажних цінах
викликана з міркувань „розсудливості і гуманності“ і що вона
не була результатом перевороту в самому способі виробництва.

По-третє: якщо ціни виробництва знижуються в наслідок
підвищення продуктивної сили праці і якщо через це знижуються
і продажні ціни, то попит часто зростає ще швидше, ніж
подання, а разом з ним зростають і ринкові ціни, так що продажні
ціни дають більше, ніж пересічний зиск.

По-четверте: який-небудь купець може знизити продажну
ціну (що завжди є не що інше, як зниження звичайного зиску,
який він накидає на ціну), щоб швидше обертати більший
капітал у своєму підприємстві. Все це речі, які стосуються
тільки до конкуренції між самими купцями.

Уже в книзі І було показано, що висота чи низькість товарних
цін не визначає ні маси додаткової вартості, яку
виробляє даний капітал, ні норми додаткової вартості; хоча
відповідно до відносної кількості товару, що його виробляє
дана кількість праці, ціна одиниці товару, а разом з тим
і частина цієї ціни, яка становить додаткову вартість, буде більша
чи менша. Ціни всякої кількості товарів, оскільки вони відповідають
вартостям, визначаються загальною кількістю упредметненої
в цих товарах праці. Якщо невелика кількість праці
упредметнюється у великій кількості товару, то ціна одиниці
товару низька і вміщена в ньому додаткова вартість незначна.
Яким чином праця, втілена в товарі, розпадається на оплачену
і неоплачену працю, отже, яка частина ціни товару представляє
додаткову вартість, це не має ніякого відношення до цієї загальної
кількості праці, отже й до ціни товару. Норма ж додаткової
вартості залежить не від абсолютної величини тієї
додаткової вартості, яка міститься в ціні окремого товару,
а від її відносної величини, від її відношення до заробітної плати,
вміщеної в тому самому товарі. Тому норма може бути висока,
хоч абсолютна величина додаткової вартості в кожній одиниці
товару невелика. Ця абсолютна величина додаткової вартості
в кожній одиниці товару залежить в першу чергу від продуктивності
праці і тільки в другу чергу від поділу праці на оплачену
і неоплачену.

Для торговельної продажної ціни навіть ціна виробництва є
наперед дана зовнішня умова.

Високі торговельні ціни товарів за попередніх часів зумовлювались:
1) високою ціною виробництва, тобто низькою продуктивністю
праці; 2) відсутністю загальної норми зиску, при чому
\parbreak{}  %% абзац продовжується на наступній сторінці
