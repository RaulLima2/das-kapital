лежної фази, яка доповнює першу. І хоча для торговельного
капіталу Т — Г промислового капіталу завжди виступає як Г —
Т — Г, проте й для нього, раз він почав функціонувати, дійсним
процесом постійно є Т — Г — Т. Але торговельний капітал одночасно
пророблює акти Т — Г і Г — Т. Тобто не тільки один капітал
перебуває в стадії Т — Г, тимчасом як другий капітал
перебуває в стадії Г — Т, але той самий капітал одночасно постійно
купує і постійно продає в наслідок безперервності процесу
виробництва; він одночасно постійно перебуває в обох
стадіях. Тимчасом як одна частина його перетворюється в гроші,
щоб пізніше знову перетворитись у товар, друга частина одночасно
перетворюється в товар, щоб знову перетворитись у гроші.

Чи функціонують при цьому гроші як засіб циркуляції чи як
засіб платежу, це залежить від форми товарообміну. В обох випадках
капіталістові постійно доводиться виплачувати гроші багатьом
особам і постійно одержувати гроші в оплату від багатьох
осіб. Ця чисто технічна операція виплачування грошей і одержування
грошей сама по собі становить працю, яка, оскільки гроші
функціонують як засіб платежу, робить необхідними балансові
розрахунки, акти вирівнення взаємних зобов’язань. Ця праця —
витрати циркуляції, праця, яка не утворює ніякої вартості. Вона
скорочується в наслідок того, що її виконує особливий підрозділ
агентів або капіталістів для всієї решти класу капіталістів.

Певна частина капіталу мусить постійно бути в наявності
як скарб, як потенціальний грошовий капітал резерв купівельних
засобів, резерв засобів платежу, незанятий капітал, який у грошовій
формі чекає свого застосування; а деяка частина капіталу
постійно припливає в цій формі назад. Крім одержування грошей,
виплачування грошей і рахівництва, це робить необхідним
зберігання скарбу, що знов таки є особлива операція. Отже,
в дійсності це є постійний розпад скарбу на засоби циркуляції
і засоби платежу і утворення його знову з грошей, одержуваних
від продажу і від платежів, яким уже настав строк; цей постійний
рух частини капіталу, яка існує у вигляді грошей, рух, відокремлений
від функції самого капіталу, ця чисто технічна операція
викликає особливу працю і витрати — витрати циркуляції.

Поділ праці приводить до того, що ці технічні операції, зумовлювані
функціями капіталу, виконуються, наскільки можливо,
для всього класу капіталістів певним підрозділом агентів або
капіталістів як. їх виключні функції або концентруються в їх
руках. Це є, як і в випадку з купецьким капіталом, поділ праці
у двоякому розумінні. Ці технічні операції стають особливим заняттям,
і в наслідок того, що воно, як особливе заняття, виконується
для грошового механізму всього класу, воно концентрується, провадиться
у великому масштабі; а в межах цього особливого заняття
знов таки відбувається поділ праці як в наслідок розпаду його
на різні, незалежні одна від одної галузі, так і в наслідок утворення
майстерні в кожній такій галузі (великі контори, чис-
