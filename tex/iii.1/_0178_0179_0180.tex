\index{iii1}{0178}  %% посилання на сторінку оригінального видання
\section{Вирівнення загальної норми зиску через
Конкуренцію. Ринкові ціни і ринкові вартості.
Надзиск}

Частина сфер виробництва має середній або пересічний склад
застосовуваного в них капіталу, тобто склад капіталу, який цілком чи приблизно збігається з складом
пересічного суспільного
капіталу.

Ціна виробництва товарів, вироблюваних у цих сферах виробництва, цілком чи приблизно збігається з їх
вартістю, вираженою
в грошах. І коли б ніяким іншим способом не можна було досягти математичної границі, то цього можна
було б досягти цим
способом. Конкуренція так розподіляє суспільний капітал між
різними сферами виробництва, що ціни виробництва в кожній
сфері утворюються на зразок цін виробництва в цих сферах
середнього складу, тобто = $k + kp'$ (витрати виробництва плюс
добуток пересічної норми зиску і витрат виробництва). Але
ця пересічна норма зиску є не що інше, як обчислений в процентах зиск у сфері виробництва середнього
складу, де, отже,
зиск збігається з додатковою вартістю. Отже, норма зиску в усіх
сферах виробництва є одна й та ж, а саме вирівнена до норми
зиску цих середніх сфер виробництва, в яких панує пересічний
склад капіталу. Тому сума зисків усіх різних сфер виробництва
мусить дорівнювати сумі додаткових вартостей і сума цін виробництва сукупного суспільного продукту
мусить дорівнювати
сумі його вартостей. Але ясно, що це вирівнювання між сферами виробництва з різним складом завжди
мусить прагнути
урівняти ці сфери з сферами середнього складу, однаково, чи
ці останні точно чи тільки приблизно відповідають пересічному
суспільному складові. У сферах виробництва, які більш чи менш
наближаються до середньої, знову таки має місце тенденція
до вирівнення, яка прагне до ідеального, тобто в дійсності не
наявного середнього рівня, тобто тенденція до вирівнення
навколо нього, як норми. Таким чином у цьому відношенні
необхідно панує тенденція зробити ціни виробництва просто
перетвореними формами вартості, або перетворити зиски в
прості частини додаткової вартості, які, однак, розподіляються
не пропорційно до додаткової вартості, створеної в кожній
окремій сфері виробництва, а пропорційно до маси капіталу,
застосовуваного в кожній сфері виробництва, так що на рівновеликі маси капіталу, хоч би який був їх
склад, припадають відповідно рівновеликі частини сукупної додаткової вартості, створеної сукупним
суспільним капіталом.

Отже, для капіталів середнього чи приблизно середнього
складу ціна виробництва збігається цілком або приблизно з вартістю,
\index{iii1}{0179}  %% посилання на сторінку оригінального видання
а зиск — із створеною ними додатковою вартістю. Всі інші
капітали, хоч би який був їх склад, під тисненням конкуренції
прагнуть зрівнятися з капіталами середнього або приблизно
середнього складу. Але через те що капітали середнього складу
е рівні або приблизно рівні пересічному суспільному капіталові,
то всі капітали, яка б не була величина створеної ними самими
додаткової вартості, прагнуть замість цієї додаткової вартості
реалізувати в цінах своїх товарів пересічний зиск, тобто прагнуть реалізувати ціни виробництва.

З другого боку, можна сказати, що повсюди, де встановлюється пересічний зиск, отже загальна норма
зиску — яким би
шляхом не досягався цей результат, — цей пересічний зиск не
може бути нічим іншим, як зиском на пересічний суспільний
капітал, зиском, сума якого дорівнює сумі додаткових вартостей,
а ціни, які утворюються в наслідок надбавки цього пересічного
зиску до витрат виробництва, не можуть бути нічим іншим, як
перетвореними в ціни виробництва вартостями. Справа ні трохи
не змінилася б, коли б капітали в певних сферах виробництва
з будь-яких причин не підлягали цьому процесові вирівнення.
Тоді пересічний зиск обчислювався б на ту частину суспільного
капіталу, яка входить у процес вирівнення. Очевидно, що пересічний зиск не може бути нічим іншим, як
сукупною масою
додаткової вартості, розподіленою в кожній сфері виробництва
між масами капіталів пропорційно до їхніх величин. Це — сума
реалізованої неоплаченої праці, і вся ця маса праці, так само як
і оплачена, мертва й жива праця, виражається в сукупній масі,
товарів і грошей, яка припадає капіталістам.

Справжня трудність питання тут ось у чому: як відбувається
це вирівнення зисків у загальну норму зиску, раз воно, очевидно, є результат і не може бути вихідним
пунктом.

Насамперед, очевидно, що оцінка товарних вартостей, наприклад, у грошах, може бути тільки
результатом обміну їх і що,
припускаючи таку оцінку, ми повинні розглядати її як результат
дійсного обміну товарної вартості на товарну вартість. Але
яким же чином може здійснитись цей обмін товарів по їх дійсних вартостях?

Припустімо, спочатку, що всі товари в різних сферах виробництва продаються по їх дійсних вартостях.
Що сталося б
тоді? Згідно з вищевикладеним, в різних сферах виробництва
тоді панували б дуже різні норми зиску. Чи продаються товари
по їх вартостях (тобто чи обмінюються вони один на один пропорційно до вміщеної в них вартості, по
цінах їх вартості),
чи продаються вони по таких цінах, що продаж їх дає рівновеликі зиски на рівновеликі маси капіталів,
авансованих на відповідне виробництво їх, — це prima facie [очевидно] цілком різні речі.

Та обставина, що капітали, які приводять в рух неоднакову
кількість живої праці, виробляють неоднакову кількість додаткової
\index{iii1}{0180}  %% посилання на сторінку оригінального видання
вартості, передбачає, принаймні до певної міри, що ступінь
експлуатації праці або норма додаткової вартості однакова, або
що існуючі в цьому відношенні ріжниці вирівнюються за допомогою
дійсних або уявних (умовних) компенсуючих причин.
Це передбачає конкуренцію між робітниками і вирівнювання
ступеня їх експлуатації в наслідок постійного переходу їх
з однієї сфери виробництва до іншої. Така загальна норма додаткової
вартості — як тенденція, подібно до всіх економічних
законів, — припускається нами як теоретичне спрощення; але
в дійсності вона є фактична передумова капіталістичного способу
виробництва, хоч вона й гальмується в більшій чи меншій
мірі практичними тертями, які викликають більш чи менш
значні місцеві ріжниці, такі є, наприклад, закони про осілість
(settlement laws) для землеробських поденників в Англії. Але
в теорії припускається, що закони капіталістичного способу
виробництва розвиваються в чистому вигляді. В дійсності існує
завжди тільки наближення; однак, це наближення тим більше,
чим більше розвинений капіталістичний спосіб виробництва і чим
більше усунене його забарвлення рештками попередніх економічних
становищ і переплетення з ними.

Вся трудність постає з того, що товари обмінюються не
просто як \emph{товари}, а як \emph{продукти капіталів}, які претендують
на пропорціональну до їх величини або, при рівній величині, на
рівну участь у сукупній масі додаткової вартості. І сукупна
ціна товарів, вироблених даним капіталом за даний період часу,
повинна задовольнити цю вимогу. Але сукупна ціна цих товарів
є просто сума цін окремих товарів, які становлять продукт
капіталу.

Punctum saliens [вирішальний пункт] виступить найбільше, якщо
ми підійдемо до справи так: Припустім, що самі робітники
володіють своїми відповідними засобами виробництва і обмінюють
свої товари один з одним. Ці товари не були б тоді
продуктами капіталу. Залежно від технічної природи їх робіт,
вартість засобів праці і матеріалів праці, застосовуваних у різних
галузях праці, була б різна; так само, незалежно від неоднакової
вартості застосовуваних засобів виробництва, потрібна була б
різна маса цих засобів виробництва для даної маси праці, залежно
від того, що один певний товар може бути виготовлений
за одну годину, а інший тільки за день і т. д. Припустімо
далі, що ці робітники пересічно працюють однакову кількість
часу, враховуючи вирівнення, які випливають з різної інтенсивності
праці та ін. Двоє робітників замістили б тоді в товарах,
що становлять продукт їх денної праці, поперше, свої
видатки, витрати (die Kostpreise) на спожиті засоби виробництва.
Ці останні були б різні залежно від технічної природи їх галузей
праці. Подруге, вони обидва створили б однакові кількості
нової вартості, а саме робочий день, доданий ними до засобів
виробництва. Ця нова вартість містила б у собі їх заробітну
\parbreak{}  %% абзац продовжується на наступній сторінці
