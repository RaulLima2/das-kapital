400 фунтів стерлінгів. Далі, через те що сталий капітал вартістю в 2000 фунтів стерлінгів потребує
для свого функціонування 500 робітників, то 400 робітників можуть привести в рух тільки сталий
капітал вартістю в 1600 фунтів стерлінгів. Отже,
для того, щоб виробництво і далі провадилося в попередніх
розмірах і щоб 1/5 машин не стояла без діла, змінний капітал
мусить бути підвищений на 100 фунтів стерлінгів, щоб, як і раніш, вживати 500 робітників; а цього
можна досягти тільки за
допомогою того, що вільний досі капітал зв’язується, при чому
та частина нагромадження, яка повинна була б служити для розширення виробництва, тепер служить
тільки для поповнення, або ж
до попереднього капіталу додається та частина, яка призначена
була для витрачання як дохід. Із збільшеною на 100 фунтів
стерлінгів витратою змінного капіталу тепер виробляється на
100 фунтів стерлінгів менше додаткової вартості. Щоб привести
в рух те саме число робітників, потрібно більше капіталу, і разом
з тим зменшується додаткова вартість, яку дає кожний окремий робітник.

Вигоди, які випливають із звільнення, і втрати, які випливають із зв’язування змінного капіталу,
існують тільки для капіталу, який уже вкладений і який, отже, репродукується при даних відношеннях.
Для нововкладуваного капіталу вигоди на
одному боці, втрати на другому зводяться до підвищення або
зниження норми додаткової вартості і відповідної, хоч і зовсім
не пропорціональної, зміни норми зиску.

Щойно досліджене звільнення і зв’язування змінного капіталу є наслідок зниження вартості або
підвищення вартості
елементів змінного капіталу, тобто витрат репродукції робочої
сили. Але змінний капітал може звільнятися й тоді, коли внаслідок розвитку продуктивної сили, при
незмінній нормі заробітної плати, потрібно менше робітників для того, щоб привести
в рух ту саму масу сталого капіталу. Так само, навпаки, зв’язування додаткового змінного капіталу
може мати місце, якщо
в наслідок зниження продуктивної сили праці потрібно більше
робітників для тієї самої маси сталого капіталу. Якщо ж, з другого боку, частина капіталу, який
раніш застосовувався як змінний капітал, застосовується тепер у формі сталого капіталу, отже, якщо
відбувається тільки зміна розподілу між складовими
частинами того самого капіталу, то, хоч це і справляє вплив
на норму додаткової вартості й зиску, але не належить до розглядуваної тут рубрики зв’язування і
звільнення капіталу.

Сталий капітал, як ми вже бачили, також може зв’язуватись
або звільнятись в наслідок підвищення вартості або зниження
вартості тих елементів, з яких він складається. Залишаючи це
осторонь, зв’язування його можливе (без перетворення будь-якої
частини змінного капіталу в сталий) тільки тоді, коли збіль-
