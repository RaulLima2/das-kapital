його праці капіталом, — ця функція, що виникає з поневолення
безпосереднього виробника, досить часто виставляється як довід
для виправдання самого цього відношення, і експлуатація, привласнення
чужої неоплаченої праці, так само часто зображається
як заробітна плата, належна власникові капіталу. Але найкраще
це робить захисник рабства у Сполучених Штатах, адвокат
О’Коннор, в своїй промові, виголошеній 19 грудня 1859 року
на мітингу в Нью-Йорку, під прапором „Справедливість до
Півдня“. — „Now, gentlemen [панове], — сказав він під гучні
оплески, — сама природа призначила негра до цього становища
раба. Він кріпкий і сильний у роботі; але природа, яка дала
йому цю силу, відмовила йому як у розумі для керівництва, так
і в бажанні працювати“. (Оплески). „І в тому і в другому йому
відмовлено! І та сама природа, яка не дала йому волі до праці,
дала йому пана для того, щоб вимусити у нього цю волю і
зробити з нього в тому кліматі, для якого він створений, корисного
слугу як для самого себе, так і для пана, що управляє
ним. Я запевняю, що немає ніякої несправедливості в тому, щоб
залишити негра в тому становищі, в яке його поставила природа;
немає ніякої несправедливості в тому, щоб дати йому пана, який
управляє ним; і в нього не віднімають жодного з його прав, коли
його примушують за це працювати і давати справедливу винагороду
своєму панові за ту працю й таланти, які цей пан застосовує
для того, щоб управляти ним і зробити його корисним
для нього самого й для суспільства“ [„New York Tribune“,
20 грудня 1859, стор. 5].

Так ось, найманий робітник, подібно до раба, теж мусить
мати пана, який примушував би його працювати і управляв би
ним. А якщо припустити це відношення панування й рабства,
то це в порядку речей, що найманого робітника примушують
виробляти свою власну заробітну плату і поверх того плату
за нагляд, компенсацію за працю панування й верховного нагляду
за ним, „і давати справедливу винагороду своєму панові
за ту працю й таланти, які цей пан застосовує для того, щоб
управляти ним і зробити його корисним для нього самого й для
суспільства“.

Праця верховного нагляду й керівництва, оскільки вона виникає
з антагоністичного характеру, з панування капіталу над працею,
і тому є спільна капіталістичному способові виробництва,
як і всім способам виробництва, що грунтуються на класовій
протилежності, — ця праця і в капіталістичній системі безпосередньо
й нероздільно зв’язана з продуктивними функціями, що
їх, як особливу працю, усяка комбінована суспільна праця покладає
на окремих індивідів. Заробітна плата якогонебудь epitropos’a
[наглядача в стародавній Греції] або regisseur’a, як його звали
у феодальній Франції, цілком відокремлюється від зиску і набирає
форми заробітної плати за вправну працю, як тільки підприємство
починає провадитись у розмірах, досить великих для того,
