нятку з нього, цей позірний виняток в дійсності був тільки
окремим випадком застосування загального закону.

Якщо в попередньому відділі виявилось, що, при незмінному
ступені експлуатації праці, із зміною вартості складових частин
сталого капіталу, а також із зміною часу обороту капіталу змінюється
норма зиску, то з цього само собою випливає, що
норми зиску різних одночасно існуючих, одна поряд одної, сфер
виробництва будуть різні, якщо при інших незмінних умовах
час обороту застосовуваних капіталів різний або якщо вартісне
відношення між органічними складовими частинами цих капіталів
у різних галузях виробництва є різне. Те, що ми раніш
розглядали як зміни, що відбуваються послідовно в часі
з тим самим капіталом, ми розглядаємо тепер як одночасно
наявні ріжниці між існуючими одно поряд одного капіталовкладеннями
в різних сферах виробництва.

При цьому нам доведеться дослідити: 1) ріжницю в органічному
складі капіталів, 2) ріжницю в часі їх обороту.

В усьому цьому дослідженні, коли ми говоримо про склад
або оборот капіталу в певній галузі виробництва, ми завжди
маємо на увазі — припущення, яке само собою зрозуміле, — пересічні
нормальні відношення капіталу, вкладеного в цю галузь
виробництва; взагалі, мова йде про пересічні відношення сукупного
капіталу, вкладеного в дану сферу, а не про випадкові
ріжниці між окремими вкладеними в цю сферу капіталами.

Через те що, далі, припускається, що норма додаткової вартості
і робочий день є незмінні, і через те що це припущення
включає також і незмінність заробітної плати, то певна кількість
змінного капіталу виражає певну кількість приведеної
в рух робочої сили, а тому й певну кількість праці, яка упредметнюється.
Отже, якщо 100 фунтів стерлінгів виражають тижневу
заробітну плату 100 робітників, тобто в дійсності 100 робочих
сил, то 100 фунтів стерлінгів × n виражають тижневу
заробітну плату 100 × n робітників, а 100 фунтів стерлінгів/n тижневу
заробітну плату 100/n робітників. Отже, змінний капітал служить
тут (як і завжди при даній величині заробітної плати) показником
маси праці, яку приводить в рух весь капітал певної величини;
тому ріжниці у величині застосовуваного змінного капіталу
служать показниками ріжниці в масі вживаної робочої сили. Якщо
100 фунтів стерлінгів представляють 100 робітників на тиждень
і, отже, при 60 годинах тижневої праці — 6000 робочих годин, то
200 фунтів стерлінгів представляють 12 000 робочих годин, а 50
фунтів стерлінгів тільки 3000 робочих годин.

Під складом капіталу ми розуміємо, як це сказано вже у
книзі першій, відношення між його активною і пасивною складовою
частиною, між змінним і сталим капіталом. При цьому
треба розглянути два відношення, які мають неоднакову важли-
