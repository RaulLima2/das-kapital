\parcont{}  %% абзац починається на попередній сторінці
\index{iii1}{0340}  %% посилання на сторінку оригінального видання
функціонує в процесі своєї репродукції як товарний капітал.
Віддання в позику і одержання позики, замість продажу й купівлі,
— це ріжниця, яка тут випливає з специфічної природи
такого товару, як капітал. Цілком так само, як і та ріжниця, що
тут сплачується процент замість ціни товару. Якщо процент
назвати ціною грошового капіталу, то це буде ірраціональна
форма ціни, цілком суперечна поняттю ціни товару.\footnote{„Вираз вартість (value), застосований до currency [засобу циркуляції], має
три значення\dots{} Подруге, currency actually in hand (гроші, які дійсно є в касі],
порівняно з тією самою сумою грошей, яка надійде пізніше. Тоді їх вартість
вимірюється розміром процента, а розмір процента визначається by the ratio
between the amount of loanable capital and the demand for it [відношенням між
сумою капіталу, що дається в позику, і попитом на нього]“ (Полковник
\emph{R. Torrens}: „On the Operation of the Bank Charter Act of 1844 etc.“, 2 вид. 1847,
[стор. 5 і далі]).} Ціна зведена
тут до її чисто абстрактної і беззмістовної форми, до того,
що вона є певна сума грошей, яка сплачується за щонебудь, що
так чи інакше фігурує як споживна вартість; тимчасом як за
своїм поняттям ціна дорівнює вираженій у грошах вартості цієї
споживної вартості.

Процент як ціна капіталу є з самого початку цілком ірраціональний
вираз. Товар має тут подвійну вартість, поперше, вартість
і, подруге, відмінну від цієї вартості ціну, тимчасом як ціна
є грошовий вираз вартості. Грошовий капітал насамперед є не
що інше, як сума грошей або вартість певної маси товарів, фіксована
як сума грошей. Якщо в позику віддається товар як
капітал, то він є тільки замаскована форма грошової суми. Бо
те, що дається в позику як капітал, це не стільки то фунтів
бавовни, а стільки то грошей, які існують у формі бавовни як
її вартість. Тому ціна капіталу відноситься до нього як до грошової
суми, хоч і не як до „currency“ [засобу циркуляції], як
це гадає пан Торренс (див. вище примітку 60). Яким же чином
сума вартості може мати ціну, крім своєї власної ціни, крім
ціни, вираженої в її власній грошовій формі? Адже ціна є вартість
товару (і це стосується також до ринкової ціни, відмінність
якої від вартості є не якісна, а тільки кількісна, тобто така, що
стосується тільки до величини вартості) в відміну від його
споживної вартості. Ціна, яка якісно відмінна від вартості —
це абсурдна суперечність.\footnote{
The ambiguity of the term value of money or of the currency, when employed
indiscriminately as it is, to signify both value in exchange for commodities
and value in use of capital, is a constant source of confusion“ [„Двояке значення
виразу вартість грошей або засобу циркуляції, коли його, як це буває, без
розрізнення вживають для позначення як мінової вартості товарів, так і споживної
вартості капіталу, є постійним джерелом плутанини“] (\emph{Tooke}: „Inquiry
into the Currency Principle“, стор. 77). — Головної плутанини (яка лежить у самій
суті справи), що вартість як така (процент) стає споживною вартістю капіталу,
Тук не бачить.
}

Капітал виявляє себе як капітал в наслідок зростання його
вартості; ступінь зростання його вартості є виразом того кількісного
\index{iii1}{0341}  %% посилання на сторінку оригінального видання
ступеня, в якому він реалізується як капітал. Створена
ним додаткова вартість або зиск — її норма або висота — може
бути вимірена тільки за допомогою порівняння її з вартістю
авансованого капіталу. Тому й більше чи менше зростання вартості
капіталу, що дає процент, може бути вимірене тільки
за допомогою порівняння суми-процента, тієї частини загального
зиску, яка припадає йому, з вартістю авансованого капіталу. Тому,
якщо ціна виражає вартість товару, то процент виражає зростання
вартості грошового капіталу і виступає через це як ціна,
яка сплачується за нього позикодавцеві. Звідси ясно, наскільки
прямо безглуздим є намагання безпосередньо прикласти сюди, як
це робить Прудон, прості відносини обміну, який відбувається за
допомогою грошей, прості відносини купівлі й продажу. Основна
передумова полягає саме в тому, що гроші функціонують як
капітал і тому можуть бути передані третій особі як капітал у
собі, як потенціальний капітал.

Але як товар капітал і тут, виступає остільки, оскільки він
пропонується на ринку і оскільки відчужується дійсно споживна
вартість грошей як капіталу. Але його споживна вартість полягає
в створюванні зиску. Вартість грошей або товарів як капіталу
визначається не їх вартістю як грошей або товарів, а тією
кількістю додаткової вартості, яку вони виробляють для свого
володільця. Продукт капіталу є зиск. Чи витрачаються гроші
як гроші, чи вони авансуються як капітал — це на основі капіталістичного
виробництва є тільки різне застосування грошей.
Гроші — або товар — є капітал у собі, потенціальний капітал, цілком
так само, як і робоча сила потенціально є капітал. Бо
1) гроші можуть бути перетворені в елементи виробництва і самі
вони, як такі, є лише абстрактний вираз елементів виробництва,
їх буття як вартість; 2) речові елементи багатства мають
властивість потенціально бути вже капіталом, тому що протилежність,
яка доповнює їх, те, що робить їх капіталом, — наймана
праця, — на основі капіталістичного виробництва є в наявності.

Антагоністична суспільна визначеність речового багатства —
його антагонізм з працею як найманою працею — є виражена,
відокремлено від процесу виробництва, уже в самій власності
на капітал, як такій. Цей момент — відокремлено від самого капіталістичного
процесу виробництва, постійним результатом якого
він є, і будучи, як такий постійний результат його, разом з тим
його постійною передумовою — виражається в тому, що гроші,
і так само товар, у собі, приховано, потенціально, є капітал, що
вони можуть бути продані як капітал і що в цій формі вони панують
над чужою працею, дають підставу претендувати на привласнення
чужої праці, а тому вони є вартість, що самозростає.
Тут також ясно виступає, що це відношення, а не якась еквівалентна
праця з боку капіталіста є підставою й засобом для привласнення
чужої праці.

Далі, капітал виступає як товар остільки, оскільки поділ зиску
\parbreak{}  %% абзац продовжується на наступній сторінці
