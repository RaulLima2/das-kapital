мого купця або дрібного торговця. В цьому випадку він одержує
додатковий зиск цілком так само, як одержують додатковий зиск
промислові капіталісти, коли вони виробляють при умовах сприятливіших,
ніж пересічні умови. Якщо конкуренція примушує до
цього, то він може продавати дешевше, ніж його товариші, не
знижуючи свого зиску нижче пересічного рівня. Якщо умови,
які дають йому можливість робити швидший оборот, є такі
умови, що йому доводиться їх купувати, наприклад, розташування
місць продажу, то він може платити за них особливу ренту,
тобто частина його додаткового зиску перетворюється в земельну
ренту.

Розділ дев’ятнадцятий

Грошево-торговельний капітал

Чисто технічні рухи, які пророблюють гроші в процесі циркуляції
промислового капіталу і, як ми тепер можемо додати,
товарно-торговельного капіталу (бо цей останній бере на себе
частину руху циркуляції промислового капіталу, роблячи його
своїм власним і специфічним рухом), — ці рухи, зробившись самостійною
функцією особливого капіталу, який виконує їх і тільки їх,
як властиві йому операції, перетворюють цей капітал у грошевоторговельний
капітал. Частина промислового капіталу і, точніше
кажучи, також товарно-торговельного капіталу постійно перебуває
в грошовій формі не тільки як грошовий капітал взагалі, але
як грошовий капітал, який виконує ці технічні функції. Від сукупного
капіталу4 відокремлюється і усамостійнюється в формі грошового
капіталу певна частина, капіталістична функція якої
полягає виключно в тому, щоб виконувати ці операції для всього
класу промислових і торговельних капіталістів. Подібно до того,
як це є з товарно-торговельним капіталом, так і тут від промислового
капіталу, який перебуває в процесі циркуляції у вигляді
грошового капіталу, відокремлюється певна частина і виконує
ці операції процесу репродукції для всієї решти сукупного капіталу.
Отже, рухи цього грошового капіталу знов таки є тільки
рухи усамостійненої частини промислового капіталу, який перебуває
в процесі своєї репродукції.

Тільки в тому випадку і остільки, коли і оскільки капітал
вкладається уперше, — що має місце і при нагромадженні, — капітал
у грошовій формі виступає як вихідний і кінцевий пункт руху.
-Але для кожного капіталу, раз він уже перебуває в своєму
процесі, вихідний пункт, як і кінцевий пункт, виступають тільки
як перехідні пункти. Оскільки промисловий капітал, з-моменту
свого виходу з сфери виробництва до того моменту, коли він
знову вступає в неї, мусить проробити метаморфозу Т' — Г — Т,
остільки Г, як це вже виявилось при простій товарній
циркуляції, в дійсності є кінцевим результатом однієї фази метаморфози
тільки для того, щоб бути вихідним пунктом проти-
