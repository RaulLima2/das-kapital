тістю, а зиск — із створеною ними додатковою вартістю. Всі інші
капітали, хоч би який був їх склад, під тисненням конкуренції
прагнуть зрівнятися з капіталами середнього або приблизно
середнього складу. Але через те що капітали середнього складу
е рівні або приблизно рівні пересічному суспільному капіталові,
то всі капітали, яка б не була величина створеної ними самими
додаткової вартості, прагнуть замість цієї додаткової вартості
реалізувати в цінах своїх товарів пересічний зиск, тобто прагнуть реалізувати ціни виробництва.

З другого боку, можна сказати, що повсюди, де встановлюється пересічний зиск, отже загальна норма
зиску — яким би
шляхом не досягався цей результат, — цей пересічний зиск не
може бути нічим іншим, як зиском на пересічний суспільний
капітал, зиском, сума якого дорівнює сумі додаткових вартостей,
а ціни, які утворюються в наслідок надбавки цього пересічного
зиску до витрат виробництва, не можуть бути нічим іншим, як
перетвореними в ціни виробництва вартостями. Справа ні трохи
не змінилася б, коли б капітали в певних сферах виробництва
з будь-яких причин не підлягали цьому процесові вирівнення.
Тоді пересічний зиск обчислювався б на ту частину суспільного
капіталу, яка входить у процес вирівнення. Очевидно, що пересічний зиск не може бути нічим іншим, як
сукупною масою
додаткової вартості, розподіленою в кожній сфері виробництва
між масами капіталів пропорційно до їхніх величин. Це — сума
реалізованої неоплаченої праці, і вся ця маса праці, так само як
і оплачена, мертва й жива праця, виражається в сукупній масі,
товарів і грошей, яка припадає капіталістам.

Справжня трудність питання тут ось у чому: як відбувається
це вирівнення зисків у загальну норму зиску, раз воно, очевидно, є результат і не може бути вихідним
пунктом.

Насамперед, очевидно, що оцінка товарних вартостей, наприклад, у грошах, може бути тільки
результатом обміну їх і що,
припускаючи таку оцінку, ми повинні розглядати її як результат
дійсного обміну товарної вартості на товарну вартість. Але
яким же чином може здійснитись цей обмін товарів по їх дійсних вартостях?

Припустімо, спочатку, що всі товари в різних сферах виробництва продаються по їх дійсних вартостях.
Що сталося б
тоді? Згідно з вищевикладеним, в різних сферах виробництва
тоді панували б дуже різні норми зиску. Чи продаються товари
по їх вартостях (тобто чи обмінюються вони один на один пропорційно до вміщеної в них вартості, по
цінах їх вартості),
чи продаються вони по таких цінах, що продаж їх дає рівновеликі зиски на рівновеликі маси капіталів,
авансованих на відповідне виробництво їх, — це prima facie [очевидно] цілком різні речі.

Та обставина, що капітали, які приводять в рух неоднакову
кількість живої праці, виробляють неоднакову кількість додат-
