співвідношенням різних частин, на які розпадається додаткова
вартість (зиск, процент, земельна рента, податки і т. д.); таким
чином тут знову виявляється, що відношенням попиту й подання
абсолютно нічого не можна пояснити, поки не розкрита та база,
на якій розвивається (spielt) це відношення.

Хоч товар і гроші, і те і друге, є єдність мінової вартості
і споживної вартості, проте, ми вже бачили (книга I, розд. 1, 3),
як в купівлі й продажу обидва ці визначення полярно розміщаються на двох крайніх пунктах, так що
товар (продавець)
репрезентує споживну вартість, а гроші (покупець) — мінову вартість. Перша передумова продажу
полягала в тому, що товар
повинен мати споживну вартість, отже, задовольняти суспільну
потребу. Друга передумова полягала в тому, що кількість
праці, вміщена в товарі, повинна репрезентувати суспільно-необхідну працю, отже, індивідуальна
вартість (і — що при цьому
припущенні є те саме — продажна ціна) товару повинна збігатися з його суспільною вартістю.28

Застосуймо це до наявної на ринку маси товарів, яка становить продукт цілої сфери виробництва.

Справу можна з’ясувати найлегше, якщо ми всю масу товарів — спочатку, отже, однієї галузі
виробництва — розглядатимемо
як один товар, а суму цін багатьох тотожних товарів як одну
сумарну ціну. В такому випадку те, що було сказано про окремий товар, буквально стосується до маси
товарів певної галузі
виробництва, яка перебуває на ринку. Відповідність індивідуальної вартості товару його суспільній
вартості здійснюється тепер
або набуває дальшого визначення в тому розумінні, що сукупна
кількість товару містить у собі працю, суспільно-необхідну для
її виробництва, і що вартість цієї маси товарів = її ринковій
вартості.

Припустімо тепер, що значна маса цих товарів вироблена
при однакових, приблизно нормальних суспільних умовах, так
що ця вартість є разом з тим індивідуальна вартість окремих
товарів, які становлять цю масу. Якщо одна порівняно незначна
частина товарів вироблена при умовах гірших, а друга — при умовах
кращих, ніж нормальні, так що індивідуальна вартість першої
частини більша, а другої менша, ніж середня вартість більшої частини цих товарів, причому обидві ці
крайності урівноважуються,
так що пересічна вартість належних до них товарів дорівнює
вартості товарів, належних до середньої маси, — то ринкова вартість визначається вартістю товарів,
вироблених при середніх
умовах.29 Вартість сукупної товарної маси дорівнює дійсній сумі
вартостей всіх окремих товарів, узятих разом — як тих, що вироблені при середніх умовах, так і тих,
що вироблені при умо-

28 K. Marx: „Zur Kritik der politischen Oekonomie“, Berlin 1859. [К. Маркс:
„До критики політичної економії“, укр. вид. 1935 р., стор. 54.]

29 Там же.
