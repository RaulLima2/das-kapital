сама дорівнює витратам виробництва плюс додаткова вартість,
отже, в даному разі дорівнює витратам виробництва плюс зиск,
а цей зиск знов таки може бути більший або менший, ніж додаткова вартість, місце якої він заступає.
Щодо змінного капіталу, то хоч пересічна денна заробітна плата завжди дорівнює вартості, виробленій
за те число годин яке робітник
мусить працювати, щоб виробити необхідні засоби існування,
однак саме число цих годин знов таки фальсифікується в наслідок того, що ціни виробництва необхідних
засобів існування
відхиляються від їх вартостей. Однак, це завжди розв’язується
таким чином, що наскільки в один товар входить більше додаткової вартості, настільки її в другий
товар входить менше,
і тому ті відхилення від вартості, які містяться в цінах виробництва товарів, взаємно знищуються.
Взагалі в цілому капіталістичному виробництві загальний закон здійснюється завжди
тільки як панівна тенденція, дуже заплутаним і приблизним
способом, тільки як якась пересічна вічних коливань, яка ніколи
не може бути точно встановлена.

Через те що загальна норма зиску утворюється з пересічної
різних норм зиску на кожні 100 авансованого капіталу за певний
період часу, скажімо, за рік, то в ній стирається також ріжниця,
викликана ріжницею в часі оборотів різних капіталів. Але ці
ріжниці є визначальним фактором для тих різних норм зиску
різних сфер виробництва, що з їх пересічної утворюється загальна норма зиску.

В попередній ілюстрації утворення загальної норми зиску
кожний капітал у кожній сфері виробництва припускався = 100,
і це було зроблено саме для того, щоб з’ясувати процентну
ріжницю в нормах зиску, а тому й ріжницю у вартостях товарів,
вироблюваних рівновеликими капіталами. Але само собою зрозуміло: дійсні маси додаткової вартості,
створювані в кожній
окремій сфері виробництва, залежать від величини застосованих
капіталів, бо в кожній такій даній сфері виробництва склад капіталу є даний. Тимчасом особлива норма
зиску кожної окремої
сфери виробництва не змінюється від того, чи застосовується
капітал в 100, 100 × m чи 100 × xm. Норма зиску однаково лишається 10% — чи становить весь зиск 10 :
100, чи 1000 : 10 000.

Але через те що норми зиску в різних сферах виробництва
є різні, бо в них залежно від відношення змінного капіталу до
всього капіталу виробляються дуже різні маси додаткової вартості, отже й зиску, то очевидно, що
пересічний зиск на кожні
100 суспільного капіталу, отже, пересічна норма зиску або загальна норма зиску, буде дуже різна
залежно від відповідної
величини капіталів, вкладених у різні сфери виробництва. Візьмімо чотири капітали А, В, C, D. Нехай
норма додаткової вартості для всіх них буде = 100%. Нехай на кожні 100 сукупного
капіталу змінного капіталу буде для А = 25, для 5 = 40, для
C = 15, для D = 10. На кожні 100 сукупного капіталу тоді при-
