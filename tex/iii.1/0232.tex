додаткової праці здовження робочого дня, — цей винахід сучасної промисловості, — не змінюючи при
цьому істотно відношення вживаної робочої сили до сталого капіталу, який вона приводить в рух, і в
дійсності скорше відносно зменшуючи сталий капітал. Взагалі ж ми вже показали, — і це становить
власне таємницю тенденції норми зиску до падіння, —-що методи виробництва відносної додаткової
вартості загалом і в цілому зводяться ось до чого: з одного боку, з даної маси праці якомога більше
перетворити в додаткову вартість, з другого боку, взагалі вживати якомога менше праці порівняно з
авансованим капіталом; так що ті самі причини, які дозволяють підвищувати ступінь експлуатації
праці, не дозволяють з тим самим сукупним капіталом експлуатувати стільки ж праці, як і раніш. Такі
є протилежні тенденції, які, викликаючи підвищення норми додаткової вартості, одночасно викликають
падіння маси додаткової вартості, вироблюваної даним капіталом, а тому й падіння норми зиску. Тут
слід також згадати про масове вживання жіночої і дитячої праці, тому що при цьому вся сім’я мусить
давати капіталові більшу масу додаткової праці, ніж раніше, навіть якщо зростає загальна сума
заробітної плати, виплачуваної цій сім’ї, — випадок, який аж ніяк не є загальним явищем. — Такий
самий вплив справляє все те, що сприяє при незмінній величині застосовуваного капіталу виробництву
відносної додаткової вартості шляхом самого тільки поліпшення методів, як у землеробстві. Хоча тут
застосовуваний сталий капітал не зростає в порівнянні з змінним, оскільки ми розглядаємо цей
останній як показник уживаної робочої сили, але маса продукту зростає порівняно з ужитою робочою
силою. Те саме має місце, коли продуктивна сила праці (однаково, чи входить її продукт у споживання
робітників, чи в елементи сталого капіталу) звільняється від перешкод, що утруднюють зносини, від
самовільних обмежень або від обмежень, які стали обтяжливими з бігом часу, взагалі від усякого роду
пут, при чому цим не зачіпається відношення змінного капіталу до сталого.

Можна було б поставити питання: чи входять у число тих причин, які гальмують падіння норми зиску,
але в кінцевому рахунку завжди прискорюють його, тимчасові, але що завжди повторюються, виявляються
то в одній, то в другій галузі виробництва підвищення додаткової вартості понад загальний рівень для
капіталіста, який використовує винаходи і т. д., поки вони ще не стали загальнопоширеними. На це
питання треба відповісти позитивно.

Маса додаткової вартості, вироблена капіталом даної величини, є добуток двох множників — норми
додаткової вартості, помноженої на число робітників, занятих при даній нормі. Отже, при даній нормі
додаткової вартості маса її залежить від числа
робітників, а при даному числі робітників — від норми додаткової вартості, тобто взагалі від
складного відношення абсолют-
