\parcont{}  %% абзац починається на попередній сторінці
\index{iii1}{0109}  %% посилання на сторінку оригінального видання
провітрюється погано. В ній є два вікна, що відчиняються, і камін,
але забитий; будь-яких спеціальних вентиляційних пристроїв немає“ (стор. 27).

Той самий звіт зауважує щодо надмірної праці модисток:
„Надмірна праця молодих жінок панує у фешенебельних модних майстернях тільки протягом 4 приблизно
місяців на рік,
але в такій потворній мірі, що це в багатьох випадках викликало хвилинне здивування й незадоволення
публіки; протягом
цих місяців у майстерні звичайно працюють повних 14 годин
щодня, а при скупченні спішних замовлень 17—18 годин на день.
В інші пори року в майстерні працюють, мабуть, 10—14 годин;
ті, що працюють дома, працюють регулярно 12 або 13 годин.
У виробництві дамських мантильок, комірців, сорочок і т. д.
число годин праці в спільній майстерні, включаючи й працю на
швацькій машині, менше і не перевищує здебільшого 10—12“;
але, каже доктор Орд, „в деяких майстернях регулярний робочий час у певні періоди здовжується окремо
оплачуваними надурочними годинами, а в інших майстернях беруть роботу додому, щоб закінчити її після
звичайного робочого часу: і та
і друга форма надмірної праці, можемо додати, часто є примусова“ (стор. 28). Джон Сімон зауважує в
примітці до цієї сторінки: „Пан Редкліф, секретар епідеміологічного товариства,
який мав особливо багато нагод досліджувати здоров’я модисток
у майстернях першого типу, знайшов на кожних 20 дівчат, які
самі вважали себе „цілком здоровими“, тільки одну здорову;
у решти виявились різні ступені фізичної перевтоми, нервового
виснаження і численних викликаних цим функціональних розладів“.
За причини цього він вважає: насамперед довжину робочого
дня, яку він визначає мінімум у 12 годин на день навіть для
тихого сезону; по-друге, „переповнення і погане провітрювання
майстерень, зіпсоване газовими ріжками повітря, недостатнє
або погане харчування і недостатнє піклування про домашній
комфорт“.

Висновок, до якого приходить голова англійського санітарного відомства, такий: „Для робітників
практично неможливо
настояти на тому, що теоретично є їх найелементарнішим правом
на здоров’я, а саме настояти, щоб підприємець, який збирає
їх для виконання будь-яких робіт, своїм власним коштом усував,
оскільки це від нього залежить, всі не необхідні в цій спільній
роботі умови, які шкідливо впливають на здоров’я; і що в той
час, як самі робітники фактично неспроможні добитись для себе
цієї санітарної справедливості, вони так само мало можуть —
не зважаючи на гаданий намір законодавця — сподіватись будь-якої ефективної допомоги від тих
урядовців, які повинні проводити в життя Nuisances Removal Acts [закони для усунення антисанітарного
стану]“ (стор. 29). — „Без сумніву, визначення точних меж, поза якими підприємці мусять підлягати
регулюванню, становитиме деякі дрібні технічні труднощі. Але... в принципі вимога
\parbreak{}  %% абзац продовжується на наступній сторінці
