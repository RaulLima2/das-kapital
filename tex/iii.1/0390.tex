[Ця шахрайська процедура практикувалась доти, поки товари з Індії та до
Індії мусили на парусниках обходити мис Доброї Надії. З того часу, як товари
стали відправляти через Суецький канал і при тому на пароплавах, цей метод
здобування фіктивного капіталу втратив свою основу: довгочасність транспортування товарів. А з того
часу, як телеграф почав у той самий день давати відомості англійському купцеві про стан індійського
ринку і індійському торговцеві про стан англійського ринку, цей метод став зовсім неможливим. — Ф.
Е.].

III. Нижченаведене взято з цитованого вже звіту „Commercial Distress“,
1847—48: „В останній тиждень квітня 1847 року Англійський банк повідомив
Royal Bank of Liverpool, що з цього моменту він наполовину зменшує свої дисконтні операції з ним. Це
повідомлення справило дуже поганий вплив, тому
що платежі в Ліверпулі за останній час далеко більше провадились векселями,
ніж готівкою, і тому що купці, які звичайно для оплати своїх акцептів вносили
в банк багато грошей готівкою, за останній час могли вносити тільки векселі,
які вони самі одержували за свою бавовну та інші продукти. Це явище дуже
поширилось, і разом з ним збільшились і труднощі в справах. Акцепти, які
банк повинен був оплачувати за купців, здебільшого видавались за кордоном
і досі покривалися в більшості випадків платежами, одержаними за продукти. Векселі, що їх тепер
подавали купці, замість колишніх грошей готівкою, були
різних строків і різного роду; значна частина їх складалася з банкових векселів на три місяці dato
[з дня видачі], велика кількість векселів була видана
під бавовну. Ці векселі акцептувались лондонськими банкірами, якщо вони були
банковими векселями, а в противному разі — всякого роду купцями, бразільськими,
американськими, канадськими, вест-індськими і т. д. фірмами... Купці не видавали векселів один на
одного, а клієнти всередині країни, які купували продукти в Ліверпулі, оплачували їх векселями на
лондонські банки, або векселями на інші фірми в Лондоні, або векселями на кого-небудь іншого.
Повідомлення Англійського банку привело до того, що для векселів під продані іноземні продукти був
скорочений строк, який до того часто перевищував три місяці“ (стор. [1], 2, 3).

Період процвітання 1844—1847 рр. в Англії, як відзначено вище, був зв’язаний
з першою великою залізничною гарячкою. Про вплив її на справи взагалі згаданий звіт каже таке: „У
квітні 1847 року майже всі торговельні фірми почали
більш чи менш виснажувати свої підприємства (to starve their business), вкладаючи частину свого
торговельного капіталу в залізниці“ (стор. 18). — „Під залізничні акції брались також і позики за
високі проценти, наприклад, по 8\%,
у приватних осіб, банкірів та страхових товариств“ (стор. 42 [43]). „Такі великі
авансування цих торговельних фірм на залізниці знов таки примушували їх брати
в банків за допомогою дисконту векселів занадто багато капіталу, щоб на нього
продовжувати ведення свого власного підприємства“ (стор. 43). — (Запитання:) „Чи
сказали б ви, що внески за залізничні акції значно сприяли пригніченню, яке
панувало“ [на грошовому ринку] „в квітні й жовтні [1847 р.]?“ (Відповідь:) „Я гадаю, що навряд чи
вони мали якийсь вплив на пригнічення в квітні. На мою думку,
вони до квітня і, мабуть, аж до літа скоріше підкріпляли, ніж ослабляли банкірів.
Бо дійсне застосування грошей зовсім не відбувалось так само швидко, як надходили внески; в наслідок
цього більшість банків мали в своїх руках на початку
року досить значну суму залізничних фондів“. [Це підтверджуються численними
свідченнями банкірів в комісії „Commercial Distress“ 1848/1857.] „Сума ця літом
помалу зменшувалась і на 31 грудня була значно менша. Одною з причин
пригнічення в жовтні було ступневе зменшення залізничних фондів у руках
банків; між 22 квітня і 31 грудня залізничні сальдо в наших руках зменшились
на третину. Такий вплив мали внески за залізничні акції в усій Великобританії;
вони помалу вичерпали вклади банків“ (стор. 19, 20). — Те саме каже і Samuel
Gurney (шеф відомої фірми Overend Gurney and С°): „В 1846 році був значно
більший попит на капітали для залізниць, але він не підвищив розміру процента. Відбулося злиття
дрібних сум у великі маси, і ці великі маси були витрачені на нашому ринку; так що загалом результат
був той, що на грошовий
ринок Сіті викидалося більше грошей, а звідти вони забиралися не так
швидко“ [стор. 135].
