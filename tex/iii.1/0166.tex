стину сукупного суспільного капіталу, вкладеного в сукупне
виробництво.\footnote{
Cherbuliez [„Riche ou Pauvre“, Paris-Genève 1840, стор. 116 і далі].
}

Отже, якщо капіталіст продає свій товар по його ціні виробництва, то він повертає собі кількість
грошей, відповідну величині вартості спожитого ним у виробництві капіталу, і добуває зиск
пропорціонально до його авансованого капіталу, просто
як до певної частини сукупного суспільного капіталу. Витрати
виробництва в кожній сфері виробництва мають специфічний
характер. Доданий до цих витрат виробництва зиск не залежить від його окремої сфери виробництва, він
є проста пересічна
на кожні 100 авансованого капіталу.

Припустімо, що п’ять різних капіталів І—V у вищенаведеному прикладі належать одній людині. Кількість
змінного і сталого капіталу, спожита на виробництво товарів у кожному окремому підрозділі І—V на
кожні 100 застосованого капіталу,
є дана; ця частина вартості товарів І—V, само собою зрозуміло, становитиме частину їх ціни, бо
принаймні ця ціна потрібна для заміщення авансованої і спожитої частини капіталу. Отже, ці витрати
виробництва були б різні для кожного роду
товарів І—V і, як такі, вони були б по-різному фіксовані їх власником. Що ж до різних мас додаткової
вартості або зиску, вироблених у підрозділах І—V, то капіталіст мав би всі підстави вважати їх за
зиск на весь свій авансований капітал, так що
на кожні 100 одиниць капіталу припадала б певна відповідна
частина. Отже, витрати виробництва товарів, вироблених в окремих підрозділах І—V, були б різні; але
в усіх цих товарів
була б однаковою частина продажної ціни, яка походить з доданого до витрат виробництва зиску на
кожні 100 одиниць капіталу. Отже, сукупна ціна товарів І—V дорівнювала б їх сукупній вартості, тобто
дорівнювала б сумі витрат виробництва
І—V плюс сума додаткової вартості, або зиску, вироблена в
І—V; отже, в дійсності ця ціна була б грошовим виразом сукупної кількості минулої і новододаної
праці, вміщеної в товарах
І—V. І таким чином, у самому суспільстві — якщо розглядати
всі галузі виробництва в їх сукупності — сума цін виробництва
вироблених товарів дорівнює сумі їх вартостей.

Цьому твердженню, здається, суперечить той факт, що в
капіталістичному виробництві елементи продуктивного капіталу
звичайно купуються на ринку, отже, ціни їх містять у собі вже
реалізований зиск, тобто ціну виробництва певної галузі промисловості разом з уміщеним в ній зиском,
так що зиск однієї
галузі промисловості входить у витрати виробництва іншої.
Але якщо ми підрахуємо на одному боці суму витрат виробництва товарів цілої країни, а на другому —
суму її зиску або додаткової вартості, то, очевидно, матимемо правильний обрахунок. Візьмімо,
наприклад, товар А; нехай витрати його