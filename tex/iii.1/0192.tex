репродукції в наслідок нагромадження капіталу, то, при інших
незмінних умовах, потрібна додаткова кількість бавовни. Те саме
і щодо засобів існування. Робітничий клас, для того, щоб і далі
жити при звичайних пересічних умовах, мусить діставати принаймні
попередню кількість необхідних засобів існування, хоч,
може, і розподілених дещо інакше між різними сортами товарів;
якщо ж узяти до уваги щорічний ріст населення, то потрібна
ще певна додаткова кількість засобів існування; те саме
з більшими чи меншими модифікаціями можна сказати і щодо
інших класів.

Отже, здається, ніби на стороні попиту є певна, даної величини
суспільна потреба, яка для свого задоволення вимагає
певної кількості товару на ринку. Але кількісна визначеність
цієї потреби цілком еластична й хитка. Вона тільки здається
фіксованою. Якби засоби існування були дешевші або грошова
заробітна плата була вища, то робітники купували б більше
засобів існування і виявилася б більша „суспільна потреба“ на
ці сорти товарів, — при чому ми зовсім залишаємо осторонь
пауперів і т. д., „попит“ яких стоїть нижче найвужчих меж їх
фізичної потреби. Коли б, з другого боку, подешевшала, наприклад,
бавовна, то попит капіталістів на бавовну виріс би, в бавовняну
промисловість було б вкладено більше додаткового
капіталу і т. д. При цьому взагалі не слід забувати, що попит
на продуктивне споживання при нашому припущенні є попит
з боку капіталіста і що справжня мета капіталіста є виробництво
додаткової вартості, так що він тільки з цією метою
виробляє певний сорт товарів. З другого боку, це не перешкоджає
тому, що капіталіст, оскільки він виступає на ринку як
покупець, наприклад, бавовни, репрезентує потребу в бавовні,
адже і для продавця бавовни байдуже, чи перетворює покупець
цю бавовну в сорочки, в бавовняний порох, чи має намір
затикати нею вуха собі і всьому світові. Але в усякому разі це
справляє великий вплив на те, якого роду покупець він є. Його
потреба в бавовні істотно модифікується тією обставиною, що
в дійсності вона тільки приховує його потребу добувати зиск. —
Межі, в яких репрезентована на ринку потреба в товарах —
попит — кількісно відрізняється від дійсної суспільної потреби,
звичайно, дуже різні для різних товарів; я маю на увазі ріжницю
між кількістю товарів, на яку є попит, і тією кількістю
їх, на яку був би попит при інших грошових цінах товарів або
при інших грошових або життьових умовах покупців.

Нема нічого легшого, як зрозуміти нерівномірності попиту
й подання та відхилення, що випливають звідси, ринкових цін
від ринкових вартостей. Справжня трудність полягає у визначенні
того, що слід розуміти під висловом: попит і подання
покриваються.

Попит і подання покриваються, якщо вони стоять у такому
відношенні одне до одного, що товарна маса певної галузі ви-
