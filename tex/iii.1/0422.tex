мерційного циклу. Разом з тим значно зростають доходи капіталістів. Споживання всюди підвищується.
Товарні ціни також
регулярно підвищуються, принаймні, в різних вирішальних галузях підприємств. В наслідок цього
зростає кількість циркулюючих грошей, принаймні в певних межах, тому що більша
швидкість циркуляції в свою чергу ставить межі зростанню кількості засобів, що циркулюють. Через те
що частина суспільних
доходів, яка складається з заробітної плати, первісно авансується промисловим капіталістом у формі
змінного капіталу
і завжди у формі грошей, то в періоди процвітання вона потребує більше грошей для своєї циркуляції.
Але ми не повинні
рахувати їх двічі: раз як гроші, потрібні для циркуляції змінного капіталу, і ще раз як гроші,
потрібні для циркуляції доходу робітників. Гроші, які виплачуються робітникам як заробітна плата,
витрачаються в роздрібній торгівлі і таким чином
приблизно щотижня повертаються назад до банків як вклади роздрібних торговців, після того, як вони
обслужать ще в дрібних
кругобігах різного роду побічні справи. В періоди процвітання
зворотний приплив грошей до промислових капіталістів відбувається гладко і таким чином їх потреба в
грошових позиках
зростає не через те, що вони повинні виплатити більше заробітної плати, потребують більше грошей для
циркуляції їх змінного капіталу.

Загальний результат є той, що в періоди процвітання кількість засобів циркуляції, яка служить для
витрачання доходів,
рішуче зростає.

Що ж до циркуляції, потрібної для передачі капіталу, отже,
тільки між самими капіталістами, то цей час жвавих справ
є разом з тим періодом найеластичнішого і найлегшого кредиту. Швидкість циркуляції між капіталістом
і капіталістом
регулюється безпосередньо кредитом, і кількість засобів циркуляції, яка потрібна для сальдування
платежів і навіть для купівель за готівку, таким чином порівняно зменшується. Абсолютно
вона може збільшитись, але відносно вона при всіх обставинах зменшується, порівняно з розширенням
процесу репродукції. З одного боку, великі масові платежі ліквідуються без усякого посередництва
грошей; з другого боку, при великому пожвавленні процесу, панує прискорений рух тієї самої кількості
грошей, — як в їх функції засобу купівлі, так і в їх функції засобу платежу. Та сама кількість
грошей опосереднює зворотний
приплив більшої кількості окремих капіталів.

Загалом, в такі періоди грошовий обіг заповнений (full),
хоч частина II (передача капіталу) скорочується, принаймні
відносно, тимчасом як частина І (витрачання доходів) абсолютно
розширюється.

Зворотні припливи грошей виражають зворотне перетворення
товарного капіталу в гроші, Г — Т — Г', як ми це бачили при
дослідженні процесу репродукції, книга II, відділ І. Завдяки кре-
