дження і таємниця її існування завуальовані і стерті. Справді,
зиск є форма виявлення додаткової вартості, і ця остання тільки
за допомогою аналізу може бути вилущена з першої. В додатковій
вартості відношення між капіталом і працею оголене; у
відношенні капіталу й зиску, — тобто капіталу і додаткової вартості,
якою вона виступає, з одного боку, як реалізований у
процесі циркуляції надлишок понад витрати виробництва товару,
а з другого боку, як надлишок, ближче визначений його
відношенням до всього капіталу, — капітал виступає як відношення
до себе самого, як відношення, в якому він як первісна
сума вартості відрізняється від нової вартості, створеної ним
самим. Що він створює цю нову вартість під час свого
руху через процес виробництва і процес циркуляції, — це є в свідомості.
Але як це стається, це тепер містифіковано і, як
здається, походить від таємних властивостей, належних самому
капіталові.

Чим далі ми стежимо за процесом зростання вартості капіталу,
тим більше містифікується капіталістичне відношення і тим
менше розкривається таємниця його внутрішнього організму.

В цьому відділі норма зиску чисельно відрізняється від
норми додаткової вартості; навпаки, зиск і додаткова вартість
розглядаються як одна й та сама числова величина, тільки
в різній формі. В дальшому відділі ми побачимо, як відчужування
йде далі і як зиск і чисельно виражається як величина,
відмінна від додаткової вартості.

Розділ третій

Відношення норми зиску до норми
Додаткової вартості

Як відзначено наприкінці попереднього розділу, ми припускаємо
тут, як і взагалі в усьому цьому першому відділі, що
сума зиску, яка припадає на даний капітал, дорівнює всій сумі
додаткової вартості, виробленої за допомогою цього капіталу
протягом даного періоду циркуляції. Отже, покищо ми залишаємо
осторонь те, що, з одного боку, ця додаткова вартість
розпадається на різні підвиди (Unterformen): процент на капітал,
земельна рента, податки і т. д., і що вона, з другого боку,
в більшості випадків зовсім не збігається з зиском, як він привласнюється
в силу загальної пересічної норми зиску, про яку
буде мова в другому відділі.

Оскільки зиск припускається кількісно рівним додатковій
вартості, його величина і величина норми зиску визначається відношеннями
простих числових величин, які в кожному окремому
випадку дані або можуть бути визначені. Отже, дослідження
рухається спочатку в чисто математичній галузі.
