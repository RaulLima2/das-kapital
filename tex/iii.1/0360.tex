Тепер дуже просто виявляються причини, чому цей поділ
гуртового зиску на процент і підприємницький дохід, раз він
став якісним, зберігає цей характер якісного розподілу для сукупного
капіталу і для всього класу капіталістів.

Поперше: це випливає вже з тієї простої емпіричної обставини,
що більшість промислових капіталістів, хоч і в різних числових
відношеннях, працює як власним, так і взятим у позику
капіталом, і що відношення між власним і взятим у позику капіталом
у різні періоди міняється.

Подруге: перетворення однієї частини гуртового зиску в форму і
процента перетворює другу його частину в підприємницький
дохід. Справді, цей останній є тільки та протилежна форма, що
її набирає надлишок гуртового зиску понад процент, коли процент
існує як особлива категорія. Все дослідження того, яким
чином гуртовий зиск поділяється на процент і підприємницький
дохід, зводиться просто до дослідження того, яким чином частина
гуртового зиску взагалі закостеніває і усамостійнюється
як процент. Але капітал, що дає процент, історично існує як
готова, успадкована форма, а тому й процент як готова підформа
додаткової вартості, виробленої капіталом, існує задовго
до того, як з’являються капіталістичний спосіб виробництва і
відповідні йому уявлення про капітал і зиск. Тому грошовий капітал,
капітал, що дає процент, в народному уявленні все ще лишається
капіталом як таким, капіталом par excellence. Звідси, з
другого боку, і те уявлення, що панувало до часів Мессі, ніби процентом
оплачуються гроші як такі. Та обставина, що даний у
позику капітал дає процент незалежно від того, чи застосовується
він дійсно як капітал, чи ні, — дає процент навіть тоді, коли він береться
в позику тільки для споживання, — зміцнює уявлення про
самостійність цієї форми капіталу. Найкращий доказ тієї самостійності,
з якою, в перші періоди капіталістичного способу
виробництва, процент виступає відносно зиску і капітал, що дає
процент, відносно промислового капіталу, полягає в тому, що
тільки в середині XVIII століття був відкритий (Мессі і за ним
Юмом) той факт, що процент є просто частина гуртового зиску,
і що взагалі доводилось робити таке відкриття.

Потретє: чи працює промисловий капіталіст власним чи
взятим у позику капіталом, це нічого не міняє в тій обставині,
що йому протистоїть клас грошових капіталістів як особливий,
вид капіталістів, грошовий капітал — як самостійний вид капіталу,
і процент — як відповідна цьому специфічному капіталові самостійна
форма додаткової вартості.

Розглядуваний щодо якості, процент є додаткова вартість,
яку дає просто власність на капітал, яку дає капітал сам по собі,
хоча власник його лишається поза процесом репродукції, — яку,
отже, дає капітал відокремлено від свого процесу.

Розглядувана щодо кількості, частина зиску, яка становить
процент, виступає так, ніби вона відноситься не до промисло-
