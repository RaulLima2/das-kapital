\parcont{}  %% абзац починається на попередній сторінці
\index{iii1}{0011}  %% посилання на сторінку оригінального видання
першого нарису всюди, де це дозволяла ясність, не викреслював
навіть окремих повторень там, де вони, як це звичайно маємо
у Маркса, кожного разу розглядають предмет з іншого боку
абож принаймні передають його в іншому способі виразу. Там, де
мої зміни або додатки не чисто редакційного характеру, або
там, де я мусив переробити даний Марксом фактичний матеріал
у власні кінцеві висновки, хоч і витримані по можливості в дусі
Маркса, — все місце взято в прямі дужки і позначено моїми ініціалами.
При моїх примітках під текстом подекуди немає дужок;
але там, де при них стоять мої ініціали, я відповідаю за всю
примітку.

В рукопису — як це само собою зрозуміло для першого нарису
— є численні вказівки на ті пункти, які треба було пізніше
розвинути, при чому не в усіх випадках виконано ці обіцянки.
Ці вказівки я зберіг, бо вони показують наміри автора щодо
майбутнього розроблення.

А тепер до деталей.

Для першого відділу головний рукопис можна було використати
тільки з великими обмеженнями. На самому початку його
вміщено все математичне обчислення відношення між нормою додаткової
вартості і нормою зиску (що становить наш 3 розділ),
тимчасом як предмет, викладений у нашому 1 розділі, розглядається
тільки пізніше і мимохідь. Тут стали в пригоді два
початки перероблення, кожний у 8 сторінок in folio; але й вони
не скрізь зв’язно розроблені. З них складається даний у цій
книзі розділ 1. Розділ 2 взято з головного рукопису. Для
розділу 3 був у наявності цілий ряд незакінчених математичних
обчислень, а також цілий майже закінчений зшиток, з семидесятих
років, який подавав у рівняннях відношення норми додаткової
вартості до норми зиску. Мій друг Самюель Мур, який зробив
також більшу частину англійського перекладу першої книги,
взявся обробити для мене цей зшиток, до чого він, як старий
кембріджський математик, був куди більше здатен. З його резюме
я потім виготував 3 розділ, користуючись принагідно головним
рукописом. — З розділу 4 був у наявності тільки заголовок.
Але тому що розглядуваний тут пункт: вплив обороту на
норму зиску, має вирішально важливе значення, то я розробив
його сам, через що весь цей розділ в тексті взято в дужки.
При цьому виявилось, що в дійсності формула розділу 3 для
норми зиску потребувала деякої модифікації для того, щоб стати
загальнозначимою. Починаючи з п’ятого розділу, головний рукопис
є єдине джерело для решти відділу, хоч і тут треба
було зробити дуже багато перестановок і доповнень.

Для дальших трьох відділів, не кажучи про стилістичну редакцію,
я майже скрізь міг триматися оригіналу рукопису. Окремі
місця, які здебільшого стосуються до впливу обороту, треба
було обробити відповідно до вставленого мною 4 розділу; ці
місця теж узяті в дужки і позначені моїми ініціалами.
