\parcont{}  %% абзац починається на попередній сторінці
\index{iii1}{0162}  %% посилання на сторінку оригінального видання
сферах однакові при рівновеликих затратах капіталу, як би не
відрізнялись між собою вироблені вартості і додаткові вартості.
Ця рівність витрат виробництва становить базу конкуренції капіталовкладень, за допомогою якої
встановлюється пересічний зиск.

Розділ дев’ятий

Утворення загальної норми зиску (пересічної норми зиску) і перетворення вартостей товарів у ціни
виробництва

Органічний склад капіталу в кожний даний момент залежить
від двох обставин: по-перше, від технічного відношення між уживаною робочою силою і масою
застосовуваних засобів виробництва; подруге, від ціни цих засобів виробництва. Його, як ми бачили,
треба розглядати в його процентному відношенні. Органічний склад капіталу, який складається на 4/5 з
сталого і на 1/5 з змінного капіталу, ми виражаємо формулою 80 c + 20 v. Далі, при
порівнянні припускається незмінна і при тому довільна норма
додаткової вартості, наприклад, в 100\%. Отже, капітал в 80 c + 20 v дає додаткову вартість в 20 m,
що становить норму зиску
в 20\% на весь капітал. Величина дійсної вартості його продукту
залежить від величини основної частини сталого капіталу і від
того, скільки з неї входить в продукт як зношування і скільки
не входить. Але через те що ця обставина зовсім не має значення для норми зиску, отже, і для даного
дослідження, то для
спрощення ми припускаємо, що сталий капітал повсюди однаково цілком входить у річний продукт цих
капіталів. Ми припускаємо далі, що капітали в різних сферах виробництва у
відношенні до величини їх змінної частини реалізують щорічно
однакову кількість додаткової вартості; отже, ми покищо залишаємо осторонь ту ріжницю, яку тут може
викликати ріжниця
в періодах обороту. Цей пункт розглядається пізніше.

Візьмімо п’ять різних сфер виробництва з різним органічним
складом вкладених у них капіталів, наприклад:

Капітали    Норма додаткової вартості    Додаткова  вартість    Вартість продукту   Норма  зиску
І. 80 c + 20 v    100\%    20    120    20\%
II. 70 c + 30 v   100\%    30    130    30\%
III. 60 c + 40 v   100\%    40   140    40\%
IV. 85 c + 15 v   100\%    15    115    15\%
V. 95 c + 5 v     100\%    5    105    5\%

Ми маємо тут для різних сфер виробництва при однаковій
експлуатації праці дуже різні норми зиску, відповідно до різного органічного складу капіталів.

Сукупна сума капіталів, вкладених у цих п’яти сферах, = 500;
сукупна сума виробленої ними додаткової вартості = 110; сукупна вартість вироблених ними товарів =
610. Якщо ми розглядатимем 500 як один єдиний капітал, що з нього І — V становлять
\parbreak{}  %% абзац продовжується на наступній сторінці
