а весь річний продукт при п’яти оборотах:

1000 с (зношування) + 5000 с + 5000 v + 5000 m = 16 000,

К = 11 000, m = 5000, р' = 5000/11000 = 45 5/11\%.

Візьмімо далі капітал III, в якому зовсім немає основного капіталу,
але є 6000 обігового сталого і 5000 змінного капіталу.
При нормі додаткової вартості в 100\% він обертається один раз
на рік. Тоді весь продукт за рік буде:

6000 с + 5000 v + 5000 m = 16 000,

К = 11000, m = 5000, р' = 5000/11000 = 45 5/11\%.

Отже, в усіх трьох випадках ми маємо однакову річну масу
додаткової вартості = 5000, а через те що весь капітал в усіх
трьох випадках теж однаковий, а саме = 11 000, то маємо
й однакову норму зиску в 45 5/11\%.

Навпаки, якщо при капіталі І ми мали б не 10, а тільки
5 річних оборотів змінної частини, то справа стояла б інакше.
Тоді продукт одного обороту був би:

200 с (зношування) + 500 с + 500 v + 500 m = 1700.

Або річний продукт:

1000 с (зношування) + 2500 с + 2500 v + 2500 m = 8500,

К = 11 000, m = 2500, р' = 2500/11000 = 22 8/11\%.

Норма зиску знизилася б наполовину, бо час обороту подвоївся.

Отже, маса додаткової вартості, привласнювана протягом року,
дорівнює масі додаткової вартості, привласнюваній за один період
обороту змінного капіталу, помноженій на число таких оборотів
за рік. Якщо привласнювану за рік додаткову вартість або зиск
ми назвемо М, привласнювану за один період обороту додаткову
вартість — m, число річних оборотів змінного капіталу — n, то
М = mn, а річна норма додаткової вартості М' = m'n, як це
вже показано в книзі II, розд. XVI, 1.

Само собою зрозуміло, що формула норми зиску р' = m' v/К =
m' v/с + v правильна тільки тоді, коли v чисельника однакове
з v знаменника. У знаменнику v є вся та частина всього капіталу,
яка пересічно застосована як змінний капітал на заробітну
плату; v чисельника насамперед визначається тільки тим, що
воно виробило і привласнило певну кількість додаткової вар-
