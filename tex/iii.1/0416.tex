мали ще деякий сенс, стають тут цілком безглуздими. Удача
і невдача однаково ведуть тут до централізації капіталів,
а тому й до експропріації в найколосальнішому масштабі. Експропріація поширюється тут з
безпосередніх виробників на
самих дрібних і середніх капіталістів. Ця експропріація — вихідний пункт капіталістичного способу
виробництва; її здійснення — його мета, і при тому в кінцевому рахунку експропріація всіх індивідів
від засобів виробництва, які з розвитком
суспільного виробництва перестають бути засобами приватного
виробництва і продуктами приватного виробництва і можуть
бути засобами виробництва тільки в руках асоційованих виробників, отже, їхньою суспільною власністю,
як вони є їхнім суспільним
продуктом. Але ця експропріація в межах самої капіталістичної
системи виражається в антагоністичній формі, як привласнення
суспільної власності небагатьма; а кредит дедалі більше надає
цим небагатьом характеру чистих авантюристів. Через те що
власність існує тут у формі акцій, її рух і передача стають
чистим результатом біржової гри, де дрібну рибу пожирають
акули, а овець — біржові вовки. В акційній справі уже існує
протилежність старій формі, в якій суспільні засоби виробництва
виступають як індивідуальна власність; але само перетворення
у форму акції лишається ще обмеженим капіталістичними рамками; тому замість того, щоб перебороти
суперечність між
характером багатства, як суспільного і як приватного багатства, воно тільки розвиває її в новій
формі.

Кооперативні фабрики самих робітників, в межах старої
форми, є перший пролом старої форми, хоч вони повсюди,
в своїй дійсній організації, звичайно репродукують і мусять
репродукувати всі хиби існуючої системи. Але протилежність між
капіталом і працею в межах цих фабрик знищена (ist... aufgehoben),
хоч спочатку тільки в такій формі, що робітники, як асоціація,
є своїм власним капіталістом, тобто застосовують засоби виробництва для використання своєї власної
праці. Ці фабрики показують, як на певному ступені розвитку матеріальних продуктивних сил і
відповідних їм суспільних форм виробництва з одного
способу виробництва з природною необхідністю розвивається і
виробляється новий спосіб виробництва. Без фабричної системи,
виниклої з капіталістичного способу виробництва, кооперативна фабрика не могла б розвинутися; і так
само вона не могла б
розвинутися без кредитної системи, виниклої з цього ж самого
способу виробництва. Кредитна система, подібно до того, як
вона становить головну основу ступневого перетворення капіталістичних приватних підприємств у
капіталістичні акційні товариства, дає так само засоби для ступневого розширення кооперативних
підприємств у більш-менш національному масштабі.
Капіталістичні акційні підприємства, цілком так само як і кооперативні фабрики, слід розглядати як
перехідні форми від капіталістичного способу виробництва до асоційованого, тільки що
