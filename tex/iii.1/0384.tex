них різноманітними шляхами. Насамперед, в їхніх руках, оскільки
вони є касирами промислових капіталістів, концентрується грошовий
капітал, який кожен виробник і купець тримає як резервний
фонд або який припливає до нього по платежах. Ці
фонди перетворюються таким чином у позиковий грошовий
капітал. Завдяки цьому резервний фонд торговельного світу,
через те що він концентрується як суспільний, обмежується
необхідним мінімумом, і частина грошового капіталу, яка інакше
лежала б без діла як резервний фонд, віддається в позику,
функціонує як капітал, що дає процент. Подруге, позиковий
капітал банків утворюється з вкладів грошових капіталістів, які
передають їм справу віддачі їх у позику. Далі, з розвитком банкової
системи, а саме, як тільки банки починають платити проценти
по вкладах, до них вкладаються грошові заощадження
і тимчасово вільні гроші всіх класів. Дрібні суми, з яких кожна
сама по собі нездатна діяти як грошовий капітал, з’єднуються
у великі маси і таким чином утворюють грошову силу. Це нагромадження
дрібних сум, як особливий результат банкової системи,
слід відрізняти від її посередницької ролі між власне грошовими
капіталістами і позичальниками. Нарешті, в банки вкладаються
й доходи, які мають споживатися тільки поступінно.

Віддача в позику (тут ми маємо справу тільки з власне торговельним
кредитом) відбувається за допомогою дисконту векселів
— перетворення їх у гроші до скінчення їх строку — і за
допомогою позик у різних формах: прямих позик під особистий
кредит, позик під заставу процентних паперів, державних фондів,
акцій усякого роду, особливо ж позик під накладні, докові
варанти та інші засвідчені документи про право власності на
товари, під вклади і т. д.

Кредит же, що його дає банкір, може даватися в різних формах
— наприклад, векселями на інші банки, чеками на них, відкриттям
кредиту того самого роду, нарешті, у банків, що випускають
банкноти, власними банкнотами банку. Банкнота є не
що інше, як вексель на банкіра, по якому пред’явник його може
в перший-ліпший час одержати гроші і яким банкір заміняє приватні
векселі. Ця остання форма кредиту здається профанові
особливо разючою і важливою, поперше, тому що цього роду
кредитні гроші переходять з простої торговельної циркуляції
в загальну циркуляцію і функціонують тут як гроші; а також
і тому, що в більшості країн головні банки, які випускають
банкноти, являючи собою дивну мішанину національного банку
і приватного банку, в дійсності мають за собою національний
кредит, а їх банкноти є більш чи менш законний засіб платежу;
тому що тут стає очевидним, що те, чим торгує банкір, є сам
кредит, бо банкнота представляє тільки кредитний знак, який перебуває
в циркуляції. Але банкір торгує кредитом і в усіх інших формах,
навіть коли дає готівкою в позику покладені до нього вкладом
гроші, В дійсності банкнота становить монету тільки для гур-
