кової вартості, передбачає, принаймні до певної міри, що ступінь
експлуатації праці або норма додаткової вартості однакова, або
що існуючі в цьому відношенні ріжниці вирівнюються за допомогою
дійсних або уявних (умовних) компенсуючих причин.
Це передбачає конкуренцію між робітниками і вирівнювання
ступеня їх експлуатації в наслідок постійного переходу їх
з однієї сфери виробництва до іншої. Така загальна норма додаткової
вартості — як тенденція, подібно до всіх економічних
законів, — припускається нами як теоретичне спрощення; але
в дійсності вона є фактична передумова капіталістичного способу
виробництва, хоч вона й гальмується в більшій чи меншій
мірі практичними тертями, які викликають більш чи менш
значні місцеві ріжниці, такі є, наприклад, закони про осілість
(settlement laws) для землеробських поденників в Англії. Але
в теорії припускається, що закони капіталістичного способу
виробництва розвиваються в чистому вигляді. В дійсності існує
завжди тільки наближення; однак, це наближення тим більше,
чим більше розвинений капіталістичний спосіб виробництва і чим
більше усунене його забарвлення рештками попередніх економічних
становищ і переплетення з ними.

Вся трудність постає з того, що товари обмінюються не
просто як товари, а як продукти капіталів, які претендують
на пропорціональну до їх величини або, при рівній величині, на
рівну участь у сукупній масі додаткової вартості. І сукупна
ціна товарів, вироблених даним капіталом за даний період часу,
повинна задовольнити цю вимогу. Але сукупна ціна цих товарів
є просто сума цін окремих товарів, які становлять продукт
капіталу.

Punctum saliens [вирішальний пункт] виступить найбільше, якщо
ми підійдемо до справи так: Припустім, що самі робітники
володіють своїми відповідними засобами виробництва і обмінюють
свої товари один з одним. Ці товари не були б тоді
продуктами капіталу. Залежно від технічної природи їх робіт,
вартість засобів праці і матеріалів праці, застосовуваних у різних
галузях праці, була б різна; так само, незалежно від неоднакової
вартості застосовуваних засобів виробництва, потрібна була б
різна маса цих засобів виробництва для даної маси праці, залежно
від того, що один певний товар може бути виготовлений
за одну годину, а інший тільки за день і т. д. Припустімо
далі, що ці робітники пересічно працюють однакову кількість
часу, враховуючи вирівнення, які випливають з різної інтенсивності
праці та ін. Двоє робітників замістили б тоді в товарах,
що становлять продукт їх денної праці, поперше, свої
видатки, витрати (die Kostpreise) на спожиті засоби виробництва.
Ці останні були б різні залежно від технічної природи їх галузей
праці. Подруге, вони обидва створили б однакові кількості
нової вартості, а саме робочий день, доданий ними до засобів
виробництва. Ця нова вартість містила б у собі їх заробітну
