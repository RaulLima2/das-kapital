\index{iii1}{0080}  %% посилання на сторінку оригінального видання
II. m' змінюється

Загальну формулу норм зиску при різних нормах додаткової
вартості, однаково, чи  v/K лишається незмінним, чи теж змінюється,
ми одержимо, коли рівняння:

p' = m' v/К

перетворимо в інше:

p'1 = m' 1 v1/K1,

де р'1, m'1, v1, і К1 означають змінені величини р', m', v і К.
Тоді ми маємо:

p': p'1 = m' v/K: m'1 v1/K1,

і звідси:

p'1 = m'1/m' × v1/v × K/K1 × p'.

1. m' змінюється, v/K не змінюється

В цьому випадку ми маємо рівняння:

p' = m' v/K; p'1 = m'1 v/K,

в обох рівняннях v/K має однакову величину. Тому одержуємо
відношення:

р': р'1 = m': m'1.

Норми зиску двох капіталів однакового складу відносяться
одна до одної, як відповідні норми додаткової вартості. Через
те що в дробу v/K важливі не абсолютні величини v і К, а тільки
відношення між ними, то це стосується й до всіх капіталів однакового
складу, яка б не була їх абсолютна величина.

80с + 20v + 20m; K = 100, m' = 100\%, p' = 20\%
160c + 40v + 20m; K = 200, m' = 50\%, p' = 10\%
100\%: 50\% = 20\%: 10\%.

Якщо абсолютні величини v і К в обох випадках однакові,
то норми зиску відносяться одна до одної, крім того, як маси
додаткової вартості:

p': p'1 = m'v: m'1v = m: m1.
