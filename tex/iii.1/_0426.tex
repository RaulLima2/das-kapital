\parcont{}  %% абзац починається на попередній сторінці
\index{iii1}{0426}  %% посилання на сторінку оригінального видання
з якою подання віддає себе в такі періоди в розпорядження
попиту, приводить до кредитної скрути під час застою. Отже,
не ріжниця у величині попиту на позики характеризує обидва
періоди.

Як уже раніш відзначено, обидва періоди відрізняються насамперед
тим, що в часи процвітання переважає попит на
засоби циркуляції між споживачами і торговцями, а в період
застою — попит на засоби циркуляції між капіталістами. В період
застою в справах перший попит скорочується, другий — збільшується.

Фуллартонові та іншим здається вирішально важливим те явище,
що в такі часи, коли securities [забезпечення, цінні папери] — застави
під позики і векселі — в Англійському банку збільшуються,
циркуляція його банкнот зменшується, і навпаки. Але сума securities
виражає розмір грошових позик, дисконтованих векселів і
позик під ходові цінні папери. Так Фуллартон у наведеному вище
місці, примітка 91, стор. 423 і далі, каже: цінні папери (securities),
які перебувають у володінні Англійського банку, здебільшого змінюються
в напрямі, протилежному до циркуляції його банкнот,
і це підтверджує випробуване приватними банками положення,
що ніякий банк не може підвищити видачу своїх банкнот
понад певну суму, визначувану потребами його клієнтів; якщо ж
банк хоче давати позики понад цю суму, то він мусить давати
їх з свого капіталу, отже, або пустити в оборот цінні папери,
або вживати для цього грошові надходження, які він інакше
вклав би в цінні папери.

Але тут виявляється також, що Фуллартон розуміє під капіталом.
Що зветься тут капіталом? Що банк не може більше
видавати позики своїми власними банкнотами, обіцянками платежу,
які йому, звичайно, нічого не коштують. Але з чого ж
він тоді дає позики? З виручки від продажу securities in reserve
[резервних цінних паперів], тобто державних паперів, акцій та
інших процентних паперів. Але за що він продає ці папери?
За гроші, золото або банкноти, оскільки ці останні є законним
засобом платежу, як банкноти Англійського банку. Отже, те,
що він віддає в позику, є при всяких обставинах гроші. Але
ці гроші становлять тепер частину його капіталу. Якщо він дає
в позику золото, то це очевидно. Якщо — банкноти, то тепер ці
банкноти представляють капітал, бо банк за них віддав в обмін
дійсну вартість, процентні папери. У приватних банків банкноти,
які припливають до них в наслідок продажу цінних паперів,
можуть бути в переважній масі тільки банкнотами Англійського
банку або їх власними банкнотами, бо інші навряд чи беруться
в оплату цінних паперів. Якщо ж це сам Англійський банк,
то його власні банкноти, які він одержує назад, коштують
йому капіталу, тобто процентних паперів. Крім того, він таким
способом вилучає з циркуляції свої власні банкноти. Якщо ж
він знову випускає ці банкноти або замість них нові на таку
\parbreak{}  %% абзац продовжується на наступній сторінці
