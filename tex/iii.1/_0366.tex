\parcont{}  %% абзац починається на попередній сторінці
\index{iii1}{0366}  %% посилання на сторінку оригінального видання
в особисту владу відносно праці і над працею. Він представляє
просту власність на капітал як засіб привласнювати собі продукти
чужої праці. Але він представляє цей характер капіталу як
щось таке, що належить йому поза процесом виробництва і що
зовсім не є результатом специфічно капіталістичної визначеності
самого цього процесу виробництва. Він представляє це не як
пряму протилежність до праці, а, навпаки, без відношення до
праці і як просте відношення одного капіталіста до другого.
Отже, як властивість, зовнішню і байдужу для відношення капіталу
до самої праці. Отже, в проценті, в цій особливій формі
зиску, в якій антагоністичний характер капіталу надає собі самостійного
виразу, він надає собі його таким чином, що цей антагонізм
тут цілком стирається і відбувається цілковите абстрагування
від нього. Процент є відношення між двома капіталістами,
а не між капіталістом і робітником.

З другого боку, ця форма процента надає другій частині
зиску якісної форми підприємницького доходу, далі плати за
нагляд. Особливі функції, які доводиться виконувати капіталістові
як такому і які припадають йому якраз у відміну від робітників
і в протилежність до робітників, зображаються як прості трудові
функції. Він створює додаткову вартість не тому, що працює
\emph{як капіталіст}, а тому, що він, незалежно від його якості як
капіталіста, \emph{теж} працює. Отже, ця частина додаткової вартості
зовсім не є вже додаткова вартість, а її протилежність, еквівалент
за виконану працю. Через те що характер відчуженості капіталу,
його протилежність до праці, переноситься по той бік дійсного
процесу експлуатації, а саме переноситься на капітал, що дає
процент, то сам цей процес експлуатації здається простим процесом
праці, в якому функціонуючий капіталіст виконує тільки
іншу працю, ніж робітник. Так що праця експлуатації і праця, яка
експлуатується, будучи та й друга працею, є тотожні. Праця
експлуатації цілком так само є праця, як і та праця, яка експлуатується.
Процент стає суспільною формою капіталу, але вираженою
в нейтральній і індиферентній формі; підприємницький
дохід стає економічною функцією капіталу, але абстрагованою
від певного, капіталістичного характеру цієї функції.

В свідомості капіталіста тут відбувається цілком те саме,
що із згаданими у відділі II цієї книги підставами компенсації
при вирівненні зиску в пересічний зиск. Ці підстави компенсації,
які визначально впливають на розподіл додаткової вартості,
перекручуються при капіталістичному способі уявлення
в причини виникнення і (суб’єктивні) підстави виправдання самого
зиску.

Уявлення про підприємницький дохід як про плату за працю
нагляду, яке виникає з протилежності між підприємницьким доходом
і процентом, знаходить собі дальшу точку опори в тому, що
частина зиску дійсно може бути відокремлена і справді відокремлюється
як заробітна плата або, точніше, навпаки, що частина
\parbreak{}  %% абзац продовжується на наступній сторінці
