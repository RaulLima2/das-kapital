багатство світу, з якого походить дохід, уже давно стало процентом
на капітал... всяка рента є тепер сплатою процентів на
капітал, раніше вкладений у землю“] („Economist", 19 Juli 1859).
Капіталові в його властивості як капіталу, що дає процент, належить
усе багатство, яке тільки взагалі може бути вироблене,
і все, що він одержував досі, це тільки платежі в розстрочку його
всезахоплюючому апетитові. За природженими йому законами
йому належить уся додаткова праця, яку тільки може колибудь
дати рід людський. Молох.

На закінчення ще така нісенітниця „романтичного“ Мюллера:
„Потворне зростання процентів на проценти д-ра Прайса, або
потворне зростання людських сил, які самі себе прискорюють,
передбачає для того, щоб викликати такі величезні наслідки,
незмінний або непорушний одноманітний порядок протягом багатьох
століть. Як тільки капітал розділяється, роздрібнюється
на багато окремих паростків, які самі по собі продовжують
зростати, загальний, описаний тут процес нагромадження сил
починається знову. Природа розподілила наростання сил на
періоди приблизно в 20—25 років, які пересічно припадають
на кожного окремого робітника (!). Після того, як цей час
минає, робітник покидає свій життьовий шлях і мусить передати
капітал, придбаний за допомогою процентів на проценти від
праці, новому робітникові, здебільшого розподілити його між
кількома робітниками або дітьми. Ці останні, раніше ніж вони
зможуть добувати власне проценти на проценти від капіталу,
що дістався їм, мусять спочатку навчитися оживляти його або
застосовувати. Далі, величезна маса капіталу, що його здобуває
громадянське суспільство, навіть у найрухливіших суспільствах,
поволі нагромаджується протягом довгих років і не
застосовується для безпосереднього розширення праці, а, навпаки,
як тільки назбирається значна сума, вона під назвою „позики“
передається іншому індивідові, робітникові, банкові, державі,
при чому одержувач цієї суми, пускаючи капітал у дійсний рух,
одержує з нього проценти на проценти і може легко зобов’язатись
платити позикодавцеві прості проценти. Нарешті, величезній
прогресії, в якій можуть збільшуватись сили людей та їх
продукт, якщо діє самий тільки закон виробництва або ощадливості,
— протидіє закон споживання, жадання, марнотратства“
(Adam Müller: „Die Elemente der Staatskunst“. Berlin 1809, III,
стор. 147—149).

Неможливо в небагатьох рядках нагородити більше найжахливішого
безглуздя. Не кажучи вже про кумедне змішання
робітника з капіталістом, вартості робочої сили з процентом
на капітал і т. д., зменшення процента на процент має бути
між іншим пояснене з того, що капітал „віддається в позику“ для
того, щоб він давав „тоді проценти на проценти“. Метод нашого
Мюллера характерний для романтики всяких професій. Зміст її
складається з ходячих передсудів, почерпнутих з найповер-
