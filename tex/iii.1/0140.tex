1861 рік. Жовтень. „Стан справ з деякого часу був дуже
пригнічений... Немає нічого неймовірного, що протягом зимніх
місяців багато фабрик дуже скоротять робочий час. Це, зрештою,
можна було передбачити... цілком незалежно від тих причин,
які припинили наш звичайний довіз бавовни з Америки і наш вивіз,
скорочення робочого часу протягом наступної зими стало б
необхідним в наслідок сильного збільшення виробництва за
останні три роки і в наслідок порушень на індійському й китайському
ринках“ („Rep. of Insp. of Fact., Oct. 1861“, стор. 19).

Бавовняні відпади. Ост-індська бавовна (Surat). Вплив на заробітну плату
робітників. Поліпшення машин. Заміна бавовни крохмальним борошном і
мінералами. Вплив цього шліхтування крохмальним борошном на робітників.
Прядільники тонких нумерів пряжі. Ошуканство фабрикантів

„Один фабрикант пише мені таке: „Щодо оцінки споживання
бавовни на одно веретено, то ви, мабуть, не досить берете до
уваги той факт, що коли бавовна дорога, кожен прядільник
звичайної пряжі (скажімо, до № 40, переважно № 12—32) пряде
такі тонкі нумери, які тільки може, тобто він прястиме № 16
замість попереднього № 12 або № 22 замість № 16 і т. д.,
і ткач, який тче з цієї тонкої пряжі, доведе свій ситець до
звичайної ваги, додаючи до нього відповідно більше шліхти.
Цим способом користуються тепер в справді ганебних розмірах.
Я чув з надійного джерела, що є звичайний Shirting [тканина
для сорочок] для експорту вагою в 8 фунтів штука, з яких
2 3/4 фунти є шліхта. В тканинах інших сортів часто є до 50%
шліхти, так що фабрикант аж ніяк не бреше, коли вихваляється,
що він багатіє, продаючи фунт своєї тканини дешевше, ніж він
заплатив за пряжу, з якої ця тканина зроблена“ („Rep. of Insp.
of Fact., April 1864“, стор. 27).

„Мені також казали, що ткачі приписують свою підвищену
захворюваність шліхті, яку застосовують для основи, випряденої
з ост-індської бавовни, і яка вже не складається, як раніше,
з чистого борошна. Однак, цей сурогат борошна дає, як кажуть,
ту велику вигоду, що він значно збільшує вагу тканини, так
що з 15 фунтів пряжі стає 20 фунтів тканини“ („Rep, of Insp.
of Fact., Oct. 1863“, стор. 63. Цим сурогатом був перемолотий
тальк, називаний China clay [китайською глиною], або гіпс, називаний
French chalk [французькою крейдою].) — „Заробіток ткачів
(тобто тут робітників) дуже зменшується в наслідок застосовування
сурогатів борошна для шліхтування основи. Ця шліхта
робить пряжу важчою, але також твердою і ламкою. Кожна
нитка основи проходить у ткацькому верстаті через так званий
реміз, міцні нитки якого тримають основу в правильному положенні;
твердо нашліхтована основа спричинює постійні розриви
ниток у ремізі; при кожному розриві ткач втрачає п’ять хвилин
на виправлення; тепер ткачеві доводиться виправляти такі
пошкодження принаймні в 10 разів частіше, ніж раніш, і вер-
