дення самостійного промислового підприємства, виявляється
так: як тільки нове дорожче промислове устаткування стає
загальнопоширеним, дрібніші капітали на майбутнє виключаються
з цього виробництва. Тільки на перших порах механічних винаходів
у різних сферах виробництва дрібніші капітали можуть
в них самостійно функціонувати. З другого боку, дуже великі
підприємства, з надзвичайно високим відношенням сталого капіталу,
як залізниці, дають не пересічну норму зиску, а тільки
частину її, процент. Інакше загальна норма зиску знизилась би
ще більше. Навпаки, і тут великі капітали, зібрані в формі акцій,
знаходять собі поле для безпосереднього застосування.

Зростання капіталу, отже, нагромадження капіталу, включає
зменшення норми зиску лиш остільки, оскільки разом з цим зростанням
настають розглянуті нами вище зміни у відношенні органічних
складових частин капіталу. Однак, не зважаючи на постійні,
повсякденні перевороти в способі виробництва, та чи інша,
більша чи менша частина всього капіталу протягом певного
часу продовжує нагромаджуватися на базі даного пересічного відношення
цих складових частин, так що з зростанням цієї частини
не сполучена ніяка органічна переміна, отже й ніякі причини
падіння норми зиску. Це постійне збільшення капіталу, а тому
й розширення виробництва на основі старих методів виробництва,
яке спокійно триває далі, тимчасом як поряд з ними
вводяться вже нові методи, знов таки є причиною того, що
норма зиску зменшується не в тій мірі, в якій зростає сукупний
капітал суспільства.

Збільшення абсолютного числа робітників, не зважаючи на
відносне зменшення змінного капіталу, витрачуваного на заробітну
плату, відбувається не в усіх галузях виробництва і не
в усіх рівномірно. В землеробстві зменшення елементу живої
праці може бути абсолютним.

Зрештою, абсолютне збільшення числа найманих робітників,
не зважаючи на його відносне зменшення, є тільки потреба капіталістичного
способу виробництва. Для нього робочі сили
стають уже зайвими, як тільки немає вже необхідності примушувати
їх працювати 12—15 годин на день. Розвиток продуктивних
сил, який зменшував би абсолютне число робітників, тобто
в дійсності давав би змогу всій нації виконувати своє сукупне
виробництво за коротший час, викликав би революцію, бо він
вивів би в тираж більшість населення. В цьому знову виявляється
специфічна межа капіталістичного виробництва, а також
те, що воно ніяк не є абсолютною формою для розвитку продуктивних
сил і створення багатства, що воно, навпаки, на певному
пункті вступає в колізію з цим розвитком. Частково така колізія
виявляється в періодичних кризах, які походять з того, що
то одна, то друга частина робітничого населення робиться зайвою
в своїй старій професії. Межа капіталістичного виробництва
— надлишковий час робітників. Абсолютний надлишковий
