\parcont{}  %% абзац починається на попередній сторінці
\index{iii1}{0323}  %% посилання на сторінку оригінального видання
багато галузей підприємств, незалежних одна від одної. Одно
підприємство виробляє тільки стільці, друге — тільки столи,
третє — тільки шафи і т. д. Але самі ці підприємства провадяться
більш чи менш ремісницьким способом, дрібним майстром з небагатьма
підмайстрами. Проте, виробництво це занадто масове для
того, щоб можна було працювати безпосередньо на окремих
приватних осіб. Покупці тут — власники мебльових крамниць.
Щосуботи майстер іде до них і продає свій продукт, при чому
про ціну торгуються цілком так само, як у ломбарді про позику
під ту чи іншу річ. Ці майстри мусять щотижня продавати вже
для того, щоб мати можливість знову купити для наступного
тижня сировинний матеріал і виплатити заробітну плату. При
таких обставинах вони власне є тільки посередники між купцем
і своїми власними робітниками. Власне капіталістом є купець,
який кладе собі в кишеню більшу частину додаткової вартості.\footnote{
З 1865 року ця система розвинулася в ще більшому масштабі. Докладніше
про це див. у „First Report from the Select Committee of the House of Lords on
the Sweating System". London 1888. — Ф. E.
}
Аналогічно стоїть справа при переході в мануфактуру тих галузей,
які раніше провадились ремісницьким способом або як
побічні галузі сільської промисловості. Залежно від технічного
розвитку цього дрібного самостійного виробництва, — коли воно
вже застосовує машини, які можливі при ремісничому виробництві,
— відбувається також перехід до великої промисловості;
машина приводиться в рух уже не рукою, а парою, як це, наприклад,
останнього часу робиться в англійському панчішному
виробництві.

Отже, перехід відбувається трояким способом: Поперше,
купець стає прямо промисловцем; це має місце в галузях промисловості,
основаних на торгівлі, особливо в галузях предметів
розкоші, галузях, які разом з сировинним матеріалом і робітниками
імпортуються купцями зза кордону, як у п’ятнадцятому
столітті в Італію з Константинополя. Подруге, купець робить
своїми посередниками (middlemen) дрібних майстрів або купує
також безпосередньо у самостійного виробника; номінально
він лишає його самостійним і лишає незмінним його спосіб виробництва.
Потрете, промисловець стає купцем і безпосередньо
виробляє у великому масштабі для торгівлі.

В середні віки купець, як правильно каже Поп, є тільки
„скупщик“ товарів, вироблених чи цеховими ремісниками, чи то
селянами. Купець стає промисловцем або, точніше, примушує
працювати на себе ремісничу, особливо ж сільську дрібну промисловість.
З другого боку, виробник стає купцем. Наприклад,
майстер-сукнороб замість того, щоб потроху, невеличкими
порціями, одержувати вовну від купця і разом з своїми підмайстрами
працювати на нього, сам купує вовну або пряжу і
продає своє сукно купцеві. Елементи виробництва входять у
\parbreak{}  %% абзац продовжується на наступній сторінці
