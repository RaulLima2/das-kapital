ми припустимо, що, крім цих 900 фунтів стерлінгів промислового
капіталу, сюди долучається ще 100 фунтів стерлінгів купецького
капіталу, який pro rata [пропорціонально] своїй величині має
таку саму частку в зиску, як і той. Згідно з припущенням,
купецький капітал становить 1/10 сукупного капіталу в 1000.
Отже, з сукупної додаткової вартості в 180 йому припадає 1/10,
і таким чином він одержує зиск нормою в 18%. Отже, зиск,
який належить поділити між рештою — 9/10 сукупного капіталу,
фактично дорівнює вже тільки 162, або на капітал в 900 він так
само = 18%. Отже, ціна, по якій Т продається володільцями промислового
капіталу в 900 торговцям товарами, = 720c + 180v + 162m = 1062.
Отже, якщо купець накине на свій капітал
в 100 пересічний зиску 18%, то він продасть товари за 1062 + 18 = 1080,
тобто по їх ціні виробництва, або — якщо розглядати
сукупний товарний капітал — по їх вартості, хоч він добуває
свій зиск тільки в циркуляції і за допомогою циркуляції, і тільки
в наслідок перевищення його продажної ціни над його купівельною
ціною. Але все ж він продає товари не вище їх вартості
або не вище їх ціни виробництва саме тому, що він купив їх
у промислових капіталістів нижче їх вартості або нижче їх ціни
виробництва.

Таким чином, в утворення загальної норми зиску купецький
капітал входить як визначальний фактор pro rata тій частині,
яку він становить у сукупному капіталі. Отже, якщо в наведеному
випадку ми кажемо: пересічна норма зиску = 18%, то вона
була б = 20%, якби 1/10 сукупного капіталу не була купецьким
капіталом і якби в наслідок цього загальна норма зиску не знизилася
на 1/10. Разом з цим з’являється точніше, обмежувальне визначення
ціни виробництва. Під ціною виробництва, як і раніш,
слід розуміти ціну товару = його витратам (вартості вміщеного
в ньому сталого + змінного капіталу) + пересічний зиск на них.
Але цей пересічний зиск визначається тепер інакше. Він визначається
сукупним зиском, що його створює сукупний продуктивний
капітал; але обчислюється він не просто на цей сукупний
продуктивний капітал, — так що, коли цей останній, як
припущено вище, = 900, а зиск = 180, то пересічна норма зиску
була б = 180/900 = 20%, — а на сукупний продуктивний капітал + торговельний
капітал, так що, коли продуктивний капітал = 900, а торговельний = 100, то пересічна норма зиску =
180/1000 = 18%.
Отже, ціна виробництва = k (витратам) + 18, замість дорівнювати
k + 20. В пересічній нормі зиску врахована вже та частина
сукупного зиску, яка припадає на торговельний капітал. Тому
дійсна вартість або ціна виробництва сукупного товарного капіталу
= k + р + h (де h є торговельний зиск). Отже, ціна виробництва,
або та ціна, по якій продає промисловий капіталіст як
такий, менша, ніж дійсна ціна виробництва товару; або, якщо
