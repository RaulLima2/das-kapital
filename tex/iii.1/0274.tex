нуючи як така в процесі циркуляції. Це — частина сукупного
капіталу, яка, коли залишити осторонь витрачання доходів,
мусить постійно циркулювати на ринку як купівельний засіб,
щоб підтримувати безперервність процесу репродукції. Вона
тим менша у відношенні до сукупного капіталу, чим швидше
йде процес репродукції і чим більше розвинена функція грошей
як платіжного засобу, тобто чим більше розвинена кредитна система. 38

Купецький капітал є не що інше, як капітал, що функціонує
в сфері циркуляції. Процес циркуляції є фаза сукупного процесу
репродукції. Але в процесі циркуляції не виробляється
ніякої вартості, а тому й ніякої додаткової вартості. В ньому
відбуваються тільки зміни форми тієї самої маси вартості. Справді,
в ньому не відбувається нічого іншого, крім метаморфози товарів,
яка як така не має ніякого відношення до творення вартості
або до зміни вартості. Якщо при продажу виробленого
товару реалізується додаткова вартість, то це тому, що ця
остання вже існує в ньому; тому при другому акті, при зворотному
обміні грошового капіталу на товар (на елементи ви-

38 Для того, щоб мати змогу класифікувати купецький капітал як виробничий
капітал, Рамсей змішує його з транспортною промисловістю і називав
торгівлю „the transport of commodities from one place to another“ [транспортуванням
товарів з одного місця до іншого] („An Essay on the Distribution of
Wealth“, стор. 19). Те саме змішування маємо вже у Веррі („Meditazioni sulla
Economia Politica“, § 4 [Мілано 1804, стор. 32]) і у Сея („Traité d’Economie
Politique“, I, 14—15). — В своїх „Elements of Political Economy“ (Andover і
New Jork 1835) І. P. Newman каже: „In the existing economical arrangements of
society, the very act which is performed by the merchant, of standing between the
producer and the consumer, advancing to the former capital and receiving products
in return, and handing over these j roducts to the latter, receiving back capital in
return, is a transaction which both facilitates the economical process of the community,
and adds value to the products in relation to which it is performed“ [„При існуючому
економічному устрої суспільства дійсний акт, який виконує купець,
стаючи між виробником і споживачем, авансуючи виробникові капітал і одержуючи
в заміну продукти, передаючи потім ці продукти споживачеві і одержуючи
за це від нього знову свій капітал, є операція, яка полегшує економічний
процес суспільства і додає вартість до продуктів, з якими вона проводилась“]
(стор. 174). Таким чином, завдяки посередництву купця виробник і споживач
заощаджують гроші й час. Така послуга вимагає авансування капіталу й праці
і мусить бути оплачена, „since it adds value to products, for the same products,
in the handset cousumers, are worth more than in the hands of producers“ [„бо вона
додає до продуктів вартість, тому що ті самі продукти мають у руках споживачів
більшу вартість, ніж у руках виробників“]. І таким чином торгівля
здається йому, цілком так само, як панові Сеєві, „strictly an act of production“
[в строгому значенні слова актом виробництва] (стор. 175). Цей погляд Ньюмена
цілком хибний. Споживна вартість товару в руках споживача є більша, ніж у
руках виробника, тому що вона взагалі тільки тут реалізується. Адже споживна
вартість товару реалізується, починає виконувати свою функцію тільки тоді,
коли товар переходить у сферу споживання. В руках виробника вона існує лиш
у потенціальній формі. Але товар не оплачують двічі: спочатку його мінову
вартість, а потім ще, крім того, його споживну вартість. Тим, що я оплачую
його мінову вартість, я привласнюю собі його споживну вартість. І мінова вартість
не дістає ні найменшого приросту від того, що товар переходить з рук
виробника або посередника-купця в руки споживача.
