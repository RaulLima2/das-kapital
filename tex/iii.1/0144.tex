ників капітал, а почасти для того, щоб забезпечити собі видушувану
з робітників квартирну плату.

„Комітети допомоги діяли в цій справі з величезною суворістю.
Коли пропонувалась робота, то робітників, яким вона пропонувалась,
викреслювали з списків, і таким способом їх примушували
брати цю роботу. Якщо робітники відмовлялися від такої роботи...
то причина цього була в тому, що їх заробіток був би тільки
номінальним, а праця — надзвичайно тяжкою“ („Rep. of Insp. of
Fact., Oct. 1863“, стор. 97).

Робітники були готові взяти всяку роботу, яку їм давали,
згідно з Public Works Act [законом про громадські роботи].
„Принципи, за якими були організовані промислові роботи, були
дуже різні в різних містах. Але навіть там, де робота під голим
небом не була виключно спробною роботою (labour test), ця
робота оплачувалась або не більше як в розмірі регулярної
допомоги, абож тільки трохи вище, так що фактично вона ставала
спробною роботою“ (стор. 69). „Public Works Act 1863 року
мав допомогти в цій біді і дати робітникові змогу заробляти свою
денну заробітну плату як незалежному поденникові. Мета цього
закону була трояка: 1) дати можливість місцевим властям позичати
гроші (за згодою президента центрального державного
комітету в справі бідних) у комісарів в справах державних позик;
2) полегшити справу впорядкування міст у бавовняних округах;
3) дати безробітним робітникам працю й достатній заробіток
(remunerative wages)“. До кінця жовтня 1863 року на підставі
цього закону дозволено було позик на суму в 883 700 фунтів
стерлінгів (стор. 70). Розпочаті роботи складалися головним
чином з каналізаційних робіт, прокладання шляхів, брукування
вулиць, влаштування водозборів для водяних двигунів і т. д.

Пан Гендерсон, президент блекбернського комітету, пише
з приводу цього до фабричного інспектора Редгрева: „З усього
того, що мені довелося бачити на протязі теперішнього часу
страждань і злиднів, ніщо не вражало мене дужче або не тішило
мене більше, як та бадьора готовість, з якою безробітні
робітники цієї округи бралися до роботи, запропонованої їм
блекбернською міською радою на підставі Public Works Act.
Ледве чи можна собі уявити більший контраст, ніж контраст
між бавовнопрядільником, який раніше був вправним робітником
на фабриці, і тим самим прядільником, що працює тепер як
поденник при копанні сточних каналів глибиною в 14—18 футів“.
[При цьому вони заробляли, залежно від складу родини, 4—12
шилінгів на тиждень; цієї колосальної суми часто мало вистачати
для родини з 8 осіб. Панове міщани мали при цьому подвійну
вигоду: поперше, вони одержували по винятково низьких
процентах гроші для поліпшення своїх димних і занедбаних
міст; подруге, вони платили робітникам далеко менше від звичайної
заробітної плати]. „Робітник, звиклий до майже тропічної
температури, до праці, при якій вправність і точність маніпу-
