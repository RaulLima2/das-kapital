вувати в своєму підприємстві як додатковий капітал і збільшувати
його вартість до повернення його банкові.

Другий випадок. — А заставив банкові цінні папери, зобов’язання
державної позики або акції, і одержав під них позику
готівкою, наприклад, до двох третин їх курсової вартості. В
цьому випадку він одержав засоби платежу, яких він потребує,
але не додатковий капітал, бо він дав у руки банку більшу
капітальну вартість, ніж одержав від нього. Але, з одного боку,
ця більша капітальна вартість не могла бути використана ним
для його потреб поточного моменту, — потреб у засобах платежу,
— бо вона була уже вкладена в певній формі з метою одержання
процента; з другого боку, у А були свої підстави не перетворювати
її безпосередньо в засоби платежу шляхом продажу.
Його цінні папери мали між іншим призначення функціонувати
як резервний капітал, і він використав їх саме у функції резервного
капіталу. Отже, між А і банком відбулася тимчасова взаємна
передача капіталів, при чому А не одержав ніякого додаткового
капіталу (навпаки!), але одержав, звичайно, потрібні йому засоби
платежу. Навпаки, для банку ця операція була тимчасовим тривким
вкладенням грошового капіталу в формі позики, перетворенням
грошового капіталу з однієї форми в другу, а таке
перетворення саме й є істотною функцією банкової справи.

Третій випадок. — А дисконтував у банку вексель і одержав
при цьому, після відрахування дисконту, певну суму готівкою.
В цьому випадку він продав банкові грошовий капітал у нетекучій
формі за суму вартості в текучій формі; вексель, якому
ще не надійшов строк, він продав за готівку. Вексель тепер
є власністю банку. Справа ні трохи не змінюється від того, що
останній індосент, А, в разі вексель не буде оплачений, відповідає
перед банком на суму векселя; цю відповідальність він
поділяє з іншими індосентами і з векселедавцем, від яких він має
право у свій час вимагати повернення відповідної суми. Отже,
тут немає ніякої позики, а цілком звичайна купівля й продаж.
Тому А нічого не повинен сплачувати банкові, банк покриває
свою видачу, інкасуючи вексель, коли настає строк платежу.
В цьому випадку теж відбулася взаємна передача капіталу між А
та банком, і при тому цілком така сама, як при купівлі й продажу
всякого іншого товару, і саме тому А не одержав ніякого
додаткового капіталу. Йому потрібні були, і він одержав засоби
платежу, і одержав він їх завдяки тому, що банк перетворив
для нього одну форму його грошового капіталу, вексель, у
другу форму, в гроші.

Отже, про дійсне авансування, про дійсну позику капіталу
може бути мова тільки в першому випадку. В другому ж і третьому
випадку хіба тільки в тому розумінні, як „авансують капітал“
при всякому капіталовкладенні. В цьому розумінні банк
авансує, позичає клієнтові А грошовий капітал; але для А він
є грошовий капітал щонайбільше в тому розумінні, що він
