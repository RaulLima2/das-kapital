де в обох випадках m' = 100%, р' = 10%, і де порівняння з попереднім
капіталом є далеко наочнішим у процентній формі.

Навпаки, коли йдеться про зміни, які відбуваються з одним
і тим самим капіталом, то процентну форму можна вживати
тільки зрідка, бо вона майже завжди стирає ці зміни. Якщо
капітал переходить з процентної форми:

80 с + 20 v + 20 m

у процентну форму:

90 с + 10 v + 10 m,

то не видно, чи цей змінений процентний склад 90 c + 10 v виник
у наслідок абсолютного зменшення в чи в наслідок абсолютного
збільшення с, чи в наслідок того й другого. Для цього ми мусимо
знати абсолютні числові величини. Але для дослідження
дальших окремих випадків змін все зводиться до того, яким
чином сталися ці зміни: чи 80 с + 20 v обернулись у 90 с + 10 v
через те що, скажімо, 12 000 с + 3000 v в наслідок збільшення
сталого капіталу при незмінній величині змінного капіталу перетворилися
в 27 000 с + 3000 v (в процентах 90 с + 10v); чи вони
набрали цієї форми при незмінному сталому капіталі в наслідок
зменшення змінного капіталу, тобто в наслідок переходу у
12000 c + 1333 1/3 v (в процентах так само 90 с + 10 v); чи, нарешті,
в наслідок зміни обох доданків, скажімо, 13 500 с + 1500 v
(в процентах знову 90 с + 10 v). Але ми будемо досліджувати
якраз всі ці випадки один за одним, і тому нам доведеться
відмовитись від зручностей процентної форми або застосовувати
її тільки в другу чергу.

1. m' і К не змінюються, v змінюється

Якщо v змінює свою величину, К може лишитися незмінним
тільки в наслідок того, що друга складова частина К, а саме
сталий капітал с, змінює свою величину на таку саму суму, що й v,
але в протилежному напрямі. Якщо К спочатку = 80 с + 20 = 100
і якщо потім v зменшується до 10, то К може лишитися = 100
тільки тоді, коли с підвищується до 90; 90 с + 10 v = 100. Загалом
кажучи: якщо v перетворюється у v +- d, у v, збільшене або
зменшене на d, то, щоб були задоволені умови даного випадку,
с мусить перетворитися в с -+ d, мусить змінитися на таку саму
суму, але в протилежному напрямі.

Цілком так само при незмінній нормі додаткової вартості m',
але при змінній величині змінного капіталу v, маса додаткової
вартості m мусить змінитися, бо m = m'v, а в m'v один множник,
v, набуває іншого значення.

Припущення нашого випадку поряд первісного рівняння:

р' = m' (v/К)
