Далі, якщо частина сукупного суспільного капіталу, яка постійно
мусить бути в розпорядженні як грошовий капітал, щоб процес
репродукції не переривався процесом циркуляції, а тривав безперервно,
— якщо цей грошовий капітал не створює ні вартості,
ні додаткової вартості, то він не може набути цих властивостей
від того, що для виконання тих самих функцій він постійно кидатиметься
в циркуляцію не промисловими капіталістами, а капіталістами
іншого підрозділу. Наскільки купецький капітал може
бути посередньо продуктивним, про це ми вже згадували і пізніше
розглянемо ще докладніше.

Отже, товарно-торговельний капітал, — якщо відкинути всі
гетерогенні функції, які можуть бути з ним зв’язані, як зберігання,
відправка, транспортування, розділ, роздрібнення, і обмежитись
його справжньою функцією купівлі ради продажу, —
не створює ні вартості, ні додаткової вартості, а тільки опосереднює
їх реалізацію, а разом з тим і дійсний обмін товарів, перехід
їх з одних рук в інші, суспільний обмін речовин. Однак,
тому що фаза циркуляції промислового капіталу цілком так само
становить фазу процесу репродукції, як і виробництво, то капітал,
який самостійно функціонує в процесі циркуляції, мусить
цілком так само давати пересічний річний зиск, як і капітал,
що функціонує в різних галузях виробництва. Коли б купецький
капітал давав у процентах вищий пересічний зиск, ніж
промисловий капітал, то частина промислового капіталу перетворилася
б у купецький капітал. Коли б він давав зиск нижчий
за пересічний, то відбувся б протилежний процес. Частина
купецького капіталу перетворилася б у промисловий капітал.
Жодний рід капіталу не може міняти свого призначення, своєї
функції з більшою легкістю, ніж купецький капітал.

Через те що купецький капітал сам не створює ніякої додаткової
вартості, то ясно, що додаткова вартість, яка припадає
йому в формі пересічного зиску, становить частину додаткової
вартості, створеної сукупним продуктивним капіталом. Але питання
тепер полягає ось у чому: яким чином купецький капітал
притягає до себе ту частину створеної продуктивним капіталом
додаткової вартості, або зиску, яка йому припадає?

Це тільки зовнішня видимість, ніби торговельний зиск є проста
надбавка, номінальне підвищення ціни товарів понад їх вартість.

Ясно, що купець може одержувати свій зиск тільки з ціни
товарів, які він продає, і ще ясніше, що цей зиск, одержуваний
ним при продажу своїх товарів, мусить дорівнювати ріжниці
між його купівельною ціною і його продажною ціною, дорівнювати
надлишкові цієї останньої порівняно з першою.

Можливо, що після купівлі товару і до його продажу в нього
входять додаткові витрати (витрати циркуляції), і так само можливо,
що таких витрат не доводиться на нього робити. Якщо
такі витрати входять у нього, то ясно, що надлишок продажної
ціни понад купівельну ціну представляє не самий тільки зиск.
