таліст одержує у своєму продукті, поперше, заміщення свого
авансованого капіталу, подруге, додатковий продукт, за який
він нічого не заплатив. Але, щоб одержати цей додатковий
продукт, він мусить авансувати свій капітал на виробництво;
тобто він мусить застосувати певну кількість упредметненої
праці, щоб мати можливість привласнити собі цей додатковий продукт.
Отже, для капіталіста цей його авансований капітал є кількість
упредметненої праці, суспільно-потрібна для того, щоб
створити йому цей додатковий продукт. Це має силу і для всякого
іншого промислового капіталіста. А тому що за законом
вартості продукти обмінюються один на один пропорціонально
до праці, суспільно-необхідної для їх виробництва, і що для капіталіста
праця, необхідна для виготовлення його додаткового продукту,
є якраз нагромаджена в його капіталі минула праця, то
з цього випливає, що додаткові продукти обмінюються пропорціонально
до капіталів, потрібних для їх виробництва, а не пропорціонально
до дійсно втіленої в них праці. Отже, частка, що
припадає на кожну одиницю капіталу, дорівнює сумі всіх вироблених
додаткових вартостей, поділеній на суму застосованих
для цього капіталів. Тому однакові капітали за однакові проміжки
часу дають однаковий зиск, і це спричинюється тим, що
вираховані так витрати виробництва (Kostpreis) додаткового
продукту, тобто пересічний зиск, додаються до витрат виробництва
оплаченого продукту, і по цій підвищеній ціні продаються
обидва, оплачений і неоплачений продукт. Встановлюється
пересічна норма зиску, не зважаючи на те, що, як
думає Шмідт, пересічні ціни окремих товарів визначаються за
законом вартості.

Конструкція надзвичайно дотепна, вона цілком на гегелівський
зразок, але вона має те спільне з більшою частиною гегелівського,
що вона неправильна. Додатковий продукт чи оплачений
продукт — це не робить ріжниці: якщо закон вартості повинен
безпосередньо мати силу і для пересічних цін, то і той і другий
продукт мусить продаватися пропорціонально до суспільнонеобхідної
праці, потрібної і спожитої на їх виготовлення. Закон
вартості з самого початку спрямований проти погляду, який перейшов
від капіталістичного способу уявлення, ніби нагромаджена
минула праця, з якої складається капітал, є не просто певна сума
готової вартості, а, як фактор виробництва й утворення зиску, має
також властивість створювати вартості, отже, є джерелом більшої
вартості, ніж має сам капітал; закон вартості твердо встановлює,
що ця властивість належить тільки живій праці. Те, що капіталісти
сподіваються рівного зиску, пропорціонального до величини
їх капіталів, отже, розглядають авансовані ними капітали як свого
роду витрати виробництва їхнього зиску — це відомо. Але якщо
Шмідт використовує це уявлення, щоб за його допомогою погодити
з законом вартості ціни, обраховані за пересічною нормою
зиску, то він скасовує (hebt... auf) самий закон вартості, при-
