Щоб спростити дослідження, ми насамперед припустимо, що
ніяких таких витрат у товар не входить.

Для промислового капіталіста ріжниця між продажною ціною
і купівельною ціною його товарів дорівнює ріжниці між їх ціною
виробництва і їх витратами виробництва, або, якщо розглядати
сукупний суспільний капітал, дорівнює ріжниці між
вартістю товарів і їх витратами виробництва для капіталістів,
що знов таки зводиться до ріжниці між сукупною кількістю
упредметненої в них праці і кількістю упредметненої в них
оплаченої праці. Товари, куплені промисловим капіталістом, пророблюють,
раніше ніж вони знову будуть кинуті на ринок як
продажні товари, процес виробництва, в якому тільки й виробляється
та складова частина їх ціни, яка пізніше має бути
реалізована як зиск. Але для торговця товарами справа стоїть
інакше. Товари перебувають у його руках тільки доти, поки
вони перебувають у процесі своєї циркуляції. Він тільки продовжує
їх продаж, початий продуктивним капіталістом, продовжує
реалізацію їх ціни, і тому не піддає їх проміжному процесові,
в якому вони могли б знову вбирати додаткову вартість. Тимчасом
як промисловий капіталіст у циркуляції тільки реалізує
раніш вироблену додаткову вартість або зиск, купець, навпаки,
повинен у циркуляції і за допомогою циркуляції не тільки реалізувати
свій зиск, але ще тільки створити його. Це можливо,
видимо, тільки завдяки тому, що товари, продані йому промисловим
капіталістом по цінах їх виробництва, або — якщо розглядати
сукупний товарний капітал — по їх вартостях, він продає
дорожче їх цін виробництва, робить номінальну надбавку до
їх цін, отже — якщо розглядати сукупний товарний капітал —
продає їх вище їх вартості і кладе собі в кишеню цей надлишок
їх номінальної вартості порівняно з їх реальною вартістю, — одним
словом, продає їх дорожче, ніж вони варті.

Цю форму надбавки зрозуміти дуже легко; наприклад, один
метр полотна коштує 2 шилінги. Якщо я від перепродажу маю
одержати 10% зиску, то я мушу накинути на ціну 1/10, отже,
мушу продавати метр по 2 шилінги 2 2/5 пенса. Ріжниця між його
дійсною ціною виробництва і його продажною ціною тоді = 2 2/5
пенса, а це становить на 2 шилінги зиск у 10%. В дійсності
я в даному разі продаю покупцеві метр полотна по такій ціні,
яка справді є ціною 1 1/10 метра. Або, що зводиться до того
самого: це цілком однаково, як коли б я продавав покупцеві за
2 шилінги тільки 10/11 метра, а 1/11 лишав би для себе. Справді, на
2 2/5 пенса я можу знову купити 1/11 метра, рахуючи ціну метра
полотна в 2 шилінги 2 2/5 пенса. Отже, це був би тільки обхідний
шлях для того, щоб за допомогою номінального підвищення
ціни товарів взяти участь в одержанні додаткової вартості і додаткового
продукту.

Така є реалізація торговельного зиску за допомогою надбавки
до ціни товарів, як вона виступає на поверхні явищ.
