\parcont{}  %% абзац починається на попередній сторінці
\index{iii1}{0069}  %% посилання на сторінку оригінального видання
через те що v, виплачена заробітна плата, лишається та сама,
то додаткова вартість підвищується з 20 до 40, і ми маємо:\footnote{
с + 12 v + 28 m; m' = 233 1/3\%, р = 28/92 = 30 10/23\%.

Отже, ми бачимо, що як здовження робочого дня (або відповідне
підвищення інтенсивності праці), так і зниження заробітної
плати підвищують масу, а тому й норму додаткової вартості;
навпаки, підвищена заробітна плата, при інших незмінних
умовах, знизила б норму додаткової вартості. Отже, коли
v зростає в наслідок підвищення заробітної плати, то це є вираз
не збільшення, а тільки дорожчої оплати певної кількості праці;
m' і р' не підвищуються, а падають.

Вже тут видно, що зміни в робочому дні, інтенсивності
праці і заробітній платі не можуть настати без одночасної зміни
v  і m та відношення між ними, отже й р' відношення m до
с + v, до всього капіталу; і так само ясно, що зміни відношення
m до v включають також зміну принаймні однієї з трьох щойно
згаданих умов праці.

В цьому виявляється якраз особливе органічне відношення
змінного капіталу до руху всього капіталу і зростання його
вартості, так само як і його відмінність від сталого капіталу.
Сталий капітал, оскільки йдеться про утворення вартості, важливий
тільки тією вартістю, яку він має; при чому для утворення
вартості цілком байдуже, чи сталий капітал у 1500 фунтів стерлінгів
представляє 1500 тонн заліза, скажім, по 1 фунту стерлінгів,
чи 500 тонн заліза по 3 фунти стерлінгів. Кількість дійсних
матеріалів, яку представляє вартість сталого капіталу, не має
ніякого значення для утворення вартості і для норми зиску, яка змінюється
у зворотному напрямі з цією вартістю, однаково, яке б
не було відношення збільшення чи зменшення вартості сталого
капіталу до маси тих речових споживних вартостей, яку він
представляє.

Цілком інакше стоїть справа з змінним капіталом. Тут насамперед
має значення не вартість, яку він має, не праця, яка
в ньому упредметнена, а ця вартість як простий показник усієї
праці, яку він приводить у рух і яка не виражена в ньому;
усієї праці, ріжниця якої від вираженої в ньому самому і, отже, від
оплаченої праці, — та частина всієї праці, яка утворює додаткову
вартість, — є якраз тим більша, чим менша є праця, що міститься
в ньому самому. Нехай робочий день у 10 годин дорівнює
десятьом шилінгам = десятьом маркам. Якщо необхідна праця,
яка заміщує заробітну плату, отже, змінний капітал, дорівнює
} с + 20  v + 40 m; m' = 200\%, р' = 40\%.

Якщо, з другого боку, при десятигодинній праці, заробітна
плата падає з 20 до 12, то ми маємо, як і спочатку, загальну
нововироблену вартість в 40, але розподіляється вона інакше;
v падає до 12 і тому лишає для m решту в 28. Отже, маємо:
\parbreak{}  %% абзац продовжується на наступній сторінці
