\parcont{}  %% абзац починається на попередній сторінці
\index{iii1}{0222}  %% посилання на сторінку оригінального видання
як старий капітал в 1000000 при нормі зиску в 40\%. Коли б він
зріс менше ніж удвоє, то він виробив би менше додаткової
вартості або зиску, ніж раніше виробляв капітал в 1000000,
який для того, щоб при своєму попередньому складі підвищити
свою додаткову вартість з 400000 до 440000, мав би зрости
з 1000000 тільки до 1100000.

Тут виявляється вже раніш викладений закон, що з відносним
зменшенням змінного капіталу, отже, з розвитком суспільної
продуктивної сили праці, потрібна все більше зростаюча маса
всього капіталу, щоб приводити в рух ту саму кількість робочої
сили і вбирати ту саму масу додаткової праці. Отже, в тій
самій мірі, в якій розвивається капіталістичне виробництво, розвивається
можливість появи відносно надлишкового робітничого
населення, не тому, що продуктивна сила суспільної праці \emph{зменшується},
а тому, що вона \emph{збільшується}, тобто не в наслідок
абсолютної невідповідності між працею і засобами існування
або засобами виробництва цих засобів існування, а в наслідок
невідповідності, яка виникає з капіталістичної експлуатації праці,
невідповідності між прогресуючим ростом капіталу і його відносно
меншаючою потребою в зростанні населення.

Якщо норма зиску падає на 50\%, то вона падає наполовину.
Тому, щоб маса зиску лишилась та сама, капітал мусить подвоїтися.
Для того, щоб при падаючій нормі зиску маса зиску лишилась
незмінною, множник, який показує зростання всього капіталу,
має бути рівний дільникові, який показує падіння норми зиску.
Якщо норма зиску падає з 40 до 20, то для того, щоб результат
лишався попереднім, весь капітал мусить, навпаки, зрости у відношенні
20 : 40. Коли б норма зиску впала з 40 до 8, то капітал
мусив би зрости у відношенні 8 : 40, тобто вп’ятеро. Капітал
в 1000000 при 40\% виробляє 400000 і капітал в 5000000 при 8\%
виробляє так само 400000. Таке зростання потрібне для того, щоб
результат лишався попереднім. Навпаки, для того щоб результат
збільшився, капітал мусить зростати в більшій пропорції, ніж
падає норма зиску. Іншими словами: для того, щоб змінна складова
частина всього капіталу не тільки абсолютно лишалась
попередньою, але й абсолютно зростала, хоч її процентне відношення
до всього капіталу падає, весь капітал мусить зростати
в більшій пропорції, ніж падає процентне відношення змінного
капіталу до всього капіталу. Він мусить зростати настільки,
щоб при його новому складі йому потрібна була для купівлі
робочої сили не тільки стара змінна частина капіталу, але ще
більша за неї. Якщо змінна частина капіталу, рівного 100,
падає з 40 до 20, то весь капітал мусить зрости більше, ніж до
200, для того щоб можна було вжити змінний капітал більший,
ніж 40.

Навіть коли б експлуатована маса робітничого населення
лишалась незмінною і збільшилася б тільки довжина і інтенсивність
робочого дня, маса вживаного капіталу мусила б зрости,
\parbreak{}  %% абзац продовжується на наступній сторінці
