\parcont{}  %% абзац починається на попередній сторінці
\index{iii1}{0063}  %% посилання на сторінку оригінального видання
Мальтус, Сеніор, Торренс і т. д., ці явища наводяться безпосередньо
як докази того, ніби капітал просто в своєму речовому
існуванні, незалежно від того суспільного відношення до
праці, в якому він саме й стає капіталом, є, поряд з працею
і незалежно від праці, самостійним джерелом додаткової вартості.
— 2) Під рубрикою витрат, куди належить заробітна плата
цілком так само, як і ціна сировинного матеріалу, зношування
машин і т. д., видушування неоплаченої праці здається тільки
заощадженням на оплаті одного з тих предметів, які входять
у витрати, тільки меншою платою за певну кількість праці;
цілком так само, як відбувається заощадження, коли дешевше
купують сировинний матеріал або зменшують зношування машин.
Таким чином видушування додаткової праці втрачає свій
специфічний характер; його специфічне відношення до додаткової
вартості затемнюється; і цьому затемнінню дуже допомагає
і дуже його полегшує, як показано в книзі І, відділ VI,
представлення вартості робочої сили в формі заробітної плати.

Через те що всі частини капіталу однаково здаються джерелами
надлишкової вартості (зиску), то капіталістичне відношення
містифікується.

Той спосіб, яким додаткова вартість за допомогою переходу
через норму зиску перетворюється в форму зиску, є, однак,
тільки дальший розвиток того переплутання суб’єкта і об’єкта,
яке відбувається уже в процесі виробництва. Вже там ми бачили,
як усі суб’єктивні продуктивні сили праці здаються продуктивними
силами капіталу. З одного боку, вартість, минула праця,
яка панує над живою працею, персоніфікується в капіталісті;
з другого боку, навпаки, робітник виступає просто як предметна
робоча сила, як товар. З цього перекрученого відношення неминуче
виникає вже в самому простому відношенні виробництва
відповідне перекручене уявлення, перенесена з цього відношення
свідомість, яка розвивається далі в наслідок перетворень і модифікацій
власне процесу циркуляції.

Спроба представити закони норми зиску безпосередньо як закони
норми додаткової вартості, або навпаки, є цілком хибна, як
у цьому можна пересвідчитися на прикладі школи Рікардо. В голові
капіталіста, звичайно, ці закони не розрізняються. У виразі m: K
додаткова вартість вимірюється вартістю всього капіталу, авансованого
на її виробництво і почасти в цьому виробництві цілком спожитого,
а почасти тільки застосованого в ньому. Відношення m: K в
дійсності виражає ступінь зростання вартості всього авансованого
капіталу, тобто, взяте відповідно до його раціонального, внутрішнього
зв’язку і природи додаткової вартості, воно показує,
яке є відношення величини, на яку змінився змінний капітал, до
величини всього авансованого капіталу.
