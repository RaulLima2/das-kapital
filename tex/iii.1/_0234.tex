\parcont{}  %% абзац починається на попередній сторінці
\index{iii1}{0234}  %% посилання на сторінку оригінального видання
самим. Ще більше підвищується вона при зростаючому населенні; і хоч це зв’язане з відносним
зменшенням числа занятих робітників порівняно з величиною всього капіталу, проте це зменшення
уміряється або затримується в наслідок підвищення норми додаткової вартості.

Раніше ніж перейти до дальшого пункту, слід ще раз підкреслити, що при даній величині капіталу \emph{норма}
додаткової вартості може зростати, хоч \emph{маса} її падає, і навпаки. Маса додаткової вартості дорівнює
її нормі, помноженій на число робітників; але норма додаткової вартості ніколи не обчислюється на
весь капітал, а тільки на змінний, в дійсності тільки на один робочий день. Навпаки, при даній
величині капітальної вартості \emph{норма зиску} ніколи не може підвищитись або впасти без того, щоб \emph{маса
додаткової вартості} так само не підвищилась або не впала.

\subsection{Зниження заробітної плати нижче її вартості}

Ми подаємо це тут тільки емпірично, бо в дійсності, воно, як і багато чого іншого, що тут слід би
було навести, не має ніякого відношення до загального аналізу капіталу, а стосується до дослідження
конкуренції, яке не входить в завдання цієї праці. Проте, воно є одною з найзначніших причин, які
затримують тенденцію норми зиску до падіння.

\subsection{Здешевлення елементів сталого капіталу}

Сюди стосується все, що було сказано в першому відділі цієї книги про причини, які підвищують норму
зиску при незмінній нормі додаткової вартості або незалежно від норми додаткової вартості, отже, і
той випадок, коли, — якщо розглядати весь капітал, — вартість сталого капіталу зростає не в такій
пропорції, як його матеріальний розмір. Наприклад, маса бавовни, яку переробляє окремий європейський
робітник-прядільник на сучасній фабриці, зросла в найколосальнішій мірі порівняно з тією масою, яку
раніше переробляв європейський прядільник за допомогою прядки. Але вартість перероблюваної бавовни
зросла не в такій пропорції, як її маса. Так само стоїть справа
з машинами та іншим основним капіталом. Коротко кажучи, той самий розвиток, який збільшує масу
сталого капіталу порівняно з змінним, зменшує вартість його елементів в наслідок підвищення
продуктивної сили праці і, значить, перешкоджає тому, щоб вартість сталого капіталу, хоч вона
постійно зростає, зростала в такій самій пропорції, як його матеріальний розмір, тобто матеріальний
розмір засобів виробництва, які приводяться в рух тією самою кількістю робочої сили. В окремих
випадках маса елементів сталого капіталу може навіть збільшитись, тимчасом як його вартість
лишається та сама або навіть падає.
