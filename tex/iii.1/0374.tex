як до основної суми, що виробила додаткову вартість. І, як ми
бачили, капітал як такий виступає для всіх активних капіталістів,
— однаково, чи функціонують вони з своїм власним чи
з узятим в позику капіталом — саме такою безпосередньо самозростаючою
вартістю.

Г — Г': тут ми маємо первісний вихідний пункт капіталу, гроші
у формулі Г — Т — Г', зведені до двох крайніх пунктів Г — Г', де
Г' — Г + ΔГ, гроші, що створюють більшу кількість грошей.
Це — первісна і загальна формула капіталу, скорочена до безглуздого
резюме. Це — готовий капітал, єдність процесу виробництва
і процесу циркуляції, який дає як готовий капітал у
певні періоди часу певну додаткову вартість. У формі капіталу,
що дає процент, це виявляється безпосередньо, без опосереднення
процесом виробництва і процесом циркуляції. Капітал
здається таємничим і самотворчим джерелом процента, свого
власного збільшення. Річ (гроші, товар, вартість) просто як річ
тепер уже є капітал, а капітал здається просто річчю; результат
сукупного процесу репродукції здається властивістю, належною
речі самій по собі; від власника грошей, тобто товару
в його завжди обмінній формі, залежить, чи витратити їх як гроші,
чи віддати в позику як капітал. Тому в капіталі, що дає процент,
цей автоматичний фетиш, самозростаюча вартість, гроші,
що породжують гроші, виробився в чистому вигляді, і в цій
формі він уже не має на собі ніякого сліду свого походження.
Суспільне відношення вивершене як відношення певної речі, грошей,
до самої себе. Замість дійсного перетворення грошей у капітал
тут виявляється тільки беззмістовнаформацьогоперетворення.
Як і в випадку з робочою силою, споживною вартістю грошей
тут стає їх властивість створювати вартість, створювати більшу
вартість, ніж вартість, що міститься в них самих. Гроші як такі
потенціально вже є самозростаюча вартість, і як така вони віддаються
в позику, що є формою продажу для цього своєрідного
товару. Утворювати вартість, давати процент стає такою
самою властивістю грошей, як властивість грушевого дерева
давати груші. І як таку річ, що дає процент, позикодавець
продає свої гроші. Але це ще не все. Як ми бачили, навіть
дійсно функціонуючий капітал виступає таким чином, ніби він
дає процент не як функціонуючий капітал, а як капітал сам по
собі, як грошовий капітал.

Перекручується і ось що: в той час як процент є тільки
частина зиску, тобто додаткової вартості, яку функціонуючий
капіталіст видушує з робітника, він, процент, виступає тепер,
навпаки, як власний плід капіталу, як щось первісне, а зиск,
який перетворився тепер у форму підприємницького доходу,
як простий аксесуар і додаток, який долучається в процесі репродукції.
Тут фетишистична форма капіталу і уявлення про
капітал-фетиш готові. В Г — Г' ми маємо ірраціональну форму
капіталу, найвищий ступінь перекручення і зречевлення відно-
