метненої в засобах виробництва, так і новододаної під час виробництва. Тому ціна одиниці товару
падає. Маса зиску, яка міститься в кожній одиниці товару, може, не зважаючи на це, збільшитись, якщо
норма абсолютної чи відносної додаткової вартості зростає. Кожний окремий товар містить у собі менше
новододаної праці, але неоплачена частина її зростає в порівнянні з оплаченою. Однак, це
відбувається тільки в певних межах. Разом з дуже значним абсолютним зменшенням новододаної до кожної
одиниці товару суми живої праці, яке відбувається в ході розвитку виробництва, зменшуватиметься
абсолютно і маса неоплаченої праці, яка міститься в ній, як би вона не зростала відносно, а саме в
порівнянні з оплаченою частиною. Маса зиску, яка припадає на кожну одиницю товару, дуже
зменшуватиметься з розвитком продуктивної сили праці, не зважаючи на зростання норми додаткової
вартості; і це зменшення цілком так само, як падіння норми зиску, тільки уповільнюється здешевленням
елементів сталого капіталу та іншими наведеними в першому відділі цієї книги обставинами, які
підвищують норму зиску при незмінній і навіть при падаючій нормі додаткової вартості.

Те, що ціна окремих товарів, з суми яких складається сукупний продукт капіталу, падає, не означає
нічого іншого, як те, що дана кількість праці реалізується в більшій масі товарів, що, отже, кожна
одиниця товару містить у собі менше праці, ніж раніше. Це відбувається навіть у тому випадку, коли
ціна якоїсь частини сталого капіталу, сировинного матеріалу та ін. зростає. За винятком окремих
випадків (наприклад, коли продуктивна сила праці рівномірно здешевлює всі елементи як сталого, так і
змінного капіталу), норма зиску знижуватиметься, не зважаючи на підвищену норму додаткової вартості,
1) тому що навіть більша неоплачена частина зменшеної загальної суми новододаної праці є менша, ніж
була менша відповідна неоплачена частина більшої загальної суми, і 2) тому що вищий склад капіталу в
окремому товарі виражається в тому, що та частина його вартості, яка взагалі представляє новододану
працю, зменшується порівняно з тією частиною вартості, яка представляє сировинний матеріал,
допоміжний матеріал і зношування основного капіталу. Ця переміна у відношенні різних складових
частин ціни окремого товару, зменшення тієї частини ціни, яка представляє новододану живу працю, і
збільшення тих частин ціни, які представляють раніше упредметнену працю, є та форма, в якій у ціні
окремого товару виражається зменшення змінного капіталу порівняно з сталим. Наскільки таке зменшення
є абсолютним для капіталу даної величини, наприклад, для 100, настільки ж воно є абсолютним для
кожного окремого товару як відповідної частини репродукованого капіталу. Однак, норма зиску, якщо
тільки обчисляти її на елементи ціни окремих товарів, виступила б іншою, ніж вона є в дійсності. І
саме з такої причини:
