заробітної плати, на основі капіталістичного способу виробництва,
здається неодмінною складовою частиною зиску. Ця частина, як
це правильно виявив уже А. Сміт, виступає в чистому вигляді,
самостійно і цілком відокремлено, з одного боку, від зиску (як
суми процента і підприємницького доходу), з другого боку —
від тієї частини зиску, яка лишається після відрахування процента
як так званий підприємницький дохід, — виступає в формі утримання
управителя в таких галузях підприємств, розмір яких
і т. д. допускає поділ праці, достатній для того, щоб встановити
окрему заробітну плату для управителя.

Праця верховного нагляду й керівництва виникає необхідно
повсюди, де безпосередній процес виробництва має форму суспільно
комбінованого процесу, а не форму роз’єднаної праці
самостійних виробників.\footnote{
„Superintendence is here (у селянина-землевласника) completely dispensed
with“ [„Тут“ (у селянина-землевласника) „можна цілком обійтися без верховного
нагляду“] (J. E. Cairnes: „The Slave Power“. London 1862, стор. 48).
} Але вона має двоякий характер.

З одного боку, в усіх роботах, в яких співробітничають багато
індивідів; зв’язок і єдність процесу необхідно представлені н
одній управляючій волі і в функціях, які стосуються не до частинних
робіт, а до сукупної діяльності майстерні, як це має місце
з дирижером оркестру. Це — продуктивна праця, яку необхідно
виконувати при всякому комбінованому способі виробництва.

З другого боку, — цілком залишаючи осторонь купецький
відділ, — ця праця верховного нагляду необхідно виникає при
всіх способах виробництва, які грунтуються на протилежності
між робітником, як безпосереднім виробником, і власником засобів
виробництва. Чим більша ця протилежність, тим більша
роль, що її відіграє ця праця верховного нагляду. Тому свого
максимуму вона досягає в системі рабства.\footnote{
„If the nature of the work requires that the workmen (саме рабів) should
be dispersed over an extended area, the number of overseers and, therefore, the
cost of the labour which requires this supervision, will be proportionately increased“
[„Якщо, характер праці вимагає розподілу робітників“ (саме рабів) „на
великому просторі, то відповідно до цього зростає число наглядачів, а тому й
витрати на працю, якої вимагає цей нагляд“] (Cairnes: там же, стор. 44).
} Але вона необхідна
і при капіталістичному способі виробництва, бо тут процес
виробництва є разом. з тим процес споживання робочої
сили капіталістом. Цілком так само, як у деспотичних державах
праця верховного нагляду і всебічного втручання уряду
охоплює обидві сторони: як виконання спільних справ, що випливають
з природи усякого суспільства, так і специфічні функції,
що випливають з протилежності між урядом і народною
масою.

В античних письменників, які безпосередньо спостерігали
систему рабства, обидві сторони праці нагляду нероздільно поєднані
в теорії, як це мало місце й на практиці, — цілком так
само, як у сучасних економістів, які вважають капіталістичний
спосіб виробництва за абсолютний спосіб виробництва. З дру-