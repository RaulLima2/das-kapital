\parcont{}  %% абзац починається на попередній сторінці
\index{iii1}{0108}  %% посилання на сторінку оригінального видання
лондонські норми смертності для осіб цього віку зовсім позбавлені значення як показники
антисанітарного стану промисловості (стор. 30).

Такий самий стан, як у кравців, маємо і в складачів, у яких
до відсутності вентиляції, до отруєного повітря і т. ін. долучається ще нічна праця. Їх звичайний
робочий час триває
12—13 годин, іноді 15—16. „Страшенний жар і задушливе повітря, як тільки запалюють газ... Нерідко
буває, що випари
з словолитні або сморід від машин чи стокових ям підіймаються
з нижчого поверху і ще більше погіршують антисанітарний стан
верхніх приміщень. Нагріте повітря нижчих приміщень нагріває
вищі вже самим тільки нагріванням їх підлоги, і коли при великому споживанні газу приміщення низькі,
то це — велике лихо.
Ще гірше стоїть справа там, де парові казани стоять у нижчому
поверсі і наповнюють весь будинок незносним жаром... Загалом можна сказати, що вентиляція абсолютно
незадовільна
і зовсім недостатня для того, щоб після заходу сонця усунути
жар та продукти згорання газу, і що в багатьох майстернях,
особливо там, де раніше були житлові приміщення, становище
надзвичайно сумне“. „У деяких майстернях, особливо в тих,
де друкуються тижневі видання і де зайняті також хлопці 12—16 років, працюють майже без перерв два
дні і одну ніч; а в інших складальних майстернях, в яких виконують „негайні“ роботи,
робітник не має відпочинку навіть у неділю, і його робочий
тиждень становить 7 днів замість 6“ (стор. 26, 28).

Про швачок і модисток (milliners and dressmakers) ми вже
казали в книзі І, розд. VIII, 3, стор. 263\footnote*{Стор. 181 рос. вид. 1935 р. \emph{Ред. укр. перекладу.}}, коли мова йшла про
надмірну працю. Їх робочі приміщення у нашому звіті описані
доктором Ордом. Навіть якщо вдень вони кращі, то в години,
коли горить газ, в них надзвичайний жар, повітря зіпсоване (foul)
і нездорове. В 34 кращих майстернях доктор Орд знайшов, що
пересічна кількість кубічних футів повітря на кожну робітницю була: „В 4 випадках більша, ніж 500; в
4 інших випадках —
400—500; в 5 — від 200 до 250; в 4 — від 150 до 200; і, нарешті,
в 9 — всього 100—150. Навіть у найбільш сприятливому з цих
випадків повітря ледве-ледве вистачає для довгої праці, якщо
приміщення недостатньо провітрюється... Навіть при добрій вентиляції увечері в майстернях стає дуже
жарко й душно через
те, що в них потрібно багато газових ріжків“. А ось зауваження
доктора Орда про одну з відвіданих ним майстерень нижчої
категорії, де робота провадиться коштом посередника (middleman):
„Кімната має 1280 кубічних футів; в ній знаходяться
14 осіб; на кожну з них припадає 91,5 кубічних футів. Робітниці мали тут спрацьований і виснажений
вигляд. Їх заробіток
визначається в 7—15 шилінгів на тиждень, крім того чай... Робочі
години — від 8 до 8. Маленька кімната, в якій скупчені ці 14 осіб,
\parbreak{}  %% абзац продовжується на наступній сторінці
