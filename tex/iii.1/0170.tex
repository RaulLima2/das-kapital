ництва, був би = 20 + x, отже, більший ніж 20, і вироблена
товарна вартість була б = k + 20 + x, більша ніж k + 20, або
більша, ніж ціна виробництва. У сферах виробництва, в яких
склад капіталу (80 + x) c + (20 — x) v, створювана протягом року
додаткова вартість, або зиск, була б = 20 — x, отже, менша, ніж 20,
а тому товарна вартість k + 20 — x була б менша, ніж ціна виробництва, яка = k + 20. Якщо залишити
осторонь можливі ріжниці в часі оборотів, то ціна виробництва товарів дорівнювала б їх вартості
тільки в тих сферах, в яких склад капіталу випадково був би = 80 c + 20 v.

В кожній окремій сфері виробництва специфічний розвиток
суспільної продуктивної сили праці є різний щодо ступеня,
вищий чи нижчий, відповідно до того, наскільки велика є кількість засобів виробництва, що їх
приводить в рух певна кількість праці, тобто, при даному робочому дні, певне число робітників; отже,
він вищий чи нижчий, відповідно до того, наскільки
мала є кількість праці, потрібна для певної кількості засобів
виробництва. Тому капітали, які містять у собі більший процент
сталого, отже, менший процент змінного капіталу, ніж пересічний суспільний капітал, ми звемо
капіталами вищого складу.
Навпаки, такі капітали, в яких сталий капітал займає відносно
менше, а змінний відносно більше місце, ніж у пересічному
суспільному капіталі, ми звемо капіталами нижчого складу.
Нарешті, ми звемо капіталами пересічного складу такі капітали,
склад яких збігається з складом пересічного суспільного капіталу. Якщо пересічний суспільний капітал
в процентах складається з 80 c + 20 v, то капітал 90 c + 10 v стоїть вище, а капітал 70 c + 30 v
нижче, ніж пересічний суспільний. Взагалі, при
складі пересічного суспільного капіталу, рівному mc + nv, де
m і n є сталі величини і m + n = 100, (m + x) c + (n — x) v репрезентує вищий, а (m — x) c + (n + x)
v — нижчий склад окремого капіталу або групи капіталів. Як функціонують ці капітали після
встановлення пересічної норми зиску, — припускаючи, що
вони обертаються один раз за рік, — це показує нижченаведена
таблиця, в якій I представляє пересічний склад, і тому пересічна норма зиску = 20%.

I. 80 c + 20 v + 20 m. Норма зиску = 20%.

Ціна продукту = 120. Вартість = 120.

II.    90 c + 10 v + 10 m. Норма зиску = 20%.

Ціна продукту = 120. Вартість = 110.

III. 70 c + 30 v + 30 m. Норма зиску = 20%.

Ціна продукту = 120. Вартість = 130.

Отже, для товарів, вироблених капіталом II, їхня вартість була б
менша, ніж їхня ціна виробництва, для товарів капіталу III ціна
виробництва була б менша, ніж вартість, і тільки для капіталів I,
