\parcont{}  %% абзац починається на попередній сторінці
\index{iii1}{0014}  %% посилання на сторінку оригінального видання
до остаточної редакції, незадовго перед друком, при чому найновіші
історичні події з завжди незмінною регулярністю давали
йому до його теоретичних положень ілюстрації бажаної актуальності.
Цитат та ілюстрацій тут, як уже і в другій книзі, значно
менше, ніж у першій. Цитати з книги І наводяться з посиланням
на сторінки\footnote{
. „Secret Committee of the House of Lords on Commercial Distress
1847. Report printed 1848. Evidence printed 1857“ (бо в 1848 році
це вважалося надто компрометуючим). — Цитовано як: „Commercial
Distress“ 1848—57.
}-го та\footnote{
. Report: Bank Acts, 1857. — Те саме, 1858. — Звіти комітету
нижньої палати про вплив банкових актів 1844 та 1845 рр.
З думками свідків. — Цитовано як: „Bank Acts“ (іноді також:
„Bank Committee“) 1857, відповідно 1858.

До четвертої книги — історія теорії додаткової вартості —
я візьмуся, як тільки це буде для мене скількинебудь можливо.

В передмові до другого тома „Капіталу“ я мусив порахуватися
з тими панами, які в той час зчинили великий галас, бо
їм хотілося знайти „в Родбертусі таємне джерело і такого попередника
Маркса, який переважає Маркса“. Я дав їм нагоду
показати, „що може зробити політична економія Родбертуса“;
я закликав їх довести, „яким чином може і повинна утворитись
однакова пересічна норма зиску не тільки без порушення закону
вартості, а скоріше на основі цього закону“. Ті самі панове,
які тоді з підстав суб’єктивних чи об’єктивних, але, як правило,
з усяких інших, тільки не з наукових, вихваляли доброго Родбертуса
як економічну зірку найпершої величини, всі без винятку
не дали ніякої відповіді. Навпаки, інші люди вважали
вартим праці зайнятися цією проблемою.

У своїй критиці II тома („Conrads Jahrbücher“, XI, 5, 1885,
стор. 452—465) професор В. Лексіс порушує це питання, хоч і не
}-го видання.\footnote*{
Тут, в українському виданні, в тексті посилання зроблено на сторінки
німецького видання Інституту Маркса — Енгельса — Леніна, а у виносках — на сторінки
російського видання 1935 року. Ред. укр. перекладу.
} Там, де в рукопису є
посилання на теоретичні судження попередніх економістів, здебільшого
наводиться тільки ім’я, саму ж цитату малося подати
при остаточному обробленні. Звичайно, я мусив це так і залишити.
З парламентських звітів використано лише чотири, але
їх використано досить широко. Ці звіти такі:

1. „Reports from Committees“ (нижньої палати), т. VIII, „Commercial
Distress“, т. І, частина І, 1847—48. Minutes of Evidence. —
Цитовано як: „Commercial Distress“, 1847—48.
\parbreak{}  %% абзац продовжується на наступній сторінці
