\parcont{}  %% абзац починається на попередній сторінці
\index{iii1}{0188}  %% посилання на сторінку оригінального видання
регульована середніми вартостями, дорівнює сумі їх індивідуальних вартостей; хоча для товарів,
вироблених на крайніх полюсах, ця вартість виступає як накинута їм пересічна вартість. Виробники, що
працюють на гіршому полюсі, мусять тоді продавати свої товари нижче індивідуальної вартості;
виробники,
що працюють на кращому полюсі, продають свої товари вище
індивідуальної вартості.

У випадку II індивідуальні маси вартості, вироблені на обох
крайніх полюсах, не урівноважуються, і справу вирішує маса
вартості, вироблена при гірших умовах. Строго кажучи, пересічна ціна або ринкова вартість кожного
окремого товару чи
кожної відповідної частини всієї маси товарів визначалася б тут
усією вартістю маси товарів, усією вартістю, що вийшла б
в результаті складання вартостей товарів, вироблених при різних умовах, і відповідною частиною цієї
суми вартостей, яка
припадала б на кожний окремий товар. Одержана таким способом ринкова вартість стояла б вище
індивідуальної вартості
не тільки товарів, належних до сприятливого полюса, але й товарів, належних до середньої групи;
проте, вона все ще стояла б
нижче індивідуальної вартості товарів, вироблених на несприятливому полюсі. Наскільки вона
наближається до цієї останньої
і чи може вона, нарешті, збігтися з нею, це цілком залежить від
розміру, що його має в даній товарній сфері маса товарів,
вироблена на несприятливому полюсі. Якщо попит хоч трохи
переважає, то ринкову ціну регулює індивідуальна вартість товарів, вироблених при несприятливих
умовах.

Якщо, нарешті, як у випадку III, кількість товарів, вироблених на сприятливому полюсі, займає більше
місце не тільки
в порівнянні з другим полюсом, але й в порівнянні з середніми
умовами, то ринкова вартість падає нижче середньої вартості.
Пересічна вартість, обчислена шляхом складання сум вартостей
обох полюсів і середини, стоїть тут нижче середньої вартості
і наближається до цієї останньої або віддаляється від неї
залежно від відносного розміру, що його має сприятливий полюс.
Якщо попит порівняно з поданням слабий, то частина товарів,
що поставлена в сприятливі умови, яка б вона не була велика,
насильно завойовує собі місце, скорочуючи свою ціну до рівня
своєї індивідуальної вартості. З цією індивідуальною вартістю
товарів, вироблених при кращих умовах, ринкова вартість ніколи
не може збігтися, крім випадку дуже значного переважання подання над попитом.

Це \emph{абстрактно} зображене тут встановлення ринкової вартості на дійсному ринку опосереднюється
конкуренцією між
покупцями, припускаючи, що попит є саме настільки великий,
щоб поглинути дану масу товарів по її встановленій таким
чином вартості. І тут ми приходимо до другого пункту.

\emph{Подруге}. Те, що товар має споживну вартість, означає тільки,
що він задовольняє якусь суспільну потребу. Поки ми мали
\parbreak{}  %% абзац продовжується на наступній сторінці
