250 000 ручних верстатів, які функціонували тоді в бавовноткацькій
промисловості Великобританії, споживали щорічно 41 мільйон
фунтів борошна для шліхтування основи. До цього треба
ще додати третину цієї кількості на біління й інші процеси.
Загальну вартість споживаного таким чином борошна він обчислює
для останніх 10 років у 342083 фунтів стерлінгів на рік.
Порівняння з цінами на борошно на континенті показало, що
переплати на самому тільки борошні, які фабриканти примушені
були робити в наслідок мита на хліб, становили щорічно
170000 фунтів стерлінгів. Для 1837 року Грег оцінює ці
переплати щонайменше в 200000 фунтів стерлінгів і вказує
одну фірму, для якої переплата на борошні становила щорічно
1000 фунтів стерлінгів. В наслідок цього „великі фабриканти,
дбайливі і рахубливі ділки, кажуть, що 10 годин щоденної
праці було б цілком досить, коли б скасували мита на хліб“
(„Rep. of Insp. of Fact., Oct. 1848“, стор. 98). Мита на хліб
були скасовані; крім того, були скасовані мита на бавовну
і інші сировинні матеріали; але ледве цього було досягнуто, як
опозиція фабрикантів десятигодинному білеві стала запеклішою,
ніж колибудь. І коли зразу після цього десятигодинний робочий
день на фабриках все ж став законом, його першим наслідком
була спроба загального зниження заробітної плати.

Вартість сировинних і допоміжних матеріалів цілком і за один
раз входить у вартість продукту, на виготовлення якого вони споживаються,
тимчасом як вартість елементів основного капіталу
входить у продукт тільки в міру свого зношування, отже, тільки
ступнево. З цього випливає, що на ціну продукту в далеко більшій
мірі впливає ціна сировинного матеріалу, ніж ціна основного
капіталу, хоч норма зиску визначається загальною сумою вартості
застосовуваного капіталу, незалежно від того, скільки
саме з нього спожито. Але ясно, — хоч про це ми згадуємо
тільки мимохідь, бо ми й тут ще припускаємо, що товари продаються
по їх вартості, отже, викликувані конкуренцією коливання
цін нас тут ще зовсім не цікавлять, — що розширення або
скорочення ринку залежить від ціни окремого товару і стоїть
у зворотному відношенні до зростання або падіння цієї ціни.
Тому в дійсності з підвищенням ціни сировинного матеріалу ціна
фабрикату підвищується не в тій самій пропорції, а при падінні
ціни сировинного матеріалу знижується не в тій самій пропорції,
як ціна сировинного матеріалу. Тому норма зиску в
одному випадку падає нижче, а в другому підіймається вище,
ніж це було б при продажу товарів по їх вартості.

Далі: маса й вартість застосовуваних машин зростає з розвитком
продуктивної сили праці, але не в тій самій пропорції,
в якій зростає ця продуктивна сила, тобто в якій ці машини
постачають збільшену кількість продукту. Отже, в тих галузях
промисловості, куди взагалі входить сировинний матеріал, тобто
де предмет праці сам уже є продукт ранішої праці, там зро-
