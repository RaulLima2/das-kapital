кісного ступеня, в якому він реалізується як капітал. Створена
ним додаткова вартість або зиск — її норма або висота — може
бути вимірена тільки за допомогою порівняння її з вартістю
авансованого капіталу. Тому й більше чи менше зростання вартості
капіталу, що дає процент, може бути вимірене тільки
за допомогою порівняння суми-процента, тієї частини загального
зиску, яка припадає йому, з вартістю авансованого капіталу. Тому,
якщо ціна виражає вартість товару, то процент виражає зростання
вартості грошового капіталу і виступає через це як ціна,
яка сплачується за нього позикодавцеві. Звідси ясно, наскільки
прямо безглуздим є намагання безпосередньо прикласти сюди, як
це робить Прудон, прості відносини обміну, який відбувається за
допомогою грошей, прості відносини купівлі й продажу. Основна
передумова полягає саме в тому, що гроші функціонують як
капітал і тому можуть бути передані третій особі як капітал у
собі, як потенціальний капітал.

Але як товар капітал і тут, виступає остільки, оскільки він
пропонується на ринку і оскільки відчужується дійсно споживна
вартість грошей як капіталу. Але його споживна вартість полягає
в створюванні зиску. Вартість грошей або товарів як капіталу
визначається не їх вартістю як грошей або товарів, а тією
кількістю додаткової вартості, яку вони виробляють для свого
володільця. Продукт капіталу є зиск. Чи витрачаються гроші
як гроші, чи вони авансуються як капітал — це на основі капіталістичного
виробництва є тільки різне застосування грошей.
Гроші — або товар — є капітал у собі, потенціальний капітал, цілком
так само, як і робоча сила потенціально є капітал. Бо
1) гроші можуть бути перетворені в елементи виробництва і самі
вони, як такі, є лише абстрактний вираз елементів виробництва,
їх буття як вартість; 2) речові елементи багатства мають
властивість потенціально бути вже капіталом, тому що протилежність,
яка доповнює їх, те, що робить їх капіталом, — наймана
праця, — на основі капіталістичного виробництва є в наявності.
Антагоністична суспільна визначеність речового багатства —
його антагонізм з працею як найманою працею — є виражена,
відокремлено від процесу виробництва, уже в самій власності
на капітал, як такій. Цей момент — відокремлено від самого капіталістичного
процесу виробництва, постійним результатом якого
він є, і будучи, як такий постійний результат його, разом з тим
його постійною передумовою — виражається в тому, що гроші,
і так само товар, у собі, приховано, потенціально, є капітал, що
вони можуть бути продані як капітал і що в цій формі вони панують
над чужою працею, дають підставу претендувати на привласнення
чужої праці, а тому вони є вартість, що самозростає.
Тут також ясно виступає, що це відношення, а не якась еквівалентна
праця з боку капіталіста є підставою й засобом для привласнення
чужої праці.

Далі, капітал виступає як товар остільки, оскільки поділ зиску
