гого боку, як я це зараз покажу на одному прикладі, апологети
сучасної системи рабства цілком так само уміють використовувати
працю нагляду як довід для виправдання рабства, як інші
економісти — для виправдання системи найманої праці.

Villicus за часів Катона: „На чолі рабовласницького господарства
(familia rustica) стояв управитель (villicus вілли), який відає
прибутками й видатками, купує і продає, дістає розпорядження
від пана і в його відсутності розпоряджається й карає...
Управитель користувався, звичайно, більшою свободою, ніж
усі інші раби; Маго в своїх книгах радить дозволяти йому одружуватись,
родити дітей і мати власні кошти, а Катон радить
одружувати його з управителькою; тільки він міг сподіватися,
в разі доброї поведінки, дістати від свого пана волю. В усьому
іншому всі становили спільне домашнє господарство.... Кожен
раб, навіть і сам управитель, одержував своє утримання від
пана в певні строки за твердо встановленими нормами, і цього
йому повинно було вистачати... Кількість регулювалась відповідно
до праці, в наслідок чого управитель, наприклад, який
виконував легшу працю, ніж раби, одержував менше, ніж ці
останні“ (Mommsen: „Römische Geschichte“. Друге видання, [Берлін]
1856, І, стор. 809—810).

Арістотель: Ο γάρ δεσπότης οὐχ ἐν τω  χτάσθαι τους δούλους, ἀλλ’ ἐν τω
χρῆσθαι δούλης. [Бо пан — капіталіст — виявляється як такий не в
набуванні рабів — власності на капітал, яка дає владу купувати
працю, — а у використанні рабів — вживанні робітників — нині
найманих робітників у процесі виробництва]. Ἔστι δέ αὕτη ἡ επιστήμη
οὐδέν μεγα ἔχουσα οὐδέ οεμνόν. [Але в цій науці немає нічого великого
або величного]; ἄ γάρ τόν δοῦλον ἔπιστασθαι δεῖ ποιεῖν, έχεῖνον δεῖ
ταῦτα ἐπίστασθαι ἐπιτάττειν [він повинен уміти наказувати те, що раб
повинен уміти виконати]. Διο οσοις ἐξουσία μή αυτούς χαχοπαθειν, επιτροπος
λαμβανει ταυτην την τιμην, αυτοι δε πολιτευονται φιλοσοφουσιν.
[Коли в панів немає потреби обтяжувати себе цим, цю честь
бере на себе наглядач, а вони самі займаються державними
справами або філософією]. (Aristoteles: „De Republica“. Вид.
Беккера. Книга І, 7 [Охоnіі 1837, стор. 10 і далі]).

Арістотель прямо говорить, що панування як у політичній,
так і в економічній галузі покладає на владарів функції панування,
тобто що в економічній галузі вони повинні вміти споживати
робочу силу, і додає до цього, що цій праці нагляду не слід надавати
великого значення, і тому пан, якщо він досить заможний,
передає „честь“ цих турбот наглядачеві.

Праця керівництва й верховного нагляду, оскільки вона не є
особлива функція, що випливає з природи всякої комбінованої
суспільної праці, а виникає з протилежності між власником засобів
виробництва і власником самої тільки робочої сили — однаково,
чи ця остання купується разом з самим робітником, як при
системі рабства, чи робітник сам продає свою робочу силу, і тому
процес виробництва є разом з тим і процесом споживання
