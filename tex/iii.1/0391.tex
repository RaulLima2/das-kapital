A. Hodgson, директор ліверпульського Joint Stock Bank, показує, в якій великій мірі векселі можуть
сприяти утворенню резервів для банкірів: „Ми мали
звичай принаймні 9/10 усіх наших вкладів і всі гроші, які ми одержували від
інших осіб, тримати в нашому портфелі у векселях, строки яких кінчалися
з дня на день... настільки, що під час кризи сума надходжень по векселях,
строк яких щодня кінчався, майже дорівнювала сумі вимог платежу, які нам
щодня ставилися“ (стор. 29).

Спекулятивні векселі. — „№ 5092. Ким головним чином акцептувалися векселі (за продану бавовну)?“ [R.
Gardner, бавовняний фабрикант, не раз згадуваний в цій праці:] „Товарними маклерами; торговець купує
бавовну,
передає її маклерові, виставляє вексель на цього маклера і дисконтує його“. — „№ 5094. І ці векселі
йдуть до ліверпульських банків і там дисконтуються? — Так, а також і в інших місцях... Коли б не
було цього надання кредитів, на яке
йшли головним чином ліверпульські банки, то бавовна була б у минулому році,
на мою думку, на 1 1/2 або 2 пенси на фунт дешевша“. — „№ 600. Ви сказали,
що в циркуляції було величезне число векселів, виставлених спекулянтами на
бавовняних маклерів у Ліверпулі; чи це саме стосується і до виданих вами
позик під векселі за інші колоніальні продукти, крім бавовни?“ — [A. Hodgson,
банкір у Ліверпулі:] „Це стосується до всіх сортів колоніальних продуктів,
особливо ж до бавовни“. — „№ 601. Чи стараєтесь ви, як банкір, уникати
такого роду векселів? — Аж ніяк; ми вважаємо їх цілком закономірними векселями, якщо тільки мати їх
у помірній кількості... По векселях цього роду
строки часто відсуваються“.

Шахрайства на ост-індсько-китайському ринку в 1847 році. — Charles
Turner (шеф однієї з першорядних ост-індських фірм у Ліверпулі): „Всі ми
знаємо випадки, які мали місце в операціях на острові Маврікія і в інших
подібних справах. Маклери звикли брати позики під товари не тільки після
їх прибуття, на покриття векселів, виданих за ці товари, що є цілком нормальна річ, і позики під
накладні... але вони брали позики під продукт раніше,
ніж він був навантажений на судна, а в деяких випадках — раніше, ніж він був
вироблений. Я, наприклад, купив з особливої нагоди в Калькутті векселів на 6000—7000 фунтів
стерлінгів; виручка від цих векселів пішла до острову Маврікія
на сприяння культурі цукру; векселі прийшли до Англії, і половина з них була
опротестована; потім, коли нарешті прийшов вантаж цукру, яким мали б бути
оплачені векселі, то виявилось, що цей цукор був уже заставлений третім особам,
раніше ніж був навантажений на судна, в дійсності навіть майже раніше, ніж він
був вироблений“ (стор. 54). „Тепер товари для ост-індського ринку доводиться
оплачувати фабрикантові готівкою; але це не має великого значення, бо якщо покупець має в Лондоні
який-небудь кредит, він виставляє вексель на Лондон
і дисконтує його в Лондоні, де дисконт стоїть тепер низько; одержаними таким
чином грішми він платить фабрикантові... минає принаймні дванадцять місяців,
поки відправник товарів до Індії зможе одержати звідти виручені за них гроші;...
людина з 10000 або 15000 фунтів стерлінгів, яка береться вести операції з Індією, одержить у
лондонської фірми кредит на значну суму; цій фірмі вона платитиме 1\% і видаватиме на неї векселі під
умовою, що виручка від відправлених до
Індії товарів надсилатиметься цій лондонській фірмі; але при цьому обидві сторони мовчки
погоджуються, що лондонська фірма не повинна давати дійсної
позики готівкою, тобто що векселі будуть пролонговані, поки не надійдуть виручені за товари гроші.
Векселі дисконтувались у Ліверпулі, Манчестері, Лондоні, деякі з них перебувають у руках
шотландських банків“ (стор. 55). — „№ 786.
Ось фірма, яка недавно збанкрутувала в Лондоні; при розгляді книг виявили таке:
існувала одна фірма в Манчестері і друга в Калькутті; вони відкрили кредит
у лондонської фірми на 200 000 фунтів стерлінгів, тобто ділові друзі цієї манчестерської фірми, які
посилали товари з Глазго й Манчестера на комісію фірмі в Калькутті, трасували на лондонську фірму в
сумі до 200000 фунтів стерлінгів; одночасно існувала угода, що калькуттська фірма видає векселів на
лондонську фірму теж на 200 000 фунтів стерлінгів; ці векселі були продані у Калькутті; на виручені
гроші були куплені інші векселі, і ці останні були відіслані до Лондона,
щоб дати можливість лондонській фірмі оплатити перші векселі, видані в Глазго
або Манчестері. Таким чином за допомогою однієї тільки цієї операції було по-
