\parcont{}  %% абзац починається на попередній сторінці
\index{iii1}{0379}  %% посилання на сторінку оригінального видання
багатство світу, з якого походить дохід, уже давно стало процентом
на капітал... всяка рента є тепер сплатою процентів на
капітал, раніше вкладений у землю“] („Economist", 19 Juli 1859).
Капіталові в його властивості як капіталу, що дає процент, належить
усе багатство, яке тільки взагалі може бути вироблене,
і все, що він одержував досі, це тільки платежі в розстрочку його
всезахоплюючому апетитові. За природженими йому законами
йому належить уся додаткова праця, яку тільки може колибудь
дати рід людський. Молох.

На закінчення ще така нісенітниця „романтичного“ Мюллера:
„Потворне зростання процентів на проценти д-ра Прайса, або
потворне зростання людських сил, які самі себе прискорюють,
передбачає для того, щоб викликати такі величезні наслідки,
незмінний або непорушний одноманітний порядок протягом багатьох
століть. Як тільки капітал розділяється, роздрібнюється
на багато окремих паростків, які самі по собі продовжують
зростати, загальний, описаний тут процес нагромадження сил
починається знову. Природа розподілила наростання сил на
періоди приблизно в 20—25 років, які пересічно припадають
на кожного окремого робітника (!). Після того, як цей час
минає, робітник покидає свій життьовий шлях і мусить передати
капітал, придбаний за допомогою процентів на проценти від
праці, новому робітникові, здебільшого розподілити його між
кількома робітниками або дітьми. Ці останні, раніше ніж вони
зможуть добувати власне проценти на проценти від капіталу,
що дістався їм, мусять спочатку навчитися оживляти його або
застосовувати. Далі, величезна маса капіталу, що його здобуває
громадянське суспільство, навіть у найрухливіших суспільствах,
поволі нагромаджується протягом довгих років і не
застосовується для безпосереднього розширення праці, а, навпаки,
як тільки назбирається значна сума, вона під назвою „позики“
передається іншому індивідові, робітникові, банкові, державі,
при чому одержувач цієї суми, пускаючи капітал у дійсний рух,
одержує з нього проценти на проценти і може легко зобов’язатись
платити позикодавцеві прості проценти. Нарешті, величезній
прогресії, в якій можуть збільшуватись сили людей та їх
продукт, якщо діє самий тільки закон виробництва або ощадливості,
— протидіє закон споживання, жадання, марнотратства“
(Adam Müller: „Die Elemente der Staatskunst“. Berlin 1809, III,
стор. 147—149).

Неможливо в небагатьох рядках нагородити більше найжахливішого
безглуздя. Не кажучи вже про кумедне змішання
робітника з капіталістом, вартості робочої сили з процентом
на капітал і т. д., зменшення процента на процент має бути
між іншим пояснене з того, що капітал „віддається в позику“ для
того, щоб він давав „тоді проценти на проценти“. Метод нашого
Мюллера характерний для романтики всяких професій. Зміст її
складається з ходячих передсудів, почерпнутих з найповерховішої
\index{iii1}{0380}  %% посилання на сторінку оригінального видання
зовнішньої видимості речей. Потім цей фальшивий і тривіальний
зміст має бути „піднесений“ і опоетизований містифікуючим
способом викладу.

Процес нагромадження капіталу може бути зображений як
нагромадження процентів на проценти лиш остільки, оскільки
можна назвати процентом ту частину зиску (додаткової вартості),
яка зворотно перетворюється в капітал, тобто служить для висмоктування
нової додаткової праці. Але:

1) Залишаючи осторонь усі випадкові порушення, більша частина
наявного капіталу протягом процесу репродукції постійно
більш чи менш знецінюється, бо вартість товарів визначається
не тим робочим часом, якого первісно коштує їх виробництво,
а тим робочим часом, якого коштує їх репродукція, а цей останній
в наслідок розвитку суспільної продуктивної сили праці постійно
зменшується. Тому на вищому ступені розвитку суспільної
продуктивності весь наявний капітал виступає не як результат
довгого процесу нагромадження капіталу, а як результат
порівняно дуже короткого часу репродукції.\footnote{
Див. Мілля і Кері, а також хибний коментарій Рошера до цього. [Пор.
J. St. Mill: „Principles of Political Economy“. 2 вид, Лондон, 1849, стор. 92. H. G.
Carey: „Principles of Social Science“, т. III. Philadelphia 1860, стор. 71 і далі.
W. Roscher: „Die Grundlagen der Nationalökonomie“. 2 вид, Штутгарт і Аугсбург
1857, стор. 70 і далі].
}

2) Як показано у відділі III цієї книги, норма зиску зменшується
в міру зростаючого нагромадження капіталу і відповідного
йому зростання продуктивної сили суспільної праці, яке
виражається якраз у зростаючому відносному зменшенні змінної
частини капіталу порівняно з сталою. Для того, щоб дати ту
саму норму зиску, коли сталий капітал, що його приводить в рух
один робітник, збільшується вдесятеро, додатковий робочий час
мусив би збільшитися вдесятеро, і скоро для цього не вистачило
б усього робочого часу, всіх 24 годин на добу, навіть
коли б цей час цілком привласнювався капіталом. Уявлення ж, що
норма зиску не зменшується, лежить в основі прогресії Прайса
і взагалі в основі „all-engrossing capital, with compound interest“
(всезахоплюючого капіталу з складними процентами)\footnote{
„It is clear, that no labour, no productive power, no ingenuity, and no art,
can answer the everwhelming demands of compound interest. But all saving is
made from the revenue of the capitalist, so that actually the e demands are constantly
made and as constantly the productive power of labour refuses to satisfy
them. A sort of balance is, therefore, constant y struck“. („Ясно, що ніяка праця,
ніяка продуктивна сила, ніякий талант і ніяка вправність не можуть задовольнити
всепоглинаючих вимог складного процента. Але всяке заощадження робиться з
доходу капіталіста, так що в дійсності ці вимоги ставляться постійно і продуктивна сила праці так
само постійно відмовляється задовольняти їх. Тому
постійно порушується певний рід рівноваги“. („Labour defended against the Claims
of Capital“, стор. 23. — Hodgskin).
}.

Тотожність додаткової вартості і додаткової праці ставить
якісну межу нагромадженню капіталу: весь робочий день, даний
у кожний момент розвиток продуктивних сил і населення, який
\parbreak{}  %% абзац продовжується на наступній сторінці
