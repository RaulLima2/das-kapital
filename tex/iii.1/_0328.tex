\index{iii1}{0328}  %% посилання на сторінку оригінального видання
В русі торговельного капіталу $Г — Т — Г'$ той самий товар
двічі або, коли купець продає купцеві, багато разів переходить
з рук в руки; але кожна така переміна місця одного й того ж
товару вказує на метаморфозу, купівлю або продаж товару,
як би часто не повторювався цей процес, поки товар остаточно
не ввійде в споживання.

З другого боку, в $Т — Г — Т$ відбувається двократна переміна
місця одних і тих самих грошей, але це вказує на повну
метаморфозу товару, який спочатку перетворюється в гроші,
а потім з грошей знову в інший товар.

Навпаки, при капіталі, що дає процент, перша переміна місця
$Г$ аж ніяк не є моментом ні в метаморфозі товару, ні в репродукції
капіталу. Таким моментом вона стає лиш при другому
витрачанні, в руках функціонуючого капіталіста, який з цими
грішми провадить торгівлю або перетворює їх у продуктивний
капітал. Перша переміна місця $Г$ не виражає тут нічого іншого,
як уступку або передачу їх від $А$ до $В$; уступку, яка звичайно
відбувається в певних юридичних формах і на певних умовах.

Цьому двократному витрачанню грошей як капіталу, при
чому перше витрачання є проста передача їх від $А$ до $В$, відповідає
і двократний зворотний приплив їх. Як $Г'$ або $Г + ΔГ$ вони
повертаються з руху назад до функціонуючого капіталіста В.
Цей останній тоді знову передає їх $А$, але вже з частиною
зиску, як реалізований капітал, як $Г + ΔГ$, де $ΔГ$ становить
не весь зиск, а тільки частину зиску, процент. До $В$ вони повертаються
тільки як те, що він витратив як функціонуючий
капітал, але як власність $А$. Тому, щоб зворотний приплив їх
закінчився, $В$ повинен знову передати їх $А$. Але, крім капітальної
суми, $В$ повинен передати $А$ під назвою процента частину
зиску, який він виробив за допомогою цієї капітальної суми,
бо $А$ дав йому гроші тільки як капітал, тобто як вартість, яка
не тільки зберігається в русі, але й створює своєму власникові
додаткову вартість. Вони лишаються в руках $В$ тільки доти,
поки вони є функціонуючий капітал. А після їх зворотного
припливу — після скінчення строку — вони перестають функціонувати
як капітал. Але як гроші, які вже більше не функціонують
як капітал, вони мусять бути знову передані назад до $А$,
який не перестав бути їх юридичним власником.

Форма позики замість форми продажу, яка властива цьому
товарові, капіталові як товарові, — вона, зрештою, трапляється
й при інших угодах, — випливає вже з того визначення, що
капітал виступає тут як товар, або що гроші як капітал стають
товаром.

Тут необхідно розрізняти таке:

Ми бачили (книга II, розд. І) і коротко нагадуємо тут про те,
що капітал у процесі циркуляції функціонує як товарний капітал
і як грошовий капітал. Але в обох формах капітал стає
товаром не як капітал.
