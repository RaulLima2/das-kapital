втрата капіталу; це — віддача певної частини благородного металу, з якого складаються світові
гроші“. — „3748. Хіба ви раніше
не сказали, що зміна норми дисконту є проста ознака зміни вартості капіталу? — Так“. — „3749. І що
норма дисконту взагалі
змінюється із зміною золотого запасу в Англійському банку? —
Так; але я вже сказав, що коливання розміру процента, які виникають із зміни кількості грошей в
країні“ [отже, під цим він
тут розуміє кількість дійсного золота] „дуже незначні...“

„3750. Отже, ви хочете сказати, що відбулося зменшення капіталу, якщо відбулося довгочасніше, але
все ж тільки тимчасове підвищення дисконту понад звичайну норму? — Зменшення
в певному розумінні слова. Змінилось відношення між капіталом
і попитом на нього; але можливо, що в наслідок збільшеного
попиту, а не в наслідок зменшення кількості капіталу“. [Але ж
тільки що капітал прирівнювався до грошей або до золота, а трохи
раніше підвищення розміру процента пояснювалось високою нормою зиску, яка виникла з розширення, а не
з скорочення справ
або капіталу].

„3751. Який це капітал ви маєте тут спеціально на увазі? —
Це цілком залежить від того, який капітал потрібен кожній
окремій людині. Це — капітал, який нація має в своєму розпорядженні, щоб продовжувати свої справи, і
якщо ці справи розростаються вдвоє, то мусить настати велике збільшення попиту
на капітал, потрібний для дальшого провадження цих справ“. [Цей
хитромудрий банкір спочатку збільшує вдвоє справи, а далі після
цього попит на капітал, яким вони мають бути подвоєні. Він
завжди бачить перед собою тільки свого клієнта, який вимагає
від пана Лойда більшого капіталу, щоб подвоїти свої справи]. —
„Капітал — те саме, що й усякий інший товар“ [але ж капітал, на думку пана Лойда, є не що інше, як
загальна сума товарів]; „він змінюється у своїй ціні“ [отже, товари двічі змінюються
в ціні: один раз — як товари, другий раз — капітал], „залежно
від попиту й подання“.

„3752. Коливання в нормі дисконту взагалі стоять в зв’язку
з коливанням суми золота у сховищах банку. Чи є це той капітал, що ви маєте на увазі? — Ні“. —
„3753. Чи можете ви навести такий приклад, коли б в Англійському банку був нагромаджений великий
запас капіталу і одночасно норма дисконту
стояла б високо? — В Англійському банку нагромаджують не
капітал, а гроші“. — „3754. Ви сказали, що розмір процента залежить від кількості капіталу; чи не
будете ласкаві сказати,
який капітал ви маєте на увазі, і чи не можете ви навести приклад, коли б у банку лежав великий
запас золота і одночасно
розмір процента стояв би високо? — Дуже ймовірно“ (ага!),
„що нагромадження золота в банку може збігатися з низьким розміром процента, бо період незначного
попиту на капітал“
[саме грошовий капітал; час, про який тут іде мова — роки 1844
і 1845 — був часом процвітання] „є період, протягом якого, зви-
