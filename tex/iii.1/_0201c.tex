\parcont{}  %% абзац починається на попередній сторінці
\index{iii1}{0201}  %% посилання на сторінку оригінального видання
спеціально ним занятих робітників, окремий капіталіст, в відміну
від своєї сфери виробництва, має в експлуатації робітників, експлуатованих
ним особисто.

З другого боку, кожна окрема сфера капіталу і кожний окремий
капіталіст однаково заінтересовані в продуктивності суспільної
праці, вживаної сукупним капіталом. Бо від цього залежать
дві обставини: поперше, маса споживних вартостей, в якій виражається
пересічний зиск; а це подвійно важливо, оскільки цей
зиск служить як фондом нагромадження нового капіталу, так
і фондом доходу, призначеного для споживання. Подруге, висота
вартості авансованого сукупного капіталу (сталого і змінного),
яка, при даній величині додаткової вартості або зиску всього
класу капіталістів, визначає норму зиску, або зиск на певну
кількість капіталу. Особлива продуктивність праці в певній
особливій сфері виробництва або в певному окремому підприємстві
цієї сфери інтересує тільки тих капіталістів, які безпосередньо
беруть участь у цій сфері або в цьому підприємстві,
оскільки така продуктивність дає можливість окремій сфері
виробництва одержувати додатковий зиск порівняно з сукупним
капіталом або окремому капіталістові — порівняно з його сферою.

Отже, ми маємо тут математично точне пояснення того,
чому капіталісти, якими б вони не виявлялись зрадливими братами
у своїй конкуренції між собою, становлять, проте, справжній
масонський союз проти робітничого класу як цілого.

Ціна виробництва включає в собі пересічний зиск. Ми назвали
її ціною виробництва; фактично вона є те саме, що А. Сміт називає
natural price [природною ціною], Рікардо — price of production,
cost of production [ціною виробництва, витратами виробництва],
фізіократи — prix nécessaire [необхідною ціною], — при
чому ніхто з них не дослідив відмінності ціни виробництва від
вартості, — бо ціна виробництва є постійна умова подання, репродукції
товарів кожної окремої сфери виробництва.\footnote{
Мальтус
} Зрозуміло
також, чому ті самі економісти, які повстають проти визначення
вартості товарів робочим часом, кількістю вміщеної в них праці,
завжди говорять про ціни виробництва, як про центри, навколо
яких коливаються ринкові ціни. Вони можуть собі дозволити це,
бо ціна виробництва є вже цілком відчуженою і prima facie [явно]
ірраціональною формою товарної вартості, формою, як вона виступає
в конкуренції, отже, у свідомості вульгарного капіталіста,
а тому також і в свідомості вульгарних економістів.

З вищевикладеного виявилось, яким чином ринкова вартість
(а все сказане про неї стосується з необхідними обмеженнями
і до ціни виробництва) містить у собі надзиск тих, що в кожній
окремій сфері виробництва виробляють при найкращих
\parbreak{}  %% абзац продовжується на наступній сторінці
