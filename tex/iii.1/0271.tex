ємства, на сталий капітал купця, складські будівлі, транспорт
і т. д.), є менший, ніж він був би в тому випадку, коли б промисловий
капітал сам мусив вести всю торговельну частину
свого підприємства;

2) що через те що купець займається виключно цією справою,
то не тільки для виробника його товар раніше перетворюється
в гроші, але й сам товарний капітал пророблює свою
метаморфозу швидше, ніж він міг би це робити у руках виробника;

3) що, коли розглядати сукупний купецький капітал відносно
промислового капіталу, то один оборот купецького капіталу
може представляти не тільки обороти багатьох капіталів в одній
сфері виробництва, але й обороти кількох капіталів у різних
сферах виробництва. Перше має місце тоді, коли, наприклад,
торговець полотном, після того як він на свої 3000 фунтів
стерлінгів купив і знову продав продукт виробника полотна,
раніше ніж той самий виробник знову кине на ринок ту саму
кількість товарів, купує і знову продає продукт іншого або декількох
інших виробників полотна, опосереднюючи таким чином
обороти різних капіталів у тій самій сфері виробництва. Друге
має місце тоді, коли купець, наприклад, після продажу полотна
купує шовк, отже, опосереднює оборот капіталу в іншій сфері
виробництва.

Взагалі слід зауважити таке. Оборот промислового капіталу
обмежується не тільки часом обігу, але й часом виробництва.
Оборот купецького капіталу, оскільки він торгує тільки товарами
певного роду, обмежується не оборотом одного промислового
капіталу, а оборотом усіх промислових капіталів однієї
і тієї ж галузі виробництва. Купець, після того як він купить
і продасть полотно одного, може потім купити і продати полотно
іншого, раніше ніж перший знову кине товар на ринок.
Отже, той самий купецький капітал може послідовно опосереднювати
різні обороти капіталів, вкладених у якійнебудь галузі
виробництва; так що його оборот не є тотожний з оборотами
якогонебудь окремого промислового капіталу, і тому він заміщає
не тільки той грошовий резерв, що його мусив би мати in
petto [напоготові] цей окремий промисловий капіталіст. Оборот
купецького капіталу в якійсь сфері виробництва, звичайно, обмежений
сукупним виробництвом цієї сфери. Але він не обмежений
границями виробництва або часом обороту одиничного
капіталу цієї сфери, оскільки цей час обороту визначається часом
виробництва. Припустім, що А постачає товар, який потребує
для свого виробництва три місяці. Після того як купець
купить його і продасть, скажімо, протягом одного місяця, він
може купити і продати такий самий продукт іншого виробника.
Або, наприклад, після того як він продасть хліб одного фермера,
він може на ті самі гроші купити й продати хліб другого
фермера і т. д. Оборот його капіталу обмежений масою
