й капіталіст, який працює власним капіталом. Обидва одержували
б однаковий пересічний зиск, а капітал, чи взятий у позику,
чи власний, діє як капітал лиш остільки, оскільки він
виробляє зиск. Умова повернення капіталу нічого не змінила б
у цьому. Чим більше розмір процента наближається до нуля,
отже, наприклад, знижується до 1\%, тим більше взятий у позику
капітал стає в однакове становище з власним капіталом.
Поки грошовий капітал має існувати як грошовий капітал,
він мусить знову й знову віддаватись у позику, і при тому
за існуючий процент, скажімо, за 1\%, і завжди тому самому
класові промислових і торговельних капіталістів. Поки ці останні
функціонують як капіталісти, ріжниця між тим, хто функціонує
за допомогою взятого у позику капіталу, і тим, хто функціонує за
допомогою власного капіталу, полягає тільки в тому, що один
повинен сплачувати проценти, а другий — ні; один кладе собі в
кишеню весь зиск р, а другий р — z, зиск мінус процент; чим
більше z наближається до нуля, тим більше р — z наближається
до р, отже, тим більше обидва капітали будуть в однаковому
становищі. Один мусить сплачувати капітал назад і знову брати
його в позику; а другий, поки його капітал має функціонувати,
теж мусить знову й знову авансовувати капітал для процесу
виробництва і не може ним порядкувати незалежно від цього
процесу. Зрештою лишається ще, єдина, сама собою зрозуміла
ріжниця, яка полягає в тому, що один з них є власник свого капіталу,
а другий — ні.

Тепер напрошується таке питання. Яким чином цей чисто
кількісний поділ зиску на чистий зиск і процент обертається
у якісний? Іншими словами, яким чином капіталіст, який застосовує
тільки свій власний капітал, а не взятий у позику, теж
підводить частину свого гуртового зиску під окрему категорію
процента і окремо обчислює його як такий? І, отже, далі, яким
чином усякий капітал, чи взятий у позику, чи ні, як капітал,
що дає процент, відрізняється від самого себе як капіталу, що
дає чистий зиск?

Відомо, що не кожний випадковий кількісний поділ зиску
такого роду обертається в якісний. Наприклад, декілька промислових
капіталістів об’єднуються в асоціацію для ведення підприємства
і потім розподіляють між собою зиск відповідно до
юридично укладеного договору. Інші провадять своє підприємство
кожний сам за себе, без associé [компаньйонів]. Ці останні
обчислюють свій зиск не за двома категоріями, — одну частину
як особистий зиск, а другу як компанійський зиск для неіснуючих
спільників. Отже, тут кількісний поділ не обертається
в якісний. Поділ відбувається, коли власник випадково складається
з декількох юридичних осіб; він не відбувається, коли
цього немає.

Щоб відповісти на це питання, нам доведеться ще дещо довше
спинитися на дійсному вихідному пункті утворення процента;
