\index{iii1}{0071}  %% посилання на сторінку оригінального видання
Якщо ми тепер визначимо відношення К і К1, а також
v і v1, припустимо, наприклад, що значення дробу К1/К = Е, а дробу
v1/v = е, то К1 = ЕК, а v1 — еv. Підставивши тепер у попередньому
рівнянні для р'1, К1 і v1 здобуті таким чином значення,
ми матимем:

р'1 = m' еv / ЕК.

Але з обох попередніх рівнянь ми можемо вивести ще й другу
формулу, перетворивши їх у пропорцію:

р': р'1 = m' v/К : m' v1/К1 = v/К : v1/К1.

Через те, що величина дробу не змінюється, коли чисельник
і знаменник помножити або поділити на те саме число, ми можемо
v/К і v1/К1 звести до процентних чисел, тобто припустити, що
К і К1 = 100. Тоді ми матимем v/К = v/100 і v1/К1 = v1/100, і можемо відкинути
у наведеній пропорції знаменники; ми одержуємо:

р' : р'1 = v : v1; або:

При двох довільно взятих капіталах, які функціонують з однаковою
нормою додаткової вартості, норми зиску відносяться
одна до одної як змінні частини капіталу, обчислені у процентах
до своїх відповідних цілих капіталів.

Ці дві формули охоплюють усі випадки змін v/К.

Раніш ніж дослідити ці випадки кожний окремо, зробимо
ще одно зауваження. Через те, що К є сума c і v, сталого і змінного
капіталу, і через те що норма додаткової вартості, як
і норма зиску, звичайно виражається у процентах, то взагалі
зручно суму c + v теж припустити рівною сотні, тобто c і v
виражати в процентах. Для визначення, — правда, не маси, а
норми зиску, — однаково, чи ми скажемо: капітал у\footnote{
000 K = 12 000 c + 3000 v (+ 3000 m)
100 K = 80 c + 20 v (+ 20 m).

В обох випадках норма додаткової вартості m' = 100\%, норма
зиску = 20\%.
Те саме, коли ми порівнюємо один з одним два капітали,
наприклад, з попереднім якийсь інший капітал:
} 000,
з них\footnote{
000 K = 10 800 c + 1200 v (+ 1200 m)
100 K = 90 c + 10 v (+ 10 m),
} 000 сталого і 3000 змінного капіталу, виробляє додаткову
вартість у 3000, чи зведемо цей капітал до процентів:
\parbreak{}  %% абзац продовжується на наступній сторінці
