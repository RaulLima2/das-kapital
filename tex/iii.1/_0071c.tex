\index{iii1}{0071}  %% посилання на сторінку оригінального видання
Якщо ми тепер визначимо відношення $К$ і $К_1$, а також
$v$ і $v_1$, припустимо, наприклад, що значення дробу $\frac{К_1}{К} = Е$, а дробу
$\frac{v_1}{v} = е$, то $К_1 = Е К$, а $v_1 = е v$. Підставивши тепер у попередньому
рівнянні для $р'_1$, $К_1$ і $v_1$ здобуті таким чином значення,
ми матимем:\[
р'_1 = m'\frac{еv}{ЕК}.
\]

Але з обох попередніх рівнянь ми можемо вивести ще й другу
формулу, перетворивши їх у пропорцію:\[
р': р'_1 = m'\frac{v}{К} : m' \frac{v_1}{K_1} = \frac{v}{К} : \frac{v_1}{К_1}.
\]
Через те, що величина дробу не змінюється, коли чисельник
і знаменник помножити або поділити на те саме число, ми можемо
$\frac{v}{К}$ і $\frac{v_1}{К_1}$ звести до процентних чисел, тобто припустити, що
$К$ і $К_1 = 100$. Тоді ми матимем $\frac{v}{К} = \frac{v}{100}$ і $\frac{v_1}{К_1} = \frac{v_1}{100}$, і можемо відкинути
у наведеній пропорції знаменники; ми одержуємо:\[
р' : р'_1 = v : v_1\text{; або:}
\]
При двох довільно взятих капіталах, які функціонують з однаковою
нормою додаткової вартості, норми зиску відносяться
одна до одної як змінні частини капіталу, обчислені у процентах
до своїх відповідних цілих капіталів.

Ці дві формули охоплюють усі випадки змін $\frac{v}{К}$.

Раніш ніж дослідити ці випадки кожний окремо, зробимо
ще одно зауваження. Через те, що $К$ є сума $c$ і $v$, сталого і змінного
капіталу, і через те що норма додаткової вартості, як
і норма зиску, звичайно виражається у процентах, то взагалі
зручно суму $c + v$ теж припустити рівною сотні, тобто $c$ і $v$
виражати в процентах. Для визначення, — правда, не маси, а
норми зиску, — однаково, чи ми скажемо: капітал у 15000,
з них 12000 сталого і 3000 змінного капіталу, виробляє додаткову
вартість у 3000, чи зведемо цей капітал до процентів:
\begin{align*}
15000 K &= 12000c + 3000 v (+ 3000 m) \\
100 K &= 80 c + 20 v (+ 20 m).
\end{align*}

В обох випадках норма додаткової вартості $m' = 100\%$, норма
зиску = 20\%.
Те саме, коли ми порівнюємо один з одним два капітали,
наприклад, з попереднім якийсь інший капітал:
\begin{align*}
12000 K &= 10800 c + 1200 v (+ 1200 m) \\
100 K &= 90 c + 10 v (+ 10 m),
\end{align*}

