виміряння додаткової вартості, — а це робиться при всякому
обчисленні зиску, — то взагалі відносне падіння додаткової вартості
і її абсолютне падіння є тотожні. Норма зиску в наведених
вище випадках знижується з 40\% до 30\% і до 20\%, бо в дійсності
маса додаткової вартості, а тому й зиску, вироблена тим
самим капіталом, падає абсолютно з 40 до 30 і до 20. Через те
що величина вартості капіталу, відносно якої вимірюється додаткова
вартість, є дана, = 100, то зменшення відношення додаткової
вартості до цієї незмінної величини може бути тільки іншим
виразом зменшення абсолютної величини додаткової вартості
й зиску. Справді, це — тавтологія. Але те, що таке зменшення
настає, випливає, як уже було показано, з природи розвитку
капіталістичного процесу виробництва.

Але, з другого боку, ті самі причини, які викликають абсолютне
зменшення додаткової вартості, а тому й зиску на даний
капітал, а тому також і обчислюваної в процентах норми зиску,
ці самі причини приводять до зростання привласнюваної суспільним
капіталом (тобто сукупністю капіталістів) абсолютної маси
додаткової вартості, а тому й зиску. Як же це мусить виразитись,
як це може виразитись, або які умови передбачаються
і цією позірною суперечністю?

Якщо кожна відповідна частина, = 100, суспільного капіталу,
отже, кожні 100 капіталу пересічного суспільного складу, є величина
дана, і тому для неї зменшення норми зиску збігається
із зменшенням абсолютної величини зиску саме через те, що
тут капітал, яким вони вимірюються, є величина стала, то,
навпаки, величина сукупного суспільного капіталу, як і капіталу,
який знаходиться в руках окремих капіталістів, є змінна
величина, яка, щоб відповідати припущеним умовам, мусить
змінюватись у зворотному відношенні до зменшення своєї змінної
частини.

В попередньому прикладі, при процентному складі капіталу
в 60c + 40v, додаткова вартість або зиск на капітал був 40,
а тому й норма зиску була 40\%. Припустім, що при цій висоті
складу сукупний капітал становив один мільйон. В такому разі
сукупна додаткова вартість, а тому й сукупний зиск становив
400000. Якщо потім склад буде = 80c + 20v, то при незмінному
ступені експлуатації праці додаткова вартість, або зиск, на кожні
100 = 20. Але через те що додаткова вартість, або зиск, як ми
показали, щодо своєї абсолютної маси зростає, незважаючи на цю
падаючу норму зиску або дедалі менше створення додаткової
вартості кожною сотнею капіталу, — наприклад, зростає, скажімо,
з 400000 до 440000, — то це можливе тільки тому, що сукупний
капітал, який утворився одночасно з цим новим складом, зріс
до 2200000. Маса приведеного в рух сукупного капіталу зросла
до 220\%, тимчасом як норма зиску впала на 50\%. Коли б капітал
тільки подвоївся, то при нормі зиску в 20\% він міг би
виробити тільки таку саму масу додаткової вартості й зиску,
