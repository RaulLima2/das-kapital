даткова вартість є надлишок вартості товару понад витрати виробництва
його. Але через те що витрати виробництва дорівнюють
вартості витраченого капіталу, в речові елементи якого вони
раз-у-раз зворотно і перетворюються, то цей надлишок вартості
є приріст вартості того капіталу, який витрачено на виробництво
товару і який повертається назад з циркуляції товару.

Раніше ми вже бачили, що хоч m, додаткова вартість, виникає
тільки з зміни вартості v, змінного капіталу, і тому первісно
є просто приріст змінного капіталу, проте після закінчення
процесу виробництва вона в такій самій мірі становить
приріст вартості c + v, усього витраченого капіталу. Формула
с + (v + m), яка вказує, що m виробляється в наслідок перетворення
певної капітальної вартості v, авансованої на робочу силу,
в текучу величину, отже, в наслідок перетворення сталої величини
в змінну, — може бути представлена також як (c + v) + m.
Перед виробництвом ми мали капітал у 500 фунтів стерлінгів.
Після виробництва ми маємо капітал у 500 фунтів стерлінгів
плюс приріст вартості в 100 фунтів стерлінгів.\footnote{
„В дійсності ми вже знаємо, що додаткова вартість є просто наслідок
тієї зміни вартості, яка відбувається з v, з частиною капіталу, перетвореною
в робочу силу, що, отже, v + m = v + Δv (v плюс приріст v). Але дійсна зміна
вартості і відношення, в якому змінюється вартість, затемнюються тією обставиною,
що в наслідок зростання своєї* варіюючої складової частини зростає
також і весь авансований капітал. Він був 500, а стає 590“ (книга І, розд.
VII, 1, стор. 222 [стор. 148 рос. вид. 1935 р.]).
}

Проте, додаткова вартість становить приріст не тільки до
тієї частини авансованого капіталу, яка входить у процес зростання
вартості, але й до тієї його частини, яка не входить у цей
процес; отже, вона становить приріст вартості не тільки до того
витраченого капіталу, який заміщається з виручених витрат виробництва
товару, але й до капіталу, взагалі застосованого у виробництві.
Перед процесом виробництва ми мали капітальну
вартість у 1680 фунтів стерлінгів: 1200 фунтів стерлінгів основного
капіталу, витраченого на засоби праці, з якого тільки
20 фунтів стерлінгів входять як зношування у вартість товару,
плюс 480 фунтів стерлінгів обігового капіталу в матеріалах виробництва
та заробітній платі. Після процесу виробництва ми
маємо 1180 фунтів стерлінгів як складову частину вартості
продуктивного капіталу плюс товарний капітал у 600 фунтів
стерлінгів. Якщо ми складемо ці дві суми вартості, то побачимо,
що капіталіст володіє тепер вартістю в 1780 фунтів стерлінгів.
Якщо він відніме від цього весь авансований капітал в 1680
фунтів стерлінгів, то в нього лишається приріст вартості в 100
фунтів стерлінгів. Отже, 100 фунтів стерлінгів додаткової вартості
в такій самій мірі становлять приріст вартості до застосованого
капіталу в 1680 фунтів стерлінгів, як і до тієї частини
його в 500 фунтів стерлінгів, яку витрачено під час виробництва.
* VII,