ництва в межах самого капіталістичного способу виробництва,
і тому суперечність, що сама себе знищує, — суперечність, яка
prima facie виступає просто як перехідний пункт до нової
форми виробництва. Як така суперечність він виступає і в
своєму виявленні. В певних сферах він відновлює монополію і
тому вимагає державного втручання. Він репродукує нову фінансову аристократію, новий вид паразитів в
образі прожектерів,
грюндерів і просто номінальних директорів; цілу систему шахрайства і обману щодо засновництва,
випуску акцій і торгівлі акціями.
Це — приватне виробництво без контролю приватної власності.

IV. Незалежно від акційної справи, — яка є знищенням (Aufhebung) капіталістичної приватної
промисловості на основі самої
капіталістичної системи і яка в тій самій мірі, в якій вона поширюється і захоплює нові сфери
виробництва, знищує (vernichtet)
приватну промисловість, — кредит віддає окремому капіталістові
або тому, хто вважається за капіталіста, в його абсолютне, в
певних межах, розпорядження чужий капітал і чужу власність,
а тому й чужу працю.\footnote{
Погляньте, наприклад, у „Times“ [див. „Times“ від 3, 5, 7 грудня 1857 р.]
на списки банкрутств за такий рік кризи, як 1857 рік, і порівняйте власне майно
банкрутів з сумою їхніх боргів. — „Воістину купівельна сила людей, які володіють
капіталом і користуються кредитом, далеко перевищує все, що можуть собі уявити
ті, хто не має практичного знайомства з спекулятивними ринками“ (Tooker
„Inquiry into the Currency Principle", стор. 79). „Людина, про яку йде слава, що
вона володіє достатнім капіталом для свого постійного діла, і яка користується в своїй галузі добрим
кредитом, якщо вона має сангвінічні погляди
про підвищення коньюнктури для того товару, яким вона оперує, і якщо на
початку і протягом її спекуляції обставини для неї сприятливі, може зробити
закупівлі колосальних розмірів порівняно з її капіталом“ (там же, стор. 136). —
 „Фабриканти, купці і т. д. — всі провадять операції, які далеко перевищують їхній
капітал... Нині капітал є скоріше основа, на якій будується добрий кредит,
ніж межа оборотів якого-небудь комерційного підприємства“ („Economist", 1847,
стор. 333).
} Розпорядження суспільним, а не власним
капіталом дає йому в розпорядження суспільну працю. Сам капітал, яким володіють дійсно або на думку
публіки, стає тільки
основою для кредитної надбудови. Це стосується особливо до
гуртової торгівлі, через руки якої проходить найзначніша частина
суспільного продукту. Всяке мірило, всякі більш-менш виправдані в межах капіталістичного способу
виробництва пояснення тут зникають. Спекулюючий гуртовий торговець
рискує не своєю власністю, а суспільною. Так само безглуздою стає фраза про походження капіталу з
заощаджень,
бо спекулюючий гуртовий торговець вимагає якраз того, щоб
інші робили для нього заощадження. [Як недавно вся Франція
заощадила півтора мільярди франків для панамських шахраїв.
З якою точністю змальоване тут усе панамське шахрайство за
цілих двадцять років до того, як воно сталося! — Ф. Е.]
Другій фразі про повздержливість б’є прямо в лице його розкіш, яка сама тепер теж стає засобом
кредиту. Уявлення, які
на менш розвиненому ступені капіталістичного виробництва