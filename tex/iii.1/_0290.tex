\parcont{}  %% абзац починається на попередній сторінці
\index{iii1}{0290}  %% посилання на сторінку оригінального видання
постачають сталий капітал тим, хто виробляє засоби існування.
Купець одержує, по-перше, заміщення цього капіталу і, по-друге,
зиск на нього. Отже в наслідок того й другого відбувається
скорочення зиску для промислового капіталіста. Але в наслідок
зв’язаної з поділом праці концентрації і економії він скорочується
в меншій мірі, ніж це було б у тому випадку, коли б самому
промисловому капіталістові доводилось авансувати цей капітал.
Зменшення норми зиску є менше, бо авансовуваний таким чином
капітал є менший.

Поки що продажна ціна складається, отже, з $В + К$ + зиск на
$В + К$. Ця частина її, після того, що ми досі виклали, не становить
ніяких труднощів. Але ось сюди долучається $b$ або авансований
купцем змінний капітал.

В наслідок цього продажна ціна стає $В + К + b$ + зиск на
$B + K$, + зиск на $b$.

$В$ заміщує тільки купівельну ціну, але, крім зиску на $В$, не
додає до цієї ціни ніякої частини. $К$ додає не тільки зиск на $К$,
але й само $К$; але $К$ + зиск на $К$, — частина витрат циркуляції,
авансована в формі сталого капіталу, + відповідний пересічний
зиск, — була б більша в руках промислового капіталіста, ніж у
руках торговельного капіталіста. Зменшення пересічного зиску
виявляється в такій формі, що обчислюється повний пересічний
зиск, — після відрахування $В + К$ з авансованого промислового
капіталу, — а відрахування з пересічного зиску, яке становить
зиск на $В + К$, виплачується купцеві, так що це відрахування
виступає як зиск особливого капіталу, купецького капіталу.

Але з $b$ + зиск на $b$, або в даному випадку, оскільки норма
зиску за припущенням = 10\%, з $b + \sfrac{1}{10}b$, справа стоїть інакше.
Саме тут ми маємо справжню трудність.

Згідно з нашим припущенням, купець купує на $b$ тільки торговельну
працю, тобто працю, необхідну для того, щоб опосереднювати
функції циркуляції капіталу, $Т — Г$ і $Г — Т$. Але торговельна
праця є праця, взагалі необхідна для того, щоб капітал
функціонував як купецький капітал, щоб він опосереднював перетворення
товару в гроші і грошей у товар. Це — праця, яка реалізує
вартості, але не створює ніяких вартостей. І лиш оскільки
який небудь капітал виконує ці функції, — отже, оскільки якийнебудь
капіталіст виконує своїм капіталом ці операції, цю працю,
— остільки цей капітал функціонує як купецький капітал
і бере участь у регулюванні загальної норми зиску, тобто одержує
свій дивіденд із сукупного зиску. Але в ($b$ + зиск на $b$) міститься,
видимо, по-перше, оплата праці (бо цілком байдуже, чи платить
промисловий капіталіст купцеві за його власну працю чи за працю
прикажчиків, оплачуваних купцем) і, по-друге, зиск на суму,
виплачену за цю працю, яку мусив би виконувати сам купець.
Купецький капітал одержує назад, по-перше, оплату $b$ і, по-друге,
зиск на нього; отже, це походить з того, що він, по-перше, примушує
оплатити собі ту працю, за допомогою якої він функціонує
\parbreak{}  %% абзац продовжується на наступній сторінці
