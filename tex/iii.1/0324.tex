процес виробництва як товари, куплені ним самим. І замість
того, щоб виробляти на окремого купця або на певних покупців,
сукнороб виробляє тепер для торговельного світу. Сам виробник
є купець. Торговельний капітал виконує вже тільки
процес циркуляції. Первісно торгівля була передумовою для
перетворення цехової і сільсько-домашньої промисловості та феодального
землеробства у капіталістичні виробництва. Вона розвиває
продукт у товар, почасти створюючи для нього ринок,
почасти постачаючи нові товарні еквіваленти, а для виробництва
нові сировинні й допоміжні матеріали, і даючи цим початок
галузям виробництва, які з самого початку грунтуються
на торгівлі: на виробництві для ринку й світового ринку, як і на
умовах виробництва, які походять з світового ринку. Як тільки
мануфактура до певної міри зміцніє, вона, а ще більше велика
промисловість, з свого боку створює собі ринок, завойовує його
своїми товарами. Тепер торгівля стає слугою промислового
виробництва, для якого постійне розширення ринку є умова
існування. Дедалі ширше масове виробництво переповнює наявний
ринок і тому постійно працює над розширенням цього
ринку, над тим, щоб пробити його рамки. Що обмежує це
масове виробництво, так це не торгівля (оскільки ця остання є
виразом тільки існуючого попиту), а величина функціонуючого
капіталу і ступінь розвитку продуктивної сили праці. Промисловий
капіталіст постійно має перед собою світовий ринок, він
порівнює і мусить постійно порівнювати свої власні витрати виробництва
з ринковими цінами не тільки своєї країни, але й усього
світу. В попередні періоди таке порівняння припадає майже виключно
купцям і забезпечує таким чином торговельному капіталові
панування над промисловим капіталом.

Перше теоретичне розроблення сучасного способу виробництва
— меркантильна система — необхідно виходило з поверхових
явищ процесу циркуляції в тому вигляді, як вони усамостійнилися
в русі торговельного капіталу, і тому торкалось
тільки зовнішньої видимості явищ. Почасти тому, що торговельний
капітал є перша вільна форма існування капіталу взагалі.
Почасти в наслідок того переважаючого впливу, який він справляв
у перший період перевороту в феодальному виробництві,
в період виникнення сучасного виробництва. Дійсна наука сучасної
економії починається лиш з того часу, коли теоретичне дослідження
переходить від процесу циркуляції до процесу виробництва.
Правда, капітал, що дає процент, є теж прадавня форма
капіталу. Але чому меркантилізм виходить не з нього, а, навпаки,
ставиться до нього полемічно, це ми побачимо пізніше.
