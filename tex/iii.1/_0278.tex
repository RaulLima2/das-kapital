\parcont{}  %% абзац починається на попередній сторінці
\index{iii1}{0278}  %% посилання на сторінку оригінального видання
І справді, все уявлення про походження зиску від номінального
підвищення ціни товарів або від продажу їх вище їх вартості
виникло з спостережень купецького капіталу.

Однак, якщо розглядати справу ближче, то відразу виявляється,
що це тільки проста видимість, і що — коли припускати
капіталістичний спосіб виробництва пануючим — торговельний
зиск реалізується не таким способом. (Тут завжди справа
йде тільки про пересічні, а не про одиничні випадки.) Чому
ми припускаємо, що торговець товарами може реалізувати зиск
на свої товари, скажімо, в 10\%, тільки продаючи їх на 10\%
вище їх цін виробництва? Тому що ми припустили, що виробник
цих товарів, промисловий капіталіст (який, як персоніфікація
промислового капіталу, відносно зовнішнього світу завжди
фігурує як „виробник“), продав їх купцеві по їх ціні виробництва.
Якщо купівельні ціни товарів, заплачені торговцем товарів, дорівнюють
їх цінам виробництва, в кінцевому рахунку дорівнюють
їх вартостям, так що, отже, ціна виробництва, в кінцевому рахунку
вартість товарів, представляє для купця витрати виробництва, то
в дійсності надлишок його продажної ціни порівняно з його купівельною
ціною — а тільки ця ріжниця між цінами становить джерело
його зиску — мусить бути надлишком їх торговельної ціни
порівняно з їх ціною виробництва, і в кінцевому рахунку купець
мусить продавати всі товари вище їх вартості. Але чому ж було
припущено, що промисловий капіталіст продає купцеві товари по
їх цінах виробництва? Або, краще, що́ означає таке припущення?
Те, що торговельний капітал (тут ми з ним маємо справу
ще тільки як з товарно-торговельним капіталом) не бере участі
в утворенні загальної норми зиску. При дослідженні загальної
норми зиску ми необхідно виходили з такого припущення, поперше,
тому, що торговельний капітал як такий тоді для нас
ще не існував; а подруге, тому, що пересічний зиск, отже й загальну
норму зиску, насамперед необхідно було розвинути як
вирівнення зисків або додаткових вартостей, які дійсно виробляються
промисловими капіталами різних сфер виробництва.
Навпаки, в купецькому капіталі ми маємо справу з капіталом,
який бере участь в одержанні зиску, не беручи участі в його
утворенні. Отже, тепер нам треба доповнити попередній виклад.

Припустім, що сукупний промисловий капітал, авансований
протягом року, = 720 c + 180 v = 900 (наприклад, мільйонів фунтів
стерлінгів), а m' = 100\%. Отже, продукт — 720 c + 180 v + 180 m.
Якщо ми потім позначимо цей продукт або вироблений товарний
капітал через Т, то його вартість або ціна виробництва
(бо для сукупності товарів вони збігаються) = 1080, а норма
зиску для всього капіталу в 900 = 20\%. Ці 20\% становлять,
згідно з викладеним раніше, пересічну норму зиску, бо додаткова
вартість тут обчислена не на той чи інший капітал особливого
складу, а на сукупний промисловий капітал з його пересічним
складом. Отже, Т = 1080, а норма зиску = 20\%. Але тепер
\parbreak{}  %% абзац продовжується на наступній сторінці
