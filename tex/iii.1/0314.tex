ний спосіб виробництва, є в дійсності історично найстаріша
вільна форма існування капіталу.

Оскільки ми вже бачили, що торгівля грішми і авансований
на неї капітал не потребують для свого розвитку нічого іншого,
крім існування гуртової торгівлі, і далі товарно-торговельного
капіталу, то ми тут розглядатимем тільки цей останній.

Через те що торговельний капітал замкнений в сфері циркуляції
і його функція полягає виключно в тому, щоб опосереднювати
обмін товарів, то — залишаючи осторонь нерозвинені форми,
які виникають з безпосередньої мінової торгівлі, — для його існування
не потрібно ніяких інших умов, крім тих, що потрібні для
простої товарної і грошової циркуляції. Або, краще сказати, ці
останні є умовою його існування. Який би не був спосіб виробництва,
на основі якого виробляються продукти, що входять
у циркуляцію як товари, — чи виробляються вони на основі первісної
громади, чи рабського виробництва, чи дрібноселянського
і дрібнобуржуазного, або капіталістичного, — це нічого не змінює
в їх характері як товарів, і як товари вони повинні проробити
процес обміну і зміни форми, які супроводять його. Крайні
члени, між якими купецький капітал є посередником, є дані для
нього, цілком так само як вони дані для грошей і для руху грошей.
Єдине необхідне полягає в тому, щоб ці крайні члени були
в наявності як товари, однаково чи виробництво в усьому своєму
обсягу є товарне виробництво, чи на ринок подається тільки
надлишок, який лишається у самостійно господарюючих виробників
після задоволення їх безпосередніх потреб їх виробництвом.
Купецький капітал тільки опосереднює рух цих крайніх членів,
товарів, як даних для нього передумов.

Розмір, в якому виробництво входить у торгівлю, проходить
через руки купців, залежить від способу виробництва і досягає
свого максимуму при повному розвитку капіталістичного
виробництва, коли продукт виробляється вже тільки як товар,
а не як безпосередній засіб існування. З другого боку, на основі
всякого способу виробництва торгівля сприяє утворенню надлишкового
продукту, призначеного входити в обмін, щоб збільшити
споживання або скарби виробників (під якими тут слід
розуміти власників продуктів); отже, вона надає виробництву
характеру виробництва, що все більше й більше має своєю метою
мінову вартість.

Метаморфоза товарів, їх рух, полягає: 1) речово в обміні
різних товарів один на один, 2) формально в перетворенні товару
в гроші, в продажу, і в перетворенні грошей у товар,
в купівлі. І до цих функцій, до обміну товарів за допомогою
купівлі й продажу, зводиться функція купецького капіталу. Отже,
він опосереднює тільки обмін товарів, який, однак, з самого початку
не можна розуміти просто як обмін товарів між безпосередніми
виробниками. При відносинах рабства, при відносинах
кріпацтва, при відносинах данництва (оскільки мається на увазі
