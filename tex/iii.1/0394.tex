нений, внаслідок неврожаю довелося б вивезти 5 мільйонів
золота і довезти хліба на таку ж суму. Циркуляція“ [це означає, як зараз виявиться, не засоби
циркуляції, а незайнятий грошовий капітал — Ф. Е.] „зменшиться на таку ж суму. Можливо, що приватні
люди матимуть ще досить засобів циркуляції,
але вклади купців у їх банках, сальдо банків у їх грошових маклерів і резерви в касах банків всі
зменшаться, і безпосереднім
наслідком цього зменшення суми незайнятого капіталу буде підвищення розміру процента, наприклад, з 4
до 6%. Тому що стан
справ здоровий, то й довір’я не буде захитане, але кредит цінуватиметься вище“ (там же, стор. 42).
„Якщо відбувається загальне
падіння товарних цін, то надлишкові гроші припливають назад
до банків у формі збільшених вкладів, надлишок незайнятого капіталу знижує розмір процента до
мінімуму, і цей стан речей
триває доти, поки вищі ціни або пожвавлення справ не покличуть
до роботи бездіяльні гроші, або поки ці гроші не будуть поглинені в купівлі іноземних цінних паперів
чи іноземних товарів“ (стор. 68).

Дальші витяги знов таки взяті з парламентського звіту про
„Commercial Distress“ 1847—1848 рр. — В наслідок неврожаю і голоду 1846—1847 років став потрібним
великий довіз засобів харчування. „Звідси велике перевищення довозу над вивозом... Звідси
значний відплив грошей з банків і посилений наплив до дисконтних маклерів осіб, яким треба було
дисконтувати векселі;
маклери починають обережніше приймати векселі. Одержуваний
досі кредит зазнав серйозного обмеження, і серед слабих фірм
почались банкрутства. Ті, що цілком покладались на кредит,
зруйнувалися вкрай. Це збільшило тривогу, яка відчувалася ще
раніше; банкіри та інші побачили, що вони не можуть з такою
певністю, як раніше, розраховувати на перетворення своїх векселів та інших цінних паперів у
банкноти, щоб виконати свої зобов’язання; вони ще більше обмежили надання позик і часто
зовсім відмовляли в цьому; в багатьох випадках вони приховували свої банкноти для майбутнього
покриття своїх власних зобов’язань; вони вважали за краще зовсім не випускати їх. Тривога
й замішання зростали з кожним днем, і якби не лист лорда
Джона Росселя, то настало б загальне банкрутство“ (стор. 50, 51).
Лист Росселя припинював чинність банкового акту. — Вищезгаданий Charles Turner свідчить: „Деякі
фірми мали великі кошти, але
вони не були вільні. Весь їх капітал міцно засів у земельній
власності на острові Маврікія або в фабриках індиго та цукрових заводах. Прийнявши на себе раніше
зобов’язання на 500 000—600 000 фунтів стерлінгів, вони не мали вільних коштів для
оплати цих векселів і кінець-кінцем виявилось, що вони могли
оплатити свої векселі тільки за допомогою свого кредиту і лиш
остільки, оскільки його вистачало“ (стор. 57). — Згаданий S. Gurney
заявив: „Нині (1848 р.) панує обмеження оборотів і великий
надмір грошей“. — „№ 1763. Я не думаю, щоб причиною, яка так
