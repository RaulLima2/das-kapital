вого і торговельного капіталу як такого, а до грошового
капіталу, і норма цієї частини додаткової вартості, норма процента
або розмір процента, закріплює це відношення. Бо, поперше,
розмір процента — не зважаючи на його залежність
від загальної норми зиску — визначається самостійно, і, подруге,
він, подібно до ринкової ціни товарів, у протилежність до невловимої
норми зиску, виступає при всякій переміні як стійке,
одноманітне, наочне і завжди дане відношення. Коли б весь
капітал був у руках промислових капіталістів, то не існувало
б ні процента, ні розміру процента. Самостійна форма,
якої набирає кількісний поділ гуртового зиску, породжує якісний
поділ. Якщо порівняти промислового капіталіста з грошовим
капіталістом, то першого відрізняє від другого тільки підприємницький
дохід, як надлишок гуртового зиску понад
пересічний процент, надлишок, який, завдяки розмірові процента,
виступає як емпірично дана величина. Якщо ж порівняти
його, з другого боку, з промисловим капіталістом, що
господарює власним, а не взятим у позику капіталом, то цей
останній відрізняється від нього тільки як грошовий капіталіст,
оскільки він сам кладе собі процент у кишеню, замість
сплачувати його. В обох випадках частина гуртового зиску, яка
відрізняється від процента, здається йому підприємницьким доходом,
а самий процент — додатковою вартістю, яку дає капітал
сам по собі і яку він через це давав би і без продуктивного
застосування.

Для окремого капіталіста це практично вірно. Від його вибору
залежить, чи віддати свій капітал — однаково, чи існує
він уже в своїй вихідній точці як грошовий капітал, чи його
ще тільки доводиться перетворити в грошовий капітал — у позику
як капітал, що дає процент, чи самому збільшувати його
вартість, застосовуючи його як продуктивний капітал. В загальному
ж розумінні, тобто в застосуванні до всього суспільного
капіталу, — як це роблять деякі вульгарні економісти, видаючи
це навіть за основу зиску, — це, звичайно, безглуздо. Припускати
можливість перетворення сукупного капіталу в грошовий капітал,
без наявності людей, що купують і збільшують вартість
засобів виробництва, у формі яких існує сукупний капітал, крім
відносно незначної частини його, існуючої у формі грошей, — це,
звичайно, безглуздя. Ще більше безглуздя гадати, що на основі
капіталістичного способу виробництва капітал давав би процент,
не функціонуючи як продуктивний капітал, тобто не створюючи
додаткової вартості, лише частину якої становить процент; що
капіталістичний спосіб виробництва міг би іти своїм шляхом без
капіталістичного виробництва. Коли б непомірно велика частина
капіталістів схотіла перетворити свій капітал у грошовий капітал,
то наслідком цього було б величезне знецінення грошового капіталу
і величезне падіння розміру процента; багато з них відразу
були б позбавлені можливості жити на свої проценти і таким
