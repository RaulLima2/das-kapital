ництва, первісно є дуже різні. Ці різні норми зиску за допомогою конкуренції вирівнюються в загальну
норму зиску, яка
є пересічною всіх цих різних норм зиску. Зиск, який відповідно
до цієї загальної норми зиску припадає на капітал даної величини, який би не був його органічний
склад, зветься пересічним
зиском. Ціна товару, яка дорівнює витратам його виробництва
плюс та частина річного пересічного зиску на застосований для
виробництва товару (а не тільки на спожитий для його виробництва) капітал, яка припадає на товар
залежно від умов його
обороту, є його ціна виробництва. Візьмімо, наприклад, капітал
в 500, в тому числі 100 основного капіталу, з якого зношується
10% протягом одного періоду обороту обігового капіталу в 400.
Припустімо, що пересічний зиск протягом цього періоду обороту становить 10%. Тоді витрати
виробництва виготовленого
протягом цього обороту продукту будуть: 10 с на зношування
плюс 400 (c + v) обігового капіталу = 410, а його ціна виробництва: 410 витрати виробництва плюс
(10% зиску на 500) 50 = 460.

Тому, хоч капіталісти різних сфер виробництва при продажу
своїх товарів повертають собі капітальні вартості, спожиті на
виробництво цих товарів, але реалізують вони не ту додаткову
вартість, отже, і не той зиск, що виробляється в їх власній
сфері при виробництві цих товарів, а лише стільки додаткової вартості, отже й зиску, скільки при
рівному розподілі
припадає на кожну відповідну частину всього капіталу суспільства з усієї додаткової вартості або
всього зиску, який виробляється сукупним капіталом суспільства за даний період часу
в усіх сферах виробництва, взятих разом. Кожен авансований
капітал, який би не був його склад, одержує кожного року або
за якийсь інший період часу стільки зиску на кожні 100, скільки
його за цей період часу припадає на кожні 100 як певну частину
сукупного капіталу. Оскільки справа стосується зиску, різні капіталісти відносяться тут один до
одного, як прості акціонери
одного акційного товариства, в якому зиск розподіляється між
ними рівномірно на кожні 100 одиниць, і тому зиски для різних
капіталістів відрізняються тільки залежно від величини капіталу,
вкладеного кожним з них у спільне підприємство, залежно від
відносного розміру участі кожного в спільному підприємстві,
залежно від числа акцій кожного з них. Отже, тимчасом як та
частина цієї товарної ціни, яка заміщає спожиті на виробництво
товарів частини вартості капіталу і за яку, отже, знову мусять
бути куплені ці спожиті капітальні вартості, — тимчасом як ця
частина, яка становить витрати виробництва, цілком визначається
видатками в межах відповідних сфер виробництва, — друга складова частина товарної ціни, зиск,
доданий до цих витрат виробництва, визначається не масою зиску, виробленою цим певним
капіталом у цій певній сфері виробництва протягом даного
часу, а тією масою зиску, яка за даний період часу пересічно припадає на кожний застосований капітал
як певну ча-
