\parcont{}  %% абзац починається на попередній сторінці
\index{iii1}{0119}  %% посилання на сторінку оригінального видання
промисловості, в які не входить сировинний матеріал у власному значенні слова, він входить як
допоміжний матеріал або як складова частина машин і т. д., і таким чином коливання його цін
впливають pro tanto на норму зиску. Якщо ціна сировинного матеріалу зменшується на суму $= d$, то $\frac{m}{K}$,
або $\frac{m}{c + v}$, переходить у $\frac{m}{K - d}$, або у
$\frac{m}{(c - d) + v}$. Тому норма зиску підвищується. Навпаки, якщо ціна сировинного матеріалу
підвищується, то $\frac{m}{K}$, або$\frac{m}{c + v}$, переходить у $\frac{m}{K + d}$,
або $\frac{m}{(c + d) + v}$; тому норма зиску
падає. Отже, при інших однакових умовах норма
зиску зменшується і підвищується у зворотному напрямі до руху ціни сировинного матеріалу. Звідси
виявляється, між іншим, наскільки важлива для промислових країн низька ціна сировинного матеріалу,
навіть в тому випадку, коли коливання цін сировинного матеріалу зовсім не супроводяться змінами у
сфері продажу продукту, отже цілком незалежно від співвідношення попиту й подання. Далі, звідси,
виходить, що зовнішня торгівля впливає на норму зиску, навіть незалежно від усякого впливу її на
заробітну плату в наслідок здешевлення необхідних засобів існування. Вона впливає саме на ціни
сировинних або допоміжних матеріалів, які входять у промисловість або землеробство. Цілком
недостатнє ще й досі розуміння природи норми зиску та її специфічної відмінності від норми
додаткової вартості є причиною того, що, з одного боку, економісти, які підкреслюють встановлений
практичним досвідом значний вплив цін сировинного матеріалу на норму зиску, пояснюють це теоретично
цілком неправильно (Торренс), тимчасом як, з другого боку, економісти, які твердо тримаються
загальних принципів, як Рікардо, не визнають, наприклад, впливу світової торгівлі на норму зиску.

Тому зрозуміло, яке велике значення має для промисловості скасування або зниження мит на сировинні
матеріали; тим то вільний по можливості довіз сировинних матеріалів був уже основним положенням
раціонально розвиненої системи охоронних мит. Поряд із скасуванням мит на хліб це було головною
метою англійських фритредерів [прихильників вільної зовнішньої торгівлі], які перш за все дбали про
те, щоб було скасоване також мито на бавовну.

Прикладом важливості зниження цін не сировинного матеріалу у власному значенні слова, а допоміжного
матеріалу, який разом з тим є головним елементом харчування, може служити споживання борошна в
бавовняній промисловості. Вже в 1837 році Р. Г. Грег\footnote{
„The Factory Question and the Ten Hours Bill“. By \emph{R. H. Greg}, London 1837, стор. 115.
} обчислив, що ті 100000 механічних і
\parbreak{}  %% абзац продовжується на наступній сторінці
