\parcont{}  %% абзац починається на попередній сторінці
\index{iii1}{0377}  %% посилання на сторінку оригінального видання
далеко позад себе фантазії алхеміків; до вигадок, в які Пітт
серйозно вірив і які він у своїх законах про sinking fund (фонд
для оплати державних боргів) зробив стовпами, на яких тримається
його фінансове господарство.

„Гроші, що дають проценти на проценти, зростають спочатку
поволі; але тому що темп зростання постійно прискорюється, він
через деякий час стає таким швидким, що перевищує всяку фантазію.
Одно пенні, віддане в позику при народженні нашого
Спасителя по 5\%, разом з процентами на проценти вже тепер
зросло б до суми більшої, ніж та, що містилася б в 150 мільйонах
земних куль з чистого золота. Але, віддане на прості проценти,
воно зросло б за той самий час тільки до 7 шилінгів 4 1/2 пенсів.
Досі наш уряд вважав за краще поліпшувати свої фінанси цим
останнім способом, а не першим“.\footnote{
Richard Price: „An Appeal to the Public on the subject of the National Debt“.
[1772], 2 вид. Лондон 1774 [стор. 18 і далі]. Він наївно пускається в дотеп: „Гроші
треба брати в позику на прості проценти, щоб збільшувати їх, віддаючи в позику
на складні проценти“. R. Hamilton: „An Inquiry concerning the Rise and
Progress of the National Debt of Great Britain“. 2 вид., Едінбург 1814 [частина III,
відділ I: „Examination of Dr. Price’s Views of Finance“, стор. 133]). Згідно з цим
одержання позик взагалі було б найпевнішим засобом збагачення і для приватних
осіб. Але якщо я беру, наприклад, 100 фунтів стерлінгів по 5\% річних,
то в кінці року я повинен сплатити 5 фунтів стерлінгів, і якщо припустити, що
ця позика триватиме 100 мільйонів років, то в цей проміжок часу я можу щороку
давати в позику завжди тільки 100 фунтів стерлінгів і так само щороку маю платити
5 фунтів стерлінгів. Таким способом я ніколи не зможу, взявши в позику
100 фунтів стерлінгів, віддати в позику 105 фунтів стерлінгів. Але з чого я можу
платити 5\%? З нових позик або, якщо я — держава, з податків. Якщо ж гроші
бере в позику промисловий капіталіст, то при зиску, скажімо, в 15\%, він повинен
5\% платити як процент, 5\% споживати (хоч його апетит зростає разом
з його доходом) і 5\% капіталізувати. Отже, доводиться припустити зиск в 15\%,
щоб можна було постійно платити 5\% як процент. Якщо процес триває далі, то
норма зиску упаде в наслідок уже викладених причин, скажімо, з 15\% до 10\%.
Але Прайс зовсім забуває, що процент в 5\% передбачає норму зиску в 15\%,
і лишає її такою самою і при нагромадженні капіталу. Йому взагалі немає ніякого
діла до дійсного процесу нагромадження, його діло тільки давати в позику
гроші, щоб вони припливали назад з процентами на проценти. Яким чином
це відбувається, йому цілком байдуже, бо це ж є природжена властивість
капіталу, що дає процент.
}

Ще вище витає він у своїх „Observations on Reversionary Payments
etc.“ London 1872: „1 шилінг, відданий у позику при народженні
нашого Спасителя“ [отже, певно, в Єрусалімському храмі]
„по 6\% з процентами на проценти, виріс би до більшої суми
золота, ніж могла б умістити вся сонячна система, перетворена
в кулю з діаметром, рівним орбіті Сатурна“. — „Тому держава
ніколи не може опинитися в скрутному становищі, бо при найменших
заощадженнях вона може сплатити найбільший борг за
такий короткий час, як того можуть вимагати її інтереси“ (стор.
XIII, XIV). Який прекрасний теоретичний вступ до англійського
державного боргу!

Прайса просто засліпила величезність числа, яка виникає з геометричної
прогресії. Через те що він розглядав капітал, не
\parbreak{}  %% абзац продовжується на наступній сторінці
