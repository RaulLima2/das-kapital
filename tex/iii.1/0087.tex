шення загальних витрат капіталу в наслідок застосування дорогих
машин і т. д., а тому й зниження норми зиску, обчислюваної
на весь капітал, то ця остання мусить підвищитись. І це, безперечно,
має місце при багатьох з найновіших успіхів металургії
і хемічної промисловості. Нововідкриті способи виготовлення
заліза й сталі — Бессемера, Сіменса, Гількріста-Томаса та інших —
скорочують до мінімуму, при відносно незначних витратах,
надзвичайно довгочасні раніш процеси. Виготовлення алізарину
або красильної речовини крапу з кам’яновугільного дьогтю дає
за кілька тижнів, і до того ж при фабричних приладдях, які
вже раніш уживалися для виготовлення фарб з кам’яновугільного
дьогтю, такий самий результат, який раніше вимагав цілих
років; один рік був потрібний для росту крапу, а потім ще
кілька років коріння лишали достигати, раніше ніж уживати
його для фарбування.

Головний засіб скорочення часу циркуляції є поліпшені
шляхи сполучення. І в цьому останні п’ятдесят років зробили
революцію, яку можна порівняти тільки з промисловою революцією
останньої половини минулого століття. На суходолі макадамізовані\footnote*{
Макадамізування — спосіб брукування шляхів за системою Мак-Адама,
при якому скальне каміння укочується круглими котками. Ред. укр. перекладу,
} шляхи відтиснені на задній план залізницею, на
морі повільне і нерегулярне вітрильне сполучення — швидким
і регулярним пароплавним сполученням, і вся земна куля обвивається
телеграфними дротами. Власне кажучи, тільки Суецький
канал і відкрив Східну Азію і Австралію для пароплавного сполучення.
Час циркуляції для товарів, що посилалися до Східної
Азії, який ще в 1847 році становив щонайменше дванадцять
місяців (див. книгу II, стор. 250 \footnote*{
Стор. 173 рос. вид. 1935 р. Ред. укр. перекладу.
}), тепер можна звести майже
до стількох же тижнів. Два великі огнища криз 1825—1857 рр.,
Америка і Індія, в наслідок цього перевороту в засобах сполучення
наблизились до європейських промислових країн на
70—90\% і тим самим утратили більшу частину своєї здатності
до вибухів. Час обороту всієї світової торгівлі скоротився
в такій самій мірі, а дієздатність капіталу, який бере в ній
участь, підвищилась більше, ніж удвоє чи утроє. Що це не
лишилось без впливу на норму зиску, зрозуміло само собою.

Щоб представити в чистому вигляді вплив обороту всього
капіталу на норму, зиску ми мусимо при порівнянні двох
капіталів припустити, що всі інші обставини однакові. Отже,
крім норми додаткової вартості і робочого дня, нехай буде
однаковий і процентний склад капіталів. Візьмім тепер капітал
А з складом 80с + 20v = 100К, що обертається двічі на рік
при нормі додаткової вартості в 100\%. Тоді річний продукт
буде:

160 с + 40 v + 40 m. Але для визначення норми зиску ми обчисляємо
ці 40 m не на капітальну вартість у 200, що оберну-