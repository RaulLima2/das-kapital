стично (наприклад, землеробство у дрібних селян), вклинюються
між капіталістичні підприємства і переплітаються з ними. Нарешті,
— велика густота населення. — Пункт 2-й передбачає скасування
всіх законів, які перешкоджають робітникам переселятись
з однієї сфери виробництва в іншу або з одного місцевого
центру виробництва до якогонебудь іншого. Байдуже ставлення
робітника до змісту його праці. Якомога більше зведення праці
в усіх сферах виробництва до простої праці. Зникнення всіх професійних
передсудів у робітників. Нарешті — і це особливо —
підпорядкування робітника капіталістичному способові виробництва.
Дальший виклад цього питання належить до спеціального
дослідження конкуренції.

Із сказаного випливає, що кожний окремий капіталіст, як
і сукупність усіх капіталістів кожної окремої сфери виробництва,
бере участь в експлуатації сукупного робітничого класу сукупним
капіталом і в ступені цієї експлуатації не тільки в силу
загальної класової симпатії, але й безпосередньо економічно,
бо — якщо припустити всі інші умови, в тому числі і вартість
сукупного авансованого сталого капіталу, даними — пересічна
норма зиску залежить від ступеня експлуатації сукупної
праці сукупним капіталом.

Пересічний зиск збігається з пересічною додатковою вартістю,
яку капітал виробляє на кожні 100 одиниць; відносно додаткової
вартості щойно сказане зрозуміле само собою. Щождо пересічного
зиску, то сюди приєднується ще як один з моментів,
які визначають норму зиску, тільки вартість авансованого капіталу.
Справді, особливий інтерес, що його має капіталіст або
капітал певної сфери виробництва в експлуатації безпосередньо
занятих ним робітників, обмежується тим, щоб за допомогою
винятково надмірної праці, або за допомогою зниження заробітної
плати нижче пересічного рівня, абож за допомогою виняткової
продуктивності вживаної праці одержати додаткову вигоду,
одержати такий зиск, що перевищує пересічний. Якщо залишити
це осторонь, то капіталіст, який зовсім не вживає у своїй сфері
виробництва змінного капіталу, отже й робітників (що в дійсності,
звичайно, неможливо), був би так само дуже заінтересований
в експлуатації робітничого класу капіталом і цілком так само
діставав би свій зиск з неоплаченої додаткової праці, як і, наприклад,
той капіталіст, який (знову таки в дійсності неможливе
припущення) вживає тільки змінний капітал, тобто витрачає весь
свій капітал на заробітну плату. Але ступінь експлуатації праці
при даному робочому дні залежить від пересічної інтенсивності
праці, а при даній інтенсивності — від довжини робочого дня.
Від ступеня експлуатації праці залежить висота норми додаткової
вартості, отже, при даній загальній масі змінного капіталу
— величина додаткової вартості, а тому й величина зиску.
Той спеціальний інтерес, що його капітал певної сфери виробництва,
в відміну від сукупного капіталу, має в експлуатації
