кової вартості лишаються в І — 10 штук = 10 фунтам стерлінгів,
в II — 20 штук = 20 фунтам стерлінгів, в III — 40 штук = 40 фунтам
стерлінгів.

Або, потретє, робочий день — різної довжини. Якщо 20 робітників
при однаковій інтенсивності працюють у І — дев’ять,
у II — дванадцять, у III — вісімнадцять годин на день, то весь їх
продукт 30 : 40 : 60 відноситься як 9 : 12 : 18, і тому що заробітна
плата в кожному випадку = 20, то знову лишається 10, відповідно
20 і 40 для додаткової вартості.

Отже, підвищення або зниження заробітної плати діє в зворотному
напрямі, підвищення або зниження інтенсивності праці
і здовження або скорочення робочого дня діє в тому самому
напрямі на висоту норми додаткової вартості, а тому, при незмінному
v/K, і на норму зиску.

2. m' і v змінюються, К не змінюється

В цьому випадку має силу пропорція:

p': p'1 = m' v/K : m'1 v1/K = m'v : m'1v1 = m : m1.

Норми зиску відносяться одна до одної, як відповідні маси
додаткової вартості.

Зміна норми додаткової вартості при незмінній величині змінного
капіталу означала зміну у величині й розподілі нововиробленої
вартості. Одночасна зміна v і m' так само завжди включає
інший розподіл, але не завжди зміну величини нововиробленої
вартості. Можливі три випадки:

a) Зміни v і m' відбуваються в протилежному напрямі, але
на однакову величину; наприклад:

80 с + 20 v + 10 m; m' = 50\%, p' = 10\%
90 с + 10 v + 20 m; m' = 200\%, p' = 20\%.

Нововироблена вартість в обох випадках однакова, отже, однакова
й кількість витраченої праці; 20 v + 10 m = 10 v + 20 m = 30.
Ріжниця тільки в тому, що в першому випадку 20 сплачується
як заробітна плата, а 10 лишається для додаткової вартості,
тимчасом як у другому випадку заробітна плата становить
тільки 10, а тому додаткова вартість — 20. Це єдиний випадок,
коли при одночасній зміні v і m' число робітників, інтенсивність
праці і довжина робочого дня лишаються незміненими.

b) Зміни m' і v відбуваються так само в протилежному напрямі,
але не на ту саму величину. Тоді перевага буде або на
стороні зміни v, або на стороні зміни m'.

I. 80 с + 20 v + 20 m; m' = 100\%, p' = 20\%
II.72 с + 28 v + 20 m; m' = 71 3/7\%, p' = 20\%
III. 84 с + 16 v + 20 m; m' = 125\%, p' = 20\%.
