\parcont{}  %% абзац починається на попередній сторінці
\index{iii1}{0114}  %% посилання на сторінку оригінального видання
обміну речовин у людини, почасти ту форму, в якій предмети
споживання лишаються після споживання їх. Отже, покидьки
виробництва в хемічній промисловості є побічні продукти, які
при незначному масштабі виробництва пропадають марно; залізні
стружки, які лишаються при фабрикації машин і знову входять
як сировинний матеріал у виробництво заліза і т. д. Екскременти споживання — це виділювані людиною
природні речовини,
рештки одягу у формі ганчірок і т. д. Екскременти споживання
мають найбільше значення для землеробства. Щодо застосування
їх, капіталістичне господарство відзначається колосальним марнотратством; у Лондоні, наприклад, воно
не знаходить кращого
застосування для екскрементів 4 \sfrac{1}{2} мільйонів людей, як з величезними витратами заражати ними Темзу.

Подорожчання сировинних матеріалів є, звичайно, спонукою
до використовування відпадів.

Загалом умовами цього повторного використання є: масовість
цих екскрементів, яка можлива тільки при роботах у великому
масштабі; поліпшення машин, завдяки чому речовини, які раніш
у своїй даній формі були непридатні до вжитку, переходять
у таку форму, в якій вони можуть бути використані в новому
виробництві; прогрес науки, особливо хемії, яка відкриває корисні властивості таких відпадів.
Правда, і в дрібному землеробстві, де поля обробляються як сади, як от у Ломбардії, південному Китаї
та Японії, також має місце значна економія
цього роду. Але, загалом кажучи, при цій системі продуктивність землеробства купується великим
марнотратством людської робочої сили, відтягуваної від інших сфер виробництва.

Так звані відпади відіграють значну роль майже в кожній
галузі промисловості. Так, наприклад, у грудневому фабричному звіті за 1863 рік [стор. 139]
наводиться як одна з головних
причин того, чому в Англії — як і в багатьох частинах Ірландії —
орендарі тільки неохоче й рідко сіють льон, ось що: „Значна
кількість відпадів... які відходять при обробітку льону в дрібних льонотіпальних фабриках, де
рушійною силою є вода
(scutch mills)... Відпадів від бавовни порівняно небагато, а при
обробленні льону їх дуже багато. Старанна робота при мочінні і механічному тіпанні льону може значно
обмежити цю
втрату... В Ірландії льон часто тіпають надзвичайно незадовільним способом, так що 28—30\% його
пропадало марно“; усе це
могло б бути усунене при застосуванні кращих машин. Костриця при цьому відпадає в такій великій
кількості, що фабричний інспектор каже: „З деяких тіпальних фабрик в Ірландії
мене повідомили, що тіпальники часто вживають у себе дома
відпади, які утворюються на цих фабриках, як паливний матеріал для своїх печей, а це ж дуже цінний
матеріал“ („Rep. of
Insp. of Fact. Oct., 1863“, стор. 140). Про відпади бавовни мова
буде далі, там, де ми розглядаємо коливання цін на сировинний
матеріал.
