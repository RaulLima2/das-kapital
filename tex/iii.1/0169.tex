падало б додаткової вартості або зиску у А = 25, B = 40, C = 15, D = 10; разом = 90; отже, якщо всі
чотири капітали є рівновеликі, пересічна норма зиску є = 90/4 = 22 1/\footnote{
) розподілом сукупного суспільного капіталу між цими різними сферами, отже, відносною величиною
капіталу, вкладеного
в кожну окрему сферу, і отже з окремою нормою зиску, тобто
відносною масою сукупного суспільного капіталу, яку поглинає
кожна окрема сфера виробництва.

В I і II книгах ми мали справу тільки з вартостями товарів.
Тепер, з одного боку, відокремились витрати виробництва, як
частина цієї вартості, з другого боку, розвинулась ціна виробництва товару, як перетворена форма
вартості товару.

Припустім, що склад пересічного суспільного капіталу є
80 c + 20 v, а норма річної додаткової вартості m' = 100\%; тоді
річний пересічний зиск для капіталу в 100 буде = 20, а загальна
річна норма зиску = 20\%. Хоч би які були витрати виробництва
товарів, вироблених за рік капіталом в 100, їх ціна виробництва була б = k + 20. У сферах
виробництва, де склад капіталу = (80 — x) c + (20 x) v, дійсно створена додаткова вартість,
відповідно річний зиск, вироблений у цій сфер і вироб-
}\%.

Але якщо загальні величини цих капіталів будуть: А = 200, B = 300, C = 1000, D = 4000, то вироблені
зиски будуть відповідно 50, 120, 150 і 400. Разом на 5500 капіталу зиску буде 720, або пересічна
норма зиску буде 13 1/11\%.

Маси всієї виробленої вартості є різні залежно від різних загальних величин відповідних капіталів,
авансованих в А, В, C, D.
Тому при утворенні загальної норми зиску справа йде не тільки
про ріжницю норм зиску в різних сферах виробництва, з яких
просто треба було б вивести пересічну, але й про відносну вагу,
з якою ці різні норми зиску входять в утворення пересічної.
Але це залежить від відносної величини капіталу, вкладеного
в кожну окрему сферу виробництва, тобто від того, яку частину
сукупного суспільного капіталу становить капітал, вкладений
в кожну окрему сферу виробництва. Дуже велика ріжниця мусить, звичайно, бути залежно від того, чи
більша чи менша
частина сукупного капіталу дає вищу або нижчу норму зиску.
Але це знов таки залежить від того, скільки капіталу вкладено
в ті сфери виробництва, в яких відношення змінного капіталу
до всього капіталу є високе або низьке. Тут справа стоїть цілком
так само, як з пересічним процентом, що його одержує лихвар,
який віддає в позику різні капітали за різні норми процента, наприклад, за 4, 5, 6, 7\% і т. д.
Пересічна норма цілком залежить
від того, скільки з свого капіталу він позичив за кожну з цих
різних норм процента.

Отже, загальна норма зиску визначається двома факторами:

1) органічним складом капіталів у різних сферах виробництва,
отже, різними нормами зиску в окремих сферах;