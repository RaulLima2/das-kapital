стат, розуміється, дає протягом робочих годин настільки ж менше
тканини“ (там же, стор. 42, 43).

„В Аштоні, Стелібріджі, Мослеї, Ольдгемі і т. д. робочий
час скорочений на цілу третину, і з кожним тижнем робочі години
скорочуються ще більше... Одночасно з цим скороченням
робочого часу в багатьох галузях відбувається також зниження
заробітної плати“ (стор. 13). — На початку 1861 року стався
страйк механічних ткачів у деяких частинах Ланкашіра. Деякі
фабриканти заявили про зниження заробітної плати на 5—7 1/2%;
робітники настоювали на тому, щоб рівень заробітної плати лишити
незмінним, а робочий день скоротити: Фабриканти на це не
згодились, і почався страйк. Через місяць робітники мусили поступитися.
Але тепер вони одержали і те і друге: „Крім зниження
заробітної плати, на що робітники кінець-кінцем згодились,
вони на багатьох фабриках працюють тепер неповний час“.
(„Rep. of Insp. of Fact., April 1861“, стор. 23).

1862 рік. Квітень. „Страждання робітників від часу мого
останнього звіту значно збільшились; але ще ніколи в історії
промисловості такі раптові і такі тяжкі страждання не переносилися
з такою мовчазною покірливістю і таким терпеливим
самовладанням“ („Rep. of Insp. of Fact., April 1862“, стор. 10). —
„Відносне число цілком безробітних робітників в даний момент,
здається, не дуже перевищує число безробітних 1848 року,
коли панувала звичайна паніка, яка, однак, була досить значною,
щоб спонукати занепокоєних фабрикантів складати такі
самі статистичні відомості про бавовняну промисловість, які
тепер публікують щотижня... В травні 1848 року з усіх бавовняних
робітників Манчестера 15% було без роботи, 12% працювало
неповний час, тоді як понад 70% працювало повний час. 28 травня
1862 року без роботи було 15%, 35% працювало неповний час,
49% — повний час... В сусідніх місцевостях, наприклад, в Стокпорті,
процент тих, що працюють неповний час, і тих, що зовсім
не працюють, вищий, процент тих, що працюють повний час,
нижчий“, бо тут випрядаються грубіші нумери, ніж у Манчестері
(стор. 16).

1862 рік. Жовтень. „За останніми офіціальними статистичними
даними, в 1861 році в Сполученому Королівстві було 2887 бавовняних
фабрик, з них 2109 в моїй окрузі (Ланкашір і Чешір). Я, звичайно,
знав, що дуже значна частина з цих 2109 фабрик моєї округи
були дрібні підприємства, які вживали небагато робітників. Я
був, однак, дуже здивований, коли виявив, як багато таких підприємств.
В 392, або 19%, рушійна сила, пара або вода, менша за
10 кінських сил; в 345, або 16%, між 10 і 20 кінськими силами;
в 1372 — 20 кінських сил і більше... Дуже значна частина цих
дрібних фабрикантів — більше ніж третина загального числа їх —
не дуже давно самі були робітниками; це — люди, які не мають
у своєму розпорядженні капіталу... Центр ваги падає, отже, на
інші 2/3“ („Rep. of Insp. of Fact., Oct. 1862“, стор. 18, 19).
