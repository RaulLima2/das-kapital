\parcont{}  %% абзац починається на попередній сторінці
\index{iii1}{0309}  %% посилання на сторінку оригінального видання
у грошовій формі, як резервний фонд платіжних і купівельних
засобів. Це — перша форма скарбу, в якій він знову з’являється
при капіталістичному способі виробництва і взагалі утворюється
при розвитку торговельного капіталу, принаймні для цього
останнього. І те й друге стосується як до внутрішньої, так і до
міжнародної циркуляції. Цей скарб перебуває постійно в текучому
стані, постійно вливається в циркуляцію і постійно вертається
з неї. Друга форма скарбу — це бездіяльний, в даний
момент незанятий капітал у грошовій формі; сюди належить
також і новонагромаджений грошовий капітал, який ще не
вкладено в діло. Функції, яких вимагає це скарботворення як
таке, є насамперед зберігання скарбу, ведення книг і т. д.

Але, \emph{подруге}, з цим зв’язане витрачання грошей при закупівлях,
одержування грошей при продажах, виплачування і одержування
грошей при платежах, вирівнювання платежів і т. д.
Торговець грішми виконує все це для купців і промислових
капіталістів насамперед як простий \emph{касир}.\footnote{
„Інститут касирів, мабуть, ніде не зберіг у такій чистоті свій первісний,
самостійний характер, як у нідерландських торговельних містах (див. про
походження касирської справи в Амстердамі \emph{Е. Luzac}: „Holland’s Rijkdom“,
[Leyden 1782.] частина III). їх функції почасти збігаються з функціями старого
Амстердамського розмінного банку. Касир одержує від купців, які користуються
його послугами, певну суму грошей і відкриває їм на цю суму „кредит“
у своїх книгах; далі вони посилають йому свої боргові вимоги, по яких він
одержує для них гроші і кредитує їх на відповідні суми; навпаки, він робить
платежі за їх переказами (kassiers briefjes) і зменшує на ці суми їх поточний
рахунок. Від цих надходжень і виплат він відраховує собі незначний процент
за комісію, який дає йому відповідну плату за його працю тільки завдяки
значним оборотам, які він опосереднює між обома сторонами. Якщо двом купцям,
яких обслуговує той самий касир, доводиться робити взаємні платежі, то
такі платежі проводяться дуже просто за допомогою взаємних записів у книгах,
в яких касири день-у-день балансують взаємні вимоги купців. Отже, в цьому
опосередненні платежів і полягає власне заняття касирів; воно, отже, виключає
промислові підприємства, спекуляції і відкриття бланкового кредиту; бо тут
мусить бути правилом, що касир не робить за того, кому він відкрив у своїх
книгах рахунок, ніяких платежів, які перевищували б розмір його активу“
(\emph{Vissering}: „Handboek van Praktische Staatshuishoudkunde". Amsterdam 1860,
стор. 134). — Про касові союзи у Венеції: „В наслідок потреби і характеру
місцевості Венеції, де перенесення готівки є труднішим, ніж в інших місцях,
гуртові торговці цього міста заснували касові союзи з належною гарантією,
контролем і управлінням, куди члени такого союзу вносили певні суми, на
які вони видавали перекази своїм кредиторам; після цього виплачена сума
списувалась у заведеній для цього книзі з рахунку боржника і додавалась до
тієї суми, яку мав у ній кредитор. Це — перші початку так званих жиро-банків.
Ці союзи, безперечно, старі. Але, коли їх відносять до XII століття, то їх
змішують з установою для державних позик, заснованою в 1171 році“ (\emph{Нüllmann}:
„Städtewesen des Mittelalters". Bonn 1826—29, І, стор. 453 і далі).
}

Торгівля грішми розвивається повністю, — і це завжди відбувається
вже й на перших початках її, — коли з іншими її функціями
сполучається функція давання і одержування позик і торгівля
в кредит. Про це в дальшому відділі, де мова йде про капітал,
що дає процент.
