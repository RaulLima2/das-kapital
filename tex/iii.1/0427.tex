саму суму, то вони, отже, представляють тепер капітал. І при
тому вони однаково представляють капітал як у тому випадку,
коли вони застосовуються для позик капіталістам, так і в тому
випадку, коли пізніше, при зменшенні попиту на такі грошові
позики, вони знову застосовуються для вкладення в цінні папери.
В усіх цих випадках слово капітал уживається тут тільки
в банкірському значенні, при чому воно означає, що банкір змушений
давати позики на суму більшу, ніж просто його кредит.

Як відомо, Англійський банк видає всі свої позики своїми
банкнотами. Якщо ж, не зважаючи на це, циркуляція банкнот
банку, як правило, зменшується в міру того, як збільшуються
в його руках дисконтовані векселі і застави під позики, отже,
збільшуються видані ним позики, — то що робиться з банкнотами,
пущеними в циркуляцію, яким чином припливають вони
назад до банку?

Насамперед, якщо попит на грошові позики викликається несприятливим
національним платіжним балансом і, отже, відпливом
золота, то справа дуже проста. Векселі дисконтуються в банкнотах.
Банкноти в самому банку, в Issue department [емісійному
департаменті] обмінюються на золото, і золото експортується.
Це те саме, як коли б банк безпосередньо при дисконті
векселів прямо платив золотом, без посередництва банкнот. Таке
підвищення попиту — яке досягає в певних випадках від 7 до
10 мільйонів фунтів стерлінгів — не додає, звичайно, до внутрішньої
циркуляції країни жодної п’ятифунтової банкноти. Якщо ж
кажуть, що банк при цьому дає в позику капітал, а не засоби
циркуляції, то це має двоякий сенс. Поперше, що він дає в позику
не кредит, а дійсну вартість, частину свого власного або
покладеного до нього як вклад капіталу. Подруге, що він дає
в позику гроші не для внутрішньої циркуляції, а для міжнародної
циркуляції, дає в позику світові гроші; а для цієї мети гроші
завжди мусять існувати в своїй формі скарбу, в своїй металічній
тілесності; в формі, в якій вони не тільки є формою вартості, але
самі дорівнюють тій вартості, грошовою формою якої вони є. Хоч
це золото як для банку, так і для експортуючого торговця золотом
представляє капітал, банкірський капітал або купецький капітал,
проте попит на нього виникає не як попит на капітал, а як попит
на абсолютну форму грошового капіталу. Він виникає саме в той
момент, коли закордонні ринки переповнені англійським товарним
капіталом, який не може бути реалізований. Отже, те, чого при
цьому вимагають, це — не капітал як капітал, а капітал як гроші, в
тій формі, в якій гроші є загальний товар світового ринку; а це є їх
первісна форма як благородного металу. Відпливи золота, отже,
не є a mere question of capital [чисте питання капіталу], як кажуть
Фуллартон, Тук та інші, a a question of money [питання грошей],
хоч і в специфічній функції. Те, що це не є питання
внутрішньої циркуляції, як твердять прихильники теорії currency,
зовсім не є доказом, що це є просте question of capital, як дума-
