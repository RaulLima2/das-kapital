Наприклад:

80 с + 20 v + 20 m; m' = 100%, p' = 20%
80 с + 20 v + 10 m; m' = 50%, p' = 10%
20% : 10% = 100 × 20 : 50 × 20 = 20 m : 10 m.

Тепер ясно, що при капіталах однакового абсолютного чи
процентного складу норми додаткової вартості можуть бути
різні тільки в тому випадку, коли різні або заробітна плата,
або довжина робочого дня, або інтенсивність праці. В трьох
випадках:

I.  80 с + 20 v + 10 m; m' = 50%, p' = 10%,
II. 80 с + 20 v + 20 m; m' = 100%, p' = 20%,
III. 80 с + 20 v + 40 m; m' = 200%, p' = 40%,

вся нововироблена вартість буде в І 30 (20 v + 10 m), в II — 40,
в III — 60. Це може статись трояким способом.

Поперше, якщо заробітні плати різні, отже, якщо 20 v в кожному
окремому випадку виражає різне число робітників. Припустім,
що в І занято 15 робітників 10 годин при заробітній
платі в 1  1/3  фунтів стерлінгів і що вони виробляють вартість
у 30 фунтів стерлінгів, з яких 20 фунтів стерлінгів заміщають
заробітну плату, а 10 фунтів стерлінгів лишаються для додаткової
вартості. Якщо заробітна плата падає до 1 фунта стерлінгів,
то можуть бути заняті 20 робітників 10 годин; тоді вони
виробляють вартість у 40 фунтів стерлінгів, з яких 20 фунтів
стерлінгів для заробітної плати і 20 фунтів стерлінгів додаткової
вартості. Якщо заробітна плата падає ще далі, до 2/3 фунтів
стерлінгів, то можуть бути заняті 30 робітників по 10 годин,
які виробляють вартість у 60 фунтів стерлінгів, що з них після
відрахування 20 фунтів стерлінгів для заробітної плати залишиться
ще 40 фунтів стерлінгів для додаткової вартості.

Цей випадок: незмінний процентний склад капіталу, незмінний
робочий день, незмінна інтенсивність праці, зміна норми
додаткової вартості, спричинена зміною заробітної плати — є
єдиний випадок, на якому справджується положення Рікардо:
„profits would be high or low, exactly in proportion as wages
would be, low or high“ [„зиск буде високий чи низький точно
в такій пропорції, в якій заробітна плата буде низька чи висока“]
(„Principles of Political Economy“, розд. І, відділ III, стор. 18.
„Works of D. Ricardo“, вид. Mac Culloch, 1852).

Або, подруге, якщо інтенсивність праці різна. Тоді, наприклад,
20 робітників при однакових засобах праці за 10 робочих
годин на день виробляють у І — 30, у II — 40, у III — 60 штук
певного товару, кожна штука якого, крім вартості спожитих
на неї засобів виробництва, представляє нову вартість в 1 фунт
стерлінгів. Через те що в кожному випадку 20 штук, = 20
фунтам стерлінгів, заміщають заробітну плату, то для додат-
