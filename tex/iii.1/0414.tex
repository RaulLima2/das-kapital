загальної норми зиску, тому що ці підприємства, в яких сталий
капітал стоїть у такій колосальній пропорції до змінного, не повинні неодмінно брати участь у
вирівненні загальної норми зиску.

[З того часу, як Маркс написав ці рядки, розвинулися, як
відомо, нові форми промислового виробництва, які являють
собою другий і третій ступінь акційного товариства. Щоденно
зростаючій швидкості, з якою нині в усіх галузях великої промисловості може бути збільшене
виробництво, протистоїть дедалі більша повільність розширення ринку для цієї збільшеної
кількості продуктів. Що промисловість виготовляє за місяці,
те ринок ледве може поглинути за кілька років. До цього долучається політика охоронних мит, за
допомогою якої кожна
промислова країна відгороджує себе від інших, і особливо від
Англії, і ще штучно підвищує вітчизняну виробничу спроможність.
Наслідком цього є загальна хронічна перепродукція, низькі ціни,
падаючий і навіть зовсім зникаючий зиск; одним словом, здавна
уславлена свобода конкуренції дійшла свого кінця і мусить
сама оповістити про своє явне скандальне банкрутство. І при
тому таким чином, що в кожній країні великі промисловці певної
галузі об’єднуються в картель для регулювання виробництва.
Комітет твердо встановлює кількість товарів, яку має виробляти кожне підприємство, і розподіляє
остаточно замовлення,
що надходять. В окремих випадках утворювались іноді навіть
міжнародні картелі, так, наприклад, між англійською і німецькою залізною промисловістю. Але й цієї
форми усуспільнення
виробництва було недосить. Протилежність інтересів окремих
фірм занадто часто проривала її і відновлювала конкуренцію. Так дійшли до того, що в окремих
галузях, де це
дозволяв ступінь виробництва, стали концентрувати сукупне
виробництво цієї галузі підприємств в одно велике акційне товариство з єдиним керівництвом. В
Америці це здійснювалось
уже не раз, в Европі найбільшим прикладом цього досі є United
Alcali Trust, який сконцентрував усе британське виробництво
калію в руках одної-єдиної фірми. Колишні власники —
більше тридцяти — окремих підприємств за всі свої капіталовкладення одержали в акціях їх встановлену
за оцінкою
вартість, загалом до 5 мільйонів фунтів стерлінгів, які представляють основний капітал тресту.
Технічна адміністрація лишається в старих руках, а комерційне керівництво сконцентроване в руках
генеральної дирекції. Обіговий капітал (floating
capital) на суму приблизно в один мільйон фунтів стерлінгів був
запропонований публіці для підписки. Отже, сукупний капітал
тресту становить 6 мільйонів фунтів стерлінгів. Таким шляхом
у цій галузі, яка становить основу всієї хемічної промисловості,
в Англії конкуренція замінена монополією і якнайзадовільніше
підготовляється майбутня експропріація всім суспільством,
нацією. — Ф. Е.]

Це — скасування (Aufhebung) капіталістичного способу вироб-
