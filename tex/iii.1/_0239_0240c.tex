\index{iii1}{0239}  %% посилання на сторінку оригінального видання
Розділ п’ятнадцятий

Розвиток внутрішніх суперечностей закону
І. Загальні зауваження

В першому відділі цієї книги ми бачили, що норма зиску завжди виражає норму додаткової вартості
нижчою, ніж вона є. Тепер ми побачили, що навіть зростаюча норма додаткової вартості має тенденцію
виражатись у падаючій нормі зиску. Норма
зиску дорівнювала б нормі додаткової вартості тільки в тому випадку, коли с було б = 0, тобто коли б
увесь капітал витрачався на заробітну плату. Падаюча норма зиску тільки тоді виражає падаючу норму
додаткової вартості, коли відношення між вартістю сталого капіталу і масою робочої сили, яка
приводить його в рух, лишається незмінним, або коли ця остання збільшується у відношенні до вартості
сталого капіталу.

Рікардо, досліджуючи, як він гадав, норму зиску, в дійсності досліджував тільки норму додаткової
вартості і цю останню тільки при тому припущенні, що робочий день щодо інтенсивності й довжини є
стала величина.

Падіння норми зиску і прискорене нагромадження лиш остільки є різні вирази одного й того ж процесу,
оскільки і те і друге є виразом розвитку продуктивної сили. Нагромадження, з свого боку, прискорює
падіння норми зиску, оскільки разом з ним
дана концентрація робіт у великому масштабі, а тому й вищий склад капіталу. З другого боку, падіння
норми зиску знову таки прискорює концентрацію капіталу і його централізацію шляхом експропріації
дрібних капіталістів, шляхом експропріації
останніх решток безпосередніх виробників, у яких лишається ще щонебудь експропріювати. В наслідок
цього, з другого боку, прискорюється — щодо маси — нагромадження, хоча з падінням норми зиску падає
і норма нагромадження.

З другого боку, оскільки норма зростання вартості всього капіталу, норма зиску, є стимулом
капіталістичного виробництва (подібно до того, як збільшення вартості капіталу є його єдиною метою),
падіння цієї норми уповільнює утворення нових самостійних капіталів і виступає таким чином як
загроза для розвитку капіталістичного процесу виробництва; воно сприяє перепродукції, спекуляції,
кризам, утворенню надмірного капіталу поряд з надмірним населенням. Отже, ті економісти, які,
подібно до Рікардо, вважають капіталістичний спосіб виробництва за абсолютний, відчувають тут, Що
цей спосіб виробництва сам собі створює межу, і тому приписують цю межу не виробництву, а природі (в
ученні про ренту). Але важливим в їх жаху перед падаючою нормою зиску є відчуття того, що
капіталістичний спосіб виробництва в розвитку продуктивних сил має таку межу, яка не стоїть ні в
якому зв’язку з виробництвом багатства як таким;
\index{iii1}{0240}  %% посилання на сторінку оригінального видання
і ця особлива межа свідчить про обмеженість і тільки
історичний, минущий характер капіталістичного способу виробництва;
свідчить про те, що він не є абсолютний спосіб виробництва
для виробництва багатства і що, навпаки, на певному
ступені він вступає в конфлікт із своїм дальшим розвитком.

Рікардо і його школа розглядають, в усякому разі, тільки промисловий
зиск, в якому міститься і процент. Але й норма земельної
ренти має тенденцію до падіння, хоч її абсолютна маса зростає
і хоч вона може зростати й відносно, порівняно з промисловим
зиском (див. Ед. Уест [„Essay on Application of Capital to Land“,
Лондон 1815], який виклав закон земельної ренти раніше від
Рікардо). Якщо ми розглядатимем сукупний суспільний капітал
К і, позначимо через р1 той промисловий зиск, який залишається
після відрахування процента й земельної ренти, через z
процент і через r земельну ренту, то m/K = p/K = (p1 + z + r)/K =
p1/K + z/K + r/K. Ми бачили, що хоч у ході розвитку капіталістичного
виробництва сукупна сума додаткової вартості, m, постійно
зростає, проте m/K так само постійно зменшується, бо К
зростає ще швидше, ніж m. Отже, немає ніякої суперечності
в тому, що p1, z і r, кожне само по собі, можуть постійно зростати,
тимчасом як m/K = p/K, а також p1/K, z/K і r/K, кожне само
по собі, постійно зменшуються, або що р1 порівняно з z, або
r порівняно з p1, абож порівняно з p1 і z відносно зростає.
При зростаючій сукупній додатковій вартості або зиску m = p, але
при одночасно падаючій нормі зиску m/K = p/K відношення величин
частин p1, z і r, на які розпадається m = p, може як завгодно
змінюватись у межах, даних сукупною сумою m, при чому
на величину m або m/K це не впливає.

Взаємна зміна p1, z і r є тільки різний розподіл m між різними
рубриками. Тому і p1/K, z/K або r/K, норма індивідуального
промислового зиску, норма процента і відношення ренти до
сукупного капіталу можуть підвищуватись одно порівняно з одним,
хоч m/K, загальна норма зиску, падає; умовою при цьому
лишається тільки те, щоб сума всіх трьох = m/K. Якщо норма
зиску падає з 50\% до 25\%, якщо, наприклад, склад капіталу,
при нормі додаткової вартості в 100\%, змінюється з 50 с + 50 v
у 75 c + 25 v, то в першому випадку капітал в 1000 дасть зиск
\parbreak{}  %% абзац продовжується на наступній сторінці
