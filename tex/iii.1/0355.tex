тих капіталістів, відношенням яких до числа позичальників визначається
норма процента“] (Th. Tooke: „History of Prices etc.
from 1793 to 1837“. London 1838, II, стор. 355 [356]). Справді, тільки
поділ капіталістів на грошових капіталістів і промислових капіталістів
перетворює частину зиску в процент, взагалі створює
категорію процента; і тільки конкуренція між цими двома видами
капіталістів створює розмір процента.

Поки капітал функціонує в процесі репродукції, — навіть при
тому припущенні, що він належить самому промисловому капіталістові,
так що цей останній не повинен повертати його позикодавцеві,
— доти в розпорядженні капіталіста як приватної
особи перебуває не сам цей капітал, а тільки зиск, який він
може витрачати як дохід. Поки його капітал функціонує як
капітал, він належить процесові репродукції, закріплений у ньому.
Хоча промисловий капіталіст і є власником капіталу, але ця власність
не дає йому змоги, поки він використовує його як капітал
для експлуатації праці, розпоряджатися ним ще якось інакше.
Цілком так само стоїть справа і з грошовим капіталістом. Поки
його капітал у позиці і тому діє як грошовий капітал, він
дає йому процент, частину зиску, але основною сумою він не
може розпоряджатись. Це виявляється кожного разу, коли грошовий
капіталіст віддає свій капітал у позику, наприклад, на
один або декілька років, і в певні строки одержує проценти,
не одержуючи при цьому назад капіталу. Але навіть повернення
йому капіталу не міняє тут справи. Якщо він одержує
його назад, то йому завжди знову доводиться віддавати його
в позику, доки він хоче зберегти для себе його діяння як
капіталу, в даному разі грошового капіталу. Доки капітал
перебуває в руках грошового капіталіста, він не дає процентів
і не діє як капітал; а доки він дає проценти і діє як капітал,
він перебуває не в його руках. Звідси можливість
віддавати капітал у позику на вічні часи. Тому цілком хибні
наведені нижче зауваження Тука проти Bosanquet. Він цитує
Bosanquet („Metallic, Paper, and Credit Currency“, стор. 73):
„Коли б рівень процента був знижений до 1\%, то взятий
у позику капітал був би поставлений майже на одну лінію (upon
a par) з власним капіталом“. До цього Тук робить таке зауваження:
„Що капітал, взятий у позику за такий або навіть ще
нижчий процент, слід вважати за такий, що стоїть майже на
одній лінії з власним капіталом, це таке дивне твердження, що
воно навряд чи заслуговувало б серйозної уваги, коли б не
йшло від такого глибокодумного і в окремих пунктах теми
добре обізнаного письменника. Невже ж він випустив з уваги
або вважає за незначну ту обставину, що його припущення
передбачає також і умову зворотної виплати?“ (Th. Tooke: „An
Inquiry into the Currency Principle“, 2 вид., Лондон 1844,
стор. 80). Коли б процент був = 0, то промисловий капіталіст,
який взяв у позику капітал, був би в такому ж становищі, що
