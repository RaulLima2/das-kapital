\parcont{}  %% абзац починається на попередній сторінці
\index{iii1}{0317}  %% посилання на сторінку оригінального видання
виробництва зв’язуються між собою за допомогою третього члена,
виражає двоякого роду обставини. З одного боку, воно виражає
те, що циркуляція ще не опанувала виробництва, а відноситься до
нього, як до даної передумови. З другого боку, те, що процес
виробництва ще не ввібрав у себе циркуляцію просто як свій
момент. Навпаки, в капіталістичному виробництві має місце і те
і друге. Процес виробництва цілком грунтується на циркуляції, а
циркуляція є лиш момент, перехідна фаза виробництва, лиш реалізація
продукту, виробленого як товар, і заміщення елементів
його виробництва, вироблюваних як товари. Форма капіталу, що
походить безпосередньо з циркуляції — торговельний капітал —
виступає тут лиш як одна з форм капіталу в його русі репродукції.

Той закон, згідно з яким самостійний розвиток купецького капіталу
стоїть у зворотному відношенні до ступеня розвитку капіталістичного
виробництва, з особливою ясністю виявляється в історії
посередницької торгівлі (carrying trade), наприклад, у венеціанців,
генуезців, голландців і т. д., отже, там, де головний бариш
добувається не за допомогою вивозу продуктів своєї країни,
а від посередництва при обміні продуктів громад, торговельно
і взагалі економічно ще не розвинених, та від експлуатації обох
країн виробництва.\footnote{
„Жителі торговельних міст привозили з багатших країн витончені мануфактурні
товари і дорогі предмети розкоші і таким чином давали поживу для
чванливості великих землевласників, які жадібно купували ці товари і сплачували
за них величезні маси сировинного продукту своїх земель. Таким чином торгівля
значної частини Европи в цей час полягала в обміні сировинного продукту
однієї країни на мануфактурні продукти країни, промислово більш розвиненої...
Як тільки цей смак став загальнопоширеним і викликав значний попит, купці,
щоб заощадити витрати провозу, почали засновувати подібні мануфактури
у своїй власній країні“ (\emph{A. Smith}: „Wealth of Nations“, книга III, розд. Ill [вид.
Wakefield, Лондон 1835/39, т. З, стор. 41 і далі]).
} Тут перед нами купецький капітал у чистому
вигляді, відокремлений від крайніх членів, від тих сфер
виробництва, між якими він є посередником. Таке є головне
джерело його утворення. Але ця монополія посередницької торгівлі,
а з нею й сама ця торгівля, занепадає в тій самій мірі,
в якій прогресує економічний розвиток тих народів, що їх вона
експлуатувала з двох сторін і нерозвиненість яких була базою
її існування. При посередницькій торгівлі це являє собою не
тільки занепад особливої галузі торгівлі, але й падіння переваги
чисто торговельних народів і їх торговельного багатства
взагалі, яке грунтувалось на базі цієї посередницької торгівлі.
Це — тільки особлива форма, в якій у ході розвитку капіталістичного
виробництва виражається підпорядкування торговельного
капіталу промисловому. Зрештою, яскравий приклад
того, як господарює купецький капітал там, де він прямо опановує
виробництво, становить не тільки колоніальне господарство
взагалі (так звана колоніальна система), але особливо господарство
старої голландсько-ост-індської компанії.
