\parcont{}  %% абзац починається на попередній сторінці
\index{iii1}{0051}  %% посилання на сторінку оригінального видання
товарної вартості, бо вона нічого не змінює в абсолютній величині
нової вартості, яку утворює текуча робоча сила. Навпаки,
така зміна впливає тільки на відношення величин тих двох складових
частин нової вартості, з яких одна становить додаткову
вартість, а друга заміщає змінний капітал і тому входить у витрати
виробництва товару.

Спільне обом частинам витрат виробництва, в нашому випадку
400 с + 100v тільки те, що обидві вони є частини товарної
вартості, які заміщають авансований капітал.

Але з точки зору капіталістичного виробництва цей дійсний
стан справи неминуче з’являється в перекрученому вигляді.

Капіталістичний спосіб виробництва відрізняється від способу
виробництва, основаного на рабстві, між іншим, тим, що
вартість, відповідно ціна робочої сили, виступає як вартість,
відповідно ціна, самої праці, або як заробітна плата (книга І,
розділ XVII). Тому змінна частина вартості авансованого капіталу
виступає як капітал, витрачений на заробітну плату,
як капітальна вартість, яка оплачує вартість, відповідно ціну,
всієї праці, витраченої на виробництво. Якщо ми припустимо,
наприклад, що пересічний суспільний робочий день в 10 годин
втілюється у грошовій масі в 6 шилінгів, то авансований
змінний капітал у 100 фунтів стерлінгів буде грошовим
виразом вартості, виробленої протягом ЗЗЗ \sfrac{1}{3} десятигодинних
робочих днів. Але ця вартість купленої робочої сили, яка фігурує
при авансуванні капіталу, не становить ніякої частини
дійсно функціонуючого капіталу. Замість неї у процес виробництва
вступає сама жива робоча сила. Якщо ступінь експлуатації
останньої становить, як у нашому прикладі, 100\%, то вона
витрачається протягом 666 \sfrac{2}{3} десятигодинних робочих днів і тому
додає до продукту нову вартість у 200 фунтів стерлінгів. Але
при авансуванні капіталу змінний капітал у 100 фунтів стерлінгів
фігурує як капітал, витрачений на заробітну плату, або як
ціна праці, яка виконується на протязі 666 \sfrac{2}{3}  десятигодинних
днів. 100 фунтів стерлінгів, поділені на 666 \sfrac{2}{3} , дають нам, як
ціну десятигодинного робочого дня, 3 шилінги, — вартість, створену
п’ятигодинною працею.

Якщо ми порівняємо тепер авансування капіталу, з одного
боку, і товарну вартість, з другого, то матимем:

I. Авансований капітал у 500 фунтів стерлінгів = 400 фунтам
стерлінгів капіталу, витраченого на засоби виробництва
(ціна засобів виробництва), + 100 фунтів стерлінгів капіталу,
витраченого на працю (ціна 666 \sfrac{2}{3}  робочих днів, або заробітна
плата за них).

II. Товарна вартість у 600 фунтів стерлінгів = витратам виробництва
у 500 фунтів стерлінгів (400 фунтів стерлінгів
ціна витрачених засобів виробництва + 100 фунтів стерлінгів
ціна витрачених 666 \sfrac{2}{3}  робочих днів) + 100 фунтів
стерлінгів додаткової вартості.
