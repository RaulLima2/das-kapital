водиться переборювати при реалізації теорії — її застосуванні до
процесу виробництва — і т. д.

Зауважмо, між іншим, що слід відрізняти загальну працю від
спільної праці. Обидві вони відіграють у процесі виробництва
свою роль, обидві переходять одна в одну, але обидві й
відрізняються між собою. Загальна праця — це всяка наукова
праця, всяке відкриття, всякий винахід. Вона зумовлюється почасти кооперацією сучасників, почасти
використанням праці попередників. Спільна праця передбачає безпосередню кооперацію індивідів.

Вищесказане дістає нове потвердження в таких часто спостережуваних обставинах:

1. У великій ріжниці між витратами першого будування нової
машини і витратами її репродукції, про що дивись у Юра
і Беббеджа.

2. У значно більших витратах, при яких узагалі ведеться підприємство, засноване на нових винаходах,
порівняно з витратами
підприємств, які пізніше виникають на його руїнах, ex suis ossibus
[на його кістках]. Це явище має таку вагу, що перші
підприємці здебільшого банкрутують і тільки пізніші підприємці,
в руки яких будівлі, машини і т. д. дістаються по дешевих цінах, процвітають. Тим-то в більшості
випадків найнікчемніша
і наймізерніша група грошових капіталістів дістає найбільшу
вигоду з усякого прогресу загальної праці людського розуму та
її суспільного застосування за допомогою комбінованої праці.

Розділ шостий

Вплив зміни цін

І. Коливання цін сировинного матеріалу, безпосередній вплив цих коливань на норму зиску

Ми припускаємо тут, як і раніше, що в нормі додаткової
вартости не відбувається ніякої зміни. Ця передумова потрібна
для того, щоб дослідити даний випадок в його чистому вигляді.
Було б, однак, можливо, що при незмінній нормі додаткової вартості певний капітал уживає до праці
зростаюче або меншаюче число робітників унаслідок тих скорочень або розширень, що їх викликають у
ньому коливання цін на сировинний матеріал, які належить тут розглянути. В цьому випадку,
маса додаткової вартості могла б змінюватися при сталій нормі
додаткової вартості. Однак, і це ми повинні тут усунути
як проміжний випадок. Якщо поліпшення машин і зміна цін
сировинного матеріалу впливають одночасно чи то на масу
робітників, вживаних даним капіталом, чи на висоту заробітної
плати, то треба тільки зіставити 1) вплив, який справляє зміна
сталого капіталу на норму зиску, 2) вплив, який справляє зміна
