єднуючи до нього як співвизначальний фактор уявлення, ділком
йому суперечне.

Або нагромаджена праця поряд з живою працею створює
вартість. Тоді закон вартості не має сили.

Або вона не створює вартості. Тоді доводи Шмідта несполучні
з законом вартості.

Шмідт збився з правильного шляху, коли він був уже дуже
близько до розв’язання проблеми, бо гадав, що треба обов’язково
знайти математичну формулу, яка дала б можливість довести
погодженість пересічної ціни кожного окремого товару з законом
вартості. Але якщо тут, бувши зовсім близько до мети, він
пішов хибним шляхом, то в усьому іншому зміст брошури показує,
з яким розумінням він зробив дальші висновки з обох перших
книг „Капіталу“. Йому належить честь самостійного відкриття
правильного пояснення непояснимої до того часу тенденції норми
зиску до зниження, пояснення, даного Марксом у третьому відділі
третьої книги; так само виведення торговельного зиску з
промислової додаткової вартості і цілий ряд уваг про процент
та земельну ренту, в яких ним передхоплені речі, розвинені у
Маркса в четвертому і п’ятому відділах третьої книги.

В одній пізнішій праці („Neue Zeit“ 1892/93, №№ 3 і 4)
Шмідт намагається розв’язати проблему іншим шляхом. Цей
шлях зводиться до того, що пересічну норму зиску встановлює
конкуренція, бо вона примушує капітал переходити з галузей
виробництва з недостатнім зиском до інших, де добувається
надзиск. Що конкуренція є велика зрівняльниця зисків, це не
новина. Але Шмідт намагається довести, що це нівелювання
зисків тотожне із зведенням продажної ціни товарів, вироблених
понад міру, до такої міри вартості, яку суспільство може
заплатити за них згідно з законом вартості. Чому і це не
могло привести до цілі, досить видно з пояснень Маркса в самій
книзі.

Після Шмідта до проблеми взявся П. Фіреман („Conrads
Jahrbücher“, Dritte Folge [1892], III, стор. 793). Я не спиняюся на
його увагах про інші сторони викладу в Маркса. Вони грунтуються
на тому непорозумінні, ніби Маркс хоче дати визначення
там, де він в дійсності розвиває, і на тому, що в Маркса взагалі
довелося б пошукати точних, готових, раз назавжди даних
дефініцій. Адже само собою зрозуміло, що там, де речі та їх
взаємовідношення розглядаються не як сталі, а як мінливі, їх
мислені відбитки, поняття, теж зазнають зміни та перетворення;
що їх не втискують у закам’янілі дефініції, а розглядають в їх
історичному або логічному процесі утворення. Після цього стане,
звичайно, ясно, чому Маркс на початку першої книги, де він
виходить з простого товарного виробництва, яке є для нього
історичною передумовою, щоб потім далі перейти від цієї бази
до капіталу, — чому він там виходить саме з простого товару,
а не з форми, логічно і історично вторинної, не з капіталістично
