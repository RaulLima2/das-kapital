творилося в гроші, ще не ввійшло як споживна вартість у споживання,
продуктивне чи особисте. Торговець полотном репрезентує
тепер на ринку той самий товарний капітал, який первісно
репрезентував на ньому виробник полотна. Для цього
останнього процес метаморфози скоротився, але тільки для того,
щоб продовжуватись у руках купця.

Коли б виробник полотна мусив чекати, поки його полотно
дійсно перестане бути товаром, поки воно перейде до останнього
покупця, продуктивного або особистого споживача, то
його процес репродукції був би перерваний. Абож для того, щоб
його не переривати, він мусив би обмежити свої операції, перетворювати
в пряжу, вугілля, працю і т. д., коротко кажучи,
в елементи продуктивного капіталу, меншу частину свого полотна,
а більшу частину його тримати в себе як грошовий резерв
для того, щоб, поки одна частина його капіталу як товар
перебуває на ринку, друга частина могла продовжувати процес
виробництва, так що коли одна частина його капіталу надходить
на ринок як товар, друга повертається назад у грошовій
формі. Цей поділ його капіталу не усувається втручанням купця.
Але без цього останнього частина капіталу циркуляції, наявна
у формі грошового резерву, завжди мусила б бути відносно
більшою порівняно з частиною, застосовуваною у формі продуктивного
капіталу, і відповідно до цього скоротився б масштаб
репродукції. Замість цього виробник може тепер постійно
застосовувати більшу частину свого капіталу на власне процес
виробництва і меншу частину як грошовий резерв.

Але зате тепер інша частина суспільного капіталу, в формі
купецького капіталу, постійно перебуває у сфері циркуляції.
Вона завжди застосовується тільки для того, щоб купувати
і продавати товари. Таким чином, видимо, відбувається тільки
переміна осіб, в руках яких перебуває цей капітал.

Коли б купець, замість того, щоб купити на 3000 фунтів
стерлінгів полотна з метою знову продати його, сам продуктивно
застосував би ці 3000 фунтів стерлінгів, то продуктивний капітал
суспільства збільшився б. Звичайно, тоді виробник полотна —
а так само й купець, який перетворився тепер у промислового
капіталіста — мусив би значнішу частину свого капіталу тримати
в себе як грошовий резерв. З другого боку, якщо купець
лишається купцем, то виробник заощаджує час, потрібний
для продажу, і може вживати його для нагляду за процесом
виробництва, тимчасом як купець мусить весь свій час уживати
для продажу.

Якщо купецький капітал не перевищує своїх необхідних пропорцій,
то слід визнати:

1) що в наслідок поділу праці капітал, який займається виключно
купівлею і продажем (а, крім грошей, потрібних для
купівлі товарів, сюди належать також гроші, які мусять витрачатись
на працю, необхідну для ведення торговельного підпри-
