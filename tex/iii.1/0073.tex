дають, в наслідок зміни v, друге рівняння:

р'1 = m' (v1/К'),

в якому v перейшло у v1 а р'1, змінена норма зиску, яка випливає
з цього, має бути знайдена.

Вона визначається за допомогою відповідної пропорції:

р': р'1 = m' (v/К): m' (v1/К) = v: v1

Або: при незмінній нормі додаткової вартості і незмінному цілому
капіталі первісна норма зиску відноситься до норми зиску,
що виникла в наслідок зміни змінного капіталу, як первісний
змінний капітал відноситься до зміненого.
Якщо капітал первісно був такий, як вище:

I. 15000 К = 12000 с + 3000 v (+ 3000 m); а тепер він:

II. 15000 К = 13000 с + 2000 v (+ 2000 m), то К = 15000

і m' = 100\% в обох випадках, а норма зиску І, 20\%, відноситься
до норми зиску II, 13 1/3\%, як змінний капітал І, 3000,
до змінного капіталу II, 2000, отже 20\%: 13 1/3\% = 3000: 2000.

Але змінний капітал може або підвищитись або зменшитись:
Візьмімо спочатку приклад, коли він підвищується. Нехай капітал
спочатку складається і функціонує так:

I. 100 с + 20 v + 10 m; К = 120, m' = 50\%, р' = 8 1/3\%

Нехай тепер змінний капітал підвищиться до 30; тоді, згідно
з припущенням, щоб весь капітал лишився незмінним = 120, сталий
капітал мусить зменшитися з 100 до 90. Вироблена додаткова
вартість, при тій самій нормі додаткової вартості в 50\%, мусить
підвищитись до 15. Отже, ми маємо:

II. 90 с + 30 v + 15 m; К = 120, m' = 50\%, р' = 12 1/2\%

Спочатку виходитимем з того припущення, що заробітна
плата не змінюється. Тоді інші фактори норми додаткової вартості,
робочий день і інтенсивність праці, теж мусять лишитись
незмінними. Отже, підвищення v (з 20 до 30) може мати тільки
те значення, що робітників уживається наполовину більше.
Тоді й уся нововироблена ними вартість підвищується наполовину,
з 30 до 45, і розподіляється цілком так само, як і раніш:
2/3 на заробітну плату і 1/3 на додаткову вартість. Але одночасно,
при збільшеному числі робітників, знизився сталий капітал,
вартість засобів виробництва, з 100 до 90. Отже, ми маємо
перед собою випадок меншаючої продуктивності праці, зв’язаний
з одночасним зменшенням сталого капіталу; чи є цей випадок
економічно можливий?
