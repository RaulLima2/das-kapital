виробництва товару = 1/2 + 17 1/2 + 2 = 20 шилінгам, пересічна
норма зиску 2/20 = 10\%, а ціна виробництва штуки товару дорівнює
його вартості = 22 шилінгам або маркам.

Припустім, що винайдено машину, яка наполовину скорочує
потрібну на кожну штуку товару живу працю, але зате
збільшує втроє ту частину вартості, яка складається з зношування
основного капіталу. Тоді справа стоятиме так: зношування
= 1 1/2 шилінгам, сировинний та допоміжний матеріал, як
і раніше, 17 1/2 шилінгів, заробітна плата 1 шилінг, додаткова
вартість 1 шилінг, разом 21 шилінг або 21 марка. Вартість товару
зменшилась тепер на 1 шилінг; нова машина безперечно підвищила
продуктивну силу праці. Але для капіталіста справа стоятиме
так: його витрати виробництва є тепер: 1 1/2 шилінги зношування,
17 1/2 шилінгів — сировинний і допоміжний матеріал,
1 шилінг — заробітна плата, разом 20 шилінгів, як і раніш.
Через те що норма зиску безпосередньо в наслідок застосування
нової машини не змінюється, він мусить одержати 10\%
понад витрати виробництва, що становить 2 шилінги; отже, ціна
виробництва лишилась незмінною = 22 шилінгам, але вона на 1 шилінг
вища вартості. Для суспільства, яке виробляє при капіталістичних
умовах, товар не став дешевшим, нова машина не являє
собою ніякого поліпшення. Отже, капіталіст не має ніякого
інтересу в тому, щоб вводити нову машину. А через те що
введенням нової машини він просто зробив би нічого невартою
свою стару, ще не зношену машину, перетворив би її просто
в старе залізо, отже, зазнав би позитивного збитку, то він дуже
стережеться такої утопічної для нього дурості.

Отже, для капіталу закон зростаючої продуктивної сили праці
має не безумовне значення. Для капіталу ця продуктивна сила
підвищується не тоді, коли взагалі заощаджується жива праця,
а тільки тоді, коли на оплачуваній частині живої праці заощаджується
більше, ніж додається минулої праці, як це вже коротко
зазначено було в книзі І, розділ XIII, 2, стор. 411*. Тут
капіталістичний спосіб виробництва впадає в нову суперечність.
Його історичне покликання — нестримний розвиток продуктивності
людської праці, підготований вперед у геометричній прогресії.
Він зраджує це покликання, оскільки він, як у даному
випадку, перешкоджає розвиткові продуктивності. Цим він тільки
знову доводить, що він хиріє від старості і все більше й більше
переживає себе.] 37

В конкуренції збільшення мінімуму капіталу, який з підвищенням
продуктивної сили стає потрібним для успішного ве-

* Стор. 297—298 рос. вид. 1935 р. Ред. укр. перекладу.

37 Вищенаведене стоїть у дужках, тому що хоч це і є переробка з примітки
оригіналу рукопису, але у викладі деяких моментів воно виходить за
межі того матеріалу, що є в оригіналі. — Ф. Е.
