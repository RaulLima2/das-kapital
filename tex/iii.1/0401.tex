характеризується тією пошаною, яку вселяла всій Англії „логіка“
мільйонера, цього dung-hill aristocrat’a [аристократа з смітника].
Зрештою, якщо висока норма зиску і розширення підприємств
можуть бути причинами високого розміру процента, то звідси
зовсім не випливає, що високий розмір процента є причина високого зиску. А питання полягає якраз у
тому, чи не тривав і далі
цей високий процент (як це дійсно показала криза) або навіть чи не піднявся він до своєї вищої точки
після того, як від
високої норми зиску вже не лишилося й помину.

„3718. Щодо значного підвищення норми дисконту, то це
обставина, яка цілком і повністю виникає із збільшеної вартості капіталу, а причину цього збільшення
вартості капіталу, я думаю, кожний може виявити з повною ясністю. Я вже згадував
той факт, що за 13 років, протягом яких мав силу цей банковий акт, торгівля Англії зросла від 45 до
120 мільйонів фунтів
стерлінгів. Подумайте про всі події, що про них говорять ці короткі цифрові дані; пригадайте
колосальний попит на капітал,
який викликав таке гігантське збільшення торгівлі; пригадайте
також, що природне джерело подання для цього величезного попиту, а саме щорічні заощадження країни,
було протягом останніх трьох або чотирьох років вичерпане на незисковні видатки
для воєнних цілей. Я, признаюсь, вражений тим, що розмір процента не підвищився ще більше; або,
іншими словами, я вражений тим, що скрута щодо капіталу в наслідок цих гігантських
операцій не стала ще більшою, ніж ви її вже спостерігали“.

Яка дивовижна плутанина слів у нашого лихварського логіка!
Він знову тут із своєю підвищеною вартістю капіталу! Як видно,
він уявляє собі, що на одному боці відбувалось це колосальне
розширення процесу репродукції, отже, нагромадження дійсного
капіталу, а на другому боці стояв „капітал“, на який виник
„колосальний попит“, щоб довести до кінця це таке гігантське
збільшення торгівлі! А хіба це гігантське збільшення виробництва
не становило самого збільшення капіталу, і якщо воно створило
попит, то хіба воно разом з тим не створило й подання, і разом
з тим навіть збільшеного подання грошового капіталу? Якщо
розмір процента піднявся дуже високо, то це ж тільки тому,
що попит на грошовий капітал ріс ще швидше, ніж подання,
що, іншими словами, зводиться до того, що з розширенням промислового виробництва розширилось і
провадження його на
базі кредитної системи. Іншими словами, дійсне розширення
промисловості спричинило збільшений попит на „позики“, і цей
останній попит є, очевидно, те, що наш банкір розуміє під „колосальним попитом на капітал“.
Звичайно, не розширення простого попиту на капітал підвищило експортну торгівлю з 45 до
120 мільйонів. І що розуміє Оверстон далі, коли каже, що
поглинені Кримською війною щорічні заощадження країни становлять природне джерело подання для цього
великого попиту? По-перше, чим нагромаджувала Англія з 1792 до 1815 року,
