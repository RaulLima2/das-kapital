Візьмімо тепер капітал, склад якого є нижчий, ніж первісний
склад пересічного суспільного капіталу 80 c + 20 v (який
перетворився тепер в 76 4/21 c + 23 17/21 v), наприклад, 50 c + 50 v.
Тут ціна виробництва річного продукту, — якщо ми для спрощення
припустимо, що весь основний капітал увійшов як зношування
в річний продукт і що час обороту такий самий, як
і в випадку I, — становила перед підвищенням заробітної плати
50 c + 50 v + 20 p = 120. Підвищення заробітної плати на 25\%
дає для тієї самої кількості приведеної в рух праці підвищення
змінного капіталу з 50 до 62 1/2. Коли б річний продукт був
проданий по попередній ціні виробництва в 120, то це дало б
50 c + 62 1/2 v + 7 1/2 p, тобто норму зиску в 6 2/3\%. Але нова пересічна
норма зиску є 14 2/7\%, і через те що ми всі інші умови
припускаємо незмінними, цей капітал в 50 c + 62 1/2 v так само
мусить дати вказаний зиск. Але капітал в 112 1/2, при нормі зиску
в 14 2/7, дає 16 1/14 зиску.* Отже, ціна виробництва вироблених
ним товарів є тепер 50 c + 62 1/2 v + 16 1/14 p = 128 8/14. Отже, в наслідок
підвищення заробітної плати на 25\% ціна виробництва
тієї самої кількості того самого товару підвищилась тут з 120
до 128 8/14, або більше ніж на 7\%.

Візьмім, навпаки, сферу виробництва вищого складу, ніж пересічний
капітал, наприклад, 92 c + 8 v. Отже, первісний пересічний
зиск і тут = 20, і якщо ми знову припустимо, що весь
основний капітал входить у річний продукт і що час обороту
такий самий, як і в випадках І і II, то ціна виробництва товару
й тут = 120.

В наслідок підвищення заробітної плати на 25\% змінний капітал
для тієї самої кількості праці зростає з 8 до 10, отже
витрати виробництва товарів зростають з 100 до 102; з другого
боку, пересічна норма зиску впала з 20\% до 14 2/7\%. Але
100 : 14 2/7 = 102: 14 4/7 **. Отже, зиск, що припадає тепер на 102,
становить 14 4/7. і тому весь продукт продається за k + kp' =
102 + 14 4/7 = 116 4/7. Отже, ціна виробництва впала з 120 до
116 4/7, або майже на 3\% ***.

Отже, в наслідок підвищення заробітної плати на 25\%:

1) для капіталу пересічного суспільного складу ціна виробництва
товару лишилась незмінною;

2) для капіталу нижчого складу ціна виробництва товару

* В першому німецькому виданні тут сказано: „в круглих числах 16 1/12
зиску“; відповідно до цього Енгельс обчислює потім ціну виробництва в 128 7/12
В рукопису Маркса дано точне число в 16 3/42, яке нами взяте з відповідним
скороченням дробу і застосоване при обчисленні ціни виробництва. Примітка
ред. нім. вид. ІМЕЛ.

** В першому німецькому виданні тут стоїть: „(приблизно)“. В рукопису
Маркса цього слова немає. В дійсності тут рівняння точне, а не тільки приблизне.
Примітка ред. нім. вид. ІМЕЛ.

*** В першому німецькому виданні тут сказано: „більше ніж на 3\%. В рукопису
Маркса стоїть: „на 3 3/7“, тобто дано абсолютне число. В процентах воно
дорівнює 2 6/7\%. Примітка ред. нім. вид. ІМЕЛ.
