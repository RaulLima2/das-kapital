поверхового уявлення про внутрішній зв’язок економічних відносин,
який виявляється в конкуренції. Це є спосіб для того, щоб від
змін, які супроводять конкуренцію, прийти до границь цих змін.
Його не можна прикласти до пересічного розміру процента.
Немає абсолютно ніякої підстави, чому середні умови конкуренції,
рівновага між позикодавцями й позичальниками, повинні давати
позикодавцеві розмір процента в 3, 4, 5\% і т. д. на його капітал
абож певну процентну частину — 20 чи 50\% — гуртового
зиску. В тих випадках, коли справу-тут вирішує конкуренція як
така, визначення само по собі є випадковим, чисто емпіричним,
і тільки педантство або фантазерство може хотіти зобразити
цю випадковість як щось необхідне.\footnote{
Так, наприклад, J. G. Opdyke: „А Treatise on Political Economy“, New
York 1851, робить надзвичайно невдалу спробу пояснити загальність розміру
процента в 5\% вічними законами. Незрівняно наївніший пан Карл Арнд в „Die
naturgemässe Volkswirtschaft gegenüber dem Monopoliengeist und dem Kommunismus
etc.“, Hanau 1845. Тут можна прочитати таке: „В природному ході виробництва
благ існує тільки одно явище, яке — в цілком культивованих країнах — до
певної міри ніби призначене регулювати розмір процента; це — відношення,
в якому збільшуються маси дерев у європейських лісах в наслідок їх щорічного
приросту. Цей приріст відбувається цілком незалежно від їх мінової вартості“
[як це комічно, що дерева організують свій приріст незалежно від своєї мінової
вартості!] „у відношенні 3—4 до 100. — Отже, згідно з цим“ [тому що приріст
дерев зовсім не залежить від їх мінової вартості, хоч і як дуже їх мінова вартість
може залежати від їх приросту] „не можна було б сподіватись падіння
нижче того рівня, що його він“ [розмір процента] „має в теперішній час у
найбагатших на гроші країнах“ (стор. 124 [125]). — Це заслуговує назви „розмір
процента лісового походження“, а його винахідник у тому самому творі здобуває
ще більшу заслугу перед „нашою наукою“ як „філософ собачого податку“
[стор. 420 і далі].
} У парламентських звітах
1857 і 1858 рр., які стосуються законодавства про банки і торговельної
кризи, немає нічого кумеднішого, як базікання директорів
Англійського банку, лондонських банкірів, провінціальних
банкірів і професіональних теоретиків про „real rate produced“
[фактично утворену норму], яке не йшло далі таких загальних
місць, як, наприклад, що „ціна, яка сплачується позиченим
капіталом, може мінятись із зміною подання цього капіталу“,
що „висока норма процента і низька норма зиску не можуть
довгий час існувати одна поряд одної“ та інші подібні банальності.\footnote{
Англійський банк підвищує і знижує норму свого дисконту залежно від
того, припливає чи відпливає золото, хоч, звичайно, він при цьому завжди бере
до уваги норму, яка панує на відкритому ринку. „By which gambling in discounts,
by anticipation of the alterations in the bank rate, has now become half the trade
ol the great heads of they money centre“ [„В наслідок цього спекуляція на зміні
дисконту, яка передхоплює зміни банкової норми, стала тепер наполовину
заняттям великих фірм грошового центру“], — тобто лондонського грошового
ринку („The Theory of the Exchanges etc.“, стор. 113).
}
Звичка, узаконена традиція і т. д. цілком так само, як
і сама конкуренція, впливають на визначення середнього розміру
процента, оскільки він існує не тільки як пересічне число,
але й як фактична величина. Середній розмір процента мусить
уже бути припущений як законний у багатьох судових справах,