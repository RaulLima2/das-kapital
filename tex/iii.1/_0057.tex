\parcont{}  %% абзац починається на попередній сторінці
\index{iii1}{0057}  %% посилання на сторінку оригінального видання
одержується зиск відповідно в 10, 20, 30, 60, 90 фунтів стерлінгів.
Між вартістю товару і витратами його виробництва, очевидно,
можливий невизначений ряд продажних цін. Чим більший
той елемент товарної вартості, який складається з додаткової
вартості, тим більші на практиці межі для цих проміжних цін.

Цим пояснюються не тільки повсякденні явища конкуренції,
як, наприклад, певні випадки продажу по знижених цінах (underselling),
ненормально низький рівень товарних цін у певних
галузях промисловості\footnote{
Пор. книгу І, розд. XVIII, стор. 574 і далі [стор. 428 і далі рос. вид.
1935 р. І.
} і т. д. Основний закон капіталістичної
конкуренції, досі незбагненний політичною економією, закон,
який регулює загальну норму зиску і визначувані нею так
звані ціни виробництва, грунтується, як побачимо пізніше, на
цій ріжниці між вартістю товару і витратами його виробництва
і на можливості, яка звідси випливає — з зиском продавати
товар нижче його вартості.

Межа мінімальної продажної ціни товару дана витратами
його виробництва. Якщо він продається нижче витрат його
виробництва, то витрачені складові частини продуктивного капіталу
не можуть бути повністю заміщені з продажної ціни.
Якщо цей процес триває далі, то авансована капітальна вартість
зникає. Уже з цієї точки зору капіталіст схильний вважати витрати
виробництва за справжню \emph{внутрішню} вартість товару, бо
вони є ціна, необхідна для простого збереження його капіталу.
Але до цього долучається ще й те, що витрати виробництва
товару є та купівельна ціна, яку сам капіталіст заплатив за
виробництво товару, отже, купівельна ціна, визначувана самим
процесом виробництва товару. Тому надлишок вартості, або
додаткова вартість, яка реалізується при продажу товару,
здається капіталістові надлишком продажної ціни товару понад
його вартість, а не надлишком його вартості понад витрати його
виробництва, так що виходить, ніби додаткова вартість, яка
міститься в товарі, не реалізується через продаж його, а виникає
з самого продажу. Ми вже висвітлили ближче цю ілюзію
в книзі І, розд. IV, 2 (Суперечності загальної формули капіталу),
але тут вертаємось на хвилину до тієї форми, в якій її
знову висунули Торренс та інші, виставляючи її як прогрес політичної
економії порівняно з Рікардо.

„Природна ціна, що складається з витрат виробництва (Ргоduktionskost)
або, інакше кажучи, з витрати капіталу на виробництво
чи фабрикацію товару, ніяк не може включати в собі
зиску... Якщо фермер витрачає на обробіток своїх полів 100
квартерів зерна і одержує за це 120 квартерів, то 20 квартерів,
як надлишок продукту понад витрати, становлять його зиск;
але було б абсурдом називати цей надлишок або зиск частиною
його витрат... Фабрикант витрачає певну кількість сировинних
\parbreak{}  %% абзац продовжується на наступній сторінці
