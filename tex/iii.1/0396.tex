загальне нагромадження грошових запасів“. Один з членів комісії
зауважує: „4691. Отже, яка б не була причина протягом
останніх 12 років, результат у кожному разі був більше на користь єврея і торговця грішми, ніж на
користь продуктивного
класу взагалі“.

В якій великій мірі торговці грішми використовують період
кризи — про це каже Тук: „В металічному виробництві Уорвікшіра та Стаффордшіра в 1847 році багато
замовлень на товари
було відхилено, бо процент, який фабрикантові доводилося платити за дисконт своїх векселів, більше
ніж поглинув би весь
його зиск“ (№ 5451).

Візьмімо тепер інший, уже раніше цитований парламентський
звіт: „Report of Select Committee on Bank Acts, communicated from
the Commons to the Lords, 1857“ (далі цитується як: „Bank Committee“ 1857). В ньому маємо таке
свідчення пана Нормана, директора Англійського банку і головного світила серед прихильників currency
principle:

„3635. Ви сказали, що додержуєтесь того погляду, що розмір
процента залежить не від кількості банкнот, а від попиту і подання капіталу. Чи не будете ласкаві
сказати, що ви розумієте під
„капіталом“, крім банкнот і металічних грошей? — Я гадаю, що звичайне визначення „капіталу“ таке:
товари або послуги, вживані
у виробництві“. — „3636. Чи, кажучи про рівень процента, ви
охоплюєте всі товари словом „капітал“? — Всі товари, вживані
у виробництві“. — „3637. Чи розумієте ви все це під словом
„капітал“, коли говорите про розмір процента? — Авжеж. Припустімо, що бавовняному фабрикантові
потрібна бавовна для його
фабрики; він, мабуть, роздобуде її таким способом, що дістане
від свого банкіра позику і поїде з одержаними таким чином
банкнотами до Ліверпуля і купить бавовну. Чого йому дійсно
потрібно, так це бавовни; банкноти або золото йому потрібні
тільки як засіб одержати бавовну. Або йому потрібні засоби,
щоб оплатити своїх робітників; тоді він знову позичає банкноти
і виплачує цими банкнотами заробітну плату своїм робітникам;
робітникам, в свою чергу, потрібні харчі й житло, а гроші є
засіб для оплати їх“. — „3638. Але за гроші платяться проценти? — Звичайно, перш за все; але
візьміть інший випадок. Припустімо, що фабрикант купує бавовну в кредит, не беручи позики
в банку; тоді ріжниця між ціною за готівку і ціною в кредит
у момент скінчення строку становить міру процента. Процент
існував би, навіть якби взагалі не існувало грошей“.

Ця самовдоволена нісенітниця цілком гідна цієї опори currency principle. Насамперед геніальне
відкриття, що банкноти
або золото є засобом щось купити і що їх беруть у позику не
задля них самих. І звідси має випливати, що розмір процента
регулюється чим? Попитом і поданням товарів, про які ми досі
знали тільки, що вони регулюють ринкові ціни товарів. Але
з однаковими ринковими цінами товарів сполучні цілком різні
