\parcont{}  %% абзац починається на попередній сторінці
\index{iii1}{0184}  %% посилання на сторінку оригінального видання
при найгірших умовах [„Principles etc“., вид. Мак-Куллоха, Лондон
1852, стор. 37, 38]).

Як би не регулювались ціни, але виходить таке:

1) Закон вартості керує рухом цін: зменшення чи збільшення
робочого часу, потрібного для виробництва, приводить до підвищення
чи падіння цін виробництва. Саме в цьому розумінні
каже Рікардо (який, звичайно, відчуває, що його ціни виробництва
відхиляються від вартостей товарів), що „the inquiry to
which he wishes to draw the reader’s attention, relates to the effect
of the variations in the relative value of commodities, and not in
their absolute value“ [дослідження, на яке він хоче звернути увагу
читача, стосується до результатів змін відносної вартості товарів,
а не їх абсолютної вартості] [Рікардо, там же, стор. 15].

2) Пересічний зиск, який визначає ціни виробництва, завжди
мусить приблизно дорівнювати тій кількості додаткової вартості,
що припадає на даний капітал, як відповідну частину сукупного
суспільного капіталу. Припустім, що загальна норма
зиску, а тому й пересічний зиск, виражається в грошовій вартості,
вищій, ніж дійсна пересічна додаткова вартість, виражена в її
грошовій вартості. Оскільки справа стосується капіталістів, байдуже,
чи нараховують вони взаємно 10 чи 15\% зиску. Ні той,
ні другий рівень процента не відповідає дійсній товарній вартості,
бо перебільшення грошового виразу є взаємне. Щождо
робітників (оскільки припускається, що вони одержують свою
нормальну заробітну плату і що, отже, підвищення пересічного
зиску не означає дійсного відрахування з заробітної плати,
тобто не виражає чогось цілком іншого, ніж нормальна додаткова
вартість капіталіста), то підвищенню товарних цін, яке
виникає в наслідок підвищення пересічного зиску, мусить відповідати
підвищення грошового виразу змінного капіталу. Справді,
таке загальне номінальне підвищення норми зиску і пересічного
зиску понад рівень, даний відношенням дійсної додаткової
вартості до всього авансованого капіталу, неможливе без того,
щоб не потягти за собою підвищення заробітної плати, а також
підвищення цін товарів, які становлять сталий капітал. Цілком
так само зниження дасть протилежний результат. Через те що
сукупна вартість товарів регулює сукупну додаткову вартість,
а ця остання регулює висоту пересічного зиску, отже й загальної
норми зиску, — регулює як загальний закон, або як закон, що
керує коливаннями, — то закон вартості регулює ціни виробництва.

Що здійснює конкуренція, спочатку в окремих сферах виробництва,
так це — встановлення однакової ринкової вартості
і ринкової ціни з різних індивідуальних вартостей товарів. Але
тільки конкуренція капіталів у різних сферах виробництва створює
ціну виробництва, яка зрівнює норми зиску між різними сферами.
Для досягнення цього останнього результату потрібен вищий
розвиток капіталістичного способу виробництва, ніж для першого.

Для того, щоб товари тієї самої сфери виробництва, того
\parbreak{}  %% абзац продовжується на наступній сторінці
