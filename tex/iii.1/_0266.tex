\parcont{}  %% абзац починається на попередній сторінці
\index{iii1}{0266}  %% посилання на сторінку оригінального видання
полотна для того, щоб знову кинути на ринок 30000 метрів вартістю в 3000 фунтів стерлінгів. Купець
не може їх знову купити, бо він ще має на складі непроданих 30 000 метрів, які ще
не перетворились для нього в грошовий капітал. Тоді настає
застій, перерив репродукції. Виробник полотна міг би, звичайно,
мати в своєму розпорядженні додатковий грошовий капітал, який
він міг би, незалежно від продажу цих 30 000 метрів, перетворити
в продуктивний капітал і таким чином продовжувати процес
виробництва. Але таке припущення зовсім не змінює справи.
Оскільки справа йде про капітал, авансований на ці 30000 метрів,
процес його репродукції є і лишається перерваним. Отже, тут
дійсно з очевидністю виявляється, що операції купця є не що
інше, як операції, які взагалі мусять бути виконані для того, щоб
перетворити товарний капітал виробника у гроші; операції, які
опосереднюють функції товарного капіталу в процесі циркуляції і репродукції. Якщо замість
незалежного купця цим продажем і, крім того, закупівлею повинен був би займатись як виключною
справою простий прикажчик виробника, то цей зв’язок ні
на одну хвилину не був би прихований.

Отже, товарно-торговельний капітал є безперечно не що
інше, як товарний капітал виробника, капітал, який повинен проробити процес свого перетворення в
гроші, виконати на ринку
свою функцію як товарний капітал; тільки тепер ця функція виступає не як побічна операція виробника,
а як виключна операція особливого роду капіталістів, торговців товарами, усамостійнюється як заняття
в особливій сфері капіталовкладення.

Зрештою, це виявляється і в специфічній формі циркуляції
товарно-торговельного капіталу. Купець купує товари і потім
продає їх: Г — Т — Г'. В простій товарній циркуляції або навіть
в циркуляції товарів, якою вона виступає як процес циркуляції
промислового капіталу, Т' — Г — Т, циркуляція опосереднюється
тим, що кожний грошовий знак двічі міняє місце. Виробник полотна продає свій товар, полотно,
перетворює його в гроші;
гроші покупця переходять у його руки. На ці самі гроші він
купує пряжу, вугілля, працю і т. д., знову витрачає ці самі
гроші, щоб зворотно перетворити вартість полотна в товари, які
становлять елементи виробництва полотна. Товар, який він купує, не той самий товар, товар не того
самого роду, який
він продає. Він продав продукти і купив засоби виробництва. Але
інакше стоїть справа в русі купецького капіталу. На 3000 фунтів стерлінгів торговець полотном купує
30 000 метрів полотна;
він продає ці самі 30000 метрів полотна, щоб одержати назад
з циркуляції грошовий капітал (3000 фунтів стерлінгів, крім
зиску). Отже, тут двічі міняє місце не той самий грошовий знак,
а той самий товар; він переходить з рук продавця в руки покупця і з рук покупця, який тепер став
продавцем, в руки
іншого покупця. Він продається двічі і може бути проданий
ще багато разів при втручанні в справу ряду купців; і якраз
\parbreak{}  %% абзац продовжується на наступній сторінці
