\parcont{}  %% абзац починається на попередній сторінці
\index{iii1}{0143}  %% посилання на сторінку оригінального видання
прийняли до школи, і скаржились при цьому, що вони не могли
заробити й 1 шилінга на тиждень. Я мав відомості про self-acting
minders [прядільників на автоматичних прядільних машинах]...
Чоловіки, що керували парою автоматів, заробили за 14 днів
повного робочого часу 8 шилінгів 11 пенсів, і з цієї суми у них відрахували
плату за наймання житла, при чому, однак, фабрикант“
[о великодушність!] „повернув їм половину „цієї плати як
подарунок. Прядільники принесли додому по 6 шилінгів 11 пенсів.
В багатьох місцях протягом останніх місяців 1862 року selfacting
minders заробляли 5—9 шилінгів на тиждень, ткачі 2—6 шилінгів
на тиждень... В даний момент стан справ багато нормальніший,
хоча в більшості округ заробіток все ще дуже знижений...
Поряд з коротким волокном індійської бавовни та її забрудненістю
багато інших причин сприяли зниженню заробітку. Так,
наприклад, тепер стало звичаєм примішувати багато бавовняних
відпадів до індійської бавовни, і це, звичайно, ще дужче збільшує
труднощі для прядільника. При короткому волокні нитки
легше рвуться при витяганні мюлів і намотуванні пряжі, і тому
не можна з такою регулярністю підтримувати рух мюлів... Так
само при тій великій увазі, яку доводиться приділяти, слідкуючи
за нитками, одна ткаля може наглядати часто лиш за одним верстатом
і тільки дуже небагато з них можуть наглядати більше,
ніж за двома верстатами... В багатьох випадках заробітну плату
робітників прямо знижено на 5, 7 1/2 і 10\%... в більшості випадків
робітникові доводиться дбати про те, як упоратися з своїм сировинним
матеріалом і добитися звичайного розміру заробітку, як
зможе. Інша трудність, з якою іноді доводиться боротися ткачам,
полягає в тому, що вони повинні з поганого матеріалу робити
добру тканину і штрафуються відрахуваннями з заробітної
плати, якщо їх робота не дає бажаних наслідків“ („Rep. of Insp.
of Fact., Oct. 1863“, стор. 41—43).

Заробітна плата була мізерна навіть там, де працювали повний
час. Робітники бавовняної промисловості брались охоче до всяких
громадських робіт, на яких їх використовували, — дренаж, прокладання
шляхів, розбивання каменю, брукування вулиць, — щоб
одержати від місцевих властей допомогу (яка фактично була допомогою
фабрикантам, див. книгу I, стор. 603*). Вся буржуазія
пильно стежила за робітниками. Коли робітникові пропонувався
найгірший, наймізерніший заробіток і він відмовлявся від нього,
то комітет допомоги викреслював його з списків на допомогу. То
був золотий час для панів фабрикантів, бо робітникам доводилось
або вмирати з голоду, або працювати за всяку найзисковнішу
для буржуа ціну, при чому комітети допомоги діяли як
їх вартові пси. Разом з тим фабриканти, в таємному порозумінні
з урядом, якомога перешкоджали еміграції, почасти для
того, щоб тримати напоготові свій існуючий в тілі й крові робіт-

* Стор. 451 рос. вид. 1935 р. Ред. укр. перекладу.
\index{iii1}{0144}  %% посилання на сторінку оригінального видання
ників капітал, а почасти для того, щоб забезпечити собі видушувану
з робітників квартирну плату.

„Комітети допомоги діяли в цій справі з величезною суворістю.
Коли пропонувалась робота, то робітників, яким вона пропонувалась,
викреслювали з списків, і таким способом їх примушували
брати цю роботу. Якщо робітники відмовлялися від такої роботи...
то причина цього була в тому, що їх заробіток був би тільки
номінальним, а праця — надзвичайно тяжкою“ („Rep. of Insp. of
Fact., Oct. 1863“, стор. 97).

Робітники були готові взяти всяку роботу, яку їм давали,
згідно з Public Works Act [законом про громадські роботи].
„Принципи, за якими були організовані промислові роботи, були
дуже різні в різних містах. Але навіть там, де робота під голим
небом не була виключно спробною роботою (labour test), ця
робота оплачувалась або не більше як в розмірі регулярної
допомоги, абож тільки трохи вище, так що фактично вона ставала
спробною роботою“ (стор. 69). „Public Works Act 1863 року
мав допомогти в цій біді і дати робітникові змогу заробляти свою
денну заробітну плату як незалежному поденникові. Мета цього
закону була трояка: 1) дати можливість місцевим властям позичати
гроші (за згодою президента центрального державного
комітету в справі бідних) у комісарів в справах державних позик;
2) полегшити справу впорядкування міст у бавовняних округах;
3) дати безробітним робітникам працю й достатній заробіток
(remunerative wages)“. До кінця жовтня 1863 року на підставі
цього закону дозволено було позик на суму в 883 700 фунтів
стерлінгів (стор. 70). Розпочаті роботи складалися головним
чином з каналізаційних робіт, прокладання шляхів, брукування
вулиць, влаштування водозборів для водяних двигунів і т. д.

Пан Гендерсон, президент блекбернського комітету, пише
з приводу цього до фабричного інспектора Редгрева: „З усього
того, що мені довелося бачити на протязі теперішнього часу
страждань і злиднів, ніщо не вражало мене дужче або не тішило
мене більше, як та бадьора готовість, з якою безробітні
робітники цієї округи бралися до роботи, запропонованої їм
блекбернською міською радою на підставі Public Works Act.
Ледве чи можна собі уявити більший контраст, ніж контраст
між бавовнопрядільником, який раніше був вправним робітником
на фабриці, і тим самим прядільником, що працює тепер як
поденник при копанні сточних каналів глибиною в 14—18 футів“.
[При цьому вони заробляли, залежно від складу родини, 4—12
шилінгів на тиждень; цієї колосальної суми часто мало вистачати
для родини з 8 осіб. Панове міщани мали при цьому подвійну
вигоду: поперше, вони одержували по винятково низьких
процентах гроші для поліпшення своїх димних і занедбаних
міст; подруге, вони платили робітникам далеко менше від звичайної
заробітної плати]. „Робітник, звиклий до майже тропічної
температури, до праці, при якій вправність і точність маніпуляції
\index{iii1}{0145}  %% посилання на сторінку оригінального видання
були для нього незрівняно важливіші, ніж мускульна сила,
звиклий до подвійної, іноді до потрійної плати, порівняно з тією,
яку він може одержати тепер, — такий робітник, згоджуючись
на пропоновану йому роботу, виявляє стільки самозречення
і розсудливості, що це робить йому найвищу честь. У Блекберні
ці люди випробувані майже чи не на всякій роботі під голим
небом: на копанні твердого, важкого глинистого грунту на
значній глибині, на осушуванні грунту, на розбиванні каменю,
на прокладанні шляхів, на копанні вуличних каналів глибиною
в 14, 16, а іноді й 20 футів. Часто їм доводиться при цьому
стояти в грязі й воді на глибині в 10—12 дюймів і вони завжди
підпадають при цьому впливові вогкого й холодного клімату,
гіршого чи навіть рівного якому взагалі не знайдеш ні в одній
окрузі Англії“ (стор. 91, 92). — „Поведінка робітників була майже
бездоганна... їх готовість працювати під голим небом і цим
перебиватися“ (стор. 69).

1864 рік. Квітень. „В різних округах час від часу чути нарікання
на недостачу робітників, головним чином у певних галузях
промисловості, наприклад, у ткацькій... але ці нарікання в однаковій
мірі є результатом як тієї незначної плати, яку можуть заробити
робітники в наслідок застосовування поганих сортів пряжі,
так і деякої дійсної недостачі самих робітників в цій особливій
галузі. Минулого місяця відбулося багато сутичок між деякими
фабрикантами і їх робітниками з приводу заробітної плати.
Я шкодую, що страйки відбувалися занадто часто... Діяння
Public Works Act’a фабриканти сприймають як конкуренцію,
і в наслідок цього місцевий комітет у Bacup’i припинив свою
діяльність, бо, хоч працюють не всі ще фабрики, а, проте, виявилась
недостача робітників“ („Rep. of Insp. of Fact., April 1864“,
стор. 9, 10). І справді, для панів фабрикантів це був крайній
час, коли вони повинні були діяти. В наслідок Public Works
Act’a попит настільки зріс, що в каменоломнях Bacup’а деякі
фабричні робітники заробляли тепер 4—5 шилінгів на день.
І тому помалу були припинені громадські роботи, це нове видання
ateliers nationaux [національних майстерень] 1848 року,
але організованих цього разу в інтересах буржуазії.

Experimente in corpore vili\footnote*{
[Експерименти над нічого не вартим тілом].
}

„Хоч я навів тут дуже знижену заробітну плату (робітників,
що працюють повний час), дійсний заробіток робітників на
різних фабриках, проте з цього ніяк не випливає, що вони
з тижня на тиждень заробляють ту саму суму. Заробіток робітників
зазнає тут великих коливань в наслідок постійного експериментування
фабрикантів з різними сортами й пропорціями бавовни
і відпадів на тій самій фабриці; ці „мішанки“, як їх звуть,
\parbreak{}  %% абзац продовжується на наступній сторінці
