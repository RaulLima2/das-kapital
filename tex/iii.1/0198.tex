з меншими витратами виробництва) або принаймні з якомога меншими
втратами вийти з цього становища, і в цьому разі він уже не
турбуватиметься про своїх сусідів, хоч його дії зачіпають не тільки
його самого, але й усіх його товаришів.\footnote{
„If each man of a class could never have more than a given share, or
aliquot part of the gains and possessions of the whole, he would readily combine
to raise the gains (він саме так і робить, якщо відношення між попитом і поданням
йому це дозволяє); this is monopoly. But where each man thinks that he
may any way increase the absolute amount of his own share, though by a process
which lessens the whole amount, he will often do it; this is competition“. [„Коли б
кожен член групи ніколи не міг одержати більше даної частки або відповідної
частини доходів і володінь всієї групи, то він охоче об’єднувався б з іншими,
щоб підвищити ці доходи“ (він саме так і робить, якщо відношення між попитом
і поданням йому це дозволяє); „це — монополія. Але якщо кожен думає,
що він якимсь способом може збільшити абсолютну суму своєї власної частки,
хоч би й шляхом зменшення цілої суми, то він часто саме так і робитиме; це —
конкуренція.] („An Inquiry into those principles respecting the nature of demand
etc.“, Лондон 1821, стор. 105).
}

Попит і подання передбачають перетворення вартості в ринкову
вартість, і, оскільки це відбувається на капіталістичній
базі, оскільки товари є продукти капіталу, передбачають капіталістичний
процес виробництва, отже, відносини, які цілком інакше
переплітаються, ніж проста купівля і продаж товарів. Тут ідеться
не про формальне перетворення вартості товарів у ціну, тобто
не про просту переміну форми; тут ідеться про певні кількісні
відхилення ринкових цін від ринкових вартостей і, далі, від цін виробництва.
При простій купівлі і продажу досить, щоб товаровиробники
як такі протистояли один одному. Попит і подання при
дальшому аналізі передбачають існування різних класів і підрозділів
класів, які розподіляють між собою сукупний дохід суспільства
і споживають його як дохід, які, отже, пред’являють попит, визначуваний
цим доходом; тимчасом як, з другого боку, для розуміння
попиту і подання, що їх утворюють між собою виробники
як такі, необхідно зрозуміти всю систему капіталістичного
процесу виробництва в цілому.

При капіталістичному виробництві справа йде не тільки про те,
щоб замість маси вартості, кинутої в циркуляцію у товарній
формі, вилучити з неї рівну масу вартості в іншій формі, —
в формі грошей або в формі іншого товару, — справа йде про те,
щоб на капітал, авансований на виробництво, здобути таку саму
додаткову вартість, або зиск, як і на кожний інший капітал такої
самої величини, або pro rata [пропорціональну до] його величини,
незалежно від того, в якій галузі виробництва він застосовується;
отже, справа йде про те, щоб продати товари мінімум
по таких цінах, які дають пересічний зиск, тобто по цінах виробництва.
В цій формі капітал сам приходить до усвідомлення
себе як суспільної сили, в якій кожний капіталіст має частину,
пропорціональну до його участі в сукупному суспільному капіталі.

Поперше, для капіталістичного виробництва самого по собі
не має значення певна споживна вартість, взагалі специфіч-