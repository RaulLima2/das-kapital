\parcont{}  %% абзац починається на попередній сторінці
\index{iii1}{0342}  %% посилання на сторінку оригінального видання
на процент і власне зиск регулюється попитом і поданням, отже,
конкуренцією, цілком так само, як ринкові ціни товарів. Але ріжниця
виступає тут так само яскраво, як і аналогія. Якщо попит
і додання взаємно покриваються, то ринкова ціна товару відповідає
його ціні виробництва; тобто його ціна регулюється тоді
внутрішніми законами капіталістичного виробництва, незалежно
від конкуренції, бо коливання попиту й подання не пояснюють
нічого, крім відхилень ринкових цін від цін виробництва, — відхилень,
які взаємно вирівнюються, так що за певні більш-менш довгі
періоди пересічні ринкові ціни дорівнюють цінам виробництва.
Якщо попит і подання взаємно покриваються, ці сили перестають
діяти, знищують одна одну, і загальний закон визначення ціни
виступає тоді як закон і для конкретного випадку; ринкова ціна
відповідає тоді уже в своєму безпосередньому бутті, а не тільки
як пересічна руху ринкових цін, ціні виробництва, яка регулюється
імманентними законами самого способу виробництва.
Те ж саме і з заробітною платою. Якщо попит і подання взаємно
покриваються, то їх вплив знищується, і заробітна плата дорівнює
тоді вартості робочої сили. Але інакше стоїть справа з процентом
на грошовий капітал. Конкуренція визначає тут не відхилення
від закону; тут просто не існує ніякого іншого закону поділу,
крім того, який диктується конкуренцією, бо, як ми це ще
побачимо далі, не існує ніякої „природної“ норми процента. Під
природною нормою процента розуміють, навпаки, саме норму,
яку встановлює вільна конкуренція. Не існує ніяких „природних“
меж норми процента. Там, де конкуренція визначає не тільки відхилення
й коливання, де, отже, при рівновазі її сил, що одна одній
протидіють, взагалі припиняється всяке визначення, там те, що
належить визначити, само по собі є чимось позбавленим закономірності
і самовільним. Докладніше про це в дальшому розділі.

При капіталі, що дає процент, все виступає як щось зовнішнє:
авансування капіталу — як проста передача його позикодавцем
позичальникові; зворотний приплив реалізованого капіталу — як
проста зворотна передача, зворотна сплата з процентами позичальником
позикодавцеві. Так само стоїть справа і з імманентним
капіталістичному способові виробництва визначенням, саме з тим,
що норма зиску визначається не тільки відношенням того зиску,
який вироблено за один оборот, до авансованої капітальної вартості,
але й тривалістю часу самого цього обороту, отже, визначається
як зиск, що його промисловий капітал дає за певні
періоди часу. Це також виступає при капіталі, що дає процент,
як щось цілком зовнішнє: позикодавцеві просто за певний період
часу сплачується певний процент.

Романтичний Адам Мюллер з своїм звичайним розумінням
внутрішнього зв’язку речей каже („Die Elemente der Staatskunst“.
Berlin 1809 [том III], стор. 138): „При визначенні ціни речей не
питають про час; при визначенні процента береться до обрахунку
головним чином час“. Він не бачить того, яким чином
\parbreak{}  %% абзац продовжується на наступній сторінці
