\parcont{}  %% абзац починається на попередній сторінці
\index{iii1}{0048}  %% посилання на сторінку оригінального видання
продукту віднімемо додаткову вартість m, то лишиться простий
еквівалент або вартість, яка заміщає в товарі капітальну вартість
c + v, витрачену як елементи виробництва.

Якщо, наприклад, виготовлення певного продукту спричиняє
витрату капіталу в 500 фунтів стерлінгів: 20 фунтів стерлінгів на
зношування засобів праці, 380 фунтів стерлінгів на матеріали
виробництва, 100 фунтів стерлінгів на робочу силу, і якщо норма
додаткової вартості становить 100\%, то вартість продукту =
400 с + 100 v + 100 m = 600 фунтам стерлінгів.

Після того, як ми віднімемо додаткову вартість в 100 фунтів
стерлінгів, лишається товарна вартість у 500 фунтів стерлінгів,
і ця остання заміщає тільки витрачений капітал у 500 фунтів
стерлінгів. Ця частина вартості товару, яка заміщає ціну спожитих
засобів виробництва і ціну вжитої робочої сили, заміщає
тільки те, чого коштує товар самому капіталістові, і тому вона
становить для нього витрати виробництва (Kostpreis) товару.

Чого коштує товар капіталістові і чого коштує само виробництво
товару, це в усякому разі дві цілком різні величини. Та
частина товарної вартості, яка складається з додаткової вартості,
нічого не коштує капіталістові саме тому, що робітникові
вона коштує неоплаченої праці. Однак, через те що на основі
капіталістичного виробництва робітник, після того як він вступив
у процес виробництва, сам становить складову частину
функціонуючого і належного капіталістові продуктивного капіталу
і, отже, дійсним виробником товару є капіталіст, то витрати
виробництва товару неминуче виступають для нього як
те, чого дійсно коштує (wirkliche Kost) самий товар. Якщо витрати
виробництва ми назвемо k, то формула: Т = с + v + m
перетворюється в формулу: T = k + m, або товарна вартість =
— витратам виробництва + додаткова вартість.

Тому підведення тих різних частин вартості товару, які тільки
заміщають витрачену на його виробництво капітальну вартість,
під категорію витрат виробництва виражає, з одного боку, специфічний
характер капіталістичного виробництва. Те, чого коштує
капіталістові (kapitalistische Kost) товар, вимірюється витратою
капіталу, те, чого дійсно коштує (wirkliche Kost) товар, —
витратою праці. Тому капіталістичні витрати виробництва товару
кількісно відрізняються від його вартості або його дійсних
витрат виробництва; вони менші, ніж товарна вартість, бо через
те що T = k + m, то k = T — m. З другого боку, витрати
виробництва товару ніяк не є такою рубрикою, яка існує тільки
в капіталістичному рахівництві. Усамостійнення цієї частини вартості
постійно виявляється на практиці в дійсному виробництві
товару, бо з її товарної форми вона, за допомогою процесу
циркуляції, знов і знов мусить бути зворотно перетворена
в форму продуктивного капіталу, отже, на виручені витрати виробництва
товару доводиться знов і знов купувати елементи
виробництва, спожиті на виробництво товару.
