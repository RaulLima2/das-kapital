\parcont{}  %% абзац починається на попередній сторінці
\index{iii1}{0408}  %% посилання на сторінку оригінального видання
втрата капіталу; це — віддача певної частини благородного металу, з якого складаються світові
гроші“. — „3748. Хіба ви раніше
не сказали, що зміна норми дисконту є проста ознака зміни вартості капіталу? — Так“. — „3749. І що
норма дисконту взагалі
змінюється із зміною золотого запасу в Англійському банку? —
Так; але я вже сказав, що коливання розміру процента, які виникають із зміни кількості грошей в
країні“ [отже, під цим він
тут розуміє кількість дійсного золота] „дуже незначні...“

„3750. Отже, ви хочете сказати, що відбулося зменшення капіталу, якщо відбулося довгочасніше, але
все ж тільки тимчасове підвищення дисконту понад звичайну норму? — Зменшення
в певному розумінні слова. Змінилось відношення між капіталом
і попитом на нього; але можливо, що в наслідок збільшеного
попиту, а не в наслідок зменшення кількості капіталу“. [Але ж
тільки що капітал прирівнювався до грошей або до золота, а трохи
раніше підвищення розміру процента пояснювалось високою нормою зиску, яка виникла з розширення, а не
з скорочення справ
або капіталу].

„3751. Який це капітал ви маєте тут спеціально на увазі? —
Це цілком залежить від того, який капітал потрібен кожній
окремій людині. Це — капітал, який нація має в своєму розпорядженні, щоб продовжувати свої справи, і
якщо ці справи розростаються вдвоє, то мусить настати велике збільшення попиту
на капітал, потрібний для дальшого провадження цих справ“. [Цей
хитромудрий банкір спочатку збільшує вдвоє справи, а далі після
цього попит на капітал, яким вони мають бути подвоєні. Він
завжди бачить перед собою тільки свого клієнта, який вимагає
від пана Лойда більшого капіталу, щоб подвоїти свої справи]. —
„Капітал — те саме, що й усякий інший товар“ [але ж капітал, на думку пана Лойда, є не що інше, як
загальна сума товарів]; „він змінюється у своїй ціні“ [отже, товари двічі змінюються
в ціні: один раз — як товари, другий раз — капітал], „залежно
від попиту й подання“.

„3752. Коливання в нормі дисконту взагалі стоять в зв’язку
з коливанням суми золота у сховищах банку. Чи є це той капітал, що ви маєте на увазі? — Ні“. —
„3753. Чи можете ви навести такий приклад, коли б в Англійському банку був нагромаджений великий
запас капіталу і одночасно норма дисконту
стояла б високо? — В Англійському банку нагромаджують не
капітал, а гроші“. — „3754. Ви сказали, що розмір процента залежить від кількості капіталу; чи не
будете ласкаві сказати,
який капітал ви маєте на увазі, і чи не можете ви навести приклад, коли б у банку лежав великий
запас золота і одночасно
розмір процента стояв би високо? — Дуже ймовірно“ (ага!),
„що нагромадження золота в банку може збігатися з низьким розміром процента, бо період незначного
попиту на капітал“
[саме грошовий капітал; час, про який тут іде мова — роки 1844
і 1845 — був часом процвітання] „є період, протягом якого, звичайно,
\index{iii1}{0409}  %% посилання на сторінку оригінального видання
можна нагромаджувати той засіб або знаряддя, за допомогою якого панують над капіталом“. —
„3755. Отже, ви гадаєте,
що не існує ніякого зв’язку між нормою дисконту і кількістю
золота у сховищах банку? — Зв’язок може існувати, але це не
принциповий зв’язок“; [проте, його банковий акт 1844 року
зводить саме в принцип Англійського банку регулювання розміру процента залежно від кількості золота,
яким володіє банк]
„вони можуть збігатися в часі (there may be a coincidence
of time)“. — „3758. Отже, ви хочете сказати, що труднощі для купців тут у країні, в періоди
недостачі грошей, в наслідок високої
норми дисконту, полягають у тому, щоб одержати капітал, а не
в тому, щоб одержати гроші? — Ви змішуєте дві речі, які я в
цій формі не об’єдную; трудність полягає в тому, щоб одержати капітал, і так само трудно одержати
гроші... Трудність одержати гроші і трудність одержати капітал — це
та сама трудність, розглядувана на двох різних ступенях
її розвитку“. — Тут рибка знову кріпко впіймалась. Перша
трудність — це дисконтувати вексель або одержати позику під
заставу товарів. Трудність ця полягає в тому, щоб перетворити
в гроші капітал або торговельний знак вартості капіталу. І ця
трудність виражається, між іншим, у високому розмірі процента.
Але раз гроші одержано, то в чому тоді полягає друга трудність? Якщо справа йде тільки про платіж,
то хіба хто-небудь
зустрічає трудність у тому, щоб позбутися своїх грошей? А якщо справа йде про купівлю, то хіба
хто-небудь у період кризи
зустрічає трудність купити товар? І якщо навіть припустити,
що це стосується окремого випадку подорожчання хліба, бавовни і т. д., то ж ця трудність могла б
виявлятись не
в вартості грошового капіталу, тобто не в розмірі процента, а
тільки в ціні товару; і ця трудність переборюється ж тим,
що в нашого ділка тепер є гроші для купівлі товару.

„3760. Але ж вища норма дисконту не збільшує трудності
одержати гроші? — Вона збільшує трудність одержати гроші, але
справа йде не про володіння грішми; гроші тільки форма“ [і ця
форма дає зиск у кишеню банкіра], „в якій виражається ця збільшена трудність одержати капітал у
складних відносинах цивілізованого порядку“.

„3763. [Відповідь Оверстона:] Банкір є посередник, який,
з одного боку, одержує вклади, а з другого боку, застосовує
ці вклади, довіряючи їх у формі капіталу в руки осіб, які
і т. д.“.

Тут ми бачимо, нарешті, що він розуміє під капіталом. Він
перетворює гроші в капітал, „довіряючи їх“ або, висловлюючись
менш евфемістично, віддаючи їх в позику за проценти.

Сказавши раніше, що зміна норми дисконту по суті не стоїть
у зв’язку із зміною суми золотого запасу банку або кількості
наявних грошей, а щонайбільше збігається з нею в часі, пан
Оверстон повторює:
