\parcont{}  %% абзац починається на попередній сторінці
\index{iii1}{0406}  %% посилання на сторінку оригінального видання
банкір передає клієнтові в позику певну суму, яка настільки ж
збільшує капітал, який є в розпорядженні клієнта.

Саме це уявлення, перенесене з банкірської контори в політичну економію, створило спірне питання,
яке вносить плутанину: чи те, що банкір дає грішми-готівкою в розпорядження
свого клієнта, є капітал чи просто гроші, засоби циркуляції,
currency? Щоб вирішити це — по суті просте — спірне питання,
ми мусимо стати на точку зору клієнта банку. Все залежить
від того, чого він вимагає і що він одержує.

Якщо банк згоджується дати клієнтові позику просто під
його особистий кредит, без забезпечення з його боку, то справа
ясна. Він безумовно одержує позику певної величини вартості
як додаток до свого капіталу, який він застосовував досі. Він
одержує позику в грошовій формі; отже, він одержує не тільки
гроші, але й грошовий \emph{капітал}.

Якщо він одержує позику, видану під заставу цінних паперів і т. д., то це є позика в тому розумінні,
що йому виплачуються гроші під умовою їх зворотної сплати. Але це не є
позика капіталу. Бо цінні папери також репрезентують капітал,
і при тому на більшу суму, ніж позика. Отже, одержувач дістає меншу капітальну вартість, ніж дає в
заставу; це для нього
ні в якому разі не є придбанням додаткового капіталу. Він робить цю операцію не тому, що йому
потрібен капітал — він уже
має його в своїх цінних паперах, а тому, що йому потрібні гроші.
Отже, в цьому випадку перед нами позика \emph{грошей}, а не капіталу.

Якщо ж позика дається під дисконт векселів, тоді зникає і \emph{форма}
позики. Перед нами проста купівля і продаж. За допомогою передатного напису вексель переходить у
власність банку, а гроші — у власність клієнта; про повернення клієнтом грошей немає й мови. Якщо
клієнт векселем або іншим подібним знаряддям
кредиту купує гроші-готівку, то це така ж позика, не більше
й не менше, як коли б він купив гроші-готівку яким-небудь
іншим своїм товаром — бавовною, залізом, хлібом. І менш за
все тут може бути мова про позику \emph{капіталу}. Всяка купівля і продаж між торговцем і торговцем є
передача капіталу.
Позика ж має місце тільки тоді, коли передача капіталу відбувається не взаємно, а однобічно і на
якийсь час. Отже, позика
капіталу за допомогою дисконту векселів може мати місце
тільки тоді, коли вексель є бронзовий вексель, який зовсім не
репрезентує проданих товарів, а такого векселя не візьме жоден
банкір, якщо він знатиме, що це за вексель. Отже, при нормальній дисконтній операції клієнт банку не
одержує ніякої позики,
пі капіталом, ні грішми; він одержує гроші за проданий товар.

Отже, випадки, коли клієнт вимагає і одержує від банку
капітал, дуже ясно відрізняються від тих випадків, коли він
одержує в позику просто гроші або купує гроші в банку. І тому
що саме пан Лойд-Оверстон мав звичай давати в позику свої
\parbreak{}  %% абзац продовжується на наступній сторінці
