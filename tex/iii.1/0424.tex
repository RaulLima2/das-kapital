Насамперед ясно, що в першому з двох вищенаведених випадків, в періоди процвітання, коли маса
засобів циркуляції, що
перебувають у циркуляції, мусить зростати, попит на них зростає. Але так само ясно, що коли
фабрикант бере з свого вкладу
в банку більше золота або банкнот, тому що він має витратити
більше капіталу в грошовій формі, то з цієї причини його попит на капітал не зростає, а зростає
тільки його попит на цю
особливу форму, в якій він витрачає свій капітал. Попит стосується тільки до технічної форми, в якій
він кидає свій капітал у циркуляцію. Цілком так само, як, наприклад, при різному
розвитку кредиту, той самий змінний капітал, та сама кількість
заробітної плати, вимагає в одній країні більшої маси засобів

everything looks prosperous, when wages are high, prices on the rise, and factories
busy, that an additional supply of currency is usually required to perform the additional functions
inseparable from the necessity of making larger and more numerous payments; whereas it is chiefly in
a more advanced stage of the commercial
cycle, when difficulties begin to present themselves, when markets are overstocked
and returns delayed, that interest rises, and a pressure comes upon the Bank for
advances of capital. It is true that there is no medium through which the Bank is
accustomed to advance capital except that of its promissory notes; and that, to
refuse the notes, therefore, is to refuse the accommodation. But, the accommodation
once granted, everything adjusts itself in conformity with the necessities of the
market; the loan remains, and the currency, if not wanted, finds its way back to
the issuer. Accordingly, a very slight examination of the Parliamentary Returns
may convince any one, that the securities in the hand of the Bank of England
fluctuate more frequently in an opposite direction to its circullation than in concert
with it, and that the example, therefore, of that great establishment furnishes no
exception to the doctrine so strongly pressed by the country bankers, to the effect
that no bank can enlarge its circulation, if that circulation be already adequate to
the purposes to which a banknote currency is commonly applied; but that every
addition to its advances, after that limit is passed, must be made from its capital,
and supplied by the sale of some of its securities in reserve, or by abstinence from
further investment in such securities. The table compiled from the Parliamentary
Returns for the interval between 1833 and 1840, to which I have referred in a preceding page,
furnishes continued examples of this truth; but two of these are so
remarkable that it will be quite unnecessary for me to go beyond them. On the
3rd January, 1837, when the resources of the Bank were strained to the uttermost
to sustain credit and meet the difficulties of the money market, we find its advances on loan and
discount carried to the enormous sum of £ 17 022 000, an
amount scarcely known sinse the war, and almost equal to the entire aggregate
issues, which, in the meanwhile, remain unmoved at so low a point as £ 7 076 000!
On the other hand, we have, on the 4th of June 1833 a circulation of £ 18 892 000
with a return of private securities in hand, nearly, if not the very lowest on
record for the last halfcentury, amounting to no more than £ 972 000!“ [„Це справді
велика помилка уявляти собі, що попит на грошову позику (тобто на позику
капіталу) є тотожний з попитом на додаткові засоби циркуляції або що обидва
ці попити часто між собою зв’язані. Кожний з цих попитів виникає з особливих і в обох випадках дуже
різних обставин. Коли все має вигляд процвітання, коли заробітна плата висока, ціни підвищуються і
фабрики добре працюють, то звичайно постає потреба в додаткових засобах циркуляції для виконання
додаткових функцій, невідділимих від необхідності робити більші і
численніші платежі; тимчасом як підвищення процента і тиск на банк, вимоги
на позики капіталу виникають головним чином на пізнішій стадії комерційного циклу, коли починають
виявлятися труднощі, коли ринки переповнені
і зворотні припливи капіталів затримуються. Це вірно, що, крім випуску банк-
