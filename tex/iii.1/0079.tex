ту саму величину, але їх речові елементи зазнають зміни вартості,
коли, отже, v означає змінену кількість приведеної у рух
праці, а с — змінену кількість приведених у рух засобів виробництва.

20 v в капіталі 80 с + 20 v + 20 m первісно представляли заробітну
плату 20 робітників, по 10 робочих годин на день. Нехай
заробітна плата кожного з них підвищиться з 1 до 1 1/4. Тоді 20 v
оплачують не 20, а тільки 10 робітників. Але, якщо ці 20 за
200 робочих годин виробляли вартість у 40, то ці 16 за 10 годин
на день, тобто разом за 160 робочих годин, вироблять вартість
лише в 32. Після того як ми віднімемо 20 v для заробітної
плати, з 32 залишиться тільки 12 для додаткової вартості; норма
додаткової вартості знизилася б з 100\% до 60\%. Але через
те що, згідно з припущенням, норма додаткової вартості мусить
лишитись незмінною, то робочий день мусив би бути здовжений
на 1/4, з 10 до 12 1/2 годин; якщо 20 робітників при 10 годинах
на день, = 200 робочим годинам, виробляють вартість у 40, то
16 робітників при 12 1/2 годинах на день, = 200 годинам, вироблять
таку саму вартість, і капітал 80 с + 20 v виробляв би, як і раніш,
додаткову вартість у 20.

Навпаки: якщо заробітна плата падає так, що 20 v; становлять
заробітну плату 30 робітників, то m' може лишитися
незмінним тільки тоді, коли робочий день скорочується з 10
до 6 2/3 годин. 10 × 20 = 6 2/3 × 30 = 200 робочим годинам.

Наскільки при таких протилежних припущеннях с щодо виразу
його вартості в грошах може лишитись незмінним і все ж
представляти змінену відповідно до змінених відносин масу
засобів виробництва, — це в істотному вже з’ясовано вище.
В своєму чистому вигляді цей випадок можливий тільки як цілком
винятковий.

Щождо зміни вартості елементів с, яка збільшує або зменшує
їх масу, але лишає незмінною суму вартості с, то ця зміна, поки
вона не веде за собою зміни величини v, не зачіпає ні норми
зиску, ні норми додаткової вартості.

Таким чином ми. вичерпали всі можливі випадки зміни v, с
і К в нашому рівнянні. Ми бачили, що при незмінній нормі додаткової
вартості норма зиску може падати, лишатись незмінною
або підвищуватись, бо найменшої зміни відношення v до с,
відповідно до К, досить для того, щоб змінити також і норму
зиску.

Далі, виявилось, що при зміні v завжди настає межа, коли незмінність
m' стає економічно неможливою. Через те що всяка
однобічна зміна с теж мусить дійти до межі, коли v не може
далі лишатись незмінним, то виявляється, що для всіх можливих
змін v / К існують межі, поза якими m' теж мусить стати змінним.
При змінах m', до дослідження яких ми тепер переходимо, ця
взаємодія різних змінних нашого рівняння виступить ще ясніше.
