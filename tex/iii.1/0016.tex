перенесення певної частини вартості сукупного продукту до
класу капіталістів“.

Отож не треба великого напруження мислі, щоб переконатися,
що це „вульгарноекономічне“ пояснення зиску на капітал
практично веде до тих самих результатів, як і теорія додаткової
вартості Маркса; що, за уявленням Лексіса, робітники
перебувають точно в такому самому „несприятливому становищі“,
як і в Маркса; що вони цілком так само обдурені, бо
кожен неробітник може продавати вище ціни, а робітник цього
не може; і що на основі цієї теорії можна збудувати принаймні
настільки ж поверховий вульгарний соціалізм, як той, що збудований
тут в Англії на основі теорії споживної вартості та
теорії граничної корисності Джевонса-Менгера. Я навіть думаю,
що коли б ця теорія зиску була відома панові Джорджеві Бернардові
Шоу, він міг би ухопитися за неї обома руками, дати
відставку Джевонсові та Карлові Менгеру і наново збудувати
на цій скелі фабіанську церкву майбутнього.

Але в дійсності ця теорія є лише парафраза теорії Маркса.
З чого ж покриваються всі надбавки до ціни? З „сукупного
продукту“ робітників. І саме в наслідок того, що товар „праця“,
або, як каже Маркс, робоча сила, мусить продаватися нижче її
ціни. Бо якщо спільна властивість усіх товарів є в тому, що їх
можна продавати дорожче витрат виробництва, а праця становить
єдиний виняток з цього і продається завжди тільки по витратах
виробництва, то вона продається якраз нижче тієї ціни, яка є
загальним правилом у цьому вульгарноекономічному світі. Надзиск,
який в наслідок цього припадає капіталістові або класові
капіталістів, полягає саме в тому і в кінцевому рахунку може
постати тільки тому, що робітник, після репродукції заміщення
ціни своєї праці, мусить ще далі виробляти продукт, за який
йому не платять, — додатковий продукт, продукт неоплаченої
праці, додаткову вартість. Лексіс — людина надзвичайно обережна
у виборі своїх висловів. Він ніде не каже прямо, що
вищенаведене розуміння є його власне; але якщо це так, то
цілком ясно, що ми тут маємо справу не з одним з тих звичайних
вульгарних економістів, про яких він сам каже, що кожний
з них в очах Маркса є „в кращому разі тільки безнадійний недоумок“,
а з марксистом, який переодягнувся вульгарним економістом.
Чи сталося це переодягнення свідомо чи несвідомо, це
є психологічне питання, яке нас тут не цікавить. Той, хто схотів
би з’ясувати це, може, дослідив би також, як могло статися, що
така безперечно розумна людина, як Лексіс, могла певний час
боронити таке безглуздя, як біметалізм.

Перший, хто дійсно намагався дати відповідь на питання, був
д-р Конрад Шмідт: „Die Durchschnittsprofitrate auf Grundlage des
Marxschen Wertgesetzes“, Stuttgart, Dietz 1889. Шмідт намагається
погодити деталі утворення ринкових цін як із законом
вартості, так і з пересічною нормою зиску. Промисловий капі-
