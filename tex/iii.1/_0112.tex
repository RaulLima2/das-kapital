\parcont{}  %% абзац починається на попередній сторінці
\index{iii1}{0112}  %% посилання на сторінку оригінального видання
Досі на охоронний клапан навішували такий тягар, що він відкривався вже при тисненні пари в 4, 6 або
8 фунтів на квадратний
дюйм; тепер виявили, що підвищенням тиснення до 14 або 20 фунтів... можна досягти дуже значного
заощадження вугілля; інакше
кажучи, фабрика почала працювати при значно меншому споживанні вугілля... Ті, що мали для цього
засоби й сміливість, стали
застосовувати систему збільшеного тиснення і розширення в повному її обсягу і застосовували
відповідно до цього збудовані парові казани, які давали пару тисненням в 30, 40, 60 і 70 фунтів на
квадратний дюйм — тиснення, при якому інженер старої школи
від страху зомлів би. Але через те що економічний результат
цього підвищеного тиснення пари... виявився дуже швидко в цілком недвозначній формі фунтів, шилінгів
і пенсів, парові казани
високого тиснення при конденсаційних машинах стали майже загальним явищем. Ті, що провели реформу
радикально, стали застосовувати вульфові машини, і це мало місце щодо більшості недавно
збудованих машин; вони стали застосовувати особливо вульфові
машини з 2 циліндрами, в одному з яких пара з казана розвиває силу в наслідок перевищення тиснення
над тисненням атмосфери і потім, замість того щоб після кожного підіймання поршня
виходити у повітря, як це було раніш, входить у циліндр
низького тиснення, приблизно вчетверо більший обсягом, і, розширившись там далі, відводиться в
конденсатор. Економічний
результат, одержуваний при таких машинах, полягає в тому, що
одна кінська сила за одну годину добувається при споживанні
3 \sfrac{1}{2}—4 фунтів вугілля; тимчасом як при машинах старої системи
для цього потрібно було від 12 до 14 фунтів. За допомогою
майстерного пристрою вульфову систему подвійного циліндра
або комбінованої машини високого й низького тиснення удалось
пристосувати до наявних уже старих машин і таким чином підвищити їх ефективність при одночасному
зменшенні споживання
вугілля. Того самого результату досягнуто протягом останніх
8—10 років за допомогою сполучення машини високого тиснення
з конденсаційною машиною таким чином, що спожита пара першої переходила в другу і пускала її в рух.
Така система корисна
в багатьох випадках“.

„Не легко було б точно встановити, наскільки збільшилась
ефективність праці тих самих колишніх парових машин, до
яких пристосовані деякі або й усі ці нові поліпшення. Але я певен, що на ту саму вагу парової машини
ми одержуємо тепер
пересічно принаймні на 50\% більше корисної роботи і що в багатьох випадках та сама парова машина,
яка в часи обмеженої
швидкості в 220 футів на хвилину давала 50 кінських сил, дає
тепер понад 100. Надзвичайно ефективні щодо економії результати
застосування пари високого тиснення при конденсаційних машинах, так само як і далеко більші вимоги,
які ставляться до старих парових машин з метою розширення підприємств, привели
за останні три роки до введення трубчастих казанів, в наслідок
\parbreak{}  %% абзац продовжується на наступній сторінці
