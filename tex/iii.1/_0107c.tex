\parcont{}  %% абзац починається на попередній сторінці
\index{iii1}{0107}  %% посилання на сторінку оригінального видання
санітарного стану; але майже всі вони переповнені, погано провітрюються і в високій мірі
несприятливі для здоров’я... В таких
кімнатах, крім усього, неодмінно жарко; а коли запалюють газ,
як це роблять удень під час туману або зимою вечорами, то температура підвищується до 80 і навіть до
90 градусів (за Фаренгейтом = 27—33° Цельсія) і викликає надзвичайне пітніння робітників
і згущення пари на шибках, так що вода безупинно стікає або
крапає з вікна в стелі, і робітники змушені держати відчиненими
кілька вікон, хоча вони при цьому неминуче простуджуються. —
Становище в 16 найзначніших майстернях лондонського Вестенду
він описує так: найбільший кубічний простір, який припадає
в цих погано провітрюваних кімнатах на одного робітника, становить 270 кубічних футів; найменший —
105 футів, пересічно —
всього тільки 156 футів на людину. В одній майстерні, яка обведена з усіх боків галереєю і має
освітлення тільки згори,
занято від 92 до 100 осіб; горить багато газових ріжків; клозети
збудовані безпосередньо коло майстерні, і на кожну людину
припадає не більше, як 150 кубічних футів простору. В другій
майстерні, в освітленому згори дворі, яку можна назвати тільки
собачою конурою і яку можна провітрювати тільки через маленьке вікно в даху, працює 5 чи 6 осіб, при
чому на кожну
з них припадає 112 кубічних футів“. І „в цих жахливих (atrocious) майстернях, які описує доктор
Сміт, кравці працюють
звичайно 12—13 годин на день, а іноді праця триває 14—16 годин“ (стор. 25, 26, 28).

Число занятих людей
958 265
22 301 чоловіків і
12 377 жінок
13 803

Галузь промисловості
і місцевість
Землеробство, Англія та
Уельс..................
Кравці, Лондон. . . .
Складачі й друкарі, Лондон...................

Норма смертності на 100 000 осіб віком
25—35 р.    35—45 р.    45—55 р.
743                805              1145
958             1262              2093
894            1747               2367

(стор. 30). Треба відзначити — і це дійсно відзначено складачем
цього звіту, завідувачем медичного відділу, Джоном Сімоном, —
що для віку в 25—35 років смертність кравців, складачів і друкарів Лондона показана применшеною, бо
в обох цих галузях
промисловості лондонські майстри одержують з села велике
число молодих людей (мабуть, до 30 років), що працюють як
учні і „improvers“, тобто для дальшого удосконалення. Вони
збільшують число занятих осіб, на яке треба обчисляти норми
смертності промислового населення Лондона; але вони не збільшують в такій самій мірі число смертей у
Лондоні, бо їх перебування в Лондоні тільки тимчасове; коли вони захворіють на протязі цього часу,
то вертаються додому на село, і смерть
їх, якщо вони умирають, реєструється там. Ця обставина ще
в більшій мірі стосується до молодшого віку, і в наслідок цього
\parbreak{}  %% абзац продовжується на наступній сторінці
