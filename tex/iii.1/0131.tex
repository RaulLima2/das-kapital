його репродукцію, і таким чином відновлюється монополія тих
країн — джерел його постачання, які виробляють при найсприятливіших
умовах, — відновлюється, може, з певними обмеженнями,
але все ж відновлюється. Правда, репродукція сировинних
матеріалів в наслідок даного поштовху відбувається в розширеному
масштабі, особливо в країнах, які в більшій чи меншій мірі
володіють монополією цього виробництва. Але та база, на якій в наслідок
збільшення кількості машин і т. д. відбувається виробництво
і яка тепер після кількох коливань має стати новою нормальною
базою, новим вихідним пунктом, дуже розширилася в наслідок
процесів, що відбувались протягом останнього циклу обороту.
При цьому, однак, в частині другорядних джерел постачання
сировинного матеріалу репродукція, яка щойно збільшилась, знову
значно гальмується. Так, наприклад, з таблиць експорту ясно
видно, як протягом останніх 30 років (до 1865 року) зростало
індійське виробництво бавовни, коли наставала недостача в американському
виробництві, і потім раптом знову починалося
більш-менш тривале скорочення. Протягом часу подорожчання
сировинного матеріалу промислові капіталісти об’єднуються,
утворюють асоціації, щоб регулювати виробництво. Так було,
наприклад, в Манчестері після підвищення цін на бавовну в
1848 році, так само як і в виробництві льону в Ірландії. Але як
тільки безпосередній привід мине і знову суверенно запанує загальний
принцип конкуренції „купувати на найдешевшому ринку“
(замість того, щоб намагатися, як це робили згадані асоціації,
підвищити виробничу здатність відповідних країн — джерел постачання
сировинного матеріалу, незалежно від безпосередньої
ціни даного моменту, по якій ці країни можуть у даний час постачати
продукт), — отже, як тільки знову суверенно запанує
принцип конкуренції, регулювати подання знову полишається
„ціні“. Всяка думка про спільний, рішучий і передбачливий контроль
над виробництвом сировинного матеріалу — контроль,
який загалом і в цілому ніяк несполучний з законами капіталістичного
виробництва і тому завжди лишається благочестивим
побажанням або обмежується винятковими спільними кроками
в моменти великої безпосередньої небезпеки й безпорадності —
поступається місцем вірі в те, що попит і подання взаємно
регулюватимуть одно одне.16 Суєвірство капіталістів тут таке
грубе, що навіть фабричні інспектори в своїх звітах знов і знов
з приводу цього здивовано розводять руками. Чергування сприят-

16 Після того, як це було написано (1865 р.), конкуренція на світовому ринку
значно посилилася в наслідок швидкого розвитку промисловості в усіх культурних
країнах, 'особливо в Америці і Німеччині. Той факт, що швидко й колосально
зростаючі сучасні продуктивні сили з кожним днем все більше переростають
закони капіталістичного товарообміну, в межах яких вони повинні
рухатись, — цей факт нині все більше й більше проникає навіть у свідомість
самих капіталістів. Це виявляється особливо в двох симптомах. Поперше, в новій
загальній манії охоронних мит, яка відрізняється від старої системи охоронних
мит особливо тим, що вона найбільше захищає якраз товари, придатні до
