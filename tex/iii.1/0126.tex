шень у методах репродукції цього основного капіталу. Вартість
машин і т. д. знижується тепер не тому, що вони швидко
витісняються або до певної міри знецінюються новими продуктивнішими
машинами і т. д., а тому, що вони тепер можуть
бути дешевше репродуковані. Це одна з причин, чому великі підприємства
часто процвітають тільки в других руках, після того
як збанкрутує перший власник, а другий, що дешево купив
підприємство, таким чином уже з самого початку починає своє
виробництво з меншими витратами капіталу.

В землеробстві особливо впадає в очі, що ті самі причини,
які підвищують або знижують ціну продукту, підвищують або
знижують також і вартість капіталу, бо цей останній у значній
частині сам складається з цього продукту — хліба, худоби і т. ін.
(Рікардо).

Тепер треба було б згадати ще про змінний капітал.

Якщо вартість робочої сили підвищується внаслідок підвищення
вартості потрібних для її репродукції засобів існування,
або, навпаки, знижується в наслідок зниження вартості цих засобів
існування, — а підвищення вартості і зниження вартості
змінного капіталу не виражає нічого іншого, крім цих обох випадків, — то при незмінній довжині
робочого дня цьому підвищенню вартості відповідає падіння додаткової вартості, а цьому
зниженню вартості — зростання додаткової вартості. Але в той
самий час з цим можуть бути зв’язані й інші обставини — звільнення і зв’язування капіталу — які не
були ще досліджені і які
треба тепер коротко розглянути.

Якщо заробітна плата знижується внаслідок падіння вартості робочої сили (з чим може бути зв’язане
навіть підвищення
реальної ціни праці), то таким чином звільняється частина капіталу, яка досі витрачалась на
заробітну плату. Відбувається
звільнення змінного капіталу. На нововкладуваний капітал це
справляє тільки той вплив, що він працює з підвищеною нормою додаткової вартості. Та сама кількість
праці приводиться
в рух за допомогою меншої кількості грошей, ніж раніше, і таким чином неоплачена частина праці
збільшується коштом
оплаченої. Але для капіталу, який був вкладений уже раніше,
не тільки підвищується норма додаткової вартості, але, крім
того, звільняється частина капіталу, яка досі витрачалась на
заробітну плату. Досі вона була зв’язана і становила постійну
частину, яка відділялась від виручки за продукт і мусила витрачатись на заробітну плату,
функціонувати як змінний капітал,
якщо підприємство мало й далі провадитися в попередніх розмірах. Тепер ця частина стає вільною, і
може, отже, бути використана як нове капіталовкладення, чи для розширення того самого підприємства,
чи для функціонування в іншій сфері
виробництва.
