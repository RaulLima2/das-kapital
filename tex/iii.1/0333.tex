тивний капіталіст може продавати товар по його вартості (вирівнення
за цінами виробництва тут, з його точки зору, не має
ніякого значення) і саме тому одержувати зиск поверх того капіталу,
який він кидає в обмін. Припустім, що ціна виробництва
100 капелюхів = 115 фунтам стерлінгів і що ця ціна виробництва
випадково дорівнює вартості капелюхів, отже, що капітал,
який виробляє капелюхи, має пересічний суспільний склад.
Якщо зиск = 15%, то капелюшник реалізує зиск у 15 фунтів стерлінгів
у наслідок того, що продає товари по їх вартості в 115.
Йому вони коштують тільки 100 фунтів стерлінгів. Якщо він
виробляв із своїм власним капіталом, то надлишок у 15 фунтів
стерлінгів він цілком кладе в свою кишеню; якщо ж він виробляв
з капіталом, взятим у позику, то з цих 15 фунтів стерлінгів
він повинен віддати, може, 5 фунтів стерлінгів як процент.
Від цього вартість капелюхів ні трохи не змінюється, а змінюється
тільки розподіл між різними особами тієї додаткової
вартості, яка вже міститься в цій вартості. Отже, — тому що
виплата процента не впливає на вартість капелюхів, — безглуздям
є таке твердження Прудона: „Через те що в торгівлі процент
на капітал долучається до заробітної плати робітника, щоб склалась
ціна товару, то робітник не може викупити продукт своєї
власної праці. Vivre en travaillant [жити працюючи] є принцип,
який, при пануванні процента, містить у собі суперечність“
(стор. 105). 57

Як мало Прудон зрозумів природу капіталу, можна бачити
з такого речення, в якому він рух капіталу взагалі описує як рух,
властивий капіталові, що дає процент: „Comme, par l’accumulation
des intérêts, le capital-argent, d’échange en échange, revient toujours
à sa source, il s’ensuit que la relocation toujours faite par la même
main, profite toujours au même personnage“ [„Через те що
в наслідок нагромадження процентів капітал-гроші після кожного
обміну завжди повертається до свого джерела, то з цього випливає,
що позика, яку постійно дає та сама особа, завжди дає зиск
тій самій особі“ [Прудон у листі від 31 грудня 1849 р., там же,
стор. 154].

Що ж для нього лишається загадковим у своєрідному русі
капіталу, що дає процент? Категорії: купівля, ціна, уступка предметів
і безпосередня форма, в якій з’являється тут додаткова

57 Тому, якби робилося так, як того хоче Прудон, то „будинок“, „гроші"
і т. д. повинні не віддаватись у позику як „капітал“, а відчужуватись як „товар...
по ціні витрат виробництва“ (стор. [43] 44). Лютер стояв трохи вище
Прудона. Він знав уже, що одержання зиску не залежить від форми позики
або купівлі: „З торгівлі теж роблять лихварство. Але за один раз це вже занадто
багато. І раз ми тепер мусимо говорити про таку річ, як лихварство
при позиках, то, якщо ми повстали (недавно) проти нього, ми хочемо віддати
по заслузі і торговельному лихварству" (М. Luther: „An die Pfarrherrn wider
den Wucher zu predigen". Wittenberg 1540 [Luthers Werke, Wittenberg 1589, частина
6, стор. 307]).
