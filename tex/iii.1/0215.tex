до всього капіталу, а не у відношенні до змінного капіталу, з
якого вона виникає. Отже, падіння норми зиску виражає спадаюче
відношення самої додаткової вартості до всього авансованого
капіталу, і тому воно незалежне від будьякого розподілу
цієї додаткової вартості між різними категоріями.

Ми бачили, що на певному ступені капіталістичного розвитку,
коли склад капіталу c : v = 50 : 100, норма додаткової вартості
в 100\% виражається в нормі зиску в 66 2/3\% і що на вищому
ступені розвитку, коли c : v як 400 : 100, та сама норма додаткової
вартості виражається в нормі зиску тільки в 20\%. Те, що
стосується до різних послідовних ступенів розвитку в одній
країні, стосується і до різних ступенів розвитку, які існують
одночасно один поряд одного в різних країнах. У нерозвиненій
країні, де перший склад капіталу є пересічний, загальна норма
зиску була б = 66 2/3\%, тимчасом як у країні другого складу капіталу,
з значно вищим ступенем розвитку, вона була б = 20\%.

Ріжниця обох національних норм зиску могла б зникнути і
навіть стати протилежною в наслідок того, що в менш розвиненій
країні праця була б менш продуктивною, тому більша
кількість праці виражалася б у меншій кількості того самого
товару, більша мінова вартість виражалася б у меншій споживній
вартості, отже, робітник мусив би вживати більшу частину
свого часу на репродукцію своїх власних засобів існування або
їх вартості і меншу частину на створення додаткової вартості,
давав би менше додаткової праці, так що норма додаткової
вартості була б нижча. Якщо, наприклад, у менш розвиненій країні
робітник працював би 2/3 робочого дня на себе самого і 1/3 на
капіталіста, то, зберігаючи припущення вищенаведеного прикладу,
та сама робоча сила оплачувалася б у розмірі 133 1/3 і дала б
надлишок тільки в 66 2/3. Змінному капіталові в 133 1/3 відповідав
би сталий капітал в 50. Отже, норма додаткової вартості становила
б тут 133 1/3 : 66 2/3 = 50\%, а норма зиску 183 1/3 : 66 2/3, або
приблизно 36 1/2\%.

Через те що ми досі ще не дослідили різних складових частин,
на які розпадається зиск, — отже, вони для нас ще не існують,
— то ми тільки для того, щоб уникнути непорозумінь,
зауважимо наперед таке. При порівнянні країн різних ступенів
розвитку, а саме країн з розвиненим капіталістичним виробництвом
і таких, де праця ще формально не підпорядкована капіталові,
хоча в дійсності робітник експлуатується капіталістом
(наприклад, в Індії, де райот господарює як самостійний селянин,
отже, його виробництво, як таке, ще не підпорядковане капіталові,
хоч лихвар може видушити з нього в формі процента
не тільки всю його додаткову працю, але навіть — капіталістично
висловлюючись — частину його заробітної плати), було б
великою помилкою, коли б хтонебудь схотів міряти висоту національної
норми зиску висотою національного рівня процента.
В такому проценті міститься весь зиск і навіть більше ніж зиск,
