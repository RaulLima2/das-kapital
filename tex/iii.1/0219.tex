бітниче населення шляхом підвищення заробітної плати, отже,
пом’якшенням згубних впливів, що скорочують приріст робітників,
і полегшенням шлюбів; а з другого боку, шляхом застосування
методів, які створюють відносну додаткову вартість (введення
й поліпшення машин), він ще далеко швидше створив би
штучне відносне перенаселення, яке з свого боку — бо в капіталістичному
виробництві злидні породжують населення, — знов таки є
теплицею дійсного швидкого збільшення чисельності населення.
Тому з природи капіталістичного процесу нагромадження —
який є тільки моментом капіталістичного процесу виробництва —
само собою випливає, що збільшена маса засобів виробництва,
призначених для перетворення в капітал, завжди знаходить під
рукою відповідно збільшене і навіть надлишкове робітниче населення,
яке можна експлуатувати. Отже, з розвитком процесу
виробництва і нагромадження мусить зростати маса придатної
до привласнення і привласнюваної додаткової праці, а тому й абсолютна
маса зиску, привласнюваного суспільним капіталом. Але ті
самі закони виробництва і нагромадження разом з масою сталого
капіталу підвищують у дедалі більшій прогресії і його вартість, —
швидше, ніж вони підвищують вартість змінної частини капіталу,
обмінюваної на живу працю. Отже, одні й ті самі закони зумовлюють
для суспільного капіталу зростаючу абсолютну масу
зиску і падаючу норму зиску.

Ми тут цілком залишаємо осторонь те, що та сама величина
вартості з прогресом капіталістичного виробництва і відповідного
йому розвитку продуктивної сили суспільної праці та при помноженні
галузей виробництва, отже й продуктів, представляє прогресивно
зростаючу масу споживних вартостей і насолод.

Хід розвитку капіталістичного виробництва і нагромадження
зумовлює процеси праці в дедалі більшому масштабі, отже, в дедалі
більших розмірах, і відповідно до цього зумовлює зростаюче
авансування капіталу на кожне окреме підприємство. Тому
зростаюча концентрація капіталів (супроводжена в той самий
час, хоч і в меншій мірі, зростанням числа капіталістів) є так
само однією з матеріальних умов капіталістичного виробництва
і нагромадження, як і одним із створюваних ним самим результатів.
Рука в руку і у взаємодії із цим відбувається прогресуюча експропріація
більш чи менш безпосередніх виробників. Таким чином
для одиничних капіталістів стає зрозумілим, що вони мають
у своєму розпорядженні дедалі зростаючі робітничі армії (як би
сильно не падав їх змінний капітал порівняно з сталим), що маса
привласнюваної ними додаткової вартості, а тому й зиску, зростає
одночасно з падінням норми зиску і не зважаючи на це падіння.
Якраз ті самі причини, які концентрують маси робітничих армій
під командою окремих капіталістів, збільшують також масу застосовуваного
основного капіталу, як і сировинних та допоміжних
матеріалів, — збільшують відносно швидше, ніж масу вживаної
живої праці.
