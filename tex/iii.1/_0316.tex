\parcont{}  %% абзац починається на попередній сторінці
\index{iii1}{0316}  %% посилання на сторінку оригінального видання
перехід одного способу виробництва в другий і дати пояснення
такого переходу.

За капіталістичного виробництва купецький капітал від свого
колишнього самостійного існування знижується до ролі особливого
моменту капіталовкладення взагалі, а вирівнення зисків
зводить його норму зиску до загального пересічного рівня. Він
функціонує уже тільки як агент продуктивного капіталу. Особливі
суспільні відносини, які утворюються з розвитком купецького
капіталу, тут уже не грають вирішальної ролі; навпаки, там,
де переважає купецький капітал, панують застарілі відносини.
Це стосується навіть до одної і тієї ж країни, де, наприклад,
чисто торговельні міста становлять більшу аналогію з минулими
відносинами, ніж фабричні міста.\footnote{
Пан В. Кіссельбах („Der Gang des Welthandels etc. im Mittelalter“ [Stuttgart]
1860) фактично все ще живе уявленнями того світу, в якому купецький
капітал є формою капіталу взагалі. Про капітал у сучасному значенні він не
має ні найменшого уявлення, як і пан Момсен, який у своїй римській історії
говорить про „капітал" і про панування капіталу. В сучасній англійській історії
власне торговельний стан і торговельні міста також виступають політично
реакційними і в союзі з земельною та фінансовою аристократією проти промислового
капіталу. Порівняй, наприклад, політичну роль Ліверпуля з роллю Манчестера
і Бірмінгама. Англійським купецьким капіталом і фінансовою аристократією
(moneyed interest) повне панування промислового капіталу визнане
тільки після скасування хлібних мит і т. д.
}

Самостійний і переважний розвиток капіталу як купецького
капіталу є рівнозначний непідпорядкуванню виробництва капіталові,
отже, рівнозначний розвиткові капіталу на основі чужої
йому і незалежної від нього суспільної форми виробництва. Самостійний
розвиток купецького капіталу стоїть, отже, у зворотному
відношенні до загального економічного розвитку суспільства.
Самостійне купецьке майно, як пануюча форма капіталу,
означає усамостійнення процесу циркуляції відносно його крайніх
членів, а ці крайні члени є самі обмінюючі виробники. Ці крайні
члени лишаються самостійними відносно процесу циркуляції, як
і цей процес відносно них. Продукт стає тут товаром завдяки
торгівлі. Саме торгівля приводить тут до того, що продукти
набирають форми товарів, а не виробництво товарів рухом цих
останніх утворює торгівлю. Отже, капітал як капітал виступає тут
уперше в процесі циркуляції. В процесі циркуляції гроші розвиваються
в капітал. В циркуляції продукт вперше розвивається
як мінова вартість, як товар і гроші. Капітал може утворитися
в процесі циркуляції і мусить утворитися в ньому, раніше ніж
він навчиться опановувати його крайні члени, різні сфери виробництва,
циркуляцію між якими він опосереднює. Грошова і товарна
циркуляція можуть обслуговувати сфери виробництва найрізноманітніших
організацій, які за своєю внутрішньою структурою
ще скеровані головним чином на виробництво споживної
вартості. Це усамостійнення процесу циркуляції, при якому сфери
\parbreak{}  %% абзац продовжується на наступній сторінці
