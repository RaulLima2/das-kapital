\parcont{}  %% абзац починається на попередній сторінці
\index{iii1}{0231}  %% посилання на сторінку оригінального видання
триває цей період вирівнення, виникає друга потреба — збільшити вкладуваний капітал; залежно від
ступеня цього зростання капіталу, капіталіст буде спроможний ужити до праці при нових умовах частину
занятого раніше числа робітників, а, може, навіть усіх або й більше, ніж раніш, число робітників,
отже виробляти ту саму або більшу масу зиску.

\section{Протидіючі причини}

Якщо взяти до уваги величезний розвиток продуктивних сил суспільної праці навіть тільки за останні
30 років, порівняно з усіма попередніми періодами, особливо якщо взяти до уваги величезну масу
основного капіталу, який, крім власне машин, входить у сукупність суспільного процесу виробництва,
то замість тієї трудності, якою досі займались економісти, а саме — трудності пояснити падіння норми
зиску, виникає протилежна трудність, а саме — трудність пояснити, чому це падіння не є більшим або
швидшим. Тут мусять діяти протилежні впливи, які перехрещують і знищують вплив загального закону і
надають йому характеру тільки тенденції, через що ми й назвали
падіння загальної норми зиску тенденцією до падіння. Найзагальніші з цих причин такі:

\subsection{Підвищення ступеня експлуатації праці}

Ступінь експлуатації праці, привласнення додаткової праці І додаткової вартості підвищується зокрема
за допомогою здовження робочого дня та інтенсифікації праці. Ці обидва пункти докладно висвітлені в
книзі І при дослідженні виробництва абсолютної й відносної додаткової вартості. Існує багато таких
моментів інтенсифікації праці, які передбачають зростання сталого капіталу порівняно із змінним,
отже, падіння норми зиску — наприклад, коли робітникові доводиться доглядати за більшим числом
машин. Тут — як і при більшості способів, що служать для виробництва відносної додаткової вартості —
ті самі причини, які викликають зростання норми додаткової вартості, можуть, — якщо розглядати дані
величини всього застосовуваного капіталу, — викликати падіння маси додаткової вартості. Але існують
інші моменти інтенсифікації, як, наприклад, прискорена швидкість машин; останні при цьому за той
самий час споживають, правда, більше сировинного матеріалу, але щодо основного капіталу, то, хоч
машини й зношуються швидше, однак це ніяк не зачіпає відношення їх вартості до ціни тієї праці, яка
приводить їх у рух. Але особливо збільшує масу привласнюваної
\parbreak{}  %% абзац продовжується на наступній сторінці
