\parcont{}  %% абзац починається на попередній сторінці
\index{iii1}{0058}  %% посилання на сторінку оригінального видання
матеріалів, знарядь і засобів існування для праці і одержує за
це певну кількість готового товару. Цей готовий товар мусить
мати вищу мінову вартість, ніж ті сировинні матеріали, знаряддя
і засоби існування, за допомогою авансування яких його
добуто“. Тому, робить висновок Торренс, надлишок продажної ціни
понад витрати виробництва, або зиск, виникає з того, що споживачі
„за допомогою безпосереднього або посереднього (circuitous)
обміну дають до певної міри більшу кількість всіх складових
частин капіталу, ніж коштує їх виробництво“.\footnote{
R. Torrens: „An Essay on the Production of Wealth“. Лондон 1821, стop.
51—53, 349.
}

Справді, надлишок понад дану величину не може становити
частини цієї величини, отже й зиск, надлишок товарної вартості
понад витрати капіталіста, не може становити частини цих витрат.
Тому, якщо в утворення вартості товару не входить
ніякий інший елемент, крім вартості, авансованої капіталістом,
то не можна зрозуміти, яким чином може вийти з виробництва
більше вартості, ніж увійшло в нього, — інакше вийшло б, що
з нічого стає щось. Цього творення з нічого Торренс уникає,
однак, тільки таким способом, що переносить його з сфери виробництва
товарів у сферу циркуляції товарів. Зиск не може постати
з виробництва, — каже Торренс, — бо інакше він містився б уже
в витратах виробництва, отже, не було б ніякого надлишку понад
ці витрати. Зиск не може постати з обміну товарів, — відповідає
йому Рамсей, — якщо його не було вже в наявності перед обміном
товарів. Сума вартості обмінюваних продуктів, очевидно,
не змінюється в наслідок обміну тих продуктів, суму вартості
яких вона становить. Вона лишається такою самою після обміну,
як і перед ним. Тут слід зазначити, що Мальтус прямо посилається
на авторитет Торренса,\footnote{
Malthus: „Definitions in Political Economy". Лондон 1853, стop. 70, 71.
} хоч сам він інакше пояснює
продаж товарів вище їх вартості, або, скорше, не пояснює його,
бо всі аргументи цього роду по суті неминуче зводяться до
славнозвісної у свій час негативної ваги флогістону.

При суспільному становищі, над яким панує капіталістичне виробництво,
капіталістичні уявлення панують і над некапіталістичним
виробником. В своєму останньому романі „Les Paysans“
[„Селяни“] Бальзак, який взагалі відзначається своїм глибоким
розумінням реальних відносин, влучно змальовує, як дрібний
селянин, щоб зберегти прихильність свого лихваря, даром виконує
для нього всяку працю і гадає, що він цим нічого не
дарує лихвареві, бо йому самому його власна праця не коштує
жодних витрат готівкою. Таким чином лихвар з свого боку
одним пострілом убиває двох зайців. Він заощаджує витрати
на заробітну плату і все глибше й глибше обплутує павутинням
лихварської сітки селянина, якого відтягання від праці на його
власному полі розоряє дедалі більше.
