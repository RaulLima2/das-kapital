\parcont{}  %% абзац починається на попередній сторінці
\index{iii1}{0332}  %% посилання на сторінку оригінального видання
гроші, з’єднується тут з грішми без опосереднюючого проміжного
руху, просто як їх характер, як їх визначеність. І в цій
визначеності вони відчужуються, коли віддаються в позику як
грошовий капітал.

У Прудона дивне розуміння ролі грошового капіталу („Gratuité
du Crédit. Discussion entre M. F. Bastiat et M. Proudhon“.
Paris 1850). Позика здається Прудонові злом тому, що вона не
є продаж. Позика за проценти є „спроможність знову й знову продавати
той самий предмет і знову й знову одержувати його ціну,
не відступаючи ніколи власності на те, що продається“\footnote*{
З першого листа до Бастіа, написаного Ф. Шеве, прихильником Прудона
і редактором „Voix du Peuple“, що відкрив дискусію (22 жовтня 1849 р.) \emph{Примітка
ред. нім. вид. ІМЕЛ.}
} (стор. 9).
Предмет, гроші, будинок і т. д. не міняють свого власника, як
це має місце при купівлі й продажу. Але Прудон не бачить,
що при віддаванні грошей у формі капіталу, що дає процент, за
них не одержують ніякого еквіваленту. В кожному акті купівлі
й продажу, оскільки взагалі відбуваються процеси обміну, об’єкт
дійсно віддається. Власність на проданий предмет кожного разу
відступається. Але вартість при цьому не віддається. При продажу
віддається товар, але не його вартість, яка повертається
у формі грошей або в формі боргового зобов’язання чи боргової
розписки, що тут є тільки іншою формою грошей. При купівлі віддаються
гроші, але не їх вартість, яка заміщається в формі товару.
На протязі всього процесу репродукції промисловий капіталіст
тримає в своїх руках ту саму вартість (залишаючи
осторонь додаткову вартість), тільки в різних формах.

Оскільки відбувається обмін, тобто обмін предметів, не відбувається
ніякої зміни вартості. Той самий капіталіст завжди
тримає в своїх руках ту саму вартість. Але, оскільки капіталістом
додаткова вартість ще тільки виробляється, обміну не відбувається;
коли ж відбувається обмін, додаткова вартість уже
міститься в товарах. Якщо ми розглядатимем не окремі акти
обміну, а весь кругобіг капіталу, $Г — Т — Г'$, то певна сума вартості
постійно авансується і ця ж сума вартості плюс додаткова вартість
або зиск вилучається назад з циркуляції. Опосереднюючої
ланки цього процесу в простих актах обміну, звичайно, не видно.
Але якраз на цьому процесі $Г$ як капіталу грунтується процент
грошового капіталіста-позикодавця, і з цього процесу він
виникає.

„Справді, — каже Прудон, — капелюшник, який продає капелюхи...
одержує за них їх вартість, не більше й не менше. Але
капіталіст-позикодавець... не тільки одержує назад свій капітал
незменшеним; він одержує більше, ніж свій капітал, більше, ніж
він кидає в обмін; поверх капіталу він одержує ще й процент“
(там же стор. 69). Капелюшник представляє тут продуктивного
капіталіста в протилежність до капіталіста-позикодавця.
Прудон, очевидно, не дійшов таємниці того, яким чином продуктивний
\index{iii1}{0333}  %% посилання на сторінку оригінального видання
капіталіст може продавати товар по його вартості (вирівнення
за цінами виробництва тут, з його точки зору, не має
ніякого значення) і саме тому одержувати зиск поверх того капіталу,
який він кидає в обмін. Припустім, що ціна виробництва
100 капелюхів = 115 фунтам стерлінгів і що ця ціна виробництва
випадково дорівнює вартості капелюхів, отже, що капітал,
який виробляє капелюхи, має пересічний суспільний склад.
Якщо зиск = 15\%, то капелюшник реалізує зиск у 15 фунтів стерлінгів
у наслідок того, що продає товари по їх вартості в 115.
Йому вони коштують тільки 100 фунтів стерлінгів. Якщо він
виробляв із своїм власним капіталом, то надлишок у 15 фунтів
стерлінгів він цілком кладе в свою кишеню; якщо ж він виробляв
з капіталом, взятим у позику, то з цих 15 фунтів стерлінгів
він повинен віддати, може, 5 фунтів стерлінгів як процент.
Від цього вартість капелюхів ні трохи не змінюється, а змінюється
тільки розподіл між різними особами тієї додаткової
вартості, яка вже міститься в цій вартості. Отже, — тому що
виплата процента не впливає на вартість капелюхів, — безглуздям
є таке твердження Прудона: „Через те що в торгівлі процент
на капітал долучається до заробітної плати робітника, щоб склалась
ціна товару, то робітник не може викупити продукт своєї
власної праці. Vivre en travaillant [жити працюючи] є принцип,
який, при пануванні процента, містить у собі суперечність“
(стор. 105).\footnote{
Тому, якби робилося так, як того хоче Прудон, то „будинок“, „гроші"
і т. д. повинні не віддаватись у позику як „капітал“, а відчужуватись як „товар...
по ціні витрат виробництва“ (стор. [43] 44). Лютер стояв трохи вище
Прудона. Він знав уже, що одержання зиску не залежить від форми позики
або купівлі: „З торгівлі теж роблять лихварство. Але за один раз це вже занадто
багато. І раз ми тепер мусимо говорити про таку річ, як лихварство
при позиках, то, якщо ми повстали (недавно) проти нього, ми хочемо віддати
по заслузі і \emph{торговельному лихварству}" (\emph{М. Luther}: „An die Pfarrherrn wider
den Wucher zu predigen". Wittenberg 1540 [Luthers Werke, Wittenberg 1589, частина
6, стор. 307]).
}

Як мало Прудон зрозумів природу капіталу, можна бачити
з такого речення, в якому він рух капіталу взагалі описує як рух,
властивий капіталові, що дає процент: „Comme, par l’accumulation
des intérêts, le capital-argent, d’échange en échange, revient toujours
à sa source, il s’ensuit que la relocation toujours faite par la même
main, profite toujours au même personnage“ [„Через те що
в наслідок нагромадження процентів капітал-гроші після кожного
обміну завжди повертається до свого джерела, то з цього випливає,
що позика, яку постійно дає та сама особа, завжди дає зиск
тій самій особі“ [Прудон у листі від 31 грудня 1849 р., там же,
стор. 154].

Що ж для нього лишається загадковим у своєрідному русі
капіталу, що дає процент? Категорії: купівля, ціна, уступка предметів
і безпосередня форма, в якій з’являється тут додаткова
\parbreak{}  %% абзац продовжується на наступній сторінці
