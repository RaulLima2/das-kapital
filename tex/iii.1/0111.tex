шої інерції маховика став значно регулярнішим. При незмінному
тисненні пари і незмінному розрідженні в конденсаторі одержується більше сили внаслідок простого
прискорення руху
поршня. Коли б ми могли, наприклад, парову машину, яка при
швидкості в 200 футів на хвилину дає 40 кінських сил, відповідними змінами привести до того, щоб
вона при тому самому тисненні
пари й розрідженні робила 400 футів на хвилину, то ми мали б
якраз подвійну кількість сили; а через те що тиснення пари
і розрідження в обох випадках однакові, то напруження окремих
частин машини, отже й небезпека нещасних випадків при прискореній швидкості не збільшується в
будь-якій значній мірі. Вся ріжниця в тому, що споживатиметься більше пари в тій самій або
приблизно в тій самій пропорції, в якій прискорюється рух поршня;
далі, трохи швидше зношуватимуться частини, які зазнають
тертя, але про це ледве чи варто говорити... Але для того, щоб
від тієї самої машини за допомогою прискореного руху поршня
добути більше сили, треба спалити під тим самим паровим казаном більше вугілля або вживати казан з
більшою здатністю паротворення, коротко — треба виробляти більше пари. Це і було
зроблено, і казани з більшою здатністю паротворення були пристосовані до старих „прискорених“ машин;
таким чином ці машини в багатьох випадках давали на 100% більше роботи. Коло
1842 року надзвичайно дешеве вироблення сили паровими машинами на копальнях Корнуоля почало
викликати до себе увагу;
конкуренція в бавовнопрядільній промисловості примушувала
фабрикантів шукати головне джерело свого зиску в заощадженнях; надзвичайна ріжниця у споживанні
вугілля за одну годину
і на одну кінську силу, яку показували корнуольські машини,
а також надзвичайна економія при застосуванні вульфових машин
з подвійними циліндрами, і в нашій місцевості висунули на перший
план питання про економію паливного матеріалу. Корнуольські
машини і машини з подвійними циліндрами постачали одну кінську
силу за годину при споживанні З 1/2 до 4 фунтів вугілля, тим часом як машини в бавовнопрядільних
округах споживали звичайно 8 або 12 фунтів вугілля на 1 кінську силу за годину.
Така значна ріжниця спонукала фабрикантів і машинобудівників
нашої округи добитись за допомогою аналогічних засобів таких же
виняткових результатів у справі економії, які в Корнуолі і у Франції стали вже звичайними, бо там
висока ціна на вугілля примушувала фабрикантів якомога більше обмежувати затрати на цю дорогу статтю
їх підприємств. Це привело до дуже важливих результатів. По-перше: багато казанів, в яких половина
поверхні в добрі
старі часи високих зисків лишалась під впливом холодного зовнішнього повітря, тепер покрили товстою
повстю або цеглою і штукатуркою та іншими матеріалами, чим запобігали утраті тепла,
добутого з такими великими витратами. Таким самим способом
почали захищати парові труби і так само обгортати повстю й деревом циліндри. По-друге, почали
вживати високе тиснення,
