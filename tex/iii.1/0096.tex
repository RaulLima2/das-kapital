шення їх матеріалу, наприклад, з заміни дерева залізом; 2) із
здешевлення машин у наслідок поліпшення фабрикації машин
взагалі; так що, хоч вартість основної частини сталого капіталу
безперервно зростає з розвитком праці у великому масштабі,
вона зростає далеко не в такій самій мірі; 12 3) з спеціальних поліпшень,
які дозволяють уже наявним машинам працювати дешевше
і ефективніше, наприклад, з поліпшення парових казанів
і т. п., про що деякі подробиці дамо пізніш; 4) із зменшення
відпадів у наслідок поліпшених машин.

Все, що зменшує зношування машин і взагалі основного капіталу
за даний період виробництва, не тільки здешевлює окремий
товар, бо кожний окремий товар репродукує в своїй ціні
відповідну частину зношування, що припадає на нього, але
й зменшує відповідні видатки капіталу за цей період. Ремонтні
роботи і т. п., в тій мірі, в якій вони стають потрібні, зараховуються
при обчисленні до первісних витрат на машини. їх зменшення,
в наслідок більшої міцності машин, зменшує pro tanto
[відповідно до цього] ціну машин.

Про всяку економію цього роду знов таки здебільшого можна
сказати, що вона можлива тільки для комбінованого робітника
і часто може здійснитись тільки при роботах в ще більшому
масштабі; що вона, отже, вимагай ще більшої комбінації робітників
безпосередньо в процесі виробництва.

Але, з другого боку, розвиток продуктивної сили праці
в одній галузі виробництва, наприклад, у виробництві заліза,
вугілля, машин, у будівельній справі і т. д., який у свою чергу
почасти може залежати від успіхів у сфері інтелектуального
виробництва, а саме природничих наук і їх застосування, являє
тут собою умову зменшення вартості засобів виробництва, а тому
й витрат на них в інших галузях промисловості, наприклад,
у текстильній промисловості або землеробстві. Це зрозуміло
само собою, бо товар, який як продукт виходить з однієї галузі
промисловості, знову входить в іншу як засіб виробництва.
Більша чи менша дешевина товару залежить від продуктивності
праці в тій галузі виробництва, з якої він виходить як продукт,
і разом з цим вона є умовою не тільки здешевлення тих товарів,
у виробництво яких він входить як засіб виробництва, але
й умовою зменшення вартості сталого капіталу, елементом якого
він тут стає, отже, і умовою підвищення норми зиску.

Характерне для цього роду економії на сталому капіталі,
яка походить з прогресивного розвитку промисловості, є те,
що тут підвищення норми зиску в одній галузі промисловості
спричинюється розвитком продуктивної сили праці в іншій галузі.
Те, що тут іде на користь капіталістові, є знов таки вигода,
яка є продуктом суспільної праці, хоч і не продуктом робітників,
експлуатованих безпосередньо самим цим капіталістом.

12 Див. Ure про прогрес у будуванні фабрик.
