ництва, а в процес циркуляції, а тому в сукупний процес репродукції.

Єдина частина цих витрат, яка нас тут цікавить, це частина,
витрачена на змінний капітал. (Крім того, слід було б дослідити:
поперше, яким чином зберігає в процесі циркуляції своє
значення закон, згідно з яким у вартість товару входить тільки
необхідна праця? Подруге, в чому виявляється нагромадження
при купецькому капіталі? Потрете, як функціонує купецький капітал
у дійсному сукупному суспільному процесі репродукції?)

Ці витрати зумовлюються економічною формою продукту як
товару.

Якщо робочий час, який втрачають самі промислові капіталісти
на те, щоб продавати свої товари безпосередньо один
одному, — отже, об’єктивно кажучи, час обігу товарів, — зовсім
не додає до цих товарів ніякої вартості, то ясно, що цей робочий
час не набуває іншого характеру від того, що його доводиться
втрачати не промисловому капіталістові, а купцеві. Перетворення
товару (продукту) в гроші і грошей у товар (у засоби
виробництва) є необхідна функція промислового капіталу і, отже,
необхідна операція капіталіста, який в дійсності є тільки персоніфікований
капітал, обдарований власною свідомістю і волею.
Але ці функції не збільшують вартості і не створюють додаткової
вартості. Виконуючи ці операції або опосереднюючи далі
функції капіталу в сфері циркуляції, після того як продуктивний
капіталіст перестав це робити, купець тільки займає місце
промислового капіталіста. Робочий час, що його коштують ці
операції, вживається на необхідні операції в процесі репродукції
капіталу, але він не додає ніякої вартості. Коли б купець
не виконував цих операцій (отже, і не витрачав би потрібного на
них робочого часу), то він не застосовував би свого капіталу
як агент циркуляції промислового капіталу, він не продовжував би
перерваної функції промислового капіталіста, і тому не брав би
участі як капіталіст pro rata [пропорціонально до] свого авансованого
капіталу в одержанні маси зиску, вироблюваної класом
промислових капіталістів. Тому, щоб брати участь в одержанні
маси додаткової вартості, щоб авансована ним сума зростала
у своїй вартості як капітал, торговельному капіталістові немає
потреби вживати найманих робітників. Якщо його підприємство
і його капітал невеликі, то він сам може бути тим єдиним робітником,
якого він уживає. Те, чим він оплачується, є частина
зиску, яка виникає для нього з ріжниці між купівельною ціною
товарів і їх дійсною ціною виробництва.

Але, з другого боку, при незначному розмірі авансовуваного
купцем капіталу, зиск, який він реалізує, може бути ні трохи
не більший і навіть менший, ніж заробітна плата краще оплачуваних
вправних найманих робітників. Справді, поряд з ним функціонують
безпосередні торговельні агенти продуктивного капіталіста,
закупники, продавці, комівояжери, які одержують стільки ж
