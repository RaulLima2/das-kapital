гроші, з’єднується тут з грішми без опосереднюючого проміжного
руху, просто як їх характер, як їх визначеність. І в цій
визначеності вони відчужуються, коли віддаються в позику як
грошовий капітал.

У Прудона дивне розуміння ролі грошового капіталу („Gratuité
du Crédit. Discussion entre M. F. Bastiat et M. Proudhon“.
Paris 1850). Позика здається Прудонові злом тому, що вона не
є продаж. Позика за проценти є „спроможність знову й знову продавати
той самий предмет і знову й знову одержувати його ціну,
не відступаючи ніколи власності на те, що продається“\footnote*{
З першого листа до Бастіа, написаного Ф. Шеве, прихильником Прудона
і редактором „Voix du Peuple“, що відкрив дискусію (22 жовтня 1849 р.) Примітка
ред. нім. вид. ІМЕЛ.
} (стор. 9).
Предмет, гроші, будинок і т. д. не міняють свого власника, як
це має місце при купівлі й продажу. Але Прудон не бачить,
що при віддаванні грошей у формі капіталу, що дає процент, за
них не одержують ніякого еквіваленту. В кожному акті купівлі
й продажу, оскільки взагалі відбуваються процеси обміну, об’єкт
дійсно віддається. Власність на проданий предмет кожного разу
відступається. Але вартість при цьому не віддається. При продажу
віддається товар, але не його вартість, яка повертається
у формі грошей або в формі боргового зобов’язання чи боргової
розписки, що тут є тільки іншою формою грошей. При купівлі віддаються
гроші, але не їх вартість, яка заміщається в формі товару.
На протязі всього процесу репродукції промисловий капіталіст
тримає в своїх руках ту саму вартість (залишаючи
осторонь додаткову вартість), тільки в різних формах.

Оскільки відбувається обмін, тобто обмін предметів, не відбувається
ніякої зміни вартості. Той самий капіталіст завжди
тримає в своїх руках ту саму вартість. Але, оскільки капіталістом
додаткова вартість ще тільки виробляється, обміну не відбувається;
коли ж відбувається обмін, додаткова вартість уже
міститься в товарах. Якщо ми розглядатимем не окремі акти
обміну, а весь кругобіг капіталу, Г — Т — Г', то певна сума вартості
постійно авансується і ця ж сума вартості плюс додаткова вартість
або зиск вилучається назад з циркуляції. Опосереднюючої
ланки цього процесу в простих актах обміну, звичайно, не видно.
Але якраз на цьому процесі Г як капіталу грунтується процент
грошового капіталіста-позикодавця, і з цього процесу він
виникає.

„Справді, — каже Прудон, — капелюшник, який продає капелюхи...
одержує за них їх вартість, не більше й не менше. Але
капіталіст-позикодавець... не тільки одержує назад свій капітал
незменшеним; він одержує більше, ніж свій капітал, більше, ніж
він кидає в обмін; поверх капіталу він одержує ще й процент“
(там же стор. 69). Капелюшник представляє тут продуктивного
капіталіста в протилежність до капіталіста-позикодавця.
Прудон, очевидно, не дійшов таємниці того, яким чином продук-