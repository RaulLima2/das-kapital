охорони здоров’я є універсальна. І в інтересах мільйонів робітників і робітниць, життя яких тепер
без потреби калічиться
і скорочується безконечними фізичними стражданнями, породжуваними тільки їх працею, я зважуюсь
висловити надію, що
санітарні умови праці так само повсюди будуть поставлені під
належний захист закону, принаймні настільки, щоб була забезпечена справжня вентиляція в усіх
закритих робочих приміщеннях і щоб у кожній галузі праці, по своїй природі шкідливій для здоров’я,
по можливості був обмежений особливо
небезпечний для здоров’я вплив“ (стор. 31).

III. Економія на здобуванні сили, на передачі сили і на будівлях

У своєму жовтневому звіті за 1852 рік Л. Горнер цитує лист
відомого інженера Джемса Насміта з Патрікрофта, винахідника
парового молота, де, між іншим, сказано:

„Публіка дуже мало знайома з колосальним приростом рушійної сили, досягнутим за допомогою таких змін
системи й поліпшень“ [у парових машинах], „як ті, про які я кажу. Сила машин нашої округи
(Ланкашіру) перебувала протягом майже 40 років
під гнітом боязкої і повної передсудів рутини, але, на щастя,
ми тепер від неї емансипувались. Протягом останніх 15 років,
особливо ж на протязі останніх 4 років“ [отже, з 1848 р.], „стались деякі дуже важливі зміни в
способі використовування конденсаційних парових машин... В результаті... ті самі машини виконують
далеко більше роботи, і до того ж при значно зменшеному споживанні вугілля... Протягом дуже багатьох
років
з часу введення парової сили на фабриках цієї округи гадали,
що швидкість, з якою можуть працювати конденсаційні парові
машини, становить приблизно 220 футів підіймання поршня на
хвилину; тобто машина з підійманням поршня в 5 футів була вже
за встановленою нормою обмежена 22 оборотами ексцентрика на
хвилину. Вважалось за недоцільне гнати машину швидше; а через
те що весь механізм був пристосований до цієї швидкості руху
поршня в 220 футів на хвилину, ця мала і безглуздо обмежена
швидкість панувала в усій промисловості протягом багатьох років.
Але, нарешті, чи в наслідок щасливого незнання встановленої норми
чи з інших, кращих причин, якийсь сміливий новатор спробував
більшу швидкість, і тому що результат був надзвичайно сприятливий, інші наслідували цей приклад;
машині, як тоді казали, попустили віжки і змінили головні колеса передатного механізму таким
чином, що парова машина могла робити 300 футів і більше на
хвилину, в той час як механізми зберігали свою колишню швидкість... Це прискорення руху парової
машини стало тепер майже
загальним, бо виявилось, що не тільки з тієї самої машини здобувалося більше корисної сили, але що й
рух в наслідок біль-
