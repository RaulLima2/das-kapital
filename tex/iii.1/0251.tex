Найбільш руйнівного впливу, і при тому найгострішого
характеру, зазнав би капітал, оскільки він має властивість
вартості, зазнали б капітальні вартості. Частина капітальної
вартості, яка перебуває просто у формі посвідок на майбутню
участь в додатковій вартості, в зиску, і яка в дійсності становить
тільки боргові зобов’язання в різних формах на виробництво,
відразу знецінюється з падінням доходів, на які вона розрахована.
Частина готівки золота й срібла лежить без діла, не функціонує
як капітал. Частина товарів, що перебувають на ринку, може
здійснити свій процес циркуляції і репродукції тільки шляхом
надзвичайного зниження своїх цін, отже, шляхом знецінення
того капіталу, який вона представляє. Цілком так само більше
чи менше знецінюються елементи основного капіталу. До цього
долучається ще й те, що певні припущені відношення цін
обумовлюють процес репродукції, і тому цей останній в наслідок
загального падіння цін приходить до застою і розладу. Цей
розлад і застій паралізує функцію грошей як платіжного засобу,
яка розвивається одночасно з розвитком капіталу і грунтується
на згаданих припущених відношеннях цін; він розриває у сотнях
місць ланцюг платіжних зобов’язань на певні строки і ще більше
загострюється в наслідок зумовленого цим краху (Zusammenbrechen)
кредитної системи, що розвинулась одночасно з капіталом,
і таким чином веде до сильних і гострих криз, до раптових
насильних знецінень і дійсного застою й розладу * процесу
репродукції, і тим самим до дійсного зменшення репродукції.

Але одночасно діяли б і інші фактори. Застій виробництва
позбавив би роботи частину робітничого класу і цим поставив би
заняту частину його в такі відносини, при яких вона мусила б
згоджуватись на зниження заробітної плати навіть нижче пересічного
рівня; обставина, яка дає для капіталу такий самий результат,
як коли б при пересічній заробітній платі була підвищена
відносна чи абсолютна додаткова вартість. Період процвітання
сприяв би шлюбам серед робітників і зменшив би смертність їх
дітей, обставини, які — хоч би й яке вони означали дійсне збільшення
населення — не означають збільшення дійсно працюючого
населення, але на відношення робітників до капіталу впливають
цілком так само, як коли б збільшилося число дійсно функціонуючих
робітників. З другого боку, падіння цін і конкурентна
боротьба спонукали б кожного капіталіста підвищувати індивідуальну
вартість свого сукупного продукту понад його загальну
вартість за допомогою застосування нових машин, нових поліпшених
методів праці, нових комбінацій, тобто підвищувати продуктивну
силу даної кількості праці, знижувати відношення
змінного капіталу до сталого, і тим самим звільняти робітників,

* В першому німецькому виданні тут стоїть: „застою й занепаду (Sturz)“;
виправлено на підставі рукопису Маркса. Примітка ред. нім. вид, ІМЕЛ.
