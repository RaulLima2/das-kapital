в падінні норми зиску такий закон, який на певному пункті
найбільш вороже виступає проти розвитку самого цього способу
виробництва і який через це мусить раз-у-раз переборюватись
кризами.

2) В тому, що для розширення чи скорочення виробництва
вирішальним є привласнення неоплаченої праці і відношення цієї
неоплаченої праці до упредметненої праці взагалі або, висловлюючись
капіталістично, що вирішальним для цього є зиск
і відношення цього зиску до застосовуваного капіталу, отже,
певна висота норми зиску, а не відношення виробництва до
суспільних потреб, до потреб суспільно розвинених людей. Тому
капіталістичне виробництво доходить до своєї межі вже на
такому ступені розширення виробництва, який, навпаки, при
інших передумовах був би далеко недостатнім. Воно припиняється
не тоді, коли цього вимагає задоволення потреб, а тоді,
коли цього припинення вимагає виробництво і реалізація зиску.

Якщо норма зиску знижується, то, з одного боку, сили
капіталу скеровуються на те, щоб окремий капіталіст за допомогою
кращих методів і т. д. знизив індивідуальну вартість
кожної одиниці своїх товарів нижче її суспільної пересічної
вартості і таким чином одержав би при даній ринковій ціні
надзиск; з другого боку, виникають грюндерські підприємства
і загальний сприятливий грунт для них в завзятих спробах застосування
нових методів виробництва, в нових капіталовкладеннях,
в нових авантюрах, щоб забезпечити хоч якийнебудь
надзиск, який не залежав би від загального пересічного рівня
і перевищував би його.

Норма зиску, тобто відносний приріст капіталу, має важливе
значення передусім для всіх нових паростків капіталу, які групуються
самостійно. І коли б утворення капіталів потрапило
виключно в руки деяких небагатьох уже наявних великих
капіталів, для яких маса зиску урівноважує його норму, то
взагалі згас би вогонь, який оживляє виробництво. Виробництво
охопив би сон. Норма зиску є рушійна сила капіталістичного
виробництва; виробляється тільки те і остільки, що і оскільки
може бути вироблене з зиском. Звідси острах англійських економістів
перед зменшенням норми зиску. Те, що вже сама тільки
можливість цього непокоїть Рікардо, свідчить якраз про його
глибоке розуміння умов капіталістичного виробництва. Якраз те,
що йому закидають, а саме, що він при розгляді капіталістичного
виробництва, не турбуючись про „людей“, звертає увагу
тільки на розвиток продуктивних сил, — яких би це не коштувало
жертв людьми і капітальними вартостями — якраз це є
в нього найвизначніше. Розвиток продуктивних сил суспільної
праці є історичне завдання і виправдання капіталу. Саме цим він
несвідомо утворює матеріальні умови вищої форми виробництва.
Рікардо непокоїть те, що нормі зиску, цьому стимулові капіталістичного
виробництва, умові й рушієві нагромадження, загро-
