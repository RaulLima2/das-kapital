зиску, ніж дрібний капіталіст, який, видимо, одержує високий зиск. Далі, найповерховіше
спостереження конкуренції показує, що при певних обставинах, коли більший капіталіст хоче захопити
для себе місце на ринку, витиснути дрібніших капіталістів, — як, наприклад, за часів кризи, — він
використовує це практично, тобто навмисно знижує свою норму зиску, щоб витиснути з ринку дрібніших
капіталістів. Так само й купецький капітал — про який ми пізніше скажемо докладніше — показує явища,
завдяки яким зниження зиску здається наслідком розширення підприємства, а разом з тим і капіталу.
Власне науковий вираз замість помилкового розуміння ми дамо пізніше. Подібні поверхові погляди є
результатом порівнення норм зиску, одержуваних в окремих галузях підприємств залежно від того, чи
підпорядковані вони режимові вільної конкуренції чи монополії. Цілком банальне уявлення, яке
створюється в головах агентів конкуренції, ми знаходимо в нашого Рошера, а саме, що таке зниження
норми зиску є „розумніше й гуманніше“.* Зменшення норми зиску представлено тут як наслідок
збільшення капіталу і зв’язаного з цим розрахунку капіталістів, що при меншій нормі зиску маса
зиску, яку вони кладуть собі в кишеню, буде більша. Все це (за винятком того, що є в А. Сміта, про
що пізніше) основане на цілковитому нерозумінні того, що таке взагалі є загальна норма зиску, і на
тому грубому уявленні, що ціни дійсно визначаються шляхом надбавки більш-менш довільної частки зиску
до дійсної вартості товарів. Хоч які грубі ці уявлення, все ж вони з необхідністю виникають з того
перекрученого способу й вигляду, в якому імманентні закони капіталістичного виробництва виявляються
в сфері конкуренції.

Закон, згідно з яким падіння норми зиску, викликуване розвитком продуктивної сили, супроводиться
збільшенням маси зиску, виражається і в тому, що падіння цін товарів, вироблюваних капіталом,
супроводиться відносним збільшенням мас зиску, які містяться в них і реалізуються через їх продаж.

Через те що розвиток продуктивної сили і відповідний цьому вищий склад капіталу приводить в рух
дедалі більшу кількість засобів виробництва за допомогою дедалі меншої кількості праці, то кожна
пропорціональна частина всього продукту, кожна одиниця товару або кожна певна окрема кількість
товару, яка служить одиницею міри для сукупної маси вироблених товарів, вбирає менше живої праці і
містить у собі, крім того, менше упредметненої праці як щодо зношення застосованого основного
капіталу,
так і щодо спожитих сировинних і допоміжних матеріалів. Отже, кожна одиниця товару містить у собі
меншу суму праці як упред-

* „Die Grundlagen der Nationalökonomie“. 2 Aufl. Stuttgart und Augsburg 1857,
стор. 190. Примітка ред. нім. вид. ІМЕЛ.
