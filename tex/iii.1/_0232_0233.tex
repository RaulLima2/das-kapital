\parcont{}  %% абзац починається на попередній сторінці
\index{iii1}{0232}  %% посилання на сторінку оригінального видання
додаткової праці здовження робочого дня, — цей винахід сучасної промисловості, — не змінюючи при
цьому істотно відношення вживаної робочої сили до сталого капіталу, який вона приводить в рух, і в
дійсності скорше відносно зменшуючи сталий капітал. Взагалі ж ми вже показали, — і це становить
власне таємницю тенденції норми зиску до падіння, —-що методи виробництва відносної додаткової
вартості загалом і в цілому зводяться ось до чого: з одного боку, з даної маси праці якомога більше
перетворити в додаткову вартість, з другого боку, взагалі вживати якомога менше праці порівняно з
авансованим капіталом; так що ті самі причини, які дозволяють підвищувати ступінь експлуатації
праці, не дозволяють з тим самим сукупним капіталом експлуатувати стільки ж праці, як і раніш. Такі
є протилежні тенденції, які, викликаючи підвищення норми додаткової вартості, одночасно викликають
падіння маси додаткової вартості, вироблюваної даним капіталом, а тому й падіння норми зиску. Тут
слід також згадати про масове вживання жіночої і дитячої праці, тому що при цьому вся сім’я мусить
давати капіталові більшу масу додаткової праці, ніж раніше, навіть якщо зростає загальна сума
заробітної плати, виплачуваної цій сім’ї, — випадок, який аж ніяк не є загальним явищем. — Такий
самий вплив справляє все те, що сприяє при незмінній величині застосовуваного капіталу виробництву
відносної додаткової вартості шляхом самого тільки поліпшення методів, як у землеробстві. Хоча тут
застосовуваний сталий капітал не зростає в порівнянні з змінним, оскільки ми розглядаємо цей
останній як показник уживаної робочої сили, але маса продукту зростає порівняно з ужитою робочою
силою. Те саме має місце, коли продуктивна сила праці (однаково, чи входить її продукт у споживання
робітників, чи в елементи сталого капіталу) звільняється від перешкод, що утруднюють зносини, від
самовільних обмежень або від обмежень, які стали обтяжливими з бігом часу, взагалі від усякого роду
пут, при чому цим не зачіпається відношення змінного капіталу до сталого.

Можна було б поставити питання: чи входять у число тих причин, які гальмують падіння норми зиску,
але в кінцевому рахунку завжди прискорюють його, тимчасові, але що завжди повторюються, виявляються
то в одній, то в другій галузі виробництва підвищення додаткової вартості понад загальний рівень для
капіталіста, який використовує винаходи і т. д., поки вони ще не стали загальнопоширеними. На це
питання треба відповісти позитивно.

Маса додаткової вартості, вироблена капіталом даної величини, є добуток двох множників — норми
додаткової вартості, помноженої на число робітників, занятих при даній нормі. Отже, при даній нормі
додаткової вартості маса її залежить від числа
робітників, а при даному числі робітників — від норми додаткової вартості, тобто взагалі від
складного відношення абсолютної
\index{iii1}{0233}  %% посилання на сторінку оригінального видання
величини змінного капіталу і норми додаткової вартості. Але ми показали, що пересічно ті самі
причини, які підвищують норму відносної додаткової вартості, зменшують масу вживаної робочої сили.
Проте, ясно, що збільшення або зменшення тут відбувається залежно від певного відношення, в якому
відбувається цей протилежний рух, і що тенденція до зменшення норми зиску послаблюється зокрема в
наслідок підвищення норми абсолютної додаткової вартості, яка походить із здовження робочого дня.

При дослідженні норми зиску ми взагалі виявили, що зниженню норми, яке відбувається в наслідок
зростання маси всього застосовуваного капіталу, відповідає збільшення маси зиску. Якщо розглядати
сукупний змінний капітал суспільства, то
створена ним додаткова вартість дорівнює створеному зискові. Разом з абсолютною масою додаткової
вартості виросла і норма додаткової вартості; перша виросла тому, що збільшилась вживана
суспільством маса робочої сили, друга — тому, що підвищився ступінь експлуатації цієї праці. Але
відносно капіталу даної величини, наприклад, 100, норма додаткової вартості може зрости, тоді як
маса її пересічно падає; бо норма визначається відношенням, в якому змінна частина капіталу зростає
в своїй вартості, а маса визначається, навпаки, тією відносною частиною, яку становить змінний
капітал в усьому капіталі.

Підвищення' норми додаткової вартості — через те що воно відбувається і при таких обставинах, коли,
як це показано вище, не відбувається ніякого збільшення або не відбувається пропорціонального
збільшення сталого капіталу порівняно з змінним — є один з факторів, яким визначається маса
додаткової вартості, а тому й норма зиску. Цей фактор не знищує загального закону. Але він робить
те, що цей закон діє більше як тенденція, тобто як закон, абсолютне здійснення якого затримується,
уповільнюється і ослаблюється протидіючими обставинами. Але через те що ті самі причини, які
підвищують норму додаткової вартості (навіть здовження робочого дня є результат великої
промисловості), мають тенденцію зменшувати вживану даним капіталом кількість робочої сили, то одні й
ті самі причини мають тенденцію зменшувати норму зиску і уповільнювати рух цього зменшення. Якщо
одному робітникові накидають таку працю, яку раціонально виконати можуть тільки двоє, і якщо це
відбувається при таких обставинах, коли цей робітник може заступити трьох, то один робітник даватиме
стількидодаткової праці, скільки раніш давало двоє, і остільки норма додаткової вартості
підвищиться. Але він не даватиме стільки, скільки раніш давало троє, і остільки маса додаткової
вартості знизиться. Але це зниження компенсується або обмежується підвищенням норми додаткової
вартості. Якщо все населення працюватиме при підвищеній нормі додаткової вартості, то маса
додаткової вартості збільшиться, хоч населення лишиться тим
\parbreak{}  %% абзац продовжується на наступній сторінці
