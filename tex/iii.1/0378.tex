беручи до уваги умов репродукції і праці, як самодіяльний
автомат, як просте число, що само собою збільшується (цілком
так само як Мальтус розглядав людей у своїй геометричній
прогресії), то він уявив, що відкрив закон зростання капіталу
в формулі s = с (1 + z) n, де s = сумі капіталу + проценти на
проценти, с = авансованому капіталові, z = розмірові процента
(вираженому у відповідних частинах 100), а n — ряд років, протягом
яких відбувається процес.

Пітт цілком серйозно приймає містифікацію д-ра Прайса.
В 1786 році палата громад ухвалила зібрати 1 мільйон фунтів стерлінгів
на громадські потреби. За Прайсом, в якого вірував Пітт,
не було нічого кращого, як оподаткувати народ, щоб „нагромадити“
одержану таким способом суму і таким чином за допомогою
таїнства складних процентів чарами позбутись державного
боргу. Після цій резолюції палати громад незабаром з ініціативи
Пітта був виданий закон, який приписував нагромаджувати по
250 000 фунтів стерлінгів доти, „поки фонд разом з відмерлими
пожиттьовими рентами зросте до 4 000 000 фунтів стерлінгів
на рік“ (Act 26, Georg III, Кар. 31 *). Пітт у своїй промові 1792 року,
в якій він пропонував збільшити суму, призначену для фонду
оплати боргів, серед причин торговельної переваги Англії наводить:
машини, кредит і т. д., але як „найпоширенішу і найтривкішу
причину — нагромадження. Принцип цей цілком розвинутий
і досить пояснений у творах Сміта, цього генія... Це нагромадження
капіталів було б викликане, коли б відкладали принаймні
частину річного зиску для того, щоб збільшити основну
суму, яка, при такому самому застосуванні її в наступному році,
давала б постійний зиск“. Таким чином за допомогою д-ра Прайса
Пітт перетворює теорію нагромадження Сміта в теорію збагачення
народу шляхом нагромадження боргів і приходить до
приємного безконечно прогресуючого збільшення позик, позик
для оплати позик.

Вже в Josias Child’a, батька сучасних банкірів, ми знаходимо,
що „100 фунтів стерлінгів з 10\% виробили б за 70 років, при
процентах на проценти, 102 400 фунтів стерлінгів“ („Traité sur le
commerce etc. par J. Child, traduit etc.“ Amsterdam et Berlin
1754, стор. 115. Написано в 1669 році).

Як погляд д-ра Прайса непродумано прохоплюється у сучасних
економістів, показує таке місце з „Economist’a“: „Capital,
with compound interes on every portion of capital saved, is so
all-engrossing that all the wealth in the world from which income
is derived, has long ago become the interest of capital... all rent is
now the payement of interest on capital previously invested in the
land“ [„Капітал із складними процентами на кожну частину заощадженого
капіталу є до такої міри всезахоплюючим, що все

* Тобто 31-й закон від 26-го року королювання Георга III. Примітка ред.
нім. вид. ІМЕЛ.
