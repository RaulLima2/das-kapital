\parcont{}  %% абзац починається на попередній сторінці
\index{iii1}{0331}  %% посилання на сторінку оригінального видання
особливою формою грошового капіталу. Адже в позику тут
завжди віддається певна сума грошей, і на цю ж таки суму обчислюється
й процент. Якщо те, що віддається в позику, не є ні
гроші, ні обіговий капітал, то й сплачується воно назад таким
способом, яким припливає назад основний капітал. Позикодавець
одержує періодично процент і частину спожитої вартості самого
основного капіталу, еквівалент періодичного зношування.
А після скінчення строку повертається назад in natura неспожита
частина позиченого основного капіталу. Якщо позичений
капітал є обіговий капітал, то він також і повертається назад
до позикодавця таким способом, яким припливає назад обіговий
капітал.

Отже, спосіб зворотного припливу в кожному окремому випадку
визначається дійсним круговим рухом капіталу, що репродукується,
і його особливих видів. Але для капіталу, відданого
в позику, зворотний приплив набирає форми зворотної виплати,
тому що авансування, відчуження цього капіталу має
форму позики.

В цьому розділі ми розглядаємо тільки власне грошовий капітал,
з якого виводяться інші форми капіталу, що віддається
в позику.

Відданий у позику капітал припливає назад двічі; у процесі
репродукції він повертається до функціонуючого капіталіста, а
потім повернення повторюється ще раз як передача позикодавцеві,
грошовому капіталістові, як зворотна виплата капіталу
його дійсному власникові, юридичній вихідній точці капіталу.

В дійсному процесі циркуляції капітал завжди виступає тільки
як товар або гроші, і його рух зводиться до ряду купівель і
продажів. Коротко кажучи, процес циркуляції зводиться до метаморфози
товару. Інша справа, якщо ми розглянемо процес репродукції
в цілому. Якщо за відпровідну точку ми візьмемо гроші
(а це те саме, як коли б ми за відпровідну точку взяли товар, бо ми
тоді виходили б з його вартості, отже, розглядали б його самого
sub specie [у формі] грошей), то побачимо, що певна сума
грошей витрачена і через певний час вона повертається з деяким
приростом. Повертається назад заміщення авансованої
суми грошей плюс додаткова вартість. Вона збереглася і збільшилась,
проробивши певний круговий рух. Але гроші, оскільки
вони віддаються в позику як капітал, віддаються в позику саме
як отака грошова сума, що зберігається і збільшується, — як
сума, яка через певний час повертається назад з додатком
і завжди може знову проробити той самий процес. Вони не витрачаються
ні як гроші, ні як товар, отже, вони ні обмінюються
на товар, коли вони авансуються як гроші, ні продаються за
гроші, коли авансуються як товар; вони витрачаються як капітал.
Відношення до самого себе, у вигляді якого виступає капітал,
якщо розглядати капіталістичний процес виробництва як цілість
і єдність, і в якому капітал виступає як гроші, що породжують
\parbreak{}  %% абзац продовжується на наступній сторінці
