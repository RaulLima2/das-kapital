\parcont{}  %% абзац починається на попередній сторінці
\index{iii1}{0421}  %% посилання на сторінку оригінального видання
а також тому, що більшість тих, хто витрачає дохід, робітники,
порівняно мало можуть купити в кредит; тимчасом як в оборотах торговельного світу, де засіб
циркуляції є грошовою
формою капіталу, гроші почасти в наслідок концентрації, почасти в наслідок переважання кредитної
системи функціонують
головним чином як засіб платежу. Але ріжниця між грішми як засобом платежу і грішми як засобом
купівлі (засобом циркуляції)
є ріжниця, належна самим грошам, а не ріжниця між грішми
і капіталом. З тієї причини, що в роздрібній торгівлі циркулює
більше міді й срібла, а в гуртовій — більше золота, ріжниця
між сріблом і міддю, з одного боку, і золотом, з другого боку,
не є ріжниця між засобом циркуляції і капіталом.

До пункту 2) про приплутання питання про кількість грошей, що циркулюють в обох функціях разом:
оскільки гроші
циркулюють як засіб купівлі чи як засіб платежу, — однаково,
в якій з обох сфер і незалежно від їх функції реалізації доходу
або капіталу, — закони, розвинені раніше при дослідженні простої товарної циркуляції (книга І, розд.
III, 2, b), зберігають свою
силу для кількості циркулюючих грошей. Ступінь швидкості
циркуляції, тобто число повторень тією самою монетою за даний
період часу тієї самої функції засобу купівлі й засобу платежу,
кількість одночасних купівель і продажів або платежів, сума
цін товарів, що циркулюють, нарешті, платіжні баланси, які
треба одночасно покрити, в обох випадках визначають масу циркулюючих грошей, currency. Чи
представляють гроші, що функціонують таким чином, для платіжника або одержувача капітал
чи дохід, це байдуже, це абсолютно нічого не міняє в стані
справи. Маса грошей визначається просто їх функцією як засобу купівлі й засобу платежу.

До пункту 3), до питання про відносні пропорції кількостей
засобів циркуляції, що циркулюють в обох функціях і тому
в обох сферах процесу репродукції. Обидві сфери циркуляції
стоять у внутрішньому зв’язку між собою, оскільки, з одного
боку, кількість доходів, що мають витрачатись, виражає розмір
споживання, а з другого боку, величина мас капіталу, що циркулюють у виробництві й торгівлі, виражає
розмір і швидкість
процесу репродукції. Не зважаючи на це, ці обставини впливають різно, і навіть у протилежному
напрямі, на кількості грошових мас, що циркулюють в обох функціях чи сферах, або,
як кажуть англійці банковою мовою, на кількості циркуляції.
І це знову дає Тукові привід до безглуздого розрізнення циркуляції і капіталу. Та обставина, що
панове прихильники теорії
currency переплутують дві різні речі, зовсім не є підставою
для того, щоб зображати їх як різні поняття.

В періоди процвітання, великого розширення, прискорення
і енергії процесу репродукції, робітники зайняті повністю. Здебільшого настає також підвищення
заробітної плати і дещо вирівнює її падіння нижче пересічного рівня в інші періоди комерційного
\index{iii1}{0422}  %% посилання на сторінку оригінального видання
циклу. Разом з тим значно зростають доходи капіталістів. Споживання всюди підвищується.
Товарні ціни також
регулярно підвищуються, принаймні, в різних вирішальних галузях підприємств. В наслідок цього
зростає кількість циркулюючих грошей, принаймні в певних межах, тому що більша
швидкість циркуляції в свою чергу ставить межі зростанню кількості засобів, що циркулюють. Через те
що частина суспільних
доходів, яка складається з заробітної плати, первісно авансується промисловим капіталістом у формі
змінного капіталу
і завжди у формі грошей, то в періоди процвітання вона потребує більше грошей для своєї циркуляції.
Але ми не повинні
рахувати їх двічі: раз як гроші, потрібні для циркуляції змінного капіталу, і ще раз як гроші,
потрібні для циркуляції доходу робітників. Гроші, які виплачуються робітникам як заробітна плата,
витрачаються в роздрібній торгівлі і таким чином
приблизно щотижня повертаються назад до банків як вклади роздрібних торговців, після того, як вони
обслужать ще в дрібних
кругобігах різного роду побічні справи. В періоди процвітання
зворотний приплив грошей до промислових капіталістів відбувається гладко і таким чином їх потреба в
грошових позиках
зростає не через те, що вони повинні виплатити більше заробітної плати, потребують більше грошей для
циркуляції їх змінного капіталу.

Загальний результат є той, що в періоди процвітання кількість засобів циркуляції, яка служить для
витрачання доходів,
рішуче зростає.

Що ж до циркуляції, потрібної для передачі капіталу, отже,
тільки між самими капіталістами, то цей час жвавих справ
є разом з тим періодом найеластичнішого і найлегшого кредиту. Швидкість циркуляції між капіталістом
і капіталістом
регулюється безпосередньо кредитом, і кількість засобів циркуляції, яка потрібна для сальдування
платежів і навіть для купівель за готівку, таким чином порівняно зменшується. Абсолютно
вона може збільшитись, але відносно вона при всіх обставинах зменшується, порівняно з розширенням
процесу репродукції. З одного боку, великі масові платежі ліквідуються без усякого посередництва
грошей; з другого боку, при великому пожвавленні процесу, панує прискорений рух тієї самої кількості
грошей, — як в їх функції засобу купівлі, так і в їх функції засобу платежу. Та сама кількість
грошей опосереднює зворотний
приплив більшої кількості окремих капіталів.

Загалом, в такі періоди грошовий обіг заповнений (full),
хоч частина II (передача капіталу) скорочується, принаймні
відносно, тимчасом як частина І (витрачання доходів) абсолютно
розширюється.

Зворотні припливи грошей виражають зворотне перетворення
товарного капіталу в гроші, $Г — Т — Г'$, як ми це бачили при
дослідженні процесу репродукції, книга II, відділ І. Завдяки кредитові
\index{iii1}{0423}  %% посилання на сторінку оригінального видання
зворотний приплив у грошовій формі стає незалежним
від часу дійсного зворотного припливу як для промислового капіталіста, так і для купця. Кожний з них
продає в кредит; отже,
товар їх відчужується раніше, ніж він зворотно перетвориться
для них у гроші, тобто повернеться до них самих назад у грошовій формі. З другого боку, кожен з них
купує в кредит, і таким
чином вартість їх товару зворотно перетворюється для них
в продуктивний капітал чи в товарний капітал раніше, ніж ця вартість дійсно перетвориться в гроші,
раніше ніж настане строк виплати ціни товару і вона буде виплачена. В такі періоди процвітання
зворотний приплив відбувається легко й гладко. Роздрібний
торговець з цілковитою певністю платить гуртовому торговцеві,
цей останній — фабрикантові, фабрикант — імпортерові сировинного матеріалу і т. д. Видимість швидких
і певних зворотних припливів капіталу завжди тримається ще довгий час після того, як у
дійсності цього вже немає — тримається завдяки налагодженому
кредитові, бо зворотні припливи в формі кредиту заступають
дійсні зворотні припливи. Банки починають передчувати біду,
як тільки їх клієнти починають платити більше векселями, ніж
грішми. Дивись вищенаведене свідчення ліверпульського директора банку, стор. 392 і далі.

Тут треба ще вставити те, про що я згадував раніше: „В періоди,
коли кредит процвітає, швидкість обігу грошей зростає скоріше,
ніж зростають ціни товарів; тимчасом як при скороченні кредиту ціни товарів падають повільніше, ніж
швидкість циркуляції“
(„Zur Kritik der politischen Oekonomie“, 1859, стор. 83, 84 [„До
критики політичної економії“, укр. вид. 1935 р., стор. 133]).

В періоди криз справа стоїть навпаки. Циркуляція № І скорочується, ціни падають, так само й
заробітні плати; число зайнятих робітників скорочується, маса оборотів зменшується. Навпаки, в
циркуляції № II з скороченням кредиту зростає потреба
в грошових позиках — обставина, яку ми зараз розглянемо докладніше.

Не підлягає ніякому сумніву, що при скороченні кредиту,
яке збігається з застоєм у процесі репродукції, маса засобів
циркуляції, потрібна для № І, витрачання доходів, зменшується,
тимчасом як маса засобів циркуляції, потрібна для № II, передачі капіталів, збільшується. Але треба
дослідити, наскільки
це положення тотожне з положенням, виставленим Фуллартоном
та іншими: „Попит на позиковий капітал і попит на додаткові
засоби циркуляції — зовсім різні речі і не часто зустрічаються
разом“.\footnote{
„А demand for capital on loan and a demand for additional circulation are
quite distinct things, and not often found associated“ (\emph{Fullarton}: „On the Regulation of
Currencies“, 2 вид., Лондон 1845, стор. 82, заголовок до розд. 5). — „It is
a great error, indeed, to imagine that the demand for pecuniary accommodation (i. e.
for the loan of capital) is identical with a demand for additional means of circulation,
or even that the two are frequently associated. Each demand originates in circumstances peculiarly
affecting itself, and very distinct from each other. It is when everything looks prosperous, when wages are high, prices on the rise, and factories
busy, that an additional supply of \emph{currency} is usually required to perform the additional functions
inseparable from the necessity of making larger and more numerous payments; whereas it is chiefly in
a more advanced stage of the commercial
cycle, when difficulties begin to present themselves, when markets are overstocked
and returns delayed, that interest rises, and a pressure comes upon the Bank for
advances of \emph{capital}. It is true that there is no medium through which the Bank is
accustomed to advance capital except that of its promissory notes; and that, to
refuse the notes, therefore, is to refuse the accommodation. But, the accommodation
once granted, everything adjusts itself in conformity with the necessities of the
market; the loan remains, and the currency, if not wanted, finds its way back to
the issuer. Accordingly, a very slight examination of the Parliamentary Returns
may convince any one, that the securities in the hand of the Bank of England
fluctuate more frequently in an opposite direction to its circullation than in concert
with it, and that the example, therefore, of that great establishment furnishes no
exception to the doctrine so strongly pressed by the country bankers, to the effect
that no bank can enlarge its circulation, if that circulation be already adequate to
the purposes to which a banknote currency is commonly applied; but that every
addition to its advances, after that limit is passed, must be made from its capital,
and supplied by the sale of some of its securities in reserve, or by abstinence from
further investment in such securities. The table compiled from the Parliamentary
Returns for the interval between 1833 and 1840, to which I have referred in a preceding page,
furnishes continued examples of this truth; but two of these are so
remarkable that it will be quite unnecessary for me to go beyond them. On the
3rd January, 1837, when the resources of the Bank were strained to the uttermost
to sustain credit and meet the difficulties of the money market, we find its advances on loan and
discount carried to the enormous sum of £ 17 022 000, an
amount scarcely known sinse the war, and almost equal to the entire aggregate
issues, which, in the meanwhile, remain unmoved at so low a point as £ 7 076 000!
On the other hand, we have, on the 4th of June 1833 a circulation of £ 18 892 000
with a return of private securities in hand, nearly, if not the very lowest on
record for the last halfcentury, amounting to no more than £ 972 000!“ [„Це справді
велика помилка уявляти собі, що попит на грошову позику (тобто на позику
капіталу) є тотожний з попитом на додаткові засоби циркуляції або що обидва
ці попити часто між собою зв’язані. Кожний з цих попитів виникає з особливих і в обох випадках дуже
різних обставин. Коли все має вигляд процвітання, коли заробітна плата висока, ціни підвищуються і
фабрики добре працюють, то звичайно постає потреба в додаткових \emph{засобах циркуляції} для виконання
додаткових функцій, невідділимих від необхідності робити більші і
численніші платежі; тимчасом як підвищення процента і тиск на банк, вимоги
на позики \emph{капіталу} виникають головним чином на пізнішій стадії комерційного циклу, коли починають
виявлятися труднощі, коли ринки переповнені
і зворотні припливи капіталів затримуються. Це вірно, що, крім випуску банкнот,
банк не має іншого засобу, за допомогою якого він звичайно дає в позику
капітал; і що через це відмова у видачі банкнот означає відмову у видачі позики.
Але, раз позика дозволена, то все пристосовується до потреб ринку; позика
лишається, а засіб циркуляції, якщо в ньому немає потреби, знаходить
свій шлях до того, хто його випустив. Відповідно до цього навіть цілком поверховий
розгляд парламентських звітів може переконати кожного, що кількість
цінних паперів, які перебувають у володінні Англійського банку, частіше коливається
в напрямі, протилежному до циркуляції його банкнот, ніж в одному
з нею напрямі, і, що, отже, приклад цієї великої установи не становить винятку
з тієї доктрини, яку так дуже обстоюють провінціальні банкіри, а саме доктрини,
що ніякий банк не може збільшити кількості своїх циркулюючих банкнот,
якщо кількість їх уже відповідає тим цілям, для яких звичайно вживається циркуляція
банкнот, і що всяке збільшення позик, які дає банк, після того як ця межа
перейдена, доводиться робити з його капіталу і постачати для цього засоби
шляхом продажу деяких його резервних цінних паперів або шляхом відмовлення
від дальших вкладень у ці папери. Таблиця, складена за даними парламентських
звітів за час від 1833 до 1840 року, на яку я посилався на одній з попередніх
сторінок, раз-у-раз дає приклади цієї істини; але два з них настільки варті уваги,
що я не можу їх обминути. На 3 січня 1837 р., коли ресурси банку були
напружені до крайності для підтримання кредиту і протидії утрудненням
грошового ринку, ми бачимо, що його позики і дисконт досягли величезної
суми в 17022000 фунтів стерлінгів, суми, яка навряд чи досягалась будьколи
після війни [1793--1815] і яка майже дорівнює сукупній сумі випущених
банкнот, яка тимчасом лишалась незмінною на такому низькому рівні,
як 17076000 фунтів стерлінгів! З другого боку, на 4 червня 1833 р. ми бачимо
циркуляцію банкнот на 18892000 фунтів стерлінгів при наявності в розпорядженні
банку приватних цінних паперів на суму не більше, як 972 000 фунтів
стерлінгів, отже на суму майже найнижчу, якщо не найнижчу, за останні
півстоліття“] (\emph{Фуллартон}, там же, стор. 97, 98). — Що зовсім немає потреби
в тому, щоб demand for pecuniary accommodation [попит на грошові позики] був
тотожним з demand for gold [попитом на золото] (що Вільсон, Тук і інші називають
капіталом), це видно з таких свідчень пана Weguelin’a, управителя Англійського
банку: „Дисконт векселів до цієї суми“ [до одного мільйона щоденно
протягом трьох днів поспіль] „не скоротив би резерву“ [банкнот], „коли б публіка
не вимагала більшої суми активної циркуляції. Банкноти, видані при
дисконті векселів, поверталися б назад через посередництво банків і шляхом
вкладів. Якщо такі операції не мають на меті вивіз золота або якщо всередині
країни не панує така паніка, що публіка міцно тримає в себе свої банкноти,
замість того, щоб платити ними банкам, то такі величезні обороти не зачепили б
резерву“. — „Банк може щодня дисконтувати на півтора мільйона, і це відбувається
постійно, ні трохи не зачіпаючи резервів банку. Банкноти повертаються
назад як вклади, і єдина зміна, яка при цьому відбувається, полягає в простому
перенесенні з одного рахунку на другий“ („Report on Bank Acts, 1857“. Свідчення
№ 241,500). Отже, банкноти служать тут тільки засобом перенесення кредитів.
}

\index{iii1}{0424}  %% посилання на сторінку оригінального видання
Насамперед ясно, що в першому з двох вищенаведених випадків, в періоди процвітання, коли маса
засобів циркуляції, що
перебувають у циркуляції, мусить зростати, попит на них зростає. Але так само ясно, що коли
фабрикант бере з свого вкладу
в банку більше золота або банкнот, тому що він має витратити
більше капіталу в грошовій формі, то з цієї причини його попит на капітал не зростає, а зростає
тільки його попит на цю
особливу форму, в якій він витрачає свій капітал. Попит стосується тільки до технічної форми, в якій
він кидає свій капітал у циркуляцію. Цілком так само, як, наприклад, при різному
розвитку кредиту, той самий змінний капітал, та сама кількість
заробітної плати, вимагає в одній країні більшої маси засобів
\index{iii1}{0425}  %% посилання на сторінку оригінального видання
циркуляції, ніж в іншій; в Англії, наприклад, більше, ніж у
Шотландії, в Німеччині більше, ніж в Англії. Так само в сільському
господарстві той самий капітал, який діє в процесі репродукції,
вимагає, для виконання своєї функції, в різні сезони
року різної кількості грошей.

Але протилежність у постановці Фуллартона невірна. Зовсім
не великий попит на позики, як він це каже, відрізняє періоди
застою від періодів процвітання, а та легкість, з якою цей попит
задовольняється в періоди процвітання, і та трудність, з якою
він задовольняється під час застою. Аджеж саме величезний
розвиток кредитної системи в періоди процвітання, отже й колосальне
підвищення попиту на позиковий капітал і готовість,
\parbreak{}  %% абзац продовжується на наступній сторінці
