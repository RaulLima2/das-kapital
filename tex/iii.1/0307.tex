ленні бухгалтери й касири, далеко проведений поділ праці).
Виплачування грошей, одержування грошей, складання балансів,
ведення поточних рахунків, зберігання грошей і т. д., виконувані
відокремлено від тих актів, в наслідок яких ці технічні операції
стають необхідними, роблять капітал, авансований на ці
функції, грошево-торговельним капіталом.

Різні операції, з усамостійнення яких в особливі підприємства
виникає торгівля грішми, випливають з різних визначеностей
самих грошей і з їх функцій, які доводиться таким чином виконувати
і капіталові у формі грошового капіталу.

Уже раніше я вказував, як гроші взагалі первісно розвиваються
при обміні продуктів між різними громадами.43

Тому торгівля грішми, торгівля грошовим товаром розвивається
насамперед з міжнародних зносин. Якщо в різних країнах
існують різні монети, то купцям, які роблять закупівлі в чужих
країнах, доводиться обмінювати монети своєї країни на місцеві
монети і навпаки, абож обмінювати різні монети на злитки чистого
срібла чи золота як на світові гроші. Звідси міняльна справа, яку
слід розглядати як одну з природно вирослих основ сучасної торгівлі
грішми.44 З міняльної справи розвинулись розмінні банки, де

43 Zur Kritik der politischen Oekonomie“, стор. 27 [„До критики політичної
економії“, укр. вид. 1935 р., стор. 74].

44 Уже з великої різноманітності монет як щодо ваги й проби, так і щодо
карбування їх багатьма князями й містами, які мають право карбування, виникла
повсюди необхідність користуватись місцевою монетою в торговельних справах,
коли треба було звести рахунки в якійсь одній монеті. Коли купці
виїжджали на іноземні ринки, вони для розплати готівкою запасалися злитками
чистого срібла, а також, звичайно, і золота. Цілком так само, від’їжджаючи
звідти, вони обмінювали одержані ними місцеві монети на злитки
срібла або золота, Тому міняльна справа, торгівля грішми, обмін злитків благородних
металів-на місцеві монети і навпаки, стала дуже поширеною і дохідною
справою“ (Hüllmann: Städtewesen des Mittelalters“. Bonn 1826—29,
І, стор. 437 [438]). — „De Wisselbank heeft hären naam niet... van den wissel, wisselbrief,
maar van wisselen van geldspecien. Lang vöör het oprigten der Amsterdarasche
wisselbank in 1609 had men in de Nederlandsche koopsteden reeds wisselaars
en wisselhuizen, zelfs wisselbanken... Het bedrijf dezer wissellaars bestond
daarin, dat zie de talrijke verscheidene muntspecien, die door vreemde handelaren
in het land gebragt worden, tegen wettelijk gangbare munt inwisselden. Langzamerhand
breidde hun werkkring zieh uit... zij werden de kassiers en bankiers van
hunne tijd. Maar in die vereeniging van de kassierderij met het wisselambt zach
de regering van Amsterdam gevaar, en om dit gevaar te keeren, werd besloten to
het stichten eener groote inrigting, die zoo wel het wisselen als de kassierderij op
openbaar gezag zou verrigten. Die inrigting was de beroemde Amsterdamsche Wisselbank
van 1609. Even zoo hebben de wisselbanken van Venetie, Genua, Stockholm,
Hamburg haar ontstaan aan de gedurlge noodzakelijkheid der verwisseling van
geldspecien te danken gehad. Van deze allen is de Hamburgsche de eenige die
nog heden bestaat, om dat de behoefte aan zulk eene inrigting zieh in deze
koopstad. die geen eigen muntstelsel heeft, nogaltijd doet gevoelen etc.“ [„Розмінний
банк дістав свою назву не... від векселя, вексельного листа, а від розміну
грошових знаків. Задовго до заснування Амстердамського розмінного банку в
1609 році в нідерландських торговельних містах були міняйли, міняльні контори,
навіть розмінні банки... Заняття цих міняйл полягало в тому, що вони
обмінювали різні численні сорти монет, які привозилися в країну іноземними
купцями, на ходячу законну монету. Помалу поле їх діяльності розширюва-
