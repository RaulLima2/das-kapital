\index{iii1}{0136}  %% посилання на сторінку оригінального видання
Ростущий попит на сировинний матеріал і переповнення
ринку фабрикатами ідуть, звичайно, рука в руку. — Між іншим,
тодішнє розширення промисловості і наступний застій не обмежились
бавовняними округами. У бредфордській камвольній
окрузі в 1836 році було тільки 318 фабрик, а в 1846 році —
490 фабрик. Ці цифри далеко не виражають дійсного росту
виробництва, бо разом з тим були дуже розширені наявні фабрики.
Особливо це стосується до льонопрядільних фабрик. „Всі
вони протягом останніх 10 років більш або менш сприяли переповненню
ринку, якому головним чином треба приписати теперішній
застій у справах... Пригнічений стан справ є цілком природним
наслідком такого швидкого збільшення числа фабрик
і машин“ („Rep. of Insp. of Fact., Oct. 1846“, стор. 30).

1847 рік. У жовтні грошова криза. Дисконт 8\%. Ще перед
тим стався крах залізничного грюндерства і спекуляції ост-індськими
векселями. Але:

„Пан Бекер наводить дуже цікаві деталі щодо підвищення
за останні роки попиту на бавовну, вовну й льон в наслідок
розширення цих галузей промисловості. Збільшений попит на ці
сировинні матеріали, особливо тому, що він настав у такий час,
коли подання їх упало далеко нижче пересічного, Бекер вважає
за майже достатній для пояснення сучасного пригніченого стану
цих галузей промисловості, навіть не беручи до уваги розладу
грошового ринку. Цей погляд цілком потверджується моїми
власними спостереженнями і тим, про що я довідався від обізнаних
людей. Ці різні галузі промисловості всі були в дуже
пригніченому стані вже тоді, коли дисконт легко можна було
робити з 5\% і менше. Навпаки, подання шовку-сирця було дуже
велике, ціни помірні, і тому справи йшли жваво до... останніх
2 або 3 тижнів, коли грошова криза, без сумніву, зачепила не
тільки самих промисловців, що переробляли шовк-сирець, але
ще більше їх головних замовників, фабрикантів модних товарів.
Досить тільки глянути на опубліковані офіціальні звіти, щоб побачити,
що бавовняна промисловість за останні три роки збільшилась
майже на 27\%. В наслідок цього бавовна підвищилась
у ціні, в круглих числах, з 4 пенсів до 6 пенсів за фунт, тимчасом
як пряжа, завдяки збільшеному поданню, стоїть тільки
трохи вище своєї попередньої ціни. Шерстяна промисловість почала
з 1836 року розширюватись; з того часу вона зросла в
Йоркшірі на 40\%, а в Шотландії ще більше. Ще більший ріст
у камвольній промисловості.\footnote{
В Англії строго розрізняють Woollen Manufacture [шерстяну промисловість],
яка тче і пряде кардну пряжу з короткорунної вовни (головний центр Лідс)
і Worsted Manufacture [камвольну промисловість], яка тче і пряде чесану пряжу
з довгорунної вовни (головний центр Бредфорд в Йоркшірі). — Ф. Е.
} Обчислення дають тут за той
самий період часу розширення більше ніж на 74\%. Тому споживання
сирової вовни було колосальне. Лляна промисловість виявляє
від 1839 року приріст приблизно в 25\% для Англії, в
\parbreak{}  %% абзац продовжується на наступній сторінці
