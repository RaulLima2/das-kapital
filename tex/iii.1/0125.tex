ровинний матеріал знову постачається великими масами, як от у землеробстві після жнив, — тим
виразніше виступає вплив зміни цін сировинного матеріалу.

В усьому нашому дослідженні ми виходимо з того припущення, що підвищення або зниження цін є вираз
дійсних коливань вартості. Але через те що тут мова йде про той вплив, який ці коливання цін
справляють на норму зиску, то в дійсності не має значення, яка є причина цих коливань; отже,
розвинуте тут має силу також і тоді, коли ціни підвищуються і падають не в наслідок коливань
вартості, а в наслідок діяння системи кредиту, конкуренції і т. д.

Через те що норма зиску дорівнює відношенню надлишку вартості продукту до вартості всього
авансованого капіталу, то підвищення норми зиску, що походить із зниження вартості авансованого
капіталу, може бути сполучене з втратою капітальної вартості; так само зниження норми зиску, що
походить з підвищення вартості авансованого капіталу, може бути сполучене з виграшем.

Щодо другої частини сталого капіталу, машин і взагалі основного капіталу, то підвищення вартості,
які тут відбуваються і стосуються саме до будівель, землі і т. д., не можуть бути розглянуті до
викладу вчення про земельну ренту і тому, вони не належать сюди. Але для зниження вартості цієї
частини капіталу загальне значення мають:

1. Постійні поліпшення, які позбавляють наявні машини, фабричне устаткування і т. д. частини їх
споживної вартості,
а тому і їх вартості. Цей процес діє з особливою силою в перший період введення нових машин, раніше
ніж вони досягають певної міри зрілості, і коли вони через це постійно стають застарілими раніше,
ніж встигають репродукувати свою вартість. Це одна з причин звичайного в такі епохи безмірного
здовження робочого часу, праці вдень і вночі почережно змінами, для того, щоб протягом коротшого
часу репродукувати вартість машин, не відраховуючи при цьому занадто багато на їх зношування. Якщо
ж, навпаки, короткий період діяльності
машин (короткий строк їх життя в зв’язку з можливими поліпшеннями) не буде таким способом
скомпенсовано, то внаслідок їх морального зношування вони переносять на продукт занадто велику
частину своєї вартості, так що не можуть конкурувати навіть з ручною працею.

Якщо машини, устаткування будівель, взагалі основний капітал досяг певної зрілості, так що протягом
довшого часу
він, принаймні в своїй основній конструкції, лишається незмінним, то подібне ж зниження вартості
настає в наслідок поліп-
