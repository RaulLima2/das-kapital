чайно, можна нагромаджувати той засіб або знаряддя, за допомогою якого панують над капіталом“. —
„3755. Отже, ви гадаєте,
що не існує ніякого зв’язку між нормою дисконту і кількістю
золота у сховищах банку? — Зв’язок може існувати, але це не
принциповий зв’язок“; [проте, його банковий акт 1844 року
зводить саме в принцип Англійського банку регулювання розміру процента залежно від кількості золота,
яким володіє банк]
„вони можуть збігатися в часі (there may be a coincidence
of time)“. — „3758. Отже, ви хочете сказати, що труднощі для купців тут у країні, в періоди
недостачі грошей, в наслідок високої
норми дисконту, полягають у тому, щоб одержати капітал, а не
в тому, щоб одержати гроші? — Ви змішуєте дві речі, які я в
цій формі не об’єдную; трудність полягає в тому, щоб одержати капітал, і так само трудно одержати
гроші... Трудність одержати гроші і трудність одержати капітал — це
та сама трудність, розглядувана на двох різних ступенях
її розвитку“. — Тут рибка знову кріпко впіймалась. Перша
трудність — це дисконтувати вексель або одержати позику під
заставу товарів. Трудність ця полягає в тому, щоб перетворити
в гроші капітал або торговельний знак вартості капіталу. І ця
трудність виражається, між іншим, у високому розмірі процента.
Але раз гроші одержано, то в чому тоді полягає друга трудність? Якщо справа йде тільки про платіж,
то хіба хто-небудь
зустрічає трудність у тому, щоб позбутися своїх грошей? А якщо справа йде про купівлю, то хіба
хто-небудь у період кризи
зустрічає трудність купити товар? І якщо навіть припустити,
що це стосується окремого випадку подорожчання хліба, бавовни і т. д., то ж ця трудність могла б
виявлятись не
в вартості грошового капіталу, тобто не в розмірі процента, а
тільки в ціні товару; і ця трудність переборюється ж тим,
що в нашого ділка тепер є гроші для купівлі товару.

„3760. Але ж вища норма дисконту не збільшує трудності
одержати гроші? — Вона збільшує трудність одержати гроші, але
справа йде не про володіння грішми; гроші тільки форма“ [і ця
форма дає зиск у кишеню банкіра], „в якій виражається ця збільшена трудність одержати капітал у
складних відносинах цивілізованого порядку“.

„3763. [Відповідь Оверстона:] Банкір є посередник, який,
з одного боку, одержує вклади, а з другого боку, застосовує
ці вклади, довіряючи їх у формі капіталу в руки осіб, які
і т. д.“.

Тут ми бачимо, нарешті, що він розуміє під капіталом. Він
перетворює гроші в капітал, „довіряючи їх“ або, висловлюючись
менш евфемістично, віддаючи їх в позику за проценти.

Сказавши раніше, що зміна норми дисконту по суті не стоїть
у зв’язку із зміною суми золотого запасу банку або кількості
наявних грошей, а щонайбільше збігається з нею в часі, пан
Оверстон повторює:
