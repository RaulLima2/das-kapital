робництва), покупцем так само не реалізується ніякої додаткової
вартості, а тільки підготовляється, за допомогою обміну грошей
на засоби виробництва і робочу силу, виробництво додаткової
вартості. Навпаки. Оскільки ці метаморфози вимагають певного
часу циркуляції — часу, протягом якого капітал взагалі не виробляє
і, отже, не виробляє й додаткової вартості, — цей час обмежує
творення вартості, і додаткова вартість, виражена як норма
зиску, стоятиме саме в зворотному відношенні до тривалості
часу циркуляції. Тому купецький капітал не створює ні вартості,
ні додаткової вартості, тобто не створює безпосередньо. Оскільки
він сприяє скороченню часу циркуляції, він посередньо може
допомагати збільшенню додаткової вартості, вироблюваної промисловим
капіталістом. Оскільки він допомагає розширювати
ринок і опосереднює поділ праці між капіталістами, отже, дає
капіталові змогу працювати в більшому масштабі, його функція
сприяє підвищенню продуктивності промислового капіталу і його
нагромадженню. Оскільки він скорочує час обігу, він підвищує
відношення додаткової вартості до авансованого капіталу, отже,
норму зиску. Оскільки він скорочує ту частину капіталу, яка
мусить постійно перебувати в сфері циркуляції як грошовий
капітал, він збільшує частину капіталу, застосовувану безпосередньо
на виробництво.

Розділ сімнадцятий
Торговельний зиск

В книзі II ми бачили, що чисті функції капіталу в сфері циркуляції
— операції, які мусить виконати промисловий капіталіст
для того, щоб, поперше, реалізувати вартість своїх товарів і,
подруге, знову перетворити цю вартість в елементи виробництва
товару, операції для опосереднення метаморфоз товарного
капіталу Т' — Г — Т, отже, акти продажу й купівлі, — що вони не
створюють ні вартості, ні додаткової вартості. Навпаки, виявилось,
що потрібний для цього час, об’єктивно — щодо товарів і суб’єктивно
— щодо капіталістів, ставить межі творенню вартості і
додаткової вартості. Те, що має силу для метаморфози товарного
капіталу самого по собі, ніяк не змінюється, звичайно, від
того, що частина цього капіталу набирає форми товарно-торговельного
капіталу, або що операції, якими опосереднюється метаморфоза
товарного капіталу, виступають як особливе заняття
особливого підрозділу капіталістів або як виключна функція частини
грошового капіталу. Якщо продаж і купівля товарів — а до
цього зводиться метаморфоза товарного капіталу Т' — Г — Т —
самими промисловими капіталістами є операції, які не створюють
ніякої вартості або додаткової вартості, то вони ні в якому разі
не можуть набути властивості створювати її від того, що вони виконуватимуться
не промисловими капіталістами, а іншими особами.
