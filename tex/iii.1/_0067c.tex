\index{iii1}{0067}  %% посилання на сторінку оригінального видання
Ми зберігаємо позначення, вжиті в першій і другій книгах.
Весь капітал К поділяється на сталий капітал с і змінний капітал
v і виробляє додаткову вартість m. Відношення цієї додаткової
вартості до авансованого змінного капіталу, отже m/v, ми
називаємо нормою додаткової вартості і позначаємо її через m'.
Отже, m/v = m', і тому m = m'v. Якщо цю додаткову вартість
віднести не до змінного капіталу, а до всього капіталу, то вона
зветься зиском (р), а відношення додаткової вартості m до
всього капіталу К, отже m/К, зветься нормою зиску р'. Звідси ми
маємо:
р' = m/К = m / (с + v'),
якщо ми замість m підставимо його знайдену вище величину
m'v, то матимем:
р' = m' (v/К) = m' (v / (c + v') )
рівняння, яке можна виразити також у пропорції:
р’: m' = v: К,
норма зиску відноситься до норми додаткової вартості, як змінний
капітал до всього капіталу.

З цієї пропорції випливає, що р', норма зиску, завжди менша
від m', норми додаткової вартості, бо v, змінний капітал, завжди
менший від К, суми v + c, змінного і сталого капіталу; за винятком
єдиного, практично неможливого випадку, коли v = K, отже,
коли капіталістом зовсім не авансувався б сталий капітал, засоби
виробництва, а тільки заробітна плата.

В нашому дослідженні треба, однак, звернути увагу ще на
ряд інших факторів, які визначально впливають на величину с,
v і m і тому мають бути коротко згадані.

Поперше, вартість грошей. Її ми можемо повсюди приймати
за сталу.

Подруге, оборот. Цей фактор ми покищо лишаємо осторонь,
бо його вплив на норму зиску розглядається окремо в одному
з дальших розділів. [Тут ми, забігаючи наперед, згадаємо тільки
про той один пункт, що формула р' = m' (v/К) є строго правильна
лиш для одного періоду обороту змінного капіталу, і що ми,
однак, можемо її зробити правильною для річного обороту, поставивши
замість m', простої норми додаткової вартості, m'n,
річну норму додаткової вартості; при чому n є число оборотів
змінного капіталу протягом одного року (див. книгу II, розд.
XVI, 1) — Ф. Е]
