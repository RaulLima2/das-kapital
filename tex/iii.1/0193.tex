робництва може бути продана по її ринковій вартості, — ні вище,
ні нижче. Ось перше, що нам кажуть.

Подруге: якщо товари можуть бути продані по їх ринковій
вартості, то попит і подання покриваються.

Якщо попит і подання взаємно покриваються, то вони перестають
діяти, і саме тому товари продаються по їх ринковій вартості.
Якщо дві сили рівномірно діють у протилежних напрямах,
то вони одна одну знищують, зовсім не діють назовні, і явища,
які відбуваються при цій умові, мусять бути пояснені якось
інакше, а не діянням цих двох сил. Якщо попит і подання взаємно
знищуються, то вони перестають щонебудь пояснювати,
не діють на ринкову вартість і залишають нас у цілковитому
невіданні того, чому ринкова вартість виражається саме в цій
сумі грошей, а не в будьякій іншій. Дійсні внутрішні закони
капіталістичного виробництва, очевидно, не можуть бути пояснені
з взаємодіяння попиту й подання (цілком незалежно від
глибшого аналізу цих двох суспільних рушійних сил, який сюди
не стосується), бо ці закони тільки тоді виявляються здійсненими
в чистому вигляді, коли попит і подання перестають
діяти, тобто взаємно покриваються. В дійсності попит і подання
ніколи не покриваються або, якщо і покриваються, то тільки
випадково, — отже, з наукового погляду такі випадки слід прирівняти
до нуля і розглядати як неіснуючі. Але в політичній
економії припускається, що вони покриваються. Чому? Це робиться
для того, щоб розглядати явища в їх закономірному, відповідному
їх поняттю вигляді, тобто розглядати їх незалежно від
того, якими вони здаються в наслідок руху попиту й подання.
З другого боку, для того, щоб знайти дійсну тенденцію їх руху, так
би мовити, фіксувати її. Бо відхилення від рівності мають протилежний
характер і, через те що вони завжди йдуть одне за одним,
вони урівноважуються завдяки своїм протилежним напрямам, завдяки
своїй суперечності. Отже, якщо попит і подання не покриваються
ні в одному випадку, то їх відхилення від рівності йдуть одне
за одним таким чином, — результат відхилення в одному напрямі
є той, що воно викликає відхилення в протилежному напрямі, —
що, коли розглядати підсумок руху за більш-менш довгий період
часу, подання і попит постійно покриваються; однак, вони покриваються
тільки як пересічне минулих уже коливань, тільки як
постійний рух їх суперечності. В наслідок цього ринкові ціни,
що відхиляються від ринкових вартостей, розглядувані щодо
їх пересічної, вирівнюються в ринкові вартості, при чому відхилення
від цих останніх взаємно знищуються як плюс і мінус.
І ця пересічна має не тільки теоретичне значення, вона має
також і практичну важливість для капіталу, вкладення якого розраховане
на коливання й вирівнювання протягом більш-менш певного
періоду часу.

Тому відношення між попитом і поданням пояснює, з одного
боку, тільки відхилення ринкових цін від ринкових вартостей
