син виробництва: форму, що дає процент, просту форму капіталу,
в якій він є передумовою свого власного процесу репродукції;
ми маємо перед собою здатність грошей, відповідно —
товару, збільшувати свою власну вартість незалежно від репродукції,
— містифікацію капіталу в найрізкішій формі.

Для вульгарної економії, яка хоче зобразити капітал як самостійне
джерело вартості, джерело вартостетворення, ця форма,
є, звичайно, знахідкою, формою, в якій уже неможливо пізнати
джерела зиску і в якій результат капіталістичного процесу
виробництва — відокремлений від самого процесу — набуває самостійного
буття.

Лиш у формі грошового капіталу капітал став товаром,
властивість якого самозростати в своїй вартості має певну ціну,
яка кожного разу позначається розміром процента.

Як капітал, що дає процент, і особливо в своїй безпосередній
формі як грошовий капітал, що дає процент (інші форми капіталу,
що дає процент, які нас тут не цікавлять, виводяться
з цієї ж форми і передбачають її), капітал набуває своєї чистої
фетишистичної форми: Г — Г' як суб’єкт, як річ, що може бути
продана. Поперше, це відбувається в наслідок його постійного
буття у формі грошей, формі, в якій усі його визначеності стерті
і в якій його реальні елементи невидимі. Гроші — це якраз форма,
в якій стерта ріжниця між товарами як споживними вартостями,
а тому й ріжниця між промисловими капіталами, які складаються з
цих товарів і умов їх виробництва; гроші — це та форма, в якій
вартість — а тут капітал — існує як самостійна мінова вартість.
В процесі репродукції капіталу грошова форма є минуща форма,
простий перехідний момент. Навпаки, на грошовому ринку капітал
завжди існує в цій формі. — Подруге, породжена ним додаткова
вартість, тут знов таки в формі грошей, здається належною
йому як такому. Подібно до того, як ріст є властивий
деревам, так і породження грошей (τοχος) здається властивим
капіталові в цій його формі грошового капіталу.

В капіталі, що дає процент, рух капіталу скорочений до
крайності; опосереднюючий процес тут випущений і таким чином
капітал — 1000 фіксується як річ, яка сама по собі = 1000 і за певний
період перетворюється в 1100, подібно до того, як поліпшує
свою споживну вартість вино, що перебувало певний час у льоху.
Капітал є тепер річ, але саме як річ він є капітал. Гроші тепер
вагітні грішми. Якщо вони віддані в позику або вкладені в процес
репродукції (оскільки вони дають функціонуючому капіталістові
як своєму власникові процент, крім підприємницького доходу), то
на них і день і ніч наростає процент, однаково, чи сплять вони,
чи пильнують, сидять дома чи подорожують. Таким чином у
грошовому капіталі, що дає процент (а всякий капітал є щодо виразу
своєї вартості грошовий капітал або вважається тепер
за вираз грошового капіталу), реалізується побожне бажання збирача
скарбів
