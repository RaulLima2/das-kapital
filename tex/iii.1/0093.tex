чається ще й те, що вартість основного капіталу таким чином репродукується
за коротший ряд періодів обороту, отже, скорочується
час, на який він мусить бути авансований, щоб одержати
певний зиск. Тому здовження робочого дня збільшує зиск навіть
тоді, коли надурочний час оплачується, а до певної міри навіть
тоді, коли він оплачується вище, ніж нормальні робочі години.
Тому постійно зростаюча при сучасній промисловій системі необхідність
збільшення основного капіталу була для ненаситно
жадливих до зиску капіталістів головним стимулом до здовження
робочого дня.11

Інші умови маємо при сталому робочому дні. В цьому випадку
для того, щоб експлуатувати більшу масу праці, треба
або збільшити число робітників, і разом з тим до певної міри
масу основного капіталу, будівель, машин і т. д. (бо ми тут
залишаємо осторонь відрахування з заробітної плати або зниження
заробітної плати нижче її нормальної висоти). Абож, якщо
збільшується інтенсивність праці чи підвищується продуктивність
праці, якщо взагалі виробляється більше відносної додаткової
вартості, то в тих галузях промисловості, які застосовують
сировинний матеріал, зростає маса обігової частини
сталого капіталу, бо за даний період часу переробляється
більше сировинного матеріалу і т. д.; і, подруге, зростає кількість
машин, які приводяться в рух тим самим числом робітників,
отже, і відповідна частина сталого капіталу. Зростання додаткової
вартості супроводиться, отже, зростанням сталого капіталу, зростаюча
експлуатація праці — подорожчанням тих умов виробництва,
за допомогою яких експлуатується праця, тобто більшими
витратами капіталу. Отже, через це норма зиску з одного
боку зменшується, тимчасом як з другого боку вона підвищується.
Цілий ряд поточних затрат лишається майже або цілком
однаковий як при довшому, так і при коротшому робочому дні.
Витрати нагляду менші при 500 робітниках і 18-годинному робочому
дні, ніж при 750 робітниках і 12-годинному робочому дні.
„Витрати ведення фабрики при десятигодинній праці майже однаково
високі, як і при дванадцятигодинній“ („Rep. of Insp. of
Fact., Oct. 1848“, стор. 37). Державні та комунальні податки,
страхування від огню, заробітна плата різних постійних службовців,
зневартнення машин і різні інші затрати фабрики лишаються
незмінними при довгому чи короткому робочому дні;
в міру того, як скорочується виробництво, вони підвищуються
коштом зиску („Rep. of Insp. of Fact., Oct. 1862“, стор. 19).

Період часу, протягом якого репродукується вартість машин
і інших складових частин основного капіталу, на практиці визна-

11 „Через те що на всіх фабриках дуже висока сума основного капіталу
вкладена в будівлі і машини, зиск буде тим більший, чим більше число годин,
протягом яких ці машини можуть бути в роботі“ („Rep. of Insp. of Fact., 31.
October 1858“, стор. 8).
