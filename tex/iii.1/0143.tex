прийняли до школи, і скаржились при цьому, що вони не могли
заробити й 1 шилінга на тиждень. Я мав відомості про self-acting
minders [прядільників на автоматичних прядільних машинах]...
Чоловіки, що керували парою автоматів, заробили за 14 днів
повного робочого часу 8 шилінгів 11 пенсів, і з цієї суми у них відрахували
плату за наймання житла, при чому, однак, фабрикант“
[о великодушність!] „повернув їм половину „цієї плати як
подарунок. Прядільники принесли додому по 6 шилінгів 11 пенсів.
В багатьох місцях протягом останніх місяців 1862 року selfacting
minders заробляли 5—9 шилінгів на тиждень, ткачі 2—6 шилінгів
на тиждень... В даний момент стан справ багато нормальніший,
хоча в більшості округ заробіток все ще дуже знижений...
Поряд з коротким волокном індійської бавовни та її забрудненістю
багато інших причин сприяли зниженню заробітку. Так,
наприклад, тепер стало звичаєм примішувати багато бавовняних
відпадів до індійської бавовни, і це, звичайно, ще дужче збільшує
труднощі для прядільника. При короткому волокні нитки
легше рвуться при витяганні мюлів і намотуванні пряжі, і тому
не можна з такою регулярністю підтримувати рух мюлів... Так
само при тій великій увазі, яку доводиться приділяти, слідкуючи
за нитками, одна ткаля може наглядати часто лиш за одним верстатом
і тільки дуже небагато з них можуть наглядати більше,
ніж за двома верстатами... В багатьох випадках заробітну плату
робітників прямо знижено на 5, 7 1/2 і 10%... в більшості випадків
робітникові доводиться дбати про те, як упоратися з своїм сировинним
матеріалом і добитися звичайного розміру заробітку, як
зможе. Інша трудність, з якою іноді доводиться боротися ткачам,
полягає в тому, що вони повинні з поганого матеріалу робити
добру тканину і штрафуються відрахуваннями з заробітної
плати, якщо їх робота не дає бажаних наслідків“ („Rep. of Insp.
of Fact., Oct. 1863“, стор. 41—43).

Заробітна плата була мізерна навіть там, де працювали повний
час. Робітники бавовняної промисловості брались охоче до всяких
громадських робіт, на яких їх використовували, — дренаж, прокладання
шляхів, розбивання каменю, брукування вулиць, — щоб
одержати від місцевих властей допомогу (яка фактично була допомогою
фабрикантам, див. книгу I, стор. 603*). Вся буржуазія
пильно стежила за робітниками. Коли робітникові пропонувався
найгірший, наймізерніший заробіток і він відмовлявся від нього,
то комітет допомоги викреслював його з списків на допомогу. То
був золотий час для панів фабрикантів, бо робітникам доводилось
або вмирати з голоду, або працювати за всяку найзисковнішу
для буржуа ціну, при чому комітети допомоги діяли як
їх вартові пси. Разом з тим фабриканти, в таємному порозумінні
з урядом, якомога перешкоджали еміграції, почасти для
того, щоб тримати напоготові свій існуючий в тілі й крові робіт-

* Стор. 451 рос. вид. 1935 р. Ред. укр. перекладу.
