раховує, отже, відшкодовує себе тим, що робить надбавку до
ціни. Абож, коли капіталовкладення, яким загрожують дуже
великі небезпеки, як, наприклад, у мореплавстві, одержують відшкодування
шляхом надбавки до ціни. Як тільки капіталістичне
виробництво, а разом з ним і страхова справа досягають певного
ступеня розвитку, небезпека фактично стає однаковою для
всіх сфер виробництва (див. Корбет); але підприємства, яким
найбільше загрожує небезпека, платять вищу страхову премію
і відшкодовують себе за це в ціні своїх товарів. На практиці
все це зводиться до того, що кожна обставина, яка робить
певне капіталовкладення менш зисковним, а друге більш зисковним,
— а всі вони в певних межах вважаються однаково необхідними,
— включається в обрахунок як раз назавжди встановлена
підстава для компенсації, при чому вже немає потреби в новій
і новій діяльності конкуренції, щоб виправдати такий мотив або
фактор обрахунку. Капіталіст забуває тільки, — або, скоріше,
не бачить, бо конкуренція йому цього не показує, — що всі ці
підстави для компенсації, які капіталісти висувають один проти
одного у взаємному обчисленні товарних цін різних галузей
виробництва, базуються просто на тому, що всі капіталісти мають
pro rata [пропорціонально] їх капіталові однакові домагання
щодо спільної здобичі, сукупної додаткової вартості. Навпаки,
через те що одержаний ними зиск відрізняється від видушеної
ними додаткової вартості, їм здається, що їх підстави для
компенсації не вирівнюють їх участі в сукупній додатковій
вартості, а створюють самий зиск, бо цей останній, мовляв,
виникає просто з так чи інакше мотивованої надбавки до витрат
виробництва товарів.

У всьому іншому і для пересічного зиску має силу те, що
було сказано в розділі VII, стор. 148, про уявлення капіталіста
щодо джерела додаткової вартості. Тут справа стоїть інакше
лиш остільки, оскільки при даній ринковій ціні товарів і даній
експлуатації праці заощадження на витратах виробництва залежить
від індивідуальної вправності, уважності і т. д.
