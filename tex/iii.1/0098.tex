винні і допоміжні матеріали. Від добротності сировинного матеріалу
почасти залежить норма зиску. Добрий матеріал дає менше
відпадів; отже, він вимагає меншої маси сировинного матеріалу
для вбирання в себе тієї самої кількості праці. Далі, опір, що
його зустрічає робоча машина, тут менший. Почасти це впливає
навіть на додаткову вартість і норму додаткової вартості. Робітникові
при поганому сировинному матеріалі потрібно більше
часу, щоб переробити ту саму кількість цього матеріалу; при
незмінній заробітній платі це веде до скорочення додаткової
праці. Далі, це дуже значно впливає на репродукцію і нагромадження
капіталу, що, як показано в першій книзі, стор. 634
і далі*, ще більше залежать від продуктивності, ніж від маси вживаної
праці.

Тому зрозумілий фанатизм капіталіста в справі економізування
засобів виробництва. Щоб ніщо не пропадало і не марнотратилось,
щоб засоби виробництва споживались тільки таким
способом, як цього потребує само виробництво, — це почасти
залежить від муштри і навченості робітників, почасти від дисципліни,
в якій капіталіст тримає комбінованих робітників і яка стає
зайвою при такому суспільному ладі, де робітники працюють
власним коштом, як вона вже тепер стає майже цілком зайвою
при відштучній платі. Цей фанатизм виявляється, з другого боку,
і в фальсифікації елементів виробництва, яка є головний засіб
знизити вартість сталого капіталу порівняно з змінним і таким
чином підвищити норму зиску; при чому до цього долучається
ще, як істотний елемент шахрайства, продаж цих елементів
виробництва вище їх вартості, оскільки ця вартість знову з’являється
в продукті. Цей момент грає вирішальну роль, особливо
в німецькій промисловості, основний принцип якої такий: людям
може бути тільки приємно, коли ми спочатку надішлемо їм добрі
зразки, а потім погані товари. А втім, ці явища, які належать до
сфери конкуренції, нас тут не цікавлять.

Треба зауважити, що таке підвищення норми зиску, викликане
зменшенням вартості, отже, і дорожнечі сталого капіталу,
зовсім не залежить від того, чи виробляє та галузь промисловості,
в якій відбувається це підвищення, предмети розкоші,
чи засоби існування, які входять у споживання робітників, чи
засоби виробництва взагалі. Остання обставина може мати значення
лиш остільки, оскільки йдеться про норму додаткової
вартості, яка істотно залежить від вартості робочої сили, тобто
від вартості звичайних засобів існування робітника. Тут, навпаки,
додаткова вартість і норма додаткової вартості припускаються
даними. Як додаткова вартість відноситься до всього
капіталу — а це визначає норму зиску — залежить, при цих
умовах, виключно від вартості сталого капіталу, але ніяк не
від споживної вартості тих елементів, з яких він складається.

Відносне здешевлення засобів виробництва, звичайно, не ви-

* Стор. 476 і далі рос. вид. 1935 р. Ред. укр. перекладу.
