\parcont{}  %% абзац починається на попередній сторінці
\index{iii1}{0248}  %% посилання на сторінку оригінального видання
пересічного робочого часу, суспільно-необхідного для виробництва
товарів. І одночасно зростає концентрація, бо за певними
межами великий капітал з невеликою нормою зиску нагромаджує
швидше, ніж невеликий капітал з великою нормою зиску. Ця
зростаюча концентрація, з свого боку, досягнувши певної висоти,
знов таки приводить до нового падіння норми зиску. Маса дрібних
розпорошених капіталів у наслідок цього штовхається на шлях
авантюр: спекуляцій, шахрайських кредитних і акційних підприємств,
криз. Так звана плетора [наддостаток] капіталу завжди
стосується головним чином до плетори такого капіталу, для якого
падіння норми зиску не урівноважується масою зиску, — а такі
завжди є новоутворювані свіжі паростки капіталу, — або до плетори
таких капіталів, які, будучи самі по собі нездатними самостійно
функціонувати, передаються в формі кредиту в розпорядження
керівників великих галузей підприємств. Ця плетора капіталу виростає
з тих самих обставин, які викликають відносне перенаселення,
і тому вона є явище, яке доповнює це останнє, хоч обоє
вони перебувають на протилежних полюсах: на одному боці — незанятий
капітал, на другому боці — незаняте робітниче населення.

Перепродукція капіталу, а не окремих товарів, — хоч перепродукція
капіталу завжди включає перепродукцію товарів, —
означає через це не що інше, як перенагромадження капіталу.
Щоб зрозуміти, що таке є це перенагромадження (докладніше
дослідження його ми подаємо нижче), досить тільки припустити
його абсолютним. Коли перепродукція капіталу була б абсолютною?
І при тому перепродукція, яка поширювалася б не на ту
чи іншу або декілька значних сфер виробництва, а була б абсолютною
в самому своєму об’ємі, отже, охоплювала б усі сфери
виробництва?

Абсолютна перепродукція капіталу була б у наявності в тому
випадку, коли додатковий капітал для цілей капіталістичного
виробництва був би = 0. Але метою капіталістичного виробництва
є збільшення вартості капіталу, тобто привласнення додаткової
праці, виробництво додаткової вартості, зиску. Отже,
коли б капітал зріс порівняно з робітничим населенням настільки,
що не можна було б ні здовжити абсолютний робочий час, що
його дає це населення, ні розширити відносний додатковий робочий
час (останнє, крім того, було б нездійсниме при таких обставинах,
коли попит на працю є такий значний, отже, коли є тенденція
до підвищення заробітної плати), тобто коли б зрослий
капітал виробляв тільки таку саму або навіть меншу масу додаткової
вартості, ніж до свого зростання, то мала б місце абсолютна
перепродукція капіталу; тобто зрослий капітал К + ΔК виробляв
би не більше зиску, або навіть менше зиску, ніж капітал К
до свого збільшення на ΔК. В обох випадках мало б також
місце значне і раптове падіння загальної норми зиску, але на
цей раз в наслідок переміни у складі капіталу, викликаної не розвитком
продуктивної сили, а підвищенням грошової вартості
\parbreak{}  %% абзац продовжується на наступній сторінці
