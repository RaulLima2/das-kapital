\parcont{}  %% абзац починається на попередній сторінці
\index{iii1}{0110}  %% посилання на сторінку оригінального видання
охорони здоров’я є універсальна. І в інтересах мільйонів робітників і робітниць, життя яких тепер
без потреби калічиться
і скорочується безконечними фізичними стражданнями, породжуваними тільки їх працею, я зважуюсь
висловити надію, що
санітарні умови праці так само повсюди будуть поставлені під
належний захист закону, принаймні настільки, щоб була забезпечена справжня вентиляція в усіх
закритих робочих приміщеннях і щоб у кожній галузі праці, по своїй природі шкідливій для здоров’я,
по можливості був обмежений особливо
небезпечний для здоров’я вплив“ (стор. 31).

III. Економія на здобуванні сили, на передачі сили і на будівлях

У своєму жовтневому звіті за 1852 рік Л. Горнер цитує лист
відомого інженера Джемса Насміта з Патрікрофта, винахідника
парового молота, де, між іншим, сказано:

„Публіка дуже мало знайома з колосальним приростом рушійної сили, досягнутим за допомогою таких змін
системи й поліпшень“ [у парових машинах], „як ті, про які я кажу. Сила машин нашої округи
(Ланкашіру) перебувала протягом майже 40 років
під гнітом боязкої і повної передсудів рутини, але, на щастя,
ми тепер від неї емансипувались. Протягом останніх 15 років,
особливо ж на протязі останніх 4 років“ [отже, з 1848 р.], „стались деякі дуже важливі зміни в
способі використовування конденсаційних парових машин... В результаті... ті самі машини виконують
далеко більше роботи, і до того ж при значно зменшеному споживанні вугілля... Протягом дуже багатьох
років
з часу введення парової сили на фабриках цієї округи гадали,
що швидкість, з якою можуть працювати конденсаційні парові
машини, становить приблизно 220 футів підіймання поршня на
хвилину; тобто машина з підійманням поршня в 5 футів була вже
за встановленою нормою обмежена 22 оборотами ексцентрика на
хвилину. Вважалось за недоцільне гнати машину швидше; а через
те що весь механізм був пристосований до цієї швидкості руху
поршня в 220 футів на хвилину, ця мала і безглуздо обмежена
швидкість панувала в усій промисловості протягом багатьох років.
Але, нарешті, чи в наслідок щасливого незнання встановленої норми
чи з інших, кращих причин, якийсь сміливий новатор спробував
більшу швидкість, і тому що результат був надзвичайно сприятливий, інші наслідували цей приклад;
машині, як тоді казали, попустили віжки і змінили головні колеса передатного механізму таким
чином, що парова машина могла робити 300 футів і більше на
хвилину, в той час як механізми зберігали свою колишню швидкість... Це прискорення руху парової
машини стало тепер майже
загальним, бо виявилось, що не тільки з тієї самої машини здобувалося більше корисної сили, але що й
рух в наслідок більшої
\index{iii1}{0111}  %% посилання на сторінку оригінального видання
інерції маховика став значно регулярнішим. При незмінному
тисненні пари і незмінному розрідженні в конденсаторі одержується більше сили внаслідок простого
прискорення руху
поршня. Коли б ми могли, наприклад, парову машину, яка при
швидкості в 200 футів на хвилину дає 40 кінських сил, відповідними змінами привести до того, щоб
вона при тому самому тисненні
пари й розрідженні робила 400 футів на хвилину, то ми мали б
якраз подвійну кількість сили; а через те що тиснення пари
і розрідження в обох випадках однакові, то напруження окремих
частин машини, отже й небезпека нещасних випадків при прискореній швидкості не збільшується в
будь-якій значній мірі. Вся ріжниця в тому, що споживатиметься більше пари в тій самій або
приблизно в тій самій пропорції, в якій прискорюється рух поршня;
далі, трохи швидше зношуватимуться частини, які зазнають
тертя, але про це ледве чи варто говорити... Але для того, щоб
від тієї самої машини за допомогою прискореного руху поршня
добути більше сили, треба спалити під тим самим паровим казаном більше вугілля або вживати казан з
більшою здатністю паротворення, коротко — треба виробляти більше пари. Це і було
зроблено, і казани з більшою здатністю паротворення були пристосовані до старих „прискорених“ машин;
таким чином ці машини в багатьох випадках давали на 100\% більше роботи. Коло
1842 року надзвичайно дешеве вироблення сили паровими машинами на копальнях Корнуоля почало
викликати до себе увагу;
конкуренція в бавовнопрядільній промисловості примушувала
фабрикантів шукати головне джерело свого зиску в заощадженнях; надзвичайна ріжниця у споживанні
вугілля за одну годину
і на одну кінську силу, яку показували корнуольські машини,
а також надзвичайна економія при застосуванні вульфових машин
з подвійними циліндрами, і в нашій місцевості висунули на перший
план питання про економію паливного матеріалу. Корнуольські
машини і машини з подвійними циліндрами постачали одну кінську
силу за годину при споживанні З 1/2 до 4 фунтів вугілля, тим часом як машини в бавовнопрядільних
округах споживали звичайно 8 або 12 фунтів вугілля на 1 кінську силу за годину.
Така значна ріжниця спонукала фабрикантів і машинобудівників
нашої округи добитись за допомогою аналогічних засобів таких же
виняткових результатів у справі економії, які в Корнуолі і у Франції стали вже звичайними, бо там
висока ціна на вугілля примушувала фабрикантів якомога більше обмежувати затрати на цю дорогу статтю
їх підприємств. Це привело до дуже важливих результатів. По-перше: багато казанів, в яких половина
поверхні в добрі
старі часи високих зисків лишалась під впливом холодного зовнішнього повітря, тепер покрили товстою
повстю або цеглою і штукатуркою та іншими матеріалами, чим запобігали утраті тепла,
добутого з такими великими витратами. Таким самим способом
почали захищати парові труби і так само обгортати повстю й деревом циліндри. По-друге, почали
вживати високе тиснення,
\parbreak{}  %% абзац продовжується на наступній сторінці
