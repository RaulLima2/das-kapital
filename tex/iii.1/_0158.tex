\parcont{}  %% абзац починається на попередній сторінці
\index{iii1}{0158}  %% посилання на сторінку оригінального видання
цілком інше джерело, а не працю, і разом з тим відпала б
усяка раціональна основа політичної економії. Якщо ми, як і раніш, припустимо, що 1 фунт стерлінгів
становить тижневу заробітну плату одного робітника за 60 робочих годин і що норма
додаткової вартості = 100\%, то ясно, що вся нововироблена
вартість, яку один робітник може дати за один тиждень, — 2 фунтам стерлінгів; отже, 10 робітників не
могли б дати більше як
20 фунтів стерлінгів; а через те що з цих 20 фунтів стерлінгів
10 фунтів стерлінгів заміщають заробітну плату, то 10 робітників не могли б створити додаткової
вартості більшої, ніж
10 фунтів стерлінгів; тимчасом як 90 робітників, весь продукт
яких = 180 фунтам стерлінгів і заробітна плата яких = 90 фунтам стерлінгів, створили б додаткову
вартість в 90 фунтів стерлінгів. Отже, норма зиску була б в одному випадку 10\%, в другому 90\%. Коли
б це було інакше, то вартість і додаткова
вартість мусили б бути чимось іншим, а не упредметненою
працею. Отже, через те що в різних сферах виробництва капітали, розглядувані в процентах — або
рівновеликі капітали —
неоднаково розподіляються на сталий і змінний елементи, приводять в рух неоднакову кількість живої
праці і тому створюють
неоднакову кількість додаткової вартості, отже й зиску, то
норма зиску, яка саме й становить процентне відношення додаткової вартості до всього капіталу, в них
різна.

Але якщо капітали різних сфер виробництва, взяті в процентному обчисленні, або рівновеликі капітали
в різних сферах
виробництва створюють неоднаковий зиск в наслідок їх різного
органічного складу, то звідси випливає, що зиски неоднакових
капіталів в різних сферах виробництва не можуть стояти у
прямому відношенні до відповідних величин цих капіталів,
що, отже, зиски в різних сферах виробництва не є пропорціональні величинам відповідних застосованих
у цих сферах капіталів. Бо таке зростання зиску pro rata [пропорціонально до] величини застосованого
капіталу припускало б, що в процентному відношенні зиски є рівні, що, отже, рівновеликі капітали
в різних сферах виробництва мають однакові норми зиску, не
зважаючи на їх різний органічний склад. Тільки в межах однієї
і тієї ж сфери виробництва, де, отже, органічний склад капіталу є даний, або в різних сферах
виробництва з однаковим
органічним складом капіталу маси зиску стоять в прямому відношенні до мас застосованих капіталів. Що
зиски неоднакових
своєю величиною капіталів пропорціональні їх величинам, означає
взагалі тільки те, що рівновеликі капітали дають рівновеликі
зиски, або що норма зиску для всіх капіталів є рівна, яка б
не була їх величина і їх органічний склад.

Все вищевикладене має місце при тому припущенні, що товари продаються по їх вартостях. Вартість
товару дорівнює
вартості вміщеного в ньому сталого капіталу, плюс вартість
репродукованого в ньому змінного капіталу, плюс приріст цього
\parbreak{}  %% абзац продовжується на наступній сторінці
