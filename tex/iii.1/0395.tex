високо погнала вгору розмір процента, була недостача капіталу;
причиною цього був страх (the alarm), трудність одержати банкноти“ [стор. 135].

В 1847 році Англія заплатила за кордон принаймні 9 мільйонів фунтів стерлінгів золотом за ввезені
харчові продукти.
З них 7 1/2 мільйонів з Англійського банку і 1 1/2 з інших джерел
(стор. 204 [277]). — Morris, управитель Англійського банку, заявив:
„23 жовтня 1847 року державні фонди і акції каналів та залізниць
були вже знецінені на 114752 225 фунтів стерлінгів“ (стор. 288).
Той самий Morris, запитаний лордом G. Bentinck: „Хіба вам невідомо, що весь капітал, вкладений у
папери й продукти всякого роду, знецінився в такій самій мірі, що сировинні матеріали,
бавовна, шовк, вовна були відправлені на континент по таких самих
бросових цінах і що цукор, кофе й чай продавалися з молотка? —
Нація неминуче мусила принести значну жертву, щоб протидіяти
відпливові золота, спричиненому величезним довозом харчових
продуктів. — Чи не думаєте ви, що було б краще зачепити ті
8 мільйонів фунтів стерлінгів, які лежали в сейфах банку, ніж
намагатися одержати назад золото з такими жертвами? — Я цього
не думаю“ [стор. 291 і далі]. — А ось коментар до цього героїзму. Дізраелі екзаменує пана W.
Cotton’a, директора і колишнього
управителя Англійського банку. „Який був дивіденд, що його
одержали акціонери банку в 1844 році? — 7\% річних. — А дивіденд 1847 року? — 9\%. — Чи платить банк у
поточному році прибутковий податок за своїх акціонерів? — Авжеж. — Чи зробив
це банк і в 1844 році? — Ні.\footnote{
Тобто раніше спочатку встановлювався дивіденд, а потім з нього при
виплаті з кожного акціонера вираховувався прибутковий податок; а після
1844 року спочатку сплачувався податок з усього зиску банку і потім вже
розподілявся дивіденд „free of Income Tax“ [вільний від прибуткового податку].
Отже, в останньому випадку той самий номінальний процент вищий на суму
податку. — Ф. Е.
} — Отже, в такому разі цей банковий
акт (1844 р.) був в дуже значній мірі в інтересах акціонерів...
Результат, отже, був такий, що від часу надання чинності новому законові дивіденд акціонерів
підвищився з 7 до 9\% і,
крім того, прибутковий податок сплачується тепер банком, тимчасом як раніше він мав сплачуватися
акціонерами? — Цілком
вірно“ (№ 4356—4361).

Відносно нагромадження грошових запасів у банках під час
кризи 1847 року Mr. Pease, провінціальний банкір, заявляє: „4605.
Через те що банк був змушений дедалі більше підвищувати
свій процент, побоювання стали загальними; провінціальні банки
збільшили свої грошові запаси, а також і суми банкнот; і багато хто з нас — ті, що звичайно тримали
в себе, може, лиш
кілька сотень фунтів золотом або банкнотами, — негайно почали нагромаджувати у своїх сейфах і
конторках тисячі фунтів, бо панувала цілковита непевність щодо дисконту і здатності векселів
циркулювати на ринку; і таким чином настало