Розділ п’ятнадцятий

Розвиток внутрішніх суперечностей закону
І. Загальні зауваження

В першому відділі цієї книги ми бачили, що норма зиску завжди виражає норму додаткової вартості
нижчою, ніж вона є. Тепер ми побачили, що навіть зростаюча норма додаткової вартості має тенденцію
виражатись у падаючій нормі зиску. Норма
зиску дорівнювала б нормі додаткової вартості тільки в тому випадку, коли с було б = 0, тобто коли б
увесь капітал витрачався на заробітну плату. Падаюча норма зиску тільки тоді виражає падаючу норму
додаткової вартості, коли відношення між вартістю сталого капіталу і масою робочої сили, яка
приводить його в рух, лишається незмінним, або коли ця остання збільшується у відношенні до вартості
сталого капіталу.

Рікардо, досліджуючи, як він гадав, норму зиску, в дійсності досліджував тільки норму додаткової
вартості і цю останню тільки при тому припущенні, що робочий день щодо інтенсивності й довжини є
стала величина.

Падіння норми зиску і прискорене нагромадження лиш остільки є різні вирази одного й того ж процесу,
оскільки і те і друге є виразом розвитку продуктивної сили. Нагромадження, з свого боку, прискорює
падіння норми зиску, оскільки разом з ним
дана концентрація робіт у великому масштабі, а тому й вищий склад капіталу. З другого боку, падіння
норми зиску знову таки прискорює концентрацію капіталу і його централізацію шляхом експропріації
дрібних капіталістів, шляхом експропріації
останніх решток безпосередніх виробників, у яких лишається ще щонебудь експропріювати. В наслідок
цього, з другого боку, прискорюється — щодо маси — нагромадження, хоча з падінням норми зиску падає
і норма нагромадження.

З другого боку, оскільки норма зростання вартості всього капіталу, норма зиску, є стимулом
капіталістичного виробництва (подібно до того, як збільшення вартості капіталу є його єдиною метою),
падіння цієї норми уповільнює утворення нових самостійних капіталів і виступає таким чином як
загроза для розвитку капіталістичного процесу виробництва; воно сприяє перепродукції, спекуляції,
кризам, утворенню надмірного капіталу поряд з надмірним населенням. Отже, ті економісти, які,
подібно до Рікардо, вважають капіталістичний спосіб виробництва за абсолютний, відчувають тут, Що
цей спосіб виробництва сам собі створює межу, і тому приписують цю межу не виробництву, а природі (в
ученні про ренту). Але важливим в їх жаху перед падаючою нормою зиску є відчуття того, що
капіталістичний спосіб виробництва в розвитку продуктивних сил має таку межу, яка не стоїть ні в
якому зв’язку з виробництвом багатства як та-
