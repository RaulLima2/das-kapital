\parcont{}  %% абзац починається на попередній сторінці
\index{iii1}{0106}  %% посилання на сторінку оригінального видання
сухот, в Блекберні й Скіптоні — 167, в Конглетоні й Бредфорді —
168, в Лейстері — 171, в Ліку — 182, в Маккльсфільді — 184,
в Больтоні — 190, в Ноттінгемі — 192, в Ронделі — 193, в Дербі —
198, в Сальфорді і Аштоні-на-Лайні — 203, в Лідсі — 218, в Престоні — 220 і в Манчестері — 263
(стор. 24). Нижченаведена таблиця дає ще разючіший приклад. Вона наводить випадки смерті
внаслідок хвороб легенів окремо для обох статей між 15 і
25 роками, обчислені на кожні 10 0000 мешканців. Вибрано такі
округи, де тільки жінки зайняті в промисловості, провадженій
у закритих приміщеннях, а чоловіки — в усяких можливих галузях праці.

Округи    Головна промисловість    Число випадків смерті від легеневих захворувань між 15 і 25
роками на 100 000 жителів
        Чоловіки    Жінки
Berkhampstead    Плетіння з соломи, працюють жінки . . . .    219    578
Leighton Buzzard    Плетіння з соломи, працюють жінки . .     309    554
Newport Pagnell    Плетіння мережива жінками . . . . . . . .         301    617
Towcester    Плетіння мережива жінками. . . . . . . . . . . . . . . .    239    577
Yeovil    Виробництво рукавичок, здебільшого працюють жінки . . .  280    409
Leek    Шовкова промисловість, переважно жінки . . . . . . . .    437    856
Congleton    Шовкова промисловість, переважно жінки . . .    566    790
Macclesfield    Шовкова промисловість, переважно жінки. .    593    890
Здорова сільська місцевість    Землеробство . . . . . . . . . .         331    333

В округах шовкової промисловості, де участь чоловіків
у фабричній праці більша, більша також і смертність серед них.
Норма смертності від сухот і т. п. як чоловіків, так і жінок
виявляє тут, як сказано в звіті, „обурливі (atrocious) санітарні
умови, за яких провадиться значна частина нашої шовкової
промисловості“. І це, якраз, та сама шовкова промисловість,
фабриканти якої, посилаючись на винятково сприятливі санітарні умови свого виробництва, вимагали і
почасти добилися
винятково довгого робочого часу для дітей, молодших 13 років,
(книга І, розд. VIII, 6, стор. 306*).

„Без сумніву, жодна з досліджених досі галузей промисловості
не дає сумнішої картини, ніж та, що її дає доктор Сміт
щодо кравецтва... Майстерні, каже він, дуже неоднакові щодо

* Стор. 214 рос. вид. 1935 р. Ред. укр. перекладу.
\parbreak{}  %% абзац продовжується на наступній сторінці
