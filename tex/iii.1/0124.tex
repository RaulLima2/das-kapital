перетворена в гроші; друга частина існує як гроші в будьякій
формі і мусить бути знову перетворена в умови виробництва;
нарешті, третя частина перебуває у сфері виробництва, почасти
у первісній формі засобів виробництва, сировинних матеріалів,
допоміжних матеріалів, куплених на ринку півфабрикатів,
машин та іншого основного капіталу, почасти як продукт, який
ще тільки виготовляється. Як діє тут підвищення вартості або
зниження вартості, це в великій мірі залежить від того відношення,
в якому стоять одні до одних ці складові частини. Щоб
спростити питання, залишмо спочатку осторонь весь основний
капітал і розгляньмо тільки ту частину сталого капіталу, яка
складається з сировинних матеріалів, допоміжних матеріалів,
півфабрикатів і товарів, які ще тільки виготовляються або вже
є готові на ринку.

Якщо підвищується ціна сировинного матеріалу, наприклад,
бавовни, то підвищується й ціна бавовняних товарів — півфабрикатів,
як от пряжа, і готових товарів, як от тканини і т. д., —
сфабрикованих з дешевшої бавовни; так само підвищується
і вартість як ще непереробленої бавовни, яка є на складі, так
і тієї, що перебуває ще в процесі оброблення. Ця остання,
через те що вона в наслідок зворотного впливу стає виразом
більшої кількості робочого часу, додає до продукту, в який
вона входить як складова частина, більшу вартість, ніж та, яку
вона первісно мала сама і яку капіталіст заплатив за неї.

Отже, якщо підвищення цін сировинного матеріалу супроводиться
наявністю на ринку значної маси готового товару, —
все одно, на якому ступені готовості, — то підвищується вартість
цього товару і разом з тим відбувається підвищення вартості
наявного капіталу. Те саме стосується і до запасів сировинного
матеріалу і т. д., які перебувають в руках виробників.
Це підвищення вартості може відшкодувати або й більш ніж
відшкодувати окремих капіталістів або навіть і цілу окрему
сферу виробництва капіталу за падіння норми зиску, яке виникає
з підвищення ціни сировинного матеріалу. Не входячи тут
у деталі впливу конкуренції, можна, однак, ради повноти відзначити,
що 1) коли запаси сировинного матеріалу, які перебувають
на складах, значні, то вони протидіють підвищенню цін,
що виникає в місці виробництва сировинного матеріалу; 2) коли
півфабрикати або готові товари, які перебувають на ринку, дуже
тиснуть на ринок, то вони заважають ціні готових товарів і півфабрикатів
зростати відповідно до ціни їх сировинного матеріалу.

Зворотне маємо при падінні цін сировинного матеріалу, яке
при інших однакових умовах підвищує норму зиску. Товари, які
перебувають на ринку, речі, які ще тільки виготовляються,
запаси сировинного матеріалу знецінюються і цим самим протидіють
одночасному підвищенню норми зиску.

Чим менші запаси, які перебувають у сфері виробництва
і на ринку, наприклад наприкінці операційного року, коли си-
