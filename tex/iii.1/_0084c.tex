\index{iii1}{0084}  %% посилання на сторінку оригінального видання
3. р' підвищується або падає в меншій пропорції, ніж m',
якщо v/K змінюється в напрямі, протилежному до зміни m', але
в меншій пропорції:\footnote{
с + 20 v + 20 m; m' = 100\%, p' = 20\%
90 с + 10 v + 15 m; m' = 150\%, p' = 15\%

m' підвищилось з 100\% до 150\%, р' зменшилось від 20\% до 15\%.
} с + 20 v + 10 m; m' = 50\%, p' = 10\%
90 с + 10 v + 15 m; m' = 150\%, p' = 15\%
50\% : 150\% > 10\% : 15\%.

4. р' підвищується, хоч m' падає, або падає, хоч m' підвищується,
якщо v/K змінюється в напрямі, протилежному до зміни
m', і в більшій пропорції, ніж m'.

5. Нарешті, р' лишається незмінним, хоч m' підвищується або
падає, якщо v/K змінює свою величину в протилежному напрямі,
але точно в тій самій пропорції, що й m'.

Тільки цей останній випадок потребує ще деякого пояснення.
Як ми бачили вище при змінах v/K, що одна й та сама норма
додаткової вартості може виражатися в найрізніших нормах
зиску, так ми бачимо тут, що в основі однієї і тієї самої норми
зиску можуть лежати дуже різні норми додаткової вартості.
Але в той час, як при незмінному m' першої-ліпшої зміни у відношенні
v до К досить було для того, щоб викликати відмінність
в нормі зиску, — при зміні величини m' мусить настати точно
відповідна зворотна зміна величини v/K для того, щоб норма
зиску лишилась та сама. Для одного й того ж капіталу або для
двох капіталів у тій самій країні це можливе тільки в дуже
виняткових випадках. Візьмімо, наприклад, капітал\footnote{
с + 16 v + 24 m; K = 96, m' = 150\%, p' = 25\%.

Отже, для того, щоб р' було, як і раніш, = 20\%, весь капітал
мусив би зрости до 120, отже, сталий — до 104:
} с + 20 v + 20 m; K = 100, m' = 100\%, p' = 20\%

і припустімо, що заробітна плата впала настільки, що тепер за
16 v можна було б мати те саме число робітників, як раніш за
20 v. Тоді ми, при інших незмінних умовах і звільненні 4 v,
маємо

104 с + 16 v + 24 m; K = 120, m' = 150\%, p' = 20\%
