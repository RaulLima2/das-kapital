\parcont{}  %% абзац починається на попередній сторінці
\index{iii1}{0260}  %% посилання на сторінку оригінального видання
час, який виграє суспільство, не цікавить капіталістичне виробництво.
Розвиток продуктивної сили для нього важливий лиш
остільки, оскільки він збільшує додатковий робочий час робітничого
класу, а не оскільки він взагалі зменшує робочий час
матеріального виробництва; таким чином капіталістичне виробництво
рухається в суперечностях.

Ми бачили, що зростаюче нагромадження капіталу включає
зростаючу концентрацію його. Таким чином зростає влада капіталу,
персоніфіковане в капіталісті усамостійнення суспільних
умов виробництва проти дійсних виробників. Капітал дедалі більше
виявляє себе як суспільна сила, яка функціонує через капіталіста
і яка не стоїть уже в ніякому відношенні до того, що
може створити праця окремого індивіда, — але як відчужена,
усамостійнена суспільна сила, що як річ, і за допомогою цієї
речі як влада капіталіста, протистоїть суспільству. Суперечність
між загальною суспільною силою, в яку перетворюється капітал,
і приватною владою окремих капіталістів над цими суспільними
умовами виробництва розвивається в дедалі більш кричущу суперечність
і включає в собі розв’язання цього відношення,
оскільки воно разом з тим передбачає вироблення умов виробництва
у загальні, колективні, суспільні умови виробництва.
Це вироблення визначається розвитком продуктивних сил при
капіталістичному виробництві і тим способом, яким відбувається
цей розвиток

Жоден капіталіст не застосовує добровільно нового способу
виробництва, хоч би наскільки він був продуктивніший і хоч би
наскільки він збільшував норму додаткової вартості, якщо він
зменшує норму зиску. Але кожен такий новий спосіб виробництва
здешевлює товари. Тому капіталіст спочатку продає
їх вище їх ціни виробництва, може, вище їх вартості. Він кладе
собі в кишеню ріжницю між їх витратами виробництва і ринковою
ціною всіх інших товарів, вироблених при вищих витратах
виробництва. Він може це робити тому, що пересічний робочий
час, суспільно потрібний для виробництва цих товарів, є більший,
ніж робочий час, потрібний при новому способі виробництва.
Його методи виробництва стоять вище пересічних суспільних.
Але конкуренція робить їх загальними і підпорядковує їх загальному
законові. Тоді починається зниження норми зиску, — спочатку,
може, в цій сфері виробництва, а потім вона вирівнюється
з іншими, — яке, отже, цілком незалежне від волі капіталістів.

З приводу цього треба ще зауважити, що цей самий закон
панує і в тих сферах виробництва, продукт яких ні безпосередньо,
ні посередньо не входить у споживання робітника або
в умови виробництва його засобів існування; отже, і в тих сферах
виробництва, в яких ніяке здешевлення товарів не може збільшити
відносну додаткову вартість, здешевити робочу силу.
\parbreak{}  %% абзац продовжується на наступній сторінці
