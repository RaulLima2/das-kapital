функціонує в процесі своєї репродукції як товарний капітал.
Віддання в позику і одержання позики, замість продажу й купівлі,
— це ріжниця, яка тут випливає з специфічної природи
такого товару, як капітал. Цілком так само, як і та ріжниця, що
тут сплачується процент замість ціни товару. Якщо процент
назвати ціною грошового капіталу, то це буде ірраціональна
форма ціни, цілком суперечна поняттю ціни товару.60 Ціна зведена
тут до її чисто абстрактної і беззмістовної форми, до того,
що вона є певна сума грошей, яка сплачується за щонебудь, що
так чи інакше фігурує як споживна вартість; тимчасом як за
своїм поняттям ціна дорівнює вираженій у грошах вартості цієї
споживної вартості.

Процент як ціна капіталу є з самого початку цілком ірраціональний
вираз. Товар має тут подвійну вартість, поперше, вартість
і, подруге, відмінну від цієї вартості ціну, тимчасом як ціна
є грошовий вираз вартості. Грошовий капітал насамперед є не
що інше, як сума грошей або вартість певної маси товарів, фіксована
як сума грошей. Якщо в позику віддається товар як
капітал, то він є тільки замаскована форма грошової суми. Бо
те, що дається в позику як капітал, це не стільки то фунтів
бавовни, а стільки то грошей, які існують у формі бавовни як
її вартість. Тому ціна капіталу відноситься до нього як до грошової
суми, хоч і не як до „currency“ [засобу циркуляції], як
це гадає пан Торренс (див. вище примітку 60). Яким же чином
сума вартості може мати ціну, крім своєї власної ціни, крім
ціни, вираженої в її власній грошовій формі? Адже ціна є вартість
товару (і це стосується також до ринкової ціни, відмінність
якої від вартості є не якісна, а тільки кількісна, тобто така, що
стосується тільки до величини вартості) в відміну від його
споживної вартості. Ціна, яка якісно відмінна від вартості —
це абсурдна суперечність.61

Капітал виявляє себе як капітал в наслідок зростання його
вартості; ступінь зростання його вартості є виразом того кіль-

60 „Вираз вартість (value), застосований до currency [засобу циркуляції], має
три значення... Подруге, currency actually in hand (гроші, які дійсно є в касі],
порівняно з тією самою сумою грошей, яка надійде пізніше. Тоді їх вартість
вимірюється розміром процента, а розмір процента визначається by the ratio
between the amount of loanable capital and the demand for it [відношенням між
сумою капіталу, що дається в позику, і попитом на нього]“ (Полковник
R. Torrens: „On the Operation of the Bank Charter Act of 1844 etc.“, 2 вид. 1847,
[стор. 5 і далі]).

61  The ambiguity of the term value of money or of the currency, when employed
indiscriminately as it is, to signify both value in exchange for commodities
and value in use of capital, is a constant source of confusion“ [„Двояке значення
виразу вартість грошей або засобу циркуляції, коли його, як це буває, без
розрізнення вживають для позначення як мінової вартості товарів, так і споживної
вартості капіталу, є постійним джерелом плутанини“] (Tooke: „Inquiry
into the Currency Principle“, стор. 77). — Головної плутанини (яка лежить у самій
суті справи), що вартість як така (процент) стає споживною вартістю капіталу,
Тук не бачить.
