ному промисловому підприємстві, повинна відповідати застосованому
в ньому змінному капіталові, бо сталий капітал не дає
ніякого зиску. Але це суперечить дійсності. Бо на практиці зиск
відповідає не змінному, а всьому капіталові. І Маркс сам бачить
це (І, розд. XI) і визнає, що зовнішньо факти суперечать
його теорії. Але як він розв’язує цю суперечність? Він відсилає
своїх читачів до подальшого тома, який ще не з’явився.
Про цей том Лоріа вже раніш сказав своїм читачам, що він
не вірить тому, що Маркс хоч би одну мить думав про те,
щоб його написати, і тепер він тріумфуючи вигукує: „Отже,
я мав рацію, коли твердив, що цей другий том, яким Маркс
весь час загрожує своїм супротивникам і який, однак, ніколи
не з’явиться, що цей том, дуже ймовірно, був хитромудрою
виверткою, якої Маркс уживав тоді, коли в нього не
вистачало наукових аргументів (un ingegnoso spediente ideato
dal Marx a sostituzione degli argomenti scientifici)“. І хто тепер
не переконався, що Маркс стоїть на такій самій висоті наукового
шахрайства, як l'illustre Лоріа, той уже цілком безнадійна
людина.

Отже, ми ось чого навчились: за паном Лоріа теорія додаткової
вартості Маркса абсолютно несполучна з фактом загальної
рівної норми зиску. Аж ось з’явилась друга книга і разом
з тим моє публічно поставлене питання саме про цей пункт.
Коли б пан Лоріа був одним з нас, соромливих німців, він би
якось збентежився. Але він — сміливий житель півдня, він походить
з гарячого клімату, де, як він може твердити, нахабність
(Unverfrorenheit) * є до певної міри природна умова. Питання
про норму зиску поставлено публічно. Пан Лоріа публічно оголосив
його нерозв’язним. І саме через це він тепер перевищить
самого себе, розв’язавши його публічно.

Це чудо сталося в „Conrads Jahrbücher“, Neue Folge, т. XX,
стор. 272 і далі, у статті про вищезгаданий твір Конрада Шмідта.
Після того, як він вичитав у Шмідта, яким чином утворюється
торговельний зиск, йому зразу все стало ясно. „Через те що
визначення вартості робочим часом дає перевагу тим капіталістам,
які вкладають більшу частину свого капіталу у заробітну
плату, то непродуктивний“ [слід сказати — торговельний] „капітал
може вимусити собі від цих капіталістів, що мають перевагу,
вищий процент“ [слід сказати — зиск] „і утворити рівність
між окремими промисловими капіталістами... Так, наприклад,
якщо промислові капіталісти А, В, С застосовують у виробництві
кожний по 100 робочих днів і відповідно 0, 100, 200
сталого капіталу, а заробітна плата за 100 робочих днів містить
у собі 50 робочих днів, то кожний капіталіст одержує додаткову
вартість у 50 робочих днів, а норма зиску становить 100\%

   * Тут гра слів: німецьке „Unverfrorenheit“ буквально можна також тлумачити
як „здатність не замерзати“. Ред. укр. перекладу.
