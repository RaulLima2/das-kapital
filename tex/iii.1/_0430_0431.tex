\parcont{}  %% абзац починається на попередній сторінці
\index{iii1}{0430}  %% посилання на сторінку оригінального видання
курсу в континентальних країнах порівняно з станом гарячкового
неспокою й тривоги, який завжди постає в Англії, коли
здається, ніби скарби банку близькі до свого повного вичерпання,
не вражаючись при цьому тією великою вигодою, яку
має в цьому відношенні металічна циркуляція“].

Якщо ж ми залишимо осторонь відплив золота, то як може
тоді банк, що випускає банкноти, отже, наприклад, Англійський
банк, збільшувати суму грошових позик, які він видає, не збільшуючи
випуску своїх банкнот?

Поза стінами банку всі банкноти, чи циркулюють вони, чи
лежать без діла в приватних сховищах, для самого банку перебувають
у циркуляції, тобто не в його володінні. Отже, якщо
банк розширює свої дисконтні і ломбардні операції, позики під
securities, то видані для цього банкноти мусять припливати до
нього назад, бо інакше вони збільшують суму циркуляції, чого
саме й не повинно бути. Цей зворотний приплив може відбуватись
двояким способом.

\emph{Поперше}: Банк платить клієнтові $А$ банкнотами за цінні папери;
$А$ оплачує ними особі $В$ вексель, якому надійшов строк,
а $В$ знову вносить ці банкноти в банк як вклад. Таким чином
циркуляція цих банкнот закінчена, але позика лишається. (The
loan remains, and the currency, if not wanted, finds its way back
to the issuer [Позика лишається, а засіб циркуляції, якщо в ньому
немає потреби, знаходить свій шлях до того, хто його випустив]
\emph{Fullarton}, стор. 97). Банкноти, які банк позичив особі $А$, повернулися
тепер до нього назад; навпаки, банк є кредитором $А$
або тієї особи, на яку виставлено вексель, дисконтований $А$,
і дебітором $В$ на суму вартості, виражену в цих банкнотах,
а $В$ може в наслідок цього порядкувати відповідною частиною
капіталу банку.

\emph{Подруге}: $А$ платить $В$, а сам $В$ або $С$, якому $В$ в свою чергу
платить цими банкнотами, оплачує ними ж банкові, безпосередньо
або посередньо, векселі, яким надійшов строк. В цьому
випадку банкові платиться його ж власними банкнотами. На
цьому тоді операція закінчується (до зворотного платежу банкові
клієнтом $А$).

В якій же мірі можна розглядати позику банку клієнтові $А$
як позику капіталу або як просту позику засобів платежу?\footnote{
Те місце оригіналу, що йде вслід за цим, незрозуміле в даному зв’язку,
і до закриття дужок воно наново перероблене упорядником. В іншому зв’язку
це питання було вже зачеплене в розділі XXVI. — \emph{Ф. Е.}
}

[Це залежить від природи самої позики. При цьому слід
дослідити три випадки.

\emph{Перший випадок}. — $А$ одержує від банку певні суми позики під
свій особистий кредит, не даючи при цьому ніякого забезпечення.
В цьому випадку він одержав у позику не тільки засоби платежу,
але, безумовно, і новий капітал, який він може застосовувати
\index{iii1}{0431}  %% посилання на сторінку оригінального видання
в своєму підприємстві як додатковий капітал і збільшувати
його вартість до повернення його банкові.

\emph{Другий випадок}. — $А$ заставив банкові цінні папери, зобов’язання
державної позики або акції, і одержав під них позику
готівкою, наприклад, до двох третин їх курсової вартості. В
цьому випадку він одержав засоби платежу, яких він потребує,
але не додатковий капітал, бо він дав у руки банку більшу
капітальну вартість, ніж одержав від нього. Але, з одного боку,
ця більша капітальна вартість не могла бути використана ним
для його потреб поточного моменту, — потреб у засобах платежу,
— бо вона була уже вкладена в певній формі з метою одержання
процента; з другого боку, у $А$ були свої підстави не перетворювати
її безпосередньо в засоби платежу шляхом продажу.
Його цінні папери мали між іншим призначення функціонувати
як резервний капітал, і він використав їх саме у функції резервного
капіталу. Отже, між $А$ і банком відбулася тимчасова взаємна
передача капіталів, при чому $А$ не одержав ніякого додаткового
капіталу (навпаки!), але одержав, звичайно, потрібні йому засоби
платежу. Навпаки, для банку ця операція була тимчасовим тривким
вкладенням грошового капіталу в формі позики, перетворенням
грошового капіталу з однієї форми в другу, а таке
перетворення саме й є істотною функцією банкової справи.

\emph{Третій випадок}. — $А$ дисконтував у банку вексель і одержав
при цьому, після відрахування дисконту, певну суму готівкою.
В цьому випадку він продав банкові грошовий капітал у нетекучій
формі за суму вартості в текучій формі; вексель, якому
ще не надійшов строк, він продав за готівку. Вексель тепер
є власністю банку. Справа ні трохи не змінюється від того, що
останній індосент, $А$, в разі вексель не буде оплачений, відповідає
перед банком на суму векселя; цю відповідальність він
поділяє з іншими індосентами і з векселедавцем, від яких він має
право у свій час вимагати повернення відповідної суми. Отже,
тут немає ніякої позики, а цілком звичайна купівля й продаж.
Тому $А$ нічого не повинен сплачувати банкові, банк покриває
свою видачу, інкасуючи вексель, коли настає строк платежу.
В цьому випадку теж відбулася взаємна передача капіталу між $А$
та банком, і при тому цілком така сама, як при купівлі й продажу
всякого іншого товару, і саме тому $А$ не одержав ніякого
додаткового капіталу. Йому потрібні були, і він одержав засоби
платежу, і одержав він їх завдяки тому, що банк перетворив
для нього одну форму його грошового капіталу, вексель, у
другу форму, в гроші.

Отже, про дійсне авансування, про дійсну позику капіталу
може бути мова тільки в першому випадку. В другому ж і третьому
випадку хіба тільки в тому розумінні, як „авансують капітал“
при всякому капіталовкладенні. В цьому розумінні банк
авансує, позичає клієнтові $А$ грошовий капітал; але для $А$ він
є \emph{грошовий капітал} щонайбільше в тому розумінні, що він
\parbreak{}  %% абзац продовжується на наступній сторінці
