Якщо ми капітали І — V знову розглядатимемо як єдиний сукупний капітал, то побачимо, що і в цьому
випадку склад суми
п’яти капіталів = 500 = 390 c + 110 v, отже, пересічний склад, = 78 c + 22 v, лишається той самий;
так само й пересічна додаткова вартість = 22\%. Розподіливши цю додаткову вартість рівномірно між
І—V, ми одержали б такі товарні ціни:

Капітали
Додаткова вартість
Вартість товарів
Витрати виробництва
Ціна товарів
Норма зиску
Відхилення ціни від вартості

І. 80 c + 20 v    20    90    70    92    22\%    + 2
II. 70 c + 30 v   30   111   81   103   22\%  — 8
III. 60 c + 40 v  40   131   91   113   22\% — 18
IV. 85 c + 15 v   15    70    55    77    22\%    + 7
V. 95 c + 5 v        5     20    15    37    22\%  + 17

В загальній сумі товари продаються на 2 + 7 + 17 = 26 вище і
на 8 + 18 = 26 нижче вартості, так що відхилення цін взаємно
знищуються в наслідок рівномірного розподілу додаткової вартості, тобто в наслідок додання
пересічного зиску в 22 на
100 одиниць авансованого капіталу до відповідних витрат виробництва товарів І—V; в тому самому
відношенні, в якому одна
частина товарів продається вище, друга продається нижче її
вартості. І тільки продаж їх по таких цінах уможливлює те, що
норма зиску для І—V є однакова, 22\%, не зважаючи на різний
органічний склад капіталів І—V. Ціни, які виникають таким чином, що з різних норм зиску різних сфер
виробництва береться
пересічна і ця пересічна додається до витрат виробництва в різних сферах виробництва, — такі ціни є
ціни виробництва. Передумовою їх є існування однієї загальної норми зиску, а ця
остання знов таки передбачає, що норми зиску в кожній окремій сфері виробництва, взяті самі по собі,
вже зведені до
відповідної кількості пересічних норм. Ці окремі норми зиску в кожній сфері виробництва = m/K, і їх
треба виводити з вартості товару, як це і було зроблено в першому відділі цієї книги. Без такого
виведення загальна норма зиску (а тому й ціна виробництва товару) була б безглуздим і ірраціональним
уявленням. Отже, ціна виробництва товару дорівнює витратам його
виробництва плюс доданий до них зиск, обчислений у процентах
відповідно до загальної норми зиску, тобто дорівнює витратам
виробництва товару плюс пересічний зиск.

В наслідок різного органічного складу капіталів, вкладених
у різні галузі виробництва, а тому в наслідок тієї обставини,
що — залежно від різного процентного відношення змінної частини до всього капіталу даної величини —
рівновеликими капіталами приводяться в рух дуже різні кількості праці, ними привласнюються також
дуже різні кількості додаткової праці, або
виробляються дуже різні маси додаткової вартості. Відповідно
до цього норми зиску, які панують в різних галузях вироб-
