Відділ п’ятий

Розпад зиску на процент і підприємницький
дохід. Капітал, що дає процент

Розділ двадцять перший

Капітал, що дає процент

При першому дослідженні загальної або пересічної норми
зиску (відділ II цієї книги) ми мали цю останню перед собою
ще не в її готовому вигляді, бо вирівнення виступало ще просто
як вирівнення промислових капіталів, вкладених у різні сфери.
Це було доповнено в попередньому відділі, де досліджувались
участь торговельного капіталу в цьому вирівненні і торговельний
зиск. Загальна норма зиску і пересічний зиск виступили
при цьому у вужчих межах, ніж раніше. В ході дальшого
дослідження слід мати на увазі, що коли ми далі говоримо
про загальну норму зиску або пересічний зиск, то це тільки
в останньому значенні, отже, тільки щодо готової форми пересічної
норми. А через те що ця норма тепер однакова для промислового
і торговельного капіталу, то немає також надалі
потреби, оскільки йдеться тільки про цей пересічний зиск,
розрізняти промисловий і торговельний зиск. Чи вкладено капітал
у сферу виробництва, як промисловий капітал, чи у сферу циркуляції,
як торговельний капітал, він однаково дає той самий
річний пересічний зиск pro rata [відповідно до] своєї величини.

Гроші, — взяті тут як самостійний вираз певної суми вартості,
однаково, чи існує вона фактично у вигляді грошей, чи у
вигляді товарів, — можуть бути перетворені на основі капіталістичного
виробництва в капітал, і в наслідок такого перетворення
стають з даної вартості самозростаючою вартістю, вартістю,
що збільшується. Вони виробляють зиск, тобто дають
капіталістові змогу добувати з робітників і привласнювати собі
певну кількість неоплаченої праці, додатковий продукт і додаткову
вартість. Таким чином, крім тієї споживної вартості, яку
вони мають як гроші, вони набувають ще додаткову споживну
вартість, саме ту, що вони функціонують як капітал. Їх спо-
