\parcont{}  %% абзац починається на попередній сторінці
\index{iii1}{0346}  %% посилання на сторінку оригінального видання
щоб робити платежі. Це разом з тим — тому що підвищенню процента
відповідає падіння ціни цінних паперів — дає людям звільним
грошовим капіталом прекрасну нагоду заволодіти за безцінь
такими процентними паперами, які при нормальному ході справ
мусять знову досягти принаймні своєї пересічної ціни, як тільки
розмір процента знову впаде.\footnote{
„An old customer of a banker was refused a loan upon a 200000 £, bond;
when about to leave to make known his suspension of payment, he was told
there was no necessity for the step, under the circumstances the banker would buy
the bond at 150000£“ Один банкір відмовив одному своєму старому клієнтові
у позиці під цінні папери в 200000 фунтів стерлінгів; коли клієнт уже збирався
піти, щоб оголосити припинення платежів, банкір йому сказав, що немає потреби
удаватись до цього кроку, що при даних обставинах він, банкір, купить
у нього ці цінні папери за 150000 фунтів стерлінгів“] („The Theory of the Exchanges.
The Banc Charter Act of 1844 etc.“ London 1864, стор. 80).
}

Але існує також тенденція до падіння розміру процента, цілком
незалежно від коливань норми зиску. І саме в наслідок
двох головних причин:

I. „Якщо навіть припустити, що капітал ніколи не береться
в позику інакше, як для продуктивного застосування, то все ж
можливо, що розмір процента мінятиметься без будь-якої зміни
в нормі гуртового зиску. Бо в міру того, як певний народ прогресує
в розвитку багатства, виникає і все більше зростає клас
людей, які завдяки праці своїх предків володіють фондами, достатніми
для того, щоб жити тільки на одержувані з них проценти.
Багато є і таких осіб, які, беручи активну участь у ділах в молодому
і в дорослому віці, відходять від діла, щоб на старості
спокійно жити з процентів від нагромаджених сум. Обидва ці
класи мають тенденцію збільшуватися з зростанням багатства
країни; бо ті, що починають уже з капіталом середнього розміру,
легше досягають незалежного становища, ніж ті, що починають
з невеликим капіталом. Тому в старих і багатих країнах
та частина національного капіталу, власники якої не хочуть самі
його застосовувати, становить порівняно з сукупним продуктивним
капіталом суспільства відносно більшу величину, ніж
у новокультивованих і бідних країнах. Який численний є клас
рантьє в Англії! В тій самій мірі, в якій зростає клас рантьє,
зростає і клас позикодавців капіталу, бо обидва вони є те саме“
(\emph{Ramsay}: „Essay on the Distribution of Wealth [Edinburgh 1836]“,
стор. 201 [202]).

II. Розвиток кредитної системи та постійно зростаюче разом
із цим порядкування промисловців і купців, за посередництвом
банкірів, усіма грошовими заощадженнями всіх класів суспільства
і прогресуюча концентрація цих заощаджень до таких розмірів,
при яких вони можуть діяти як грошовий капітал, — це
так само мусить справляти тиснення на розмір процента.
Докладніше про це мова буде далі.

Щодо визначення норми процента Рамсей каже, що вона „залежить
почасти від норми гуртового зиску, почасти від того
\parbreak{}  %% абзац продовжується на наступній сторінці
