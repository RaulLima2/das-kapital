від додаткової вартості, але й від багатьох інших обставин: від
купівельних цін засобів виробництва, від методів виробництва,
продуктивніших, ніж пересічні, від економії сталого капіталу
і т. д. І, залишаючи осторонь ціну виробництва, від особливих
коньюнктур, а при кожній окремій операції від більшої чи меншої
спритності і підприємливості капіталіста залежить, чи купує
він і продає, і в якій мірі, вище або нижче ціни виробництва,
отже, чи привласнює він собі в процесі циркуляції більшу чи
меншу частину сукупної додаткової вартості. Але в усякому разі
кількісний поділ гуртового зиску перетворюється тут в якісний,
і це тим більше, що сам кількісний поділ залежить від того, що
належить розподілити, як активний капіталіст господарює капіталом
і який гуртовий зиск він дає йому як функціонуючий капітал,
тобто в наслідок функцій капіталіста як активного капіталіста.
Функціонуючий капіталіст припускається тут як невласних капіталу.
Власність на капітал представлена відносно нього позикодавцем,
грошовим капіталістом. Отже, процент, який він
сплачує цьому останньому, виступає як частина гуртового зиску,
яка припадає власності на капітал як такій. Протилежно до
цього та частина зиску, яка припадає активному капіталістові,
виступає тепер як підприємницький дохід, що ніби виникає виключно
з тих операцій або функцій, які він виконує в процесі
репродукції за допомогою капіталу, отже, спеціально з тих функцій,
які він як підприємець виконує в промисловості або торгівлі.
Отже, відносно нього процент виступає просто як плід власності
на капітал, плід капіталу самого по собі, абстрагованого від процесу
репродукції капіталу, як плід капіталу, оскільки він „не працює“,
не функціонує; тимчасом як підприємницький дохід здається
йому виключно плодом тих функцій, які він виконує з капіталом,
плодом руху капіталу, руху, який здається йому тепер його
власного діяльністю протилежно до недіяльності, неучасті грошового
капіталіста в процесі виробництва. Це якісне відокремлення
одне від одного двох частин гуртового зиску, завдяки якому
процент є плід капіталу самого по собі, плід власності на капітал,
незалежно від процесу виробництва, а підприємницький
дохід — плід капіталу, що пророблює процес, що діє в процесі
виробництва, і тому плід тієї активної ролі, яку застосовних
капіталу грає в процесі репродукції, — це якісне відокремлення
зовсім не є просто суб’єктивне уявлення грошового капіталіста,
з одного боку, і промислового капіталіста, з другого. Воно
основане на об’єктивному факті, бо процент припливає до грошового
капіталіста, до позикодавця, який є просто власником
капіталу, отже, просто представником власності на капітал до
процесу виробництва і поза процесом виробництва; а підприємницький
дохід припливає просто до функціонуючого капіталіста,
який є невласних капіталу.

Таким чином, як для промислового капіталіста, оскільки він
працює з узятим в позику капіталом, так і для грошового капі-
