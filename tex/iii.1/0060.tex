ліста, це не сам відчутний продукт, а надлишок вартості продукту
понад вартість спожитого на нього капіталу. Капіталіст
авансує весь капітал, не звертаючи уваги на ті різні ролі, що їх
відіграють складові частини капіталу у виробництві додаткової
вартості. Він однаково авансує всі ці складові частини капіталу
не тільки для того, щоб репродукувати авансований капітал, але
і для того, щоб виробити певний надлишок вартості понад
цей капітал. Він може перетворити вартість змінного капіталу,
який він авансує, у вищу вартість тільки через обмін його на
живу працю, через експлуатацію живої праці. Але він може експлуатувати
працю тільки в тому разі, коли він одночасно авансує
умови для здійснення цієї праці — засоби праці і предмет
праці, машини і сировинний матеріал, тобто коли він ту суму
вартості, якою він володіє, перетворює в форму умов виробництва;
як і взагалі, він тільки тому є капіталіст, тільки тому взагалі
може взятися до процесу експлуатації праці, що він як власник
умов праці протистоїть робітникові як володільцеві тільки робочої
сили. Вже раніше, в першій книзі, було показано, що саме
те, що цими засобами виробництва володіють не-робітники, перетворює
робітників у найманих робітників, а не-робітників — у капіталістів.
Капіталістові байдуже, чи розглядати справу так, що він
авансує сталий капітал для того, щоб здобути бариш із змінного,
чи так, що він авансує змінний капітал для того, щоб збільшити
вартість сталого; чи так, що він витрачає гроші на заробітну
плату для того, щоб надати машинам і сировинному матеріалові
вищу вартість, чи так, що він авансує гроші на машини та сировинний
матеріал для того, щоб мати можливість експлуатувати працю.
Хоч додаткову вартість утворює лише змінна частина капіталу,
проте вона утворює її тільки при тій умові, що авансуються
й інші частини, виробничі умови праці. Через те що капіталіст
може експлуатувати працю тільки за допомогою авансування сталого
капіталу, що він може збільшити вартість сталого капіталу
тільки за допомогою авансування змінного, то в його уявленні
ці капітали збігаються, і це тим більше, що дійсний рівень його
баришу визначається відношенням не до змінного капіталу, а
до всього капіталу, не нормою додаткової вартості, а нормою
зиску, яка, як ми побачимо, може лишатись однаковою і все ж
виражати різні норми додаткової вартості.

До витрат виготовлення (Kosten) продукту належать усі
складові частини його вартості, які капіталіст оплатив або еквівалент
яких він кинув у виробництво. Ці витрати мусять бути
заміщені для того, щоб капітал просто зберігся або репродукувався
в своїй первісній величині.

Вартість, яка міститься в товарі, дорівнює тому робочому
часові, якого коштує його виготовлення, а сума цієї праці складається
з оплаченої і неоплаченої праці. Навпаки, для капіталіста
витрати виготовлення товару складаються тільки з тієї
