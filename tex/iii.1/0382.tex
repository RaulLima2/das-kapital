покриваються, вони абсолютно функціонують як гроші, бо при
цьому не відбувається кінцевого перетворення їх у гроші. Подібно
до того як ці взаємні авансування, виробників і купців
один одному становлять власне основу кредиту, так і їх знаряддя
циркуляції, вексель, становить базу власне кредитних
грошей, банкнот і т. д. Ці останні грунтуються не на грошовій
циркуляції металічних грошей або державних паперових грошей,
а на вексельній циркуляції.

W. Leatham (банкір у Йоркшірі): „Letters on the Currency“, 2 вид., Лондон 1840:
„Я думаю, що загальна сума векселів за весь 1839 рік становила 528 493 842 фунтів
стерлінгів“ [він рахує суму закордонних векселів приблизно в 1/6 всієї суми],
„а сума векселів, які одночасно циркулювали в тому самому році, становила
132 123 460 фунтів стерлінгів“ (стор. 56). „Векселі є складова частина циркуляції,
розміром своїм більша, ніж усі інші частини, разом узяті“ (стор. З [4]). — „Ця
величезна надбудова з векселів грунтується (!) на основі, утвореній сумою
банкнот і золота; і якщо в ході подій ця основа занадто звужується, її міцності
і навіть її існуванню загрожує небезпека“ (стор. 8). — „Якщо оцінити всю
циркуляцію“ [він має на думці банкноти] „і суму зобов’язань усіх банків,
по яких може постати потреба негайного платежу готівкою, то я думаю, що
це становитиме суму в 153 мільйони, перетворення якої в золото можна вимагати
за законом, а для задоволення цієї вимоги є тільки 14 мільйонів золотом“
(стор. 11). — „Векселі не можуть бути поставлені під контроль, хіба тільки
коли буде ужито заходів проти надміру грошей і низького розміру процента
або дисконту, який породжує частину цих векселів і підохочує до цього великого
й небезпечного поширення їх. Неможливо встановити, скільки з них виникло
з дійсних операцій, наприклад, з дійсних купівель і продажів, і яка частина їх
штучно зроблена (fictitious) і складається тільки з бронзових векселів (Reitwechseln) *, тобто коли
вексель видається, щоб замінити поточний вексель до прострочення платежу і таким чином створити
фіктивний капітал за допомогою утворення простих засобів циркуляції. Мені відомо, що за часів, коли
грошей є надмірна кількість і коли вони дешеві, це практикується в колосальних розмірах“ (стор. 43,
44). J. W. Bosanquet: „Metallic, Paper, and Credit Currency“, London 1842: „Пересічна сума платежів,
зроблених у розрахунковій палаті [де лондонські банкіри взаємно обмінюються оплаченими чеками й
векселями, яким надійшов строк платежу], становить у кожний операційний день понад 3 мільйони фунтів
стерлінгів, а потрібний для цієї мети щоденний грошовий запас — дещо більший,
ніж 200 000 фунтів стерлінгів“ (стор. 86). [В 1889 році загальний оборот розрахункової палати
становив 7618 3/4 мільйонів фунтів стерлінгів або, при круглому числі в 300 операційних днів,
пересічно 251/2 мільйонів щоденно. — Ф. Е.]. „Векселі є, безперечно, засіб циркуляції (currency),
незалежно від грошей,
оскільки вони передають власність з рук у руки за допомогою передатного
напису“ (стор. 92 [93]). „Пересічно можна вважати, що кожний вексель,
який перебуває в циркуляції, має на собі два передатні написи і що кожний
вексель, таким чином, до скінчення його строку пересічно покриває два
платежі. Таким чином, можна вважати, що за допомогою самих тільки передатних
написів на протязі 1839 року векселі передали з рук у руки власність
на суму вдвоє більшу за 528 мільйонів, тобто на 1056 мільйонів фунтів стерлінгів,
— більше, ніж 3 мільйони щоденно. Тому можна з певністю сказати, що
векселі і вклади, разом узяті, шляхом передачі власності з рук у руки і без
допомоги грошей виконують грошові функції на суму щонайменше у 18 мільйонів
фунтів стерлінгів щоденно“ (стор. 93).

Про кредит взагалі Тук каже таке: „У своєму найпростішому виразі кредит
є добре чи погано обгрунтоване довір’я, завдяки якому одна особа довіряє
другій певну суму капіталу, в грошах, або в товарах, оцінених у певній гро-

* Фіктивний вексель, який не відбиває ніякої дійсної операції, чи то позики,
чи купівлі-продажу в кредит. Ред. укр. перекладу.
