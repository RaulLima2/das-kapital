\parcont{}  %% абзац починається на попередній сторінці
\index{iii1}{0167}  %% посилання на сторінку оригінального видання
виробництва містять у собі зиски від B, C, D, так само як витрати виробництва B, C, D і т. д., в
свою чергу, містять у собі
зиск від A. Отже, якщо зробимо підрахунок, то зиску від А не
буде в його власних витратах виробництва, і так само зисків
від В, C, D і т. д. не буде в їх власних витратах виробництва.
Ніхто не залічує свого власного зиску до своїх витрат виробництва. Отже, якщо є, наприклад, n сфер
виробництва, і в кожній з них добувається зиск, рівний p, то витрати виробництва в усіх них разом =
k — np. Отже, розглядаючи весь обрахунок
в цілому, ми бачимо, що оскільки зиски однієї сфери виробництва входять у витрати виробництва іншої,
остільки ці зиски
введені вже в обрахунок для загальної ціни остаточного кінцевого
продукту і не можуть вдруге з’явитись у графі зиску. Якщо ж
вони з’являються в цій графі, то тільки тому, що сам даний
товар був остаточним продуктом, отже, ціна його виробництва
не входить у витрати виробництва іншого товару.

Якщо у витрати виробництва якогось товару входить сума = p, яка становить зиск виробників засобів
виробництва, і якщо
на ці витрати виробництва набавляється зиск, = p1, то весь зиск
P = p + p1. Загальні витрати виробництва товару, якщо абстрагуватись від усіх частин ціни, що
входять у склад зиску, дорівнюють таким чином його власним витратам виробництва мінус P.
Якщо ці витрати виробництва назвемо k, то, очевидно, k + P = k + p + p1. При дослідженні додаткової
вартості в книзі I,
розд. VII, 2, стор. 229*, ми бачили, що продукт кожного капіталу можна розглядати таким чином, ніби
одна його частина
тільки заміщає капітал, а друга тільки виражає додаткову вартість. Застосовуючи це обчислення до
сукупного продукту
суспільства, ми повинні зробити певні поправки, бо, якщо
розглядати суспільство в цілому, зиск, вміщений, наприклад,
у ціні льону, не може фігурувати двічі — як частина ціни полотна
і разом з тим як частина зиску виробника льону.

Між зиском і додатковою вартістю немає ріжниці, оскільки
додаткова вартість, наприклад, капіталіста А входить у сталий
капітал В. Адже для вартості товарів зовсім не має значення, чи
складається вміщена в них праця з оплаченої чи неоплаченої праці.
Це показує тільки, що В оплачує додаткову вартість А. В загальному підсумку додаткову вартість А не
можна рахувати двічі.

Але ріжниця полягає ось у чому. Крім того, що ціна продукту,
виробленого, наприклад, капіталом В, відхиляється від його вартості, бо реалізована в В додаткова
вартість може бути більша
або менша, ніж зиск, доданий в ціні продуктів В, ця сама обставина
знов таки має силу й для тих товарів, що становлять сталу
частину капіталу В, а посередньо — як засоби існування робітників — і його змінну частину. Щодо
сталої частини, то вона

* Стор. 153—154 рос. вид. 1935 р. Ред. укр. перекладу.
\parbreak{}  %% абзац продовжується на наступній сторінці
