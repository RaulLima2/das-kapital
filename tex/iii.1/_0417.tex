\parcont{}  %% абзац починається на попередній сторінці
\index{iii1}{0417}  %% посилання на сторінку оригінального видання
в одних суперечність усунена негативно, а в других — позитивно.
Досі ми розглядали розвиток кредитної системи — і вміщене
в ній приховане скасування (Aufhebung) капіталістичної власності — головним чином щодо промислового
капіталу. В дальших розділах ми розглядаємо кредит щодо капіталу, що дає процент, як
такого, а також його вплив на цей останній, як і ту форму, що
її він при цьому набирає; при цьому нам взагалі доведеться ще
зробити декілька зауважень специфічно економічного характеру.

Але попереду ще таке:

Якщо кредитна система виступає як головна підойма перепродукції та надмірної спекуляції в торгівлі,
то тільки тому,
що процес репродукції, який по своїй природі є еластичний,
форсується тут до крайніх меж, і при тому форсується тому,
що значна частина суспільного капіталу застосовується його
невласниками, які через це пускаються в справи цілком інакше,
ніж власник, який, оскільки він функціонує сам, боязливо зважує
межі свого приватного капіталу. Це тільки показує, що зростання вартості капіталу, основане на
антагоністичному характері
капіталістичного виробництва, допускає дійсний, вільний розвиток тільки до певного пункту, отже, в
дійсності утворює для виробництва імманентні окови й межі, які постійно прориваються
кредитною системою.\footnote{
Th. Chalmers. [„On Political Economy etc.“ London 1832.)
} Тому кредитна система прискорює матеріальний розвиток продуктивних сил і
утворення світового ринку,
доведення яких як матеріальних основ нової форми виробництва
до певного ступеня розвитку є історичним завданням капіталістичного способу виробництва. Одночасно
кредит прискорює насильні вибухи цієї суперечності, кризи, і тим самим посилює елементи розкладу
старого способу виробництва.

Імманентний кредитній системі двобічний характер: з одного
боку, розвивати рушійну силу капіталістичного виробництва,
збагачення експлуатацією чужої праці, до найчистішої і найколосальнішої системи гри й шахрайства і
дедалі більше обмежувати число тих небагатьох, що експлуатують суспільне багатство; а з другого
боку, становити перехідну форму до нового
способу виробництва, — ця двобічність і надає головним провісникам кредиту від Ло до Ісаака Перейри
їхнього приємного мішаного характеру шахрая і пророка.
