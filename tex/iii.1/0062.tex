тість, нічого не змінює ні в величині, ні в природі додаткової
вартості. В дійсному процесі циркуляції не тільки відбуваються
перетворення, які ми розглянули в книзі II, але вони збігаються
з дійсною конкуренцією, з купівлею і продажем товарів вище
або нижче їх вартості, так що для окремого капіталіста реалізована
ним самим додаткова вартість залежить так само від
взаємного ошукування, як і від безпосередньої експлуатації
праці.

В процесі циркуляції поряд робочого часу починає діяти час
циркуляції, який цим самим обмежує масу додаткової вартості,
яку можна реалізувати за певний період. На безпосередній
процес виробництва впливають визначально ще й інші моменти,
які виникають з циркуляції. І те і друге, безпосередній
процес виробництва і процес циркуляції, постійно переходять
один в один, пронизують один одного, і тим самим постійно
перекручують свої характерні відмінні ознаки. Виробництво додаткової
вартості, як і вартості взагалі, набуває в процесі циркуляції,
як показано вище, нових визначень; капітал перебігає
круг своїх перетворень; нарешті, він вступає з свого, так би
мовити, внутрішнього органічного життя в зовнішні життьові
відносини, у відносини, де один одному протистоять не капітал
і праця, а з одного боку капітал і капітал, з другого боку
індивіди знов таки просто як покупці і продавці; час циркуляції
і робочий час перехрещуються на своєму шляху, і таким
чином здається, ніби вони обидва в однаковій мірі визначають
додаткову вартість; та первісна форма, в якій протистоять один
одному капітал і наймана праця, замасковується в наслідок втручання
відносин, які, як здається, незалежні від неї; сама додаткова
вартість здається не продуктом привласнення робочого часу,
а надлишком продажної ціни товарів понад витрати їх виробництва,
в наслідок чого витрати виробництва легко можуть здаватися
дійсною вартістю (vaLeur intrinsèque) товарів, так що зиск
здається надлишком продажної ціни товарів понад їх імманентну
вартість.

Правда, під час безпосереднього процесу виробництва природа
додаткової вартості постійно доходить до свідомості капіталіста,
як це вже при розгляді додаткової вартості показала
нам його жадоба до чужого робочого часу і т. д. Але: 1) сам
безпосередній процес виробництва є тільки минущий момент,
який постійно переходить у процес циркуляції, як і цей останній
переходить у нього, так що ясніше чи туманніше проблискуюча
в процесі виробництва догадка про джерело здобутого у ньому
баришу, тобто про природу додаткової вартості, щонайбільше
виступає як момент рівноправний з тим уявленням, ніби реалізований
надлишок походить від руху, який не залежить від
процесу виробництва і виникає з самої циркуляції, отже, руху,
який належить капіталові незалежно від його відношення до
праці. Адже навіть сучасними економістами, як от Рамсей,
