\parcont{}  %% абзац починається на попередній сторінці
\index{iii1}{0362}  %% посилання на сторінку оригінального видання
чином були б примушені знову перетворитись у промислових
капіталістів. Але, як уже сказано, для окремого капіталіста
це — факт. Тому він, навіть коли господарює власним капіталом,
необхідно розглядає ту частину свого пересічного зиску,
яка дорівнює пересічному процентові, як плід свого капіталу
як такого, незалежно від процесу виробництва; і в протилежність
до цієї частини, яка усамостійнилась у проценті, він розглядає
надлишок гуртового зиску понад процент просто як підприємницький
дохід.

Почетверте: [Прогалина у рукопису].

Отже, виявилось, що та частина зиску, яку функціонуючий
капіталіст повинен сплачувати простому власникові взятого в позику
капіталу, перетворюється в самостійну форму тієї частини
зиску, яку дає під назвою процента всякий капітал як
такий, однаково, чи взято його в позику, чи ні. Величина цієї
частини залежить від висоти пересічного розміру процента.
На її походження вказує тепер тільки те, що функціонуючий
капіталіст, оскільки він є власником свого капіталу, не конкурує
— принаймні не конкурує активно — при визначенні розміру
процента. Чисто кількісний поділ зиску між двома особами, які
мають різні юридичні титули на нього, перетворився в якісний
поділ, який, як здається, виникає з самої природи капіталу
й зиску. Бо, як ми бачили, коли частина зиску взагалі набирає
форми процента, ріжниця між пересічним зиском і процентом,
або надлишкова понад процент частина зиску, перетворюється
в протилежну процентові форму, в форму підприємницького
доходу. Ці дві форми, процент і підприємницький дохід, існують
тільки у своїй протилежності. Отже, обидві вони стоять у певному
співвідношенні не з додатковою вартістю, що тільки частинами
її, фіксованими в різних категоріях, рубриках і назвах,
вони є, а в співвідношенні одна з одною. Через те що одна
частина зиску перетворюється в процент, друга частина виступає
як підприємницький дохід.

Під зиском ми завжди розуміємо тут пересічний зиск, бо
відхилення як індивідуального зиску, так і зиску в різних сферах
виробництва, — отже, ті чи інші зміни в розподілі пересічного
зиску або додаткової вартості, які відбуваються в зв’язку
з конкурентною боротьбою та іншими обставинами, — для нас
тут цілком не мають значення. Це взагалі стосується до всього
цього дослідження.

Отож, процент, як його визначає Рамсей, є чистий зиск, що
його дає власність на капітал як така, чи то простому позикодавцеві,
який лишається поза процесом репродукції, чи власникові,
який сам продуктивно застосовує свій капітал. Але й цьому
останньому капітал дає цей чистий зиск не як функціонуючому
капіталістові, а як грошовому капіталістові, який позичив свій
власний капітал як капітал, що дає процент, самому собі як
функціонуючому капіталістові. Подібно до того, як перетворення
\parbreak{}  %% абзац продовжується на наступній сторінці
