тової торгівлі, а найважливіше значення для банків завжди мають вклади. Найкращий доказ цьому дають
шотландські банки.

Для нашої мети немає потреби докладніше розглядати особливі види кредитних установ, як і особливі
форми самих банків.

„Банкірська справа двояка... 1) Збирати капітал від тих, хто не знаходить
для нього безпосереднього застосування, і розподіляти й передавати його
тим, хто може його використати. 2) Приймати вклади з доходів своїх клієнтів
і виплачувати їм суми в міру того, як вони їм стають потрібні для видатків
на предмети споживання. Перше є циркуляція капіталу, останнє — циркуляція
грошей (currency)“. — „Перше є концентрація капіталу, з одного боку, і розподіл його, з другого
боку, друге є управління циркуляцією для місцевих
цілей округи“. — Tooke: „Inquiry into the Currency Principle“, стор. 36, 37. Ми
повернемось до цього місця в XXVIII розділі.

„Reports of Committees“, т. VIII. „Commercial Distress“, т. І, частина I, 1847—48.
Minutes of Evidence. — (Далі цитується як: „Commercial Distress“, 1847—48). В сорокових роках при
дисконті векселів у Лондоні в незчисленних випадках замість банкнот приймали векселі одного банку на
другий строком на 21 день.
(Свідчення J. Pease, провінціального банкіра, № 4636 і 4645). Згідно з тим самим звітом, банкіри
мали звичай, коли грошей ставало мало, регулярно платити
своїм клієнтам такими векселями. Якщо одержувач хотів банкнот, то він мусив
знов дисконтувати цей вексель. Для банків це дорівнювало привілеєві робити
гроші. Пани Jones, Loyd and С° платили таким способом „з незапам’ятних часів“,
якщо грошей ставало мало і розмір процента перевищував 5\%. Клієнт був
радий одержати такі Bankers Bills [банківські векселі], бо векселі Jones, Loyd
and С° можна було легше дисконтувати, ніж свої власні; вони часто переходили також через 20—30 рук
(там же, № 901—905, 992).

Всі ці форми служать для того, щоб зробити передаваним право на одержання платежу. „Навряд чи існує
яка-небудь форма кредиту, в якій йому часами
не доводилося б виконувати функції грошей; чи є цією формою банкнота, чи
вексель, чи. чек, процес по суті є той самий і результат по суті є той самий“. — Fullarton: „On the
Regulation of Currencies“, 2 вид., Лондон 1845, стор. 38. — „Банкноти — дрібні кредитні гроші“
(стор. 51).

Нижченаведене з J. W. Gilbart: „The History and Principles of Banking“,
London 1834: „Капітал банку складається з двох частин — з основного капіталу
(invested capital) і банкового капіталу (banking capital), взятого в позику“ (стор. 117).
„Банковий капітал або капітал, взятий у позику, одержується трьома шляхами:
1) прийманням вкладів, 2) випуском власних банкнот, 3) видачею векселів. Якщо
хтонебудь схоче позичити мені задарма 100 фунтів стерлінгів, а я позичу ці
100 фунтів стерлінгів комусь іншому по 4\% то за рік я на цій справі одержу
4 фунти стерлінгів доходу. Так само якщо хтонебудь схоче взяти моє платіжне
зобов’язання“ (I promise to pay [я обіцяю заплатити] — звичайна формула англійських банкнот) „і
наприкінці року поверне мені його, заплативши мені за це 4\%, цілком так само, як коли б я позичив
йому 100 фунтів стерлінгів, то я на цій справі одержу 4 фунти стерлінгів доходу; і, далі, якщо
хто-небудь у провінціальному місті принесе мені 100 фунтів стерлінгів з умовою, щоб я через 21 день
заплатив цю суму в Лондоні третій особі, то всякий процент, який я за проміжний час зможу одержати
на ці гроші, буде моїм зиском. До цього по суті справи зводяться операції банку і той шлях, яким
утворюється банковий капітал за допомогою вкладів, банкнот і векселів“ (стор. 117). „Загалом зиски
банкіра пропорційні сумі одержаного ним у позику капіталу або банкового капіталу. Щоб визначити
дійсний зиск банку, треба з гуртового зиску відняти процент на основний капітал. Остача є банковий
зиск“ (стор. 118). „Позики банкіра своїм клієнтам робляться грішми інших людей“ (стор. 146). „Саме
ті банкіри, які не випускають банкнот, створюють банковий капітал за допомогою дисконтування
векселів. Вони збільшують свої вклади за допомогою своїх дисконтних операцій. Лондонські банкіри
дисконтують векселі тільки для тих фірм, які мають у них рахунок вкладів“ (стор. 119). „Фірма, яка
дисконтує векселі у своєму банку і платить проценти на всю суму цих векселів, мусить принаймні
частину цієї суми залишити
