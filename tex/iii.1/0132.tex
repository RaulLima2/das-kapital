ливих і несприятливих років, звичайно, знов таки приводить до
здешевлення сировинного матеріалу. Незалежно від того безпосереднього
впливу, який ця обставина справляє на розширення
попиту, сюди долучається ще як стимул вищезгаданий
вплив на норму зиску. І зазначений вище процес ступневого
випереджання виробництва сировинних матеріалів виробництвом
машин і т. д. повторюється тоді в ширшому масштабі. Дійсне
поліпшення сировинного матеріалу, так щоб він постачався не
тільки в потрібній кількості, але й потрібної якості, наприклад,
бавовна американської якості з Індії, вимагало б тривалого, регулярно
зростаючого і постійного попиту з боку Европи (цілком
залишаючи осторонь ті економічні умови, в які поставлений індійський
виробник на своїй батьківщині). Але при таких умовах
сфера виробництва сировинних матеріалів змінюється тільки
стрибками, то раптом розширюється, то знову дуже скорочується.
Все це, як і дух капіталістичного виробництва взагалі, можна
дуже добре вивчати на бавовняному голоді 1861—1865 років, до
якого долучалася ще й та обставина, що часами зовсім не було
сировинного матеріалу, одного з найістотніших елементів репродукції.
Ціна може підвищуватись навіть і тоді, коли подання
цілком достатнє, але достатнє при тяжчих умовах. Або може
мати місце справжня недостача сировинного матеріалу. Під час
бавовняної кризи спочатку мала місце така недостача сировинного
матеріалу.

Отже, чим більше ми наближаємось в історії виробництва
до безпосередньої сучасності, тим регулярніше ми знаходимо,
особливо у вирішальних галузях виробництва, постійне повторення
чергувань відносного подорожчання і виникаючого з нього
пізнішого знецінення сировинних матеріалів органічного походження.
Ілюстрації до вищесказаного дано в наведених нижче
прикладах, взятих із звітів фабричних інспекторів.

Мораль історії, яку можна здобути також з дослідження землеробства
взагалі, полягає в тому, що капіталістична система
протидіє раціональному землеробству, або що раціональне землеробство
несполучне з капіталістичною системою (хоч ця остання
і сприяє його технічному розвиткові) і потребує або руки самостійно
працюючого дрібного селянина, або контролю асоційованих
виробників.

Тепер ми наводимо щойно згадані ілюстрації з англійських
фабричних звітів.

експорту. Подруге, в картелях (trusts) фабрикантів цілих великих сфер виробництва
для регулювання виробництва і разом з тим цін і зисків. Само собою
зрозуміло, що ці експерименти здійснимі тільки при відносно сприятливій
економічній погоді. Перша ж буря повинна їх зруйнувати і довести, що, хоч
виробництво і потребує регулювання, але, без сумніву, не капіталістичний клас
покликаний здійснити його. Покищо ці картелі мають лиш одну мету —
дбати про те, щоб дрібні капіталісти пожиралися великими ще швидше, ніж
досі. — Ф. Е.
