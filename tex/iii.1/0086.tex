2) при однаковій нормі додаткової вартості і неоднаковому
процентному складі, при чому норми зиску відносяться одна до
одної, як змінні частини капіталу;

3) при неоднаковій нормі додаткової вартості і неоднаковому
процентному складі, при чому норми зиску відносяться одна до
одної, як добутки m'v, тобто як маси додаткової вартості,
взяті в процентному відношенні до всього капіталу.\footnote{
В рукопису є ще дуже докладні обчислення щодо різності між нормою
додаткової вартості і нормою зиску (m' — р'); ця різність має різні цікаві особливості,
рух її показує випадки, коли обидві норми віддаляються одна від
одної або наближаються одна до одної. Ці рухи можуть бути зображені у формі
кривих. Я відмовляюсь від відтворення цього матеріалу, бо для ближчих цілей
цієї книги він менш важливий; тут досить просто звернути на це увагу тих
читачів, які захочуть далі простежити це питання. — Ф. Е.
}

Розділ четвертий

Вплив обороту на норму зиску

[Вплив обороту на виробництво додаткової вартості, отже
й зиску, з’ясовано в другій книзі. Його можна коротко зрезюмувати
таким чином, що в наслідок того, що на оборот потрібен
певний час, на виробництво не може бути застосований одночасно
весь капітал; що, отже, частина капіталу постійно лежить без
діла, чи то в формі грошового капіталу, запасних сировинних
матеріалів, готового, але ще не проданого товарного капіталу,
чи в формі боргових вимог, для яких ще не настав строк платежу;
що капітал, який діє в активному виробництві, тобто при
створенні і привласненні додаткової вартості, постійно зменшується
на цю частину, при чому в такій самій пропорції постійно
зменшується створювана і привласнювана додаткова вартість.
Чим коротший час обороту, тим меншою порівняно з усім
капіталом стає ця частина капіталу, яка лежить без діла; і тим
більшою, отже, стає, при інших незмінних умовах, привласнювана
додаткова вартість.

Уже в другій книзі докладно розвинуто, як скорочення часу
обороту або одного з двох його підрозділів, часу виробництва
і часу циркуляції, підвищує масу вироблюваної додаткової вартості.
Але через те що норма зиску виражає тільки відношення
виробленої маси додаткової вартості до всього капіталу, занятого
в її виробництві, то очевидно, що всяке таке скорочення підвищує
норму зиску. Те, що раніше, в другому відділі другої
книги, розвинуто щодо додаткової вартості, в такій самій мірі
стосується і до зиску та норми зиску і не потребує тут повторення.
Ми хочемо відзначити лиш декілька головних моментів.

Головний засіб скорочення часу виробництва є підвищення
продуктивності праці, що звичайно називають прогресом промисловості.
Якщо цим одночасно не викликається значне збіль-