в руках банку, не одержуючи за неї процентів. Таким шляхом банкір одержує на гроші, які він дає в
позику, вищий за поточний розмір процента і
створює собі банковий капітал за допомогою тієї остачі, що лишається в його руках“ (стор. 120). —
Економія резервних фондів, вкладів, чеків: „Депозитні банки за
допомогою трансферту активів заощаджують на вживанні засобів циркуляції і з
незначною сумою дійсних грошей виконують операції на великі суми. Звільнені таким чином гроші
вживаються банкіром на позики своїм клієнтам за
допомогою дисконту і т. д. Тому трансферт активів підвищує діяльність депозитної системи“ (стор.
123). „Чи мають обидва клієнти, які ведуть між собою
справи, свої рахунки в одного й того ж банкіра, чи в різних банкірів, — це
однаково. Бо банкіри обмінюють між собою свої чеки в Clearing House [розрахунковій палаті]. За
допомогою трансферту депозитна система, могла б
таким чином досягти такого ступеня поширення, що зовсім витиснула б
уживання металічних грошей. Якби кожен мав у банку рахунок вкладів і робив би всі свої платежі за
допомогою чеків, то ці чеки стали б єдиним засобом циркуляції. В цьому випадку довелося б
припускати, що банкіри мають
гроші в своїх руках, інакше чеки не мали б ніякої цінності“ (стор. 124). Централізація місцевого
обігу в руках банків проводиться за допомогою 1) філіальних банків. Провінціальні банки мають філії
в невеличких містах своєї округи;
лондонські банки — в різних частинах міста Лондона; 2) за допомогою агентур.
„Кожний провінціальний банк має агента в Лондоні, щоб там оплачувати свої
банкноти або векселі і одержувати гроші, що їх сплачують жителі Лондона за рахунок осіб, які живуть
у провінції“ (стор. 127). Кожний банкір збирає банкноти
іншого банку і вже не видає їх. В кожному більш-менш великому місті
вони сходяться раз або двічі на тиждень і обмінюються банкнотами. Сальдо
виплачується переказом на Лондон (стор. 134). „Мета банків — полегшувати
справи. Все, що полегшує справи, полегшує і спекуляцію. Справи і спекуляція
в багатьох випадках так тісно зв’язані між собою, що важко сказати, де кінчаються справи і
починається спекуляція... Всюди, де є банки, капітал можна
одержувати легше й дешевше. Дешевина капіталу дає поштовх спекуляції,
подібно до того, як дешевина м’яса й пива дає поштовх прожерливості й пияцтву“ (стор. 137, 138).
„Тому що банки, які випускають власні банкноти, завжди
платять цими банкнотами, то може здатися, що їх дисконтні операції робляться
виключно за допомогою капіталу, одержаного таким способом; але це не так.
Банкір, звичайно, може всі дисконтовані ним векселі оплатити своїми власними
банкнотами, і все ж 9/10 векселів, які є в його портфелі, можуть репрезентувати дійсний капітал. Бо,
хоч він сам за ці векселі видав тільки свої власні паперові гроші, вони зовсім не повинні
обов’язково лишатись у циркуляції до
скінчення строку векселів. Векселі можуть мати тримісячний строк, а банкноти
можуть повернутися через три дні“ (стор. 172). „Покриття рахунку клієнтами
становить нормальну ділову операцію. Це в дійсності та мета, для якої
гарантується кредит готівкою... Кредити готівкою забезпечуються не тільки
особистою гарантією, але й вкладом цінних паперів“ (стор. 174, 175). „Капітал,
даний у позику під заставу товарів, справляє той самий вплив, що й капітал,
даний у позику під дисконт векселів. Якщо хтонебудь бере в позику 100 фунтів
стерлінгів під забезпечення своїми товарами, то це те саме, як коли б він продав
їх за вексель у 100 фунтів стерлінгів і дисконтував би його в банкіра.
Але позика дає йому змогу придержати свої товари до кращого стану ринку і
уникнути жертв, яких інакше йому довелося б зазнати, щоб одержати гроші
на невідкладні справи“ (стор. 180, 181).

„The Currency Theory Reviewed etc.“ [Edinburgh 1845] стор. 62, 63: „Незаперечна правда, що 1000
фунтів стерлінгів, які я сьогодні вклав до А, завтра
знову будуть видані і стануть вкладом у В. Позавтра вони знову можуть бути
видані В, стати вкладом у С, і так далі до безконечності. Отже, ті самі
1000 фунтів стерлінгів грошей можуть за допомогою ряду передач помножитись до абсолютно невизначної
суми вкладів. Тому можливо, що дев’ять десятих усіх вкладів в Англії зовсім не існують поза
бухгалтерськими рубриками
в книгах банкірів, кожен з яких відповідає за свою частину... Так, у Шотландії, де гроші, які є в
циркуляції“ [до того ж майже виключно паперові гроші!],
„ніколи не перевищують 3 мільйонів фунтів стерлінгів, вклади становлять 27 міль-
