\parcont{}  %% абзац починається на попередній сторінці
\index{iii1}{0252}  %% посилання на сторінку оригінального видання
коротко кажучи, створювати штучне перенаселення. Далі, знецінення
елементів сталого капіталу само стало б елементом,
який включав би підвищення норми зиску. Маса застосовуваного
сталого капіталу зросла б порівняно з змінним, але вартість
цієї маси могла б зменшитись. Насталий застій виробництва
підготував би пізніше розширення виробництва в капіталістичних
межах.

І таким чином круг пророблювався б знову. Частина капіталу,
яка в наслідок припинення функціонування знецінилась,
знову набула б своєї колишньої вартості. Зрештою, при розширених
умовах виробництва, при розширеному ринку і при підвищеній
продуктивній силі був би знову пророблений той самий
порочний кругобіг.

Але навіть при зробленому нами крайньому припущенні абсолютна
перепродукція капіталу не є абсолютна перепродукція
взагалі, абсолютна перепродукція засобів виробництва. Вона
є перепродукція засобів виробництва лиш остільки, оскільки ці
останні \emph{функціонують як капітал}, і тому — відносно їх вартості,
яка збільшується разом із збільшенням їх маси — передбачають
зростання цієї вартості, повинні породжувати додаткову
вартість.

Але, не зважаючи на те, це все ж була б перепродукція,
бо капітал був би неспроможний експлуатувати працю в тому
ступені, який зумовлюється „здоровим“, „нормальним“ розвитком
капіталістичного процесу виробництва, в тому ступені експлуатації,
при якому із зростанням маси застосовуваного капіталу
збільшується принаймні маса зиску і який, отже, виключає
можливість того, щоб норма зиску падала в тій самій мірі,
в якій зростає капітал, і, особливо, щоб норма зиску падала
швидше, ніж зростає капітал.

Перепродукція капіталу ніколи не означає нічого іншого, як
перепродукцію засобів виробництва, — засобів праці і засобів
існування, — які можуть функціонувати як капітал, тобто можуть
бути застосовані для експлуатації праці в даному ступені експлуатації;
а падіння цього ступеня експлуатації нижче певного
даного пункту викликає порушення й застої капіталістичного процесу
виробництва, кризи, руйнування капіталу. Немає ніякої суперечності
в тому, що ця перепродукція капіталу супроводиться
більш чи менш значним відносним перенаселенням. Ті самі обставини,
які підвищили продуктивну силу праці, збільшили масу
товарних продуктів, розширили ринки, прискорили нагромадження
капіталу, як щодо маси, так і щодо вартості, і знизили норму
зиску, — ці самі обставини породили і постійно породжують відносне
перенаселення, перенаселення робітників, які не вживаються
надлишковим капіталом в наслідок низького ступеня експлуатації
праці, при якому вони тільки й могли б бути вжиті, або, принаймні,
в наслідок низької норми зиску, яку вони давали б при даному
ступені експлуатації.
