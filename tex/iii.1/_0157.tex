\parcont{}  %% абзац починається на попередній сторінці
\index{iii1}{0157}  %% посилання на сторінку оригінального видання
дорожчі, ніж в $А$. В цьому випадку на 100 фунтів стерлінгів
змінного капіталу в $А$ припадало б, наприклад, 200 фунтів
стерлінгів сталого капіталу, а в $В$ 400. Тоді при нормі додаткової вартості в 100\% вироблена
додаткова вартість в обох випадках дорівнює 100 фунтам стерлінгів; отже й зиск в обох випадках
дорівнює 100 фунтам стерлінгів. Але в $А$ $\frac{100}{200 c + 100 v} =
\sfrac{1}{3} = 33 \sfrac{1}{3}\%$, тимчасом як в $В$ $\frac{100}{400 c + 100 v }= \sfrac{1}{5}=20\%$. Дійсно,
якщо ми в обох випадках візьмемо певні відповідні частини
всього капіталу, то в $В$ з кожних 100 фунтів стерлінгів тільки
20 фунтів стерлінгів, або \sfrac{1}{5}, становить змінний капітал, тимчасом як в $А$ з кожних 100 фунтів
стерлінгів 33\sfrac{1}{3} фунтів стерлінгів, або \sfrac{1}{3} становить змінний капітал. $В$ виробляє на кожні 100
фунтів стерлінгів менше зиску, бо приводить в рух менше
живої праці, ніж $А$. Отже, ріжниця норм зиску зводиться тут
знов таки до ріжниці мас зиску, вироблених на кожні 100 одиниць вкладеного капіталу, бо маси зиску
тотожні з масами додаткової вартості.

Ріжниця цього другого прикладу від попереднього є тільки
така: в другому випадку вирівнення між $А$ і $В$ вимагало б тільки
зміни вартості сталого капіталу, чи то в $А$, чи то в $В$, при незмінній технічній базі; навпаки, в
першому випадку сам технічний склад в обох сферах виробництва є різний і для вирівнення він мусив би
зазнати перетворення.

Отже, різний органічний склад капіталів не залежить від їх
абсолютної величини. Питання завжди тільки в тому, скільки
з кожних 100 одиниць є змінного капіталу і скільки сталого.

Отож, капітали різної в процентному обчисленні величини
або, що зводиться до того самого, капітали однакової величини
створюють при однаковому робочому дні і однаковому ступені
експлуатації праці дуже різні кількості зиску, бо створюють
дуже різні кількості додаткової вартості, і це саме тому, що
залежно від різного органічного складу капіталів у різних сферах виробництва їх змінна частина є
різна, отже, різні й кількості живої праці, яку вони приводять в рух, отже й кількості
привласнюваної ними додаткової праці, — субстанції додаткової
вартості, а тому й зиску. Рівновеликі частини всього капіталу
в різних сферах виробництва містять у собі нерівновеликі джерела додаткової вартості, а єдиним
джерелом додаткової вартості є жива праця. При однаковому ступені експлуатації праці маса праці,
приведеної в рух капіталом, рівним 100, а тому й
маса привласнюваної ним додаткової праці, залежить від величини
його змінної складової частини. Коли б капітал, який в процентах складається з $90 c + 10 v$, при
однаковому ступені експлуатації праці виробляв стільки ж додаткової вартості або зиску, як капітал,
що складається з $10 c + 90 v$, то було б ясно, як день,
що додаткова вартість, а тому й вартість взагалі мусять мати
\parbreak{}  %% абзац продовжується на наступній сторінці
