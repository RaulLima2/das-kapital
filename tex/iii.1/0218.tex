з двох мільйонів до трьох. Проте, не зважаючи на це зростання
абсолютної маси додаткової праці, а тому й додаткової
вартості на 50\%, відношення змінного капіталу до сталого
впало б з 2 : 4 до 3 : 15, і відношення додаткової вартості до
всього капіталу було б таке (в мільйонах):

I. 4с + 2v + 2m; K = 6, p' = 33 1/3\%.

II. 15с + 3v' + 3m; K = 18, p' = 16 2/3\%.

Тимчасом як маса додаткової вартості підвищилась наполовину,
норма зиску впала наполовину порівняно з попередньою.
Але зиск є тільки додаткова вартість, обчислена на суспільний
капітал, і тому маса зиску, його абсолютна величина, розглядувана
з точки зору всього суспільства, дорівнює абсолютній величині
додаткової вартості. Отже, абсолютна величина зиску, його
сукупна маса, зросла б на 50\%, не зважаючи на величезне зменшення
цієї маси зиску відносно авансованого сукупного капіталу
або не зважаючи на величезне зменшення загальної норми зиску.
Отже, число вживаних капіталом робітників, тобто абсолютна
маса праці, яка ним приводиться в рух, тому й абсолютна маса
вбираної ним додаткової праці, тому й маса виробленої ним додаткової
вартості, тому й абсолютна маса виробленого ним зиску
може зростати і зростати прогресивно, не зважаючи на прогресивне
падіння норми зиску. Це не тільки може бути. Це — залишаючи
осторонь минущі коливання — мусить так бути на базі
капіталістичного виробництва.

Капіталістичний процес виробництва є разом з тим істотно і
процес нагромадження. Ми показали, як з розвитком капіталістичного
виробництва маса вартості, яка мусить бути просто
репродукована, збережена, збільшується і зростає разом з
підвищенням продуктивності праці, навіть якщо вживана робоча
сила лишається незмінною. Але з розвитком суспільної продуктивної
сили праці ще більше зростає маса вироблюваних споживних
вартостей, частину яких становлять засоби виробництва.
А добавна праця, через привласнення якої це додаткове багатство
може бути знову перетворене в капітал, залежить не від вартості,
а від маси цих засобів виробництва (включаючи й засоби
існування), бо в процесі праці робітник має справу не з вартістю,
а з споживною вартістю засобів виробництва. Однак, само
нагромадження і дана разом з ним концентрація капіталу є
матеріальний засіб підвищення продуктивної сили. Але це зростання
засобів виробництва передбачає зростання робітничого
населення, створення населення робітників, яке відповідає додатковому
капіталові і загалом і в цілому навіть завжди перевищує
його потреби, отже, створення перенаселення робітників.
Тимчасовий надлишок додаткового капіталу порівняно з робітничим
населенням, яке є в його розпорядженні, справляв би
двоякий вплив. З одного боку, він ступнево збільшував би ро-
