грошей і взагалі вартості в капітал є постійний результат капіталістичного
процесу виробництва, цілком так само буття їх
як капіталу є постійна передумова капіталістичного процесу
виробництва. Завдяки своїй здатності перетворюватись у засоби
виробництва, вони постійно командують над неоплаченою
працею і тому перетворюють процес виробництва й циркуляції
товарів у виробництво додаткової вартості для свого володільця.
Отже, процент є тільки вираз того, що вартість взагалі, —
упредметнена праця в її загальносуспільній формі, — вартість,
яка в дійсному процесі виробництва набирає вигляду засобів виробництва,
протистоїть живій робочій силі як самостійна сила
і є засобом привласнювати собі неоплачену працю; і що вона є такою
силою завдяки тому, що вона протистоїть робітникові як
чужа власність. Однак, з другого боку, в формі процента ця
протилежність найманій праці стерта; бо капітал, що дає процент,
як такий має своєю протилежністю не найману працю,
а функціонуючий капітал; капіталіст-позикодавець як такий прямо
протистоїть дійсно функціонуючому в процесі репродукції капіталістові,
а не найманому робітникові, у якого саме на основі
капіталістичного виробництва експропрійовано засоби виробництва.
Капітал, що дає процент — це капітал як власність у
протилежність до капіталу як функції. Але оскільки капітал
не функціонує, він не експлуатує робітників і не вступає в антагонізм
з працею.

З другого боку, підприємницький дохід становить протилежність
не до найманої праці, а тільки до процента.

Поперше: якщо припустити пересічний зиск як дану величину,
то норма підприємницького доходу визначається не заробітною
платою, а розміром процента. Вона буде вища чи нижча
у зворотному відношенні до розміру процента.73

Подруге: функціонуючий капіталіст виводить свою претензію
на підприємницький дохід, отже, і самий підприємницький дохід,
не з своєї власності на капітал, а з функції капіталу в протилежність
до тієї визначеності, в якій він існує тільки як бездіяльна
власність. Це виступає як безпосередньо наявна протилежність
у тих випадках, коли він оперує взятим у позику капіталом,
коли, отже, процент і підприємницький дохід дістаються
двом різним особам. Підприємницький дохід виникає з функції
капіталу в процесі репродукції, тобто в наслідок операцій, діяльності,
якою функціонуючий капіталіст опосереднює ці функції
промислового й торговельного капіталу. Але бути представником
функціонуючого капіталу — це не синекура, подібна до представництва
капіталу, що дає процент. На основі капіталістичного
виробництва капіталіст управляє процесом виробництва, як і про-

73 „The profits of enterprise depend upon the net profits of capital, not the
latter upon the former“. [„Підприємницькі зиски залежать від чистих зисків капіталу,
а не останні від перших“). (Ramsay: „An Essay on the'Distribution of Wealth“,
стор. 214. Net profits [чистий зиск] у Рамсея завжди = процентові.)
