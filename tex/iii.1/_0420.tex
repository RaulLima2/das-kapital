\parcont{}  %% абзац починається на попередній сторінці
\index{iii1}{0420}  %% посилання на сторінку оригінального видання
в тому металі, що служить мірилом вартості, отже, в Англії
стерлінги і навіть банкноти, а саме банкноти на незначні суми,
отже, наприклад, в п’ять або десять фунтів. Ці золоті монети
і банкноти разом з надлишковою для нього розмінною монетою він щодня або щотижня вносить як вклад у
свій банк
і оплачує ними свої закупівлі чеками на свій банковий вклад.
Але ті самі золоті монети й банкноти так само постійно знову
забираються з банків прямо або посередньо (наприклад, дрібні
гроші фабрикантами для виплати заробітної плати) як грошова
форма доходу всією публікою, як споживачем, і постійно припливають назад до роздрібних торговців,
для яких вони таким чином знову реалізують частину їх капіталу, але разом
з тим і частину їх доходу. Ця остання обставина є важлива,
а Тук її зовсім випускає з уваги. Тільки тоді, коли гроші
витрачаються як грошовий капітал,  на початку процесу репродукції (книга II, відділ І), капітальна
вартість існує як така в
чистому вигляді. Бо у виробленому товарі міститься не тільки
капітал, але вже й додаткова вартість; вироблений товар є не
тільки капітал у собі, але вже здійснений капітал, капітал із
втіленим в ньому джерелом доходу. Отже, те, що роздрібний
торговець віддає за гроші, які припливають до нього назад,
його товар, є для нього капітал плюс зиск, капітал плюс дохід.

Але далі, гроші, що циркулюють, припливаючи назад до
роздрібного торговця, відновлюють грошову форму його капіталу.

Отже, перетворювати ріжницю між циркуляцією як циркуляцією доходу і як циркуляцією капіталу в
ріжницю між циркуляцією і капіталом — це абсолютно неправильно. Цей мовний
зворот з’являється у Тука тому, що він просто стає на точку
зору банкіра, який випускає власні банкноти. Сума його банкнот, яка постійно (хоч вона і складається
завжди з інших банкнот) перебуває в руках публіки і функціонує як засіб циркуляції, нічого йому не
коштує, крім паперу й друку. Це —
виставлені на нього самого циркулюючі боргові зобов’язання
(векселі), які однак приносять йому гроші і таким чином служать засобом збільшення вартості його
капіталу. Але вони відмінні від його капіталу, однаково, чи є це капітал власний, чи
взятий у позику. Звідси для нього виникає спеціальна ріжниця
між циркуляцією і капіталом, яка однак не має нічого спільного
з визначеннями цих понять як таких, і менше за все з тими
визначеннями, що їх дав Тук.

Різне призначення грошей — чи функціонують вони як грошова форма доходу чи капіталу — насамперед
нічого не міняє
в характері грошей як засобу циркуляції; цей характер вони
зберігають, однаково, чи виконують вони ту чи другу функцію.
Зрештою, якщо гроші виступають як грошова форма доходу,
вони функціонують більше як власне засіб циркуляції (монета,
засіб купівлі), в наслідок роздрібнення цих купівель і продажів,
\parbreak{}  %% абзац продовжується на наступній сторінці
