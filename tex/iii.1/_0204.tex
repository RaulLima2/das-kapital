\parcont{}  %% абзац починається на попередній сторінці
\index{iii1}{0204}  %% посилання на сторінку оригінального видання
підвищилась, хоч і не в такій самій пропорції, в якій зменшився
зиск;

3) для капіталу вищого складу ціна виробництва товару впала,
хоч теж не в такій самій пропорції, як зиск.

Через те що ціна виробництва товарів пересічного капіталу
лишилась тією самою, рівною вартості продукту, то й сума цін
виробництва продуктів усіх капіталів лишилась тією самою,
рівною сумі вартостей, вироблених сукупним капіталом; для
сукупного капіталу підвищення на одному боці, зниження на
другому вирівнюються до рівня суспільного пересічного капіталу.
Якщо ціна виробництва товарів у прикладі II підвищується,
а в III падає, то вже ця протилежна дія, яку викликає падіння
норми додаткової вартості або загальне підвищення заробітної
плати, показує, що тут не може йтися про відшкодування в
ціні товару за підвищення заробітної плати, бо у випадку III падіння
ціни виробництва ніяк не може відшкодувати капіталістові
падіння зиску, а у випадку II підвищення ціни не перешкоджає
падінню зиску. Навпаки, в обох випадках, і там, де ціна підвищується,
і там, де вона падає, зиск є такий самий, як і для
пересічного капіталу, де ціна лишилась незмінною. Як у випадку
II, так і в випадку III зиск є однаковий, а саме пересічний
зиск, який зменшився на 5 5/7, тобто трохи більше ніж на 25\%.
Звідси випливає, що коли б у випадку II ціна не підвищувалась,
а в випадку III не падала, то у випадку II товари продавалися б
з зиском нижчим, а в випадку III — з вищим, ніж новий знижений
пересічний зиск. Само собою зрозуміло, що залежно від
того, 50, 25 чи 10 процентів капіталу витрачається на працю,
підвищення заробітної плати мусить дуже різно зачіпати
того, хто витрачає на заробітну плату 1/10, і того, хто витрачає
на неї 1/4 або 1/2 свого капіталу. Підвищення цін виробництва
— з одного боку, зниження їх — з другого, залежно від
того, вищий чи нижчий є склад капіталу порівняно з суспільним
пересічним складом, відбувається тільки в наслідок вирівнення
зиску до рівня нового зниженого пересічного зиску.\footnote*{
В першому німецькому виданні тут є таке речення: «Зрозуміло, що коли
в наслідок встановлення загальної норми зиску вартості при перетворенні їх
у ціни виробництва знижуються для капіталів нижчого складу (де v вище пересічного),
то для капіталів вищого складу вони підвищуються“. В списку друкарських
помилок до першого видання Енгельс з приводу цього речення
пише: „Все речення від слів „Зрозуміло“ до слів „підвищуються“ треба викреслити.
Речення правильне, але в цьому контексті воно може заплутати
справу“. Примітка ред. нім. вид. ІМЕН.
}

Як же вплине загальне зниження заробітної плати і відповідне
йому загальне підвищення норми зиску, а тому й пересічного
зиску, на ціни виробництва товарів, які є продуктом
капіталів, що відхиляються від суспільного пересічного складу
в протилежних напрямах? Щоб відповісти на це питання (яке
\parbreak{}  %% абзац продовжується на наступній сторінці
