\parcont{}  %% абзац починається на попередній сторінці
\index{iii1}{0280}  %% посилання на сторінку оригінального видання
розглядати сукупність товарів, то ціни, по яких їх продає клас
промислових капіталістів, менші, ніж їх вартості. Так, у вищенаведеному випадку: 900 (витрати) + 18\%
на 900, або 900 + 162 =
1062. Продаючи товар, який коштує йому 100, за 118, купець, звичайно,
накидає 18\%; але тому що товар, який він купив за 100,
вартий 118, то він таким чином продає його не дорожче його
вартості. Ми вживатимем вираз ціна виробництва у вищевикладеному
ближчому його значенні. В такому разі ясно, що зиск
промислового капіталіста дорівнює надлишкові ціни виробництва
товару понад його витрати виробництва і що, в відміну від цього
промислового зиску, торговельний зиск дорівнює надлишкові
продажної ціни понад ціну виробництва товару, яка для купця
є купівельною ціною товару; але ясно, що дійсна ціна товару =
його ціні виробництва + купецький (торговельний) зиск. Подібно
до того, як промисловий капітал тільки реалізує зиск, який міститься
вже у вартості товарів як додаткова вартість, так і торговельний
капітал реалізує зиск тільки тому, що ще не вся
додаткова вартість, або зиск, реалізована в ціні товару, реалізованій
промисловим капіталом.\footnote{
Джон Беллерс.
} Таким чином, ціна, по якій купець
продає, стоїть вище його купівельної ціни не тому, що
перша стоїть вище всієї вартості, а тому, що друга стоїть нижче її.

Отже, купецький капітал бере участь у вирівненні додаткової
вартості в пересічний зиск, хоча не бере участі у виробництві
цієї додаткової вартості. Тому загальна норма зиску вже
передбачає відрахування з додаткової вартості, яке припадає купецькому
капіталові, отже, відрахування з зиску промислового
капіталу.

З вищевикладеного випливає:

1) Чим більший купецький капітал порівняно з промисловим
капіталом, тим менша норма промислового зиску, і навпаки;

2) Якщо в першому відділі виявилось, що норма зиску завжди
виражає меншу норму, ніж норма дійсної додаткової вартості,
тобто завжди виражає ступінь експлуатації праці занадто низьким,
— наприклад, у наведеному вище випадку 720 с + 180 v + 180 m
норма додаткової вартості в 100\% виражається як норма зиску
тільки в 20\%, — то ці відношення розходяться ще більше,
оскільки тепер сама пересічна норма зиску, якщо врахувати
частку, яка припадає купецькому капіталові, в свою чергу виявляється
меншою, — в даному випадку 18\% замість 20\%. Отже,
пересічна норма зиску безпосередньо експлуатуючого капіталіста
виражає норму зиску меншою, ніж вона є в дійсності.

Якщо припустити всі інші умови незмінними, відносний розмір
купецького капіталу (при чому, однак, одна його різновидність,
капітал роздрібних торговців, становить виняток) стоятиме
у зворотному відношенні до швидкості його обороту, отже,
у зворотному відношенні до енергії процесу репродукції взагалі.
\parbreak{}  %% абзац продовжується на наступній сторінці
