Сама торгівля злитками, перевезення золота або срібла з однієї
країни до іншої, є тільки результат товарної торгівлі, визначуваний
вексельним курсом, який виражає стан міжнародних платежів
і рівня процента на різних ринках. Торговець злитками
як такий є тільки посередником при таких результатах товарної
торгівлі.

При дослідженні грошей, а саме того, яким чином з простої
товарної циркуляції розвиваються їх рухи і визначеності форми,
ми бачили (книга І, розд. ІІІ), що рух маси грошей, які циркулюють
як купівельний засіб і як платіжний засіб, визначається
метаморфозою товарів, її розміром і швидкістю, — метаморфозою,
яка, як ми тепер знаємо, сама є тільки моментом сукупного
процесу репродукції. Щодо одержання грошового матеріалу —
золота й срібла — з джерел його добування, то воно зводиться
до безпосереднього товарообміну, до обміну золота й срібла як
товарів на інші товари, отже, воно само цілком так само є моментом
товарообміну, як одержання заліза чи інших металів. Щождо
руху благородних металів на світовому ринку (ми залишаємо
тут осторонь цей рух, оскільки він виражає перенесення капіталу
в формі позики, перенесення, яке відбувається і в формі
товарного капіталу), то він цілком так само визначається міжнародним
товарообміном, як рух грошей як купівельного і платіжного
засобу всередині країни визначається товарообміном
всередині країни. Еміграція і імміграція благородних металів з
однієї національної сфери циркуляції до іншої, оскільки вони
викликаються тільки знеціненням місцевої монети або подвійною
валютою, не мають ніякого відношення до грошової циркуляції
як такої і є тільки виправленням довільних порушень, зроблених
державною владою. Нарешті, щодо утворення скарбів, оскільки
воно становить собою резервний фонд купівельних чи платіжних
засобів для внутрішньої або для зовнішньої торгівлі, а також
оскільки воно є простою формою тимчасово бездіяльного капіталу,
— воно в обох випадках є тільки неминучий осад процесу
циркуляції.

Якщо вся грошова циркуляція своїм розміром, своїми формами
і своїм рухом є простий результат товарної циркуляції,
яка з капіталістичної точки зору сама виражає тільки процес
циркуляції капіталу (а цей процес включає обмін капіталу на
дохід і доходу на дохід, оскільки витрачання доходу реалізується
в роздрібній торгівлі), то само собою розуміється, що торгівля
грішми опосереднює не тільки простий результат і спосіб
виявлення товарної циркуляції, не тільки грошову циркуляцію.
Сама ця грошова циркуляція, як момент товарної циркуляції, є
дана для торгівлі грішми. Торгівля грішми опосереднює технічні
операції грошової циркуляції, які вона концентрує, скорочує
і спрощує. Вона не утворює скарбів, а дає технічні засоби для
того, щоб це скарботворення, оскільки воно відбувається добровільно
(отже, оскільки воно не є виразом незанятості капіталу
