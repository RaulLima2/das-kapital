Середній розмір процента являє собою в кожній країні для
більш-менш довгих періодів постійну величину, бо загальна норма
зиску змінюється тільки протягом довших періодів, не зважаючи
на постійні зміни окремих норм зиску, при чому, однак, зміна в
одній сфері урівноважується протилежною зміною в другій. І відносна
постійність загальної норми зиску виявляється саме в цьому
більш-менш постійному характері середнього розміру процента
(average rate or common rate of interest).

Щождо ринкової норми процента, яка постійно коливається,
то для кожного моменту вона, подібно до ринкової ціни товарів,
є дана як фіксована величина, бо на грошовому ринку весь
капітал, що дається в позику, завжди протистоїть функціонуючому
капіталові як сукупна маса, отже, відношенням між поданням
капіталу, який дається в позику, з одного боку, і
попитом на нього, з другого, кожний раз визначається ринковий
рівень процента. Це має місце тим більше, чим більше
розвиток і зв’язана з ним концентрація кредитної справи
надають капіталові, що дається в позику, загальносуспільного
характеру і відразу, одночасно кидають його на грошовий
ринок. Навпаки, загальна норма зиску завжди існує тільки як
тенденція, як рух до вирівнення окремих норм зиску. Конкуренція
капіталістів — яка сама є цим рухом вирівнення —
полягає тут у тому, що вона ступнево відтягає капітал від тих
сфер, в яких зиск довгий час стоїть нижче пересічного рівня,
і так само ступнево притягає його до тих сфер, в яких зиск стоїть
вище пересічного рівня; або й у тому, що додатковий капітал
помалу розподіляється в різних пропорціях між цими сферами.
Відносно цих різних сфер це є постійні коливання у припливі й
відтяганні капіталу, але це ніколи не є одночасний вплив усієї
маси капіталу, як при визначенні розміру процента.

Ми бачили, що капітал, який дає процент, хоч він і є категорія,
абсолютно відмінна від товару, стає товаром sui generis
[особливого роду] і в наслідок цього процент стає його ціною,
яка, подібно до ринкової ціни звичайного товару, кожен раз
фіксується попитом і поданням. Тому ринкова норма процента,
хоч вона постійно коливається, виступає в кожний даний момент
так само постійно фіксованою і одноманітною, як і ринкова
ціна товару в кожному окремому випадку. Грошові капіталісти
пропонують цей товар, а функціонуючі капіталісти купують
його, утворюють попит на нього. Цього не відбувається
при вирівненні зиску в загальну норму зиску. Якщо ціни товарів
у якійнебудь сфері стоять нижче або вище ціни виробництва
(при чому залишаються осторонь коливання, властиві кожному
підприємству і зв’язані з різними фазами промислового циклу),
то вирівнення відбувається за допомогою розширення або звуження
виробництва, тобто за допомогою збільшення або зменшення
товарних мас, що їх промислові капіталісти кидають на
ринок, що досягається за допомогою імміграції або еміграції
