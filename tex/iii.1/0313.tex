передбачає як свою базу передусім товарну циркуляцію, а тому
й грошову циркуляцію, — вивести їх як форми, що необхідно
виникають з процесу виробництва як такого.

Якби між товарно-торговельним капіталом і грошево-торговельним
капіталом, з одного боку, і виробництвом зерна, з
другого, не було ніякої іншої ріжниці, крім такої, яка є між цим
останнім і скотарством та мануфактурою, то було б цілком
ясно, що виробництво і капіталістичне виробництво взагалі
тотожні, і що зокрема й розподіл суспільних продуктів між членами
суспільства як для продуктивного, так і для особистого
споживання так само мусить вічно відбуватись за допомогою
купців і банкірів, як споживання м’яса за допомогою скотарства
і споживання одягу за допомогою його фабрикації.46

Великі економісти, як Сміт, Рікардо і т. д., в наслідок того,
що вони розглядали основну форму капіталу, капітал як промисловий
капітал, а капітал циркуляції (грошовий капітал і товарний
капітал) фактично розглядали лиш остільки, оскільки він
сам є фазою в процесі репродукції всякого капіталу, попали в
скрутне становище з торговельним капіталом, як особливим видом
капіталу. Положення про утворення вартості, про зиск та
ін., безпосередньо виведені з розгляду промислового капіталу,
не можуть бути застосовані безпосередньо до купецького капіталу.
Тому ці економісти в дійсності лишають купецький капітал
цілком осторонь і згадують про нього тільки як про вид
промислового капіталу. Там, де вони окремо говорять про
нього, як от Рікардо в зв’язку з зовнішньою торгівлею, вони
намагаються довести, що він не утворює ніякої вартості (отже
й додаткової вартості). Але те, що має силу для зовнішньої
торгівлі, стосується й до внутрішньої.

Досі ми розглядали купецький капітал з точки зору і в межах
капіталістичного способу виробництва. Однак не тільки торгівля,
але і торговельний капітал старіший за капіталістич-

46 Мудрий Рошер [„Die Grundlagen der Nationalökonomie“, 2 вид., Штутгарт
і Аугсбург 1857, стор. 102] додумався до того, що коли дехто характеризує
торгівлю як „посередництво“ між виробниками й споживачами, то з таким самим
успіхом можна характеризувати й саме виробництво як „посередництво“
споживання (між ким?), з чого, звичайно, випливає, що торговельний капітал
є частина продуктивного капіталу подібно до землеробського і промислового
капіталів. Отже, якщо можна сказати, що людина може опосереднювати своє
споживання тільки виробництвом (а це вона мусить зробити, навіть не здобувши
освіти в Лейпцігу), або що праця потрібна для присвоєння природи (що можна
назвати „посередництвом“), то звідси, звичайно, виходить, що суспільне „посередництво“, яке випливає
з специфічної суспільної форми виробництва, — тому
що воно є посередництво, — має такий самий абсолютний характер необхідності,
такий самий ранг. Слово посередництво вирішує все. Зрештою, адже купці зовсім
не є посередники між виробниками й споживачами (споживачів у відміну від виробників, споживачів, які
не виробляють, ми спочатку не беремо до уваги), а посередники при обміні продуктів цих виробників
між собою; вони тільки проміжні
особи при обміні, який все ж у тисячі випадків відбувається без них.
