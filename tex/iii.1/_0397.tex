\parcont{}  %% абзац починається на попередній сторінці
\index{iii1}{0397}  %% посилання на сторінку оригінального видання
норми процента. — І далі така мудрість. На слушне зауваження:
„Але за гроші платяться проценти“, яке, звичайно, містить у собі
запитання: яке відношення має процент, який одержує банкір,
що зовсім не торгує товарами, до цих товарів? І хіба не одержують грошей за однакові проценти
фабриканти, які витрачають
ці гроші на зовсім різних ринках, отже, на ринках, де панує
цілком різне відношення між попитом і поданням уживаних у виробництві товарів? — на це питання наш
тріумфуючий геній зауважує: якщо фабрикант купує бавовну в кредит, „тоді ріжниця між ціною за
готівку і ціною в кредит у момент скінчення строку становить міру процента“. Навпаки. Існуюча норма
процента, регулювання якої геній Нормана повинен пояснити, є масштаб ріжниці
між ціною за готівку і ціною в кредит до скінчення строку. Спочатку треба продати бавовну по її ціні
за готівку, а ця ціна визначається ринковою ціною, яка сама регулюється станом попиту
й подання. Припустім, що ціна = 1000 фунтам стерлінгів. На цьому
між фабрикантом і бавовняним маклером справа закінчується,
оскільки йдеться про купівлю й продаж. Але сюди долучається
друга операція. Операція між позикодавцем і позичальником. Вартість у 1000 фунтів стерлінгів дається
фабрикантові в позику бавовною, а він повинен повернути її грішми, скажімо, через три місяці. А
проценти за 1000 фунтів стерлінгів за три місяці, визначувані
ринковою нормою процента, становитимуть тоді надбавку до і поверх ціни готівкою. Ціна бавовни
визначається попитом і поданням.
Але ціна позиченої вартості бавовни, ціна 1000 фунтів стерлінгів на
три місяці, визначається нормою процента. І це — а саме, що сама
бавовна перетворюється таким чином у грошовий капітал — служить для пана Нормана доказом того, що
процент існував би, навіть
якби взагалі не існувало грошей. Якби взагалі не існувало грошей,
то в усякому разі не існувало б ніякої загальної норми процента.

Насамперед, вульгарне уявлення про капітал як про „товари,
вживані у виробництві“. Оскільки ці товари фігурують як капітал, їх вартість як капіталу, в відміну
від їх вартості як товарів, виражається в зиску, який одержується від їх продуктивного або
торговельного застосування. І норма зиску безумовно
завжди має деяке відношення до ринкової ціни куплених товарів
і до їх попиту й подання, але визначається вона ще зовсім іншими
обставинами. І що норма процента взагалі має свою межу в нормі
зиску, — в цьому немає ніякого сумніву. Але нехай пан Норман
прямо скаже нам, як визначається ця межа. А визначається
вона попитом і поданням грошового капіталу в відміну його від
інших форм капіталу. Але можна було б далі запитати: Як визначається попит і подання грошового
капіталу? Що існує прихований зв’язок між поданням речового капіталу і поданням грошового капіталу,
— в цьому немає ніякого сумніву, так само як
і в тому, що попит промислових капіталістів на грошовий капітал визначається обставинами дійсного
виробництва. Замість
того, щоб пояснити нам це, Норман повчає нас тієї премудрості,
\parbreak{}  %% абзац продовжується на наступній сторінці
