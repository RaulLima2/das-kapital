\parcont{}  %% абзац починається на попередній сторінці
\index{iii1}{0300}  %% посилання на сторінку оригінального видання
купецький капітал привласнював собі значно більшу частину додаткової
вартості, ніж припадало б йому при загальній рухливості
капіталів. Отже, припинення цього становища, розглядуваного
з того і другого боку, є результатом розвитку капіталістичного
способу виробництва.

В різних галузях торгівлі обороти купецького капіталу є довші
чи коротші, отже, число їх протягом року є більше чи менше.
В одній і тій же галузі торгівлі оборот є швидший чи повільніший
на різних фазах економічного циклу. Проте, відбувається
певне пересічне число оборотів, яке пізнається з досвіду.

Ми вже бачили, що оборот купецького капіталу відмінний
від обороту промислового капіталу. Це випливає з природи
справи; одна окрема фаза в обороті промислового капіталу виступає
як повний оборот власне купецького капіталу або принаймні
частини його. Крім того, оборот купецького капіталу стоїть
в іншому відношенні до визначення зиску й ціни.

У промисловому капіталі оборот виражає, з одного боку, періодичність
репродукції, і тому від нього залежить маса товарів,
які протягом певного часу викидаються на ринок. З другого
боку, час обігу становить певну межу, хоча й розтягливу, яка
більш чи менш обмежуюче впливає на утворення вартості й додаткової
вартості, бо впливає на розміри процесу виробництва.
Тому оборот визначально впливає — не як позитивний елемент,
а як обмежуючий елемент — на масу щорічно вироблюваної
додаткової вартості, а тому й на утворення загальної норми
зиску. Навпаки, пересічна норма зиску для купецького капіталу
є величина дана. Він не бере безпосередньо участі в створенні
зиску або додаткової вартості, і визначально впливає на утворення
загальної норми зиску лиш остільки, оскільки одержує — відповідно
до того, яку він становить частину в сукупному капіталі —
свої дивіденди з маси зиску, виробленої промисловим капіталом.

Чим більше число оборотів промислового капіталу, — при
умовах, викладених в книзі II, відділ II, — тим більша маса зиску,
яку він утворює. Правда, тепер, в наслідок встановлення
загальної норми зиску, сукупний зиск розподіляється між різними
капіталами не пропорціонально до їх безпосередньої участі
в його виробництві, а відповідно до того, які вони становлять
частини в сукупному капіталі, тобто пропорціонально до їх величини.
Однак, це ні трохи не змінює суті справи. Чим більше
число оборотів сукупного промислового капіталу, тим більша
маса зиску, маса щорічно вироблюваної додаткової вартості,
а тому, при інших однакових умовах, і норма зиску. Інакше
стоїть справа з купецьким капіталом. Для нього норма зиску є
величина дана, визначувана, з одного боку, масою вироблюваного
промисловим капіталом зиску, а з другого боку, відносною величиною
сукупного торговельного капіталу, його кількісним відношенням
до суми капіталу, авансованого на процес виробництва
і процес циркуляції. Звичайно, число його оборотів
\parbreak{}  %% абзац продовжується на наступній сторінці
