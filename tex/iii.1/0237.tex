чені галузі капіталовкладень підлягають законам вільної конкуренції. Навпаки, Рікардо вбачається
таке: на гроші, одержані за кордоном від продажу по вищій ціні, там купуються товари і
відправляються на заміну додому; отже, ці товари продаються
всередині країни, і тому це може становити, принаймні тимчасово, особливу, порівняно з іншими
сферами, невигоду для сфер виробництва, які є в сприятливих умовах. Ця ілюзія відпадає, як тільки ми
абстрагуємось від грошової форми. Країна,
яка перебуває в сприятливіших умовах, одержує назад більше праці в обмін за меншу кількість праці,
хоч ця ріжниця, цей надлишок, як взагалі при обміні між працею і капіталом, привласнюється певним
класом. Отже, оскільки норма зиску є вища, тому
що вона взагалі вища в колоніальній країні, це при сприятливих природних умовах цієї країни може йти
рука в руку з низькими товарними цінами. Вирівнення відбувається, але вирівнення не за старим
рівнем, як гадає Рікардо.

Але та сама зовнішня торгівля розвиває всередині країни капіталістичний спосіб виробництва і тим
самим веде до зменшення змінного капіталу порівняно з сталим; з другого боку, вона створює
перепродукцію відносно закордону і тому в дальшому перебігу знов таки справляє протилежний вплив.

І    таким чином взагалі виявляється, що ті самі причини, які приводять до падіння загальної норми
зиску, викликають протилежні впливи, які гальмують, уповільнюють і почасти паралізують це падіння.
Вони не знищують закону, але ослаблюють його діяння. Без цього було б незрозумілим не падіння
загальної норми зиску, а, навпаки, відносна повільність цього падіння. Таким чином закон діє тільки
як тенденція, вплив якої виразно виступає тільки при певних обставинах і на протязі довгих періодів
часу.

Раніше, ніж піти далі, ми, щоб уникнути непорозумінь, повторимо ще два, вже не раз розвинуті
положення.

Поперше: Той самий процес, який в ході розвитку капіталістичного способу виробництва породжує
здешевлення товарів, породжує також зміну в органічному складі суспільного капіталу, застосовуваного
для виробництва товарів, а в наслідок цього і падіння норми зиску. Отже, зменшення відносних витрат
на окремий товар, а також тієї частини цих витрат, яка містить у собі зношування машин, не слід
ототожнювати з зростанням вартості сталого капіталу порівняно з змінним, хоча, навпаки, всяке
зменшення відносних витрат на сталий капітал, при незмінному або зростаючому розмірі його речових
елементів, впливає на підвищення норми зиску, тобто на зменшення pro tanto [відповідно до цього]
вартості сталого капіталу порівняно з змінним капіталом, застосовуваним в дедалі менших пропорціях.

Подруге: Та обставина, що в окремих товарах, сукупність яких становить продукт капіталу, відношення
додаваної живої праці, яка міститься в них, до уміщених в них матеріалів праці
