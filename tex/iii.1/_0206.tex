\parcont{}  %% абзац починається на попередній сторінці
\index{iii1}{0206}  %% посилання на сторінку оригінального видання
В обох випадках — як при підвищенні, так і при падінні заробітної плати — припускалося, що робочий
день лишається незмінним, так само як і ціни всіх необхідних засобів існування. Отже, падіння
заробітної плати тут можливе тільки в тому випадку, коли вона або раніше стояла вище нормальної ціни
праці, або тепер падає нижче неї. Як модифікується справа, коли підвищення або падіння заробітної
плати походить від зміни вартості, а тому й ціни виробництва товарів, що звичайно входять у
споживання робітника, це почасти буде досліджено далі, у відділі про земельну ренту. Проте, тут слід
раз назавжди зазначити:

Якщо підвищення або падіння заробітної плати походить від зміни вартості необхідних засобів
існування, то вищесказане може потребувати модифікації лиш остільки, оскільки товари, зміна цін яких
підвищує або знижує змінний капітал, входять також у сталий капітал як конститутивні елементи і,
отже, діють не на саму тільки заробітну плату. Оскільки ж вони діють на саму тільки заробітну плату,
вищенаведені міркування містять у собі все, що тут слід сказати.

В усьому цьому розділі встановлення загальної норми зиску, пересічного зиску і, отже, перетворення
вартостей у ціни виробництва припускалося як даний факт. Питання полягало тільки в тому, яким чином
загальне підвищення або зниження заробітної плати впливає на ціни виробництва товарів, які припущені
як дані. Це — дуже другорядне питання порівняно з усіма іншими важливими пунктами, розглянутими в
цьому відділі. Але
це єдине з зачеплених тут питань, яке досліджує Рікардо, і при тому, як ми побачимо далі, досліджує
однобічно й хибно.

\section{Додатки}

\subsection{Причини, що зумовлюють зміни в ціні виробництва}

Ціна виробництва товару може змінюватися тільки з двох причин:

\emph{Поперше}. Змінюється загальна норма зиску. Це можливе тільки в наслідок того, що змінюється сама
пересічна норма додаткової вартості, або — при незмінній пересічній нормі додаткової вартості — в
наслідок того, що змінюється відношення суми привласнюваних додаткових вартостей до суми сукупного
авансованого суспільного капіталу.

Оскільки зміна норми додаткової вартості базується не на зниженні заробітної плати нижче або
підвищенні її понад її нормальний рівень, — а подібні рухи слід розглядати тільки як коливання, —
зміна ця може статися або тільки в наслідок того, що вартість робочої сили знизилася або
підвищилась; те й друге однаково неможливе без зміни в продуктивності праці, яка створює
\parbreak{}  %% абзац продовжується на наступній сторінці
