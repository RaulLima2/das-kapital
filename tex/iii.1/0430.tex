курсу в континентальних країнах порівняно з станом гарячкового
неспокою й тривоги, який завжди постає в Англії, коли
здається, ніби скарби банку близькі до свого повного вичерпання,
не вражаючись при цьому тією великою вигодою, яку
має в цьому відношенні металічна циркуляція“].

Якщо ж ми залишимо осторонь відплив золота, то як може
тоді банк, що випускає банкноти, отже, наприклад, Англійський
банк, збільшувати суму грошових позик, які він видає, не збільшуючи
випуску своїх банкнот?

Поза стінами банку всі банкноти, чи циркулюють вони, чи
лежать без діла в приватних сховищах, для самого банку перебувають
у циркуляції, тобто не в його володінні. Отже, якщо
банк розширює свої дисконтні і ломбардні операції, позики під
securities, то видані для цього банкноти мусять припливати до
нього назад, бо інакше вони збільшують суму циркуляції, чого
саме й не повинно бути. Цей зворотний приплив може відбуватись
двояким способом.

Поперше: Банк платить клієнтові А банкнотами за цінні папери;
А оплачує ними особі В вексель, якому надійшов строк,
а В знову вносить ці банкноти в банк як вклад. Таким чином
циркуляція цих банкнот закінчена, але позика лишається. (The
loan remains, and the currency, if not wanted, finds its way back
to the issuer [Позика лишається, а засіб циркуляції, якщо в ньому
немає потреби, знаходить свій шлях до того, хто його випустив]
Fullarton, стор. 97). Банкноти, які банк позичив особі А, повернулися
тепер до нього назад; навпаки, банк є кредитором А
або тієї особи, на яку виставлено вексель, дисконтований А,
і дебітором В на суму вартості, виражену в цих банкнотах,
а В може в наслідок цього порядкувати відповідною частиною
капіталу банку.

Подруге: А платить В, а сам В або С, якому В в свою чергу
платить цими банкнотами, оплачує ними ж банкові, безпосередньо
або посередньо, векселі, яким надійшов строк. В цьому
випадку банкові платиться його ж власними банкнотами. На
цьому тоді операція закінчується (до зворотного платежу банкові
клієнтом А).

В якій же мірі можна розглядати позику банку клієнтові А
як позику капіталу або як просту позику засобів платежу?\footnote{
Те місце оригіналу, що йде вслід за цим, незрозуміле в даному зв’язку,
і до закриття дужок воно наново перероблене упорядником. В іншому зв’язку
це питання було вже зачеплене в розділі XXVI. — Ф. Е.
}

[Це залежить від природи самої позики. При цьому слід
дослідити три випадки.

Перший випадок. — А одержує від банку певні суми позики під
свій особистий кредит, не даючи при цьому ніякого забезпечення.
В цьому випадку він одержав у позику не тільки засоби платежу,
але, безумовно, і новий капітал, який він може застосо-