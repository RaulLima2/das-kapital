тимчасом як у країнах розвиненого капіталістичного виробництва
процент виражає тільки відповідну частину виробленої додаткової
вартості або зиску. З другого боку, тут рівень процента переважно
визначається такими відносинами (позики лихварів знаті,
власникам земельної ренти), які не мають нічого спільного з
зиском, а, навпаки, показують тільки, в якій мірі лихвар привласнює
собі земельну ренту.

В країнах різного ступеня розвитку капіталістичного виробництва
і тому різного органічного складу капіталу норма
додаткової вартості (один з факторів, що визначають норму зиску)
може стояти вище в тій країні, де нормальний робочий день
коротший, ніж у тій країні, де він довший. Поперше, якщо
англійський робочий день у 10 годин в наслідок своєї вищої
інтенсивності дорівнює австрійському робочому дневі в 14 годин,
то при однаковому розподілі робочого дня 5 годин додаткової
праці англійця можуть на світовому ринку представляти
вищу вартість, ніж 7 годин австрійця. А подруге, в Англії
додаткову працю може становити більша частина робочого дня,
ніж в Австрії.

Закон спадаючої норми зиску, в якій виражається та сама або
навіть зростаюча норма додаткової вартості, означає, інакше
кажучи, таке: якщо взяти якусь певну кількість пересічного
суспільного капіталу, наприклад, капітал в 100, то частина
його, представлена в засобах праці, дедалі зростає, а частина,
представлена в живій праці, дедалі зменшується. Отже, через
те що вся маса живої праці, додаваної до засобів виробництва,
зменшується порівняно з вартістю цих засобів виробництва,
то порівняно з вартістю всього авансованого капіталу
зменшується також і неоплачена праця і та частина вартості,
в якій вона виражається. Або: з усього витраченого капіталу
все менша й менша частина перетворюється в живу працю,
і тому весь цей капітал вбирає порівняно з своєю величиною
все менше й менше додаткової праці, хоч одночасно з цим відношення
неоплаченої частини вживаної праці до її оплаченої частини
може зростати. Відносне зменшення змінного і збільшення
сталого капіталу, хоч обидві ці частини абсолютно зростають,
є, як ми вже сказали, тільки інший вираз зростаючої продуктивності
праці.

Припустім, що капітал в 100 складається з 80с + 20v, а ці
останні = 20 робітникам. Норма додаткової вартості нехай буде
100%, тобто робітники працюють півдня на себе, півдня на капіталіста.
Нехай у другій, менш розвиненій країні капітал буде
20c + 80v, і ці останні = 80 робітникам. Але цим робітникам потрібно
2/3 робочого дня для себе й тільки 1/3 вони працюють на
капіталіста. При всіх інших однакових умовах, у першому випадку
робітники виробляють вартість в 40, у другому — в 120.
Перший капітал виробляє 80с + 20v + 20m = 120; норма зиску =
20%; другий капітал 20с + 80v + 40m = 140; норма зиску =
