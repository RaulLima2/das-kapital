виробництва, з другого боку — як індивідуальну вартість товарів,
які виробляються при пересічних умовах даної сфери і які становлять
значну масу продуктів цієї сфери. Тільки при виняткових
комбінаціях товари, вироблені при найгірших умовах або
при найсприятливіших умовах, регулюють ринкову вартість, яка становить,
з свого боку, центр коливань для ринкових цін, —
які, однак, є ті самі для товарів того самого роду. Якщо подання
товарів по пересічній вартості, отже, по середній вартості
тієї маси, яка лежить між обома крайніми полюсами,
задовольняє звичайний попит, то товари, індивідуальна вартість
яких стоїть нижче ринкової вартості, реалізують наддодаткову
вартість або надзиск, тимчасом як товари, індивідуальна вартість
яких стоїть вище ринкової вартості, не можуть реалізувати
частини вміщеної в них додаткової вартості.

Ні трохи не допомагає й те, коли сказати, що продаж товарів,
вироблених при найгірших умовах, доводить, що вони
потрібні для покриття попиту\footnote*{
В першому німецькому виданні і в рукопису Маркса тут стоїть: „подання“;
очевидна рукописна помилка, як це випливає з дальших рядків. Примітка
ред. нім. вид. ІМЕЛ.
}. Коли б у припущеному випадку
ціна була вища за середню ринкову вартість, то попит був би
менший\footnote*{
В першому німецькому виданні тут стоїть: „більший“; виправлено на
підставі рукопису Маркса. Примітка ред, нім. вид. ІМЕЛ.
}. При певних цінах даний рід товару може займати
на ринку якесь певне місце; при зміні цін це місце лишається
тим самим тільки тоді, коли вища ціна збігається з меншою
кількістю товарів, а нижча ціна — з більшою кількістю товарів.
Якщо ж, навпаки, попит такий значний, що він не скорочується
і тоді, коли ціна регулюється вартістю товарів, вироблених при
найгірших умовах, то ці останні визначають ринкову вартість.
Це можливо тільки тоді, коли попит перевищує звичайний рівень
або коли подання падає нижче звичайного рівня. Нарешті,
якщо маса вироблених товарів більша за ту кількість їх, яка
знаходить собі збут по середніх ринкових вартостях, то ринкову
вартість регулюють товари, вироблені при найкращих
умовах. Ці останні, наприклад, можуть бути продані цілком або
приблизно по їх індивідуальній вартості, при чому може статися,
що товари, вироблені при найгірших умовах, не реалізують навіть
своїх витрат виробництва, тимчасом як товари, вироблені
при середніх, пересічних умовах, можуть реалізувати тільки частину
вміщеної в них додаткової вартості. Те, що сказано тут
про ринкову вартість, стосується і до ціни виробництва, якщо
вона заступає місце ринкової вартості. Ціна виробництва регулюється
в кожній сфері виробництва, і регулюється так само
залежно від певних обставин. Але сама вона знов таки є центром,
навколо якого коливаються щоденні ринкові ціни і відповідно
до якого вони за певні періоди часу вирівнюються (див.
у Рікардо про визначення ціни виробництва працею, виконуваною