бо поруч з дійсним управителем і над ним з’являється багато
членів правління і ради нагляду, для яких управління й нагляд
стають тільки приводом для грабування акціонерів і самозбагачення.
Про це можна знайти дуже цікаві подробиці в „The
City or the Physiology of London Business; with Sketches on’Change,
and the Coffee Houses“, London 1845. „Що виграють банкіри
й купці від того, що вони беруть участь в правліннях
восьми або дев’яти різних товариств, можна бачити з такого
прикладу: приватний баланс пана Timothy Abraham Curtis’a, поданий
після його банкрутства до комерційного суду, показував
під рубрикою директорство 800—900 фунтів стерлінгів річного
доходу. Тому що пан Curtis був директором Англійського банку
і Ост-Індської компанії, то кожне акційне товариство вважало за
щастя залучити його до товариства як директора“ (стор. [81] 82). —
Винагорода директорів таких товариств за одне засідання на
тиждень дорівнює щонайменше одній гінеї (21 марка). Розгляд
справ в комерційному суді показує, що плата за цей нагляд
звичайно стоїть у зворотному відношенні до нагляду, що його
в дійсності виконують ці номінальні директори.

Розділ двадцять четвертий

Виділення капіталістичного відношення у формі
капіталу, що дає процент

В капіталі, що дає процент, капіталістичне відношення досягає
своєї найбільш зовнішньої і фетишистичної форми. Ми
маємо тут Г — Г', гроші, що породжують більшу кількість грошей,
самозростаючу вартість, без того процесу, який є опосереднюючою
ланкою між двома крайніми пунктами. В купецькому
капіталі, Г — Т — Г', є в наявності принаймні загальна форма
капіталістичного руху, хоч він відбувається тільки в сфері циркуляції;
тому зиск виступає просто як зиск від відчуження, продажу;
але все ж він виступає як продукт суспільного відношення, а не
як продукт просто речі. Форма купецького капіталу все ж
являє собою процес, єдність протилежних фаз, рух, який розпадається
на два протилежні акти — на купівлю і продаж товарів.
В Г — Г', у формі капіталу, що дає процент, це стерто. Якщо,
наприклад, 1000 фунтів стерлінгів віддаються капіталістом у позику,
і розмір процента є 5%, то вартість 1000 фунтів стерлінгів
як капіталу за 1 рік = К + Кz', де К — капітал, а z' —
розмір процента; отже тут 5% = 5/100 = 1/20, 1000 + 1000 × 1/20 =
1050 фунтів стерлінгів. Вартість 1000 фунтів стерлінгів як
капіталу = 1050 фунтів стерлінгів, тобто капітал не є проста
величина. Він є відношення величин, відношення основної суми
як даної вартості до себе самої як до самозростаючої вартості,
