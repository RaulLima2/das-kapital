\index{iii1}{0388}  %% посилання на сторінку оригінального видання
А ці власні справи здебільшого теж були вже переобтяжені боргами. Привабливий високий зиск спокусив
до далеко
ширших операцій, ніж це дозволяли наявні вільні засоби. Але до
послуг був кредит, легко доступний і, крім того, дешевий. Банковий дисконт стояв на низькому рівні:
в 1844 році 1\sfrac{3}{4}--2\sfrac{3}{4}\%, в 1845 році до жовтня нижче 3\%, потім на короткий час підвищився до 5\%
(лютий 1846), потім знов упав до 3\sfrac{1}{4}\% в грудні
1846 року. У своїх підвалах банк мав запас золота нечуваних
розмірів. Всі вітчизняні біржові цінності стояли так високо, як ніколи раніше. Отже, чому не не
використати прекрасну нагоду,
чому б не взятися якнайшвидше до справ? Чому б не надіслати
на іноземні ринки, що жадають англійських фабрикатів, усякі
товари, які тільки можна виробити? І чому б самому фабрикантові не покласти собі в кишеню подвійний
зиск, який утворювався від продажу пряжі й тканин на Далекому Сході і від
продажу в Англії привезених звідти товарів?

Так виникла система масових відсилань товарів під аванс, до
Індії й Китаю, яка дуже скоро розвинулася в систему відсилань
товарів виключно ради одержання авансу, як це детально змальовано в нижченаведених замітках, і яка
неминуче мусила кінчитися масовим переповненням ринків і крахом.

Цей крах вибухнув у наслідок неврожаю 1846 року. Англія
і особливо Ірландія потребували величезного привозу засобів
існування, особливо хліба й картоплі. Але ті країни, що постачали ці продукти, тільки в
найнезначнішій частині могли бути
оплачені за ці продукти продуктами англійської промисловості;
необхідно було платити благородним металом; золота пішло за
кордон щонайменше на 9 мільйонів. З цього золота цілих
7\sfrac{1}{2} мільйонів взято було з запасів готівкою Англійського банку,
в наслідок чого свобода руху Англійського банку на грошовому
ринку була значно паралізована; всі інші банки, резерви яких
лежали в Англійському банку і фактично були тотожні з резервами цього банку, так само мусили
обмежити надання грошових
позик; потік платежів, який швидко й легко припливав до банку,
припинився спочатку в окремих пунктах, потім повсюди. Банковий дисконт, який ще в січні 1847 року
стояв на рівні 3--3\sfrac{1}{2}\%, підвищився у квітні, коли вибухла перша паніка, до 7\%; потім влітку
ще раз настало невелике тимчасове полегшення (6,5\%, 6\%), але
як тільки виявився новий неврожай, вибухла нова, ще дужча паніка. Офіціальний мінімальний дисконт
Англійського банку підвищився в жовтні до 7, в листопаді до 10\%, тобто величезну більшість векселів
можна було дисконтувати тільки під колосальні
лихварські проценти або їх взагалі не можна було дисконтувати;
загальне припинення платежів привело до банкрутства ряд першорядних фірм і дуже багато середніх і
дрібних; самому Англійському банкові загрожувала небезпека банкрутства в наслідок
обмежень, покладених на нього хитромудрим банковим актом
1844 року, — тоді уряд, на загальну вимогу, 25 жовтня припинив
\parbreak{}  %% абзац продовжується на наступній сторінці
