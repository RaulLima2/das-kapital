\parcont{}  %% абзац починається на попередній сторінці
\index{iii1}{0214}  %% посилання на сторінку оригінального видання
показано, чому це зниження виступає не в цій абсолютній формі,
а більше в тенденції до прогресивного падіння.) Отже, прогресуюча
тенденція загальної норми зиску до зниження є тільки
\emph{властивий} \emph{капіталістичному способові виробництва вираз}
прогресуючого розвитку суспільної продуктивної сили праці.
Цим не сказано, що норма зиску не може тимчасово падати і
з інших причин, але цим доведено, як само собою зрозумілу
з суті капіталістичного способу виробництва необхідність, що
з розвитком цього способу виробництва загальна пересічна норма
додаткової вартості мусить виражатись у падаючій загальній
нормі зиску. Через те що маса вживаної живої праці постійно
зменшується порівняно з масою упредметненої праці, яку вона
приводить в рух, порівняно з масою продуктивно споживаних
засобів виробництва, то й відношення тієї частини цієї живої
праці, яка неоплачена і упредметнюється в додатковій вартості,
до розміру вартості всього вживаного капіталу мусить постійно
зменшуватись. Але це відношення маси додаткової вартості до
вартості всього вживаного капіталу становить норму зиску, яка
через це мусить постійно падати.

Хоч і яким простим здається цей закон після того, що ми досі
розвинули, проте всій дотеперішній політичній економії не вдалося
відкрити його, як ми це побачимо в одному з дальших відділів.
Вона бачила явище і мучилася в суперечливих спробах
пояснити його. Але при тій великій важливості, яку цей закон
має для капіталістичного виробництва, можна сказати, що він
становить таємницю, над розв’язанням якої б’ється вся політична
економія від часів Адама Сміта, і що ріжниця між різними школами
від часів А. Сміта полягає в різних спробах розв’язати цю
таємницю. З другого ж боку, якщо взяти до уваги, що дотеперішня
політична економія хоч напомац і підходила до розрізнення
сталого і змінного капіталу, але ніколи не спромоглась
ясно сформулювати його; що вона ніколи не представляла додаткову
вартість відокремлено від зиску, а зиск взагалі ніколи
не представляла у чистому вигляді в відміну від його різних
усамостійнених одна проти одної складових частин, — як промисловий
зиск, торговельний зиск, процент, земельна рента; що
вона ніколи грунтовно не аналізувала ріжниці в органічному
складі капіталу, а тому й утворення загальної норми зиску, —
то перестає бути загадковим те, що їй ніколи не вдавалося розв’язати
цю загадку.

Ми навмисно виклали цей закон раніше, ніж показали розпад
зиску на різні усамостійнені одна проти одної категорії. Незалежність
цього викладу від розпаду зиску на різні частини,
які припадають різним категоріям осіб, прямо доводить незалежність
закону в його всезагальності від такого розпаду і від
взаємних відношень між категоріями зиску, які виникають з цього
розпаду. Зиск, про який ми тут говоримо, є тільки інша назва
самої додаткової вартості, яка тільки представлена у відношенні
\parbreak{}  %% абзац продовжується на наступній сторінці
