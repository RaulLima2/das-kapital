\parcont{}  %% абзац починається на попередній сторінці
\index{iii1}{0334}  %% посилання на сторінку оригінального видання
вартість; коротко кажучи, те явище, що тут капітал як капітал
став товаром і що через це продаж перетворився в позику,
ціна — в частину зиску.

Повернення капіталу до своєї вихідної точки є взагалі характерним
рухом капіталу в його цілому кругобігу. Це властиве
не тільки капіталові, що дає процент. Для нього характерна
тільки зовнішня форма повернення, відокремлена від опосереднюючого
кругобігу. Капіталіст-позикодавець віддає свій
капітал, передає його промисловому капіталістові, не одержуючи
еквіваленту. Ця віддача взагалі не є актом дійсного процесу
кругобігу капіталу, а тільки підготовляє цей кругобіг, що повинен
бути виконаний промисловим капіталістом. Ця перша переміна
місця грошей не виражає акту метаморфози, ні купівлі, ні
продажу. Власність не відступається, бо не відбувається ніякого
обміну, не одержується ніякого еквіваленту. Повернення грошей
з рук промислового капіталіста назад в руки капіталіста-позикодавця
просто доповнює перший акт віддачі капіталу. Авансований
у грошовій формі капітал за допомогою процесу кругобігу
повертається знову до промислового капіталіста в грошовій
формі. Але тому що капітал не належав йому при витрачанні,
він не може належати йому і при поверненні. Проходження
цього капіталу через процес репродукції ні в якому разі не
може перетворити його у власність промислового капіталіста.
Отже, він повинен повернути його позикодавцеві. Перше витрачання
капіталу, яке переносить його з рук позикодавця в руки позичальника,
є юридична угода, яка не має нічого спільного з дійсним
процесом репродукції капіталу, а тільки підготовляє його.
Зворотна сплата, яка знову переносить капітал, що зворотно
приплив, з рук позичальника в руки позикодавця, є друга юридична
угода, доповнення першої; перша підготовляє дійсний
процес, друга є акт, що доповнює цей самий процес. Отже,
вихідна точка і точка повернення, віддача в позику і повернення
позиченого капіталу, виступають як самовільні рухи, опосереднювані
юридичними угодами, рухи, які відбуваються до і після
дійсного руху капіталу і які не мають ніякого відношення до самого
цього руху. Для цього останнього не мало б значення,
коли б капітал з самого початку належав промисловому капіталістові
і тому повертався б до нього просто як його власність.

В першому вступному акті позикодавець віддає свій капітал
позичальникові. В другому додатковому і кінцевому акті позичальник
повертає капітал позикодавцеві. Оскільки до уваги береться
тільки угода між обома, — і залишаючи покищо процент
осторонь, — оскільки, отже, справа йде тільки про рух самого
позиченого капіталу між позикодавцем і позичальником, обидва
ці акти (відділені один від одного довшим чи коротшим періодом
часу, на який припадає дійсний рух репродукції капіталу)
охоплюють весь цей рух. А цей рух: віддача під умовою повернення,
є взагалі рух віддання в позику і одержання позики,
\parbreak{}  %% абзац продовжується на наступній сторінці
