= 40\%. Отже, в другому випадку вона вдвоє більша, ніж у першому,
хоч у першому випадку норма додаткової вартості, = 100\%,
вдвоє більша, ніж у другому випадку, де вона становить тільки
50\%. Але зате однакової величини капітал привласнює собі в першому
випадку додаткову працю тільки 20, а в другому 80 робітників.
Закон прогресуючого падіння норми зиску або відносного
зменшення привласнюваної додаткової праці порівняно з масою
упредметненої праці, яка приводиться в рух живою працею, аж
ніяк не виключає зростання абсолютної маси праці, яка приводиться
в рух і експлуатується суспільним капіталом, а тому й зростання
абсолютної маси привласнюваної ним додаткової праці; так само
цей закон не виключає того, що капітали, які є в розпорядженні
окремих капіталістів, командують дедалі більшою масою праці,
а тому й додаткової праці, — останнє навіть у тому випадку,
коли число робітників, якими вони командують, не зростає.

Якщо взяти робітниче населення даної чисельності, наприклад,
два мільйони, якщо взяти, далі, як дані, довжину і інтенсивність
пересічного робочого дня, а також заробітну плату,
а разом з тим і відношення між необхідною і додатковою працею,
то сукупна праця цих двох мільйонів, а також їх додаткова
праця, яка виражається в додатковій вартості, завжди виробляє
вартість однакової величини. Але з зростанням маси сталого
— основного і обігового — капіталу, який приводиться в рух
цією працею, падає відношення цієї величини вартості до вартості
цього капіталу, яка зростає разом з його масою, хоч і не в тій
самій пропорції. Це відношення, а тому й норма зиску, падає, хоч
капітал командує такою самою масою живої праці, як і раніше,
і вбирає таку саму масу додаткової праці. Відношення змінюється
не тому, що зменшується маса живої праці, а тому, що збільшується
маса упредметненої вже праці, яку вона приводить в рух.
Зменшення тут відносне, не абсолютне, і в дійсності нічим
не зв’язане з абсолютною величиною приведеної в рух праці
й додаткової праці. Падіння норми зиску виникає не з абсолютного,
а тільки з відносного зменшення змінної складової частини
всього капіталу, з її зменшення. порівняно з сталою складовою
частиною.

Те саме, що має значення для даної маси праці і маси додаткової
праці, має значення і для зростаючого числа робітників, а тому, при
даних припущеннях, і для зростаючої маси праці, яка взагалі
є в розпорядженні, і зокрема для її неоплаченої частини, для
додаткової праці. Якщо робітниче населення зростає з двох мільйонів
до трьох, якщо змінний капітал, виплачений йому в формі
заробітної плати, так само становив раніше два мільйони, а тепер
становить три мільйони, а сталий капітал, навпаки, підвищується
з 4 до 15 мільйонів, то при даних припущеннях (незмінний
робочий день і незмінна норма додаткової вартості) маса додаткової
праці, додаткової вартості зростає наполовину, на 50\%,
