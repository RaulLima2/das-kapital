5 шилінгів“. Таким чином, індійський ринок переповнений Англією, а англійський — Індією в однаковій
мірі. І цей випадок мав місце якраз влітку 1857 року,
через неповних десять років після гіркого досвіду 1847 року!

Розділ двадцять шостий

Нагромадження грошового капіталу; його вплив
На розмір процента

„В Англії відбувається постійне нагромадження додаткового
багатства, яке має тенденцію, кінець-кінцем, набрати грошової
форми. Але бажання набувати гроші супроводиться найнастійливішим бажанням знову звільнитись від них
шляхом якого-небудь
застосування, що дає процент або зиск; бо гроші як гроші не
дають нічого. Тому, якщо одночасно з цим постійним припливом надлишкового капіталу не відбувається
ступневого і достатнього розширення поля діяльності для нього, то у нас
періодично нагромаджуються гроші, які шукають застосування,
при чому ці нагромадження, залежно від обставин, мають
більше чи менше значення. Протягом довгого ряду років державні борги були головним засобом
поглинення надлишкового
багатства Англії. З того часу, як державний борг досяг у
1816 році свого максимуму і більше вже не поглинає багатства, щороку виявлялася сума принаймні в 27
мільйонів, яка
шукала іншого застосування. До того ж відбувались ще зворотні
виплати капіталу різного роду... Підприємства, які для свого
здійснення потребують великих капіталів і які час від часу
відтягають надлишок незайнятого капіталу... принаймні для нашої країни абсолютно необхідні для того,
щоб відводити періодичні нагромадження надлишкового багатства суспільства, які не
можуть знайти собі місця в звичайних галузях застосування“
(„The Currency Theory Reviewed“, London 1845, стор. 32 [33, 34]).
Про 1845 рік сказано там же: „Протягом дуже короткого періоду ціни від найнижчої точки депресії
підскочили вгору... трипроцентна державна позика стоїть майже al pari [на рівні номінальної
вартості]... золото в підвалах Англійського банку перевищує всяку суму, яка будь-коли там
нагромаджувалась. На
акції всякого роду стоять ціни, майже ніколи нечувані, а розмір
процента так упав, що він майже номінальний... Все це доводить,
що в Англії тепер знову наявне тяжке нагромадження незайнятого
багатства, що в недалекому майбутньому знову матимемо період
спекулятивної гарячки“ (там же, стор. 36).

„Хоча довіз золота не є певною ознакою зисків у зовнішній
торгівлі, все ж частина цього золотого довозу, за відсутністю
іншого способу пояснення, репрезентує prima facie такий зиск“
(J. G. Hubbard: „The Currency and the Country“. London 1843,
стор. [40] 41). „Припустімо, що в такий період, коли справи
весь час добрі, ціни зисковні і грошовий обіг добре запов-
