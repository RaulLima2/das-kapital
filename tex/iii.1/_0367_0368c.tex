\parcont{}  %% абзац починається на попередній сторінці
\index{iii1}{0367}  %% посилання на сторінку оригінального видання
заробітної плати, на основі капіталістичного способу виробництва,
здається неодмінною складовою частиною зиску. Ця частина, як
це правильно виявив уже А. Сміт, виступає в чистому вигляді,
самостійно і цілком відокремлено, з одного боку, від зиску (як
суми процента і підприємницького доходу), з другого боку —
від тієї частини зиску, яка лишається після відрахування процента
як так званий підприємницький дохід, — виступає в формі утримання
управителя в таких галузях підприємств, розмір яких
і т. д. допускає поділ праці, достатній для того, щоб встановити
окрему заробітну плату для управителя.

Праця верховного нагляду й керівництва виникає необхідно
повсюди, де безпосередній процес виробництва має форму суспільно
комбінованого процесу, а не форму роз’єднаної праці
самостійних виробників.\footnote{
„Superintendence is here (у селянина-землевласника) completely dispensed
with“ [„Тут“ (у селянина-землевласника) „можна цілком обійтися без верховного
нагляду“] (J. E. Cairnes: „The Slave Power“. London 1862, стор. 48).
} Але вона має двоякий характер.

З одного боку, в усіх роботах, в яких співробітничають багато
індивідів; зв’язок і єдність процесу необхідно представлені н
одній управляючій волі і в функціях, які стосуються не до частинних
робіт, а до сукупної діяльності майстерні, як це має місце
з дирижером оркестру. Це — продуктивна праця, яку необхідно
виконувати при всякому комбінованому способі виробництва.

З другого боку, — цілком залишаючи осторонь купецький
відділ, — ця праця верховного нагляду необхідно виникає при
всіх способах виробництва, які грунтуються на протилежності
між робітником, як безпосереднім виробником, і власником засобів
виробництва. Чим більша ця протилежність, тим більша
роль, що її відіграє ця праця верховного нагляду. Тому свого
максимуму вона досягає в системі рабства.\footnote{
„If the nature of the work requires that the workmen (саме рабів) should
be dispersed over an extended area, the number of overseers and, therefore, the
cost of the labour which requires this supervision, will be proportionately increased“
[„Якщо, характер праці вимагає розподілу робітників“ (саме рабів) „на
великому просторі, то відповідно до цього зростає число наглядачів, а тому й
витрати на працю, якої вимагає цей нагляд“] (Cairnes: там же, стор. 44).
} Але вона необхідна
і при капіталістичному способі виробництва, бо тут процес
виробництва є разом. з тим процес споживання робочої
сили капіталістом. Цілком так само, як у деспотичних державах
праця верховного нагляду і всебічного втручання уряду
охоплює обидві сторони: як виконання спільних справ, що випливають
з природи усякого суспільства, так і специфічні функції,
що випливають з протилежності між урядом і народною
масою.

В античних письменників, які безпосередньо спостерігали
систему рабства, обидві сторони праці нагляду нероздільно поєднані
в теорії, як це мало місце й на практиці, — цілком так
само, як у сучасних економістів, які вважають капіталістичний
спосіб виробництва за абсолютний спосіб виробництва. З другого
\index{iii1}{0368}  %% посилання на сторінку оригінального видання
боку, як я це зараз покажу на одному прикладі, апологети
сучасної системи рабства цілком так само уміють використовувати
працю нагляду як довід для виправдання рабства, як інші
економісти — для виправдання системи найманої праці.

Villicus за часів Катона: „На чолі рабовласницького господарства
(familia rustica) стояв управитель (villicus вілли), який відає
прибутками й видатками, купує і продає, дістає розпорядження
від пана і в його відсутності розпоряджається й карає...
Управитель користувався, звичайно, більшою свободою, ніж
усі інші раби; Маго в своїх книгах радить дозволяти йому одружуватись,
родити дітей і мати власні кошти, а Катон радить
одружувати його з управителькою; тільки він міг сподіватися,
в разі доброї поведінки, дістати від свого пана волю. В усьому
іншому всі становили спільне домашнє господарство.... Кожен
раб, навіть і сам управитель, одержував своє утримання від
пана в певні строки за твердо встановленими нормами, і цього
йому повинно було вистачати... Кількість регулювалась відповідно
до праці, в наслідок чого управитель, наприклад, який
виконував легшу працю, ніж раби, одержував менше, ніж ці
останні“ (Mommsen: „Römische Geschichte“. Друге видання, [Берлін]
1856, І, стор. 809—810).

Арістотель: Ο γάρ δεσπότης οὐχ ἐν τω  χτάσθαι τους δούλους, ἀλλ’ ἐν τω
χρῆσθαι δούλης. [Бо пан — капіталіст — виявляється як такий не в
набуванні рабів — власності на капітал, яка дає владу купувати
працю, — а у використанні рабів — вживанні робітників — нині
найманих робітників у процесі виробництва]. Ἔστι δέ αὕτη ἡ επιστήμη
οὐδέν μεγα ἔχουσα οὐδέ οεμνόν. [Але в цій науці немає нічого великого
або величного]; ἄ γάρ τόν δοῦλον ἔπιστασθαι δεῖ ποιεῖν, έχεῖνον δεῖ
ταῦτα ἐπίστασθαι ἐπιτάττειν [він повинен уміти наказувати те, що раб
повинен уміти виконати]. Διο οσοις ἐξουσία μή αυτούς χαχοπαθειν, επιτροπος
λαμβανει ταυτην την τιμην, αυτοι δε πολιτευονται φιλοσοφουσιν.
[Коли в панів немає потреби обтяжувати себе цим, цю честь
бере на себе наглядач, а вони самі займаються державними
справами або філософією]. (Aristoteles: „De Republica“. Вид.
Беккера. Книга І, 7 [Охоnіі 1837, стор. 10 і далі]).

Арістотель прямо говорить, що панування як у політичній,
так і в економічній галузі покладає на владарів функції панування,
тобто що в економічній галузі вони повинні вміти споживати
робочу силу, і додає до цього, що цій праці нагляду не слід надавати
великого значення, і тому пан, якщо він досить заможний,
передає „честь“ цих турбот наглядачеві.

Праця керівництва й верховного нагляду, оскільки вона не є
особлива функція, що випливає з природи всякої комбінованої
суспільної праці, а виникає з протилежності між власником засобів
виробництва і власником самої тільки робочої сили — однаково,
чи ця остання купується разом з самим робітником, як при
системі рабства, чи робітник сам продає свою робочу силу, і тому
процес виробництва є разом з тим і процесом споживання
\parbreak{}  %% абзац продовжується на наступній сторінці
