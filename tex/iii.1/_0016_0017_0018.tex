\parcont{}  %% абзац починається на попередній сторінці
\index{iii1}{0016}  %% посилання на сторінку оригінального видання
перенесення певної частини вартості сукупного продукту до
класу капіталістів“.

Отож не треба великого напруження мислі, щоб переконатися,
що це „вульгарноекономічне“ пояснення зиску на капітал
практично веде до тих самих результатів, як і теорія додаткової
вартості Маркса; що, за уявленням Лексіса, робітники
перебувають точно в такому самому „несприятливому становищі“,
як і в Маркса; що вони цілком так само обдурені, бо
кожен неробітник може продавати вище ціни, а робітник цього
не може; і що на основі цієї теорії можна збудувати принаймні
настільки ж поверховий вульгарний соціалізм, як той, що збудований
тут в Англії на основі теорії споживної вартості та
теорії граничної корисності Джевонса-Менгера. Я навіть думаю,
що коли б ця теорія зиску була відома панові Джорджеві Бернардові
Шоу, він міг би ухопитися за неї обома руками, дати
відставку Джевонсові та Карлові Менгеру і наново збудувати
на цій скелі фабіанську церкву майбутнього.

Але в дійсності ця теорія є лише парафраза теорії Маркса.
З чого ж покриваються всі надбавки до ціни? З „сукупного
продукту“ робітників. І саме в наслідок того, що товар „праця“,
або, як каже Маркс, робоча сила, мусить продаватися нижче її
ціни. Бо якщо спільна властивість усіх товарів є в тому, що їх
можна продавати дорожче витрат виробництва, а праця становить
єдиний виняток з цього і продається завжди тільки по витратах
виробництва, то вона продається якраз нижче тієї ціни, яка є
загальним правилом у цьому вульгарноекономічному світі. Надзиск,
який в наслідок цього припадає капіталістові або класові
капіталістів, полягає саме в тому і в кінцевому рахунку може
постати тільки тому, що робітник, після репродукції заміщення
ціни своєї праці, мусить ще далі виробляти продукт, за який
йому не платять, — додатковий продукт, продукт неоплаченої
праці, додаткову вартість. Лексіс — людина надзвичайно обережна
у виборі своїх висловів. Він ніде не каже прямо, що
вищенаведене розуміння є його власне; але якщо це так, то
цілком ясно, що ми тут маємо справу не з одним з тих звичайних
вульгарних економістів, про яких він сам каже, що кожний
з них в очах Маркса є „в кращому разі тільки безнадійний недоумок“,
а з марксистом, який переодягнувся вульгарним економістом.
Чи сталося це переодягнення свідомо чи несвідомо, це
є психологічне питання, яке нас тут не цікавить. Той, хто схотів
би з’ясувати це, може, дослідив би також, як могло статися, що
така безперечно розумна людина, як Лексіс, могла певний час
боронити таке безглуздя, як біметалізм.

Перший, хто дійсно намагався дати відповідь на питання, був
д-р \emph{Конрад Шмідт}: „Die Durchschnittsprofitrate auf Grundlage des
Marxschen Wertgesetzes“, Stuttgart, Dietz 1889. Шмідт намагається
погодити деталі утворення ринкових цін як із законом
вартості, так і з пересічною нормою зиску. Промисловий капіталіст
\index{iii1}{0017}  %% посилання на сторінку оригінального видання
одержує у своєму продукті, поперше, заміщення свого
авансованого капіталу, подруге, додатковий продукт, за який
він нічого не заплатив. Але, щоб одержати цей додатковий
продукт, він мусить авансувати свій капітал на виробництво;
тобто він мусить застосувати певну кількість упредметненої
праці, щоб мати можливість привласнити собі цей додатковий продукт.
Отже, для капіталіста цей його авансований капітал є кількість
упредметненої праці, суспільно-потрібна для того, щоб
створити йому цей додатковий продукт. Це має силу і для всякого
іншого промислового капіталіста. А тому що за законом
вартості продукти обмінюються один на один пропорціонально
до праці, суспільно-необхідної для їх виробництва, і що для капіталіста
праця, необхідна для виготовлення його додаткового продукту,
є якраз нагромаджена в його капіталі минула праця, то
з цього випливає, що додаткові продукти обмінюються пропорціонально
до капіталів, потрібних для їх виробництва, а не пропорціонально
до \emph{дійсно} втіленої в них праці. Отже, частка, що
припадає на кожну одиницю капіталу, дорівнює сумі всіх вироблених
додаткових вартостей, поділеній на суму застосованих
для цього капіталів. Тому однакові капітали за однакові проміжки
часу дають однаковий зиск, і це спричинюється тим, що
вираховані так витрати виробництва (Kostpreis) додаткового
продукту, тобто пересічний зиск, додаються до витрат виробництва
оплаченого продукту, і по цій підвищеній ціні продаються
обидва, оплачений і неоплачений продукт. Встановлюється
пересічна норма зиску, не зважаючи на те, що, як
думає Шмідт, пересічні ціни окремих товарів визначаються за
законом вартості.

Конструкція надзвичайно дотепна, вона цілком на гегелівський
зразок, але вона має те спільне з більшою частиною гегелівського,
що вона неправильна. Додатковий продукт чи оплачений
продукт — це не робить ріжниці: якщо закон вартості повинен
\emph{безпосередньо} мати силу і для пересічних цін, то і той і другий
продукт мусить продаватися пропорціонально до суспільнонеобхідної
праці, потрібної і спожитої на їх виготовлення. Закон
вартості з самого початку спрямований проти погляду, який перейшов
від капіталістичного способу уявлення, ніби нагромаджена
минула праця, з якої складається капітал, є не просто певна сума
готової вартості, а, як фактор виробництва й утворення зиску, має
також властивість створювати вартості, отже, є джерелом більшої
вартості, ніж має сам капітал; закон вартості твердо встановлює,
що ця властивість належить тільки живій праці. Те, що капіталісти
сподіваються рівного зиску, пропорціонального до величини
їх капіталів, отже, розглядають авансовані ними капітали як свого
роду витрати виробництва їхнього зиску — це відомо. Але якщо
Шмідт використовує це уявлення, щоб за його допомогою погодити
з законом вартості ціни, обраховані за пересічною нормою
зиску, то він скасовує (hebt... auf) самий закон вартості, приєднуючи
\index{iii1}{0018}  %% посилання на сторінку оригінального видання
до нього як співвизначальний фактор уявлення, ділком
йому суперечне.

Або нагромаджена праця поряд з живою працею створює
вартість. Тоді закон вартості не має сили.

Або вона не створює вартості. Тоді доводи Шмідта несполучні
з законом вартості.

Шмідт збився з правильного шляху, коли він був уже дуже
близько до розв’язання проблеми, бо гадав, що треба обов’язково
знайти математичну формулу, яка дала б можливість довести
погодженість пересічної ціни кожного окремого товару з законом
вартості. Але якщо тут, бувши зовсім близько до мети, він
пішов хибним шляхом, то в усьому іншому зміст брошури показує,
з яким розумінням він зробив дальші висновки з обох перших
книг „Капіталу“. Йому належить честь самостійного відкриття
правильного пояснення непояснимої до того часу тенденції норми
зиску до зниження, пояснення, даного Марксом у третьому відділі
третьої книги; так само виведення торговельного зиску з
промислової додаткової вартості і цілий ряд уваг про процент
та земельну ренту, в яких ним передхоплені речі, розвинені у
Маркса в четвертому і п’ятому відділах третьої книги.

В одній пізнішій праці („Neue Zeit“ 1892/93, №№ 3 і 4)
Шмідт намагається розв’язати проблему іншим шляхом. Цей
шлях зводиться до того, що пересічну норму зиску встановлює
конкуренція, бо вона примушує капітал переходити з галузей
виробництва з недостатнім зиском до інших, де добувається
надзиск. Що конкуренція є велика зрівняльниця зисків, це не
новина. Але Шмідт намагається довести, що це нівелювання
зисків тотожне із зведенням продажної ціни товарів, вироблених
понад міру, до такої міри вартості, яку суспільство може
заплатити за них згідно з законом вартості. Чому і це не
могло привести до цілі, досить видно з пояснень Маркса в самій
книзі.

Після Шмідта до проблеми взявся \emph{П. Фіреман} („Conrads
Jahrbücher“, Dritte Folge [1892], III, стор. 793). Я не спиняюся на
його увагах про інші сторони викладу в Маркса. Вони грунтуються
на тому непорозумінні, ніби Маркс хоче дати визначення
там, де він в дійсності розвиває, і на тому, що в Маркса взагалі
довелося б пошукати точних, готових, раз назавжди даних
дефініцій. Адже само собою зрозуміло, що там, де речі та їх
взаємовідношення розглядаються не як сталі, а як мінливі, їх
мислені відбитки, поняття, теж зазнають зміни та перетворення;
що їх не втискують у закам’янілі дефініції, а розглядають в їх
історичному або логічному процесі утворення. Після цього стане,
звичайно, ясно, чому Маркс на початку першої книги, де він
виходить з простого товарного виробництва, яке є для нього
історичною передумовою, щоб потім далі перейти від цієї бази
до капіталу, — чому він там виходить саме з простого товару,
а не з форми, логічно і історично вторинної, не з капіталістично
\parbreak{}  %% абзац продовжується на наступній сторінці
