ної величини змінного капіталу і норми додаткової вартості. Але ми показали, що пересічно ті самі
причини, які підвищують норму відносної додаткової вартості, зменшують масу вживаної робочої сили.
Проте, ясно, що збільшення або зменшення тут відбувається залежно від певного відношення, в якому
відбувається цей протилежний рух, і що тенденція до зменшення норми зиску послаблюється зокрема в
наслідок підвищення норми абсолютної додаткової вартості, яка походить із здовження робочого дня.

При дослідженні норми зиску ми взагалі виявили, що зниженню норми, яке відбувається в наслідок
зростання маси всього застосовуваного капіталу, відповідає збільшення маси зиску. Якщо розглядати
сукупний змінний капітал суспільства, то
створена ним додаткова вартість дорівнює створеному зискові. Разом з абсолютною масою додаткової
вартості виросла і норма додаткової вартості; перша виросла тому, що збільшилась вживана
суспільством маса робочої сили, друга — тому, що підвищився ступінь експлуатації цієї праці. Але
відносно капіталу даної величини, наприклад, 100, норма додаткової вартості може зрости, тоді як
маса її пересічно падає; бо норма визначається відношенням, в якому змінна частина капіталу зростає
в своїй вартості, а маса визначається, навпаки, тією відносною частиною, яку становить змінний
капітал в усьому капіталі.

Підвищення' норми додаткової вартості — через те що воно відбувається і при таких обставинах, коли,
як це показано вище, не відбувається ніякого збільшення або не відбувається пропорціонального
збільшення сталого капіталу порівняно з змінним — є один з факторів, яким визначається маса
додаткової вартості, а тому й норма зиску. Цей фактор не знищує загального закону. Але він робить
те, що цей закон діє більше як тенденція, тобто як закон, абсолютне здійснення якого затримується,
уповільнюється і ослаблюється протидіючими обставинами. Але через те що ті самі причини, які
підвищують норму додаткової вартості (навіть здовження робочого дня є результат великої
промисловості), мають тенденцію зменшувати вживану даним капіталом кількість робочої сили, то одні й
ті самі причини мають тенденцію зменшувати норму зиску і уповільнювати рух цього зменшення. Якщо
одному робітникові накидають таку працю, яку раціонально виконати можуть тільки двоє, і якщо це
відбувається при таких обставинах, коли цей робітник може заступити трьох, то один робітник даватиме
стількидодаткової праці, скільки раніш давало двоє, і остільки норма додаткової вартості
підвищиться. Але він не даватиме стільки, скільки раніш давало троє, і остільки маса додаткової
вартості знизиться. Але це зниження компенсується або обмежується підвищенням норми додаткової
вартості. Якщо все населення працюватиме при підвищеній нормі додаткової вартості, то маса
додаткової вартості збільшиться, хоч населення лишиться тим
