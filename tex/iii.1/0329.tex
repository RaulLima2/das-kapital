Як тільки продуктивний капітал перетворився в товарний капітал,
він мусить бути кинутий на ринок, проданий як товар.
Тут він функціонує просто як товар. Капіталіст виступає тут
тільки як продавець товару, подібно до того як покупець —
тільки як покупець товару. Як товар продукт мусить у процесі
циркуляції, через свій продаж, реалізувати свою вартість, набрати
свого перетвореного вигляду, вигляду грошей. В наслідок
цього цілком байдуже також, чи купується цей товар споживачем
як засоби існування, чи капіталістом як засоби виробництва,
як складова частина капіталу. В акті циркуляції товарний
капітал функціонує тільки як товар, не як капітал. Це — товарний капітал у відміну від простого
товару: 1) тому що він
уже вагітний додатковою вартістю, отже, реалізація його вартості
є разом з тим реалізація додаткової вартості; але це нічого не
змінює в тому, що він існує просто як товар, як продукт певної
ціни; 2) тому що ця його функція як товару є момент процесу
репродукції його як капіталу, і тому його рух як товару,
будучи тільки частиною пророблюваного ним руху, разом з тим
є його рухом як капіталу; але вона стає такою не в наслідок
самого акту продажу, а тільки в наслідок зв’язку цього акту
з сукупним рухом цієї певної суми вартості як капіталу.

Як грошовий капітал, він фактично так само діє просто тільки
як гроші, тобто як засіб купівлі товарів (елементів виробництва).
Що ці гроші тут є разом з тим грошовим капіталом, формою капіталу,
це випливає не з акту купівлі, не з дійсної функції, яку
він виконує тут як гроші, а із зв’язку цього акту з сукупним
рухом капіталу, бо цей акт, який він виконує як гроші, є вступ
до капіталістичного процесу виробництва.

Але оскільки товарний капітал і грошовий капітал дійсно
функціонують, дійсно грають свою роль у процесі, товарний капітал
діє тут тільки як товар, грошовий капітал — тільки як гроші.
Ні в одному з окремих моментів метаморфози, розглядуваних самі
по собі, капіталіст не продає покупцеві товар як капітал, хоч
для нього товар представляє капітал, і не відчужує продавцеві
грошей як капітал. В обох випадках він відчужує товар
просто як товар і гроші просто як гроші, як засіб купівлі
товару.

Капітал виступає в процесі циркуляції як капітал тільки в
загальному зв’язку всього процесу, в тому моменті, в якому вихідна
точка являє собою разом з тим точку повернення назад, в Г — Г' або Т — T' (тимчасом як у процесі
виробництва він виступає як
капітал в наслідок підпорядкування робітника капіталістові і в
наслідок виробництва додаткової вартості). Але в цьому моменті
повернення до вихідної точки опосереднююча ланка зникла.
Що тут наявне, так це Г' або Г + ΔГ (байдуже, чи існує тепер
сума вартості, збільшена на ΔГ, у формі грошей, чи товару,
чи елементів виробництва), грошова сума, рівна первісній авансованій
грошовій сумі + певний надлишок понад неї, реалізо-
