\parcont{}  %% абзац починається на попередній сторінці
\index{iii1}{0147}  %% посилання на сторінку оригінального видання
45—50). „Хоч подекуди ост-індська бавовна, можливо, перероблялася
з зиском для фабрикантів, проте, ми бачимо (див. розрахунковий
лист заробітків, стор. 53), що робітники порівняно
з 1861 роком терплять від цього. Якщо вживання сурату укріпиться,
то робітники вимагатимуть такого самого заробітку, як
в 1861 році; але таке підвищення заробітної плати серйозно
відбилося б на зиску фабриканта, якщо тільки воно не скомпенсується
ціною або бавовни, або фабрикатів“ (стор. 105).

\emph{Квартирна плата}. „Квартирна плата робітників, в тих випадках,
коли котеджі, в яких вони живуть, належать фабрикантові,
часто відраховується фабрикантом із заробітної плати,
навіть коли робітники працюють неповний час. Не зважаючи
на це, вартість цих будівель знизилась, і хатки тепер можна
мати на 25—50\% дешевше, ніж раніше; котедж, який раніше
коштував 3 шилінги 6 пенсів за тиждень, тепер можна мати за шилінги 4 пенси, а іноді ще дешевше“ (стор. 57).

\emph{Еміграція}. Фабриканти, звичайно, були проти еміграції робітників,
почасти тому, що вони, „чекаючи кращих часів для бавовняної
промисловості, хотіли зберегти в своїх руках засоби
для того, щоб провадити виробництво на своїх фабриках якнайвигідиішим
способом“. Але, крім того, „багато фабрикантів
є власники будинків, в яких живуть заняті ними робітники,
і принаймні деякі з них безумовно розраховують на те, що
пізніше одержать частину квартирної плати, яку їм заборгували
робітники“ (стор. 96).

Пан Берноль Осборн каже в одній з промов до своїх виборців
у парламент від 22 жовтня 1864 року, що робітники
Ланкашіра поводились як античні філософи (стоіки). Чи не як вівці?

\section{Додатки}

Припустімо, як ми це робимо в цьому відділі, що маса
зиску, привласнювана в кожній окремій сфері виробництва, дорівнює
сумі додаткової вартості, яку створює весь капітал, вкладений
у цю сферу. Все ж буржуа не сприйматиме зиск як тотожний
з додатковою вартістю, тобто з неоплаченою додатковою
працею, і саме з таких причин:

1) В процесі циркуляції він забуває процес виробництва.
Реалізація вартості товарів — яка включає і реалізацію вміщеної
в них додаткової вартості — йому здається утворенням додаткової
вартості. [Незаповнена прогалина в рукопису вказує на те,
що Маркс мав намір докладніше розвинути цей пункт. — \emph{Ф. Е.}]

2) Якщо припустити незмінний ступінь експлуатації праці, то,
як уже виявилось, норма зиску, незалежно від усіх викликаних
кредитною системою модифікацій, від усякого взаємного ошуканства
і шахрайства капіталістів, незалежно, далі, від усякого
\parbreak{}  %% абзац продовжується на наступній сторінці
