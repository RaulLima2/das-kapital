\index{iii1}{0100}  %% посилання на сторінку оригінального видання
\emph{Подруге}: Оскільки ці засоби виробництва в капіталістичному
процесі виробництва є разом з тим засобами експлуатації праці,
відносна дешевина чи дорожнеча цих засобів експлуатації так
само мало турбує робітника, як, наприклад, коня та обставина,
чи дорогими чи дешевими вудилами правлять ним.

\emph{Нарешті}, як ми бачили раніше, робітник дійсно ставиться
до суспільного характеру своєї праці, до її комбінації з працею
інших для спільної мети, як до якоїсь чужої йому сили; умовою
здійснення цієї комбінації є чужа йому власність, марнування
якої для нього було б цілком байдужим, коли б його не примушували
економізувати її. Цілком інакше стоїть ця справа
на фабриках, що належать самим робітникам, наприклад, у Рочделі.

Отже, ледве чи треба згадувати, що оскільки продуктивність
праці в одній галузі виробництва виявляється як подешевшення
й поліпшення засобів виробництва в іншій галузі і тим самим
підвищує норму зиску, цей загальний зв’язок суспільної праці
виступає як щось цілком чуже робітникам, що на ділі торкається
тільки капіталіста, оскільки тільки він купує і привласнює
собі ці засоби виробництва. Те, що він купує продукт робітників чужої галузі виробництва на продукт
робітників своєї власної галузі виробництва, і тому розпоряджається продуктом чужих робітників лиш
остільки, оскільки він даром привласнив собі продукт своїх власних робітників, — це є таке
відношення,
яке щасливо приховується процесом циркуляції і т. д.

Сюди долучається ще те, що подібно до того, як виробництво у великому масштабі розвивається вперше в
капіталістичній формі, так і жадоба до зиску, з одного боку, і, з другого — конкуренція, яка
примушує до якомога дешевшого виробництва товарів, надають цій економії в застосуванні сталого
капіталу такого вигляду, ніби вона є специфічною властивістю
капіталістичного способу виробництва, а тому функцією капіталіста.

Подібно до того, як капіталістичний спосіб виробництва,
з одного боку, спонукає до розвитку продуктивних сил суспільної праці, він, з другого боку, спонукає
до економії в застосуванні сталого капіталу.

Справа не обмежується, однак, цим відношенням відчуженості
й байдужості між робітником, носієм живої праці, і економним,
тобто раціональним і ощадним, застосуванням умов його праці.
В силу своєї суперечливої, антагоністичної природи капіталістичний спосіб виробництва йде далі і
приводить до того, що само
марнування життя і здоров’я робітника, зниження умов його
існування залічується до економії в застосуванні сталого капіталу, і тим самим до засобів підвищення
норми зиску.

Через те що робітник найбільшу частину свого життя проводить у процесі виробництва, умови процесу
виробництва
в значній частині є умовами його активного життьового процесу, умовами його існування, і економія на
цих умовах існування
\parbreak{}  %% абзац продовжується на наступній сторінці
