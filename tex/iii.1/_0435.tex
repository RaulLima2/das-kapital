\parcont{}  %% абзац починається на попередній сторінці
\index{iii1}{0435}  %% посилання на сторінку оригінального видання
і необхідних супутників низького вексельного курсу; попит у таких
випадках є попит не на циркуляцію“ (слід було б сказати:
не на циркуляцію як на засоби купівлі), „а для утворення скарбу,
попит з боку частини стривожених банкірів і капіталістів, який
звичайно виникає в останньому акті кризи“ (отже, попит на резерв
засобів платежу) „після довгочасного відпливу золота і який є
провісником скінчення цього відпливу“] (Fullarton, стор. 130).

При дослідженні грошей як засобу платежу (книга І, розд. III,
З, b) було вже показано, як при раптовому перериві ланцюга платежів
гроші перетворюються відносно товарів з просто ідеальної
форми у речову і разом з тим абсолютну форму вартостi.
Там же ми навели деякі приклади цього, в примітках 100 та 101.
Цей перерив сам є почасти наслідок, а почасти причина потрясіння
кредиту і тих обставин, які супроводять його: переповнення
ринків, знецінення товарів, припинення виробництва і т. д.

Ясно, однак, що Фуллартон перетворює ріжницю між грішми як
засобом купівлі і грішми як засобом платежу в помилкову ріжницю
між currency [засобом циркуляції] і капіталом. Але в основі
цього знову таки лежить обмежене банкірське уявлення про
циркуляцію. —

Можна було б ще запитати: чого ж не вистачає в такі часи
скрути — капіталу чи грошей, в їх визначеності як засобу платежу?
А в цьому, як відомо, і полягає спір.

Насамперед, оскільки скрута виявляється у відпливі золота,
ясно, що вимагають саме міжнародних засобів платежу. Але
гроші, в їх визначеності як міжнародного засобу платежу, є
золото в його металічній реальності, як сама по собі дорогоцінна
субстанція, маса вартості. Разом з тим вони є капітал, але капітал
не як товарний капітал, а як грошовий капітал, капітал не
в формі товару, а в формі грошей (і при тому грошей у тому
переважному значенні слова, в якому вони існують як загальний
товар світового ринку). Тут немає протилежності між попитом,
на гроші як засобом платежу і попитом на капітал. Протилежність
існує між капіталом у його формі грошей і капіталом у
його формі товару; і та форма, в якій його тут вимагають і в
якій він тільки й може функціонувати, є його грошова форма.

Залишаючи осторонь цей попит на золото (або срібло), не
можна сказати, що в такі періоди кризи відчувається будьяким
чином недостача капіталу. При надзвичайних обставинах, як,
наприклад, подорожчання хліба, недостача бавовни і т. д., це
може мати місце; але ці обставини ніяк не є необхідні або звичайні
супутники таких періодів; і тому не можна з самого початку
робити висновок про існування такої недостачі капіталу
з тієї обставини, що існує великий попит на грошові позики.
Навпаки. Ринки переповнені, наводнені товарним капіталом.
Отже, в усякому разі скрута викликається не недостачею товарного
капіталу. Ми повернемось до цього питання пізніше.
\parbreak{}  %% абзац продовжується на наступній сторінці
