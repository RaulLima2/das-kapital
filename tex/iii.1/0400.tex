достачі в засобах циркуляції, в наявних грошах. Але одна й та сама
сума грошового капіталу може бути віддана в позику за допомогою дуже різних кількостей засобів
циркуляції.

Візьмімо його приклад 1847 року. Офіціальний банковий процент був: у січні 3—3 1/2%, в лютому 4—4
1/2%, в березні здебільшого 4%, У квітні (паніка) 4—7 1/2%, в травні 5—5 1/2%, в червні загалом 5%,
в липні 5%, в серпні 5—5 1/2%, у вересні 5% з незначними
коливаннями до 5 1/4, 5 1/2, 6%, у жовтні 5, 5 1/2, 7%, в листопаді 7—10%, у грудні 7—5%. — В цьому
випадку процент підвищувався,
тому що зиски зменшувались і грошові вартості товарів надзвичайно впали. Отже, якщо Оверстон каже
тут, що розмір процента
в 1847 році підвищився, тому що вартість капіталу підвищилась,
то під вартістю капіталу він може тут розуміти тільки вартість
грошового капіталу, а вартість грошового капіталу є саме розмір
процента і ніщо інше. Але потім показується лисячий хвіст,
і вартість капіталу ототожнюється з нормою зиску.

Щодо високого розміру процента, який платився в 1856 році,
то Оверстон дійсно не знав, що він почасти був симптомом того,
що з’явився такий вид рицарів кредиту, які сплачували процент
не з зиску, а з чужого капіталу; всього лише за декілька місяців
перед кризою 1857 року він твердив, що „стан справ цілком
здоровий“.

Далі він каже: „3722. Уявлення, ніби зиск підприємства знищується в наслідок підвищення розміру
процента, в найвищій
мірі помилкове. По-перше, підвищення розміру процента рідко
буває довгочасним; по-друге, якщо воно й буває довгочасним
і значним, то по суті справи воно є підвищенням вартості капіталу; а чому підвищується вартість
капіталу? Тому що підвищилась норма зиску“. — Отже, тут ми, нарешті, дізнаємося,
який сенс має „вартість капіталу“. Зрештою, норма зиску може
протягом довгого часу лишатись високою, але підприємницький
дохід — упасти, а розмір процента підвищитись, так що процент поглине найбільшу частину зиску.

„3724. Підвищення розміру процента було наслідком колосального
розширення в ділах нашої країни і великого підвищення норми зиску; і коли скаржаться, що підвищений
розмір
процента руйнує ті самі дві речі, які були його власною причиною, то це логічний абсурд, про який не
знаєш, що й сказати“. — Це якраз настільки ж логічно, як коли б він сказав:
Підвищена норма зиску була наслідком підвищення товарних
цін спекуляцією, і коли скаржаться, що підвищення цін руйнує
свою власну причину, а саме спекуляцію, то це логічний абсурд
і т. д. Що річ може кінець-кінцем зруйнувати свою власну причину, це логічний абсурд тільки для
лихваря, закоханого у високий процент. Величність римлян була причиною їхніх завоювань,
а їхні завоювання зруйнували їхню величність. Багатство — причина
розкоші, а розкіш руйнуюче впливає на багатство. Отакий
мудрець! Ідіотизм сучасного буржуазного світу якнайкраще
