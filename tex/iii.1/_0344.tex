\index{iii1}{0344}  %% посилання на сторінку оригінального видання
Припустімо спочатку, що існує постійне відношення між усім
зиском і тією його частиною, яка повинна бути сплачена як процент
грошовому капіталістові. В такому разі очевидно, що процент
підвищуватиметься або падатиме разом з усім зиском, а цей
останній визначається загальною нормою зиску та її коливаннями.
Коли б, наприклад, пересічна норма зиску була = 20\%, а процент
= 1/4 зиску, то розмір процента = 5\%; коли б пересічна
норма зиску була = 16\%, то процент = 4\%. При нормі зиску в
20\% процент міг би підвищитись до 8\% і промисловий капіталіст
все ж одержував би той самий зиск, що й при нормі зиску
= 16\% і розмірі процента = 4\%, а саме 12\%. Коли б процент
підвищився тільки до 6 або 7\%, то він все ще залишав би собі
більшу частину зиску. Коли б процент дорівнював якійсь постійній
частині пересічного зиску, то з цього випливало б, що
чим вища загальна норма зиску, тим більша абсолютна ріжниця
між усім зиском і процентом, тим більша, отже, та частина
всього зиску, яка дістається функціонуючому капіталістові, і навпаки.
Припустім, що процент = 1/5 пересічного зиску. 1/5 від 10
є 2; ріжниця між усім зиском і процентом = 8. 1/5 від 20 = 4;
ріжниця = 20 — 4 = 16; 1/5 від 25 = 5; ріжниця = 25 — 5 = 20; 1/5 від
30 = 6; ріжниця = 30—6 = 24; 1/5 від 35—7; ріжниця = 35—7 = 28.
Різні норми процента в 4, 5, 6, 7\% тут весь час виражали б
тільки 1/5, або 20\%, всього зиску. Отже, якщо норми зиску є різні,
то різні норми процента можуть виражати ту саму відповідну
частину всього зиску або ту саму процентну частину всього
зиску. При такому постійному відношенні процента промисловий
зиск (ріжниця між усім зиском і процентом) був би тим більший,
чим вища загальна норма зиску, і навпаки.

При інших однакових умовах, тобто припускаючи відношення
між процентом і всім зиском за більш-менш постійне, функціонуючий
капіталіст буде спроможний і згодиться платити вищий
або нижчий процент у прямому відношенні до висоти норми
зиску.\footnote{
„The natural rate of interest із governed by the profits of trade to particulars“
(„Природна норма процента регулюється зиском окремих підприємств“]
(Massie, там же, стор. 51).
} Ми вже бачили, що висота норми зиску стоїть у зворотному
відношенні до розвитку капіталістичного виробництва;
звідси випливає, що вищий чи нижчий розмір процента в даній
країні стоїть у такому самому зворотному відношенні до висоти
промислового розвитку, якщо тільки ріжниця в розмірі
процента дійсно виражає ріжницю норм зиску. Пізніше ми побачимо,
що немає ніякої необхідності в тому, щоб це завжди
було так. В цьому розумінні можна сказати, що процент регулюється
зиском, точніше, загальною нормою зиску. І цей спосіб
його регулювання поширюється навіть на його пересічний розмір.

В усякому разі, пересічну норму зиску слід розглядати як
остаточно визначальну максимальну межу процента.
