\parcont{}  %% абзац починається на попередній сторінці
\index{iii1}{0116}  %% посилання на сторінку оригінального видання
матеріалу перетвориться в процесі виробництва у відпади. Нарешті, це залежить від якості самого
сировинного матеріалу. А ця остання знов таки залежить почасти від розвитку добувної промисловості і
землеробства, які виробляють сировинний матеріал (від прогресу культури у власному значенні цього
слова), почасти від виробленості тих процесів, які проходить
сировинний матеріал перед тим, як вступає в мануфактуру.

„Пармантьє довів, що з не дуже давніх часів, наприклад, з часу Людовіка XIV, умілість молоти зерно у
Франції дуже
значно удосконалилась, так що нові млини порівняно з старими
можуть давати з тієї самої кількості зерна майже наполовину
більше хліба. Дійсно, річне споживання на одного жителя Парижа
спочатку рахували в 4 сетьє зерна, потім у 3, нарешті в 2, тим часом як тепер воно становить тільки
1 1/3 сетьє, або приблизно
342 фунти на душу... В Перше, де я довго жив, грубо збудовані млини, які мали жорна з граніту й
трапу, здебільшого перебудовано за правилами механіки, яка зробила такі великі успіхи за останні 30
років. їх устаткували добрими жорнами La Ferté,
зерно почали молоти двічі, ситу надали колового руху, внаслідок чого кількість продукту борошном з
тієї самої кількості
зерна збільшилась на 1/6. Отже, я легко пояснюю собі величезну
ріжницю між щоденним споживанням зерна у римлян і в нас;
причина полягає просію в недостатній досконалості способів
помолу зерна і виготовлення хліба. Цим самим я мушу пояснити
також той дивовижний факт, що його наводить Пліній, XVIII,
розд. 20, 2... „Борошно продавалося в Римі, залежно від якості,
по 40, 48 або 96 асів за модій“. Ці ціни, такі високі порівняно
з одночасними цінами на зерно, пояснюються недосконалим станом млинів, які перебували тоді ще в
періоді дитинства, і значними витратами помолу, які випливали з цього“ (Dureau de la Malle:
„Economie Politique des Romains“. Paris 1840. І, стор.
280 [281]).

V. Економія внаслідок винаходів

Ці заощадження в застосуванні основного капіталу є, як уже
сказано, результатом того, що умови праці застосовуються
у великому масштабі, коротко, що вони служать умовами безпосередньо суспільної, усуспільненої праці,
або безпосередньої
кооперації в процесі виробництва. Це є, з одного боку, та умова,
при якій тільки й можуть бути застосовані механічні й хімічні
винаходи без підвищення ціни товару, а ця остання обставина
завжди є conditio sine qua non [неодмінною умовою]. З другого
боку, тільки при великому масштабі виробництва стають можливими ті заощадження, які випливають із
спільного продуктивного
споживання. Нарешті, тільки досвід комбінованого робітника відкриває і показує, де і як треба
економізувати, як найпростіше
реалізувати вже зроблені відкриття, які практичні труднощі доводиться
\index{iii1}{0117}  %% посилання на сторінку оригінального видання
переборювати при реалізації теорії — її застосуванні до
процесу виробництва — і т. д.

Зауважмо, між іншим, що слід відрізняти загальну працю від
спільної праці. Обидві вони відіграють у процесі виробництва
свою роль, обидві переходять одна в одну, але обидві й
відрізняються між собою. Загальна праця — це всяка наукова
праця, всяке відкриття, всякий винахід. Вона зумовлюється почасти кооперацією сучасників, почасти
використанням праці попередників. Спільна праця передбачає безпосередню кооперацію індивідів.

Вищесказане дістає нове потвердження в таких часто спостережуваних обставинах:

1. У великій ріжниці між витратами першого будування нової
машини і витратами її репродукції, про що дивись у Юра
і Беббеджа.

2. У значно більших витратах, при яких узагалі ведеться підприємство, засноване на нових винаходах,
порівняно з витратами
підприємств, які пізніше виникають на його руїнах, ex suis ossibus
[на його кістках]. Це явище має таку вагу, що перші
підприємці здебільшого банкрутують і тільки пізніші підприємці,
в руки яких будівлі, машини і т. д. дістаються по дешевих цінах, процвітають. Тим-то в більшості
випадків найнікчемніша
і наймізерніша група грошових капіталістів дістає найбільшу
вигоду з усякого прогресу загальної праці людського розуму та
її суспільного застосування за допомогою комбінованої праці.

Розділ шостий

Вплив зміни цін

І. Коливання цін сировинного матеріалу, безпосередній вплив цих коливань на норму зиску

Ми припускаємо тут, як і раніше, що в нормі додаткової
вартости не відбувається ніякої зміни. Ця передумова потрібна
для того, щоб дослідити даний випадок в його чистому вигляді.
Було б, однак, можливо, що при незмінній нормі додаткової вартості певний капітал уживає до праці
зростаюче або меншаюче число робітників унаслідок тих скорочень або розширень, що їх викликають у
ньому коливання цін на сировинний матеріал, які належить тут розглянути. В цьому випадку,
маса додаткової вартості могла б змінюватися при сталій нормі
додаткової вартості. Однак, і це ми повинні тут усунути
як проміжний випадок. Якщо поліпшення машин і зміна цін
сировинного матеріалу впливають одночасно чи то на масу
робітників, вживаних даним капіталом, чи на висоту заробітної
плати, то треба тільки зіставити 1) вплив, який справляє зміна
сталого капіталу на норму зиску, 2) вплив, який справляє зміна
\parbreak{}  %% абзац продовжується на наступній сторінці
