\parcont{}  %% абзац починається на попередній сторінці
\index{iii1}{0070}  %% посилання на сторінку оригінального видання
5 годинам = 5 шилінгам, то додаткова праця = 5 годинам і додаткова
вартість = 5 шилінгам; якщо ж необхідна праця дорівнює
4 годинам = 4 шилінгам, то додаткова праця = 6 годинам і додаткова
вартість = 6 шилінгам.

Отже, якщо величина вартості змінного капіталу перестає
бути показником маси праці, приведеної ним у рух, і, навпаки,
змінюється сама міра цього показника, то разом з тим норма
додаткової вартості змінюється в протилежному напрямі і в зворотному
відношенні.

Тепер ми переходимо до того, щоб застосувати вищенаведене
рівняння норми зиску: р' = m' v/К до різних можливих випадків.
Ми будемо почережно змінювати вартість окремих факторів
m' v/К і встановлювати вплив цих змін на норму зиску. Таким
чином ми одержимо різні ряди випадків, які ми можемо розглядати
або як послідовні зміни умов діяння одного й того
самого капіталу, або як різні, одночасно існуючі один поряд
одного і притягнені до порівняння капітали, наприклад, капітали
в різних галузях промисловості або в різних країнах. Тому,
якщо розуміння деяких наших прикладів, як прикладів послідовних
у часі станів одного й того ж капіталу, здається вимушеним
або практично неможливим, то це заперечення відпаде,
якщо ми розглядатимем їх як порівняння незалежних капіталів.

Отже, ми розкладаємо добуток m' v/К на його обидва множники,
m' і v/К; спочатку ми розглядатимем m' як сталу величину
і дослідимо вплив можливих змін v/К; потім ми припустимо, що
дріб v/К є стала величина і дамо m' проробити можливі зміни;
нарешті, ми припустимо, що всі фактори змінюються, і вичерпаємо
цим усі випадки, з яких можуть бути виведені закони,
що стосуються норми зиску.

І. m' не змінюється, v/К змінюється

Для цього випадку, який охоплює декілька часткових випадків,
можна скласти загальну формулу. Якщо ми маємо два
капітали К і К1, з відповідними змінними складовими частинами
v і v1, зі спільною їм обом нормою додаткової вартості m' і нормами
зиску р' і р'1, то

р' = m' v/К; р'1 = m' v1 / К1
