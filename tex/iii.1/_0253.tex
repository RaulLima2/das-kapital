\index{iii1}{0253}  %% посилання на сторінку оригінального видання
Якщо капітал відправляється за кордон, то це стається не
тому, що він абсолютно не міг би бути застосований всередині
країни. Це стається тому, що за кордоном він може бути застосований
при вищій нормі зиску. Але цей капітал є абсолютно
надлишковий капітал для занятого робітничого населення і для
даної країни взагалі. Він існує як такий поряд з відносно надлишковим
населенням, і це є приклад того, як надлишковий
капітал і відносно надлишкове населення існують одне поряд
одного і взаємно одне одного зумовлюють.

З другого боку, зв’язане з нагромадженням падіння норми зиску
необхідно викликає конкурентну боротьбу. Компенсація падіння
норми зиску збільшенням маси зиску має силу тільки для сукупного
капіталу суспільства і для великих капіталістів з уже
влаштованими підприємствами. Новий, самостійно функціонуючий
додатковий капітал не знаходить для себе таких умов компенсації,
він ще тільки мусить завоювати їх собі, і тому падіння
норми зиску викликає конкурентну боротьбу між капіталами,
а не навпаки. Ця конкурентна боротьба супроводиться, звичайно,
минущим підвищенням заробітної плати і дальшим тимчасовим
зниженням норми зиску, яке випливає з цього підвищення. Те
саме виявляється в перепродукції товарів, у переповненні ринків.
Через те що метою капіталу є не задоволення потреб, а виробництво
зиску, і що цієї мети він досягає тільки методами,
при яких маса продуктів визначається масштабом виробництва,
а не навпаки, то постійно мусить виникати суперечність
(Zwiespalt) між обмеженими розмірами споживання на капіталістичній
базі і виробництвом, яке постійно прагне переступити
ці імманентні йому межі. А втім, адже капітал складається з товарів,
і тому перепродукція капіталу включає перепродукцію
товарів. Тому дивно, що ті самі економісти, які заперечують
перепродукцію товарів, визнають перепродукцію капіталу. Коли
кажуть, що має місце не загальна перепродукція, а диспропорція
між різними галузями виробництва, то це не означає нічого
іншого, як тільки те, що при капіталістичному виробництві
пропорціональність окремих галузей виробництва виступає з диспропорціональності
як постійний процес, бо тут зв’язок сукупного
виробництва нав’язується агентам виробництва як сліпий
закон, а не як збагнутий їх колективним розумом і тому опанований
закон, що підпорядковує процес виробництва їх спільному
контролеві. Далі, разом з цим вимагають, щоб країни, в яких
капіталістичний спосіб виробництва не розвинений, споживали
й виробляли в такій мірі, як це відповідає країнам капіталістичного
способу виробництва. Коли кажуть, що перепродукція є
тільки відносна, то це цілком правильно; але весь капіталістичний
спосіб виробництва є саме тільки відносний спосіб виробництва,
межі якого є не абсолютні межі, але для нього, на його базі,
вони є абсолютні. Інакше яким чином могло б не бути попиту на
ті самі товари, в яких має нужду маса народу, і яким чином було б
\parbreak{}  %% абзац продовжується на наступній сторінці
