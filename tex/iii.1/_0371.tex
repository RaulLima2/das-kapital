\index{iii1}{0371}  %% посилання на сторінку оригінального видання
Плата за управління, яку одержують управителі, як торговельних,
так і промислових підприємств, є цілком відокремленою від
підприємницького доходу як у кооперативних фабриках робітників,
так і в капіталістичних акційних підприємствах. Відокремлення
плати за управління від підприємницького доходу, яке в інших
випадках є випадковим, тут є постійне. У кооперативній фабриці
антагоністичний характер праці нагляду відпадає, бо управитель
оплачується робітниками, а не протистоїть їм як представник
капіталу. Акційні підприємства — які розвиваються разом
з кредитною справою — взагалі мають тенденцію все більше й
більше відокремлювати цю працю управління як функцію від
володіння капіталом, своїм власним, чи взятим у позику; цілком
так само, як з розвитком буржуазного суспільства функції суду
й управління відокремлюються від землеволодіння, атрибутами
якого вони були за феодальних часів. Але коли, з одного
боку, простому власникові капіталу, грошовому капіталістові,
протистоїть функціонуючий капіталіст, і з розвитком кредиту
цей грошовий капітал сам набирає суспільного характеру, концентрується
в банках і віддається в позику ними, а не його безпосередніми
власниками; коли, з другого боку, простий
управитель, який не володіє капіталом ні під яким титулом,
ні позиково, ні якнебудь інакше, виконує всі реальні функції,
які припадають функціонуючому капіталістові як такому, — тоді
лишається тільки службовець, а капіталіст, як зайва особа,
зникає з процесу виробництва.

З опублікованих звітів\footnote{
Наведені тут дані звітів доходять щонайбільше до 1864 року, бо вищесказане
було написано в 1865 році. — \emph{Ф. Е.}
} кооперативних фабрик в Англії видно,
що — після відрахування плати управителя, яка становить частину
витраченого змінного капіталу цілком так само, як і плата
всіх інших робітників — зиск був більший, ніж пересічний зиск,
не зважаючи на те, що кооперативні фабрики подекуди платили
далеко вищий процент, ніж приватні фабриканти. Причиною
вищого зиску в усіх цих випадках була більша економія
в застосуванні сталого капіталу. Але нас при цьому цікавить те,
що тут пересічний зиск (= процент + підприємницький дохід)
фактично і наочно виступає як величина, цілком незалежна від
плати за управління. Тому що зиск був тут більший за пересічний
зиск, то й підприємницький дохід був більший, ніж взагалі.

Те саме явище спостерігається і в деяких капіталістичних акційних
підприємствах, наприклад, в акційних банках (Joint Stock
Banks). London and Westminster Bank у 1863 році виплатив 30\%
річного дивіденду, Union Bank of London та інші — 15\%. З гуртового
зиску тут, крім плати управителям, відходить процент,
який сплачується за вклади. Високий зиск пояснюється тут тим,
що вкладений в підприємства капітал становить незначну величину
порівняно з вкладами. Наприклад, в London and Westminster
\parbreak{}  %% абзац продовжується на наступній сторінці
