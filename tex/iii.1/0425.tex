циркуляції, ніж в іншій; в Англії, наприклад, більше, ніж у
Шотландії, в Німеччині більше, ніж в Англії. Так само в сільському
господарстві той самий капітал, який діє в процесі репродукції,
вимагає, для виконання своєї функції, в різні сезони
року різної кількості грошей.

Але протилежність у постановці Фуллартона невірна. Зовсім
не великий попит на позики, як він це каже, відрізняє періоди
застою від періодів процвітання, а та легкість, з якою цей попит
задовольняється в періоди процвітання, і та трудність, з якою
він задовольняється під час застою. Аджеж саме величезний
розвиток кредитної системи в періоди процвітання, отже й колосальне
підвищення попиту на позиковий капітал і готовість,

нот, банк не має іншого засобу, за допомогою якого він звичайно дає в позику
капітал; і що через це відмова у видачі банкнот означає відмову у видачі позики.
Але, раз позика дозволена, то все пристосовується до потреб ринку; позика
лишається, а засіб циркуляції, якщо в ньому немає потреби, знаходить
свій шлях до того, хто його випустив. Відповідно до цього навіть цілком поверховий
розгляд парламентських звітів може переконати кожного, що кількість
цінних паперів, які перебувають у володінні Англійського банку, частіше коливається
в напрямі, протилежному до циркуляції його банкнот, ніж в одному
з нею напрямі, і, що, отже, приклад цієї великої установи не становить винятку
з тієї доктрини, яку так дуже обстоюють провінціальні банкіри, а саме доктрини,
що ніякий банк не може збільшити кількості своїх циркулюючих банкнот,
якщо кількість їх уже відповідає тим цілям, для яких звичайно вживається циркуляція
банкнот, і що всяке збільшення позик, які дає банк, після того як ця межа
перейдена, доводиться робити з його капіталу і постачати для цього засоби
шляхом продажу деяких його резервних цінних паперів або шляхом відмовлення
від дальших вкладень у ці папери. Таблиця, складена за даними парламентських
звітів за час від 1833 до 1840 року, на яку я посилався на одній з попередніх
сторінок, раз-у-раз дає приклади цієї істини; але два з них настільки варті уваги,
що я не можу їх обминути. На 3 січня 1837 р., коли ресурси банку були
напружені до крайності для підтримання кредиту і протидії утрудненням
грошового ринку, ми бачимо, що його позики і дисконт досягли величезної
суми в 17022000 фунтів стерлінгів, суми, яка навряд чи досягалась будьколи
після війни [1793—1815] і яка майже дорівнює сукупній сумі випущених
банкнот, яка тимчасом лишалась незмінною на такому низькому рівні,
як 17076000 фунтів стерлінгів! З другого боку, на 4 червня 1833 р. ми бачимо
циркуляцію банкнот на 18892000 фунтів стерлінгів при наявності в розпорядженні
банку приватних цінних паперів на суму не більше, як 972 000 фунтів
стерлінгів, отже на суму майже найнижчу, якщо не найнижчу, за останні
півстоліття“] (Фуллартон, там же, стор. 97, 98). — Що зовсім немає потреби
в тому, щоб demand for pecuniary accommodation [попит на грошові позики] був
тотожним з demand for gold [попитом на золото] (що Вільсон, Тук і інші називають
капіталом), це видно з таких свідчень пана Weguelin’a, управителя Англійського
банку: „Дисконт векселів до цієї суми“ [до одного мільйона щоденно
протягом трьох днів поспіль] „не скоротив би резерву“ [банкнот], „коли б публіка
не вимагала більшої суми активної циркуляції. Банкноти, видані при
дисконті векселів, поверталися б назад через посередництво банків і шляхом
вкладів. Якщо такі операції не мають на меті вивіз золота або якщо всередині
країни не панує така паніка, що публіка міцно тримає в себе свої банкноти,
замість того, щоб платити ними банкам, то такі величезні обороти не зачепили б
резерву“. — „Банк може щодня дисконтувати на півтора мільйона, і це відбувається
постійно, ні трохи не зачіпаючи резервів банку. Банкноти повертаються
назад як вклади, і єдина зміна, яка при цьому відбувається, полягає в простому
перенесенні з одного рахунку на другий“ („Report on Bank Acts, 1857“. Свідчення
№ 241,500). Отже, банкноти служать тут тільки засобом перенесення кредитів.
