ших сферах. Лиш оскільки ціна лишається незмінною, сфера
середнього складу зберігає свій рівень зиску однаковим з іншими сферами. Отже, на практиці в цій
сфері справа відбувається цілком так само, як коли б продукти цієї сфери продавались по їх дійсній
вартості. А саме, якщо товари продаються по їх дійсних вартостях, то очевидно, що при інших
однакових
 умовах підвищення або зниження заробітної плати викликає відповідне зниження або підвищення зиску,
але не викликає ніякої зміни вартості товарів, і що при всіх обставинах підвищення або зниження
заробітної плати ніколи не може вплинути на вартість товарів, а завжди тільки на величину додаткової
вартості.

        III. Підстави капіталіста для компенсації

Уже було сказано, що, конкуренція вирівнює норми зиску різних сфер виробництва в пересічну норму
зиску і саме тим перетворює вартості продуктів цих різних сфер виробництва в ціни виробництва. І це
стається саме в наслідок постійного перенесення капіталу з однієї сфери виробництва до іншої, де в
даний момент зиск стоїть вище пересічного рівня; при цьому, однак, слід взяти до уваги коливання
зиску, зв’язані з чергуванням худих і ситих років в даній галузі промисловості на протязі даного
періоду часу. Ця безперервна еміграція та імміграція капіталу, яка відбувається між різними сферами
виробництва, породжує висхідні і низхідні рухи норми зиску, які більше чи менше взаємно
урівноважуються і через це мають тенденцію
повсюди зводити норму зиску до того самого спільного й загального рівня.

Цей рух капіталів завжди викликається в першу чергу станом ринкових цін, які в одному місці
підвищують зиск понад загальний пересічний рівень, в другому — знижують його нижче цього рівня. Ми
покищо залишаємо осторонь купецький капітал, з яким ми тут ще не маємо справи і який, як це
показують пароксизми спекуляції з певними улюбленими товарами, що раптово вибухають, може з
надзвичайною швидкістю витягати маси капіталу з одної галузі застосування і так само швидко кидати
їх до іншої. Але в кожній сфері виробництва у власному розумінні слова — в промисловості,
землеробстві, рудниках і т. д. — перенесення капіталу з однієї сфери в іншу становить значні
труднощі, особливо в наслідок наявності основного капіталу. До того ж досвід показує, що коли
яканебудь галузь промисловості,
наприклад, бавовняна промисловість, в певний час дає надзвичайно високий зиск, то вона потім, в
інший час, дає дуже незначний зиск, а то навіть і збиток, так що за певний цикл років пересічний
зиск в ній приблизно такий самий, як і в інших галузях. І капітал швидко привчається зважати на цей
досвід.

Але чого конкуренція не показує, так це визначення вартості, яке керує рухом виробництва; так це
вартостей, які стоять за
