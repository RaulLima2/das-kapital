купівлями й продажами, так що ці операції розвиваються в форму
особливого заняття, відокремленого від усіх інших функцій промислового капіталу і тому самостійного.
Це — особлива форма
суспільного поділу праці, так що певна частина функції, яка
взагалі має бути виконана в особливій фазі процесу репродукції
капіталу, — в даному випадку у фазі циркуляції, — виступає як
виключна функція особливого агента циркуляції, відмінного від
виробника. Однак, від цього це особливе заняття ні в якому
разі не виступало б ще як функція особливого капіталу, відмінного від промислового капіталу, що
перебуває в процесі своєї
репродукції, і самостійного відносно промислового капіталу; це
заняття і в дійсності не виступає як таке в тих випадках, коли
торгівля товарами провадиться за допомогою простих комівояжерів або інших безпосередніх агентів
промислового капіталіста.
Отже, до цього мусить долучитися ще й другий момент.

Подруге: Цей другий момент полягає в тому, що самостійний агент циркуляції, купець, авансує в цьому
своєму становищі грошовий капітал (власний або позичений). Те, що для промислового капіталу, який
перебуває в процесі своєї репродукції, виступає просто як Т — Г, як перетворення товарного
капіталу в грошовий капітал або простий продаж, те для купця
виступає як Г — Т — Г', як купівля і продаж того самого товару,
отже, як повернення до нього за допомогою продажу того грошового капіталу, який від нього віддалився
при купівлі.

Те, що для купця, оскільки він авансує капітал на купівлю
товару у виробника, виступає як Г — Т — Г, це є завжди Т — Г,
перетворення товарного капіталу в грошовий капітал; це завжди
перша метаморфоза товарного капіталу, хоча той самий акт
може для виробника або для промислового капіталу, який перебуває в процесі своєї репродукції,
виступати як Г — Т, як зворотне перетворення грошей у товар (в засоби виробництва) або як друга фаза
метаморфози. Для виробника полотна першою
метаморфозою було Т — Г, перетворення товарного капіталу
в грошовий капітал. Цей акт для купця виступає як Г — Т, як
перетворення його грошового капіталу в товарний капітал. Якщо ж
він продає полотно білільникові, то для білільника це становить Г — Т, перетворення грошового
капіталу в продуктивний
капітал, або другу метаморфозу його товарного капіталу; але
для купця це становить Т — Г, продаж купленого ним полотна.
Але в дійсності товарний капітал, вироблений фабрикантом полотна, тільки тепер остаточно проданий,
або це Г — Т — Г купця
становить тільки опосереднюючий процес для Т — Г між двома
виробниками. Або припустімо, що фабрикант, який виробляє
полотно, на частину вартості проданого полотна купує пряжу
в торговця пряжею. Таким чином це є для нього Г — Т. Але для
купця, який продає пряжу, це є Т — Г, перепродаж пряжі; а відносно самої пряжі як товарного капіталу
це є тільки остаточний
продаж її, в наслідок якого вона переходить із сфери цирку-
