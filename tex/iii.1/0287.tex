вельний капітал функціонує в процесі репродукції як капітал,
і тому як функціонуючий капітал одержує частину з виробленої
сукупним капіталом додаткової вартості. Для кожного окремого
купця маса його зиску залежить від маси капіталу, яку він
може вжити в цьому процесі, а він тим більше може вжити
з неї на купівлю й продаж, чим більша є неоплачена праця його
прикажчиків. Саму функцію, в силу якої його гроші є капітал,
торговельний капіталіст примушує здебільшого виконувати своїх
робітників. Неоплачена праця його прикажчиків, хоч вона й не
створює додаткової вартості, створює однак йому привласнення
додаткової вартості, що своїм результатом є для цього капіталу
цілком те саме; отже, ця неоплачена праця є для нього
джерелом зиску. Інакше торговельне підприємство ніколи не
можна було б вести у великому масштабі, ніколи не можна
було б вести по-капіталістичному.

Подібно до того, як неоплачена праця робітника безпосередньо
створює для продуктивного капіталу додаткову вартість, цілком
так само неоплачена праця торговельних найманих робітників створює
для торговельного капіталу участь в цій додатковій вартості.

Трудність полягає ось у чому: оскільки робочий час і праця
самого купця не є вартостетворча праця, хоч і створює йому
участь у виробленій вже додатковій вартості, то як стоїть справа
з тим змінним капіталом, який він витрачає на купівлю торговельної
робочої сили? Чи слід цей змінний капітал прирахувати
як витрати до авансованого купецького капіталу? Якщо ні, то це,
видимо, суперечить законові вирівнення норми зиску; який капіталіст
авансовував би 150, коли б він міг рахувати як авансований
капітал тільки 100? Якщо ж слід, то це, видимо, суперечить
сутності торговельного капіталу, бо цей сорт капіталу
функціонує як капітал не в наслідок того, що він подібно до
промислового капіталу приводить в рух чужу працю, а в наслідок
того, що він сам працює, тобто виконує функції купівлі й продажу,
і саме тільки за це і цим переносить на себе частину
додаткової вартості, створеної промисловим капіталом.

(Отже, нам треба дослідити такі пункти: змінний капітал купця;
закон необхідної праці в циркуляції; яким чином праця купця
зберігає вартість його сталого капіталу; роль купецького капіталу
в сукупному процесі репродукції; нарешті, роздвоєння на
товарний капітал і грошовий капітал — з одного боку, і на товарно-торговельний
капітал і грошево-торговельний капітал —
з другого боку.)

Якби кожний купець мав лиш стільки капіталу, скільки він
міг би обертати особисто своєю власною працею, то мало б
місце безконечне роздрібнення купецького капіталу; це роздрібнення
мусило б зростати в міру того, як продуктивний капітал,
з розвитком капіталістичного способу виробництва, виробляє
в дедалі більшому масштабі і оперує дедалі більшими масами.
Отже, ми мали б зростаючу невідповідність між тим і другим.
