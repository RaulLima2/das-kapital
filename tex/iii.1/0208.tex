II. Ціна виробництва товарів середнього складу

       Ми бачили, яким чином відхилення цін виробництва від вартостей постає в наслідок того:
       1) що до витрат виробництва товару додається не додаткова вартість, вміщена в ньому, а
пересічний зиск;
       2) що ціна виробництва товару, яка таким чином відхиляється від вартості, входить як елемент
у витрати виробництва інших товарів, в наслідок чого, отже, вже у витратах виробництва товару може
міститись відхилення від вартості спожитих на нього засобів виробництва, незалежно від того
відхилення, що може постати для самого цього товару в наслідок ріжниці між пересічним зиском і
додатковою вартістю.

Таким чином, можливо, що і в товарів, вироблених капіталами середнього складу, витрати виробництва
відхилятимуться від суми вартості елементів, з яких складається ця складова частина їх ціни
виробництва. Припустім, що середній склад є 80 c + 20 v. Можливо, що в дійсних капіталах, які мають
такий склад, 80  c більше або менше вартості с, сталого капіталу, бо це с складається з товарів,
ціна виробництва яких відхиляється від їх вартості. Так само 20 v  могли б відхилятися від своєї
вартості, якщо в споживання заробітної плати входять товари, ціна виробництва яких відрізняється від
їх вартості; отже, робітник, щоб купити ці товари (замістити їх), мусить витратити більше або менше
робочого часу, отже, мусить виконати більше або менше необхідної праці, ніж потрібно було б, коли б
ціни виробництва
необхідних засобів існування збігалися з їх вартостями.

        Однак, ця можливість зовсім не міняє правильності положень, встановлених для товарів
середнього складу. Кількість зиску, що припадає на ці товари, дорівнює кількості вміщеної в них
самих додаткової вартості. Наприклад, при наведеному вище капіталі з складом у 80 с + 20 v для
визначення додаткової вартості важливе не те, чи ці числа є вирази дійсних вартостей, а те, як вони
відносяться одне до одного; а саме, що v = 1/5, а с = 4/5  всього капіталу. Якщо це так, то
додаткова вартість, вироблена v, дорівнює, як ми це припустили вище, пересічному зискові. З другого
боку: через те що додаткова вартість дорівнює пересічному зискові, ціна виробництва = витратам
виробництва + зиск = k + p = k + m, на практиці дорівнює вартості товару. Тобто підвищення або
зниження заробітної плати лишає k + p  в цьому випадку так само незмінним, як воно лишило б
незмінною вартість товару, і викликає тільки відповідний зворотний рух,
зниження або підвищення, на стороні норми зиску. А саме, якщо в наслідок підвищення або зниження
заробітної плати тут змінилася б ціна товарів, то норма зиску в цих сферах середнього складу стала б
вищою або нижчою порівняно з її рівнем в ін-
