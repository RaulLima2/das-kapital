\parcont{}  %% абзац починається на попередній сторінці
\index{iii1}{0304}  %% посилання на сторінку оригінального видання
пророблюються з їх капіталами, і вирівненням загальної норми
зиску.\footnote{
Ось одно дуже наївне, але разом з тим дуже правильне зауваження: „Тому,
безперечно, причиною тієї обставини, що один і той же товар можна одержати
у різних продавців по цілком різних цінах, дуже часто є неправильна калькуляція“
(\emph{Feller} und \emph{Odermann}: „Das Ganze der kaufmännischen Arithmetik“, 7 вид.,
1859 [стор. 451]). Це показує, як визначення ціни стає чисто теоретичним,
тобто абстрактним.
} Конкуренція для цих голів неминуче грає також цілком
перекручену роль. Якщо межі вартості й додаткової вартості
дані, то легко зрозуміти, яким чином конкуренція капіталів перетворює
вартості в ціни виробництва і потім у торговельні ціни
а додаткову вартість — у пересічний зиск. Але, не знаючи цих
меж, абсолютно не можна зрозуміти, чому конкуренція приводить
загальну норму зиску до цієї, а не до іншої межі, до 15\%, а не
до 1500\%. Адже вона, щонайбільше, може приводити її до \emph{одного}
рівня. Але в ній немає абсолютно ніякого елементу, який визначав
би самий цей рівень.

Отже, з точки зору купецького капіталу сам оборот виступає
так, ніби він визначає ціни. З другого боку, тимчасом як швидкість
обороту промислового капіталу, оскільки вона дає можливість
даному Капіталові експлуатувати більше чи менше праці, визначально
і обмежувально впливає на масу зиску, а тому й на
загальну норму зиску, — для торговельного капіталу норма зиску
є дана іззовні, і внутрішній зв’язок її з утворенням додаткової
вартості цілком стертий. Якщо той самий промисловий капітал
при інших незмінних умовах і особливо при однаковому органічному
складі обертається протягом року чотири рази замість
двох, то він виробляє удвоє більше додаткової вартості, а тому
й зиску; і це наочно виявляється, якщо і поки цей капітал володіє
монополією поліпшеного способу виробництва, який дає
йому можливість так прискорювати оборот. Навпаки, різна тривалість
обороту в різних галузях торгівлі виявляється в тому,
що зиск, одержуваний на оборот певного товарного капіталу,
стоїть у зворотному відношенні до числа оборотів грошового
капіталу, за допомогою якого обертається цей товарний капітал.
Small profits and quick returns [невеликі зиски і швидкі обороти]
являють собою для shopkeeper’a [крамаря] саме той принцип,
якого він додержується з принципу.

Зрештою, само собою зрозуміло, що цей закон оборотів купецького
капіталу в кожній галузі торгівлі, — і залишаючи осторонь
чергування швидших і повільніших оборотів, які один одного компенсують,
— має силу тільки для пересічних оборотів усього купецького
капіталу, вкладеного в цю галузь. Капітал $А$, який функціонує
в тій самій галузі, що й $В$, може робити більше чи менше оборотів
порівняно з пересічним числом оборотів. В цьому випадку
інші капітали роблять менше чи більше оборотів. Це нічого не
змінює в обороті загальної маси купецького капіталу, вкладеної
в цю галузь. Але це має вирішально важливе значення для окремого
\index{iii1}{0305}  %% посилання на сторінку оригінального видання
купця або дрібного торговця. В цьому випадку він одержує
додатковий зиск цілком так само, як одержують додатковий зиск
промислові капіталісти, коли вони виробляють при умовах сприятливіших,
ніж пересічні умови. Якщо конкуренція примушує до
цього, то він може продавати дешевше, ніж його товариші, не
знижуючи свого зиску нижче пересічного рівня. Якщо умови,
які дають йому можливість робити швидший оборот, є такі
умови, що йому доводиться їх купувати, наприклад, розташування
місць продажу, то він може платити за них особливу ренту,
тобто частина його додаткового зиску перетворюється в земельну
ренту.

\section{Грошево-торговельний капітал}

Чисто технічні рухи, які пророблюють гроші в процесі циркуляції
промислового капіталу і, як ми тепер можемо додати,
товарно-торговельного капіталу (бо цей останній бере на себе
частину руху циркуляції промислового капіталу, роблячи його
своїм власним і специфічним рухом), — ці рухи, зробившись самостійною
функцією особливого капіталу, який виконує їх і тільки їх,
як властиві йому операції, перетворюють цей капітал у грошевоторговельний
капітал. Частина промислового капіталу і, точніше
кажучи, також товарно-торговельного капіталу постійно перебуває
в грошовій формі не тільки як грошовий капітал взагалі, але
як грошовий капітал, який виконує ці технічні функції. Від сукупного
капіталу4 відокремлюється і усамостійнюється в формі грошового
капіталу певна частина, капіталістична функція якої
полягає виключно в тому, щоб виконувати ці операції для всього
класу промислових і торговельних капіталістів. Подібно до того,
як це є з товарно-торговельним капіталом, так і тут від промислового
капіталу, який перебуває в процесі циркуляції у вигляді
грошового капіталу, відокремлюється певна частина і виконує
ці операції процесу репродукції для всієї решти сукупного капіталу.
Отже, рухи цього грошового капіталу знов таки є тільки
рухи усамостійненої частини промислового капіталу, який перебуває
в процесі своєї репродукції.

Тільки в тому випадку і остільки, коли і оскільки капітал
вкладається уперше, — що має місце і при нагромадженні, — капітал
у грошовій формі виступає як вихідний і кінцевий пункт руху.
-Але для кожного капіталу, раз він уже перебуває в своєму
процесі, вихідний пункт, як і кінцевий пункт, виступають тільки
як перехідні пункти. Оскільки промисловий капітал, з-моменту
свого виходу з сфери виробництва до того моменту, коли він
знову вступає в неї, мусить проробити метаморфозу $Т' — Г — Т$,
остільки $Г$, як це вже виявилось при простій товарній
циркуляції, в дійсності є кінцевим результатом однієї фази метаморфози
тільки для того, щоб бути вихідним пунктом протилежної
\index{iii1}{0306}  %% посилання на сторінку оригінального видання
фази, яка доповнює першу. І хоча для торговельного
капіталу $Т — Г$ промислового капіталу завжди виступає як $Г$ —
$Т — Г$, проте й для нього, раз він почав функціонувати, дійсним
процесом постійно є $Т — Г — Т$. Але торговельний капітал одночасно
пророблює акти $Т — Г$ і $Г — Т$. Тобто не тільки \emph{один} капітал
перебуває в стадії $Т — Г$, тимчасом як другий капітал
перебуває в стадії $Г — Т$, але той самий капітал одночасно постійно
купує і постійно продає в наслідок безперервності процесу
виробництва; він одночасно постійно перебуває в обох
стадіях. Тимчасом як одна частина його перетворюється в гроші,
щоб пізніше знову перетворитись у товар, друга частина одночасно
перетворюється в товар, щоб знову перетворитись у гроші.

Чи функціонують при цьому гроші як засіб циркуляції чи як
засіб платежу, це залежить від форми товарообміну. В обох випадках
капіталістові постійно доводиться виплачувати гроші багатьом
особам і постійно одержувати гроші в оплату від багатьох
осіб. Ця чисто технічна операція виплачування грошей і одержування
грошей сама по собі становить працю, яка, оскільки гроші
функціонують як засіб платежу, робить необхідними балансові
розрахунки, акти вирівнення взаємних зобов’язань. Ця праця —
витрати циркуляції, праця, яка не утворює ніякої вартості. Вона
скорочується в наслідок того, що її виконує особливий підрозділ
агентів або капіталістів для всієї решти класу капіталістів.

Певна частина капіталу мусить постійно бути в наявності
як скарб, як потенціальний грошовий капітал резерв купівельних
засобів, резерв засобів платежу, незанятий капітал, який у грошовій
формі чекає свого застосування; а деяка частина капіталу
постійно припливає в цій формі назад. Крім одержування грошей,
виплачування грошей і рахівництва, це робить необхідним
зберігання скарбу, що знов таки є особлива операція. Отже,
в дійсності це є постійний розпад скарбу на засоби циркуляції
і засоби платежу і утворення його знову з грошей, одержуваних
від продажу і від платежів, яким уже настав строк; цей постійний
рух частини капіталу, яка існує у вигляді грошей, рух, відокремлений
від функції самого капіталу, ця чисто технічна операція
викликає особливу працю і витрати — витрати циркуляції.

Поділ праці приводить до того, що ці технічні операції, зумовлювані
функціями капіталу, виконуються, наскільки можливо,
для всього класу капіталістів певним підрозділом агентів або
капіталістів як. їх виключні функції або концентруються в їх
руках. Це є, як і в випадку з купецьким капіталом, поділ праці
у двоякому розумінні. Ці технічні операції стають особливим заняттям,
і в наслідок того, що воно, як особливе заняття, виконується
для грошового механізму всього класу, воно концентрується, провадиться
у великому масштабі; а в межах цього особливого заняття
знов таки відбувається поділ праці як в наслідок розпаду його
на різні, незалежні одна від одної галузі, так і в наслідок утворення
майстерні в кожній такій галузі (великі контори, численні
\index{iii1}{0307}  %% посилання на сторінку оригінального видання
бухгалтери й касири, далеко проведений поділ праці).
Виплачування грошей, одержування грошей, складання балансів,
ведення поточних рахунків, зберігання грошей і т. д., виконувані
відокремлено від тих актів, в наслідок яких ці технічні операції
стають необхідними, роблять капітал, авансований на ці
функції, грошево-торговельним капіталом.

Різні операції, з усамостійнення яких в особливі підприємства
виникає торгівля грішми, випливають з різних визначеностей
самих грошей і з їх функцій, які доводиться таким чином виконувати
і капіталові у формі грошового капіталу.

Уже раніше я вказував, як гроші взагалі первісно розвиваються
при обміні продуктів між різними громадами.\footnote{
Zur Kritik der politischen Oekonomie“, стор. 27 [„До критики політичної
економії“, укр. вид. 1935 р., стор. 74].
}

Тому торгівля грішми, торгівля грошовим товаром розвивається
насамперед з міжнародних зносин. Якщо в різних країнах
існують різні монети, то купцям, які роблять закупівлі в чужих
країнах, доводиться обмінювати монети своєї країни на місцеві
монети і навпаки, абож обмінювати різні монети на злитки чистого
срібла чи золота як на світові гроші. Звідси міняльна справа, яку
слід розглядати як одну з природно вирослих основ сучасної торгівлі
грішми.\footnote{
Уже з великої різноманітності монет як щодо ваги й проби, так і щодо
карбування їх багатьма князями й містами, які мають право карбування, виникла
повсюди необхідність користуватись місцевою монетою в торговельних справах,
коли треба було звести рахунки в якійсь одній монеті. Коли купці
виїжджали на іноземні ринки, вони для розплати готівкою запасалися злитками
чистого срібла, а також, звичайно, і золота. Цілком так само, від’їжджаючи
звідти, вони обмінювали одержані ними місцеві монети на злитки
срібла або золота, Тому міняльна справа, торгівля грішми, обмін злитків благородних
металів-на місцеві монети і навпаки, стала дуже поширеною і дохідною
справою“ (\emph{Hüllmann}: Städtewesen des Mittelalters“. Bonn 1826—29,
І, стор. 437 [438]). — „De Wisselbank heeft hären naam niet... van den wissel, wisselbrief,
maar van wisselen van geldspecien. Lang vöör het oprigten der Amsterdarasche
wisselbank in 1609 had men in de Nederlandsche koopsteden reeds wisselaars
en wisselhuizen, zelfs wisselbanken... Het bedrijf dezer wissellaars bestond
daarin, dat zie de talrijke verscheidene muntspecien, die door vreemde handelaren
in het land gebragt worden, tegen wettelijk gangbare munt inwisselden. Langzamerhand
breidde hun werkkring zieh uit... zij werden de kassiers en bankiers van
hunne tijd. Maar in die vereeniging van de kassierderij met het wisselambt zach
de regering van Amsterdam gevaar, en om dit gevaar te keeren, werd besloten to
het stichten eener groote inrigting, die zoo wel het wisselen als de kassierderij op
openbaar gezag zou verrigten. Die inrigting was de beroemde Amsterdamsche Wisselbank
van 1609. Even zoo hebben de wisselbanken van Venetie, Genua, Stockholm,
Hamburg haar ontstaan aan de gedurlge noodzakelijkheid der verwisseling van
geldspecien te danken gehad. Van deze allen is de Hamburgsche de eenige die
nog heden bestaat, om dat de behoefte aan zulk eene inrigting zieh in deze
koopstad. die geen eigen muntstelsel heeft, nogaltijd doet gevoelen etc.“ [„Розмінний
банк дістав свою назву не... від векселя, вексельного листа, а від розміну
грошових знаків. Задовго до заснування Амстердамського розмінного банку в
1609 році в нідерландських торговельних містах були міняйли, міняльні контори,
навіть розмінні банки... Заняття цих міняйл полягало в тому, що вони
обмінювали різні численні сорти монет, які привозилися в країну іноземними
купцями, на ходячу законну монету. Помалу поле їх діяльності розширювалось... вони стали касирами і банкірами свого часу. Але в поєднанні професій
касира і міняйла власті Амстердама побачили небезпеку і, щоб запобігти цій
небезпеці, було вирішено заснувати велику установу, яка повинна була б провадити
операції як міняйл, так і касирів, і діяти відкрито, згідно з статутом.
Такою установою був знаменитий Амстердамський розмінний банк 1609 року.
Цілком так само розмінні банки Венеції, Генуї, Стокгольма, Гамбурга виникли
в наслідок постійної потреби в розміні грошових знаків. З усіх цих банків
існує ще й тепер один тільки гамбурзький, бо в цьому торговельному місті, яке
не має своєї власної монетної системи, все ще відчувається потреба в такій
установі і т. д.“). (\emph{S. Vissering}: „Handboek van Praktische Staatshuishoudkunde“.
Amsterdam 1860, І, стор. 247).
} З міняльної справи розвинулись розмінні банки, де
\index{iii1}{0308}  %% посилання на сторінку оригінального видання
срібло (або золото) функціонує у відміну від ходячої монети
як світові гроші, — тепер як банкові гроші або торговельні гроші.
Вексельна справа, оскільки вона зводилась просто до видачі мандрівникам
міняйлом якоїсь країни свідоцтва на одержаний грошей
від інших міняйл, розвинулась уже в Римі й Греції з власне
міняльної справи.

Торгівля золотом і сріблом як товарами (як сировинними
матеріалами для виготовлення предметів розкоші) становить
природно вирослу базу для торгівлі злитками (bullion trade) або
торгівлі, яка опосереднює функції грошей як світових грошей.
Ці функції, як це вже було з’ясовано раніше (книга І, розд. III,
З, с), двоякого роду: переміщення грошей між різними національними
сферами циркуляції для вирівнювання міжнародних
платежів і при мандруванні капіталу, що шукає процентів;
поруч із цим рух від джерел добування благородних металів
по світовому ринку і розподіл видобутку між різними національними
сферами циркуляції. В Англії ще протягом більшої
частини XVII століття функції банкірів виконували майстри
по золоту. Яким чином вирівнювання міжнародних платежів
розвивається далі у вексельну торгівлю та ін., це ми тут залишаємо
цілком осторонь, як і все те, що стосується до операцій
з цінними паперами, коротко кажучи, усі особливі форми
кредитної справи, яка нас тут ще не стосується.

Як світові гроші, національні гроші скидають із себе свій
місцевий характер; гроші однієї країни виражаються в грошах
іншої країни, і таким чином усі вони зводяться до вміщеного
в них золота або срібла, тимчасом як золото й срібло, як два
товари, що циркулюють як світові гроші, разом з тим зводяться
до взаємного відношення їх вартостей, яке постійно
міняється. Опосереднення цих операцій торговець грішми робить
своїм особливим заняттям. Таким чином, міняльна справа
і торгівля злитками — це є найбільш первісні форми торгівлі
грішми і виникають вони з двояких функцій грошей: як місцевих
монет і як світових грошей.

З капіталістичного процесу виробництва, як з торгівлі взагалі,
навіть при докапіталістичному способі виробництва, випливає:

\emph{Поперше}, збирання грошей як скарбу, тобто в теперішній
час як частини капіталу, яка завжди мусить бути в наявності
\parbreak{}  %% абзац продовжується на наступній сторінці
