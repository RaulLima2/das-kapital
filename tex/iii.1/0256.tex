жує небезпека в наслідок розвитку самого виробництва. А кількісне
відношення тут — усе. В дійсності в основі цього лежить щось
глибше, про що він тільки догадується. В цьому виявляється
чисто економічним способом, тобто з буржуазної точки зору,
в межах капіталістичного розуміння, з точки зору самого капіталістичного
виробництва, обмеженість капіталістичного виробництва,
його відносність, те, що воно не є абсолютний, а тільки
історичний спосіб виробництва, відповідний певній обмеженій
епосі розвитку матеріальних умов виробництва.

IV. Додатки

Через те що розвиток продуктивної сили праці відбувається
дуже нерівномірно в різних галузях промисловості, і не тільки
нерівномірно щодо ступеня, а часто і в протилежному напрямі,
то звідси випливає, що пересічна маса зиску (= додаткової
вартості) мусить стояти далеко нижче тієї висоти, якої можна
було б сподіватися відповідно до розвитку продуктивної сили
в найбільш розвинених галузях промисловості. Те, що розвиток
продуктивної сили в різних галузях промисловості відбувається
не тільки в дуже різних пропорціях, але часто і в протилежному
напрямі, виникає не тільки з анархії конкуренції і своєрідності
буржуазного способу виробництва. Продуктивність праці
зв’язана також з природними умовами, які часто стають менш
вигідними в тій самій мірі, в якій зростає продуктивність, оскільки
вона залежить від суспільних умов. Звідси протилежний рух
в цих різних сферах — прогрес в одних, регрес в інших. Досить
тільки згадати, наприклад, про вплив сезонів року, від чого
залежить кількість найбільшої частини всіх сировинних матеріалів,
про вичерпання лісів, кам’яновугільних і залізнорудних
копалень і т. д.

Якщо обігова частина сталого капіталу, сировинний матеріал
і т. д. постійно зростає в своїй масі в міру розвитку продуктивної
сили праці, то цього не можна сказати про основний
капітал, будівлі, машини, пристрої для освітлення, опалення і т. д.
Хоч машина з зростанням її розмірів стає абсолютно дорожчою,
але відносно вона стає дешевшою. Якщо п’ятеро робітників
виробляють удесятеро більше товарів, ніж раніш, то
з цієї причини витрати на основний капітал не збільшуються
вдесятеро; хоч вартість цієї частини сталого капіталу зростає
з розвитком продуктивної сили, але вона зростає далеко не в такій
самій пропорції. Ми вже не раз відзначали ріжницю між
відношенням сталого капіталу до змінного, як воно виражається
в падінні норми зиску, і тим самим відношенням, як воно,
з розвитком продуктивності праці, виражається щодо одиничного
товару та його ціни.

[Вартість товару визначається всім робочим часом, минулим
і живим, що входить у цей товар. Підвищення продуктивності праці
