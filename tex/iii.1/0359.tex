таліста, оскільки він не сам застосовує свій капітал, простий
кількісний поділ гуртового зиску між двома різними особами,
які мають різні юридичні титули на той самий капітал, а тому
й на вироблений ним зиск, обертається в якісний поділ. Одна
частина зиску виступає тепер як плід капіталу самого по собі
в одному визначенні його, як процент; друга частина виступає
як специфічний плід капіталу в протилежному визначенні, і
тому як підприємницький дохід; одна — як плід виключно власності
на капітал, друга — як плід виключно функціонування
з цим капіталом, як плід капіталу, що пророблює процес,
або як плід тих функцій, які виконує активний капіталіст. І це
скостеніння і усамостійнення обох частин гуртового зиску одної
проти одної, як коли б вони походили з двох істотно різних
джерел, мусить тепер встановитись для всього класу капіталістів
і для сукупного капіталу. І при тому однаково, чи застосовуваний
активним капіталістом капітал взято в позику, чи ні, або
чи належний грошовому капіталістові капітал застосовується
ним самим, чи ні. Зиск від усякого капіталу, отже й пересічний
зиск, оснований на вирівненні капіталів між собою, розпадається
або може бути розкладений на дві якісно різні, одна відносно
одної самостійні і одна від одної незалежні частини, процент
і підприємницький дохід, які обидві — і та й друга — визначаються
особливими законами. Капіталіст, який працює власним
капіталом, так само як і той, що працює капіталом,
взятим у позику, ділить свій гуртовий зиск на процент, який
припадає йому як власникові, як тому, хто самому собі позичив
свій власний капітал, і на підприємницький дохід, який припадає
йому як активному, функціонуючому капіталістові. Таким чином
для цього поділу, як поділу якісного, не має значення, чи повинен
капіталіст дійсно поділитися з іншим капіталістом, чи ні.
Застосовний капіталу, навіть коли він працює власним капіталом,
розпадається на дві особи — на простого власника капіталу
і на застосовника капіталу; сам його капітал щодо категорій
зиску, які він дає, розпадається на капітал-власність, капітал
поза процесом виробництва, капітал, що сам по собі дає процент,
і на капітал у процесі виробництва, який як капітал, що
пророблює процес, дає підприємницький дохід.

Отже, процент закріплюється таким чином, що він тепер виступає
не як байдужий для виробництва поділ гуртового зиску,
який має місце тільки принагідно, саме коли промисловець працює
чужим капіталом. Навіть коли він працює власним капіталом,
його зиск розпадається на процент і підприємницький
дохід. Тим самим просто кількісний поділ стає якісним; він має
місце незалежно від тієї випадкової обставини, чи є промисловець
власник, чи невласник свого капіталу. Це не тільки частини
зиску, розподілювані між різними особами, але дві різні
категорії зиску, які стоять у різному відношенні до капіталу,
отже, мають відношення до різних визначеностей капіталу.
