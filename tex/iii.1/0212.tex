Відділ третій

Закон тенденції норми зиску до падіння

Розділ тринадцятий

Закон як такий

При даній заробітній платі і при даному робочому дні змінний
капітал, наприклад, в 100, представляє певне число приведених
у рух робітників; він є показник цього числа. Припустімо,
наприклад, що 100 фунтів стерлінгів становлять заробітку плату
100 робітників, скажімо, за 1 тиждень. Якщо ці 100 робітників
виконують стільки ж необхідної праці, скільки додаткової праці,
якщо вони, отже, щодня працюють стільки ж часу на себе
самих, тобто для репродукції своєї заробітної плати, скільки
на капіталістів, тобто для виробництва додаткової вартості, то
вся вироблена ними вартість буде = 200 фунтам стерлінгів,
а вироблена ними додаткова вартість становитиме 100 фунтів
стерлінгів. Норма додаткової вартості m/v була б = 100%. Однак,
ця норма додаткової вартості, як ми бачили, виражалася б у дуже
різних нормах зиску, залежно від різного розміру сталого капіталу
c, а тому й усього капіталу K, бо норма зиску = m/K. При нормі
додаткової вартості в 100%,

якщо c = 50, v = 100, то р' = 100/150 = 66 2/3%;
якщо c = 100, v = 100, то р' = 100/200 = 50%;
якщо c = 200, v = 100, то р' = 100/300 = 33 1/3%;
якщо c = 300, v = 100, то р' = 100/400 = 25%;
якщо c = 400, v = 100, то р' = 100/500 = 20%.

Таким чином при незмінному ступені експлуатації праці та
сама норма додаткової вартості виражалася б у падаючій нормі
зиску, бо разом з матеріальним розміром сталого капіталу зро-
