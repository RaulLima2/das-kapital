ність товару, який воно виробляє. В кожній сфері виробництва
мета полягає тільки в тому, щоб виробляти додаткову
вартість, привласнювати собі в продукті праці певну кількість
неоплаченої праці. І так само в природі підлеглої капіталові
найманої праці лежить те, що вона байдуже ставиться до
специфічного характеру своєї праці, мусить змінюватись відповідно
до потреб капіталу і переходити з однієї сфери виробництва
до іншої.

Подруге, кожна сфера виробництва дійсно є остільки ж добра
і остільки ж погана, як і будьяка інша; кожна дає той самий
зиск і кожна була б безцільною, коли б вироблювані нею товари
не задовольняли будьякої суспільної потреби.,

Але якщо товари продаються по їх вартостях, то, як це вже
показано, в різних сферах виробництва виникають дуже різні
норми зиску, залежно від різного органічного складу вкладених
у ці сфери мас капіталу. Але капітал вилучається з сфери виробництва
з нижчою нормою зиску і кидається в іншу, яка дає
вищий зиск. В наслідок цієї постійної еміграції та імміграції,
одним словом, в наслідок свого розподілу між різними сферами
виробництва, залежно від того, де норма зиску падає і де підвищується,
капітал здійснює таке відношення між попитом і поданням,
що в різних сферах виробництва пересічний зиск
стає однаковий, і тому вартості перетворюються в ціни виробництва.
Це вирівнення капіталові вдається здійснити тим повніше,
чим вищий капіталістичний розвиток в даному національному
суспільстві, тобто чим більше відносини даної країни пристосовані
до капіталістичного способу виробництва. З прогресом капіталістичного
виробництва розвиваються і умови його; воно підпорядковує
своєму специфічному характерові і своїм імманентним
законам усю сукупність суспільних передумов, в межах яких
відбувається процес виробництва.

Постійне вирівнювання постійних нерівностей відбувається
тим швидше, 1) чим рухливіший капітал, тобто чим легше він
може бути перенесений з однієї сфери і з одного місця в інші;
2) чим 'швидше робоча сила може бути перекинута з однієї
сфери в іншу і з одного місцевого центру виробництва до
іншого. Пункт 1-й передбачає повну свободу торгівлі всередині
суспільства і усунення всіх монополій, крім природних, особливо
тих, що виникають з самого капіталістичного способу виробництва.
Далі, передбачається розвиток кредитної системи, яка
концентрує в руках окремих капіталістів неорганізовану масу
вільного суспільного капіталу; нарешті — підпорядкування різних
сфер виробництва капіталістам. Це останнє включене вже
у припущені нами передумови, раз ми допустили, що справа
йде про перетворення вартостей у ціни виробництва в усіх капіталістично
експлуатованих сферах виробництва; однак, само
це вирівнювання наштовхується на значніші перешкоди, коли
численні і масові сфери виробництва, проваджені некапіталі-
