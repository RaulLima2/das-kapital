виробництва вимагають застосування масового капіталу. Це зумовлює
також централізацію капіталу, тобто поглинення дрібних
капіталістів великими і втрату першими своїх капіталів. Це знов
таки є відокремлення, хоча тільки вторинного порядку, умов праці
від виробників, до яких ще належать ці дрібні капіталісти, бо
у них власна праця грає ще певну роль; взагалі праця капіталіста
стоїть у зворотному відношенні до величини його капіталу, тобто
до тієї міри, в якій він є капіталіст. Це є те відокремлення одне
від одного умов праці з одного боку і виробників з другого,
яке становить поняття капіталу, яке починається з первісним
нагромадженням (книга І, розд. XXIV), потім виявляється як постійний
процес в нагромадженні і концентрації капіталу і, нарешті,
виражається тут як централізація вже наявних капіталів у небагатьох
руках і втрата капіталів (так змінюється тепер експропріація)
багатьма. Цей процес швидко привів би капіталістичне
виробництво до краху, коли б постійно поряд з доцентровою
силою знов і знов децентралізаційно не діяли протидіючі тенденції.

II. Конфлікт між розширенням виробництва
і зростанням вартості

Розвиток суспільної продуктивної сили праці виявляється
двояко: поперше, у величині вже вироблених продуктивних сил,
в розмірі вартості і в розмірі маси виробничих умов, при яких
відбувається нове виробництво, і в абсолютній величині нагромадженого
вже продуктивного капіталу; подруге, у відносно
незначній величині витрачуваної на заробітну плату частини
капіталу порівняно з усім капіталом, тобто у відносно незначній
кількості живої праці, потрібної для репродукції і збільшення
вартості даного капіталу, для масового виробництва. А це передбачає
разом з тим концентрацію капіталу.

Відносно вживаної робочої сили розвиток продуктивної сили
виявляється знов таки двояко: поперше, в збільшенні додаткової
праці, тобто в скороченні необхідного робочого часу, потрібного
для репродукції робочої сили. Подруге, в зменшенні кількості
робочої сили (числа робітників), яка взагалі вживається для
того, щоб привести в рух даний капітал.

Обидва ці рухи не тільки йдуть рука в руку, але взаємно
зумовлюють один одного; обидва вони є явища, в яких
виражається один і той самий закон. Проте, вони діють на
норму зиску в протилежному напрямі. Сукупна маса зиску
дорівнює сукупній масі додаткової вартості, норма зиску =
m/K = додаткова вартість/сукупний авансований капітал. Але додаткова вартість, як сукупна сума,
визначається, поперше, її нормою, а подруге,
масою одночасно вживаної при цій нормі праці або, що є те
саме, величиною змінного капіталу. З одного боку, підвищується
один фактор, норма додаткової вартості; з другого боку, змен-
