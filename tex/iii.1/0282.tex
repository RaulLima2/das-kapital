Якого б роду не були ці витрати циркуляції, — чи виникають
вони з чисто купецького підприємства як такого, отже, належать
до специфічних витрат циркуляції купця, чи представляють
затрати, які виникають з додаткових процесів виробництва, що
долучаються під час процесу циркуляції, як відправка, транспортування,
зберігання і т. д., — на боці купця вони завжди передбачають,
крім авансованого на купівлю товарів грошового капіталу,
додатковий капітал, авансований на купівлю і оплату цих
засобів циркуляції. Оскільки цей елемент витрат складається
з обігового капіталу, він як додатковий елемент входить цілком
у продажну ціну товарів; оскільки ж він складається з основного
капіталу, він входить у неї як додатковий елемент в міру
свого зношування; але він входить у неї як елемент, який утворює
номінальну вартість, навіть якщо він, як от чисто купецькі
витрати циркуляції, не становить ніякого дійсного додатку до
вартості товару. Але весь цей додатковий капітал, обіговий
чи основний, бере участь в утворенні загальної норми зиску.

Чисто купецькі витрати циркуляції (отже, за винятком витрат
відправки, транспортування, зберігання і т. д.) зводяться до тих
витрат, які потрібні для того, щоб реалізувати вартість товару,
перетворити її з товару в гроші або з грошей у товар, опосереднити
обмін між ними. При цьому цілком залишаються осторонь
можливі процеси виробництва, які продовжуються протягом
акту циркуляції і від яких торговельне підприємство може
існувати цілком відокремлено; подібно до того, як в дійсності,
наприклад, власне транспортна промисловість та відправка можуть
бути і є галузі промисловості, цілком відмінні від торгівлі,
так само й товари, які мають бути куплені й продані,
можуть лежати в доках та інших громадських приміщеннях, при
чому витрати, які виникають з цього, оскільки купцеві доводиться
їх авансувати, нараховуються на нього третіми особами.
Все це має місце у власне гуртовій торгівлі, де купецький
капітал виступає в найчистішому вигляді і найменше переплітається
з іншими функціями. Підприємець-перевізник, директор
залізниці, судновласник — не „купці“. Витрати, які ми тут розглядаємо,
це витрати купівлі й продажу. Ми вже раніш відзначили,
що вони зводяться до обрахунків, ведення книг, ринкових
видатків, кореспонденції і т. д. Потрібний для цього сталий
капітал складається з контори, паперів, поштових знаків
і т. д. Інші витрати зводяться до змінного капіталу, який авансується
на вживання торговельних найманих робітників. (Витрати
відправки, витрати транспортування, авансування на оплату
мита і т. д. почасти можна розглядати таким чином, ніби купець
авансує їх на закупівлі товарів, і що тому вони для нього входять
у купівельну ціну.)

Всі ці витрати робляться не при виробництві споживної вартості
товарів, а при реалізації їх вартості; це — чисті витрати
циркуляції. Вони входять не в безпосередній' процес вироб-
