В формулі р' = m'n (v/K), як сказано, m'n є те, що в другій
книзі названо річною нормою додаткової вартості. У вищенаведеному
випадку вона становить 153 11/13\% × 8 1/2, або точно
1307 9/13\%. Отже, якщо якийсь бравий чоловік сплеснув руками
з приводу потворності річної норми додаткової вартості в 1000\%,
наведеної в одному прикладі в другій книзі, то він, може, заспокоїться
на факті річної норми додаткової вартості понад
1300\%, який наведено йому тут з живої практики Манчестера.
В часи найвищого розквіту, яких ми, правда, давно вже не переживали,
така норма аж ніяк не є рідкість.

До речі сказати, ми маємо тут приклад дійсного складу капіталу
в сучасній великій промисловості. Весь капітал поділяється на
12182 фунти стерлінгів сталого і 318 фунтів стерлінгів змінного
капіталу, разом 12 500 фунтів стерлінгів. Або в процентах:
97 1/2с + 2 1/2v = 100 K. Тільки сорокова частина всього капіталу,
але, повторно обертаючись більше ніж вісім разів на рік, служить
для виплати заробітної плати.

Через те що, звичайно, тільки небагатьом капіталістам спадає
на думку робити такі обчислення щодо свого власного підприємства,
то статистика майже абсолютно мовчить про відношення
сталої частини всього суспільного капіталу до змінної
частини. Тільки американський перепис дає те, що можливе при
сучасних відносинах: суму заробітної плати, виплаченої в кожній
галузі підприємств, і одержаних зисків. Хоч і які підозрілі ці
дані, — бо вони основані тільки на неперевірених повідомленнях
самих промисловців, — проте вони надзвичайно цінні і становлять
усе, що ми маємо про цей предмет. В Европі ми занадто делікатні,
щоб вимагати від наших великих промисловців подібних
викрить. — Ф. Е.]

Розділ п’ятий

Економія в застосуванні сталого капіталу

І. Загальні положення

Збільшення абсолютної додаткової вартості, або здовження
додаткової праці, отже й робочого дня, при незмінній величині
змінного капіталу, тобто при вживанні того самого числа робітників
за ту саму номінально заробітну плату, — при чому байдуже,
чи оплачується надурочний час чи ні, — відносно знижує
вартість сталого капіталу порівняно з вартістю всього капіталу
і змінного капіталу і підвищує цим норму зиску, знов таки
незалежно від зростання й маси додаткової вартості і можливого
підвищення норми додаткової вартості. Розмір основної
частини сталого капіталу, фабричних будівель, машин тощо лишається
той самий, однаково, чи працюють за його допомогою 16,
чи 12 годин. Здовження робочого дня не вимагає ніяких нових
затрат на цю найдорожчу частину сталого капіталу. До цього долу-
