\parcont{}  %% абзац починається на попередній сторінці
\index{iii1}{0432}  %% посилання на сторінку оригінального видання
взагалі є частина його капіталу. І він потребує і застосовує його
не як власне капітал, а як власне засіб платежу. Інакше довелося
б кожний звичайний продаж товару, за допомогою якого
добуваються засоби платежу, також вважати за одержання капіталу
в позику. — \emph{Ф. Е.}]

Для приватного банку, який випускає банкноти, ріжниця полягає
в тому, що в тому разі, коли його банкноти не лишаються
в місцевій циркуляції і не повертаються до нього в формі
вкладів або при оплаті векселів, яким надійшов строк, ці банкноти
потрапляють у руки тих осіб, яким він в обмін на них
мусить платити золото або банкноти Англійського банку.
Таким чином, у цьому випадку позика його банкнот в дійсності
репрезентує позику банкнот Англійського банку, або — що для
цього банку є те саме — позику золота, отже, частини його
банкового капіталу. Те саме має місце і в тому випадку, коли
сам Англійський банк або якийнебудь інший банк, який підлягає
закону про максимум випуску банкнот, мусить продавати
цінні папери, щоб вилучити з циркуляції свої власні банкноти
і потім знову віддавати їх у позику, в цьому випадку
його власні банкноти репрезентують частину його мобілізованого
банкового капіталу.

Коли б навіть циркуляція була чисто металічною, то одночасно
1) відплив золота [тут, очевидно, мається на увазі такий
відплив золота, коли принаймні частина його відпливає за кордон
— \emph{Ф. Е.}] міг би спорожнити сховища банку, і 2) тому що
банк потребував би золота головним чином тільки для сальдування
платежів (для завершення минулих операцій), то його позики
під цінні папери могли б дуже зрости, але поверталися б до
нього у формі вкладів або при оплаті векселів, яким надійшов
строк; так що, з одного боку, при збільшенні кількості цінних
паперів у портфелі банку загальна сума його скарбу зменшилася
б, а, з другого боку, ту саму суму, яку він раніше тримав
як її власник, він тримав би тепер як боржник своїх вкладників,
і, нарешті, зменшилася б загальна маса засобів циркуляції.

Досі ми припускали, що позики видаються в банкнотах,
отже, ведуть за собою принаймні тимчасове збільшення випуску
банкнот, яке, правда, одразу знов зникає. Але в цьому
немає потреби. Замість видачі паперових банкнот банк може
відкрити клієнтові $А$ кредит у своїх книгах, при чому $А$, боржник
банку, стає ніби його вкладником. $А$ платить своїм кредиторам
чеками на банк, а одержувач цих чеків платить ними
далі своєму банкірові, який обмінює їх у розрахунковій палаті на
чеки, виписані на нього. В цьому випадку операція відбувається
без усякої участі банкнот, і вся операція обмежується тим, що
вимоги, які банк повинен задовольнити, покриваються чеком на
нього самого, а його дійсна винагорода полягає в кредитній вимозі
на $А$. В цьому випадку банк дав йому в позику частину
\parbreak{}  %% абзац продовжується на наступній сторінці
