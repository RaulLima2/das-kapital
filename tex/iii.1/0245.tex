шується (відносно або абсолютно) другий фактор, число робітників.
Оскільки розвиток продуктивних сил зменшує оплачувану
частину вживаної праці, він підвищує додаткову вартість, підвищуючи
її норму; проте, оскільки він зменшує всю масу
праці, вживаної даним капіталом, він зменшує другий фактор,
число робітників, на яке треба помножити норму додаткової
вартості, щоб одержати її масу. Двоє робітників, які працюють
по 12 годин на день, не можуть дати такої ж маси додаткової вартості,
як 24 робітники, які працюють тільки по 2 години кожний,
навіть якби вони могли живитись самим повітрям і якби їм через це
зовсім не доводилось працювати на самих себе. Отже, в цьому
відношенні компенсація зменшеного числа робітників підвищенням
ступеня експлуатації праці має певні непереступні межі;
тому вона може, звичайно, затримати падіння норми зиску, але
вона не може його усунути.

Отже, з розвитком капіталістичного способу виробництва
норма зиску падає, тимчасом як маса його із збільшенням маси
застосовуваного капіталу підвищується. При даній нормі абсолютна
маса, на яку зростає капітал, залежить від його величини
в даний момент. Але, з другого боку, якщо цю величину дано,
то відношення, в якому він зростає, норма його зростання, залежить
від норми зиску. Безпосередньо підвищення продуктивної
сили (яке, крім того, як уже згадано, завжди йде рука в руку із
знеціненням наявного капіталу) може збільшити величину вартості
капіталу тільки в тому випадку, коли воно, підвищуючи
норму зиску, збільшує ту частину вартості річного продукту,
яка зворотно перетворюється в капітал. Оскільки мова йде про
продуктивну силу праці, це може статися тільки в тому випадку
(бо ця продуктивна сила безпосередньо не має ніякого відношення
до вартості наявного капіталу), коли в наслідок підвищення
продуктивної сили або збільшується відносна додаткова
вартість, або зменшується вартість сталого капіталу, отже, здешевлюються
товари, які входять або в репродукцію робочої сили,
або в елементи сталого капіталу. Але і те і друге означає також
знецінення наявного капіталу; і те і друге йде рука в руку із
зменшенням змінного капіталу порівняно з сталим. І те і друге
зумовлює падіння норми зиску і уповільнює це падіння. Далі,
оскільки підвищена норма зиску спричиняє підвищений попит на
працю, вона впливає на збільшення робітничого населення і разом
з тим на збільшення матеріалу, придатного для експлуатації, який
тільки й робить капітал капіталом.

Але посередньо розвиток продуктивної сили праці сприяє
збільшенню наявної капітальної вартості, збільшуючи масу й різноманітність
споживних вартостей, в яких представлена та сама
мінова вартість і які становлять матеріальний субстрат, речові елементи
капіталу, матеріальні предмети, з яких складається сталий
капітал безпосередньо і змінний, принаймні, посередньо. З тим
самим капіталом і тією самою працею створюється більше речей,
