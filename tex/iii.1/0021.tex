не визнавати комічності становища. До цього я ще додаю тільки
ось що: з тією самою сміливістю, з якою він уже тоді міг сказати,
що „Маркс у третьому томі безсумнівно довів“, він користується
нагодою, щоб розповісти — мабуть, професорську —
плітку, ніби вищезгаданий твір Конрада Шмідта „безпосередньо
інспірований Енгельсом“. Пане Юліус Вольф! В тому світі, в якому
ви живете і дієте, може й водиться таке, що людина, яка публічно
ставить перед іншими проблему, нишком відкриває своїм
особистим друзям її розв’язання. Що ви на це здатні, я вам охоче
вірю. Що в тому світі, в якому я обертаюся, немає потреби
опускатися до такої мерзоти, це доводить вам оця передмова. —

Ледве помер Маркс, як пан Ахілл Лоріа спішно опублікував
статтю про нього в „Nuova Antologia“ (квітень 1883): спочатку
біографія, переповнена брехливими даними, потім критика
громадської, політичної і літературної діяльності. Матеріалістичне
розуміння історії Маркса тут сфальсифіковане і перекручене
з таким апломбом, який дозволяє угадати якусь велику
мету. І ця мета була досягнута: в 1886 році той самий пан
Лоріа опублікував книгу: „La teoria есоnomіса della constituzione
politica“, в якій він оповістив здивованому світові його сучасників,
як своє власне відкриття, історичну теорію Маркса, так
грунтовно і так навмисно перекручену ним в 1883 році. Звичайно,
теорію Маркса він звів тут до досить філістерського
рівня; історичні ілюстрації й приклади теж рясніють такими помилками,
яких не простили б і учневі четвертого класу; але
хіба це все має якесь значення? Відкриття, що політичні становища
і події скрізь і завжди знаходять своє пояснення у відповідних
економічних становищах, зроблене, як доведено цією книгою
Лоріа, аж ніяк не Марксом у 1845 році, а паном Лоріа
в 1886 році. Принаймні він щасливо упевнив у цьому своїх земляків,
а з того часу, як його книга з’явилась французькою мовою,
і деяких французів, і може тепер чванитись в Італії як
автор нової епохальної історичної теорії, поки тамошні соціалісти
знайдуть час повискубувати в illustre [славетного] Лоріа
крадені павині пера.

Але це тільки один маленький зразочок маніри пана Лоріа.
Він запевняє нас, що всі теорії Маркса грунтуються на свідомому
софізмі (un consaputo sofisma); що Маркс не відступав
перед паралогізмами навіть тоді, коли він визнавав їх за такі
(sapendoli tali) і т. д. І після того, як він в цілому ряді подібних
підлих побрехеньок дав своїм читачам усе потрібне для
того, щоб вони побачили в Марксі якогось кар’єриста à la Лоріа,
який досягає своїх дрібних ефектів за допомогою таких самих
дрібних негідних шахрайських засобів, як наш падуанський професор,
— він може тепер відкрити їм важливу таємницю, а тим
самим і нас приводить назад до норми зиску.

Пан Лоріа каже: За Марксом маса додаткової вартості (яку
пан Лоріа ототожнює тут з зиском), вироблена в капіталістич-
