\parcont{}  %% абзац починається на попередній сторінці
\index{iii1}{0399}  %% посилання на сторінку оригінального видання
товарами, вищеописаний примусовий вивіз до Індії та примусовий
довіз з Індії з однією тільки метою здобути бронзові векселі.
Всі ці обставини, перепродукція в промисловості, так само як
і недопродукція в землеробстві, отже цілком різні причини, викликали підвищення попиту на грошовий
капітал, тобто на кредит і гроші. Підвищений попит на грошовий капітал мав свої
причини в ході самого процесу виробництва. Але, яка б не була
причина, саме попит на грошовий капітал викликав підвищення
розміру процента, вартості грошового капіталу. Якщо Оверстон хоче сказати, що вартість грошового
капіталу підвищилась, тому що вона підвищилась, то це — тавтологія. Якщо ж
він під „вартістю капіталу“ розуміє тут підвищення норми зиску
як причину підвищення розміру процента, то помилковість цього
зразу ж виявиться. Попит на грошовий капітал, а тому „вартість капіталу“, можуть підвищитись, хоч
зиск знижується; як
тільки відносне подання грошового капіталу знижується, „вартість“ його підвищується. Оверстон хоче
довести, що криза 1847 року і висока норма процента, яка супроводила її, ні трохи
не залежали від „кількості наявних грошей“, тобто від постанов інспірованого ним банкового акта 1844
року; хоча в дійсності вони залежали від неї, оскільки страх перед вичерпанням
банкового резерву — витвір Оверстона — долучив до кризи 1847—1848 рр. грошову паніку. Але тепер
питання не в цьому.
В наявності була нужда в грошовому капіталі, яка була спричинена надмірними розмірами операцій,
порівняно з наявними засобами, і яка вибухла через порушення процесу репродукції в наслідок
неврожаю, надмірного капіталовкладення в залізниці, перепродукції, особливо бавовняних товарів,
індійської і китайської
шахрайської торгівлі, спекуляції, надмірного довозу цукру і т. д.
Тим, хто купив хліб, коли він коштував 120 шилінгів за квартер, бракувало тоді, коли він впав до 60
шилінгів, саме тих
60 шилінгів, які вони переплатили, і відповідного кредиту під
заставу цього хліба. Зовсім не недостача в банкнотах заважала
їм обернути свій хліб в гроші по старій ціні в 120 шилінгів.
Так само стояла справа і з тими, хто довіз занадто багато
цукру, який потім майже не можна було продати. Так само й
у тих панів, які міцно вклали свій обіговий капітал (floating capital)
у залізниці і щодо заміщення його в своєму „законному“ підприємстві поклалися на кредит. Все це для
Оверстона виражається в „моральній свідомості підвищеної вартості його грошей“ („а moral sense of
the enhanced value of his money“). Але цій
підвищеній вартості грошового капіталу безпосередньо відповідала на другому боці знижена грошова
вартість реального капіталу
(товарного капіталу і продуктивного капіталу). Вартість капіталу
в одній формі підвищилась, тому що вартість капіталу в другій
формі знизилась. А Оверстон намагається обидві ці вартості різних
родів капіталу ототожнити в єдиній вартості капіталу взагалі, і
при тому таким способом, що протиставляє обидві ці вартості недостачі
\index{iii1}{0400}  %% посилання на сторінку оригінального видання
в засобах циркуляції, в наявних грошах. Але одна й та сама
сума грошового капіталу може бути віддана в позику за допомогою дуже різних кількостей засобів
циркуляції.

Візьмімо його приклад 1847 року. Офіціальний банковий процент був: у січні 3—3 1/2\%, в лютому 4—4
1/2\%, в березні здебільшого 4\%, У квітні (паніка) 4—7 1/2\%, в травні 5—5 1/2\%, в червні загалом 5\%,
в липні 5\%, в серпні 5—5 1/2\%, у вересні 5\% з незначними
коливаннями до 5 1/4, 5 1/2, 6\%, у жовтні 5, 5 1/2, 7\%, в листопаді 7—10\%, у грудні 7—5\%. — В цьому
випадку процент підвищувався,
тому що зиски зменшувались і грошові вартості товарів надзвичайно впали. Отже, якщо Оверстон каже
тут, що розмір процента
в 1847 році підвищився, тому що вартість капіталу підвищилась,
то під вартістю капіталу він може тут розуміти тільки вартість
грошового капіталу, а вартість грошового капіталу є саме розмір
процента і ніщо інше. Але потім показується лисячий хвіст,
і вартість капіталу ототожнюється з нормою зиску.

Щодо високого розміру процента, який платився в 1856 році,
то Оверстон дійсно не знав, що він почасти був симптомом того,
що з’явився такий вид рицарів кредиту, які сплачували процент
не з зиску, а з чужого капіталу; всього лише за декілька місяців
перед кризою 1857 року він твердив, що „стан справ цілком
здоровий“.

Далі він каже: „3722. Уявлення, ніби зиск підприємства знищується в наслідок підвищення розміру
процента, в найвищій
мірі помилкове. По-перше, підвищення розміру процента рідко
буває довгочасним; по-друге, якщо воно й буває довгочасним
і значним, то по суті справи воно є підвищенням вартості капіталу; а чому підвищується вартість
капіталу? Тому що підвищилась норма зиску“. — Отже, тут ми, нарешті, дізнаємося,
який сенс має „вартість капіталу“. Зрештою, норма зиску може
протягом довгого часу лишатись високою, але підприємницький
дохід — упасти, а розмір процента підвищитись, так що процент поглине найбільшу частину зиску.

„3724. Підвищення розміру процента було наслідком колосального
розширення в ділах нашої країни і великого підвищення норми зиску; і коли скаржаться, що підвищений
розмір
процента руйнує ті самі дві речі, які були його власною причиною, то це логічний абсурд, про який не
знаєш, що й сказати“. — Це якраз настільки ж логічно, як коли б він сказав:
Підвищена норма зиску була наслідком підвищення товарних
цін спекуляцією, і коли скаржаться, що підвищення цін руйнує
свою власну причину, а саме спекуляцію, то це логічний абсурд
і т. д. Що річ може кінець-кінцем зруйнувати свою власну причину, це логічний абсурд тільки для
лихваря, закоханого у високий процент. Величність римлян була причиною їхніх завоювань,
а їхні завоювання зруйнували їхню величність. Багатство — причина
розкоші, а розкіш руйнуюче впливає на багатство. Отакий
мудрець! Ідіотизм сучасного буржуазного світу якнайкраще
\parbreak{}  %% абзац продовжується на наступній сторінці
