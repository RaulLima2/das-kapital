лом. Якщо я куплю на 1000 фунтів стерлінгів вина із строком
платежу через 3 місяці і продам це вино за готівку до того, як
минуть ці три місяці, то для цієї операції не доводиться авансовувати
жодної копійки. В цьому випадку якнайнаочніше ясно
також, що грошовий капітал, який тут фігурує як купецький
капітал, є безперечно не що інше, як сам промисловий капітал
у своїй формі грошового капіталу, в процесі свого повернення
до самого себе у формі грошей. (Та обставина, що виробник,
який продав на 1000 фунтів стерлінгів товарів із строком платежу
через 3 місяці, може дисконтувати в банкіра одержаний при цьому
вексель, тобто боргове зобов’язання, ні трохи не змінює справи
і не має ніякого відношення до капіталу торговця товарами.)
Якщо за цей проміжок часу ринкові ціни товару впадуть,
скажемо, на 1/10, то купець не тільки не одержить ніякого
зиску, а взагалі виручить тільки 2700 фунтів стерлінгів замість
3000 фунтів стерлінгів. Він мусив би додати 300 фунтів стерлінгів
для того, щоб сплатити борг. Ці 300 фунтів стерлінгів
функціонували б тільки як резерв для вирівнювання ріжниці в
ціні. Але те саме стосується і до виробника. Коли б він сам продавав
по знижених цінах, то він теж утратив би 300 фунтів стерлінгів
і не міг би без резервного капіталу знову почати виробництво
в попередньому масштабі.

Торговець полотном купує у фабриканта на 3000 фунтів
стерлінгів полотна; фабрикант з цих 3000 фунтів стерлінгів платить,
наприклад, 2000 фунтів стерлінгів, щоб купити пряжу;
він купує цю пряжу в торговця пряжею. Гроші, якими фабрикант
платить торговцеві пряжею, не є гроші торговця полотном,
бо цей останній одержав за них товар на таку саму суму. Це —
грошова форма його власного капіталу. В руках торговця пряжею
ці 2000 фунтів стерлінгів виступають тепер як грошовий
капітал, що повернувся до нього; але в якій мірі вони є таким
грошовим капіталом, як відмінні від цих 2000 фунтів стерлінгів
як грошової форми, яку скинуло з себе полотно і набрала
пряжа? Якщо торговець пряжею купив у кредит і продав за
готівку до того, як минув строк платежу, то в цих 2000 фунтах
стерлінгів не міститься жодної копійки купецького капіталу,
відмінного від тієї грошової форми, якої набирає сам промисловий
капітал в процесі свого кругобігу. Товарно-торговельний
капітал, оскільки він, отже, не є проста форма промислового
капіталу, який перебуває в вигляді товарного капіталу або грошового
капіталу в руках купця, є не що інше, як частина грошового
капіталу, яка належить самому купцеві і застосовується
в купівлі й продажу товарів. Ця частина в зменшеному масштабі
представляє частину капіталу, авансованого на виробництво,
яка завжди мусила б перебувати в руках промисловців
як грошовий резерв, як купівельний засіб, і завжди мусила б
циркулювати як їх грошовий капітал. Ця частина перебуває тепер,
зменшеною, в руках капіталістів-купців, постійно функціо-
