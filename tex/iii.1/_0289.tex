\parcont{}  %% абзац починається на попередній сторінці
\index{iii1}{0289}  %% посилання на сторінку оригінального видання
тим, що $В + b$ є взагалі скорочення первісного $В$, представляє
менший купецький капітал, ніж той, іцо був би потрібний без $b$.
Але ця продажна ціна мусить бути достатньою 2) для того, щоб,
крім зиску на $b$, який тепер з’являється додатково, замістити також
виплачену заробітну плату, замістити самий змінний капітал
купця = $b$. Це останнє й становить труднощі. Чи становить $b$
нову складову частину ціни, чи воно є тільки частиною зиску,
одержаного за допомогою $В + b$, яка виступає як заробітна плата
тільки відносно торговельного робітника, а відносно самого
купця як просте заміщення його змінного капіталу? В останньому
випадку одержаний купцем зиск на його авансований капітал
$В + b$ дорівнював би тільки зискові, який відповідно до загальної
норми припадає на $В$, плюс $b$, яке він виплачує у формі
заробітної плати, але яке само не дає ніякого зиску.

Справді, справа зводиться до того, щоб знайти границі $b$ (в математичному
розумінні). Перш за все ми хочемо точно встановити,
в чому полягає трудність. Позначимо капітал, витрачуваний безпосередньо
на закупівлі та продаж товарів, через $В$, сталий капітал,
споживаний при цій функції (речові торговельні витрати),
через $К$ і змінний капітал, витрачуваний купцем, через $b$.

Заміщення $В$ зовсім не становить ніяких труднощів. Воно є
тільки реалізованою купівельною ціною для купця або ціною
виробництва для фабриканта. Цю ціну купець платить, а при
перепродажу він одержує $В$ назад, як частину своєї продажної
ціни; крім цього $В$, він одержує зиск на $В$, як це пояснено раніше.
Наприклад, товар коштує 100 фунтів стерлінгів. Припустім, що зиск
на нього становить 10\%. Таким чином, товар продається за
110. Товар уже раніше коштував 100; купецький капітал в 100
додає до нього тільки 10.

Далі, якщо ми візьмемо $К$, то воно, щонайбільше, такої ж
величини, — в дійсності ж менше, — як частина сталого капіталу,
яку споживав би виробник для продажу й купівлі; але вона становила
б тоді додаток до сталого капіталу, який уживається виробником
безпосередньо у виробництві. Проте, ця частина мусить
постійно заміщатися з ціни товару, або, що є те саме, відповідна
частина товару мусить пост|йно витрачатись у цій формі, мусить
— якщо розглядати сукупний капітал суспільства — постійно
репродуковуватись у цій формі. Ця частина авансовуваного сталого
капіталу, так само як і вся маса його, вкладена безпосередньо
у виробництво, обмежуюче впливала б на норму зиску. Оскільки
промисловий капіталіст передає торговельну частину свого підприємства
купцеві, він не має потреби авансувати цю частину
капіталу. Замість нього її авансує купець. Але це тільки номінально;
купець ні виробляє, ні репродукує споживаний ним сталий
капітал (речові торговельні витрати). Отже, виробництво
цього капіталу являє собою особливе підприємство або, принаймні,
частину підприємства певних промислових капіталістів, які таким
чином грають таку саму роль, як промислові капіталісти, що
\parbreak{}  %% абзац продовжується на наступній сторінці
