стання продуктивної сили праці виражається якраз в тому відношенні,
в якому збільшується кількість сировинного матеріалу,
що вбирає в себе певну кількість праці, отже, в зростаючій масі
сировинного матеріалу, яка, наприклад, за одну робочу годину перетворюється
в продукт, перероблюється на товар. Отже, в
міру розвитку продуктивної сили праці вартість сировинного
матеріалу становить все зростаючу складову частину вартості
товарного продукту, і не тільки тому, що вона цілком входить
у цю останню, але й тому, що в кожній відповідній частині
цілого продукту постійно зменшується і частина, яка відповідає
зношуванню машин, і частина, яку створює новододана праця.
В наслідок цього спадного руху відносно зростає друга частина
вартості, утворювана сировинним матеріалом, якщо це зростання
не знищується відповідним зменшенням вартості на боці сировинного
матеріалу, яке випливає з ростущої продуктивності
праці, застосовуваної для виготовлення самого цього сировинного
матеріалу.

Далі: через те що сировинні й допоміжні матеріали цілком
так само, як і заробітна плата, становлять складові частини
обігового капіталу, отже, мусять постійно цілком заміщатися
з кожного продажу продукту, тимчасом як щодо машин треба
заміщати тільки зношування, і до того ж на перший час у формі
резервного фонду, — при чому в дійсності зовсім неістотно, чи
дає кожний окремий продаж відповідну частину для цього
резервного фонду, якщо тільки весь річний продаж дає для
цього фонду відповідну річну частину, — то тут знову виявляється,
що підвищення ціни сировинного матеріалу може урізати
або загальмувати весь процес репродукції, якщо виручена
від продажу товарів ціна недостатня для заміщення всіх елементів
товару, або якщо ця ціна робить неможливим продовження
процесу в розмірах, відповідних його технічній основі,
так що або тільки частина машин може працювати абож усі
машини не можуть працювати звичайний повний час.

Нарешті, витрати, спричинювані відпадами, змінюються в прямому
відношенні до коливань ціни сировинного матеріалу: підвищуються,
якщо вона підвищується, падають, якщо вона падає. Але
й тут є певна межа. Ще в 1850 році було написано: „Одно з джерел
значних втрат, що виникають з підвищення ціни сировинного
матеріалу, ледве чи буде помітне для кожного, хто не є
прядільником-практиком, а саме втрата на відпадах. Мене повідомляють,
що коли ціна бавовни підвищується, то витрати прядільника,
особливо при виготовленні пряжі низької якості, зростають
у більшій мірі, ніж це показує виплачена надбавка до
ціни. Відпади при прядінні грубої пряжі становлять понад 15\%;
отже, якщо цей процент спричинює втрату в 1/2 пенса на фунт
при ціні бавовни в 3 1/2 пенса, то при підвищенні ціни бавовни
до 7 пенсів за фунт він підвищує цю втрату до 1 пенса на
фунт“ („Rep. of Insp. of Fact., April 1850“, стор. 17). — Але в
