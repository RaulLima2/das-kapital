сприяють нагромадженню, тобто утворенню додаткового капіталу,
і якщо кожен додатковий капітал приводить в рух добавну
працю і виробляє добавну додаткову вартість; якщо, з другого
боку, просте зниження норми зиску включає вже і той
факт, що сталий капітал, а тому й весь старий капітал, зріс, —
то весь цей процес перестає бути таємничим. Ми далі побачимо,
до яких умисних фальшувань в обчисленнях вдаються
для того, щоб по-шахрайському відкинути можливість збільшення
маси зиску при одночасному зменшенні норми зиску.

Ми показали, як ті самі причини, які викликають тенденцію
загальної норми зиску до падіння, зумовлюють прискорене нагромадження
капіталу, а тому й зростання абсолютної величини або
загальної маси привласнюваної ним додаткової праці (додаткової
вартості, зиску). Як усе в конкуренції, а тому й у свідомості
агентів конкуренції, так і цей закон — я маю на думці цей внутрішній
і необхідний зв’язок між двома явищами, які, як здається,
одне одному суперечать — виступає у перекрученому вигляді.
Очевидно, що в межах вищенаведених пропорцій капіталіст,
який розпоряджається великим капіталом, одержує більшу масу

when 100 000 £; still greater when 300 000  £; and so on, increasing, though at a
diminishing rate, with every increase of capital. This progression, howewer, is only
true for a certain time; thus, 19 per cent, on 200 000 £ is more than 20 on 100 000 £;
again 18 per cent on 300 000 £ is more than 19 per cent, on 200 000 £; but after
capital has accumulated to a large amount, and profits have fallen, the further
accumulation diminishes the aggregate of profits. Thus, suppose the accumulation
should be 1 000 000 £, and the profits 7 per cent., the whole amount of profits will be
70000 £; now if an addition of 100000£ capital bemade to the million, and profits should
fall to 6 per cent., 66 000 £ or a diminution of 4000 £ will be received by the owners
of stock, although the whole amount of stock will be increased from 1 000 000 £ to
1 100 000£.“ [„Нам слід, отже, сподіватися, що хоча норма зиску на капітал може
зменшитися в наслідок нагромадження капіталу в країні і підвищення заробітної
плати, однак загальна сума зиску збільшиться. Так, якщо ми припустимо, що при
послідовному нагромадженні 100 000 фунтів стерлінгів норма зиску впаде з 20\% до
19\%, до 18\% і до 17\%, тобто постійно зменшуватиметься, то слід сподіватися,
що вся сума зиску, одержувана цими послідовними власниками капіталу, постійно
зростатиме; що вона буде більша при капіталі в 200 000 фунтів стерлінгів,
ніж при капіталі в 100 000 фунтів стерлінгів, і ще більша при капіталі
в 300 000 фунтів стерлінгів і т. д., зростаючи з кожним збільшенням капіталу,
не зважаючи на зменшення норми. Однак, таке зростання має місце тільки на
протязі певного часу; так, 19\% від 200000 фунтів стерлінгів є більше, ніж 20\%
від 100 000 фунтів стерлінгів, 18\% від 300 000 фунтів стерлінгів знов таки
більше, ніж 19\% від 200 000 фунтів стерлінгів; але після того, як капітал уже
нагромадився до великої суми, а зиски зменшились, дальше нагромадження
зменшує загальну суму зиску. Так, якщо припустимо, що нагромадження
становить 1 000 000 фунтів стерлінгів, а зиск 7\%, то загальна сума зиску становитиме
70 000 фунтів стерлінгів; якщо тепер до капіталу в мільйон буде
додано 100 000 фунтів стерлінгів і зиск знизиться до 6\%, то власники капіталу
одержать 66 000 фунтів стерлінгів, або на 4000 фунтів стерлінгів менше, хоч
загальна сума капіталу зросла з 1000 000 фунтів стерлінгів до 1 100 000 фунтів
стерлінгів“. Ricardo: „Principles of Political Economy“, розд. VII („Works“
вид. Мак-Куллоха, 1852, стор. 68 [69]). В дійсності тут припускається, що капітал
зростає з 1 000 000 до 1 100 000, тобто на 10\%, тимчасом як норма зиску
падає з 7 до 6, тобто на 14 2/7\%. Hinc illae lacrimae [звідси ці сльози].
