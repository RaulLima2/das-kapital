\parcont{}  %% абзац починається на попередній сторінці
\index{iii1}{0189}  %% посилання на сторінку оригінального видання
справу тільки з окремими товарами, ми могли припускати, що
існує потреба в цьому певному товарі, — кількість якого вже
дана в його ціні, — не вдаючись до ближчого визначення кількісної сторони потреби, яку належить
задовольнити. Але ця
кількісна сторона стає істотним моментом, якщо на одному боці
стоїть продукт цілої галузі виробництва, а на другому — суспільна потреба. Тепер треба розглянути
міру, тобто кількість цієї суспільної потреби.

В даних вище визначеннях ринкової вартості припускається,
що маса вироблених товарів лишається незмінною, що вона
є дана; що змінюється тільки відношення між складовими частинами цієї маси, виробленими при різних
умовах, і що тому
ринкова вартість тієї самої маси товарів регулюється по-різному.
Припустімо, що ця маса товарів становить звичайні розміри
подання, при чому ми залишаємо осторонь ту можливість, що
частина вироблених товарів може час від часу забиратися
з ринку. Якщо попит на цю масу товарів теж лишається звичайним, то товар продається по його ринковій
вартості, хоч би
який з досліджених раніше трьох випадків регулював цю ринкову вартість. Товарна маса не тільки
задовольняє певну потребу, але задовольняє її в її суспільному розмірі. Якщо, навпаки, кількість
товарів є менша чи більша, ніж попит на них,
то мають місце відхилення ринкової ціни від ринкової вартості.
І перше відхилення полягає в тому, що коли кількість товарів
занадто мала, то ринкову вартість завжди регулюють товари,
вироблені при найгірших умовах, а якщо кількість їх занадто
велика, то ринкову вартість регулюють товари, вироблені при
найкращих умовах; що, отже, ринкову вартість визначає один
з крайніх полюсів, не зважаючи на те, що відповідно до самого
тільки відношення між масами, виробленими при різних умовах,
мусив би бути інший результат. Якщо ріжниця між попитом
і кількістю продуктів значніша, то ринкова ціна так само ще
значніше відхилятиметься вгору або вниз від ринкової вартості.
Але ріжниця між кількістю вироблених товарів і тією кількістю
їх, при якій вони продаються по їх ринковій вартості, може
мати двояку причину. Або змінюється сама ця кількість, стає
занадто малою чи занадто великою, тобто репродукція відбувається в іншому масштабі, ніж той, що
регулював дану
ринкову вартість. В цьому випадку змінюється подання, хоч
попит лишається той самий, і в наслідок цього настає відносна
перепродукція або недопродукція. Або ж репродукція, тобто
подання, лишається та сама, а попит падає чи підвищується,
що може статися з різних причин. Хоч при цьому абсолютна
величина подання лишилась та сама, але його відносна величина,
його величина порівняно з потребою, або вимірювана потребою,
змінилася. Результат той самий, що й у першому випадку,
тільки зворотного напряму. Нарешті: якщо відбуваються зміни
на одній і на другій стороні, але в протилежному напрямі,
\parbreak{}  %% абзац продовжується на наступній сторінці
