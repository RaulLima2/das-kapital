тості = m, відношення якої до нього, m/v, є норма додаткової
вартості m'. Тільки таким шляхом рівняння р' = m/ (c + v) перетворилось
в друге: р' = m' (v/ (c + v) ). Тепер v чисельника ближче визначається
тим, що воно мусить бути рівне v знаменника, тобто
всій змінній частині капіталу К. Інакше кажучи, рівняння
р' = m/K можна тільки тоді без помилки перетворити в друге рівняння
р' = m' (v/ (c + v) ), коли m означає додаткову вартість, вироблену
за один період обороту змінного капіталу. Якщо m охоплює
тільки частину цієї додаткової вартості, то хоч m — m'v є
правильне рівняння, але це v тут менше, ніж v в K = с + v, бо
воно менше, ніж весь змінний капітал, витрачений на заробітну
плату. Якщо ж m охоплює більше, ніж додаткову вартість від
одного обороту v, то частина цього v або навіть все v функціонує
двічі: спочатку в першому, потім у другому або в другому
й дальших оборотах; отже, це v, яке виробляє додаткову
вартість і яке становить суму всієї виплаченої заробітної плати,
є більше, ніж v в c + v, і тому обчислення стає неправильним.

Для того, щоб формула річної норми зиску стала цілком
правильною, ми повинні замість простої норми додаткової вартості
поставити річну норму додаткової вартості, тобто замість
m' поставити М', або m'n. Інакше кажучи, ми повинні помножити
m', норму додаткової вартості — або, що зводиться до
того самого, вміщену в К змінну частину капіталу v, — на n,
число оборотів цього змінного капіталу за рік, і таким чином
ми одержуємо: р' = m'm (v/K), формулу для обчислення річної
норми зиску.

Але яка є величина змінного капіталу в певному підприємстві,
цього в більшості випадків не знає і сам капіталіст.
У восьмому розділі другої книги ми бачили і побачимо ще
далі, що єдина ріжниця в капіталі капіталіста, яка нав’язується
йому як істотна, є ріжниця основного й обігового капіталу.
З каси, в якій знаходиться частина обігового капіталу, яку він
має в своїх руках у грошовий формі, — оскільки вона не лежить
у банку, — він бере гроші для заробітної плати, з тієї самої
каси він бере гроші для сировинних і допоміжних матеріалів
і записує ті і другі на той самий рахунок каси. А коли б йому
й довелося вести окремий рахунок виплачуваної заробітної
плати, то цей рахунок в кінці року, правда, показав би виплачену
на заробітну плату суму, тобто vn, але не показав би
самого змінного капіталу v. Щоб визначити цей останній, капі-
