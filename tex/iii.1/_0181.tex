\parcont{}  %% абзац починається на попередній сторінці
\index{iii1}{0181}  %% посилання на сторінку оригінального видання
плату плюс додаткову вартість, додаткову працю понад їх необхідні
потреби, при чому, однак, результати її належали б їм
самим. Висловлюючись капіталістичною мовою, обидва робітники
одержують рівну заробітну плату плюс рівний зиск, але разом
з тим і вартість, виражену, наприклад, у продукті десятигодинного
робочого дня. Але, поперше, вартості їх товарів були б
різні. Нехай, наприклад, з уміщеної в товарі І вартості на спожиті
засоби виробництва припадає більша частина вартості, ніж у товарі
II, і — щоб урахувати тут усі можливі ріжниці — припустімо,
що товар І вбирає більше живої праці, отже, потребує довшого
робочого часу для свого виготовлення, ніж товар II. Отже, вартість
цих товарів І і II дуже різна. Так само різні й суми товарних вартостей,
які є продуктом праці, виконаної за даний час робітником
І і робітником II. Норми зиску теж дуже різні для І і II,
якщо ми назвемо тут нормою зиску відношення додаткової вартості
до всієї вартості витрачених засобів виробництва. Засоби
існування, які щодня споживаються робітниками І і II протягом
виробництва і які представляють заробітну плату, становлять
тут ту частину авансованих засобів виробництва, яку ми в інших
випадках звемо змінним капіталом. Але додаткові вартості за
однаковий робочий час були б для І і II однакові, або ще точніше:
через те що І і II одержують кожний вартість продукту
одного робочого дня, вони одержують — якщо відрахувати вартість,
авансованих „сталих“ елементів — однакові вартості, одну
частину яких можна розглядати як заміщення спожитих у виробництві
засобів споживання, а другу — як додаткову вартість,
яка лишається понад це. Якщо І зробив більше витрат, то вони
заміщаються більшою частиною вартості його товару, яка заміщає
цю „сталу“ частину, і тому він повинен також більшу
частину всієї вартості свого продукту перетворити знову в речові
елементи цієї сталої частини, тимчасом як II, якщо він
менше одержав як заміщення, повинен зате настільки ж менше
знову перетворити в елементи сталої частини. Отже, при цьому
припущенні ріжниця в нормах зиску була б байдужою обставиною,
цілком так само, як нині для найманого робітника байдуже,
в якій нормі зиску виражається видушена з нього кількість
додаткової вартості, і цілком так само, як у міжнародній торгівлі
ріжниця норм зиску у різних націй є байдужа обставина
для їх товарообміну.

Отже, для обміну товарів по їх вартостях, або приблизно
по їх вартостях, потрібен значно нижчий ступінь, ніж для обміну
по цінах виробництва, для якого потрібна певна висота капіталістичного
розвитку.

Яким би чином не встановлювались або регулювались первісно
ціни різних товарів одного відносно одного, закон вартості
керує їх рухом. Де зменшується робочий час, потрібний для
виробництва товарів, там падають і ціни; де він збільшується,
там підвищуються, при інших незмінних умовах, і ціни.
