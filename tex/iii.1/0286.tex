З одного боку, такий торговельний робітник є такий самий
найманий робітник, як і всякий інший. Поперше, оскільки його
праця купується на змінний капітал купця, а не на ті гроші, що
витрачаються як дохід; отже, оскільки вона купується не для
особистих послуг, а з метою самозростання вартості авансованого
на це капіталу. Подруге, оскільки вартість його робочої
сили, а тому і його заробітна плата, визначається, як і в усіх
інших найманих робітників, витратами виробництва і репродукції
його специфічної робочої сили, а не продуктом його праці.

Але між ним і робітниками, безпосередньо вживаними промисловим
капіталом, мусить існувати така сама ріжниця, яка існує
між промисловим капіталом і торговельним капіталом, а тому й
між промисловим капіталістом і купцем. Через те що купець, як
простий агент циркуляції, не виробляє ні вартості, ні додаткової
вартості (бо та добавна вартість, яку він додає до товарів своїми
витратами, зводиться до додання вартостей, які вже раніш існували,
хоч тут нав’язується питання: яким чином він удержує, зберігає
цю вартість свого сталого капіталу?), то й торговельні робітники,
вживані ним для виконання тих самих функцій, не можуть
безпосередньо створювати для нього додаткову вартість. Тут, як
і тоді, коли справа йшла про продуктивних робітників, ми припускаємо,
що заробітна плата визначається вартістю робочої сили,
отже, купець не збагачується відрахуваннями з заробітної плати,
так що в обрахунок своїх витрат він вносить не таке авансування
на працю, яке оплачувало б її тільки почасти, — іншими словами,
він збагачується не тим, що обшахровує своїх прикажчиків і т. п.

Труднощі при вивченні питання про торговельних найманих
робітників полягають зовсім не в тому, щоб пояснити, яким чином
вони виробляють зиск безпосередньо для свого наймача, хоч
безпосередньо вони не виробляють додаткової вартості (а зиск
є тільки перетворена форма її). Це питання в дійсності розв’язане
вже загальним аналізом торговельного зиску. Подібно до
того, як промисловий капітал одержує зиск в наслідок того,
що продає вміщену в товарах і реалізовану працю, за яку він
не заплатив ніякого еквіваленту, цілком так само і торговельний
капітал одержує зиск в наслідок того, що він оплачує продуктивному
капіталові не всю неоплачену працю, яка міститься
в товарі (в товарі, оскільки капітал, витрачений на його виробництво,
функціонує як відповідна частина сукупного промислового
капіталу); навпаки, при продажу товарів він примушує
заплатити собі за цю неоплачену ним частину праці, яка ще
міститься в товарах. Відношення купецького капіталу до додаткової
вартості інше, ніж відношення промислового капіталу.
Останній виробляє додаткову вартість шляхом безпосереднього
привласнювання неоплаченої чужої праці. Перший привласнює
собі частину цієї додаткової вартості, примушуючи промисловий
капітал відступати йому цю частину.

Тільки за допомогою своєї функції реалізації вартостей торго-
