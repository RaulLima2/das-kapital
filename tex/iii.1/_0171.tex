\parcont{}  %% абзац починається на попередній сторінці
\index{iii1}{0171}  %% посилання на сторінку оригінального видання
капіталів тієї галузі виробництва, в якій склад капіталу випадково є суспільно-пересічний, вартість
і ціна виробництва були б
рівні. А втім, прикладаючи ці означення до певних випадків,
треба, звичайно, брати до уваги, наскільки не ріжниця в технічному складі, а проста зміна вартості
елементів сталого капіталу
відхиляє відношення між $c$ і $v$ від загального пересічного.

Те, що ми тут розвинули, безперечно, модифікує визначення
витрат виробництва товарів. Первісно ми припускали, що витрати виробництва товару дорівнюють
\emph{вартості} товарів, спожитих на його виробництво. Але ціна виробництва якогось товару для покупця
товару є витрати виробництва цього товару
і може таким чином увійти як витрати виробництва в утворення
ціни іншого товару. Через те що ціна виробництва товару може
відхилятись від його вартості, то й витрати виробництва товару,
в яких включена ця ціна виробництва іншого товару, також
можуть бути вищі або нижчі тієї частини всієї його вартості,
яка утворюється вартістю засобів виробництва, що входять в
товар. Треба пам’ятати про це модифіковане значення витрат
виробництва, отже, пам’ятати, що завжди можлива помилка,
якщо в якійсь окремій сфері виробництва витрати виробництва товару прирівнюються до вартості
спожитих на його виробництво
засобів виробництва. Для цього нашого дослідження немає потреби докладніше розглядати цей пункт. При
цьому завжди лишається правильним положення, що витрати виробництва товарів
завжди менші, ніж їх вартість. Справді, як би не відхилялись
витрати виробництва товару від вартості спожитих на нього
засобів виробництва, для капіталіста ця минула помилка не має
ніякого значення. Витрати виробництва товару є дані, вони є
незалежна від його, капіталіста, виробництва передумова, тим часом як результат його виробництва є
товар, який містить у
собі додаткову вартість, отже, певний надлишок вартості понад
витрати виробництва товару. А втім, положення, що витрати
виробництва є менші, ніж вартість товару, практично перетворилось тепер у положення, що витрати
виробництва є менші,
ніж ціна виробництва. Для сукупного суспільного капіталу, для
якого ціна виробництва дорівнює вартості, це положення є
тотожне з попереднім: що витрати виробництва менші, ніж вартість. Хоч для окремих сфер виробництва
воно має мінливе значення, проте, основою його завжди лишається той факт, що при розгляді сукупного
суспільного капіталу витрати виробництва
вироблених ним товарів є менші, ніж вартість, або в даному разі
для сукупної маси вироблених товарів, менші, ніж тотожна з цією
вартістю ціна виробництва. Витрати виробництва товару відповідають тільки кількості вміщеної в ньому
оплаченої праці,
вартість же — всій кількості вміщеної в ньому оплаченої і неоплаченої праці; ціна виробництва — сумі
оплаченої праці плюс
певна, для кожної окремої сфери виробництва від неї самої незалежна, кількість неоплаченої праці.
