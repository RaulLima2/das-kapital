цесом циркуляції. Експлуатація продуктивної праці коштує зусиль,
однаково, чи займається нею сам капіталіст, чи хто інший від його
імени. Отже, в протилежність до процента, його підприємницький
дохід здається йому незалежним від власності на капітал,
скоріше результатом його функцій як невласника, як — робітника.

Тому в його мозку необхідно виникає уявлення; що його підприємницький
дохід дуже далекий від того, щоб становити будьяку
протилежність до найманої праці і бути лише неоплаченою чужою
працею, а, навпаки, сам є заробітною платою, платою за нагляд,
wages of superintendence of labour, вищою платою, ніж плата звичайного
найманого робітника, 1) тому що це складніша праця, 2)
тому що він сам собі виплачує заробітну плату. Що його функція
як капіталіста полягає в тому, щоб добувати додаткову вартість,
тобто неоплачену працю, і до того ж при найекономніших умовах,
— це зовсім забувається в наслідок тієї суперечності, що процент
дістається капіталістові, хоч би він і не виконував ніякої
функції як капіталіст, а був би тільки власником капіталу, і що,
навпаки, підприємницький дохід дістається функціонуючому капіталістові,
хоч би він і не був власником капіталу, з яким він
функціонує. За антагоністичною формою обох частин, на які
розпадається зиск, тобто додаткова вартість, забувається, що
обидві вони є просто частини додаткової вартості і що її поділ
нічого не може змінити ні в її природі, ні в її походженні та
умовах її існування.

В процесі репродукції функціонуючий капіталіст виступає
відносно найманих робітників представником капіталу як чужої
власності, і грошовий капіталіст, будучи представлений функціонуючим
капіталістом, бере участь в експлуатації праці. Що
активний капіталіст може виконувати свою функцію, яка полягає
в тому, щоб заставляти робітників працювати на нього або
заставляти засоби виробництва функціонувати як капітал, — що
він може виконувати цю функцію тільки як представник засобів
виробництва у протилежність робітникам, це забувається
в наслідок протилежності між функцією капіталу в процесі
репродукції і простою власністю на капітал поза процесом
репродукції.

Справді, в тій формі, яку обидві частини зиску, тобто додаткової
вартості, набирають як процент і підприємницький дохід,
не виражене ніяке відношення до праці, тому що це відношення
існує тільки між нею і зиском або, точніше, додатковою вартістю
як сумою, як цілим, як єдністю обох цих частин. Відношення, в
якому ділиться зиск, і різні юридичні титули, на основі яких відбувається
це ділення, передбачають зиск як готовий, передбачають
існування зиску. Тому, якщо капіталіст є власник того капіталу,
з яким він функціонує, то він кладе собі в кишеню весь зиск
або додаткову вартість; для робітника цілком байдуже, чи капіталіст
кладе собі в кишеню весь зиск, чи він повинен частину
сплачувати третій особі, як юридичному власникові. Таким
