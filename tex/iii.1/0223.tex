бо вона мусить зрости навіть для того, щоб при зміненому
складі капіталу можна було вживати ту саму масу праці при
попередніх відношеннях експлуатації.

Отже, той самий розвиток суспільної продуктивної сили
праці виражається з прогресом капіталістичного способу виробництва,
з одного боку, в тенденції до прогресуючого падіння
норми зиску, а з другого боку, в постійному зростанні абсолютної
маси привласнюваної додаткової вартості або зиску; так
що загалом відносному зменшенню змінного капіталу і зиску
відповідає абсолютне збільшення обох. Ця двобічна дія, як ми
вже показали, може виразитись тільки в зростанні всього капіталу
в швидшій прогресії, ніж та, в якій падає норма зиску. Для
того, щоб при вищому складі капіталу або при відносно сильнішому
збільшенні сталого капіталу можна було вжити абсолютно
зрослий змінний капітал, весь капітал мусить зрости не
тільки відповідно до вищого складу, але ще швидше. З цього
випливає, що чим більше розвивається капіталістичний спосіб
виробництва, тим більша й більша маса капіталу потрібна для
того, щоб уживати ту саму робочу силу, і ще більша для того,
щоб уживати вирослу робочу силу. Отже, зростаюча продуктивна
сила праці на капіталістичній базі з необхідністю створює
постійне позірне перенаселення робітників. Якщо змінний капітал
становить тільки 1/6 всього капіталу замість колишньої
1/2, то, щоб можна було вжити ту саму робочу силу, весь
капітал мусить потроїтись; а для того, щоб можна було вжити
подвійну робочу силу, він мусить пошестеритись.

Дотеперішня політична економія, яка не зуміла була пояснити
закон падіння норми зиску, вказувала на підвищення маси
зиску, зростання абсолютної величини зиску, чи то для окремих
капіталістів, чи для суспільного капіталу, як на свого роду
підставу для утішення, але й вона базується на самих тільки
загальних місцях і можливостях.

Те, що маса зиску визначається двома факторами, поперше,
нормою зиску і, подруге, масою капіталу, вжитого для одержання
цієї норми зиску, — це просто тавтологія. Тому та обставина,
що зростання маси зиску можливе, не зважаючи на одночасне
падіння норми зиску, є тільки вираз цієї тавтології і
не допомагає ні на крок посунутися вперед, бо цілком так само
можливе й те, що капітал зростатиме без зростання маси зиску
і що він може навіть зростати і в тому випадку, коли вона
падає. 100 при 25\% дає 25, 400 при 5\% дає тільки 20. 35 Але
якщо ті самі причини, які викликають падіння норми зиску,

35 „We should also expect that, however the rate of the profits of stock
might diminish in consequence of the accumulation of capital on the land and the
rise of wages, yet the aggregate amount of profits would increase. Thus, supposing
that, with repeated accumulations of 100000 £, the rate of profits should fall from
20 to 19, to 18, to 17 per cent., a constantly diminishing rate; we should expect that
the whole amount of profits received by those successive owners of capital would be
always progressive; that it would be greater when the capital was 200000 £, than
