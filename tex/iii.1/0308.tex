срібло (або золото) функціонує у відміну від ходячої монети
як світові гроші, — тепер як банкові гроші або торговельні гроші.
Вексельна справа, оскільки вона зводилась просто до видачі мандрівникам
міняйлом якоїсь країни свідоцтва на одержаний грошей
від інших міняйл, розвинулась уже в Римі й Греції з власне
міняльної справи.

Торгівля золотом і сріблом як товарами (як сировинними
матеріалами для виготовлення предметів розкоші) становить
природно вирослу базу для торгівлі злитками (bullion trade) або
торгівлі, яка опосереднює функції грошей як світових грошей.
Ці функції, як це вже було з’ясовано раніше (книга І, розд. III,
З, с), двоякого роду: переміщення грошей між різними національними
сферами циркуляції для вирівнювання міжнародних
платежів і при мандруванні капіталу, що шукає процентів;
поруч із цим рух від джерел добування благородних металів
по світовому ринку і розподіл видобутку між різними національними
сферами циркуляції. В Англії ще протягом більшої
частини XVII століття функції банкірів виконували майстри
по золоту. Яким чином вирівнювання міжнародних платежів
розвивається далі у вексельну торгівлю та ін., це ми тут залишаємо
цілком осторонь, як і все те, що стосується до операцій
з цінними паперами, коротко кажучи, усі особливі форми
кредитної справи, яка нас тут ще не стосується.

Як світові гроші, національні гроші скидають із себе свій
місцевий характер; гроші однієї країни виражаються в грошах
іншої країни, і таким чином усі вони зводяться до вміщеного
в них золота або срібла, тимчасом як золото й срібло, як два
товари, що циркулюють як світові гроші, разом з тим зводяться
до взаємного відношення їх вартостей, яке постійно
міняється. Опосереднення цих операцій торговець грішми робить
своїм особливим заняттям. Таким чином, міняльна справа
і торгівля злитками — це є найбільш первісні форми торгівлі
грішми і виникають вони з двояких функцій грошей: як місцевих
монет і як світових грошей.

З капіталістичного процесу виробництва, як з торгівлі взагалі,
навіть при докапіталістичному способі виробництва, випливає:

Поперше, збирання грошей як скарбу, тобто в теперішній
час як частини капіталу, яка завжди мусить бути в наявності

лось... вони стали касирами і банкірами свого часу. Але в поєднанні професій
касира і міняйла власті Амстердама побачили небезпеку і, щоб запобігти цій
небезпеці, було вирішено заснувати велику установу, яка повинна була б провадити
операції як міняйл, так і касирів, і діяти відкрито, згідно з статутом.
Такою установою був знаменитий Амстердамський розмінний банк 1609 року.
Цілком так само розмінні банки Венеції, Генуї, Стокгольма, Гамбурга виникли
в наслідок постійної потреби в розміні грошових знаків. З усіх цих банків
існує ще й тепер один тільки гамбурзький, бо в цьому торговельному місті, яке
не має своєї власної монетної системи, все ще відчувається потреба в такій
установі і т. д.“). (S. Vissering: „Handboek van Praktische Staatshuishoudkunde“.
Amsterdam 1860, І, стор. 247).
