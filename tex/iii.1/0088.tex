лась, а на авансовану капітальну вартість у 100, і таким чином
одержуємо: р' = 40\%.

Порівняймо з цим капітал В = 160 с + 40 v = 200 К, який
функціонує при такій самій нормі додаткової вартості в 100\%,
але обертається тільки один раз на рік. Тоді річний продукт
буде, як і вище:

160 с + 40 v + 40 m. Але на цей раз ці 40 m слід обчислити на
авансований капітал у 200; це дає для норми зиску тільки 20\%,
отже, тільки половину норми для А.

Звідси випливає: при капіталах однакового процентного
складу, при однаковій нормі додаткової вартості і однаковому
робочому дні, норми зиску двох капіталів стоять у зворотному
відношенні до часу їх оборотів. Якщо в двох порівнюваних випадках
неоднаковий склад, або норма додаткової вартості, або
робочий день, або заробітна плата, то цим, звичайно, будуть
породжені й дальші ріжниці в нормі зиску; але вони незалежні
від обороту і тому нас не цікавлять тут; вони вже розглянуті
в розділі III.

Безпосередній вплив скороченого часу обороту на виробництво
додаткової вартості, отже й зиску, полягає в підвищеній
діяльності, яка таким способом надається змінній частині капіталу,
про що див. книгу II, розділ XVI: Оборот змінного
капіталу. Там виявилось, що змінний капітал у 500, який обертається
десять разів на рік, привласнює за цей час стільки ж додаткової
вартості, як і змінний капітал у 5000, який при однаковій
нормі вартості і однаковій заробітній платі обертається
тільки один раз на рік.

Візьмемо капітал І, що складається з 10 000 основного капіталу,
— річне зношування якого становить 10\% = 1000, — 500
обігового сталого і 500 змінного капіталу. При нормі додаткової
вартості в 100\% цей змінний капітал обертається десять
разів на рік. Задля спрощення ми припускаємо в усіх дальших
прикладах, що обіговий сталий капітал обертається за той
самий час, як і змінний, що й на практиці здебільшого приблизно
так і буває. Тоді продукт одного такого періоду обороту буде:

100 с (зношування) + 500 с + 500 v + 500 m = 1600,
а продукт цілого року з десятьма такими оборотами:

1000 с (зношування) + 5000 с + 5000 v + 5000 m = 16 000,

 К = 11 000; m = 5000, р' = 5000/11000 + 45 5/11\%.

Візьмемо тепер капітал II: основний капітал — 9000, його річне
зношування — 1000, обіговий сталий капітал — 1000, змінний
капітал — 1000, норма додаткової вартості — 100\%, число річних
оборотів змінного капіталу — 5. Отже, продукт кожного періоду
обороту змінного капіталу буде:

200 c (зношування) + 1000 c + 1000 v + 1000 m = 3200,
