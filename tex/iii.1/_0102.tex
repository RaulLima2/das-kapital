\index{iii1}{0102}  %% посилання на сторінку оригінального видання
Це треба тепер коротко пояснити окремими ілюстраціями.
Ми починаємо з кінця, з економії на умовах виробництва,
оскільки ці останні разом з тим становлять собою умови існування й життя робітника.

\subsection{Заощадження на умовах праці коштом робітника}

\emph{Кам’яновугільні копальні. Нехтування найнеобхідніших витрат.}

„При тій конкуренції, яка панує... між володільцями кам’яновугільних копалень, не робиться витрат
більше, ніж потрібно
для того, щоб подолати найочевидніші фізичні труднощі; а при
конкуренції між шахтними робітниками, яких звичайно є надмірна кількість, ці останні охоче піддають
себе значним небезпекам і найшкідливішим впливам за плату, тільки трохи вищу,
ніж плата сусідніх сільських поденників, бо праця в копальнях
дозволяє з вигодою використовувати ще й їхніх дітей. Цієї подвійної конкуренції цілком досить... щоб
привести до того, що
більша частина копалень експлуатується при надто недосконалих осушенні і вентиляції; часто при
погано збудованих шахтах,
з поганими скріпами, при нездатних машиністах, погано прокладених і погано збудованих штольнях і
відкотних шляхах; а це спричиняє руйнування життя і здоров’я робітників або калічення їх,
про що статистика дала б жахливу картину“ („First Report
on Children’s Employment in Mines and Collieries etc. 21 April
1829“, стор. 102). В англійських кам’яновугільних копальнях коло
1860 року щотижня вбивалося пересічно 15 чоловіка. За звітом
про Coal Mines Accidents [нещасні випадки в кам’яновугільних
копальнях] (6 лютого 1862 р.), за 10 років, з 1852 до 1861,
було вбито разом 8466 чоловіка. Але це число, як каже сам
звіт, значно применшене, бо в перші роки, коли фабричні інспектори були тількищо настановлені, а їх
округи були занадто
великі, про велике число нещасних і смертельних випадків зовсім не повідомлялось. Саме та обставина,
що, не зважаючи на
все ще дуже великий убій робітників і недостатнє число та
незначну владу інспекторів, число нещасних випадків, відколи
засновано інспекцію, дуже зменшилось, — саме це показує природну тенденцію капіталістичної
експлуатації. — Ці людські
жертви є здебільшого результат брудної скнарості володільців
шахт, які часто, наприклад, викопують одну тільки шахту, так
що не тільки неможлива ніяка справжня вентиляція, але неможливо і вийти з шахти, коли цей єдиний
вихід завалюється.

Капіталістичне виробництво, якщо ми розглядатимем його
відокремлено, абстрагуючись від процесу циркуляції і спустошень конкуренції, поводиться надзвичайно
економно із здійсненою, упредметненою в товарах працею. Навпаки, воно далеко
більше, ніж всякий інший спосіб виробництва, є марнотратником
людей, живої праці, марнотратником не тільки тіла й крові, але
\parbreak{}  %% абзац продовжується на наступній сторінці
