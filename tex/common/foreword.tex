\nonumsection{Чи застарів «застарілий» Маркс?}{~}{Іван Дзюба}

Почати з того, що Маркс застарівав уже не один раз. Спершу — ще після 
революцій 1848 року, які розвивалися не за логікою «Комуністичного 
маніфесту». Потім — після невдачі Паризької Комуни. Далі — коли його 
відмодельовували в протилежні боки «ревізіоністи» (від Бернштейна до 
Каутського) і Ленін, більшовики, Сталін\ldots{} А скільки в нього 
застарілих окремих формул і тез! Наприклад, колись знамените про 
«ідіотизм сільського життя». Це ж як воно звучить тепер, коли світ 
потерпає від набагато глибшого і страшнішого ідіотизму мегаполісів?!


Про те, що Карл Маркс застарів, знає у нас навіть той, хто взагалі 
нічого не знає (особливо він). Але, хоч як дивно, всупереч нашому знанню 
і незнанню, у західних університетах його праці поважно вивчають 
(звісно, з певної критичної позиції), про нього пишуть видатні 
соціологи й філософи як про одного з великих мислителів людства, його 
перевидають і шукають у нього стежок до пояснення економічних криз 
сучасного світу тощо, — а 5-го травня цього року широко відзначалося 
його 200-річчя. Але все це — «за бугром». У нас же будь-який 
малоосвічений публіцист може при нагоді поглузувати з «двох 
німецьких гномів» (це довелося зустріти недавно в інтернетному 
дописі). Таке от бачення (власне, небачення) історії, такий рівень 
культури мислення, таке розуміння динаміки інтелектуального розвитку 
людства, за якої насправді нові осягнення виростають із 
«застарілого», а заперечуване відходить, тільки стимулювавши саме 
заперечення, або й повертається невпізнане.


Ще одна біда — коли говорити про «масову людину» — брак історичного 
підходу в поцінуванні культурних явищ та феноменів думки, понятійних 
категорій. Багато хто в простоті душевній гадає, що це Маркс придумав 
класи, класову боротьбу, пролетаріат, революції та інший клопіт, отож 
усі біди від нього, Маркса. Таким чином на нього ніби падає 
відповідальність за століття (або й тисячоліття) соціальних 
конфліктів і майнових битв на нашій планеті. Хай так. Але на 
виправдання Маркса можна сказати, що в нього було багато попередників. 
Не будемо зазирати в біблійні, або античні, чи й середньовічні часи, а 
звернімося до ХIХ ст., в якому й визрівало те, що згодом дістало назву 
марксизму.


Отож: у межах європейських феодальних монархій народжується і набирає 
сил буржуазія, що здобуває економічні й політичні позиції, 
використовуючи суперечності між монархом і його васалами та свої 
фінансово-майнові важелі; відбувається нагромадження капіталу, 
розвивається фабрична промисловість, змінюється характер виробничих 
і суспільних відносин та способи експлуатації робітника, колишні 
дрібні власники й обезземелені селяни стають знекоріненою і 
безправною «робочою силою», зростає безробіття. Пролетаризація 
охоплює цілі суспільні верстви, збільшуючи зубожіння люду й набираючи 
катастрофічного характеру. Як відповідь на біди, що їх приніс масі 
населення, насамперед трудовому людові, бурхливий розвиток 
жорстокого й хижацького капіталізму; як відповідь на загострення 
соціальних антагонізмів та масових злиднів — виникають, з одного 
боку, стихійні бунти, наприклад, луддитів та інших руйначів машин, 
потім і бунти та масові революційні рухи, а з другого — народжуються 
соціальні міфи й утопії та спроби окремих гуманістично мислячих, з 
чутливою соціальною совістю особистостей, як правило з освічених, 
упривілейованих станів, запропонувати моделі подолання кричущих 
суспільних дисгармоній і шляхи влаштування справедливих відносин, 
бодай у окремих локальних осередках, якщо не взагалі у світі. Так 
народжується утопічний соціалізм, яскрава плеяда теоретиків якого 
(К.А. Сен-Симон, Ш. Фур'є, Р.Оуен, а також Т.Мюнцер, Т.Кампанелла, Мореллі, 
Ж.Мельє, Дж.Уїнстлі, Г.Б.Маблі, Г.Бабьйоф, Т.Дезамі) за всієї відмінності 
поміж собою в конкретних позиціях і національному представництві 
були суголосними в критиці реального стану супільств як 
невідповідного поняттям про гідне життя, справедливість, моральність 
і доцільність, — а тому й неприйнятного для людського розуму й 
совісті. Їхні проекти ідеального суспільства базувалися на 
ідеалістичних уявленнях про рівність, свободу і братерство, про 
нібито добрі від природи моральні засади людини. На зміну приватній 
власності мало прийти велике колективне виробництво із справедливим 
розподілом і забезпеченням потреб кожного; в такому суспільстві буде 
подолано різницю між розумовою і фізичною працею, суперечність між 
містом і селом. (Мрія ця супроводжувала і ще супроводжуватиме чи не всю 
історію світу, вона позачасова!) Щоб прийти до такого суспільства 
справедливості, до здійснення цієї споконвічної мрії людства, треба 
було всього лиш переконати людність у його перевагах. Одначе ця проста 
і зрозуміла справа чомусь не вдавалася, як не вдавалися і спроби 
жертовних мрійників подати власний приклад організацією 
соціалістичних комун або фаланстерів. Побудувати комунізм чи 
соціалізм в окремо взятій громаді виявилося неможливим.


За цих умов і постає необхідність в іншому, неутопічному, 
реалістичному підході, обгрунтованому не моральною риторикою, а 
економічно, ідеологічно, політично, з орієнтацією на докорінну, 
найпевніше силову, перебудову всього суспільства. А на яку соціальну 
силу можна покладатися?


1842 року з'являється праця німецького вченого-юриста Лоренца фон 
Штайна (1815--1890) «Соціалізм і комунізм у сьогоднішній Франції». Це було 
за шість років до європейських революцій 1848-го і до появи 
«Комуністичного маніфесту» Маркса й Енгельса. Лоренц фон Штайн був 
одним із перших, хто розробляв теорію пролетаріату (невдовзі, 1845-го, 
з'являється праця Маркса і Енгельса «Свята родина», присвячена цій 
темі, але вже із розробленням стратегії дій революційного 
пролетаріату). Штайн показав, що клас пролетарів неминуче з'являється 
внаслідок появи і діяльності класу капіталістів. За вільноринкової 
економіки свободу і права мають власники, а не робітники. Учений-юрист 
ліберальних переконань, з лівих молодогегельянців, він гадав, що певні 
правові норми могли б допомогти пролетарям урівнятися з 
капіталістами і, таким чином, соціальної справедливості можна було б 
досягти без революції, шляхом реформ.

\subsection*{Молодий Маркс}

По-іншому розумів справу Карл Маркс. Він також вийшов із 
молодогегельянства, але швидко переріс його (праця Маркса і Енгельса 
«Німецька ідеологія», 1845--1846, містила розгорнуту критику ідеалізму 
Гегеля й непослідовності матеріалізму Феєрбаха). В його особі 
визначилися і рідкісно поєдналися філософ, ідеолог, соціолог, 
політичний діяч, журналіст-пропагандист, а згодом і економіст. Він уже 
був відомий як автор численних журналістських публікацій та наукових 
праць, присвячених обговоренню політичних і філософських проблем, 
полеміці з іншими теоретиками й ідеологами, аналізові тогочасного 
буржуазного суспільства. Досвід практичної роботи, широкий світогляд, 
філософська системність критичного мислення і потужний інтелект дали 
йому можливість узагальнити й переосмислити здобутки німецької 
філософії, французьких і англійських соціалістичних та комуністичних 
теорій, англійської політекономії (як відомо, Енгельс називав три 
джерела марксизму: німецька філософія, англійська політекономія, 
праці французьких істориків) — і прийти до принципово нових 
висновків. Вони чітко викладені в «Комуністичному маніфесті» 
авторства Маркса й Енгельса, по суті полемічному щодо утопічних або 
ретроградних ідей попередників. Не реформи, не регулювання ринку, не 
обмеження приватної власності на засоби виробництва, а повна їх 
націоналізація й одержавлення способів розподілу, що — уявлялося — 
зробить неможливою експлуатацію людини людиною, приведе до 
ліквідації класів та створення безкласового суспільства, в якому 
вільний розвиток кожного буде умовою вільного розвитку всіх. Для 
цього пролетаріат має взяти владу в свої руки революційним шляхом. Хоч 
є у Маркса й неоднозначні думки на цю тему. Революція відбудеться тоді, 
коли пролетаріат стане більшістю в суспільстві, але тоді він може 
прийти до влади й мирним шляхом. Зокрема, припускалося, що в країнах, де 
вже утвердився парламентський лад (Англія, США), пролетаріат може 
прийти до влади, перемігши на виборах. Саме на це згодом орієнтувалися 
соціал-демократичні партії II Інтернаціоналу, але які цього так і не 
дочекалися.


Картина майбутнього соціалістичного суспільства та шляхи його 
творення в «Комуністичному маніфесті» не обговорені скільки-небудь 
конкретно. Це була не так наукова праця, як 
політично-пропагандистський документ узагальнювального характеру. 
До речі, не слід забувати, що «Комуністичний маніфест» Маркс і Енгельс 
написали не з власного задуму, а на прохання міжнародної робітничої 
організації «Союз Комуністів». І точна його назва —  «Маніфест 
Комуністичної Партії». Тобто: вже існував досить організований 
робітничий рух, який потребував ідеологічного осмислення, і цю 
потребу мали задовольнити Маркс і Енгельс, вибір на яких упав, 
звичайно ж, не випадково. Але факт, що від самого початку не вони 
інспірували організований робітничий рух (у чому їх подеколи 
«звинувачували»), а робітничий рух їх «інспірував». Інша річ, що вони 
своїми ідеями надали нової якості й потужної енергії цьому рухові. 


«Маніфест Комуністичної Партії» завершував першу фазу діяльності 
«молодого» Маркса, в якій означилися основні його ідеї, що дістануть 
дальший розвиток, але вже й тоді своєю сукупністю були новим (хоч і не 
беззаперечним, і не беззаперечно новим у всьому) словом у науці й стали 
відомі під назвою історичний матеріалізм. Це, зокрема, погляд на 
історію людства крізь призму класової боротьби, соціальних 
антагонізмів, які і є рушієм розвитку (тут Маркс поглибив поняття 
класової боротьби, введене в обіг французькими істориками). Це 
твердження про неминучість революційних змін у суспільствах 
унаслідок суперечності між зростанням засобів виробництва й 
інерційністю суспільних відносин, боротьби між класом експлуататорів 
і класом експлуатованих. Це погляд на суму економічних відносин у 
суспільстві як на той базис, на якому виростає складна світоглядна, 
юридична, політична, ідеологічна, художня та ін. надбудова, що 
змінюється із зміною базису (теза, яка зазнавала і зазнає спростувань, 
почасти і через її профанацію вульгаризаторами марксизму: сам Маркс 
мав на увазі не пряму підпорядкованість надбудви базисові, а складну й 
багатоетапну опосередкованість зв'язку між базисом і надбудовою, хоча 
точних меж між одним і другим він не визначив, як і не наголосив 
зворотного впливу надбудови на базис). Далі, це важлива думка про те, що 
старий лад не відходить доти, доки не вичерпає своїх можливостей, а 
новий не приходить йому на зміну, доки не визріли передумови для нього. 
Навколо цих та інших Марксових ідей десятиліттями точилися суперечки 
між марксистами й антимарксистами, між ортодоксами й ревізіоністами, 
догматиками й реформаторами тощо.


Критики Маркса здебільше не охоплювали сукупності його поглядів та 
їхньої діалектики, їхньої часом вільної гри в концерті Марксових ідей. 
Так, один із непримиренних його негаторів, видатний мислитель ХХ ст. 
Арнольд Дж. Тойнбі у «Дослідженні історії» писав: «Німецький єврей 
Карл Маркс (1818--1883 рр.) намалював у барвах, які запозичив з 
апокаліптичних видінь відкинутої ним традиційної релігії, 
страхітливу картину відокремлення пролетаріату й класової війни, яку 
він розв'яже. Величезне враження, яке справив цей марксистський 
матеріалістичний апокаліпсис на стільки мільйонів умів, почасти 
пояснюється політичною войовничістю Марксової схеми, бо хоч вона й 
становить ядро загальної філософії історії, вона також являє собою 
революційний заклик до збройної боротьби» (Арнольд Дж. Тойнбі. 
Дослідження історії. Т.1. -- К., 1995. -- С. 362). 


Мусимо визнати, що ущиплива іронія Тойнбі стосується Марксової 
риторики чи метафорики, але не зачіпає суті, змісту його послання. Так 
само небагато дає і ревний пошук юдаїстських коренів у марксизмі. 
«Маркс поставив богиню «Історична Необхідність» на місце Єгови, а на 
місце євреїв, богообраного народу, — внутрішній пролетаріат 
західного світу. Його Мессіанське Царство — це диктатура 
Пролетаріату, але грандіозна будівля Єврейського Апокаліпсису легко 
вгадується під цим благеньким укриттям» (там само, с. 391). Безперечно! 
Але розпізнавання цієї метафорики не є спростуванням марксизму, бо ця 
метафорика давно вже стала складником європейського мислення, — хіба 
що за всіма ідеалами комунізму доведеться бачити проповіді Христа і 
зводити справу до цього. Не випадково ж існує християнський соціалізм, 
був християнський комунізм, який заперечувано ще в «Комуністичному 
маніфесті». 


Власне, Тойнбі іронізує фактично з «молодого» Маркса, часів до 
написання «Капіталу», і, як історик, бере до уваги насамперед його 
узагальнені історіософські моделі, що не вкладалися в циклопічну 
будову тойнбівского «Дослідження історії», яке охоплювало не одне 
тисячоліття і в масштабі якого марксизм міг здаватися епізодом.


\subsection*{Марксів «Капітал»}


\ldots{}«Молодий» Маркс був філософом, ідеологом, політичним публіцистом, 
але ще не економістом. «Зрілий» Маркс, критично опанувавши досягнення 
сучасної йому економічної науки, насамперед англійської, розпочинає 
фундаментальне дослідження капіталізму як формації, 
капіталістичного способу виробництва, — типологічно, за Марксовим 
визначенням, відмінного від азійського, античного й феодального 
розвитком продуктивних сил та способом експлуатації людини людиною 
(це дуже важлива частина Марксового вчення), — його очевидних та 
прихованих механізмів, його «таємниць» і перспектив та меж. Так 
з'являється перший том його «Капіталу» — праці, що справила 
величезний вплив на розвиток людської думки і на політичну історію 
людства. (Свій задум Маркс не встиг довести до кінця, і другий та третій 
томи «Капіталу» готував Енгельс з Марксових чернеток.)  


Маркс показав, що капіталізм — принципово новий історичний і 
економічний феномен: у тому сенсі, що для нього характерний не обмін 
товарів за допомогою грошей, як це було на докапіталістичних етапах 
історії людства, а обмін грошей за допомогою товарів. Через це метою 
капіталіста є грошовий прибуток, заради якого він готовий на все. А що 
є джерелом прибутку? Як створюється додаткова вартість? Це, сказати б, 
головна «таємниця» капіталізму, без розкриття якої не можна мати 
адекватного бачення його і не можна опрозорити його міфологію. Маркс 
зосереджується на цій «таємниці» і створює теорію вартості, теорію 
заробітної платні і теорію додаткової вартості — найбільшої 
«таємниці» капіталізму, що відтак перестає бути таємницею. 
Скрупульозний Марксів аналіз показує, що робочий день трудівника 
складається з двох частин — праці, яка повернеться в його зарплатню, і 
додану працю. Тобто, додаткова вартість — це неоплачена частина праці 
робітника. Праця робітника — товар, але дивовижний товар, єдиний 
товар, який виробляє вартість, вищу за власну вартість! Звідси — 
прибутки капіталіста, які тим більші, чим вище співвідношення між 
доданою вартістю і заробітною платнею. Це співвідношення є 
\emph{нормою}\emph{ }\emph{експлуатації}.


Можна, мабуть. сказати, що до Маркса категорія \emph{праці} виступала у 
суспільній свідомості (принаймні у вульгарно-матеріалістичному 
мисленні або в моралістичному) узагальнено, нерозчленованою: як 
джерело усякого багатства. На таке уявлення впливала не в останню 
чергу й протестантська трудова мораль. Певні ілюзії існували і в 
німецькому робітничому русі. Так, філософ-робітник Іосиф Дицген 
вважав, що праця — це Рятівник, удосконалення праці зробить те, чого не 
зміг досягти жоден Визволитель. Натомість Маркс не тільки показав, 
кому реально дістаються плоди праці, а й проаналізував економічні 
«складові» праці, її місце в процесі експлуатації робітника. 


Марксова демістифікація капіталізму, розкриття його механізму 
експлуатації мали не тільки наукове й політичне значення, але не в 
останню чергу й етичне, гуманістичне. Вони дали потужний поштовх 
робітничому революційному рухові спочатку в Європі, а потім і в усьому 
світі. Вони змінили світ. Зрештою змінили і самий капіталізм. І коли 
кажуть, що капіталізм давно вже не той, про який писав Маркс, то треба 
додати, шо став він «не тим» (хоч і не зовсім «не тим») завдяки зокрема й 
Марксу: капіталізмові нічого не залишалося, як змінитися під потужним 
тиском робітничого революційного руху, профспілкового руху, впливу на 
суспільства комуністичних і соціал-демократичних партій, — зрештою, 
і, мабуть, не в останню чергу, внаслідок власних внутрішніх 
суперечностей як джерела руху і завдяки невикористаним резервам, про 
можливість яких говорив Маркс (пригадаймо його тезу про те, що старий 
лад ніколи не сходить зі сцени, поки не вичерпає всіх своїх 
можливостей, певна річ, і здатності до змін).


Тут не буду говорити про те, як інтерпретували Маркса його 
послідовники (сам він якось саркастично сказав, що не хотів би бути 
марксистом), як розвивав марксизм В.І. Ленін і як на місці марксизму 
утворилося нове вчення — \emph{марксизм-ленінізм}. Це окрема велика тема. 
Але нагадаю про те, що в перше десятиліття радянської влади над 
вивченням Маркса й Енгельса трудилися спеціально створені солідні 
наукові інституції, які публікували свої праці, відбувалися дискусії 
тощо. В московському Інституті Маркса-Енгельса під керівництвом 
філософів-марксистів Д. Рязанова та І. Рубіна досліджували 
першоджерела, публікували невідомі твори. Тобто, в автентичному 
марксизмі бачили джерело ідей, що могли допомогти зрозуміти реальні 
суспільно-політичні процеси, орієнтуватися в будівництві нового 
суспільства. Ще жили такі ілюзії. Історична школа М. Покровського з 
марксистських позицій гостро викривала російський імперіалізм. В 30-і 
роки, коли Сталін утвердив свій спрощений (ще набагато спрощеніший, 
ніж ленінський) варіант марксизму, всі ці структури ліквідовано, 
провідні вчені, дослідники й популяризатори Маркса були репресовані 
то як меншовики, то як троцькісти, а єдиним законним речником 
марксизму зробився сам Сталін.


Не менш цікаве й те, що коїлося з Марксом-Енгельсом і з марксизмом 
після розвалу СРСР у нашій самостійній Україні. Їхні твори опинилися у 
спецфондах. Посилатися на них — моветон, ознака «совковості», 
відсталості мислення й антипатріотизму. Та про це далі. А спочатку про 
те, яке місце посідав Маркс у політичній свідомості видатних 
українців минулого, чи мав він якусь «причетність» до визвольної 
боротьби українців?


\subsection*{Від Франка до української діаспори}

Дивно було б припускати, що Маркс, який став «душею» всіх 
комуністичних і соціалістичних рухів, залишиться «чужим» для України, 
яка шукала вирішення своїх національних проблем, що були водночас і 
соціальними. Марксом поважно цікавилися М. Драгоманов, М. Павлик, І. 
Франко, Леся Українка. Іван Франко 1879 року зробив перший український 
переклад частини «Капіталу» (фрагмент друкується у цьому виданні). 
Професор Київського університету Микола Зібер, видатний економіст і 
соціолог, перший в Україні й Росії популяризував ідеї Маркса. Він 
зустрічався з Марксом і Енгельсом у Лондоні. 1885 року опублікував працю 
«Д. Рикардо и К. Маркс в их общественно-экономических исследованиях», 
яку Маркс читав і прихильно цитував. Учень М. Зібера Сергій 
Подолинський також зустрічався з Марксом і Енгельсом та листувався з 
ними; він був автором перших марксистських праць — популярних брошур 
— українською мовою, в яких застосовував Марксові ідеї до аналізу 
проблем українського селянства: «Про хліборобство» (1874), «Парова 
машина» (1875), «Про багатство та бідність» (1876), «Життя і здоров'я людей 
на Україні» (1879), «Ремесла і фабрики на Україні» (1880) та ін. Вони 
друкувалися, зрозуміло, в Галичині, але розповсюджувалися по всій 
Україні зусиллями Драгоманова, Павлика і київської «Громади». У 
Львові й Чернівцях 1892 року друкуються брошурами українські переклади 
з Маркса й Енгельса, а перший український переклад «Комуністичного 
маніфесту» виходить 1902 року у Львові. Цікавий етап у розповсюдженні 
марксистських ідей у Російській імперії — це розквіт т. зв. 
«легального марксизму», найяскравіше представленого у Києві: В. 
Кістяковський, С. Булгаков, М. Ратнер, М. Туган-Барановський (пізніше 
виступав з критикою Маркса). З «легальним марксизмом» уперто боровся 
Ленін, який бачив у ньому джерело ревізіонізму.


Якщо на перших порах популяризацією марксизму захоплювалися ліберали 
й народники, то з розвитком в Україні соціал-демократичного руху він 
стає елементом партійних програм. До марксизму апелювала створена 1905 
року на основі РУПу (Революційної Української Партії) — УСДРП 
(Українська Соціал-Демократична Робітнича Партія), визначними діячами 
якої були В. Винниченко, С. Петлюра, Д. Антонович, Л. Юркевич, М. 
Ковальський, М. Тимченко та ін. При цьому україноцентричні 
соціал-демократи звертаються до марксизму для висвітлення 
колоніального становища України та обстоювання ідеї національного й 
соціального визволення України. Видатним науковцем і політичним 
діячем цього гатунку був Микола Порш, один із чільних діячів РУПу та 
УСДРП, міністр в урядах УНР, соціолог і статистик, автор праць «Із 
статистики України» (1907), «Пролетаріат на Україні» (1907), «Про автономію 
України» (1907), «Автономія України і соціал-демократія» (1917), «Україна і 
Росія на робітничому ринку» (1918), «Україна в державному бюджеті Росії» 
(1918) та ін. Він же переклав українською мовою перший том «Капіталу» 
Маркса (не був виданий). Напередодні першої світової війни в Україні 
зростає мережа соціал-демократичної преси: «Дзвін» у Києві, «Воля», 
«Вперед», «Робітник», «Наш голос» — у Львові. Одним із організаторів і 
активних публіцистів у них був Володимир Левицький, автор книжок 
«Нарис розвитку українського робітничого руху в Галичині» (1914), 
«Царская Россия и украинский вопрос» (1919), «Соціалістичний 
інтернаціонал і поневолені народи» та ін. Українські марксисти 
зберігали європейське обличчя марксизму і відмежовувалися від 
марксизму ленінського. Про такий «лібералізований» марксизм можна 
говорити і стосовно Володимира Винниченка та інших лідерів УСДРП. 


Під час Світової війни українська соціал-демократична преса Галичини 
(в підросійській Україні вона була заборонена) не просто осуджувала 
варварське кровопролиття, а викривала загарбницький, 
імперіалістичний характер війни. Глибокий аналіз її причин з 
марксистських позицій дали В. Левицький та М. Залізняк. 


Тут слід нагадати, що в цей самий час значна частина європейських 
соціал-демократів, як відомо, розбіглася по «національних квартирах» 
і так чи інакше ставала по боці «своїх» урядів. 


Антивоєнна й правдиво інтернаціоналістична позиція українських 
соціал-демократів парадоксальним чином обернулася проти них у 
визвольну добу 1918--1919 рр., коли вони відігравали провідну роль у 
Центральній Раді та в урядах УНР. Як соціалісти і марксистські 
налаштовані політики, вони сподівалися на розуміння і мирну 
домовленість із соціалістами й марксистами «великого братнього» 
народу. Але виявилося, що то зовсім інакший «соціалізм» і зовсім 
інакший «марксизм». Ставлення до загрози російсько-більшовицької 
агресії в різних колах УСДРП було різне, і це спричинило внутрішню 
боротьбу й розколи, що також додалося до причин поразки. 


Ще у складнішому становищі опинилися українські соціал-демократи 
після приходу більшовиків в Україну. Вони не могли ігнорувати того 
факту, що більшовицькі гасла неабияк впливали на українські маси, а 
Радянська Росія немовби очолила світовий революційний і 
соціалістичний рух, за яким бачилося майбутнє. Невблаганний 
історичний процес диктував необхідність стратегії і тактики, 
відповідної до нових і непередбачуваних умов, здатної забезпечити 
можливість впливати на події і не бути відкинутими на задвірки 
історії. Тут неминучими стали нові незгоди й розколи. Частина 
вчорашніх соціал-революціонерів та соціал-демократів обирає 
співпрацю, на певних умовах, з більшовиками, сподіваючись таким чином 
впливати на характер перетворень і обстоювати українські національні 
інтереси, як вони їх уявляли. На перших порах ці сподівання почасти 
справджувалися, оскільки більшовики потребували підтримки досить 
сильної партії боротьбистів і йшли на деякі поступки. Але в міру 
зміцнення своєї влади вони дедалі більше утискували своїх 
ситуативних союзників. Тим часом і серед більшовиків, у тодішній 
КП(б)У, були, хоч і не переважальні, «націонал-ухильницькі» (на 
офіційному парткерівному жаргоні) сили, пов'язані зі своїм народом і 
відповідальні за його долю принаймні в тому розумінні, що хотіли 
бачити його рівноправним з іншими в уявлюваному комуністичному 
суспільстві, яке мало привести до вільного розвитку всіх народів. 
Зрештою, сума вагомих чинників — спротив українського села, опозиція 
національної інтелігенції, наявність різних ідеологічних елементів 
та різного бачення історичної перспективи у самій партії, слабкість 
її позицій в українському та інших національних суспільствах, — а не в 
останню чергу й претензія на роль маяка антиколоніальних революцій на 
Сході, для яких комуністична Україна мала стати переконливим і 
звабливим прикладом, — змусила більшовицьку партію, на виконання 
нового курсу Леніна, вдатися до політики «українізації», ширше —
«коренізації», оскільки йшлося й про інші колонізовані народи. Тобто, 
це був пошук надійного опертя в неросійських народах. В Україні ця 
політика пов'язувалася з лідерами націонал-комунізму, давніми 
партійними діячами Олександром Шумським та Миколою Скрипником, які 
прагнули дати їй марксистське обгрунтування. Відповідні дослідження 
проваджувано в Українському Інституті Марксизму та Ленінізму (УІМЛ, 
1922--1931). Професійна марксистська методологія з різною мірою успіху 
впроваджувалася в різних галузях суспільних наук. Серед яскравих 
представників цього штибу мислення можна назвати А. Річицького, 
одного із засновників УКП (Української Комуністичної Партії), 
сподвижника М.Скрипника, наукового працівника УІМЛ, автора праць на 
літературні й Марксівські теми, редактора першого видання «Капітала» 
Маркса українською мовою (1927--1929); філософа-марксиста, поета, 
публіциста і літературознавця В. Юринця; історика М. Яворського, школа 
якого працювала до погрому 30-х років. 


Доля УІМЛ була такою ж, як і московського Інституту Маркса-Енгельса, 
хіба що набагато трагічнішою, бо стала частиною тотальних репресій 
проти українських наукових і культурних установ та їхніх діячів — під 
моторошний акомпанемент голодомору. 


Зрозуміло, що після цього будь-які серйозні роботи в галузі марксизму 
стали неможливими і втратили сенс, черга реорганізацій закінчилася 
створенням Інституту історії партії при ЦК КПУ — як філіалу Інституту 
марксизму-ленінізму при ЦК КПРС. 


Марксистська фразеологія стала способом придушення самостійного 
мислення, і не дивно, що на час розвалу СРСР марксизм був у суспільстві 
остаточно скомпрометований, хоча долинали ще якісь відгомони 
європейського неомарксизму й були спроби створювати нелегальні 
робітничі гуртки з орієнтацією на «справжній марксизм» (про це ми 
могли довідатися з великим запізненням, у 90-і роки, — після того, як СБУ 
опублікувала секретні матеріали провокаційної кагебістської «Справи 
,,Блок``», по якій «проходили» не тільки «українські буржуазні 
націоналісти», а й широкий спектр інших «підривних елементів»). 


Прикметно, що з настанням горбачовської «гласності» та після здобуття 
Україною незалежності Маркса у нас зовсім не стало. Його уникали, мов 
якогось «совєтського» маркера, і офіціоз, і рухівська опозиція. Щодо 
офіціозу зрозуміло: йому з Марксом не було про що говорити, та й 
некомфортно. А \mbox{РУХ}ом він залишився непрочитаний. На мій погляд, велика 
помилка \mbox{РУХ}у й одна з причин його досить швидкого занепаду — 
абсолютизація національного питання й невміння розкрити всю 
конкретність його пов'язаності з соціальним. Це саме те, чого можна 
було повчитися у Маркса. Але Маркс вважався завербованим у офіційну 
совєтчину (пригадую: коли я в своєму самвидавському опусі 
«Інтернаціоналізм чи русифікація» рясно посилався на погляди Маркса 
й Енгельса, як і Леніна, з національного питання, на його листування, в 
якому фактично заперечується його власна, з «Комуністичного 
маніфесту», теза про те, що пролетаріат не має вітчизни, і говориться 
протилежне: щоб успішно вести свою боротьбу, пролетаріат повинен 
насамперед визволити чи об'єднати свою вітчизну, — багато хто навіть 
із прихильних до мене були подивовані, часом і неприємно, або 
сприймали це як курйоз чи риторичний прийом). Можна зрозуміти: Маркс 
такий далекий, а українське національне питання таке пекуче, що 
багатьом воно здавалося самодостатнім. Серед рухівців панувало 
стихійне переконання: національне — головне, соціальне — 
підпорядковане. Вся історія людства, м'яко кажучи, не підтверджувала 
цього, але гіркі розчарування варті того, щоб їх пережити самому. 
Вкотре наочно виявилося, що в дилемі національне-соціальне (до якої, 
власне, і не повинен допускати розумний політик!) національне стає 
пріоритетом для героїчної меншості, а соціальне — для решти. Героїчна 
меншість може творити революції, але парламенти обирає негеролїчна 
більшість. Тож український виборець у масі своїй голосував не за 
безкорисливих патріотів української мови (або й історії), а за хижих 
демагогів, які обіцяли швидке і фантастичне полагодження житейських 
проблем. І в захисники трудящих на політичній арені перевдягалися 
їхні найжорстокіші експлуататори, захребетники — як оті донецькі 
вугільно-металургійні барони, які невеличку частину прибутків, 
здертих з каторжної праці робітників, витрачали на фінансування 
організованих ними експедицій цих робітників під Верховну Раду чи 
Кабмін, для стукання шахтарськими касками, — так, ніби це українські 
урядовці, а не вони, донецькі барони, винні в несплаті заробітку і в 
жахливому занедбанні техніки безпеки та постійних катастрофах. 


Ні, не став РУХ реальним захисником трудового люду. Як не стали ним і 
профспілки, спосіб організації яких, структура і зміст роботи, права і 
можливості залишаються далекими від прозорості. Може, я помиляюся, але 
мені здається, що ні в кого немає ніякої концепції — ані марксистської, 
ані неомарксистської, ані просто немарксистської, ані навіть 
антимарксистської — захисту трудящих за умов нашого дикого 
капіталізму. Ані концепції, ані продуманих ідей, ані якоїсь — 
політичної чи моральної — гуманістичної настанови, Про це останнє 
кажу тому, що автентичний марксизм — насамперед гуманістичне вчення! 
Воно має глибоке коріння в історії людських борінь за справедливість, 
виразно перегукується з етикою шукань істини, пропонує свого роду 
соціологію пізнання. Як філософ і письменник (у широкому значенні 
слова), Маркс не чужий феноменології, в нього знаходять елементи 
екзистенціалізму. Може, найважливіше чи, принаймні, найцікавіше для 
гуманітаристики в «Капіталі» —це дослідження товарного фетишизму й 
відчуження праці, які фундаментальним чином діють у напрямку 
збіднення світу людини, її знелюднення. Це чинники універсальні, яким 
людство ще не знайшло противаги і не знати, чи колись знайде. Тут, може, 
найважливіші з Марсових відкриттів, і вони варті не меншої уваги, ніж 
його суто економічні осягнення. 


Ще\ldots{} Про Маркса часто говорили й писали, говорять і пишуть, що він 
нібито зневажав духовну творчість або ставився до неї догматично. Як 
на мене, це прикре непорозуміння. Маркс добре знав історію культури, 
його твори не бідні на апеляцію до її фактів та на глибокі думки про 
літературу, великих письменників минулого і сучасників. А прочитайте 
його листування, прилучіться до обсягу його естетичних переживань і 
читацьких реакцій! Отут іще один незнаний нам Маркс!


\ldots{}На закінчення варто додати, що в той час, як в Україні «набридлий» 
за радянські часи Маркс для науковців перестав бути актуальним (хоча б 
для цитування, про вивчення й мови немає), а політологам і публіцистам 
було не до Маркса (та й попиту ніякого), — «націоналісти», а власне 
інтелектуали в українській діаспорі про нього не забували. Крім тих, 
хто поважно студіював Маркса (Р. Роздольський, відомий як один із 
кращих знавців економічних і національних поглядів Маркса, в 
молодості один із організаторів КПЗУ, якийсь час співробітник, під 
керівництвом Д. Рязанова й І. Рубіна, московського Інституту 
Маркса-Енгельса, після його ліквідації працював у архівах Відня, 
Львова, Кракова, у 1942--1945 — в'язень німецьких концтаборів, від 1945 — у 
Детройті, автор багатьох досліджень про Маркса, зокрема й 
неомарксистського тлумачення «Капіталу» — от така дивовижна людина, 
варта біографічного роману або кіно- чи телесеріалу; Панас Феденко, 
один із організаторів УСДРП та лідер її в еміграції; історик і 
публіцист В. Голубничий; були ті, хто розумів його роль у розвитку 
суспільних наук та в політичній історії людства і знаходив йому 
належне місце в системі своїх ідеологічних оцінок постатей і 
феноменів сучасності, як-от Іван Лисяк-Рудницький. Можна говорити про 
певну пов'язаність з марксизмом Івана Багряного та інших ідеологів 
УРДП — Української Революційно-Демократичної Партії, яка за складних 
умов політичного розбрату в еміграції мужньо обстоювала ідею єдності 
українців на основі не «філологічного паріотизму», а спільності 
корінних інтересів соціальної справедливості й прагнення до свободи. 
Ідеї Івана Багряного могли б уберегти український політикум, 
насамперед рухівців та пізніших «правих», від прикрих помилок та 
неуваги до соціальної сторони української проблематики, — але, на 
жаль, вони не знайшли належного відгуку та й просто місця у вузькому 
кругозорі наших «патріотів» (не кажучи вже про байдужих до України). 


Окремо слід сказати про солідну працю видатного українського 
історика в США (власне, й американського історика) Романа Шпорлюка 
«Націоналізм і комунізм» (Оксфорд, 1988; український переклад Гр. 
Касьянова — К., «Основи», 1998). Праця має підзаголовок: «Карл Маркс проти 
Фрідріха Ліста», але фактично Марксова полеміка з німецьким 
економістом і теоретиком націоналізму Лістом — це лише сюжетний 
стрижень праці, який обростає великим фактичним матеріалом та 
інтелектуальними розважаннями автора й лектурою на тему взаємодії 
марксизму й націоналізму як двох великих проектів модернізації 
суспільств — проектів принципово суперечних один одному, але й 
суголосних багато в чому та навіть «повчальних» один до одного — аж 
такою мірою, що в процесі суспільного розвитку ХIХ--ХХ ст. марксизм 
помітно «націоналізувався», а націоналізм — почасти «омарксизмився».


Праця Романа Шпорлюка, на мій погляд, особливо важлива для українців 
тим, що виводить уявлення про націоналізм з провінційних вимірів у 
глобальні, знайомить нашого читача з інтерпретацією націоналізму 
широким колом сучасних європейських істориків, соціологів, філософів; 
те ж саме стосується і марксизму, якого українське суспільство — 
виглядає — так і не освоїло, хоч у ньому є ще немало інтелектуальних 
резервів для нас. Вони ждуть свого «будителя». І, може, якимось 
імпульсом стане сподівана публікація українського перекладу 
«Капіталу». 

\begin{flushright}
  \emph{Іван Дзюба}
\end{flushright}

{\small 22 липня 2018 р.}
