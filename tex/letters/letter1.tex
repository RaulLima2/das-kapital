\documentclass{kapital}
%% should be a class option
\renewcommand{\parbreak}{\unskip}
\renewcommand{\parcont}{\unskip}
\usepackage{pgffor}
\renewcommand{\labelitemi}{\textemdash{}}
\pagestyle{empty}

\def\names{
Дорога Ірино,
Дорогий Денисе,
Дорогий Антоне,
Дорогий Ернесте,
Дорога Анастасіє,
Дорогий Сергію,
Дорога Ірино,
Дорогий Мар‘яне,
Дорогий Павле,
Дорога Ірино,
Дорогий Леве,
Дорога Вікторіє,
Дорогий Тарасе,
Дорогий Миколо,
Дорога Софіє-Ольго,
Дорогий Андрію,
Дорога Юліє,
Дорогий Вікторе,
Дорога Юстино,
Дорогий Михайле,
Дорогий Василю,
Дорога Ніно,
Дорога Христино,
Дорога Вікторіє%
}

\begin{document}

\foreach \name in \names {
  \section*{\name{}!}
  \thispagestyle{empty}

  Дякуємо за те, що погодились взяти участь у тестовому читанні Марксового «Капіталу». Насолоджуйтесь! Але пам’ятайте, що в тексті ще залишилися помилки — ми надіємося на вашу уважність.

  \subsubsection*{Помилки та орфографія}

  Умовно на кожних 10 сторінках тексту залишилось близько 4 помилок. Розподілені вони нерівномірно: може бути 10 сторінок без жодної помилки, а за ними – сторінка-катастрофа з 12 помилками. Отож, будьте пильними.

  Наш спільний проект — повторне видання текстів 30-х років, тому орфографія
може здаватися незвичною. Але це — не помилки. Найчастіше вам будуть
зустрічатися:
  \begin{itemize}
  \item я замість а в запозичених словах (\emph{соціялний, клясичний, Голляндія})
  \item ґ замість г в запозичених словах (\emph{проґрес, реґулюють, континґент})
  \item архаїчні граматичні форми (працю \emph{вимірюється} безпосередньо часом, при \emph{машиновому} вибиванні, раптом \emph{порушиться} звичну рівновагу)
  \end{itemize}

  \noindent{}Якщо у вас є бажання звіритися з оригінальним виданням — скан книжки
знаходиться за посиланням \underline{bit.ly/marx\_book}.

  \subsubsection*{Зворотній зв'язок}

  Найшвидший шлях отримати відповідь на будь-які питання – написати нам
у фейсбук \underline{fb.me/marx.in.ua}. Відмічайте помилки, які ви помітили в тексті, та повідомляйте
нам:
 
  \begin{itemize}
  \item Сфотографувати або відсканувати сторінки з помилками. Відправте їх
нам на пошту marx.ukr@gmail.com (або завантажте через google photo).
  \item Відправити назад сторінки з помилками новою поштою. На ім’я Антон Потапов, 093~770~01~45, м.~Київ, Відділення~№58
  \end{itemize}

  \subsubsection*{Дякуємо!}

  Ми розпочали працювати над цим проектом 3 роки тому, з часом до нас доєдналося ще 80 волонтерів та волонтерок. Разом на читання, редагування та ін. вони виділили майже 1000 годин свого часу. Ми тішимось з того, скільки людей готові працювати спільно заради спільної мети – перевидання «Капіталу» Маркса. 

  \bigskip{}

  \noindent{}Волонтери «Капіталу»,

  \medskip{}

  \noindent{}Денис Потапов

  \noindent{}Ірина Зробок

  \noindent{}Антон Потапов
}

\end{document}

