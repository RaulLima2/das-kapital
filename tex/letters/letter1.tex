\documentclass{kapital}
%% should be a class option
\renewcommand{\parbreak}{\unskip}
\renewcommand{\parcont}{\unskip}
\renewcommand{\labelitemi}{\textemdash{}}
\begin{document}

\section*{Дорогий Василю!}

Ми вдячні вам за те, що погодились взяти участь у тестовому читанні Марксовго «Капіталу».
Насолоджуйтесь! Але пам'ятайте, що в тексті ще залишились помилки які ми хочемо знайти з вашою допомогою.

\subsubsection*{Помилки та орфографія}

На кожні 10 сторінок залишилось близько 4 помилки. Розподілені вони не рівномірно, може бути 10 сторінок без жодної помилки, а за ними сторінка-катастрофа з 12 помилками. Будьте пильні.

Наш спільний проект — повторне видання текстів 30-х років. І орфографія може здаватися незвичною. Але це — не помилки. Найчастіше вам будуть зустрічатися:
\begin{itemize}
\item я замість а в запозичених словах (соціялний, клясичний, Голляндія)
\item ґ замість г в запозичених словах (проґрес, еґулюють, континґент)
\item архаїчні граматичні форми ( працю вимірюється безпосередньо часом, при машиновому вибиванні, раптом порушиться звичну рівновагу)
\end{itemize}

\noindent{}Якщо у вас є бажання звіритися з оригінальним виданням — скан книжки знаходиться за посиланням додати сюди посилання.

\subsubsection*{Зворотній зв'язок}

Найшвидший шлях отримати відповідь на будь-які запитання написати нам у фейсбуці лінк. Відмічайте помилки, які ви помітили в тексті. Та повідомте нам:

\begin{itemize}
\item Сфотографувати або відсканувати сторінки з помилками. Відправте ім нам нам на пошту marx.ukr@gmail.com (Або завантажте через google photo).
\item Відправити назад сторінки з помилками новою поштою. На ім’я Антон Потапов, 0937700145, м. Київ, Відділення №58
\end{itemize}

\subsubsection*{Дякуємо!}

Ми почали цей проект 3 роки. До проекту вже долучилось 80 волонтерів та волонтерок. Разом вони витратили на проект майже 1000 годин. Тож Марксів «Капітал» показав, яка велика кількість інтелектуалів серед нас, які готові діяти спільно. 

\medskip{}

\noindent{}Волонтери Марксового «Капіталу»,

\noindent{}Денис Потапов

\noindent{}Ірина Зробок

\noindent{}Антон Потапов

\end{document}

