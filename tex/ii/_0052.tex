\parcont{}  %% абзац починається на попередній сторінці
\index{ii}{0052}  %% посилання на сторінку оригінального видання
дорівнює величина авансованого грошового капіталу 422 ф. стерл., хоч
500 ф. стерл., це не змінює нічого в тому, що він з’являється як проста
капітальна вартість. $Г'$ існує вже не як вирослий своєю вартістю або
запліднений додатковою вартістю капітал, не як капіталістичне відношення.
Воно ($Г'$) ще лише має збільшити свою вартість у процесі. Це саме має
силу щодо $П\dots{} П'$; $П'$ мусить і далі завжди функціонувати і поновлювати
кругобіг як $П$, як капітальна вартість, що повинна продукувати додаткову
вартість. Навпаки, кругобіг товарового капіталу починається не
капітальною вартістю, а капітальною вартістю, що вже виросла в товаровій
формі, а тому з самого початку містить у собі кругобіг не лише капітальної
вартости, яка є в товаровій формі, але також і додаткової вартости.
Тому, коли в цій формі відбувається проста репродукція, то в кінцевому
пункті виступає $Т'$ такої самої величини, як і в початковому пункті.
Коли в кругобіг капіталу входить частина додаткової вартости, то хоч
наприкінці й буде $Т''$ замість $Т'$, тобто більше за $Т'$, однак, наступний кругобіг
знову починається з $Т'$, яке тепер є лише більше $Т'$, ніж воно
було в попередньому кругобігу, і починає свій новий кругобіг з більшою
акумульованою капітальною вартістю, а тому й з відповідно більшою
новоутвореною додатковою вартістю. В усіх випадках $Т'$ завжди починає
кругобіг як товаровий капітал, що дорівнює капітальній вартості плюс
додаткова вартість.

$Т'$ виступає як $Т$ в кругобігу поодинокого промислового капіталу не
як форма цього капіталу, а як форма іншого промислового капіталу,
оскільки засоби продукції являють собою продукт цього останнього.
Акт $Г — Т$ (тобто $Г — Зп$) першого капіталу є для цього другого капіталу
$Т' — Г'$.

В акті циркуляції $Г — Т\splitfrac{Р}{Зп}$ $Р$ і $Зп$ відіграють тотожню ролю остільки,
оскільки вони є товари в руках своїх продавців, в одному випадку
робітників, що продають свою робочу силу, а в другому — власників
засобів продукції, що продають їх. Для покупця, що його гроші
функціонують тут як грошовий капітал, вони функціонують
лише як товари доти, доки він їх не купив, отже, доти, доки
вони, як товари, іншим належні, протистоять його капіталові, що
існує в грошовій формі. $Зп$ і $Р$ відрізняються тут лише остільки,
оскільки $Зп$ в руках свого продавця дорівнюють $Т'$, отже, можуть бути
капіталом, коли $Зп$ є товарова форма його капіталу, тимчасом як для
робітника $Р$ завжди є лише товар і стає капіталом лише в руках покупця,
як складова частина $П$.

Тому $Т'$ ніколи не може почати кругобіг як просте $Т$, як проста
товарова форма капітальної вартости. Як товаровий капітал воно завжди
має подвійне значення. З погляду споживної вартости воно є продукт
функції $П$ — в даному разі пряжа, елементи якої $Р$ і $Зп$, що як товари
походять із сфери циркуляції, функціонували лише як продуктотворчі
елементи цього продукту. Подруге, з погляду вартости, воно є продукт
\parbreak{}  %% абзац продовжується на наступній сторінці
