\index{ii}{0203}  %% посилання на сторінку оригінального видання
Другий період обороту, тижні 8--16, має в собі другий робочий
період, тижні 8--14. З них потреби 8-го й 9-го тижнів покривається
капіталом II. Наприкінці 9-го тижня повертаються давніші 700 ф. стерл.;
з них пускається в роботу до кінця робочого періоду (тижні 10--14)
500 ф. стерл. 200 ф. стерл. лишаються вільні для ближчого наступного
робочого періоду. Другий період обігу триває протягом 15-го й 16 тижнів;
наприкінці 16-го тижня знову повертаються назад 700 ф. стерл.
З цього моменту в кожному робочому періоді повторюється те саме
явище. Потреба в капіталі протягом перших двох тижнів покривається
за допомогою 200 ф. стерл., що звільнились наприкінці попереднього
робочого періоду; наприкінці 2-го тижня повертаються назад 700 ф.
стерл.; але робочий період налічує ще тільки 5 тижнів, так що на нього
можна авансувати лише 500 ф. стерл.; отже, 200 ф. стерл. завжди лишаються
вільні для наступного робочого періоду.

Отже, виявляється, що в нашому випадку, де ми припускали, що робочий
період більший, ніж період обігу, наприкінці кожного робочого
періоду при всяких обставинах є звільнений грошовий капітал, такої
саме величини, як капітал II, авансований на період циркуляції. В наших
трьох прикладах капітал II дорівнював: в першому — 300 ф. стерл., в
другому — 400 ф. стерл., в третьому — 200 ф. стерл.; відповідно до
цього капітал, що звільнявся наприкінці кожного робочого періоду, був
послідовно 300, 400, 200 ф. стерл.

\subsection{Робочий період менший від часу обігу}

Спочатку ми знову припустимо період обороту в 9 тижнів: з них
З тижні становлять робочий період, що для нього є в розпорядженні
капітал  І = 300 ф. стерл. Період обігу хай буде 6 тижнів. Для цих
6 тижнів потрібен додатковий капітал в 600 ф. стерл., який ми знову
можемо розподілити на два капітали по 300 ф. стерл., що з них кожен
заповнює один робочий період. Тоді ми маємо три капітали по 300 ф.
стерл., з них 300 ф. стерл. завжди зайнято в продукції, тимчасом як
600 ф. стерл. циркулюють.

\begin{table}[h]
  \begin{center}
  \caption*{Таблиця III.}

  \caption*{Капітал І.}
  \begin{tabular}{r@{\hspace{1}} c@{\hspace{1}} r@{\textendash{}} l c@{\hspace{1}} r@{\textendash{}} l c@{\hspace{1}} r@{\textendash{}} l}
  \toprule
  \multicolumn{4}{c}{Періоди обороту} & \multicolumn{3}{c}{Робочі періоди} & \multicolumn{3}{c}{Періоди обігу}\\
  \cmidrule(r){1-4}
  \cmidrule(r){5-7}
  \cmidrule{8-10}
  І.  & Тижні         & 1 & 9   & Тижні         & 1 & 3   & Тижні & 4 & 9\\
  ІІ. & \ditto{Тижні} & 10 & 18 & \ditto{Тижні} & 10 & 12 & \ditto{Тижні} & 13 & 18\\
  III.& \ditto{Тижні} & 19 & 27 & \ditto{Тижні} & 19 & 21 & \ditto{Тижні} & 22 & 27\\
  IV. & \ditto{Тижні} & 28 & 36 & \ditto{Тижні} & 28 & 30 & \ditto{Тижні} & 31 & 36\\
  V.  & \ditto{Тижні} & 37 & 45 & \ditto{Тижні} & 37 & 39 & \ditto{Тижні} & 40 & 45\\
  VI. & \ditto{Тижні} & 46 & [54] & \ditto{Тижні}& 46 & 48 & \ditto{Тижні} & 49 & [54]\\
  \end{tabular}
\end{center}
\end{table}
