\parcont{}  %% абзац починається на попередній сторінці
\index{ii}{0097}  %% посилання на сторінку оригінального видання
\subsubsection{Власне товаровий запас}

Ми вже бачили, що на основі капіталістичної продукції товар стає
загальною формою продукту і стає нею то більше, що більше ця продукція
розвивається розмірами і глибиною. Отже, навіть при однакових
розмірах продукції, значно більша частина продукту існує як товар, порівнючи
чи то до попередніх способів продукції, чи то до капіталістичного
способу продукції на менш розвиненому щаблі. Але кожен товар,
— отже, також і кожен товаровий капітал, що є лише товар, але товар
як форма буття капітальної вартости, — оскільки він переходить із
сфери своєї продукції не безпосередньо в сферу продуктивного або
особистого споживання, отже, оскільки він певний переміжний час
перебуває на ринку, він становить елемент товарового запасу. Тому,
при незмінних розмірах продукції, з розвитком капіталістичної продукції
зростає й товаровий запас сам по собі (тобто це усамостійнення й фіксування
товарової форми продукту). Ми вже бачили, що це є лише зміна
форми запасу, тобто, що на одному боці більшає запас у товаровій
формі тому, що на другому боці в формі безпосереднього запасу для
продукції або споживання він меншає. Це є лише змінена суспільна
форма запасу. А коли разом з тим більшає не лише відносна величина
товарового запасу порівняно з сукупним суспільним продуктом, але також
і його абсолютна величина, то це тому, що з розвитком капіталістичної
продукції зростає й маса сукупного продукту.

З розвитком капіталістичної продукції маштаб продукції дедалі менше
визначається безпосереднім попитом на продукт і дедалі більше визначається
розмірами капіталу, що ним порядкує індивідуальний капіталіст,
прагненням його капіталу зростати своєю вартістю і доконечністю безперервности
й поширення його продукційного процесу. Разом з тим у
кожній окремій галузі продукції неодмінно зростає маса продукту, що
перебуває на ринку як товар або шукає збуту. Зростає маса капіталу, що
на більш або менш довгий час фіксована у формі товарового капіталу.
Тому зростає товаровий запас.

Нарешті, більша частина суспільства перетворюється на найманих
робітників, на людей, що живуть з дня на день, одержують свою заробітну
плату щотижня і щоденно витрачають її — отже, на людей, що
мусять знаходити свої засоби існування як запас. Хоч як можуть рухатися
поодинокі елементи цього запасу, все ж деяка частина їх мусить
завжди бути нерухома, щоб увесь запас завжди був рухомий.

Усі ці моменти випливають із форми продукції і властивого їй
перетворення форми, що його мусить проробити продукт у процесі
циркуляції.

Хоч яка буде суспільна форма запасу продуктів, на його зберігання потрібні
витрати: будівлі, вмістища тощо, які становлять сховища на продукти;
треба також, залежно від природи продукту, більше або менше засобів продукції
та праці, які мусять витрачатися, щоб запобігти шкідливим
впливам. Що вища суспільна концентрація запасів, то відносно менші
\parbreak{}  %% абзац продовжується на наступній сторінці
