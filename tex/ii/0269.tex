складові частини капіталу (основний і обіговий) пророблюють кругобіг форм
в різні періоди часу й різним способом; з другого боку, ми дослідили обставини,
що ними зумовлюється різний протяг робочого періоду й періоду
циркуляції. Ми показали, як впливає період кругобігу й різне відношення
його складових частин на розмір самого продукційного процесу і на
річну норму додаткової вартости. В дійсності, коли в першому відділі
розглядалось переважно послідовні форми, що їх у своєму кругобігу капітал
постійно набирає й скидає, то в другому відділі ми розглянули, як
у межах цього руху й послідовности форм капітал даної величини одночасно,
хоч і в змінному розмірі, поділяється на різні форми — на продуктивний
капітал, грошовий капітал і товаровий капітал, так, що ці
форми не лише чергуються одна з однією, але різні частини сукупної
капітальної вартости постійно одна поряд однієї перебувають і функціонують
у цих різних станах. Саме грошовий капітал при цьому виявив
особливість, яка не виявлялась в книзі першій. Ми виявили ті певні закони,
що згідно з ними різні величиною складові частини даного капіталу,
відповідно до умов обороту, постійно мусять авансуватись і відновлюватись
у формі грошового капіталу для того, щоб підтримувати
продуктивний капітал даного розміру в безперервному функціонуванні.

Але і в першому і в другому відділі мова була завжди тільки про індивідуальний
капітал, про рух усамостійненої частини суспільного капіталу.

Але кругобіги індивідуальних капіталів переплітаються один з одним,
являють передумову і зумовлюють один одного і саме в цьому сплетінні
й становлять рух сукупного суспільного капіталу. Як при простій товаровій
циркуляції уся метаморфоза одного товару виступала як ланка
ряду метаморфоз товарового світу, так тепер метаморфоза індивідуального
капіталу виступає як ланка ряду метаморфоз суспільного капіталу. Але
коли проста циркуляція товарів зовсім не включає неодмінно циркуляції
капіталу, — бо товарова циркуляція може відбуватись на основі некапіталістичної
продукції, — то кругобіг сукупного суспільного капіталу включає,
як уже зазначено, і товарову циркуляцію, що не входить в кругобіг
індивідуального капіталу, тобто включає циркуляцію товарів, які не є
капітал.

Тепер ми повинні розглянути процес циркуляції (а він у своїй сукупності
є форма процесу репродукції) індивідуальних капіталів, як складових
частин сукупного суспільного капіталу, отже, розглянути процес
циркуляції цього суспільного сукупного капіталу.

II. Роля грошового капіталу

[Хоч дальший виклад належить до пізнішої частини цього відділу,
все ж ми зараз дослідимо це, тобто грошовий капітал, розглядуваний
як складова частина суспільного сукупного капіталу].

При розгляді обороту індивідуального капіталу грошовий капітал виявив
себе з двох боків.
