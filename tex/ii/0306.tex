кляса капіталістів II знову перетворила свій сталий капітал, рівний 2000,
з форми засобів споживання на форму засобів продукції засобів споживання,
на форму, що в ній він може знову функціонувати як чинник
процесу праці і як стала капітальна вартість для зростання вартости. З
другого боку, в наслідок цього еквівалент робочої сили в І (1000 Iv)
і додаткова вартість капіталістів I (1000 Іm) реалізувались у засобах
споживання; і те й друге з своєї натуральної форми засобів продукції перетворилися
на таку натуральну форму, що в ній їх можна спожити як дохід.

Але таке взаємне перетворення здійснюється за допомогою грошової
циркуляції, що так само упосереднює його, як і утруднює його розуміння;
однак, вона відіграє вирішально важливу ролю, бо змінна частина капіталу
знову й знов мусить виступати в грошовій формі, як грошовий капітал,
що з грошової форми перетворюється на робочу силу. В усіх галузях
підприємств, що одночасно працюють одне біля одного на периферії
суспільства, все одно, чи належать вони до категорії I чи до II, змінний
капітал доводиться авансувати в грошовій формі. Капіталіст купує
робочу силу раніш ніж вона входить у процес продукції, але оплачує її
лише в строки, визначені умовою, після того, як її вже витрачено на
продукцію споживної вартости. Так само, як і інші частини вартости
продукту, йому належить і та частина її, що є лише еквівалент грошей
витрачених на оплату робочої сили, тобто та частина вартости продукту,
що репрезентує змінну капітальну вартість. Цією частиною вартости продукту
робітник уже дав капіталістові еквівалент своєї заробітної плати. Але
лише зворотне перетворення товару на гроші, продаж товару, відновлює
капіталістові його змінний капітал як грошовий капітал, що його він
знову може авансувати на закуп робочої сили.

Отже, в підрозділі I капіталіст, розглядуваний як збірний капіталіст,
виплатив робітникам 1000 ф. стерл. (я кажу ф. стерл. лише для того,
щоб зазначити, що це — вартість у грошовій формі) = 1000v за
ту вартість, яка вже існує як частина v вартости продукту I, тобто спродукованих
робітниками засобів продукції. На ці 1000 ф. стерл. робітники
купують у капіталістів II засобів споживання на таку саму вартість і таким
чином перетворюють половину сталого капіталу II на гроші; капіталісти II
із свого боку купують на ці 1000 ф. стерл. засоби продукції вартістю
на 1000 у капіталістів I; тим самим змінну капітальну вартість останніх
= 1000 v, що існує як частина їхнього продукту в натуральній формі
засобів продукції, знову перетворено на гроші, і тепер в руках капіталістів
І знову може вона функціонувати як грошовий капітал, що перетворюється
на робочу силу, отже, на найпосутніший елемент продуктивного
капіталу. Таким чином, в наслідок реалізації частини їхнього товарового
капіталу, до них зворотно припливає їхній змінний капітал у
грошовій формі.

Щодо грошей, потрібних для обміну частини m товарового капіталу
І на другу половину сталої частини капіталу II, то їх можна авансувати
різними способами. На ділі ця циркуляція охоплює незчисленну силу поодиноких
актів купівлі й продажу, що їх переводять індивідуальні капі-
