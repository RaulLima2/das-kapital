І. Перша стадія. Г — Т2)

Г — Т являє перетворення певної грошової суми на певну суму товарів; для покупця — перетворення його
грошей на товар, для продавців — перетворення їхніх товарів на гроші. Що перетворює цей акт
загальної товарової циркуляції разом з тим на функціонально визначений відділ в самостійному
кругобігу індивідуального капіталу, це — насамперед не форма цього акту, а його речовий зміст,
специфічний характер
споживання тих товарів, що обмінюються місцем з грішми. Це, з одного боку, засоби продукції, а з
другого — робоча сила, речові та особові чинники товарової продукції, що їхній особливий рід,
звичайно, мусить відповідати тому ґатункові предметів, що мають виробляти. Коли ми позначимо робочу
силу Р, засоби продукції Зп, то сума товарів, що їх мають купити є: Т = Р + Зп, або коротше Т <  P
Зп. Отже, розглядуваний щодо свого змісту акт Г — Т являє собою Г — Т< Р Зп; тобто Г — Т
розпадається на Г — Р і Г — Зп; грошова сума Г розпадається на дві частини, що з них одна купує
робочу силу, а друга засоби продукції. Ці два ряди купівель належать до цілком різних ринків: один
до власне товарового ринку, а другий — до ринку праці.

Але, крім такого якісного розщеплення тієї суми товарів, що на неї перетворюється Г, акт Г — Т  < Р
Зп являє собою ще надзвичайно характеристичне кількісне відношення.

Ми знаємо, що вартість, зглядно ціна робочої сили, власникові її, що продає її як товар, сплачується
в формі заробітної плати, тобто як ціну певної кількости праці, що має в собі додаткову працю; так
що, коли, напр., денна вартість робочої сили дорівнює 3 маркам, продуктові п’ятигодинної праці, то в
угоді між покупцем і продавцем ця сума фігурує як ціна або плата, напр., за десятигодинну працю.
Коли таку угоду складено, напр., з 50 робітниками, то вони повинні загалом дати покупцеві протягом
одного дня 500 робочих годин, що з них половина, 250 робочих годин, тобто 25 десятигодинних робочих
днів, є чиста додаткова праця. Кількість, а також і об’єм тих засобів продукції,
що їх треба купити, мусять бути достатні, щоб можна було вжити цю кількість праці.

Отже, Г — Т < Р Зп виражає не лише якісне відношення, не лише те, що певна сума грошей, напр., 422
фунти стерл., перетворюється на відповідні одне одному засоби продукції та робочу силу, але також
ви-

2) Відси рукопис VII, розпочатий 2 липня 1878.
