одного боку, робочу силу, з другого — засоби праці. І справа тут стоїть
так само, як з часом на купівлю й продаж.

Як єдність у своїх кругобігах, як вартість, що процесує, хоч перебуває
вона в сфері продукції, хоч в обох фазах сфери циркуляції, капітал
існує лише ідеально в формі розрахункових грошей, насамперед у голові
товаропродуцента, зглядно капіталістичного товаропродуцента. Бухгальтерією,
куди належить також визначення цін або обчислення товарових
цін (калькуляція цін), рух цей фіксується й контролюється. Таким
чином рух продукції, а особливо зростання вартости — при чому товари
фігурують лише як носії вартости, як назви речей, що їхнє ідеальне
вартісне буття фіксується в рахункових грошах — набуває символічного
образу в уявленні. Доки поодинокий товаропродуцент веде свою книгу
тільки в своїй голові (як, напр., селянин; лише капіталістичне хліборобство
породжує фармера, що веде рахункову книгу) або веде книгу своїх
видатків і прибутків, термінів виплат тощо лише між іншим, у вільний від
роботи час, доти очевидно, що ця його функція та засоби праці, на це
зужитковані, прим., папір тощо, являють додаткову витрату робочого часу
й засобів праці, хоч і доконечних, але все ж таких, що становлять одбаву
так з часу, що його він може спожити продуктивно, як і з засобів праці, що
функціонують у справжньому процесі продукції і беруть участь у творенні
продукту й вартости\footnote{
За середніх віків книги хліборобські вели тільки в манастирях. Однак ми
бачили (кн. І, розд. XII, 4), що вже в староіндійських громадах був рахівничий
у хліборобстві. Бухгальтерію тут усамостійнено у виключну функцію громадського урядовця. В наслідок
цього розподілу праці зберігається час, роботу та витрати, але все ж продукція і бухгальтерія щодо
продукції лишаються такими самими різними речами, як навантаження пароплавів і писання квитків на
вантаж. В особі бухгальтера частину робочої сили громади відібрано від продукції, і витрати його
функціонування покривались не його власною працею, а одбавою з громадськоґо
продукту. З бухгальтером у капіталіста справа стоїть mutatis mutandis\footnote*{
Змінивши те, що треба змінити, або з відповідними змінами. Ред.
} так само, як з бухгальтером
індійської громади. (З рукопису II).
}. Природа самої функції не змінюється ні в наслідок
того розміру, що його вона набуває, концентруючися в руках капіталістичного
товаропродуцента і роблячись не функцією багатьох дрібних товаропродуцентів,
а функцією одного капіталіста, функцією в процесі продукції
широкого маштабу; ні в наслідок того, що її відокремлено від продуктивних
функцій, що до них вона становила додаток, ні в наслідок її усамостійнення
як функції особливих аґентів, що їм виключно доручається її.

Поділ праці, усамостійнення якоїбудь функції, не робить ще з неї
функції, яка утворює продукт і вартість, якщо вона не була такою сама
собою, отже, ще до свого усамостійнення. Коли капіталіст вкладає
вперше свій капітал, він мусить вкласти частину капіталу на те, щоб
найняти бухгальтера тощо і купити засоби для ведення книг. Коли його
капітал уже функціонує, перебуваючи в постійному процесі репродукції,
то капіталіст мусить частину товарового продукту, за допомогою перетворення
на гроші, повсякчас зворотно перетворювати на бухгальтера,
конторників тощо. Цю частину капіталу відтягується від процесу про-