\parcont{}  %% абзац починається на попередній сторінці
\index{ii}{0339}  %% посилання на сторінку оригінального видання
відрізняється від дикуна не тим, що, — як гадає Сеніор\footnote{
„Коли дикун робить лук, то він займається промисловістю, але не практикує
поздержливости“ (Senior: „Principes fondamentaux de l’Economie politique“,
trad. Arrivabene, Paris) 1836 і p. 342). „Що більше розвивається суспільство, то
більшої поздержливости потребує воно“ (там само, стор. 342.) Пор. „Капітал“,
т. І. розділ XXIII, 3.
}, — дикунові
належить привілей і властивість витрачати свою працю протягом певного
часу так, що вона йому не дає продуктів, які можна розкласти (обміняти)
на дохід, тобто на засоби споживання. Ріжниця ця ось у чому:

а) Капіталістичне суспільство більшу частину праці, що нею воно
порядкує, застосовує на продукцію засобів продукції (ergo, сталого
капіталу), що їх не можна розкласти на доходи, ні в формі заробітної
плати, ні в формі додаткової вартости, а можуть вони функціонувати
тільки як капітал.

б) Коли дикун робить лук, стріли, кам’яний молот, сокиру, кіш тощо,
то він добре знає, що зужитий таким чином час він застосовує не на
виготовлення засобів споживання, що він таким чином покриває свою
потребу в засобах продукції й нічого більше. Крім того, дикун допускається
тяжкого економічного гріха через свою повну байдужість до того,
скільки він часу витрачає, а іноді, як розповідає Тайлор, він витрачає,
напр., цілий місяць, щоб виготовити одну стрілу\footnote{
Е. В. Tyler: „Forschungen über die Urgeschichte der Menschheit, übersetzt
von H. Müller, Leipzig, стор. 240.
}.

Поширене пласке уявлення, що за допомогою його частина політикоекономів
намагається позбутись теоретичних труднощів, тобто розуміння
справжнього зв’язку, — уявлення, ніби те, що для одного є капітал, для
другого є дохід, і навпаки, — почасти правильне, але стає цілком хибним
(отже, містить у собі цілковите нерозуміння цілого процесу перетворень,
що відбувається підчас річної репродукції, отже, і нерозуміння фактичної
основи того, що є почасти правильне в цьому уявленні), скоро йому
надають загального значення.

Тепер ми даємо зіставлення дійсних відношень, що на них ґрунтується
часткова правильність цього уявлення; при цьому одразу ж виявиться
помилковість у розумінні цих відношень.

1) Змінний капітал функціонує як капітал у руках капіталіста і функціонує
як дохід у руках найманого робітника.

Змінний капітал існує спочатку в руках капіталіста як грошовий
капітал; він функціонує як грошовий капітал, коли капіталіст
купує на нього робочу силу. Поки він лишається в його руках у грошовій
формі, він є не що інше, як дана вартість, що існує в грошовій
формі, отже, він є стала, а зовсім не змінна величина. Це — лише потенціяльно
змінний капітал, саме в наслідок того, що його можна перетворити
на робочу силу. Справжнім змінним капіталом він стає, лише скинувши
свою грошову форму, після того, як перетворено його на робочу
силу, а вона почне функціонувати в капіталістичному процесі як складова
частина продуктивного капіталу.
