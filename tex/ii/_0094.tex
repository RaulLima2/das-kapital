\parcont{}  %% абзац починається на попередній сторінці
\index{ii}{0094}  %% посилання на сторінку оригінального видання
може зберегтись лише через зберігання продукту, через зберігання самої
споживної вартости. Споживна вартість тут не підвищується і не збільшується,
навпаки, вона зменшується. Але її зменшення обмежується,
вона зберігається. Авансована вартість, що існує в товарі, тут теж не
і підвищується. Але нова праця, зречевлена й жива, долучається до неї.

Далі треба тепер розглянути, наскільки ці затрати (Unkosten)
походять із своєрідного характеру товарової продукції взагалі і товарової
продукції вії загальній, абсолютній формі, тобто капіталістичної продукції;
наскільки, з другого боку, вони спільні всякій суспільній продукції і
лише тут у межах капіталістичної продукції набирають особливого вигляду,
особливої форми виявлення.

А. Сміс висловив той казковий погляд, що утворення запасу є явище,
властиве капіталістичній продукції\footnote{
Wealth of Nation, Book II, Introduction.
}. Новітні економісти, напр. Ляльор,
навпаки, твердять, що з розвитком капіталістичної продукції утворення запасу
меншає. Сісмонді навіть вбачає в цьому тіньову сторону цієї продукції.

В дійсності запас існує в трьох формах: у формі продуктивного
капіталу, в формі фонду особистого споживання і в формі товарового
запасу або товарового капіталу. Коли збільшується запас в одній формі,
то запас в іншій формі відносно зменшується, хоч своєю абсолютною
величиною він може одночасно зростати в усіх трьох формах.

Само собою зрозуміло, що там, де продукцію безпосередньо скеровано
на задоволення власної потреби, і де лише в незначній частині
продукується на обмін або продаж, отже, де суспільний продукт зовсім
не набирає форми товару або лише в меншій частині набирає такої форми,
там запас у формі товару або товаровий запас становить лише незначну
й непомітну частину багатства. Щодо фонду споживання, то він
тут відносно великий, особливо фонд власне засобів існування. Досить
шіянути лише на старовинне селянське господарство. Переважна частина продукту
перетворюється тут безпосередньо, не утворюючи товарового запасу
— саме тому, що вона лишається в руках свого власника — на запасні
засоби продукції або запасні засоби існування. Вона не набирає форми
товарового запасу, і саме тому в суспільствах, що ґрунтуються на такому
способі продукції, згідно з А. Смісом, немає жодного запасу. А. Сміс
сплутує форму запасу із самим запасом і вважає, що суспільство до
цього часу жило з дня на день або покладалося на випадковості прийдешнього
дня\footnote{
Утворення запасу не є наслідок перетворення продукту на товар і споживного запасу на товаровий
запас, як це уявляє собі Д. Сміс, — навпаки,
ця зміна форми зумовлює найгостріші кризи в господарстві продуцентів підчас
переходу від продукції для власного споживання до товарової продукції. В Індії,
напр., зберігся до найосганнішого часу „звичай складати в амбари багато хліба, що
зі нього в урожайні роки небагато можна було б одержати“. (Return. Bengal and
Orissa Famine. H. of C. 1867. I, p. 230, № 74). Коли в наслідок американської громадянської
війни підвищився раптом попит на бавовну, джут тощо, то в багатьох
частинах Індії це спричинило велике скорочення культури рижу, підвищення цін
на риж і продаж старих запасів рижу, що був у продуцентів. До цього долучився
}. Та це — дитяче непорозуміння.
