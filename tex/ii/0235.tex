Ріжниця випливає з неоднаковости періодів обороту, тобто тих періодів,
що в них вартість, яка замішує змінний капітал, застосований протягом
певного часу, знову може функціонувати як капітал, отже, як новий
капітал. У В, як і в А, однаково заміщується вартість змінного капіталу,
застосованого протягом однакових періодів. Так само протягом однакових
періодів відбувається однаковий приріст додаткової вартости. Але хоч у
В й заміщується що п’ять тижнів вартість в 500 ф. стерл., та ще й наростає
500 ф. стерл. додаткової вартости, однак ця вартість, що являє собою
заміщення V, не є ще новий капітал, бо вона перебуває не в грошовій
формі. У А не лише стару капітальну вартість заміщується новою, а її
відновлюється в її грошовій формі, отже, її заміщується як новий, здібний
функціонувати капітал.

Чи раніше, чи пізніше відбувається перетворення вартости, що являє
собою заміщення, на гроші, а тому на форму, що в ній авансується
змінний капітал, — це, очевидно, цілком байдужа обставина для самої
продукції додаткової вартости. Ця продукція залежить від величини застосованого
змінного капіталу й від ступеня експлуатації праці. Але обставина
ця модифікує величину того грошового капіталу, що його треба
авансувати, щоб протягом року пустити в рух певну кількість робочої
сили, а тому вона визначає річну норму додаткової вартости.

III. Оборот змінного капіталу, розглядуваного з суспільного погляду

Погляньмо на хвилинку на справу з суспільного погляду. Припустімо,
що один робітник коштує на тиждень 1 ф. стерл., а робочий день = 10 годинам.
У А, як і у В, протягом року працюють 100 робітників (100 ф.
стерл. на тиждень на 100 робітників становлять за 5 тижнів 500 ф. стерл., а
за 50 тижнів — 5000 ф. стерл.); припустімо, що вони працюють 6 днів на
тиждень, по 60 робочих годин кожен. Отже, 100 робітників працюватимуть
протягом тижня 6000 робочих годин, а протягом 50 тижнів, 300000 робочих
годин. І А, і В захопили цю робочу силу, отже, суспільство не може витрачати
її на щось інше. Щодо цього, то з суспільного погляду справа
така сама в А, як і у В. Далі у А, як і у В, кожні 100 робітників одержують
на рік 5000 ф. ст. заробітної плати (отже, всі 200 робітників
одержують разом 10000 ф. стерл.) і беруть у суспільства засобів існування
на цю суму. І щодо цього справа з суспільного погляду така сама в
А, як і у В. Що робітники в обох випадках одержують заробітну плату
щотижня, то щотижня вони беруть у суспільства й засоби існування, за
які вони в обох випадках щотижня пускають у циркуляцію грошовий
еквівалент. Але відси починається ріжниця.

Поперше. Гроші, що їх пускає в циркуляцію робітник А, є не
тільки грошова форма вартости його робочої сили, як для робітника В
(у дійсності — засіб виплати за вже виконану роботу); починаючи з другого
періоду обороту, рахуючи з відкриття підприємства, вони вже є
