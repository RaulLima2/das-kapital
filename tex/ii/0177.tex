часом і часом продукції, то й час зуживання вкладеного основного капіталу раз-у-раз переривається на
більш-менш протяжні періоди, як, напр., у хліборобстві при вживанні робочої худоби, знарядь праці та
машин. Оскільки цей основний капітал складається з робочої худоби, він потребує завжди однакових або
майже однакових витрат на корм і т. ін., все одно, чи в роботі вона, чи не в роботі. Щодо мертвих
засобів праці, то коли їх не вживається, вони теж дещо зневартнюються. Тому продукт взагалі
дорожчає, бо передачу вартости на продукт обчислюється не на той час, коли основний капітал
функціонує, але на той час, коли він втрачає вартість. В цих галузях продукції бездіяльність
основного капіталу, хоч сполучена вона з поточними витратами, хоч ні, становить так само умову
нормального його вжитку, як, наприклад, втрата певної кількосте бавовни в процесі прядіння; так само
в кожному процесі праці непродуктивна — але неминуча — витрата робочої сили, що відбувається в
нормальних технічних умовах, береться на увагу так само, як і продуктивна. Кожне поліпшення, що
зменшує непродуктивну витрату засобів праці, сировинного матеріялу та робочої сили, зменшує також і
вартість продукту.

В сільському господарстві поєднуються й порівняно довгий робочий період і велика ріжниця між робочим
часом і часом продукції. Годскін слушно зауважує про це: „Ріжниця в часі [хоч він тут і не відрізняє
робочого часу й часу продукції], потрібному на те, щоб виготовити продукти в сільському
господарстві, і тим часом, що потрібен в інших галузях праці, є головна причина великої залежности
сільських господарств. Вони не можуть подавати свої товари на ринок раніше, ніж через рік. Протягом
цілого цього часу вони мусять боргуватись у шевця, кравця, коваля, колісника та різних інших
продуцентів, що їхніх продуктів вони потребують, і що їхні продукти можна виготувати протягом
небагатьох днів або тижнів. В наслідок цієї природної обставини і в наслідок швидкого збільшення
багатства в інших галузях праці, землевласники, що монополізували землю цілої держави, хоч вони,
крім цього, захопили й монополію законодавства, все ж таки не можуть врятувати себе й своїх слуг
фармерів від долі найбільш залежних людей в країні“. (Thomas Hodgskin, Popular Political Economy,
London, 1827, p. 147, примітка).

Всі методи, що ними в хліборобстві почасти рівномірніше розподіляється на цілий рік витрати на
заробітну плату й засоби праці, почасти скорочується оборот у наслідок культивування різноманітних
продуктів, яке уможливлює кілька зборів урожаю на рік, — всі ці методи потребують збільшення
авансовуваного обігового капіталу, витрачуваного на заробітну плату, добриво, насіння тощо. Так
буває, коли переходять від трипільного господарства з паром до сівозмінного без пару. Так буває у
Фляндрії при cultures dérobées\footnote*{
Culture dérobée — дослівно: „потайна культура“. Так зветься культура корінняків, що їх засівають
після збору основної культури; назва походить з того, що така культура, потребуючи менше часу,
вистигає між двома основними культурами, ніби потай. Ред.
}“. „В culture dérobée застосовуютькорінняки; те саме поле спочатку
дає збіжжя, льон, рапс на задоволення потреб лю-