\parcont{}  %% абзац починається на попередній сторінці
\index{ii}{0022}  %% посилання на сторінку оригінального видання
формі відкрив процес, був авансований у грошовій формі. Як ми бачили,
зворотнє перетворення на грошову форму є функція товарового капіталу
$Т'$, а не грошового капіталу. А щодо різности між $Г'$ і Г, то вона є (г)
лише грошова форма т, приросту Т; $Г'$ лише тому = $Г + г$, що $Т'$
дорівнювало $Т + т$. Отже, ця різність і відношення капітальної вартости
до породженої нею додаткової вартости існують і виражаються в $Т'$ раніше,
ніж обидві ці вартості перетворились на $Г'$, на грошову суму, що в
ній обидві частини вартости самостійно протистоять одна одній і тому
можуть застосовуватись на самостійні, між собою різні, функції.

$Г'$ є лише результат реалізації $Т'$. Обидва вони, і $Т'$ і $Г'$, є лише різні
форми, товарова форма й грошова форма, вирослої капітальної вартости,
обидва мають те спільне, що вони є виросла капітальна вартість. Обидва
вони є здійснений капітал, бо тут капітальна вартість як така існує разом
з додатковою вартістю як відмінним від першої і через неї одержаним
витвором, хоч це відношення й виражається лише в іраціональній
формі відношення двох частин тієї самої грошової суми або тієї самої
товарової вартости. Але як вирази капіталу в його відношенні до
додаткової вартости і в його відмінності від додаткової вартости, ним
утвореної, отже, як вирази, вирослої вартости, $Г'$ і $Т'$ є те саме й виражають
те саме, але лише в різній формі; вони відрізняються одне від одного
не як грошовий капітал і товаровий капітал, а як гроші й товар.
Оскільки вони являють вирослу вартість, капітал, що функціонував як
капітал, вони виражають лише результат функціонування продуктивного
капіталу, єдиного функціонування, що в ньому капітальна вартість вилуплює
вартість. Спільне в них те, що обидва вони, грошовий капітал і
товаровий капітал, є форми існування капіталу. Один є капітал в грошовій
формі, другий — капітал в товаровій формі. Відмінності специфічних
функцій, що відрізняють їх, не можуть тому становити нічого
іншого, крім відмінностей між функцією грошей і функцією товару. Товаровий
капітал, як безпосередній продукт капіталістичного процесу продукції,
нагадує про це своє походження, а тому в своїй формі
він більш раціональний, менш незрозумілий, ніж грошовий капітал,
що в ньому згас будь-який слід цього процесу, так само як і
взагалі в грошах згасає всяка особливість споживної форми товару. Тому
лише там, де само $Г'$ функціонує як товаровий капітал, де воно
є безпосередній продукт продукційного процесу, а не перетворена
форма цього продукту, — лише там зникає його дивна форма, отже, в
продукції самого грошового матеріялу. Для продукції золота, напр., формула була б така: $Г — Т \splitfrac{Р}{Зп}\dots{}  П\dots{} Г'$ (Г + г), де $Г'$ фігурує як товаровий
продукт тому, що П дає золота більше, ніж авансовано в
першому Г, грошовому капіталі, на елементи продукції золота. Отже,
тут зникає іраціональність виразу $Г\dots{} Г'$ (Г + г), де одна частина
грошової суми виступає як мати другої частини тієї самої грошової
суми.
