ділові гроші, але не дала потім жодного товару. Щодо цих 2/3
авансованих нею грошей, частина 1 виступає проти підрозділу I лише як
покупець, але не виступає ще потім як продавець. Отже, ці гроші
не можуть повернутись до частини 1: інакше сталось би, що вона одержала
елементи основного капіталу в подарунок від I. — Щодо останньої
третини авансованих нею грошей, частина 1 виступає спочатку
як покупець обігових складових частин свого сталого капіталу. На ці
самі гроші підрозділ I купує в частини 1 решту її товару вартістю
в 100. Отже, гроші повертаються до неї (до частини 1 підрозділу II)
назад, бо вона виступає як продавець товарів одразу після того, як
виступала покупцем. Коли б вони не повернулись, то сталося б, що
підрозділ II (частина 1) дав підрозділові I за товари в сумі на 100 спочатку
100 грішми, а потім ще 100 товаром, отже, подарував би йому
свій товар.

Навпаки, до частини 2, яка витратила 100 грішми, повертається
300 грішми: 100 — тому, що вона спочатку як покупець подала в циркуляцію
100 грішми, а потім одержала їх назад як продавець; 200 —
тому, що вона функціонує тільки як продавець товарів, на суму вартости
в 200, але не як покупець. Отже, гроші не можуть повернутись до I.
Отже, зношування основного капіталу покривається грішми, що їх II
(частина 1) подав у циркуляцію на закуп елементів основного капіталу;
але вони потрапляють до рук частини 2 не як гроші частини 1, а як
гроші, що належать підрозділові I.

b) При цьому припущенні решта II c розподіляється так, що частина
1 має 200 грішми, а частина 2—400 в товарах.

Частина 1 продала всі свої товари, але 200 в грошах є перетворена
форма основної складової частини її сталого капіталу, яку треба відновити
in natura. Отже, частина 1 виступає тут лише як покупець і замість
своїх грошей одержує на ту саму суму вартости товари І в формі натуральних
елементів основного капіталу. Частині 2 доводиться подати в
циркуляцію (коли I не авансував грошей для обміну товарів між I і II)
maximum лише 200 ф. стерл., бо в розмірі половини своєї товарової
вартости вона є лише продавець підрозділові I, а не покупець у
підрозділу I.

З циркуляції повертаються 400 ф. стерл. до частини 2; 200 — тому,
що вона їх авансувала як покупець і одержує їх назад як продавець
товарів на 200; 200 — тому, що вона продає підрозділові I товарів вартістю
на 200, не одержуючи за це товарового еквіваленту від I.

c) Частина 1 має 200 в грошах і 200 c в товарах; частина 2 —
200 c (d) в товарах.

Частина 2 при цьому припущенні не має авансувати грішми нічого,
бо вона проти підрозділу I взагалі вже функціонує не як покупець, а
лише як продавець, отже, їй треба чекати, поки в неї куплять.

Частина 1 авансує 400 ф. стерл. грішми; 200 для взаємного обміну
товарами з I, 200 — просто як покупець у I. На ці останні 200 ф. стерл.
грішми вона купує елементи основного капіталу.
