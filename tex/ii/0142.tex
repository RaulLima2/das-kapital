з формами товарового капіталу і грошового капіталу. Але обидві ці форми,
товаровий капітал і грошовий капітал, є носії вартости так само
основної, як і поточної частини продуктивного капіталу. Обидві вони є
капітал циркуляції протилежно до капіталу продуктивного, а не обіговий
(поточний) капітал протилежно до основного.

Нарешті, цілком неправильне пояснення утворення зиску основним
та обіговим капіталом, а саме, що перший ніби утворює зиск, лишаючись
у процесі продукції, а другий — виходячи з нього та циркулюючи далі, —
призводить до того, що через однаковість форми, що її в обороті
мають змінний капітал і поточна складова частина сталого капіталу, приховується
їхня посутня ріжниця в процесі зростання вартости й
утворення додаткової вартости, отже, уся таємниця капіталістичної продукції
ще більше затемнюється. Через загальне означення: „обіговий капітал“,
знімається (wird aufgehoben) цю посутню ріжницю; це повело до
того, що пізніші економісти пішли ще далі: за посутню й єдино відмінну
вони визнавали протилежність не між змінним і сталим капіталом, а
протилежність між основним і обіговим капіталом.

Визначивши основний і обіговий капітал як два різні способи приміщувати
капітал, що з них кожен, сам по собі розглядуваний, дає зиск,
А. Сміс каже: „Жодний основний капітал не може давати зиску без допомоги
обігового капіталу. Найкорисніші машини та знаряддя праці нічого
не випродукують без обігового капіталу, що дає матеріяли, ними
оброблювані, і дає утримання робітникам, які їх застосовують *).

Тут виявляється, що значать попередні вирази: yield a revenue make
a profit **) і т. ін., а саме — це значить, що обидві частини капіталу служать
як продуктотворчі.

А. Сміс наводить потім такий приклад: „Частина капіталу фармерового,
вкладена в господарські знаряддя, є основний капітал, а частина,
вкладена в заробітну плату і утримання слуг-робітників, є обіговий капітал“.
(Тут ріжницю між основним й обіговим капіталом правильно зведено
тільки до різної циркуляції, до різного обороту різних складових
частин продуктивного капіталу). „Фармер має зиск від першого, поки має
його в своєму розпорядженні, і від другого, віддаючи його від себе.
Ціна або вартість його робочої худоби є основний капітал“ (тут знову таки
правильно те, що за основу різниці береться вартість, а не речовий
елемент) „так само, як і ціна або вартість господарських знарядь; засоби
утримання її (робочої худоби) є обіговий капітал так само, як і засоби
утримання слуг-робітників. Фармер одержує зиск, лишаючи в своєму
розпорядженні робочу худобу й віддаючи продукти, що є худобі за засіб
існування“. (Фармер лишає корм худобі, не продає його. Він зуживає
його як корм худобі, зуживаючи саму худобу як знаряддя праці.
Ріжниця лише ось у чому: корм для худоби, що зуживається на утри-

*) „No fixed capital can yield any revenue but by means of a circulating capital.
The most useful machines and instruments of trade will produce nothing wit
hout the circulating capital which affords the materials they are employed upon, and
the maintenance of the workmen who employ them“ (p. 188).

**) Давати дохід, творити зиск.
