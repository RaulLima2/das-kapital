\parcont{}  %% абзац починається на попередній сторінці
\index{ii}{0209}  %% посилання на сторінку оригінального видання
затримає їх цілком або почасти як грошовий капітал. З другого боку,
само собою зрозуміло, що частину, витрачувану на заробітну плату,
рівну 200 ф. стерл., затримується в грошовій формі. Капіталіст, купивши
робочу силу, не може складати її на складах як сировинний матеріял.
Він мусить ввести її в процес продукції та оплачує її наприкінці тижня.
Отже, із звільненого капіталу в 300 ф. стерл. в усякому разі ці 100 ф.
стерл. матимуть форму звільненого грошового капіталу, тобто непотрібного
для робочого періоду. Отже, капітал, що звільнився в формі грошового
капіталу, мусить дорівнювати принаймні змінній частині капіталу,
витраченій на заробітну плату; в максимумі цей грошовий капітал може
досягти суми всього звільненого капіталу. А в дійсності величина його
постійно коливається між цим мінімумом і максимумом.

Грошовий капітал, що звільнився таким чином у наслідок самого
лише механізму руху оборотів (поряд грошового капіталу, що утворюється
через послідовний зворотний приплив основного капіталу, і поряд грошового
капіталу, потрібного в кожному процесі праці для капіталу змінного),
мусить відігравати чималу ролю, скоро тільки розвивається кредитова
система, і разом з тим мусить бути за одну з основ її.

Припустімо, що в нашому прикладі час циркуляції скорочується
з 3 тижнів до 2. Це не нормальне явище, а лише наслідок сприятливого
моменту для підприємства, скорочених термінів виплат тощо. Капітал
в 600 ф. стерл., витрачений протягом робочого періоду, повертається на
тиждень раніше, ніж треба, отже, він звільняється на цей тиждень. Далі
в середині робочого періоду, як і раніше, звільняється 300 ф. стерл.
(частина тих 600 ф. стерл.), але звільняється на 4 тижні замість 3.
Отже, на грошовому ринку протягом одного тижня перебуває 600 ф.
стерл., і 300 ф. стерл. перебувають протягом 4 тижнів замість 3. А що це
стосується не до одного лише капіталіста, а до багатьох, і відбувається по
різних галузях підприємств, в різні періоди, то в наслідок цього на ринку
стає більше вільного грошового капіталу. Коли такий стан триває порівняно
довго, то продукція поширюється, там, де це можливо; капіталісти,
що роблять позиченим капіталом, ставитимуть менший попит на грошовому
ринку, а це полегшує стан грошового ринку так само, як збільшене
подання; або, нарешті, суми, що стали надлишковими для механізму
обороту, остаточно викинеться на грошовий ринок.

В наслідок скорочення часу циркуляції\footnote*{
В нім. тексті, очевидно, помилково стоїть „час обороту“. Ред.
} з 3 до 2 тижнів, а тому й
періоду обороту з 9 до 8 тижнів, 1/9 цілого авансованого капіталу
стає надлишковою; шеститижневий робочий період може тепер так само
безперервно перебігати при 800 ф. стерл., як раніш при 900 ф. стерл.
Тому частина вартости товарового капіталу, рівна 100 ф. стерл., зворотно
перетворившись на гроші, залишається в цьому стані, як грошовий
капітал, не функціонуючи далі як частина капіталу, авансованого на
процес продукції. Тимчасом як продукцію й далі провадиться в попередніх
розмірах і в інших незмінних умовах, як от щодо цін та ін.,
\parbreak{}  %% абзац продовжується на наступній сторінці
