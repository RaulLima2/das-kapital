\parcont{}  %% абзац починається на попередній сторінці
\index{ii}{0167}  %% посилання на сторінку оригінального видання
припустімо, що бавовну й паровози фабрикується на замовлення й оплачується,
коли приставляється готовий продукт. Наприкінці тижня, здавши
готову пряжу, фабрикант-прядільник (ми лишаємо тут осторонь додаткову
вартість) одержує назад витрачений обіговий капітал і разом з тим зношену
частину основного капіталу, яка ввійшла в вартість пряжі. Отже,
він може з тим самим капіталом знову повторити той самий кругобіг.
Капітал тут вивершив свій оборот. Навпаки, фабрикант паровозів мусить
протягом трьох місяців щотижня вкладати новий капітал в заробітну
плату й сировинні матеріяли, і тільки по трьох місяцях, коли здасть паровоза,
обіговий капітал, постійно витрачуваний протягом цього періоду
на один і той самий продукційний акт, на виготовлення одного й того
самого товару, знову набирає форми, що в ній він знову може почати
свій кругобіг; так само зношування машин протягом цих трьох місяців
в нього покривається тільки тепер. Один витрачає свій капітал на тиждень,
а витрата другого дорівнює щотижневій витраті, помноженій на 12.
Коли припустити, що всі інші умови однакові, то один мусить мати в
своєму розпорядженні в дванадцять разів більший обіговий капітал, ніж
другий.

Та обставина, що капітали, авансовувані щотижня, є рівні, не має
тут жодного значення. Хоч яка буде величина авансованого капіталу, в
першому разі його авансовано тільки на тиждень, в другому — на дванадцять
тижнів; і лише коли мине цей час, з ним можна оперувати
знову, повторюючи ту саму операцію або починаючи якусь іншу.

Ріжниця в швидкості обороту, або в періоді, на який треба авансувати
різні капітали, і до скінчення якого та сама капітальна вартість не
може знову служити для нового процесу праці або для нового процесу
утворення вартости, постає тут ось із чого:

Припустімо, що будування паровоза або будь-якої машини коштує
100 робочих днів. Щодо робітників, зайнятих у прядінні і машинобудівництві,
ці 100 робочих днів однаково становлять переривну (diskrete) величину,
що складається, згідно з припущенням, з 100 послідовних окремих
десятигодинних процесів праці. Але щодо продукту-машини 100
робочих днів становлять безперервну величину, один робочий день у
1000 робочих годин, єдиний і суцільний акт продукції. Такий робочий
день, утворений з ряду послідовних, більш-менш численних і зв’язаних
поміж себе робочих днів, я називаю робочим періодом. Кажучи
про робочий день, ми маємо на увазі довжину робочого часу, що протягом
його робітник мусить щоденно витрачати свою робочу силу, щоденно
працювати. А коли ми кажемо, навпаки, про робочий період, то
ми маємо на увазі число зв’язаних поміж себе робочих днів, потрібних
у певній галузі продукції для того, щоб дати готовий продукт. Продукт
кожного робочого дня є тут лише частковий продукт, що день у день
його твориться далі і лише наприкінці довшого або коротшого періоду робочого
часу він набирає викінченої форми, стає готовою споживною вартістю.

Тому перерви, порушення суспільного продукційного процесу, напр.,
в наслідок криз, впливають дуже неоднаково на ті продукти праці, що
\parbreak{}  %% абзац продовжується на наступній сторінці
