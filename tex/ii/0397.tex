ного капіталу, і при постійному повторюванні цього процесу це могло б
стати нормальним джерелом утворення скарбів, а значить, і джерелом
утворення віртуального додаткового грошового капіталу кляси II. Звичайно,
тут, де йдеться про нормальне утворення капіталу, ми лишаємо
осторонь випадковий зиск з шахрайства. Але не треба забувати, що
справді виплачувану нормальну заробітну плату (а вона ceteris paribus\footnote*{
В інших однакових умовах. Ред.
}
визначає величину змінного капіталу) виплачується зовсім не з ласки
капіталістів; її виплачується тому, що при даних відношеннях вона
мусить бути виплачена. Таким чином, цей спосіб пояснення усувається.
Коли ми припускаємо, що змінний капітал, який має витратити кляса II,
становить 376 v, то для того, щоб розв’язати новопосталу проблему, ми
не можемо одразу висунути гіпотезу, що кляса II авансує, напр., тільки
350 v, а не 376 v.

2) Але, з другого боку, як ми вже сказали, кляса II, розглядувана як
ціле, має ту перевагу проти кляси І, що як покупець робочої сили вона
разом з тим є продавець своїх товарів своїм власним робітникам. Як це
можна визискувати, — яким чином можна номінально виплачувати нормальну
заробітну плату, а в дійсності частину її загарбати собі без відповідного
еквіваленту, інакше кажучи, украсти в робітників; як це можна робити
почасти за допомогою truck system, а почасти через фалшування засобів
циркуляції (хоч його не завжди дається виявити юридично), — про
це є цілком наочні дані в кожній промисловій країні, напр., в Англії та
Сполучених Штатах. (З цього приводу треба це розвинути на влучно
обраних прикладах). Це — та сама операція, що в 1), тільки замаскована
і пророблювана обкружним шляхом. Отже, її треба тут відкинути так
само, як і ту. Тут ідеться не про номінальну, а про справді виплачувану
заробітну плату.

Ми бачимо, що при об’єктивній аналізі капіталістичного механізму не
можна скористатися з деяких ганебних плям, які екстраординарно ще
гніздяться в ньому, для того, щоб викрутом усунути теоретичні труднощі.
Але більшість моїх буржуазних критиків якось чудно здіймає галас, ніби
я, прим., в І книзі „Капіталу“ своїм припущенням, що капіталіст виплачує
дійсну вартість робочої сили — а цього він здебільша не робить —
зробив велику кривду цим самим капіталістам! (Тут можна з тією самою
великодушністю, яку приписується мені, цитувати Шефле).

Отже, з 376 II v для згаданої мети нічого не вдієш.

Але ще більші труднощі, здається, постають щодо 376 II m. Тут протистоять
один одному лише капіталісти тієї самої кляси, що продають
один одному й купують один в одного продуковані ними засоби споживання.
Гроші, потрібні для цього обміну, функціонують тільки як засоби
циркуляції, і при нормальному перебігу мусять повертатись назад
до учасників у тій мірі, в якій вони їх авансували для циркуляції, а потім
знову й знов переходити той самий шлях.