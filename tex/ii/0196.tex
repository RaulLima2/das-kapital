тина весь час мусить бути в періоді циркуляції. Або, інакше кажучи, одна
частина може функціонувати як продуктивний капітал лише з тією умовою,
що другу частину, в формі товарового або грошового капіталу, вилучено
з власне продукції. Недобачати це — значить взагалі недобачати значення
й ролі грошового капіталу.

Нам треба тепер дослідити, яка ріжниця буде в обороті залежно від
того, чи будуть обидва відділи періоду обороту — робочий період і період
циркуляції — рівні один одному, чи робочий період буде більший
або менший, ніж період циркуляції, а потім дослідити, як це впливає на
закріплення капіталу в формі грошового капіталу.

Припустімо, що авансовуваний щотижня капітал в усіх випадках дорівнює
100 ф. стерл., а період обороту — 9 тижням; отже, капітал, який
треба авансувати на кожен період обороту, дорівнює 900 ф. стерл.

І. Робочий період дорівнює періодові циркуляції

Цей випадок, хоч, насправді, він трапляється тільки як рідкісний виняток,
мусить бути за вихідний пункт у дослідженні, бо відношення тут виступають
якнайпростіше та якнайнаочніше.

Два капітали (капітал І, авансований на перший робочий період, і додатковий
капітал II, що функціонує протягом періоду циркуляції капіталу І)
чергуються один по одному в своєму русі, не сплітаючись один з одним.
Тому, за винятком першого періоду, кожний із обох капіталів авансується
лише на свій власний період обороту. Період обороту хай буде,
як у наступних прикладах, 9 тижнів; отже, робочий період і період обігу
буде по 4 1/2 тижні. Тоді ми маємо таку схему року:

Таблиця 1

Капітал І

Періоди обороту    Робочі періоди    Авансовано    Періоди циркуляції
І. Тижні 1—9    Тижні 1—4 1/2                 450 ф. ст.        Тижні 4 1/2—9
II.    „   10—18        „      10—13 1/2             450 „ „                  13 1/2—18
III.   „   19—27       „       19—22 1/2             450 „ „                22 1/2—27
IV.   „    28—36      „        28—31 1/2             450 „ „               31 1/2—36
V.     „    37—45      „        37—40 1/2             450 „ „               40 1/2—45
VI.   „    46—[54]  „        46—49 1/2            450 „ „            49 1/2—[54] 31

31) Тижні, що припадають на другий рік обороту, взято в дужки.
