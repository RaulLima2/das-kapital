чено справу не з 1800, а лише з 1780. Отже, в такому разі матимемо:

I.    220 m

II. (1) 220 с (грішми) + (2) 200 с (товаром).

ІІ с, частина 1, на 220 ф. стерл. грішми купує 220 І m, а І на
200 ф. стерл. купує потім 200 II с (2) товаром. Але тоді на боці І
лишається 20 ф. стерл. грішми — така частина додаткової вартости, яку
I може лише затримати в грошах, а не витрачати на засоби споживання.
Таким чином труднощі лише переміщено з II с (частина 1) на І m.

Припустімо тепер, з другого боку, що ІІ с, частина 1, менше, ніж ІІ с
(частина 2), отже:

Другий випадок:

I.    200 m (товаром).

II. (1) 180 с (грішми) + (2) 200 с (товаром).

II (частина 1), на 180 ф. стерл. грішми купує товари 180 І m; на ці
гроші І купує в II (частини 2) товари такої самої вартости, тобто 180

II    с (2); на одному боці лишається 20 І m, що їх не сила продати, і так
само — 20 II с (2) на другому боці; товари вартістю в 40 не сила перетворити
на гроші.

Коли б ми припустили, що остача I = 180, це нам ані трохи не допомогло
б; правда, тоді в І не залишилося б жодного надлишку, але в II с
(частині 2), як і раніш, був би надлишок в 20, що його не сила продати,
перетворити на гроші.

В першому випадку, де II (1) більше, ніж II (2), на боці II с (1)
лишається надлишок в грошах, що його не сила перетворити знову на
основний капітал, або, коли ми припустимо, що остача І m = ІІ с (1) на
боці І m буде такий самий надлишок у грошах, не перетворюваний на
засоби споживання.

В другому випадку, де ІІ с (1) менше, ніж ІІ с (2), виявляється грошовий
дефіцит на боці 200 І m і II с (2) і на обох боках товаровий надлишок
однакової величини, або коли припустити, що остача І ш = ІІ с (2), дефіцит
в грошах і надлишок у товарі на боці II с (2).

Коли б ми припустили, що остачі І завжди дорівнюють ІІ с (1), — бо
продукція визначається замовленнями і в репродукції нічого не змінюється,
коли поточного року випродукувано більше основних складових частин
капіталу, а другого наступного року більше обігових складових частин
сталого капіталу II і І, — то в першому випадку І m можна було б знову
перетворити на засоби споживання лише тоді, коли б І купив на І m
частину додаткової вартости у II, отже, коли б І її не споживав, а нагромаджував
як гроші; в другому випадку лихові можна було б запобігти
лише тоді, коли б І сам витратив гроші, — а цю гіпотезу ми
відкинули.

Коли ІІ с (1) більше, ніж ІІ с (2), то для реалізації грошового надлишку
в І m потрібен довіз закордонних товарів. Коли ІІ с (1) менше, ніж ІІ с
(2), то для того, щоб реалізувати зношену частину II с в засобах про-
