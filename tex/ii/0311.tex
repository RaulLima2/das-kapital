Коли ми тепер припустимо пропорційно однаковий поділ витрат
доходу на доконечні засоби існування та засоби розкошів у капіталістів
II а і II b — припустимо, що й ті й ці витрачають по 3/5 на доконечні
засоби існування, по 2/5 на речі розкошів, то капіталісти підкляси II а
3/5 своєї додаткової вартости, свого доходу в 400 m, отже, 240, витрачатимуть
на свої власні продукти, на доконечні засоби існування, і
2/5 = 160 — на речі розкошів. Капіталісти підкляси II b розподілятимуть
свою додаткову вартість 100 m таким самим способом: 3/5 = 60 на доконечні
засоби і 2/5 = 40 на речі розкошів; ці останні продукується й
обмінюється в межах її власної підкляси.

Ті 160 в засобах розкошів, що їх одержує (II а) m, припливають до
капіталістів II а таким чином: з (II а) 400 m, як ми бачили, 100, що є у
формі доконечних засобів існування, обмінюється на рівну суму (II b) v,
яка існує в засобах розкошів, а дальші 60 в доконечних засобах існування
обмінюється на (II b) 60 m в засобах розкошів. Отже, загальний
підсумок буде такий:

II а: 400 v + 400 m; II b : 100 v + 100 m.

1) 400 v (а) споживають робітники II a, що частину їхнього продукту
(доконечні засоби існування) становлять ці 400 v (a); робітники купують
їх у капіталістичних продуцентів свого власного підрозділу. У наслідок
цього до цих капіталістичних продуцентів повертаються 400 ф. стерл.
грішми, повертається їхня змінна капітальна вартість в 400, сплачена як
заробітна плата цим самим робітникам; на цю вартість капіталісти можуть
знову купити робочу силу.

2) Частину 400 m (a), рівну 100 v (b), отже, 1/4 додаткової вартости (а),
реалізується в речах розкошів таким чином: робітники (b) одержують від
капіталістів свого підрозділу (b) 100 ф. стерл. як заробітну плату; на
цю суму вони купують m (a), тобто товари, що складаються з доконечних
засобів існування; капіталісти а купують на ці гроші речей
розкошів на таку саму суму вартости = 100 v (b), тобто половину всієї
продукції речей розкошів. У наслідок цього до капіталістів b повертається
в грошовій формі їхній змінний капітал, і вони, відновивши закуп
робочої сили, можуть знову почати свою репродукцію, бо ввесь сталий
капітал всієї кляси II вже заміщено через обмін І (v + m) на ІІ с. Отже,
робочу силу робітників, що продукують речі розкошів, тільки тому
можна продати знову, що частину їхнього власного продукту, утворену
як еквівалент їхньої власної заробітної плати, взяли капіталісти II a в
свій споживний фонд, продано їм. (Це саме має силу й для продажу робочої
сили підрозділу І: бо те ІІ с, що на нього обмінюється І (v + m), складається
і з речей розкошів і з доконечних засобів існування, а те, що
відновлюється через І (v + m), складається з засобів продукції так речей
розкошів, як і доконечних (засобів існування).

3) Переходимо до обміну між а і b, оскільки він є обмін лише між
капіталістами обох підвідділів. До цього часу закінчено справу з змінним
