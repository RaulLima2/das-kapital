їхньої додаткової вартости, зглядно їхнього зиску = 400 ф. стерл., то
ці 400 ф. стерл. перетворюються, напр., на 500 ф. стерл. в наслідок
того, що кожен співвласник цих 400 ф. стерл. продає свою частину другому
дорожче на 25%. Що всі роблять так, то наслідок такий самий,
як коли б вони навзаєм продавали один одному за дійсною вартістю.
Тільки для циркуляції товарової вартости в 400 ф. стерл. їм потрібна
маса грошей в 500 ф. стерл., а це є, здається, скорше метод збіднення,
ніж збагачення, бо їм доводиться чималу частину всього свого майна
непродуктивно зберігати в некорисній формі засобів циркуляції. Все
сходить на те, що кляса капіталістів, не зважаючи на всебічне номінальне
підвищення цін їхніх товарів, може розподіляти поміж себе для свого
особистого споживання лише запас товарів вартістю в 400 ф.
стерл., але вони роблять один одному приємність, пускаючи в циркуляцію
400 ф. стерл. товарової вартости за допомогою такої маси грошей,
яка потрібна для 500 ф. стерл. товарової вартости.

Ми зовсім лишаємо осторонь, що тут припускається „частину їхнього
зиску“, і значить, взагалі запас товарів, що в ньому виражається зиск.
А проте Детю хотів саме з’ясувати нам, відки походить цей зиск. Маса
грошей, потрібна для його циркуляції, це питання цілком другорядне. Та маса
товарів, яка репрезентує зиск, здається, походить від того, що капіталісти
не лише продають її один одному, хоч уже й це дуже добре й
глибоко розумно, але що вони продають один одному дуже дорого.
Отже, ми знаємо тепер одне джерело збагачення капіталістів. Воно сходить
до таємниці „ентспектора Брезіґа“, що великі злидні походять з великої
pauvreté *).

2) Далі, ті самі капіталісти продають „найманим робітникам, так тим,
що їх оплачують вони сами, як і тим, що їх оплачують капіталісти-нероби;
таким чином, вони одержують назад від цих робітників всю їхню заробітну
плату, за винятком хіба невеликих заощаджень“.

Зворотний приплив до капіталістів того грошового капіталу, що
в формі його вони авансували заробітну плату робітникові, є, за паном
Детю, друге джерело збагачення цих капіталістів.

Отже, коли кляса капіталістів виплатить робітникам, напр., 100 ф. стерл.,
як заробітну плату, а потім ті самі робітники купують товари такої самої
вартости в 100 ф. стерл. у тієї самої кляси капіталістів, і тому сума
в 100 ф. стерл., авансована капіталістами як покупцями робочої сили,
припливає до них назад при продажу цим робітникам товарів на 100 ф.
стерл., то капіталісти в наслідок цього збагачуються. З погляду
доброго розуму виходить, що капіталісти за допомогою цієї процедури
знову мають ті 100 ф. стерл., що були в них до цієї процедури. На
початку процедури в них було 100 ф. стерл. грішми, на ці 100 ф. стерл.
вони купили робочу силу. За ці 100 ф. стерл. грішми куплена праця
продукує товари вартістю, оскільки ми знаємо до цього часу, в 100 ф.
стерл. В наслідок того, що робітникам продано ці 100 ф. стерл. в то-

*) Бідности. Ред
