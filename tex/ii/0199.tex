цесі продукції і 450 ф. стерл. ввесь час у процесі циркуляції, чи 900 ф.
стерл. функціонують протягом 4 1/2 тижнів у процесі продукції, а протягом
наступних 4 1/2 тижнів — у процесі циркуляції.

Навпаки, коли ми розглядаємо періоди обороту, то

Капітал І 450 × 5 2/3 = 2550 ф. стерл.

„II „ 450 × 5 1/6 = 2325 ф. стерл.

Отже, оборот цілого капіталу 900 × 5 5/12 = 4875 ф. стерл.

Бо число оборотів цілого капіталу дорівнює сумі підсумків оборотів капіталів
І і II, поділеній на суму капіталу І і II.

Треба зазначити, що капітали І і II, коли б були вони самостійні
один проти одного, все ж становили б лише різні самостійні частини суспільного
капіталу, авансованого в тій самій сфері продукції. Отже, коли б
суспільний капітал у цій сфері продукції складався лише з І і II, то для
обороту суспільного капіталу в цій сфері мало б силу те саме обчислення,
що тут має силу для обох складових частин, І і II, того самого
приватного капіталу. Йдучи далі, можна зробити таке обчислення для
кожної частини цілого суспільного капіталу, вкладеної в будь-яку особливу
сферу продукції. Нарешті, число оборотів цілого суспільного капіталу
дорівнює сумі капіталу, що обернувся в різних сферах продукції,
поділеній на суму капіталу, авансованого в цих сферах продукції.

Далі треба зауважити, що так само, як тут у тому самому приватному
підприємстві капітали І і II, точно кажучи, мають різні роки обороту
(що цикл оборотів капіталу II починається на 4 1/2 тижні пізніше, ніж
цикл оборотів капіталу І, то рік капіталу І закінчується на 4 1/2 тижні
раніше, ніж рік капіталу II), так і різні приватні капітали в тій самій
сфері продукції починають свою роботу в цілком різні моменти часу,
а тому й закінчують свій річний оборот в різні часи року. Тут досить
зробити таке саме пересічне обчислення, що його ми вище застосували
до капіталів І і II, щоб роки обороту різних самостійних частин суспільного
капіталу звести до одного загального року обороту.

II. Робочий період більший, ніж період циркуляції

Замість чергуватися один по одному, робочі періоди й періоди обороту
капіталу І і II перехрещуються один з одним. Разом з тим постає
тут звільнення капіталу, чого не було в вище розглянутому випадку.

Але від цього нічого не змінюється в тому, що тепер, як і раніше,
1) число робочих періодів цілого авансованого капіталу дорівнює сумі
вартости річного продукту обох авансованих частин капіталу, поділеній
на весь авансований капітал, і 2) число оборотів цілого капіталу дорівнює
сумі підсумків обох оборотів, поділеній на суму обох авансованих
капіталів. Ми мусимо й тут розглядати обидві частини капіталу так, ніби
вони пророблювали цілком незалежні один від одного рухи обороту.

Отже, ми знову припускаємо, що на процес праці треба щотижня
авансовувати 100 ф. стерл. Робочий період триває 6 тижнів, отже, кожного
