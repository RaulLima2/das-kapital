тивного капіталу, ніж 500 ф. стерл., бо вона по черзі править за грошовий
фонд різних продуктивних капіталів. Отже, цей спосіб пояснення
припускає, що вже є наявні ті гроші, що їх наявність він має з’ясувати.

Далі можна було б сказати так: капіталіст А продукує такі речі, що
їх капіталіст В споживає особисто, непродуктивно. Отже, гроші В перетворюють
на гроші товаровий капітал А, і таким чином та сама грошова
сума служить для перетворення на гроші додаткової вартости В і
обігового сталого капіталу А. Але тут ще пряміше припускається розв’язаним
те саме питання, що на нього треба дати відповідь. А саме, відки
В бере гроші на покриття свого доходу? Яким чином він сам перетворив
на гроші цю частину додаткової вартости свого продукту?

Потім можна було б сказати, що частина обігового змінного капіталу,
яку А постійно авансує своїм робітникам, постійно повертається до нього
з циркуляції; і тільки деяка змінна частина її завжди лишається закріплена
в нього самого для видачі заробітної плати. Однак між моментом
витрачання й моментом зворотного припливу минає деякий час, що
протягом нього гроші, витрачені на заробітну плату, можуть, між іншим,
служити і для перетворення на гроші додаткової вартости. — Але, поперше,
ми знаємо, що як довший цей час, то й більша мусить бути маса грошового
запасу, що її капіталіст А постійно мусить зберігати in petto\footnote*{
In petto — дослівно: в серці своєму, тут у значенні: з собою, при собі. Ред.
}.
Подруге, робітник витрачає гроші, купує на них товари й тим самим
pro tanto перетворює на гроші додаткову вартість, що міститься в цих
товарах. Отже, ті самі гроші, що їх авансується в формі змінного капіталу,
pro tanto служать і для перетворення на гроші додаткової вартости.
Не вглиблюючись у це питання ще далі, тут досить лише зауважити, що
споживання цілої кляси капіталістів і залежних від неї непродуктивних
осіб відбувається рівнобіжно й одночасно з споживанням робітничої кляси;
отже, одночасно з тим, як робітники подають у циркуляцію гроші,
мусять пускати гроші в циркуляцію і капіталісти, щоб витрачати свою
додаткову вартість як дохід; отже, для цього треба вилучати гроші з
циркуляції. Таким чином, щойно наведене пояснення лише зменшувало б
кількість потрібних грошей, але не усунуло б потреби в них. —

Нарешті, можна було б сказати: однак в циркуляцію при першому
вкладенні основного капіталу постійно подається велику кількість грошей,
що їх той, хто пустив їх в циркуляцію, знову вилучає з неї лише поступінно,
частинами, протягом багатьох років. Хіба цієї суми не досить,
щоб перетворити на гроші додаткову вартість? — На це треба відповісти,
що сума в 500 ф. стерл. (яка включає й скарботворення для потрібного
резервного фонду) можливо вже включає й застосовування цієї суми як
основного капіталу, якщо не тим, хто пустив її в циркуляцію, то кимось
іншим. Крім того вже припускається, що в сумі, витрачуваній на
придбання продуктів, що служать як основний капітал, оплачено й додаткову
вартість, що міститься в цих товарах, і питання саме в тому, відки
беруться ці гроші.