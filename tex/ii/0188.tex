гроші більшими масами. Вони припливають назад хоч швидше, хоч повільніше
— залежно від обороту капіталу, — але завжди лише частинами.
Частину їх так само постійно витрачається знову в невеликі переміжкн
часу, а саме ту частину, що знову перетворюється на заробітну плату.
Але другу частину, ту, що її треба знову перетворити на сировинний
матеріял тощо, треба нагромаджувати протягом довшого часу як запасний
фонд, хоч для закупів, хоч для виплат. Отже, ця частина існує в формі
грошового капіталу, хоч змінюється розмір, що в ньому вона існує в
такій формі.

З наступного розділу ми побачимо, як інші обставини, — хоч випливають
вони з процесу продукції, хоч з процесу циркуляції, — неминуче
зумовлюють отаке перебування певної частини авансованого капіталу в
грошовій формі. Взагалі ж треба зазначити, що економісти мають великий
нахил забувати, що частина потрібного в підприємстві капіталу не лише
постійно перебігає послідовно три форми: грошового капіталу, продуктивного
капіталу й товарового капіталу, але що різні частини його постійно
перебувають одна поряд однієї в цих трьох формах, хоч відносна
величина цих частин постійно змінюється. Вони забувають саме про ту
частину, яка постійно існує як грошовий капітал, хоч саме ця обставина
дуже важлива для розуміння буржуазного господарства, а тому має значення
також і на практиці.

Розділ п’ятнадцятий

Вплив часу обороту на величину авансованого
капіталу

В цьому та наступному шістнадцятому розділі ми досліджуємо вплив
часу обороту на самозростання вартости капіталу.

Візьмімо товаровий капітал, що є продукт робочого періоду, напр.,
дев’ятьох тижнів. Лишімо покищо осторонь частину вартости продукту,
долучену до нього в наслідок пересічного зношування основного капіталу,
а також і додаткову вартість, долучену до нього під час продукційного
процесу; тоді вартість цього продукту дорівнюватиме вартості
авансованого на його продукцію поточного капіталу, тобто заробітної плати
й зужиткованих на його продукцію сировинних і допоміжних матеріялів.
Припустімо, що ця вартість дорівнює 900 ф. стерл., так що тижнева
витрата становить 100 ф. стерл. Отже, періодичний час продукції, що
збігається тут з робочим періодом, становить 9 тижнів. При цьому байдуже,
чи припускається, що тут ідеться про робочий період для неподільного
продукту, чи про безперервний робочий період для продукту
подільного, скоро тільки кількість подільного продукту, що його воднораз
подається на ринок, коштує 9 тижнів праці. Припустімо, що час
обігу триває 3 тижні. Отже, весь період обороту триває 12 тижнів.
По 9 тижнях авансований продуктивний капітал перетворюється на това-
