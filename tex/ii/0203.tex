Другий період обороту, тижні 8—16, має в собі другий робочий
період, тижні 8—14. З них потреби 8-го й 9-го тижнів покривається
капіталом II. Наприкінці 9-го тижня повертаються давніші 700 ф. стерл.;
з них пускається в роботу до кінця робочого періоду (тижні 10—14)
500 ф. стерл. 200 ф. стерл. лишаються вільні для ближчого наступного
робочого періоду. Другий період обігу триває протягом 15-го й 16 тижнів;
наприкінці 16-го тижня знову повертаються назад 700 ф. стерл.
З цього моменту в кожному робочому періоді повторюється те саме
явище. Потреба в капіталі протягом перших двох тижнів покривається
за допомогою 200 ф. стерл., що звільнились наприкінці попереднього
робочого періоду; наприкінці 2-го тижня повертаються назад 700 ф.
стерл.; але робочий період налічує ще тільки 5 тижнів, так що на нього
можна авансувати лише 500 ф. стерл.; отже, 200 ф. стерл. завжди лишаються
вільні для наступного робочого періоду.

Отже, виявляється, що в нашому випадку, де ми припускали, що робочий
період більший, ніж період обігу, наприкінці кожного робочого
періоду при всяких обставинах є звільнений грошовий капітал, такої
саме величини, як капітал II, авансований на період циркуляції. В наших
трьох прикладах капітал II дорівнював: в першому — 300 ф. стерл., в
другому — 400 ф. стерл., в третьому — 200 ф. стерл.; відповідно до
цього капітал, що звільнявся наприкінці кожного робочого періоду, був
послідовно 300, 400, 200 ф. стерл.

III. Робочий період менший від часу обігу

Спочатку ми знову припустимо період обороту в 9 тижнів: з них
З тижні становлять робочий період, що для нього є в розпорядженні
капітал І = 300 ф. стерл. Період обігу хай буде 6 тижнів. Для цих
6 тижнів потрібен додатковий капітал в 600 ф. стерл., який ми знову
можемо розподілити на два капітали по 300 ф. стерл., що з них кожен
заповнює один робочий період. Тоді ми маємо три капітали по 300 ф.
стерл., з них 300 ф. стерл. завжди зайнято в продукції, тимчасом як
600 ф. стерл. циркулюють.

Таблиця III.

Капітал І.

Періоди обороту    Робочі періоди    Періоди обігу
І. Тижні 1—9    Тижні 1—3    Тижні 4—9
ІІ. „10—18 „10—12 „13—18
III. „19—27 „19—21 „22—27
IV. „28—36 „28—30 „31—36
V. „37—45 „37—39 „40—45
VI. „46 — [54] „46—48 „49 — [54]
