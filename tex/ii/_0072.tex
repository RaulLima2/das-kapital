\parcont{}  %% абзац починається на попередній сторінці
\index{ii}{0072}  %% посилання на сторінку оригінального видання
між авансованою заробітною платою і купівельною ціною, що її платить
останній споживач, повинна являти зиск з капіталу. Він розподіляється
між фабрикантом, гуртовим купцем і роздрібним торговцем, з того часу,
як вони розподілили між собою свої функції, а виконана робота лишилась
та сама, хоч здійснили її три особи й три гатунки капіталу
замість одного“\footnote*{
Le commerce emploie un capital considérable qui paraît, au premier coup
d’oeil, ne point faire partie de celui dont nous avons détaillé la marche. La valeur
des draps accumulés dans les magasins du marchand-drapier semble d'abord tout-à-fait étrangère à
cette partie de la production annuelle que le riche donne au
pauvre comme salaire pour le faire travailler. Ce capital n’a fait cependant que
remplacer celui dont nous avons parlé. Pour saisir avec clarté le progrès de la
richesse, nous l’avons prise à sa création, et nous l’avons suivie jusqu’à sa consommation.
Alors le capital employé dans la manufacture de draps, par exemple, nous a paru
toujours le même; échangé contre le revenu du consommateur, il ne s’est partagé
qu’en deux parties: l’une a servi de revenu au fabricant comme profit, l’autre a servi de revenu aux
ouvriers comme salaire, tandis qu’ils fabriquaient de nouveau drap. —

„Mais on trouva bientôt que, pour l’avantage de tous, il valait mieux que les
diverses parties de ce capital se remplaçassent l’une l’autre, et que, si cent mille
écus suffisaient à faire toute la circulation entre le fabricant et le consommateur,
ces cent mille écus se partageassent également entre le fabricant, le marchand en
gros et le marchand en détail. Le premier, avec le tiers seulement, fit le même
ouvrage qu’il aurait fait avec la totalité, parcequ’au moment où sa fabrication était
terminée, il trouvait le marchand acheteur beaucoup plus tôt qu’il n’aurait trouvé le consommateur.
Le capital du marchand en gros se trouvait de son côté beaucoup
plus tôt remplacé par celui du marchand en détail... La différence entre les sommes, des salaires
avancés et le prix d’achat du dernier consommateur devait faire le profit des capitaux. Elle se
répartit entre le fabricant, le marchand et le détaillant depuis qu’ils eurent divisé entre eux
leurs fonctions, et l’ouvrage accompli fut le même quoiqu il eût employé trois personnes et trois
fractions de capitaux, au lieu d’une.
(„Nouveaux Principes d’Economie Politique, Livre II, ch. VIII, éd. 1827. p. 138—140).
}.

„Всі (торговці) посередньо сприяли продукції, бо вона має на меті
споживання, і тому її можна вважати за вивершену лише тоді, коли
вона подала спродуковану річ до розпорядження споживачеві“\footnote*{
Tous concouraient indirectement à la production; car celle-ci, avant pour
objet la consommation, ne peut être considérée comme accomplie que quand elle a
mis la chose produite à la portée du consommateur*, (lb., p. 137).
}.

Розглядаючи загальні форми кругобігу, і взагалі в усій цій другій
книзі, ми беремо гроші як металеві гроші й лишаємо осторонь
символічні гроші — звичайні знаки вартости, що є лише виключно приналежність
деяких держав, а також кредитові гроші, які ще не розвинулись.
Це, поперше, відповідає історичному розвиткові; кредитові гроші
не відіграють жадної ролі, або лише незначну ролю, в першу добу капіталістичної
продукції. Подруге, доконечність такого порядку дослідження
обґрунтовується також теоретично тим, що всі критичні досліди над
циркуляцією кредитових грошей, що їх маємо з боку Тука й інших,
примушували їх завжди повертатись до розгляду того, як стояли б справи
на основі чистої металевої циркуляції. Але не треба забувати, що металеві
гроші можуть так само правити за купівельний засіб, як і за виплатний
засіб. Дбаючи про спрощення, ми взагалі в цій II книзі беремо їх лише
в першій функціональній формі.
