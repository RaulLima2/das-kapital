\index{ii}{0407}  %% посилання на сторінку оригінального видання
З) Заміщення II с при акумуляції

Отже, при обміні І (v + m) на II с можливі різні випадки.

При простій репродукції обидві ці величини мусять дорівнювати одна
одній й заміщувати одна одну, бо інакше, як ми бачили вище, проста
репродукція не може відбуватись без порушень.

При акумуляції треба звернути увагу насамперед на норму акумуляції.
До цього часу ми в усіх випадках припускали, що норма акумуляції в
I =\footnote{
І (v + 1/2m) = II с, яке, отже, менше, ніж І (v + m). Це завджди
мусить бути так, інакше І не акумулював би.
}/\footnote{
І (v + 1/2m) більше, ніж ІІс. В цьому випадку заміщення досягається
тим, що до ІІс долучається відповідна частина з IIm, так що
ця сума = І (v + 1/2 m). Тут заміщення для II є не проста репродукція
його сталого капіталу, а вже акумуляція, збільшення цього сталого капіталу
на частину його додаткового продукту, що її він обмінює на засоби
продукції І; це збілішення разом з тим включає, що II, крім того, відповідно
збільшує свій змінний капітал з свого власного додаткового продукту,
} m І, і що вона в різні роки лишалась стала. Ми припускали
тільки зміну відношення, що в ньому цей акумульований капітал поділяється
на сталий і змінний. При цьому ми мали три випадки:

3) І (v + 1/2m) менше, ніж ІІс. В цьому випадку II за допомогою
обміну не цілком репродукує свій сталий капітал, отже, він мусить покрити
недостачу купівлею в І. Та це не потребує дальшої акумуляції
змінного капіталу II, бо такою операцією тільки цілком репродукується
за величиною його сталий капітал. З другого боку, та частина капіталістів
І, яка акумулює лише додатковий грошовий капітал, через такий
обмін почасти вже здійснила акумуляцію такого роду.

Припущення простої репродукції, а саме, що І (v + m) = IIc не лише
не узгоджується з капіталістичною продукцією, — це, однак, не виключає
того, що в промисловому циклі в 10—11 років сукупна продукція одного
якогось року часто буває менша, ніж попереднього року, отже,
порівняно з попереднім роком не відбувається навіть простої репродукції,
— але, крім того, при природному річному прирості людности проста
репродукція могла б відбуватись лише остільки, оскільки відповідно
більше, непродуктивного службового люду брало б участь у споживанні
тих 1500, що репрезентують сукупну додаткову вартість. Навпаки, акумуляція
капіталу, тобто справжня капіталістична продукція при цьом] /
була б неможлива. Отже, факт капіталістичної акумуляції виключає мо® = >
ливість того, що II с = І (v + m).

Однак, навіть при капіталістичній акумуляції могло б статись, що в
наслідок перебігу процесу акумуляції, який відбувався протягом цілого
ряду попередніх періодів продукції, II с було б не лише рівне, але й навіть
більше, ніж І (v + m). Це була б перепродукція в II, і її можна було б
вирівняти лише веіиким крахом, що в наслідок його капітал з II перемістився
б в І. — Відношення І (v + m) до II с зовсім не зміниться від того,
коли частина сталого капіталу II репродукує сама себе, як, наприклад,
\parbreak{}  %% абзац продовжується на наступній сторінці
