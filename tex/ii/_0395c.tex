\parcont{}  %% абзац починається на попередній сторінці
\index{ii}{0395}  %% посилання на сторінку оригінального видання
Ми взяли суму меншу, ніж у схемі І, саме для того, щоб унаочнити,
що репродукція в поширеному маштабі (а ми її розуміємо тут як продукцію,
проваджувану з більшим капіталовкладенням) не має жодного
чинення до абсолютної величини продукту, що для даної маси товарів
вона має за передумову тільки інше розміщення або інше функціональне
призначення різних елементів даного продукту, отже, за величиною вартости,
вона є спочатку лише проста репродукція. Змінюється не кількість,
а якісне призначення даних елементів простої репродукції, і ця зміна є
матеріяльна передумова пізніше посталої репродукції в поширеному
маштабі\footnote{
Це раз назавжди кладе край суперечці про акумуляцію капіталу між Джемсом
Міллом і S. Ваiley’єм, розглянутій в кн. І (розділ XXII, 5, примітка 65) з іншого
погляду; а саме — суперечкі про те, в якій мірі може розширюватися діяння промислового
капіталу при незмінній величині його. Пізніше слід до цього повернутись.
}.

Ми могли б подати цю схему інакше, при іншому відношенні між
змінним і сталим капіталом; напр., так:

Схема b)

I. 4000 c + 875 v + 875 m = 5750
II. 1750 c + 376 v + 376 m = 2502 Сума = 8252.

В такому вигляді вона виступала б як побудована для простої репродукції,
так що всю додаткову вартість витрачалося б як дохід, а не нагромаджувалось.
В обох випадках, і при а) і при b), ми маємо річний
продукт однакової величини вартости, тільки в схемі b) його елементи
групуються за своїми функціями так, що знову починається репродукція
в попередньому маштабі, тимчасом як в схемі а) утворюється матеріяльна
основа для репродукції в поширеному маштабі. А саме в схемі b)
(875 v + 875 m) I = 1750 І (v + m) обмінюються без остачі на
1750 II с, тимчасом як в схемі а) (1000 v + 1000 m) I = 2000 І
(v + m) при обміні на 1500 II с дають остачу в 500 І m для акумуляції
в клясі І.

Тепер до ближчої аналізи схеми а). Припустімо, що і в І і в II половину
додаткової вартости, замість витрачати як дохід, акумулюється, тобто
перетворюється на елемент додаткового капіталу. А що половину 1000
І m = 500 повинно акумулювати в тій або іншій формі, вкласти як додатковий
грошовий капітал, тобто перетворити на додатковий продуктивний
капітал, то як дохід витрачається тільки (1000 v + 500 m) І. Тому
як нормальна величина II с тут фігурують теж лише 1.500. Обмін між
1500 І (v + m) і 1500 II с не треба досліджувати далі, бо ми вже описали
його як процес простої репродукції; так само не треба розглядати
4000 І с, бо поновне розміщення його для репродукції, що знову починається
(а вона тепер відбувається в поширеному маштабі), ми теж розглянули
як процес простої репродукції.

Отже, нам лишається дослідити тут тільки одне, а саме 500 І m і
(376 v + 376 m) II, оскільки розглядається, з одного боку, внутрішнє відношення
так в І, як і в II, а з другого боку — рух між ними обома. А що при-
\parbreak{}  %% абзац продовжується на наступній сторінці
