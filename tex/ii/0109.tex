роду. Підчас свого функціонування вони не зберігають своєї самостійної
споживної форми. Отже, підчас їхнього функціонування жодна частина
капітальної вартости не лишається фіксована в своєму першому споживному
вигляді, в своїй натуральній формі. Та обставина, що ця частина
допоміжних матеріялів речово не входить в продукт, але ввіходить у
вартість продукту лише своєю вартістю, як частина його вартости, і що
в зв’язку з цим функціонування таких матеріялів міцно прикріплено до
сфери продукції, — призвело деяких економістів, напр., Рамсая, до того,
що вони (одночасно сплутуючи основний і сталий капітал) зачислили їх
до категорії основного капіталу.

Частина засобів продукції, що речово ввіходить у продукт, отже,
сировинний матеріял і т. ін., набуває в наслідок цього почасти таких
форм, що в них вона може пізніше ввійти в особисте споживання як
засоби споживання. Власне засоби праці, речові носії основного капіталу,
споживається лише продуктивно, і не можуть вони ввійти в особисте
споживання, бо вони не ввіходять у продукт або в ту споживну вартість,
що її вони допомагають утворити, а, навпаки, зберігають проти
неї свою самостійну форму ввесь час, поки вони цілком зносяться. Виняток
становлять тільки засоби транспорту. Корисний ефект, що його
вони дають підчас свого продуктивного функціонування, тобто підчас
перебування в сфері продукції, — зміна місця, ввіходить одночасно в особисте
споживання, напр., пасажира. Він також оплачує тут споживання,
як оплачує користування з інших засобів споживання. Ми бачили, що
сировинний матеріял і допоміжні матеріяли іноді зливаються один з
одним, напр., у хемічній фабрикації. Те саме буває й з засобами праці
та допоміжними матеріялами й сировинним матеріялом. Напр., в хліборобстві
речовини, вкладені для поліпшення ґрунту, почасти ввіходять як продуктотворчі
елементи в рослинний продукт. З другого боку, їхня дія
розподіляється на відносно довгий період, напр., 4—5 років. Тому частина
їх речово ввіходить у продукт і разом з тим переносить свою
вартість на продукт, тимчасом як друга частина зберігає свою стару
споживну форму і фіксує в ній свою вартість. Вона і далі існує як засіб
продукції і тому набирає форми основного капіталу. Як робоча худоба
віл є основний капітал. А якщо його з’їдають, він функціонує вже не як
засіб праці, отже, не як основний капітал.

Причина (Bestimmung), що надає частині капітальної вартости, витраченій
на засоби продукції, характеру основного капіталу, є виключно в
своєрідному способові циркуляції цієї вартости. Цей особливий спосіб
циркуляції випливає з того особливого способу, що ним засоби праці
віддають свою вартість продуктові, або з того способу, в який вони виступають
як вартіснотворчі чинники підчас продукційного процесу. А цей останній
й собі випливає з особливого способу функціонування різних засобів
праці в процесі праці.

Відомо, що та сама споживна вартість, що виходить як продукт з
одного процесу праці, ввіходить у другий як засіб продукції. Тільки
функціонування продукту як засобу праці в продукційному процесі ро-
