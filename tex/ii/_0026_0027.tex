\parcont{}  %% абзац починається на попередній сторінці
\index{ii}{0026}  %% посилання на сторінку оригінального видання
всякого іншого товару, визначається вартістю витрачених на нього елементів
продукції (робочої сили та засобів продукції) плюс додаткова
вартість, утворена додатковою працею робітників, що працюють у транспортовій
промисловості. Щодо споживання корисного ефекту транспортової
промисловости, то він і тут не відрізняється нічим від інших товарів.
Коли його споживається особисто, то разом з споживанням зникає його
вартість, коли його споживається продуктивно, так що він сам являє стадію
продукції товарів, що перебувають у транспорті, тоді його вартість
переноситься як додаток вартости на товари. Отже, формула для транспортової
промисловости була б така: Г — Т Р Зп... П... Г', бо тут оплачується і
споживається самий процес продукції, а не продукт, що його можна від
нього відокремити. Отже, вона має майже таку саму форму, як формула для
продукції благородних металів, тільки тут Г' являє перетворену форму
корисного ефекту, утвореного підчас продукційного процесу, а не
натуральну форму золота або срібла, видобутих підчас цього процесу й
виштовхнутих з нього.

Промисловий капітал є єдина форма буття капіталу, де функція
капіталу є не лише привлащення додаткової вартости, зглядно додаткового
продукту, але й разом з тим її утворення. Тому промисловий
капітал зумовлює капіталістичний характер продукції; його наявність
включає наявність клясового противенства капіталістів і найманих робітників.
У міру того, як він опановує суспільну продукцію, відбувається
переворот у техніці й суспільній організації процесу праці, а
разом з тим і в економічно-історичному типі суспільства. Інші відміни
капіталу, які існували до нього серед обставин суспільної продукції, що
вже минули, або серед обставин суспільної продукції, що гинуть, не
лише стають йому підпорядковані й у механізмі своїх функцій відповідно
до нього змінені, але й рухаються лише на його основі, отже, живуть
і вмирають, стоять і падають разом з цією основою. Грошовий
капітал і товаровий капітал, оскільки вони з своїми функціями виступають
поряд промислового капіталу як носії особливих галузей підприємств,
є лише усамостійнені в наслідок суспільного поділу праці
та однобічно розвинені відміни існування різних функціональних форм,
що їх промисловий капітал то набирає, то скидає в сфері циркуляції.

Кругобіг Г — Г', з одного боку, переплітається з загальною товаровою
циркуляцією, виходить з неї, входить у неї, і становить її частину. З другого
боку, для індивідуального капіталіста він становить особливий самостійний
рух капітальної вартости, рух, що почасти відбувається в межах загальної
товарової циркуляції, а почасти поза нею але завжди зберігає свій самостійний
характер. Поперше, тому, що обидві фази руху, що відбуваються
в сфері циркуляції, Г — Т і Т' — Г' мають функціонально визначений
характер як фази руху капіталу; Т в Г — Т речево визначено як
робоча сила і засоби продукції; в Т' — Г' реалізується капітальна вартість
плюс додаткова вартість. Подруге, П, процес продукції, охоплює продуктивне
\index{ii}{0027}  %% посилання на сторінку оригінального видання
споживання. Потретє, поворот грошей до їхнього вихідного пункту
робить рух Г... Г' кругобігом, замкненим у собі самому.

Отже, з одного боку, кожний індивідуальний капітал у своїх обох
половинах циркуляції, Г-Т і Т' — Г', є чинник загальної товарової
циркуляції, що в ній він функціонує або вплітається то як гроші, то як
товар; і таким чином сам він є член у загальному ряді метаморфоз товарового
світу. З другого боку, він пророблює в межах загальної циркуляції свій
власний самостійний кругобіг, що в ньому сфера продукції є перехідна
стадія і що в ньому він повертається до свого вихідного пункту в тій
самій формі, в якій вийшов з нього. У межах свого власного кругобігу,
що має в собі його реальну метаморфозу в продукційному процесі,
капітал змінює також величину своєї вартости. Він повертається не лише
як грошова вартість, а як збільшена, виросла грошова вартість.

Нарешті, коли ми розглядатимемо Г — Т... П... Т' — Г' як специфічну
форму процесу кругобігу капіталу поряд інших форм, що їх дослідиться
в дальшому, то воно відзначатиметься ось чим.

1) Воно являє кругобіг грошового капіталу, бо промисловий капітал
в його грошовій формі, як грошовий капітал, є вихідний і кінцевий
пункт цілого його процесу. Сама формула виражає, що гроші тут не
витрачається як гроші, а лише авансується, а тому є лише грошова
форма капіталу, грошовий капітал. Вона, крім того, виражає, що мінова
вартість, а не споживна вартість, є самоціль, що визначає рух. Саме тому,
що грошова форма вартости є її самостійна, наочна форма виявлення, —
саме тому форма циркуляції Г... Г', що її вихідний і кінцевий
пункти є справжні гроші, і виражає якнайнаочніше движний чинник
капіталістичної продукції, роблення грошей. Продукційний процес
є лише неминуча посередня ланка, неодмінне лихо для роблення грошей.
[Тому всі нації з капіталістичним способом продукції періодично захоплюються
шахрайством, що за допомогою його вони намагаються робити
гроші без продукційного процесу]. .

2) Стадія продукції, функція П, становить у цьому кругобігу перерву
між двома фазами циркуляції Г — Т... Т' — Г', яка знову таки є лише
посередня ланка простої циркуляції Г — Т — П. Продукційний процес у
самій цій формі процесу кругобігу формально й виразно виступає як
те, чим він є в капіталістичному способі продукції, як звичайний спосіб
збільшувати авансовану вартість, отже, збагачення, як таке, виступає як
самоціль продукції.

3) Що ряд фаз починається фазою Г — Т, то другий член циркуляції
є Т' — Г'; отже, вихідний пункт — Г, грошовий капітал, що має збільшити
свою вартість; кінцевий пункт — Г', вирослий у своїй вартості грошовий
капітал Г + г, де Г фігурує як реалізований капітал поряд свого
паросту г. Це відрізняє кругобіг Г від двох інших кругобігів П і Т; відрізняє
двома сторонами. З одного боку, грошовою формою обох крайніх членів;
а гроші с самостійна, наочна форма існування вартости, вартість продукту
в ії самостійній формі, де зник будь-який слід споживної вартости
товарів. З другого боку, форма П... П не неодмінно перетворюється на
\parbreak{}  %% абзац продовжується на наступній сторінці
