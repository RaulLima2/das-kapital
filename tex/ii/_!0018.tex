\parcont{}  %% абзац починається на попередній сторінці
\index{ii}{*0018}  %% посилання на сторінку оригінального видання
може спокійно лежати, поки вийде друком рукопис Марксів, що її критикує\footnote*{
Цю книгу видано за редакцією К. Кавтського під назвою: „Theorien über den Mehrwert“. Zweiter
Band. Erster Teil. („Теорії додаткової вартости“. Том II, част. І). \emph{Ред.}
}... Нарешті, його проєкти
щодо визволення старопруського землеволодіння від гніту капіталу знову цілком утопічні; а саме вони
обминають єдине практичне питання, про яке тут ідеться, — питання про те, як старопруський юнкер,
одержуючи, приміром, 20.000 марок щороку й витрачаючи, приміром, 30.000 марок, може все таки не
робити боргів?

Школа Рікардо біля 1830 року скрахувала на додатковій вартості. Те, чого вона не могла розв’язати,
лишилось то більше нерозв’язним для її наступниці, вульґарної економії. Два пункти, що на них вона
зазнала загибелі, такі.

По-перше. Праця є міра вартости. Але жива праця в обміні на капітал має меншу вартість, ніж
зречевлена праця, що на неї її обмінюється. Заробітна плата, вартість певної кількости живої праці,
завжди менша, ніж вартість продукту, утворюваного цією самою кількістю живої праці, або того
продукту, що в ньому ця праця втілюється. Коли
так уявляти це питання, то його в дійсності не можна розв’язати. Маркс поставив його правильно й
тому дав на нього відповідь. Не праця має вартість. Як діяльність, що утворює вартість, вона так
само не може мати особливої вартости, як важкість не може мати особливої ваги, тепло — особливої
температури, електрика особливої сили струму. Купується й продається, як товар, не праця, а робоча
сила. Скоро вона стає товаром, її вартість вимірюється працею, втіленою в ній, як у суспільному
продукті; ця вартість дорівнює праці, суспільно доконечній для її продукції та репродукції. Отже,
купівля і продаж робочої сили на основі такої її вартости зовсім не суперечать економічному законові
вартости.

Подруге. Згідно з законом вартости Рікардо, два капітали, що вживають однакової кількости однаково
оплачуваної живої праці, за всіх інших однакових умов, продукують протягом однакового часу продукти
однакової вартости, а також додаткову вартість або зиск однакового розміру. А коли вони вживають
неоднакової кількости живої праці, то не можуть продукувати додаткову вартість, або, як кажуть
рікардіянці, зиск однакового розміру. А в дійсності маємо протилежне. У дійсності однакові
капітали протягом однакового часу продукують пересічно однаковий зиск незалежно від того, чи багато,
чи мало вживають вони живої праці. Отже, тут маємо суперечність законові вартости, що її помітив ще
Рікардо, і яку його школа теж не могла розв’язати. Родбертус також не міг не помітити цієї
суперечности, але замість розв’язати її, він зробив з неї один з вихідних пунктів своєї утопії („Zur
Erkenntniss“, S. 131). Цю суперечність розв’язав Маркс уже в рукопису „Zur Kritik“; за пляном
„Капіталу“ це розв’язання подається в книзі III. До її опублікування пройдуть іще місяці. Отже,
економісти, які хочуть відкрити в особі Родбертуса таємне джерело й незрівняного попередника Маркса,
\parbreak{}  %% абзац продовжується на наступній сторінці
