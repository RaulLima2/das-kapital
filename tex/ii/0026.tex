всякого іншого товару, визначається вартістю витрачених на нього елементів
продукції (робочої сили та засобів продукції) плюс додаткова
вартість, утворена додатковою працею робітників, що працюють у транспортовій
промисловості. Щодо споживання корисного ефекту транспортової
промисловости, то він і тут не відрізняється нічим від інших товарів.
Коли його споживається особисто, то разом з споживанням зникає його
вартість, коли його споживається продуктивно, так що він сам являє стадію
продукції товарів, що перебувають у транспорті, тоді його вартість
переноситься як додаток вартости на товари. Отже, формула для транспортової
промисловости була б така: Г — Т Р Зп... П... Г', бо тут оплачується і
споживається самий процес продукції, а не продукт, що його можна від
нього відокремити. Отже, вона має майже таку саму форму, як формула для
продукції благородних металів, тільки тут Г' являє перетворену форму
корисного ефекту, утвореного підчас продукційного процесу, а не
натуральну форму золота або срібла, видобутих підчас цього процесу й
виштовхнутих з нього.

Промисловий капітал є єдина форма буття капіталу, де функція
капіталу є не лише привлащення додаткової вартости, зглядно додаткового
продукту, але й разом з тим її утворення. Тому промисловий
капітал зумовлює капіталістичний характер продукції; його наявність
включає наявність клясового противенства капіталістів і найманих робітників.
У міру того, як він опановує суспільну продукцію, відбувається
переворот у техніці й суспільній організації процесу праці, а
разом з тим і в економічно-історичному типі суспільства. Інші відміни
капіталу, які існували до нього серед обставин суспільної продукції, що
вже минули, або серед обставин суспільної продукції, що гинуть, не
лише стають йому підпорядковані й у механізмі своїх функцій відповідно
до нього змінені, але й рухаються лише на його основі, отже, живуть
і вмирають, стоять і падають разом з цією основою. Грошовий
капітал і товаровий капітал, оскільки вони з своїми функціями виступають
поряд промислового капіталу як носії особливих галузей підприємств,
є лише усамостійнені в наслідок суспільного поділу праці
та однобічно розвинені відміни існування різних функціональних форм,
що їх промисловий капітал то набирає, то скидає в сфері циркуляції.

Кругобіг Г — Г', з одного боку, переплітається з загальною товаровою
циркуляцією, виходить з неї, входить у неї, і становить її частину. З другого
боку, для індивідуального капіталіста він становить особливий самостійний
рух капітальної вартости, рух, що почасти відбувається в межах загальної
товарової циркуляції, а почасти поза нею але завжди зберігає свій самостійний
характер. Поперше, тому, що обидві фази руху, що відбуваються
в сфері циркуляції, Г — Т і Т' — Г' мають функціонально визначений
характер як фази руху капіталу; Т в Г — Т речево визначено як
робоча сила і засоби продукції; в Т' — Г' реалізується капітальна вартість
плюс додаткова вартість. Подруге, П, процес продукції, охоплює продук-
