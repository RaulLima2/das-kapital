\index{ii}{0024}  %% посилання на сторінку оригінального видання
Дві форми, що їх набирає капітальна вартість у своїх стадіях циркуляції,
є форми грошового капіталу й товарового капіталу, її форма,
що належить до стадії продукції, є форма продуктивного капіталу.
Капітал, то протягом цілого свого кругобігу набирає, а потім скидає ці
форми і в кожній з них виконує відповідну їй функцію, є промисловий
капітал — промисловий тут у тому значінні, що він охоплює кожну
галузь продукції, провадженої капіталістично.

Отже, грошовий капітал, товаровий капітал, продуктивний капітал
означають тут не самостійні ґатунки капіталу, що їхні функції являють
зміст теж самостійних і відокремлених одна від однієї галузей підприємств.
Вони означають тут лише особливі функціональні форми промислового
капіталу, що послідовно набирає їх усі три одну по одній.

Кругобіг капіталу відбувається нормально лише доти, доки різні фази
його без затримок переходять одна в одну. Коли капітал затримується
на першій фазі $Г — Т$, то грошовий капітал затвердіває в скарб;
коли на продукційній фазі — то на одному боці лежать засоби продукції
не функціонуючи, тимчасом як на другому боці робоча сила лишається
незайнятою, коли на останній фазі $Т' — Г'$, то нерозпродані товари
скупчуються й захаращують перебіг циркуляції.

З другого боку, з самої суті справи, самий кругобіг зумовлює фіксацію
капіталу на певний строк в окремих фазах кругобігу. В кожній із
своїх фаз промисловий капітал зв’язаний з якоюбудь певною формою,
як грошовий капітал, продуктивний капітал, товаровий капітал. Тільки
після того як він виконає функцію, що відповідає тій формі, яку він
має кожного разу, він набирає форму, що в ній може ввійти в нову
фазу перетворень. Щоб унаочнити це, ми припустили в нашому прикладі,
що капітальна вартість товарової маси, утвореної на продукційній стадії,
дорівнює всій сумі вартости, первісно авансованої в грошовій формі,
інакше кажучи, що вся капітальна вартість, авансована як гроші, одним
заходом увіходить з однієї стадії в наступну. Але ми бачили (книга І,
розділ VI), що частина сталого капіталу, власне знаряддя праці (прим.,
машини), в більшому або меншому числі повторюваних процесів продукції
придаються знову й знову, а тому лише частинами передають свою вартість
продуктові. Далі виявиться, як ця обставина відмінює процес кругобігу
капіталу. Тут обмежимось ось чим: у нашому прикладі вартість
продуктивного капіталу, 422 ф. стерл., має в собі лише пересічно обчислене
зношування робітних будівель, машин тощо, отже, тільки ту частину
вартости, що її вони підчас перетворення 10.000 ф. бавовни на 10.000 ф.
пряжі переносять на цю останню, на продукт тижневого процесу прядіння
протягом 60 годин. Тому в засобах продукції, що на них перетворюється
авансований сталий капітал в 372 ф. стерл., знаряддя праці, будівлі,
машини тощо також фігурують так, ніби їх брали на прокат на ринку,
оплачуючи тижневими ратами. Однак, це аніскільки не змінює справи.
Досить буде нам кількість пряжі в 10.000 ф., спродуковану протягом
тижня, помножити на число тижнів, що складають ряд років, — і вся
вартість куплених і спожитих протягом цього часу знарядь праці перенесеться
\index{ii}{0025}  %% посилання на сторінку оригінального видання
на пряжу. Відси зрозуміло, що авансований грошовий капітал
спочатку треба перетворити на ці знаряддя, отже, він мусить вийти а
першої стадії $Г — Т$ раніш, ніж матиме змогу функціонувати як продуктивний
капітал $П. Т$ак само зрозуміло в нашому прикладі, що сума
капітальної вартости в 422 ф. стерл., долучена до пряжі протягом продукційного
процесу, не може ввійти в фазу циркуляції $Т' — Г'$ як складова
частина вартости 10.000 ф. пряжі, поки ця пряжа не буде готова.
Не можна продати пряжі, поки її не напряли.

В загальній формулі продукт П, тобто продуктивного капіталу розглядається
як матеріяльна річ, відмінна від елементів продуктивного капіталу, як
предмет, що існує відокремлено від продукційного процесу й має споживну
форму, відмінну від елементів продукції. Його розглядається так завжди, коли
результат продукційного процесу виступає як річ, навіть і тоді, коли частина
продукту знову входить у відновлювану продукцію як її елемент. Так,
збіжжя придається на засів для зернової продукції; але продукт складається
лише із збіжжя, отже, має форму, відмінну від інших елементів, що
їх разом застосувалось — робочої сили, інструментів, добрива. Але є
самостійні галузі індустрії, де продукт продукційного процесу не є
новий речевий продукт, не є товар. З них економічно важлива лише
промисловість комунікаційна, хоч буде то промисловість власне транспортова
для товарів і людей, хоч для пересилання просто повідомлень,
листів, телеграм тощо.

А. Чупров\footnote{
А. Чупров „Железнодорожное хозяйство“. Москва, 1875, стор. 75, 76.
} каже про це так: „Фабрикант може спочатку спродукувати
предмети, а потім шукати для них споживачів“ [його продукт, виштовхнутий
готовим з продукційного процесу, переходить у циркуляцію як
відокремлений від нього товар]. „Таким чином продукція і споживання є
два акти, відокремлені в просторі й часі. У транспортовій промисловості,
що не утворює нових продуктів, а тільки переміщує людей і речі, ці
обидва акти зливаються; послуги“ [переміщення] „повинні споживатися в
той самий момент, коли їх продукується. Тому район, де залізничні шляхи
можуть шукати споживачів, поширюється найбільш на 50 верстов
(53 кілометри) в обидва боки“.

Результат — все одно, чи перевозять людей, чи товари — є зміна перебування,
напр., пряжа тепер перебуває в Індії, а не в Англії, де її спродуковано.
Але транспортова промисловість продає саме переміщення. Корисний
ефект, що його вона дає, нерозривно сполучається з процесом транспорту,
тобто з продукційним процесом транспортової промисловости.
Люди й товари їдуть разом з засобами транспорту, і ця їхня їзда, цей
рух у просторі, і є продукційний процес, здійснюваний цими засобами. Корисний
ефект можна споживати тільки протягом продукційного процесу;
цей ефект не існує як відмінна рід цього процесу річ споживання,
що, лише бувши спродукована, фігурує як предмет торговлі,
циркулює як товар. Але мінова вартість цього корисного ефекту, як і
\parbreak{}  %% абзац продовжується на наступній сторінці
