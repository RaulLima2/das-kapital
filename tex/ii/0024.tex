Дві форми, що їх набирає капітальна вартість у своїх стадіях циркуляції,
є форми грошового капіталу й товарового капіталу, її форма,
що належить до стадії продукції, є форма продуктивного капіталу.
Капітал, то протягом цілого свого кругобігу набирає, а потім скидає ці
форми і в кожній з них виконує відповідну їй функцію, є промисловий
капітал — промисловий тут у тому значінні, що він охоплює кожну
галузь продукції, провадженої капіталістично.

Отже, грошовий капітал, товаровий капітал, продуктивний капітал
означають тут не самостійні ґатунки капіталу, що їхні функції являють
зміст теж самостійних і відокремлених одна від однієї галузей підприємств.
Вони означають тут лише особливі функціональні форми промислового
капіталу, що послідовно набирає їх усі три одну по одній.

Кругобіг капіталу відбувається нормально лише доти, доки різні фази
його без затримок переходять одна в одну. Коли капітал затримується
на першій фазі Г — Т, то грошовий капітал затвердіває в скарб;
коли на продукційній фазі — то на одному боці лежать засоби продукції
не функціонуючи, тимчасом як на другому боці робоча сила лишається
незайнятою, коли на останній фазі Т' — Г', то нерозпродані товари
скупчуються й захаращують перебіг циркуляції.

З другого боку, з самої суті справи, самий кругобіг зумовлює фіксацію
капіталу на певний строк в окремих фазах кругобігу. В кожній із
своїх фаз промисловий капітал зв’язаний з якоюбудь певною формою,
як грошовий капітал, продуктивний капітал, товаровий капітал. Тільки
після того як він виконає функцію, що відповідає тій формі, яку він
має кожного разу, він набирає форму, що в ній може ввійти в нову
фазу перетворень. Щоб унаочнити це, ми припустили в нашому прикладі,
що капітальна вартість товарової маси, утвореної на продукційній стадії,
дорівнює всій сумі вартости, первісно авансованої в грошовій формі,
інакше кажучи, що вся капітальна вартість, авансована як гроші, одним
заходом увіходить з однієї стадії в наступну. Але ми бачили (книга І,
розділ VI), що частина сталого капіталу, власне знаряддя праці (прим.,
машини), в більшому або меншому числі повторюваних процесів продукції
придаються знову й знову, а тому лише частинами передають свою вартість
продуктові. Далі виявиться, як ця обставина відмінює процес кругобігу
капіталу. Тут обмежимось ось чим: у нашому прикладі вартість
продуктивного капіталу, 422 ф. стерл., має в собі лише пересічно обчислене
зношування робітних будівель, машин тощо, отже, тільки ту частину
вартости, що її вони підчас перетворення 10.000 ф. бавовни на 10.000 ф.
пряжі переносять на цю останню, на продукт тижневого процесу прядіння
протягом 60 годин. Тому в засобах продукції, що на них перетворюється
авансований сталий капітал в 372 ф. стерл., знаряддя праці, будівлі,
машини тощо також фігурують так, ніби їх брали на прокат на ринку,
оплачуючи тижневими ратами. Однак, це аніскільки не змінює справи.
Досить буде нам кількість пряжі в 10.000 ф., спродуковану протягом
тижня, помножити на число тижнів, що складають ряд років, — і вся
вартість куплених і спожитих протягом цього часу знарядь праці пере-
