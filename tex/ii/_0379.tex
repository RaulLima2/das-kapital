\parcont{}  %% абзац починається на попередній сторінці
\index{ii}{0379}  %% посилання на сторінку оригінального видання
бітниками. Вони продають їм усі товари дорожче, напр., на 20\%. Тут можливі два випадки. Нероби, крім
тих 100 ф. стерл., що їх вони одержують щорічно від промисловців, мають ще інші грошові засоби, або
не мають їх. В першому разі промисловці продають їм свої товари вартістю в 100 ф. стерл. за ціну,
прим., в 120 ф. стерл. Отже, через продаж їхніх товарів до них повертаються назад не лише ті 100 ф.
стерл., що їх вони заплатили неробам, але, крім того, ще 20 ф. стерл., які справді являють для них
нову вартість. Як же тепер стоїть справа з розрахунком? Вони дурно віддали товарів на 100 ф. стерл.,
бо ті 100 ф. стерл. грішми, що ними їм виплатили частину суми, були їхні власні гроші. Отже, їхній
власний товар їм оплачено їхніми власними грішми. Отже, 100 ф. стерл. втрати. Але вони одержали,
крім того, 20 ф. стерл. як надлишок ціни над вартістю. Отже, 20 ф. стерл. бариша; при 100 ф. стерл.
втрат, це становить 80 ф. стерл. втрат; завжди лишається мінус, ніколи не буде плюса. Шахрайство,
спричинене неробам, зменшило втрати промисловців, але від цього втрата багатства не перетворилась
для них на засіб до збагачення. Але такий метод не можна вживати протягом довгого часу, бо нероби не
можуть щороку платити 120 ф. стерл. грішми, одержуючи щороку лише 100 ф. стерл. грішми.

Тоді маємо другий спосіб: промисловці продають товарів вартістю в 80 ф. стерл. за ті 100 ф. стерл.
грішми, що їх вони заплатили неробам. В цьому разі, як і раніш, вони дурно віддають 80 ф. стерл.
грішми у формі ренти, проценту й т. ін. Таким шахрайством вони зменшили свою данину неробам, але
вона лишається, як і раніше, і, згідно з тією самою теорією, що ціни залежать від доброї волі
продавців, нероби можуть у майбутньому вимагати 120 ф. стерл. ренти, процентів і т. ін. за свою
землю та капітал, а не 100 ф. стерл., як до цього часу.

Цей блискучий дослід цілком гідний глибокого мислителя, який на одній сторінці списує в А. Смійа. що
„праця є джерело всякого багатства“ (стор. 242), що промислові капіталісти „застосовують свій
капітал, щоб оплачувати працю, яка репродукує капітал з зиском“ (стор. 246), а на другій сторінці
робить висновок, що ці промислові капіталісти „годують
усіх інших людей, що тільки вони збільшують суспільне майно і утворюють усі засоби для нашого
споживання“ (стор. 242), що не робітники годують капіталістів, а капіталісти робітників, і це з тієї
чудової причини, що гроші, якими оплачується робітників, не лишаються в їхніх руках, а постійно
повертаються назад до капіталістів як плата за спродуковані робітниками товари. „Вони одержують лише
однією рукою, а другою віддають назад. Отже, їх споживання треба розглядати як породжене тими, хто
платить їм утримання"\dots{} (стор. 235).

Після такого вичерпного опису суспільної репродукції та споживання, як вони відбуваються за
посередництвом грошової циркуляції, Детю каже далі: „Ось чим поповнюється це perpetuum mobile\footnote*{Прилад, що безупинно рухається без впливу зовнішньої движної сили. \emph{Ред.}}
багатства, рух.
\parbreak{}  %% абзац продовжується на наступній сторінці
