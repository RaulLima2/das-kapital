Розділ дев’ятнадцятий\footnote{
Тут починається рукопис VIII.
}

Давніші уявлення про предмет
І. Фізіократи

Кене в Tableau économique кількома широкими рисами показує, як
річний продукт національної продукції певної вартости розподіляється через
циркуляцію так, що, за інших незмінних обставин, може відбуватися проста
репродукція цього продукту, тобто репродукція в попередньому маштабі.
За вихідний пункт періоду продукції є по суті врожай останнього року.
Незчисленні індивідуальні акти циркуляції тут з самого початку об’єднуються
в характеристично-суспільний масовий рух, — в циркуляцію між
великими, функціонально визначеними економічними клясами суспільства.
Нас тут цікавить ось що: частина сукупного продукту — що, як і всяка
інша частина його, як предмет споживання, являє новий результат праці
минулого року — є разом з тим лише носій старої капітальної вартости,
що знову з’являється в попередній натуральній формі. Вона не циркулює,
а лишається в руках її продуцентів, кляси фармерів, щоб там знову почати
служити їм як капітал. До цієї сталої частини річного продукту
Кене залічує також неналежні сюди елементи, але він схоплює суть
справи завдяки обмеженості свого кругогляду, що за ним хліборобство е
та єдина сфера застосовання людської праці, яка продукує додаткову
вартість, тобто з капіталістичного погляду, єдина справді продуктивна
сфера застосовання праці. Економічний процес репродукції, хоч який
буде її специфічно-суспільний характер, завжди переплітається в цій
галузі (в хліборобстві) з природним процесом репродукції. Цілком очевидні
умови цього останнього пояснюють умови першого й не припускають
до хибних висновків, що до них призводить марево циркуляції.
Етикетка системи відрізняється від етикетки інших товарів, між
іншим, тим, що вона обдурює не лише покупця, а часто і продавця.
Сам Кене та його ближчі учні вірили в свою февдальну вивіску. Так
само вірять у неї досі наші шкільні вчені. А в дійсності система фізіократів
є перша систематична концепція капіталістичної продукції. Представник
промислового капіталу — кляса фармерів — є керівник цілого
економічного руху. Хліборобство провадиться капіталістично, тобто як
підприємство капіталістичного фармера в широкому маштабі; безпосередній
обробник землі є найманий робітник. Продукція створює не лише
предмети споживання, а й вартість їхню; але рушійним мотивом продукції
є здобування додаткової вартости, що місце її зародження є
сфера продукції, а не сфера циркуляції. З тих трьох кляс, що фігурують
як носії суспільного процесу репродукції, упосереднюваного циркуляцією
безпосередній визискувач „продуктивної“ праці, продуцент додаткової