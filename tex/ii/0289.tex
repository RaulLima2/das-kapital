двоїстого характеру самої праці: праці, оскільки вона як витрата робочої сили утворює вартість, і
оскільки вона як конкретна корисна праця утворює предмети споживання (споживну вартість). Загальна
сума виготовлених протягом року товарів, тобто ввесь річний продукт, є продукт корисної праці, яка
діяла протягом останнього року; всі ці
товари існують лише в наслідок того, що суспільно застосовану працю витрачено в різноманітно
розгалуженій системі різних видів корисної праці; тільки тому в їхній сукупній вартості вартість
засобів продукції, зужиткована на їх продукцію, збереглася, знову з’явившись в новій натуральній
формі. Отже, ввесь річний продукт є результат корисної праці, витраченої протягом року. Але протягом
року утворюється знову лише деяка частина вартости річного продукту; ця частина є новоспродукована
річна вартість, що в ній втілено суму праці, пущеної в рух протягом даного року.

Отже, коли А. Сміс у щойно цитованому місці каже: „Річна праця кожної нації є той фонд, що первісно
дає їй усі засоби існування, які вона споживає протягом року і т. ін.“, то він однобічно стає на
погляд просто корисної праці, яка, щоправда, надала всім цим засобам існування форму придатну для
споживання. Але він забуває при цьому, що це
було б неможливо без участи засобів праці й предметів праці, переданих від минулих років, і що в
наслідок цього „річна праця“, оскільки вона утворювала вартість, ні в якому разі не утворила всієї
вартости виготовленого нею продукту; він забуває, що новопродукована вартість менша, ніж вартість
продукту.

Коли А. Смісові й не можна закинути, що в цій аналізі він ішов лише до тих меж, як і всі його
наслідувачі (хоч спробу правильно розв’язати питання маємо вже в фізіократів), то все ж треба
сказати, що далі він губиться в хаосі, головним чином, тому, що „езотеричне“ розуміння товарової
вартости в нього взагалі завжди переплітається з екзотеричним, а це останнє в нього здебільша й
переважає, тимчасом як його науковий інстинкт час від часу знову й знов приводить його до
езотеричного погляду.

4) Капітал і дохід у А. Сміса

Частина вартости кожного товару (а тому й річного продукту), яка становить лише еквівалент
заробітної плати, дорівнює капіталові, авансованому капіталістом на заробітну плату, тобто дорівнює
змінній складовій частині цілого авансованого ним капіталу. Цю складову частину авансованої
капітальної вартости капіталіст одержує назад через новоспродуковану складову частину вартости
товару, виробленого найманими робітниками. Хоч авансується змінний капітал в тому розумінні, що
капіталіст виплачує грішми ту частину ще неготового для продажу продукту, яка припадає робітникові,
або хоч готового, але ще не проданого капіталістом; хоч платить він робітникові грішми, вже
одержаними від продажу виробленого робітниками товару, хоч він за допомогою кредиту анти-
