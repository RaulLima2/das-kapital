\parcont{}  %% абзац починається на попередній сторінці
\index{ii}{0085}  %% посилання на сторінку оригінального видання
меншу віддаль від місця його продукції його можна перемістити, то
вужча просторова сфера його циркуляції, то більш місцеве значення має
ринок його збуту. Тому, що нетриваліший товар, що вужчі в наслідок
його фізичних властивостей абсолютні межі його обігу як товару, то
менш здатний він бути за предмет капіталістичної продукції. Такий
товар капіталістична продукція може захопити лише в дуже залюднених
місцях, або в міру того як розвиток транспортових засобів скорочує
віддаль. А концентрація продукції якогобудь товару в небагатьох руках
та в місцевостях, густо залюднених, може утворити відносно великий ринок
і для таких товарів, як, напр., продукти великих броварень, молочних
фарм тощо.

\section{Витрати циркуляції}

\subsection{Чисті витрати циркуляції}

\subsubsection{Час купівлі й продажу}

Перетворення форм капіталу з товару на гроші й з грошей на
товар є разом з тим операції капіталіста, акти купівлі й продажу. Час,
що протягом його відбуваються ці перетворення форм капіталу, суб’єктивно
з погляду капіталіста, є час продажу й час купівлі, час, що протягом
його капіталіст функціонує на ринку як продавець і покупець. Так само,
як час обігу капіталу становить неодмінну частину часу його репродукції,
так само час, що протягом його капіталіст купує і продає, гасаючи на
ринку, становить неодмінну частину того часу, коли він функціонує як
капіталіст, тобто як персоніфікований капітал. Цей час становить частину
його ділового часу.

[А що ми припустили, що товари купується й продається за їхньою
вартістю, то при цих процесах справа йде лише про перетворення тієї
самої вартости з однієї форми на іншу, з товарової форми на грошову
і з грошової форми на товарову, — справа йде лише про зміну стану.
Коли товари продається за їхньою вартістю, то величина вартости
лишається незмінна так у руках покупця, як і в руках продавця; змінюється
лише її форма буття. А коли товари продається не за їхньою вартістю,
то незмінна лишається сума перетворених вартостей: те, що є плюс на
одному боці, є мінус на другому.

Але метаморфози $Т — Г$ і $Г — Т$ є операції, що відбуваються між
покупцем і продавцем; їм потрібен час, щоб договоритись, то більше,
що тут відбувається боротьба, де кожна сторона хоче взяти гору над
іншою, де стоять один проти одного ділки, a „when greek meets greek
then comes the tug of war“\footnote*{
Коли з греком грек зустрінеться, доходиться жорстокої війни. \emph{Ред.}
}. Зміна стану коштує часу та робочої сили.
\parbreak{}  %% абзац продовжується на наступній сторінці
