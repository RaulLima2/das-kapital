\parcont{}  %% абзац починається на попередній сторінці
\index{ii}{0168}  %% посилання на сторінку оригінального видання
з природи своєї подільні, і на ті, що потребують для виготовлення порівняно
довшого зв’язного періоду. В одному випадку по сьогоднішній
продукції певної кількости пряжі, вугілля тощо, завтра не наступає нової
продукції пряжі, вугілля та ін. Інша справа з кораблями, будівлями, залізницями
тощо. Тут не тільки робота припиняється, але припиняється
і зв’язний акт продукції. Коли будову не будується далі, то зужиті на
неї засоби продукції та працю витрачено марно. Навіть коли будову знову
розпочнуть будувати, то в проміжний період вона завжди зазнає ушкодження.

Протягом цілого робочого періоду нагромаджується, наверствовуючись,
та частина вартости, що її основний капітал щоденно віддає продуктові,
поки останній цілком достигне. І тут разом з тим виявляється практична
важливість ріжниці між основним та обіговим капіталом. Основний капітал
авансується на порівняно довший час на процес продукції, його не
доводиться поновлювати раніше, ніж мине цей, може кількарічний період.
Чи переносить парова машина свою вартість щоденно частинами на пряжу,
продукт подільного робочого процесу, чи вона протягом трьох місяців
віддає її паровозові, продуктові безперервного продукційного акту, —
ця обставина абсолютно нічого не змінює у витраті капіталу, потрібного
на закуп парової машини. В одному разі її вартість припливає назад маленькими
частками, напр., щотижня, а в другому разі — більшими масами,
напр., що три місяці. Але в обох випадках парову машину поновлюється
лише один раз, може, лише через двадцять років. Доки кожен окремий
період, що протягом його вартість парової машини через продаж продукту
припливає частинами назад, коротший, ніж період її власного існування,
та сама машина функціонує й далі в процесі продукції протягом
кількох робочих періодів.

Інакше стоїть справа з обіговими складовими частинами авансованого
капіталу. Робочу силу, куплену на цей тиждень, витрачено протягом
цього тижня, і вона зречевилась у продукті. Її треба оплатити наприкінці
цього тижня. І така витрата капіталу на робочу силу щотижня повторюється
протягом трьох місяців, так що витрата частини капіталу на
один тиждень не забезпечує капіталіста проти закупу робочої сили на
наступний тиждень. На оплату робочої сили щотижня треба витрачати
новий додатковий капітал, і, якщо залишити осторонь всі кредитові відносини,
капіталіст мусить мати спроможність витрачати заробітну плату
протягом трьох місяців, хоч він і виплачує її щотижневими порціями. Так
само стоїть справа з іншими частинами обігового капіталу, сировинними
та допоміжними матеріялами. Один по одному шари праці накладаються на
продукт. Не лише вартість витраченої робочої сили, а й додаткову
вартість безперервно переноситься на продукт протягом процесу праці,
але на продукт, що ще неготовий, що не має ще форми готового товару,
а, значить, і не є ще здатний до циркуляції. Все це має силу й для капітальної
вартости, що переноситься на продукт шарами з сировинних та
допоміжних матеріялів.

Залежно від довшого або коротшого протягу робочого періоду, що
його потребує специфічна природа продукту на виготовлення його або
\parbreak{}  %% абзац продовжується на наступній сторінці
