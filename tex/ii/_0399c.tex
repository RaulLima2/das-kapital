\parcont{}  %% абзац починається на попередній сторінці
\index{ii}{0399}  %% посилання на сторінку оригінального видання
його змінний капітал зростає з 750 до 800. Це збільшення сталого, а
також і змінного капіталу II, загалом на 150, покривається з його додаткової
вартости; отже, з 750 II m тільки $600 m$ лишаються як споживний
фонд капіталістів II, що їхній річний продукт розподіляється тепер
так:

II. 1600 с + $800 v + 600m$ (споживний фонд) = 3000.    $150 m$,

спродуковані як засоби споживання й обмінені тут на (100 с + $50 v$) II,
в своїй натуральній формі цілком ідуть на споживання робітників: 100
споживають робітники І (100 I v), а 50 — робітники II (50 II v), як це
показано вище. В дійсності в II, де ввесь продукт його виготовляється
в формі, потрібній для акумуляції, більша на 150 частина додаткової
вартости мусить бути репродукована в формі доконечних засобів
споживання. Коли дійсно починається репродукція в поширеному
маштабі, то 100 змінного грошового капіталу І через руки робітничої
кляси І повертаються до II; навпаки, II передає $100 m$, як товаровий
запас, підрозділові І і разом з тим передає 50, як товаровий запас своїм
власним робітникам.

Розміщення, змінене з метою акумуляції, тепер є таке:

I.    4400 с + 1.$100 v$ + 500 фонд споживання = 6000

II.    1600 с + $800 v$ + 600 фонд споживання = 3000

Сума\dotfill 9000, як вище.

З цього маємо капіталу:

І. 4400 с + $1100 v$ (грішми) = 5500

II. 1600 с + $800 v$ (грішми) = 2400
= 7900, ’

тимчасом, як на початку продукції було:

І. 4000 с + $1000 v$ = 5000
II. 1500 с + $750 v$ = 2250
= 7250.

Коли справжня акумуляція відбувається тепер на цій основі, тобто,
коли продукцію дійсно провадять з таким збільшеним капіталом, то
наприкінці другого року матимемо:

І. 4400с + $1100 v + 1100 m$ = 6600

II. 1600 с + $800 v + 800 m$ = 3200
= 9.800.

Акумуляція в І триватиме далі в тій самій пропорції; отже, $550 m$
витрачатиметься як дохід, а $550 m$ акумулюватиметься. В такому разі насамперед
1100 I v заміститься черга 1100 ІІ с; далі ще треба реалізувати
550 I m в товарах II рівної вартости; отже, загалом 1.650 І
($v + m$). Але сталий капітал II, що його треба замістити, становить
тільки 1600, отже, решту 50 доводиться поповнити з 800 II m. Коли ми
\parbreak{}  %% абзац продовжується на наступній сторінці
