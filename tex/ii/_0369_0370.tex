\parcont{}  %% абзац починається на попередній сторінці
\index{ii}{0369}  %% посилання на сторінку оригінального видання
нього дорівнює нулеві. З цього неодмінно випливає, що грошей, потрібних
для циркуляції річного доходу, досить і для циркуляції сукупного
річного продукту; що, отже, в нашому випадку, грошей, потрібних для
циркуляції засобів споживання вартістю в 3.000, досить і для циркуляції
сукупного річного продукту вартістю в 9.000. Такий справді погляд
А. Сміса, і Т. Тук повторює його. Це хибне уявлення про відношення
маси грошей, потрібної, щоб перетворити дохід на гроші, до маси грошей,
потрібної для циркуляції сукупного суспільного продукту, є неминучий
результат незрозумілого, непродуманого уявлення про спосіб, що
ним репродукуються й щороку заміщуються різні речові й вартісні елементи
сукупного річного продукту. Тому його вже й збито.

Послухаймо самого Сміса й Тука.

Сміс каже (книга II, розділ 2): „Циркуляцію кожної країни можна
розподілити на дві частини: циркуляцію між самими торговцями й циркуляцію
між торговцями й споживачами. Хоч ті самі грошові одиниці, — паперові
або металеві, — можуть застосовуватись то в одній, то в другій
циркуляції, однак, і та й друга безупинно відбуваються одночасно одна поряд
однієї, і тому кожна з них потребує певної маси грошей того або
іншого роду, щоб і далі продовжувати свій рух. Вартість товарів, що циркулюють
між різними торговцями, ніколи не може перевищити вартости
товарів, що циркулюють між торговцями й споживачами; бо, хоч що купують
торговці, вони мусять усе це, кінець-кінцем, продати споживачам.
А що циркуляція між торговцями відбувається en gros\footnote*{
Гуртом, оптом. \emph{Ред.}
}, то вона взагалі потребує
досить великих сум для кожного поодинокого обміну. Навпаки, циркуляція
між торговцями й споживачами відбувається здебільша en détail\footnote*{
На роздріб. \emph{Ред.}
}
і часто потребує лише дуже незначних грошових сум; часто досить одного
шилінґа або навіть половини пенні. Але невеличкі суми циркулюють
куди швидше, ніж великі\dots{} Тому, хоч річні закупи всіх споживачів принаймні
(чудове це „принаймні“) дорівнюють вартістю закупам усіх
торговців, однак, їх звичайно можна переводити куди меншою масою
грошей“ і т. ін.

До цього місця Адама Т. Тук („Аn Inquiry into the Currency Principle.
London, 1844“, crop. 34--36 passim) зауважує: „Не викликає жодного
сумніву, що ця подана тут ріжниця в суті правильна\dots{} Обмін між торговцями
й споживачами охоплює також і виплату заробітної плати, яка
являє головний дохід (the principal means) споживачів\dots{} Всі обміни між
торговцями, тобто всі продажі, починаючи від продуцента або імпортера,
переходячи всі щаблі посередніх процесів мануфактури й т. інш. і закінчуючи
роздрібним торговцем або купцем-експортером, можна звести
до рухів переміщення капіталу. Але переміщення капіталу не мають собі
зa неодмінну передумову й на практиці справді при більшості обмінів не
призводять до того, щоб підчас переміщення справді передавалось банкноти
\index{ii}{0370}  %% посилання на сторінку оригінального видання
або монети, — я маю на увазі матеріяльну, а не фіктивну передачу\dots{}
Загальна сума взаємних обмінів між торговцями мусить, кінець-кінцем,
визначатись і обмежуватись сумою обмінів між торговцями й споживачами“.

Коли б у Тука остання теза була висловлена відокремлено, то можна
було б думати, що він просто констатує, що є співвідношення між обмінами
поміж самими торговцями і обмінами поміж торговцями й споживачами, —
інакше кажучи, співвідношення між вартістю сукупного річного доходу й
вартістю капіталу, що за допомогою його продукується дохід. Однак це
не так. Він прямо пристає на погляд А. Сміса. Тому зайве було б
критикувати зокрема його теорію циркуляції.

2) Кожен промисловий капітал при відкритті підприємства одним заходом
подає в циркуляцію гроші на всю свою основну складову частину, яку
він змову вилучає лише поступінно, протягом ряду років, продаючи свій
річний продукт. Отже, спочатку він подає в циркуляцію більше грошей,
ніж вилучає з неї. Це повторюється кожного разу при відновленні цілого
капіталу in natura; не повторюється щороку для певного числа підприємств,
що їхні основні капітали доводиться відновлювати in natura; частинно
це повторюється при кожному ремонті, при кожному лише частинному
відновленні основного капіталу. Отже, коли одна сторона
вилучає з циркуляції більше грошей, ніж подає в неї, то друга сторона —
навпаки.

В усіх галузях промисловости, де період продукції (як величина
відмінна від робочого періоду) охоплює порівняно довгий час, капіталістичні
продуценти протягом цього періоду ввесь час подають гроші в циркуляцію,
— почасти на оплату застосованої робочої сили, почасти на закуп
засобів продукції, що їх треба застосувати; таким чином, засоби продукції
безпосередньо вилучаються з ринку, а засоби споживання почасти посередньо
через робітників, які витрачають свою заробітну плату, почасти
безпосередньо самими капіталістами, які зовсім не відкладають свого
споживання, і при цьому ці капіталісти спочатку не подають на ринок
жодного еквіваленту товарами. Гроші, що їх вони подають в циркуляцію,
протягом цього періоду служать для перетворення на гроші товарової
вартости, а втім і вміщеної в ній додаткової вартости. Дуже важливий
стає цей момент при розвиненій капіталістичній продукції, в довгочасних
підприємствах, що їх засновують акційні товариства і т. ін., як
от будування залізниць, каналів, доків, великих міських споруд, залізних
пароплавів, дренування ґрунту в широких розмірах і т. ін.

3) Тимчасом як інші капіталісти, — лишаючи осторонь витрати на
основний капітал, — вилучають з циркуляції більше грошей, ніж подали в
неї, купуючи робочу силу та обігові елементи, капіталісти, що продукують
золото й срібло, — лишаючи осторонь благородний металь, що служить як
сировинний матеріял, — подають в циркуляцію тільки гроші, а вилучають
з неї тільки товари. Сталий капітал, за винятком зношеної частини,
більшу частину змінного капіталу й усю додаткову вартість, за винятком
скарбу, який, можливо, нагромаджується в їхніх власних руках, — усе це
як гроші подається в циркуляцію.
