періоду. Так в дійсності й є. Наприкінці 6-го тижня продукт вартістю
в 600 ф. стерл. входить у циркуляцію і наприкінці 9-го тижня повертається
назад в грошовій формі. Разом з тим на початку 7-го тижня
входить у роботу капітал II і покриває потреби наступного робочого
періоду протягом тижнів 7—9. Але, згідно з нашим припущенням, наприкінці
9 тижня робочий період пророблено лише на половину. Отже, на
початку 10-го тижня знову входить у роботу капітал І в 600 ф. стерл.,.
який шойно повернувся назад, і своїми 300 ф. стерл. він покриває авансування,
потрібні для тижнів 10—12. Цим завершується другий робочий
період. В циркуляції є продукт вартістю в 600 ф. стерл., і повертаються
вони назад наприкінці 15-го тижня; але, крім того, є 300 ф. стерл.
вільних — величина первісного капіталу II, і можуть вони функціонувати
в першу половину наступного робочого періоду, отже, протягом тижнів
13—15. Коли минуть вони, знову повертаються назад 600 ф. стерл.;
з них 300 ф. стерл. вистачить до кінця цього робочого періоду, а
300 ф. стерл. лишаються вільні для наступного.

Отже, справа перебігає так:

І. Період обороту: тижні\footnote{
. Робочий період: тижні 1—6. Функціонує капітал І, 600 ф. стерл.
}—9.

1. Період циркуляції: тижні 7—9. Наприкінці 9-го тижня повертаються
назад 600 ф. стерл.

II. Період обороту: тижні 7—15.\footnote{
. Період циркуляції: тижні 13—15. Наприкінці 15-го тижня повертаються
назад в грошовій формі 600 ф. стерл. (складені наполовину
з капіталу І, наполовину з капіталу II).

III. Період обороту: тижні 13—21.
}. Робочий період: тижні 7—12.

Перша половина: тижні 7—9. Функціонує капітал II, 300 ф. стерл.
Наприкінці 9 тижня повертаються назад 600 ф. стерл. в грошовій формі
(капітал І).

Доуга половина: тижні 10—12. Функціонують 300 ф. стерл. капіталу
І. Решта 300 ф. стерл. капіталу І лишаються вільні.\footnote{
. Період циркуляції: тижні 19—21, що наприкінці їх знову зворотно
припливають 600 ф. стерл. в грошовій формі; в цих 600 ф. стерл.
капітал І і капітал II тепер злито так, що їх годі відрізнити один од одного.

Таким чином, до кінця 51-го тижня відбувається вісім повних оборотів
капіталу в 600 ф. стерл. (І: тижні 1—9; II: 7—15; III: 13—21;
IV: 19—27; V: 25—33; VI: 31—39; VII: 37—45; VIII: тижні
43—51). А що тижні 49—51 припадають на восьмий період цирку-
}. Робочий період: тижні 13—18.

Перша половина: тижні 13—15. Вільні 300 ф. стерл. входять у
роботу. Наприкінці 15-го тижня повертаються назад 600 ф. стерл. в грошовій
формі.

Друга половина: тижні 16—18. З 600 ф. стерл., що повернулись,
функціонують 300 ф. стерл., а решта 300 ф. стерл. знову лишаються
вільні.