ву таки сплутує поточний капітал і товаровий капітал. Продукт є речовий
носій товарового капіталу. Але, звичайно, лише та частина продукту,
яка справді входить в циркуляцію і не входить знову безпосередньо в
той самий процес продукції, відки вона вийшла, як продукт.

Чи береться зерно, як частина, безпосередньо з продукту, чи продається
ввесь продукт і частину його вартости за допомогою купівлі перетворюється
на чуже зерно, і в тому і в другому випадку маємо лише заміщення
вартости, і цим заміщенням не утворюється жодного зиску. В
першому випадку зерно разом з рештою продукту входить як товар в
циркуляцію, а в другому випадку воно фігурує лише в бухгальтерії як
складова частина вартости авансованого капіталу. Але в обох випадках
воно лишається поточною складовою частиною продуктивного капіталу.
Його зуживається цілком, щоб виготовити продукт, і воно мусить цілком
заміститися з нього, щоб уможливити репродукцію.

„Сировинні й допоміжні матеріяли втрачають ту самостійну форму, в
якій вони ввійшли в процес праці як споживні вартості. Інша справа
з власне засобами праці. Інструмент, машина, фабричний будинок, посуд
і т. ін. служать у процесі праці лише доти, доки зберігають вони
свою первісну форму, доки й завтра можуть вони входити в процес праці
у тій самій формі, що й учора. І як за свого життя, тобто протягом
процесу праці, вони зберігають проти продукту свою самостійну форму,
так само зберігають її вони й після своєї смерти. Трупи машин, майстерень,
фабричних будівель і далі все ще існують самостійно, окремо
від продуктів, творенню яких вони допомагали“. (Капітал, кн. І, розд. VI).

Ці різні способи застосування засобів продукції для створення продукту,
— при чому одні засоби продукції зберігають свою самостійну форму
проти продукту, а інші змінюють або цілком втрачають її, — цю ріжницю,
властиву процесові праці, як такому, отже, ріжницю, що так само
властива й такому процесові праці, що має на меті задовольнити лише
власні потреби, прим., патріярхальної сім’ї, без якогобудь обміну, без
товарової продукції, — А. Сміс освітлює неправильно, бо: 1) він притягує
зовсім неналежне сюди визначення зиску: що одні засоби продукції дають
своєму власникові зиск, зберігаючи свою форму, а інші дають зиск,
втрачаючи її; 2) зміни частини елементів продукції в процесі праці він
сплутує з тією зміною форми, яка властива обмінові продуктів, товаровій
циркуляції (купівлі та продажеві), і яка разом з тим включає зміну
власности на товари, що циркулюють.

Оборот має собі за передумову репродукцію, упосереднювану циркуляцією,
тобто продажем продукту, перетворенням його на гроші і зворотним
перетворенням з грошей на елементи його продукції. Але оскільки
капіталістичному підприємцеві частина його власного продукту безпосередньо
сама знову служить як засіб продукції, продуцент виступає як
продавець того самого продукту самому собі і саме так фігурує ця операція
в його бухгальтерії. Отже, ця частина репродукції не упосереднюється
циркуляцією, а відбувається безпосередньо. Але та частина продукту,
що таким чином знову служить як засіб продукції, заміщує
