лось Т' — Г' — Т, потрібна справжня репродукція того Т, що на нього
перетворюється Т'; а ця репродукція зумовлена процесами репродукції,
які є поза процесом репродукції індивідуального капіталу, що його
репрезентує Т'.

У формі I акт Г — Т Р Зп підготовлює лише перше перетворення
грошового капіталу на продуктивний капітал; у формі II він підготовлює
зворотне перетворення з товарового капіталу на продуктивний капітал;
отже, оскільки вкладування промислового капіталу лишається те саме,
він підготовляє зворотне перетворення товарового капіталу на ті самі
елементи продукції, що з них він постав. Тому тут, як і в формі I, цей
акт з’являється як підготовча фаза продукційного процесу, але як
поворот до нього, як відновлення його, а тому — як предтеча процесу
репродукції, отже, також повторення процесу зростання вартости.

Треба тут ще раз зауважити, що акт Г — Р є не простий товарообмін,
а купівля товару Р, який повинен служити для продукції додаткової
вартости так само, як Г — Зп є лише процедура, матеріяльно неминуча
для здійснення цієї мети.

Після того як відбувся акт Г — Т Р Зп, Г перетворюється знову
на продуктивний капітал, на П, і знову починає кругобіг.

Отже, розгорнута форма кругобігу П... Т' — Г' — Т... П така:

П … Т' (Т + т) — — (Г + г) — Т Р Зп … П. — т

Перетворення грошового капіталу на продуктивний капітал є купівля
товарів для продукції товарів. Лише оскільки споживання є продуктивне
споживання, воно входить у кругобіг самого капіталу; умова цього споживання
та, що за посередництвом споживаних у такий спосіб товарів утворюється
додаткова вартість. І це є щось дуже відмінне від тієї продукції,
і навіть від товарової продукції, що її мета — існування продуцента;
отаке зумовлене продукцією додаткової вартости заміщення товару
товаром є щось цілком відмінне від обміну продуктів самого по собі,
обміну, що лише упосереднюється грішми. Але так розглядають справу
економісти, щоб довести, що неможлива будь-яка перепродукція.

Крім продуктивного споживання Г, що перетворюється на Р і Зп,
кругобіг містить у собі перший член Г — Р, що для робітника є Р — Г =
Т — Г. З циркуляції робітника Р — Г — Т, що має в собі його споживання,
в кругобіг капіталу ввіходить лише перший член як результат
Г — Р. Другий акт, а саме Г — Т, не входить у циркуляцію індивіду-
