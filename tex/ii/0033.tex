процес циркуляції промислового капіталу, ввесь його рух у межах фази
циркуляції становить лише перерву, а, значить, лише посередню ланку між
продуктивним капіталом, що як перший крайній член розпочинає кругобіг
і як останній закінчує його в тій самій формі, тобто в формі, що в ній
може відновитись. Власне циркуляція виступає лише як посередня ланка
періодично поновлюваної і через це поновлення безупинної репродукції.

По-друге. Ціла циркуляція з'являється тут у формі, цілком протилежній
тій, що її вона має в кругобігу грошового капіталу. Там ця форма,
лишаючи осторонь визначення вартости, була така: Г — Т — Г (Г — Т.
Т — Г); а тут, знову таки, лишаючи осторонь визначення вартости, вона
така: Т — Г — Т (Т — Г. Г — Т), отже, форма простої товарової циркуляції.

І. Проста репродукція

Розгляньмо насамперед процес Т' — Г' — Т, що відбувається в сфері
циркуляції між крайніми членами П... П.

Вихідний пункт цієї циркуляції є товаровий капітал: Т' = Т + т
= П + т. Функцію товарового капіталу Т' — Г' (реалізація капітальної вартости,
що міститься в ньому, = П, що існує тепер як складова частина
товару Т, а також реалізація додаткової вартости, яка міститься в
ньому, та, маючи вартість т, існує як складова частина тієї таки
товарової маси) розглянули ми в першій формі кругобігу. Але там становила
вона другу фазу перерваної циркуляції і кінцеву фазу цілого кругобігу.
Тут вона становить другу фазу кругобігу, але першу фазу циркуляції.
Перший кругобіг закінчується Г', а що Г' так само, як і первісне
Г, може знову як грошовий капітал почати другий кругобіг, то спочатку
не було потреби розглядати, чи продовжуватимуть Г і г (додаткова
вартість), що містяться в Г', свій шлях спільно, чи кожне з них перебігатиме
свій відмінний шлях. Це було б потрібно зробити лише тоді,
коли б ми простежили перший кругобіг у його дальшому відновленні.
Але в кругобігу продуктивного капіталу цей пункт мусить бути розв’язаний,
бо від цього розв’язання залежить уже визначення його першого
кругобігу, а також і тому, що Т' — Г' є в ньому перша фаза циркуляції,
що її треба доповнити через Г — 'Г. Від цього розв’язання залежить,
чи позначає формула просту репродукцію, чи репродукцію в поширеному
маштабі. Отже, залежно від цього розв’язання змінюється й характер
кругобігу.

Отже, візьмімо спочатку просту репродукцію продуктивного капіталу
і при цьому, як і в першому розділі, припустімо, що обставини лишаються
незмінні і товари купується й продається за їхньою вартістю. Ціла
додаткова вартість при такому припущенні ввіходить у сферу особистого
споживання капіталіста. Скоро товаровий капітал Т' перетворився на
гроші, то частина грошової суми, що репрезентує капітальну вартість,
циркулює далі в кругобігу промислового капіталу; друга частина, перетворена
на золото додаткова вартість, увіходить у загальну товарову
циркуляцію та являє собою грошову циркуляцію, що виходить від капі-
