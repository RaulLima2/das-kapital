творюється задля репродукції робочої сили змінний капітал, після того як
його виплачено робітникам), хоч яке буде їхнє походження, суспільна
форма продукційного процесу, що з нього вони походять, — протистоять
вже самому промисловому капіталові в формі товарового капіталу, в формі
товарово-торговельного або купецького капіталу, а цей останній з самої
природи своєї охоплює товари всяких способів продукції.

Так само, як капіталістичний спосіб продукції припускає широкі розміри
продукції, так само неминуче припускає він широкі розміри продажу;
отже, продаж купцеві, а не поодинокому споживачеві. Оскільки цей
споживач сам є продуктивний споживач, тобто промисловий капіталіст,
оскільки, отже, промисловий капітал однієї галузі дає засоби продукції
для другої галузі, остільки (в формі замовлення тощо) відбувається
також безпосередній продаж товарів одного промислового капіталіста
багатьом іншим. В цьому розумінні кожен промисловий капіталіст є безпосередній
продавець, сам для себе купець, що ним він є між іншим і
продаючи товар купцеві.

Товарова торговля як функція купецького капіталу припускається
капіталістичною продукцією і чимраз більше розвивається з розвитком
цієї продукції. Отже, принагідно для ілюстрації окремих боків капіталістичного
процесу циркуляції, ми припускаємо наявність товарової торговлі,
а в загальній аналізі капіталістичного процесу циркуляції ми
припускаємо безпосередній продаж без втручання купця, бо це останнє
затемнює різні моменти руху.

Звернімось до Сісмонді, який дещо наївно освітлює цю справу:

„Торговля застосовує чималий капітал, що на перший погляд,
здається, не становить жодної частини того капіталу, що його перебіг
ми розглянули в подробицях. Вартість сукна, нагромадженого в крамницях
торговця сукном, як здається спочатку, є щось цілком відмінне
від тієї частини річної продукції, що її багатий дає бідному як заробітну
плату, щоб примусити його робити. Однак, цей капітал лише заміщує
той, що про нього ми казали. Щоб добре уявити собі розвиток багатства,
ми взяли його в момент його утворення і простежили аж до його
споживання. При цьому, капітал, вкладений, напр., у суконну мануфактуру,
здавався нам завжди тим самим; обмінений на дохід споживача, він
лише поділився на дві частини: одна у формі зиску була доходом фабриканта,
друга у формі заробітної плати — доходом робітників на час,
протягом якого вони виробляли нове сукно.

„Але скоро виявилось, що для всіх було б краще, коли б різні частини
цього капіталу заміщували одна одну, і коли б, якщо сотні тисяч екю
досить для всієї циркуляції між фабрикантом і споживачем, ці сто тисяч
екю рівномірно розподілились між фабрикантом, гуртовим торговцем і
роздрібним торговцем. Перший, маючи лише третину цієї суми, виробляв
стільки, скільки раніше, мавши цілу суму, бо в момент закінчення
своєї продукції, він находив торговця-покупця куди раніше, ніж
він знайшов би споживача. Так само і капітал гуртового торговця
куди швидше заміщено капіталом роздрібного торговця... Різність
