ному маштабі. Капіталістична продукція, як ми бачили, продукує не
лише товари і додаткову вартість; вона репродукує, і до того в чимраз
більшому розмірі, клясу найманих робітників і перетворює величезну
більшість безпосередніх продуцентів на найманих робітників. Тому
Г — Т... П... Т' — Г', маючи за першу передумову для свого перебігу
постійну наявність кляси найманих робітників, припускає вже наявність
капіталу в формі продуктивного капіталу, а тому й форму кругобігу
продуктивного капіталу.

II. Друга стадія. Функція продуктивного капіталу

Розглядуваний тут кругобіг капіталу починається з акту циркуляції
Г — Т, з перетворення грошей на товар, з купівлі. Отже, циркуляція
мусить бути доповнена протилежною метаморфозою Т — Г, перетворенням
товару на гроші, продажем. Але безпосередній наслідок акту
Г — Т Р Зп є перерва циркуляції капітальної вартости, авансованої в
грошовій формі. Що грошовий капітал перетворився на продуктивний
капітал, то капітальна вартість набула такої натуральної форми, що в
ній вона не може далі циркулювати, а мусить увійти в споживання, а
саме в продуктивне споживання. Споживання робочої сили, працю, можна
реалізувати лише в процесі праці. Капіталіст не може знову продати
робітника як товар, бо він не його раб, і купив він не що інше, як
користання з його робочої сили протягом певного часу. З другого боку,
він може скористатися з робочої сили, лише примусивши її використовувати
засоби продукції як товаротворчі елементи. Отже, наслідок першої
стадії це — перехід у другу, у продуктивну стадію капіталу.

Рух капіталу має вигляд Г — Т Р Зп... П, де крапки позначають, що
циркуляцію капіталу перервано, але процес його кругобігу триває далі,
бо із сфери товарової циркуляції він переходить до сфери продукції.
Отже, перша стадія, перетворення грошового капіталу на продуктивний
капітал, являє лише попередню і вступну фазу до другої стадії, до
функціонування продуктивного капіталу.

Г — Т Р Зп має собі за передумову, що індивідуум, який виконує цей
акт, не тільки володіє вартостями в першій-ліпшій споживній формі,
але володіє цими вартостями в грошовій формі, що він є власник грошей.
Але акт цей сходить саме на віддачу грошей, і тому той індивідуум
може лишитись власником грошей лише остільки, оскільки гроші
implicite\footnote*{
Дослівно: що приховано міститься. Тут у розумінні, що акт віддачі грошей приховано містить у собі 
зворотній приплив їх. Ред.
} зворотно припливають до нього в наслідок самого акту від-