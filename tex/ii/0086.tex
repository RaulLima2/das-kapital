але не для того, щоб утворити вартість, а для того, щоб перетворити
вартість з однієї форми на іншу, при чому справа не змінюється від
взаємних намагань при цій нагоді присвоїти собі надлишкову кількість
вартости. Ця робота, зловмисно перебільшувана обома сторонами, так само не
утворює вартости, як і робота, витрачена на судовому процесі, не
збільшує вартости предмету позову. З цією роботою, — яка є неодмінний
момент капіталістичного продукційного процесу в його сукупності, коли
він містить у собі циркуляцію або сам міститься в ній, — справа стоїть
так само, як, напр., з роботою горіння речовини, що нею користуються для
видобування тепла. Ця робота горіння не створює тепла, хоч вона є
неодмінний момент процесу горіння. Щоб, напр., зужити вугілля як
паливо, я мушу сполучити його з киснем і для цього перевести його з
твердого стану в газуватий (бо у вуглекислому газі, наслідку горіння,
вугілля перебуває в стані газуватому), отже, я мушу перевести зміну
фізичної форми буття або стану. Відділення молекуль вуглецю, сполучених
в одне тверде тіло, і розпад самих молекуль вуглецю на окремі атоми,
мусить відбутись раніш, ніж постане нова сполука, а для цього треба
прикласти деяку силу, що, отже, не перетворюється на тепло, а
відбирається від нього. Тому, коли товаровласники не є капіталісти, а
самостійні безпосередні продуценти, то час, витрачений від них на купівлю
й продаж, є одбава з їхнього робочого часу. Ось чому вони завжди
(і в старовину і в середні віки) дбали про те, щоб такі операції відкладати
на святкові дні.

Розміри, що їх доходить перетворення товарів у руках капіталістів, не
можуть звичайно перетворити цю роботу, яка не утворює жодної вартости,
а впосереднює лише зміну форм вартости, на роботу, що утворює вартість.
Чудо такого перетворення так само мало може постати в наслідок
перекладання, тобто в наслідок того, що промислові капіталісти замість
самим виконувати оту „роботу горіння“, роблять з неї виключне зайняття
оплачуваних ними третіх осіб. Ці треті особи, звичайно, не дають їм
своєї робочої сили заради краси їхніх очей. Для збирача орендної плати,
що служить у якогобудь землевласника, або для банківського службовця
так само байдуже, що їхня праця ні на шеляг не збільшує величини
вартости ані ренти, ані зливків золота, що їх переноситься в мішках до
іншого банку\footnote{
Заведене в дужки взято з примітки наприкінці рукопису VIII.
}].

Для капіталіста, що примушує інших робити на себе, купівля й
продаж стають головною функцією. Що він привлащує продукт багатьох
у широкому суспільному маштабі, то в такому самому маштабі має він
продавати цей продукт, а потім знову зворотно перетворювати з грошей
на елементи продукції. Але, як і раніше, час купівлі, та продажу не
утворює жодної вартости. Ілюзія постає тут у наслідок функції купецького
капіталу. Але покищо, не розглядаючи цього ближче, само собою
зрозуміло таке. Коли якась функція, що сама собою непродуктивна, але
становить доконечний момент репродукції, в наслідок поділу праці пере-