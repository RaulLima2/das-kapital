\parcont{}  %% абзац починається на попередній сторінці
\index{ii}{0401}  %% посилання на сторінку оригінального видання
797 1/2; лишається 676 1/2 II m. Отже, II перетворює на сталий капітал
ще 121, і для цього треба йому 60 1/2 нового змінного капіталу: його
так само береться з 676 1/2; для споживання лишається 616.

Тоді матимемо капіталу:

I. Сталого 4840 + 484 = 5324

Змінного 1210 + 121 = 1331

II. Сталого 1760 + 55 + 121 = 1936
Змінного 880 + 27 1/2 + 60 1/2 = 968

Разом: І. 5324 с + $1331 v$ = 6655

II. 1936 с + $968 v$ = 2904
= 9559,

а наприкінці року матимемо продукту:

I.    5324 с + $1331 v + 1331 m$ = 7986

II.    1936 с + $968 v + 968 m$ = 3872
= 11858.

Повторюючи це обчислення й заокруглюючи дроби, матимемо наприкінці
наступного року продукту:

I.    5856 с + $1464 v + 1464 m$ = 8784

II.    $2129 c + 1065 v + 1065 m$ = 4259 \footnote*{
В нім. тексті тут, як і подекуди далі, єаритметичні помилки. Ці помилки ми
виправили. Ред.

К. Маркс. Капітал, т. II
}
= 13043.

А наприкінці наступного року:

І. 6442 с + $1610 v + 1610 m$ = 96624
II. 2342 с + $1171 v + 1171 m$) = 468
= 14346.

Протягом чотирилітньої репродукції в поширеному маштабі ввесь
капітал І і II збільшився з 5500 с + $1750 v$ = 7250 до 8784 с + $2781 v$ =
11565, отже, у відношенні 100: 160. Вся додаткова вартість спочатку
становила 1750, тепер вона становить 2781. Спожита додаткова вартість
спочатку була 500 для І і 600 для II, а разом 1100; вона була в

26
\parbreak{}  %% абзац продовжується на наступній сторінці
