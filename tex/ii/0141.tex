рядь праці, що переходить на продукт в наслідок зношування знарядь
праці. З цього заміщення ще зовсім не утворюється зиск. Залежно від
того, чи заміщується в наслідок продажу продукту авансовану на його
продукцію вартість цілком чи частинами, разом чи поступінно, може змінитися
лише спосіб і час заміщення; але це ніяк не може перетворити
спільне обом випадкам заміщення вартости на утворення додаткової вартости.
В основі маємо тут звичайне уявлення, що додаткова вартість —
тому що її реалізується лише через продаж продукту, через його циркуляцію,
— виникає лише з продажу, з циркуляції. А справді різні
способи постання зиску є тут лише неправильні вислови того, що різні
елементи продуктивного капіталу відіграють різну ролю, неоднаково
функціонують в процесі праці як продуктивні елементи. Нарешті, ця
ріжниця висновується в нього не з процесу праці, зглядно процесу зростання
вартости, не з функції самого продуктивного капіталу, а повинна
мати лише суб’єктивне значення для поодинокого капіталіста, що для
нього одна частина капіталу корисна цим, а друга — тим способом.

Навпаки, Кене висновує ці ріжниці з самого процесу репродукції
та його доконечности. Для того, щоб цей процес був безперервний,
вартість річних авансів мусить цілком заміщуватись з вартости річного
продукту, і навпаки — вартість основного капіталу мусить заміщуватись
лише частинами, так що лише протягом ряду років, прим., десятиліття,
її цілком заміщується, а тому й репродукується цілком (заміщується новими
екземплярами того самого роду). А. Сміс, отже, робить великий
крок назад порівняно з Кене.

Таким чином, для визначення основного капіталу А. Смісові не лишається
зовсім нічого іншого, як сказати, що це — засоби праці, які, протилежно
до продуктів, що їх утворенню вони допомагають, не змінюють
своєї форми в процесі продукції та функціонують далі в продукції, поки
зносяться. При цьому забувають, що всі елементи продуктивного капіталу
в своїй натуральній формі (як засоби праці, матеріяли й робоча сила)
постійно протистоять продуктові, і продуктові, що циркулює як товар,
і що ріжниця між частиною, яка складається з матеріялів та робочої
сили, і частиною, яка складається з засобів праці, лише в тому, що
робочу силу завжди купується наново (а не на ввесь час її існування,
як купується засоби праці), тимчасом як у процесі праці функціонують
не ті самі тотожні, а завжди нові екземпляри матеріялів того самого роду.
Разом з тим постає ілюзія, ніби вартість основного капіталу не циркулює,
хоч А. Сміс звичайно зазначав раніше, що зношування основного
капіталу ввіходить як частина в ціну продуктів.

При визначенні обігового капіталу як протилежности до основного,
не пояснюється, що обіговий капітал має цю протилежність лише як та
складова частина продуктивного капіталу, яка мусить цілком заміститись
з вартости продукту й тому мусить цілком брати участь у його
метаморфозах, тимчасом як із основним капіталом цього немає. Замість
пояснити це, А. Сміс сплутує обіговий капітал з тими формами, що їх
набирає капітал, переходячи зі сфери продукції в сферу циркуляції, —
