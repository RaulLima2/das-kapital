Розділ десятий

Теорії про основний та обіговий капітал

Фізіократи і Адам Сміс

У Кене ріжниця між основним і обіговим капіталом з’являється як
ріжниця між avances primitives\footnote*{
Аванси первинні. Ред.
} і avances annuelles\footnote*{
Аванси річні. Ред.
}. Він правильно визначає
цю ріжницю як ріжницю в межах продуктивного капіталу, тобто
капіталу, вкладеного в безпосередній процес продукції. А що для нього
єдиним справді продуктивним капіталом є капітал, застосовуваний в хліборобстві,
тобто капітал фармера, то й ці ріжниці подає він тільки для
капіталу фармера. Цим самим пояснюється, чому він для однієї частини
капіталу бере річний період обороту, для другої — довший (десятирічний).
В дальшому розвитку свого вчення фізіократи почали мимохідь переносити
ці ріжниці й на інші відміни капіталу, на промисловий капітал взагалі.
Для суспільства ріжниця між авансуваннями щорічними й багаторічними
така важлива, що багато економістів, навіть після Адама Сміса,
повертались до цього визначення.

Ріжниця між обома відмінами авансів постає лише тоді, коли авансовані
гроші перетворено на елементи продуктивного капіталу. Ця ріжниця
існує виключно в рамцях продуктивного капіталу. Тому Кене й не спадає
на думку залічувати гроші до первинних або щорічних авансів. Як аванси
для продукції, тобто як продуктивний капітал, обидві ці категорії протистоять
так само й грошам, як і наявним на ринку товарам. Далі, у Кене
ріжниця між цими двома елементами продуктивного капіталу правильно
сходить на ріжницю між способами, що ними ці елементи входять у вартість
готового продукту, а значить, на ріжницю між способами циркуляції
їхньої вартости разом з вартістю продукту, а тому й на ріжницю
між способами їхнього заміщення або їхньої репродукції, коли вартість
одного елемента щорічно заміщується цілком, а вартість другого — частинами
протягом довших періодів\footnote{
Порівн. Quesnay, Analyse du Tableau Economique. (Physiocrates, éd. Daire,
I. Partie, Paris. 1846). Ми читаємо там, напр. „Щорічні аванси складаються з витрат,
що їх робиться щороку на обробіток землі; ці аванси треба відрізняти від первинних
авансів, що становлять фонд організації сільського господарства“). Les
avances annuelles consistent dans les dépenses qui se font annuellement pour le
travail de la culture; ces avances doivent être distinguées des avances primitives,
qui forment les fonds de l’établissement de la culture. P. 59). У пізніших фізіократів
аванси часто зветься вже просто капіталом: „Capital ou avances“ Dupont de
}.

Єдиний успіх, що його зробив А. Сміс, — це узагальнення зазначених
категорій. Він прикладає їх уже не лише до спеціяльної форми капіталу,
до капіталу фармера, але взагалі до всякої форми продуктивного капіталу.
Відси само собою зрозуміло, що замість ріжниці між однорічним і багаторічним
оборотом, різниці, запозиченої від хліборобства, виступає взагалі
ріжниця різночасних оборотів, так що оборот основного капіталу завжди