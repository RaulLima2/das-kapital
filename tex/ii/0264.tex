нюється лише їхня функція. З грошей, що циркулюють, вони перетворюються
на лятентний грошовий капітал, що поступінно утворюється.

Гроші, нагромаджувані при цьому, є грошова форма проданих товарів,
а саме форма тієї частини їхньої вартости, яка репрезентує для їхніх
власників додаткову вартість. (Тут припускається, що кредитова система
не існує). Капіталіст, що нагромадив ці гроші, pro tanto продавав, не
купуючи.

Коли уявити собі цей процес, як окремий випадок, а не як загальний,
то він не потребує жодних пояснень. Частина капіталістів затримує частину
грошей, вторгованих від продажу своїх продуктів, не купуючи на
них продукту на ринку. Навпаки, друга частина капіталістів перетворює
на продукт усі свої гроші за винятком потрібного для продукції грошового
капіталу, що завжди повертається. Частина продукту, що її, як носія
додаткової вартости, подається на ринок, складається з засобів продукції
або з реальних елементів змінного капіталу, з доконечних засобів існування.
Отже, вона може одразу придатись для поширення продукції.
Ми бо зовсім не припускаємо, що одна частина капіталістів нагромаджує
грошовий капітал, у той час, як друга частина цілком споживає всю свою
додаткову вартість; ми лише припускаємо, що одна частина капіталістів
провадить свою акумуляцію у грошовій формі, утворює лятентний грошовий
капітал, тим часом як друга справді акумулює, тобто поширює розміри
продукції, справді збільшує свій продуктивний капітал. Маси наявних
грошей завжди досить для потреб циркуляції, коли навіть по черзі одна
частина капіталістів акумулює гроші, тим часом як друга частина поширює
маштаб продукції, і навпаки. Крім того, нагромадження грошей на одному
боці може відбуватись і без наявних грошей, шляхом самого лише нагромадження
боргових вимог.

Але труднощі постають тоді, коли ми припускаємо акумуляцію грошового
капіталу не як окремий випадок, а як загальну акумуляцію грошового
капіталу в кляси капіталістів. Згідно з нашим припущенням —
загальне й виключне панування капіталістичної продукції — поза цією
клясою взагалі немає жодних інших кляс, крім робітничої кляси. Все,
що купує робітнича кляса, дорівнює сумі її заробітної плати, дорівнює
сумі змінного капіталу, авансованого цілою клясою капіталістів. До цих
останнніх ці гроші припливають назад тому, що вони продають свій
продукт робітничій клясі. В наслідок цього їхній змінний капітал знову
набирає грошової форми. Припустімо, що сума цього змінного капіталу,
тобто сума змінного капіталу, не просто авансованого протягом року, а
справді застосованого, дорівнює 100 ф. стерл. × х; для розглядуваного тут
питання не має жодного значення, чи багато чи мало, залежно від швидкости
обороту, треба грошей для того, щоб авансувати протягом року
змінний капітал такої вартости. Цими 100 ф. стерл. × х капіталу кляса капіталістів
купує певну масу робочої сили або сплачує заробітну плату
певному числу робітників — перша оборудка. Робітники на цю саму суму
купують у капіталістів деяку кількість товарів, в наслідок цього сума
100 ф. стерл. × х зворотно припливає до капіталістів — друга оборудка.
