ваної ним, так само не продукують вартости, як і його вісім годин
доконечної праці, хоч у наслідок цих останніх до нього переходить
частина суспільного продукту. Поперше, з суспільного погляду, протягом
усіх десятьох годин робочу силу використовується, як і раніше, на цю
просту функцію циркуляції. Її не можна вживати на що інше, не
можна вживати на продуктивну працю. Подруге, суспільство не оплачує
цих двох годин додаткової праці, хоч їх і витратила особа, що
працювала протягом цього часу. Суспільство не одержує через це жодного
додаткового продукту або вартости. Але витрати циркуляції, що їх
ця особа репрезентує, зменшуються на одну п’яту частину: з десятьох
годин до вісьмох. Суспільство не виплачує жодного еквіваленту за п’яту
частину того активного часу циркуляції, що його агентом є ця особа.
А коли це капіталіст, що вживає таких агентів, то неоплачені дві години
зменшують ті витрати циркуляції його капіталу, які становлять одбаву з його
прибутків. Для нього це — позитивний виграш, бо неґативні межі зростання
вартости його капіталу вужчають. Поки дрібні самостійні товаропродуценти
витрачають частину свого власного часу на купівлю й продаж,
він являє або час, витрачуваний у промежках їхньої продуктивної
діяльности, або час, що віднімається від їхнього часу продукції.

За всіх обставин час, витрачений на це, є витрати циркуляції, що
нічого не додають до перетворених вартостей. Це є витрати, — потрібні
для того, щоб перетворити вартості з товарової форми на грошову.
Оскільки капіталістичний товаропродуцент є аґент циркуляції, він відрізняється
від безпосереднього товаропродуцента лише тим, що продає й
купує в ширших розмірах, а тому й функціонує як аґент циркуляції в
ширшому маштабі. Але коли розмір його підприємства примушує його
або дозволяє йому купувати (наймати) власних аґентів циркуляції, як
найманих робітників, то суть справи від цього не змінюється. Робочу
силу й робочий час треба до певного ступеня витратити на процес циркуляції
(оскільки він є просте перетворення форми). Однак тепер ця витрата
являє додаткову витрату капіталу; частину змінного капіталу доводиться
витрачати на закуп цієї робочої сили, що функціонує лише в
циркуляції. Таке авансування капіталу не утворює ні продукту, ні
вартости. Воно зменшує pro tanto*) розміри, що в них авансований
капітал функціонує продуктивно. Це те саме, ніби частину продукту
перетворили на машину, що купувала б і продавала решту продукту. Ця
машина зумовлює одбаву з продукту. Вона не співдіє в процесі продукції,
хоч може зменшити робочу силу тощо, витрачувану на циркуляцію.
Вона становить лише частину витрат циркуляції.

2) Бухгальтерія

Поряд справжніх купівель і продажів робочий час витрачається на
ведення книг, куди, крім того, входить і зречевлена праця: пера, чорнила,
папір, бюрко, витрати на контору. Отже, на цю функцію витрачається, з

*) Відповідно. Ред.
