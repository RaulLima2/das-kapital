\parcont{}  %% абзац починається на попередній сторінці
\index{ii}{0392}  %% посилання на сторінку оригінального видання
тому щодо цієї частини постає перепродукція, яка, знову таки щодо цієї
частини, гальмує репродукцію — навіть у незмінному маштабі.

Отже, у цьому випадку додатковий віртуальний грошовий капітал на
боці А (І) є, правда, перетворена на золото форма додаткового продукту
(додаткової вартости); але додатковий продукт (додаткова вартість), розглядуваний
як такий, є тут явище простої репродукції, але ще не явище
репродукції в поширеному маштабі. І ($v + m$) — а для нього це в усякому
разі має силу щодо деякої частини m — кінець-кінцем, мусить обмінятись
на II с, щоб репродукція II с могла відбуватися в незмінному маштабі.
А (І), продавши В (II) свій додатковий продукт, дав йому відповідну
частину вартости сталого капіталу в натуральній формі, але разом з тим,
вилучаючи гроші з циркуляції й не доповнюючи свого продажу наступною
купівлею, унеможливив продаж частини товарів В (II) такої самої
вартости. Отже, коли ми маємо на увазі всю суспільну репродукцію, яка
однаково охоплює капіталістів І і II, то перетворення додаткового продукту
А (І) на віртуальний грошовий капітал виражає неможливість зворотного
перетворення рівного величиною вартости товарового капіталу
В (II) на продуктивний (сталий) капітал; отже, виражає не віртуальну
продукцію в поширеному маштабі, а гальмування простої репродукції,
тобто дефіцит у простій репродукції. Що утворення й продаж додаткового
продукту А (І) сами собою є нормальні явища простої репродукції,
то ми маємо тут уже на основі простої репродукції такі навзаєм зумовлювані
явища: утворення віртуально додаткового грошового капіталу у
кляси І (тому недостатнє споживання з погляду II); затримка у кляси II
товарових запасів, що їх не можна знову перетворити на продуктивний
капітал (отже, відносна перепродукція в II); надмірний грошовий капітал
в І і дефіцит у репродукції в II.

Не зупиняючись далі на цьому пункті, зауважмо лише ось що.
Коли описувалось просту репродукцію, то припускалось, що всю додаткову
вартість І і II витрачається як дохід. Але в дійсності одну частину
додаткової вартости витрачається як дохід, а другу частину перетворюється
на капітал. Тільки при такому припущенні відбувається дійсна
акумуляція. Вважати, що акумуляція відбувається коштом споживання, —
коли це розуміти в такій загальній формі, — є ілюзія, яка суперечить суті
капіталістичної продукції, бо це припускає, що мета й движний мотив
її є споживання, а не здобування додаткової вартости й капіталізація її,
тобто акумуляція.

\pfbreak

Розгляньмо тепер трохи ближче акумуляцію в підрозділі II.

Перша трудність щодо II $с$, тобто щодо його зворотного перетворення
з складової частини товарового капіталу II на натуральну форму
сталого капіталу II, стосується до простої репродукції. Візьмімо попередню
схему:
\begin{center}
  ($1000 v + 1000 m$) І обмінюються на
  2000 II $с$.
\end{center}
