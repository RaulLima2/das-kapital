Далі з цього випливає: річна норма додаткової вартости завжди =
m'n, тобто дорівнює справжній нормі додаткової вартости, спродукованої
в один період обороту змінним капіталом, зужитим протягом цього
періоду, помноженій на число оборотів цього змінного капіталу протягом
цього року, або помноженій (що те саме) на обернений дріб його часу
обороту, обчисленого на рік, що береться за одиницю. (Коли
змінний капітал обертається 10 разів на рік, то час його обороту = 1/10 року;
отже, обернений дріб його часу обороту = 10/1 = 10).

Далі з наведеного випливає: М' = m', коли n = 1. М' більше за m',
коли n більше за 1, тобто, коли авансований капітал обертається більше
як один раз на рік, або коли капітал, що обернувся, більший, ніж капітал
авансований.

Нарешті, М' менше за m', коли n менше за 1, тобто коли капітал,
що обернувся протягом року, є лише частина авансованого капіталу, отже,
коли період обороту триває більш як рік.

Зупинімось трохи на цьому останньому випадку.

Ми зберігаємо всі припущення нашого попереднього прикладу, хай
тільки період обороту продовжиться до 55 тижнів. Процес праці потребує
щотижня 100 ф. стерл. змінного капіталу, отже, 5500 ф. стерл. для періоду
обороту, і продукує щотижня 100 m; отже, m, як і перше, = 100%.
Число оборотів n дорівнює тут 55: 50 = 10: 11, бо час обороту (беручи рік
50 тижнів) = 1 + 1/10 року = 11: 10 року. М' = 100%X5500Х10/11: 5500 =
100%Х10/11 = 1000: 11% = 90Х10/11%, отже, менше, ніж 100%. Справді,
коли б річна норма додаткової вартости була 100%, то 5500 v протягом
року мусили б випродукувати 5500 m, тимчасом як для цього треба 11: 10
року. Ці 5500 v продукують протягом року лише 5000 m, отже, річна
норма додаткової вартости = 5000 m: 5000 v = 10: 11 = 90X10: 11%.

Тому річна норма додаткової вартости, або відношення між додатковою
вартістю, спродукованою протягом року, і взагалі авансованим
змінним капіталом (на відміну від змінного капіталу, що обернувся
протягом року), не є просте суб’єктивне відношення, а самий
справжній рух капіталу викликає це зіставлення. Наприкінці року до
власника капіталу А повернувся авансований ним змінний капітал, рівний
500 ф. стерл., і крім того 5000 ф. стерл. додаткової вартости. Величину
авансованого ним капіталу виражає не та маса капіталу, що її він застосував
протягом року, а та, що періодично до нього повертається. В
розглядуваному питанні не має жодного значення, чи існує капітал наприкінці
року почасти як продукційний запас, чи почасти як товаровий
або грошовий капітал, і в якому відношенні розподіляється він на ці
різні частини. Для власника капіталу В повернулись 5000 ф. стерл., аван-
