\index{ii}{0367}  %% посилання на сторінку оригінального видання
Щодо І з, то І з завжди може виступати тут як покупець; він подає
в циркуляцію своє m як золото і вилучає за це засоби споживання II с;.
в II золото почасти зуживається як матеріял, а тому функціонує як справжній
елемент сталої складової частини с продуктивного капіталу II; а
оскільки цього не постає, воно знову таки стає елементом для утворення
скарбу як частини II т, яка затримується в грошовій формі. Відси
видно — навіть лишаючи осторонь І с, що його ми розглянемо потім,\footnote{
В рукопису немає досліду про обмін новопродукованого золота, який відбувається
в межах сталого капіталу підрозділу І. Ф. Е.
} —
як навіть проста репродукція, хоч тут і виключено акумуляцію
у власному значенні слова, тобто репродукцію в поширеному маштабі,
все ж неминуче включає нагромаджування грошей або утворення скарбу.
А що це знову повторюється щороку, то цим пояснюється припущення,
що було нам за вихідний пункт при вивченні капіталістичної продукції,
а саме припущення, що на початку репродукції в руках кляси капіталістів
І і II є відповідна товаровому обмінові маса грошових засобів. Таке
нагромадження відбувається навіть, коли відлічити золото, втрачуване в
наслідок зношування грошей, що циркулюють.

Само собою зрозуміло, що чим старіша капіталістична продукція, тим
більша всюди нагромаджена маса грошей, і значить, тим відносно менша
та частина, що її нова річна продукція золота долучає до цієї маси, хоч
абсолютна величина цієї додачі може бути значна. Повернімося в загальних
рисах ще раз до заперечення, зробленого Тукові: як можливо, щоб кожен капіталіст
вилучав з річного продукту додаткову вартість грішми, тобто вилучав
з циркуляції більше грошей, ніж подав до неї, — як це можливо, якщо,
кінець-кінцем, саму клясу капіталістів доводиться розглядати як те джерело,
що з нього взагалі гроші подається в циркуляцію?

Підсумовуючи вище (розд. XVII) розвинуте, даємо на це таку відповідь:

1) Єдине, потрібне тут припущення: що взагалі є досить грошей для
того, щоб обміняти різні елементи маси річної репродукції, — зовсім не
порушується тією обставиною, що частина товарової вартости складається
з додаткової вартости. Коли б припустити, що вся продукція належить
самим робітникам і додаткова праця є, отже, додаткова праця лише для
них самих, а не для капіталістів, то маса товарової вартости в циркуляції'
лишалась би та сама і, при інших незмінних обставинах, потребувала б
тієї самої маси грошей для своєї циркуляції. Отже, в обох випадках питання
ось у чому: відки беруться гроші для обміну всієї цієї товарової
вартости? — А зовсім не в тому, відки беруться гроші для перетворення
додаткової вартости на гроші.

Звичайно, — повертаємось ще раз до цього — кожен поодинокий товар
складається з $с + v + m$, і значить, для циркуляції всієї товарової маси
потрібна, з одного боку, певна грошова сума для циркуляції капіталу с + v,
а з другого боку, потрібна друга грошова сума для циркуляції доходу
капіталістів, додаткової вартости m. Як для поодиноких капіталістів, так
і для цілої кляси гроші, що в них вона авансує капітал, відрізняються.
\parbreak{}  %% абзац продовжується на наступній сторінці
