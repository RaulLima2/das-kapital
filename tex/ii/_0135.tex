\parcont{}  %% абзац починається на попередній сторінці
\index{ii}{0135}  %% посилання на сторінку оригінального видання
продуктивного капіталу та їхній вплив на характер обороту. Ба навіть
він одразу наводить, як приклад, купецький капітал у такому питанні,
де йдеться виключно про ріжниці частин продуктивного капіталу
в процесі утворення продукту й вартости — ріжниці, що й собі утворюють
ріжниці в обороті й репродукції капіталу.

Він каже далі: „Капітал, застосовуваний таким способом, не дає своєму
власникові доходу або зиску, поки він лишається в його посіданні або
зберігає ту саму форму“\footnote*{
„The capital employed in this manner yields no revenue or profit to its employer,
while it either remains in his possession or continues in the same shape“.
}. — Капітал, застосовуваний таким способом!
Але ж А. Сміс каже про капітал, вкладений у сільське господарство або
промисловість, і далі каже нам, що приміщений таким способом капітал
розподіляється на основний та обіговий. Отже, приміщення капіталу цим
способом само собою не може зробити його ні основним, ні обіговим.

Але, може, він хотів сказати, що капітал, застосований для того, щоб
продукувати товари й продавати ці товари з зиском, мусить, по перетворенні
на товари, продаватись і через продаж, поперше, переходити з
власности продавця у власність покупця, а подруге, зміняти свою натуральну
форму товару на грошову форму, і тому капітал не є корисний
для свого власника, поки він лишається в його посіданні або зберігає —
для нього — ту саму форму? Однак тоді справа сходить ось на що: та
сама капітальна вартість, яка раніш функціонувала в формі продуктивного
капіталу, в формі належній до продукційного процесу, функціонує
тепер як товаровий капітал і грошовий капітал, — в формах капіталу, належних
до процесу циркуляції, і тому вона вже не є ні основний, ні поточний
капітал. І це має силу так само для тих елементів вартости, що
долучаються сировинними та допоміжними матеріялами, отже, поточним
капіталом, як і для тих, що долучаються в наслідок зношування засобів
праці, отже, основним капіталом. Таким чином, ми тут ні на крок не
наблизились до висвітлення ріжниці між основним і поточним капіталом.

Далі: „Товари торговця не дають йому жодного доходу або зиску,
поки він не продасть їх за гроші, і гроші так само мало дають йому,
поки він знову не обміняє їх на товари. Його капітал безупинно одходить
від нього в одній формі й повертається до нього в другій і тільки
за допомогою такої циркуляції або послідовних актів обміну може дати
йому будь-який зиск. Тому такі капітали можна назвати у власному значенні
слова обіговими капіталами“\footnote*{
„The goods of the merchant yield him no revenue or profit tilt he sells them
for money, and the money yields him as little till it is again exchanged for goods.
His capital is continually going from him in one shape, and returning to him in
another, and it is only by means of such circulation, or successive exchanges, that
it can yield him any profit. Such capitals, therefore, may very properly be called
circulating capitals“.
}.

Те, що А. Сміс визначає тут як обіговий капітал, я хочу назвати
капіталом циркуляції (Zirkulationskapital). Це капітал в формі,
належній до процесу циркуляції, капітал, що змінює форму за допомогою
обміну (зміни речовин і зміни власника), отже, товаровий капітал і грошовий
\parbreak{}  %% абзац продовжується на наступній сторінці
