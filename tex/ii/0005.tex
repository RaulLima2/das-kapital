ражає і кількісне відношення між частиною грошей, витраченою на робочу
силу Р, і частиною, витраченою на засоби продукції Зп, — відношення,
що визначається насамперед сумою надлишкової додаткової праці,
що її має витратити певне число робітників.

Коли, отже, напр., в якійсь пральні тижнева заробітна плата 50 робітників
становить 50 фунтів стерл., то на засоби продукції треба витратити
372 фунт, стерл., припускаючи, що це є вартість засобів продукції,
перетворюваних на пряжу тижневою працею в 3.000 годин, що з них
1.500 годин є додаткова праця.

Тут цілком байдуже, до якої міри в різних галузях промисловости
додаткове вживання праці зумовлює додаткову витрату вартости в
формі засобів продукції. Тут має значіння лише те, щоб, за всіх умов,
витрачуваної на засоби продукції частини грошей — засобів продукції,
куплених в акті Г — Зп — було достатньо, отже, заздалегідь обчислено й
подано у відповідній пропорції. Інакше кажучи, маса засобів продукції
мусить бути достатня, щоб увібрати масу праці, щоб через неї перетворитись
на продукт. Коли б у наявності не було достатньої кількости
засобів продукції, то надлишкова праця, що нею порядкує покупець,
не могла б застосуватись; право покупця порядкувати цією
працею не призвело б ні до чого. Коли б у наявності засобів продукції
було більше, ніж праці, що нею можна порядкувати, то ці засоби продукції
не наситилось би працею, не перетворилось би на продукт.

Скоро відбувся акт Г — Т < Р Зп,покупець порядкує не лише засобами продукції та робочою силою, що
потрібні для продукції певного корисного
предмету. Він порядкує більшою кількістю робочої сили, що її
можна реалізувати як працю, або більшою кількістю праці, ніж треба
на покриття вартости робочої сили, і разом з тим порядкує засобами
продукції, які потрібні для реалізації або зречевлення цієї суми
праці: отже, порядкує він чинниками для продукції предметів більшої
вартости, ніж вартість елементів їх продукції, або чинниками
продукції такої кількости товарів, що в ній буде й додаткова вартість.
Отже, вартість, що її він авансував у грошовій формі, перебуває тепер
в такій натуральній формі, що в ній вона може реалізуватись як вартість,
яка вилуплює додаткову вартість (у вигляді товарів). Інакше кажучи:
вона перебуває в стані, або в формі продуктивного капіталу, який має
здібність функціонувати так, що утворює вартість і додаткову вартість.
Капітал у цій формі ми позначимо П.

Але вартість П дорівнює вартості Р + Зп, тобто дорівнює Г,
перетвореному на Р і Зп. Г є така сама капітальна вартість,
як і П, тільки форма її буття інша, а саме це є капітальна
вартість у грошовому стані або в грошовій формі — грошовий капітал.
Тому Г — Т < Р Зп, або в своїй загальній формі Г — Т, сума всіх товарокупі-
