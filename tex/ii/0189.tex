ровий капітал, але потім він ще три тижні перебуває в періоді циркуляції.
Отже, новий період продукції може початись знову тільки на початку
13-го тижня, і продукція мала б припинитись на три тижні, або на
четверту частину цілого періоду обороту. Тут знову таки байдуже, чи
припускаємо ми, що це припинення пересічно триває доти, доки товар
буде проданий, чи припускаємо, що воно зумовлене віддаленістю ринку
або строками виплат за проданий товар. Що три місяці продукція
припиняється на три тижні, отже, протягом року вона припиняється на
4X3 = 12 тижнів = 3 місяцям = 1/4 річного періоду обороту. Тому
провадити продукцію безперервно тиждень у тиждень у тому самому
маштабі можна лише двома способами.

Або треба скоротити маштаб продукції так, щоб 900 ф. стерл. вистачало
на те, щоб тримати роботу в русі так протягом робочого періоду,
як і протягом часу обігу першого обороту. Тоді з початком
10-го тижня відкривається другий робочий період, отже, й другий період
обороту, — відкривається раніше, ніж закінчиться перший період обороту,
бо період обороту дванадцятитижневий, а робочий період дев’ятитижневий.
900 ф. стерл., розподілені на 12 тижнів, дають 75 ф. стерл. на тиждень.
Насамперед очевидно, що такий скорочений маштаб підприємства має
собі за передумову зміну розмірів основного капіталу, а значить, і взагалі
скорочення розмірів підприємства. Подруге, сумнівно, чи можна взагалі
провести таке скорочення, бо відповідно до розвитку продукції в різних
підприємствах є певний нормальний мінімум капіталовкладення, і коли
воно нижче від цього мінімуму, то підприємство не може витримати конкуренції.
Самий цей нормальний мінімум з розвитком капіталістичної
продукції теж раз-у-раз зростає і, значить, не є сталий. Але між даним
кожного разу нормальним мінімумом і дедалі більшим нормальним максимумом
є численні проміжні щаблі, — середина, що припускає дуже різні ступені
капіталовкладень. В межах цієї середини, отже, також можна провести
скорочення, що межа його є самий кожноразовий нормальний мінімум.
При затриманнях у продукції, переповненні ринку, подорожчанні сировинного
матеріялу тощо, скорочення нормальних витрат обігового капіталу,
за даної величини основного капіталу, постає через обмеження
робочого часу, через те, що роблять, приміром, тільки півдня; так само
в часи розцвіту за даної величини основного капіталу надмірне збільшення
обігового капіталу постає почасти через подовження робочого часу,
почасти через його інтенсифікацію. В підприємствах, заздалегідь розрахованих
на такі коливання, дають собі раду почасти вищезазначеними
способами, почасти одночасним уживанням більшого числа робітників,
а це сполучається з застосуванням запасного основного капіталу, напр.,
запасних паровозів на залізницях тощо. Але тут, припускаючи нормальні
відношення, ми не будемо брати на увагу таких ненормальних коливань.

Отже, тут, щоб зробити продукцію безперервною, витрату того самого
обігового капіталу розподіляється на довший час, на 12 тижнів замість 9.
Отже, в кожний даний переміжок часу функціонує вменшений продуктивний
капітал; поточна частина продуктивного капіталу зменшується
