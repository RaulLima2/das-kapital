репродукції вартість машини поступінно акумулюється насамперед в формі резервного грошового фонду.

Інші елементи продуктивного капіталу складаються почасти з елементів сталого капіталу, які є в
допоміжних матеріялах та сировинних матеріялах, а почасти із змінного капіталу, витраченого на
робочу силу.

Аналіза процесу праці й процесу зростання вартости (книга І, розділ V) виявила, що ці різні складові
частини відіграють цілком різну ролю в утворенні продукту і в утворенні вартости. Вартість тієї
частини сталого капіталу, яка складається з допоміжних та сировинних матеріялів — цілком так само,
як і вартість тієї його частини, яка складається з засобів праці — знову з’являється в вартості
продукту, як лише перенесена вартість, тимчасом як робоча сила за посередництвом процесу праці додає
до продукту еквівалент своєї вартости, або дійсно репродукує свою вартість. Далі, одну частину
допоміжних матеріялів, — вугілля на опалення, світильний газ тощо, — зужитковується в процесі праці,
при чому речово вона не увіходить у продукт, тимчасом як друга частина їх своїм тілом увіходить у
продукт і становить матеріял його субстанції. Але всі ці відмінності не мають значення для
циркуляції, а тому й для способу обороту. Коли допоміжні й сировинні матеріяли цілком зужитковується
під час утворення певного продукту, то вони цілком переносять свою вартість на продукт. Тому вона
через продукт цілком подається в циркуляцію, перетворюється на гроші, а з грошей знову на елементи
продукції товару. Її оборот не переривається, як оборот основного капіталу, але безупинно перебігає
весь кругобіг своїх форм, так що ці елементи продуктивного капіталу постійно відновлюються in
natura.

Щодо змінної складової частини продуктивного капіталу, витрачуваної на робочу силу, то робочу силу
купується на певний час. Коли капіталіст купить її і введе в продуктивний процес, то вона утворює
складову частину його капіталу, а саме — його змінну частину. Вона діє щоденно певний час, що
протягом його вона додає до продукту не лише всю свою денну вартість, а також ще деяку надлишкову
додаткову вартість, яку ми покищо залишаємо осторонь. Після того, як робочу силу куплено, й вона
діяла, напр., протягом тижня, закуп її мусить постійно відновлюватися у певні терміни. Той
еквівалент її вартости, що його робоча сила долучає до продукту протягом свого функціонування і що в
наслідок циркуляції продукту перетворюється на гроші, мусить завжди знову перетворюватись з грошей
на робочу силу або завжди мусить пророблювати повний кругобіг своїх форм, тобто завжди обертатись,
щоб не перервався кругобіг безперервної продукції.

Отже, частина вартости продуктивного капіталу, авансована на робочу силу, цілком переходить на
продукт (додаткову вартість ми залишаємо ввесь час осторонь), разом з ним перебігає обидві
метаморфози, що належать до сфери циркуляції, і в наслідок цього постійного відновлення завжди
лишається зв’язана з продукційним процесом. Отже, хоч як у всьому іншому робоча сила відрізняється
щодо утворення вартости від тих складових частин сталого капіталу, які не становлять ос-
