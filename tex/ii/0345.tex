рення потенціяльного на справжній змінний капітал, тобто для купівлі
й продажу робочої сили. Робочий ринок уже не становить частини того
товарового ринку, що тут є перед нами. Тут робітник не тільки вже продав
свою робочу силу, а й дав у товарі, крім додаткової вартости,
еквівалент ціни своєї робочої сили; з другого боку, заробітна плата є
вже в його кишені і в обміні він фігурує лише як покупець товару
(засобів споживання). Але далі річний продукт мусить мати в собі всі
елементи репродукції, мусить відновити всі елементи продуктивного капіталу,
— отже, насамперед, найважливіший елемент його — змінний капітал.
І ми справді бачили, що відносно до змінного капіталу результат обміну
такий: робітник як покупець товару, витрачаючи свою заробітну плату
й споживаючи куплений товар, зберігає й репродукує свою робочу силу
як єдиний товар, що його він може продавати; як гроші, авансовані
капіталістом на закуп цієї робочої сили, повертаються до капіталіста, так
і робоча сила, як товар, обмінюваний на ці гроші, повертається на робочий
ринок; в наслідок цього ми тут, а саме для 1000 I v, маємо таке:
на боці капіталістів І — 1000v грішми; на протилежному боці, на
боці робітників І — робоча сила вартістю в 1000, отже, ввесь процес
репродукції І може початися знову. Це — один результат процесу
обміну.

З другого боку, витрачення заробітної плати робітників І забрало
в II засобів споживання на суму 1000с і таким чином перетворило їх з
товарової форми на грошову форму; II з цієї грошової форми перетворив
їх знову на натуральну форму свого сталого капіталу за допомогою
закупу товарів на суму 1000 V у І; в наслідок цього до І повертається
його змінна капітальна вартість знову в грошовій формі.

Змінний капітал І пророблює три перетворення, що зовсім не виявляються
при обміні річного продукту, або виявляються лише як натяк.

1) Перша форма, 1000 Iv в грошах, які перетворюються на робочу
силу того ж розміру вартости. Саме це перетворення не виявляється в
товаровому обміні між І і II, але його результат виявляється в тому, що
кляса робітників І з 1000 в грошах протистоїть продавцеві товарів II,
цілком так само, як кляса робітників II з 500 в грошах протистоїть
продавцеві товарів — 500 IIv в товаровій формі.

2) Друга форма, — єдина, що в ній змінний капітал справді змінюється,
функціонує як змінний, що в ній вартостетворча сила виступає замість
обміненої на неї даної вартости, — належить виключно до продукційного
процесу, що лежить позаду нас.

3) Третя форма, що в ній в наслідок продукційного процесу змінний
капітал виявляв себе як такий, є новоспродукована вартість, отже,
в І = 1000 v + 1000 m = 2000 І (v + m). Замість його первісної вартости =
1000 грішми виступає вдвоє більша вартість = 2000 в товарах.
А тому змінна капітальна вартість = 1000 в товарах становить лише
половину тієї нової вартости, що її утворив змінний капітал як елемент
продуктивного капіталу. Ці 1000 Iv в товарах є точний еквівалент тієї
змінної за її призначенням частини всього капіталу, яку первісно І аван-
