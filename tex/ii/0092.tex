таліст. Отже, витрати, що удорожчують товар, нічого не додаючи до його
споживної вартости, отже, витрати, що для суспільства належать до faux
frais продукції, можуть становити джерело збагачення індивідуального
капіталіста. З другого боку, ці витрати циркуляції не втрачають характеру
непродуктивних витрат, від того, що надвишка, яку вони додають до
ціни товару, лише рівномірно розподіляє ці витрати. Напр., страхові
товариства розподіляють утрати індивідуальних капіталістів між клясою
капіталістів. Однак це не заваджає тому, що вирівняні таким способом
утрати все ж, як і раніш, з погляду суспільного сукупного капіталу, є втрати.

1) Утворення запасу взагалі

Протягом того часу, коли продукт існує у формі товарового капіталу
або перебуває на ринку, отже, протягом усього часу між процесом
продукції, відки він виходить, і процесом споживання, в який він увіходить,
він становить товаровий запас. Як товар на ринку, а тому й у формі
запасу, товаровий капітал з’являється двічі в кожному кругобігу, — один
раз як товаровий продукт самого капіталу, що процесує, — капіталу, що
його кругобіг розглядається; другий раз, навпаки, як товаровий продукт
іншого капіталу, як продукт, що мусить бути на ринку, щоб його
можна було купити й перетворити на продуктивний капітал. Звичайно,
можливо, що цей останній товаровий капітал продукується лише на
замовлення. Тоді постає перерва, що триває доти, доки його спродукують.
Але перебіг процесу продукції та репродукції потребує, щоб деяка маса
товарів (засобів продукції) завжди була на ринку і, значить, становила запас.
Так само продуктивний капітал охоплює й закуп робочої сили, і грошова
форма є тут лише форма вартости засобів існування, що їх більшість робітник
мусить знаходити на ринку. В цьому параграфі ми докладніше зупинимось
на цьому питанні. Але вже й тепер ми доходимо такого пункту:
коли дивитися з погляду капітальної вартости, що процесує, капітальної
вартости, яка перетворилась на товаровий продукт і мусить тепер продатись,
тобто знову перетворитись на гроші, яка, отже, функціонує тепер на ринку
як товаровий капітал, то той стан її, що в ньому вона утворює запас, є
недоцільне вимушене перебування на ринку. Що швидше відбувається
продаж, то швидше перебігає процес репродукції. Затримка в
перетворенні форми Т' — Г' заваджає реальному обмінові речовин, що
мусить відбуватися в кругобігу капіталу, так само, як і дальшому функціонуванні
його в ролі продуктивного капіталу. З другого боку, постійна
наявність товару на ринку, товаровий запас, є для Г — Т умова перебігу
процесу репродукції, а також і вкладання нового або додаткового
капіталу.

Шоб товаровий капітал міг лишатись на ринку як товаровий запас,
потрібні будівлі, магазини, приміщення для товарів, товарові склади, отже,
погрібна витрата сталого капіталу; так само потрібна і оплата робочої
сили для складання товарів у приміщення. Крім того, товари псуються
і зазнають шкідливих стихійних впливів. Щоб уберегти їх від цього,
