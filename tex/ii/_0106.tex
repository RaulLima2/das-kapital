\parcont{}  %% абзац починається на попередній сторінці
\index{ii}{0106}  %% посилання на сторінку оригінального видання
harvesting its ending. — S. P. Newman. Elements of Pol. Econ.; Andover
and New-York, p. 81). Інші починають з T' (III форми): „Можна вважати,
що світ продукційного обміну рухається колом, що його ми можемо
назвати економічним циклом, і де кожний обіг вивершується, скоро підприємство,
проробивши ряд послідовних оборудок, знову доходить пункта,
відки вийшло. За початок можна вважати той пункт, коли капіталіст
одержує надходження, що за їх посередництвом до нього повертається
його капітал, з цього пункту він знову переходить до того, щоб навербувати
собі робітників і розподілити між ними в формі заробітної плати
засоби їхнього існування або скорше силу, потрібну на придбання цих
засобів; одержати від них готові речі, що їх він виробляє; подати ці
речі на ринок і там довести до кінця кругобіг ряду цих рухів, продаючи
ці речі й одержуючи в уторгованих за товар грошах покриття всіх своїх
капітальних витрат“. (Th. Chalmers, on Pol. Econ., 2. ed., London, 1832 p.
84 і далі).

Скоро тільки вся капітальна вартість, вкладена поодинокими капіталістами
в якубудь галузь продукції, вивершить у своєму русі кругобіг,
вона знову опиняється в своїй початковій формі і може тепер повторити
той самий процес. Щоб вартість увічнилась і далі зростала, як капітальна
вартість, вона мусить повторювати цей кругобіг. В житті капіталу поодинокий
кругобіг становить лише один постійно повторюваний відділ,
отже, період. Наприкінці періоду $Г\dots{} Г'$ капітал знову перебуває в формі
грошового капіталу, що знову перебігає ряд перетворень форми, які
охоплюють процес його репродукції, зглядно процес зростання вартости.
При закінченні періоду $П\dots{} П$ капітал знову перебуває в формі елементів
продукції, які є передумова для відновлення його кругобігу. Кругобіг
капіталу, визначуваний не як поодинокий акт, а як періодичний процес,
зветься оборотом капіталу. Протяг цього обороту дано сумою часу
його продукції та часу його обігу. Ця сума часу становить час обороту
капіталу. Отже, вона охоплює переміжок часу від одного періоду кругобігу
цілої капітальної вартости до наступного, вона позначає періодичність
у життьовому процесі капіталу, або, коли хочете, час відновлення, повторення
процесу зростання вартости, зглядно процесу продукції тієї самої
капітальної вартости.

Лишаючи осторонь індивідуальні пригоди, що можуть для окремого
капіталу подовжити або скоротити час обороту, цей час є різний залежно
від різних сфер приміщення капіталу.

Так само, як для функції робочої сили за природну вимірну одиницю
є робочий день, так само рік є природна вимірна одиниця для оооротів капіталу,
що процесує. Природну основу такої одиниці виміру являє та обставина,
що в помірній смузі, батьківщині капіталістичної продукції, найважливіші
плоди земні є річні продукти.

Коли рік як вимірну одиницю часу обороту ми позначимо О, час
обороту певного капіталу — о, число його оборотів п, то п = \sfrac{О}{о} Отже,
\parbreak{}  %% абзац продовжується на наступній сторінці
