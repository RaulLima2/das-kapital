ність будь-якої акумуляції або репродукції в поширеному маштабі є
неймовірне припущення, а з другого боку, відношення, що в них відбувається
продукція, в різні роки не лишаються абсолютно незмінні (а таке
є наше припущення). Наше припущення те, що суспільний капітал даної
вартости, як минулого року, так і цього року знов дає таку саму масу товарових вартостей і
задовольняє таку саму кількість потреб, хоча б форми товарів і змінилися в процесі репродукції. А
проте, оскільки відбувається акумуляція, проста репродукція завжди становить частину останньої,
отже, її можна розглядати окремо, вона — реальний чинник акумуляції. Вартість річного продукту може
зменшитись, хоч маса споживних вартостей лишається та сама, вартість може лишатись та сама, хоч маса
споживних вартостей меншає; маса вартости й маса репродукованих
споживних вартостей можуть одночасно меншати. Все це залежить
від того, що репродукція відбувається або при сприятливіших умовах,
ніж були раніше, або при гірших умовах, а останні можуть призвести до
неповної — недостатньої — репродукції. Однак усе це стосується лише до
кількісного боку різних елементів репродукції, а не до тієї ролі, що її
вони відіграють в цілому процесі як капітал, що його репродукується,
або як дохід, уже репродукований.

II. Два підрозділи суспільної продукції 44)

Цілий продукт, отже, і вся продукція суспільства, розпадається на
два великі підрозділи:

I. Засоби продукції, товари, які мають таку форму, що в ній
вони мусять ввійти або принаймні можуть ввійти в продуктивне споживання.

II. Засоби споживання, товари, які мають таку форму, що в
ній вони входять в особисте споживання кляси капіталістів і кляси робітників.

В кожному з цих підрозділів усі різні галузі продукції, належні до
того або того підрозділу, становлять єдину велику галузь продукції,
в одному разі — продукції засобів продукції, в другому — засобів споживання.
Ввесь капітал, застосований в кожній з цих двох галузей продукції,
становить окремий великий підрозділ суспільного капіталу.

У кожному підрозділі капітал розпадається на дві складові частини:

1) Змінний капітал. Розглядуваний щодо вартости він дорівнює
вартості суспільної робочої сили, застосованої в цій галузі продукції,
отже, дорівнює сумі заробітної плати, сплаченої за цю робочу силу.
Розглядуваний з речового боку, він складається з самої діючої робочої
сили, тобто з живої праці, пущеної в рух цією капітальною вартістю.

2) Сталий капітал, тобто вартість усіх засобів продукції, застосованих
для продукції в цій галузі. Вони й собі розпадаються на основний
капітал: машини, знаряддя праці, будівлі, робочу худобу і т. ін.,

44) В головному рукопису II. Схема рукопису VIII.
