рового капіталу, що мандрують до різних ринків. Поліпшені вітрильні
судна та пароплави, напр., що скорочують шлях, однаково скорочують
його так до близьких, як і до далеких портів. Відносна ріжниця лишається,
хоч часто зменшена. Але відносні ріжниці можуть у наслідок розвитку
засобів транспорту й зв’язку змінюватись таким способом, який не відповідає
природним віддаленням. Напр., залізниця, що веде від місця продукції до
головного внутрішнього залюдненого центру, може зробити ближчий
унутрішній пункт, що до нього немає залізниці, абсолютно або відносно
більш віддаленим порівняно з пунктом, куди віддаленішим географічно; так
само, в наслідок цієї самої обставини, може змінюватись навіть відносна
віддаленість місць продукції від більших ринків збуту, і цим пояснюється
занепад старих і постання нових центрів продукції рівнобіжно з зміною
засобів транспорту й зв’язку. (До цього ще долучається відносно більша
дешевина транспорту на великі дистанції порівняно з невеликими). Разом
з розвитком засобів транспорту не тільки збільшується швидкість
переміщення, і в наслідок цього просторова віддаль зменшується в часі.
Розвивається не лише маса засобів комунікації, так що, напр., одночасно
багато суден виходять до того самого порту, кілька поїздів одночасно
йдуть різними залізницями між тими самими двома пунктами, але, напр.,
у різні послідовні дні тижня товарові судна виходять з Ліверпулу на
Нью-Йорк, або товарові поїзди в різні години доби йдуть з Менчестера
до Лондону. Правда, абсолютна швидкість — отже, і відповідна частина
часу обігу — в наслідок цієї останньої обставини, за даної провізної спроможности
засобів транспорту, не змінюється. Але все ж послідовні партії товарів
можна відправляти через коротші переміжки часу, що йдуть один по
одному, і таким чином вони можуть послідовно надходити на ринок, не
нагромаджуючись великими масами як потенціяльний товаровий капітал,
поки їх дійсно відправиться. Тому й зворотний приплив розподіляється
на коротші послідовні періоди часу, так що одна частина постійно перетворюється
на грошовий капітал, тимчасом як друга частина циркулює
як товаровий капітал. В наслідок такого розподілу зворотного
припливу на кілька послідовних періодів скорочується весь час обігу, а
тому скорочується й оборот. Насамперед розвивається більша чи менша
частість функціонування засобів транспорту, — напр., численність поїздів
на залізниці розвивається, з одного боку, разом із тим, як осередок
продукції продукує дедалі більше, стає більшим центром продукції, і
розвивається вона в напрямку до вже наявних ринків збуту, отже, в напрямку
до великих центрів продукції та залюднення, до вивізних портів
тощо. Але, з другого боку, ця особлива легкість сполучень і зумовлений
нею прискорений оборот капіталу (оскільки його зумовлює час обігу)
призводить, навпаки, до прискореної концентрації, з одного боку, центру
продукції, а з другого — його ринку збуту. Разом із прискореною
таким чином концентрацією в певних пунктах маси людей та капіталів,
розвивається концентрація цих мас капіталів у небагатьох руках. Разом
з тим знову пересуваються й переміщуються осередки продукції та ринки
в наслідок їх зміненого відносного положення, зумовленого зміною
