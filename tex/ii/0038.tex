капіталу Г — Т. Не зважаючи на різні місця, функція грошового капіталу,
що на нього тепер перетворився товаровий капітал, є та сама: його
перетворення на Зп та Р, на засоби продукції та робочу силу.

Отже, одночасно з т — г капітальна вартість у функції товарового
капіталу Т' — Г' проробила фазу Т — Г і вступає тепер у додатковуфазу Г — Т Р Зп; отже, її ціла
циркуляція є Т — Г — Т Р Зп.

По-перше, грошовий капітал Г виступив у формі І (кругобіг Г... Г')
як первісна форма, що в ній авансується капітальна вартість; а тут він
виступає з самого початку як частина грошової суми, що на неї перетворився
товаровий капітал у першій фазі циркуляції Т' — Г', отже, з самого
початку виступає як перетворення П, продуктивного капіталу, на грошову
форму, — перетворення, що упосереднюється продажем товарового продукту.
Грошовий капітал існує тут з самого початку не як первісна та
не як кінцева форма капітальної вартости, бо фаза Г — Т, що вивершує
фазу Т — Г, може бути пророблена, лише через подруге скидання грошової
форми. Тому частина Г — Т, яка одночасно є Г — Р, виступає тут
уже не як просте авансування грошей у формі закупу робочої сили, але
як таке авансування, що в ньому робочій силі авансується в формі
грошей ті самі 1000 ф. пряжі вартістю в 50 ф. стерл., які становлять
частину товарової вартости, утвореної робочою силою. Гроші, авансовані
тут робітникові, є лише перетворена еквівалентна форма частини
вартости товару, що його спродукував сам робітник. І вже тому акт
Г — Т, оскільки він є акт Г — Р, зовсім не є лише заміщення товару в
грошовій формі товаром у споживній формі, але має в собі інші
елементи, незалежні від загальної товарової циркуляції як такої.

Г' є перетворена форма Т', яке саме є продукт минулого функціонування
П, продукційного процесу; тому вся грошова сума Г' є грошовий
вираз минулої праці. У нашому прикладі: 10.000 ф. пряжі = 500 ф. стерл.,
продуктові процесу прядіння; з них 7440 ф. пряжі = авансованому
сталому капіталові c = 372 ф. стерл.; 1000 ф. пряжі = авансованому
змінному капіталові v = 50 ф. стерл., і 1560 ф. пряжі = додатковій
вартості m = 78 ф. стерл. Коли з Г' знову авансується, за інших незмінних
умов, лише первісний капітал = 422 ф. стерл., то робітник одержує
в найближчий тиждень в акті Г — Р як аванс лише частину 10.000 ф.
пряжі (грошову вартість 1000 ф. пряжі), спродуковану протягом цього
тижня. Як результат акту Т — Г гроші завжди є вираз минулої праці.
Оскільки на товаровому ринку відразу відбувається додатковий акт Г — Т,
тобто Г обмінюється на наявні товари, що є на ринку, то знову таки
тут маємо перетворення минулої праці з однієї форми (грошової) на
другу (товарову). Але Г — Т і Т — Г відрізняються щодо часу. Вони
можуть винятково відбуватись одночасно, коли, наприклад, капіталіст,
що переводить Г — Т, і капіталіст, що для нього цей акт є акт
Т — Г, одночасно передають один одному свої товари, і Г потім лише
вирівнює ріжницю. Ріжниця щодо часу здійснення Т — Г і часу здійс-
