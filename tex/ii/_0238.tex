\parcont{}  %% абзац починається на попередній сторінці
\index{ii}{0238}  %% посилання на сторінку оригінального видання
можуть вийти лише з певних галузей, як, напр., сільське господарство тощо,
де працюють виключно дужі парубки. Це діється й після того, як нові підприємства
стали вже постійною галуззю продукції і, значить, після того,
як уже утворилась потрібна для них бродяча робітнича кляса. Напр.,
коли залізниця раптом почне будуватись у ширшому від пересічного
маштабі. Тоді вбирається частину резервної армії робітників, що її тиск
тримав заробітну плату на порівняно низькому рівні. Тоді заробітна плата
скрізь підвищується, навіть у тих частинах робочого ринку, де робітники
й раніш легко знаходили собі працю. Це триває доти, доки неминучий
крах знову звільняє резервну армію робітників, і заробітну плату
знову знижується до її мінімуму й нижче.\footnote{
В рукопису тут вставлено таку замітку, щоб пізніш її розвинути: „Суперечність
в капіталістичному способі продукції: робітники як покупці товару,
важать для ринку. Але як продавців свого товару — робочої сили капіталістичне
суспільство має тенденцію обмежувати їх мінімумом ціни. Дальша суперечність:
ті епохи, коли капіталістична продукція напружує всі свої сили, регулярно з’являються
як епохи перепродукції, бо продуктивні сили ніколи не можна застосувати
так, щоб у наслідок цього можна було не лише випродукувати, а й зреалізувати
більше вартости; але продаж товарів, реалізація товарового капіталу, отже,
і додаткової вартости, обмежена не просто споживними потребами суспільства
взагалі, з споживними потребами такого суспільства, що його переважна
більшість завжди бідна й мусить завжди лишатися бідною. Однак це стосується
лише до наступного відділу.“ Ф. Е.
}

Оскільки більший або менший протяг періоду обороту залежить від
робочого періоду у власному значенні, тобто від періоду, потрібного на
те, щоб виготувати продукт для ринку, він ґрунтується на кожного
разу даних речових умовах продукції різних капіталовкладень, на
умовах, що в хліборобстві мають більше характер природних умов продукції,
а в мануфактурі і в більшій частині видобувної промисловости
змінюються разом із суспільним розвитком самого продукційного процесу.

Оскільки протяг робочого періоду ґрунтується на величині поставок
(на кількісному розмірі, що в ньому продукт звичайно подається на ринок
як товар), він має умовний характер. Але сама ця умовність має за
матеріяльну базу розміри продукції, а тому вона є випадкова лише остільки,
оскільки ми розглядаємо її ізольовано.

Нарешті, оскільки протяг періоду обороту залежить від протягу періоду
циркуляції, він почасти зумовлюється постійною зміною ринкових
коньюнктур, більшою або меншою легкістю продажу і неминучою, відси
посталою, потребою подавати частину продукту на ближчий або дальший
ринок. Лишаючи осторонь розмір попиту взагалі, рух цін відіграє
тут головну ролю, оскільки при зниженні цін продаж навмисно обмежується,
тимчасом як продукція розвивається далі; навпаки буває при
підвищенні цін, коли продукція та продаж не відстають одне від одного,
або коли продаж може відбуватися заздалегідь. Однак, за власне матеріяльну
базу треба вважати справжнє віддалення місця продукції від
ринку збуту.
