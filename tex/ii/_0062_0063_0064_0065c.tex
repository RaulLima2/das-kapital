\parcont{}  %% абзац починається на попередній сторінці
\index{ii}{0062}  %% посилання на сторінку оригінального видання
його вартости; але разом з тим вона утворює додатковий матеріял для
зростання вартости.

$Т'\dots{} Т'$ становить основу Tableau économique Кене, і те, що він
обрав протилежно до форми $Г\dots{} Г'$ (форми, що її виключно дотримувалась
меркантильна система) саме цю форму, а не $П\dots{} П$, свідчить про
його великий і правильний такт.

\section{Три фігури процесу кругобігу}

Коли Ск позначає сукупний процес циркуляції, то три фігури можна
зобразити так:
% TODO: make centering
  \begin{table}[h]
    \begin{tabularx}{\textwidth}{r l}
      I. & $Г — Т\dots{} П\dots{} Т' — Г'$\\

      II. & $П\dots{} Ск\dots{} П$\\

      III. & $Ск\dots{} П (Т')$.\\
    \end{tabularx}
  \end{table}
Коли ми всі три форми сполучимо, то всі передумови процесу виступають,
як його результат, як передумова, ним самим утворена. Кожен
момент виступає як вихідний пункт, перехідний пункт і пункт повороту.
Сукупний процес виступає як єдність процесу продукції та процесу
циркуляції; продукційний процес стає посередньою ланкою в процесі
циркуляції і навпаки.

Всі три кругобіги мають спільну рису, а саме — зростання вартости як
визначальна мета, як движний мотив. В І це виражено в самій формі.
Форма II починається з $П$, з самого процесу зростання вартости.
В III кругобіг починається з вирослої вартости й закінчується нововирослою
вартістю, навіть коли рух повторюється в незмінному маштабі.
Оскільки $Т — Г$ є для покупця $Г — Т$, а $Г — Т$ є для продавця $Т — Г$,
остільки циркуляція капіталу являє лише звичайну товарову метаморфозу,
і тут мають силу розвинені щодо неї закони про кількість грошей, що
циркулюють (кн. І, розділ III, 2). Але коли не зупинятись на цьому
формальному боці, а розглядати реальний зв’язок між метаморфозами
різних індивідуальних капіталів, отже, дійсно зв’язок кругобігів індивідуальних
капіталів як частинні рухи в процесі репродукції сукупного
суспільного капіталу, то цей зв’язок не можна з’ясувати простою зміною
форм грошей і товару.

В постійно рухомому колі кожен пункт є одночасно пункт вихідний
і пункт повороту. Коли переривається круговий рух, то не кожний
вихідний пункт є пункт повороту. Наприклад, ми бачили, що не лише
кожен окремий кругобіг (implicite) припускає інший, але також, що
повторення кругобігу в одній формі включає пророблення кругобігу й
в інших формах. Таким чином, уся ріжниця виступає як суто-формальна
\index{ii}{0063}  %% посилання на сторінку оригінального видання
або навіть суто-суб’єктивна, як ріжниця, що існує лише для спостерігача.
Оскільки кожен з цих кругобігів розглядається як особливу форму
руху, що в ній перебувають різні індивідуальні промислові капітали, остільки
й ця ріжниця завжди існує лише як індивідуальна ріжниця. Але в дійсності
кожний індивідуальний промисловий капітал одночасно перебуває
в усіх трьох кругобігах. Три кругобіги, форми репродукції трьох відмін
капіталу, безупинно відбуваються один поряд одного. Напр., одна частина
капітальної вартости, яка функціонує тепер як товаровий капітал, перетворюється
на грошовий капітал, але одночасно друга частина виходить
з продукційного процесу в циркуляцію як новий товаровий капітал. Так
завжди перебігає кругова форма $Т'\dots{} Т'$, так само і обидві інші. Репродукція
капіталу в кожній з його форм і на кожній з його стадій
є так само безперервна, як і метаморфоза цих форм і послідовний перебіг
цих трьох стадій. Отже, ввесь кругобіг є тут дійсна єдність усіх трьох
його форм.

У нашому досліді припускалось, що вся капітальна вартість у цілому розмірі
своєму виступає як грошовий капітал, або як продуктивний капітал, або як
товаровий капітал. Напр., 422 ф. стерл. було в нас спочатку як грошовий
капітал, потім, знову таки в цілому розмірі, перетворились вони на
продуктивний капітал і, нарешті, на товаровий капітал: пряжу, вартістю
в 500 ф. стерл. (з них 78 ф. стерл. додаткової вартости). Тут ці різні
стадії становлять стільки ж перерв. Напр., поки 422 ф. стерл. лишаються
в грошовій формі, тобто поки не відбулись купівлі $Г — Т (Р + Зп)$,
сукупний капітал існує і функціонує лише як грошовий капітал. А скоро
він перетворюється на продуктивний капітал, він уже не функціонує ні
як грошовий капітал, ні як товаровий капітал. Увесь процес його циркуляції
переривається, так само, як, з другого боку, переривається ввесь
процес його продукції, коли він починає функціонувати в одній з двох
стадій циркуляції, чи то як $Г$, чи то як $Т'$. Таким чином, кругобіг $П\dots{} П$
являв би собою тоді не лише періодичне відновлення продуктивного
капіталу, але так само й перерву його функції, процесу продукції, перерву,
що тривала, б доки буде закінчений процес циркуляції; замість відбуватись
безперервно, продукція відбувалась би скоками й поновлювалась би лише
по переміжках невизначеного часу, залежно від того, оскільки швидка
або повільно здійснюються обидві стадії процесу циркуляції. Так, напр.,
стоїть справа в китайського ремісника, що робить тільки на приватних
замовників і припиняє процес продукції, поки не буде нових замовлень.

У дійсності це має силу й для кожної окремої частини капіталу,
що є в русі, і всі частини капіталу послідовно пророблюють цей рух.
Напр., 10.000 ф. пряжі є тижневий продукт прядуна. Ці 10.000 ф.
пряжі цілком виходять із сфери продукції в сферу циркуляції; капітальна
вартість, що міститься в них, мусить цілком перетворитисьна грошовий капітал,
і поки вона лишається в формі грошового капіталу, вона не може ввійти
знову в продукційний процес; вона мусить спочатку ввійти в цируляцію
і знову перетворитись на елементи продуктивного капіталу $Р + Зп$, Процес
\index{ii}{0064}  %% посилання на сторінку оригінального видання
кругобігу капіталу є повсякчасні перерви, вихід з однієї стадії, перехід
у наступну; скидання однієї форми, буття в другій формі; кожна з цих
стадій не лише зумовлює другу, але разом з тим і виключає її.

Але безперервність є характеристична ознака капіталістичної продукції
і зумовлюється технічною основою її, хоч не завжди безумовно досяжна.
Подивімось, як у дійсності стоїть справа. В той час, коли, напр., 10.000 ф.
пряжі виступають на ринок як товаровий капітал і перетворюються на
гроші (хоч ці гроші будуть виплатним засобом, хоч купівельним
засобом, або навіть розрахунковими грішми), місце їх заступає в продукційному
процесі нова бавовна, вугілля тощо, отже, пряжа тут знову
зворотно перетворилася з грошової й товарової форми на форму продуктивного
капіталу, і в цій формі починає свою функцію; тимчасом як у
той самий час перші 10.000 ф. пряжі перетворюються на гроші, раніш спродуковані
10.000 ф. пряжі вже перебігають другу стадію своєї циркуляції
i зворотно перетворюються з грошей на елементи продуктивного капіталу.
Всі частини каіпталу по черзі пророблюють процес кругобігу, перебувають
одночасно на різних стадіях його. Таким чином промисловий капітал
у своєму безперервному кругобігу перебуває одночасно на всіх стадіях
його й у відповідних їм різних функціональних формах. Для тієї частини,
яка вперше перетворюється з товарового капіталу на гроші, починається
кругобіг $Т'\dots{} Т'$, тимчасом як для промислового капіталу, як для цілого,
що перебуває в русі, кругобіг $Т'\dots{} Т'$ уже перейдено. Однією рукою гроші
авансується, другою одержується; початок кругобігу $Г\dots{} Г'$ на одному
пункті є разом з тим його поворот на другому. Те ж саме має силу
й для продуктивного капіталу.

Справжній кругобіг промислового капіталу в своїй безперервності
є тому не лише єдність процесу циркуляції та процесу продукції, але
також і єдність усіх його трьох кругобігів. Але такою єдністю може він
бути лише остільки, оскільки кожна з різних частин капіталу може по
черзі переходити послідовні фази кругобігу, переходити з однієї фази,
з однієї функціональної форми в іншу, оскільки, отже, промисловий капітал,
як сукупність цих частин, одночасно перебуває в різних фазах
і функціях і таким чином одночасно пророблює усі три кругобіги. Чергування
кожної частини в часі зумовлюється тут чергуванням частин у
просторі, тобто поділом капіталу. Напр., в розчленованій фабричній
системі продукт завжди перебуває на різних ступенях процесу свого
творення й так само завжди переходить з однієї фази продукції до іншої,
А що індивідуальний промисловий капітал являє певну величину, що
залежить від засобів капіталіста й має для кожної галузі промисловости
певну мінімальну величину, то при поділі його мусять існувати певні числові
відношення. Величина наявного капіталу зумовлює розміри продукційного
процесу, його розміри зумовлюють розмір товарового й грошового
капіталу, оскільки вони функціонують поряд процесу продукції. Чергування
в просторі, що ним зумовлюється безперервність продукції, існує, однак,
тільки завдяки рухові частин капіталу, що в ньому вони одна по одній переходять
різні стадії. Чергування в просторі саме є лише результат чергування
\index{ii}{0065}  %% посилання на сторінку оригінального видання
в часі. Напр., коли товари не можна продати, коли рух $Т' — Г'$
припиняється для однієї частини, то кругобіг цієї частини переривається
і її не покривається засобами її продукції; зміна функцій наступних
частин, що виходять як $Т'$ з процесу продукції, затримується
попередніми. Коли це триває деякий час, то продукція обмежується, і
цілий процес припиняється. Кожна зупинка в послідовному русі частин
порушує послідовність їх у просторі, кожна зупинка на одній стадії
зумовлює більшу або меншу зупинку в цілому кругобігу не лише тієї
частини капіталу, що її рух спинився, але також і цілого індивідуального
капіталу.

Найближча форма, що в ній виявляється процес, є така послідовність
фаз, при якій перехід капіталу в нову фазу зумовлюється його виходом
з іншої фази. Тому кожний поодинокий кругобіг має також за вихідний
пункт і за пункт повороту одну з функціональних форм капіталу.
З другого боку цілий процес є в дійсності єдність трьох кругобігів, що є
різні форми, в яких виражається безперервність процесу. Для кожної
функціональної форми капіталу цілий кругобіг є її специфічний кругобіг,
і при цьому кожен з цих кругобігів зумовлює безперервність цілого процесу;
круговий рух однієї функціональної форми зумовлює круговий рух
інших форм. Для сукупного процесу продукції, особливо для суспільного
капіталу, є та неодмінна умова, що процес продукції є разом з
тим і процес репродукції, а значить, і процес кругобігу кожного з його
моментів. Різні частини капіталу послідовно перебігають різні стадії та
функціональні форми. В наслідок цього кожна функціональна форма, хоч
вона і репрезентує раз-у-раз іншу частину капіталу, пророблює одночасно
з іншими свій власний кругобіг. Одна частина капіталу, але
така, що завжди міняється, завжди репродукується, існує як товаровий
капітал, який перетворюється на гроші; друга частина існує як грошовий
капітал, який перетворюється на продуктивний; третя частина —
як продуктивний капітал, що перетворюється на товаровий капітал.
Постійну наявність усіх трьох форм упосереднює кругобіг цілого капіталу,
що саме й переходить ці три фази.

Отже, капітал, як ціле, одночасно перебуває в своїх різних фазах,
послідовно розміщених у просторі. Але кожна частина завжди переходить
по черзі з однієї фази, з однієї функціональної форми в іншу,
і так функціонує по черзі в усіх формах. Ці форми є такі поточні форми,
що їхню одночасність упосереднює їхня послідовність. Кожна форма
йде по другій й передує їй, так що поворот однієї частини капіталу
до однієї форми зумовлено поворотом другої частини до другої форми.
Кожна частина безупинно пророблює свій власний обіг, але в цій
формі завжди перебуває інша частина капіталу, і ці осібні обіги становлять
лише одночасні й послідовні моменти сукупного руху.

Лише в єдності трьох кругобігів здійснюється безперервність сукупного
процесу замість змальованої вище переривчастости. Сукупний
суспільний капітал завжди має цю безперервність, і його процес завжди
є єдність трьох кругобігів.
