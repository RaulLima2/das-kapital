\parcont{}  %% абзац починається на попередній сторінці
\index{ii}{0053}  %% посилання на сторінку оригінального видання
капітальної вартости $П п$люс додаткова вартість m, спродукована в
наслідок функціонування $П.

Т$ільки в кругобігу самого $Т'$ частина його $Т = П$ = капітальній вартості
може й мусить відокремитись від тієї частини $Т'$, що в ній існує додаткова
вартість, від додаткового продукту, що в ньому міститься додаткова
вартість, — може й мусить відокремитись незалежно від того, чи обидва
вони справді подільні, як от пряжа, чи ні, як от машина. Вони стають
подільні кожного разу, скоро $Т' п$еретворюється на $Г'$.

Коли цілий товаровий продукт можна поділити на самостійні однорідні
частинні продукти, як, прим., наші 10.000 ф. пряжі, і коли в
наслідок цього акт $Т' — Г'$ можна зобразити як суму послідовно вчинених
продажів, то капітальна вартість може функціонувати в товаровій
формі як Т, може відокремитись від $Т'$ раніше, ніж реалізується додаткова
вартість, отже, раніш, ніж реалізовано цілком усе $Т'$.

В 10.000 ф. пряжі в 500 ф. стерл., вартість 8.440 ф. пряжі = 422 ф.
стерлін. = капітальній вартості, відокремленій від додаткової вартости.
Коли капіталіст продає спочатку 8.440 ф. пряжі за 422 ф. стерл., то
ці 8.440 ф. пряжі репрезентують Т, капітальну вартість у товаровій
формі; додатковий продукт, що є, крім того, в $Т'$, а саме 1560 ф.
пряжі = додатковій вартості в 78 ф. стерл. ввійде в циркуляцію лише
пізніше; капіталіст міг би здійснити $Т — Г — Т Р Зп п$еред циркуляцією
додаткового продукту $т — г — т$.

Або коли б він спочатку продав 7440 ф. пряжі вартістю в 372 ф.
стерл., а потім 1000 ф. пряжі вартістю в 50 ф. стер., то першою частиною
Т можна було б покрити засоби продукції (сталу частину капіталу
с), а другою частиною Т — змінну частину капіталу v, робочу
силу, — а далі все відбувалось би, як і раніш.

Але коли відбуваються такі послідовні продажі і коли умови кругобігу
це дозволяють, то капіталіст замість поділити $Т'$ на с + у + m,
може зробити такий поділ і в аліквотних частинах $Т'$.

Наприклад, 7440 ф. пряжі = 372 ф. стерл., що як частини $Т'$
(10000 ф. пряжі = 500 ф. стерл.) репрезентують сталу частину капіталу,
своєю чергою можуть бути розкладені на 5535,360 ф. пряжі вартістю
в 276,768 ф. стерл., що покривають лише сталу частину капіталу, вартість
засобів продукції, зужиткованих у виготовленні 7.440 ф. пряжі; на 744 ф. пряжі
вартістю в 37,200 ф. стер., що покривають лише змінний капітал; на
1160.640 ф. пряжі вартістю в 58,032 ф. стерл., що, як додатковий продукт,
є носії додаткової вартости. Отже, з проданих 7.440 ф. пряжі
капіталіст може покрити вміщену в них капітальну вартість, продавши
6279,360 ф. пряжі за 313,968 ф. стерл., а вартість додаткового продукту
1160.640 ф. пряжі = 58,032 ф. стерл. витратити як дохід.

Так само може він і далі розкласти й відповідно до цього продати
1000 ф. пряжі = 50 ф. стерл. = змінній капітальний вартості: 744 ф.
пряжі за 37,200 ф. стерл. — стала капітальна вартість 1000 ф. пряжі;
\parbreak{}  %% абзац продовжується на наступній сторінці
