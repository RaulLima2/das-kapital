талісти обох категорій, але при цьому гроші в усякому разі мають походити
від цих капіталістів, бо ми вже закінчили обчислення з тими
грішми, що їх подали в циркуляцію робітники. Тут буває або так, що
капіталіст категорії II із свого грошового капіталу, який існує поряд капіталу
продуктивного, купує засоби продукції в капіталістів категорії І;
або, навпаки, капіталіст категорії І із свого грошового фонду, призначеного
на особисті витрати, не на витрачання як капітал, купує засоби
споживання в капіталістів категорії II. Як ми вже показали в відділі 1 і
II, ми мусимо в усякому разі припустити, що в руках капіталістів поряд
продуктивного капіталу є певні грошові запаси, — чи то для авансування
капіталу, чи то для витрачання доходу. Припустімо — пропорція не має
жодного значення для нашої цілі, — що капіталісти II авансують половину
грошей на закуп засобів продукції, щоб замістити свій сталий капітал,
а другу половину капіталісти І витрачають на споживання, а саме: підрозділ
II авансує 500 ф. стерл. і купує на них у І засоби продукції,
тим самим він заміщує in natura (разом з вищезгаданими 1000 ф. стерл.,
то походять від робітників) 3/4 свого сталого капіталу; підрозділ І на
одержані таким чином 500 ф. стерл. купує у II засоби споживання
і закінчує разом з тим циркуляцію т — г — т для половини тієї частини
свого товарового капіталу, яка складається з т, реалізує цей продукт
свій в фонді споживання. В наслідок цього другого процесу 500 ф. стерл.
повертаються назад до рук II як грошовий капітал, що його капіталісти
II мають поряд свого продуктивного капіталу. З другого боку, І для
половини тієї частини т свого товарового капіталу, що лежить ще у
нього як продукт, антиципує — раніш, ніж цю частину продано — витрачання
грошей в розмірі 500 ф. стерл. на закуп засобів споживання у II.
На ці самі 500 ф. стерл. II купує засоби продукції в І і таким чином
заміщує in natura ввесь свій сталий капітал (1000 + 500 + 500 = 2000),
тимчасом як І реалізував у засобах споживання всю свою додаткову
вартість. В загальному підсумку обмін товарів на суму в 4000 ф. стерл.
відбувся б за грошової циркуляції в 2000 ф. стерл. і при цьому вона
досягає такої величини лише тому, що, як подано у нас, увесь річний
продукт обмінюється разом, небагатьма великими порціями. Важлива
при цьому лише та обставина, що II підрозділ не лише знову перетворив
на форму засобів продукції свій сталий капітал, репродукований
у формі засобів споживання, але що до нього, крім того, повертаються
ті 500 ф. стерл., що їх він авансував для циркуляції на закуп засобів
продукції; і що І підрозділ так само не лише знову одержав у грошовій
формі, як грошовий капітал, що його можна безпосередньо перетворити
на робочу силу, свій змінний капітал, репродукований ним у формі засобів
продукції, але що до нього, крім того, повертаються знову ті
500 ф. стерл., що їх він витратив на закуп засобів споживання, перед продажем
частини додаткової вартости від свого капіталу, антиципуючи цей
продаж. Але вони повертаються до нього назад не в наслідок витрати,
що вже відбулася, а в наслідок дальшого продажу частини його товарового
продукту, що є носій половини його додаткової вартости.
