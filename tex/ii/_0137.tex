\parcont{}  %% абзац починається на попередній сторінці
\index{ii}{0137}  %% посилання на сторінку оригінального видання
дохід або зиск, не змінюючи власника, не циркулюючи далі. Тому такі
капітали можна назвати основними капіталами у власному значенні цього
слова. Різні підприємства потребують поділу вкладеного в них капіталу
на основний та обіговий в дуже різних пропорціях... Кожен ремісник
або фабрикант мусить деяку частину свого капіталу зв’язати в формі
засобів праці своєї галузі. Ця частина однак іноді дуже мала, іноді дуже
велика... Куди більша частина капіталу всіх цих ремісників (кравців, шевців,
ткачів) перебуває однак в циркуляції, то як заробітна плата їхніх
робітників, то як ціна їхнього сировинного матеріялу, і її треба оплатити
з зиском в ціні їхніх продуктів“\footnote*{
Secondly, it (capital) may be employed in the improvement of land, in the
purchase of useful machines and instruments of trade, or in such like things as
yield a revenue or profit without changing masters, or circulating any further. Such
capitals therefore, may very properly be called fixed capitals. Different occupations
require very different proportions between the fixed and circulating capitals employed
in them... Some part of the capital of every master artificer or manufacturer
must be fixed in the instruments of his trade. This part, however, is very small in
some, and very great in others.. The far greater part of the capital of all such master
artificers however is circulated, either in the wages of their workmen, or in the
price of their materials, and to be repaid with a profit by the price of the work".
}.

Не кажучи вже про дитяче визначення джерела зиску, хибність і заплутаність
видно ось з чого: для фабриканта-машинобудівника, напр.,
машина є продукт, що циркулює як товаровий капітал, або, кажучи словами
А. Сміса: „is parted with, changes masters, circulates further“ (відокремлюється,
змінює власника, циркулює далі). Отже, машина згідно з
його власним визначенням була б не основним, а обіговим капіталом. Ця
плутанина знову таки постає тому, що Сміс сплутує ріжницю між основним
і поточним капіталом, яка постає в наслідок неоднакових способів
циркуляції різних елементів продуктивного капіталу, з відмінностями
форми, що їх перебігає той самий капітал, оскільки він функціонує
в продукційному процесі як продуктивний капітал, а в сфері
циркуляції, навпаки, як капітал циркуляції, тобто як товаровий капітал
або як грошовий капітал. Тому ті самі речі, залежно від того місця, що
його вони мають у життьовому процесі капіталу, можуть, за А. Смісом,
функціонувати і як основний капітал (як засоби праці, елементи продуктивного
капіталу) і як „обіговий“ капітал, товаровий капітал (як продукт,
виштовхнутий з сфери продукції в сферу циркуляції).

Але А. Сміс сплутує разом з тим самі основи цього розподілу й суперечить
тому, з чого він почав кількома рядками вище цілий свій
дослід. Це саме сталось у реченні: „Є два способи застосувати капітал
так, щоб він давав своєму власникові дохід або зиск“, а саме — застосувати
його або як обіговий або як основний капітал. Тут мають на увазі,
очевидно, різні способи застосування різних і незалежних один від одного
капіталів, як, напр., капітали, що їх можна вкласти або в промисловість,
або в хліборобство. Але далі ми читаємо: „Різні підприємства
потребують поділу вкладеного в них капіталу на основний та обіговий
в дуже різних пропорціях“. Тепер основний та обіговий капітал є вже
\parbreak{}  %% абзац продовжується на наступній сторінці
