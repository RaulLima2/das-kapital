нення Г — Т може бути більш або менш значна. Хоч Г, як результат акту
Т — Г, репрезентує минулу працю, все ж Г для акту Г — Т може
репрезентувати перетворену форму товарів, що їх на ринку ще немає,
що лише будуть там у майбутньому, бо акт Г — Т має відбуватися
лише після того, як знову вироблено Т. Так само Г може
репрезентувати товари, що їх продукується одночасно з тим Т, що його
грошовий вираз воно є. Наприклад, в обміні Г — Т (купівля засобів
продукції) вугілля може купуватись раніше, ніж його видобудеться із
шахти. Оскільки г фігурує як акумульовані гроші, а не витрачається
як дохід, то воно може репрезентувати бавовну, що її випродукується
лише в наступному році. Так само стоїть справа з витрачанням доходу
в капіталіста, в акті г — т. Так само із заробітною платою Р = 50 ф.
стерл.; ці гроші є не лише грошова форма минулої праці робітників, але
разом з тим і асиґната на одночасну або майбутню працю, що
саме тепер реалізується або має реалізуватися в майбутньому. Робітник
може купити на них сурдут, що його зробиться лише протягом найближчого
тижня. Саме так стоїть справа щодо дуже великого числа доконечних
життьових засобів, що їх доводиться споживати майже безпосередньо в
момент їх продукції, щоб запобігти їхньому псуванню. Таким чином
робітник одержує в грошах, що в них видається йому його заробітну плату,
перетворену форму своєї власної майбутньої праці або праці інших робітників.
У частині його минулої праці капіталіст дає йому асиґнату на його
власну майбутню працю. Його власна одночасна або майбутня праця
являє той ще не наявний запас, що ним йому платиться за його минулу
працю. Тут цілком зникає уявлення про утворення запасу.

По-друге. У циркуляції Т — Г — Т Р Зп ті самі гроші переміщуються двічі; капіталіст одержує їх
спочатку як продавець і передає їх далі як покупець;
перетворення товару на грошову форму придається лише для того,
щоб з грошової форми знову перетворити його на товарову форму;
тому грошова форма капіталу, його буття як грошового капіталу,
є в цьому русі лише минущий момент; інакше кажучи, поки триває цей
рух, грошовий капітал, якщо він придається як купівельний засіб, виступає
лише як засіб циркуляції; як власне засіб виплати він виступає тоді,
коли капіталісти навзаєм купують один в одного, а потім доводиться
лише вирівняти балянс виплат.

По-третє, функціонування грошового капіталу, все одно, чи придається
він як простий засіб циркуляції, чи як засіб виплати, упосереднює лише
заміщення Т на Р і Зп, тобто заміщення пряжі, товарового продукту,
що є результат (відлічивши додаткову вартість, що її вживається як дохід)
продуктивного капіталу, на елементи продукції; отже, воно упосереднює
зворотне перетворення капітальної вартости з її товарової форми на
витворчі елементи цього товару; отже, кінець-кінцем, це функціонування
упосереднює лише зворотне перетворення товарового капіталу на продуктивний
капітал.
