синах, могли б самі собою та із своєї природи бути капіталом в
тій або іншій певній формі, основним або обіговим. Ми бачили в книзі
І, розділ V, що засоби продукції в кожному процесі, хоч при яких суспільних
умовах він відбувається, завжди поділяються на засоби праці й
предмет праці. Але тільки за капіталістичного способу продукції обидва
вони робляться капіталом, саме „продуктивним капіталом“, як це
визначено в попередньому розділі. Разом з тим ріжниця між засобом
праці й предметом праці, яка ґрунтується на природі процесу праці,
відбивається в новій формі, як ріжниця між основним капіталом та обіговим.
Лише відтепер річ, що функціонує як засіб праці, робиться основним
капіталом. Якщо вона своїми речовими властивостями може придаватись
і в інших функціях, крім функцій засобів праці, то вона буде
основним капіталом або не буде, залежно від відмінности свого функціонування.
Худоба як робоча худоба, є основний капітал; худоба на заріз
є сировинний матеріял, що, кінець-кінцем, як продукт, входить у
циркуляцію — отже, це не основний капітал, а обіговий.

Простий стан довгочасної фіксованости якогобудь засобу продукції
в повторюваних процесах праці, що між собою зв’язані й являють
безперервний ряд і тому становлять період продукції — тобто ввесь час
продукції, потрібний на те, щоб виготувати продукт, — цей стан довгочасної
фіксованости зумовлює цілком так само, як і основний капітал,
авансування з боку капіталіста на довший або коротший час, але не
перетворює його капіталу на основний капітал. Насіння, напр., зовсім не
є основний капітал, а лише сировинний матеріял, що його майже на
цілий рік фіксується в процесі продукції. Всякий капітал, поки він функціонує
як продуктивний капітал, фіксується в процесі продукції, отже,
фіксуються і всі елементи продуктивного капіталу, хоч яка буде їхня речова
форма, їхня функція та спосіб циркуляції їхньої вартости. Чи це фіксування
триває довший, чи коротший час, залежно від способу продукційного
процесу або бажаного корисного ефекту, не це утворює ріжницю між
основним та обіговим капіталом 20).

Частина засобів праці, — куди належать і загальні умови праці — або
прикріплюється до певного місця, коли вона як засіб праці входить у процес
продукції, зглядно, коли її підготовлюється до продуктивної функції, як
напр., машини. Або частину засобів праці з самого початку продукується
в такій нерухомій, зв’язаній з місцем формі, як, напр., земельні
меліорації, фабричні будівлі, домни, канали, залізниці тощо. Постійна
зв’язаність засобів праці з продукційним процесом, що в ньому вони
повинні функціонувати, зумовлюється тут уже речовим способом їхнього
існування. З другого боку, засоби праці можуть фізично завжди переміщуватись,
рухатись, і все ж бути завжди в продукційному процесі, як,
напр., локомотив, судно, робоча худоба і т. ін. Ні нерухомість не надає
їм у першому випадку характеру основного капіталу, ні рухомість не

20) Що дуже важко дати визначення основного та обігового капіталу, то пан
Льоренц Штайн каже, що ця ріжниця придається лише для популярности викладу.
