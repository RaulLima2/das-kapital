\parcont{}  %% абзац починається на попередній сторінці
\index{ii}{0302}  %% посилання на сторінку оригінального видання
те, що не відбувається жодних переворотів у вартості складових частин
продуктивного капіталу. А втім, щодо відхилення цін від вартостей, то
ця обставина не може справити будь-якого впливу на рух суспільного
капіталу. При цьому в цілому обмінювалось би ті самі маси продуктів,
що й раніше, хоч поодиноким капіталістам при цьому дістались би пайки
вартости, вже непропорціональні їхнім відповідним авансуванням і тим масам
додаткової вартости, що кожен із них випродукував. Щодо переворотів
у вартості, то коли вони мають загальний характер і розподіляються
рівномірно, вони не спричиняють жодних змін у відношеннях між
складовими частинами вартости сукупного річного продукту. Навпаки, коли
вони мають частинний характер і розподіляються нерівномірно, то являють
собою розлади, що їх, поперше, можна зрозуміти як такі, лише розглядаючи
їх, як відхили від незмінних відношень вартости; але, подруге,
коли визначено закон, що згідно з ним одна частина річного продукту заміщує
сталий, а друга — змінний капітал, то в цьому законі нічого не
змінила б революція в вартости хоч сталого, хоч змінного капіталу. Вона
змінила б лише відносну величину тих частин вартости, що функціонують
у тій або іншій якості, бо на місце первісних вартостей виступили
б вартості іншої величини.

Поки ми розглядали продукцію вартости та вартість продукту капіталу,
як індивідуального капіталу, для нашої аналізи натуральна форма
товарового продукту була цілком байдужа, — було цілком байдуже, напр.,
чи складається він з машин, чи з хліба, чи з дзеркал. Всі ці натуральні
форми були б просто прикладом для нас, і перша-ліпша галузь продукції
однаково була б придатна, як ілюстрація. Нам доводилось мати
справу безпосередньо з самим процесом продукції, що в кожному пункті
виступав як процес індивідуального капіталу. Оскільки ми розглядали
репродукцію капіталу, нам досить було того припущення, що частина
товарового продукту, яка являє капітальну вартість, має в сфері циркуляції
змогу зворотно перетворитись на елементи її продукції, і значить,
на форму продуктивного капіталу; цілком так само, як досить було нам
того припущення, що робітник і капіталіст знаходять на ринку товари,
на які вони витрачають заробітну плату й додаткову вартість. Але цей
суто-формальний спосіб викладу вже недостатній, коли ми розглядаємо
сукупний суспільний капітал і вартість його продукту. Зворотне перетворення
однієї частини вартости продукту на капітал, перехід другої частини в
сферу особистого споживання кляси капіталістів і кляси робітників, становить
рух у межах самої вартости продукту, яка є результат сукупного
капіталу; і цей рух є не лише заміщення вартости, а й заміщення речовини,
а тому він так само зумовлюється співвідношенням складових
частин вартости суспільного продукту, як і споживною їхньою вартістю,
їхньою речовою формою.

Проста репродукція\footnote{
З рукопису VIII.
} в незмінному маштабі являє абстракцію в тому
розумінні, що, з одного боку, на базі капіталістичної продукції відсутність
\index{ii}{0303}  %% посилання на сторінку оригінального видання
будь-якої акумуляції або репродукції в поширеному маштабі є
неймовірне припущення, а з другого боку, відношення, що в них відбувається
продукція, в різні роки не лишаються абсолютно незмінні (а таке
є наше припущення). Наше припущення те, що суспільний капітал даної
вартости, як минулого року, так і цього року знов дає таку саму масу товарових вартостей і
задовольняє таку саму кількість потреб, хоча б форми товарів і змінилися в процесі репродукції. А
проте, оскільки відбувається акумуляція, проста репродукція завжди становить частину останньої,
отже, її можна розглядати окремо, вона — реальний чинник акумуляції. Вартість річного продукту може
зменшитись, хоч маса споживних вартостей лишається та сама, вартість може лишатись та сама, хоч маса
споживних вартостей меншає; маса вартости й маса репродукованих
споживних вартостей можуть одночасно меншати. Все це залежить
від того, що репродукція відбувається або при сприятливіших умовах,
ніж були раніше, або при гірших умовах, а останні можуть призвести до
неповної — недостатньої — репродукції. Однак усе це стосується лише до
кількісного боку різних елементів репродукції, а не до тієї ролі, що її
вони відіграють в цілому процесі як капітал, що його репродукується,
або як дохід, уже репродукований.

II. Два підрозділи суспільної продукції\footnote{
В головному рукопису II. Схема рукопису VIII.
}

Цілий продукт, отже, і вся продукція суспільства, розпадається на
два великі підрозділи:

I. Засоби продукції, товари, які мають таку форму, що в ній
вони мусять ввійти або принаймні можуть ввійти в продуктивне споживання.

II. Засоби споживання, товари, які мають таку форму, що в
ній вони входять в особисте споживання кляси капіталістів і кляси робітників.

В кожному з цих підрозділів усі різні галузі продукції, належні до
того або того підрозділу, становлять єдину велику галузь продукції,
в одному разі — продукції засобів продукції, в другому — засобів споживання.
Ввесь капітал, застосований в кожній з цих двох галузей продукції,
становить окремий великий підрозділ суспільного капіталу.

У кожному підрозділі капітал розпадається на дві складові частини:

1) Змінний капітал. Розглядуваний щодо вартости він дорівнює
вартості суспільної робочої сили, застосованої в цій галузі продукції,
отже, дорівнює сумі заробітної плати, сплаченої за цю робочу силу.
Розглядуваний з речового боку, він складається з самої діючої робочої
сили, тобто з живої праці, пущеної в рух цією капітальною вартістю.

2) Сталий капітал, тобто вартість усіх засобів продукції, застосованих
для продукції в цій галузі. Вони й собі розпадаються на основний
капітал: машини, знаряддя праці, будівлі, робочу худобу і т. ін.,
\parbreak{}  %% абзац продовжується на наступній сторінці
