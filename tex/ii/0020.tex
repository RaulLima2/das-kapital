грошовий капітал, воно виражається як вартість, що сама з себе зросла
в своїй вартості, отже, як вартість, що має властивість зростати в своїй
вартості, породжувати більше вартости, ніж вона сама має. Г стало капіталом
у наслідок свого відношення до другої частини Г', як частини,
утвореної ним, посталої з нього як причини, — через своє відношення до
неї, як до наслідку, що Г є його причина. Таким чином Г' з’являється
як сума вартости, що сама в собі диференційована, що в собі самій функціонально
(в понятті) себе відрізняє, — сума вартости, що виражає капіталістичне
відношення.

Але це тільки виражено як результат, без посередництва того процесу,
що зумовив цей результат.

Частини вартости як такі не відрізняються якісно між себе за винятком
того, коли вони виступають як вартості різних предметів, конкретних
речей, отже, в різних споживних формах, а, значить і, як вартості
різних товарових тіл — ріжниця, що постає не з них самих як простих
частин вартости. У грошах згасає всяка відмінність товарів, бо вони є
саме для всіх їх спільна еквівалентна форма. Грошова сума в 500 ф.
стерл. складається цілком з однойменних елементів по 1 ф. стерл. А що
в простому бутті цієї грошової суми зникла посередня ланка її походження
і зник будь-який слід специфічної ріжниці, що її мають різні
складові частини капіталу в продукційному процесі, то ріжниця існує
між головною сумою (англійською мовою — principal), що дорівнює
авансованому капіталові в 422 ф. стерл., і надлишковою сумою вартости
в 78 ф. стерл. лише в понятті. Припустімо, напр., що Г' = 110 ф. стерл.,
і з них 100 = Г, головній сумі, а 10 = М, додатковій вартості. Тут
є абсолютна однорідність, отже, безвідмінність у поняттях, між обома
складовими частинами суми в 110 ф. стерл. Які завгодно 10 ф. стерл.
є завжди 1/11 цілої суми в 110 ф. стерл., все одно, чи вони є 1/10 авансованої
головної суми в 100 ф. стерл., чи надлишок над нею в 10 ф. стерл.
Тому головна сума й приростова сума, капітал і додаткова сума, можуть
бути виражені як дробові частини цілої суми; в нашому прикладі 10/11 становлять
головну суму або капітал, 1/11 становить додаткову суму. Отже, це є
іраціональний вираз капіталістичного відношення, що в ньому тут наприкінці
свого процесу реалізований капітал з’язляється в своєму грошовому виразі.

Звичайно це має силу й щодо Т' (= Т + т). Але з тією ріжницею, що
Т', в якому Т і т являють теж лише пропорційні частини вартости тієї
самої однорідної товарової маси, вказує на своє походження з П, що Т'
безпосередньо є його продукт, тимчасом як в Г', формі, що походить
безпосередньо з циркуляції, зник прямий зв’язок з П.

Ця іраціональна ріжниця між головною і приростовою сумою, — ріжниця,
яка є в Г', оскільки воно виражає результат руху Г... Г', відразу зникає,
скоро Г' знову активно починає функціонувати як грошовий капітал, отже,
коли воно не фіксується як грошовий вираз промислового капіталу, ви-
