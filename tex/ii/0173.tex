швидше припливає назад в грошовій формі еквівалент зношеної її частини.
Інша справа з обіговим капіталом. Не тільки капітал треба вкладати на
довший час відповідно до протягу робочого періоду, але треба також
повсякчас авансувати новий капітал на заробітну плату, сировинні та
допоміжні матеріяли. Отже, уповільнений зворотний приплив впливає
неоднаково на основний і обіговий капітал. Хоч буде зворотний приплив
повільніший, хоч швидший, основний капітал і далі діє. Навпаки, обіговий
капітал при уповільненому зворотному припливі стає нездатний до функціонування,
якщо його закріплено в формі непроданого або неготового,
ще непридатного до продажу продукту, і якщо немає наявного додаткового
капіталу, щоб відновити його in natura. — „Тимчасом як селянин
голодує, худоба його росте й гладшає. Було досить дощів, і паша стала
буйна. Індійський селянин умре з голоду біля свого жирного бика. Приписи
забобонів суворі проти поодиноких людей, але вони підтримують
суспільство; зберігання робочої худоби забезпечує поступ хліборобства,
а тим самим і джерела майбутніх засобів існування й майбутнього багатства.
Можливо, це звучить жорстоко й сумно, але це так: в Індії легше
замінити людину, ніж бика“. (Return, East India. Madras and Orissa
Famine. № 4, p. 4). Порівняйте з цим таке речення Манара-ДармаСестри,
розділ X, стор. 862: „Жертва життям без нагороди, щоб зберегти
життя жерцеві або корові... може забезпечити блаженство цих
родів низького походження“.

Звичайно неможливо подати на ринок п’ятилітню тварину, раніше
ніж їй буде п’ять років. Але в певних межах можна, змінюючи догляд
за худобою, підгодувати її протягом коротшого часу до її призначення.
Саме це й зробив Беквел. Раніше англійські вівці, як і французькі ще
1855 року, не були готові на заріз до четвертого або п’ятого року. За
системою Беквела, вівцю можна відгодувати протягом одного року і в
усякому разі вона цілком достигає до двох років. Старанно добираючи вівці,
Беквел, фармер з Дішлей Ґренджа, довів кістяк овець до мінімуму, потрібного
для їхнього існування. Ці його вівці зветься ньюлейстерські.
„Вівчар може тепер подати на ринок три вівці за той самий час, за який
раніше давав одну, і ці його вівці товщі, кругліші й розвиненіші
в тих частинах, що дають найбільше м’яса. Майже ціла вага їхня є чисте
м’ясо.“ (Lavergne, The Rural Economy of England etc. 1855, p. 20)

Методи, що скорочують робочий період, в різних галузях продукції
можна застосовувати дуже неоднаковою мірою, й вони не вирівнюють
ріжниці в часі різних робочих періодів. Щоб залишитись при нашому
прикладі, хай через застосування нових робочих машин абсолютно скорочується
робочий період, потрібний на виготовлення одного паровоза.
Але коли в наслідок удосконалення процесу прядіння кількість щоденно
й щотижнево вироблюваного готового продукту збільшиться ще швидше,
ніж у машинобудівництві, то відносно, порівняно з прядінням, довжина
робочого періоду в машинобудівництві збільшиться.
