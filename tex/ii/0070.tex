Г — Т Р Зп, що з них складаються Зп, засоби продукції, також буде
чужим діющим товаровим капіталом. Отже, з погляду продавця, тут
відбувається Т' — Г', перетворення товарового капіталу на грошовий. Але
це не має абсолютного значення. Навпаки. В процесі своєї циркуляції,
де промисловий капітал функціонує або як гроші, або як товар, кругобіг
промислового капіталу, — хоч він виступає як грошовий капітал, хоч як
товаровий капітал — перехрещується з циркуляцією товарів, найрізнорідніших
способів суспільної продукції, оскільки вони є разом з тим товарова
продукція. Хоч ці товари є продукт такої продукції, яка ґрунтується
на рабстві, хоч продукт селян (китайці, індійські райоти),
громадської продукції (голландська Ост-Індія), державної продукції (як
основана на кріпацтві продукція, що трапляється в раніші періоди російської
історії), хоч продукції напівдиких мисливських народів і т. ін.
— все одно: як товари й гроші протистоять вони грошам і товарам,
що в них виявляється промисловий капітал і входять у кругобіг його так само,
як і в кругобіг додаткової варюсти, що міститься в товаровому капіталі,
оскільки її витрачається як дохід, — отже, входять в обидві галузі
циркуляції товарового капіталу. Характер продукційного процесу, що
з нього вони походять, не має значення; як товари вони функціонують
на ринку, як товари входять вони в кругобіг промислового капіталу, так
само, як і в циркуляцію додаткової вартости, що є в ньому. Отже, всебічний
характер їхнього походження, наявність ринку як світового ринку
— ось що є відзначна риса процесу циркуляції промислового капіталу.
Те, що має силу для чужих товарів, має силу також і для чужих грошей: так
само, як товаровий капітал протистоїть грошам лише як товар, так і ці
гроші протистоять йому лише як гроші; гроші функціонують тут як
світові гроші.

Але тут треба зазначити дві обставини.

Поперше. Товари (Зп), скоро закінчився акт Г — Зп, перестають бути
товаром і стають одним з способів буття промислового капіталу в його
функціональній формі П, продуктивним капіталом. Але разом з тим
зникають сліди їхнього походження; вони існують далі тільки як форми
існування промислового капіталу, вони є в його складі. Однак, щоб
їх замістити, потрібна репродукція їх; в цьому розумінні капіталістичний
спосіб продукції зумовлено способами продукції, що перебувають на
іншій, ніж він, стадії розвитку. Але його тенденція в тому, щоб по змозі
всяку продукцію перетворити на товарову продукцію; його головний засіб
для цього є саме оце втягнення цих способів продукції в його процес
циркуляції; а розвинена товарова продукція вже сама є капіталістична
продукція. Втручання промислового капіталу всюди прискорює це перетворення,
а разом з тим і перетворення всіх безпосередніх продуцентів
на найманих робітників.

Подруге. Товари, що входять у процес циркуляції промислового
капіталу (сюди належать і доконечні засоби існування, що на них пере-
