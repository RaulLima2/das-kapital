купець товару, так і капіталіст І проти II функціонує тут лише як продавець товару. І на 1000
грошей, призначених функціонувати як змінний капітал, спочатку купив робочу силу вартістю в 1000;
отже, він одержав еквівалент за свої 1000 v, віддані в грошовій формі; тепер гроші належать
робітникові, що витрачає їх на акти купівлі в II; ці гроші, що потрапили таким чином до каси II, І
може одержати знову, лише виловлюючи їх назад через продаж товарів на таку саму суму вартости.

Спочатку І мав певну грошову суму = 1000, призначену функціонувати
як змінна частина капіталу; вона функціонує як така в наслідок перетворення її на робочу силу такого
ж розміру вартости. Але робітник дав йому, як результат продукційного процесу, певну масу товарів
(засобів продукції) вартістю в 6000, що з них 1/6, або 1000, своєю вартістю
являє еквівалент авансованої в грошах змінної частини капіталу. Як перше,
в своїй грошовій формі, так і тепер в своїй товаровій формі, змінна
капітальна вартість не функціонує як змінний капітал; вона може так
функціонувати лише після того, як перетвориться на живу робочу силу
і лише протягом того часу, поки ця остання функціонує в продукційному
процесі. В грошовій формі, змінна капітальна вартість була лише потенціяльним
змінним капіталом. Але ця вартість перебувала в такій формі,
що в ній її можна було перетворити безпосередньо на робочу силу.
В товаровій формі, та сама змінна капітальна вартість є покищо лише
потенціяльна грошова вартість; її можна знову відновити в первісній
грошовій формі лише через продаж товару, отже, в даному разі, в наслідок
того, що II купує на 1000 товару в І. Рух циркуляції тут такий:
1000 v (гроші) — робоча сила вартістю в 1000—1000 в товарі (еквівалент
змінного капіталу) — 1000 v (гроші); отже, Г — Т... Т — Г (= Г — Р...
Т — Г). Самий процес продукції, що припадає між Т... Т, не належить до
сфери циркуляції; він не з’являється в обміні різних елементів річної
репродукції одних на одні, хоч цей обмін включає репродукцію всіх
елементів продуктивного капіталу, так його сталого елементу, як і змінного,
робочої сили. Всі аґенти цього обміну виступають як лише покупці
або продавці, або як ті й ці; робітники виступають в ньому лише як
покупці товару; капіталісти — навперемінки як покупці й продавці, а в
певних межах — лише однобічно як покупці товару або однобічно як продавці
товару.

Результат такий: І має змінну частину вартости свого капіталу знову
в грошовій формі, що тільки з неї й можна перетворити цю частину
вартости безпосередньо на робочу силу, тобто знову має її в тій єдиній
формі, що в ній її справді можна авансувати як змінний елемент його
продуктивного капіталу. З другого боку, щоб мати змогу знову виступити
як покупець товару, робітник тепер мусить уперед знову виступити
як продавець товару, як продавець своєї робочої сили.

Щодо змінного капіталу категорії II (500 II v) процес циркуляції
між капіталістами й робітниками тієї самої кляси продукції виступає в
безпосередній формі — оскільки ми розглядаємо його як процес, що відбувається
між збірним капіталістом II і збірним робітником II.
