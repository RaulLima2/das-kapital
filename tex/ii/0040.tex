Щоб кругобіг відбувався нормально, Т' мусить продаватись за своєю
вартістю і геть усе без остачі. Далі, Т — Г — Т має в собі не лише заміщення
одного товару другим, але заміщення при тих самих відношеннях
вартости. Ми припустили, що це тут саме так і відбувається. Але в дійсності
вартості засобів продукції змінюються; саме капіталістичному способові
продукції властива постійна зміна відношень вартости в наслідок постійних
змін у продуктивності праці, що й характеризує капіталістичну
продукцію. Тут ми лише згадуємо про цю зміну в вартості чинників
продукції, а дослідимо її далі. Перетворення елементів продукції на
товаровий продукт, П на Т', відбувається в сфері продукції; зворотне
перетворення Т' на П — у сфері циркуляції. За посередника йому
править проста товарова метаморфоза. Але своїм змістом це зворотнє
перетворення є момент у процесі репродукції, розглядуваному як ціле.
Т — Г — Т як форма циркуляції капіталу має в собі функціонально визначений
обмін речовин. Далі перетворенням Т — Г — Т зумовлено те, що Т
= елементам продукції товарової маси Т', і що ці їхні взаємні первісні
відношення вартостей лишаються незмінні. Отже, тут припускається не лише
те, що товари купується за їхню вартість, а також і те, що протягом
кругобігу вони не зазнають жадної зміни вартости; а де цього немає,
там процес не може відбуватись нормально.

Г в Г... Г' є первісна форма капітальної вартости, яка скидає з себе
цю форму, щоб знову потім її набрати. Г в П... Т' — Г' — Т... П
є форма, що її набирається лише в процесі і в межах цього самого
процесу знову скидається. Грошова форма з’являється тут лише як минуща
самостійна форма вартости капіталу; капітал у формі Т' так само прагне
набути цієї грошової форми, як капітал у формі Г', скоро тільки він перетворився
на цю форму нібито на лялечку, прагне скинути її, щоб знову перетворитися
на форму продуктивного капіталу. Поки цей капітал застигає в
грошовій формі, він не функціонує як капітал, а тому не зростає в своїй
вартості; капітал лежить без діла. Г діє тут як засіб циркуляції, але як засіб
циркуляції капіталу. Позірна самостійність, що її має грошова форма
капітальної вартости в першій формі їі кругобігу (грошового капіталу),
зникає в цій другій формі, що таким чином править за критику форми І і
зводить її лише на особливу форму. Коли друга метаморфоза Г — Т наражається
на перешкоди (напр., коли немає на ринку засобів продукції),
то кругобіг, перебіг процесу репродукції переривається так само, як
і тоді, коли капітал лежить нерухомо у формі товарового капіталу. Але
ріжниця тут ось яка: капітал у грошовій формі може довший час чекати,
ніж у минущій товаровій формі. Не функціонуючи як грошовий капітал,
він, однак, не перестає бути грішми; але він перестає бути товаром
і взагалі споживною вартістю, коли занадто довго затримується в своїй
функції товарового капіталу. Подруге, в грошовій формі може він,
замість своєї первісної форми продуктивного капіталу, набрати іншої,
тимчасом як у формі Т' він взагалі не може зрушити з місця.

Т' — Г' — Т має в собі лише для Т', відповідно до його форми,
акти циркуляції, що являють моменти його репродукції. Але, щоб відбу-
