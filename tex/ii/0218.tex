продукт, та якщо цей продукт знову входить як елемент продукції у
другу галузь продукції і pro tanto звільняє тут капітал. В обох випадках
капітал, втрачений для X — для заміщення якого X справляє тиск на грошовий
ринок — можуть дати йому товариші в ділових справах як новий
додатковий капітал. В такому разі відбувається лише переміщення.

Навпаки, коли ціна продукту підвищується, то з сфери циркуляції
привлащується частину капіталу, що її не авансовано. Вона не є
органічна частина капіталу, авансованого на процес продукції, а тому,
коли підприємство не поширюється, вона становить виділений капітал.
А що тут припущено, що ціни елементів продукту дано раніш, ніж він
як товаровий капітал вступив на ринок, то до підвищення цін тут могла б
спричинитись справжня зміна вартости, оскільки ця зміна вартости мала б
зворотний вплив, напр., коли б сировинні матеріяли в дальшому
подорожчали. В цьому разі капіталіст X виграв би на своєму продукті,
що циркулює як товаровий капітал, і на продукційному запасі, що є в
нього. Цей виграш дав би йому додатковий капітал, тепер потрібний для
того, щоб провадити далі підприємство при нових підвищених цінах елементів
продукції.

Або підвищення цін є лише тимчасове. Тоді те, що на боці капіталіста
X потрібне як додатковий капітал, виступає на боці другого капіталіста
як звільнений капітал, оскільки його продукт є елемент продукції
для інших галузей підприємств. Що один втратив, те інший виграв.

Розділ шістнадцятий

Оборот змінного капіталу

І. Річна норма додаткової вартости

Припустімо обіговий капітал в 2500 ф. стерл., а саме 4/5 = 2000 ф.
стерл. сталого капіталу (матеріяли продукції) і 1/5 = 500 ф. стерл. змінного
капіталу, витрачуваного на заробітну плату.

Період обороту хай дорівнює 5 тижням; робочий період = 4 тижням;
період циркуляції = 1 тижневі. Тоді капітал І = 2000 ф. стерл. і складається
з 1600 ф. стерл. сталого капіталу і 400 ф. стерл. змінного; капітал
ІІ = 500 ф. стерл., з них 400 ф. стерл. сталого капіталу і 100 ф.
стерл. змінного капіталу. Протягом кожного робочого тижня витрачається
капітал в 500 ф. стерл. Протягом року, що складається з 50 тижнів,
виготовлюється річний продукт в 500 X 50 = 25000 ф. стерл. Отже, капітал
І в 2000 ф. стерл., що весь час застосовується в робочому періоді,
обертається 12,5 разів. 2000 X 12,5 = 25000 ф. стерл. З цих 25000 ф.
стерл. 4/5 = 20000 ф. стерл. сталого капіталу, витраченого на засоби
продукції, і 1/5 = 5000 ф. стерл. змінного капіталу, витраченого на за
