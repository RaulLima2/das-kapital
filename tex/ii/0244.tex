тих, хто їх утримував протягом трьох років. Як 3 роки стосуються
до протягу життя одного здорового покоління, прим., до 40 років,
так стосується величина й значення дійсного багатства, акумульований
капітал, навіть найбагатшої країни, до її продуктивної сили, до продуктивних
сил одного лише покоління людей; не до того, що вони могли б
випродукувати за розумного ладу однакової для всіх забезпечености, а
особливо при кооперованій праці, а до того, що вони дійсно абсолютно
продукують за недосконального ладу незабезпечености, що приводить до
збентеження!.. І для того, щоб зберегти й увічнити в сучасному стані вимушеного
розподілу цю на позір величезну масу наявного капіталу, або радше,
щоб зберегти й увічнити здобуту за її допомогою владу й монополію
над продуктом річної праці, мусить увічнитись весь цей страшенний
механізм, порочність, злочинність і злидні незабезпечености. Нічого не
можна акумулювати, поки не задовольниться неодмінні потреби, а великий
потік людських нахилів прямує до задоволення; звідси порівняно незначний
розмір дійсного багатства суспільства в кожний даний момент.
Це — вічний кругобіг продукції та споживання. При такій величезній масі
річної продукції та споживання навряд чи можна було б обійтися без
пригорщі дійсної акумуляції; і все ж головну увагу звернуто не на
масу продуктивних сил, а на цю пригорщ акумуляції. Але небагато
людей захопили цю пригорщ і перетворили її на знаряддя, щоб
привласнювати рік-у-рік відновлювані продукти праці великої маси людей...
Звідси надзвичайна важливість такого знаряддя для цих небагатьох...
Близько третини національного річного продукту відбирається тепер від
продуцентів під назвою громадських податків, і споживають її непродуктивно
люди, що не дають за те жодного еквіваленту, тобто такого, що
мав би значення еквіваленту для продуцентів... Маса людей з подивом
дивиться на акумульовані багатства, особливо коли зосереджені вони в
руках небагатьох осіб. Але щорічно продуковані маси продуктів, як
вічні та незчисленні хвилі могутнього потоку, ринуть далі й зникають у
забутному океані споживанння. І однак це вічне споживання зумовлює
не лише всі втіхи, а й існування цілого людського роду. Кількість та
розподіл цього річного продукту насамперед повинні бути за об’єкт дослідження.
Дійсна акумуляція має цілком другорядне значення та й його
вона набирає майже виключно в наслідок свого впливу на розподіл річного
продукту... Дійсну акумуляцію та розподіл завжди розглядається тут
(у досліді Томпсона) у зв'язку з продуктивними силами і як їм підпорядковану.
Майже в усіх інших системах продуктивні сили розглядалось
в зв’язку та підпорядковано акумуляції та увічненню наявного
способу розподілу. Порівняно з збереженням цього наявного способу
розподілу вважається за неварті уваги завжди відновлювані злидні або
добробут цілого людського роду. Увічнення здобутків насильства, обману
й випадковости назвали забезпеченістю, і щоб зберегти цю вигадану
забезпеченість, без жалю приносять у жертву всі продуктивні сили
людського роду“. (Там же, стор. 440—443).
