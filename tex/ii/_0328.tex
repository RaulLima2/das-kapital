\parcont{}  %% абзац починається на попередній сторінці
\index{ii}{0328}  %% посилання на сторінку оригінального видання
спродукованій протягом року в підрозділі II) плюс змінна капітальна
вартість І, репродукована протягом року, і новоспродукована додаткова
вартість І (тобто плюс вартість, спродукована протягом року в підрозділі
І).

Отже, припускаючи просту репродукцію, вся вартість спродукованих
протягом року засобів споживання дорівнює новоспродукованій річній
вартості, тобто дорівнює всій вартості, спродукованій суспільною працею
протягом року, і мусить їй дорівнювати, бо при простій репродукції
споживається всю цю вартість.

Цілий суспільний робочий день розпадається на дві частини: 1) доконечна
праця; протягом року вона утворює вартість в $1500 v$; 2) додаткова
праця; вона утворює додаткову вартість в $1500 m$. Сума цих
вартостей = 3000, дорівнює вартості спродукованих протягом року засобів
споживання в 3000. Отже, вся вартість спродукованих протягом
року засобів споживання дорівнює всій вартості, що її продукує цілий
суспільний робочий день протягом року, дорівнює вартості суспільного змінного
капіталу плюс суспільна додаткова вартість, дорівнює цілому річному
новому продуктові.

Але ми знаємо, що хоч ці дві величини вартости збігаються, а проте,
зовсім не всю вартість товарів II, засобів споживання, спродуковано
в цьому підрозділі суспільної продукції. Вони збігаються, бо стала
капітальна вартість, що знову з’являється в II, дорівнює вартості (змінній
капітальній вартості плюс додаткова вартість), новоспродукованій
в І; тому І ($v + m$) може купити в II ту частину продукту, яка для продуцентів
його (в підрозділі II) репрезентує сталу капітальну вартість. Відси
видно, чому, хоч для капіталістів II вартість їхнього продукту розпадається
на $c + v + m$, з погляду суспільства вартість цього продукту
можна розкласти на $v + m$. Але справа стоїть так лише тому, що тут
II c дорівнює І ($v + m$), і що ці дві складові частини суспільного продукту
через обмін навзаєм обмінюються своїми натуральними формами,
так що після такого обміну II c знову існує в формі засобів продукції,
а І ($v + m$), навпаки, — в формі засобів споживання.

Саме ця обставина й дала А. Смісові нагоду твердити, що вартість
річного продукту розкладається на $v + m$. Це правильно 1) лише для
тієї частини річного продукту, яка складається з засобів споживання, і
2) правильно не в тому розумінні, що всю цю вартість спродуковано в II і
що тому вартість продукту II дорівнює змінній капітальній вартості, авансованій
в II, плюс додаткова вартість спродукована, в II. А лише в тому
розумінні, що II ($c + v + m$) = II ($v + m$) + І ($v + m$), або тому, що
II c = І ($v + m$).

З цього далі випливає ось що:

Хоч суспільний робочий день (тобто праця, витрачена цілою робітничою
клясою протягом цілого року), так само, як і кожен індивідуальний
робочий день, розпадається лише на дві частини, а саме — на доконечну
працю плюс додаткову працю, отже, хоч і вартість, спродукована цим
робочим днем, теж розпадається лише на дві частини, а саме — на змінну
\parbreak{}  %% абзац продовжується на наступній сторінці
