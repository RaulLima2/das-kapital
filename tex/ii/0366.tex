рити їх на банкноти — і тоді ввесь його змінний капітал прямо, без
дальшого посередництва II, знову опиняється в його руках у грошовій
формі.

Але вже при цьому першому процесі річної репродукції сталася зміна
в кількості грошей, що дійсно або віртуально належать циркуляції. Ми
припустили, що IIс купив 5v (І з) як матеріял, а 3 як грошову форму
змінного капіталу І з знову витратив в межах II. Отже, з тієї маси грошей,
що її дано в наслідок нової продукції грошей, 3 лишились в межах
II й не повернулись до І. Згідно з нашим припущенням, II задовольнив
свою потребу в грошовому матеріялі. 3 лишаються в його руках як
золотий скарб. А що вони не можуть становити будь-якого елемента
його сталого капіталу й що, далі, II вже раніш мав достатній грошовий
капітал на закуп робочої сили; що, далі, за винятком елемента зношування,
цими додатковими 3 з не доводиться виконувати жодної функції в межах II,
на частину якого їх обмінено (вони могли б служити лише для того,
щоб pro tanto покривати елемент зношування тоді, коли IIс (І) менше,
ніж ІІс (2), а це буває випадково); що, з другого боку, саме за винятком
елемента зношування, ввесь товаровий продукт ІІс треба обміняти на
засоби продукції I (v + m), — то ці гроші цілком доводиться перенести
з ІІс в IIm, хоч це останнє буде в доконечних засобах існування або в
засобах розкошів, і, навпаки, відповідну товарову вартість доводиться
перенести з IIm в IIс. Результат: частина додаткової вартости нагромаджується
як грошовий скарб.

На другий рік репродукції, коли таку саму частину щорічно продукованого
золота й далі зуживається як матеріял, 2 знову повернуться
до І з, а 3 заміститься in natura, тобто знову звільняться у II як скарб
і т. ін.

Взагалі щодо змінного капіталу: капіталістові І з, як і всякому
іншому, завжди доводиться авансувати цей капітал в грошах на закуп
праці. На це v не йому, а його робітникам доводиться купувати в II; отже,
ніколи не може бути такого випадку, щоб він виступав як покупець,
тобто подав гроші в II без ініціятиви II. Але оскільки II купує в нього
матеріял, оскільки II мусить перетворювати свій сталий капітал IIс на золотий
матеріял, частина (І з) v повертається від II до І з таким самим шляхом,
як і до інших капіталістів І; а оскільки цього не постає, він заміщує своє v
золотом безпосередньо з свого продукту. Але в тій самій мірі, що в ній v,
авансоване в грошовій формі, не повертається до нього від II, частина
вже наявних засобів циркуляції (гроші, що приплили від І до II й не повернулись
до І) перетворюється в II на скарб, і тому частину додаткової
вартости II не витрачається на засоби споживання. Що постійно відкриваються
нові золоті копальні, або відновлюються роботи на старих,
то певна частина грошей, що їх Із повинен витрачати на v, завжди
становить частину тієї маси грошей, яка була вже до нової продукції
золота, яку Із за допомогою своїх робітників подає в II, і, оскільки
вона не повертається з II до Із, вона становить там елемент для
утворення скарбів.
