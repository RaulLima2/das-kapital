дини, а по жнивах його засівають корінняками на годівлю худоби. Ця система, за якої рогата худоба
може ввесь час перебувати в стійлі, дає чималі запаси угноєння і стає таким чином за основу
сівозмінного господарства. В піскуватих місцевостях більше, ніж третину оброблюваної землі
відводиться під cultures dérobées, а наслідок такий, ніби оброблюваної землі побільшало на третину“.
Поряд корінняків тут культивують також конюшину та інші кормові рослини. „Рільництво доведене таким
чином до того пункту, де воно перетворюється на городництво, потребує, звичайно, порівняно чималого
основного капіталу (Anlagekapital). В Англії основний капітал обчислюється в 250 франків на гектар.
У Фляндрії основний капітал в 500 франків на гектар наше селянство, мабуть, визнало б за дуже
низький“. (Essais sur l’Economie Rurale de la Belgique par Emile de Laveleye. Paris, 1863, p. 59,
60, 63).

Візьмімо нарешті лісівництво. — „Продукція дерева посутньо відрізняється від більшости інших
продукцій тим, що тут сила природи діє самостійно і при природному поновленні не потребує сили
людської або сили капіталу. А проте, навіть там, де ліси розводять штучно, застосування сили
людської та капіталу порівняно з дією сил природи є лише незначне. Крім того ліс може добре рости на
таких ґрунтах і місцях, де хліб не удається або продукція його не оплачується. Але для
лісорозведення при правильному господарюванні потрібна також більша площа, ніж для культури хліба,
бо на маленьких парцелях не можна розбити ліс на правильні дільниці, побічних плодів майже не можна
використати, важче зберігати дерево й т. д. Однак процес продукції тут сполучено також з такими
довгими періодами, що він виходить поза пляни приватного господарства, а іноді навіть поза межі
людського життя. Капітал,
витрачений на закуп землі“ [при громадській продукції цей капітал відпадає, і справа лише в тому,
скільки землі може громада відібрати під ліс від поля та пасовиська], „дає помітні плоди лише через
довгий час і обертається тільки почасти, а цілий оборот при деяких ґатунках дерев потребує до 150
років. Крім того, для правильної продукції дерева треба, щоб був запас живого дерева в 10–40 разів
більший, ніж щорічне споживання. Тому той, хто не має інших прибутків і посідає чимало площі лісу,
не може вести правильне лісове господарство“ (Kirchhof, р. 58).

Довгий час продукції (що має в собі відносно лише незначну частку робочого часу) і сполучений з ними
довгий період обороту робить лісівництво несприятливим для приватних, а значить, і для
капіталістичних підприємств, бо останні суттю своєю є приватні підприємства, хоча б замість
поодинокого капіталіста виступав капіталіст асоційований. Розвиток культури і взагалі промисловости
остільки енергійно виявив себе щодо знищення лісів, що порівняно з цим усе, зроблене ним для
підтримання й насадження лісу, є цілком незначна величина.

Особливо треба зауважити в цитаті Кірхгофа таке місце: „Крім того, для правильної продукції дерева
треба, щоб був запас живого дерева в10–40 разів більший, ніж щорічне споживання“. — Отже, один
оборот дорівнює 10–40 і більше рокам.
