повує з циркуляції більше грошей, ніж подає в неї, то частина капіталістів,
— та, що продукує золото — постійно напомповує в неї більше
грошей, ніж вилучає з неї в засобах продукції.

Хоч частина цього продукту, золота, в 500 ф. стерл., є додаткова
вартість продуцентів золота, однак цілу цю суму призначається лише на
заміщення грошей, потрібних для циркуляції товарів; при цьому байдуже,
скільки з цієї суми йде на перетворення на гроші додаткової вартости
товарів і скільки на перетворення на гроші інших складових частин
вартости товарів.

Якщо продукцію золота перенести з даної країни в інші країни, то
це нічого не змінює в справі. Частину суспільної робочої сили й суспільних
засобів продукції в країні А перетворено на продукт, прим., на
полотно, вартістю в 500 ф. стерл., що його вивозиться в країну В, щоб
там купити золото. Продуктивний капітал, застосований таким чином у
країні А, так само не подає на ринок країни А товарів — на відміну
від грошей — як коли б його безпосередньо застосовувалось на продукцію
золота. Цей продукт А репрезентовано в 500 ф. золота, і він
надходить в циркуляцію країни А лише як гроші. Частина суспільної
додаткової вартости, що є в цьому продукті, існує безпосередньо як
гроші, і для країни А ніколи не існує інакше, як у формі грошей. Хоч
для капіталістів, що продукують золото, лише частина продукту репрезентує
додаткову вартість, а друга частина — покриття капіталу, однак питання
про те, яка кількість цього золота покриває, крім обігового сталого
капіталу, змінний капітал, і яка кількість репрезентує додаткову
вартість залежить виключно від тих відповідних відношень,
що в них заробітна плата й додаткова вартість перебувають до
вартости товарів, що циркулюють. Частина, що становить додаткову
вартість, розподіляється між різними членами кляси капіталістів. Хоч
вони постійно витрачають її на особисте споживання й знов одержують
її через продаж нового продукту, — саме ця купівля й продаж взагалі
лише і зумовлює циркуляцію між ними грошей, потрібних для перетворення
на гроші додаткової вартости, — однак деяка частина суспільної
додаткової вартости, хоч і в змінних кількостях, перебуває в формі
грошей в кишені капіталістів, цілком так само, як частина заробітної
плати, принаймні протягом кількох днів тижня, затримується в формі грошей
в кишенях робітників. І ця частина не обмежена тією частиною грошового
продукту, що первісно становила додаткову вартість капіталістів, які
продукують золото; як сказано, вона обмежена тією пропорцією, що в
ній вищеназваний продукт в 500 ф. стерл. взагалі розподіляється між
капіталістами і робітниками, і що в ній запас товарів, призначених для
циркуляції, складається з додаткової вартости та з інших складових частин
вартости.

А проте, частина додаткової вартости, яка існує не в інших товарах,
а в грошах поряд цих інших товарів, лише остільки складається з частини
щорічно продукованого золота, оскільки частина річної продукції золота
йде в циркуляцію для реалізації додаткової вартости. Друга частина
