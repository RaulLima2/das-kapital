\parcont{}  %% абзац починається на попередній сторінці
\index{ii}{0312}  %% посилання на сторінку оригінального видання
капіталом ($400 v$) і частиною додаткової вартости ($100 m$) в підвідділі а
і з змінним капіталом ($100 v$) у підвідділі b. В дальшому ми припускаємо,
що пересічна пропорція між витратами доходу в капіталістів обох кляс
є така: 2/5 на речі розкошів і 3/5 на доконечні засоби існування.
Тому крім 100, уже витрачених на речі розкошів, усій підклясі а припадає
ще 60 на речі розкошів і в такій самій пропорції, тобто 40 припадає
підклясі b.

Отже, (II а) m розподіляється так: 240 на засоби існування і 160 на
речі розкошів = $240 + 160 = 400 m$ (II а).

(II b) m розподіляється так: 60 на засоби існування й 40 на речі
розкошів: $60 + 40 = 100 m$ (II b). Останні 40 ця кляса бере для споживання
з свого власного продукту (2/5 своєї додаткової вартости); 60 в
засобах існування вона одержує, обмінюючи 60 свого додаткового продукту
на $60 m$ (а).

Отже, для цілої кляси капіталістів II ми маємо (при цьому $v + m$ в
підвідділі а існують у доконечних засобах існування, в підвідділі b — в
речах розкошів):

II а ($400 v + 400 m$) + II b ($100 v + 100 m$) = 1.000; через рух усе
це реалізується так: $500 v$ (а + b) [реалізуються в $400 v$ (а) і $100 m$ + (а)] +
$500m$ (а + b) [реалізуються в $300 m$ (а) + $100 v$ (b) + $100 m$ (b)] =
1000.

Для а і b, розглядуваних окремо, реалізація відбувається таким
чином:

а) $v / 400 v$ (a) + $m 240 m$ (a) + $100 v$ (b) + $60 m$ (b) ... = 800

в) $v / 100 m$ (a) + $m 60 m$ (a) + $40 m$ (b) ... = 200 / 1000

Коли ми спрощення ради додержуватимемось для обох підвідділів однакового
відношення між змінним і сталим капіталом (що, до речі, зовсім не неодмінно),
то на $400 v$ (а) припаде сталий капітал = 1600, а на 100 (b)
сталий капітал = 400, і для II будуть такі два підрозділи а і b:

II а) $1600 c + 400 v + 400 m$ = 2400

II b) $400 c + 100 v + 100 m$ = 600

a разом

$2000 c + 500 v + 500 m$ = 3000.

Відповідно до цього з 2000 II с в засобах споживання, які обмінюють
на 2000 І ($v + m$), 1600 обмінюються на засоби продукції доконечних
засобів існування і 400 — на засоби продукції речей розкошів.

Отже, ці 2000 І ($v + m$) і собі поділяться на ($800 v + 800 m$) І,
призначених для а = 1600 засобів продукції доконечних засобів існування,
\index{ii}{0313}  %% посилання на сторінку оригінального видання
і на ($200 v + 200 m$) І, призначених для b, = 400 засобів продукції
речей розкошів.

Чимала частина не лише власне засобів праці, а й сировинних та
допоміжних матеріялів тощо, однорідна в обох підрозділах. Але щодо
обміну різних частин вартости цілого продукту I ($v + m$), то цей поділ
на підрозділи не має жодного значення. Так згадані вище 800 I v, як і
200 I v реалізується в наслідок того, що заробітну плату витрачається
на засоби споживання 1000 ІІ с, отже, грошовий капітал, авансований на
неї, повертаючись, розподіляється рівномірно між капіталістами продуцентами
I і pro rata заміщує в грошах авансований ними змінний капітал:
з другого боку, щодо реалізації 1000 I m, то і тут капіталісти рівномірно
(пропорційно величині їхнього m) візьмуть засоби споживання з
усієї другої половини ІІ с = 1000: 600 II а і 400 II b; отже, ті, що заміщують
стали; капітал II а:

480 (3/5) з 600 с (II а) і 320 (2/5*) з 400 с (II b) = 800; ті, що заміщують
сталий капітал II b:

120 (3/5) від 600 с (II а) і 80 (2/5) від 400 с (ІІ b) = 200. Сума = 1000.

Що тут узято довільно і для І і для II, так це — відношення змінного
капіталу до сталого, а також однаковість цього відношення в
І і в II і в їхніх підвідділах. Щодо цієї однаковости, то її тут припущено
лише для спрощення; припущення різних пропорцій абсолютно нічого
не змінило б в умовах проблеми та в її розв’язанні. Але, коли припустити
просту репродукцію, то як доконечний результат з цього випливає таке:

1) Нова вартість, утворена річною працею в натуральній формі засобів
продукції (яка розпадається на $v + m$), дорівнює репродукованій
у формі засобів споживання сталій капітальній вартості с від вартости
продукту, утвореного другою частиною річної праці. Коли б ця нова вартість
була менша за II с, то II не міг би повнотою замістити свій сталий капітал;
коли б вона була більша, то надлишок не використовувалось би. В
обох випадках порушувалось би наше припущення простої репродукції.

2) Щодо річного продукту, репродукованого в формі засобів споживання,
то змінний капітал v, авансований в грошовій формі, можуть реалізувати
його одержувачі — оскільки вони є робітники, що продукують
речі розкошів — лише в тій частині доконечних засобів існування, що в
ній prima facie втілено додаткову вартість капіталістичних продуцентів
цих засобів; отже, v, витрачене на продукцію речей розкошів, розмірами
своєї вартости дорівнює відповідній частині m, спродукованій у
формі доконечних засобів існування, отже, воно мусить бути менше, ніж
усе це m — а саме (II а) m — і лише через реалізацію цього v в тій частині
m до капіталістичних продуцентів речей розкошів повертається в грошовій
формі авансований ними змінний капітал. Це явище цілком аналогічне
до реалізації I ($v + m$) в ІІ с; ріжниця лише та, що в другому випадку
(IІ b) v реалізується в частині (I Iа) m, яка величиною вартости до-

*) Зазначені тут у дужках дроби 3/5 і 2/5 є частина від усієї другої половини
сталого капіталу II а, тобто від 800, так само, як при II b вони є частини від усієї
другої половини сталого капіталу II b, тобто від 200. Peд
\index{ii}{0314}  %% посилання на сторінку оригінального видання
рівнює (II b) V. Ці відношення лишаються якісно вирішальні при всякому
розподілі всього річного продукту, оскільки він дійсно входить у процес
річної репродукції, упосереднюваної циркуляцією I ($v + m$) можна реалізувати
лише в II с, так само, як ІІ с в його функції складової частини
продуктивного капіталу можна відновити лише за допомогою цієї реалізації;
так само (II b) v можна реалізувати лише в частині (II а) т, і лише
таким способом (II b) v можна знову перетворити на його форму грошового
капіталу. Звичайно, це має силу лише за тієї умови, що все це дійсно
є результат самого процесу репродукції, отже, коли, напр., капіталісти
II b не одержують грошового капіталу для v за допомогою кредиту з
якихось інших джерел. Навпаки, щодо кількісного боку, то обміни різних
частин річного продукту можуть відбуватися з такою пропорційністю, як
подано вище, лише остільки, оскільки маштаб та відношення вартости
продукції лишаються незмінні, і оскільки ці точно визначені відношення
не зазнають змін в наслідок зовнішньої торгівлі.

Коли, за прикладом А. Сміса, казали, що I ($v + m$), розкладається на
II с, II с розкладається на I ($v + m$), або, як він часто каже своїм
звичаєм іще недоладніше, що I ($v + m$) становлять складові частини ціни,
зглядно вартости, він каже value in exchange II с, а II с становить усю
складову частину вартости I ($v + m$), то можна й треба було б сказати
також, що (II b) v розкладається на (II а) m, або (II а) m на (II b) v,
або що (II b) v становить складову частину додаткової вартости II а, і
навпаки: додаткова вартість розкладалась би таким чином на заробітну
плату, зглядно на змінний капітал, а змінний капітал становив би „складову
частину“ додаткової вартости. І справді, така недоречність дійсно
є у А. Сміса, бо заробітна плата визначається в нього вартістю доконечних
засобів існування, і ці товарові вартості знову таки визначаються
вартістю вміщених у них заробітної плати (змінного капіталу) і додаткової
вартости. Він до того захопився тими частинами, на які при капіталістичній
основі продукції можна розкласти вартість, спродуковану протягом
одного робочого дня, — а саме $v + m$, — що цілком забуває про те, що при
простому товаровому обміні цілком байдуже, чи складаються еквіваленти,
які існують в різних натуральних формах, з оплаченої чи неоплаченої праці:
бо в обох випадках вони коштують однакову кількість праці, витраченої
на їхню продукцію; і що так само байдуже, чи є товар якогось А засоби
продукції, а товар якогось В — засоби споживання, чи має один
товар функціонувати після продажу як складова частина капіталу, а
другий, навпаки, входить у фонд споживання його, і, за Адамом, споживається
як дохід. В який спосіб індивідуальний покупець вживає свій
товар, це не має жодного чинення до обміну товарів, до сфери циркуляції,
і не стосується вартости товару. Це ані трохи не змінюється
від того, що при аналізі циркуляції всього річного суспільного продукту
треба взяти на увагу певний характер вживання, момент споживання
різних складових частин цього продукту.

При вище констатованому обміні (II b) v на рівновартісну частину
(II a) m і при дальших обмінах між (II a) m і (II b) v зовсім не припускається,
\index{ii}{0315}  %% посилання на сторінку оригінального видання
що капіталісти — хоч поодинокі капіталісти II а і II b, хоч відповідні
категорії капіталістів в їхній сукупності — в однаковому відношенні
розподіляють свою додаткову вартість між доконечними предметами споживання
й засобами розкошів. Один може більше витрачати на одні
предмети споживання, другий — на другі. Лишаючись на ґрунті простої
репродукції, ми припускаємо тільки, що суму вартости, рівну всій додатковій
вартості, реалізується в фонді споживання. Отже, межі тут
дано. В межах кожного підрозділу один може більше витрачати на а,
другий на b; тут можлива взаємна компенсація, так що кляси капіталістів
а і b, взяті кожна як ціле, будуть в однаковій мірі брати участь
в а і b. Але відношення вартостей — пропорційна участь в цілій вартості
продукту II обох категорій продуцентів а і b — а значить і певне кількісне
відношення між галузями продукції, що дають ці продукти — ці
відношення неодмінно є дані для кожного конкретного випадку: гіпотетичне
є лише відношення, що фігурує в прикладі; коли припустити
інше відношення, то від цього ніщо не зміниться в якісних моментах;
змінилися б лише кількісні визначення. Але коли б в наслідок тих або
інших обставин постала справжня зміна у відносних величинах а й b, то
відповідно змінились би й умови простої репродукції.

З тієї обставини, що (II b) v реалізується в еквівалентній частині
(ІІ а) m, випливає, що тією самою мірою, як більшає частина річного
продукту, яка припадає на речі розкошів, отже, тією самою мірою, як
більшає маса робочої сили, що її поглинає продукція засобів розкошів,
такою самою мірою зворотне перетворення авансованого на (II b) v
змінного капіталу в грошовий капітал, що знову функціонує як грошова
форма змінного капіталу, а в наслідок цього й існування і репродукція
частини робітничої кляси, занятої в II b — одержання цією частиною робітничої
кляси доконечних засобів споживання — зумовлюється марнотратством
кляси капіталістів, перетворенням значної частини їхньої додаткової
вартости на речі розкошів.

Кожна криза моментально зменшує споживання речей розкошів; вона
уповільнює, затримує зворотне перетворення (II b) v на грошовий капітал,
лише почасти допускає це перетворення й тим самим викидає частину робітників,
які виробляють речі розкошів, на брук, а з другого боку, саме через
це вона призводить до застою й скорочення продажу доконечних засобів
споживання. Ми залишаємо цілком осторонь звільнених разом з цим
непродуктивних робітників, які за свої послуги одержують від капіталістів
частину їхніх витрат на розкоші (самі ці робітники pro tanto є
предмети розкошів) і беруть також дуже велику участь у споживанні
доконечних засобів існування тощо. Протилежне маємо в періоди процвітання
і особливо підчас спекулятивного процвітання, коли відносна,
виражена в товарах вартість грошей, падає вже з інших причин (при
йому не відбувається дійсного перевороту в вартості), а тому ціна товарів,
незалежно від їхньої власної вартости, підвищується. При цьому
\parbreak{}  %% абзац продовжується на наступній сторінці
