творюється з бічної функції багатьох на виключну функцію небагатьох,
на їхнє особливе зайняття, то від цього самий характер функції не зміниться.
Можливо, що купець (тут розглядуваний просто як аґент перетворення
форми товарів, лише як покупець і продавець) своїми операціями
скорочує для багатьох продуцентів час, витрачуваний на купівлю
й продаж. Тоді його треба розглядати як машину, що зменшує некорисну
витрату сили, або допомагає врятувати час для продукції\footnote{
Торгові витрати, хоч і доконечні, все ж повинно вважати за обтяжливі видатки“.
(Les frais de commerce, quoique nécessaires, doivent être regardés comme une
dépense onéreuse. — Quesnay, „Analyse du Tableau Economique“, y Daire, Physiocrates,
1-е partie, Paris, 1846, p. 71). — Згідно з Кене, „зиск“, що його дає конкуренція
між торговцями, примушуючи їх зробити поступку з своєї винагороди або бариша...,
власне кажучи, є лише усунення втрати для продавця з перших
рук і для покупця-споживача. Але усунення втрати, що її спричиняють торгові
витрати, не є реальний продукт, або збільшення багатства в наслідок торговлі,
розглядуваної в собі самій просто як обмін, незалежно від транспортових витрат
або разом з цими витратами“ (à mettre leur rétribution ou leur gain au rabais...
n’est rigoureusement parlant, qu’une, privation de perte pour le vendeur de la première
main et pour l’acheteur-consommateur. Or, une privation de perte sur les frais du
commerce n'est pas un produit réel ou un accroît de richesses obtenu par le commerce,
considérée en lui-même simplement comme échange, indépendemment des frais de transport, ou envisagé
conjointement avec les frais de transport) (145, 146 стор.).
„Торгові витрати завжди оплачується коштом продавця продуктів, що був би
одержував усю ціну, яку дають за них покупці, коли б не було жодних витрат
на посередництво“ („Les frais du commerce sont toujours payés aux dépens des
vendeurs des productions qui jouiraient de tout le prix qu’en payent les acheteurs,
s’il n’y avait point de frais intermédiaires) (p. 163). Власники й продуценти є
„salariants“ — ті, хто оплачують; купці — „salariés", оплачувані, наймані (стор. 164)
(Quesnay, Problèmes économiques, у Daire, Physiocrates, 1-е partie, Paris, 1846).
}.

Щоб спростити справу (бо ми лише пізніше розглядатимемо купця
як капіталіста і купецький капітал), ми припустимо, що аґент купівлі та
продажу є людина, яка продає свою працю. Він витрачає свою робочу
силу і свій робочий час на ці операції Т — Г і Г — Т. Він живе з цього
так само, як, напр., інший живе з прядіння або готування пілюль. Він
виконує доконечну функцію, бо самий процес репродукції має в собі
непродуктивні функції. Він працює так само, як і інший, але зміст його
праці не утворює ні вартости, ні продукту. Він сам належить до faux
frais\footnote*{
Faux frais (франц.) — фалшиві витрати, тобто непродуктивні витрати. Ред.
} продукції. Користь від нього не в тому, що він перетворює
непродуктивну функцію на продуктивну або непродуктивну працю на
продуктивну. Було б чудо, коли б таке перетворення сталось у наслідок
перекладання функції від однієї особи на іншу. Скоріше, він дає користь
тим, що меншу частину робочої сили й робочого часу суспільства
зв’язується цією непродуктивною функцією. Навіть більше. Припустімо,
що він простий найманий робітник, хоч і краще оплачуваний.
Хоч би    як оплачувалось його працю, все ж частину свого часу
він, як    найманий робітник, працює    задурно. Можливо, він
одержує щодня вартість, спродуковану протягом вісьмох робочих годин, а
працює протягом десятьох годин. Дві години додаткової праці, викону-