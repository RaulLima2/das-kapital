\index{ii}{0341}  %% посилання на сторінку оригінального видання
Збірний робітник І продав свою робочу силу збірному капіталістові
І на 1000; цю вартість виплачено йому грішми в формі заробітної плати.
На ці гроші він купує в II засоби споживання на ту саму суму
вартости. Капіталіст II протистоїть йому лише як продавець товарів, і
нічого більше, хоч би робітник купував у свого власного капіталіста, як
напр., вище (стор. 311) в обміні 500 II v. Форма циркуляції що її пророблює
його товар, робоча сила, це форма простої циркуляції товарів, спрямованої
виключно на задоволення потреб, на споживання Т (робоча сила) —
Г — Т (засоби споживання, товар II). Результат цього акту циркуляції
той, що робітник зберіг себе як робочу силу для капіталіста І, як таку,
і щоб зберегти себе як робочу силу надалі, робітник мусить знову та
знову повторювати процес Р (Т) — Г — Т. Його заробітна плата реалізується
в засобах споживання, її витрачається як дохід, і, беручи
робітничу клясу в цілому, завжди знову й знов витрачається як дохід.

Розгляньмо тепер той самий обмін IV на II c з погляду капіталіста. Ввесь
товаровий продукт II складається з засобів споживання, отже, з речей,
призначених на те, щоб увійти в річне споживання, тобто служити комубудь
— в даному разі збірному робітникові І — для реалізації доходу. Але
для збірного капіталіста ІI частина його товарового продукту, = 2000,
являє тепер перетворену на товар форму сталої капітальної вартости
його продуктивного капіталу, що його з цієї товарової форми треба
знову перетворити на ту натуральну форму, в якій він може знову
функціонувати як стала частина продуктивного капіталу. До цього часу
капіталіст II досяг того, що половину (= 1000) своєї сталої капітальної
вартости, репродукованої в товаровій формі (в засобах споживання) він
знову перетворив на грошову форму через продаж робітникові І. Отже,
на цю першу половину сталої капітальної вартости II с перетворився не
змінний капітал Іv, а гроші, які в обміні на робочу силу функціонували
для І як грошовий капітал і потрапили таким чином у посідання продавця
робочої сили, для якого вони являють зовсім не капітал, а дохід
у грошовій формі, тобто він їх витрачає як купівельний засіб на предмети
споживання. З другого боку, гроші = 1000, що приплили від робітників
І до капіталістів II, не можуть функціонувати як сталий елемент
продуктивного капіталу II. Це покищо лише грошова форма його товарового
капіталу, що її ще лише треба перетворити на основні або обігові
складові частини сталого капіталу. Отже, II на гроші, вторговані від
робітників І, покупців його товару, купує в І засоби продукції на 1000.
У наслідок цього стала капітальна вартість II на половину всієї своєї
величини відновлюється в тій натуральній формі, що в ній вона знову
може функціонувати як елемент продуктивного капіталу II. Формою
циркуляції при цьому було Т — Г — Т: засоби споживання вартістю в 1000 —
гроші = 1000 — засоби продукції вартістю в 1000.

Але Т — Г — Т в даному разі є рух капіталу. Т, продане робітникам,
перетворюється на Г, а це Г перетворюється на засоби продукції; це —
зворотне перетворення з товару на речові творчі елементи цього товару.
З другого боку, так само, як капіталіст II проти І функціонує лише як покупець
\index{ii}{0342}  %% посилання на сторінку оригінального видання
товару, так і капіталіст І проти II функціонує тут лише як продавець товару. І на 1000
грошей, призначених функціонувати як змінний капітал, спочатку купив робочу силу вартістю в 1000;
отже, він одержав еквівалент за свої $1000 v$, віддані в грошовій формі; тепер гроші належать
робітникові, що витрачає їх на акти купівлі в II; ці гроші, що потрапили таким чином до каси II, І
може одержати знову, лише виловлюючи їх назад через продаж товарів на таку саму суму вартости.

Спочатку І мав певну грошову суму = 1000, призначену функціонувати
як змінна частина капіталу; вона функціонує як така в наслідок перетворення її на робочу силу такого
ж розміру вартости. Але робітник дав йому, як результат продукційного процесу, певну масу товарів
(засобів продукції) вартістю в 6000, що з них 1/6, або 1000, своєю вартістю
являє еквівалент авансованої в грошах змінної частини капіталу. Як перше,
в своїй грошовій формі, так і тепер в своїй товаровій формі, змінна
капітальна вартість не функціонує як змінний капітал; вона може так
функціонувати лише після того, як перетвориться на живу робочу силу
і лише протягом того часу, поки ця остання функціонує в продукційному
процесі. В грошовій формі, змінна капітальна вартість була лише потенціяльним
змінним капіталом. Але ця вартість перебувала в такій формі,
що в ній її можна було перетворити безпосередньо на робочу силу.
В товаровій формі, та сама змінна капітальна вартість є покищо лише
потенціяльна грошова вартість; її можна знову відновити в первісній
грошовій формі лише через продаж товару, отже, в даному разі, в наслідок
того, що II купує на 1000 товару в І. Рух циркуляції тут такий:
$1000 v$ (гроші) — робоча сила вартістю в 1000—1000 в товарі (еквівалент
змінного капіталу) — $1000 v$ (гроші); отже, Г — Т... Т — Г (= Г — Р...
Т — Г). Самий процес продукції, що припадає між Т... Т, не належить до
сфери циркуляції; він не з’являється в обміні різних елементів річної
репродукції одних на одні, хоч цей обмін включає репродукцію всіх
елементів продуктивного капіталу, так його сталого елементу, як і змінного,
робочої сили. Всі аґенти цього обміну виступають як лише покупці
або продавці, або як ті й ці; робітники виступають в ньому лише як
покупці товару; капіталісти — навперемінки як покупці й продавці, а в
певних межах — лише однобічно як покупці товару або однобічно як продавці
товару.

Результат такий: І має змінну частину вартости свого капіталу знову
в грошовій формі, що тільки з неї й можна перетворити цю частину
вартости безпосередньо на робочу силу, тобто знову має її в тій єдиній
формі, що в ній її справді можна авансувати як змінний елемент його
продуктивного капіталу. З другого боку, щоб мати змогу знову виступити
як покупець товару, робітник тепер мусить уперед знову виступити
як продавець товару, як продавець своєї робочої сили.

Щодо змінного капіталу категорії II (500 II v) процес циркуляції
між капіталістами й робітниками тієї самої кляси продукції виступає в
безпосередній формі — оскільки ми розглядаємо його як процес, що відбувається
між збірним капіталістом II і збірним робітником II.
