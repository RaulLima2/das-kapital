матеріяли на бажану форму, не можуть становити жодної частини цього
доходу. Ціна цієї праці звичайно може становити частину цього доходу,
бо робітники, зайняті цією працею, можуть усю вартість своєї заробітної
плати вкласти в запас, призначений для їхнього безпосереднього споживання.
Але в інших галузях праці в цей фонд споживання входить так
ціна“ (тобто плата, видана за цю працю), як і „продукт“ (що в
ньому цю працю втілено); „ціна входить у споживний фонд робітників,
продукт у споживний фонд інших людей, що їхнє існування, комфорт
та втіхи підвищуються в наслідок праці цих робітників“. (Кн. II, розд. 2,
стор. 190, 191).

А. Сміс натрапляє тут на дуже важливу ріжницю між робітниками,
що працюють у продукції засобів продукції, і тими, що працюють
безпосередньо в продукції засобів споживання. Вартість
товарового продукту першої категорії ротітників має в собі складову
частину, рівну сумі заробітних плат, тобто вартості частини капіталу,
витраченої на закуп робочої сили; ця частина вартости існує тілесно як
певна частина засобів продукції, спродукованих цими робітниками. Гроші,
одержані ними як заробітна плата, становлять для них дохід, але їхня
праця не дала ні для них самих, ні для інших людей, продуктів, придатних
на споживання. Отже, самі ці продукти не становлять елемента
тієї частини річного продукту, яка призначена давати суспільний споживний
фонд, той фонд, що в ньому лише й може реалізуватися „чистий
дохід“. А. Сміс забуває тут додати, що сказане тут про заробітну
плату однаково має силу й для тієї складової частини вартости засобів
продукції, яка — як додаткова вартість — становить у вигляді категорій
зиску та ренти дохід (у першу чергу) промислових капіталістів. Ці складові
частини вартости теж існують у формі засобів продукції, непридатних
до споживання. Тільки перетворившись на гроші, вони можуть
зрушити відповідну до їхньої ціни кількість засобів споживання, спродукованих
другою категорією робітників, і перенести її в фонд особистого
споживання власників цих засобів продукції. То більше слід було б
А. Смісові зрозуміти, що частина вартости щорічно утворюваних засобів
продукції, яка дорівнює вартості засобів продукції, діющих у цій сфері
продукції, засобів продукції, що ними виробляється засоби продукції, —
отже, частина вартости, яка дорівнює вартості застосованого тут сталого
капіталу, абсолютно виключається з кожної складової частини вартости,
що становить дохід — виключається не лише в наслідок тієї натуральної
форми, що в ній ця частина існує, а й тому, що функціонує вона як
капітал.

Щодо другої категорії робітників, — тих, які безпосередньо продукують
засоби споживання — то визначення А. Сміса не зовсім точні.
А саме, він каже, що в цих галузях праці входять в фонд безпосереднього
споживання (go to) обидва елементи — і ціна праці й продукт:
„ціна (тобто гроші, одержані як заробітна плата) — в споживний фонд
робітників, а продукт — в споживний фонд інших людей
(that of other people), що їхнє існування, комфорт та втіхи підвищуються
