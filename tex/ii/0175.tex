талу. Отже, залежно від протягу тієї частини часу продукції, яка не є
робочий час, подовжується й період обороту капіталу. Оскільки час продукції,
надмірний порівняно з робочим часом, не визначено раз назавжди
даними законами природи, як от при достиганні хліба, рості дуба тощо,
період обороту часто можна більш-менш скоротити, штучно скорочуючи
час продукції. Напр., коли заводиться хемічне біління замість біління
на полі, — чинніші сушні апарати у процесах сушіння. Так в чинбарстві,
де за старими методами треба було від 6 до 18 місяців, щоб
чинбарська кислота пройняла шкіри, ці операції, за нової методи, коли
почали застосовувати повітряну помпу, скоротились до 1 1/2—2 місяців.
(I. G. Courcelle-Saneuil. Traité théorique et pratique des Entreprises industrielles
etc. Paris, 1857, 2 éd.).

Найяскравіший приклад штучного скорочення часу продукції, заповненого
виключно природними процесами, подає історія залізоробної
продукції і особливо перероблення чавуна на сталь за останні 100 років,
починаючи з відкритого 1780 року пудлінґування й до сучасного бессемерівського
процесу та інших, заведених з того часу найновіших методів.
Час продукції скорочено надзвичайно, але такою самою мірою збільшились
і вкладення основного капіталу.

Своєрідний приклад того, як відхиляється час продукції від робочого
часу, подає американська фабрикація копил на чоботи. Тут чимала частина
затрат постає тому, що дерево має сохнути до 18 місяців, щоб
готове копило не дубилось і не змінювало своєї форми. Протягом цього
часу дерево не підпадає жодному іншому процесові праці. Період обороту
вкладеного тут капіталу визначається, отже, не лише часом, потрібним
на виготовлення самих копил, а й часом, що протягом його капітал
лежить без діла в дереві, що висихає. Дерево перебуває 18 місяців
у процесі продукції, поки воно, нарешті, ввійде у власне робочий процес.
Разом з тим цей приклад показує, які різні можуть бути періоди обороту
різних частин цілого обігового капіталу в наслідок обставин, що постають
не в сфері циркуляції, а в продукційному процесі.

Особливо виразно виступає ріжниця між часом продукції і робочим
часом у сільському господарстві. В нашому помірному підсонні земля дає
збіжжя раз на рік. Скорочення або продовження періоду продукції (пересічно
дев’ятимісячного для озимого засіву) і собі залежить від зміни
сприятливих і несприятливих років, а тому не можна його точно наперед
визначити й контролювати, як у власне промисловості. Лише бічні продукти,
напр., молоко, сир і т. ін. завжди можна продукувати й продавати
протягом більш-менш коротких періодів. Щождо робочого часу, то тут
справа така: „В різних місцевостях Німеччини, залежно від кліматичних
та інших чинних умов, число робочих днів для трьох головних періодів
праці буде таке: у весняному періоді, з середини березня або початку
квітня до середини травня 50—60 робочих днів; в літньому періоді, з початку
червня до кінця серпня 65—80; в осінньому періоді, з початку
вересня до кінця жовтня або середини або кінця листопада 55—75 робочих
днів. На зиму припадають лише такі роботи, що їх можна вико-
