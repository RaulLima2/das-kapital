його вартости; але разом з тим вона утворює додатковий матеріял для
зростання вартости.

Т'... Т' становить основу Tableau économique Кене, і те, що він
обрав протилежно до форми Г... Г' (форми, що її виключно дотримувалась
меркантильна система) саме цю форму, а не П... П, свідчить про
його великий і правильний такт.

Розділ четвертий

Три фігури процесу кругобігу

Коли Ск позначає сукупний процес циркуляції, то три фігури можна
зобразити так:

I. Г — Т... П... Т' — Г'

II. п... ск... п

III. Ск... П (Т').

Коли ми всі три форми сполучимо, то всі передумови процесу виступають,
як його результат, як передумова, ним самим утворена. Кожен
момент виступає як вихідний пункт, перехідний пункт і пункт повороту.
Сукупний процес виступає як єдність процесу продукції та процесу
циркуляції; продукційний процес стає посередньою ланкою в процесі
циркуляції і навпаки.

Всі три кругобіги мають спільну рису, а саме — зростання вартости як
визначальна мета, як движний мотив. В І це виражено в самій формі.
Форма II починається з П, з самого процесу зростання вартости.
В III кругобіг починається з вирослої вартости й закінчується нововирослою
вартістю, навіть коли рух повторюється в незмінному маштабі.
Оскільки Т — Г є для покупця Г — Т, а Г — Т є для продавця Т — Г,
остільки циркуляція капіталу являє лише звичайну товарову метаморфозу,
і тут мають силу розвинені щодо неї закони про кількість грошей, що
циркулюють (кн. І, розділ III, 2). Але коли не зупинятись на цьому
формальному боці, а розглядати реальний зв’язок між метаморфозами
різних індивідуальних капіталів, отже, дійсно зв’язок кругобігів індивідуальних
капіталів як частинні рухи в процесі репродукції сукупного
суспільного капіталу, то цей зв’язок не можна з’ясувати простою зміною
форм грошей і товару.

В постійно рухомому колі кожен пункт є одночасно пункт вихідний
і пункт повороту. Коли переривається круговий рух, то не кожний
вихідний пункт є пункт повороту. Наприклад, ми бачили, що не лише
кожен окремий кругобіг (implicite) припускає інший, але також, що
повторення кругобігу в одній формі включає пророблення кругобігу й
в інших формах. Таким чином, уся ріжниця виступає як суто-фор-
