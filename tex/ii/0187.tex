винного матеріялу. Так, напр., у Лондоні що три місяці бувають великі
авкціони шерсти, які реґулюють шерстяний ринок, тимчасом як ринок
бавовни від урожаю до врожаю поновлюється в цілому безупинно, хоч
і не завжди рівномірно. Такі періоди визначають головні строки закупу
цих сировинних матеріялів і особливо впливають на спекулятивні закупи,
що зумовлюють більш або менш протяжні авансування на ці елементи
продукції, — впливають цілком так само, як природа випродукуваних
товарів впливає на спекулятивне, навмисне, довше або коротше затримування
продукту в формі потенціяльного товарового капіталу. „Отже,
сільський господар теж мусить до певної міри бути спекулянтом і тому
утримуватись від продажу своїх продуктів, зважаючи на обставини часу“...
Далі ідуть деякі загальні правила... „Тимчасом при збуті продуктів найголовніше
все ж таки залежить від особи, від самого продукту й місцевости.
Коли людина, крім кмітливости та вдачі (!), має достатній капітал для продукції
(Betriebskapital), їй не можна докоряти, якщо при незвичайно низьких цінах
вона залишить лежати свій зібраний хліб ще цілий рік; навпаки, кому
бракує обігового капіталу або взагалі (!) духа спекуляції, той дбатиме
про те, щоб взяти звичайну пересічну ціну і, значить, муситиме продавати,
скоро матиме нагоду до цього. Коли вовну зберігати довше, ніж протягом
одного року, то це майже завжди зробить тільки шкоду; тимчасом як
зернові хліба та олійне насіння можна зберігати кілька років, і при цьому
не псуються їхні властивості й добротність. Ті продукти, що зазнають
звичайно протягом короткого часу великого піднесення та падіння цін як
от, прим., олійне насіння, хміль, ворсувальні шишки тощо, небезпідставно
залишають лежати в ті роки, коли ціни на них нижчі від цін їхньої продукції.
Найменше слід відкладати продаж таких продуктів, що потребують
щоденних витрат на їхнє утримання, як от відгодована худоба, або таких,
що псуються, як от фрукти, картопля і т. ін. В деяких місцевостях у певну
добу року ціна продукту пересічно є найнижча, а іншого часу, навпаки,
найвища. Так, напр., пересічно ціна на зерно на Мартіна в деяких місцевостях
нижча, ніж між різдвом і великоднем. Далі, в деяких місцевостях
деякі продукти можна добре продати тільки певного часу, як, напр.,
вовну на вовняних ярмарках у таких місцевостях, де, крім ярмарок, звичайно
дуже мало торгують вовною“. (Kirchhof, стор. 302).

Розглядаючи другу половину часу обігу, що протягом його гроші
знову перетворюються на елементи продуктивного капіталу, треба взяти
на увагу не лише це перетворення само собою; не лише час, що протягом
його гроші припливають назад, залежно від віддалености того
ринку, де продається продукт; треба взяти насамперед на увагу й розміри
тієї частини авансованого капіталу, яка постійно мусить перебувати в
грошовій формі, в стані грошового капіталу.

Лишаючи осторонь усяку спекуляцію, розмір закупів тих товарів, які
постійио мають бути наявні як продуктивний запас, залежить від строків
поновлення цього запасу, отже, від обставин, що й собі залежать від
ринкових відносин, і які тому є різні для різних сировинних матеріялів
тощо; отже, тут доводиться час від часу одним заходом авансовувати
