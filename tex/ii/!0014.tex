креслює, що interest of capital він розуміє, як загальну форму
додаткової праці на відміну від її особливих форм, ренти, проценту і
підприємецького зиску... Але назву однієї з цих особливих форм, interest,
він усе ж бере як назву загальної форми. І цього досить, щоб він знову
заплутався в економічній тарабарщині“ (в рукопису стоїть „slang“).

Цей останній пункт якнайточніше стосується й до Родбертуса. І він
також у полоні економічних категорій, що були до нього. І він зве
додаткову вартість ім’ям однієї з її перетворених підпорядкованих їй форм,
ім’ям ренти, — ренти, що її він до того ж зробив зовсім невизначеною.
Наслідок цих двох помилок був той, що він знову вдається в економічну
тарабарщину, не прокладає критично шляхів далі за Рікардо і замість
того піддається спокусі зробити з своєї недоробленої теорії, що не
вилупилася ще з шкаралупи, основу утопії, з якою він, як і завжди,
прийшов дуже пізно. Памфлет, виданий 1821 року, цілком упередив
„ренту“ Родбертуса від 1842 року.

Наш памфлет є лише крайній аванпост тієї багатої літератури, що
двадцятими роками обернула теорію вартости й додаткової вартости
Рікардо в інтересах пролетаріату проти капіталістичної продукції, била
буржуазію її власною зброєю. Весь Оуенівський комунізм, оскільки він
бере участь в економічній полеміці, спирається на Рікардо. Але поряд
нього був ще цілий ряд письменників, що з них Маркс уже 1847 р. в
полеміці проти Прудона („Misère de la philosophie“, p. 49) згадує лише деяких:
Едмонда, Томпсона, Годскіна і т. ін. і т. ін., „і ще чотири сторінки et
cetera“. З цих численних праць я беру одну, першу-ліпшу: „Ап Inquiry
into the Principles of the Distribution of Wealth, most conducive to Human
Happiness, by William Thompson; a new edition, London 1850“. Цей твір,
написаний 1822 p., вперше видано 1827 р. Багатство, що його привлащують
непродуктивні кляси, тут теж усюди визначається, як відрахування з
продукту робітника, і це в досить енергійних висловах. „Повсякчасне
намагання того, що ми звемо суспільством, було в тому, щоб обманою
або умовлянням, залякуванням або примусом спонукати продуктивного
робітника виконувати працю за якомога меншу частину продукту його
власної праці“ (стор. 28). „Чому ж робітник не може одержувати абсолютно
ввесь продукт своєї роботи? " (ст. 32). „Цю компенсацію, що її
капіталісти вимушують від продуктивних робітників під назвою земельної
ренти, або зиску, вимагають за користування землею або іншими речами...
Що всі фізичні матеріяли, на яких або за допомогою яких позбавлений
власности продуктивний робітник, що нічого не має, крім своєї здібности
продукувати, тільки й може виявити цю свою продуктивну здібність, —
що всі ці матеріяли є в посіданні інших осіб, котрих інтереси протилежні
інтересам робітника, а згода їх є передумова його діяльности, — то чи не
залежить і чи не повинно залежати від ласки цих капіталістів те, яку
частину витворів його власної праці вони побажають дати йому, в
нагороду за цю працю? (ст. 125)... порівняно з величиною утриманого
продукту, все одно, чи зветься він податком, зиском або крадіжкою...
ці відрахування“ (ст. 126) і т. ін.
