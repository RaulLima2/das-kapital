Отже, суть справи, яка становить тут основу акту Г — Т Р Зп, є
розподіл; не розподіл у звичайному розумінні як розподіл засобів споживання,
а розподіл елементів самої продукції, що з них речові чинники
є сконцентровані на одному боці, а робоча сила, відокремлена від них, —
на другому.

Отже, засоби продукції, речова частина продуктивного капіталу,
мусять протистояти робітникові як такі, як капітал раніше, ніж акт Г — Р
може стати загальним суспільним актом.

Ми вище бачили, що капіталістична продукція, скоро вона вже постала,
в своєму розвитку не лише репродукує це відокремлення, але поширює
його в дедалі більших розмірах, поки воно не зробиться загальним домінантним
суспільним станом. Але в цій справі є ще другий бік. Щоб
капітал міг утворитись і опанувати продукцію, для цього повинна бути
передумова — певний щабель у розвитку торговлі, значить, і в розвитку
товарової циркуляції, а тим самим і товарової продукції; бо речі не
можуть увійти в циркуляцію як товари, коли їх продукується не для
продажу, отже, не як товари. Але лише на основі капіталістичної продукції
товарова продукція з’являється як нормальний, домінантний характер
продукції.

Російські землевласники, які в наслідок так званого визволення селян
провадять тепер своє сільське господарство найманими робітниками, а
не кріпаками, підневільними робітниками, скаржаться на дві обставини:
поперше, на брак грошового капіталу. Наприклад, вони кажуть: раніше,
ніж продати врожай, треба робити великі виплати найманим робітникам,
і тут бракує першої умови, готівки. Щоб капіталістично провадити
продукцію, капітал у формі грошей мусить завжди бути в наявності, саме
для видачі заробітної плати. Однак землевласники можуть з цього приводу
не журитись. З плином часу можна зривати рожі, і промисловий капіталіст
порядкує вже не лише власними грішми, але також і l’argent des autres*).

Але характеристичніша є друга скарга, а саме: хоча б і були
гроші, все ж немає в достатніх розмірах і в який завгодно час робочої
сили, що ії можна було б купити, бо російський сільський
робітник у наслідок спільної власности на землю в земельній громаді ще
не цілком відокремлений від своїх засобів продукції, а тому й не являє
ще „вільного найманого робітника“ в повному значінні цього слова.
А наявність останнього в широкому суспільному маштабі є неодмінна
умова для того, щоб Г — Т, перетворення грошей на товар, могло являти
собою перетворення грошового капіталу на продуктивний капітал.

Тому само собою зрозуміло, що формула кругобігу грошового капіталу:
Г — Т... П... Т' — Г' є сама собою зрозуміла форму кругобігу капіталу
лише на основі вже розвиненої капіталістичної продукції, бо вона
має за передумову наявність кляси найманих робітників в суспіль-
