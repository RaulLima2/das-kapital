сував в 1000v грішми; але в товаровій формі вони є гроші лише потенціяльно
(дійсними грішми вони стають тільки в наслідок продажу), отже,
ще менше вони є безпосередньо змінний грошовий капітал. Кінець-кінцем,
вони стають ним через продаж товару 1000 Iv покупцеві IIc і через
те, що робоча сила одразу знову з’являється як продажний товар, як
матеріял, що на нього можуть перетворитись l000v грішми.

Підчас усіх цих перетворень капіталіст І постійно має в своїх руках
змінний капітал: 1) спочатку як грошовий капітал; 2) потім як елемент
його продуктивного капіталу; 3) ще пізніше як частину вартости його
товарового капіталу, тобто в товаровій вартості; 4) нарешті, знову в грошах,
що їм знову протистоїть робоча сила, на яку їх можна перетворити.
Протягом процесу праці капіталіст має в своїх руках змінний капітал як
діющу вартостетворчу робочу силу, а не як вартість даної величини;
що капіталіст завжди оплачує робітника лише після того, як сила його
діяла вже певний коротший або довший час, то перш ніж оплатити її,
він уже одержує в свої руки утворену нею вартість як еквівалент її
самої плюс додаткова вартість.

А що змінний капітал в тій або іншій формі постійно
лишається в руках капіталіста, то ні в якому
разі не можна сказати, що він перетворюється для будького
на дохід. Навпаки, 1000 Iv в товарі перетворюється на гроші
через продаж покупцеві II, що для нього таким чином заміщується
in natura половина його сталого капіталу.

Не змінний капітал І, l000v в грошах, сходить на дохід. Ці гроші,
скоро вони перетворені на робочу силу, перестають функціонувати як
грошова форма змінного капіталу І, — так само, як і гроші всякого іншого
продавця товарів перестають репрезентувати щось належне йому,
скоро він їх перетворить на товар якогось продавця. Перетворення, що
їх пророблюють в руках робітничої кляси гроші, одержані як заробітна
плата, є перетворення не змінного капіталу, а перетвореної на гроші
вартости робочої сили робітничої кляси; цілком так само, як перетворення
новоутвореної робітником вартости [2000 І (v + m)] є лише перетворення
належного капіталістові товару, перетворення, яке зовсім
не стосується до робітника. Але капіталіст, — а ще більше його теоретичний
тлумач, політикоеконом силу в силу може визволитись від уявлення,
ніби гроші, виплачені робітникові, все ще є його, капіталіста, гроші.
Коли капіталіст є продуцент золота, то змінна частина вартости, тобто
той еквівалент у товарі, що заміщує йому купівельну ціну праці, сама
безпосередньо з’являється в грошовій формі, а тому знову, без обкружних
шляхів зворотного припливу, може функціонувати як змінний грошовий
капітал. Але щодо робітника в II, — оскільки ми лишаємо осторонь
робітників, що продукують речі розкошів — то саме 500v існує в товарах,
призначених на споживання робітникові, і їх він, розглядуваний як
збірний робітник, безпосередньо знову купує в того самого збірного
капіталіста, що йому він продав свою робочу силу. Змінна частина
вартости капіталу II своєю натуральною формою складається з засобів
