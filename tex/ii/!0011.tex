сують його так, що посадять за роботу старанних людей, дадуть їм
сировинний матеріял і засоби існування для того, щоб здобути зиск,
продаючи продукти їхньої праці, або того, що їхня праця додала до
вартости того сировинного матеріялу... Вартість, що її робітники
додають до сировинного матеріялу, поділяється тут на дві частини —
з них однією оплачується їхню заробітну плату, а друга становить
зиск підприємця на всю суму, що її він авансував на сировинний
матеріял та заробітну плату.“ І трохи далі: „Скоро вся земля в якійбудь
країні стане приватною власністю, землепосідачі, як і інші люди,
воліють за краще жати там, де вони не сіяли, і вимагають земельної
ренти навіть за природні продукти землі... Робітник... мусить відступити
землепосідачеві деяку частину з того, що він зібрав або виробив
своєю працею. Ця частина, або — що те саме — ціна цієї частини становить
земельну ренту“.

З приводу цього місця Маркс зауважує в зазначеному рукопису
„Zur Kritik“ etc., на стор. 253: „Отже, А. Сміс розуміє додаткову вартість,
— а саме додаткову працю, надлишок виконаної та зречевленої в
товарі праці проти оплаченої праці, тобто проти праці, що одержала
свій еквівалент у заробітній платні, — як загальну категорію, при чому
власно зиск і земельна рента являють лише її відгалуження“.

Далі А. Сміт каже в книзі І, розд. VIII: „Скоро земля стала приватною
власністю, землепосідач вимагає частину майже всіх продуктів, що
їх робітник може виробити або зібрати на ній. Його земельна рента
становить перше відрахування з продукту праці, прикладеної до землі.
Але хлібороб рідко коли має засоби утримувати себе до збору врожаю.
Його утримання звичайно авансується йому з капіталу (stock) підприємця,
орендаря, що не мав би жодного інтересу давати йому роботу, коли б
хлібороб не ділився з ним продуктом своєї праці або не повертав йому
капіталу разом з зиском. Цей зиск е друге відрахування з праці, прикладеної
до землі. Продукт майже всякої праці зазнає такого самого
відрахування для зиску. В усіх галузях промисловости для більшости робітників
потрібен підприємець, який авансував би їм до закінчення праці
сировинний матеріял і заробітну плату та утримання. Цей підприємець
поділяв з ними продукт їхньої праці, або вартість, що її вони додають
до перероблюваного сировинного матеріялу, і ця частка становить його
зиск“.

Маркс додає до цього (рукопис, ст. 256): „Отже, А. Сміс по-простому
визначає тут земельну ренту і зиск на капітал як прості відрахування з
продукту робітника або з вартости його продукту, що дорівнює праці,
долученій ним до сировинного матеріялу. Але це відрахування, як довів
раніше сам А. Сміс, може складатися лише з частини праці, яку робітник
додав до матеріялу понад ту кількість праці, яка оплачує тільки
його заробітну плату або дає еквівалент його заробітної плати, отже, з
додаткової праці, з неоплаченої частини його праці“.

„Відки виникає додаткова вартість капіталіста“ і, крім того, землевласника,
це знав, як бачимо, ще А. Сміс; Маркс відверто визнає це ще
