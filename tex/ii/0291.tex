бочої сили (лишаючи осторонь додаткову вартість); отже, він репродукує капіталістові в товаровій
формі ту частину капіталу, що ЇЇ капіталіст йому авансував або має авансувати як заробітну плату;
продукує йому еквівалент цієї плати; отже, він продукує капіталістові капітал, що його капіталіст
може знову „авансувати“ на закуп робочої сили.

Потрете. При продажу товару частина його продажної ціни повертає капіталістові авансований ним
змінний капітал і цим дає йому змогу знову купувати робочу силу, а робітникові — знову продавати її.

При всіх актах купівлі й продажу товарів — оскільки розглядається лише ці оборудки — цілком байдуже,
що зробить продавець з уторгованими за свій товар грішми, і що зробить покупець з купленими
предметами споживання. Отже, оскільки розглядається лише процес циркуляції, цілком не має значення
також та обставина, що куплена капіталістом робоча сила репродукує йому капітальну вартість, і що, з
другого боку,
гроші, вторговані як купівельна ціна робочої сили, становлять дохід робітника. На величину вартости
предмета торгівлі робітника, його робочої сили, не впливає ані те, що вона становить його „дохід“,
ані те, що споживання цього його предмета торгівлі покупцем репродукує цьому покупцеві капітальну
вартість.

Через те, що вартість робочої сили, — тобто адекватна продажна ціна цього товару — визначається
кількістю праці, потрібного для її репродукції, а саму цю кількість праці визначається тут тією
кількістю праці, яка потрібна для продукції потрібних засобів існування робітника, отже, кількістю
праці, потрібного для підтримання його життя, заробітна плата стає доходом, що з нього має жити
робітник.

Цілком неправильне твердження А. Сміса (стор. 223): „Частина капіталу, витрачена на утримання
продуктивної праці, після того як вона служила йому [капіталістові] в функції капіталу... становить
їх [робітників] дохід“. Гроші, що ними капіталіст оплачує куплену ним робочу силу „служать йому в
функції капіталу“, оскільки він за допомогою їх долучав робочу силу до речових складових частин
свого капіталу і тільки цим взагалі ставить свій капітал в умови, що в них він може функціонувати як
продуктивний капітал. Треба відрізняти таке: робоча сила в руках робітника є товар, а не капітал;
вона становить для нього дохід остільки, оскільки він може постійно повторювати її продаж; вона
функціонує як капітал після продажу в руках капіталіста, підчас самого процесу продукції. Робоча
сила служить тут подвійно; в руках робітника, як товар, продаваний за його вартістю; в руках
капіталіста, що купив її, як сила, що продукує вартість і споживну вартість. Але гроші, що їх
одержує робітник від капіталіста, він одержує лише після того, як він дав йому вжиток своєї робочої
сили, після того, як вона вже реалізована в вартості продукту праці. Капіталіст має цю вартість у
своїх руках,
перш ніж оплатить її. Отже, не гроші є те, що двічі функціонує: спочатку як грошова форма змінного
капіталу, а потім як заробітна плата. Двічі функціонує робоча сила: поперше, як товар при продажу
робочої сили (при визначенні розміру заробітної плати, що її треба
