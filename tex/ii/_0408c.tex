\parcont{}  %% абзац починається на попередній сторінці
\index{ii}{0408}  %% посилання на сторінку оригінального видання
вживання в хліборобстві зерна власної продукції. При обміні між І і II
цю частину ІІс теж не доводиться брати на увагу, як і Іс. Справа зовсім
не змінюється й тоді, коли частина продуктів II теж може ввійти в І
як засоби продукції. їх покривається частиною засобів продукції, поданих
підрозділом І, і цю частину треба з самого початку виключити на обох
сторонах, коли ми хочемо дослідити в чистому, незатемненому вигляді
обмін між двома великими клясами суспільної продукції, між продуцентами
засобів продукції та продуцентами засобів споживання.

Отже, при капіталістичній продукції І (у + m) не може дорівнювати
II с, або обидва вони при обміні не можуть покривати один одного.

Навпаки, коли І m x є та частина І m, що її як дохід витрачають капіталісти І,
то I ($v + m$ x) може бути рівне, більше або менше, ніж II c; але I ($v + m$ x)
завжди мусить бути менше, ніж II (с + m), а саме — менше на ту частину
II m, що її кляса капіталістів II сама мусить споживати за всяких обставин.
Треба відзначити, що при такому зображенні акумуляції не точно
зображено вартість сталого капіталу, оскільки він є частина вартости
товарового капіталу, що в його продукції цей сталий капітал бере участь.
Основна частина новоакумульованого сталого капіталу входить у товаровий
капітал лише поступінно й періодично, відповідно до різної природи
цих основних елементів; тому там, де сировинний матеріял, напівфабрикат
і т. ін. масами входять у товарову продукцію, найбільша частина
цього товарового капіталу складається з заміщення обігової сталої
складової частини та змінного капіталу. (Однак так можна було робити,
зважаючи на оборот обігових складових частин; таким чином, припускається,
що обігова частина разом з переміщеною на неї частиною вартости основного
капіталу обертається протягом року так часто, що вся сума вартости
виготовлених товарів дорівнює вартості цілого капіталу, який входить у
річну продукцію). Але там, де при машинній продукції входить не сировинний
матеріял, а лише допоміжні матеріяли, — елемент праці, рівний v,
мусить знов з’явитися в товаровому капіталі як його найбільша складова
частина. Тимчасом як у нормі зиску додаткову вартість обчислюється
на ввесь капітал, незалежно від того, багато чи мало вартости періодично
передають продуктові основні складові частини, — до вартости кожного
періодично вироблюваного товарового капіталу основну частину сталого
капіталу треба прираховувати лише остільки, оскільки вона в наслідок
пересічного зношування передає вартість самому продуктові.

IV. Додаткові зауваження

За первісне грошове джерело для II є $v + m$ золотопромисловости І,
обмінюване на частину II с; лише оскільки продуцент золота нагромаджує
додаткову вартість або перетворює її на засоби продукції І, отже,
поширює свою продукцію, його у + m не входить в II; з другого боку,
оскільки акумуляція грошей самим продуцентом золота, кінець-кінцем,
\parbreak{}  %% абзац продовжується на наступній сторінці
