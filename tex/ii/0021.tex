рослого в своїй вартості. Кругобіг грошового капіталу ніколи не може
починатися з Г' (хоч Г' тепер функціонує як Г), а тільки з Г; тобто
він ніколи не може починатись як вираз капіталістичного відношення, а
лише як форма авансування капітальної вартости. Скоро тільки 500 ф.
стерл. авансовано знову як капітал, для нового самозростання, вони
являють вихідний пункт замість кінцевого пункту. Замість капіталу в
422 ф. стерл. тепер авансовано капітал в 500 ф. стерл. — більше
грошей, ніж раніш, більше капітальної вартости, але відношення між
двома складовими частинами відпало цілком так само, як і первісно
могла б функціонувати як капітал сума в 500 ф. стерл. замість
422 ф. стерл.

З’являтись як Г' не є активна функція грошового капіталу; його
власне з’явлення в формі Г' є скорше функція Т'. Вже в простій товаровій
циркуляції: 1) Т1 — Г, 2) Г — Т2, Г активно функціонує лише в
другому акті Г — Т2; з’явлення його у вигляді Г є лише результат першого
акту, що лише силою його воно виступає як перетворена форма
Т1. Капіталістичне відношення, що міститься в Г', відношення однієї його
частини як капітальної вартости до другої його частини як приросту
цієї вартости, набуває, правда, функціонального значіння остільки, оскільки,
при постійному повторенні кругобігу Г... Г', Г' розподіляється між двома
циркуляціями — циркуляцією капіталу і циркуляцією додаткової вартости,
отже, обидві частини виконують не лише кількісно, але і якісно
різні функції, Г інші функції, ніж г. Але розглядувана сама по собі форма
Г... Г' не має в собі споживання капіталіста, а має лише виразно
самозростання й акумуляцію, оскільки остання виражається насамперед
у періодичному прирості знову та знову авансовуваного грошового
капіталу.

Хоч Г' = Г + г і є іраціональна форма капіталу, але разом з тим воно
є грошовий капітал в його реалізованій формі, як гроші, що породили
гроші. Але в цьому треба вбачати ріжницю від функції грошового
капіталу в першій стадії Г — Т Р Зп. В цій першій стадії Г циркулює
як гроші. Воно функціонує як грошовий капітал тільки тому, що
лише в своєму грошовому стані воно може виконувати функцію грошей,
перетворитись на елементи П, Р і Зп, що протистоять йому як товари.
В цьому акті циркуляції Г функціонує лише як гроші; а що цей акт
становить першу стадію капітальної вартости, що процесує, то він одночасно
є і функція грошового капіталу в наслідок специфічної споживної
форми купованих тут товарів Р і Зп. Навпаки, Г', що складається з Г, капітальної
вартости, і г, виробленої нею додаткової вартости, виражає вирослу
капітальну вартість, — мету й результат, функцію цілого процесу
кругобігу капіталу. Те, що воно виражає цей результат у грошовій формі,
як реалізований грошовий капітал, випливає не з того, що воно є грошова
форма капіталу, грошовий капітал, але, навпаки, з того, що воно є
грошовий капітал, капітал у грошовій формі, з того, що капітал у цій
