\index{ii}{*0005}  %% посилання на сторінку оригінального видання
Передмова

Підготувати до друку другу книгу „Капіталу“, та ще так, щоб вона
являла, з одного боку, суцільний і по змозі викінчений твір, а з другого
— твір виключно автора, а не упорядника, було не легкою роботою.
Це завдання ускладнялось великим числом наявних, здебільша фраґментарних
рукописів. Цілком готовий до друку був щонайбільше лише один
(рукопис IV), оскільки він міг придатись; але й у ньому більша частина
застаріла в наслідок перероблень пізнішого часу. Основна маса матеріялу,
хоч і була по суті переважно оброблена, але вона не була оброблена щодо
мови; матеріял викладено такою мовою, якою Маркс мав звичай робити свої
виписи: недбайливий стиль, фамільярні, часто грубо-гумористичні вислови
та звороти, англійські та французькі технічні назви, часто цілі речення
й навіть сторінки англійською мовою; це — виклад думок у тій формі,
що в ній вони щоразу розвивалися в голові автора. Поряд окремих,
докладно поданих частин, інші, не менш важливі, лише накреслено; фактичний
ілюстраційний матеріял зібрано, але ледве згруповано, не кажучи
вже про оброблення; наприкінці розділу, під тиском бажання перейти до
наступного розділу, часто маємо лише кілька уривчастих речень, які накреслюють
думку, але не розвивають її повнотою; нарешті, відома рука,
часто нерозбірна й для самого автора.

Я задовольнився з того, що відтворив рукописи по змозі дослівно,
змінюючи в стилі тільки те, що й сам Маркс був би змінив, і вставивши
лише пояснювальні речення та переходи там, де це було конче потрібно
й де, крім того, зміст не викликав жодних сумнівів. Ті речення, що їхній
зміст наводив хоч на деякі найвіддаленіші сумніви, я вважав за краще
друкувати дослівно. Мої перероблення та вставки не складають разом і
десятьох друкованих сторінок, і всі вони лише формального характеру.

Самий лише перелік залишеного Марксом рукописного матеріялу до
II книги доводить, з якою незрівняною сумлінністю, з якою суворою
самокритикою намагався він розробити до найвищої досконалости свої
великі економічні відкриття, раніш ніж оголошувати їх; ця самокритика
рідко коли давала йому змогу пристосувати свій виклад змістом і формою
до свого кругозору, дедалі ширшого в наслідок нових студій. Ці
матеріяли складаються ось із чого.

По-перше, є рукопис „Zur Kritik der politischein Ökonomiе“, 1472 сторінки
чвертьаркушевих, у 23 зшитках, написані від серпня 1861 р. де
червня 1863 р. Це є продовження першого зшитку, опублікованого в
\parbreak{}  %% абзац продовжується на наступній сторінці
