\parcont{}  %% абзац починається на попередній сторінці
\index{ii}{0228}  %% посилання на сторінку оригінального видання
вартости виражає не що інше, як відношення застосованого протягом
певного часу змінного капіталу до спродукованої протягом того самого
часу додаткової вартости; або — масу тієї неоплаченої праці, що її пускає
в рух змінний капітал, застосований протягом цього часу. Вона абсолютно
не має чинення до тієї частини змінного капіталу, яку авансовано,
але протягом певного часу не застосовується, отже, так само не
має ніякою чинення вона й до відношення між частиною капіталу, авансованого
в певний протяг часу, і тією частиною його, що її застосовано
протягом цього самого часу — відношення, що для різних капіталів
під впливом періодів обороту модифікується й є різне.

З наведеного вище скорше випливає, що річна норма додаткової вартости
лише в одному єдиному випадку збігається з справжньою нормою
додаткової вартости, яка виражає ступінь експлуатації праці: а саме в
тому разі, коли авансований капітал обертається тільки один раз на рік,
коли тому авансований капітал дорівнює капіталові, що обернувся протягом
року, а відношення маси додаткової вартости, спродукованої протягом
року, до капіталу, застосованого на її продукцію протягом року, збігається
і є тотожне з відношенням маси додаткової вартости, спродукованої
протягом року, до капіталу, авансованого протягом року.

A) Річна норма додаткової вартости дорівнює:

маса додаткової вартости, спродукованої протягом року: авансований змінний капітал

Але маса додаткової вартости, спродукованої протягом року, дорівнює
справжній нормі додаткової вартости, помноженій на змінний капітал,
застосований на її продукцію. Капітал, застосований на продукцію
річної маси додаткової вартости, дорівнює авансованому капіталові, помноженому
на число оборотів його, яке ми позначатимемо n. Тому формула А) перетворюється на таку:

B) Річна норма додаткової вартости дорівнює:

справжня норма додаткової вартости × аванс. змінний капітал × n: авансований змінний капітал

Наприклад, для капіталу В = 100\%Х500Х1: 5000 або 100\%. Тільки коли
n = 1, тобто, коли авансований змінний капітал обертається тільки
один раз на рік, отже, дорівнює застосованому протягом року капіталові,
або капіталові, що обернувся протягом року, — тільки тоді річна норма додаткової
вартости, дорівнює справжній нормі додаткової вартости.

Коли ми позначимо річну норму додаткової вартости М', справжню норму
додаткової вартости m', авансований змінний капітал — v, число оборотів
— n, то М' = m'vn: $v = m$'n; отже, М' = m'n, і лише тоді = m', коли
n = 1; отже, М' = m'X$1 = m$'.
\parbreak{}  %% абзац продовжується на наступній сторінці
