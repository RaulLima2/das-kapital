ні з другого боку подавати гроші в циркуляцію для оплати ріжниці.
Отже, проблема виступає в своєму чистому вигляді лише тоді, коли ми
викреслимо на обох боках, І і II, товар 200 Im і його еквівалент товар
200 ІІс (частини 1).

Отже, усунувши ці дві товарові величини рівної вартости (І і II),
що навзаєм одна одну урівноважують, матимемо решту обміну, в якому
проблема виступає в чистому вигляді, а саме:

I    200m товаром.

II (1) 200c грішми + (2) 200c товаром.

Тут очевидно: II, частина 1, на 200 грішми купує складові частини
свого основного капіталу, 200 Іm, в наслідок цього основний капітал II,
частини 1, відновлено in natura, а додаткова вартість І, вартістю в 200,
з товарової форми (засоби продукції, — а саме, елементи основного
капіталу) перетворена на грошову форму. На ці гроші І купує засоби
споживання у II, частини 2, а результат для II такий, що для частини 1
відновлено in natura основну складову частину її сталого капіталу; і що
для частини 2 друга складова частина (яка замішує зношування основного
капіталу) осіла в формі грошей, і це щороку повторюється доти,
доки й цю складову частину треба буде відновити in natura.

Попередня умова тут, очевидно, в тому, щоб ця основна складова частина
сталого капіталу II, яка в розмірі всієї своєї вартости зворотно перетворюється
на гроші, і яку, отже, кожного року треба відновлювати in natura
(частина 1), — щоб вона була рівна річному зношуванню другої основної
складової частини сталого капіталу II, яка все ще й далі функціонує в
своїй старій натуральній формі, і зношування якої — втрату вартости,
переношувану на товари, що в їхній продукції функціонує ця частина —
спочатку треба замістити грішми. Тому така рівновага з’являється як
закон репродукції в незмінному маштабі; це значить, інакше кажучи, що
пропорційний поділ праці в клясі 1, яка продукує засоби продукції,
мусить лишатись незмінний, оскільки вона дає, з одного боку, обігові,
а з другого — основні складові частини сталого капіталу підрозділові II.

Перш ніж ближче дослідити це, ми повинні розглянути, який вигляд
матиме справа, коли решта суми IIс (1) не дорівнюватиме решті IIс
(2); вона може бути більша або менша за цю останню. Візьмімо один
по одному обидва ці випадки.

Перший випадок:

I.    200m.

II. (1) 220с (грішми) + (2) 200с (товаром).

Тут IIс (1) на 220 ф. стерл. грішми купує товари 200 Im, а І на
ті самі гроші купує товари 200 ІІс (2), тобто ту складову частину
основного капіталу, яка має осісти в грошовій формі; її перетворено
таким чином на гроші. Але 20 IIс (1) грішми не сила перетворити на
основний капітал in natura.

Цьому лихові можна, здається, запобігти, коли ми припустимо, що решта
Im дорівнює не 200, а 220, так що з суми 2000 І попереднім обміном закін-
