\parcont{}  %% абзац починається на попередній сторінці
\index{ii}{0157}  %% посилання на сторінку оригінального видання
від того, чи сплачує він її щотижня, щомісяця, чи раз на три місяці.
В дійсності справа стоїть не так, а якраз навпаки. Робітник авансує
капіталістові свою роботу на тиждень, на місяць, на три місяці, залежно
від того, як виплачується йому заробітну плату — щотижня, щомісяця або
раз на три місяці. Коли б капіталіст купував робочу силу замість
оплачувати її після зужитку, отже, коли б він платив робітникові заробітну
плату за день, тиждень, місяць, або три місяці наперед, то можна
було б казати про авансування на ці періоди часу. Але що він платить
після того, як праця вже тривала дні, тижні, місяці, замість купити й
оплатити працю за період часу, що протягом нього вона повинна тривати,
то в цілому ми маємо капіталістичне qui pro quo\footnote*{
Qui pro quo — дослівно „один замість другого“, тобто сплутування, що
випливає з неправильної заміни однієї речі або явища на іншу або інше. Ред.
}, і аванс, виданий
робітником капіталістові в вигляді праці, перетворюється на аванс, що
його видає капіталіст робітникові в формі грошей. Справа зовсім не змінюється
від того, що капіталіст одержує назад із сфери циркуляції або
реалізує самий продукт або його вартість (разом з вміщеною в ньому
додатковою вартістю) тільки після довшого або коротшого періоду, залежно
від різного протягу часу, потрібного на виготовлення продукту,
а також від різного протягу часу, потрібного для його циркуляції. Продавцеві
цілком байдуже, що саме хоче зробити з його товаром покупець.
Капіталістові машина не дістається дешевше від того, що він мусить
одразу авансувати всю її вартість, тимчасом як до нього ця вартість
зворотно припливає з процесу циркуляції лише поступінно й частинами;
так само він не платить дорожче за бавовну в наслідок того, що її
вартість входить цілком у вартість виготовленого з неї продукту, а
значить, цілком і одним заходом покривається через продаж продукту.

Повернімось до Рікардо:

1) Характеристична властивість змінного капіталу в тому, що певну,
дану (отже, і, як таку, сталу) частину капіталу, дану суму вартости
(припускається, що вона дорівнює вартості робочої сили, хоч у даному
разі все одно, чи дорівнює заробітна плата вартості робочої сили, чи
більша, чи менша від неї) обмінюється на силу, що сама з себе
зростає вартістю, утворює вартість, — на робочу силу, яка не лише репродукує
виплачену капіталістом вартість її, але разом із тим продукує
додаткову вартість, вартість, що не було її раніш і не була вона
оплачена будь-яким еквівалентом. Ця характеристична властивість частини
капіталу, витраченої на заробітну плату, властивість, що відрізняє її як
змінний капітал toto соеіо\footnote*{
Toto соеіо — всюди, всіма сторонами, цілком. Ред.
} від капіталу сталого, зникає, якщо частина
капіталу, витрачена на заробітну плату, розглядається виключно з погляду
процесу циркуляції і виступає таким чином як обіговий капітал
протилежно до основного капіталу, витраченого на засоби праці. Це
випливає вже з того, що тоді під тією самою рубрикою — під рубрикою
обігового капіталу — частина капіталу, витрачена на робочу силу разом із
складовою частиною сталого капіталу, витраченого на матеріял праці,
\parbreak{}  %% абзац продовжується на наступній сторінці
