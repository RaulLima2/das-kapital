тус і в зв’язку з своєю працею „Zur Erkenntniss unserer staatswirtschaftlichen
Zustände“ (1842) пише Й. Целлерові („Tübinger Zeitschrift für
die Gesamte Staatswissenschaft“, 1879, S. 219) ось що: „Ви побачите, що
в цього“ (з розвинутих тут думок) „непогано скористався... Маркс, звичайно,
не посилаючись на мене“. Це повторює, не довго думаючи, за ним
і його посмертний видавець, Т. Козак („Das Kapital von Rodbertus“, Berlin
1884, Einleitung, S. XV). — Нарешті, у виданих 1881 року Р. Маєром
„Briefe und sozialpolitischen Aufsätze von Dr. Rodbertus-Jagetzow“ Родбертус
прямо каже: „Тепер я бачу, як мене обікрали Шефле й Маркс, не
посилаючись на мене“ (Brief № 60, S. 134). А в другому місці претенсія
Родбертуса набирає виразнішої форми: „Відхи виникає додаткова
вартість капіталіста, це я показав у моєму третьому „Соціяльному листі“
посутньо так само, як Маркс, тільки коротше та виразніше“ (Brief
№ 48, S. 111).

Про всі ці обвинувачення в плягіяті Маркс ніколи нічого не знав.
В його примірнику „Emanzipationskampf“ — розрізано тільки частину,
що стосується до Інтернаціоналу, а решту книги розрізав уже я після
його смерти. Тюбінґенської „Zeitschrift“ він ніколи не бачив. „Briefe“ etc.
до Р. Маєра також були йому невідомі, і мою увагу на місце, що стосується
„обкрадання“, лише 1887 року, ласкаво звернув сам п. д-р Маєр.
Навпаки, лист №48 був Марксові відомий; п. Маєр сам з своєї ласки
подарував ориґінал молодшій дочці Марксовій. Маркс, що до нього звичайно
дійшло потайне шушукання про таємні джерела його критики, які
треба шукати у Родбертуса, показав мені цього листа й додав, що тут
він має, нарешті, автентичне свідчення про те, на що, власне, претендує
сам Родбертус; коли він не каже нічого більш, то для нього, тобто для
Маркса, справа йде на добре; а що Родбертус уважає свій виклад за
коротший та виразніший, то він може дати йому й це задоволення.
Маркс справді гадав, що цим листом Родбертуса вичерпано всю справу.

І так можна було гадати то більше, що, як я добре знаю, вся літературна
діяльність Родбертуса лишалась невідома Марксові до 1859 року,
коли його власна критика політичної економії не лише в основному, але
й у найважливіших подробицях була готова. Свої економічні студії він
почав 1843 року в Парижі, вивчаючи великих англійців і французів; з
німців він знав лише Рав і Ліста, і цього йому було досить. Ні Маркс,
ні я не знали нічогісінько про існування Родбертуса, поки 1848 року
не довелось нам критикувати в „Neue Rheinische Zeitung“ його промову,
як берлінського депутата, і його вчинки, як міністра. Ми були так необізнані,
що запитували райнських депутатів, хто це такий Родбертус,
що так швидко зробився міністром. Але й вони не могли нам нічого
сказати про економічні праці Родбертуса. Навпаки, що Маркс і без допомоги
Родбертуса вже тоді дуже добре знав, не лише звідки, але також
і як „виникає додаткова вартість капіталіста“, — це доводять його
„Misère de la Philosophie“ 1847 року і лекції про найману працю та капітал,
прочитані 1847 р. в Брюсселі й опубліковані 1849 р. в „Neue
Rheinische Zeitung“, під № 264—69. Тільки щось 1859 р. Маркс до-
