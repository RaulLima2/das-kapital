ції. Коли конечність цього для капіталіста постає не в Т' — Г', то постає
вона для нього в Г — Т; якщо не для його товарового капіталу, то
для товарового капіталу інших капіталістів, що продукують засоби продукції
для нього й засоби існування для його робітників.

Здавалось би, суть справи ніяк не може змінитись від того, чи утворюється
запас добровільно, чи недобровільно, тобто, чи навмисно товаропродуцент
тримає запас, чи його товари утворюють запас у наслідок того опору, що його
обставини самого процесу циркуляції протиставлять продажеві товарів. Усе
ж, щоб розв’язати це питання, корисно знати, чим відрізняється добровільне
утворення запасу від недобровільного. Недобровільне утворення запасу
випливає з затримки в циркуляції або є тотожне із затримкою в циркуляції,
що є незалежна від передбачення товаропродуцента і перешкоджає його
волі. Що характеризує добровільне утворення запасу? Завжди продавець
хоче яко мога швидше збути свій товар. Він завжди подає свій продукт як
товар. Коли він утримується від продажу, то продукт утворює лише можливий
(δυναμει), а не справжній (ενεργεια) елемент товарового запасу.
Товар як такий, як і раніше, є для нього лише носій мінової вартости,
і як мінова вартість він може діяти, лише скинувши з себе товарову
форму й набравши грошової форми.

Товаровий запас мусить доходити певних розмірів, щоб протягом
даного періоду задовольняти розміри попиту. При цьому треба зважувати
й на постійне збільшення кола покупців. Напр., щоб вистачити на один день,
частина товарів, яка є на ринку, мусить завжди затримуватись у товаровій
формі, тимчасом як друга тече, перетворюється на гроші. Та частина,
яка затримується, поки друга тече, правда, постійно меншає, так само,
як зменшуються розміри самого запасу, аж поки його ввесь продадуть.
Отже, застій товару тут зважено, як доконечну умову продажу
товару. Далі, розміри його мусять бути більші, ніж середні розміри
продажу або середні розміри його попиту. Бо інакше надлишок над
середнім розміром попиту не можна було б задовольнити. З другого боку,
запас має постійно поновлюватись, бо він постійно витрачається. Це
відновлення може, кінець-кінцем, походити лише з продукції, з подання
товару. Чи його довозиться з-за кордону, чи ні, це не змінює суті
справи. Відновлення залежить від протягу часу, потрібного на репродукцію
товарів. На ввесь цей час має вистачати товарового запасу. Що
він не лишається в руках первісних продуцентів, а переходить через різні
сховища, починаючи від гуртового торговця й до роздрібного торговця
— це змінює лише зовнішність, а не суть справи. З погляду суспільства,
як і раніш, частина капіталу лишається в формі товарового запасу доти,
доки товар увійде в сферу продуктивного або особистого споживання. Сам
продуцент намагається мати на складі товаровий запас відповідно до його
пересічного попиту, щоб не залежати безпосередньо від продукції й
забезпечити собі постійне коло покупців. Відповідно до періодів продукції
встановлюються строки купівель, і протягом довшого або коротшого
часу товар становить запас, поки його заступлять нові екземпляри того
таки ґатунку. Тільки таким утворенням запасу забезпечується сталість
