ляції мусять оплачувати аґенти продукції. Але коли капіталісти, купуючи
й продаючи один одному, не утворюють цим актом ні продуктів, ні
вартостей, то це зовсім не змінюється від того, що розміри їхнього
підприємства дають їм змогу і примушують їх перекладати ці функції на
інших. У деяких підприємствах закупників і продавців оплачується
тантьємами із зиску. Коли кажуть, ніби їх оплачують споживачі, то це
нічого не пояснює. Споживачі можуть оплачувати лише остільки, оскільки
вони сами, як аґенти продукції, продукують еквівалент у товарах або
привлащують його собі від аґентів продукції, хоч на основі правного титула
(як члени товариства тощо), хоч на основі особистих послуг.

Між Т — Г і Г — Т є ріжниця, що не має ніякого чинення до ріжниці
форми товару і грошей, а випливає з капіталістичного характеру
продукції. Сами собою Т — Г так само, як і Г — Т, є прості перетворення
даної вартости з однієї форми на іншу. Але Т' — Г', є разом з тим
реалізація додаткової вартости, що міститься в Т'. Інакше справа стоїть з
Г — Т. Тому продаж важливіший за купівлю. Г — Т в нормальних умовах
є акт, потрібний для зростання вартости, вираженої в Г, але він не є
реалізація додаткової вартости; це вступ до її продукції, а не доповнення
до неї.

Для циркуляції товарового капіталу Т' — Г' певні межі кладуться
формою існування самих товарів, їхнім буттям як споживних вартостей.
Але останні з своєї природи минущі. Тому, коли протягом певного часу
вони не ввійдуть у продуктивне або особисте споживання, залежно від
їхнього призначення, коли, інакше кажучи, їх не продасться протягом
певного часу, то вони псуються й разом зі своєю споживною вартістю
втрачають властивість бути носіями мінової вартости. Вміщена в них
капітальна взртість, зглядно і приросла до неї додаткова вартість, втрачається.
Споживні вартості лишаються носіями капітальної вартости, яка
зберігається протягом років і зростає в своїй вартості лише остільки,
оскільки вони постійно відновлюються та репродукуються, заміщуючись
на нові споживні вартості того самого або іншого ґатунку. Але продаж
їх у формі готових товарів, отже, перехід їхній за допомогою продажу
в сферу продуктивного або особистого споживання, є завжди поновлювана
умова їхньої репродукції. Вони мусять протягом певного часу
перемінити свою стару споживну форму, щоб далі існувати в новій.
Мінова вартість зберігається лише в наслідок цього постійного поновлення
її тіла. Споживні вартості різних товарів псуються швидше
або повільніше; тому між їхньою продукцією та споживанням може
минути довший або коротший час; отже, вони можуть, не знищуючись
більш або менш довгий час, лишатись у фазі циркуляції. Т — Г
як товаровий капітал, можуть витримати як товари більш або менш
довгий час обігу. Межі часу обігу товарового капіталу, зумовлені псуванням
самого товарового тіла, є абсолютні межі цієї частини часу обігу,
або того часу обігу, що протягом його товаровий капітал може існувати
як товаровий капітал. Що нетриваліший товар, що швидше треба
спожити його безпосередньо по продукції, а, значить, і продати, то на
