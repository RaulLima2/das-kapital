несеться на пряжу. Відси зрозуміло, що авансований грошовий капітал
спочатку треба перетворити на ці знаряддя, отже, він мусить вийти а
першої стадії Г — Т раніш, ніж матиме змогу функціонувати як продуктивний
капітал П. Так само зрозуміло в нашому прикладі, що сума
капітальної вартости в 422 ф. стерл., долучена до пряжі протягом продукційного
процесу, не може ввійти в фазу циркуляції Т' — Г' як складова
частина вартости 10.000 ф. пряжі, поки ця пряжа не буде готова.
Не можна продати пряжі, поки її не напряли.

В загальній формулі продукт П, тобто продуктивного капіталу розглядається
як матеріяльна річ, відмінна від елементів продуктивного капіталу, як
предмет, що існує відокремлено від продукційного процесу й має споживну
форму, відмінну від елементів продукції. Його розглядається так завжди, коли
результат продукційного процесу виступає як річ, навіть і тоді, коли частина
продукту знову входить у відновлювану продукцію як її елемент. Так,
збіжжя придається на засів для зернової продукції; але продукт складається
лише із збіжжя, отже, має форму, відмінну від інших елементів, що
їх разом застосувалось — робочої сили, інструментів, добрива. Але є
самостійні галузі індустрії, де продукт продукційного процесу не є
новий речевий продукт, не є товар. З них економічно важлива лише
промисловість комунікаційна, хоч буде то промисловість власне транспортова
для товарів і людей, хоч для пересилання просто повідомлень,
листів, телеграм тощо.

А. Чупров\footnote{
А. Чупров „Железнодорожное хозяйство“. Москва, 1875, стор. 75, 76.
} каже про це так: „Фабрикант може спочатку спродукувати
предмети, а потім шукати для них споживачів“ [його продукт, виштовхнутий
готовим з продукційного процесу, переходить у циркуляцію як
відокремлений від нього товар]. „Таким чином продукція і споживання є
два акти, відокремлені в просторі й часі. У транспортовій промисловості,
що не утворює нових продуктів, а тільки переміщує людей і речі, ці
обидва акти зливаються; послуги“ [переміщення] „повинні споживатися в
той самий момент, коли їх продукується. Тому район, де залізничні шляхи
можуть шукати споживачів, поширюється найбільш на 50 верстов
(53 кілометри) в обидва боки“.

Результат — все одно, чи перевозять людей, чи товари — є зміна перебування,
напр., пряжа тепер перебуває в Індії, а не в Англії, де її спродуковано.
Але транспортова промисловість продає саме переміщення. Корисний
ефект, що його вона дає, нерозривно сполучається з процесом транспорту,
тобто з продукційним процесом транспортової промисловости.
Люди й товари їдуть разом з засобами транспорту, і ця їхня їзда, цей
рух у просторі, і є продукційний процес, здійснюваний цими засобами. Корисний
ефект можна споживати тільки протягом продукційного процесу;
цей ефект не існує як відмінна рід цього процесу річ споживання,
що, лише бувши спродукована, фігурує як предмет торговлі,
циркулює як товар. Але мінова вартість цього корисного ефекту, як і