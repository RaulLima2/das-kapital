\parcont{}  %% абзац починається на попередній сторінці
\index{ii}{0281}  %% посилання на сторінку оригінального видання
складатись лише з його зисків. Отже, частина його товарового продукту,
яка заміщує його капітал, не розкладається на складові частини вартости,
які для нього становлять дохід.

2) Обіговий капітал кожного індивідуального капіталіста становить
частину обігового капіталу суспільства цілком так само, як і кожен індивідуальний
основний капітал.

3) Обіговий капітал суспільства, хоч і є він лише сума індивідуальних
обігових капіталів, має характеристичну особливість, яка відрізняє
його від обігового капіталу кожного індивідуального капіталіста. Останній
ніколи не може становити частини його доходу; частина першого
(а саме та, що складається з засобів споживання), навпаки, може разом
з тим становити частину доходу суспільства, або, як раніш казав
А. Сміс, ця частина капіталу не мусить неодмінно зменшувати чистий
дохід суспільства на частину річного продукту. На ділі те, що А. Сміс
зве тут обіговим капіталом, складається з щорічно продукованого товарового
капіталу, що його капіталісти, які продукують засоби споживання,
щороку подають у циркуляцію. Ввесь цей їхній річний товаровий продукт
складається з предметів споживання й тому становить фонд, що в ньому
реалізується або що на нього витрачається чисті доходи суспільства (включно
й заробітну плату). Замість обирати як приклад товари, що є в крамниці
дрібного торговця, А. Смісові треба було б взяти маси продуктів, що
лежать на товарових складах промислових капіталістів.

Якби А. Сміс резюмував усі уривки думок, які поставали в нього
раніш при розгляді репродукції того капіталу, що його він зве основним,
а тепер при розгляді репродукції того капіталу, що його він зве
обіговим капіталом, то він дійшов би такого результату.

I. Суспільний річний продукт складається з двох відділів: перший
охоплює засоби продукції, другий — засоби споживання; кожен з цих
відділів треба розглядати окремо.

II. Загальна вартість тієї частини річного продукту, що складається
з засобів продукції, розподіляється так: частина вартости є
лише вартість засобів продукції, зужиткованих на виготовлення цих засобів
продукції; отже, це є капітальна вартість, що лише знову з’явилася
в новій формі: друга частина дорівнює вартості капіталу, витраченого
на робочу силу, або дорівнює сумі заробітних плат, виданих капіталістами
цієї сфери продукції. Нарешті, третя частина вартости становить
джерело зиску — включно й земельну ренту — промислових капіталістів
цієї категорії.

Перша складова частина, за А. Смісом репродукована частина основного
капіталу всіх зайнятих у цьому першому відділі індивідуальних
капіталів, „очевидно, виключається й ніколи не може становити частини
чистого доходу“, хоч індивідуальних капіталістів, хоч суспільства. Вона
завжди функціонує як капітал і ніколи не функціонує як дохід. У цьому
— „основний капітал“ кожного індивідуального капіталіста нічим не
відрізняється від основного капіталу суспільства. Але інші частини вартости
річного продукту суспільства, що складається з засобів продукції —
\parbreak{}  %% абзац продовжується на наступній сторінці
