ми повинні припустити, що частина новоутвореного грошового капіталу,
яку можна перетворити на змінний капітал, завжди знаходить робочу
силу, на яку вона має перетворитись. Так само в книзі І\footnote*{
Див. „Капітал“, т. I, розділ 22, §4. — Ред.
} з’ясовано,
як даний капітал може без акумуляції поширювати в певних межах розміри
своєї продукції. Але тут ідеться про акумуляцію капіталу в специфічному
значенні, так що поширення продукції зумовлюється перетворенням
додаткової вартости на додатковий капітал, отже, зумовлюється
поширенням капітальної бази продукції.

Золотопромисловець може акумулювати деяку частину своєї золотої
додаткової вартости як віртуальний грошовий капітал; скоро він досягає
потрібних розмірів, золотопромисловець може безпосередньо перетворити
його на новий змінний капітал, для цього йому не доводиться спочатку
продавати свій додатковий продукт; так само він може перетворити свій
додатковий продукт на елементи сталого капіталу. Однак у цьому останньому
випадку він мусить знайти ці речові елементи свого сталого капіталу,
все одно, чи робить кожен продуцент, як ми припускали в попередньому
викладі, про запас, а потім виносить свої готові товари на ринок,
чи він робить на замовлення. В обох випадках припускається реальне
поширення продукції, тобто припускається додатковий продукт, в
одному випадку як справді наявний, а в другому як віртуально наявний,
що може бути поданий.

II. Акумуляція в підрозділі II

До цього часу ми припускали, що А, А, ' А" (І) продають свій до
датковий продукт В, В, ' В" і т. д „що належать до того самого підрозділу
І. Але припустімо, що А (І), перетворює на золото свій додатковий
продукт, продаючи його В з підрозділу II. Це може статися
лише в наслідок того, що А (І), продавши В (II) засоби продукції,
не купує потім засобів споживання; отже, лише в наслідок однобічного
продажу з його боку. Оскільки II с з форми товарового капіталу можна
перетворити на натуральну форму продуктивного сталого капіталу лише
таким способом, що не тільки I V, але принаймні, і деяка частина I m
обмінюється на деяку частину II с, яке існує у формі засобів споживання;
але тепер А перетворює на золото своє І ш в такий спосіб, що такого
обміну не постає; навпаки, наш А вилучає з циркуляції гроші, вторговані
від II через продаж свого І ш, замість повертати їх на закуп засобів
споживання II с, — то хоч на боці А (І) утворюється додатковий
віртуальний грошовий капітал, але на другому боці закріплюється в формі
товарового капіталу рівна величиною вартости частина сталого капіталу
В (II), яка не може перетворитись на натуральну форму продуктивного
сталого капіталу. Інакше кажучи, не сила продати частину товарів
В (II), а саме prima facie ту частину, що без її продажу В (II) не може
знову перетворити ввесь свій сталий капітал на продуктивну форму;