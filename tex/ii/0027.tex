тивне споживання. Потретє, поворот грошей до їхнього вихідного пункту
робить рух Г... Г' кругобігом, замкненим у собі самому.

Отже, з одного боку, кожний індивідуальний капітал у своїх обох
половинах циркуляції, Г-Т і Т' — Г', є чинник загальної товарової
циркуляції, що в ній він функціонує або вплітається то як гроші, то як
товар; і таким чином сам він є член у загальному ряді метаморфоз товарового
світу. З другого боку, він пророблює в межах загальної циркуляції свій
власний самостійний кругобіг, що в ньому сфера продукції є перехідна
стадія і що в ньому він повертається до свого вихідного пункту в тій
самій формі, в якій вийшов з нього. У межах свого власного кругобігу,
що має в собі його реальну метаморфозу в продукційному процесі,
капітал змінює також величину своєї вартости. Він повертається не лише
як грошова вартість, а як збільшена, виросла грошова вартість.

Нарешті, коли ми розглядатимемо Г — Т... П... Т' — Г' як специфічну
форму процесу кругобігу капіталу поряд інших форм, що їх дослідиться
в дальшому, то воно відзначатиметься ось чим.

1) Воно являє кругобіг грошового капіталу, бо промисловий капітал
в його грошовій формі, як грошовий капітал, є вихідний і кінцевий
пункт цілого його процесу. Сама формула виражає, що гроші тут не
витрачається як гроші, а лише авансується, а тому є лише грошова
форма капіталу, грошовий капітал. Вона, крім того, виражає, що мінова
вартість, а не споживна вартість, є самоціль, що визначає рух. Саме тому,
що грошова форма вартости є її самостійна, наочна форма виявлення, —
саме тому форма циркуляції Г... Г', що її вихідний і кінцевий
пункти є справжні гроші, і виражає якнайнаочніше движний чинник
капіталістичної продукції, роблення грошей. Продукційний процес
є лише неминуча посередня ланка, неодмінне лихо для роблення грошей.
[Тому всі нації з капіталістичним способом продукції періодично захоплюються
шахрайством, що за допомогою його вони намагаються робити
гроші без продукційного процесу]. .

2) Стадія продукції, функція П, становить у цьому кругобігу перерву
між двома фазами циркуляції Г — Т... Т' — Г', яка знову таки є лише
посередня ланка простої циркуляції Г — Т — П. Продукційний процес у
самій цій формі процесу кругобігу формально й виразно виступає як
те, чим він є в капіталістичному способі продукції, як звичайний спосіб
збільшувати авансовану вартість, отже, збагачення, як таке, виступає як
самоціль продукції.

3) Що ряд фаз починається фазою Г — Т, то другий член циркуляції
є Т' — Г'; отже, вихідний пункт — Г, грошовий капітал, що має збільшити
свою вартість; кінцевий пункт — Г', вирослий у своїй вартості грошовий
капітал Г + г, де Г фігурує як реалізований капітал поряд свого
паросту г. Це відрізняє кругобіг Г від двох інших кругобігів П і Т; відрізняє
двома сторонами. З одного боку, грошовою формою обох крайніх членів;
а гроші с самостійна, наочна форма існування вартости, вартість продукту
в ії самостійній формі, де зник будь-який слід споживної вартости
товарів. З другого боку, форма П... П не неодмінно перетворюється на
