\parcont{}  %% абзац починається на попередній сторінці
\index{ii}{0165}  %% посилання на сторінку оригінального видання
пише: „Далі треба зазначити, що обіговий капітал може циркулювати
або повертатися до свого власника в дуже різні періоди часу. Пшениця,
куплена фармером на засів, є основний капітал, порівняно з пшеницею,
що її купив пекар, щоб перетворити її на хліб. Один лишає пшеницю
в землі й може одержати її знову лише за рік, другий може віддати її
змолоти на борошно й продати як хліб своїм покупцям, так що протягом
одного тижня капітал його знову стає вільний, і може він знову почати
з ним ту саму або будь-яку іншу операцію“\footnote{
„It is also to be observed that the circulating capital may circulate, or be
returned to its employer, in very unequal times. The wheat bought by a farmer to
sow is comparatively a fixed capital to the wheat purchased by a baker to make
into loaves. The one leaves it in the ground, and can obtain no return for a year; the
other can get it ground into flour, sell it as bread to his customers, and have his capital
free, to renew the same, or commence any other employement in a week“. (Ricardo,
1. c., p. 26, 27).
}.

Тут характерно те, що пшениця — хоч вона як зерно служить не як
засіб існування, а як сировинний матеріял, є, поперше, обіговий капітал,
бо вона сама по собі є засіб існування, і, подруге, основний капітал, бо
її зворотний приплив відбувається через рік. Але не повільніший або
швидший зворотний приплив робить даний засіб продукції основним капіталом,
основним капіталом його робить певний спосіб перенесення його
вартости на продукт.

Плутанина понять, що походить від А. Сміса, призводить до таких
наслідків:

1) Ріжницю між основним і поточним капіталом сплутується з ріжницею
між продуктивним капіталом і товаровим капіталом. Так, напр.,
та сама машина є обіговий капітал, коли вона як товар перебуває на
ринку, і основний капітал, коли її введено в процес продукції. При
цьому лишається абсолютно незрозуміле, чому певний рід капіталу треба
вважати більше за основний або більше за обіговий, ніж інший, рід капіталу.

2) Всякий обіговий капітал ототожнюється з капіталом, що його витрачено
або треба витратити на заробітну плату. Так каже Дж. Ст. Мілл та ін.

3) Ріжницю між змінним і сталим капіталом, що її вже Бартон, Рікардо
та ін. сплутують з ріжницею між обіговим і основним капіталом, зводиться,
кінець-кінцем, цілком на цю останню, як, напр., у Рамсея, що в нього
всі засоби продукції, сировинні матеріяли і т. ін., так само і засоби праці,
є основний капітал, і лише капітал, витрачений на заробітну плату, є
обіговий капітал. А що це зведення робиться саме в такій формі, то й
лишається незрозуміла справжня ріжниця між сталим і змінним капіталом.

4) У новітніх англійських, особливо шотландських економістів, що розглядають
усе з невимовно обмеженого погляду банкірського прикажчика,
у Маклеода, Патерсона та ін. ріжниця між основним і обіговим капіталом
перетворюється на ріжницю між money at call і money not at call
(грошові вклади, що їх одержується назад без попереднього повідомлення,
і грошові вклади, що їх одержується назад лише після попереднього
повідомлення).
