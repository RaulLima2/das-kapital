\parcont{}  %% абзац починається на попередній сторінці
\index{ii}{0234}  %% посилання на сторінку оригінального видання
як продуктивний капітал, в даному разі зглядно як змінний капітал.
Правда, його вартість заміщено новою вартістю, отже, відновлено, але
форму його вартости (тут абсолютну форму вартости, її грошову форму)
не відновлено.

Отже, для другого п’ятитижневого періоду (і таким самим робом
послідовно для кожних п’ятьох тижнів протягом року) мусять так
само бути в запасі дальші 500 ф. стерл., як і для першого періоду.

Отже, залишаючи осторонь кредитові відносини, на початку року мусять
бути в запасі 5000 ф. стерл., як латентний авансований грошовий капітал,
хоч в дійсності їх витрачатиметься, перетворюватиметься на робочу
силу протягом року лише поступінно.

Навпаки, у $А$ через те, що кругобіг, оборот авансованого капіталу,
вже закінчено, вартість для заміщення вже по скінченні перших п’ятьох
тижнів перебуває в такій формі, в якій вона може пустити в рух нову
робочу силу на 5 тижнів: вона перебуває у своїй первісній грошовій формі.

Протягом другого п’ятитижневого періоду в $А$, як і в $В$, зуживається
нову робочу силу, і для оплати цієї робочої сили витрачається новий капітал
в 500 ф. стерл. Засоби існування робітників, куплені на перші 500 ф.
стерл., зникли, в усякому разі вартість їхня зникла з рук капіталіста.
На другі 500 ф. стерл. купується нову робочу силу, нові засоби існування
вилучається з ринку. Коротко кажучи, витрачається новий капітал в 500 ф.
стерл., а не старий. Але в $А$ цей новий капітал в 500 ф. стерл. є грошова
форма новоспродукованої вартости, що заміщує раніше витрачені 500 ф.
стерл. У $В$ ця вартість, що являє собою заміщення $v$, перебуває в такій
формі, що в ній вона не може функціонувати як змінний капітал. Вона є
наявна, але не в формі змінного капіталу. Тому для дальшого провадження
продукційного процесу на ближчі п’ять тижнів мусить бути наявний неодмінно
в грошовій формі додатковий капітал в 500 ф. стерл. і мусить він авансуватись.
Таким чином, протягом 50 тижнів і $А$, і $В$ витрачають однаковий
змінний капітал, оплачують і споживають однакову кількість робочої сили.
Але $В$ мусить оплатити її, авансувавши капітал, що дорівнює всій її
вартості = 5000 ф. стерл., тимчасом як $А$ оплачує її послідовно раз-у-раз
відновлюваною грошовою формою вартости, яка продукується протягом
кожних 5 тижнів і заміщує капітал в 500 ф. стерл., авансовуваний на кожні
5 тижнів. Отже, тут ніколи не авансується грошовий капітал більш як
на 5 тижнів, тобто ніколи не авансується капітал більший, ніж капітал
у 500 ф. стерл., авансований на перші п’ять тижнів. Цих 500 ф. стерл. вистачає
на цілий рік. Тому, очевидно, що за однакового ступеня експлуатації
праці, за однакової справжньої норми додаткової вартости, річні
норми для $А$ і В мусять бути у зворотному відношенні до величин змінних
грошових капіталів, що їх треба було авансувати для того, щоб
протягом року пустити в рух ту саму масу робочої сили.
\[
А: \frac{5000 m}{500 v} = 1000\% \text{ і } B: \frac{5000 m}{5000 v} = 100\%
\]

Але $500 v: 5000 v = 1: 10 = 100\%: 1000\%$.
