зношування в наслідок залізничного руху, скільки від якости дерева, заліза та будівельного
матеріялу, що підпадають під вплив атмосферних чинників. Один суворий зимовий місяць заподіє більше
шкоди залізниці, ніж цілий рік залізничного руху* (R. Р. Williams. „On the Maintenance of Permanent
Way“. Доповідь в Institute of Civil Engineers, восени 1867 p.).

Нарешті тут, як і всюди в великій промисловості, відіграє ролю моральне зношування; по 10 роках
звичайно можна купити за 30.000 ф. ст. стільки ж вагонів і паротягів, скільки раніш коштувало 40.000
ф. стерл. Отже, на цей матеріял треба рахувати 25% зниження з ринкової ціни, хоча б не відбулося
жодного зниження споживної вартости (Lardner,
„Railway Economy“).

„Трубчасті мости в їхній теперішній формі не поновлюються“. (Тепер бо є кращі форми таких мостів).
„Звичайний ремонт, зняття й заміна поодиноких частин недоцільні“ (W. Р. Adams, „Road and Rails“.
London, 1862). В наслідок поступу промисловости в засобах праці здебільша відбуваються постійні
перевороти. Тому їх замінюється не в їх первісній формі, а в формі, що зазнала перевороту. З одного
боку, та обставина,
що маса основного капіталу вкладається в певній натуральній формі і повинна в ній протриматися
протягом певного пересічного життьового часу, становить причину того, що нові машини та ін.
вводиться лише поступінно, а тому ця обставина перешкоджає швидкому й загальному запровадженню
удосконалених засобів праці. З другого боку, конкуренційна боротьба, особливо підчас рішучих
переворотів, примушує заміняти
старі засоби праці ще до їхньої природної смерти на нові. Катастрофи, кризи — ось що, головним
чином, примушує до такого передчасного поновлення технічних знарядь в широкому суспільному маштабі.

Зношування (залишаючи осторонь моральне) є та частина вартости, що її основний капітал в наслідок
свого зуживання поступінно передає продуктові, в тому пересічному розмірі, що в ньому він втрачає
свою споживну вартість.

Почасти це зуживання таке, що основний капітал має певний пересічний час життя; його цілком
авансується на цей час; а як він мине, то треба його цілком замінити. Для живих засобів праці,
напр., коней, час репродукції визначається самими законами природи. Пересічний час життя їхнього як
засобів праці визначається законами природи. Скоро цей час мине, зужиті екземпляри треба замінювати
на нові. Кінь не може замінюватись частинами, а тільки на нового коня.

Щодо інших елементів основного капіталу, то тут можливе періодичне або частинне поновлення. І тут
треба відрізняти частинне або періодичне заміщення від поступінного поширення підприємства.

Основний капітал складається почасти з однорідних складових частин, що неоднаково довго тривають, а
поновлюються частинами в різні переміжки часу. Прим., рейки біля станції, що їх доводиться
поновлювати частіше, ніж на решті залізничної лінії. Так само й злежні, що їх на бельгійських
залізницях 50-ми роками, за Ларднером, доводилось поновлювати щорічно на 8%, що, отже, цілком
поновлювались протягом 12 ро-
