його продавців і разом з тим продуцентів, не може цього року зменшитись
до 0 так, щоб наступний рік почався нулем, — цілком так само, як
цього не може бути при переході від сьогоднішнього дня до завтрішнього,
А що новоутворення таких товарових запасів мусить відбуватись постійно,
хоч і в змінюваних розмірах, то наші капіталістичні продуценти II мусять
мати запасний грошовий капітал, який давав би їм змогу безперервно
провадити їхній процес продукції, хоч би частина їхнього продуктивного
капіталу деякий час затримувалась у товаровій формі. Адже вони, згідно
з припущенням, сполучають у своїх руках усю торговельну справу
з справою продукції; отже, вони мусять мати в своєму розпорядженні
й такий додатковий грошовий капітал, що, при усамостійненні окремих
функцій процесу репродукції між різними категоріями капіталістів, находиться
в руках торговців.

Проти цього треба сказати ось що: 1) таке утворення запасів і доконечність
такого запасу має силу для всіх капіталістів і І і II. Розглядувані
як звичайні продавці товарів, вони відрізняються один від одного
лише тим, що продають товари різних ґатунків. Запас товарів у II припускає
попередній запас товарів у І. Якщо ми не візьмемо на увагу
цього запасу на одному боці, то ми мусимо зробити те саме й щодо
другого. А коли ми візьмемо їх на увагу й тут і там, то проблема ані
трохи не зміниться. — 2) Коли на боці II цей рік закінчується з товаровим
запасом для наступного року, то й починається він з товаровим запасом
на тому самому боці, переданим від попереднього року.
Отже, при аналізі річної репродукції, зведеної до її найабстрактнішого виразу,
ми в обох випадках мусимо викреслити товаровий запас. Коли ми всю
продукцію цього року залічимо до цього ж року, отже, і те, що він передає
наступному рокові як товаровий запас, але, з другого боку, відлічимо
також з нього товаровий запас, одержаний ним від попереднього року,
то ми справді матимемо сукупний продукт за пересічний рік як предмет
нашої аналізи. — 3) Та проста обставина, що, досліджуючи просту
репродукцію, ми не наражались на труднощі, які тепер треба переборювати,
доводить, що тут ідеться про специфічне явище, викликане тільки
іншим групуванням (щодо репродукції) елементів І, зміненим групуванням,
що без нього взагалі не може відбуватись жодна репродукція в поширеному
маштабі.

III. Схематичне зображення акумуляції

Розгляньмо тепер репродукцію за такою схемою:

Схема а)

I.    4000 с + 1000 v + 1000 m = 6000

II.    1500 с + 376 v + 376 m = 2252  Сума = 8252.

Насамперед бачимо, що сукупна сума річного суспільного продукту
= 8252, менша, ніж у першій схемі, де вона дорівнювала 9000.
Ми могли б так само взяти куди більшу суму, напр., могли б подесятерити її.
