що потрібна певна більша або менша кількість потенціального продуктивного капіталу, тобто певна
кількість призначених для продукції засобів продукції, що їх треба мати у запасі в більших або
менших масах, щоб могли вони помалу входити в процес продукції. При цьому ми зазначили, що в даному
підприємстві або в капіталістичному підприємстві даних розмірів величина цього продукційного запасу
залежить від більшої або меншої важкості його поновлення, від відносної близькости ринків набування
його, розвитку засобів транспорту й комунікації і т. ін. Всі ці обставини впливають на мінімум
капіталу, що мусить бути наявний в формі продуктивного запасу, отже, і на протяг часу, що на нього
треба авансувати капітал, і на розмір капіталу, що його треба авансувати одним заходом. Цей розмір,
що впливає, отже, на оборот, зумовлено більшим або меншим протягом часу, що на нього закріплюється
обіговий капітал в формі продуктивного запасу, як лише потенціальний продуктивний капітал. З другого
боку, оскільки це закріплення залежить від більшої або меншої можливости швидкого заміщення, від
ринкових умов тощо, воно саме і собі зумовлюється часом обігу, обставинами, що належать до сфери
циркуляції. „Далі всі предмети реманенту або прилади, як ручний струмент, решета, кошівниці,
вірьовки, дьоготь, гвіздки тощо, на випадок негайного заміщення мають бути в запасі то більшому, що
менша змога швидко дістати їх поблизу. Нарешті, щорічно протягом зими ввесь реманент треба пильно
переглянути й подбати про те, щоб його відповідно поповнити й полагодити. Оскільки великі мають бути
взагалі запаси щодо реманенту, це залежить, головним чином, від місцевих умов. Там, де близько немає
ремісників і крамниць, треба мати більший запас, ніж там, де вони є на місці або близько. А коли при
інших однакових умовах потрібні запаси закуповується разом чималими масами, то звичайно мають ту
перевагу, що купують дешевше, особливо, коли для цього обирають влучний час; правда, при цьому з
обігового капіталу підприємства воднораз береться чималу суму, а без неї не завжди може обійтись
господарство“ (Kirchhof, p. 301).

Ріжниця між часом продукції і робочим часом може, як ми бачили, поставати в дуже різних випадках.
Обіговий капітал може бути в періоді продукції раніш, ніж він увійде з процес праці у власному
значенні слова (виробництво копил); або він перебуває в періоді продукції після того як проробив
власне процес праці (вино, засівне зерно), або час продукції деколи переривається робочим часом
(хліборобство, лісівництво); чимала частина обігоздатного продукту лишається втіленою в
продукційному процесі, тимчасом як куди менша частина ввіходить у річну циркуляцію (лісівництво й
скотарство); довший або коротший час, що на нього треба витратити обіговий капітал в формі
потенціяльного продуктивного капіталу, отже, більша або менша маса капіталу, що його треба витратити
воднораз — це зумовлюється почасти родом продукційного процесу (хліборобство), а почасти залежить
від близькости ринків і т. ін., коротко кажучи, від обставин, які належать до сфери циркуляції.

Далі (книга III) ми бачимо, до яких безглуздих теорій призвела Мак-
