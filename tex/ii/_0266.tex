\index{ii}{0266}  %% посилання на сторінку оригінального видання
Тому, коли з одного боку, частину реалізованої в грошах додаткової
вартости вилучається з циркуляції і нагромаджується як скарб, то одночасно
другу частину додаткової вартости постійно перетворюється на
продуктивний капітал. За тим винятком, коли між клясою капіталістів
розподіляється додатковий благородний металь, нагромадження в грошовій
формі ніколи не відбувається одночасно в усіх пунктах.

Щодо тієї частини річного продукту, яка репрезентує додаткову
вартість у товаровій формі, то для неї має силу цілком те саме, що й
для другої частини річного продукту. Для її циркуляції потрібна певна
сума грошей. Ця сума грошей так само належить клясі капіталістів, як
і щороку продукована маса товарів, яка репрезентує додаткову вартість.
Її спочатку подає в циркуляцію сама кляса капіталістів. Вона завжди
знову розподіляється між ними через самий процес циркуляції. Як і взагалі
при циркуляції монет, одна частина цієї суми завжди затримується
раз в тому, раз в іншому пункті, тимчасом як друга частина безупинно
циркулює. Справа зовсім не змінюється від того, що частину цього нагромадження
робиться навмисно, щоб утворити грошовий капітал.

Ми тут залишили осторонь ті пригоди в циркуляції, що в наслідок
їх один капіталіст захоплює частину додаткової вартости, ба навіть
капіталу другого капіталіста, і коли, отже, постає однобічна акумуляція
й централізація так грошового, як і продуктивного капіталу. Так, напр.,
частина здобутої додаткової вартости, нагромаджувана як грошовий
капітал капіталістом А, може бути частиною додаткової вартости капіталіста
В, яка не повертається до нього.
