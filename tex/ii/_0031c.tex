\parcont{}  %% абзац починається на попередній сторінці
\index{ii}{0031}  %% посилання на сторінку оригінального видання
вилучає з циркуляції, ніж туди подано підчас його купівлі. Навпаки,
$Г — Т\dots{} П\dots{} Т' — Г'$, зафіксоване як виключна форма, становить основу
розвиненої меркантильної системи, що її неодмінний елемент є не лише
товарова циркуляція, але й товарова продукція.

Ілюзорний характер формули $Г — Т\dots{} П\dots{} Т' — Г'$ і відповідне їй
ілюзорне тлумачення лишаються доти, доки ця форма фіксується як
одноразова, а не як поточна, завжди поновлювана; отже, поки її вважається
не за одну з форм кругобігу, а за виключну його форму. Але
вона сама по собі вказує інші форми.
Поперше, весь цей кругобіг має собі за передумову капіталістичний
характер самого продукційного процесу, а тому як базу — цей
продукційний процес разом із специфічним, зумовленим ним станом
суспільства. $Г — Т = Г — Т\splitfrac{Р}{Зп}$; але $Г — Р$ має собі за передмову найманого робітника, а тому й
засоби продукції як частину продуктивного
капіталу, отже, й те, що процес праці та процес зростання вартосте,
продукційний процес, є вже функція капіталу.

Подруге, коли акт $Г\dots{} Г' п$овторюється, то поворот до грошової
форми є так само минущий, як і грошова форма на першій стадії. $Г — Т$
зникає, щоб дати місце $П. П$остійно повторюване авансування грошей,
так само як і постійне повертання авансованої суми в формі грошей,
сами виступають лише як моменти, що зникають у кругобігу.

Потретє,

$Г — Т\dots{} П\dots{} Т' — Г'. Г — Т\dots{} П\dots{} Т' — Г'. Г — Т\dots{} П$\dots{} і т. ін.

Уже при другому повторенні кругобігу виступає кругобіг $П\dots{} Т' — Г'.
Г — Т\dots{} П$ раніш, ніж закінчиться другий кругобіг Г, і таким чином
усі дальші кругобіги можна розглядати в формі $П\dots{} Т' — Г — Т\dots{} П$, так
що $Г — Т$ як перша фаза першого кругобігу становить лише минуще підготування
до постійно повторюваного кругобігу продуктивного капіталу,
і так воно дійсно буває в тих випадках, коли промисловий капітал уперше
вкладається в формі грошового капіталу.

З другого боку, раніше, ніж закінчиться другий кругобіг П, вже
закінчився перший кругобіг $Т' — Г'. Г — Т\dots{} П\dots{} Т'$ (скорочено $Т' — Т'$),
кругобіг товарового капіталу. Таким чином уже перша форма має в собі
обидві інші, і так зникає грошова форма, оскільки вона є не просто
вираз вартости, але вираз її в еквівалентній формі, в грошах.

Нарешті, коли ми візьмемо індивідуальний капітал, який виступає
вперше і вперше пророблює кругобіг $Г — Т\dots{} П\dots{} Т' — Г'$, то
$Г — Т$ є підготовча фаза, предтеча першого продукційного процесу, що
його пророблює цей індивідуальний капітал. Тому ця фаза $Г — Т$ не є
наперед дана, а, навпаки, вона дана або зумовлюється продукційним
процесом. Але це має силу лише щодо цього індивідуального капіталу.
Загальна форма кругобігу промислового капіталу є кругобіг грошового
\parbreak{}  %% абзац продовжується на наступній сторінці
