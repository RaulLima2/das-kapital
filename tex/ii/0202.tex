ляції, то протягом цього періоду мусять ввійти в роботу й підтримувати
продукцію в русі 300 ф. стерл. звільненого капіталу. Разом з тим наприкінці
року оборот має такий вигляд: 600 ф. стерл. вісім разів зробили
свій кругобіг, що дає 4800 ф. стерл. До цього долучається продукт
останніх 3 тижнів (49—51), який проробив лише третину свого дев’ятитижнгвого
кругобігу, отже, у суму обороту він увіходить лише третиною
своєї величини, 100 ф. стерл. Отже, коли річний продукт, рахуючи рік в
51 тиждень, дорівнює 5100 ф. стерл., то капітал, що обернувся, становитиме
тільки 4800 + 100—4900 ф. стерл.; отже, ввесь авансований капітал
в 900 ф. стерл. обернувся 5 4/9 раза, тобто на незначну величину більше,
ніж у випадку І.

В цьому прикладі припускався такий випадок, коли робочий час =\footnote{
. Робочий період: тижні 6—10.

Перший відділ: тижні 6—9. Функціонує капітал II = 400 ф. стерл.
Наприкінці 9 тижня зворотно припливає капітал 1 = 500 ф. стерл. в грошовій
формі.

Другий відділ: 10 тиждень. З 500 ф. стерл., що повернулися, функціонують
100 ф. стерл. Решта 400 ф. стерл. лишаються вільні для наступного
робочого періоду.
}/3,
а час обігу = 1/3 періоду обороту, отже, робочий час є просте кратне
часу обігу. Треба з’ясувати, чи констатоване вище звільнення капіталу
буде й в інших умовах.

Припустімо, що робочий період дорівнює 5 тижням, час обігу = 4 тижням,
щотижнево авансовуваний капітал = 100 ф. стерл.

І. Період обороту: тижні 1—9.

1. Робочий період: тижні 1—5. Функціонує капітал 1 = 500 ф. стерл.

1. Період циркуляції: тижні 6—9. Наприкінці 9 тижня припливають
назад в грошовій формі 500 ф. стерл.

ІІ. Період обороту: тижні 6—14.

2. Період циркуляції: тижні 11—14. Наприкінці 14 тижня 500 ф.
стерл. зворотно припливають у грошовій формі.

До кінця 14-го тижня (11—14) функціонують раніш звільнені 400 ф.
стерл.; із 500 ф. стерл., що потім повернулись, 100 ф. стерл. поповнюють
недостачу для потреб третього робочого періоду (тижні 11—15),
так що знову звільняються 400 ф. стерл. для четвертого робочого періоду.
Те саме явище повторюється в кожному робочому періоді; на
початку його є 400 ф. стерл., і їх досить на перші 4 тижні. Наприкінці
4-го тижня припливають назад 500 ф. стерл. в грошовій формі, що з
них тільки 100 ф. стерл. потрібні для останнього тижня, а решта 400 ф.
стерл. лишаються вільні для наступного робочого періоду.

Припустімо далі робочий період в 7 тижнів з капіталом І в 700 ф.
стерл.; час обігу в два тижні з капіталом II в 200 ф. стерл.

В такому разі перший період обороту триває протягом тижнів 1—9.
з них перший робочий період протягом тижнів 1—7, з авансуванням
в 700 ф. стерл., і перший період циркуляції протягом тижнів 8—9. Наприкінці
9-го тижня 700 ф. стерл. зворотно припливають у грошовій формі.
