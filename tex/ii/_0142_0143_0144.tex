\parcont{}  %% абзац починається на попередній сторінці
\index{ii}{0142}  %% посилання на сторінку оригінального видання
з формами товарового капіталу і грошового капіталу. Але обидві ці форми,
товаровий капітал і грошовий капітал, є носії вартости так само
основної, як і поточної частини продуктивного капіталу. Обидві вони є
капітал циркуляції протилежно до капіталу продуктивного, а не обіговий
(поточний) капітал протилежно до основного.

Нарешті, цілком неправильне пояснення утворення зиску основним
та обіговим капіталом, а саме, що перший ніби утворює зиск, лишаючись
у процесі продукції, а другий — виходячи з нього та циркулюючи далі, —
призводить до того, що через однаковість форми, що її в обороті
мають змінний капітал і поточна складова частина сталого капіталу, приховується
їхня посутня ріжниця в процесі зростання вартости й
утворення додаткової вартости, отже, уся таємниця капіталістичної продукції
ще більше затемнюється. Через загальне означення: „обіговий капітал“,
знімається (wird aufgehoben) цю посутню ріжницю; це повело до
того, що пізніші економісти пішли ще далі: за посутню й єдино відмінну
вони визнавали протилежність не між змінним і сталим капіталом, а
протилежність між основним і обіговим капіталом.

Визначивши основний і обіговий капітал як два різні способи приміщувати
капітал, що з них кожен, сам по собі розглядуваний, дає зиск,
А. Сміс каже: „Жодний основний капітал не може давати зиску без допомоги
обігового капіталу. Найкорисніші машини та знаряддя праці нічого
не випродукують без обігового капіталу, що дає матеріяли, ними
оброблювані, і дає утримання робітникам, які їх застосовують\footnote*{
„No fixed capital can yield any revenue but by means of a circulating capital.
The most useful machines and instruments of trade will produce nothing wit
hout the circulating capital which affords the materials they are employed upon, and
the maintenance of the workmen who employ them“ (p. 188).
}.

Тут виявляється, що значать попередні вирази: yield a revenue make
a profit\footnote*{
Давати дохід, творити зиск.
} і т. ін., а саме — це значить, що обидві частини капіталу служать
як продуктотворчі.

А. Сміс наводить потім такий приклад: „Частина капіталу фармерового,
вкладена в господарські знаряддя, є основний капітал, а частина,
вкладена в заробітну плату і утримання слуг-робітників, є обіговий капітал“.
(Тут ріжницю між основним й обіговим капіталом правильно зведено
тільки до різної циркуляції, до різного обороту різних складових
частин продуктивного капіталу). „Фармер має зиск від першого, поки має
його в своєму розпорядженні, і від другого, віддаючи його від себе.
Ціна або вартість його робочої худоби є основний капітал“ (тут знову таки
правильно те, що за основу різниці береться вартість, а не речовий
елемент) „так само, як і ціна або вартість господарських знарядь; засоби
утримання її (робочої худоби) є обіговий капітал так само, як і засоби
утримання слуг-робітників. Фармер одержує зиск, лишаючи в своєму
розпорядженні робочу худобу й віддаючи продукти, що є худобі за засіб
існування“. (Фармер лишає корм худобі, не продає його. Він зуживає
його як корм худобі, зуживаючи саму худобу як знаряддя праці.
Ріжниця лише ось у чому: корм для худоби, що зуживається на утримання
\index{ii}{0143}  %% посилання на сторінку оригінального видання
робочої худоби, споживається цілком і мусить постійно замішуватись
новим кормом безпосередньо з продукту хліборобства або з продажу
його; тимчасом як саму худобу заміщується лише в міру того, як
поодинокі екземпляри її по черзі стають непрацездатні). „І ціна й утримання
худоби, купленої не для роботи, а на відгодовування, є обіговий капітал.
Фармер одержує зиск, віддаючи його“. (Кожен товаропродуцент,
а, значить, і капіталістичний товаропродуцент, продає свій продукт,
результат свого процесу продукції, але через це цей продукт
ще не стає ні основною, ні поточною складовою частиною його
продуктивного капіталу. Навпаки, продукт має тепер ту форму, що
в ній він виштовхується з процесу продукції й мусить функціонувати як товаровий
капітал. Відгодовувана худоба функціонує в процесі продукції як
сировинний матеріял, а не як знаряддя праці, не як робоча худоба. Вона,
отже, входить речово в продукт, і вся її вартість переходить цілком
на цей продукт, як і вартість допоміжних матеріялів [її корму]. Саме тому
вона й є поточна частина продуктивного капіталу, а зовсім не тому,
що проданий продукт — відгодована худоба — має тут ту саму натуральну
форму, що й сировинний матеріял, тобто ще не відгодована худоба. Це —
цілком випадкова обставина. Але разом з тим Сміс міг би побачити з
цього прикладу, що не речова форма елемента продукції, а лише його
функція в межах продукційного процесу надають вартості, що міститься
в ньому, характеру основної або поточної частини). „Вся вартість засівного
зерна є теж основний капітал. Хоч воно завжди переходить з землі
в комори й назад, але воно ніколи не змінює власника, а тому в
дійсності й не циркулює. Фармер одержує свій зиск не тому, що продає
його, а тому, що кількість його більшає\footnote*{
„That part of the capital of the farmer which is employed in the implements
of agriculture is a fixed, that which is employed in the wages and maintenance of
his labouring servants is a circulating capital. He makes a profit of the one by
keeping it in his own possession, and of the other by parting with it. The price or
value of his labouring cattle is a fixed capital, in the same manner as that of the
instruments of husbandry; their maintenance is a circulating capital, in the same way
as that of the labouring servants. The farmer makes his profit by keeping the labouring
cattle, and by parting with their maintenance. Both the price and the maintenance
of the cattle which are bought in and fattened, not for labour but for sale,
are a circulating capital. The farmer makes his profit by parting with them. The whole
value of the seed, too is a fixed capital. Though it goes backwards and
forwards between the ground and the granary, it never changes masters, and therefore
it does not properly circulate. The farmer makes his profit not by its sale, but
by its increase“.
}.

Тут особливо яскраво виявляється вся безглуздість Смісового відрізнення.
За його теорією, засівне зерно було б основним капіталом, якби не
відбулося change of masters\footnote*{
Зміна власника. \emph{Ред.}
}, тобто, коли засівне зерно безпосередньо
заміщується з річного продукту, береться з нього. Навпаки,
воно було б обіговим капіталом, коли б продавалось увесь продукт і на
частину вартости його купувалось засівне зерно в другого власника. В
одному разі відбувається change of masters, в другому ні. Сміс тут знову
\index{ii}{0144}  %% посилання на сторінку оригінального видання
таки сплутує поточний капітал і товаровий капітал. Продукт є речовий
носій товарового капіталу. Але, звичайно, лише та частина продукту,
яка справді входить в циркуляцію і не входить знову безпосередньо в
той самий процес продукції, відки вона вийшла, як продукт.

Чи береться зерно, як частина, безпосередньо з продукту, чи продається
ввесь продукт і частину його вартости за допомогою купівлі перетворюється
на чуже зерно, і в тому і в другому випадку маємо лише заміщення
вартости, і цим заміщенням не утворюється жодного зиску. В
першому випадку зерно разом з рештою продукту входить як товар в
циркуляцію, а в другому випадку воно фігурує лише в бухгальтерії як
складова частина вартости авансованого капіталу. Але в обох випадках
воно лишається поточною складовою частиною продуктивного капіталу.
Його зуживається цілком, щоб виготовити продукт, і воно мусить цілком
заміститися з нього, щоб уможливити репродукцію.

„Сировинні й допоміжні матеріяли втрачають ту самостійну форму, в
якій вони ввійшли в процес праці як споживні вартості. Інша справа
з власне засобами праці. Інструмент, машина, фабричний будинок, посуд
і т. ін. служать у процесі праці лише доти, доки зберігають вони
свою первісну форму, доки й завтра можуть вони входити в процес праці
у тій самій формі, що й учора. І як за свого життя, тобто протягом
процесу праці, вони зберігають проти продукту свою самостійну форму,
так само зберігають її вони й після своєї смерти. Трупи машин, майстерень,
фабричних будівель і далі все ще існують самостійно, окремо
від продуктів, творенню яких вони допомагали“. (Капітал, кн. І, розд. VI).

Ці різні способи застосування засобів продукції для створення продукту,
— при чому одні засоби продукції зберігають свою самостійну форму
проти продукту, а інші змінюють або цілком втрачають її, — цю ріжницю,
властиву процесові праці, як такому, отже, ріжницю, що так само
властива й такому процесові праці, що має на меті задовольнити лише
власні потреби, прим., патріярхальної сім’ї, без якогобудь обміну, без
товарової продукції, — А. Сміс освітлює неправильно, бо: 1) він притягує
зовсім неналежне сюди визначення зиску: що одні засоби продукції дають
своєму власникові зиск, зберігаючи свою форму, а інші дають зиск,
втрачаючи її; 2) зміни частини елементів продукції в процесі праці він
сплутує з тією зміною форми, яка властива обмінові продуктів, товаровій
циркуляції (купівлі та продажеві), і яка разом з тим включає зміну
власности на товари, що циркулюють.

Оборот має собі за передумову репродукцію, упосереднювану циркуляцією,
тобто продажем продукту, перетворенням його на гроші і зворотним
перетворенням з грошей на елементи його продукції. Але оскільки
капіталістичному підприємцеві частина його власного продукту безпосередньо
сама знову служить як засіб продукції, продуцент виступає як
продавець того самого продукту самому собі і саме так фігурує ця операція
в його бухгальтерії. Отже, ця частина репродукції не упосереднюється
циркуляцією, а відбувається безпосередньо. Але та частина продукту,
що таким чином знову служить як засіб продукції, заміщує
\parbreak{}  %% абзац продовжується на наступній сторінці
