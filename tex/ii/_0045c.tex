\parcont{}  %% абзац починається на попередній сторінці
\index{ii}{0045}  %% посилання на сторінку оригінального видання
еквівалент не входить у процес репродукції даного кругобігу, так само
як не входять у нього гроші, вкладені в процентодайні папери тощо, хоч
може він увіходить у кругобіг іншого індивідуального промислового
капіталу.

Весь характер капіталістичної продукції визначається зростанням
авансованої капітальної вартости, отже, в першу чергу, продукцією якомога
більшої додаткової вартости; а, подруге (див. кн. І, розділ XXII),
— продукцією капіталу, тобто перетворенням додаткової вартости на
капітал. Акумуляція, або продукція в поширеному маштабі, що, як
засіб до дедалі більшого поширення продукції додаткової вартости, а тому
й до збагачення капіталіста, являє особисту мету його, і ввіходить у
загальну тенденцію капіталістичної продукції, — стає далі, як це
показано в першій книзі, в наслідок свого розвитку доконечністю
для кожного поодинокого капіталіста. Постійне збільшення його капіталу
стає умовою зберігання його. Однак ми не будемо повертатися до раніш
уже з’ясованого.

Спочатку ми розглядали просту репродукцію і при цьому припускали,
що всю додаткову вартість витрачається як дохід. У дійсності в нормальних
умовах одна частина додаткової вартости завжди мусить витрачатися як
дохід, а друга капіталізуватися, і при цьому цілком байдуже, що додаткову
вартість, спродуковану протягом певного періоду, то споживається
цілком, то капіталізується цілком. Беручи рух у його пересічному — а
загальна формула може зображати лише пересічний рух — відбувається
і те і друге. Однак, щоб не ускладняти формули, краще припустити,
що акумулюється всю додаткову вартість. Формула\[
П\dots{} Т' — Г' — Т'\splitfrac{Р}{3п} \dots{} П'
\]
виражає продуктивний капітал, що репродукується в поширеному маштабі
і з більшою вартістю, і що як вирослий продуктивний капітал починає
свій другий кругобіг, або, що сходить на те саме, відновлює свій
перший кругобіг. Скоро починається цей другий кругобіг, ми маємо
знову $П$ як вихідний пункт; тільки це $П$ є продуктивний капітал
більших розмірів, ніж було перше $П$. Так само, коли в формулі $Г\dots{} Г'$
другий кругобіг починається з $Г'$, то це $Г'$ функціонує як $Г$, як авансований
грошовий капітал певної величини; це — грошовий капітал, більший,
ніж той, що ним почався перший кругобіг; але, скоро він виступає
в функції авансованого грошового капіталу, зникає будь-який натяк на
те, що він виріс у наслідок капіталізації додаткової вартости. У формі
грошового капіталу, що в ній він починає кругобіг, зникають сліди цього
походження. Те саме й з $П'$, коли воно є вихідний пункт нового
кругобігу.

Коли ми порівняємо $П\dots{} П'$ з $Г\dots{} Г'$, або з першим кругобігом, то
побачимо, що вони мають неоднакове значіння. $Г\dots{} Г'$, взяте само по
\parbreak{}  %% абзац продовжується на наступній сторінці
