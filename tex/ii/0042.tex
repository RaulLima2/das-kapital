ального капіталу, хоч і походить з неї. Але постійна наявність кляси
робітників потрібна для кляси капіталістів, а тому й споживання робітників,
яке упосереднюється через Г — Т.

Акт Т' — Г' щодо продовження кругобігу капітальної вартости, а
також щодо споживання додаткової вартости від капіталіста, припускає
лише те, що Т' перетворено на гроші, продано. Його купують звичайно
лише тому, що предмет являє споживну вартість, а, значить, придатний
до якогобудь споживання, хоч продуктивного, хоч особистого. А коли
Т' і далі циркулює, прим., у руках купця, що купив пряжу, то
це насамперед ніяк не зачіпає продовження кругобігу того індивідуального
капіталу, що спродукував пряжу й продав її купцеві. Цілий процес
триває далі, а разом з ним триває й зумовлене ним особисте споживання
капіталіста й робітника. Ця обставина має велике значіння при вивченні
криз.

Скоро Т' продано, перетворено на гроші, воно може зворотно перетворитись
на реальні чинники процесу праці, а, значить, і процесу репродукції.
Чи купив Т' остаточний споживач, чи купець, який хоче знову продати
його, це безпосередньо не змінює справи. Об’єм товарових мас, утворюваних
капіталістичною продукцією, визначається маштабом цієї продукції
та потребою постійно поширювати її, але зовсім не наперед визначеним
розміром попиту й подання, не розміром потреб, що їх треба задовольнити.
За безпосереднього покупця масової продукції може бути, крім
інших промислових капіталістів, лише гуртовий покупець. У певних
межах процес репродукції може відбуватись у попередньому або й
поширеному розмірі, хоч виштовхнуті з нього товари в дійсності не
ввійшли в особисте або продуктивне споживання. Споживання товарів не
включено в той кругобіг капіталу, що з нього вони постали. Напр.,
скоро пряжу продано, кругобіг капітальної вартости, втіленої в пряжі,
може початися знову, незалежно від того, що сталося з проданою пряжею.
Доки продукт продається, то з погляду капіталістичного продуцента
все йде своїм нормальним порядком. Кругобіг капітальної вартости, що
її репрезентує продукт, не переривається. А коли цей процес поширюється
— що включає поширене продуктивне споживання засобів продукції,
— то ця репродукція капіталу може супроводитися поширеним
особистим споживанням (а, значить, попитом) робітників, бо продуктивне
споживання підготовлює й упосереднює цей процес. Таким
чином продукція додаткової вартости, а разом з нею і особисте
споживання капіталіста, може зростати, ввесь процес репродукції може
перебувати в стані найбільшого розцвіту, і все ж більшість товарів
може переходити в сферу споживання лише позірно, а в
дійсності лежати непроданою у перекупників, отже, фактично перебувати
ще на ринку. Але потік товарів котиться один по одному, і нарешті
виявляється, що попередній потік увібрано в споживання лише позірно.
Товарові капітали змагаються один з одним за своє місце на ринку. Ті,
що прийшли пізніше, продають нижчою ціною, аби тільки продати.
Попередні потоки ще не збуто, а вже надходять строки виплат за них.
