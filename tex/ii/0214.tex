постійно витрачається на матеріяли для продукції і 1/5 = 120 ф. стерл.
на заробітну плату. Отже, щотижня 80 ф. стерл. на матеріяли для продукції,
20 ф. стерл. на заробітну плату. Отже, капітал II = 300 ф. стерл.
так само мусить поділитись на 4/5 = 240 ф. стерл. для продукційних матеріялів
і на 1/5 = 60 ф. стерл. для заробітної плати. Капітал, витрачуваний
на заробітну плату, завжди мусить авансуватись у грошовій формі.
Скоро товаровий продукт вартістю в 600 ф. стерл. зворотно перетворюється
на грошову форму, скоро його продано, — 480 ф. стерл. з цієї
суми можна перетворити на матеріяли для продукції (на продуктивний
запас), але 120 ф. стерл. зберігають свою грошову форму, щоб служити
для виплати заробітної плати протягом 6 тижнів. Ці 120 ф. стерл. являють
той мінімум приплилого назад капіталу в 600 ф. стерл., який завжди
мусить поповнюватись і заміщуватись у формі грошового капіталу, а
тому й мусить він завжди бути наявний як діюща в грошовій формі частина
авансованого капіталу.

Коли тепер з тих 300 ф. стерл., що періодично звільняються на З
тижні й так само розпадаються на 240 ф. стерл. для продуктивного запасу
й на 60 ф. стерл. для заробітної плати, в наслідок скорочення часу
обігу виділюється 100 ф. стерл. у формі грошового капіталу, зовсім викидаємся
з механізму обороту, то постає питання: відки береться гроші
для цих 100 ф. стерл. грошового капіталу? Лише на п’яту частину вони
складаються з грошового капіталу, що періодично звільняється в межах
оборотів. Але 4/5 = 80 ф. стерл. уже заміщено додатковим продуктивним
запасом тієї самої вартости. Яким же чином цей додатковий продуктивний
запас перетворюється на гроші, і відки береться гроші на це перетворення?
Якщо постало скорочення часу обігу, то з вищезгаданих 600 ф. стерл.
на продуктивний запас замість 480 ф. стерл. перетворюється лише 400 ф.
стерл. Решту 80 ф. стерл. зберігається в їхній грошовій формі, і
разом з вищезгаданими 20 ф. стерл., призначеними для заробітної плати,
вони становлять цей виділений капітал в 100 ф. стерл. Хоч ці 100 ф.
стерл. приходять з циркуляції в наслідок продажу товарового капіталу в
600 ф. стерл. і тепер їх вилучається з циркуляції, бо їх не витрачається
знову на заробітну плату й елементи продукції, однак, не треба забувати,
що в грошовій формі вони знову є в тій самій формі, що в ній їх
первісно кинуто в циркуляцію. Спочатку на продукційний запас і на заробітну
плату витрачалось 900 ф. стерл. грішми. Щоб подати той
самий процес продукції, треба тепер вже лише 800 ф. стерл. Виділені в
наслідок цього в грошовій формі 100 ф. стерл. становлять тепер новий
грошовий капітал, що шукає приміщення, нову складову частину грошового
ринку. Щоправда, вони й раніш періодично перебували у формі
звільненого грошового капіталу й додаткового продуктивного капіталу,
але цей лятентний стан сам був умовою провадження процесу продукції,
бо він був умовою його безперервности. Тепер їх уже не треба для
цього, а тому вони становлять новий грошовий капітал і одну з складових
частин грошового ринку, хоч вони зовсім не є ні додатковий еле-
