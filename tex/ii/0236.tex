грошова форма новоутвореної ним самим вартости (= ціні робочої
сили плюс додаткова вартість) першого періоду обороту, що нею
оплачується його працю протягом другого періоду обороту. У В справа
інша. Хоч щодо робітника гроші й тут є засіб виплати за вже виконану
ним працю, але цю виконану вже працю оплачується не новоутвореною
нею вартістю, перетвореною на золото (не грошовою формою вартости,
спродукованої самою цією працею). Такий спосіб оплати може постали,
починаючи лише з другого року, коли робітника В оплачується спродукованою
ним в минулому році вартістю, перетвореною на золото.

Що коротший період обороту капіталу, — що коротші, отже, переміжки,
що в них протягом року поновлюються терміни його репродукції, —
то швидше змінна частина капіталу, первісно авансована в грошовій формі
капіталістом, перетворюється на грошову форму тієї новоутвореної
вартости (яка, крім того, містить у собі й додаткову вартість), що її
утворив робітник на заміщення цього змінного капіталу; то коротший,
отже, час, на який капіталіст мусить авансувати гроші з свого власного
фонду, то менший, порівняно з даними розмірами маштабу продукції, той
капітал, що його він взагалі авансує; і то більша порівняно та маса додаткової
вартости, що її він за даної норми додаткової вартости видушує
протягом року, бо він то частіше може знову й знову купувати
робітника на грошову форму вартости продукту цього ж таки робітника
й пускати в рух його працю.

За даних розмірів продукції абсолютна величина авансованого змінного
грошового капіталу (як і взагалі обігового капіталу) меншає, а річна
норма додаткової вартости більшає пропорційно до скорочення періоду
обороту. За даної величини авансованого капіталу розміри продукції,
а тому за даної норми додаткової вартости й абсолютна маса додаткової
вартости, утвореної протягом одного періоду обороту, зростають разом
з підвищенням річної норми додаткової вартости, що його зумовлює
скорочення періоду репродукції. Взагалі, з нашого досліду виявилось, що
відповідно до різного протягу періоду обороту доводиться авансовувати грошовий
капітал дуже різної величини для того, щоб при тому самому
ступені експлуатації праці пускати в рух однакову масу продуктивного
обігового капіталу та однакову масу праці.

Подруге — і це має зв’язок з першою ріжницею — робітник капіталіста
В, як і А, платить за куповані ним засоби існування змінним
капіталом, що перетворився в його руках на засіб циркуляції. Він не
тільки, напр., бере з ринку пшеницю, а й заміщує її грошовим еквівалентом.
А що гроші, що ними робітник В оплачує засоби свого існування,
вилучаючи їх з ринку, не є грошова форма новоутвореної вартости,
подаваної ним на ринок протягом року, як у робітника А, то хоч
він і дає гроші продавцеві його засобів існування, але не дає він жодного
товару, — ні засобів продукції, ні засобів існування, — що їх той
міг би купити за вторговані гроші, як це, навпаки, маємо в випадку А.
Тому з ринку береться робочу силу, засоби існування для цієї робочої
сили, основний капітал у формі засобів праці й продукційних матеріялів,
