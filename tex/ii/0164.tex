дження додаткової вартости, що є в продукті, лишається цілком поза
межами поля зору.

Далі, тут завершується властивий буржуазній політичній економії фетишизм,
що перетворює суспільний, економічний характер, накладуваний
на речі суспільним процесом продукції, на природний, з самої речової
природи цих речей посталий характер. Напр., „засоби праці є основний
капітал“ — схоластичне визначення, що призводить до суперечностей і плутанини.
Цілком так само, як при вивченні процесу праці („Капітал“, книга
І, розділ V) показано, що від тієї ролі, яку в кожному окремому
випадку відіграють речові складові частини у певному процесі праці,
від їхньої функції, цілком залежить те, чи будуть вони виступати як
засіб праці, чи як матеріял праці, або як продукт; цілком так само засоби
праці тільки тоді є основний капітал, коли процес продукції є взагалі
капіталістичний продукційний процес і, значить, засоби продукції
взагалі є капітал, коли вони мають економічну визначеність, суспільний
характер капіталу; і, подруге, вони є основний капітал лише тоді, коли
вони свою вартість переносять на продукт певним способом. Коли цього
немає, вони лишаються засобами праці, не являючи основного капіталу.
Так само допоміжні матеріяли, напр., добриво, коли вони передають
свою вартість тим самим особливим способом, що й більша частина засобів
праці, стають основним капіталом, хоч вони й не є засоби праці.
Тут ідеться не про визначення, що під нього можна підводити речі. Тут
ідеться про певні функції, що їх виражається в певних категоріях.

Коли засобам існування самим по собі при всяких обставинах приписується
властивість бути капіталом, витраченим на заробітну плату, то
ця властивість „підтримувати працю“, to support labour (Рікардо, ст. 25),
стає також характеристичною властивістю цього „обігового“ капіталу.
Отже, виходить, що коли б засоби існування не були „капіталом“, то
вони не підтримували б робочої сили; тимчасом характер капіталу надає
їм саме властивости підтримувати капітал за допомогою чужої праці.

Якщо засоби існування сами по собі є обіговий капітал — після того
як цей останній перетворився на заробітну плату, — то відси випливає
далі, що величина заробітної плати залежить від відношення між числом
робітників і даною масою обігового капіталу — улюблена засада економістів;
тимчасом як у дійсності маса засобів існування, що її робітник
бере з ринку, і маса засобів існування, що її має капіталіст для власного
споживання, залежить від відношення між додатковою вартістю й ціною
праці.

Рікардо, як і Бартон 29), сплутують усюди відношення між змінним і сталим
капіталом із відношенням між обіговим і основним капіталом. Ми побачимо
далі, якої хибности набирає в наслідок цього дослід над нормою зиску.

Далі, Рікардо ототожнює ріжницю між основним і обіговим капіталом із
ріжницями, що постають в процесі обороту підо впливом інших причин. Він

29) „Observations on the Circumstances which influences the Condition of the
Labouring Classes of Society. London, 1817“. Відповідне, сюди належне, місце
подано в 1 кн. „Капіталу“, розд. XXIII, 3, примітка 79.
