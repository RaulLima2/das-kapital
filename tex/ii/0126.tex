дукції: те, що для залізничної колії повинно в певний час робити одним заходом, а саме, коли лінію
цілком переклалається наново, — це в рухомій частині робиться поступінно з року на рік. Її існування
вічне, вона завжди омолоджується“ (Lardner, p. 116).

Цей процес, як його описує тут Ларднер щодо залізниць, не підходить до поодинокої фабрики, але
змальовує нам картину постійної, частинної, переплетеної з ремонтом репродукції основного капіталу в
межах якоїсь цілої галузі промисловости, або взагалі в межах сукупної продукції, розглядуваної у
суспільному масштабі.

Наводимо тут ще одну вказівку, що пояснює, в яких широких розмірах спритні управління можуть
орудувати поняттями ремонт і заміщення, щоб здобувані дивіденди. Згідно з вище цитованою доповіддю
Р. Б. Вільямса, різні англійські залізничні товариства пересічно за ряд років списували з рахунку
доходів такі суми на ремонт та витрати на підтримання залізничної колії та будівель (на англійську
милю довжини колії щороку):

London and North Western........370 ф. стерл.
Midland........225 „„
London and South Western........257 „„
Great Northern........360 „„
Lancashire and Yorkshire........377 „„
South Eastern........263 „„
Brignton........266 „
Manchester and Sheffield........200 „„

Ці ріжниці лише дуже мало залежать від неоднакового розміру дійсно зроблених витрат: вони походять
майже виключно з неоднаковости в способах обчислення, з того, чи залічується статті видатків на
рахунок капіталу, чи на рахунок доходів. Вільямс прямо каже: «Меншу цифру витрат вибирається тому,
що це потрібно для доброго дивіденда, а більшуцифру подається тому, що є досить високий дохід, який
може витримати це“.

Іноді зношування, отже, і заміщення його стає величиною практично зникомою, так що на увагу береться
лише витрати на ремонт. Те, що Ларднер каже далі про works of art * на залізницях, має силу взагалі
для всіх таких довготривалих споруд, як канали, доки, залізні та кам’яні мости і т. ін. —
„Зношування, що постає в наслідок повільного впливу часу на солідніших спорудах, діє майже непомітно
протягом невеликих переміжків часу; а коли минає більше часу, прим., століття, то воно мусить
призвести до поновлювання, повного або частинного, навіть для найсолідніших споруд. Це непомітне
зношування, порівняно з помітнішими зношуваннями інших частин залізниці, можна прирівняти до вікових
і періодичних відхилів у русі світових тіл. Вплив часу на масивніші споруди залізниці —

* Works of art — будівельні споруди. Ред.
