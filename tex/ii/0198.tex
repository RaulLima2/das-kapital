роту. Точний зміст вислову, що авансований капітал в 450 ф. стерл. зробив
5 2/3 обороту лише той, що він зробив п’ять повних оборотів і тільки
2/3 шостого. Навпаки, в вислові: капітал, що обернувся, дорівнює
авансованому капіталові, взятому 5 2/3 раза, тобто в наведеному вище прикладі
дорівнює 450 ф. стерл. × 5 2/3 = 2550 ф. стерл., — правильне те, що
коли б цей капітал в 450 ф. стерл. не доповнювався б другим капіталом
в 450 ф. стерл., то в дійсності одна частина його мусила б бути в процесі
продукції, а друга — в процесі циркуляції. Коли ми хочемо час обороту
виразити в масі капіталу, що обернувся, то можемо виразити його
виключно в масі наявної вартости (в дійсності — в масі готового продукту).
Та обставина, що авансований капітал не перебуває в такому
стані, в якому він знову може почати процес продукції, виражається в
тому, що лише частина його перебуває в стані, придатному для продукції,
або в тому, що капітал, коли він має бути в стані безперервної продукції,
треба поділити на частини, що з них одна постійно була б у періоді
продукції, а друга — постійно в періоді циркуляції, залежно від взаємного
відношення цих періодів. Це той самий закон, що згідно з ним
масу постійно діющого продуктивного капіталу визначається відношенням
часу обігу до часу обороту.

Наприкінці 51-го тижня — а ми його беремо тут як кінець року
150 ф. стерл. з капіталу II авансовано на продукцію недоробленого ще
продукту. Ще деяка частина перебуває в формі поточного сталого капіталу
— сировинного матеріялу тощо — тобто в такій формі, що в ній
вона може функціонувати в процесі продукції як продуктивний капітал.
Але третя частина перебуває в грошовій формі, а саме, принаймні, сума
заробітної плати за решту робочого періоду (3 тижні), що оплачується
лише наприкінці кожного тижня. Хоч на початку нового року, отже, нового
циклу оборотів, ця частина капіталу перебуває не в формі продуктивного
капіталу, а в формі грошового капіталу, що в ній вона не може
ввійти в процес продукції, все ж, коли починається новий оборот, поточний,
змінний капітал, тобто жива робоча сила, уже діє в процесі продукції.
Це явище випливає з того, що хоч робочу силу купується на початку
робочого періоду, напр., щотижня, і так само зуживається, але
оплачується її лише наприкінці тижня. Гроші функціонують тут як засіб
виплати. Тому вони, з одного боку, як гроші перебувають ще в руках
капіталіста, тимчасом як, з другого боку, робоча сила, товар, що на
нього їх перетворюється, вже діє в продукційному процесі; отже, та сама
капітальна вартість з’являється тут двічі.

Коли ми розглядаємо лише робочі періоди, то:

Капітал І випродукував 450 × 6 = 2700 ф. стерл.

„II „ 450 × 5 1/3 = 2400 ф. стерл.

Отже, разом................ 900 × 5 2/3 = 5100 ф. стерл.

Отже, увесь авансований капітал в 900 ф. стерл. за рік функціонував
5 2/3 раза як продуктивний капітал. Для продукції додаткової вартости
байдуже, чи функціонують навперемінку 450 ф. стерл. ввесь час у про-
