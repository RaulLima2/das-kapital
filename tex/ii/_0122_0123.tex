\parcont{}  %% абзац починається на попередній сторінці
\index{ii}{0122}  %% посилання на сторінку оригінального видання
витрачати на медичну допомогу, ніж людині в розквіті сил. Отже, незважаючи на випадковий характер
ремонтних робіт, вони розподіляються нерівномірно на різні життьові періоди основного капіталу.

З цього, а також взагалі випадкового характеру ремонтних робіт, що їх потребує машина, випливає
таке:

З одного боку, справжня витрата на робочу силу й засоби праці для ремонтних робіт є випадкові, як
випадкові й самі обставини, що роблять потрібними ці ремонтні роботи; число потрібних полагоджень
розподіляється нерівномірно на різні життьові періоди основного капіталу. З другого боку, коли
обчислюють пересічний життьовий період основного капіталу, то припускається, що його постійно
підтримується в діяльному
стані, — почасти чищенням (сюди належить і тримання в чистоті приміщень), почасти ремонтом, що його
робиться в разі потреби. Перенесення вартости в наслідок зношування основного капіталу розраховано
на його пересічний життьовий період, але й самий цей пересічний період життя розрахований на те, що
весь час авансуватиметься додатковий капітал, потрібний на його підтримання в доброму стані.

З другого боку, так само зрозуміло, що вартість, долучувана в наслідок цієї додаткової витрати
капіталу й праці, не може входити в ціну товарів одночасно з цими витратами. Коли, напр., у
прядільника на цьому тижні поламалось колесо або розірвався пас, то він не може цього тижня
продавати свою пряжу дорожче, ніж продавав минулого. Загальні витрати прядіння ніяк не змінились в
наслідок такого нещасного випадку на одній фабриці. Тут, як і взагалі при визначенні вартости,
вирішувальне значення має пересічна величина. Досвід виявляє середнє число таких нещасних випадків і
пересічний розмір робіт на підтримання і ремонт, потрібних протягом пересічного життьового періоду
основного капіталу, вкладеного в певну галузь підприємства. Ці пересічні витрати розподіляються на
пересічний життьовий періоді відповідними аліквотними частинами їх долучається до ціни продукту, а
тому й покривається через його продаж.

Додатковий капітал, таким чином заміщуваний, належить до поточного капіталу, хоч спосіб витрат
нерегулярний. А що дуже важливо виправляти кожне ушкодження машини негайно, то при кожній великій
фабриці є, крім власне фабричних робітників, відповідний персонал інженерів, теслярів, механіків,
слюсарів і т. ін. їхня заробітна плата становить частину змінного капіталу, і вартість їхньої праці
розподіляється на
продукт. З другого боку, потрібні видатки на засоби продукції визначаються за пересічним розрахунком
і відповідно до нього ввесь час входять у продукт, як частина його вартости, хоч фактично їх
авансується нереґулярно, а, значить, нереґулярно входять вони в продукт, зглядно в основний капітал.
Цей капітал, витрачуваний власне на ремонт, з певного погляду є капітал особливого роду, що його не
можна залічити ні до поточного, ні до основного капіталу, але більше до першого, бо він належить до
категорії поточних витрат.

Система бухгальтерії, звичайно, нічого не змінює в дійсному зв’язку речей, що про них ведеться ці
книги. Але важно відзначити, що в багатьох
\index{ii}{0123}  %% посилання на сторінку оригінального видання
галузях підприємств є звичка разом обчислювати витрати на ремонт з справжнім зношуванням
основного капіталу в такий спосіб. Хай авансовий основний капітал буде 10.000 ф. стерл., а його
життьовий період 15 років; річне зношування дорівнює тоді 666 2/3 ф. стерл. Але в дійсності
зношування обчислюють лише на 10 років, тобто до ціни вироблюваних товарів щороку додають на
зношування основного капіталу 1.000 ф. стерл. замість 666 2/3 ф. стерл.; інакше кажучи, на ремонтні
роботи тощо утворюється резервний фонд в ЗЗЗ 1/3 ф. стерл. (Числа 10 і 15 беремо лише, як приклад).
Отже, таку суму пересічно витрачається на ремонт для того, щоб основний капітал існував 15 років.
Такий спосіб обчислення, звичайно, не заважає, щоб основний капітал і витрачуваний на ремонт
додатковий капітал становили різні категорії. На ґрунті цього способу обчислення, напр.,
припускається, що мінімальна додача витрат на підтримання й поновлення пароплавів становить річно
15\%, отже, час репродукції дорівнює 6 2/3 рокам. 60-х років управління англійської Peninsular and
Oriental С° обчислювало щорічні витрати на це в 16\%, що відповідає часові репродукції в 6 1/4 років.
На залізницях середній час життя паровоза є 10 років, але, беручи на увагу ремонт, зношування беруть
в 12 1/2\%, і тому час життя сходить до 8 років. Для пасажирських і товарових вагонів зношування
обчислюється в 9\%, отже, час життя беруть в 11 1/9 років.

Законодавство щодо контрактів про винаймання будинків та інших речей, які є для їхніх власників
основний капітал і здаються ними як такий, визнає повсюди ріжницю між нормальним зношуванням, що
його спричиняють час, вплив природних сил і саме нормальне користання, і випадковими полагодженнями,
які час від часу потрібні в нормальному протязі життя будинку й при нормальному користанні, на
підтримання будинку в нормальному стані. Як загальне правило, ремонт першої відміни покладається на
власників, а другої — на наймачів. Ремонтні роботи далі поділяється на звичайні й капітальні.
Останні є часткове поновлення основного капіталу в його натуральній формі й теж покладається їх на
власників, якщо тільки контракт не вимагає протилежного. Напр., згідно з англійськими законами:

„Наймач лише повинен тримати будівлі рік-у-рік в такому стані, щоб вони не пропускали вітру й води,
оскільки це можливо без капітального ремонту; і взагалі він повинен дбати лише про такі
полагодження, що їх можна назвати звичайними. Але навіть і тут доводиться брати на увагу вік і
загальний стан відповідних частин будівлі в той час, коли наймач їх прийняв, бо він не повинен
старий і зношений матеріял замінювати на новий, ні відшкодовувати зневартнення, щонеминуче постає
під впливом часу і нормального користання“ (Holdsworth, „Law of Landlord and Tenant“, p. 90, 91).

Цілком відмінне так від покриття зношування, як і від робіт для зберігання й ремонту, є страхування,
яке поширюється на руйнацію в наслідок незвичайних природних явищ, пожеж, поводей тощо. Воно
\parbreak{}  %% абзац продовжується на наступній сторінці
