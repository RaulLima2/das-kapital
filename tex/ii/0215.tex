мент до вже наявного суспільного грошового запасу (бо вони вже були
на початку заснування підприємства, й воно пустило їх в циркуляцію),
ні новоакумульований скарб.

Тепер ці 100 ф. стерл. дійсно вилучається з циркуляції, оскільки вони
є частина авансованого грошового капіталу, що її тепер уже не застосовується
в тому самому підприємстві. Але таке вилучення можливе
тільки тому, що перетворення товарового капіталу на гроші, а цих грошей
— на продуктивний капітал, Т' — Г — Т, прискорюється на один тиждень,
отже, прискорюється й обіг діющих у цьому процесі грошей. Їх вилучено
з циркуляції, бо вони більше непотрібні для обороту капіталу X.

Тут припускається, що авансований капітал належить тому, хто його
застосовує. Коли б він був позичений, справа через це ані трохи не змінилась
би. Із скороченням часу обігу, підприємцеві треба було б замість
900 ф. стерл. лише 800 ф. стерл. позиченого капіталу. 100 ф. стерл.,
повернені позикодавцеві, становлять, як і раніше, новий грошовий капітал
в 100 ф. стерл. тільки вже не в руках X, а в руках Y. Далі, коли
капіталіст X одержував наборг свої продукційні матеріяли вартістю в
480 ф. стерл., так що сам він мав авансувати грішми тільки 120 ф. стерл.
на заробітну плату, то тепер він має брати наборг продукційних матеріялів
на 80 ф. стерл. менше, отже, ці 80 ф. стерл. становлять надлишковий
товаровий капітал для капіталіста, що дає наборг, тимчасом як
для капіталіста X виділилось би 20 ф. стерл. грішми.

Додатковий продукційний запас зменшився тепер на 1/3. Являючи 4/5
від 300 ф. стерл., він дорівнював додатковому капіталові II—240 ф.
стерл., тепер він дорівнює лише 160 ф. стерл., тобто являє додатковий
запас на 2 тижні замість 3. Тепер він відновлюється що 2 тижні замість
що 3, але також тільки на два тижні замість 3. Закупи, напр., на ринку
бавовни повторюється, таким чином, частіше й меншими пайками. З ринку
береться ту саму кількість бавовни, бо маса продукту лишається та
сама. Але закупи розподіляються інакше в часі й розтягується їх на довший
час. Припустімо, напр., що йдеться про 3 місяці й про 2 місяці; хай
річне споживання бавовни буде 1200 пак. В першому випадку продаватиметься:

1 січня 300 пак, на складі лишається 900 пак
1 квітня 300 „„    600 „

1 липня 300 „„    300 „

1 жовтня 300 „„    0 „

Навпаки, в другому випадку:

1 січня продається 200, на    складі    1000    пак

1 березня „200 „„    800 „

1 травня „200 „„    600 „

1 липня „200 „„    400 „

1 вересня „200 „„    200 „

1 листопада „200 „„    0 „

Отже, витрачені на бавовну гроші цілком повертаються лише на місяць
пізніше, в листопаді замість жовтня. Отже, коли в наслідок скорочення
