\parcont{}  %% абзац починається на попередній сторінці
\index{ii}{0061}  %% посилання на сторінку оригінального видання
то це з самого початку показує, що різні, виражені в частинах продукту складові частини вартости $Т'$
мусять зайняти неоднакове місце, залежно від того, чи має силу $Т'\dots{} Т'$ як форма руху сукупного
суспільного капіталу, чи як самостійний рух індивідуального промислового капіталу. Всіма такими
особливостями цей кругобіг показує, що він є кругобіг, який виходить поза межі поодинокого кругобігу
просто індивідуального капіталу.

У фігурі $Т'\dots{} Т$ рух товарового капіталу, тобто сукупного капіталістично спродукованого продукту, не
лише являє передумову самостійного кругобігу індивідуального капіталу, але й собі зумовлюється ним.
Тому, коли схоплено своєрідність цієї фігури, то вже не досить заспокоїтись на тому, що метаморфози
$Т' — Г'$ та $Г — Т$ є, з одного боку, функціонально визначені частини в метаморфозі капіталу, а, з
другого боку, — члени загальної товарової циркуляції. Постає потреба виразно визначити переплітання
метаморфоз одного індивідуального капіталу з метаморфозами інших індивідуальних капіталів і з
частиною сукупного продукту, призначеною для особистого споживання. Тому в аналізі кругобігу
індивідуального промислового капіталу ми беремо за основу переважно перші дві форми.

Кругобіг $Т'\dots{} Т'$ як форма поодинокого індивідуального капіталу виступає, прим., у хліборобстві, де
розрахунки робиться від жнив до жнив. У фігурі II за вихідний пункт є засів, а в фігурі III — жнива,
або, як кажуть фізіократи, в першому — avances\footnote*{
Авансування. \emph{Ред.}
}, а в другому — reprises\footnote*{
Надходження. \emph{Ред.}
}. Рух капітальної
вартости в III з самого початку виступає лише як частина руху загальної маси продуктів, тимчасом як
в I і II рух $Т'$ становить лише один момент у русі відокремленого капіталу.

У фігурі III наявні на ринку товари становлять постійну передумову процесу продукції та репродукції.
Тому коли звертати увагу тільки на цю фігуру, то здається, що всі елементи продукційного процесу
походять із товарової циркуляції і складаються лише з товарів. При такому однобічному розумінні
недобачають тих елементів продукційного процесу, що не залежать від товарових елементів.

Що в $Т'\dots{} Т'$ вихідний пункт є сукупний продукт (сукупна вартість), то виявляється тут, що
(залишаючи осторонь закордонну торговлю) репродукція в поширених розмірах, при незмінній взагалі
продуктивності, може відбуватись лише тоді, коли в частині додаткового продукту, що її мають
капіталізувати, вже містяться речові елементи додаткового продуктивного капіталу; що, отже, оскільки
продукція одного року є за і передумову продукції наступного року, або оскільки ця продукція може
відбуватись протягом одного року, одночасно з простим процесом репродукції, остільки додатковий
продукт безпосередньо продукується в такій формі, яка дає йому спромогу функціонувати в ролі
додаткового капіталу. Збільшена продуктивність може збільшити лише речовину капіталу, не підвищуючи
цим
\parbreak{}  %% абзац продовжується на наступній сторінці
