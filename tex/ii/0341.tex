Збірний робітник І продав свою робочу силу збірному капіталістові
І на 1000; цю вартість виплачено йому грішми в формі заробітної плати.
На ці гроші він купує в II засоби споживання на ту саму суму
вартости. Капіталіст II протистоїть йому лише як продавець товарів, і
нічого більше, хоч би робітник купував у свого власного капіталіста, як
напр., вище (стор. 311) в обміні 500 II v. Форма циркуляції що її пророблює
його товар, робоча сила, це форма простої циркуляції товарів, спрямованої
виключно на задоволення потреб, на споживання Т (робоча сила) —
Г — Т (засоби споживання, товар II). Результат цього акту циркуляції
той, що робітник зберіг себе як робочу силу для капіталіста І, як таку,
і щоб зберегти себе як робочу силу надалі, робітник мусить знову та
знову повторювати процес Р (Т) — Г — Т. Його заробітна плата реалізується
в засобах споживання, її витрачається як дохід, і, беручи
робітничу клясу в цілому, завжди знову й знов витрачається як дохід.

Розгляньмо тепер той самий обмін IV на II c з погляду капіталіста. Ввесь
товаровий продукт II складається з засобів споживання, отже, з речей,
призначених на те, щоб увійти в річне споживання, тобто служити комубудь
— в даному разі збірному робітникові І — для реалізації доходу. Але
для збірного капіталіста ІI частина його товарового продукту, = 2000,
являє тепер перетворену на товар форму сталої капітальної вартости
його продуктивного капіталу, що його з цієї товарової форми треба
знову перетворити на ту натуральну форму, в якій він може знову
функціонувати як стала частина продуктивного капіталу. До цього часу
капіталіст II досяг того, що половину (= 1000) своєї сталої капітальної
вартости, репродукованої в товаровій формі (в засобах споживання) він
знову перетворив на грошову форму через продаж робітникові І. Отже,
на цю першу половину сталої капітальної вартости II с перетворився не
змінний капітал Іv, а гроші, які в обміні на робочу силу функціонували
для І як грошовий капітал і потрапили таким чином у посідання продавця
робочої сили, для якого вони являють зовсім не капітал, а дохід
у грошовій формі, тобто він їх витрачає як купівельний засіб на предмети
споживання. З другого боку, гроші = 1000, що приплили від робітників
І до капіталістів II, не можуть функціонувати як сталий елемент
продуктивного капіталу II. Це покищо лише грошова форма його товарового
капіталу, що її ще лише треба перетворити на основні або обігові
складові частини сталого капіталу. Отже, II на гроші, вторговані від
робітників І, покупців його товару, купує в І засоби продукції на 1000.
У наслідок цього стала капітальна вартість II на половину всієї своєї
величини відновлюється в тій натуральній формі, що в ній вона знову
може функціонувати як елемент продуктивного капіталу II. Формою
циркуляції при цьому було Т — Г — Т: засоби споживання вартістю в 1000 —
гроші = 1000 — засоби продукції вартістю в 1000.

Але Т — Г — Т в даному разі є рух капіталу. Т, продане робітникам,
перетворюється на Г, а це Г перетворюється на засоби продукції; це —
зворотне перетворення з товару на речові творчі елементи цього товару.
З другого боку, так само, як капіталіст II проти І функціонує лише як по-
