Загальна сума:

Сталий капітал.  .  .  .      7440 ф. пряжі = 372 ф. стерл.
Змінний „.  .  .  .      1000 „„ = 50 „„
Додаткова вартість.  .     1560 „„ = 78 „„

                      Разом. . .    10.000 ф. пряжі = 500 ф. стерл.

Т' — Г' само по собі є не що інше, як продаж 10.000 ф. пряжі.
10.000    ф. пряжі є товар, як і всяка інша пряжа. Для покупця має значення
ціна в 1 шил. за фунт, або 500 ф. стерл. за 10.000 ф. пряжі. Коли
підчас торгу він і звертає увагу на склад вартости, то лише маючи
хитрий намір довести, що 1 ф. можна було б продати дешевше, ніж за
1 шил., і що навіть у цьому разі продавець усе ж зробить вигідну оборудку.
Але кількість товару, що його він купує, залежить від його
потреб; наприклад, коли він власник ткацького підприємства, то ця
кількість залежить від складу його власного капіталу, що функціонує в
підприємстві, але не від складу капіталу того прядуна, що в нього він
купує. Відношення, що в них Т' повинне, з одного боку, покрити
зужиткований у процесі його продукції капітал (тобто різні його складові
частини), а, з другого боку, правити за додатковий продукт, призначений
чи то на витрачання додаткової вартости, чи то на акумуляцію капіталу,
існують лише в кругобігу капіталу, що його товарову форму являють
10.000 ф. пряжі. З продажем, як таким, вони не мають нічого спільного.
Тут, крім того припускається, що Т' продається по своїй вартості, тобто
справа сходить лише на перетворення його з товарової форми на грошову.
Для Т', як для функціональної форми в кругобігу цього індивідуального
капіталу, — форми, що з неї треба покрити продуктивний капітал, має,
природно, вирішувальне значення, чи відхиляються та до якої міри відхиляються
одне від одного ціна і вартість при продажу, але тут, розглядаючи
самі лише ріжниці щодо форм, нам не потрібно досліджувати
цього.

У формі І, Г... Г', процес продукції відбувається посередині між двома
протилежними, що одна одну доповнюють, фазами циркуляції капіталу;
і він закінчується раніше, ніж надійде кінцева фаза Т' — Г'. Гроші авансується
як капітал, спочатку на елементи продукції, з них вони перетворюються
на товаровий продукт, і цей товаровий продукт знову перетворюється
на гроші. Це — цілком вивершений цикл оборудок, що результат
його є на все й для кожного придатні гроші. Таким чином відновлення
процесу дано лише в можливості. Г... ГІ... Г' може бути так само останнім
кругобігом, що вивершує функціонування індивідуального капіталу, який
виходить з підприємства, як і першим кругобігом капіталу, що вперше
вступає у функціонування. Загальний рух тут є Г... Г', від грошей до більшої
суми грошей.

У формі II, тобто у формі П... Т' — Р — Т... П (ГГ) увесь процес циркуляції
йде за першим П і поперед другого, але відбувається він зворотним
порядком проти форми І. Перше П є продуктивний капітал, і його функ-
