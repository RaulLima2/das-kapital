засобів комунікації. Який-будь осередок продукції, що мав особливі переваги,
в наслідок того, що він містився на великому шляху або каналі,
тепер опиняється близько однісінького залізничного рукава, який функціонує
з порівняно великими перервами, тимчасом як інший осередок,
що був раніш зовсім осторонь від головних шляхів сполучення, тепер
опиняється у вузловому пункті кількох залізниць. Другий осередок розвивається,
перший занепадає. Отже, зміна в засобах транспорту зумовлює
місцеві відмінності в часі обігу товарів, в умовах купівлі, продажу
тощо, або вона інакше розподіляє вже наявні місцеві відмінності. Важливість
цієї обставини для обороту капіталу виявляється в суперечках між
представниками купців та промисловців різних місцевостей з управлінням
залізниць. (Див., напр., вище цитовану Синю книгу Railway Committee).

Тому всі галузі продукції, що, відповідно до природи своїх продуктів,
розраховані переважно на місцевий збут, як напр., броварні, розвиваються
до велетенських розмірів у головних залюднених центрах.
Швидший оборот капіталу почасти урівноважує тут більше подорожчання
деяких умов продукції, місця під будівлю тощо.

Коли, з одного боку, з поступом капіталістичної продукції розвиток
засобів транспорту й комунікації скорочує час обігу для даної кількости
товарів, то той самий поступ і дана з розвитком засобів транспорту
й комунікації можливість, навпаки, зумовлює доконечність роботи
на чимраз віддаленіші ринки, коротко кажучи, на світовий ринок. Маса
товарів, що перебувають у дорозі, відправлених до віддалених пунктів,
надзвичайно зростає, а тому абсолютно й відносно зростає і та частина
суспільного капіталу, яка постійно протягом довшого часу перебуває в
стадії товарового капіталу, перебуває в періоді обігу. Одночасно зростає
в наслідок цього й та частина суспільного багатства, що, замість безпосередньо
служити засобом продукції, витрачається на засоби транспорту
й зв’язку та на основний і обіговий капітал, потрібний для їхньої
роботи.

Відносний протяг подорожі товару від місця продукції до місця
збуту зумовлює ріжницю не лише в першій частині часу обігу, в часі
продажу, а і в другій частині, в зворотному перетворенні грошей
на елементи продуктивного капіталу, в часі купівлі. Напр., товар відправляють
в Індію. Це триває, припустімо, чотири місяці. Хай час продажу
дорівнює нулеві, тобто товар надсилається на замовлення й гроші
виплачується аґентові продуцента під час здачі товару. Зворотна пересилка
грошей (форма, з якій їх пересилається, тут не має значення) триває
знову таки чотири місяці. Отже, минає загалом вісім місяців, раніш
ніж той самий капітал має змогу знову функціонувати як продуктивний капітал,
— раніш ніж з ним можна знову розпочати ту саму операцію.
Спричинені таким чином відмінності в обороті становлять одну з матеріяльних
основ для різних кредитових строків, подібно до того, як
морська торгівля, напр., у Венеції та Ґенуї взагалі становить одно з
джерел кредитової системи у власному розумінні слова. „Криза 1847 р.
дала банкам і торговим підприємствам того часу змогу скоротити ін-
