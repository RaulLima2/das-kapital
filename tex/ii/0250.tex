відповідно авансованих ними продуктивних капіталів; і так само постійно
доводиться їм розподіляти між собою ту суму вартости, що її вони
з усіх боків подають у циркуляцію в товаровій формі, як відповідний
надлишок товарової вартости проти вартости її елементів продукції.

Але товаровий капітал, перш ніж він перетвориться знову на продуктивний
капітал, і перш ніж витратиться вміщену в ньому додаткову
вартість, треба перетворити на гроші. Відки беруться гроші для цього?
На перший погляд питання це видається складним, і ні Тук, ні хто інший
до цього часу не дали на нього відповіді.

Припустімо, що обіговий капітал в 500 ф. стерл. авансований у формі
грошового капіталу, — хоч який буде період його обороту, — є сукупний
обіговий капітал суспільства, тобто кляси капіталістів. Додаткова вартість
хай буде 100 ф. стерл. Яким же чином ціла кляса капіталістів може постійно
вилучати з циркуляції 600 ф. стерл., постійно подаючи в неї
лише 500 ф. стерл.?

Після того, як грошовий капітал в 500 ф. стерл. перетворився на
продуктивний капітал, цей останній у процесі продукції перетворюється
на товарову вартість в 600 ф. стерл. і таким чином в циркуляції перебуває
не лише товарова вартість в 500 ф. стерл., рівна первісно авансованому
грошовому капіталові, а й новоспродукована додаткова
вартість в 100 ф. стерл.

Цю новододану додаткову вартість в 100 ф. стерл. подано в циркуляцію
в товаровій формі. В цьому немає жодного сумніву. Але в наслідок
цієї операції не здобувається додаткових грошей для циркуляції
цієї новододаної товарової вартости.

Не слід намагатися обминати цієї трудности за допомогою зовнішньопристойних
викрутів.

Наприклад: щодо сталого обігового капіталу, то очевидно, що його
не всі витрачають одночасно. У той час, коли капіталіст А продає
свій товар, отже, коли авансований ним капітал набирає для нього грошової
форми, для покупця В його капітал, що перебуває в грошовій
формі, набирає, навпаки, форми його засобів продукції, саме тих, що їх
продукує А. Тим самим актом, що ним А знову надає грошової форми
своєму спродукованому товаровому капіталові, В знову надає продуктивної
форми своєму капіталові, перетворює його з грошової форми на засоби
продукції та робочу силу; та сама сума грошей функціонує в двобічному
процесі, як при всякій простій купівлі Т — Г. З другого боку, коли
А знову перетворює гроші на засоби продукції, він купує їх в С, а
цей платить тими самими грішми В і т. ін. Тоді справу з’ясувалось би. Але:

Всі закони, викладені нами (кн. І, розд. Ill) щодо кількости грошей,
які циркулюють при товаровій циркуляції, зовсім не змінюються в наслідок
капіталістичного характеру продукційного процесу.

Отже, коли кажуть, що обіговий капітал суспільства, який треба
авансувати в грошовій формі, становить 500 ф. стерл., то при цьому вже
взято на увагу, що це, з одного боку, є така сума, яку авансовано одночасно,
але що, з другого боку, сума ця пускає в рух більше продук-
