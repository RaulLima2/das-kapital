\parcont{}  %% абзац починається на попередній сторінці
\index{ii}{0195}  %% посилання на сторінку оригінального видання
що зменшується розміри продукції. Порівняно з маштабом продукції, капітал,
закріплений в грошовій формі, тут ще більше зростає.

Таким поділом капіталу на первісний продуктивний і додатковий капітал
взагалі досягається безперервна послідовність робочих періодів, постійне
функціонування однаково великої частини авансованого капіталу,
як продуктивного капіталу.

Придивімось до прикладу II. Капітал, що постійно перебуває в процесі
продукції, є 500 ф. стерл. А що робочий період дорівнює 5 тижням,
то протягом 50 тижнів (а їх ми беремо, як рік) цей капітал буде в
роботі 10 разів. Тому й продукт, — лишаючи осторонь додаткову вартість
— дорівнює 500X10—5000 ф.  стерл. Отже, з погляду капіталу,
безпосередньо і безупинно діющого в прсдукційному процесі, — з погляду
капітальної вартости в 500 ф. стерл., — час обігу, здається, цілком
знищується. Період обороту збігається з робочим періодом; час обігу прирівнюється
нулеві.

Коли б, навпаки, продуктивну діяльність капіталу в 500 ф. стерл. регулярно
перепинялося п’ятитижневим періодом обігу, так що він ставав
би знову продукційноздатним лише по закінченні цілого десятитижневого
періоду обороту, то протягом 50 тижнів року ми мали б 5 десятитижневих
оборотів; в них було б 5 п’ятитижневих періодів продукції, отже,
разом 25 тижнів продукції з загальною кількістю продукту на 500 X 5 = 2500
ф. стерл.; 5 п’ятитижневих періодів обігу, отже, цілого часу обігу теж
25 тижнів. Коли ми тут кажемо, що капітал в 500 ф. стерл. обернувся
5 разів протягом року, то очевидно й зрозуміло, що протягом половини
кожного періоду обороту цей капітал в 500 ф. стерл. зовсім не функціонував
як продуктивний капітал, і що в підсумку він функціонував тільки
протягом півроку, а другу половину року зовсім не функціонував.

В нашому прикладі на час цих п’ятьох періодів обігу входить у роботу
додатковий капітал в 500 ф. стерл., і в наслідок цього оборот підвищується
з 2500 ф. стерл. до 5000 ф. стерл. Але й авансований капітал
тепер є 1000 ф. стерл. замість 500 ф. стерл. 5000 поділені на 1000
дорівнює 5. Отже, замість 10 оборотів маємо 5. Так справді й рахують.
Однак, коли кажуть, що капітал 1000 ф. стерл. обернувся 5 разів протягом
року, то в пустій голові капіталіста зникає спогад про час обігу,
і постає сплутане уявлення, ніби цей капітал протягом 5 послідовних
оборотів постійно функціонував у процесі продукції. Але, коли ми кажемо,
що капітал 1000 ф. стерл. обернувся п’ять разів, то сюди ввіходить
і час обігу й час продукції. Справді, коли б 1000 ф. стерл. безперервно
функціонували в процесі продукції, то при наших припущеннях продукт
мусив би бути 10000 ф. стерл. замість 5000. Але для того, щоб завжди
мати в процесі продукції 1000 ф. стерл., довелось би взагалі авансувати
2000 ф. стерл. Економісти, що в них взагалі не знайти нічого виразного
про механізм обороту, завжди недобачають той головний момент, що
продукція може відбуватися безперервно лише тоді, коли в процесі продукції
завжди буде фактично зайнята тільки частина промислового капіталу.
Тимчасом як одна частина перебуває в періоді продукції, друга частина
\index{ii}{0196}  %% посилання на сторінку оригінального видання
весь час мусить бути в періоді циркуляції. Або, інакше кажучи, одна
частина може функціонувати як продуктивний капітал лише з тією умовою,
що другу частину, в формі товарового або грошового капіталу, вилучено
з власне продукції. Недобачати це — значить взагалі недобачати значення
й ролі грошового капіталу.

Нам треба тепер дослідити, яка ріжниця буде в обороті залежно від
того, чи будуть обидва відділи періоду обороту — робочий період і період
циркуляції — рівні один одному, чи робочий період буде більший
або менший, ніж період циркуляції, а потім дослідити, як це впливає на
закріплення капіталу в формі грошового капіталу.

Припустімо, що авансовуваний щотижня капітал в усіх випадках дорівнює
100 ф. стерл., а період обороту — 9 тижням; отже, капітал, який
треба авансувати на кожен період обороту, дорівнює 900 ф. стерл.

І. Робочий період дорівнює періодові циркуляції

Цей випадок, хоч, насправді, він трапляється тільки як рідкісний виняток,
мусить бути за вихідний пункт у дослідженні, бо відношення тут виступають
якнайпростіше та якнайнаочніше.

Два капітали (капітал І, авансований на перший робочий період, і додатковий
капітал II, що функціонує протягом періоду циркуляції капіталу І)
чергуються один по одному в своєму русі, не сплітаючись один з одним.
Тому, за винятком першого періоду, кожний із обох капіталів авансується
лише на свій власний період обороту. Період обороту хай буде,
як у наступних прикладах, 9 тижнів; отже, робочий період і період обігу
буде по 4 1/2 тижні. Тоді ми маємо таку схему року:

Таблиця 1

Капітал І

Періоди обороту    Робочі періоди    Авансовано    Періоди циркуляції
І. Тижні 1—9    Тижні 1—4 1/2                 450 ф. ст.        Тижні 4 1/2—9
II.    „   10—18        „      10—13 1/2             450 „ „                  13 1/2—18
III.   „   19—27       „       19—22 1/2             450 „ „                22 1/2—27
IV.   „    28—36      „        28—31 1/2             450 „ „               31 1/2—36
V.     „    37—45      „        37—40 1/2             450 „ „               40 1/2—45
VI.   „    46—[54]  „        46—49 1/2            450 „ „            49 1/2—[54] 31

31) Тижні, що припадають на другий рік обороту, взято в дужки.
