\parcont{}  %% абзац починається на попередній сторінці
\index{ii}{0060}  %% посилання на сторінку оригінального видання
що під неї можна підвести кожний індивідуальний промисловий капітал (крім тих випадків, коли капітал
вкладається вперше), отже, не лише як форму руху, спільну всім індивідуальним промисловим капіталам,
але разом з тим і як форму руху суми індивідуальних капіталів, тобто сукупного капіталу кляси
капіталістів, — як рух, що в ньому рух кожного індивідуального промислового капіталу є лише
частинний рух, який переплітається з рухами інших капіталів і ними зумовлюється. Коли ми, напр.,
розглядаємо сукупний річний товаровий продукт якоїсь країни й аналізуємо рух, що за допомогою його
одна частина цього продукту покриває продуктивний капітал у всіх індивідуальних підприємствах, а
друга частина входить в особисте споживання різних кляс, то ми розглядаємо $Т'\dots{} Т'$ як форму руху
так суспільного капіталу, як і утвореної ним додаткової вартости, зглядно додаткового продукту. Та
обставина, що суспільний капітал дорівнює сумі індивідуальних капіталів (залічуючи сюди акційні
капітали, зглядно державний капітал, оскільки уряди вживають продуктивну найману працю в гірництві,
залізницях тощо, функціонують як промислові капіталісти) і що загальний рух суспільного капіталу
дорівнює альґебричній сумі рухів
індивідуальних капіталів, зовсім не виключає того, що цей рух, як рух ізольованого індивідуального
капіталу, виявляє інші явища, ніж той самий рух, розглядуваний з погляду частини сукупного руху
суспільного капіталу, тобто в зв’язку його з рухами інших частин його, — і що він разом з тим
розв’язує такі проблеми, що їх розв’язання мусить припускатися як дане при розгляді кругобігу
поодинокого індивідуального капіталу, а не виводитися з нього.

$Т'\dots{} Т'$ є єдиний кругобіг, де первісно авансована капітальна вартість становить лише частину
крайнього члена, що починає рух, і де рух з самого початку виявляється як сукупний рух промислового
капіталу; так само, як рух тієї частини продукту, яка покриває продуктивний капітал, так і тієї
частини, що являє додатковий продукт і пересічно витрачається почасти як дохід, а почасти має стати
елементом акумуляції. Оскільки витрачання додаткової вартости як доходу входить у цей кругобіг,
остільки в нього входить також особисте споживання. Але воно входить ще й тому, що вихідний пункт Т,
товар, існує в вигляді будь-якого предмету споживання; а кожний капіталістично спродукований предмет
є товаровий капітал, усе одно, чи його споживна форма призначає його для продуктивного, чи для
особистого споживання, або для обох разом. $Г\dots{} Г' п$оказує лише вартісний бік зростання авансованої
капітальної вартости як мету цілого процесу; $П\dots{} П$ (П') показує процес продукції капіталу як процес
репродукції, при чому величина продуктивного капіталу лишається однакова або зростає (акумуляція);
$Т'\dots{} Т'$, виявляючу себе вже на своєму початковому пункті як форма капіталістичної товарової
продукції, з самого початку охоплює продуктивне та особисте споживання; продуктивне споживання і
закладене в ньому зростання вартости є лише частина його руху. Нарешті, що $Т'$ може існувати в такій
споживній формі, яка знову не може ввійти в будь-який процес продукції,
\parbreak{}  %% абзац продовжується на наступній сторінці
