\parcont{}  %% абзац починається на попередній сторінці
\index{ii}{0237}  %% посилання на сторінку оригінального видання
застосованих у В, і на заміщення цього всього подається на ринок еквівалент
в формі грошей; але протягом цього року на ринок не подається
жодного продукту, щоб замістити взяті з ринку речові елементи продуктивного
капіталу. Коли ми уявимо собі не капіталістичне суспільство,
а комуністичне, то, насамперед, зовсім відпадає грошовий капітал, а значить,
і всі ті маскування оборудок, які постають через грошовий капітал.
Справа сходить просто на те, що суспільство мусить наперед обчислити,
скільки праці, засобів продукції та засобів існування воно може без якої-будь
шкоди витрачати на такі галузі продукції, що, як от, напр, будування
залізниць, довгий час, рік або й більше, не дають ні засобів
продукції, ні засобів існування, ні взагалі будь-якого корисного ефекту,
але звичайно відбирають від цілої річної продукції працю, засоби продукції
і засоби існування. Навпаки, в капіталістичному суспільстві, де
суспільний розум завжди виявляє себе тільки post festum\footnote*{
Post festum — дослівно: „після свята“, коли справу вже вакінчено. \emph{Ред.}
}, можуть і мусять
завжди поставати великі порушення. З одного боку, тиск на грошовий
ринок, тимчасом як гарний стан грошового ринку, навпаки, і собі покликає
до життя багато таких підприємств, отже, призводить саме до таких обставин,
що потім зумовлюють тиск на грошовий ринок. Грошовий ринок
зазнає тиску, бо при цьому треба постійно авансувати великий грошовий
капітал на довгий час. Ми вже зовсім не кажемо про те, що промисловці
й торговці кидають на залізничні спекуляції тощо грошовий капітал,
потрібний їм для провадження власних підприємств, і заміщують його позиками
на грошовому ринку. — З другого боку, зазнає тиску продуктивний капітал,
що є в розпорядженні суспільства. А що елементи продуктивною капіталу
постійно вилучається з ринку і натомість на ринок подається лише
грошовий еквівалент, то більшає виплатоспроможний попит, який, із свого
боку, не має в собі жодних елементів подання. Відси підвищення цін
і на засоби існування, і на продукційні матеріяли. До цього долучається
ще й те, що під такий час звичайно розвивається шахрайство і переміщується
чимало капіталу. Зграя спекулянтів, постачальників, інженерів,
адвокатів тощо збагачується. Вони спричиняють на ринку великий попит
на речі споживання, поряд цього підвищується заробітна плата. Щодо
попиту на харчові засоби, то він звичайно підганяє й сільське господарство.
А що цих харчових засобів не можна збільшити одразу, протягом
року, то більшає довіз їх, як і взагалі довіз екзотичних харчових
засобів (кави, цукру, вина тощо) та речей розкошів. Звідси надмірний
довіз і спекуляція в цій галузі імпортної торговлі. З другого боку, в
тих галузях промисловости де продукцію можна швидко збільшити (власне
мануфактура, гірництво тощо), підвищення цін призводить до раптового
поширення, що по ньому скоро настає крах. Такий самий вплив
справляється на робочий ринок, щоб притягти до нових галузей підприємств
великі маси латентного відносного надміру людности і навіть робітників,
уже зайнятих. Взагалі такі підприємства великого маштабу, як
от залізниці, відтягують від робочого ринку певну кількість сил, що
\parbreak{}  %% абзац продовжується на наступній сторінці
