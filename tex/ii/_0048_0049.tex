\index{ii}{0048}  %% посилання на сторінку оригінального видання
\subsection{Акумуляція грошей}

Чи може $г$, додаткова вартість, перетворена на золото, відразу знову
прилучитись до капітальної вартости, що процесує, і таким чином разом
з капіталом $Г$ як величина $Г'$ увійти в процес кругобігу, — це залежить
від обставин, незалежних від простої наявности $г$. Коли $г$ повинно правити
за грошовий капітал у другому, самостійному підприємстві, що його
мають закласти поряд першого, то, очевидно, $г$ можна вжити на це
лише тоді, коли воно має мінімальну величину, потрібну для такого підприємства.
Коли $г$ треба вжити на поширення початкового підприємства,
то знову таки відношення речевих чинників $П$ і відношення їхньої
вартости потребують певної мінімальної величини $г$. Між усіма засобами
продукції, що діють у цьому підприємстві, є не лише певне якісне, але
й певне кількісне відношення, певна пропорційність розмірів. Ці
речеві відношення і відношення вартости, що їх носіями є речеві відношення
тих чинників, які входять у продуктивний капітал, визначають
мінімальний розмір, що його мусить мати $г$ для того, щоб його можна
було перетворити як приріст продуктивного капіталу на додаткові засоби
продукції та додаткову робочу силу, або лише на перші. Напр., прядун
не може збільшити числа своїх веретен, коли він не набуде одночасно
відповідне число чухральних і тіпальних верстатів, не кажучи вже про
збільшені витрати на бавовну та заробітну плату, що їх зумовлює таке
поширення підприємства. А щоб перевести це поширення в життя, додаткова
вартість мусить уже становити чималу суму (звичайно нові
витрати обчислюються в 1 ф. стерл. на веретено). Доки $г$ не має цього
мінімального розміру, кругобіг капіталу мусить кілька разів повторитись,
поки сума послідозно ним спороджених г матиме змогу функціонувати
разом з $Г$, тобто $Г' — Т' —\splitfrac{Р}{Зп}$. Навіть прості зміни деталів, напр., у прядильних машинах, оскільки
вони роблять їх продуктивнішими, потребують
збільшених витрат прядильного матеріялу, поширення машин, що
обробляють бавовну до прядіння тощо. Отже, у проміжний період $г$
нагромаджується, і його нагромадження є не його власна функція, а
результат повторних $П\dots{} П$. Його власна функція є його перебування в
грошовому стані, доки воно з повторних кругобігів, що дають приріст
вартости, тобто із-зовні, матиме достатній приріст, щоб досягти мінімальної
величини, потрібної для його активного функціонування, — величини,
що, тільки мавши її, воно як грошовий капітал, у даному разі, як акумульована
частина грошового діющого капіталу $Г$, дійсно може ввійти у функціонування
цього останнього. У проміжний період г нагромаджується й
існує лише в формі скарбу, що перебуває в процесі свого утворення, зростання.
Отже, грошова акумуляція, утворення скарбу виступає тут як процес,
що тимчасово супроводить справжню акумуляцію, поширення того маштабу,
що в ньому діє промисловий капітал. Тимчасово супроводить, бо
поки скарб перебуває в стані скарбу, він не функціонує як капітал, не
\index{ii}{0049}  %% посилання на сторінку оригінального видання
бере участи в процесі зростання вартости, лишається грошовою сумою,
яка зростає лише тому, що гроші, які існують поза її участю, складається
до тієї самої скрині.

Форма скарбу є лише форма грошей, що не перебувають в циркуляції,
грошей, що їхню циркуляцію перервано, і тому їх зберігається
в їхній грошовій формі. Щодо самого процесу утворення скарбу, то
він властивий усякій товаровій продукції і відіграє ролю як самоціль
лише в нерозвинених передкапіталістичних формах товарової продукції.
Але в нашому прикладі скарб виступає як форма грошового капіталу, а
утворення скарбу — як процес, що тимчасово супроводить акумуляцію
капіталу, тому і остільки, що і оскільки гроші фігурують тут як
\emph{лятентний грошовий капітал}; бо утворення скарбу, перебування в стані
скарбу тієї додаткової вартосте, яка є в грошовій формі, являє тут функціонально
визначену підготовчу стадію для перетворення додаткової
вартосте на справді діющий капітал, — стадію, що відбувається поза
кругобігом капіталу. Отже, своїм призначенням це є лятентний грошовий
капітал; тому й розміри, що їх мусить дійти скарб, щоб увійти в
процес, визначається кожного разу вартісним складом продуктивного
капіталу. Але поки гроші лишаються в стані скарбу, вони ще не функціонують
як грошовий капітал, вони ще є бездіяльний капітал, не тому,
що, як раніше, їхню функцію перервано, а тому, що вони ще не здатні
до своєї функції.

Ми беремо тут нагромадження грошей в його первісній реальній
формі, як дійсний грошовий скарб. Але воно може існувати і в
формі простих боргових документів, боргових вимог капіталіста,
що продав $Т'$. Щодо інших форм, коли цей лятентний грошовий
капітал і в проміжний період існує в формі грошей, що вилуплюють
гроші, напр., як вклади в який-будь банк на проценти, в векселях
або цінних паперах будь-якого ґатунку, то ці форми сюди не стосуються.
Додаткова вартість, реалізована в грошах, виконує тоді особливі функції
капіталу поза кругобігом того промислового капіталу, що з нього
вона походить; функції, що, поперше, не мають нічого спільного з кругобігом
цього капіталу як таким і, подруге, припускають функції капіталу,
відмінні від функцій промислового капіталу, які тут ще не досліджені.

\subsection{Резервний фонд}

Скарб у щойно розгляненій формі, як формі існування додаткової
вартости, є грошовий фонд акумуляції, грошова форма, що її тимчасово
має акумуляція капіталу, і яка остільки сама є умова акумуляції.
Але цей акумуляційний фонд може виконувати і особливі побічні послуги,
тобто входити в процес кругобігу капіталу, не набираючи форми $П\dots{} П'$,
а, значить, не поширюючи розмірів капіталістичної репродукції.

Коли процес $Т' — Г' т$риває понад свій нормальний протяг, коли, отже,
перетворення товарового капіталу на грошову форму ненормально затримується;
або коли, після того як це перетворення відбулося, ціна, напр.,
\parbreak{}  %% абзац продовжується на наступній сторінці
