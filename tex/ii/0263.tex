грошового обігу“. (Карл Маркс, „До критики політичної економії“
1859 р., стор. 105—106. — Вираз „монета“ протилежно до грошей вжито
тут на позначення грошей у їхній функції як простого засобу циркуляції,
протилежно до інших їхніх функцій).

Коли всіх цих засобів не досить, то доводиться додатково продукувати
золото або — що сходить на те саме — частину додаткового продукту
обмінюється безпосередньо або посередньо на золото, на продукт країн,
що продукують благородні металі.

Вся сума робочої сили та суспільних засобів продукції, витрачуваних
на щорічну продукцію золота й срібла як знаряддя циркуляції, становить
чималу частину faux frais капіталістичного способу продукції і взагалі
способу продукції, що ґрунтується на товаровій продукції. Ця продукція
відтягує від суспільного користання відповідну суму можливих,
додаткових засобів продукції та споживання, тобто справжнього багатства.
Оскільки за незмінного даного маштабу продукції або за даного ступеня
її поширення меншають витрати на цей дорогий механізм циркуляції,
остільки ж підвищується в наслідок цього продуктивна сила суспільної
праці. Отже, оскільки так впливають допоміжні засоби, що розвиваються
разом з кредитовою системою, вони безпосередньо збільшують капіталістичне
багатство, або тим, що більшу частину процесу суспільної продукції
та процесу суспільної праці провадиться в наслідок цього без
якоїбудь інтервенції справжніх грошей, або тим, що підвищується
функціональну спроможність грошової маси, яка справді функціонує.

Цим розв’язується безглузде питання про те, чи можлива була б
капіталістична продукція в її теперішніх розмірах без кредитової системи
(коли навіть розглядати її тільки з цього погляду), тобто при самій
металевій циркуляції. Очевидно, ні. Навпаки, вона була б обмежена розміром
продукції благородних металів. З другого боку, не треба складати собі
містичних уявлень про продуктивну силу кредитової системи, оскільки
вона дає в розпорядження грошовий капітал або пускає його в рух.
Однак дальший розвиток цього сюди не стосується.

Тепер ми повинні розглянути той випадок, коли відбувається не
справжня акумуляція, тобто безпосереднє поширення розмірів продукції,
а коли частину реалізованої додаткової вартости на більш-менш довгий
час акумулюється як грошовий резервний фонд, щоб пізніше перетворити
його на продуктивний капітал.

Оскільки гроші, що їх акумулюється таким чином, є додаткові гроші,
справа сама собою зрозуміла. Вони можуть бути лише частиною надлишкового
золота, довезеного з країн, що продукують золото. При цьому
треба зазначити, що в країні немає вже того національного продукту,
що за нього довезено це золото. Його віддано за кордон в обмін на
золото.

Навпаки, коли припустити, що в країні лишається, як і раніш, та сама
маса грошей, то нагромаджувані гроші припливають з циркуляції; змі-
