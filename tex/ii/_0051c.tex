\index{ii}{0051}  %% посилання на сторінку оригінального видання
Коли $П = П$, то Г в 2) дорівнює $Г'$ мінус г; коли $П = П'$, то Г в 2)
більше, ніж $Г'$ мінус г; це значить, що г цілком або почасти перетворилось
на грошовий капітал.

Кругобіг продуктивного капіталу є та форма, що в ній клясична
економія розглядає процес кругобігу промислового капіталу.

Розділ третій

Кругобіг товарового капіталу

Загальна формула для кругобігу товарового капіталу така:

$Т' — Г' — Т... П... Т'.

Т'$ з’являється не лише як продукт, але також як і припущення обох
попередніх кругобігів, бо те, що для одного капіталу є $Г — Т$, уже містить
у собі $Т' — Г'$ для другого, — принаймні остільки, оскільки частина засобів
продукції сама є товаровий продукт інших індивідуальних капіталів,
які пророблюють свій кругобіг. У нашому випадку, приміром,
вугілля, машини і т. ін. є товаровий капітал вуглепромисловця, капіталістичного
машинобудівника і т. ін. Далі, в розділі І, 4, вже показано,
що вже при першому повторенні $Г... Г'$, раніше, ніж закінчиться цей
другий кругобіг грошового капіталу, припускається не лише кругобіг
$П... П$, але також і кругобіг $Т'... Т'$.

Коли репродукція відбувається в поширеному розмірі, то кінцеве $Т'$
більше, ніж початкове $Т'$, а тому" його треба позначити Т".

Ріжниця третьої форми від перших двох виявляється в тому, поперше,
що тут кругобіг починається циркуляцією, як цілим, з двома її протилежними
фазами, тимчасом як у формі І циркуляцію уривається процесом
продукції, а в формі II вся циркуляція з її двома фазами, що одна одну
доповнюють, виступає лише як посередня ланка для процесу репродукції,
а тому становить посередній рух між $П... П. П$ри $Г... Г'$ форма
циркуляції є $Г — Т... Т' — Г' = Г — Т — Г. П$ри $П... П$ вона зворотна:
$Т' — Г'. Г — Т = Т — Г — Т$. В $Т'... Т'$ вона також має цю останню
форму.

Подруге: коли повторюються кругобіги І і II, навіть коли кінцеві
пункти $Г'$ і $П'$ являють початкові пункти відновленого кругобігу, зникає
форма, що в ній вони були спродуковані. $Г' = Г + г$, $П' = П + п п$очинають
новий процес знову як Г і П. Але в формі III вихідний пункт Т
мусить позначатись як $Т'$, навіть коли кругобіг поновлюється в попередніх
розмірах і ось чому. В формі І, скоро $Г'$ як таке починає новий
кругобіг, функціонує воно як грошовий капітал Г, як авансована в грошовій
формі капітальна вартість, що має зростати своєю вартістю.
Величина авансованого грошового капіталу, зростаючи в наслідок акумуляції,
що відбулась протягом першого кругобігу, збільшилась. Але хоч
\parbreak{}  %% абзац продовжується на наступній сторінці
