Ця праця, що на її значення повинен був би звернути увагу вже один
вислів: Surplus produce or capital, є памфлет на 40 сторінок, що його
Маркс видобув з непам’яті; там сказано:

„Хоч як це випадало б капіталістові [з погляду капіталіста], він
завжди може привласнювати лише додаткову працю (surplus labour) робітника,
бо робітник повинен жити“ (ст. 23). Але як робітник живе,
і тому оскільки велика може бути додаткова праця, що її привлащує
капіталіст, — це дуже відносно. „Коли вартість капіталу зменшується
не в такому відношенні, як збільшується його маса, то капіталіст буде,
витискувати з робітника продукт кожної робочої години понад той мінімум,
що з нього може існувати робітник... капіталіст може, кінець-кінцем
сказати робітникові: не треба тобі їсти хліб, бо можна прожити й на
буряках та картоплі; і ми вже дійшли цього“ (ст. 24). „Коли робітника
можна довести до такого стану, що він харчуватиметься картоплею замість
хліба, то, безперечно, правильно, що при цьому можна більше здерти з
його праці; тобто, коли, харчуючись хлібом, він повинен був на утриманая
себе та своєї сім’ї залишати для себе працю понеділка й вівторка,
то, годуючись картоплею, він матиме для себе тільки половину понеділка;
а друга половина понеділка і ввесь вівторок звільняться або на користь
державі або для капіталістів“ (ст. 26). „Безперечно (it is admitted),
що сплачувані капіталістам інтереси, чи в формі ренти, проценту або
підприємецького зиску, сплачуються з праці інших“ (ст. 23). Отже, тут
ми маємо цілком Родбертусову „ренту“, тільки замість „ренти" сказано
„інтереси“.

Маркс робить таке до цього зауваження (рукопис „Zur Kritik“,
ст. 852): „Цей мало відомий памфлет, — а видано його тоді, коли почав
звертати на себе увагу „неймовірний латальник“ Мак Куллох, — являє
великий крок наперед порівняно з Рікардо. Додаткову вартість або
„зиск", як зве її Рікардо (часто також додатковий продукт, surplus
produce), або interest, як називає її автор памфлету, останній визначає як
surplus labour, як додаткову працю, — працю, що її робітник виконує безплатно,
виконує понад ту кількість праці, що нею покривається вартість
його робочої сили, тобто що нею продукується еквівалент його заробітної
плати. Так само, як важливо було звести вартість до праці, так
само важливо було додаткову вартість (surplus value), виражену додатковому
продукті (surplus produce), звести до додаткової праці
(surplus labour). Це власно сказав уже А. Сміс, і це становить головний
момент у тому, що розвинув Рікардо. Але ніде в них це не
висловлено в абсолютній формі й не установлено точно. Потім далі, на
стор. 859 рукопису, сказано: „А, проте, автора полонили ті економічні
категорії, що були до нього. Так само, як у Рікардо сплутування додаткової
вартости й зиску призводить до неприємних суперечностей, так
само сталось і з ним тому, що він назвав додаткову вартість
інтересом капіталу. А, проте, він стоїть вище від Рікардо тією стороною, що
він перший зводить усяку додаткову вартість до додаткової праці і, хоч зве
додаткову вартість інтересом капіталу, однак, разом з тим під-
