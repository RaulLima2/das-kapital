засобами споживання своїх робітників, — в truck-system’i, що одною з
багатьох галузей її є винаймання помешкань робітникам, так що капіталіст
є разом з тим квартировласник своїх робітників.

Той самий Друммонд, що його прекрасна душа мріє про капіталістичні
спроби поліпшити стан робітничої кляси, в тому самому звіті розповідає,
між іншим, про зразкові бавовнопрядільні Lowell and Lawrence Mills,
їдальні й помешкання робітниць належать акційному товариству, якому
належать і сами фабрики; завідательки цих будинків служать у того
самого товариства, яке приписує їм правила поведінки; жодна робітниця
не сміє повертатись додому пізніше, ніж о 10-ій годині вечора. Але
ось перл: патрулі спеціяльної поліції товариства доглядають у навкольності,
щоб не порушувалось цей житловий порядок. Після 10-ої години
вечора жодну робітницю не випускають з будинку й не впускають туди.
Жодна робітниця не сміє мешкати десь інде, крім території, що належить
товариству; кожний будинок дає товариству щотижня приблизно 10 доларів
плати за помешкання; і тут ми бачимо раціональних споживачів у всій
славі. „Що в багатьох кращих будинках для робітниць є повсюди піяніно,
то музика, співи й танці відіграють значну ролю, принаймні для тих з
них, кому після нудної безперервної десятигодинної праці біля ткацького
варстату, більш потрібна переміна, ніж справжній відпочинок“ (стор. 412).
Але головна таємниця, як з робітника зробити раціонального споживача,
ще далі. Пан Друммонд завітав на фабрику ножів Turners Falls (Connecticut
River), і пан Окмен, скарбник акційного товариства, розповівши
йому, що американські столові ножі якістю переважають англійські, додав:
„Ми поб’ємо Англію і щодо цін; ми вже й тепер перевищуємо її якістю,
це визнано; але ми мусимо мати й дешевші ціни, і ми досягнемо цього
так само, як ми одержали дешевше нашу сталь і знизили плату за нашу
працю!“ (стор. 427). Зниження заробітної плати й довгий робочий день —
у цьому вся суть раціональних і здорових заходів, що повинні піднести
робітника до ранґу раціональоного споживача, щоб він утворив ринок
для маси предметів, зроблених для нього приступними культурою та поступом
у винаходах.

Отже, як І повинен дати з свого додаткового продукту додатковий
сталий капітал для II, так II дає в цьому розумінні додатковий змінний
капітал для І. Оскільки ходить про змінний капітал, II акумулює для І і
для себе самого, репродукуючи більшу частину всього свого продукту,
отже, і свого додаткового продукту, в формі доконечних засобів споживання.
При продукції на дедалі більшій капітальній базі І (v + m) мусить
дорівнювати II с плюс та частина додаткового продукту, яка знову долучається
до капіталу, плюс додаткова частина сталого капіталу, потрібна
для поширення продукції в II; а мінімум цього поширення такий, що
без нього не можлива справжня акумуляція, тобто справжнє поширення
продукції в самому І.
88
