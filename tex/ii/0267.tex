Відділ третій

Репродукція й циркуляція сукупного
суспільного капіталу

Розділ вісімнадцятий\footnote{
З рукопису II.
}

Вступ

1. Предмет досліду

Безпосередній процес продукції капіталу є процес праці й процес
зростання його вартости, процес, що наслідок його є товаровий продукт,
а визначальний мотив — продукція додаткової вартости.

Процес репродукції капіталу охоплює так цей безпосередній процес
продукції, як і обидві фази власне процесу циркуляції, тобто він охоплює
ввесь кругобіг, що як періодичний процес, — процес знову та знов
повторюваний через певні періоди — становить оборот капіталу.

Хоч розглядатимемо ми кругобіг у формі Г... Г', хоч у формі П... П,
безпосередній процес продукції П завжди становить лише одну ланку
цього кругобігу. В одній формі він виступає як посередня ланка процесу
циркуляції, в другій формі процес циркуляції виступає як посередня
ланка для нього. Постійне відновлення цього процесу, постійну повторювану
появу капіталу в формі продуктивного капіталу, в обох випадках
зумовлено його перетвореннями в процесі циркуляції. З другого боку,
постійно поновлюваний процес продукції є умова перетворень, що їх
знову та знов пророблює капітал у сфері циркуляції — є умова його
поперемінної появи то як грошового капіталу, то як товарового
капіталу.

Однак кожний поодинокий капітал становить лише усамостійнену, так
би мовити, обдаровану індивідуальним життям, частину сукупного суспільного
капіталу, так само, як кожен поодинокий капіталіст є лише індивідуальний
елемент кляси капіталістів. Рух суспільного капіталу складається
з сукупности рухів його усамостійнених уламків, з сукупности оборотів
індивідуальних капіталів. Як метаморфоза поодинокого товару є ланка
в ряді метаморфоз товарового світу — товарової циркуляції, — так