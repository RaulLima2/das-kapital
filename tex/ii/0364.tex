пускаючи, що інші умови лишились ті самі, — не змінюється; отже, вся
продукція засобів продукції мусила б в одному випадку поширитись, в
другому скоротитись. Цьому можна було б запобігти лише постійною
відносною перепродукцією; з одного боку, продукується основного
капіталу на певну кількість більше, ніж безпосередньо треба; з другого
боку, продукується такий запас сировинного матеріялу та інш., що
перевищує безпосередні річні потреби (це особливо стосується до засобів
існування). Такий рід перепродукції рівнозначний контролеві суспільства
над речовими засобами його власної репродукції. Але в капіталістичному
суспільстві вона є анархічний елемент.

Цей приклад з основним капіталом — при незмінному маштабі репродукції
— є разючий. Непропорційність у продукції основного та обігового
капіталу це — одна з улюблених економістами причин, що ними вони
пояснюють кризи. А що така непропорційність може й мусить поставати
при простому підтриманні основного капіталу, що вона може й
мусить поставати при припущенні ідеальної нормальної продукції, при
простій репродукції уже діющого суспільного капіталу, це для них — щось
нове.

XII. Репродукція грошового матеріялу

До цього часу ми зовсім не звертали уваги на один момент, а саме
на річну репродукцію золота й срібла. Як простий матеріял для речей
розкошів, позолочування тощо, вони так само, як і всякі інші продукти,
не заслуговували б тут на особливу згадку. Навпаки, як грошовий
матеріял, а тому і як потенціяльні гроші, вони відіграють важливу ролю.
Для спрощення ми будемо вважати тут за грошовий матеріял тільки золото.

За старими даними вся річна продукція золота становила 800 —
900 тисяч фунтів = заокруглюючи 1100 або 1250 мільйонів марок. Навпаки,
за Зетбеером 53) пересічно за 1871—75 роки лише 170675 кг вартістю
в округлих цифрах 476 мільйонів марок. З цього давали: Австралія
округло 167, Сполучені Штати 166, Росія 93 мільйони марок. Решта
розподіляється між різними країнами на суму меншу, ніж 10 мільйонів
марок на кожну. Річна продукція срібла за той самий період становила
трохи менш, ніж 2 мільйони кілограмів вартістю на 354 1/2 мільйони
марок; з цього Мехіко давало округло 108, Сполучені Штати 102,
Південна Америка 67, Німеччина 26 мільйонів і т. ін.

З країн, де панує капіталістична продукція, лише Сполучені Штати
є продуценти золота й срібла; європейські капіталістичні країни майже
все своє золото й переважну більшість свого срібла одержують з
Австралії, Сполучених Штатів, Мехіко, Південної Америки та Росії.

Але ми переносимо золоті копальні в ту країну капіталістичної продукції,
що її річну репродукцію ми тут аналізуємо, і робимо так ось з
яких міркувань.

Капіталістична продукція взагалі не існує без зовнішньої торговлі.
Але коли ми припускаємо нормальну річну репродукцію в даному маш-53

53) Ad. Soetbeer, "Edelmetall-Produktion". 1879, S. 112.
