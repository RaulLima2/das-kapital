звело б до підвищеної конкуренції за вільний капітал. Коли б деяка частина
його лежала без діла, те її pro tanto покликали б до діяльности.

Але, потретє, за даних розмірів продукції, за незмінної швидкости
обороту та незмінних цін елементів поточного продуктивного капіталу,
ціна продуктів підприємства X може знизитись або підвищитись. Коли
ціна товарів, що їх подає підприємство X, знижується, то спадає ціна
його товарового капіталу з 600 ф. стерл., що їх воно завжди подавало
в циркуляцію, напр., до 500 ф. стерл. Отже, шоста частина вартости авансованого
капіталу не припливає назад з процесу циркуляції (додаткову вартість,
що є в товаровому капіталі, тут не береться на увагу); вона пропадає
марно в цьому процесі. Але що вартість, зглядно ціна елементів продукції
лишається та сама, то цих 500 ф. стерл., які приплили назад, вистачить
лише на те, щоб замістити 5/6 капіталу в 600 ф. стерл, ввесь час
занятого в процесі продукції. Отже, для того, щоб і далі провадити підприємство
в тому самому маштабі, довелось би витратити 100 ф. стерл.
додаткового грошового капіталу.

Навпаки: коли ціна продуктів підприємства X підвищиться, то підвищиться
й ціна товарового капіталу з 600 ф. стерл., напр., до 700 ф.
стерл. Сьома частина його ціни, рівна 100 ф. стерл., приходить не з
процесу продукції, не була авансована на нього, а припливає сюди з
процесу циркуляції. Однак, для заміщення продуктивних елементів треба
лише 600 ф. стерл.; отже, 100 ф. стерл. звільняються.

Дослідження причин, чому в першому випадку період обороту
скорочується або подовжується, в другому випадку ціни на сировинний
матеріял та працю, і в третьому ціни поданих продуктів підвищуються
або падають, — дослідження цих причин не входить у межі цього досліду.

Але ось що входить у межі його:

I випадок. Незмінний маштаб продукції, незмінні ціни елементів
продукції та продуктів, зміна в періоді циркуляції, а значить, і в періоді
обороту.

Згідно з припущенням у нашому прикладі, в наслідок скорочення періоду
циркуляції, треба авансувати всього капіталу менше на 1/9; тому
капітал цей зменшується з 900 до 800 ф. стерл., і виділюється грошовий
капітал в 100 ф. стерл.

Як і раніш, підприємство X дає той самий шеститижневий продукт такої
самої вартости в 600 ф. стерл., а що роблять цілий рік безперервно, то
воно протягом 51-го тижня дає ту саму масу продукту, вартістю в 5100 ф.
стерл. Отже, в масі та ціні продукту, що його подає підприємство в
циркуляцію, немає жодної зміни, немає її і в тих строках, що в них підприємство
подає продукт на ринок. Але виділилось 100 ф. стерл., бо
через скорочення періоду циркуляції процес насичено авансуванням капіталу
лише в 800 ф. стерл. замість попередніх 900 ф. стерл. Ці 100 ф.
стерл. виділеного капіталу існують у формі грошового капіталу. Але вони
зовсім не репрезентують тієї частини авансованого капіталу, що постійно
мусить функціонувати в формі грошового капіталу. Припустімо, що з
авансованого поточного капіталу 1 = 600 ф. стерл. 4/5 = 480 ф. стерл.
