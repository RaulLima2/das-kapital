По-перше, він являє ту форму, що в ній кожний індивідуальний капітал
виступає на кін, починає свій процес як капітал. Тому він виступає
як primus motor\footnote*{
Перший рушій. Ред.
}, що надає руху цілому процесові.

По-друге. Відповідно до різного протягу періоду обороту і різного
відношення між обома складовими частинами його — робочим періодом і
періодом циркуляції — складова частина авансованої капітальної вартости,
що її завжди треба авансувати і відновлювати в грошовій формі, є різна
у відношенні до продуктивного капіталу, що його вона пускає в рух,
тобто у відношенні до безперервного розміру продукції. Але хоч яке
це буде відношення, за всіх обставин та частина капітальної вартости,
що процесує, що може постійно функціонувати як продуктивний капітал,
обмежується тією частиною авансованої капітальної вартости, яка мусить
завжди існувати в грошовій формі поряд продуктивного капіталу. Тут
ідеться лише про нормальний оборот, про абстрактну пересічну величину.
При цьому ми лишаємо осторонь додатковий грошовий капітал,
потрібний, щоб вирівнювати застої циркуляції.

До першого пункту. Товарова продукція припускає товарову
циркуляцію, а товарова циркуляція припускає виявлення товару в грошах,
грошову циркуляцію; двоїсте буття товару: як товару, і як грошей,
є закон виявлення продукту як товару. Так само капіталістична товарова
продукція, — хоч суспільно, хоч індивідуально розглядувана —
припускає капітал у грошовій формі або грошовий капітал як primus
motor для кожного новопосталого підприємства, і як постійний рушій. Обіговий
капітал зокрема припускає, що через короткі переміжки постійно
знову й знов з’являється грошовий капітал як рушій. Всю авансовану
капітальну вартість, тобто всі складові частини капіталу, що складаються
з товарів, робочої сили, засобів праці й матеріялів продукції, постійно
доводиться знову й знов купувати на гроші. Те, що тут сказано про індивідуальний
капітал, має силу й щодо суспільного капіталу, який функціонує
лише в формі багатьох індивідуальних капіталів. Але, як уже показано
в І книзі, з цього зовсім не випливає, щоб поле функціонування
капіталу, маштаб продукції, навіть на капіталістичній основі, в своїх
абсолютних розмірах залежав від розміру діющого грошового капіталу.

В капітал заведено елементи продукції, що їхня здатність розширюватись,
у певних межах, не залежить від величини авансованого грошового
капіталу. При однаковій оплаті робочої сили її можна екстенсивно
або інтенсивно більше визискувати. Якщо із збільшенням визиску збільшується
грошовий капітал (тобто підвищується заробітну плату), то не
пропорційно до збільшення визиску, отже, pro tanto він зовсім не збільшується.
Продуктивно експлуатований матеріял природи — що зовсім не являє
собою елементу вартости капіталу — земля, море, руди, ліси тощо, при
більшому напруженні тієї самої кількостн робочої сили може інтенсивно
або екстенсивно більше експлуатуватись без збільшеного авансування гро-