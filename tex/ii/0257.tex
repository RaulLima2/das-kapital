велика, щоб могла вона пристосуватись до такої змінности в подовженні
і скороченні періодів обороту.

Коли далі припустимо інші незмінні умови, — а між ними незмінну
довжину, інтенсивність і продуктивність робочого дня, але змінний
розподіл новоствореної вартости між робітниками та додатковою
вартістю, так, що або перша підвищується, а друга меншає,
або, навпаки, то це не справить жодного впливу на масу грошей в
циркуляції. Така зміна може відбуватися без будь-якого збільшення або
зменшення маси грошей, що перебувають в циркуляції. Розгляньмо особливо
той випадок, коли стається загальне підвищення заробітної плати,
а тому — при вищеприпущених умовах — загальне зниження норми додаткової
вартости; при цьому — також згідно з припущенням — не відбувається
жодної зміни в вартості товарової маси, яка циркулює. В цьому
випадку, звичайно, зростає грошовий капітал, що його треба авансувати
як змінний капітал, отже, зростає маса грошей, що служить у цій
функції. Але саме настільки, наскільки зростає маса грошей, потрібних
для функції змінного капіталу, саме на стільки меншає додаткова вартість,
отже, й маса грошей, потрібних для її реалізації. На суму грошей,
потрібних для реалізації товарової вартости, це так само не справляє
жодного впливу, як і на саму цю товарову вартість. Ціна витрат*) на
товар підвищується для поодинокого капіталіста, але його суспільна ціна
продукції **) лишається незмінна. Змінюється при цьому тільки те відношення,
що в ньому — лишаючи осторонь сталу частину вартости — ціна
продукції товарів поділяється на заробітну плату й зиск.

Але, можуть сказати, більша витрата змінного грошового капіталу
(вартість грошей, звичайно, припускається за незмінну) значить те саме,
що й збільшення грошових засобів у руках робітників. Звідси випливає
підвищення попиту на товари з боку робітників. Дальший наслідок буде
підвищення цін товарів. Або можуть сказати: коли підвищується заробітна
плата, то капіталісти підвищують ціни на свої товари. В обох випадках
загальне підвищення заробітної плати спричиняється до підвищення ціни
товарів. Тому для циркуляції товарів потрібна більша маса грошей, хоч
у який спосіб пояснюватимуть підвищення цін.

Відповідь на перше міркування: в наслідок підвищення заробітної
плати підвищиться саме попит робітників на доконечні засоби існування.
Куди менше збільшиться попит їхній на речі розкошів або постане попит
на такі речі, що раніш не ввіходили в сферу їхнього споживання. Підвищення
попиту на доконечні засоби існування, що постає раптом та у
великих розмірах, безперечно, зараз же підвищить їхню ціну. Наслідок
цього буде той, що більшу частину суспільного капіталу застосовуватиметься
на продукцію доконечних засобів існування, а меншу — на про-

*) Про визначення терміну „ціна витрат“ (Kostenpreis або Kostpreis, як Маркс
вживає в книзі III (дивись „Капітал“, т. III, ч. І, розділ 1). Ред.

**) Про визначення терміну „ціна продукції“ (Produktionspreis) дивись „Капітал“,
т. III, ч. І, розділ дев’ятий. Ред.
