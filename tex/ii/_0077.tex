\parcont{}  %% абзац починається на попередній сторінці
\index{ii}{0077}  %% посилання на сторінку оригінального видання
значно меншу вартість, ніж вартість того товарового капіталу, що
його він подає.

Щодо його попиту на робочу силу, то він, щодо вартости, визначається
відношенням його змінного капіталу до його цілого капіталу,
тобто дорівнює v: С, і тому зростає в капіталістичній продукції
відносно менше, ніж його попит на засоби продукції. Капіталіст
у дедалі більшій мірі більше купує Зп, ніж купує Р.

Що робітник перетворює свою заробітну плату переважно на засоби
існування, а найбільшу частину її — на доконечні засоби існування, то попит
капіталіста на робочу силу посередньо є разом з тим попит на засоби споживання,
що ввіходять у споживання робітничої кляси. Але цей попит
дорівнює v, і він на жоден атом не є більший від v (коли робітник зоощаджує
з своєї заробітної плати — кредитові відносини ми повинні лишити
тут осторонь — то це значить, що він частину своєї заробітної плати
перетворює на скарб і pro tanto\footnote*{
Остільки, відповідно до цього. Ред.
**) Норма зиску це є відношення маси додаткової вартости до цілого авансованого
капіталу. Про це дивись: Маркс, „Капітал“, т. III, ч. І, розд. II. —
Ред.
***) Про поняття „оборот капіталу“ див. далі розділ VII. — Ред.
} вже не виступає як особа, що ставить
попит, не як покупець). Максимальна межа попиту капіталіста дорівнює
C = c + v, а його подання дорівнює c + v + m; отже, коли будова
його товарового капіталу є 80c + 20v + 20m, то попит його дорівнює
80с + 20v, отже, розглядуваний щодо його вартости, він на 1/5 менший
від його подання. Що більше відсоткове відношення спродукованої від нього
маси m (норма зиску),**) то менший стає його попит порівняно з його
поданням. Хоч попит капіталіста на робочу силу, а тому, посередньо, і
на доконечні засоби існування, з розвитком продукції дедалі меншає порівняно
з його попитом на засоби продукції, все ж, з другого боку, не
треба забувати, що його попит на Зп завжди менший, ніж його капітал,
обчислюючи з дня на день. Отже, попит його на засоби продукції
мусить завжди бути меншої вартости, ніж товаровий продукт капіталіста,
що постачає йому ці засоби продукції та працює з однаковим
капіталом і за однакових інших обставин. Та обставина, що є
багато капіталістів, а не один, справи аж ніяк не змінює. Припустімо,
що його капітал дорівнює 1000 ф. стерл., стала частина його дорівнює
800 ф. стерл. Тоді попит його до всіх капіталістів дорівнює 800 ф. стерл.,
а всі вони разом на кожні 1000 ф. стерл. (хоч скільки з цієї суми припадає
на кожного з них зокрема і хоч яку частину цілого капіталу його
становить сума, яка припадає кожному) постачають, за однакової норми
зиску, засобів продукції вартістю в 1200 ф. стерл.; отже, його попит покриває
лише 2/3 їхнього подання, тимчасом як увесь його власний попит,
розглядуваний щодо величини його вартости, дорівнює лише 4/5
його власного подання.

Тепер ми ще мусимо, забігаючи наперед, розглянути, між іншим, оборот
капіталу***). Припустімо, що ввесь капітал даного капіталіста дорівнює
\parbreak{}  %% абзац продовжується на наступній сторінці
