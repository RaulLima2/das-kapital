\parcont{}  %% абзац починається на попередній сторінці
\index{ii}{0248}  %% посилання на сторінку оригінального видання
досить, щоб постійно оплачувати робочу силу. Тут, у продукції грошей
вистачить такої самої суми; але зворотно приплилі 100 ф. стерл., що ними
кожні 5 тижнів оплачується робочу силу, є не перетворена форма продукту
цієї робочої сили, а частина цього самого постійно відновлюваного
продукту. Золотопромисловець платить своїм робітникам безпосередньо
частиною золота, що вони сами його випродукували. Тому 1000 ф. стерл.,
щорічно витрачувані таким чином на робочу силу й подавані робітниками
в циркуляцію, не повертаються через циркуляцію до свого вихідного
пункту.

Далі, щодо основного капіталу, то при першому заснуванні підприємства
треба витратити порівняно великий грошовий капітал, що його,
отже, пускається в циркуляцію. Як кожний основний капітал, він повертається
назад лише частинами протягом кількох років. Але він повертається
назад як безпосередня частина продукту, золота, не через продаж
продукту, не через перетворення його таким способом на золото. Отже,
він поступінно набирає своєї грошової форми не через вилучення грошей
з циркуляції, а через нагромадження відповідної частини продукту.
Відновлений таким чином грошовий капітал не є грошова сума, поступінно
вилучувана з циркуляції на покриття грошової суми, первісно кинутої
в циркуляцію на придбання основного капіталу. Це — додаткова
маса грошей.

Нарешті, щодо додаткової вартости, то вона так само дорівнює тій
частині нового продукту — золота, яку в кожний новий період обороту
пускається в циркуляцію, щоб, згідно з нашим припущенням, витратити
Її непродуктивно, на оплату засобів існування та речей розкошів.

Але згідно з нашим припущенням, вся ця річна продукція золота —
що нею постійно вилучається з ринку робочу силу й матеріяли продукції,
але не вилучається з нього грошей, а постійно подається додаткові гроші
— вся ця річна продукція золота тільки заміщує гроші, зношувані протягом
року, отже, лише підтримує в суспільстві сповна ту кількість грошей,
яка постійно, хоч і в змінних долях, існує в двох формах — в формі
скарбу і в формі грошей, що перебувають у циркуляції.

Згідно з законом товарової циркуляції загальна маса грошей мусить
дорівнювати масі грошей, потрібних для циркуляції, плюс кількість грошей,
що перебуває в формі скарбу, а ця остання кількість більшає або
меншає залежно від скорочення або поширення циркуляції; вона ж служить
також і для утворення потрібного резервного фонду засобів виплати.
Вартість товарів мусить сплачуватись грішми, оскільки виплати
взаємно не урівноважуються. Та обставина, що частина цієї вартости
складається з додаткової вартости, тобто нічого не коштувала продавцеві
товарів, абсолютно нічого не змінює в справі. Припустімо, що всі
продуценти є самостійні власники їхніх засобів продукції, отже, що циркуляція
відбувається між самими безпосередніми продуцентами. Коли залишити
осторонь сталу частину їхнього капіталу, то їхній річний додатковий
продукт за аналогією з капіталістичним станом, можна було б поділити
на дві частини: одну — а, що тільки заміщує потрібні засоби
\parbreak{}  %% абзац продовжується на наступній сторінці
