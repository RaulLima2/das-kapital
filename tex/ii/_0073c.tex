\index{ii}{0073}  %% посилання на сторінку оригінального видання
Процес циркуляції промислового капіталу, що становить лише частину
процесу його індивідуального кругобігу, визначається раніш розвиненими
загальними законами (книга I, розділ III), оскільки він являє лише ряд
актів у межах загальної товарової циркуляції. Та сама маса грошей, напр.,
500 ф. стерл., втягує по черзі в циркуляцію то більше промислових капіталів
(або також індивідуальних капіталів в їхній формі товарових капіталів), що
більша обігова швидкість грошей, що швидше, отже, кожний поодинокий
капітал перебігає ряд своїх товарових або грошових метаморфоз. Тому
та сама кількість капітальної вартости потребує для своєї циркуляції то
менше грошей, що більше гроші функціонують як засіб виплати, отже,
що більше, прим., при заміщенні товарового капіталу засобами його продукції,
доводиться оплачувати лише різність і що коротші строки виплати,
прим., при виплаті заробітної плати. З другого боку, коли швидкість
циркуляції та всі інші обставини дано як незмінні, то кількість грошей,
що мусить циркулювати як грошовий капітал, визначається сумою товарових
цін (ціна, помножена на кількість товарів), або, коли дано кількість
і вартість товарів, — вартісно самих грошей.

Але закони загальної товарової циркуляції мають силу лише остільки,
оскільки процес циркуляції капіталу утворює ряд актів простої циркуляції,
і не мають сили, коли ці акти становлять функціонально визначені
етапи в кругобігу індивідуальних промислових капіталів.

Щоб пояснити це, найкраще розглядати процес циркуляції в його
безперервному зв’язку, яким він з’являється в обох формах:

II) $П... Т' Т — —Г' т —  Г — Т Р Зп... П$ (П')

III) $Т' Т — —Г' т —  Г — Т Р Зп... П... Т' г — т$

Як ряд актів циркуляції взагалі, процес циркуляції (хоч є він $Т — Г — Т$,
хоч $Г — Т — Г$) являє лише два протилежні ряди товарових метаморфоз,
що з них кожна окрема метаморфоза знову таки має собі й протилежну
метаморфозу на боці чужого товару або чужих грошей, що протистоять
даному товарові.

Те, що з боку товаровласника $Т — Г$, з боку покупця є $Г — Т$; перша
метаморфоза товару в $Т — Г$е друга метаморфоза товару, що виступає
як Г; в формулі $Г — Т$ справа стоїть навпаки. Отже, все, що сказано
про те, як метаморфоза товару на одній стадії переплітається з метаморфозою
товару на другій стадії, має силу для циркуляції капіталу,
оскільки капіталіст виконує функції покупця і продавця товарів, і
оскільки в наслідок цього його капітал функціонує як гроші проти
чужих товарів або як товар проти чужих грошей. Але це переплітання
метаморфоз не є разом з тим вираз для переплітання метаморфоз капіталів.
