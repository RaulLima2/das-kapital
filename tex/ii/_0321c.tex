\index{ii}{0321}  %% посилання на сторінку оригінального видання
З другого боку, та натуральна форма, що на неї мусить перетворитись
змінний капітал, який існує в грошовій формі, — тобто робоча сила, —
в наслідок споживання зберігається, репродукується і знову є наявна,
як той єдиний предмет торговлі її посідачів, що його вони мусять продавати,
коли хочуть жити. Отже, репродукується й відношення між найманими
робітниками й капіталістами.

2) Сталий капітал II заміщено in natura, і 500 ф. стерл., авансовані
цим підрозділом II для циркуляції, позернулися до нього назад.

Для робітників І циркуляція є проста циркуляція $Т — Г — Т$. 1/Т (робоча
сила) — 2/Г (1000 ф. стерл., грошова форма змінного капіталу І) — 3/Т
(доконечні засоби існування в сумі 1000 ф. стерл.); ці 1000 ф. стерл.
перетворюють на гроші на таку саму величину вартости сталий капітал II,
який існує у формі товару — засобів існування.

Для капіталістів II цей процес є $Т — Г$, перетворення частини їхнього
товарового продукту на грошову форму, що з неї він перетворюється
знову на елементи продуктивного капіталу, а саме на частину потрібних
цим капіталістам засобів продукції.

Авансуючи Г (500 ф. стерл.) на закуп другої частини засобів продукції,
капіталісти II антиципують грошову форму тієї частини II с, яка
ще існує в товаровій формі (у формі засобів споживання); в акті $Г — Т$,
коли II купує за Г, а І продає Т, гроші (II) перетворюються на частину
продуктивного капіталу, тимчасом як Т (І) пророблює акт $Т — Г$, перетворюється
на гроші; але ці гроші репрезентують для І не складову частину
капітальної вартости, а перетворену на гроші додаткову вартість, що її
витрачається лише на засоби споживання.

В циркуляції $Г — Т\dots{} П\dots{} Т' — Г' п$ерший акт, $Г — Т$, є акт одного капіталіста,
останній, $Т — Г'$, є акт (або частина акту) іншого. Чи репрезентує
це Т, що за допомогою його $Г п$еретворюється на продуктивний капітал,
для продавця Т (який, отже, перетворює це Т на гроші) складову частину
сталого капіталу, чи складову частину змінного капіталу, чи додаткову
вартість, це не має жодного значення для самої товарової циркуляції.

Що стосується до кляси І, щодо складової частини $v + m$ її товарового
продукту, то ця кляса вилучила з циркуляції більше грошей, ніж подала
в неї. Поперше, до неї повертаються 1000 ф. стерл. її змінного капіталу;
подруге, вона продає (див. вище, обмін під № 4) засобів продукції
на 500 ф. стерл., у наслідок цього перетворюється на гроші половина
її додаткової вартости; потім (обмін під № 6) вона знову продає засобів
продукції на 500 ф. стерл., другу половину своєї додаткової вартости,
і в наслідок цього всю додаткову вартість вилучено з циркуляції в грошовій
формі. Отже, маємо послідовно: 1) змінний капітал перетворюється
знову на гроші = 1000 ф. стерл.; 2) половина додаткової вартости перетворюється
на гроші = 500 ф. стерл.; 3) друга половина додаткової
вартости = 500 ф. стерл.; отже, перетворено на гроші суму
$1000 v + 1000 m$ = 2000 ф. стерл. Хоч І (лишаючи осторонь обміни,
що їх розглянеться потім, і що упосереднюють репродукцію І с) подав
\parbreak{}  %% абзац продовжується на наступній сторінці
