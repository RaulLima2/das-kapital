\parcont{}  %% абзац починається на попередній сторінці
\index{ii}{0403}  %% посилання на сторінку оригінального видання
капіталу, потрібного для акумуляції в межах II, цілком так само, як II
повинен дати матеріял для змінного капіталу, що має пустити в рух ту
частину додаткового продукту І, яку сам І застосовує як додатковий
сталий капітал. Ми знаємо, що справжній змінний капітал, а значить, і
додатковий, складається з робочої сили. Тепер не капіталіст І купує в II
доконечні засоби існування про запас і зберігає їх для додаткової робочої
сили, що йому її треба буде застосувати в майбутньому, як це
мусив робити рабовласник. Сами робітники мають справу з II. Та це не
заваджає тому, що з погляду капіталіста засоби споживання додаткової
робочої сили є лише засоби продукції та зберігання його евентуальної
додаткової робочої сили, отже, натуральна форма його змінного капіталу.
Його власна найближча операція, в даному разі виконувана І, сходить
лише на те, що він нагромаджує новий грошовий капітал, потрібний для
закупу додаткової робочої сили. Скоро тільки він долучає її до свого
капіталу, гроші стають для цієї робочої сили засобом до купівлі
товарів II, отже, вона мусить знайти засоби свого споживання.

Між іншим. Пан капіталіст, як і його преса, часто бувають незадоволені
з того способу, яким робоча сила витрачає свої гроші, і з
тих товарів II, що в них вона реалізує ці гроші. Він філософує з цього
приводу, розводить теревені про культуру та удає філантропа, як це,
прим., робить п. Друммонд, секретар англійського посольства в Вашінґтоні.
„The Nation“ (газета) наприкінці жовтня 1879 р. вмістила цікаву
статтю, де, між іншим, сказано: „Робітники відстали в культурі від поступу
в винаходах; для них стала приступна маса речей, що їх вживати
вони не вміють, і що для них вони, отже, не являють ринку“.
(Кожен капіталіст, звичайно, хоче, щоб робітник купував його товар).
„Немає жодної підстави для того, щоб робітник не хотів жити з таким
самим комфортом, як піп, адвокат або лікар, що одержує стільки ж, як
і він“. (Справді багатенько комфорту можуть дозволити собі з свого бажання
такі адвокати, попи й лікарі!). „Але він цього не робить. Питання
все ще в тому, якими раціональними здоровими заходами підвищити його
рівень як споживача; це питання не легке, бо все його шанолюбство не
йде далі скорочення робочих годин, і демагоги радше підбурюють його
до цього, ніж до поліпшення його стану через удосконалення його розумових
і моральних здібностей“. (Reports of Н. M-s Secretaries of Embassy
and Legation on the Manufactures, Commerce etc. of the Countries in
which they reside. London 1879, p. 404).

Довгий робочий день являє, певно, таємницю раціональних і здорових
заходів, що повинні поліпшити стан робітника, удосконалюючи його розумові
й моральні здібності, та зробити з нього раціонального споживача.
Щоб стати раціональним споживачем товарів капіталістів він мусить
насамперед почати — але цьому заваджає демагог! — з того, щоб дозволити
своєму власному капіталістові споживати його робочу силу
нераціональним і шкідливим для здоров’я способом. Як розуміє
капіталіст раціональне споживання, це виявляється там, де ласка капіталіста
доходить того, що він безпосередньо береться до торговлі
\index{ii}{0404}  %% посилання на сторінку оригінального видання
засобами споживання своїх робітників, — в truck-system’i, що одною з
багатьох галузей її є винаймання помешкань робітникам, так що капіталіст
є разом з тим квартировласник своїх робітників.

Той самий Друммонд, що його прекрасна душа мріє про капіталістичні
спроби поліпшити стан робітничої кляси, в тому самому звіті розповідає,
між іншим, про зразкові бавовнопрядільні Lowell and Lawrence Mills,
їдальні й помешкання робітниць належать акційному товариству, якому
належать і сами фабрики; завідательки цих будинків служать у того
самого товариства, яке приписує їм правила поведінки; жодна робітниця
не сміє повертатись додому пізніше, ніж о 10-ій годині вечора. Але
ось перл: патрулі спеціяльної поліції товариства доглядають у навкольності,
щоб не порушувалось цей житловий порядок. Після 10-ої години
вечора жодну робітницю не випускають з будинку й не впускають туди.
Жодна робітниця не сміє мешкати десь інде, крім території, що належить
товариству; кожний будинок дає товариству щотижня приблизно 10 доларів
плати за помешкання; і тут ми бачимо раціональних споживачів у всій
славі. „Що в багатьох кращих будинках для робітниць є повсюди піяніно,
то музика, співи й танці відіграють значну ролю, принаймні для тих з
них, кому після нудної безперервної десятигодинної праці біля ткацького
варстату, більш потрібна переміна, ніж справжній відпочинок“ (стор. 412).
Але головна таємниця, як з робітника зробити раціонального споживача,
ще далі. Пан Друммонд завітав на фабрику ножів Turners Falls (Connecticut
River), і пан Окмен, скарбник акційного товариства, розповівши
йому, що американські столові ножі якістю переважають англійські, додав:
„Ми поб’ємо Англію і щодо цін; ми вже й тепер перевищуємо її якістю,
це визнано; але ми мусимо мати й дешевші ціни, і ми досягнемо цього
так само, як ми одержали дешевше нашу сталь і знизили плату за нашу
працю!“ (стор. 427). Зниження заробітної плати й довгий робочий день —
у цьому вся суть раціональних і здорових заходів, що повинні піднести
робітника до ранґу раціональоного споживача, щоб він утворив ринок
для маси предметів, зроблених для нього приступними культурою та поступом
у винаходах.

Отже, як І повинен дати з свого додаткового продукту додатковий
сталий капітал для II, так II дає в цьому розумінні додатковий змінний
капітал для І. Оскільки ходить про змінний капітал, II акумулює для І і
для себе самого, репродукуючи більшу частину всього свого продукту,
отже, і свого додаткового продукту, в формі доконечних засобів споживання.
При продукції на дедалі більшій капітальній базі І (v + m) мусить
дорівнювати II с плюс та частина додаткового продукту, яка знову долучається
до капіталу, плюс додаткова частина сталого капіталу, потрібна
для поширення продукції в II; а мінімум цього поширення такий, що
без нього не можлива справжня акумуляція, тобто справжнє поширення
продукції в самому І.
88
