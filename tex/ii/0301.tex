дукції від процесу репродукції індивідуального капіталу, і які риси
спільні їм обом. Річний продукт охоплює так ті частини суспільного
продукту, які заміщують капітал, суспільну репродукцію, як і ті частини,
що входять у фонд споживання, що їх споживають робітники й капіталісти,
отже, охоплює так продуктивне, як і особисте споживання. Воно
охоплює також і репродукцію (тобто зберігання) кляси капіталістів і
робітничої кляси, а тому й репродукцію капіталістичного характеру сукупного
процесу продукції.

Зрозуміло, що нам треба аналізувати формулу циркуляції
Т' — (Г — Т... П... Т') г — т, при чому споживання неодмінно відіграє в ній
певну ролю; бо вихідний пункт Т' = Т + т, товаровий капітал, має в
собі так сталу і змінну капітальну вартість, як і додаткову вартість.
Тому його рух охоплює й особисте й продуктивне споживання. В кругобігах
Г — Т... П... Т' — Г' і П... Т' — Г' — Т... П вихідний і кінцевий
пункт є рух капіталу. Правда, це включає і споживання, бо товар,
продукт, треба продати. Але коли припускається, що цього вже досягнуто,
то для руху поодинокого капіталу буде байдуже, що далі зробиться
з цим товаром. Навпаки, в русі Т'... Т' умови суспільної репродукції
виявляються саме в тому, що тут треба показати, що зробиться з кожною
частиною вартости цього сукупного продукту Т'. Сукупний процес
репродукції тут так само включає процес споживання, упосереднюваний
циркуляцією, як і власне процес репродукції капіталу.

Маючи на увазі мету нашу, ми повинні розглянути процес репродукції
з погляду заміщення так вартости, як і речовини поодиноких складових
частин Т'. Тепер нам уже не досить, як то було при аналізі вартости
продукту поодинокого капіталу, припустити, що поодинокий
капіталіст, через продаж свого товарового продукту, може спочатку перетворити
складові частини свого капіталу на гроші, а потім, знову купуючи
на товаровому ринку елементи продукції, перетворити знову ці
складові частини на продуктивний капітал. Ці елементи продукції, оскільки
вони мають речовий характер, так само становлять складову частину
суспільного капіталу, як і індивідуальний готовий продукт, обмінюваний
на них і заміщуваний ними. З другого боку, рух тієї частини суспільного
товарового продукту, що її споживає робітник, витрачаючи свою
заробітну плату, і капіталіст, витрачаючи додаткову вартість, становить не
лише складову ланку руху цілого продукту, а й переплітається з рухом
індивідуальних капіталів, і тому цього процесу не можна пояснити тим,
що його просто припускають.

Питання, що безпосередньо постає перед нами, таке: як капітал,
спожитий в продукції, заміщується щодо вартости своєї з річного продукту
й як процес цього заміщення переплітається із споживанням додаткової
вартости капіталістами й заробітної плати робітниками? Отже,
насамперед йдеться про репродукцію в простому маштабі. Далі припускається
не лише те, що продукти обмінюється за їхньою вартістю, а й
