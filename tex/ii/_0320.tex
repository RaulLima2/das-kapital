\parcont{}  %% абзац починається на попередній сторінці
\index{ii}{0320}  %% посилання на сторінку оригінального видання
засобів споживання на таку саму суму; в наслідок цього ці 500 ф. стерл.
припливають назад до II; капіталісти цього підрозділу тепер, як і раніше,
мають 500 ф. стерл. в грошах і 2000 ф. стерл. в сталому капіталі,
однак, останній знову перетворено з форми товарового капіталу на продуктивний
капітал. Циркуляція маси товарів на 5000 ф. стерл. відбулась
за посередництвом 1500 ф. стерл. грошей, а саме:\footnote{
І платить 1000 ф. стерл. грішми за робочу силу, отже, за товар
= 1000 ф. стерл.
} І виплачує робітникам
1000 ф. стерл. за робочу силу такої самої величини вартости;\footnote{
Робітники на свою заробітну плату в сумі 1000 ф. стерл. грішми
купують засоби споживання в II; отже, товар = 1000 ф. стерл.
} робітники на ці 1000 ф. стерл. купують у II засоби існування;\footnote{
II на вторговані від робітників 1000 ф. стерл. купує в І засоби
продукції такої ж вартости; отже, товар = 1000 ф. стерл.

В наслідок цього до І повернулись 1000 ф. стерл. в грошах як грошова
форма змінного капіталу.
} II на ті самі гроші купує засоби продукції в І, що в нього таким
чином відновлюється в грошовій формі змінний капітал в 1000 ф. стерл.;\footnote{
II купує в І на 500 ф. стерл. засоби продукції, тобто товар =
500 ф. стерл.
} II купує на 500 ф. стерл. засоби продукції у І;\footnote{
І купує на ці самі 500 ф. стерл. засоби споживання у II; отже,
товар = 500 ф. стерл.
} І купує на ці
самі 500 ф. стерл. засоби споживання у II;\footnote{
II купує на ці самі 500 ф. стерл. засоби продукції в І, отже,
товар = 500 ф. стерл.
} II купує на ті самі 500 ф.
стерл. засоби продукції у І;\footnote{
І купує на ті самі 500 ф. стерл. засоби споживання в II; отже,
товар = 500 ф. стерл.

Сума обмінених товарових вартостей = 5000 ф. стерл.
} І купує на ті самі 500 ф. стерл. засоби
існування у II. До II повернулись назад 500 ф. стерл., що їх він подав у
циркуляцію понад 2000 ф. стерл. у своєму товарі й що за них він не
вилучив з циркуляції жодного еквіваленту в товарі\footnote{
Тут виклад трохи відхиляється від вище поданого (стор. 306—307). Там і
І підрозділ подав в циркуляцію додаткову суму в 500. Тут тільки II підрозділ дає
додатковий грошовий матеріял для циркуляції. Однак, це нічого не змінює в
кінцевому наслідку. — Ф. Е.
}.

Отже, обмін відбувається так:

500 ф. стерл., що їх II підрозділ авансував на купівлю, повернулись
до нього назад.

Результат такий:

І) І підрозділ має змінний капітал в грошовій формі величиною в 1000 ф.
стерл., що їх він первісно авансував для циркуляції; крім того, він витратив
на своє особисте споживання 1000 ф. стерл. у своєму власному
товаровому продукті; тобто витратив ті гроші, що їх він одержав від продажу
засобів продукції вартістю в 1000 ф. стерл.
