вання, і на (200 v + 200 m) І, призначених для b, = 400 засобів продукції
речей розкошів.

Чимала частина не лише власне засобів праці, а й сировинних та
допоміжних матеріялів тощо, однорідна в обох підрозділах. Але щодо
обміну різних частин вартости цілого продукту I (v + m), то цей поділ
на підрозділи не має жодного значення. Так згадані вище 800 I v, як і
200 I v реалізується в наслідок того, що заробітну плату витрачається
на засоби споживання 1000 ІІ с, отже, грошовий капітал, авансований на
неї, повертаючись, розподіляється рівномірно між капіталістами продуцентами
I і pro rata заміщує в грошах авансований ними змінний капітал:
з другого боку, щодо реалізації 1000 I m, то і тут капіталісти рівномірно
(пропорційно величині їхнього m) візьмуть засоби споживання з
усієї другої половини ІІ с = 1000: 600 II а і 400 II b; отже, ті, що заміщують
стали; капітал II а:

480 (3/5) з 600 с (II а) і 320 (2/5*) з 400 с (II b) = 800; ті, що заміщують
сталий капітал II b:

120 (3/5) від 600 с (II а) і 80 (2/5) від 400 с (ІІ b) = 200. Сума = 1000.

Що тут узято довільно і для І і для II, так це — відношення змінного
капіталу до сталого, а також однаковість цього відношення в
І і в II і в їхніх підвідділах. Щодо цієї однаковости, то її тут припущено
лише для спрощення; припущення різних пропорцій абсолютно нічого
не змінило б в умовах проблеми та в її розв’язанні. Але, коли припустити
просту репродукцію, то як доконечний результат з цього випливає таке:

1) Нова вартість, утворена річною працею в натуральній формі засобів
продукції (яка розпадається на v + m), дорівнює репродукованій
у формі засобів споживання сталій капітальній вартості с від вартости
продукту, утвореного другою частиною річної праці. Коли б ця нова вартість
була менша за II с, то II не міг би повнотою замістити свій сталий капітал;
коли б вона була більша, то надлишок не використовувалось би. В
обох випадках порушувалось би наше припущення простої репродукції.

2) Щодо річного продукту, репродукованого в формі засобів споживання,
то змінний капітал v, авансований в грошовій формі, можуть реалізувати
його одержувачі — оскільки вони є робітники, що продукують
речі розкошів — лише в тій частині доконечних засобів існування, що в
ній prima facie втілено додаткову вартість капіталістичних продуцентів
цих засобів; отже, v, витрачене на продукцію речей розкошів, розмірами
своєї вартости дорівнює відповідній частині m, спродукованій у
формі доконечних засобів існування, отже, воно мусить бути менше, ніж
усе це m — а саме (II а) m — і лише через реалізацію цього v в тій частині
m до капіталістичних продуцентів речей розкошів повертається в грошовій
формі авансований ними змінний капітал. Це явище цілком аналогічне
до реалізації I (v + m) в ІІ с; ріжниця лише та, що в другому випадку
(IІ b) v реалізується в частині (I Iа) m, яка величиною вартости до-

*) Зазначені тут у дужках дроби 3/5 і 2/5 є частина від усієї другої половини
сталого капіталу II а, тобто від 800, так само, як при II b вони є частини від усієї
другої половини сталого капіталу II b, тобто від 200. Peд
