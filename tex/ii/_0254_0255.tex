\parcont{}  %% абзац починається на попередній сторінці
\index{ii}{0254}  %% посилання на сторінку оригінального видання
циркуляцію, а вилучає з циркуляції 6000 ф. стерл.: 5000 ф. стерл. капіталу й
1000 ф. стерл. додаткової вартости. Ці 1000 ф. стерл. перетворились на
гроші за допомогою тих грошей, що їх вона сама пустила в циркуляцію
не як капіталіст, а як споживач, що їх вона не авансувала, а витратила.
Тепер вони повертаються до неї знову як грошова форма спродукованої
нею додаткової вартости. І з цього часу ця операція повторюється
щороку. Але, починаючи з другого року, 1000 ф. стерл., витрачувані нею,
завжди є вже перетворена форма, грошова форма спродукованої нею
додаткової вартости. Вона витрачає їх щороку і щороку ж вони повертаються
до неї назад.

Коли б капітал цього капіталіста обертався протягом року частіше,
то справа від цього ані трохи не змінилась би, але, звичайно, змінився
б протяг часу, а тому й величина тієї грошової суми, що її капіталістові
понад авансований ним грошовий капітал довелось би пускати в
циркуляцію на своє особисте споживання.

Ці гроші капіталіст пускає в циркуляцію не як капітал. Але, зрозуміло,
така вже властивість капіталіста, що доки до нього повернеться з
циркуляції додаткова вартість, він може існувати на ті засоби, які є в
його посіданні.

В цьому випадку ми припускали, що грошова сума, яку капіталіст
пускає в циркуляцію на покриття свого особистого споживання до першого
зворотного припливу його капіталу, точно дорівнює спродукованій ним
додатковій вартості, і тому має бути перетворена на гроші. Очевидно,
що таке припущення відносно поодинокого капіталіста довільне. Але
воно мусить бути правильне для цілої кляси капіталістів, якщо ми
припускаємо просту репродукцію. Воно виражає лише те, що є в цьому
припущенні, а саме, що всю додаткову вартість, але і тільки її, споживається
непродуктивно; що, отже, жодної частини первісного капіталу
не споживається непродуктивно.

Ми вище припускали, що цілої продукції благородних металів
(припустімо = 500 ф. стерл.) досить лише для того, щоб покрити зношування
грошей.

Капіталісти, що продукують золото, мають увесь свій продукт у
формі золота, так ті частини його, що покривають сталий і змінний
капітал, як і ту частину його, яка складається з додаткової вартости.
Отже, частина суспільної додаткової вартости складається з золота, а
не з такого продукту, що перетворюється на золото лише в циркуляції.
Вона з самого початку складається з золота, й пускається її в циркуляцію
для того, щоб вилучити з циркуляції продукти. Це саме тут стосується
до заробітної плати, змінного капіталу і до покриття авансованого
сталого капіталу. Отже, коли одна частина кляси капіталістів пускає в
циркуляцію товарову вартість більшу (на величину додаткової вартости),
ніж авансований ними грошовий капітал, то друга частина капіталістів
пускає в циркуляцію більшу грошову вартість (більшу на додаткову
вартість), ніж товарова вартість, що її вони постійно вилучають з циркуляції
для продукції золота. Якщо частина капіталістів постійно випомповує
\index{ii}{0255}  %% посилання на сторінку оригінального видання
з циркуляції більше грошей, ніж подає в неї, то частина капіталістів,
— та, що продукує золото — постійно напомповує в неї більше
грошей, ніж вилучає з неї в засобах продукції.

Хоч частина цього продукту, золота, в 500 ф. стерл., є додаткова
вартість продуцентів золота, однак цілу цю суму призначається лише на
заміщення грошей, потрібних для циркуляції товарів; при цьому байдуже,
скільки з цієї суми йде на перетворення на гроші додаткової вартости
товарів і скільки на перетворення на гроші інших складових частин
вартости товарів.

Якщо продукцію золота перенести з даної країни в інші країни, то
це нічого не змінює в справі. Частину суспільної робочої сили й суспільних
засобів продукції в країні \emph{А} перетворено на продукт, прим., на
полотно, вартістю в 500 ф. стерл., що його вивозиться в країну \emph{В}, щоб
там купити золото. Продуктивний капітал, застосований таким чином у
країні \emph{А}, так само не подає на ринок країни \emph{А} товарів — на відміну
від грошей — як коли б його безпосередньо застосовувалось на продукцію
золота. Цей продукт \emph{А} репрезентовано в 500 ф. золота, і він
надходить в циркуляцію країни \emph{А} лише як гроші. Частина суспільної
додаткової вартости, що є в цьому продукті, існує безпосередньо як
гроші, і для країни А ніколи не існує інакше, як у формі грошей. Хоч
для капіталістів, що продукують золото, лише частина продукту репрезентує
додаткову вартість, а друга частина — покриття капіталу, однак питання
про те, яка кількість цього золота покриває, крім обігового сталого
капіталу, змінний капітал, і яка кількість репрезентує додаткову
вартість залежить виключно від тих відповідних відношень,
що в них заробітна плата й додаткова вартість перебувають до
вартости товарів, що циркулюють. Частина, що становить додаткову
вартість, розподіляється між різними членами кляси капіталістів. Хоч
вони постійно витрачають її на особисте споживання й знов одержують
її через продаж нового продукту, — саме ця купівля й продаж взагалі
лише і зумовлює циркуляцію між ними грошей, потрібних для перетворення
на гроші додаткової вартости, — однак деяка частина суспільної
додаткової вартости, хоч і в змінних кількостях, перебуває в формі
грошей в кишені капіталістів, цілком так само, як частина заробітної
плати, принаймні протягом кількох днів тижня, затримується в формі грошей
в кишенях робітників. І ця частина не обмежена тією частиною грошового
продукту, що первісно становила додаткову вартість капіталістів, які
продукують золото; як сказано, вона обмежена тією пропорцією, що в
ній вищеназваний продукт в 500 ф. стерл. взагалі розподіляється між
капіталістами і робітниками, і що в ній запас товарів, призначених для
циркуляції, складається з додаткової вартости та з інших складових частин
вартости.

А проте, частина додаткової вартости, яка існує не в інших товарах,
а в грошах поряд цих інших товарів, лише остільки складається з частини
щорічно продукованого золота, оскільки частина річної продукції золота
йде в циркуляцію для реалізації додаткової вартости. Друга частина
\parbreak{}  %% абзац продовжується на наступній сторінці
