\parcont{}  %% абзац починається на попередній сторінці
\index{ii}{0020}  %% посилання на сторінку оригінального видання
грошовий капітал, воно виражається як вартість, що сама з себе зросла
в своїй вартості, отже, як вартість, що має властивість зростати в своїй
вартості, породжувати більше вартости, ніж вона сама має. Г стало капіталом
у наслідок свого відношення до другої частини Г', як частини,
утвореної ним, посталої з нього як причини, — через своє відношення до
неї, як до наслідку, що Г є його причина. Таким чином Г' з’являється
як сума вартости, що сама в собі диференційована, що в собі самій функціонально
(в понятті) себе відрізняє, — сума вартости, що виражає капіталістичне
відношення.

Але це тільки виражено як результат, без посередництва того процесу,
що зумовив цей результат.

Частини вартости як такі не відрізняються якісно між себе за винятком
того, коли вони виступають як вартості різних предметів, конкретних
речей, отже, в різних споживних формах, а, значить і, як вартості
різних товарових тіл — ріжниця, що постає не з них самих як простих
частин вартости. У грошах згасає всяка відмінність товарів, бо вони є
саме для всіх їх спільна еквівалентна форма. Грошова сума в 500 ф.
стерл. складається цілком з однойменних елементів по 1 ф. стерл. А що
в простому бутті цієї грошової суми зникла посередня ланка її походження
і зник будь-який слід специфічної ріжниці, що її мають різні
складові частини капіталу в продукційному процесі, то ріжниця існує
між головною сумою (англійською мовою — principal), що дорівнює
авансованому капіталові в 422 ф. стерл., і надлишковою сумою вартости
в 78 ф. стерл. лише в понятті. Припустімо, напр., що Г' = 110 ф. стерл.,
і з них 100 = Г, головній сумі, а 10 = М, додатковій вартості. Тут
є абсолютна однорідність, отже, безвідмінність у поняттях, між обома
складовими частинами суми в 110 ф. стерл. Які завгодно 10 ф. стерл.
є завжди 1/11 цілої суми в 110 ф. стерл., все одно, чи вони є 1/10 авансованої
головної суми в 100 ф. стерл., чи надлишок над нею в 10 ф. стерл.
Тому головна сума й приростова сума, капітал і додаткова сума, можуть
бути виражені як дробові частини цілої суми; в нашому прикладі 10/11 становлять
головну суму або капітал, 1/11 становить додаткову суму. Отже, це є
іраціональний вираз капіталістичного відношення, що в ньому тут наприкінці
свого процесу реалізований капітал з’язляється в своєму грошовому виразі.

Звичайно це має силу й щодо Т' (= Т + т). Але з тією ріжницею, що
Т', в якому Т і т являють теж лише пропорційні частини вартости тієї
самої однорідної товарової маси, вказує на своє походження з П, що Т'
безпосередньо є його продукт, тимчасом як в Г', формі, що походить
безпосередньо з циркуляції, зник прямий зв’язок з П.

Ця іраціональна ріжниця між головною і приростовою сумою, — ріжниця,
яка є в Г', оскільки воно виражає результат руху Г... Г', відразу зникає,
скоро Г' знову активно починає функціонувати як грошовий капітал, отже,
коли воно не фіксується як грошовий вираз промислового капіталу, вирослого
\index{ii}{0021}  %% посилання на сторінку оригінального видання
в своїй вартості. Кругобіг грошового капіталу ніколи не може
починатися з Г' (хоч Г' тепер функціонує як Г), а тільки з Г; тобто
він ніколи не може починатись як вираз капіталістичного відношення, а
лише як форма авансування капітальної вартости. Скоро тільки 500 ф.
стерл. авансовано знову як капітал, для нового самозростання, вони
являють вихідний пункт замість кінцевого пункту. Замість капіталу в
422 ф. стерл. тепер авансовано капітал в 500 ф. стерл. — більше
грошей, ніж раніш, більше капітальної вартости, але відношення між
двома складовими частинами відпало цілком так само, як і первісно
могла б функціонувати як капітал сума в 500 ф. стерл. замість
422 ф. стерл.

З’являтись як Г' не є активна функція грошового капіталу; його
власне з’явлення в формі Г' є скорше функція Т'. Вже в простій товаровій
циркуляції: 1) Т1 — Г, 2) Г — Т2, Г активно функціонує лише в
другому акті Г — Т2; з’явлення його у вигляді Г є лише результат першого
акту, що лише силою його воно виступає як перетворена форма
Т1. Капіталістичне відношення, що міститься в Г', відношення однієї його
частини як капітальної вартости до другої його частини як приросту
цієї вартости, набуває, правда, функціонального значіння остільки, оскільки,
при постійному повторенні кругобігу Г... Г', Г' розподіляється між двома
циркуляціями — циркуляцією капіталу і циркуляцією додаткової вартости,
отже, обидві частини виконують не лише кількісно, але і якісно
різні функції, Г інші функції, ніж г. Але розглядувана сама по собі форма
Г... Г' не має в собі споживання капіталіста, а має лише виразно
самозростання й акумуляцію, оскільки остання виражається насамперед
у періодичному прирості знову та знову авансовуваного грошового
капіталу.

Хоч Г' = Г + г і є іраціональна форма капіталу, але разом з тим воно
є грошовий капітал в його реалізованій формі, як гроші, що породили
гроші. Але в цьому треба вбачати ріжницю від функції грошового
капіталу в першій стадії Г — Т Р Зп. В цій першій стадії Г циркулює
як гроші. Воно функціонує як грошовий капітал тільки тому, що
лише в своєму грошовому стані воно може виконувати функцію грошей,
перетворитись на елементи П, Р і Зп, що протистоять йому як товари.
В цьому акті циркуляції Г функціонує лише як гроші; а що цей акт
становить першу стадію капітальної вартости, що процесує, то він одночасно
є і функція грошового капіталу в наслідок специфічної споживної
форми купованих тут товарів Р і Зп. Навпаки, Г', що складається з Г, капітальної
вартости, і г, виробленої нею додаткової вартости, виражає вирослу
капітальну вартість, — мету й результат, функцію цілого процесу
кругобігу капіталу. Те, що воно виражає цей результат у грошовій формі,
як реалізований грошовий капітал, випливає не з того, що воно є грошова
форма капіталу, грошовий капітал, але, навпаки, з того, що воно є
грошовий капітал, капітал у грошовій формі, з того, що капітал у цій
\parbreak{}  %% абзац продовжується на наступній сторінці
