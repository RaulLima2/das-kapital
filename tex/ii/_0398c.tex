\index{ii}{0398}  %% посилання на сторінку оригінального видання
Вилучати з циркуляції ці гроші, щоб утворювати віртуально додатковий
грошовий капітал, можна, здається, лише двома способами. Або
одна частина капіталістів II обдурює другу й таким чином грабує гроші.
Як ми знаємо, для утворення нового грошового капіталу не треба жодного
попереднього збільшення засобів циркуляції; для цього не треба
нічого іншого, а тільки те, щоб гроші з певних сторін вилучалися з
циркуляції та нагромаджувались як скарб. Що гроші можна украсти і
тому утворення додаткового грошового капіталу в однієї частини капіталістів
II може сполучатися з позитивною втратою грошей у другої частини,
— це зовсім не мало б значення для справи. Обдуреній частині
капіталістів II довелось би трохи менш розкошувати, та й годі.

Або ж частина II $m$, що втілюється в доконечних засобах існування,
безпосередньо перетворюється на новий змінний капітал в межах підрозділу
II. Як це стається, ми дослідимо наприкінці цього розділу
(№ IV).

\emph{1) Перший приклад}

\begin{center}
\so{А) Схема простої репродукції}

$
 \text{Схема а)} \left.\begin{aligned}
        \text{I. }4000 с + 1000 v + 1000 m = 6000\\
        \text{II. }2000 с + \phantom{0}500 v + \phantom{0}500 m = 3000
       \end{aligned}
 \right\}
   \qquad \text{Сума = 9000.}

$
\end{center}

\begin{center}
\so{В) Вихідна схема для акумуляції в поширеному маштабі}

$
 \text{Схема а)} \left.\begin{aligned}
        \text{I. }4000 с + 1000 v + 1000 m = 6000\\
        \text{II. }1500 с + \phantom{0}750 v + \phantom{0}750 m = 3000
       \end{aligned}
 \right\}
  \qquad \text{Сума = 9000.}
$
\end{center}

Припустивши, що в схемі В акумулюється половина додаткової вартости
І, тобто 500, ми насамперед матимемо, що ($1000 v + 500 m$) І,
або 1500 І ($v + m$), мусять заміститись через 1500 II $с$; в І лишається
тоді $4000 с + 500 m$, що з них останні мають бути акумульовані.
Заміщення ($1000 v + 500 m$) І через 1500 II $c$ є процес простої репродукції,
і ми пояснили його, досліджуючи її.

Припустімо, що 400 з 500 І $m$ мають бути перетворені на сталий
капітал, 100 — на змінний. Переміщення в межах І підрозділу цих $400 m$,
що таким чином мають бути капіталізовані, ми вже розглянули; отже,
ці $400 m$ без жодних вагань можна прилучити до І $с$, і тоді матимемо
для І:

$4400 с + 1000 v + 100 m$ (що їх треба обміняти на $100 v$).

II з свого боку купує в I з метою акумуляції ці 100 І $m$ (які існують
в засобах продукції), що становлять тепер додатковий сталий капітал II,
тимчасом як 100 грішми, виплачені ним за ці 100 І $m$, перетворюються
на грошову форму додаткового змінного капіталу І. Тоді для І маємо
капітал в $4400 с + 1100 v$ (останні в грошах) = 5500.

II мав тепер для сталого капіталу $1600 с$; щоб його обробити, він
мусить додати ще $50 v$ грішми на закуп нової робочої сили, так що
\parbreak{}  %% абзац продовжується на наступній сторінці
