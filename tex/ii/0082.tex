дукує споживні вартості і сам зростає вартістю, а тому функціонує як
продуктивний капітал, хоч цей час продукції має в собі й той час, коли
капітал або перебуває в латентному стані, або продукує продукти, не
зростаючи своєю вартістю.

В сфері циркуляції капітал перебуває як товаровий капітал і грошовий
капітал. Обидва його процеси циркуляції в тому, що він перетворюється
з товарової форми на грошову та з грошової форми на товарову. Та
обставина, що перетворення товару на гроші є тут разом з тим реалізація
додаткової вартости, вміщеної в товарі, і що перетворення грошей
на товар є разом з тим перетворення або зворотне перетворення капітальної
вартости на форму елементів її продукції, нічого не змінює в
тому, що ці процеси, як процеси циркуляції, є процеси простої метаморфози
товарів.

Час обігу і час продукції навзаєм виключають один одного. Протягом
часу свого обігу капітал не функціонує як продуктивний капітал і тому
не продукує ні товару, ні додаткової вартости. Якщо ми розглядаємо
кругобіг у найпростішій формі, коли вся капітальна вартість кожного разу
одним заходом переходить з однієї фази в іншу, то очевидно, що процес
продукції, а, значить, і самозростання капіталу переривається доти, доки
триває час його обігу, і що залежно від протягу останнього процес продукції
буде відновлюватися швидше або повільніше. Навпаки, коли різні
частини капіталу пророблюють кругобіг одна по одній, так що кругобіг
цілої капітальної вартости здійснюється послідовно в кругобігу її
різних частин, то очевидно, що як довше аліквотні частини капітальної
вартости постійно перебувають у сфері циркуляції, то менша
мусить бути та її частина, яка завжди функціонує в сфері
продукції. Тому збільшення або скорочення часу обігу впливає тут як
негативна межа для скорочення або збільшення часу продукції або тих
розмірів, що в них капітал даної величини функціонує як продуктивний
капітал. Що більше метаморфози циркуляції капіталу є лише ідеальні,
тобто що більше час обігу дорівнює нулеві або до нього наближається,
то більше функціонує капітал, то вища стає його продуктивність і самозростання
його вартости. Коли, напр., капіталіст працює на замовлення, так
що плату він одержує, здаючи продукт, при чому виплату цю робиться
засобами його власної продукції, — то час циркуляції наближається до
нуля.

Отже, час обігу капіталу взагалі обмежує час продукції його, а тому
й процес зростання його вартости. І обмежує його саме пропорційно
до свого протягу. Цей протяг може більшати або меншати дуже неоднаково,
а тому й у дуже неоднаковій мірі обмежувати час
продукції капіталу. Але політична економія бачить тут лише зовнішній
вигляд явища, а саме вплив часу обігу на процес зростання капітальної
вартости взагалі, Вона вважає цей неґативний вплив за позитивний, бо
наслідки його позитивні, і то більше тримається цієї позірности, що ця позірність
дає ніби доказ того, що капітал має в собі, незалежне від його
процесу продукції, а, значить, і від експлуатації праці, містичне джерело
