\parcont{}  %% абзац починається на попередній сторінці
\index{ii}{0409}  %% посилання на сторінку оригінального видання
призводить до поширеної репродукції, частина додаткової вартосте золотопромисловосте,
витрачувана не як дохід, а як додатковий змінний
капітал, входить в II, сприяє тут новому утворенню скарбів або дає нові
засоби купувати в І, безпосередньо знову не продаючи йому. З грошей,
які походять від цього І (у + m) золотопромисловосте, відходить частина
золота, яку деякі галузі продукції II потребують як сировинний матеріял
тощо, коротко кажучи, як елемент, що заміщує їхній сталий капітал.
Елемент для попереднього утворення скарбу — з метою майбутнього розширення
репродукції — при обміні між І і II є: для І тільки в тому разі,
коли частину І m продається підрозділові II однобічно, без зустрічної
купівлі, і служить вона тут для додаткового сталого капіталу II; для II
в тому разі, коли те саме маємо в І для додаткового змінного капіталу;
далі, в тому разі, коли частину додаткової вартости, витраченої підрозділом
І як дохід, не покривається за допомогою II с, отже, на неї купується
частина II m, яка в наслідок цього перетворюється на гроші. Коли
I ($v + m$ x) більше, ніж II с, то для своєї простої репродукції II с не
доводитеся заміщувати товарами з І те, що І взяв для споживання з II m.
Постає питання, до якої міри може утворитись скарб в межах обміну
капіталістів II між собою, — обміну, що може бути лише взаємним
обміном II m. Ми знаємо, що в межах II безпосередня акумуляція відбувається
тому, що частину II m безпосередньо перетворюється на змінний
капітал (цілком так само, як в І частину I m безпосередньо перетворюється
на сталий капітал). При різному часі пливу акумуляції в різних галузях
підприємств підрозділу II і в межах кожної поодинокої галузі для поодиноких
капіталістів, справа пояснюється, mutatis mutandis, цілком так само,
як і в підрозділі І. Одні перебувають ще на стадії утворення скарбів,
продають, не купуючи, інші вже досягли пункту справжнього розширення
репродукції, купують, не продаючи. Звичайно, додатковий змінний грошовий
капітал спочатку витрачають на додаткову робочу силу; але робітники
купують засоби існування в тих власників додаткових засобів
споживання, — засобів, що входять в споживання робітників, — які ще
утворюють скарб. Від цих власників засобів споживання гроші, відповідно
до утворення скарбів у них, не повертаються до свого вихідного
пункту; вони нагромаджують їх.
\parbreak{}  %% абзац продовжується на наступній сторінці
