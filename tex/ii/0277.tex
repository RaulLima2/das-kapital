мельна рента є три первісні джерела всякого доходу, так само,
як і всякої мінової вартости“. Далі ми розглянемо докладніше це
вчення А. Сміса про „складові частини ціни товарів“, зглядно про „всяку
мінову вартість“. Далі він каже: „Що все це має силу для всякого
поодинокого товару, взятого окремо, то повинно воно мати силу й для
всіх товарів, разом узятих, які становлять увесь річний продукт
землі та праці кожної країни. Вся ціна або мінова вартість
цього річного продукту мусить розкладатись на ці самі три частини
та розподілятись між різними жителями країни або як плата за
їхню працю, або як зиск їхнього капіталу, або як рента з їхнього
землеволодіння". (Кн. II, розд. 2, ст. 190).

Після того, як А. Сміс і ціну всіх товарів, узятих окремо, і „всю ціну
або мінову вартість... річного продукту землі та праці кожної країни"
розклав таким чином на три джерела доходів: доходів найманого робітника,
капіталіста й земельного власника, на заробітну плату, зиск і земельну
ренту, він все ж мусить контрабандою ввести обхідним шляхом
четвертий елемент, а саме елемент капіталу. Це робиться через відрізнення
між гуртовим і чистим доходом. „Гуртовий дохід усіх жителів великої
країни охоплює ввесь річний продукт їхньої землі та їхньої праці;
чистий дохід — частину, що лишається в їхньому розпорядженні,
відлічивши втрати на підтримання, поперше, їхнього основного,
а подруге, їхнього поточного капіталу, тобто
частину, що її вони можуть, не порушуючи свого капіталу, залічити
до свого споживного запасу або витратити на своє утримання, комфорт
і втіхи. Справжнє їхнє багатство теж пропорційне не їхньому гуртовому,
а чистому їхньому доходові". (Там само, ст. 190).

На це ми зауважимо ось що:

1) А. Сміс тут виразно розглядає тільки просту репродукцію, а не
репродукцію в поширеному маштабі, або акумуляцію; він каже лише про
видатки на підтримання (maintening) діющого капіталу. „Чистий“ дохід
дорівнює тій частині річного продукту — хоч суспільства, хоч індивідуального
капіталіста — яка може ввійти в „фонд споживання“, але розміри
цього фонду не повинні порушити діющого капіталу (encroach upon capital).
Отже, частина вартости, так індивідуального, як і суспільного продукту
не сходить ні на заробітну плату, ні на зиск або земельну ренту,
а сходить на капітал.

2) А. Сміс ховається від своєї власної теорії за допомогою гри слів,
за допомогою розмежування між gross і net revenue — гуртовим і чистим
доходом. Поодинокий капіталіст, як і ціла кляса капіталістів, або так
звана нація, замість зужиткованого в продукції капіталу, одержує товаровий
продукт, що його вартість — її можна визначити в пропорційних частках
цього самого продукту — з одного боку, покривав витрачену капітальну
вартість, а тому становить дохід або, буквально, revenue (revenu — дієприкметник
від revenir, повертатись), однак, nota bene, являє capital-revenue або
дохід на капітал; з другого боку, маємо складові частини вартости, що
їх „розподіляється між різними жителями країни або як плату за їхню
