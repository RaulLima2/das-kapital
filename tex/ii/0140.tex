ше формальну метаморфозу товару, що її пророблює продукт, товаровий
капітал, в сфері циркуляції, та що упосереднює переміщення товару,
з рук до рук, і ту речову метаморфозу, що її пророблюють різні елементи
продуктивного капіталу протягом процесу продукції. Перетворення
товару на гроші й грошей на товар, купівлю й продаж, він, не довго
думаючи, сплутує тут з перетворенням елементів продукції на продукт.
Його приклад для обігового капіталу є купецький капітал, що перетворюється
з товару на гроші, з грошей на товар; це зміна форми Т — Г — Т,
належна до товарової циркуляції. Але ця зміна форми в межах циркуляції
має для діющого промислового капіталу те значення, що товари, на
які зворотно перетворюються гроші, є елементи продукції, засоби праці
й робоча сила; отже, те значення, що вона упосереднює безперервність
функціонування промислового капіталу, процес продукції, як безперервний
процес або як процес репродукції. Вся ця зміна форм відбувається
в циркуляції; саме вона упосереднює справжній перехід товарів з
одних рук до інших. Навпаки, метаморфози, що їх перебігає продуктивний
капітал в своєму процесі продукції, є метаморфози, належні до
процесу праці, доконечні, щоб перетворити елементи продукції на
згаданий продукт. А. Сміс спиняється на тому, що одна частина засобів
продукції (власне засоби праці) функціонує в процесі праці (що він
неправильно висловлює, кажучи: „дає зиск їхньому власникові“ — yields а
pronfit to their master), не змінюючи своєї натуральної форми, зношується
лише поступінно, тимчасом як друга частина, матеріяли, змінюється, і
саме в наслідок своєї зміни виконує вона своє призначення як засіб продукції.
Але це різне поводження елементів продуктивного капіталу в
процесі праці становить лише вихідний пункт ріжниці між основним і неосновним
капіталом, а не саму цю ріжннцю, і це видно вже з того, що
таке поводження однаково існує за всіх способів продукції, капіталістичних
і некапіталістичних. Але цьому різному речовому поводженнювідповідає
віддача вартости продуктові, а цій віддачі знову таки відповідає
заміщення вартости за допомогою продажу продукту; і лише це
заміщення утворює ту ріжницю. Капітал, отже, є основний не тому, що
його зафіксовано в засобах праці, а тому, що частина його вартости,
вкладеної в засоби праці, лишається фіксована в них, тимчасом як друга
частина циркулює як складова частина вартости продукту.

„Коли він (капітал) застосовується для того, щоб утворити в майбутньому
зиск, то він мусить утворити цей зиск, або перебуваючи у
нього (власника), або одходячи від нього. В першому разі це основний,
в другому — обіговий капітал“\footnote*{
„If it (the stock) is employed in procuring future profit, it must procure this
profit by staying with him (the employer), or by going from him. In the one case
it is a fixed, in the other it is a circulating capital“ (p. 189).
}.

Тут насамперед впадає в очі грубо емпіричне — перейняте з уявлення
звичайного капіталіста — уявлення про зиск, яке цілком суперечить ліпшому
езотеричному поглядові А. Сміса. В ціні продукту заміщується і ціну
матеріялів, і ціну робочої сили, але так само і ту частину вартости зна-