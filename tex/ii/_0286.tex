\parcont{}  %% абзац починається на попередній сторінці
\index{ii}{0286}  %% посилання на сторінку оригінального видання
величини цілком не залежить від її розподілу між трьома групами осіб.
Може здаватися, що потрібна четверта частина, щоб замістити капітал
фармера або замістити зношування його робочої худоби та інших
його хліборобських знарядь. Але треба взяти на увагу, що ціна якогобудь
хліборобського знаряддя, напр., ціна робочого коня, теж складається
з вищезгаданих трьох частин: ренти на землю, де його вирощено,
праці догляду за конем і зиску фармера, що авансує й ренту з цієї
землі й плату за цю працю. Тому, хоч ціна зерна й може покрити так
ціну, як і витрати на утримання коня, все ж ціла ціна безпосередньо
або кінець-кінцем розкладається на ті таки три частини: земельну ренту,
працю (він має на думці заробітну плату) і зиск“. (Кн. І, розд. 6..
crop. 42).

Оце буквально все, що подає А. Сміс, обґрунтовуючи свою дивовижну
доктрину. Його доказ сходить просто на повторення того самого
твердження. Напр., він допускає, що ціна зерна складається не лише з
$v + m$, але, крім того, і з ціни засобів продукції, зужиткованих на продукцію
зерна, отже, з капітальної вартости, що її фармер витратив не
на робочу силу. Однак — каже він — ціни всіх цих засобів продукції й
собі розкладаються, як і ціна зерна, на $v + m$; А. Сміс забуває тільки
додати: і крім того, на ціну засобів продукції, зужиткованих на їхню
власну продукцію. Від однієї галузі продукції він відсилає до другої, а
від другої знову відсилає до третьої. Твердження, що вся ціна товарів
„безпосередньо“ або „кінець-кінцем" (ultimately) розкладається на $v + m$,
лише тоді не було б марною викруткою, коли б довести, що товарові
продукти, ціна котрих безпосередньо розкладається на с (ціна зужиткованих
засобів продукції) + $v + m$ кінець-кінцем, компенсується товаровими
продуктами, які заміщують ці „зужитковані засоби продукції“ в
цілому їх обсязі, і які з свого боку виробляється, протилежно до перших
товарових продуктів, через витрату лише змінного капіталу, тобто
капіталу, витрачуваного на робочу силу. В такому разі ціна останніх
товарових продуктів безпосередньо була б = $v + m$. Тому й ціну перших
товарових продуктів, $c + v + m$, де с фігурує як стала частина
капіталу, кінець-кінцем, можна було б розкласти на $v + m$. А. Сміс сам
не гадав, що він дав такий доказ, подаючи приклад з збирачами scotch
pebbles, які, проте, за його словами, 1) не дають жодної додаткової вартости,
а продукують лише власну заробітну плату; 2) не вживають жодних
засобів продукції (а все ж і вони мають засоби продукції, як от
кошики, мішки та інші вмістища, щоб забирати камінці).

Ми вже раніше бачили, що А. Сміс далі сам розбиває свою власну
теорію, не усвідомлюючи однак своїх суперечностей. Однак, джерела їх
треба шукати саме в його наукових вихідних пунктах. Капітал, перетворений
на працю, продукує більшу вартість, ніж його власна вартість.
Яким чином? В наслідок того, каже А. Сміс, що робітники підчас процесу
продукці втілюють в оброблювані ними речі таку вартість, яка,
крім еквіваленту за їхню власну купівельну ціну, утворює додаткову вартість
(зиск і ренту), що дістається не їм, а тим, хто застосовує їхню працю.
\parbreak{}  %% абзац продовжується на наступній сторінці
