Припустімо тепер, навпаки, незмінну величину періоду обороту, незмінний
маштаб продукції, але, з другого боку, зміну цін, тобто падіння або
підвищення цін на сировинні та допоміжні матеріяли й працю, або перших
двох з цих елементів. Припустімо, що ціна сировинних та допоміжних
матеріялів, так само, як і заробітна плата, зменшилась на половину.
Тоді в нашому прикладі треба було б авансованого капіталу щотижня
50 ф. стерл. замість 100, а для дев'ятитижневого періоду обороту — 450 ф.
стерл. замість 900 ф. стерл. 450 ф. стерл. авансованої капітальної
вартости виділюється насамперед як грошовий капітал, але процес продукції
триватиме й далі в тому самому маштабі, з тим самим періодом
обороту і з тим самим поділом останнього. Річна маса продукту лишається
теж та сама, але вартість її на половину зменшилась. Цю зміну, яку
супроводить і зміна в поданні та в попиті на грошовий капітал, спричиняє
не прискорення обігу й не зміна маси грошей, що циркулюють. Навпаки.
Зниження вартости, зглядно ціни елементів продуктивного капіталу
наполовину справило б насамперед той вплив, що авансувалось би капітальну
вартість, зменшену наполовину для того, шоб вести підприємство
X у попередніх розмірах, а що підприємство X авансує цю капітальну
вартість насамперед у формі грошей, тобто як грошовий капітал,
то, значить, воно мало б викидати на ринок лише половину попередньої
кількости грошей. Маса грошей, поданих в циркуляцію, зменшилась би
тому, що знизились ціни елементів продукції. Такий був би перший вплив.

Але подруге: половина первісно авансованої капітальної вартости в
900 ф. стерл. = 450 ф. стерл., яка а) по черзі перебігала форму грошового
капіталу, продуктивного капіталу й товарового капіталу, б) яка одночасно
постійно перебувала почасти в формі грошового капіталу, почасти
в формі продуктивного капіталу, почасти в формі товарового капіталу,
в одній поряд однієї, — ця половина виділилась би з
кругобігу підприємства X і тому надійшла б як додатковий грошовий
капітал на грошовий ринок, впливаючи на нього як додаткова складова
частина. Ці звільнені гроші, 450 ф. стерл., впливають як грошовий капітал
не тому, що вони є гроші, які стали надлишкові для продовження
підприємства X, а тому, що вони є складова частина первісної капітальної
вартости, і тому повинні й далі діяти як капітал, а не витрачатись як
простий засіб циркуляції. Найближчий спосіб надати їм чинности капіталу,
це подати їх на грошовий ринок як грошовий капітал. З другого боку,
можна було б також збільшити розміри продукції, залишаючи осторонь
основний капітал, вдвоє. Авансуючи той самий капітал в 900 ф.
стерл., можна було б провадити процес продукції в подвоєному розмірі.

З другого боку, коли б ціни поточних елементів продуктивного капіталу
підвищились наполовину, то щотижня замість 100 ф. стерл. треба
було б 150 ф. стерл., отже, замість 900 ф. стерл. — 1350 ф. стерл. Щоб
провадити підприємство в тому самому маштабі, треба було б 450 ф.
стерл. додаткового капіталу, і це залежно від стану грошового ринку
справляло б на нього pro tanto більший або менший тиск. Коли б на
ввесь вільний капітал, що є та ринку, ставилося вже попит, то це при
