новного капіталу, спосіб обороту вартости є спільний у робочої сили з цими складовими частинами,
протилежно до основного капіталу. Ці складові частини продуктивного капіталу, — а саме ті частини
його вартости, що їх витрачається на робочу силу й засоби продукції, які не становлять основного
капіталу — в наслідок цієї спільности характеру їхнього обороту, — протистоять основному капіталові,
як обіговий або поточний капітал.

Як ми бачили раніше, гроші, що їх капіталіст сплачує робітникові за вживання робочої сили, справді є
лише загальна еквівалентна форма доконечних робітникові засобів існування. Остільки й змінний
капітал речово складається з засобів існування. Але тут, розглядаючи оборот, ідеться про форму.
Капіталіст купує не засоби існування робітника, але саму його робочу силу. Змінну частину його
капіталу являють не засоби існування робітника, але його діюща робоча сила. В процесі праці
капіталіст продуктивно споживає саму робочу силу, а не засоби існування робітника. Сам робітник
перетворює на засоби існування ті гроші, що їх він одержав за свою робочу силу, щоб потім
перетвороти знову ці засоби існування на робочу силу й підтримати своє існування, цілком так само,
як, напр., капіталіст перетворює на засоби свого існування деяку частину додаткової вартости, що є в
товарі, який він продає за гроші, і, не зважаючи на це, зовсім не можна сказати, що покупець його
товарів сплачує йому засобами існування. Навіть коли робітникові сплачується частину його заробітної
плати в засобах існування in natura, то це за наших часів є вже друга оборудка. Він продає свою
робочу силу за певну ціну і при цьому умовляється, що частину цієї ціни він одержить в засобах
існування. Цим змінюється лише форма виплати, але не змінюється та обставина, що він дійсно продає
свою робочу силу. Це є друга оборудка, що відбувається вже не між робітником і капіталістом, а між
робітником як покупцем товару і капіталістом як продавцем товару; тимчасом як у першій оборудці
робітник є продавець товару (своєї робочої сили), а капіталіст її покупець. Цілком так само, як коли
б капіталіст, продаючи свій товар, прим., машину на гамарню, захотів мати за неї товар — залізо.
Отже, не засоби існування робітника визначаються як обіговий капітал протилежно до основного. А
також і не робоча сила його, а частина вартости продуктивного капіталу, витрачена на робочу силу,
яка через форму свого обороту набуває цього характеру обігового капіталу, спільно з деякими
складовими частинами сталого капіталу і протилежно до деяких інших складових частин сталого
капіталу.

Вартість поточного капіталу — в робочій силі та засобах продукції — авансується лише на той час, що
протягом його виготовляється продукт, залежно від маштабу продукції, визначуваного розміром
основного капіталу. Ця вартість цілком увіходить у продукт, а тому після продажу продукту цілком
повертається з циркуляції, і можна знову її авансувати. Робоча сила й засоби продукції, що в них
існує поточна складова частина капіталу, вилучається з циркуляції в розмірі, потрібному на
вироблення й продаж готового продукту, але їх завжди треба замінювати
