або покриттю всього того, що можна вважати за продукт рук людських.
Вона рідко коли менша, ніж чверть, і часто більша, ніж третина цілого
продукту. Жодна однакова маса продуктивної праці, застосована в мануфактурі,
ніколи не може зумовити такої великої репродукції. В мануфактурі
природа не робить нічого, людина — все; а репродукція завжди
мусить бути пропорційна потужності аґентів, що її переводять. Тому
капітал, вкладений у хліборобство, не лише пускає в рух більшу масу
продуктивної праці, ніж якийбудь інший, однаковий величиною капітал,
застосований у мануфактурі, але, порівняно з масою зайнятої ним продуктивної
праці, він додає куди більшу вартість до річного продукту
землі та праці даної країни, — до цього справжнього багатства і доходу
її жителів“. (Кн. II, розд. 5, стор. 242).

А. Сміс каже в II книзі, розд. І; „Вся вартість засівного матеріялу
теж є власне основний капітал“. Отже, тут капітал = капітальній вартості;
він існує в „основній“ формі. „Хоч засівний матеріял завжди переходить
з поля до комори й навпаки, він ніколи не змінює свого власника, а тому
в дійсності не циркулює. Фармер здобуває свій зиск не через його продаж,
а через його приріст“, (ст. 186). Обмеженість тут у тому, що Сміс
не бачить, як то вже бачив Кене, що вартість сталого капіталу знову
з’являється в відновленій формі, отже, не бачить важливого моменту
процесу репродукції, а бачить лише ще одну ілюстрацію — та до того ж
і фалшиву — свого відрізнювання між обіговим капіталом і основним
капіталом. Перекладаючи „avances primitives“ і „avances annuelles“
виразами „fixed capital“ і „circulating capital“, Сміс робить крок наперед
щодо вживання слова „капітал“, поняття якого узагальнюється і стає
незалежне від особливого застосування його фізіократами до сфери
„хліборобської“; крок назад у тому, що ріжниці між „основним“ капіталом
і „обіговим“ капіталом розглядається і їх додержується як вирішальних
ріжниць.

II. Адам Сміс

1) Загальні погляди А. Сміса

А. Сміс каже в книзі І, розд. 6, стор. 42; „В усякому суспільстві
ціна кожного товару кінець-кінцем розкладається або на ту або на другу
з цих трьох частин (заробітна плата, зиск, земельна рента), або на всі
три частини; і в усякому розвиненому суспільстві всі вони три, більш
або менш, увіходять як складові частини в ціну переважної більшости
товарів* 38); або, як сказано далі, стор. 63: „Заробітна плата, зиск і зе.

38) Щоб читача не ввів у помилку вислів „ціна переважної більшости товарів*,
наведемо витяг про те, як сам А. Сміс розуміє цей вислів. Напр., в ціну морської
риби рента не входить, а входить лише заробітна плата й зиск; в ціяу Scotch
pebbles (шотландської ріні) входить лише заробітна плата: „В деяких частинах
Шотландії бідняки промишляють тим, що збирають на морському березї різнокольорові
камінці, так звану шотландську рінь. Ціна, що її платять їм за ці камінці
різьбарі, складається тільки з їхньої заробітної плати, бо ні земельна рента, ні
зиск не становлять жодної частини її".
