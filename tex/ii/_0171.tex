\parcont{}  %% абзац починається на попередній сторінці
\index{ii}{0171}  %% посилання на сторінку оригінального видання
то звичайно руйнується все підприємство; в кращому разі будинки
лишаються недобудовані до ліпших часів, а в найгіршому їх продається
з авкціону й за півціни. Без спекуляційних будов, та до того ше в широких
розмірах, не може тепер обійтись жоден підприємець. Зиск від
самих будов надзвичайно малий; його головний бариш у підвищенні
земельної ренти, у влучному виборі та використанні забудовуваної площі.
Таким способом, тобто спекуляцією, що антиципує попит на будинки,
збудовано майже цілу Бельґравію й Тібурнію, а також багато тисяч вілл
в околицях Лондону. (Скорочений виклад з „Report from the Select
Committee on Bank Acts. Part I, 1857. Evidence. Запитання 5413—18,
5435—36).

Виконання робіт, що потребують дуже довгого робочого періоду й
широкого маштабу, цілком потрапляє в руки капіталістичної
продукції лише тоді, коли концентрація капіталу вже дуже велика, і
коли, з другого боку, розвиток кредитової системи дає капіталістові зручний
засіб авансувати чужий капітал замість власного, а тому й ризикувати
чужим капіталом. Однак, само собою зрозуміло, що на швидкість обороту
й на час обороту капіталу не впливає зовсім та обставина, чи належить
капітал, чи не належить тому, хто авансує його на підприємство.

Обставини, що збільшують продукт окремого робочого дня, як от кооперація,
поділ праці, застосування машин, разом з тим скорочують робочий
період при актах продукції, що зв’язані між собою. Напр., машини
скорочують час будування будинків, мостів тощо; жатки, молотарки
тощо скорочують робочий період, потрібний на те, щоб перетворити
достигле зерно на готовий товар. Удосконалене суднобудівництво, збільшуючи
швидкість суден, скорочує час обороту капіталу, витраченого на
судноплавство. Однак, ці поліпшення, що скорочують робочий період, а
через це й час, що на нього треба авансувати обіговий капітал, сполучаються
здебільша із збільшеною витратою основного капіталу. З другого
боку, в деяких галузях робочий період може скоротитись в наслідок
простого поширення кооперації: будування залізниць скорочується
в наслідок того, що в роботу пускається великі армії робітників, які
беруться до роботи одночасно в багатьох пунктах. Час обороту скорочується
тут в наслідок збільшення авансованого капіталу. Більше засобів
продукції і більше робочої сили мусять об’єднатись під командою
капіталіста.

Отже, якщо скорочення робочого періоду здебільша сполучається зі
збільшенням капіталу, авансованого на коротший час, так що в міру того,
як скорочується час авансування, більшає маса авансованого капіталу, — то
треба тут згадати, що, незалежно від того, яка взагалі є маса суспільного
капіталу, справа сходить на те, якою мірою засоби продукції й засоби
існування, зглядно порядкування ними, є розпорошені або зосереджені
в руках поодиноких капіталістів, отже, якого розміру вже
досягла концентрація капіталів. Оскільки кредит упосереднює концентрацію
капіталу в одних руках, прискорює та підвищує її, остільки він
сприяє скороченню робочого періоду, а тим самим і скороченню часу обороту.
