\parcont{}  %% абзац починається на попередній сторінці
\index{ii}{0169}  %% посилання на сторінку оригінального видання
специфічна природа корисного ефекту, що його треба досягти, — потрібні
постійні додаткові витрати обігового капіталу (заробітної плати, сировинних
та допоміжних матеріялів), що з них жодна частина не перебуває в формі
здатній до циркуляції, а, значить, і не може служити для відновлення тієї
самої операції; навпаки, кожну частину послідовно закріплюється в сфері
продукції як складову частину вироблюваного продукту, зв’язується в
формі продуктивного капіталу. Але час обороту дорівнює сумі часу продукції
та часу циркуляції капіталу. Отже, подовження часу продукції в
тій самій мірі зменшує швидкість обороту капіталу, як і подовження часу
циркуляції. Але тут треба звернути увагу на дві обставини:

Поперше: довше перебування в сфері продукції. Напр., капітал, авансований
протягом першого тижня на працю, сировинний матеріял тощо,
так само, як і частини вартости основного капіталу, перенесені на продукт,
протягом цілого тримісячного періоду лишаються зв’язані в сфері
продукції, і, бувши втілені в продукт ще лише утворюваний, ще не готовий,
не можуть увійти в процес циркуляції як товари.

Подруге: що робочий період, потрібний для продукційного акту, триває
три місяці і в дійсності становить лише один зв’язний процес праці,
то постійно треба долучати кожного тижня нові порції обігового капіталу
до попередніх. Отже, разом із подовженням робочого періоду
зростає маса авансованих один по одному додаткових капіталів.

Ми припустили, що капітали, вкладені в прядіння та машинобудівництво,
однакові величиною; що капітали ці в однаковій пропорції розподілено
на сталий і змінний капітал, а також на основний і обіговий,
що робочі дні однакові довжиною, — коротко кажучи, що однакові всі
обставини, крім протягу робочого періоду. Протягом першого тижня витрати
в обох виробництвах однакові, але продукт прядіння можна продати
і на вторговані гроші можна знову купити сировинний матеріял і
робочу силу, коротко кажучи, продукцію можна провадити далі в тому
самому маштабі. Навпаки, фабрикант машин лише по трьох місяцях, коли
його продукт уже цілком готовий, може знову перетворити на гроші обіговий
капітал, витрачений протягом першого тижня, й знову оперувати
ним. Отже, поперше, ріжниця є в зворотному припливі тієї самої витраченої
кількости капіталу. А подруге, протягом трьох місяців у прядінні
й машинобудівництві застосовано однакові величиною продуктивні капітали,
але величина витрат капіталу цілком різна для прядуна й машинобудівника,
бо в одному разі той самий капітал швидко відновлюється
й ту саму операцію можна тому знову повторювати, а в другому разі
він відновлюється порівняно лише повільно і тому, поки його відновиться,
треба постійно долучати нові кількості капіталу до старих. Отже, періоди,
що протягом їх відновлюється певні кількості капіталу, або періоди,
що на них авансується капітал, є різні, так само різні є й маси капіталу
(хоч щоденно або щотижня застосовуються однакові капітали), що їх до
водиться авансувати залежно від протягу робочого процесу. Цю обставину
треба взяти на увагу тому, що, як у випадках, що їх ми розглянемо
в наступному розділі, протяг періодів, що на них авансується
\parbreak{}  %% абзац продовжується на наступній сторінці
