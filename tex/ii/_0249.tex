\parcont{}  %% абзац починається на попередній сторінці
\index{ii}{0249}  %% посилання на сторінку оригінального видання
їхнього існування, другу — b, що її вони почасти витрачають на речі
розкошів, а почасти застосовують на поширення продукції; а — в такому
разі репрезентує змінний капітал, b — додаткову вартість. Але такий поділ
не мав би жодного впливу на величину тієї маси грошей, яка потрібна
для циркуляції цілого їхнього продукту. За інших незмінних умов, вартість
товарової маси, що циркулює, була б та сама, а значить, і маса
потрібних для цього грошей була б та сама. Крім того, при однаковому
поділі періодів обороту продуценти мусили б мати такі самі грошові запаси,
тобто постійно мати в грошовій формі таку саму частину свого
капіталу, бо, згідно з нашим припущенням, їхня продукція, як і раніш,
була б товаровою продукцією. Отже, та обставина, що частина товарової
вартости складається з додаткової вартости, абсолютно не змінює маси
грошей доконечних для провадження підприємства.

Один з супротивників Тука, що тримається формули $Г — Т — Г$, запитує
його, як капіталістові вдається постійно вилучати з циркуляції більше
грошей, ніж він подає туди. Це цілком зрозуміло. Тут ідеться не про
утворення додаткової вартости. Останнє, являючи єдину таємницю, з
капіталістичного погляду само собою зрозуміле. Застосована бо сума вартости
не була б капіталом, коли б вона не збагачувалась додатковою
вартістю. А що згідно з припущенням вона є капітал, то додаткова вартість
сама собою зрозуміла.

Отже, питання не в тім, відки береться додаткова вартість, а в тім,
відки беруться гроші, що на них вона перетворюється.

Але для буржуазної економії існування додаткової вартости зрозуміле
само собою. Отже, її не лише припускається, але разом з нею припускається
й те, що частина товарової маси, пущеної в циркуляцію, складається
з додаткового продукту, отже, репрезентує таку вартість, що її капіталіст
не кинув у циркуляцію, кидаючи туди свій капітал; що, отже, капіталіст,
разом з своїм продуктом кидає в циркуляцію певний надлишок
порівняно з своїм капіталом, а потім знову вилучає з неї цей надлишок.

Товаровий капітал, що його капіталіст подає в циркуляцію, має більшу
вартість (звідки це постає, не пояснюється або не розуміється, але з
погляду буржуазної економії c’est un fait\footnote*{
Це — факт. \emph{Ред.}
} ), ніж продуктивний капітал,
що його він вилучив з циркуляції в формі робочої сили плюс засоби
продукції. Тому при цьому припущенні ясно, чому не лише капіталіст
А, але й В, С, D і т. ін. можуть постійно вилучати з циркуляції через
обмін своїх товарів більшу вартість, ніж вартість їхнього первісно авансованого
капіталу, що його потім знову й знову авансується. А, В, С,
D і т. ін. завжди подають в циркуляцію в формі товарового капіталу, —
а ця операція так само багатобічна, як і самостійно діющі капітали, —
більшу товарову вартість, ніж та, що її вони вилучають з циркуляції в
формі продуктивного капіталу. Отже, їм постійно доводиться розподіляти
між собою суму вартости (тобто кожному доводиться вилучати для себе
з циркуляції продуктивний капітал), що дорівнює сумі вартости їхніх
\parbreak{}  %% абзац продовжується на наступній сторінці
