\parcont{}  %% абзац починається на попередній сторінці
\index{ii}{0154}  %% посилання на сторінку оригінального видання
відрізняється лише остільки, оскільки вона разом з однією частиною об’єктивних
продуктотворців („матеріялів“, за загальним позначенням Сміса)
належить до категорії обігового капіталу протилежно до другої частини
об’єктивних продуктотворців, що належить до категорії основного капіталу.

Та обставина, що частина капіталу, витрачена на заробітну плату, належить
до поточної частини продуктивного капіталу, має властивість поточности
— протилежно до основної частини продуктивного капіталу —
спільно з частиною речових продуктотворців, як от сировинний матеріял
тощо, ця обставина абсолютно не має ніякого чинення до тієї ролі, що
її в процесі зростання вартости відіграє ця змінна частина капіталу протилежно
до сталої. Тут річ лише в тім, яким чином ця частина авансованої
капітальної вартости мусить покритись, відновитись, тобто репродукуватись
з вартости продукту за допомогою циркуляції. Повторювані
акти купівлі робочої сили належать до процесу циркуляції. Але тільки
в процесі продукції вартість, витрачена на робочу силу, з певної, сталої
величини, перетворюється (не для робітника, а для капіталіста) на змінну —
і, значить, взагалі тільки в наслідок цього авансована вартість перетворюється
на капітальну вартість, на капітал, на вартість, що сама з себе
зростає. Але через те, що — як це маємо в Сміса — не на робочу силу
витрачену вартість визначається як поточну складову частину продуктивного
капіталу, а вартість, витрачену на засоби існування робітника, то й
не можна через це зрозуміти ріжницю між змінним і сталим капіталом, а
значить, зрозуміти процес капіталістичної продукції взагалі. Визначення
цієї частини капіталу як капіталу змінного, протилежно до сталого капіталу,
витраченого на речових продуктотворців, цілком поховано тут
визначенням, що, згідно з ним, частина капіталу, витрачена на робочу силу,
щодо своєї ролі в обороті належить до поточної частини продуктивного
капіталу. Таке поховання стає тим певніше, що, замість робочої
сили, як елемент продуктивного капіталу наводиться засоби існування
робітника. Чи авансується вартість робочої сили в грошах, чи безпосередньо
в засобах існування, це не має значення. Хоч, звичайно, останнє
на основі капіталістичної продукції може бути лише винятком\footnote{
До якої міри сам А. Сміс закриває собі шлях розуміння ролі робочої
сили в процесі зростання вартости, доводить таке речення, де працю робітників,
цілком на зразок фізіократів, прирівнюється до праці худоби: „Не лише його
(фармера) слуги, що працюють, а й його робоча худоба є продуктивні робітники“.
(Not only his (the farmer’s) labouring servants, but his labouring cattle are
productive labourers“. — Book II, chap. V, p. 243).
}.

В наслідок того, що визначення обігового капіталу А. Сміс таким
чином закріпив як вирішувальне для капітальної вартости, вкладеної в робочу
силу — визначення фізіократів без засновків фізіократів — він щасливо
довів своїх наслідувачів до неспроможности зрозуміти частину капіталу,
витрачену на робочу силу, як змінний капітал. Глибші й правильніші
думки, розкидані в А. Сміса в інших місцях, не перемогли, перемогла
саме ця помилка. Більше за те, пізніші письменники пішли ще
\parbreak{}  %% абзац продовжується на наступній сторінці
