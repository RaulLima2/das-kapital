требувати їх переміщення, тобто додаткового продукційного процесу
транспортової промисловости. Отже, вкладений в неї продуктивний капітал
додає вартість до транспортованого продукту, почасти через перенесення
вартости транспортових засобів, почасти тому, що вартість додається працею
транспорту. Ця остання додана вартість розкладається, як взагалі в
капіталістичній продукції, на покриття заробітної плати й на додаткову
вартість.

В кожному продукційному процесі велику ролю відіграє переміщення
предмету праці й потрібні на це засоби праці й робоча сила — напр.,
бавовну переміщується з чесальної майстерні до прядільні, вугілля підіймається
з шахти на поверхню. Перехід готового продукту як готового
товару з одного місця самостійної продукції на друге, просторово віддалене
від нього, показує нам те саме явище, тільки в ширшому маштабі.
Після перевозу продуктів з одного місця продукції в інше відбувається
перевіз готових продуктів із сфери продукції в сферу споживання.
Продукт тільки тоді готовий для споживання, коли він закінчить це
переміщення.

Як показано раніше, загальний закон товарової продукції такий: продуктивність
праці та утворювання нею вартости перебувають у зворотному
відношенні. Це має силу для транспортової промисловости, як і
для кожної іншої. Що менше мертвої та живої праці треба для перевозу
товару на дану віддаль, то вища продуктивна сила праці, і навпаки18).

Абсолютна величина вартости, додавана до товарів транспортом, за
інших незмінних обставин, стоїть у зворотному відношенні до продуктивної
сили транспортової промисловости і в прямому відношенні до віддалей,
що на них товари переміщуються.

Відносна частина вартости, що її, за інших незмінних обставин,
долучають до ціни товару витрати на транспорт, стоїть у прямому відношенні
до просторової величини і ваги товару. Але є багато обставин,
що модифікують справу. Для перевозу, напр., потрібні більші або менші

18) Рікардо цитує Сея, який вважає за щастя для торговлі те, що вона удорожнює
продукти й підвищує їхню вартість в наслідок транспортових витрат.
„Торговля, — каже Сей, — дозволяє нам одержувати товар у місці його постання й
перевозити його в інше місце споживання; отже, вона дозволяє нам збільшувати вартість товару на всю
ріжницю між його ціною в першому місці та в другому“.
Рікардо каже з цього приводу: „Правильно. Але як долучається до неї цей додаток
вартости? Через додачу до витрат продукції, поперше, витрат на транспорт,
а подруге, зиску на капітал, авансований торговцем. Товар має більшу вартість
лише з тієї причини, з якої може збільшитись вартість кожного товару, коли на
його продукцію та перевіз, раніш ніж купиться його, витратиться більше праці.
А це не можна вважати за одну з переваг торговлі“. („True, but how is the additional
value given to it? By adding to the cast of production, first, the expences of
conveyance, secondly, the profit on the advances of capital made by the merchant.
The commodity is only more valuable, for the same reason that every other commodity
may become more valuable, because more labour is expended on its production
and conveyance, before it is purchased by the consumer. This must not be
mentioned as one of the advantages of commerce*. (Ricardo. Principles of Pol. Econ.,
3-rd ed., London, 1821, ст. 309, 310).
