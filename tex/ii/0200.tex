разу він потребує авансування в 600 ф. стерл. (капітал І). Період циркуляції
3 тижні; отже, період обороту, як і раніш, 9 тижнів. Капітал II
в 300 ф. стерл. ввіходить у роботу протягом тритижневого періоду циркуляції
капіталу І. Коли розглядати їх обидва, як капітали, незалежні
один від одного, то схема річного обороту матиме такий вигляд:

Таблиця II.

Капітал І. 600 ф. стерл.

Періоди обороту    Робочі періоди    Авансовано    Періоди циркуляції
І. Тижні 1—9    Тижні 1—6    600 ф. стерл. Тижні 7—9
11. „10—18 „10—15    600 „„ 16—18
III. „19—27 „19—24    600 „„ 25—27
IV. „28—36 „28—33    600 „34—36
V. „37—45 „37—42    600 „„ 43—45
VI. „45 — [54] „46—51    600 „„ [52—54]

Додатковий капітал II. 300 ф. стерл.

Періоди обороту    Робочі періоди    Авансовано    Періоди циркуляції
І. Тижні 7—15    Тижні 7—9    300 ф. стерл. Тижні 10—15
II. „16—24 „16—18    300 „„ „19—24
III. „25—33 „25—27    300 „„ „28—33
IV. „34—42 „34—36    300 „„ „37—42
V. „43—51 „43—45    300 „„ „45—51

Процес продукції відбувається цілий рік безперервно в однакових
розмірах. Обидва капітали І і II лишаються цілком відокремлені. Але
для того, щоб подати їх так відокремленими, нам довелось роз’єднати
їхні справжні схрещування й переплітання, а через це змінити й число
оборотів. А саме, згідно з вище наведеною таблицею, обертається:

Капітал І 600 × 5 2/3 = 3400 ф. стерл.

„II 300 × 5 = 1500 ф. стерл.

отже, ввесь капітал    900 X 5 4/9 = 4900 ф. стерл.

Але це неправильно, бо, як ми побачимо, справжні періоди продукції
та циркуляції не абсолютно збігаються з цими періодами вище наведеної
схеми, де головне було в тому, щоб подати обидва капітали, І і II, незалежними
один від одного.

В дійсності саме капітал II не має ані особливого робочого періоду, ані особливого
періоду циркуляції, відокремлених від цих періодів капіталу І. Робочий
період триває 6 тижнів, період циркуляції 3 тижні. Що капітал II дорівнює
тільки 300 ф. стерл., то він може виповнити лише частину робочого
