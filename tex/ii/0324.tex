куляцію ті гроші — а саме, витрачаючи їх на засоби споживання — що
ними перетворюється на гроші або, інакше кажучи, реалізується його
додаткова вартість. Звичайно, справа тут не в тих самих монетах, а
в сумі дзвінкої монети, рівній тій сумі (або рівній частині тієї суми),
що її він подав у циркуляцію на задоволення особистих потреб.

На практиці це стається двома способами: коли підприємство відкрито
лише поточного року, то мине чимало часу, в кращому разі кілька місяців,
перш ніж капіталіст матиме змогу витрачати на своє особисте споживання
гроші з доходів самого підприємства. Але через це він ні на хвилину
не відкладає свого споживання. „Він сам собі авансує (чи з своєї
власної кишені, чи з чужої в кредит, тут ця обставина зовсім не має
значення) гроші під додаткову вартість, яку він іще лише має здобути;
але цим самим він авансує і засоби циркуляції для реалізації додаткової
вартости, що її треба буде реалізувати пізніше. Навпаки, коли підприємство
вже давно йде правильним ходом, то виплати й надходження розподіляються
на різні строки протягом року. Що відбувається безперервно,
так це споживання капіталіста, яке антиципується і своїми розмірами
розраховується в певній пропорції до звичайних або передбачуваних надходжень.
В продажі кожної партії товару реалізується й частину додаткової
вартости, що її треба видобути протягом року. Але коли б протягом
цілого року продали лише стільки спродукованого товару, скільки
треба для заміщення сталої й змінної капітальної вартости, що є в ньому,
або коли б ціни спали так, що, продавши ввесь річний товаровий продукт,
можна було б реалізувати лише авансовану капітальну вартість, що
міститься в ньому, то у витрачанні грошей виразно виступило б антиципування,
надія на майбутню вартість. Коли наш капіталіст збанкротує,
то його кредитори й суд досліджуватимуть, чи були його антициповані
особисті витрати в правильному відношенні до розмірів його підприємства
й до надходжень додаткової вартости, що звично або нормально відповідають
цим розмірам.

Але коли ми візьмемо цілу клясу капіталістів, то теза, що вона сама
мусить подати в циркуляцію гроші для реалізації своєї додаткової вартости
(зглядно й для циркуляції свого капіталу, сталого й змінного), не
лише не є парадоксальна, але є неодмінна умова цілого механізму; тут
бо є лише дві кляси: робітнича кляса, що тільки й має свою робочу силу,
і кляса капіталістів, що в її монопольному володінні є засоби суспільної
продукції, так само, як і гроші. Парадокс був би тоді, коли б робітнича
кляса з самого початку авансувала з власних коштів гроші, потрібні для
реалізації додаткової вартости, що міститься в товарах. Але поодинокий
капіталіст робить це авансування завжди лише в такій формі, що він діє
як покупець, витрачає гроші на закуп засобів споживання, або
авансує гроші на закуп елементів свого продуктивного капіталу, чи то
робочої сили, чи то засобів продукції. Він завжди віддає гроші лише
за еквівалент. Гроші він авансує циркуляції лише таким самим способом,
яким авансує їй товари. І в тому, і в цьому разі він діє як вихідний пункт
циркуляції товарів та грошей.
