слідок централізації їх в небагатьох руках, при чому суспільна сума цих
капіталів не зростає. Тут маємо лише змінний розподіл поодиноких капіталів.
Нарешті, в попередньому розділі показано, що скорочення періоду
обороту дозволяє або пускати в рух з меншим грошовим капіталом той
самий продуктивний капітал, або з тим самим грошовим капіталом —
більший продуктивний капітал.

Але все це, очевидно, не стосується власне до питання про грошовий
капітал. Це показує лише, що авансований капітал — дана сума вартости,
що в своїй вільній формі, в своїй формі вартости, складається з
певної суми грошей, — по своєму перетворенні на продуктивний капітал
має в собі продуктивні потенції, що їхні межі визначаються не величиною
його вартости, а можуть, навпаки, до певної міри діяти з різною екстенсивністю
або інтенсивністю. Коли дано ціни елементів продукції — засобів
продукції та робочої сили — то цим визначено величину грошового
капіталу, потрібного на закуп певної кількости цих елементів продукції,
наявних у вигляді товарів. Інакше кажучи, визначено величину вартости
того капіталу, що його треба авансувати. Але розміри, що в них цей
капітал діє як вартостетворець і продуктотворець, елястичні й змінні.

До другого пункту. Само собою зрозуміло, що та частина суспільної
праці та засобів продукції, яку доводиться щорічно витрачувати
на продукцію або закуп золота, щоб замістити зужиту монету, є pro
tanto зменшення розміру суспільної продукції. Щождо грошової вартости,
яка функціонує почасти як засіб обігу, а почасти як скарб, то раз
вона вже є, скоро її здобуто, вона перебуває поряд з робочою силою,
спродукованими засобами продукції та природними джерелами багатства.
Її не можна розглядати, як щось, що обмежує все це. Перетворенням її
на елементи продукції, обміном з іншими народами, можна було б розширити
розміри продукції. Але для цього треба, щоб гроші тут, як і
раніше, відігравали ролю світових грошей.

Залежно від величини періоду обороту потрібна більша або менша
маса грошового капіталу, щоб пустити в рух продуктивний капітал. Так
само ми бачили, що поділ періоду обороту на робочий час і час циркуляції
зумовлює збільшення лятентного в грошовій формі капіталу, або
капіталу, що його застосовання відкладається.

Оскільки період обороту визначається протягом робочого періоду, остільки
його за інших незмінних умов, визначається матеріяльною природою
процесу продукції, отже, не специфічним суспільним характером
цього процесу продукції. Однак на основі капіталістичної продукції довготриваліші
широкі операції зумовлюють більші авансування грошового
капіталу на довший час. Отже, продукція в таких галузях залежить від
тих меж, що в них поодинокий капіталіст порядкує грошовим капіталом.
В цих обмеженнях пробиває вилам система кредиту і зв’язані з нею асоціації,
прим., акційні товариства. Тому порушення на грошовому ринку
припиняють діяльність таких підприємств, тимчасом як ці самі підприємства
і собі зумовлюють порушення на грошовому ринку.
