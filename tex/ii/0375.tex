варах, капіталісти одержують знову 100 ф. стерл. грішми. Отже, у капіталістів
знову є 100 ф. стерл. грішми, а в робітників — на 100 ф. стерл.
товару, що його вони сами спродукували. Важко зрозуміти, як могли б
капіталісти збагатитись на цьому. Коли б 100 ф. стерл. грішми не припливали
до них назад, то їм довелось би, поперше, заплатити робітникам
за їхню працю 100 ф. стерл. грішми, і, подруге, безплатно віддати
їм продукт цієї праці, засоби споживання на 100 ф. сгерл. Отже, зворотний
приплив грошей міг би, щонайбільш, пояснити, чому капіталісти
не збіднюються в наслідок цієї операції, але ні в якому разі не міг би
пояснити, чому вони з неї збагачуються.

Звичайно, друге питання, звідки капіталісти беруть ці 100 ф. стерл.
грішми, і чому робітники мусять обмінювати свою робочу силу на ці 100 ф.
стерл., замість самим продукувати товари власним коштом. Але це є щось
само собою зрозуміле для мислителів типу Детю.

Детю сам не цілком задоволений з такого розв’язання. Він бо не
сказав нам, що збагачення постає тому, що витрачають грошову суму
в 100 ф. стерл. і потім знову одержують грошову суму в 100 ф. стерл
отже, не в наслідок зворотного припливу 100 ф. стерл. грішми, який
з’ясовує лише, чому ці 100 ф. стерл. грішми не втрачається. Він сказав
нам, що капіталісти збагачуються, „продаючи все продуковане ними
дорожче, ніж коштував їм закуп цього“.

Отже, капіталісти в своїй оборудці з робітниками мусять збагачуватись
тому, що вони продають робітникам дуже дорого. Чудово! „Вони
виплачують заробітну плату... і все це зворотно припливає до них в наслідок
витрат всіх цих людей, що платять за них“ (за продукти) „дорожче,
ніж вони коштували їм“ (капіталістам) „при такій заробітній
платі“ (стор. 240). Отже, капіталісти платять робітникам 100 ф.
стерл. заробітної плати, а потім продають робітникам власний продукт
останніх за 120 ф. стерл., так що до капіталістів не лише припливають
назад ці 100 ф. сгерл, , а ще виграється 20 ф. стерл.? Це неможливо.
Робітники можуть заплатити лише тими грішми, що їх вони одержали
в формі заробітної плати. Коли вони одержали від капіталістів 100 ф.
стерл. заробітної плати, то вони можуть купити лише на 100 ф. стерл.
а не на 120 ф. стерл. Отже, цим способом питання не розв’язується. Але
є ще один спосіб. Робітники купують у капіталістів товару на 100 ф.
стерл., а в дійсності одержують товар вартістю лише на 80 ф. стерл.
Тому їх, безперечно, обшахраяли на 20 ф. стерл. А капіталіст, безперечно,
збагатився на 20 ф. стерл., бо він фактично оплатив робочу силу
на 20\% нижче від її вартости або обкружним шляхом зробив одрахування
в 20\% з номінальної заробітної плати.

Кляса капіталістів досягла б цього самого, якби вона з самого початку
виплатила робітникам заробітної плати лише 80 ф. стерл., а потім
дала б їм за ці 80 ф. стерл. грішми товарову вартість дійсно на 80 ф.
стерл. Ось такий, — коли взяти цілу клясу, — здається, нормальний спосіб,
бо, за висловом самого пана Детю, робітнича кляса мусить одержувати
„достатню заробітну плату“ (ст. 219), бо цієї заробітної плати
