\parcont{}  %% абзац починається на попередній сторінці
\index{ii}{0089}  %% посилання на сторінку оригінального видання
одного боку, робочу силу, з другого — засоби праці. І справа тут стоїть
так само, як з часом на купівлю й продаж.

Як єдність у своїх кругобігах, як вартість, що процесує, хоч перебуває
вона в сфері продукції, хоч в обох фазах сфери циркуляції, капітал
існує лише ідеально в формі розрахункових грошей, насамперед у голові
товаропродуцента, зглядно капіталістичного товаропродуцента. Бухгальтерією,
куди належить також визначення цін або обчислення товарових
цін (калькуляція цін), рух цей фіксується й контролюється. Таким
чином рух продукції, а особливо зростання вартости — при чому товари
фігурують лише як носії вартости, як назви речей, що їхнє ідеальне
вартісне буття фіксується в рахункових грошах — набуває символічного
образу в уявленні. Доки поодинокий товаропродуцент веде свою книгу
тільки в своїй голові (як, напр., селянин; лише капіталістичне хліборобство
породжує фармера, що веде рахункову книгу) або веде книгу своїх
видатків і прибутків, термінів виплат тощо лише між іншим, у вільний від
роботи час, доти очевидно, що ця його функція та засоби праці, на це
зужитковані, прим., папір тощо, являють додаткову витрату робочого часу
й засобів праці, хоч і доконечних, але все ж таких, що становлять одбаву
так з часу, що його він може спожити продуктивно, як і з засобів праці, що
функціонують у справжньому процесі продукції і беруть участь у творенні
продукту й вартости\footnote{
За середніх віків книги хліборобські вели тільки в манастирях. Однак ми
бачили (кн. І, розд. XII, 4), що вже в староіндійських громадах був рахівничий
у хліборобстві. Бухгальтерію тут усамостійнено у виключну функцію громадського
урядовця. В наслідок цього розподілу праці зберігається час, роботу та
витрати, але все ж продукція і бухгальтерія щодо продукції лишаються
такими самими різними речами, як навантаження пароплавів і писання квитків на
вантаж. В особі бухгальтера частину робочої сили громади відібрано від
продукції, і витрати його функціонування покривались не його власною
працею, а одбавою з громадськоґо продукту. З бухгальтером у капіталіста
справа стоїть mutatis mutandis\footnote*{
Змінивши те, що треба змінити, або з відповідними змінами. \emph{Ред.}
} так само, як з бухгальтером індійської громади. (З рукопису II).
}. Природа самої функції не змінюється ні в наслідок
того розміру, що його вона набуває, концентруючися в руках капіталістичного
товаропродуцента і роблячись не функцією багатьох дрібних товаропродуцентів,
а функцією \emph{одного} капіталіста, функцією в процесі продукції
широкого маштабу; ні в наслідок того, що її відокремлено від продуктивних
функцій, що до них вона становила додаток, ні в наслідок її усамостійнення
як функції особливих аґентів, що їм виключно доручається її.

Поділ праці, усамостійнення якоїбудь функції, не робить ще з неї
функції, яка утворює продукт і вартість, якщо вона не була такою сама
собою, отже, ще до свого усамостійнення. Коли капіталіст вкладає
вперше свій капітал, він мусить вкласти частину капіталу на те, щоб
найняти бухгальтера тощо і купити засоби для ведення книг. Коли його
капітал уже функціонує, перебуваючи в постійному процесі репродукції,
то капіталіст мусить частину товарового продукту, за допомогою перетворення
на гроші, повсякчас зворотно перетворювати на бухгальтера,
конторників тощо. Цю частину капіталу відтягується від процесу продукції,
\index{ii}{0090}  %% посилання на сторінку оригінального видання
і належить вона до витрат циркуляції, відраховань із загального
виторгу (включаючи й саму робочу силу, що її вживають виключно на
ці функції).

Все ж є деяка ріжниця між витратами, що випливають з бухгальтерії,
зглядно непродуктивною витратою на це робочого часу, з одного боку, і витратами
часу просто на купівлю й продаж, з другого боку. Ці останні випливають
лише з певної суспільної форми продукційного процесу, з того,
що це — процес продукції товару. Ведення книг, як контроль та ідеальне
об’єднання процесу стає то доконечніше, що більше процес поширюється
до суспільних розмірів і втрачає суто-індивідуальний характер.
Отже, ведення книг доконечніше в капіталістичній продукції, ніж
в розпорошеному ремісничому й селянському виробництві, доконечніше
у колективній продукції, ніж у капіталістичній. Але витрати на ведення
книг скорочуються з концентрацією продукції і скорочуються то більше,
що більше воно перетворюється на суспільне ведення книг.

Тут ідеться лише про загальний характер витрат циркуляції, що постають
тільки з простої формальної метаморфози. І не треба тут вдаватись у
всі детальні форми цих витрат. Але наскільки форми, належні
до простого перетворення форми вартости, що, отже, випливають з певної
суспільної форми продукційного процесу, — форми, які в індивідуального
товаропродуцента є лише минущі й ледве помітні
моменти, що перебігають поряд його продуктивних функцій
або переплітаються з ними, — наскільки ці
форми можуть вражати як масові витрати циркуляції, це ми побачимо,
поглянувши на просте приймання й
видачу грошей, коли вони усамостійнюються й концентруються в широкому маштабі,
як виключна функція банків тощо, або як виключна функція
скарбників в індивідуальних підприємствах. Але треба пам’ятати,
що ці витрати циркуляції, змінюючи свій вигляд, не змінюють свого характеру.

\subsubsection{Гроші}

Хоч продукт продукується як товар, хоч не як товар, він завжди є
речова форма багатства, споживна вартість, призначена для особистого
або продуктивного споживання. Його вартість як товару існує ідеально
в ціні, яка нічого не змінює в його справжній споживній формі. Але та
обставина, що певні товари, напр., золото й срібло, функціонують як
гроші, і як такі перебувають виключно в процесі циркуляції (навіть як
скарб, резерв і т. ін., лишаються вони, хоч і лятентно, в сфері
циркуляції) є цілком продукт певної суспільної форми продукційного
процесу, процесу продукції товарів. Що на основі капіталістичної продукції
товар стає загальною формою продукту; що переважна кількість
продуктів продукується як товар, і тому мусить вона набрати грошової
форми; що, отже, товарова маса, тобто частина суспільного багатства,
яка функціонує як товар, безупинно зростає, — то тут більшає й кількість
золота й срібла, що, функціонують як засіб циркуляції, засіб виплати,
резерв тощо. Ці товари, що функціонують як гроші, не входять ні в
особисте, ні в продуктивне споживання. Це — суспільна праця, зафіксована
\index{ii}{0091}  %% посилання на сторінку оригінального видання
в такій формі, що в ній вона править лише за машину для циркуляції.
Крім того, що частина суспільного багатства закріплюється в цій
непродуктивній формі, зношення грошей потребує постійного заміщування
їх або перетворення дедалі більшої кількости суспільної праці — в
формі продукту — на дедалі більше золота й срібла. Ці витрати заміщування
доходять чималих розмірів у капіталістично розвинених націй, бо
в них взагалі чималу частину багатства закріплено в формі грошей.
Золото й срібло, як грошові товари, становлять для суспільства витрати
циркуляції, що походять лише із суспільної форми продукції. Це — faux frais
товарової продукції взагалі, які зростають з розвитком товарової продукції,
а особливо капіталістичної продукції. Це — частина суспільного
багатства, що її доводиться жертвувати процесові циркуляції\footnote{
„Гроші, що циркулюють у якійсь країні, є певна частина капіталу
країни, цілком вилучена з продуктивних призначень для того, щоб полегшити
або збільшити продуктивність решти. Отже, певна сума багатства доконечна
для того, щоб золото могло правити за засіб циркуляції, так само, як доконечна
вона для того, щоб виготовити машину, що полегшує якусь іншу
продукцію“.

(The money circulating in a country is a certain portion of the capital of the
country, absolutely withdrawn from productive purposes, in order to facilitate
or increase the productiveness of the remainder; a certain amount of wealth
is. therefore, as necessary, in order to adopt gold as a circulating medium,
as it is to make a machine, in order to facilitate any other production (Economist,
vol. V, стор. 519).
}.

\subsection{Витрати зберігання}

Витрати циркуляції, що випливають з простої зміни форми вартости,
з циркуляції, розглядуваної ідеально, не входять у вартість товарів.
Витрачувані на них частини капіталу, оскільки ми маємо на увазі капіталіста,
становлять лише одбави з продуктивно витрачуваного капіталу.
Іншої природи є ті витрати циркуляції, що ми їх зараз розглядаємо. Вони
можуть, походити з процесів продукції, що лише тривають далі в циркуляції,
і що їхній продуктивний характер, отже, лише замасковується формою
циркуляції. З другого боку, із суспільного погляду, вони можуть бути
просто витратами, непродуктивними витратами хоч живої, хоч зречевленої
праці, але саме в наслідок цього вони можуть діяти так, що утворюють
вартість для індивідуального капіталіста, становлять надвишку до продажної
ціни його товару. Це випливає вже з того, що ці витрати різні в різних галузях
продукції, а місцями різні й для різних індивідуальних капіталів у межах
тієї самої галузі продукції. В наслідок того, що їх додається до ціни
товару, вони розподіляються відповідно до того, в яких розмірах припадають
вони індивідуальним капіталістам. Але кожна праця, що додає
вартість, може додавати також і додаткову вартість і на капіталістичній
основі завжди буде додавати додаткову вартість, бо утворювана цією працею
вартість залежить від її власної величини, а утворювана нею додаткова
вартість залежить від того розміру, в якому оплачує її капіталіст.
\index{ii}{0092}  %% посилання на сторінку оригінального видання
Отже, витрати, що удорожчують товар, нічого не додаючи до його
споживної вартости, отже, витрати, що для суспільства належать до faux
frais продукції, можуть становити джерело збагачення індивідуального
капіталіста. З другого боку, ці витрати циркуляції не втрачають характеру
непродуктивних витрат, від того, що надвишка, яку вони додають до
ціни товару, лише рівномірно розподіляє ці витрати. Напр., страхові
товариства розподіляють утрати індивідуальних капіталістів між клясою
капіталістів. Однак це не заваджає тому, що вирівняні таким способом
утрати все ж, як і раніш, з погляду суспільного сукупного капіталу, є втрати.

\subsubsection{Утворення запасу взагалі}

Протягом того часу, коли продукт існує у формі товарового капіталу
або перебуває на ринку, отже, протягом усього часу між процесом
продукції, відки він виходить, і процесом споживання, в який він увіходить,
він становить товаровий запас. Як товар на ринку, а тому й у формі
запасу, товаровий капітал з’являється двічі в кожному кругобігу, — один
раз як товаровий продукт самого капіталу, що процесує, — капіталу, що
його кругобіг розглядається; другий раз, навпаки, як товаровий продукт
іншого капіталу, як продукт, що мусить бути на ринку, щоб його
можна було купити й перетворити на продуктивний капітал. Звичайно,
можливо, що цей останній товаровий капітал продукується лише на
замовлення. Тоді постає перерва, що триває доти, доки його спродукують.
Але перебіг процесу продукції та репродукції потребує, щоб деяка маса
товарів (засобів продукції) завжди була на ринку і, значить, становила запас.
Так само продуктивний капітал охоплює й закуп робочої сили, і грошова
форма є тут лише форма вартости засобів існування, що їх більшість робітник
мусить знаходити на ринку. В цьому параграфі ми докладніше зупинимось
на цьому питанні. Але вже й тепер ми доходимо такого пункту:
коли дивитися з погляду капітальної вартости, що процесує, капітальної
вартости, яка перетворилась на товаровий продукт і мусить тепер продатись,
тобто знову перетворитись на гроші, яка, отже, функціонує тепер на ринку
як товаровий капітал, то той стан її, що в ньому вона утворює запас, є
недоцільне вимушене перебування на ринку. Що швидше відбувається
продаж, то швидше перебігає процес репродукції. Затримка в
перетворенні форми $Т' — Г'$ заваджає реальному обмінові речовин, що
мусить відбуватися в кругобігу капіталу, так само, як і дальшому функціонуванні
його в ролі продуктивного капіталу. З другого боку, постійна
наявність товару на ринку, товаровий запас, є для $Г — Т$ умова перебігу
процесу репродукції, а також і вкладання нового або додаткового
капіталу.

Шоб товаровий капітал міг лишатись на ринку як товаровий запас,
потрібні будівлі, магазини, приміщення для товарів, товарові склади, отже,
потрібна витрата сталого капіталу; так само потрібна і оплата робочої
сили для складання товарів у приміщення. Крім того, товари псуються
і зазнають шкідливих стихійних впливів. Щоб уберегти їх від цього,
\parbreak{}  %% абзац продовжується на наступній сторінці
