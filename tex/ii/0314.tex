рівнює (II b) V. Ці відношення лишаються якісно вирішальні при всякому
розподілі всього річного продукту, оскільки він дійсно входить у процес
річної репродукції, упосереднюваної циркуляцією I (v + m) можна реалізувати
лише в II с, так само, як ІІ с в його функції складової частини
продуктивного капіталу можна відновити лише за допомогою цієї реалізації;
так само (II b) v можна реалізувати лише в частині (II а) т, і лише
таким способом (II b) v можна знову перетворити на його форму грошового
капіталу. Звичайно, це має силу лише за тієї умови, що все це дійсно
є результат самого процесу репродукції, отже, коли, напр., капіталісти
II b не одержують грошового капіталу для v за допомогою кредиту з
якихось інших джерел. Навпаки, щодо кількісного боку, то обміни різних
частин річного продукту можуть відбуватися з такою пропорційністю, як
подано вище, лише остільки, оскільки маштаб та відношення вартости
продукції лишаються незмінні, і оскільки ці точно визначені відношення
не зазнають змін в наслідок зовнішньої торгівлі.

Коли, за прикладом А. Сміса, казали, що I (v + m), розкладається на
II с, II с розкладається на I (v + m), або, як він часто каже своїм
звичаєм іще недоладніше, що I (v + m) становлять складові частини ціни,
зглядно вартости, він каже value in exchange II с, а II с становить усю
складову частину вартости I (v + m), то можна й треба було б сказати
також, що (II b) v розкладається на (II а) m, або (II а) m на (II b) v,
або що (II b) v становить складову частину додаткової вартости II а, і
навпаки: додаткова вартість розкладалась би таким чином на заробітну
плату, зглядно на змінний капітал, а змінний капітал становив би „складову
частину“ додаткової вартости. І справді, така недоречність дійсно
є у А. Сміса, бо заробітна плата визначається в нього вартістю доконечних
засобів існування, і ці товарові вартості знову таки визначаються
вартістю вміщених у них заробітної плати (змінного капіталу) і додаткової
вартости. Він до того захопився тими частинами, на які при капіталістичній
основі продукції можна розкласти вартість, спродуковану протягом
одного робочого дня, — а саме v + m, — що цілком забуває про те, що при
простому товаровому обміні цілком байдуже, чи складаються еквіваленти,
які існують в різних натуральних формах, з оплаченої чи неоплаченої праці:
бо в обох випадках вони коштують однакову кількість праці, витраченої
на їхню продукцію; і що так само байдуже, чи є товар якогось А засоби
продукції, а товар якогось В — засоби споживання, чи має один
товар функціонувати після продажу як складова частина капіталу, а
другий, навпаки, входить у фонд споживання його, і, за Адамом, споживається
як дохід. В який спосіб індивідуальний покупець вживає свій
товар, це не має жодного чинення до обміну товарів, до сфери циркуляції,
і не стосується вартости товару. Це ані трохи не змінюється
від того, що при аналізі циркуляції всього річного суспільного продукту
треба взяти на увагу певний характер вживання, момент споживання
різних складових частин цього продукту.

При вище констатованому обміні (II b) v на рівновартісну частину
(II a) m і при дальших обмінах між (II a) m і (II b) v зовсім не припус-
