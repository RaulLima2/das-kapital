не різні самостійні вкладання капіталу, а різні частини того самого продуктивного
капіталу, які в різних сферах приміщення становлять різні
частини сукупної вартости цього капіталу. Отже, це — ріжниці, що випливають
з розподілу самого продуктивного капіталу відповідно до
обставин і тому мають силу лише для цього останнього. Але цьому знову
суперечить та обставина, що торговельний капітал, як виключно обіговий,
протиставиться основному, бо сам Сміс каже: „капітал торговця є
цілком обіговий капітал“. А справді це — капітал, що функціонує лише
в межах сфери циркуляції, і як такий він взагалі протистоїть продуктивному
капіталові, вкладеному в процес продукції, але саме тому його не
можна протиставляти як поточну (обігову) складову частину продуктивного
капіталу основній складовій частині продуктивного капіталу.

В прикладах, що їх наводить Сміс, він визначає, як основний капітал,
знаряддя праці; як обіговий капітал — ту частину капіталу, що витрачена
на заробітну плату й сировинний матеріял, зараховуючи сюди й допоміжні
матеріяли, „оплачувані із зиском в ціні продуктів“ („repaid with а
profit by the price of work“).

Отже, насамперед, за вихідний пункт тут є лише різні елементи процесу
праці: робоча сила (праця) й сировинний матеріял на одному боці,
знаряддя праці — на другому. Але все це є складові частини капіталу, бо
в них вкладено суму вартости, що має функціонувати як капітал. Остільки
це є речові елементи, способи буття продуктивного капіталу, тобто
капіталу, що функціонує в продукційному процесі. Чому ж одна частина
зветься основною? Тому що „деякі частини капіталу мусять бути фіксовані
в засобах праці“ („some parts of the capital must be fixed in the
instruments of trade“). Але друга частина теж є фіксована в заробітній
платі й сировинному матеріялі. Тимчасом, машини і „знаряддя праці... та
подібні речі... дають дохід або зиск, не змінюючи власника, не циркулюючи
далі. Тому такі капітали можна назвати основними капіталами у
власному значенні цього слова“.

Візьмімо, напр., гірничу справу. Сировинного матеріялу тут зовсім не
застосовується, бо предмет праці, прим., мідь, є продукт природи, що
його треба лише видобути за допомогою праці. Мідь, що її лише треба
видобути, — продукт процесу, що потім циркулює як товар, зглядно як
товаровий капітал, не становить жодного елемента продуктивного капіталу.
Жодна частина його вартости не вкладена в нього. З другого боку,
інші елементи продукційного процесу, робоча сила й допоміжні матеріяли,
як от вугілля, вода і т. ін. так само не входять речово в продукт.
Вугілля зуживається цілком, і тільки його вартість ввіходить у продукт,
цілком так само, як частина вартости машин тощо ввіходить у продукт.
Нарешті, робітник лишається так само самостійним проти продукту, міді,
як і машина. Тільки вартість, спродукована його працею, є тепер складова
частина вартости міді. Отже, в цьому прикладі жодна з складових
частин продуктивного капіталу не змінює власника (masters) і жодна з
них не циркулює далі, бо жодна з них не ввіходить речово в продукт.
Де ж тут обіговий капітал? Згідно з власним визначенням А. Сміса до-
