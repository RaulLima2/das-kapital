чена на працю, є змінний капітал, і те, що, згідно з ним, вона є обіговий
капітал протилежно до основного.

Вже з самого початку очевидно, що визначення капіталу, витраченого
на робочу силу, як обігового або поточного, є другорядне визначення,
в якому зникають його differentia specifica в продукційному процесі;
бо, з одного боку, при такому визначенні капітали, витрачений на працю
й витрачений на сировинний матеріял і т. ін., вважається за рівнозначні;
рубрика, що ототожнює частину сталого капіталу зі змінним, цілком ігнорує
differentia specifica змінного капіталу протилежно до сталого. З другого
боку, частини капіталу, витрачені на працю і на засоби праці, хоч
і протиставиться одна одній, але зовсім не в тому розумінні, що вони
цілком різним способом входять у продукцію вартости, а лише в тому
розумінні, що обидві вони переносять на продукт свою дану вартість
тільки в різні переміжки часу.

В усіх цих випадках ідеться тільки про те, як дана вартість, що її
витрачається на процес продукції товару — хоч то буде заробітна плата,
ціна сировинного матеріялу або ціна засобів праці, — переноситься на
продукт, а, значить, і як вона циркулює за допомогою продукту і в наслідок
продажу його повертається до свого вихідного пункту, або як
покривається її. Єдина ріжниця тут у цьому „як“, в особливому способі
перенесення, а, значить, і циркуляції цієї вартости.

Чи виплачується кожного разу заздалегідь визначену контрактом ціну
робочої сили грішми, чи засобами існування — це нічого не змінює в її
характері, а саме в тому, що вона є певна дана ціна. А проте, при виплаті
заробітної плати грішми цілком очевидно, що самі гроші не входять
у процес продукції, як входять засоби продукції, що в них не лише
вартість, а й сама речовина входить у продукційний процес. А коли ж,
навпаки, засоби існування, що їх робітник купує на свою заробітну плату,
безпосередньо підводиться як речову форму обігового капіталу під
одну рубрику з сировинними матеріялами тощо й протиставиться засобам
праці, то це надає справі іншого вигляду. Коли вартість цих речей,
тобто засобів продукції, в процесі праці переноситься на продукт, то
вартість тих других речей, тобто засобів існування, знову з’являється в
робочій силі, що їх спожила, і через функціонування робочої сили її знову
таки переноситься на продукт. І в тому, і в другому разі однаково
йдеться про просту повторну появу в продукті вартостей, авансованих
підчас продукції. (Фізіократи брали це серйозно, а тому не визнавали,
що промислова праця створює додаткову вартість). Напр., Вейленд
пише в цитованому вже місці: „Байдуже, в якій саме формі з’являється
знову капітал... різні відміни харчу, одягу й житла, потрібні для існування
й добробуту людей, також змінюються. Їх зуживається з плином
часу, і вартість їхня з’являється знову і т. ін.“. (Elements of Political
Economy, ст. 31, 32). Капітальні вартості, авансовані в формі
засобів продукції й засобів існування для продукції тут однаково знову
з’являються в вартості продукту. Так щасливо досягається перетворення
капіталістичного процесу продукції на цілковиту містерію, а похо-
