\index{ii}{0040}  %% посилання на сторінку оригінального видання
Щоб кругобіг відбувався нормально, $Т'$ мусить продаватись за своєю
вартістю і геть усе без остачі. Далі, $Т — Г — Т$ має в собі не лише заміщення
одного товару другим, але заміщення при тих самих відношеннях
вартости. Ми припустили, що це тут саме так і відбувається. Але в дійсності
вартості засобів продукції змінюються; саме капіталістичному способові
продукції властива постійна зміна відношень вартости в наслідок постійних
змін у продуктивності праці, що й характеризує капіталістичну
продукцію. Тут ми лише згадуємо про цю зміну в вартості чинників
продукції, а дослідимо її далі. Перетворення елементів продукції на
товаровий продукт, П на $Т'$, відбувається в сфері продукції; зворотне
перетворення $Т'$ на П — у сфері циркуляції. За посередника йому
править проста товарова метаморфоза. Але своїм змістом це зворотнє
перетворення є момент у процесі репродукції, розглядуваному як ціле.
$Т — Г — Т$ як форма циркуляції капіталу має в собі функціонально визначений
обмін речовин. Далі перетворенням $Т — Г — Т$ зумовлено те, що Т
= елементам продукції товарової маси $Т'$, і що ці їхні взаємні первісні
відношення вартостей лишаються незмінні. Отже, тут припускається не лише
те, що товари купується за їхню вартість, а також і те, що протягом
кругобігу вони не зазнають жадної зміни вартости; а де цього немає,
там процес не може відбуватись нормально.

Г в $Г... Г'$ є первісна форма капітальної вартости, яка скидає з себе
цю форму, щоб знову потім її набрати. Г в $П... Т' — Г' — Т... П$
є форма, що її набирається лише в процесі і в межах цього самого
процесу знову скидається. Грошова форма з’являється тут лише як минуща
самостійна форма вартости капіталу; капітал у формі $Т' т$ак само прагне
набути цієї грошової форми, як капітал у формі $Г'$, скоро тільки він перетворився
на цю форму нібито на лялечку, прагне скинути її, щоб знову перетворитися
на форму продуктивного капіталу. Поки цей капітал застигає в
грошовій формі, він не функціонує як капітал, а тому не зростає в своїй
вартості; капітал лежить без діла. Г діє тут як засіб циркуляції, але як засіб
циркуляції капіталу. Позірна самостійність, що її має грошова форма
капітальної вартости в першій формі їі кругобігу (грошового капіталу),
зникає в цій другій формі, що таким чином править за критику форми І і
зводить її лише на особливу форму. Коли друга метаморфоза $Г — Т$ наражається
на перешкоди (напр., коли немає на ринку засобів продукції),
то кругобіг, перебіг процесу репродукції переривається так само, як
і тоді, коли капітал лежить нерухомо у формі товарового капіталу. Але
ріжниця тут ось яка: капітал у грошовій формі може довший час чекати,
ніж у минущій товаровій формі. Не функціонуючи як грошовий капітал,
він, однак, не перестає бути грішми; але він перестає бути товаром
і взагалі споживною вартістю, коли занадто довго затримується в своїй
функції товарового капіталу. Подруге, в грошовій формі може він,
замість своєї первісної форми продуктивного капіталу, набрати іншої,
тимчасом як у формі $Т'$ він взагалі не може зрушити з місця.

$Т' — Г' — Т$ має в собі лише для $Т'$, відповідно до його форми,
акти циркуляції, що являють моменти його репродукції. Але, щоб відбулось
\index{ii}{0041}  %% посилання на сторінку оригінального видання
$Т' — Г' — Т$, потрібна справжня репродукція того Т, що на нього
перетворюється $Т'$; а ця репродукція зумовлена процесами репродукції,
які є поза процесом репродукції індивідуального капіталу, що його
репрезентує $Т'$.

У формі I акт $Г — Т\splitfrac{Р}{Зп} п$ідготовлює лише перше перетворення
грошового капіталу на продуктивний капітал; у формі II він підготовлює
зворотне перетворення з товарового капіталу на продуктивний капітал;
отже, оскільки вкладування промислового капіталу лишається те саме,
він підготовляє зворотне перетворення товарового капіталу на ті самі
елементи продукції, що з них він постав. Тому тут, як і в формі I, цей
акт з’являється як підготовча фаза продукційного процесу, але як
поворот до нього, як відновлення його, а тому — як предтеча процесу
репродукції, отже, також повторення процесу зростання вартости.

Треба тут ще раз зауважити, що акт $Г — Р$ є не простий товарообмін,
а купівля товару Р, який повинен служити для продукції додаткової
вартости так само, як $Г — Зп$ є лише процедура, матеріяльно неминуча
для здійснення цієї мети.

Після того як відбувся акт $Г — Т\splitfrac{Р}{Зп}$, $Г п$еретворюється знову
на продуктивний капітал, на П, і знову починає кругобіг.

Отже, розгорнута форма кругобігу $П... Т' — Г' — Т... П т$ака:

$П … Т'$ (Т + т) — — (Г + г) — $Т\splitfrac{Р}{Зп} … П. — т

П$еретворення грошового капіталу на продуктивний капітал є купівля
товарів для продукції товарів. Лише оскільки споживання є продуктивне
споживання, воно входить у кругобіг самого капіталу; умова цього споживання
та, що за посередництвом споживаних у такий спосіб товарів утворюється
додаткова вартість. І це є щось дуже відмінне від тієї продукції,
і навіть від товарової продукції, що її мета — існування продуцента;
отаке зумовлене продукцією додаткової вартости заміщення товару
товаром є щось цілком відмінне від обміну продуктів самого по собі,
обміну, що лише упосереднюється грішми. Але так розглядають справу
економісти, щоб довести, що неможлива будь-яка перепродукція.

Крім продуктивного споживання Г, що перетворюється на Р і Зп,
кругобіг містить у собі перший член $Г — Р$, що для робітника є $Р — Г =
Т — Г$. З циркуляції робітника $Р — Г — Т$, що має в собі його споживання,
в кругобіг капіталу ввіходить лише перший член як результат
$Г — Р$. Другий акт, а саме $Г — Т$, не входить у циркуляцію індивідуального
\index{ii}{0042}  %% посилання на сторінку оригінального видання
капіталу, хоч і походить з неї. Але постійна наявність кляси
робітників потрібна для кляси капіталістів, а тому й споживання робітників,
яке упосереднюється через $Г — Т$.

Акт $Т' — Г'$ щодо продовження кругобігу капітальної вартости, а
також щодо споживання додаткової вартости від капіталіста, припускає
лише те, що $Т' п$еретворено на гроші, продано. Його купують звичайно
лише тому, що предмет являє споживну вартість, а, значить, придатний
до якогобудь споживання, хоч продуктивного, хоч особистого. А коли
$Т'$ і далі циркулює, прим., у руках купця, що купив пряжу, то
це насамперед ніяк не зачіпає продовження кругобігу того індивідуального
капіталу, що спродукував пряжу й продав її купцеві. Цілий процес
триває далі, а разом з ним триває й зумовлене ним особисте споживання
капіталіста й робітника. Ця обставина має велике значіння при вивченні
криз.

Скоро $Т' п$родано, перетворено на гроші, воно може зворотно перетворитись
на реальні чинники процесу праці, а, значить, і процесу репродукції.
Чи купив $Т'$ остаточний споживач, чи купець, який хоче знову продати
його, це безпосередньо не змінює справи. Об’єм товарових мас, утворюваних
капіталістичною продукцією, визначається маштабом цієї продукції
та потребою постійно поширювати її, але зовсім не наперед визначеним
розміром попиту й подання, не розміром потреб, що їх треба задовольнити.
За безпосереднього покупця масової продукції може бути, крім
інших промислових капіталістів, лише гуртовий покупець. У певних
межах процес репродукції може відбуватись у попередньому або й
поширеному розмірі, хоч виштовхнуті з нього товари в дійсності не
ввійшли в особисте або продуктивне споживання. Споживання товарів не
включено в той кругобіг капіталу, що з нього вони постали. Напр.,
скоро пряжу продано, кругобіг капітальної вартости, втіленої в пряжі,
може початися знову, незалежно від того, що сталося з проданою пряжею.
Доки продукт продається, то з погляду капіталістичного продуцента
все йде своїм нормальним порядком. Кругобіг капітальної вартости, що
її репрезентує продукт, не переривається. А коли цей процес поширюється
— що включає поширене продуктивне споживання засобів продукції,
— то ця репродукція капіталу може супроводитися поширеним
особистим споживанням (а, значить, попитом) робітників, бо продуктивне
споживання підготовлює й упосереднює цей процес. Таким
чином продукція додаткової вартости, а разом з нею і особисте
споживання капіталіста, може зростати, ввесь процес репродукції може
перебувати в стані найбільшого розцвіту, і все ж більшість товарів
може переходити в сферу споживання лише позірно, а в
дійсності лежати непроданою у перекупників, отже, фактично перебувати
ще на ринку. Але потік товарів котиться один по одному, і нарешті
виявляється, що попередній потік увібрано в споживання лише позірно.
Товарові капітали змагаються один з одним за своє місце на ринку. Ті,
що прийшли пізніше, продають нижчою ціною, аби тільки продати.
Попередні потоки ще не збуто, а вже надходять строки виплат за них.
\parbreak{}  %% абзац продовжується на наступній сторінці
