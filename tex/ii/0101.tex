але його розміри більшають. Отже, об’єм товарового запасу, що
бубнявіє в наслідок застою циркуляції, можна помилково вважати за
ознаку поширення процесу репродукції, і це особливо тоді, коли з розвитком
кредитової системи справжній рух може містифікуватись.

Витрати на утворення запасу складаються: 1) з кількісного зменшення
маси продукту (напр., запасу борошна), 2) з якісного псування, 3) із зречевленої
або живої праці, що потрібна для зберігання запасу.

III. Витрати на транспорт

Тут нам не треба вдаватись у всі подробиці щодо витрат циркуляції,
як, прим., пакування, сортування тощо. Загальний закон той, що всі
витрати циркуляції, які випливають лише з перетворення
форми товару, не додають до нього жодної
вартости. Це — лише витрати на реалізацію вартости або на перетворення
її з однієї форми на іншу. Капітал, витрачений на ці видатки
(разом із працею, що є під його орудою), належить до faux frais капіталістичної
продукції. Покриття цих витрат мусить відбуватись з додаткового
продукту і становить, розглядаючи цілу капіталістичну клясу, одбаву
з додаткової вартости або додаткового продукту цілком так само,
як час, потрібний робітникові на закуп засобів його існування, є для нього
змарнований час. Але витрати на транспорт відіграють дуже важливу
ролю, й тому на них треба тут трохи зупинитись.

У межах кругобігу капіталу й метаморфози товарів, що становить
відділ цього кругобігу, відбувається обмін речовин суспільної праці. Цей
обмін речовин може зумовлювати переміщення продуктів, їхній справжній
рух з одного місця на інше. Але циркуляція товарів може відбуватись і
без їхнього фізичного руху, а транспорт продуктів — без товарової циркуляції,
ба навіть без безпосереднього обміну продуктів. Будинок, що
його А продає В, циркулює як товар, але лишається на тому самому
місці. Рухомі товарові вартості, прим., бавовна або чавун, лишаються
на тому самому товаровому складі в той самий час, як вони перебігають
десятки різних процесів циркуляції, купуються спекулянтами й знову
продаються17). Справді тут рухається лише титул власности на річ, а не
сама річ. З другого боку, напр., у царстві інків, транспортова промисловість
відігравала велику ролю, хоч суспільний продукт не циркулював як
товар і не розподілялось його за допомогою мінової торговлі.

Тому, хоч транспортова промисловість на основі капіталістичної продукції
видається причиною витрат циркуляції, однак ця особлива форма
виявлення їх зовсім не змінює справи.

Маса продуктів не більшає в наслідок перевозу їх. Всі зміни, спричинені
перевозом у природних властивостях продуктів, за деякими винятками,
є не навмисний корисний ефект, а неминуче лихо. Але споживна вартість
речей реалізується лише в їх споживанні, а споживання їх може по-

17) Шторх зве цю циркуляцію factice.
