вель, — цей процес загальної товарової циркуляції становить разом з тим,
як стадія в процесі самостійного кругобігу капіталу, перетворення капітальної
вартости з її грошової форми на її продуктивну форму, або
коротше — перетворення грошового капіталу на продуктивний капітал.
Отже, у тій схемі кругобігу, що її ми насамперед тут розглядаємо,
гроші з’являються як перший носій капітальної вартости, а тому грошовий
капітал — як форма, що в ній авансується капітал.

Як грошовий капітал, він перебуває в такому стані, що може виконувати
функції грошей, в даному разі функції загального купівельного
засобу і загального засобу виплати. (Останнє остільки, оскільки робочу
силу, хоч і раніш куплену, оплачується лише після того, як вона функціонувала.
Оскільки готових продукційних засобів немає на ринку, а
треба їх замовляти, в процесі Г — Зп гроші теж правлять за засіб виплати).
Ця здібність випливає не з того, що грошовий капітал є капітал, а з
того, що він — гроші.

З другого боку, капітальна вартість у грошовому стані може виконувати
лише функції грошей і жадних інших. Що перетворює функції
грошей на функції капіталу, це — їхня певна роля в русі капіталу, а тому
і зв’язок стадії, що в ній вони з’являються, з іншими стадіями його
кругобігу. Наприклад, у випадку, який тут насамперед розглядається,
гроші перетворюються на товари, що їхнє сполучення утворює натуральну
форму продуктивного капіталу, отже, форму, що лятентно, в
змозі, вже таїть у собі наслідок капіталістичного процесу продукції.

Частина грошей, що в Г — Т < Р Зп виконує функцію грошового капіталу, вивершивши цю циркуляцію, сама
переходить до такої функції, що в
ній зникає її характер капіталу, але лишається її характер грошей. Циркуляція
грошового капіталу Г розпадається на Г — Зп і Г — Р, купівлю
засобів продукції та купівлю робочої сили. Розгляньмо останній процес
сам по собі. Г — Р є купівля робочої сили з боку капіталіста; з боку
робітника, власника робочої сили, це є продаж робочої сили — ми можемо
сказати тут — продаж праці, бо тут наперед припускається форму заробітної
плати. Те, що для покупця є Г — Т (= Г — Р), тут, як і в усякій
купівлі, для продавця (робітника) є Р — Г (= Т — Г), продаж його робочої
сили. Це — перша стадія циркуляції, або перша метаморфоза товару
(книга І, розд. III, 2а); це — з боку продавця праці — перетворення його
товару на його грошову форму. Одержані таким чином гроші робітник
поступінно витрачає на певну суму товарів, що задовольняють його
потреби, на предмети споживання. Отже, ціла циркуляція його товару
виявляється як Р — Г — Т, тобто поперше Р — Г (= Т — Г), а подруге Г — Т,
отже, у загальній формі простої товарової циркуляції Т — Г — Т, де гроші
фігурують як простий минущий засіб циркуляції, як простий посередник
в обміні товару на товар.

Г — Р є характеристичний момент перетворення грошового капіталу
на продуктивний капітал, бо це є істотна умова для того, щоб вартість,
