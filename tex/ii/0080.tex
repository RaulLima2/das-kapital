дукції\footnote{
Час продукції тут взято в активному значенні: час продукції засобів продукції
є тут не час, що протягом його їх продукується, а час, що протягом його
вони беруть участь у процесі продукції товарового продукту.
} та його часом функціонування. Отже, час продукції засобів
продукції взагалі охоплює 1) час, що протягом його вони функціонують
як засоби продукції, тобто придаються в процесі продукції;
2) павзи, що протягом їх продукційний процес, а значить, функціонування
належних йому засобів продукції, переривається; 3) час, що
протягом його вони, хоч і є напоготові як умови процесу, і тому являють
уже продуктивний капітал, але ще не ввійшли в процес продукції.

Розглядувана досі ріжниця завжди є ріжниця між часом перебування
продуктивного капіталу в сфері продукції і часом перебування його в
процесі продукції. Але сам процес продукції може зумовлювати перерви
в процесі праці, а тому й у часі праці, зумовлювати переміжки, коли
предмет праці зазнає впливу фізичних процесів без дальшого прикладання
людської праці. Продукційний процес, а тому й функціонування засобів
продукції в цьому разі триває далі, хоч процес праці, а значить, і функціонування
засобів продукції як засобів праці, перервано. Так буває, напр.,
з зерном, що його висіяно, з вином, що ферментує в льоху, з матеріялом
праці в багатьох мануфактурах, як, напр., на чинбарнях, де цей матеріял
зазнає впливу хемічних процесів. Час продукції тут більший, ніж час
праці. Ріжниця між ними — це лишок часу продукції над часом праці.
Цей надлишок ґрунтується завжди на тому, що продуктивний капітал
перебуває в лятентному стані в сфері продукції, не функціонуючи в
самому процесі продукції, або на тому, що він функціонує в процесі
продукції, але не перебуває в процесі праці.

Та частина лятентного продуктивного капіталу, що її наготовлено
лише як умову для продукційного процесу, напр., бавовна, вугілля й
т. ін., в прядільні, не функціонує ні як продуктотворча, ні як вартостетворча.
Це — бездіяльний капітал, хоч така його бездіяльність становить
умову для безперервного перебігу продукційного процесу. Будівлі, апарати
тощо, потрібні як сховища продуктивного запасу (лятентного капіталу),
є умови продукційного процесу, а тому становлять складові частини авансованого
продуктивного капіталу. Вони виконують свою функцію, як сховища
продуктивних складових частин в попередній стадії. Оскільки на цій стадії
потрібні процеси праці, остільки вони удорожнюють сировинний матеріял
тощо, але вони є продуктивна праця і утворюють додаткову
вартість, бо частину цієї праці, як і всякої іншої найманої праці, не
оплачується. Нормальні перерви цілого продукційного процесу, тобто
переміжки, що в них продуктивний капітал не функціонує, не продукують
ні вартости, ні додаткової вартости. Відси постає намагання примусити
працювати навіть уночі (книга І, розд, VIII, 4). — Переміжки в часі праці,
що їх мусить перейти предмет праці під час самого продукційного процесу,
не утворюють ні вартости, ні додаткової вартости; але вони розвивають
продукт, становлять частину його життя, процес, що його він