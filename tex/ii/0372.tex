лише як засіб циркуляції (купівельні засоби). Це — цілковита протилежність
натуральному господарству, як воно переважає на основі всіх систем
залежности (втім і кріпацької) і ще більше на основі більш-менш примітивних
громад, усе одно — хоч вони сполучені з відносинами залежности
й рабства, хоч ні.

При системі рабства грошовий капітал, витрачуваний на закуп робочої
сили, відіграє ролю грошової форми основного капіталу, замішуваного
лише поступінно, поки мине активний період життя раба. Тому
в атенян бариш, що його одержує рабовласник безпосередньо, промислово
використовуючи свого раба, або посередньо, винаймаючи його іншим
промисловцям (напр., на працю в копальнях), розглядалось лише як
процент (разом з амортизацією) на авансований грошовий капітал, —
цілком так само, як при капіталістичній продукції частину додаткової
вартости плюс зношування основного капіталу промисловий капіталіст залічує
до проценту й заміщення свого основного капіталу; цілком так само, як
це звичайно роблять капіталісти, що винаймають основний капітал (будинки,
машини тощо). Простих хатніх рабів, хоч вони служать для виконування
доконечних робіт, хоч для демонстрування розкошів, тут не береться на
увагу, вони відповідають нашій клясі слуг. Але й система рабства —
оскільки вона є панівна форма продуктивної праці в хліборобстві, мануфактурі,
судноплавстві тощо, як було в розвинених державах Греччини та
в Римі, — зберігає елемент натурального господарства. Самий ринок
рабів постійно поповнюється товаром-робочою силою за допомогою воєн,
піратства і т. ін.; з свого боку це піратство не упосереднюється процесом
циркуляції, а є натуральне привлащення чужої робочої сили
безпосереднім фізичним примусом. Навіть в Сполучених Штатах, після
того як проміжна територія між північними штатами найманої праці й
південними штатами рабства перетворилась на країну вирощування рабів
для півдня, де, отже, сами раби, подавані на невільничий ринок, стали
елементом річної репродукції, цього через деякий час було вже не досить,
і для поповнення ринку ще довго — оскільки це було можливе, — й
далі провадили африканську работорговлю.

b) Відпливи й зворотні припливи грошей при перетворенні річного
продукту, які відбуваються стихійно на основі капіталістичної продукції;
одночасні авансування основних капіталів на всю величину їхньої вартости
— і поступінне, що триває протягом довголітніх періодів, вилучення
їхньої вартости з циркуляції, отже, постійне відновлення їх у грошовій
формі за допомогою щорічного скарботворення, яке в суті цілком
відмінне від того скарботворення, що відбувається рівнобіжно з ним і
ґрунтується на щорічній новій продукції золота; різний протяг часу, що
на нього, залежно від довжини періодів продукції товарів, авансується
гроші, отже, і потреба завжди знову нагромаджувати їх, раніше ніж
можна буде вилучати їх назад з циркуляції через продаж товарів; ріжниця
в протязі часу, що на нього авансується гроші, яка постає в наслідок
хоча б різної віддалі місця продукції від ринку збуту; так само ріжниця
в величині та строках зворотного припливу грошових сум, залежно від
