\parcont{}  %% абзац починається на попередній сторінці
\index{ii}{0329}  %% посилання на сторінку оригінального видання
капітальну вартість, тобто ту частину вартости, що на неї робітник купує
засоби репродукції самого себе, і додаткову вартість, що її капіталіст
може витрачати на своє власне особисте споживання, — все ж з суспільного
погляду, частину суспільного робочого дня витрачається виключно на
продукцію свіжого сталого капіталу, а саме — продуктів, призначених
виключно для того, щоб функціонувати в процесі праці як засоби
продукції, отже, і для того, щоб функціонувати як сталий капітал у
процесі зростання вартости, що ним супроводиться цей процес праці.
Згідно з нашим припущенням, ввесь суспільний робочий день виражається в
грошовій вартості в 3000, з чого лише 1/3 = 1000 продукується в підрозділі
II, який продукує засоби споживання, тобто товари, що в них, кінецькінцем,
реалізується вся змінна капітальна вартість і вся додаткова вартість
суспільства. Отже, згідно з цим припущенням, 2/3 суспільного робочого
дня вживається на продукцію нового сталого капіталу. Хоч з погляду
індивідуальних капіталістів і робітників підрозділу І ці 2/3 суспільного
робочого дня служать лише для продукції змінної капітальної вартости
плюс додаткова вартість цілком так само, як і остання третина суспільного
робочого дня в підрозділі II, однак з суспільного погляду — а
також розглядувані щодо споживної вартости продукту — ці 2/3 суспільного
робочого дня продукують лише заміщення сталого капіталу, який
перебуває або вже зужиткований в процесі продуктивного споживання.
Також з індивідуального погляду, хоч ціла вартість, що її продукують ці
2/3 робочого дня, дорівнює лише змінній капітальній вартості плюс додаткова
вартість для її продуцентів, однак вони зовсім не продукують
споживних вартостей такого роду, щоб на них можна було витрачати
заробітну плату або додаткову вартість; продукт цих 2/3 робочого дня є
засоби продукції.

Насамперед треба зазначити, що жодна частина суспільного робочого
дня ні в І, ні в II не служить для продукції вартости сталого капіталу,
застосованого в цих двох великих сферах продукції та діющого в них.
Вони продукують лише новододавану вартість, 2000 І ($v + m$) + 1000
II ($v + m$), додатково до сталої капітальної вартости = 4000 І c + 2000
II c. Нова вартість, спродукована в формі засобів продукції, ще не є
сталий капітал. Вона має лише призначення функціонувати як сталий
капітал в майбутньому.

Цілий продукт підрозділу II — засоби споживання — розглядуваний щодо
його споживної вартости, в його конкретній натуральній формі, є продукт,
спродукований в підрозділі II однією третиною суспільного дня;
це є продукт праці в її конкретних формах, праця ткача, праця пекаря
і т. ін., що застосовується в цьому підрозділі, продукт цієї праці, оскільки
вона функціонує як суб’єктивний елемент процесу праці. Навпаки,
щодо сталої частини вартости цього продукту II, то вона лише знову
з’являється в новій споживній вартості, в новій натуральній формі, у
формі засобів споживання, тимчасом як раніше вона існувала в формі засобів
продукції. В наслідок процесу праці вартість цієї частини перенесено
з її попередньої натуральної форми на її нову натуральну форму. Але
\parbreak{}  %% абзац продовжується на наступній сторінці
