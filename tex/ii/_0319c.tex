\parcont{}  %% абзац починається на попередній сторінці
\index{ii}{0319}  %% посилання на сторінку оригінального видання
повторюються так, що вони по черзі протистоять один одному раз як
купці, раз як продавці товарів. Капіталіст II оплачує робочу силу грішми;
таким чином він вводить робочу силу в склад свого капіталу і лише
в наслідок цього акту циркуляції, що є для нього тільки перетвір грошового
капіталу на продуктивний капітал, він протистоїть як промисловий
капіталіст робітникові як своєму найманому робітникові. Але
потім робітник, що на першій стадії виступав як продавець, торговець
своєю власною робочою силою, на другій стадії протистоїть як покупець,
як власник грошей, капіталістові як продавцеві товару; в наслідок цього
до капіталіста повертаються гроші, витрачені на заробітну плату. Оскільки
продаж цих товарів не сполучається з шахрайством і т. ін., отже,
оскільки в товарах і грошах тут обмінюється еквіваленти, він не є процес,
що за допомогою його збагачується капіталіст. Він не оплачує
робітника двічі: спочатку грішми, а потім товаром; його гроші повертаються
до нього, скоро робітник купить в нього на ці гроші товар.

Але грошовий капітал, перетворений на змінний капітал, — тобто гроші,
авансовані на заробітну плату, — відіграє головну ролю в самій грошовій
циркуляції, бо — через те, що робітнича кляса мусить перебиватися з
дня на день і тому не може кредитувати промислових капіталістів на
довгий час — в незчисленних, територіально різних пунктах суспільства
змінний капітал одночасно мусить авансуватися в грошах через певні
короткі строки, приміром, щотижня тощо, через порівняно швидко
повторювані переміжки часу (що коротші ці переміжки, то порівняно
менша може бути вся сума грошей, кожного разу подаваних у
циркуляцію через цей канал), — хоч які різні будуть періоди обороту
капіталів у різних галузях промисловости. Авансований таким чином грошовий
капітал становить у кожній країні капіталістичної продукціі відносно
вирішальну частину цілої циркуляції, то більше, що ці самі гроші,
перш ніж повернутись до вихідного пункту, циркулюють у найрізноманітніших
каналах і функціонують як засіб циркуляції у безмежному числі
інших операцій.
\pfbreak
Розгляньмо тепер циркуляцію між І ($v + m$) і II $c$ з іншого погляду.

Капіталісти І авансують на видачу заробітної плати 1000 ф. стерл.,
що на них робітники купують на 1000 ф. стерл. засоби існування в капіталістів
II, а ті теж купують на ті самі гроші засоби продукції в капіталістів
І. Тепер до останніх повернувся в грошовій формі їхній змінний
капітал, тимчасом як капіталісти II перетворили половину свого сталого
капіталу з форми товарового капіталу знову на продуктивний капітал.
Капіталісти II авансують дальші 500 ф. стерл. грішми, щоб дістати засоби
продукції в І; капіталісти І витрачають ці гроші на засоби споживання
II; таким чином ці 500 ф. стерл. припливають назад до капіталістів II;
вони знову авансують ці гроші, щоб перетворити знову на продуктивну
натуральну форму останню чверть свого сталого капіталу, перетвореного
на товар. Ці гроші знову повертаються до І і знову забирають у II
\parbreak{}  %% абзац продовжується на наступній сторінці
