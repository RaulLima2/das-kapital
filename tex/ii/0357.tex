талістів II, яка відновлює основний капітал in natura, за допомогою відповідної
зношуванню частини тієї товарової маси, що від неї тут фігурує
лише остача, можливо, вже таким чином реалізувала частину його зношеної
вартости; але їм лишається ще реалізувати таким чином
200 в грошах).

Далі, щодо другої половини (= 200) тих 400 ф. стерл., що їх II
подав у циркуляцію при цій прикінцевій операції, то на неї купується
у I обігові складові частини сталого капіталу. Частину цих 200 ф. стерл.
подали в циркуляцію, можливо, обидві частини капіталістів II або тільки
та частина, яка не відновлює in natura основної складової частини
вартости.

Отже, за допомогою 400 ф. стерл. з I підрозділу вилучено: 1) на
суму в 200 ф. стерл. таких товарів, що складаються лише з елементів
основного капіталу, 2) на суму в 200 ф. стерл. таких товарів, що заміщують
in natura лише елементи обігової частини сталого капіталу II.
I продав тепер увесь свій річний товаровий продукт, оскільки його доводиться
продати II підрозділові; але вартість однієї п’ятої цього продукту
400 ф. стерл. тепер існує в його руках у грошовій формі. Однак ці
гроші є перетворена на гроші додаткова вартість, яку доводиться витратити
як дохід на засоби споживання. Отже, I на ці 400 ф. стерл. купує
в II всю товарову вартість = 400. Таким чином, гроші допливають назад
до II, вилучаючи його товари.

Припустімо тепер три випадки. При цьому ту частину капіталістів II,
яка заміщує основний капітал in natura, ми називаємо „частина 1“, а
ту, що нагромаджує в грошовій формі вартість зношування основного
капіталу, називаємо „частина 2“. Три випадки такі: a) певна частина тих
400, що як остача існують ще в II підрозділі в товарах, має замістити певну
частину обігових частин сталого капіталу для „частини 1“ і „частини 2“
(наприклад, по 1/2); b) „частина 1“ уже продала ввесь свій товар, отже,
„частина 2“ ще повинна продати 400; c) „частина 2“ продала все, крім
тих 200, що є носії вартости зношування.

Тоді маємо такі розподіли:

a) 3 товарової вартости = 400 с, яка ще лишається в руках II, частині
1 належить 100 і частині 2—300; 200 з цих 300 репрезентують
зношування. В цьому разі з тих 400 ф. стерл. грішми, що їх I тепер
подає назад, щоб одержати товари II, частина 1 спочатку витратила 300,
— а саме 200 грішми, що ними вона вилучила з I елементи основного
капіталу in natura, і 100 грішми для упосереднення свого обміну товарами
з І; навпаки, частина 2 з цих 400 авансувала тільки 1/4, тобто
100 — так само для упосереднення свого товарового обміну з I.

Отже, з цих 400 грішми частина 1 авансувала 300 і частина
2—100.

Але з цих 400 повертаються назад:

До частини 1 : 100, отже, лише 1/3 авансованих нею грошей. Але
замість решти, 2/3, вона має відновлений основний капітал вартістю в 200.
За цей основний елемент капіталу вартістю в 200 вона дала І підроз-
