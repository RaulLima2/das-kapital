\parcont{}  %% абзац починається на попередній сторінці
\index{ii}{0330}  %% посилання на сторінку оригінального видання
вартість цих \sfrac{2}{3} вартости річного продукту = 2000 спродуковано не
в процесі зростання вартости II протягом поточного року.

Подібно до того, як розглядуваний з погляду процесу праці продукт
II є результат новодіющої живої праці та даних припущених при цьому
засобів продукції, що в них, як у своїх речових умовах, ця праця здійснюється,
— цілком так само з погляду процесу зростання вартости вартість
продукту II = 3000 складається з нової вартости, спродукованої новодолученою
\sfrac{1}{3} суспільного робочого дня ($500 v + 500m$ = 1000) і з сталої
вартости, що в ній зречевлено \sfrac{2}{3} минулого суспільного робочого дня,
що минув до початку розглядуваного тут продукційного процесу II. Ця
частина вартости продукту II виражається в частині самого продукту.
Вона існує в певній масі засобів споживання вартістю в 2000 = \sfrac{2}{3} суспільного
робочого дня. Це — та нова споживна форма, що в ній знову
з’являється ця частина вартости. Отже, обмін частини засобів споживання =
2000 ІІ с на засоби продукції І = І ($1000 v + 1000 m$) в дійсності є обмін
\sfrac{2}{3} цілого робочого дня, що не мають жодної частини праці цього року
й минули до початку цього року, на \sfrac{2}{3} робочого дня новододаного протягом
цього року. \sfrac{2}{3} суспільного робочого дня цього року не могли б
бути вжиті на продукцію сталого капіталу і разом з тим становити змінну
капітальну вартість плюс додаткова вартість для продуцентів цього капіталу,
коли б їх не обмінювалося з тією частиною вартости щорічно
споживаних засобів споживання, що в них міститься \sfrac{2}{3} робочого дня,
витраченого й реалізованого, до цього року, не на протязі цього
року. Це — обмін \sfrac{2}{3} робочого дня цього року на \sfrac{2}{3} робочого дня,
витрачені до цього року, обмін між робочим часом цього року й
торішнім робочим часом. Отже, це пояснює нам загадку, чому вартість,
новоутворена протягом цілого суспільного робочого дня, може розкластись
на змінну капітальну вартість плюс додаткова вартість, хоч
\sfrac{2}{3} цього робочого дня витрачається не на продукцію речей, що в
них міг би реалізуватись змінний капітал або додаткова вартість, а
на продукцію засобів продукції, які заміщують капітал, зужиткований
протягом року. Це пояснюється просто тим, що ті \sfrac{2}{3} вартости продукту
II, що в ньому капіталісти й робітники І реалізують спродуковану
ними змінну капітальну вартість плюс додаткова вартість (а вони
становлять разом \sfrac{2}{3} цілої вартости річного продукту), являють, розглядувані
щодо вартости, продукт \sfrac{2}{3} суспільного робочого дня, що
минув до цього року.

Хоч сума суспільного продукту І і II, засоби продукції та засоби
споживання, розглядувані конкретно з погляду їхньої споживної вартости,
їхньої натуральної форми є продукт праці поточного року, однак це має
силу лише остільки, оскільки цю працю розглядається як корисну конкретну
працю, а не як витрату робочої сили, як вартостетворчу працю.
А проте, навіть продуктом праці поточного року вони є в тому лише розумінні,
що засоби продукції перетворилися на новий продукт, на продукт
цього року тільки за допомогою долученої до них живої праці, яка орудувала
ними. Навпаки, праця поточного року не могла б перетворитись на продукт без
\parbreak{}  %% абзац продовжується на наступній сторінці
