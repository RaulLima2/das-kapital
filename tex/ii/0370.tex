ноти або монети, — я маю на увазі матеріяльну, а не фіктивну передачу...
Загальна сума взаємних обмінів між торговцями мусить, кінець-кінцем,
визначатись і обмежуватись сумою обмінів між торговцями й споживачами“.

Коли б у Тука остання теза була висловлена відокремлено, то можна
було б думати, що він просто констатує, що є співвідношення між обмінами
поміж самими торговцями і обмінами поміж торговцями й споживачами, —
інакше кажучи, співвідношення між вартістю сукупного річного доходу й
вартістю капіталу, що за допомогою його продукується дохід. Однак це
не так. Він прямо пристає на погляд А. Сміса. Тому зайве було б
критикувати зокрема його теорію циркуляції.

2) Кожен промисловий капітал при відкритті підприємства одним заходом
подає в циркуляцію гроші на всю свою основну складову частину, яку
він змову вилучає лише поступінно, протягом ряду років, продаючи свій
річний продукт. Отже, спочатку він подає в циркуляцію більше грошей,
ніж вилучає з неї. Це повторюється кожного разу при відновленні цілого
капіталу in natura; не повторюється щороку для певного числа підприємств,
що їхні основні капітали доводиться відновлювати in natura; частинно
це повторюється при кожному ремонті, при кожному лише частинному
відновленні основного капіталу. Отже, коли одна сторона
вилучає з циркуляції більше грошей, ніж подає в неї, то друга сторона —
навпаки.

В усіх галузях промисловости, де період продукції (як величина
відмінна від робочого періоду) охоплює порівняно довгий час, капіталістичні
продуценти протягом цього періоду ввесь час подають гроші в циркуляцію,
— почасти на оплату застосованої робочої сили, почасти на закуп
засобів продукції, що їх треба застосувати; таким чином, засоби продукції
безпосередньо вилучаються з ринку, а засоби споживання почасти посередньо
через робітників, які витрачають свою заробітну плату, почасти
безпосередньо самими капіталістами, які зовсім не відкладають свого
споживання, і при цьому ці капіталісти спочатку не подають на ринок
жодного еквіваленту товарами. Гроші, що їх вони подають в циркуляцію,
протягом цього періоду служать для перетворення на гроші товарової
вартости, а втім і вміщеної в ній додаткової вартости. Дуже важливий
стає цей момент при розвиненій капіталістичній продукції, в довгочасних
підприємствах, що їх засновують акційні товариства і т. ін., як
от будування залізниць, каналів, доків, великих міських споруд, залізних
пароплавів, дренування ґрунту в широких розмірах і т. ін.

3) Тимчасом як інші капіталісти, — лишаючи осторонь витрати на
основний капітал, — вилучають з циркуляції більше грошей, ніж подали в
неї, купуючи робочу силу та обігові елементи, капіталісти, що продукують
золото й срібло, — лишаючи осторонь благородний металь, що служить як
сировинний матеріял, — подають в циркуляцію тільки гроші, а вилучають
з неї тільки товари. Сталий капітал, за винятком зношеної частини,
більшу частину змінного капіталу й усю додаткову вартість, за винятком
скарбу, який, можливо, нагромаджується в їхніх власних руках, — усе це
як гроші подається в циркуляцію.
