\index{ii}{0371}  %% посилання на сторінку оригінального видання
4) З одного боку, циркулюють як товари різні речі, спродуковані
не цього року, земельні ділянки, будинки тощо; далі продукти, що
період їхньої продукції охоплює більш, ніж рік: худоба, ліс, вино й т. ін.
Для цих та інших явищ важливо пам’ятати, що, крім грошової суми,
потрібної для безпосередньої циркуляції, певна частина завжди перебуває
в латентному стані, не функціонуючи, але, коли дати їй поштовх, вона
може почати функціонувати. Вартість таких продуктів часто циркулює
також частинами й поступінно, як от вартість будинків при винайманні
протягом ряду років.

З другого боку, не всі рухи процесу репродукції упосереднюється
грошовою циркуляцією. Ввесь процес продукції, скоро його елементи
придбано, виключається з циркуляції. Далі, виключається ввесь продукт,
що його безпосередньо споживає сам продуцент, — хоч особисто, хоч
продуктивно; і сюди належить також харчування сільських робітників
натурою.

Отже, та маса грошей, що за її допомогою циркулює річний продукт,
вже є в суспільстві, вона нагромаджувалась поступінно. За винятком золота,
яке заміщує, напр., зношені монети, вона не належить до вартости,
спродукованої цього року.

У нашому викладі припускається циркуляцію виключно благородних
металевих грошей, і до того найпростіша її форма — купівля й продаж
за готівку, хоч на основі простої металевої циркуляції гроші можуть
функціонувати і як засіб виплати і справді історично так функціонували,
і хоч на цій основі розвинулись кредитова система й певні сторони
її механізму.

Це припущення робиться не лише з методичних міркувань, що їх важливість
виявляється вже в тому, що так Тук і його школа, як і їхні противники,
в своїх контроверсіях, досліджуючи циркуляцію банкнот, постійно
мусіли повертатись до гіпотези суто-металевої циркуляції. Вони мусіли
це робити post festum, але робили це дуже поверхово, до того ж під
тиском, бо вихідний пункт відігравав при цьому лише ролю другорядного
моменту в аналізі.

Але найпростіший розгляд грошової циркуляції, поданої в її первісній
формі, — а грошова циркуляція є тут іманентний момент річного процесу
репродукції, — показує:

а) Коли ми припустимо розвинену капіталістичну продукцію, а значить,
і панування системи найманої праці, то грошовий капітал, очевидно,
відіграє головну ролю, оскільки він є форма, що в ній авансується
змінний капітал. В міру того, як розвивається система найманої праці,
кожен продукт перетворюється на товар, а тому — за деякими важливими
винятками — він мусить також цілком проробити перетворення на
гроші як фазу свого руху. Маса грошей, що циркулюють, мусить бути
достатня для цього перетворення товарів на гроші, і найбільшу частину
цієї маси подається в формі заробітної плати, грошей, що їх як грошову
форму змінного капіталу авансував промисловий капітал на оплату робочої
сили, і що функціонують в руках робітників — більшою частиною —
\parbreak{}  %% абзац продовжується на наступній сторінці
