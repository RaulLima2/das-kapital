\parcont{}  %% абзац починається на попередній сторінці
\index{ii}{0201}  %% посилання на сторінку оригінального видання
періоду. Так в дійсності й є. Наприкінці 6-го тижня продукт вартістю
в 600 ф. стерл. входить у циркуляцію і наприкінці 9-го тижня повертається
назад в грошовій формі. Разом з тим на початку 7-го тижня
входить у роботу капітал II і покриває потреби наступного робочого
періоду протягом тижнів 7--9. Але, згідно з нашим припущенням, наприкінці
9 тижня робочий період пророблено лише на половину. Отже, на
початку 10-го тижня знову входить у роботу капітал І в 600 ф. стерл.,.
який шойно повернувся назад, і своїми 300 ф. стерл. він покриває авансування,
потрібні для тижнів 10--12. Цим завершується другий робочий
період. В циркуляції є продукт вартістю в 600 ф. стерл., і повертаються
вони назад наприкінці 15-го тижня; але, крім того, є 300 ф. стерл.
вільних — величина первісного капіталу II, і можуть вони функціонувати
в першу половину наступного робочого періоду, отже, протягом тижнів
13--15. Коли минуть вони, знову повертаються назад 600 ф. стерл.;
з них 300 ф. стерл. вистачить до кінця цього робочого періоду, а
300 ф. стерл. лишаються вільні для наступного.

Отже, справа перебігає так:

І. Період обороту: тижні\footnote{
. Робочий період: тижні 1--6. Функціонує капітал І, 600 ф. стерл.
}—9.

1. Період циркуляції: тижні 7--9. Наприкінці 9-го тижня повертаються
назад 600 ф. стерл.

II. Період обороту: тижні 7--15.\footnote{
. Період циркуляції: тижні 13--15. Наприкінці 15-го тижня повертаються
назад в грошовій формі 600 ф. стерл. (складені наполовину
з капіталу І, наполовину з капіталу II).

III. Період обороту: тижні 13--21.
}. Робочий період: тижні 7--12.

Перша половина: тижні 7--9. Функціонує капітал II, 300 ф. стерл.
Наприкінці 9 тижня повертаються назад 600 ф. стерл. в грошовій формі
(капітал І).

Доуга половина: тижні 10--12. Функціонують 300 ф. стерл. капіталу
І. Решта 300 ф. стерл. капіталу І лишаються вільні.\footnote{
. Період циркуляції: тижні 19--21, що наприкінці їх знову зворотно
припливають 600 ф. стерл. в грошовій формі; в цих 600 ф. стерл.
капітал І і капітал II тепер злито так, що їх годі відрізнити один од одного.

Таким чином, до кінця 51-го тижня відбувається вісім повних оборотів
капіталу в 600 ф. стерл. (І: тижні 1--9; II: 7--15; III: 13--21;
IV: 19--27; V: 25--33; VI: 31--39; VII: 37--45; VIII: тижні
43--51). А що тижні 49--51 припадають на восьмий період цирку-
}. Робочий період: тижні 13--18.

Перша половина: тижні 13--15. Вільні 300 ф. стерл. входять у
роботу. Наприкінці 15-го тижня повертаються назад 600 ф. стерл. в грошовій
формі.

Друга половина: тижні 16--18. З 600 ф. стерл., що повернулись,
функціонують 300 ф. стерл., а решта 300 ф. стерл. знову лишаються
вільні.
\index{ii}{0202}  %% посилання на сторінку оригінального видання
ляції, то протягом цього періоду мусять ввійти в роботу й підтримувати
продукцію в русі 300 ф. стерл. звільненого капіталу. Разом з тим наприкінці
року оборот має такий вигляд: 600 ф. стерл. вісім разів зробили
свій кругобіг, що дає 4800 ф. стерл. До цього долучається продукт
останніх 3 тижнів (49--51), який проробив лише третину свого дев’ятитижнгвого
кругобігу, отже, у суму обороту він увіходить лише третиною
своєї величини, 100 ф. стерл. Отже, коли річний продукт, рахуючи рік в
51 тиждень, дорівнює 5100 ф. стерл., то капітал, що обернувся, становитиме
тільки 4800 + 100--4900 ф. стерл.; отже, ввесь авансований капітал
в 900 ф. стерл. обернувся 5\sfrac{4}{9} раза, тобто на незначну величину більше,
ніж у випадку І.

В цьому прикладі припускався такий випадок, коли робочий час =\footnote{
. Робочий період: тижні 6--10.

Перший відділ: тижні 6--9. Функціонує капітал II = 400 ф. стерл.
Наприкінці 9 тижня зворотно припливає капітал 1 = 500 ф. стерл. в грошовій
формі.

Другий відділ: 10 тиждень. З 500 ф. стерл., що повернулися, функціонують
100 ф. стерл. Решта 400 ф. стерл. лишаються вільні для наступного
робочого періоду.
}/3,
а час обігу = \sfrac{1}{3} періоду обороту, отже, робочий час є просте кратне
часу обігу. Треба з’ясувати, чи констатоване вище звільнення капіталу
буде й в інших умовах.

Припустімо, що робочий період дорівнює 5 тижням, час обігу = 4 тижням,
щотижнево авансовуваний капітал = 100 ф. стерл.

І. Період обороту: тижні 1--9.

1. Робочий період: тижні 1--5. Функціонує капітал 1 = 500 ф. стерл.

1. Період циркуляції: тижні 6--9. Наприкінці 9 тижня припливають
назад в грошовій формі 500 ф. стерл.

ІІ. Період обороту: тижні 6--14.

2. Період циркуляції: тижні 11--14. Наприкінці 14 тижня 500 ф.
стерл. зворотно припливають у грошовій формі.

До кінця 14-го тижня (11--14) функціонують раніш звільнені 400 ф.
стерл.; із 500 ф. стерл., що потім повернулись, 100 ф. стерл. поповнюють
недостачу для потреб третього робочого періоду (тижні 11--15),
так що знову звільняються 400 ф. стерл. для четвертого робочого періоду.
Те саме явище повторюється в кожному робочому періоді; на
початку його є 400 ф. стерл., і їх досить на перші 4 тижні. Наприкінці
4-го тижня припливають назад 500 ф. стерл. в грошовій формі, що з
них тільки 100 ф. стерл. потрібні для останнього тижня, а решта 400 ф.
стерл. лишаються вільні для наступного робочого періоду.

Припустімо далі робочий період в 7 тижнів з капіталом І в 700 ф.
стерл.; час обігу в два тижні з капіталом II в 200 ф. стерл.

В такому разі перший період обороту триває протягом тижнів 1--9.
з них перший робочий період протягом тижнів 1--7, з авансуванням
в 700 ф. стерл., і перший період циркуляції протягом тижнів 8--9. Наприкінці
9-го тижня 700 ф. стерл. зворотно припливають у грошовій формі.
