велось би визнати, що ввесь капітал, застосований на розробку мідних
копалень, є лише основний капітал.

Візьмімо, навпаки, іншу промисловість, яка застосовує сировинний
матеріял, що становить субстанцію продукту, і допоміжні матеріяли, які
своєю речовиною, а не лише вартістю — як от вугілля, що йде на опалення
— увіходять у продукт. Разом з продуктом, напр., пряжею, сировинний
матеріял, що з нього складається продукт, напр., бавовна, змінює
власника й переходить з процесу продукції в процес споживання. Але
поки бавовна функціонує як елемент продуктивного капіталу, власник не
продає її, а обробляє, наказує робити з неї пряжу. Власник не випускає
її з рук, або, вживаючи грубофалшивого тривіяльного вислову Смісового,
власник не здобуває жодного зиску, „коли продукт відокремлюється від
нього, коли змінюється його хазяїн або коли він циркулює“ (by parting
with it, by its changing masters, or by circulating it). Він так само мало
пускає в циркуляцію свої матеріяли, як і свої машини. Вони фіксовані
в продукційному процесі цілком так само, як прядільні машини та фабричні
будівлі. Частина продуктивного капіталу мусить навіть бути завжди
фіксована в формі вугілля, вовни тощо, так само, як і в формі засобів
праці.

Ріжниця лише та, що бавовну, вугілля та ін., потрібні, напр., для
щотижневої продукції пряжі, завжди цілком зужитковується на продукцію
тижневого продукту, і, значить, їх треба замінювати на нові екземпляри
бавовни, вугілля, тощо; отже, ці елементи продуктивного капіталу,
хоч вони лишаються тотожні своїм родом, постійно складаються з нових
екземплярів того самого роду, тимчасом як та сама поодинока прядільна
машина, та сама поодинока фабрична будівля й далі беруть участь
у цілому ряді повторюваних тижневих процесів продукції, не заміщуючись
на інші екземпляри того самого роду. Як елементи продуктивного
капіталу, всі його складові частини завжди фіксовані в процесі
продукції, бо без них він не може відбуватися. І всі елементи продуктивного
капіталу, основні й поточні, як продуктивний капітал, однаково
протистоять капіталові циркуляції, тобто товаровому капіталові й грошовому
капіталові.

Так само стоїть справа й щодо робочої сили. Частина продуктивного
капіталу завжди мусить бути фіксована в ній, і той самий капіталіст
протягом більш або менш довгого часу застосовує ті самі тотожні
поміж себе робочі сили, на зразок того, як застосовує ті самі машини.
Ріжниця між ними й машинами тут не в тому, що машину купується раз
назавжди (хоч цього не буває, коли, напр., за неї сплачують рати), а
робітника не на завжди, а в тому, що праця, яку витрачає робітник,
цілком входить у вартість продукту, тимчасом як вартість машини — лише
частинами.

Сміс плутає різні визначення, коли він про обіговий капітал, протилежно
до основного, каже таке: „Капітал, застосований таким способом,
не дає своєму власникові доходу або зиску, поки лишається в його посіданні
або зберігає ту саму форму“. Він ставить на один рівень ту ли-
