обігу, постійно вирівнюється з первісною вартістю. Коли усамостійнення
вартости проти вартостетворчої сили, робочої сили, починається в акті
Г — Р (купівля робочої сили) і здійснюється в процесі продукції як
експлуатація робочої сили, то це усамостійнення вартости не виявляється
знову в тому кругобігу, де гроші, товар, елементи продукції являють
лише почережні форми капітальної вартости, що процесує, і де попередня
величина вартости вирівнюється з теперішньою зміненою величиною
вартости капіталу.

„Вартість, — каже Бейлі, заперечуючи усамостійнення вартости, яке
характеризує капіталістичний спосіб продукції, і яке він трактує як
ілюзію деяких економістів, — вартість є відношення між одночасно наявними
товарами, бо лише такі товари можна обмінювати один на один“ *).
Він це каже, заперечуючи можливість порівняння товарових вартостей в
різні доби, порівняння, яке — в разі грошову вартість фіксовано для
кожної доби — означає лише порівняння витрати праці, потрібної в різні
доби для продукції товарів однакового сорту. Це випливає з його загального
хибного уявлення, що згідно з ним мінова вартість дорівнює
вартості, а форма вартости є сама вартість; отже, товарові вартості не
можуть порівнюватись, коли вони не функціонують активно як мінові вартости,
тобто коли їх не можна realiter **) обміняти одна на одну. Таким
чином, йому й на думку не спадає, що вартість функціонує як капітальна
вартість або капітал лише остільки, оскільки вона в різних фазах свого
кругобігу — а вони зовсім не є contemporary ***), а постають одна по
одній — лишається ідентична самій собі і сама з собою порівнюється.

Щоб дослідити формулу кругобігу в чистому вигляді, не досить лише
того припущення, що товари продається за їхньою вартістю, але треба
ще припустити й те, що це відбувається за інших незмінних обставин.
Візьмімо, напр., форму П... П не вважаючи на всякі революції у техніці
продукційного процесу, що можуть зневартнити продуктивний капітал певного
капіталіста, а також не вважаючи на всякий зворотний вплив, що
його може справити зміна вартости елементів продуктивного капіталу на
вартість наявного товарового капіталу, яка може підвищитись або
знизитись, коли є запас такого капіталу. Хай Т', 10.000 ф. пряжі,
продається за їхньою вартістю за 500 ф. стерл.; 8.440 ф. пряжі =
422 ф. стерл. покривають капітальну вартість, що є в Т'. Але коли
вартість бавовни, вугілля і т. ін. підвищилась (ми тут не беремо на увагу
звичайного коливання цін), то можливо цих 422 ф. стерл. буде не досить,
щоб повнотою покрити елементи продуктивного капіталу; потрібен додатковий
грошовий капітал — грошовий капітал зв’язується. Навпаки, коли
ті ціни падають, грошовий капітал звільняється. Процес перебігає цілком
нормально тільки тоді, коли відношення вартости лишаються сталі; в дійсності
він перебігає нормально доти, доки перешкоди в повторенні круго-

*) «Value is a relation between contemporary commodities, because such only
admit of being exchanged with each other».

**) Realiter — в дійсності, реально. Ред.

***) Cotemporary — одночасно; є contemporary — є одночасні. Ред.
