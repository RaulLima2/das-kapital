\parcont{}  %% абзац починається на попередній сторінці
\index{ii}{0036}  %% посилання на сторінку оригінального видання
це є лише ідеальний дріб цілого будинку, і все ж досить реальний, щоб
бути забезпеченням для додаткової позики. (Див. про це далі, розділ XII).

По-третє. Коли рух капітальної вартости й додаткової вартости, що
був спільний в Т і Г, відокремлюється лише почасти (так що частину
додаткової вартости не витрачається як дохід) або зовсім не відокремлюється,
то сама капітальна вартість зазнає зміни ще підчас її кругобігу,
до його закінчення. В нашому прикладі вартість продуктивного капіталу
дорівнювала 422 ф. стерл. Отже, коли він пророблює й далі свій шлях
$Г — Т$, наприклад, як 480 ф. стерл. або 500 ф. стерл., то він перебігає
останні стадії кругобігу як вартість на 58 ф. стерл. або на 78 ф. стерл.
більша, ніж початкова вартість. Разом з тим це може сполучатися з зміною
його складу щодо вартости.

$Т' — Г'$, друга стадія циркуляції й кінцева стадія кругобігу І (Г... $Г'$),
в нашому кругобігу являє другу стадію його й першу стадію товарової
циркуляції. Отже, оскільки справа стосується циркуляції, це $Т' — Г'$ мусить
доповнюватись через $Г' — Т'$. Але $Т' — Г'$ не лише вже має позаду себе процес
зростання вартости (тут функцію П, першу стадію), але й результат його,
товаровий продукт $Т'$, уже зреалізовано. Отже, через $Т' — Г'$ вивершується
процес зростання вартости капіталу, а також реалізація того товарового продукту,
що в ньому виражається виросла капітальна вартість.

Отже, ми припустили просту репродукцію, тобто припустили, що
$г-т$ цілком відокремлюється від $Г — Т$. А що обидві циркуляції,
$т — г — т т$ак само, як і $Т — Г — Т$, своєю загальною формою належать
до товарової циркуляції (а тому й не виявляють жадної відміни щодо
вартости між крайніми членами), то відси легко, як це робить вульґарна
політична економія, вбачати в капіталістичному процесі просту
продукцію товарів, споживних вартостей, призначених для того або іншого
споживання, — вартостей, що їх капіталіст продукує для того, щоб замінити
їх на товари іншої споживної вартости, або обміняти на ці товари, як
про це фалшиво говорить вульґарна політична економія.

$Т'$ з самого початку виступає як товаровий капітал, і мета цілого
процесу, збагачення (зростання вартости), не лише не виключає дедалі
більшого, відповідно до збільшення додаткової вартости (а, значить, і
капіталу), споживання капіталіста, але саме включає його.

У циркуляції доходу капіталіста спродукований товар т (або ідеально
відповідна до нього частина товарового продукту $Т'$) у дійсності придається
лише на те, щоб перетворити цей дохід спочатку на гроші, а з цих
грошей на ряд інших товарів, які придаються для особистого споживання.
Але при цьому не треба забувати тієї маленької обставини, що т є товарова
вартість, яка нічого не коштувала капіталістові, втілення додаткової
праці, а тому воно виходить на кін спочатку як складова частина товарового
капіталу $Т'$. Отже, це т самим своїм існуванням зв’язане з кругобігом
капітальної вартости, що перебуває в процесі руху; і коли рух
її припиниться або якось інакше порушиться, то скорочується або зовсім
припиняється не лише споживання т, але разом з тим і збут товарового
\parbreak{}  %% абзац продовжується на наступній сторінці
