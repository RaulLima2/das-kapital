на грошову форму. Але як особлива стадія в кругобігу певного індивідуального
капіталу той самий акт є реалізація капітальної вартости
в 422 ф. стерл., що міститься в товарі, плюс додаткова вартість в 78 ф.
стерл., яка міститься в тому самому товарі, отже, являє Т' — Г', перетворення
товарового капіталу з його товарової форми на грошову форму 4).

Функція Т' тепер така сама, як і всякого іншого товарового продукту:
перетворитись на гроші, бути проданим, проробити фазу циркуляції
Т — Г. Поки вирослий вартістю капітал застигає у формі товарового
капіталу, нерухомо лежить на ринку, продукційний процес припиняється.
Капітал не діє ні як продуктотворець, ні як вартостетворець. Залежно від
різного ступеня швидкости, що з нею капітал скидає з себе свою товарову
форму й набирає грошової форми, або, залежно від швидкости продажу,
та сама капітальна вартість дуже неоднаковою мірою буде правити за
продуктотворця й вартостетворця, і маштаб репродукції буде розширюватись
або скорочуватись. У першій книзі показано, що міра дії даного
капіталу зумовлена потенціями продукційного процесу, які до деякої
міри не залежать від величини його власної вартости. Тепер виявляється,
що процес циркуляції пускає в рух нові потенції, незалежні від величини
вартости капіталу, — потенції, що зумовлюють міру його дії, його поширення
та скорочення.

Далі товарова маса Т', як носій капіталу, що сам із себе зріс у своїй
вартості, мусить уся проробити метаморфозу Т' — Г'. Кількість проданого
має тут посутнє значіння. Поодинокий товар фігурує лише як інтеґральна
частина цілої маси. 500 ф. стерл. вартости існують в 10.000 ф. пряжі.
Коли капіталістові вдасться продати лише 7.440 ф. пряжі за її вартістю
в 372 фунт, стерл., то він поверне тільки частину свого сталого капіталу,
вартість витрачених засобів продукції; а коли 8440 ф., — то лише величину
вартости цілого авансованого капіталу. Він мусить більше продати, щоб
реалізувати додаткову вартість, і він мусить продати всі 10.000 ф.
пряжі, щоб реалізувати додаткову вартість в 78 ф. стерл. (= 1.560 ф.
пряжі). Отже, в 500 ф. стерл. грошей він одержує лише рівновартість
за проданий товар; його операція в межах циркуляції є просте Т — Г.
Коли б він заплатив своїм робітникам 64 ф. стерл. замість 50 ф. стерл.
заробітної плати, то його додаткова вартість була б лише 64 ф. стерл.
замість 78 ф. стерл., а ступінь експлуатації лише 100\% замість
156\%; але, як і раніш, вартість його пряжі залишилась би незмінною;
тільки співвідношення різних частин її змінилось би; акт циркуляції Т — Г
тепер, як і раніше, був би продажем 10.000 ф. пряжі за 500 ф. стерл.,
за її вартість.

Т' = Т + т (= 422 ф. стерл. + 78 ф. стерл.). — Т дорівнює вартості П,
або вартості продуктивного капіталу, а вона дорівнює вартості Г, авансованого
в Г — Т, в купівлі елементів продукції; в нашому прикладі вона
дорівнює 422 ф. стерл. Коли товарова маса продається за її вартістю, то
Т = 422 ф. стерл. і т = 78 ф. стерл., тобто вартості додаткового продукту

4) До цього місця рукопис VI. Відси рукопис V.
