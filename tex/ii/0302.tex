те, що не відбувається жодних переворотів у вартості складових частин
продуктивного капіталу. А втім, щодо відхилення цін від вартостей, то
ця обставина не може справити будь-якого впливу на рух суспільного
капіталу. При цьому в цілому обмінювалось би ті самі маси продуктів,
що й раніше, хоч поодиноким капіталістам при цьому дістались би пайки
вартости, вже непропорціональні їхнім відповідним авансуванням і тим масам
додаткової вартости, що кожен із них випродукував. Щодо переворотів
у вартості, то коли вони мають загальний характер і розподіляються
рівномірно, вони не спричиняють жодних змін у відношеннях між
складовими частинами вартости сукупного річного продукту. Навпаки, коли
вони мають частинний характер і розподіляються нерівномірно, то являють
собою розлади, що їх, поперше, можна зрозуміти як такі, лише розглядаючи
їх, як відхили від незмінних відношень вартости; але, подруге,
коли визначено закон, що згідно з ним одна частина річного продукту заміщує
сталий, а друга — змінний капітал, то в цьому законі нічого не
змінила б революція в вартости хоч сталого, хоч змінного капіталу. Вона
змінила б лише відносну величину тих частин вартости, що функціонують
у тій або іншій якості, бо на місце первісних вартостей виступили
б вартості іншої величини.

Поки ми розглядали продукцію вартости та вартість продукту капіталу,
як індивідуального капіталу, для нашої аналізи натуральна форма
товарового продукту була цілком байдужа, — було цілком байдуже, напр.,
чи складається він з машин, чи з хліба, чи з дзеркал. Всі ці натуральні
форми були б просто прикладом для нас, і перша-ліпша галузь продукції
однаково була б придатна, як ілюстрація. Нам доводилось мати
справу безпосередньо з самим процесом продукції, що в кожному пункті
виступав як процес індивідуального капіталу. Оскільки ми розглядали
репродукцію капіталу, нам досить було того припущення, що частина
товарового продукту, яка являє капітальну вартість, має в сфері циркуляції
змогу зворотно перетворитись на елементи її продукції, і значить,
на форму продуктивного капіталу; цілком так само, як досить було нам
того припущення, що робітник і капіталіст знаходять на ринку товари,
на які вони витрачають заробітну плату й додаткову вартість. Але цей
суто-формальний спосіб викладу вже недостатній, коли ми розглядаємо
сукупний суспільний капітал і вартість його продукту. Зворотне перетворення
однієї частини вартости продукту на капітал, перехід другої частини в
сферу особистого споживання кляси капіталістів і кляси робітників, становить
рух у межах самої вартости продукту, яка є результат сукупного
капіталу; і цей рух є не лише заміщення вартости, а й заміщення речовини,
а тому він так само зумовлюється співвідношенням складових
частин вартости суспільного продукту, як і споживною їхньою вартістю,
їхньою речовою формою.

Проста репродукція43) в незмінному маштабі являє абстракцію в тому
розумінні, що, з одного боку, на базі капіталістичної продукції відсут-

43) З рукопису VIII.
