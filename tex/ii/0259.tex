Всі заперечення — це є сліпий вистріл капіталістів та їхніх економістів
сикофантів.

Факти, що дають нагоду для такого сліпого вистрілу, є троякого роду.

1) Загальний закон грошової циркуляції той, що коли сума цін товарів,
що циркулюють, підвищується — все одно чи це збільшення суми
цін постає для тієї самої маси товарів, чи для збільшеної, — то за інших
незмінних обставин зростає маса грошей, що циркулюють.

Тут наслідок сплутують з причиною. Заробітна плата підвищується (хоч
і рідко підвищується, а пропорційно до підвищення цін вона підвищується
тільки в виняткових випадках) із підвищенням цін доконечних засобів існування.
Її підвищення є наслідок, а не причина підвищення цін товарів.

2) При частковому або місцевому підвищенні заробітної плати, тобто
при підвищенні її тільки в поодиноких галузях продукції — може в наслідок
цього постати місцеве підвищення цін на продукти цієї галузі. Але навіть
це залежить від багатьох обставин. Напр., від того, що заробітна плата
тут не була надто низька і норма зиску тому не була надто висока, що
в наслідок підвищення цін ринок для цих товарів не скорочується
(отже, для підвищення їхніх цін не треба попереднього зменшення
подання їх) і т. ін.

3) При загальному підвищенні заробітної плати підвищується ціна товарів,
продукованих в тих галузях промисловости, де переважає змінний капітал,
але зате спадає в тих, де переважає сталий, зглядно основний капітал.

При дослідженні простої товарової циркуляції (книга І, розд. III, 2) виявилось,
що хоч у процесі циркуляції будь-якої певної кількости товарів її
грошова форма є лише минуща, однак, гроші, зникаючи при метаморфозі
товару в руках однієї особи, неодмінно переходять до рук іншої; отже,
товари насамперед не лише всебічно обмінюються або заміщуються один
одним, але це заміщення упосереднюється й супроводиться всебічним
осіданням грошей. „У наслідок заміщення одного товару іншим товаром
до рук третьої особи одночасно в’язне товар-гроші. Циркуляція постійно
спливає грошовим потом“ (кн. І, розд. III, 2, а). Той самий тотожній
факт на основі капіталістичної товарової продукції виражається в тому,
що частина капіталу постійно існує в формі грошового капіталу, а частина
додаткової вартости так само постійно перебуває в руках її власника
в грошовій формі.

Лишаючи це осторонь, кругобіг грошей — тобто зворотний приплив
грошей до свого вихідного пункту — оскільки він становить момент
обороту капіталу, є цілком відмінне явище, навіть протилежне обігові
грошей33), який виражає постійне віддалення їх від вихідного

33) Хоч фізіократи ще сплутують обидва ці явища, однак вони перші звернули
увагу на зворотний приплив грошей до свого вихідного пункту, як на важливу
форму циркуляції капіталу, як на форму циркуляції, що упосереднює репродукцію.
„Погляньте на „Tableau Économique“, і ви побачите, що продуктивна кляса дає
гроші на які інші кляси купують у неї продукти, і що вони повертають їй ці
