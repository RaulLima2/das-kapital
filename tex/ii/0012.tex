дачі. Але гроші можуть зворотно припливати до нього в наслідок продажу товарів. Отже, цей акт з
самого початку припускає, що даний індивідуум — товаропродуцент.

Г — Р. Найманий робітник живе тільки з продажу робочої сили. Її збереження — його самозбереження —
потребує щоденного споживання. Отже, виплати йому мусять завжди повторюватись через короткі
промежки, щоб він міг повторювати закупки, потрібні для його самозбереження — акт Р — Г — Т або Т —
Г — Т. Тому капіталіст завжди мусить протистояти йому як грошовий капіталіст, а його капітал — як
грошовий капітал. Але, з другого боку, щоб маса безпосередніх продуцентів, найманих
робітників, могла чинити акт Р — Г — Т, доконечні засоби існування повинні протистояти їм завжди в
такій формі, щоб їх можна було купити, тобто в товаровій формі. Отже, цей стан потребує вже
високорозвиненої циркуляції продуктів як товарів, отже, і високорозвиненої товарової продукції.
Скоро продукція за допомогою найманої праці є загальна, товарова продукція мусить бути загальною
формою продукції.
З свого боку товарова продукція, — коли припускається, що вона має загальний характер, — призводить
до дедалі більшого розподілу суспільної праці, тобто до дедалі більшого відокремлення продукту, що
його продукує певний капіталіст як товар, — призводить до того, що продукційні процеси, які один
одного доповнюють, дедалі більше розщеплюються на самостійні процеси. Тому такою самою мірою, як
розвивається Г — Р, розвивається Г — Зп, тобто в такому ж самому обсязі продукування засобів
продукції відокремлюється від продукування товару, що для нього вони правлять за засоби продукції, і
сами вони протистоять кожному товаропродуцентові як товари, що їх він не виробляє, а купує для свого
певного продукційного процесу. Вони
виходять з галузей продукції, самостійно проваджуваних, цілком відокремлених
від його власної, і входять в його галузь продукції як товари; отже, їх мусять купувати. Речові
умови товарової продукції протистоять товаропродуцентові в дедалі більшому обсязі як продукти інших
товаропродуцентів, як товари. І в такому самому обсязі капіталіст мусить виступати як грошовий
капіталіст, тобто поширюється той маштаб, що в ньому його капітал мусить функціонувати як грошовий
капітал.

З другого боку, ті самі обставини, що утворюють основну умову капіталістичної продукції — наявність
кляси найманих робітників — спонукають до переходу цілої товарової продукції на капіталістичну
товарову продукцію. Що більше остання розвивається, то більше впливає вона руйнаційно й розкладово
на всяку старішу форму продукції, яка, бувши розрахована переважно на безпосереднє задоволення
власних потреб, перетворює на товар тільки надлишок продукту. Вона робить продаж продукту
переважним інтересом, при чому на початку вона виразно й не зачіпає самого способу продукції, —
такий був, наприклад, спочатку вплив капіталістичної світової торговлі на такі народи, як китайці,
індійці, араби тощо. Але далі, там, де вона вкорінюється, вона руйнує всі форми товарової продукції,
що ґрунтуються або на власній праці продуцентів,
