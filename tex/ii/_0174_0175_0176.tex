\index{ii}{0174}  %% посилання на сторінку оригінального видання
Розділ тринадцятий

Час продукції

Робочий час завжди є час продукції, тобто час, що протягом його
капітал зв’язано в сфері продукції. Але, навпаки, не увесь час, що протягом
його капітал перебуває в процесі продукції, є в наслідок цього
доконечно також робочий час.

Тут ідеться не про ті перерви у процесі праці, що зумовлені природними
межами самої робочої сили, хоч вже й виявилось, якою поважною
спонукою незвичайного подовження процесу праці та заведення
денної й нічної роботи є та лише обставина, що основний капітал
фабричні будівлі, машини тощо — стоять без ужитку під час перерв у процесі\footnote*{
Див. „Капітал“, кн. І, розд. VIII, 4 та розд. XIII, 3 b. \emph{Ред.}
}.
Тут ідеться про перерву, незалежну від протягу процесу праці,
зумовлену самою природою продукту та способом його виготовлення,
про перерву, що протягом її предмет праці підпадає більш-менш протяжним
природним процесам, мусить зазнати фізичних, хемічних і фізіологічних
змін, перерву, що протягом її процес праці цілком або почасти
припиняється.

Напр., щойно видавлене вино мусить деякий час шумувати, а потім
протягом деякого часу стояти, щоб набути певного ступеня досконалости.
В багатьох галузях промисловости продукт мусить сушитись, напр., у
ганчарстві, або підпадати певним впливам, що змінюють його хемічні
властивості, як от у білильнях. Озимим хлібам треба аж дев’ять місяців
вистигати. Між посівом і жнивами процес праці майже цілком припиняється.
В лісівництві після посіву та потрібних для нього підготовчих
робіт треба, може, сто років, щоб насіння перетворилося на готовий продукт;
а протягом усього цього часу потрібно прикладати відносно лише
дуже мало праці.

В усіх таких випадках протягом більшої частини часу продукції новододаваної
праці потрібно прикладати лише зрідка. Описані в попередньому
розділі умови, що за них до капіталу, вже зв’язаного в процесі
продукції, треба долучити новий додатковий капітал і новододавану працю,
здійснюються тут лише з більшими або меншими перервами.

Отже, в усіх цих випадках час продукції авансованого капіталу складається
з двох періодів: перший період, коли капітал перебуває в процесі
праці; другий період, коли форма існування капіталу — форма ще неготового
продукту — підпадає впливові природних процесів, не перебуваючи
в процесі праці. Справа ані трохи не змінюється від того, що обидва ці
періоди можуть почасти перехрещуватись та вклинюватись один в один.
Робочий період і період продукції тут не збігаються. Період продукції
є довший, ніж робочий період. Але тільки по закінченні періоду продукції
продукт є готовий, достиглий, отже, тільки тоді його можна перетворити
з форми продуктивного капіталу на форму товарового капіталу.
\index{ii}{0175}  %% посилання на сторінку оригінального видання
Отже, залежно від протягу тієї частини часу продукції, яка не є
робочий час, подовжується й період обороту капіталу. Оскільки час продукції,
надмірний порівняно з робочим часом, не визначено раз назавжди
даними законами природи, як от при достиганні хліба, рості дуба тощо,
період обороту часто можна більш-менш скоротити, штучно скорочуючи
час продукції. Напр., коли заводиться хемічне біління замість біління
на полі, — чинніші сушні апарати у процесах сушіння. Так в чинбарстві,
де за старими методами треба було від 6 до 18 місяців, щоб
чинбарська кислота пройняла шкіри, ці операції, за нової методи, коли
почали застосовувати повітряну помпу, скоротились до 1\sfrac{1}{2}—2 місяців.
(I. G. Courcelle-Saneuil. Traité théorique et pratique des Entreprises industrielles
etc. Paris, 1857, 2 éd.).

Найяскравіший приклад штучного скорочення часу продукції, заповненого
виключно природними процесами, подає історія залізоробної
продукції і особливо перероблення чавуна на сталь за останні 100 років,
починаючи з відкритого 1780 року пудлінґування й до сучасного бессемерівського
процесу та інших, заведених з того часу найновіших методів.
Час продукції скорочено надзвичайно, але такою самою мірою збільшились
і вкладення основного капіталу.

Своєрідний приклад того, як відхиляється час продукції від робочого
часу, подає американська фабрикація копил на чоботи. Тут чимала частина
затрат постає тому, що дерево має сохнути до 18 місяців, щоб
готове копило не дубилось і не змінювало своєї форми. Протягом цього
часу дерево не підпадає жодному іншому процесові праці. Період обороту
вкладеного тут капіталу визначається, отже, не лише часом, потрібним
на виготовлення самих копил, а й часом, що протягом його капітал
лежить без діла в дереві, що висихає. Дерево перебуває 18 місяців
у процесі продукції, поки воно, нарешті, ввійде у власне робочий процес.
Разом з тим цей приклад показує, які різні можуть бути періоди обороту
різних частин цілого обігового капіталу в наслідок обставин, що постають
не в сфері циркуляції, а в продукційному процесі.

Особливо виразно виступає ріжниця між часом продукції і робочим
часом у сільському господарстві. В нашому помірному підсонні земля дає
збіжжя раз на рік. Скорочення або продовження періоду продукції (пересічно
дев’ятимісячного для озимого засіву) і собі залежить від зміни
сприятливих і несприятливих років, а тому не можна його точно наперед
визначити й контролювати, як у власне промисловості. Лише бічні продукти,
напр., молоко, сир і т. ін. завжди можна продукувати й продавати
протягом більш-менш коротких періодів. Щождо робочого часу, то тут
справа така: „В різних місцевостях Німеччини, залежно від кліматичних
та інших чинних умов, число робочих днів для трьох головних періодів
праці буде таке: у весняному періоді, з середини березня або початку
квітня до середини травня 50--60 робочих днів; в літньому періоді, з початку
червня до кінця серпня 65--80; в осінньому періоді, з початку
вересня до кінця жовтня або середини або кінця листопада 55--75 робочих
днів. На зиму припадають лише такі роботи, що їх можна виконати
\index{ii}{0176}  %% посилання на сторінку оригінального видання
в цей час, прим., вивіз добрива, дрів, продуктів на ринок, будівельних
матеріялів тощо“ (F. Kirchhof, Handbuch der landwirtschaftlichen
Betriebslehre. Dresden, 1852, S. 160).

Отже, що несприятливіше підсоння, то коротший робочий період у
сільському господарстві, а тому коротший і той час, коли вкладається
капітал і працю. Візьмімо, напр., Росію. Там у деяких північних місцевостях
польові роботи можливі тільки протягом 130--150 днів на рік.
Легко зрозуміти, якою втратою було б для Росії, коли б 50 мільйонів
із 65 мільйонів її європейської людности лишалось без роботи протягом
шости або восьми зимових місяців, коли мусять припинитись всі польові
роботи. Крім 200.000 селян, що роблять на 10.500 фабриках Росії, по
селах там скрізь розвинулась своя хатня промисловість. Там є села, де
всі селяни з роду в рід ткачі, чинбарі, шевці, слюсарі, ножівники і т.
ін.; особливо це стосується до губерень Московської, Владимирської, Калузької,
Костромської та Петербурзької. До речі буде зауважити, що цю
хатню промисловість дедалі більше примушується служити капіталістичній
продукції, напр., ткачам основу та піткання постачають торговці або
безпосередньо або через факторів. (Скорочено за Reports by Н. М. Secretaries
of Embassy and Legation, on the Manufactures, Commerce etc.,
№ 8, p. 86, 87). Відси видно, як розходження періоду продукції і робочого
періоду — а останній становить лише частину періоду продукції, —
утворює природну основу для об’єднання хліборобства з сільськими підсобними
промислами і як, з другого боку, ці останні дають точку опори
капіталістам, які спочатку протискуються сюди, як торговці. Рівнобіжно
з тим, як пізніше капіталістична продукція відокремлює мануфактуру від
хліборобства, сільський робітник підпадає під дедалі більшу залежність
від суто-випадкового, підсобного промисла, і його стан через це погіршується.
Для капіталу, як ми побачимо далі, всі ріжниці в обороті вирівнюються.
Для робітника вони не вирівнюються.

Тимчасом як у більшості галузей власне промисловости, як от у гірництві,
транспорті і т. ін., процес продукції відбувається рівномірно, рік-у-рік
рівномірно витрачається робочий час, і, лишаючи осторонь, як ненормальні
перерви, коливання цін, розлади в перебігу справ і т. ін., рівномірно
розподіляються витрати на капітал, що входять у повсякденний процес
циркуляції, тимчасом як за інших незмінних ринкових відносин
зворотний приплив обігового капіталу або відновлення його так само
розподіляється на рівномірні переміжки часу — в тих галузях приміщення
капіталу, де робочий час становить лише частину часу продукції, обіговий
капітал витрачається в різні періоди року дуже нерівномірно, а назад
він припливає разом, в момент, визначуваний природними умовами. Отже,
тут, при однаковому маштабі підприємства, тобто при однаковій величині
авансованого обігового капіталу, цей останній треба авансувати одним
заходом більшими масами й на довший час, ніж у підприємствах з безперервними
робочими періодами. Життьова тривалість основного капіталу
тут також більше відрізняється від того часу, що протягом його він
справді функціонує продуктивно. Звичайно, коли є ріжниця між робочим
\parbreak{}  %% абзац продовжується на наступній сторінці
