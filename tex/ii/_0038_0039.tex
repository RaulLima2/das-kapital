\parcont{}  %% абзац починається на попередній сторінці
\index{ii}{0038}  %% посилання на сторінку оригінального видання
капіталу $Г — Т$. Не зважаючи на різні місця, функція грошового капіталу,
що на нього тепер перетворився товаровий капітал, є та сама: його
перетворення на $Зп$ та $Р$, на засоби продукції та робочу силу.

Отже, одночасно з $т — г$ капітальна вартість у функції товарового
капіталу $Т' — Г'$ проробила фазу $Т — Г$ і вступає тепер у додаткову
фазу $Г — Т\splitfrac{Р}{Зп}$; отже, її ціла
циркуляція є $Т — Г — Т\splitfrac{Р}{Зп}$.

Поперше, грошовий капітал Г виступив у формі І (кругобіг $Г\dots{} Г'$)
як первісна форма, що в ній авансується капітальна вартість; а тут він
виступає з самого початку як частина грошової суми, що на неї перетворився
товаровий капітал у першій фазі циркуляції $Т' — Г'$, отже, з самого
початку виступає як перетворення $П$, продуктивного капіталу, на грошову
форму, — перетворення, що упосереднюється продажем товарового продукту.
Грошовий капітал існує тут з самого початку не як первісна та
не як кінцева форма капітальної вартости, бо фаза $Г — Т$, що вивершує
фазу $Т — Г$, може бути пророблена, лише через подруге скидання грошової
форми. Тому частина $Г — Т$, яка одночасно є $Г — Р$, виступає тут
уже не як просте авансування грошей у формі закупу робочої сили, але
як таке авансування, що в ньому робочій силі авансується в формі
грошей ті самі 1000 ф. пряжі вартістю в 50 ф. стерл., які становлять
частину товарової вартости, утвореної робочою силою. Гроші, авансовані
тут робітникові, є лише перетворена еквівалентна форма частини
вартости товару, що його спродукував сам робітник. І вже тому акт
$Г — Т$, оскільки він є акт $Г — Р$, зовсім не є лише заміщення товару в
грошовій формі товаром у споживній формі, але має в собі інші
елементи, незалежні від загальної товарової циркуляції як такої.

$Г'$ є перетворена форма $Т'$, яке саме є продукт минулого функціонування
$П$, продукційного процесу; тому вся грошова сума $Г'$ є грошовий
вираз минулої праці. У нашому прикладі: 10.000 ф. пряжі = 500 ф. стерл.,
продуктові процесу прядіння; з них 7440 ф. пряжі = авансованому
сталому капіталові $c = 372$ ф. стерл.; 1000 ф. пряжі = авансованому
змінному капіталові $v = 50$ ф. стерл., і 1560 ф. пряжі = додатковій
вартості $m = 78$ ф. стерл. Коли з $Г'$ знову авансується, за інших незмінних
умов, лише первісний капітал = 422 ф. стерл., то робітник одержує
в найближчий тиждень в акті $Г — Р$ як аванс лише частину 10.000 ф.
пряжі (грошову вартість 1000 ф. пряжі), спродуковану протягом цього
тижня. Як результат акту $Т — Г$ гроші завжди є вираз минулої праці.
Оскільки на товаровому ринку відразу відбувається додатковий акт $Г — Т$,
тобто $Г$ обмінюється на наявні товари, що є на ринку, то знову таки
тут маємо перетворення минулої праці з однієї форми (грошової) на
другу (товарову). Але $Г — Т$ і $Т — Г$ відрізняються щодо часу. Вони
можуть винятково відбуватись одночасно, коли, наприклад, капіталіст,
що переводить $Г — Т$, і капіталіст, що для нього цей акт є акт
$Т — Г$, одночасно передають один одному свої товари, і $Г$ потім лише
вирівнює ріжницю. Ріжниця щодо часу здійснення $Т — Г$ і часу здійснення
\index{ii}{0039}  %% посилання на сторінку оригінального видання
$Г — Т$ може бути більш або менш значна. Хоч $Г$, як результат акту
$Т — Г$, репрезентує минулу працю, все ж $Г$ для акту $Г — Т$ може
репрезентувати перетворену форму товарів, що їх на ринку ще немає,
що лише будуть там у майбутньому, бо акт $Г — Т$ має відбуватися
лише після того, як знову вироблено $Т$. Так само $Г$ може
репрезентувати товари, що їх продукується одночасно з тим $Т$, що його
грошовий вираз воно є. Наприклад, в обміні $Г — Т$ (купівля засобів
продукції) вугілля може купуватись раніше, ніж його видобудеться із
шахти. Оскільки $г$ фігурує як акумульовані гроші, а не витрачається
як дохід, то воно може репрезентувати бавовну, що її випродукується
лише в наступному році. Так само стоїть справа з витрачанням доходу
в капіталіста, в акті $г — т$. Так само із заробітною платою $Р = 50$ ф.
стерл.; ці гроші є не лише грошова форма минулої праці робітників, але
разом з тим і асиґната на одночасну або майбутню працю, що
саме тепер реалізується або має реалізуватися в майбутньому. Робітник
може купити на них сурдут, що його зробиться лише протягом найближчого
тижня. Саме так стоїть справа щодо дуже великого числа доконечних
життьових засобів, що їх доводиться споживати майже безпосередньо в
момент їх продукції, щоб запобігти їхньому псуванню. Таким чином
робітник одержує в грошах, що в них видається йому його заробітну плату,
перетворену форму своєї власної майбутньої праці або праці інших робітників.
У частині його минулої праці капіталіст дає йому асиґнату на його
власну майбутню працю. Його власна одночасна або майбутня праця
являє той ще не наявний запас, що ним йому платиться за його минулу
працю. Тут цілком зникає уявлення про утворення запасу.

Подруге. У циркуляції $Т — Г — Т\splitfrac{Р}{Зп}$ ті самі гроші переміщуються двічі; капіталіст одержує їх
спочатку як продавець і передає їх далі як покупець;
перетворення товару на грошову форму придається лише для того,
щоб з грошової форми знову перетворити його на товарову форму;
тому грошова форма капіталу, його буття як грошового капіталу,
є в цьому русі лише минущий момент; інакше кажучи, поки триває цей
рух, грошовий капітал, якщо він придається як купівельний засіб, виступає
лише як засіб циркуляції; як власне засіб виплати він виступає тоді,
коли капіталісти навзаєм купують один в одного, а потім доводиться
лише вирівняти балянс виплат.

Потретє, функціонування грошового капіталу, все одно, чи придається
він як простий засіб циркуляції, чи як засіб виплати, упосереднює лише
заміщення $Т$ на $Р$ і $Зп$, тобто заміщення пряжі, товарового продукту,
що є результат (відлічивши додаткову вартість, що її вживається як дохід)
продуктивного капіталу, на елементи продукції; отже, воно упосереднює
зворотне перетворення капітальної вартости з її товарової форми на
витворчі елементи цього товару; отже, кінець-кінцем, це функціонування
упосереднює лише зворотне перетворення товарового капіталу на продуктивний
капітал.
