самозростання вартости, самозростання, що надходить до нього із сфери
циркуляції. Далі ми побачимо, як навіть наукова економія дозволяє собі
через цю позірність допуститись помилки. Як ми доведемо далі, ця позірність
зміцнюється різними явищами: 1) капіталістичним способом обчислювати
зиск, при якому неґативна причина фігурує як позитивна, бо для капіталів
у різних сферах приміщення, де лише час обігу різний, довший час
обігу діє як причина підвищення цін, — коротко кажучи, як одна з причин
вирівнювання зиску. 2) Час обігу становить тільки один момент часу обороту;
а цей останній має в собі час продукції, зглядно час репродукції.
Те, що завдячує останньому, здається, ніби завдячує воно часові обігу.

3) Перетворення товарів на змінний капітал (заробітну плату) зумовлено
попереднім перетворенням їх на гроші. Отже, при акумуляції капіталу
перетворення на додатковий змінний капітал відбувається в сфері циркуляції
перед або протягом часу обігу. Тому й здається, що акумуляція,
яка відбувається разом із цим, завдячує обігові.

У сфері циркуляції капітал перебігає в тій або іншій послідовності
дві протилежні фази Т — Г і Г — Т. Його час обігу розпадається, отже,
на дві частини, — час, що його він потребує, щоб перетворитися з товару
на гроші, і час, що його він потребує, щоб перетворитися з грошей на
товар. Ми вже знаємо з аналізи простої товарової циркуляції (кн. І,
розділ III), що Т — Г, продаж, є найважча частина його метаморфози,
і тому в звичайних обставинах він становить більшу частину часу
обігу. Вартість у формі грошей перебуває в такій формі, що її завжди
можна перетворити на іншу. Але у формі товару вона лише по перетворенні
на гроші мусить набрати форми безпосередньої вимінности,
а тому постійної готовости до діяльности. Однак у процесі циркуляції
капіталу на його стадії Г — Т справа йде про перетворення його на
товари, що становлять певні елементи продуктивного капіталу в даному
підприємстві. Може статися, що засобів продукції немає ще на ринку, і
що треба їх спочатку ще лише випродукувати або довезти з віддалених
ринків, або можуть статися порушення в їхньому звичайному поданні,
зміни цін тощо, коротко кажучи, може статися ряд обставин, що
їх не видно в простій зміні форми Г — Т, але які також для цієї частини
фази циркуляції потребують більшого або меншого часу. Так само як
Т — Г і Г — Т відділені один від одного в часі, так само вони можуть
бути відділені і в просторі: ринок купівлі і ринок продажу можуть бути
просторово різними ринками. Напр., на фабриках закупник і продавець
часто є дві різні особи. За товарової продукції циркуляція так само
потрібна, як і сама продукція, отже, аґенти циркуляції так само потрібні,
як і аґенти продукції. Процес репродукції містить у собі обидві функції
капіталу, а тому й доконечність представництва цих функцій, чи в особі
самого капіталіста, чи в особі найманого робітника, аґента капіталіста.
Однак це зовсім не може бути підставою для того, щоб сплутувати
аґентів циркуляції з аґентами продукції, так само, як не може воно
бути підставою для того, щоб сплутувати функції товарового капіталу
й грошового капіталу з функціями продуктивного капіталу. Аґентів цирку-
