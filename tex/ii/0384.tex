вже за простої товарової циркуляції, за довгий час до того, як ця циркуляція
уґрунтується на капіталістичній товаровій продукції; кількість
наявних у суспільстві грошей завжди більша за ту частину їх, яка є
в активній циркуляції, хоч ця частина, залежно від обставин більшає
або меншає. Тут ми знову маємо такі самі скарби й таке саме утворення
скарбів, але тепер уже як момент, іманентний капіталістичному процесові
продукції.

Можна зрозуміти таку приємність, коли при системі кредиту всі ці
потенціальні капітали, концентруючись у банках і т. ін., стають капіталом,
що ним можна порядкувати, „loanable capital“ *), грошовим капіталом,
і саме вже не пасивним, не музикою майбутнього, а активним,
Wucher-капіталом (тут слово Wucher**) — в розумінні зростання).

Але А переводить таке утворення скарбу лише остільки, оскільки він —
щодо свого додаткового продукту — виступає тільки як продавець, не виступаючи
потім, як покупець. Отже, послідовна продукція додаткового
продукту, — носія його додаткової вартости, що її треба перетворити
на золото, — є для нього передумова утворення скарбу. В даному разі,
коли ми розглядаємо циркуляцію лише в межах категорії І, натуральна
форма додаткового продукту, як і всього продукту, що з нього додатковий
продукт являє частину, є натуральна форма одного з елементів
сталого капіталу І, тобто належить до категорії засобів продукції для
засобів продукції. Що з нього стає, тобто для якої функції він служить
в руках покупців В, В', В" і т. ін. це ми зараз побачимо.

Але тут ми насамперед повинні пам’ятати ось що: хоч А на свою
додаткову вартість вилучає гроші з циркуляції й нагромаджує їх як
скарб, він, з другого боку, подає в циркуляцію товари, не вилучаючи
з неї за них інших товарів, в наслідок чого В, В, ' В" і т. ін. і собі
можуть подавати в циркуляцію гроші і натомість вилучати з неї лише
товари. В даному разі цей товар і своєю натуральною формою і своїм
призначенням входить як основний або поточний елемент в сталий капітал
В, В', і т. ін. Про останній ми скажемо докладніше, коли матимемо
справу з покупцями додаткового продукту, В, В' і т. ін.

Між іншим, зазначмо тут ось що: як і раніше, при досліді простої
репродукції, ми тут знову бачимо, що перетворення різних складових
частин річного продукту, тобто їхня циркуляція (а вона мусить разом
з тим охоплювати й репродукцію капіталу, а саме його відновлення
в різних його визначенностях: сталого, змінного, основного, обігового,
грошового капіталу й товарового капіталу) зовсім не має за передумову
простої купівлі товару, доповненої наступним продажем, або продажу,
доповненого наступною купівлею, так щоб в дійсності тільки обмінювалось
товари один на один, як це припускає політична економія, а саме фрит-

*) Позиковий капітал. Ред.

**) В німецькій мові слово Wucher має значення, крім звичайного — лихварство,
діє й значення — зростання. Ред.
