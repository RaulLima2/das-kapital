сований ним капітал, та ще 5000 ф. стерл. додаткової вартости. Для власника
капіталу C (останнього розглянутого нами капіталу в 5500 ф. стерл.)
спродуковано протягом року 5000 ф. стерл. додаткової вартости (при витраті
5000 ф. стерл. і нормі додаткової вартости в 100\%), але авансований
ним капітал, а так само і спродукована додаткова вартість ще не повернулись
до нього.

М' = m'n виражає, що норма додаткової вартости, яка має силу
для змінного капіталу, застосованого протягом одного періоду обороту:

маса додаткової вартости, створена протягом одного періоду обороту:
змінний капітал, застосований протягом одного періоду обороту

має бути помножена на число періодів обороту або на число періодів
репродукції авансованого змінного капіталу — на число періодів, що протягом
їх він відновлює свій кругобіг.

В книзі І, розділ IV („Перетворення грошей на капітал“), а потім у
книзі І, розділ XXI („Проста репродукція“) ми бачили вже, що капітальну
вартість взагалі авансується, а не витрачається, бо ця вартість,
проробивши різні фази свого кругобігу, знову повертається до свого
вихідного пункту, та ще й збагачена додатковою вартістю. Це характеризує
її, як авансовану вартість. Час, що минає від її вихідного пункту
до пункту її повороту, і є той час, що на нього авансується її. Ввесь
кругобіг, що його перебігає капітальна вартість, вимірюваний часом від її
авансування де її повороту, становить її оборот, а час тривання цього обороту
становить період обороту. Коли цей період закінчився і кругобіг
вивершено, то та сама капітальна вартість може знову почати той самий
кругобіг, отже, і знову зростати в своїй вартості, утворювати додаткову
вартість. Коли змінний капітал, як в А, обертається десять разів на рік,
то протягом року тим самим авансованим капіталом десять разів утворюється
таку масу додаткової вартости, яка відповідає одному періодові
обороту.

Треба з’ясувати природу авансування з погляду капіталістичного суспільства.
Капітал А, що обертається десять разів протягом року, авансується
протягом року десять разів. На кожний новий період обороту його авансується
знову. Але разом з тим протягом року А ніколи не авансує нічого
більшого, ніж ту саму капітальну вартість в 500 ф. стерл., і дійсно,
для розглядуваного нами продукційного процесу він ніколи не має в
своєму розпорядженні нічого більшого понад ці 500 ф. стерл. Скоро тільки
ці 500 ф. стерл. закінчують один кругобіг, капіталіст А повертає їх
знову на такий самий кругобіг: капітал з природи своєї зберігає характер
капіталу лише тому, що він завжди в повторюваних процесах продукції
функціонує як капітал. Його тут ніколи не авансується на довший час,
ніж 5 тижнів. Коли оборот триватиме довший час, то капіталу не вистачить.
Коли час обороту скорочується, то частина капіталу стає надлишковою.
Тут авансується не десять капіталів по 500 ф. стерл., а один
