\parcont{}  %% абзац починається на попередній сторінці
\index{ii}{0109}  %% посилання на сторінку оригінального видання
роду. Підчас свого функціонування вони не зберігають своєї самостійної
споживної форми. Отже, підчас їхнього функціонування жодна частина
капітальної вартости не лишається фіксована в своєму першому споживному
вигляді, в своїй натуральній формі. Та обставина, що ця частина
допоміжних матеріялів речово не входить в продукт, але ввіходить у
вартість продукту лише своєю вартістю, як частина його вартости, і що
в зв’язку з цим функціонування таких матеріялів міцно прикріплено до
сфери продукції, — призвело деяких економістів, напр., Рамсая, до того,
що вони (одночасно сплутуючи основний і сталий капітал) зачислили їх
до категорії основного капіталу.

Частина засобів продукції, що речово ввіходить у продукт, отже,
сировинний матеріял і т. ін., набуває в наслідок цього почасти таких
форм, що в них вона може пізніше ввійти в особисте споживання як
засоби споживання. Власне засоби праці, речові носії основного капіталу,
споживається лише продуктивно, і не можуть вони ввійти в особисте
споживання, бо вони не ввіходять у продукт або в ту споживну вартість,
що її вони допомагають утворити, а, навпаки, зберігають проти
неї свою самостійну форму ввесь час, поки вони цілком зносяться. Виняток
становлять тільки засоби транспорту. Корисний ефект, що його
вони дають підчас свого продуктивного функціонування, тобто підчас
перебування в сфері продукції, — зміна місця, ввіходить одночасно в особисте
споживання, напр., пасажира. Він також оплачує тут споживання,
як оплачує користування з інших засобів споживання. Ми бачили, що
сировинний матеріял і допоміжні матеріяли іноді зливаються один з
одним, напр., у хемічній фабрикації. Те саме буває й з засобами праці
та допоміжними матеріялами й сировинним матеріялом. Напр., в хліборобстві
речовини, вкладені для поліпшення ґрунту, почасти ввіходять як продуктотворчі
елементи в рослинний продукт. З другого боку, їхня дія
розподіляється на відносно довгий період, напр., 4--5 років. Тому частина
їх речово ввіходить у продукт і разом з тим переносить свою
вартість на продукт, тимчасом як друга частина зберігає свою стару
споживну форму і фіксує в ній свою вартість. Вона і далі існує як засіб
продукції і тому набирає форми основного капіталу. Як робоча худоба
віл є основний капітал. А якщо його з’їдають, він функціонує вже не як
засіб праці, отже, не як основний капітал.

Причина (Bestimmung), що надає частині капітальної вартости, витраченій
на засоби продукції, характеру основного капіталу, є виключно в
своєрідному способові циркуляції цієї вартости. Цей особливий спосіб
циркуляції випливає з того особливого способу, що ним засоби праці
віддають свою вартість продуктові, або з того способу, в який вони виступають
як вартіснотворчі чинники підчас продукційного процесу. А цей останній
й собі випливає з особливого способу функціонування різних засобів
праці в процесі праці.

Відомо, що та сама споживна вартість, що виходить як продукт з
одного процесу праці, ввіходить у другий як засіб продукції. Тільки
функціонування продукту як засобу праці в продукційному процесі робить
\index{ii}{0110}  %% посилання на сторінку оригінального видання
його основним капіталом. Навпаки, коли він ще тільки сам виходить
з процесу, він зовсім не є основний капітал. Напр., машина, як продукт,
зглядно товар фабриканта-машинобудівника належить до його
товарового капіталу. Основним капіталом вона стає лише в руках покупця,
капіталіста, що продуктивно її вживає.

Припускаючи всі інші умови за однакові, ступінь зв’язаности основного
капіталу зростає разом із тривалістю засобів праці. Саме від цієї
тривалости залежить величина ріжниці між капітальною вартістю, фіксованою
в засобах праці, і тією частиною капітальної вартости, яку вона
в повторюваних процесах праці віддає продуктові. Що повільніше відбувається
ця передача вартости, — а вартість передається з засобів праці
при всякому повторенні того самого процесу праці, — то більший фіксований
капітал, то більша ріжниця між капіталом, застосованим у продукційному
процесі, і капіталом, що в ньому зужитковується. Скоро ця
ріжниця зникає, це значить, що засіб праці віджив свій час і разом із
своєю споживною вартістю втратив свою вартість. Він перестав бути
носієм вартости. Через те, що засіб праці, як і кожний інший речовий
носій сталого капіталу, віддає свою вартість продуктові лише в тих розмірах,
в яких разом із споживною вартістю він втрачає і вартість, то
очевидно, що як повільніше втрачає він свою споживну вартість, що
довше він перебуває в продукційному процесі, то й довший буде період,
протягом якого вартість сталого капіталу лишається в ньому фіксована.

Коли засіб продукції, що не є засіб праці у власному розумінні,
напр., допоміжний матеріял, сировинний матеріял, напівфабрикат тощо,
перенесенням своєї вартости, а тому й способом циркуляції своєї вартости
відіграє таку саму ролю як засоби праці, то він так само є речовий
носій, форма існування основного капіталу. Так буває при вищезгаданих
земельних меліораціях, коли в ґрунт додається хемічні складові
частини, що їхнє діяння поширюється на багато продукційних періодів
або років. Тут частина вартости і далі існує поряд продукту в
своїй самостійній формі або в формі основного капіталу, тимчасом як друга
частина вартости передається на продукт, а тому разом з ним циркулює.
В цьому разі в продукт входить не лише частина вартости основного капіталу,
а й споживна вартість, та субстанція, що в ній існує ця частина вартости.

Лишаючи осторонь основну помилку — сплутування категорій: основний
і обіговий капітал з категоріями: сталий і змінний капітал — плутанина
в дотеперішньому визначенні понять в економістів ґрунтується насамперед
на таких пунктах.

Певні властивості, речово належні засобам праці, вони перетворюють
на безпосередні властивості основного капіталу, напр., таку, як фізична
нерухомість хоча б будинку. Але завжди легко довести, що інші засоби
праці, що, як такі, теж є основний капітал, мають протилежні властивості,
напр., фізична рухомість хоча б корабля.

Але економічну визначеність форми, що походить з циркуляції вартости,
вони сплутують з речовою властивістю; ніби речі, які самі собою
взагалі не є капітал, а робляться ним лише в певних суспільних відносинах,
\index{ii}{0111}  %% посилання на сторінку оригінального видання
могли б самі собою та із своєї природи бути капіталом в
тій або іншій певній формі, основним або обіговим. Ми бачили в книзі
І, розділ V, що засоби продукції в кожному процесі, хоч при яких суспільних
умовах він відбувається, завжди поділяються на засоби праці й
предмет праці. Але тільки за капіталістичного способу продукції обидва
вони робляться капіталом, саме „продуктивним капіталом“, як це
визначено в попередньому розділі. Разом з тим ріжниця між засобом
праці й предметом праці, яка ґрунтується на природі процесу праці,
відбивається в новій формі, як ріжниця між основним капіталом та обіговим.
Лише відтепер річ, що функціонує як засіб праці, робиться основним
капіталом. Якщо вона своїми речовими властивостями може придаватись
і в інших функціях, крім функцій засобів праці, то вона буде
основним капіталом або не буде, залежно від відмінности свого функціонування.
Худоба як робоча худоба, є основний капітал; худоба на заріз
є сировинний матеріял, що, кінець-кінцем, як продукт, входить у
циркуляцію — отже, це не основний капітал, а обіговий.

Простий стан довгочасної фіксованости якогобудь засобу продукції
в повторюваних процесах праці, що між собою зв’язані й являють
безперервний ряд і тому становлять період продукції — тобто ввесь час
продукції, потрібний на те, щоб виготувати продукт, — цей стан довгочасної
фіксованости зумовлює цілком так само, як і основний капітал,
авансування з боку капіталіста на довший або коротший час, але не
перетворює його капіталу на основний капітал. Насіння, напр., зовсім не
є основний капітал, а лише сировинний матеріял, що його майже на
цілий рік фіксується в процесі продукції. Всякий капітал, поки він функціонує
як продуктивний капітал, фіксується в процесі продукції, отже,
фіксуються і всі елементи продуктивного капіталу, хоч яка буде їхня речова
форма, їхня функція та спосіб циркуляції їхньої вартости. Чи це фіксування
триває довший, чи коротший час, залежно від способу продукційного
процесу або бажаного корисного ефекту, не це утворює ріжницю між
основним та обіговим капіталом\footnote{
Що дуже важко дати визначення основного та обігового капіталу, то пан
Льоренц Штайн каже, що ця ріжниця придається лише для популярности викладу.
}.

Частина засобів праці, — куди належать і загальні умови праці — або
прикріплюється до певного місця, коли вона як засіб праці входить у процес
продукції, зглядно, коли її підготовлюється до продуктивної функції, як
напр., машини. Або частину засобів праці з самого початку продукується
в такій нерухомій, зв’язаній з місцем формі, як, напр., земельні
меліорації, фабричні будівлі, домни, канали, залізниці тощо. Постійна
зв’язаність засобів праці з продукційним процесом, що в ньому вони
повинні функціонувати, зумовлюється тут уже речовим способом їхнього
існування. З другого боку, засоби праці можуть фізично завжди переміщуватись,
рухатись, і все ж бути завжди в продукційному процесі, як,
напр., локомотив, судно, робоча худоба і т. ін. Ні нерухомість не надає
їм у першому випадку характеру основного капіталу, ні рухомість не
\parbreak{}  %% абзац продовжується на наступній сторінці
