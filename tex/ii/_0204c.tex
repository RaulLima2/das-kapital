\index{ii}{0204}  %% посилання на сторінку оригінального видання

\begin{table}[h]
  \begin{center}

  \caption*{Капітал II.}
  \begin{tabular}{r@{\hspace{1}} c@{\hspace{1}} r@{\textendash{}} l c@{\hspace{1}} r@{\textendash{}} l c@{\hspace{1}} r@{\textendash{}} l}
  \toprule
  \multicolumn{4}{c}{Періоди обороту} & \multicolumn{3}{c}{Робочі періоди} & \multicolumn{3}{c}{Періоди обігу}\\
  \cmidrule(r){1-4}
  \cmidrule(r){5-7}
  \cmidrule{8-10}
  І.  & Тижні         & 4 & 12   & Тижні         & 4 & 6   & Тижні & 7 & 12\\
  ІІ. & \ditto{Тижні} & 13 & 21 & \ditto{Тижні} & 13 & 15 & \ditto{Тижні} & 16 & 21\\
  III.& \ditto{Тижні} & 22 & 30 & \ditto{Тижні} & 22 & 24 & \ditto{Тижні} & 25 & 30\\
  IV. & \ditto{Тижні} & 31 & 39 & \ditto{Тижні} & 31 & 33 & \ditto{Тижні} & 34 & 39\\
  V.  & \ditto{Тижні} & 40 & 48 & \ditto{Тижні} & 40 & 42 & \ditto{Тижні} & 43 & 48\\
  VI. & \ditto{Тижні} & 49 & [57] & \ditto{Тижні}& 49 & 51 & \ditto{Тижні} & [52 & 57]\\
  \end{tabular}
\end{center}
\end{table}

\begin{table}[h]
  \begin{center}

  \caption*{Капітал III.}
  \begin{tabular}{r@{\hspace{1}} c@{\hspace{1}} r@{\textendash{}} l c@{\hspace{1}} r@{\textendash{}} l c@{\hspace{1}} r@{\textendash{}} l}
  \toprule
  \multicolumn{4}{c}{Періоди обороту} & \multicolumn{3}{c}{Робочі періоди} & \multicolumn{3}{c}{Періоди обігу}\\
  \cmidrule(r){1-4}
  \cmidrule(r){5-7}
  \cmidrule{8-10}
  І.  & Тижні         & 7 & 15   & Тижні         & 7 & 9   & Тижні & 10 & 15\\
  ІІ. & \ditto{Тижні} & 16 & 24 & \ditto{Тижні} & 16 & 18 & \ditto{Тижні} & 19 & 24\\
  III.& \ditto{Тижні} & 25 & 33 & \ditto{Тижні} & 25 & 27 & \ditto{Тижні} & 28 & 33\\
  IV. & \ditto{Тижні} & 34 & 42 & \ditto{Тижні} & 34 & 36 & \ditto{Тижні} & 37 & 42\\
  V.  & \ditto{Тижні} & 43 & 51 & \ditto{Тижні} & 43 & 45 & \ditto{Тижні} & 46 & 51\\
  \end{tabular}
\end{center}
\end{table}

Тут ми маємо точну подобу випадку І, з тією лише ріжницею, що
тепер чергуються три капітали, замість двох. Схрещування або переплітання
капіталів немає; кожен поодинокий капітал можна простежити
окремо до кінця року. Отже, тут, так само, як і в випадку І, наприкінці
робочого періоду, не постає звільнення капіталу. Капітал І, що його цілком
витрачено на кінець 3-го тижня, припливає цілком назад наприкінці 9-го
тижня і знову починає функціонувати на початку 10-го тижня. Так само і з
капіталами II і III. Правильне й повне чергування капіталів виключає
будь-яке звільнення.

Весь оборот обчисляється так:
\begin{table}[h]
  \begin{center}
  \begin{tabular}{c@{ } r@{ } c@{ } l@{ }  c@{ } c@{ } c@{ } c@{ } c}
  Капітал & І & = & 300 & ф. стерл. & × & 5\sfrac{2}{3} & = & 1700 ф. стерл. \\

  \ditto{Капітал} & II & = & 300 & \ditto{ф.} \ditto{стерл.}   & × & 5\sfrac{1}{3} & = & 1600 ф. стерл. \\

  \ditto{Капітал} & III & = & 300 & \ditto{ф.} \ditto{стерл.} & × & 5 & = & 1500 ф. стерл. \\
  \midrule
  Ввесь капітал & & & 900 & \ditto{ф.} \ditto{стерл.}  & × & 5\sfrac{1}{3} & = & 4800 ф. стерл.\\
  \end{tabular}
  \end{center}
\end{table}

Візьмімо тепер ще один приклад, де період обігу не є точне кратне
робочому періоду. Напр., робочий період 4 тижні, період циркуляції
5 тижнів; отже, в такому разі відповідні розміри капіталу були б:
капітал І = 400 ф. стерл., капітал II = 400 ф. стерл., капітал III = 100 ф. стерл.
\begin{table}[h]
\begin{center}
\begin{tabular}{c@{ } r@{ } c@{ } c@{ } c }
Капітал & I & = & 400 & ф. стерл.\\

\ditto{Капітал} & II & = & 400 & ф. стерл.\\

\ditto{Капітал} & III & = & 400 & ф. стерл.\\
\end{tabular}
\end{center}
\end{table}
