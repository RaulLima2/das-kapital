ровий капітал, хоч ні, все ж вони такі самі товари, як Т' і відносяться один до одного як товари. Це
саме має силу й щодо другого т в т — г — т. Отже, оскільки Т' = Т (Р+Зп), остільки товари є творчі
елементи самого Т' і остільки воно само мусить замінюватися в циркуляції на такі самі товари; і в т
— г — т друге т теж мусить замінюватись у циркуляції так само на інші товари.

Крім того, на основі капіталістичного способу продукції, як панівного, кожен товар у руках продавця
мусить бути товаровим капіталом. Він і далі лишається таким у руках купця або стає таким в його
руках, коли не був ним раніше. Або — як, напр., довізні товари — він мусить бути товаром, що
замістив первісний товаровий капітал і тому надав йому лише іншу форму буття.

Товарові елементи Р і Зп, що з них складається продуктивний капітал П, як форма буття П мають не той
самий вигляд, що був у них на тих різних товарових ринках, де їх придбали. Тепер їх сполучено, і в
такому своєму сполученні вони можуть функціонувати як продуктивний капітал.

Той факт, що лише в цій III формі, в межах самого кругобігу, Т з’являється як передумова Т, походить
із того, що за вихідний пункт є капітал у товаровій формі. Кругобіг починається перетворенням Т'
(оскільки вона функціонує як капітальна вартість — хоч збільшена додатковою вартістю, хоч ні) на
товари, що являють елементи його продукції. Але це перетворення охоплює цілий процес циркуляції Т —
Г — Т (= Р+Зп) і є результат його. Отже, тут Т стоїть на обох крайніх пунктах, але другий крайній
пункт, що набуває своєї форми через акт Г
— Т іззовні, з товарового ринку, не є останній пункт кругобігу, а лише останній пункт його двох
перших стадій, що охоплюють процес циркуляції. Його результат є П, що його функція, процес
продукції, починається після цього. Лише як результат цього процесу, отже, не як результат процесу
циркуляції, Т з’являється як завершення кругобігу і в тій самій формі, як і початковий пункт Т'.
Навпаки, в Г... Г', П... П, кінцеві крайні пункти Г' і П є безпосередні результати процесу
циркуляції. Отже, тут лише наприкінці кругобігу припускається, що в чужих руках перебуває одного
разу Г', другого разу П. Оскільки кругобіг відбувається між крайніми пунктами, остільки ні Г в
першому випадку, ані П в другому, — тобто ні буття Г як чужих грошей, ані буття П як чужого
продукційного процесу — не є передумова цих кругобігів. Навпаки, Т'... Т' припускає Т (= Р+Зп) як
чужі товари,
що перебувають у чужих руках, втягуються в кругобіг через увідний процес циркуляції і перетворюються
на продуктивний капітал, а як результат функціонування цього капіталу Т' тепер знову стає кінцевою
формою кругобігу.

Але саме тому, що кругобіг Т... Т припускає в своїх межах інший промисловий капітал у формі Т (=
Р+Зп) (а Зп охоплюють різноманітні інші капітали, напр., у даному разі — машини, вугілля, мастиво
тощо), то він сам призводить до того, що його розглядають не лише як загальну форму кругобігу, тобто
не лише як таку суспільну форму,
