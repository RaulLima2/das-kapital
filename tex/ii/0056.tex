дія е його процес продукції, як попередня умова наступного процесу
циркуляції. Навпаки, кінцеве П не є процес продукції; воно є лише
повторне буття промислового капіталу в його формі продуктивного
капіталу. А саме воно е результат перетворення, що сталося в
останній фазі циркуляції — перетворення капітальної вартости на Р + Зп,
на суб’єктивні й об’єктивні чинники, які в своєму сполученні становлять
форму буття продуктивного капіталу. Капітал, хоч буде він П, хоч П',
кінець-кінцем, є знову наявний у такій формі, що в ній він мусить знову
функціонувати як продуктивний капітал, здійснювати процес продукції.
Загальна форма руху, П... П, є форма репродукції і не показує, як це показує
Г... Г', що зростання вартости є мета цього процесу. Тому то більше полегшує
вона клясичній економії змогу не звертати уваги на певну капіталістичну
форму продукційного процесу й зображати продукцію як таку
за мету процесу, яка є ніби в тім, щоб яко мога більше й дешевше
продукувати й обмінювати продукт на якнайрізноманітніші інші продукти,
почасти для відновлення продукції (Г — Т), почасти для споживання
(г — т). А що при цьому Г і г з’являються лише як минущий засіб
циркуляції, то особливостей грошей і грошового капіталу не помічається, і
ввесь процес здається простим і природним, тобто має природність
плаского раціоналізму. Так само, розглядаючи товаровий капітал, забувають
іноді зиск, і коли мовиться про кругобіг продукції в цілому, то капітал
фігурує лише як товар; але коли мова йде про складові частини вартости,
то він фігурує як товаровий капітал. Акумуляція зображається, природно,
таким самим способом, як і продукція.

У формі III, Т' — Г' — Т... П... Т', кругобіг починають дві фази
процесу циркуляції, і саме таким самим порядком, як і у формі II,
П... П.; потім іде П, саме так, як і в формі І, із своєю функцією,
з процесом продукції; разом з результатом цього процесу, Т', кругобіг
закінчується. Так само, як у формі II, він закінчується П, як простим
повторним буттям продуктивного капіталу, так само тут він закінчується
Т', повторним буттям товарового капіталу; так само, як у
формі II капітал у своїй кінцевій формі П знову мусить розпочати
процес як процес продукції, так само й тут разом з повторною появою
промислового капіталу в формі товарового капіталу, кругобіг мусить
знову початись фазою циркуляції Т' — Г'. Обидві форми кругобігу тут
невивершені, бо їх ще не вивершило Г', перетворена на гроші виросла
капітальна вартість. Обидві, отже, вони мусять провадитись далі, а тому
містять у собі репродукцію. Цілий кругобіг у формі III є Г... Т'.

Третя форма від двох попередніх відрізняється тим, що лише в цьому
кругобігу за вихідний пункт зростання вартости є виросла капітальна
вартість, а не первісна капітальна вартість, що лише ще має зростати.
Т' як капіталістичне відношення є тут вихідний пункт і як таке визначально
впливає на ввесь кругобіг, бо вже на першій фазі
своїй воно містить у собі кругобіг капітальної вартости та кругобіг
додаткової вартости, а додаткова вартість, коли й не в кожному окремому
кругобігу, то пересічно, мусить витрачатись почасти як дохід, про-
