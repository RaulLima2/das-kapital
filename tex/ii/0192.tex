3-й робочий період: тижні 11—15 (500 ф. стерл. товару повертаються
наприкінці 20 тижня).

4-й робочий період: тижні 16—20 (500 ф. стерл. товару повертаються
наприкінці 25 тижня).

5-й робочий період: тижні 21—25 (500 ф. стерл. товару повертаються
наприкінці 30 тижня) і т. д.

Коли час обігу дорівнює 0, отже, коли період обороту дорівнює
робочому періодові, то число оборотів дорівнює числу робочих періодів
на рік.
Отже, при п’ятитижневому робочому періоді воно було б = 50/5
тижнів, тобто 10, а вартість капіталу, що обернувся, була б 6 = 500X10 =
5000.
В таблиці, де час обігу припущено в 5 тижнів, так само щороку
продукується товарів вартістю в 5000 ф. стерл., але з них 1/10 = 500 ф.
стерл. завжди перебуває у вигляді товарового капіталу й повертається
назад лише по 5 тижнях. Наприкінці року продукт десятого робочого
періоду (46—50 робочі тижні) закінчив лише половину свого часу
обороту, при чому його час обігу припадає на перші 5 тижнів наступного року.

Візьмімо ще третій приклад: робочий період 6 тижнів, час обігу
З тижні, щотижневе авансування на процес праці 100 ф. стерл.

1-й робочий період: тижні 1—6. Наприкінці 6-го тижня є товаровий
капітал в 600 ф. стерл., він повертається наприкінці 9-го тижня.

2-й робочий період: тижні 7—12. Протягом 7—9-го тижнів авансовано
300 ф. стерл. додаткового капіталу. Наприкінці 9-го тижня повертаються
назад 600 ф. стерл. З них протягом 10—12 тижнів авансовано 300 ф.
стерл.; отже, наприкінці 12 тижня є вільних 300 ф. стерл. і в товаровому
капіталі є 600 ф. стерл., що повертаються наприкінці 15 тижня.

3-й робочий період: тижні 13—18. Протягом 13—15-го тижнів авансується
вищезгадані 300 ф. стерл., потім повертаються назад 600 ф. стерл.,
з них 300 ф. сторл. авансується на 16—18 тижні. Наприкінці 18-го тижня
є вільних 300 ф. стерл. грішми; 600 ф. стерл. є в товаровому капіталі,
що повертаються наприкінці 21 тижня (див. докладніший виклад
цього випадку далі під II).

Отже, протягом 9 робочих періодів (= 54 тижням) продукується товару
на 600×9 = 5400 ф. стерл. Наприкінці дев’ятого робочого періоду
капіталіст має 300 ф. стерл. грішми і 600 ф. стерл. товаром, що не
проробив ще свого часу обігу.

Порівнюючи ці три приклади, ми бачимо, поперше, що лише в другому
прикладі відбуваеті ся послідовна зміна капіталу І в 500 ф. стерл.
і додаткового капіталу II так само в 500 ф. стерл., так що ці дві частини
капіталу рухаються відокремлено одна від однієї і саме лише тому, що
тут припускається цілком винятковий випадок, що робочий період і час
обігу становлять дві однакові половини періоду обороту. В усіх інших
випадках, хоч яка буде нерівність між двома періодами цілого періоду
обороту, рух обох капіталів навзаєм переплітаємся, як у прикладах
І і III, вже починаючи з другого періоду обороту. В цих випадках додат-
