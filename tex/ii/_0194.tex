\parcont{}  %% абзац починається на попередній сторінці
\index{ii}{0194}  %% посилання на сторінку оригінального видання
купувати тільки що два тижні. А що при цьому кредит ще виключається,
то ця частина капіталу, оскільки вона є до розпорядку не в формі продуктивного
запасу, мусить лишатись вільною у формі грошей, щоб в разі
потреби перетворюватись на продуктивний запас. Це нічого не змінює у
величині вартости сталого обігового капіталу, авансованого на 6 тижнів.
Навпаки — лишаючи осторонь грошовий запас на непередбачені витрати,
власне запасний фонд для вирівнювання порушень — заробітну плату видається
в коротші строки, здебільша щотижня. Отже, якщо тільки капіталіст
не примушує робітника авансувати йому свою працю на довший час,
капітал, потрібний для заробітної плати, мусить бути в грошовій формі.
Отже, при зворотному припливі капіталу частину його треба зберігати в
грошовій формі для оплати праці, тимчасом як другу частину можна перетворити
на продуктивний запас.

Додатковий капітал розподіляється цілком так само, як первісний.
Але відрізняє його від капіталу І те, що він (лишаючи осторонь кредитові
відносини) — для того, щоб ним можна було порядкувати, для його
власного робочого періоду — мусить бути вже авансований на весь
час тривання першого робочого періоду капіталу І, періоду, що в нього
він не входить. Протягом цього часу він може, принаймні почасти,
перетворитися на сталий обіговий капітал, що його авансується на ввесь
період обороту. Якою мірою він набирає цієї форми, або якою мірою
він залишається у формі додаткового грошового капіталу до того моменту,
коли це перетворення стане потрібне, це залежить почасти від
особливих продукційних умов у певній галузі підприємств, почасти від
місцевих обставин, почасти від коливань цін на сировинний матеріял тощо.
Якщо розглядати цілий суспільний капітал, то більш-менш значна частина
цього додаткового капіталу завжди перебуватиме в стані грошового
капіталу протягом довшого часу. Навпаки, щодо тієї частини капіталу II,
яку треба авансовувати на заробітну плату, то її завжди перетворюється
на робочу силу лише поступінно, у міру того, як закінчуються й оплачуються
коротші робочі періоди. Отже, ця частина капіталу II протягом
цілого робочого періоду існує в формі грошового капіталу, поки вона
через перетворення на робочу силу не увійде у функціонування продуктивного
капіталу.

Отже, це долучення додаткового капіталу, потрібного для перетворення
часу обігу капіталу І на час продукції, не лише збільшує величину
авансованого капіталу і протяг часу, що на нього неодмінно авансується
ввесь капітал, але воно також збільшує, зокрема, ту частину
авансованого капіталу, що існує як грошовий запас, отже, перебуває
в стані грошового капіталу й має форму потенціального грошового
капіталу.

Це так само відбувається і тоді — все одно, чи маємо авансування в
формі продуктивного запасу, чи в формі грошового запасу — коли зумовлений
часом обігу поділ капіталу на дві частини: на капітал для першого
робочого періоду й на капітал, що заступає його протягом часу
обігу, здійснюється не тим, що збільшується витрачуваний капітал, а тим,
\parbreak{}  %% абзац продовжується на наступній сторінці
