\parcont{}  %% абзац починається на попередній сторінці
\index{ii}{0374}  %% посилання на сторінку оригінального видання
їхньої додаткової вартости, зглядно їхнього зиску = 400 ф. стерл., то
ці 400 ф. стерл. перетворюються, напр., на 500 ф. стерл. в наслідок
того, що кожен співвласник цих 400 ф. стерл. продає свою частину другому
дорожче на 25\%. Що всі роблять так, то наслідок такий самий,
як коли б вони навзаєм продавали один одному за дійсною вартістю.
Тільки для циркуляції товарової вартости в 400 ф. стерл. їм потрібна
маса грошей в 500 ф. стерл., а це є, здається, скорше метод збіднення,
ніж збагачення, бо їм доводиться чималу частину всього свого майна
непродуктивно зберігати в некорисній формі засобів циркуляції. Все
сходить на те, що кляса капіталістів, не зважаючи на всебічне номінальне
підвищення цін їхніх товарів, може розподіляти поміж себе для свого
особистого споживання лише запас товарів вартістю в 400 ф.
стерл., але вони роблять один одному приємність, пускаючи в циркуляцію
400 ф. стерл. товарової вартости за допомогою такої маси грошей,
яка потрібна для 500 ф. стерл. товарової вартости.

Ми зовсім лишаємо осторонь, що тут припускається „частину їхнього
зиску“, і значить, взагалі запас товарів, що в ньому виражається зиск.
А проте Детю хотів саме з’ясувати нам, відки походить цей зиск. Маса
грошей, потрібна для його циркуляції, це питання цілком другорядне. Та маса
товарів, яка репрезентує зиск, здається, походить від того, що капіталісти
не лише продають її один одному, хоч уже й це дуже добре й
глибоко розумно, але що вони продають один одному дуже дорого.
Отже, ми знаємо тепер одне джерело збагачення капіталістів. Воно сходить
до таємниці „ентспектора Брезіґа“, що великі злидні походять з великої
pauvreté\footnote*{
Бідности. Ред
}.

2) Далі, ті самі капіталісти продають „найманим робітникам, так тим,
що їх оплачують вони сами, як і тим, що їх оплачують капіталісти-нероби;
таким чином, вони одержують назад від цих робітників всю їхню заробітну
плату, за винятком хіба невеликих заощаджень“.

Зворотний приплив до капіталістів того грошового капіталу, що
в формі його вони авансували заробітну плату робітникові, є, за паном
Детю, друге джерело збагачення цих капіталістів.

Отже, коли кляса капіталістів виплатить робітникам, напр., 100 ф. стерл.,
як заробітну плату, а потім ті самі робітники купують товари такої самої
вартости в 100 ф. стерл. у тієї самої кляси капіталістів, і тому сума
в 100 ф. стерл., авансована капіталістами як покупцями робочої сили,
припливає до них назад при продажу цим робітникам товарів на 100 ф.
стерл., то капіталісти в наслідок цього збагачуються. З погляду
доброго розуму виходить, що капіталісти за допомогою цієї процедури
знову мають ті 100 ф. стерл., що були в них до цієї процедури. На
початку процедури в них було 100 ф. стерл. грішми, на ці 100 ф. стерл.
вони купили робочу силу. За ці 100 ф. стерл. грішми куплена праця
продукує товари вартістю, оскільки ми знаємо до цього часу, в 100 ф.
стерл. В наслідок того, що робітникам продано ці 100 ф. стерл. в товарах,
\index{ii}{0375}  %% посилання на сторінку оригінального видання
капіталісти одержують знову 100 ф. стерл. грішми. Отже, у капіталістів
знову є 100 ф. стерл. грішми, а в робітників — на 100 ф. стерл.
товару, що його вони сами спродукували. Важко зрозуміти, як могли б
капіталісти збагатитись на цьому. Коли б 100 ф. стерл. грішми не припливали
до них назад, то їм довелось би, поперше, заплатити робітникам
за їхню працю 100 ф. стерл. грішми, і, подруге, безплатно віддати
їм продукт цієї праці, засоби споживання на 100 ф. сгерл. Отже, зворотний
приплив грошей міг би, щонайбільш, пояснити, чому капіталісти
не збіднюються в наслідок цієї операції, але ні в якому разі не міг би
пояснити, чому вони з неї збагачуються.

Звичайно, друге питання, звідки капіталісти беруть ці 100 ф. стерл.
грішми, і чому робітники мусять обмінювати свою робочу силу на ці 100 ф.
стерл., замість самим продукувати товари власним коштом. Але це є щось
само собою зрозуміле для мислителів типу Детю.

Детю сам не цілком задоволений з такого розв’язання. Він бо не
сказав нам, що збагачення постає тому, що витрачають грошову суму
в 100 ф. стерл. і потім знову одержують грошову суму в 100 ф. стерл
отже, не в наслідок зворотного припливу 100 ф. стерл. грішми, який
з’ясовує лише, чому ці 100 ф. стерл. грішми не втрачається. Він сказав
нам, що капіталісти збагачуються, „продаючи все продуковане ними
дорожче, ніж коштував їм закуп цього“.

Отже, капіталісти в своїй оборудці з робітниками мусять збагачуватись
тому, що вони продають робітникам дуже дорого. Чудово! „Вони
виплачують заробітну плату\dots{} і все це зворотно припливає до них в наслідок
витрат всіх цих людей, що платять за них“ (за продукти) „дорожче,
ніж вони коштували їм“ (капіталістам) „при такій заробітній
платі“ (стор. 240). Отже, капіталісти платять робітникам 100 ф.
стерл. заробітної плати, а потім продають робітникам власний продукт
останніх за 120 ф. стерл., так що до капіталістів не лише припливають
назад ці 100 ф. сгерл, , а ще виграється 20 ф. стерл.? Це неможливо.
Робітники можуть заплатити лише тими грішми, що їх вони одержали
в формі заробітної плати. Коли вони одержали від капіталістів 100 ф.
стерл. заробітної плати, то вони можуть купити лише на 100 ф. стерл.
а не на 120 ф. стерл. Отже, цим способом питання не розв’язується. Але
є ще один спосіб. Робітники купують у капіталістів товару на 100 ф.
стерл., а в дійсності одержують товар вартістю лише на 80 ф. стерл.
Тому їх, безперечно, обшахраяли на 20 ф. стерл. А капіталіст, безперечно,
збагатився на 20 ф. стерл., бо він фактично оплатив робочу силу
на 20\% нижче від її вартости або обкружним шляхом зробив одрахування
в 20\% з номінальної заробітної плати.

Кляса капіталістів досягла б цього самого, якби вона з самого початку
виплатила робітникам заробітної плати лише 80 ф. стерл., а потім
дала б їм за ці 80 ф. стерл. грішми товарову вартість дійсно на 80 ф.
стерл. Ось такий, — коли взяти цілу клясу, — здається, нормальний спосіб,
бо, за висловом самого пана Детю, робітнича кляса мусить одержувати
„достатню заробітну плату“ (ст. 219), бо цієї заробітної плати
\parbreak{}  %% абзац продовжується на наступній сторінці
