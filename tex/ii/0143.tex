мання робочої худоби, споживається цілком і мусить постійно замішуватись
новим кормом безпосередньо з продукту хліборобства або з продажу
його; тимчасом як саму худобу заміщується лише в міру того, як
поодинокі екземпляри її по черзі стають непрацездатні). „І ціна й утримання
худоби, купленої не для роботи, а на відгодовування, є обіговий капітал.
Фармер одержує зиск, віддаючи його“. (Кожен товаропродуцент,
а, значить, і капіталістичний товаропродуцент, продає свій продукт,
результат свого процесу продукції, але через це цей продукт
ще не стає ні основною, ні поточною складовою частиною його
продуктивного капіталу. Навпаки, продукт має тепер ту форму, що
в ній він виштовхується з процесу продукції й мусить функціонувати як товаровий
капітал. Відгодовувана худоба функціонує в процесі продукції як
сировинний матеріял, а не як знаряддя праці, не як робоча худоба. Вона,
отже, входить речово в продукт, і вся її вартість переходить цілком
на цей продукт, як і вартість допоміжних матеріялів [її корму]. Саме тому
вона й є поточна частина продуктивного капіталу, а зовсім не тому,
що проданий продукт — відгодована худоба — має тут ту саму натуральну
форму, що й сировинний матеріял, тобто ще не відгодована худоба. Це —
цілком випадкова обставина. Але разом з тим Сміс міг би побачити з
цього прикладу, що не речова форма елемента продукції, а лише його
функція в межах продукційного процесу надають вартості, що міститься
в ньому, характеру основної або поточної частини). „Вся вартість засівного
зерна є теж основний капітал. Хоч воно завжди переходить з землі
в комори й назад, але воно ніколи не змінює власника, а тому в
дійсності й не циркулює. Фармер одержує свій зиск не тому, що продає
його, а тому, що кількість його більшає\footnote*{
„That part of the capital of the farmer which is employed in the implements
of agriculture is a fixed, that which is employed in the wages and maintenance of
his labouring servants is a circulating capital. He makes a profit of the one by
keeping it in his own possession, and of the other by parting with it. The price or
value of his labouring cattle is a fixed capital, in the same manner as that of the
instruments of husbandry; their maintenance is a circulating capital, in the same way
as that of the labouring servants. The farmer makes his profit by keeping the labouring
cattle, and by parting with their maintenance. Both the price and the maintenance
of the cattle which are bought in and fattened, not for labour but for sale,
are a circulating capital. The farmer makes his profit by parting with them. The whole
value of the seed, too is a fixed capital. Though it goes backwards and
forwards between the ground and the granary, it never changes masters, and therefore
it does not properly circulate. The farmer makes his profit not by its sale, but
by its increase".
}.

Тут особливо яскраво виявляється вся безглуздість Смісового відрізнення.
За його теорією, засівне зерно було б основним капіталом, якби не
відбулося change of masters\footnote*{
Зміна власника. Ред.
}, тобто, коли засівне зерно безпосередньо
заміщується з річного продукту, береться з нього. Навпаки,
воно було б обіговим капіталом, коли б продавалось увесь продукт і на
частину вартости його купувалось засівне зерно в другого власника. В
одному разі відбувається change of masters, в другому ні. Сміс тут зно-