їхньої вартости. Як вартість кожного іншого, товару, вартість робочої
сили визначається кількістю праці, доконечної для її репродукції; те, що
ця кількість праці визначається вартістю доконечних засобів існування
робітника, отже, дорівнює праці, доконечній для репродукції засобів його
власного існування, є характеристичне для цього товару (робочої сили);
але не характеристичніше за те, що вартість в’ючної худоби визначається
вартістю засобів існування, доконечних для її утримання, отже,
масою людської праці, потрібної для того, щоб випродукувати ці засоби
існування.

Саме ця категорія „доходу“ і спричиняє тут усе лихо в А. Сміса.
Різні відміни доходів становлять у нього „component parts“, складові
частини щорічно продукованої, новоутворюваної товарової вартости, тимчасом
як, навпаки, ті дві частини, що на них розкладається ця товарова
вартість для капіталіста — еквівалент його змінного капіталу, авансованого
в грошовій формі підчас закупу праці, і друга частина вартости, яка також
належить йому, хоч нічого йому й не коштувала, додаткова вартість —
становлять джерела доходів. Еквівалент змінного капіталу знов авансується
на робочу силу і остільки становить дохід для робітника у формі його
заробітної плати. Друга частина — додаткова вартість — не має заміщувати
капіталістові жодного авансування капіталу, і тому він може витратити її
на засоби споживання (доконечні, а також речі розкошів), може спожити її
як дохід, замість перетворювати її на капітальну вартість будь-якого роду.
Передумова цього доходу є сама товарова вартість, і її складові частини
лише остільки відрізняються для капіталіста, оскільки вони являють або
еквівалент за авансовану ним змінну капітальну вартість, або надлишок
над авансованою ним змінною капітальною вартістю. Обидві частини складаються
не з чого іншого, як з робочої сили, витраченої підчас продукції
товару, пущеної в рух у процесі праці. Вони складаються з витрати,
не з надходження або доходу, а з витрати праці.

Після цього qui pro quo, де дохід стає джерелом товарової вартости
замість товаровій вартості бути джерелом доходу, товарова
вартість виступає тепер „складеною“ з різних відмін доходів. їх
визначається незалежно одну від однієї, і всю вартість товару визначається
доданням величин вартости цих доходів. Але запитаймо тепер,
як визначається вартість кожного з цих доходів, що з них має постати
товарова вартість? Щодо заробітної плати, то її можна визначити, бо вона
є вартість відповідного товару, робочої сили, а цю останню визначається
(як і вартість всякого іншого товару) працею, потрібного на репродукцію
цього товару. Але як визначається додаткову вартість або радше, за А.
Смісом, дві її форми — зиск і земельну ренту? Тут усе сходить на порожню
балаканину. А. Сміс то подає заробітну плату й додаткову вартість
(зглядно — заробітну плату й зиск) як складові частини, що з них складається
товарова вартість, зглядно ціна, то — і часто майже безпосередньо
по цьому — як частини, що на них „розкладається“ (resolves itself) товарова
ціна; а це значить, навпаки, що товарова вартість є наперед дана,
і що різні частини цієї даної вартости в формі різних доходів дістаються
