\parcont{}  %% абзац починається на попередній сторінці
\index{ii}{0282}  %% посилання на сторінку оригінального видання
отже, частини вартости, які теж існують, як аліквотні частини цієї всієї
маси засобів продукції, становлять, правда, разом з тим доходи для
всіх аґентів, що беруть участь у цій продукції: заробітну
плату робітників, зиск і ренту капіталістів. Але для суспільства
вони становлять не дохід, а капітал, хоч річний продукт суспільства
складається лише із суми продуктів індивідуальних капіталістів, що належать
до цього суспільства. Більшість цих продуктів уже з самої природи
своєї може функціонувати лише як засоби продукції, і навіть ті з них,
що в разі потреби могли б функціонувати як засоби споживання, призначені
служити як сировинний або допоміжний матеріял для нової продукції.
Вони функціонують як такий — отже, як капітал — але не в руках
їхніх продуцентів, а в руках тих, хто їх застосовує, а саме:

III. Капіталістів другого відділу, безпосередніх продуцентів засобів
споживання. Ними заміщується капітал, зужиткований на продукцію
засобів споживання (оскільки цей капітал не перетворюється на робочу
силу, тобто оскільки він не становить суми заробітних плат робітників
цього другого відділу), тимчасом як цей зужиткований капітал, що тепер
у формі засобів споживання перебуває в руках капіталістів, які продукують
засоби споживання, і собі, — отже, з суспільного погляду — також
становить споживний фонд, що в ньому капіталісти
й робітники першого відділу реалізують свої доходи.

Коли б А. Сміс продовжив свою аналізу до цього пункту, він мало
не розв’язав би цілої проблеми. Він майже схопив суть справи, бо він
уже помітив, що певні частини вартости одного ґатунку (засобів продукції)
товарового капіталу, з яких складається ввесь річний продукт
суспільства, становлять, правда, дохід для індивідуальних робітників і капіталістів,
зайнятих в їхній продукції, але не становлять складової частини
доходу суспільства; тимчасом як частина вартости другого ґатунку
(засобів споживання), хоч і становить капітальну вартість для індивідуальних
власників цієї частини, — для капіталістів, зайнятих у цій сфері
застосовання капіталу — але все ж становить лише частину суспільного
доходу.

Але з усього наведеного вище випливає ось що:

Поперше: хоч суспільний капітал дорівнює лише сумі індивідуальних
капіталів, а тому й річний товаровий продукт (або товаровий капітал)
суспільства дорівнює сумі товарових продуктів цих індивідуальних капіталів;
отже, хоч розклад товарової вартости на її складові частини, який має
силу для кожного індивідуального товарового капіталу, мусить мати силу
і, кінець-кінцем, справді має силу й для капіталу цілого суспільства, все ж
та форма прояву, що в ній цей розклад товарової вартости виявляється
в сукупному суспільному процесі репродукції, є інша.

Подруге: навіть на основі простої репродукції відбувається не лише
продукція заробітної плати (змінного капіталу) та додаткової вартости,
а й безпосередня продукція нової сталої капітальної вартости, хоч робочий
день складається лише з двох частин: з однієї, що протягом її
робітних покриває змінний капітал, дійсно продукує еквівалент витрат на
\parbreak{}  %% абзац продовжується на наступній сторінці
