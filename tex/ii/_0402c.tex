\parcont{}  %% абзац починається на попередній сторінці
\index{ii}{0402}  %% посилання на сторінку оригінального видання
останньому році 732 для І і 746 для II, разом 1478. Отже, вона
зросла у відношенні $100: 134$.

\subsubsection{Другий приклад}

Візьмім тепер річний продукт в 9000, що цілком перебуває в руках
кляси промислових капіталістів як товаровий капітал, у формі, де загальне
пересічне відношення змінного й сталого капіталу становить $1: 5$.
Це має за передумову: уже значний розвиток капіталістичної продукції
й відповідний цьому розвиток продуктивної сили суспільної праці; далі
це має за передумову значне, вже раніш постале поширення маштабу
продукції, нарешті, розвиток усіх умов, що спричиняють відносне перелюднення
в робітничій клясі. Річний продукт буде тоді розподілятись по
заокругленні дробів так:

\begin{center}

 \left.\begin{aligned}
        \text{I. }5000 с + 1000 v + 1000 m = 7000\\
        \text{II. }1430 с + \phantom{0}285 v + \phantom{0}285 m = 2000
       \end{aligned}
 \right\}
 \text{= 9000.}

\end{center}

Припустімо тепер, що кляса капіталістів І половину додаткової вартости
= 500 споживає, а другу половину акумулює. Тоді ($1000 v +
500 m) \text{I} = 1500$ треба було б замістити через $1500 \text{ II} с$. А що $\text{ II} с$
дорівнює тут лише 1430, то 70 треба додати з додаткової вартости;
відлічуючи їх з $285 \text{ II} m$, маємо остачу $215 \text{ II} m$.

Отже, маємо:

I.  $5000 с + 500 m\text{(що їх треба капіталізувати)} + 1500 (v + m)$ в споживному
фонді капіталістів і робітників.

II.  $1430 c + 70 m\text{(що їх треба капіталізувати)} + 285 v + 215 m$.

А що при цьому $70 \text{ II} m$ безпосередньо долучаються до $\text{ II} c$, то для
того, щоб пустити в рух цей додатковий сталий капітал, треба змінного
капіталу в \frac{70}{5} = 14; ці 14 знову береться з $215 \text{ II} с$; лишається $201 \text{ II} m$,
і ми маємо:

II. $(1430с + 70с) + (285 v + 14 v) + 201 m$.

Обмін 1500 І ($v + \sfrac{1}{2} m$) на $1500 \text{ ІІ} с$ є процес простої репродукції,
і тому з ним закінчено. Однак ми повинні тут зазначити ще деякі особливості,
які випливають з того, що при репродукції сполученій з акумуляцією,
$\text{І} (v + \sfrac{1}{2} m)$ заміщується не самим лише $\text{ II} с$, а $\text{ ІІ} с$ плюс частина
$\text{ II} m$.

Зрозуміло само собою, що коли припускається акумуляцію, то $\text{І} (v + m)$
більше за $\text{ ІІ} с$, а не дорівнює $\text{ ІІ} с$, як при простій репродукції, бо
1) І заводить частину свого додаткового продукту в свій власний
продуктивний капітал і перетворює \sfrac{5}{6} цієї частини на сталий капітал,
отже, він не може разом з тим замістити ці \sfrac{5}{6} засобами споживання II;
2) І з свого додаткового продукту повинен дати матеріял для сталого
\parbreak{}  %% абзац продовжується на наступній сторінці
