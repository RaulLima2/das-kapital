Розділ тринадцятий

Час продукції

Робочий час завжди є час продукції, тобто час, що протягом його
капітал зв’язано в сфері продукції. Але, навпаки, не увесь час, що протягом
його капітал перебуває в процесі продукції, є в наслідок цього
доконечно також робочий час.

Тут ідеться не про ті перерви у процесі праці, що зумовлені природними
межами самої робочої сили, хоч вже й виявилось, якою поважною
спонукою незвичайного подовження процесу праці та заведення
денної й нічної роботи є та лише обставина, що основний капітал
фабричні будівлі, машини тощо — стоять без ужитку під час перерв у процесі*).
Тут ідеться про перерву, незалежну від протягу процесу праці,
зумовлену самою природою продукту та способом його виготовлення,
про перерву, що протягом її предмет праці підпадає більш-менш протяжним
природним процесам, мусить зазнати фізичних, хемічних і фізіологічних
змін, перерву, що протягом її процес праці цілком або почасти
припиняється.

Напр., щойно видавлене вино мусить деякий час шумувати, а потім
протягом деякого часу стояти, щоб набути певного ступеня досконалости.
В багатьох галузях промисловости продукт мусить сушитись, напр., у
ганчарстві, або підпадати певним впливам, що змінюють його хемічні
властивості, як от у білильнях. Озимим хлібам треба аж дев’ять місяців
вистигати. Між посівом і жнивами процес праці майже цілком припиняється.
В лісівництві після посіву та потрібних для нього підготовчих
робіт треба, може, сто років, щоб насіння перетворилося на готовий продукт;
а протягом усього цього часу потрібно прикладати відносно лише
дуже мало праці.

В усіх таких випадках протягом більшої частини часу продукції новододаваної
праці потрібно прикладати лише зрідка. Описані в попередньому
розділі умови, що за них до капіталу, вже зв’язаного в процесі
продукції, треба долучити новий додатковий капітал і новододавану працю,
здійснюються тут лише з більшими або меншими перервами.

Отже, в усіх цих випадках час продукції авансованого капіталу складається
з двох періодів: перший період, коли капітал перебуває в процесі
праці; другий період, коли форма існування капіталу — форма ще неготового
продукту — підпадає впливові природних процесів, не перебуваючи
в процесі праці. Справа ані трохи не змінюється від того, що обидва ці
періоди можуть почасти перехрещуватись та вклинюватись один в один.
Робочий період і період продукції тут не збігаються. Період продукції
є довший, ніж робочий період. Але тільки по закінченні періоду продукції
продукт є готовий, достиглий, отже, тільки тоді його можна перетворити
з форми продуктивного капіталу на форму товарового капі-

*) Див. „Капітал“, кн. І, розд. VIII, 4 та розд. XIII, 3 b. Ред.
