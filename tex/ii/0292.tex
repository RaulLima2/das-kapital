виплатити, гроші відіграють ролю лише ідеальної міри вартости, і при
цьому ще зовсім не потрібно, щоб вони були в руках капіталіста); подруге,
в процесі продукції, де робоча сила функціонує в руках капіталіста
як капітал, тобто як елемент, що утворює споживну вартість і вартість.
Вона вже дала в товаровій формі той еквівалент, що його треба виплатити
робітникові, дала еквівалент цей раніше, ніж капіталіст виплатить
його в грошовій формі робітникові. Отже, робітник сам утворює виплатний
фонд, що з нього капіталіст оплачує його. Та це ще не все.

Робітник витрачає одержувані гроші на утримання своєї робочої
сили, отже, — коли розглядати клясу капіталістів і клясу робітників у
їхній сукупності, — робітник витрачає ці гроші, щоб зберегти капіталістові
те знаряддя, що за допомогою його лише й може він лишатись
капіталістом.

Отже, постійна купівля й продаж робочої сили увічнює, з одного
боку, робочу силу як елемент капіталу; в наслідок цього капітал з’являється
як творець товарів, предметів споживання, що мають вартість;
далі, в наслідок цього ж ту частину капіталу, яка купує робочу силу, постійно
відновлюється продуктом цієї робочої сили, і значить, сам робітник постійно
утворює той фонд капіталу, що з нього йому платять. З другого
боку, постійний продаж робочої сили стає повсякчас поновлюваним джерелом
засобів існування робітника й таким чином його робоча сила
з’являється як здатність, що через неї він одержує дохід, з якого він
живе. Дохід тут значить не що інше, як зумовлюване постійно повторюваним
продажем товару (робочої сили) привласнення вартостей, при
чому самі ці вартості служать лише для постійної репродукції продаваного
товару. І остільки має А. Сміс рацію казати, що джерелом доходу
робітника стає та частина вартости утвореного самим робітником продукту, за
яку капіталіст дає йому еквівалент у формі заробітної плати. Але це так
само нічого не змінює в природі або величині цієї частини вартости товару,
як нічого не змінює у вартості засобів продукції та обставина, що
вони функціонують як капітальні вартості, або так само, як нічого не
змінюється в природі й величині прямої лінії від того, чи буде вона правити
за основу трикутника чи за діяметр еліпси. Вартість робочої сили,
як і перше, визначається також незалежно від цієї обставини, як і вартість
засобів продукції. Ця частина вартости товару ані складається з доходу,
як одного з її складових самостійних чинників, ані розкладається на
дохід. Хоч ця нова вартість, постійно репродуковувана робітником, і становить
для нього джерело доходу, однак, навпаки, його дохід не становить
складової частини продукованої ним нової вартости. Величина виплачуваної
йому частини, утворюваної ним нової вартости визначає розмір
вартости його доходу, а не навпаки. Та обставина, що ця частина нової
вартости становить для нього дохід, свідчить лише про те, що з нею робиться,
про спосіб вжитку її, але це так само не має жодного чинення до створення
її, як і до створення всякої іншої вартости. Коли я щотижня
одержую десять талерів, то самий факт цього щотижневого одержання
нічого не змінює ні в природі вартости десятьох талерів, ні в величині
