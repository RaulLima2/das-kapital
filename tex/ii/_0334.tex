\index{ii}{0334}  %% посилання на сторінку оригінального видання
Інша справа з продуктом сукупного суспільного капіталу. Всі речові
елементи репродукції мусять в своїй натуральній формі бути частинами
цього самого продукту. Зужитковану сталу частину капіталу можна замістити
за допомогою цілої продукції лише в тому разі, коли вся новоз’явлена
в продукті стала частина капіталу з’являється знову в натуральній
формі нових засобів продукції, що дійсно можуть функціонувати як сталий
капітал. Тому, — якщо ми припустимо просту репродукцію, — вартість
тієї частини продукту, яка складається з засобів продукції, мусить дорівнювати
сталій частині вартости суспільного капіталу.

Далі, коли розглядати справу з індивідуального погляду, то новодолученою
працею капіталіст продукує в вартості свого продукту лише свій змінний
капітал, плюс додаткова вартість, тимчасом як стала частина вартости
переноситься на продукт в наслідок конкретного характеру новодолученої
праці.

З суспільного погляду, та частина суспільного робочого дня, яка продукує
засоби продукції, отже, долучає до них нову вартість і переносить
на них вартість засобів продукції, зужиткованих на їх продукцію, не продукує
нічого іншого, крім нового сталого капіталу, призначеного
замістити сталий капітал, зужиткований в формі старих засобів продукції,
спожитий так в І, як і в II підрозділах. Ця частина продукує лише
продукт, що має ввійти в продуктивне споживання. Отже, ціла вартість
цього продукту є вартість, що може знову функціонувати лише як сталий
капітал, що на неї можна знову купити тільки сталий капітал у його
натуральній формі і що, отже, коли розглядати справу з суспільного погляду,
не розкладається ні на змінний капітал, ні на додаткову вартість.
— З другого боку, та частина суспільного робочого дня, яка продукує
засоби споживання, не продукує жодної частини для заміщення суспільного
капіталу. Вона продукує тільки продукти, що в їхній натуральній
формі призначені на те, щоб реалізувати вартість змінного капіталу
та додаткову вартість в І і II.

Кажучи про суспільний погляд, отже, розглядаючи ввесь суспільний
продукт, який включає так репродукцію суспільного капіталу, як і особисте
споживання, не треба вдаватись у маніру, запозичену Прудоном
у буржуазної економії, і розглядати справу так, ніби суспільство капіталістичного
способу продукції, розглядуване en bloc, як ціле, втрачає цей
свій специфічний, історично-економічний характер. Навпаки. При суспільному
погляді доводиться мати справу з колективним капіталістом. Увесь
капітал виступає як акційний капітал усіх поодиноких капіталістів, узятих
разом. І таке акційне товариство має те спільне з багатьма іншими
акційними товариствами, що кожен знає, що він вклав, та не знає, що
він одержить назад.

\subsection{Ретроспективний погляд на А. Сміса, Шторха і Рамсая}

Сукупна вартість суспільного продукту становить 9000 = $6000 с +
1500 v + 1500 m$, інакше кажучи: 6000 репродукують вартість засобів
\parbreak{}  %% абзац продовжується на наступній сторінці
