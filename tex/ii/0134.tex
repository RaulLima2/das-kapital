охоплює більше, ніж один оборот капіталу обігового, хоч який буде
протяг цих оборотів обігового капіталу: річний, більш ніж річний або
менш ніж річний. Таким чином, у Сміса avances annuelles перетворюються
на обіговий, a avances primitives — на основний капітал. Але цим узагальненням
категорій і обмежується його крок наперед. Щодо виконання він
лишається далеко позаду Кене.

Вже той грубий емпіричний спосіб, що ним він розпочинає свій дослід,
породжує плутанину: „Є два способи застосувати капітал так, щоб
він давав своєму власникові дохід або зиск“ *.

Способи приміщувати вартість так, щоб вона функціонувала як капітал,
щоб давала своєму власникові додаткову вартість, так само різні
і так само різноманітні, як і сфери приміщення капіталу. Це є питання
про різні галузі продукції, куди можна вкласти капітал. Але питання, так
зформульоване, поширюється далі. Воно захоплює й питання про те, як
вартість, навіть, коли вона не вкладена в продуктивний капітал, може для
її власника функціонувати, напр., як процентодайний капітал, купецький
капітал тощо. Отже, тут ми безмежно віддалились від справжнього предмету
аналізи, від питання: як розподіл продуктивного капіталу на
його різні елементи впливає на його оборот, незалежно від різних сфер
його приміщення.

А. Сміс безпосередньо по цьому каже: „Насамперед його можна застосувати
в сільському господарстві, мануфактурі або на закуп благ і дальший
продаж з зиском“ **. А. Сміс каже тут лише те, що капітал можна
застосувати в сільському господарстві, мануфактурі й торговлі. Отже, він
каже лише про різні сфери приміщення капіталу, і між іншим про такі, де,
як у торговлі, капітал не ввіходить у безпосередній процес продукції, тобто
не функціонує як продуктивний капітал. Тим самим він покидає той грунт,
що на ньому стояли фізіократи, визначаючи відмінності різних частин

Nemours, „Origine et Progrès d’une science nouvelle“, 1767 (Daire, I, p. 291 ***),
далі Le Trosne пише: „В наслідок більшої або меншої довготривалости продуктів
праці, нація має чималий фонд багатств, незалежний від його щорічної
репродукції; фонд, що становить капітал, нагромаджений протягом довгого
часу, первинно оплачений продуктами, постійно поновлюваний і збільшуваний“
(„Au moyen de la durée plus ou moins grande des ouvrages demain d'œuvre, une
nation possède un fonds considérable de richesses, indépendant de sa reproduction
annuelle, qui forme un capital accumulé de longue main, et originairement payé
avec des productions, qui s'entretient et s’augmente toujours“ (Daire, I, p. 928).
Тюрґо вже систематично вживає слова капітал замість аванси і ще повніше ототожнює
аванси мануфактуристів з авансами фармерів (Turgot, „Rétlexions sur la
Formation et la Distribution des Richesses“, 1766).

* „There are two different ways in which a capital may be employed so as to
yield a revenue or profit to its employer“. (Wealth of Nations. Book II, ch. 1, p. 189.
Kdit. Aberdeen, 1848).

** „First, it may be employed in raising, manufacturing, or purchasing goods and
selling them again with a profit“.

*** Цитоване місце є не в статті „Origine et Progrès“, 1767 (Daire, I, p. 291),
a в статті „Maximes du docteur Quesnay“ (Daire, I, p. 391). Ред.
