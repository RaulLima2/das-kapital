ваного в формі засобів праці, допоміжних матеріялів і т. ін. при фабрикації
сорочок, почасти в наслідок витраченої на цю фабрикацію праці,
що долучає вартість заробітної плати робітників, які роблять сорочки,
плюс додаткова вартість фабриканта сорочок. Припустімо, що ввесь цей
продукт — сорочки коштують, кінець-кінцем, 100 ф. стерл., і що це є
та частина всієї вартости річного продукту, яку суспільство витрачає на
сорочки. Споживачі сорочок оплачують 100 ф. стерл., отже, вартість усіх
засобів продукції, що є в сорочках, а також заробітну плату плюс додаткова
вартість льонізника, прядуна, ткача, білильника, фабриканта сорочок,
а також і всіх транспортерів. Це цілком слушно. Це така справа, що й
дитина зрозуміє її. Але потім сказано: так само стоїть справа й щодо
dартости всіх інших товарів. Треба було б сказати: так само стоїть справа
й щодо вартости всіх засобів споживання, щодо вартости тієї
частини суспільного продукту, яка входить у фонд споживання, отже, з
тією частиною вартости суспільного продукту, яку можна витратити як
дохід. Сума вартости всіх цих товарів справді дорівнює вартості всіх
зужиткованих на них засобів продукції (сталих частин капіталу) плюс
вартість, утворена працею, долученою востаннє (заробітна плата плюс
додаткова вартість). Отже, сукупність споживачів може оплатити всю цю
суму вартости, бо хоч вартість кожного окремого товару складається
з c + v + m, але сума вартости всіх товарів, що входять у фонд
споживання, разом узята, в максимальній величині, може дорівнювати лише
тій частині вартости суспільного продукту, яка розкладається на v + m,
тобто може дорівнювати лише тій вартості, що її долучила витрачена
протягом року праця до вже наявних засобів продукції, до вартости сталого
капіталу. Але щодо сталої капітальної вартости, то ми бачили, що
її заміщується з маси суспільного продукту двояким способом. Поперше,
через обмін капіталістів II, які продукують засоби споживання, з капіталістами
І, які продукують засоби продукції. Тут і є джерело тієї фрази,
ніби те, що для одних є капітал, для інших є дохід. Але справа в дійсності
стоїть не так. Ті 2000 II с, що існують у засобах споживання вартістю
в 2000, становлять для кляси капіталістів II сталу капітальну вартість.
Отже, сами капіталісти II не можуть спожити цю вартість, хоч продукт
за його натуральною формою і призначено для споживання. З другого
боку, 2000 І (v + m) є спродукована клясою капіталістів і робітників І
заробітна плата плюс додаткова вартість. Вони існують у натуральній
формі засобів продукції, речей, що в них їхню власну вартість не можна
спожити. Отже, ми маємо тут суму вартости в 4000, що з них половина,
— і до обміну й після обміну — заміщує лише сталий капітал,
а друга половина становить лише дохід. Але, подруге, сталий капітал
підрозділу І заміщується in natura, почасти через обмін між капіталістами
І, почасти через заміщення in natura в кожному поодинокому підприємстві.
Фраза, ніби вся вартість річного продукту, кінець-кінцем, має бути
оплачена споживачами, була б правильна тільки тоді, коли б споживачів
мислили, як два цілком різні сорти: індивідуальних споживачів і
