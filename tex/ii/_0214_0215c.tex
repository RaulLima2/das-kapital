\parcont{}  %% абзац починається на попередній сторінці
\index{ii}{0214}  %% посилання на сторінку оригінального видання
постійно витрачається на матеріяли для продукції і 1/5 = 120 ф. стерл.
на заробітну плату. Отже, щотижня 80 ф. стерл. на матеріяли для продукції,
20 ф. стерл. на заробітну плату. Отже, капітал II = 300 ф. стерл.
так само мусить поділитись на 4/5 = 240 ф. стерл. для продукційних матеріялів
і на 1/5 = 60 ф. стерл. для заробітної плати. Капітал, витрачуваний
на заробітну плату, завжди мусить авансуватись у грошовій формі.
Скоро товаровий продукт вартістю в 600 ф. стерл. зворотно перетворюється
на грошову форму, скоро його продано, — 480 ф. стерл. з цієї
суми можна перетворити на матеріяли для продукції (на продуктивний
запас), але 120 ф. стерл. зберігають свою грошову форму, щоб служити
для виплати заробітної плати протягом 6 тижнів. Ці 120 ф. стерл. являють
той мінімум приплилого назад капіталу в 600 ф. стерл., який завжди
мусить поповнюватись і заміщуватись у формі грошового капіталу, а
тому й мусить він завжди бути наявний як діюща в грошовій формі частина
авансованого капіталу.

Коли тепер з тих 300 ф. стерл., що періодично звільняються на З
тижні й так само розпадаються на 240 ф. стерл. для продуктивного запасу
й на 60 ф. стерл. для заробітної плати, в наслідок скорочення часу
обігу виділюється 100 ф. стерл. у формі грошового капіталу, зовсім викидаємся
з механізму обороту, то постає питання: відки береться гроші
для цих 100 ф. стерл. грошового капіталу? Лише на п’яту частину вони
складаються з грошового капіталу, що періодично звільняється в межах
оборотів. Але 4/5 = 80 ф. стерл. уже заміщено додатковим продуктивним
запасом тієї самої вартости. Яким же чином цей додатковий продуктивний
запас перетворюється на гроші, і відки береться гроші на це перетворення?
Якщо постало скорочення часу обігу, то з вищезгаданих 600 ф. стерл.
на продуктивний запас замість 480 ф. стерл. перетворюється лише 400 ф.
стерл. Решту 80 ф. стерл. зберігається в їхній грошовій формі, і
разом з вищезгаданими 20 ф. стерл., призначеними для заробітної плати,
вони становлять цей виділений капітал в 100 ф. стерл. Хоч ці 100 ф.
стерл. приходять з циркуляції в наслідок продажу товарового капіталу в
600 ф. стерл. і тепер їх вилучається з циркуляції, бо їх не витрачається
знову на заробітну плату й елементи продукції, однак, не треба забувати,
що в грошовій формі вони знову є в тій самій формі, що в ній їх
первісно кинуто в циркуляцію. Спочатку на продукційний запас і на заробітну
плату витрачалось 900 ф. стерл. грішми. Щоб подати той
самий процес продукції, треба тепер вже лише 800 ф. стерл. Виділені в
наслідок цього в грошовій формі 100 ф. стерл. становлять тепер новий
грошовий капітал, що шукає приміщення, нову складову частину грошового
ринку. Щоправда, вони й раніш періодично перебували у формі
звільненого грошового капіталу й додаткового продуктивного капіталу,
але цей лятентний стан сам був умовою провадження процесу продукції,
бо він був умовою його безперервности. Тепер їх уже не треба для
цього, а тому вони становлять новий грошовий капітал і одну з складових
частин грошового ринку, хоч вони зовсім не є ні додатковий елемент
\index{ii}{0215}  %% посилання на сторінку оригінального видання
до вже наявного суспільного грошового запасу (бо вони вже були
на початку заснування підприємства, й воно пустило їх в циркуляцію),
ні новоакумульований скарб.

Тепер ці 100 ф. стерл. дійсно вилучається з циркуляції, оскільки вони
є частина авансованого грошового капіталу, що її тепер уже не застосовується
в тому самому підприємстві. Але таке вилучення можливе
тільки тому, що перетворення товарового капіталу на гроші, а цих грошей
— на продуктивний капітал, Т' — Г — Т, прискорюється на один тиждень,
отже, прискорюється й обіг діющих у цьому процесі грошей. Їх вилучено
з циркуляції, бо вони більше непотрібні для обороту капіталу X.

Тут припускається, що авансований капітал належить тому, хто його
застосовує. Коли б він був позичений, справа через це ані трохи не змінилась
би. Із скороченням часу обігу, підприємцеві треба було б замість
900 ф. стерл. лише 800 ф. стерл. позиченого капіталу. 100 ф. стерл.,
повернені позикодавцеві, становлять, як і раніше, новий грошовий капітал
в 100 ф. стерл. тільки вже не в руках X, а в руках Y. Далі, коли
капіталіст X одержував наборг свої продукційні матеріяли вартістю в
480 ф. стерл., так що сам він мав авансувати грішми тільки 120 ф. стерл.
на заробітну плату, то тепер він має брати наборг продукційних матеріялів
на 80 ф. стерл. менше, отже, ці 80 ф. стерл. становлять надлишковий
товаровий капітал для капіталіста, що дає наборг, тимчасом як
для капіталіста X виділилось би 20 ф. стерл. грішми.

Додатковий продукційний запас зменшився тепер на\footnote{
січня 300 пак, на складі лишається 900 пак
1 квітня 300 „„    600 „
}/3. Являючи 4/5
від 300 ф. стерл., він дорівнював додатковому капіталові II—240 ф.
стерл., тепер він дорівнює лише 160 ф. стерл., тобто являє додатковий
запас на 2 тижні замість 3. Тепер він відновлюється що 2 тижні замість
що 3, але також тільки на два тижні замість 3. Закупи, напр., на ринку
бавовни повторюється, таким чином, частіше й меншими пайками. З ринку
береться ту саму кількість бавовни, бо маса продукту лишається та
сама. Але закупи розподіляються інакше в часі й розтягується їх на довший
час. Припустімо, напр., що йдеться про 3 місяці й про 2 місяці; хай
річне споживання бавовни буде 1200 пак. В першому випадку продаватиметься:\footnote{
жовтня 300 „„    0 „

Навпаки, в другому випадку:
} липня 300 „„    300 „\footnote{
березня „200 „„    800 „
} січня продається 200, на    складі    1000    пак\footnote{
липня „200 „„    400 „
} травня „200 „„    600 „\footnote{
листопада „200 „„    0 „

Отже, витрачені на бавовну гроші цілком повертаються лише на місяць
пізніше, в листопаді замість жовтня. Отже, коли в наслідок скорочення
} вересня „200 „„    200 „
\parbreak{}  %% абзац продовжується на наступній сторінці
