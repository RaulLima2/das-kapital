\index{ii}{0104}  %% посилання на сторінку оригінального видання
\chapter{Оборот капіталу}

\section{Час обороту й число оборотів}

Ми бачили: сукупний час циркуляції даного\footnote*{
Термін „сукупний час циркуляції“ тут Маркс вживає в тому самому розумінні,
в якому він далі в цьому ж розділі вживає термін „час обороту“, тимчасом
як взагалі він з цій книзі термін „час циркуляції“ вживає в тому самому
розумінні, що і „час обігу“, тобто в розумінні того часу, що протягом його капітал
перебуває в сфері циркуляції. (Дивись розділ V). \emph{Ред.}
} капіталу дорівнює сумі
часу його обігу та часу його продукції. Це є відтинок часу від моменту
авансування капітальної вартости в певній формі до моменту, коли капітальна
вартість, що процесує, повертається в тій самій формі.

Мета, що визначає капіталістичну продукцію, завжди є зростання
авансованої вартости, чи авансовано цю вартість в її самостійній формі,
тобто в грошовій формі, чи в формі товару, так що його форма вартости
має лише ідеальну самостійність у ціні авансованих товарів.
В обох випадках ця капітальна вартість перебігає протягом свого кругобігу
різні форми існування. Її тотожність з самою собою констатується
в книгах капіталіста або в формі рахункових грошей.

Хоч візьмемо ми форму $Г\dots{} Г'$, хоч форму $П\dots{} П$, обидві форми
значать: 1) що авансована вартість функціонувала як капітальна вартість
і зросла своєю вартістю; 2) що по закінченні процесу вона повернулась
до тієї форми, в якій почала його. Зростання авансованої вартости Г і
разом з тим поворот капіталу до цієї форми (до грошової форми) виразно
помітно в $Г\dots{} Г'$. Але те саме відбувається і в другій формі. Бо
вихідний пункт для П є наявність елементів продукції, товарів даної
вартости. Ця форма має в собі зростання цієї вартости (Т' і $Г'$) і поворот
до первісної форми, бо в другому П авансована вартість
знову має форму елементів продукції, що в ній її первісно авансовано.

Раніше ми бачили: "Якщо продукція має капіталістичну форму, то
і репродукція має ту саму форму. Як процес праці за капіталістичного
способу продукції є лише засіб для процесу зростання вартости, так
\parbreak{}  %% абзац продовжується на наступній сторінці
