\parcont{}  %% абзац починається на попередній сторінці
\index{ii}{0117}  %% посилання на сторінку оригінального видання
роду, коли старі цілком зужитковано на вироблення готового продукту. Так само в процесі продукції
завжди є і робоча сила, але тільки в наслідок постійного поновлювання її закупу, при чому часто
змінюються й особи. Навпаки, підчас повторюваних оборотів поточного капіталу в тих самих
повторюваних процесах продукції й далі функціонують ті самі будівлі, машини тощо.

\subsection{Складові частини, заміщення, ремонт, акумуляція основного капіталу}
В тих самих капіталовкладеннях життьова тривалість поодиноких елементів основного капіталу різна, а
тому різний і час їх обороту. Напр., на залізниці, час функціонування й час репродукції рейок,
злежнів, земляних споруд, станційних будинків, мостів, тунелів, льокомотивів і вагонів різний, а
тому й різний час обороту авансованого на них капіталу. Протягом багатьох років будівлі, плятформи,
водоймища, віядуки, тунелі,
земляні виїмки й насипи, — коротше кажучи, все те, що в англійському залізничному господарстві
зветься works of art, не потребує жодного поновлення. Речі, що найбільше зношуються — це рейковий
шлях і рухома частина (rolling stock).

Первісно, коли будували сучасні залізниці, панував той погляд, — його поширювали видатні
інженери-практики, — ніби залізниця своєю тривалістю вічна, а зношування рейок таке непомітне, що
його можна й зовсім не брати на увагу в усіх фінансових і практичних розрахунках; життьову
тривалість добрих рейок тоді обчислювалось в 100--150 років. Але незабаром виявилось, що життьова
тривалість рейки, звичайно залежна від швидкости паротягів, ваги та числа потягів, грубини самих
рейок та багатьох інших бічних обставин, пересічно не перевищує 20 років.
На деяких станціях, центрах великого обороту, рейки зношуються навіть щорічно. Щось близько 1867
року почали вводити сталеві рейки, які коштували майже вдвоє дорожче, ніж залізні, але зате й
тривають більш як удвоє. Життьова тривалість дерев’яних злежнів становила 12--15 років. Щодо рухомої
частини, то товарові вагони зношуються куди швидше, ніж пасажирські. Життьову тривалість паротягів
1867 року обчислювалось в 10--12 років.

Зношування постає, поперше, в наслідок самого уживання. Взагалі рейки зношуються пропорційно до
числа потягів (R. C. № 17655)\footnote{
Цитати, позначені R. С., взято з Royal Commission on Railways. Minutes of Evidence taken before
the Commissioners. Presented to both Houses of Parliament. London, 1867. — Запитання й відповіді
перенумеровано й нумери тут зазначено.
}. Коли швидкість збільшувалась, то зношування зростало більш ніж
відповідно до квадратів швидкости, тобто, коли швидкість потягів збільшувалась удвоє, то зношування
зростало більше ніж у чотири рази (R. С. № 17046).

Далі, зношування постає в наслідок впливу природних сил. Приміром, злежні псуються не лише в
наслідок дійсного зношування, але й через гниття. „Витрати на утримання залізниці залежать не
стільки від
\parbreak{}  %% абзац продовжується на наступній сторінці
