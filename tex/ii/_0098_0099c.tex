\parcont{}  %% абзац починається на попередній сторінці
\index{ii}{0098}  %% посилання на сторінку оригінального видання
ці витрати. Ці витрати завжди становлять частину суспільної праці в зреневленій
або живій формі — отже, в капіталістичній формі вони є затрати
капіталу, — що не беруть участи в утворенні продукту, отже, вони є одбави
з продукту. Вони є доконечні, це — затрати (Unkosten) суспільного
багатства. Це — витрати на зберігання суспільного продукту, все одно,
чи походить існування його як елементу товарового запасу лише
із суспільної форми продукції, отже, з товарової форми та її доконечного
перетворення форми, чи ми розглядаємо товаровий запас лише як спеціальну
форму запасу продуктів, спільного всім суспільствам, хоч би
такий запас не мав форми товарового запасу, цієї форми запасу продуктів,
належної до процесу циркуляції.

Тепер запитаймо, якою мірою ці витрати входять у вартість товарів.

Коли капіталіст свій капітал, авансований на засоби продукції й робочу
силу, перетворив на продукт, на готову для продажу масу товарів, і вона
лишається непродана на складах, то на цей час не лише припиняється
процес зростання вартости його капіталу. Видатки, що їх потребує зберігання
цього запасу в приміщеннях, видатки на новододавану працю тощо, становлять
позитивну втрату. Покупець, що, кінець-кінцем, прийшов би, висміяв
би його, коли б капіталіст сказав: мій товар не купувався протягом шістьох
місяців, і коли зберігалось його протягом цих шістьох місяців, то не
тільки лежало марно стільки й стільки капіталу, але це спричинило мені,
крім того, х затрат (Unkosten). Tant pis pour vous\footnote*{
То гірше для вас. \emph{Ред.}
}, скаже покупець. Бо
поряд вас є інший продавець, що його товар вироблено лише позавчора.
Ваш товар є заваль і, мабуть, більш або менш попсувався від часу. Отже,
ви мусите продавати дешевше, ніж ваш суперник. — Умови існування
товару зовсім не змінюються від того, чи є товаропродуцент справжній
продуцент свого товару, чи капіталістичний продуцент, тобто в суті
лише представник справжніх продуцентів. Йому треба перетворити свою
річ на гроші. Затрати (Unkosten), що їх спричиняє фіксування її
в товаровій формі, належать до його особистих справ, і до них його
покупцеві байдуже. Він не оплачує йому час циркуляції його товарів.
Навіть коли капіталіст навмисно тримає свій товар поза ринком, підчас
справжньої або передбачуваної революції у вартості, то й тоді від
справжнього постання цієї революції у вартості, від правильности чи
неправильности його спекуляції залежить, чи реалізує він свої додаткові
затрати (Unkosten). Але революція у вартості не є наслідок його затрат.
Отже, оскільки утворення запасу являє собою затримку циркуляції, спричинені
цим витрати не додають до товару жодної вартости. З другого
боку, жоден запас не може існувати без перебування в сфері циркуляції,
без довшого або коротшого перебування капіталу в його товаровій
формі; отже, жоден запас не буває без затримки циркуляції, так само,
як не можлива грошова циркуляція без утворення грошового резерву.
Отже, без товарового запасу не може бути жодної товарової циркуляції.
\index{ii}{0099}  %% посилання на сторінку оригінального видання
Коли конечність цього для капіталіста постає не в $Т' — Г'$, то постає
вона для нього в $Г — Т$; якщо не для його товарового капіталу, то
для товарового капіталу інших капіталістів, що продукують засоби продукції
для нього й засоби існування для його робітників.

Здавалось би, суть справи ніяк не може змінитись від того, чи утворюється
запас добровільно, чи недобровільно, тобто, чи навмисно товаропродуцент
тримає запас, чи його товари утворюють запас у наслідок того опору, що його
обставини самого процесу циркуляції протиставлять продажеві товарів. Усе
ж, щоб розв’язати це питання, корисно знати, чим відрізняється добровільне
утворення запасу від недобровільного. Недобровільне утворення запасу
випливає з затримки в циркуляції або є тотожне із затримкою в циркуляції,
що є незалежна від передбачення товаропродуцента і перешкоджає його
волі. Що характеризує добровільне утворення запасу? Завжди продавець
хоче яко мога швидше збути свій товар. Він завжди подає свій продукт як
товар. Коли він утримується від продажу, то продукт утворює лише можливий
(δυναμει), а не справжній (ενεργεια) елемент товарового запасу.
Товар як такий, як і раніше, є для нього лише носій мінової вартости,
і як мінова вартість він може діяти, лише скинувши з себе товарову
форму й набравши грошової форми.

Товаровий запас мусить доходити певних розмірів, щоб протягом
даного періоду задовольняти розміри попиту. При цьому треба зважувати
й на постійне збільшення кола покупців. Напр., щоб вистачити на один день,
частина товарів, яка є на ринку, мусить завжди затримуватись у товаровій
формі, тимчасом як друга тече, перетворюється на гроші. Та частина,
яка затримується, поки друга тече, правда, постійно меншає, так само,
як зменшуються розміри самого запасу, аж поки його ввесь продадуть.
Отже, застій товару тут зважено, як доконечну умову продажу
товару. Далі, розміри його мусять бути більші, ніж середні розміри
продажу або середні розміри його попиту. Бо інакше надлишок над
середнім розміром попиту не можна було б задовольнити. З другого боку,
запас має постійно поновлюватись, бо він постійно витрачається. Це
відновлення може, кінець-кінцем, походити лише з продукції, з подання
товару. Чи його довозиться з-за кордону, чи ні, це не змінює суті
справи. Відновлення залежить від протягу часу, потрібного на репродукцію
товарів. На ввесь цей час має вистачати товарового запасу. Що
він не лишається в руках первісних продуцентів, а переходить через різні
сховища, починаючи від гуртового торговця й до роздрібного торговця
— це змінює лише зовнішність, а не суть справи. З погляду суспільства,
як і раніш, частина капіталу лишається в формі товарового запасу доти,
доки товар увійде в сферу продуктивного або особистого споживання. Сам
продуцент намагається мати на складі товаровий запас відповідно до його
пересічного попиту, щоб не залежати безпосередньо від продукції й
забезпечити собі постійне коло покупців. Відповідно до періодів продукції
встановлюються строки купівель, і протягом довшого або коротшого
часу товар становить запас, поки його заступлять нові екземпляри того
таки ґатунку. Тільки таким утворенням запасу забезпечується сталість
\parbreak{}  %% абзац продовжується на наступній сторінці
