з В, В', В" і т. д. (І) справа стоїть інакше. 1) Тільки в їхніх руках
додатковий продукт А, А', А" і т. д. буде функціонувати активно як
додатковий сталий капітал (другий елемент продуктивного капіталу, додаткову
робочу силу, отже, додатковий змінний капітал, ми. покищо
лишаємо осторонь); 2) для того, щоб додатковий продукт потрапив до
їхніх рук, потрібен акт циркуляції, вони повинні купити цей додатковий
продукт.

До пункту (І) тут треба зауважити, що чимала частина додаткового
продукту (віртуально додаткового сталого капіталу), продукованого
А, А', А" (І), хоч її спродуковано поточного року, може активно функціонувати
як промисловий капітал в руках В, В', В" (І) тільки наступного
року або навіть пізніше; до пункту 2) постає питання, відки беруться
гроші, потрібні для процесу циркуляції?

Оскільки продукти, що їх продукують В, В', В" і т. д. (І), сами
знову входять in natura в їхній процес продукції, то само собою зрозуміло,
що pro tanto частина їхнього власного додаткового продукту безпосередньо
(без посередництва циркуляції) переміщується в їхній продуктивний
капітал і входить в нього як додатковий елемент сталого капіталу.
Але pro tanto вони й не перетворюють на золото додаткового
продукту А, А' і т. д. (І). Лишаючи це осторонь, відки ж беруться
гроші? Ми знаємо, що В, В', В" і т. д. (І) утворили свій скарб, як і
А, А' і т. д., через продаж їхніх відповідних додаткових продуктів і
тепер досягли тієї мети, коли їхній нагромаджений як скарб, лише віртуально
грошовий капітал, повинен тепер справді функціонувати як
додатковий грошовий капітал. Але так ми лише блукаємо в зачарованому
колі. Все ж лишається питання, відки беруться гроші, що їх раніше
вилучили з циркуляції та нагромадили капіталісти В, В', В" і т. д. (І)?

Однак уже з досліду простої репродукції ми знаємо, що в руках
капіталістів І і II мусить бути певна маса грошей для того, щоб перетворювати
їхній додатковий продукт. Гроші, що служили лише для витрат
як дохід на засоби споживання, повертались там назад до капіталістів
в міру того, як вони авансували їх для обміну своїх власних товарів;
тут знову з’являються ті самі гроші, але функція їхня змінилась.
Капіталісти А, А' і т. д. І В, В' і т. д. (І) навперемінку подають гроші
для перетворення додаткового продукту на додатковий віртуальний грошовий
капітал, і навперемінку знову подають у циркуляцію новоутворений
грошовий капітал як купівельний засіб.

Єдине припущення при цьому те, що наявної в країні маси грошей
(швидкість обігу та ін. припускається однакові) досить так для активної
циркуляції, як і для утворення запасного скарбу; отже — те саме припущення,
що, як ми бачили, мусить бути здійснене і при простій товаровій
циркуляції. Тільки функція скарбу тут інша. Крім того, маса наявних
грошей мусить бути більша: 1) бо при капіталістичній продукції
кожен продукт (за винятком новоспродукованих благородних металів та
небагатьох продуктів, що їх споживає сам продуцент) продукується як
товар, отже, мусить проробити грошове залялькування; 2) бо на капіта-
