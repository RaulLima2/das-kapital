капіталу, потрібного для акумуляції в межах II, цілком так само, як II
повинен дати матеріял для змінного капіталу, що має пустити в рух ту
частину додаткового продукту І, яку сам І застосовує як додатковий
сталий капітал. Ми знаємо, що справжній змінний капітал, а значить, і
додатковий, складається з робочої сили. Тепер не капіталіст І купує в II
доконечні засоби існування про запас і зберігає їх для додаткової робочої
сили, що йому її треба буде застосувати в майбутньому, як це
мусив робити рабовласник. Сами робітники мають справу з II. Та це не
заваджає тому, що з погляду капіталіста засоби споживання додаткової
робочої сили є лише засоби продукції та зберігання його евентуальної
додаткової робочої сили, отже, натуральна форма його змінного капіталу.
Його власна найближча операція, в даному разі виконувана І, сходить
лише на те, що він нагромаджує новий грошовий капітал, потрібний для
закупу додаткової робочої сили. Скоро тільки він долучає її до свого
капіталу, гроші стають для цієї робочої сили засобом до купівлі
товарів II, отже, вона мусить знайти засоби свого споживання.

Між іншим. Пан капіталіст, як і його преса, часто бувають незадоволені
з того способу, яким робоча сила витрачає свої гроші, і з
тих товарів II, що в них вона реалізує ці гроші. Він філософує з цього
приводу, розводить теревені про культуру та удає філантропа, як це,
прим., робить п. Друммонд, секретар англійського посольства в Вашінґтоні.
„The Nation“ (газета) наприкінці жовтня 1879 р. вмістила цікаву
статтю, де, між іншим, сказано: „Робітники відстали в культурі від поступу
в винаходах; для них стала приступна маса речей, що їх вживати
вони не вміють, і що для них вони, отже, не являють ринку“.
(Кожен капіталіст, звичайно, хоче, щоб робітник купував його товар).
„Немає жодної підстави для того, щоб робітник не хотів жити з таким
самим комфортом, як піп, адвокат або лікар, що одержує стільки ж, як
і він“. (Справді багатенько комфорту можуть дозволити собі з свого бажання
такі адвокати, попи й лікарі!). „Але він цього не робить. Питання
все ще в тому, якими раціональними здоровими заходами підвищити його
рівень як споживача; це питання не легке, бо все його шанолюбство не
йде далі скорочення робочих годин, і демагоги радше підбурюють його
до цього, ніж до поліпшення його стану через удосконалення його розумових
і моральних здібностей“. (Reports of Н. M-s Secretaries of Embassy
and Legation on the Manufactures, Commerce etc. of the Countries in
which they reside. London 1879, p. 404).

Довгий робочий день являє, певно, таємницю раціональних і здорових
заходів, що повинні поліпшити стан робітника, удосконалюючи його розумові
й моральні здібності, та зробити з нього раціонального споживача.
Щоб стати раціональним споживачем товарів капіталістів він мусить
насамперед почати — але цьому заваджає демагог! — з того, щоб дозволити
своєму власному капіталістові споживати його робочу силу
нераціональним і шкідливим для здоров’я способом. Як розуміє
капіталіст раціональне споживання, це виявляється там, де ласка капіталіста
доходить того, що він безпосередньо береться до торговлі
