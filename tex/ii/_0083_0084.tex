\parcont{}  %% абзац починається на попередній сторінці
\index{ii}{0083}  %% посилання на сторінку оригінального видання
самозростання вартости, самозростання, що надходить до нього із сфери
циркуляції. Далі ми побачимо, як навіть наукова економія дозволяє собі
через цю позірність допуститись помилки. Як ми доведемо далі, ця позірність
зміцнюється різними явищами: 1) капіталістичним способом обчислювати
зиск, при якому неґативна причина фігурує як позитивна, бо для капіталів
у різних сферах приміщення, де лише час обігу різний, довший час
обігу діє як причина підвищення цін, — коротко кажучи, як одна з причин
вирівнювання зиску. 2) Час обігу становить тільки один момент часу обороту;
а цей останній має в собі час продукції, зглядно час репродукції.
Те, що завдячує останньому, здається, ніби завдячує воно часові обігу.
3) Перетворення товарів на змінний капітал (заробітну плату) зумовлено
попереднім перетворенням їх на гроші. Отже, при акумуляції капіталу
перетворення на додатковий змінний капітал відбувається в сфері циркуляції
перед або протягом часу обігу. Тому й здається, що акумуляція,
яка відбувається разом із цим, завдячує обігові.

У сфері циркуляції капітал перебігає в тій або іншій послідовності
дві протилежні фази $Т — Г$ і $Г — Т$. Його час обігу розпадається, отже,
на дві частини, — час, що його він потребує, щоб перетворитися з товару
на гроші, і час, що його він потребує, щоб перетворитися з грошей на
товар. Ми вже знаємо з аналізи простої товарової циркуляції (кн. І,
розділ III), що $Т — Г$, продаж, є найважча частина його метаморфози,
і тому в звичайних обставинах він становить більшу частину часу
обігу. Вартість у формі грошей перебуває в такій формі, що її завжди
можна перетворити на іншу. Але у формі товару вона лише по перетворенні
на гроші мусить набрати форми безпосередньої вимінности,
а тому постійної готовости до діяльности. Однак у процесі циркуляції
капіталу на його стадії $Г — Т$ справа йде про перетворення його на
товари, що становлять певні елементи продуктивного капіталу в даному
підприємстві. Може статися, що засобів продукції немає ще на ринку, і
що треба їх спочатку ще лише випродукувати або довезти з віддалених
ринків, або можуть статися порушення в їхньому звичайному поданні,
зміни цін тощо, коротко кажучи, може статися ряд обставин, що
їх не видно в простій зміні форми $Г — Т$, але які також для цієї частини
фази циркуляції потребують більшого або меншого часу. Так само як
$Т — Г$ і $Г — Т$ відділені один від одного в часі, так само вони можуть
бути відділені і в просторі: ринок купівлі і ринок продажу можуть бути
просторово різними ринками. Напр., на фабриках закупник і продавець
часто є дві різні особи. За товарової продукції циркуляція так само
потрібна, як і сама продукція, отже, аґенти циркуляції так само потрібні,
як і аґенти продукції. Процес репродукції містить у собі обидві функції
капіталу, а тому й доконечність представництва цих функцій, чи в особі
самого капіталіста, чи в особі найманого робітника, аґента капіталіста.
Однак це зовсім не може бути підставою для того, щоб сплутувати
аґентів циркуляції з аґентами продукції, так само, як не може воно
бути підставою для того, щоб сплутувати функції товарового капіталу
й грошового капіталу з функціями продуктивного капіталу. Аґентів циркуляції
\index{ii}{0084}  %% посилання на сторінку оригінального видання
мусять оплачувати аґенти продукції. Але коли капіталісти, купуючи
й продаючи один одному, не утворюють цим актом ні продуктів, ні
вартостей, то це зовсім не змінюється від того, що розміри їхнього
підприємства дають їм змогу і примушують їх перекладати ці функції на
інших. У деяких підприємствах закупників і продавців оплачується
тантьємами із зиску. Коли кажуть, ніби їх оплачують споживачі, то це
нічого не пояснює. Споживачі можуть оплачувати лише остільки, оскільки
вони сами, як аґенти продукції, продукують еквівалент у товарах або
привлащують його собі від аґентів продукції, хоч на основі правного титула
(як члени товариства тощо), хоч на основі особистих послуг.

Між $Т — Г$ і $Г — Т$ є ріжниця, що не має ніякого чинення до ріжниці
форми товару і грошей, а випливає з капіталістичного характеру
продукції. Сами собою $Т — Г т$ак само, як і $Г — Т$, є прості перетворення
даної вартости з однієї форми на іншу. Але $Т' — Г'$, є разом з тим
реалізація додаткової вартости, що міститься в $Т'$. Інакше справа стоїть з
$Г — Т. Т$ому продаж важливіший за купівлю. $Г — Т$ в нормальних умовах
є акт, потрібний для зростання вартости, вираженої в Г, але він не є
реалізація додаткової вартости; це вступ до її продукції, а не доповнення
до неї.

Для циркуляції товарового капіталу $Т' — Г' п$евні межі кладуться
формою існування самих товарів, їхнім буттям як споживних вартостей.
Але останні з своєї природи минущі. Тому, коли протягом певного часу
вони не ввійдуть у продуктивне або особисте споживання, залежно від
їхнього призначення, коли, інакше кажучи, їх не продасться протягом
певного часу, то вони псуються й разом зі своєю споживною вартістю
втрачають властивість бути носіями мінової вартости. Вміщена в них
капітальна взртість, зглядно і приросла до неї додаткова вартість, втрачається.
Споживні вартості лишаються носіями капітальної вартости, яка
зберігається протягом років і зростає в своїй вартості лише остільки,
оскільки вони постійно відновлюються та репродукуються, заміщуючись
на нові споживні вартості того самого або іншого ґатунку. Але продаж
їх у формі готових товарів, отже, перехід їхній за допомогою продажу
в сферу продуктивного або особистого споживання, є завжди поновлювана
умова їхньої репродукції. Вони мусять протягом певного часу
перемінити свою стару споживну форму, щоб далі існувати в новій.
Мінова вартість зберігається лише в наслідок цього постійного поновлення
її тіла. Споживні вартості різних товарів псуються швидше
або повільніше; тому між їхньою продукцією та споживанням може
минути довший або коротший час; отже, вони можуть, не знищуючись
більш або менш довгий час, лишатись у фазі циркуляції. $Т — Г$
як товаровий капітал, можуть витримати як товари більш або менш
довгий час обігу. Межі часу обігу товарового капіталу, зумовлені псуванням
самого товарового тіла, є абсолютні межі цієї частини часу обігу,
або того часу обігу, що протягом його товаровий капітал може існувати
як товаровий капітал. Що нетриваліший товар, що швидше треба
спожити його безпосередньо по продукції, а, значить, і продати, то на
\parbreak{}  %% абзац продовжується на наступній сторінці
