Подаємо лише перші три обороти.

Таблиця IV.

Капітал І.

Періоди обороту Робочі періоди    Періоди обігу

І. Тижні 1—9 Тижні 1—4    Тижні 5—9
ІІ. „9—17 „9. 10—12 „13—17
ІІІ. „17—25 „17. 18—20 „21—25

Капітал II.

Періоди обороту Робочі періоди    Періоди обігу

I. Тижні 5—13  Тижні 5—8  Тижні 9—13
II. „13—21 „13. 14—16 „17—21
III. „21—29 „21. 22—24 „25—29

Капітал III.

Періоди обороту Робочі періоди    Періоди обігу

І. Тижні 9—17   Тижні 9 Тижні 10—17
II. „17—25 „17 „18—25
III. „25—33 „25 „26—33

Переплітання капіталів тут є остільки, оскільки робочий період капіталу
III, що не має самостійного робочого періоду, бо його вистачає
тільки на один тиждень, збігається з першим робочим тижнем капіталу І.
Але зате наприкінці робочого періоду і капіталу І й капіталу II
звільняється рівна капіталові III сума в 100 ф. стерл. А саме, коли капітал
III виповнює перший тиждень другого та всіх наступних робочих періодів
капіталу І, а наприкінці цього першого тижня назад припливає ввесь
капітал І,400 ф. стерл., то для решти робочого періоду капіталу І лишається
тільки 3 тижні, і відповідна витрата капіталу буде 300 ф. стерл. Звільнених
таким чином 100 ф. стерл. буде досить для першого тижня безпосередньо
наступного робочого періоду капіталу II; наприкінці цього тижня повертається
назад увесь капітал II в 400 ф. стерл.; а що розпочатий робочий
період може ввібрати ще тільки 300 ф. стерл., то наприкінці його лишають-
