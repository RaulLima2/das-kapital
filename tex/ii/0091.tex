вана в такій формі, що в ній вона править лише за машину для циркуляції.
Крім того, що частина суспільного багатства закріплюється в цій
непродуктивній формі, зношення грошей потребує постійного заміщування
їх або перетворення дедалі більшої кіль кости суспільної праці — в
формі продукту — на дедалі більше золота й срібла. Ці витрати заміщування
доходять чималих розмірів у капіталістично розвинених націй, бо
в них взагалі чималу частину багатства закріплено в формі грошей.
Золото й срібло, як грошові товари, становлять для суспільства витрати
циркуляції, що походять лише із суспільної форми продукції. Це — faux frais
товарової продукції взагалі, які зростають з розвитком товарової продукції,
а особливо капіталістичної продукції. Це — частина суспільного
багатства, що її доводиться жертвувати процесові циркуляції13).

II. Витрати зберігання

Витрати циркуляції, що випливають з простої зміни форми вартосте,
з циркуляції, розглядуваної ідеально, не входять у вартість товарів.
Витрачувані на них частини капіталу, оскільки ми маємо на увазі капіталіста,
становлять лише одбави з продуктивно витрачуваного капіталу.
Іншої природи є ті витрати циркуляції, що ми їх зараз розглядаємо. Вони
можуть, походити з процесів продукції, що лише тривають далі в циркуляції,
і що їхній продуктивний характер, отже, лише замасковується формою
циркуляції. З другого боку, із суспільного погляду, вони можуть бути
просто витратами, непродуктивними витратами хоч живої, хоч зречевленої
праці, але саме в наслідок цього вони можуть діяти так, що утворюють
вартість для індивідуального капіталіста, становлять надвишку до продажної
ціни його товару. Це випливає вже з того, що ці витрати різні в різних галузях
продукції, а місцями різні й для різних індивідуальних капіталів у межах
тієї самої галузі продукції. В наслідок того, що їх додається до ціни
товару, вони розподіляються відповідно до того, в яких розмірах припадають
вони індивідуальним капіталістам. Але кожна праця, що додає
вартість, може додавати також і додаткову вартість і на капіталістичній
основі завжди буде додавати додаткову вартість, бо утворювана цією працею
вартість залежить від її власної величини, а утворювана нею додаткова
вартість залежить від того розміру, в якому оплачує її капі-

13) „Гроші, що циркулюють у якійсь країні, є певна частина капіталу
країни, цілком вилучена з продуктивних призначень для того, щоб полегшити
або збільшити продуктивність решти. Отже, певна сума багатства доконечна
для того, щоб золото могло правити за засіб циркуляції, так само, як доконечна
вона для того, щоб виготовити машину, що полегшує якусь іншу
продукцію“.

(The money circulating in a country is a certain portion of the capital of the
country, absolutely withdrawn from productive purposes, in order to facilitate
or increase the productiveness of the remainder; a certain amount of wealth
is. therefore, as necessary, in order to adopt gold as a circulating medium,
as it is to make a machine, in order to facilitate any other production (Economist,
vol. V, стор. 519).
