капіталом (400 v) і частиною додаткової вартости (100 m) в підвідділі а
і з змінним капіталом (100 v) у підвідділі b. В дальшому ми припускаємо,
що пересічна пропорція між витратами доходу в капіталістів обох кляс
є така: 2/5 на речі розкошів і 3/5 на доконечні засоби існування.
Тому крім 100, уже витрачених на речі розкошів, усій підклясі а припадає
ще 60 на речі розкошів і в такій самій пропорції, тобто 40 припадає
підклясі b.

Отже, (II а) m розподіляється так: 240 на засоби існування і 160 на
речі розкошів = 240 + 160 = 400 m (II а).

(II b) m розподіляється так: 60 на засоби існування й 40 на речі
розкошів: 60 + 40 = 100 m (II b). Останні 40 ця кляса бере для споживання
з свого власного продукту (2/5 своєї додаткової вартости); 60 в
засобах існування вона одержує, обмінюючи 60 свого додаткового продукту
на 60 m (а).

Отже, для цілої кляси капіталістів II ми маємо (при цьому v + m в
підвідділі а існують у доконечних засобах існування, в підвідділі b — в
речах розкошів):

II а (400 v + 400 m) + II b (100 v + 100 m) = 1.000; через рух усе
це реалізується так: 500 v (а + b) [реалізуються в 400 v (а) і 100 m + (а)] +
500m (а + b) [реалізуються в 300 m (а) + 100 v (b) + 100 m (b)] =
1000.

Для а і b, розглядуваних окремо, реалізація відбувається таким
чином:

а) v / 400 v (a) + m 240 m (a) + 100 v (b) + 60 m (b) ... = 800

в) v / 100 m (a) + m 60 m (a) + 40 m (b) ... = 200 / 1000

Коли ми спрощення ради додержуватимемось для обох підвідділів однакового
відношення між змінним і сталим капіталом (що, до речі, зовсім не неодмінно),
то на 400 v (а) припаде сталий капітал = 1600, а на 100 (b)
сталий капітал = 400, і для II будуть такі два підрозділи а і b:

II а) 1600 c + 400 v + 400 m = 2400

II b) 400 c + 100 v + 100 m = 600

a разом

2000 c + 500 v + 500 m = 3000.

Відповідно до цього з 2000 II с в засобах споживання, які обмінюють
на 2000 І (v + m), 1600 обмінюються на засоби продукції доконечних
засобів існування і 400 — на засоби продукції речей розкошів.

Отже, ці 2000 І (v + m) і собі поділяться на (800 v + 800 m) І,
призначених для а = 1600 засобів продукції доконечних засобів існу-
