Шторх, який теж у принципі приймає вчення А. Сміса, вважає однак,
що застосовання цього вчення в Сея не витримує критики. „Коли
допустити, що дохід нації дорівнює її гуртовому продуктові, тобто що з
нього не треба робити якогобудь відрахування капіталу“ (це має значити
сталого капіталу), „то доведеться також допустити, що ця нація може
непродуктивно спожити всю вартість її річного продукту, не зменшивши
ні на крихту свого майбутнього доходу... Продукти, що становлять“
(сталий) „капітал нації, не можуть споживатись“ (Storck:. Considérations
sur la nature du revenu national, Paris, 1824“, p. 150).

Але Шторх забув сказати, як погодити існування цієї сталої частини
капіталу з аналізою цін, що її він узяв у Сміса, аналізою, що згідно з
нею товарова вартість містить у собі лише заробітну плату й додаткову
вартість, але не містить жодної частини сталого капіталу. Лише завдяки
Сеєві йому стає ясно, що ця аналіза ціни призводить до абсурдних
результатів, і його власне кінцеве слово про це звучить так: „неможливо
розкласти доконечну ціну на її найпростіші елементи“. („Cours d’Economie
Politique“, Petersbourg, 1815. II, p. 140).

Сісмонді, що особливо досліджував відношення між капіталом і доходом
і своє особливе розуміння цього відношення в дійсності перетворив
на differentia specifica своїх „Nouveaux Principes“, не сказав ж о д н о г о
наукового слова, не додав ж о д н о г о атома для висвітлення проблеми.

Бартон, Рамсай і Шербюльє роблять спроби піднестись понад Смісове
розуміння. Але це їм не вдається, бо вони з самого початку ставлять
проблему однобічно, не відмежовуючи виразно ріжниці між сталою
та змінною капітальною вартістю від ріжниці між основним капіталом та
капіталом обіговим.

Також і Джон Стюарт Мілл із звичайною повагою відтворює доктрину,
що перейшла в спадщину від А. Сміса до його наслідувачів.

Результат: Смісова плутанина понять існує й далі до нашого часу, і
догма Смісова є ортодоксальний символ віри політичної економії.

Розділ двадцятий

Проста репродукція

І. Постава питання

Коли ми розглянемо\footnote{
З рукопису II.
} річне функціонування суспільного капіталу щодо
його результату, — отже, функціонування сукупного капіталу, що в ньому
індивідуальні капітали становлять лише частини, рух яких є так їхній
індивідуальний рух, як і разом з тим складова ланка руху цілого капіталу,
— тобто, коли ми розглянемо товаровий продукт, що його дає суспільство
протягом року, то мусить виявитись, як відбувається процес
репродукції суспільного капіталу, які риси відрізняють цей процес репро