\parcont{}  %% абзац починається на попередній сторінці
\index{ii}{0220}  %% посилання на сторінку оригінального видання
25000 / 5 = 5000 ф. стерл. Коли поділити ці 5000 ф. стерл. на 500, то матимемо число оборотів 10,
цілком таке саме, як і для цілого капіталу в 2500 ф. стерл.

Це пересічне обчислення, що за ним вартість річного продукту ділиться на вартість авансованого
капіталу, а не на вартість частини цього капіталу, постійно застосовуваної в одному робочому періоді
(отже, в нашому прикладі, не на 400, а на 500, не на капітал І, а на капітал І + капітал II), — це
пересічне обчислення тут, де йдеться лише про продукцію додаткової вартости, є абсолютно точне. Далі
ми побачимо, що, з іншого погляду, воно не зовсім точне, як і взагалі це пересічне обчислення не
зовсім точне. Інакше кажучи, воно задовільне для практичних цілей капіталіста,
але воно не виражає точно й гаразд усіх реальних обставин обороту.

Досі ми одну частину вартости товарового капіталу лишали цілком осторонь, а саме вміщену в ньому
додаткову вартість, спродуковану та долучену до продукту протягом процесу продукції. На неї тепер і
треба нам звернути увагу.

Коли припустити, що витрачуваний щотижня змінний капітал в 100 ф. стерл., продукує додаткову
вартість в 100\% = 100 ф. стерл., то змінний капітал в 500 ф. стерл.,  витрачуваний протягом
п’ятитижневого періоду обороту, випродукує додаткову вартість в 500 ф. стерл., тобто половина
робочого дня складається з додаткової праці.

Але коли 500 ф. стерл. змінного капіталу продукують 500 ф. стерл. додаткової вартости, то 5000 ф.
стерл. випродукують її 500 х 10 = 5000 ф. стерл. Але авансований змінний капітал = 500 ф. стерл.
Відношення всієї маси додаткової вартости, спродукованої протягом року, до суми вартости
авансованого змінного капіталу ми звемо річною нормою додаткової вартости. Отже, в даному випадку,
вона = 5000 / 500 = 1000\%.

Коли ближче аналізувати цю норму, то виявиться, що вона дорівнює тій нормі додаткової вартости, яку
авансований змінний капітал продукує протягом одного періоду обороту, помноженій на число оборотів
змінного капіталу (а воно збігається з числом оборотів цілого обігового капіталу).

Авансований протягом одного періоду обороту змінний капітал в даному випадку = 500 ф. стерл.;
створена ним додаткова вартість теж = 500 ф. стерл. Тому норма додаткової вартости протягом одного
періоду обороту = $500 m / 500 v$ = 100\%. Ці 100\%, помножені на 10, на число оборотів протягом року,
дають $5000 m / 5000 v$ = 1000\%.

Це має силу щодо річної норми додаткової вартости. Щождо маси додаткової вартости, здобуваної
протягом певного періоду обороту, то ця маса дорівнює вартості авансованого протягом цього періоду
змінного капіталу — в даному випадку = 500 ф. стерл., помноженій на норму
\parbreak{}  %% абзац продовжується на наступній сторінці
