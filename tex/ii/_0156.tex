\parcont{}  %% абзац починається на попередній сторінці
\index{ii}{0156}  %% посилання на сторінку оригінального видання
відновлення витраченого капіталу, або з другого погляду — до періоду,
що на нього авансовано капітал. Коли ми замість виявити внутрішній
рух капіталістичного процесу продукції, будемо розглядати його з погляду
вже посталих явищ, то ці ріжниці, справді, будуть рівнозначні. При
розподілі суспільної додаткової вартости між капіталами, вкладеними
в різні галузі продукції, ріжниці в періодах, що на них авансується
капітал (отже, напр. різна життьова тривалість основного капіталу) і
ріжниці в органічному складі капіталу (а, значить, і ріжниця в циркуляції
сталого й змінного капіталу) однаково впливають при вирівнюванні загальної
норми зиску й перетворенні вартостей на ціни продукції.

Подруге. З погляду процесу циркуляції на одному боці є засоби
праці: основний капітал, на другому боці — матеріял праці й заробітна
плата: поточний капітал. Навпаки, з погляду процесу праці та процесу
зростання вартости на одному боці є засоби продукції (засоби праці й
матеріял праці), тобто сталий капітал; на другому боці — робоча сила, тобто
змінний капітал. Для органічного складу капіталу („Капітал“. Книга 1,
XXIII, 2\footnote*{
Див. також: „Капітал“, т. III, ч. І, розд. VIII, і т. III, ч. II, розд. 45, ДВОУ,
1930 р., ст. 200--201. \emph{Ред.}
}) цілком не має значення, чи складається дана кількість вартости
сталого капіталу з чималої маси засобів праці й невеликої маси
матеріялу праці, чи, навпаки, з чималої маси матеріялу праці й невеликої
маси засобів праці: органічний склад капіталу цілком залежить від відношення
між капіталом, витраченим на засоби продукції, та капіталом, витраченим
на робочу силу. Навпаки, з погляду процесу циркуляції, з погляду
ріжниці між основним і обіговим капіталом, також цілком байдуже,
в якому відношенні дана кількість вартости обігового капіталу розпадається
на матеріял праці й заробітну плату. З одного погляду, матеріял
праці залічується в одну категорію з засобами праці, протилежно до
капітальної вартости, витраченої на робочу силу. З другого погляду,
частину капіталу, витрачену на робочу силу, разом з частиною капіталу,
витраченою на матеріял праці, залічується до однієї рубрики й протиставиться
частині капіталу, витраченій на засоби праці.

Тому в Рікардо частина капітальної вартости, витрачена на матеріяли
праці (сировинні й допоміжні матеріяли) не фігурує ні на тому, ні на іншому
боці. Вона зникає цілком. Її не можна залічити до категорії основного
капіталу, бо вона способом своєї циркуляції цілком збігається
з частиною капіталу, витраченою на робочу силу. З другого боку, її не
можна залічити до категорії обігового капіталу, бо разом з цим зникла
б сама собою змога ототожнювати протиставлення „основний і обіговий
капітал“ з протиставленням „сталий і змінний капітал“, — а таке ототожнення
помітне у А. Сміса, і на нього Рікардо мовчазно пристає. Рікардо
має дуже великий логічний інстинкт, щоб не відчути цього, а тому згадана
частина капіталу зовсім зникає у нього.

Тут треба зазначити, що, як каже політична економія, капіталіст
авансує капітал, витрачений на заробітну плату, в різні терміни, залежно
\parbreak{}  %% абзац продовжується на наступній сторінці
