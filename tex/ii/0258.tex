дукцію речей розкошів, бо ці останні дешевшають в наслідок зменшення
додаткової вартости і зумовленого цим зменшення попиту капіталістів на
речі розкошів. Навпаки, оскільки робітники сами купують речі розкошів,
підвищення їхньої заробітної плати не справить — в цих межах — впливу на
підвищення ціни доконечних засобів існування, а лише змінить склад
покупців речей розкошів. Речей розкошів тепер більше йде, ніж раніш,
на споживання робітників і порівняно менш — на споживання капіталістів.
Voilà tout *). Після деяких коливань у циркуляції буде маса товарів такої
самої вартости, як і раніш. — Щождо короткочасних коливань, то наслідок
їх буде лише той, що вільний грошовий капітал, який досі шукав собі
застосування в спекулятивних біржових підприємствах або за кордоном,
тепер надійде в циркуляцію в середині країни.

Відповідь на друге міркування: коли б капіталістичні продуценти
мали змогу з свого бажання підвищувати ціни своїх товарів, то вони
могли б робити це й робили б без усякого підвищення заробітної плати.
Заробітна плата ніколи не підвищувалась би при зниженні цін товарів.
Кляса капіталістів ніколи не ставила б опору тред-юньйонам, бо вона
завжди та за всяких умов могла б робити те, що вона в дійсності робить
тепер, як виняток, в певних особливих, сказати б, місцевих умовах:
а саме, вона могла б використовувати кожне підвищення заробітної плати
для того, щоб куди більше підвищувати ціни товарів, отже, щоб покласти
собі до кишені більший зиск.

Твердження, що капіталісти можуть підвищувати ціни речей розкошів,
бо попит на них меншає (в наслідок зменшеного попиту капіталістів, що
їхні купівельні засоби на це поменшали), це твердження було б цілком
оригінальним застосуванням закону попиту й подання. Оскільки не постає
простої переміни покупців речей розкошів, заміни капіталістів робітниками, —
а оскільки така заміна постає, попит робітників не зумовлює підвищення
цін доконечних засобів існування, бо робітники не можуть витрачати на
доконечні засоби існування тієї частини додаткового заробітку, яку вони
витрачають на речі розкошів, — остільки ціни речей розкошів знижуються
в наслідок зменшеного попиту. У наслідок цього капітал вилучається з
продукції речей розкошів доти, доки їхнє подання зменшиться до таких
розмірів, що відповідають зміненій ролі їх в суспільному процесі продукції.
При такій скороченій продукції ціни їх, за незмінної вартости, знову
підвищуються до свого нормального рівня. Якщо відбувається таке
скорочення, або такий процес вирівнювання, то протягом його при підвищенні
цін на засоби існування у продукцію цих останніх постійно
подаватиметься стільки ж капіталів, скільки їх вилучатиметься з іншої
галузі продукції, поки насититься попит. Тоді знову постає рівновага, і
кінець цілого процесу той, що суспільний капітал, а тому й грошовий
капітал, розподіляється між продукцією доконечних засобів існування й
продукцією речей розкошів в зміненій пропорції.

*) От і все. Ред.
