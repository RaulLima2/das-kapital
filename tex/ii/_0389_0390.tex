\parcont{}  %% абзац починається на попередній сторінці
\index{ii}{0389}  %% посилання на сторінку оригінального видання
з В, В', В" і т. д. (І) справа стоїть інакше. 1) Тільки в їхніх руках
додатковий продукт А, А', А" і т. д. буде функціонувати активно як
додатковий сталий капітал (другий елемент продуктивного капіталу, додаткову
робочу силу, отже, додатковий змінний капітал, ми. покищо
лишаємо осторонь); 2) для того, щоб додатковий продукт потрапив до
їхніх рук, потрібен акт циркуляції, вони повинні купити цей додатковий
продукт.

До пункту (І) тут треба зауважити, що чимала частина додаткового
продукту (віртуально додаткового сталого капіталу), продукованого
А, А', А" (І), хоч її спродуковано поточного року, може активно функціонувати
як промисловий капітал в руках В, В', В" (І) тільки наступного
року або навіть пізніше; до пункту 2) постає питання, відки беруться
гроші, потрібні для процесу циркуляції?

Оскільки продукти, що їх продукують В, В', В" і т. д. (І), сами
знову входять in natura в їхній процес продукції, то само собою зрозуміло,
що pro tanto частина їхнього власного додаткового продукту безпосередньо
(без посередництва циркуляції) переміщується в їхній продуктивний
капітал і входить в нього як додатковий елемент сталого капіталу.
Але pro tanto вони й не перетворюють на золото додаткового
продукту А, А' і т. д. (І). Лишаючи це осторонь, відки ж беруться
гроші? Ми знаємо, що В, В', В" і т. д. (І) утворили свій скарб, як і
А, А' і т. д., через продаж їхніх відповідних додаткових продуктів і
тепер досягли тієї мети, коли їхній нагромаджений як скарб, лише віртуально
грошовий капітал, повинен тепер справді функціонувати як
додатковий грошовий капітал. Але так ми лише блукаємо в зачарованому
колі. Все ж лишається питання, відки беруться гроші, що їх раніше
вилучили з циркуляції та нагромадили капіталісти В, В', В" і т. д. (І)?

Однак уже з досліду простої репродукції ми знаємо, що в руках
капіталістів І і II мусить бути певна маса грошей для того, щоб перетворювати
їхній додатковий продукт. Гроші, що служили лише для витрат
як дохід на засоби споживання, повертались там назад до капіталістів
в міру того, як вони авансували їх для обміну своїх власних товарів;
тут знову з’являються ті самі гроші, але функція їхня змінилась.
Капіталісти А, А' і т. д. І В, В' і т. д. (І) навперемінку подають гроші
для перетворення додаткового продукту на додатковий віртуальний грошовий
капітал, і навперемінку знову подають у циркуляцію новоутворений
грошовий капітал як купівельний засіб.

Єдине припущення при цьому те, що наявної в країні маси грошей
(швидкість обігу та ін. припускається однакові) досить так для активної
циркуляції, як і для утворення запасного скарбу; отже — те саме припущення,
що, як ми бачили, мусить бути здійснене і при простій товаровій
циркуляції. Тільки функція скарбу тут інша. Крім того, маса наявних
грошей мусить бути більша: 1) бо при капіталістичній продукції
кожен продукт (за винятком новоспродукованих благородних металів та
небагатьох продуктів, що їх споживає сам продуцент) продукується як
товар, отже, мусить проробити грошове залялькування; 2) бо на капіталістичній
\index{ii}{0390}  %% посилання на сторінку оригінального видання
основі маса товарового капіталу й величина його вартости не
лише абсолютно більша, але й зростає з куди більшою швидкістю; 3) дедалі більший змінний капітал завжди мусить перетворюватись на
грошовий капітал; 4) бо рівнобіжно з поширенням продукції утворюються
нові капітали, отже, мусить бути наявний і матеріял для їхнього нагромадження
в формі скарбу. — Якщо це має силу для першої фази капіталістичної
продукції, коли і кредитова система супроводиться переважно
металевою циркуляцією, то й для найрозвиненішої фази кредитової
системи це має силу остільки, оскільки її базою лишається металева
циркуляція. З одного боку, додаткова продукція благородних металів
може тут, оскільки вона навперемінку буває буйніша або бідніша,
викликати порушення в товарових цінах не лише протягом довгих, а й
в межах дуже коротких періодів часу; з другого боку, ввесь кредитовий
механізм постійно дбає про те, щоб всілякими операціями, методами,
технічними засобами обмежити справжню металеву циркуляцію відносно
дедалі меншим мінімумом, наслідком чого відповідно збільшується також
штучність цілого механізму й шанси на порушення нормального його
перебігу.

Різні В, В', В'' і т. д. (І), що їхній віртуальний новий грошовий
капітал вступає в операції як активний капітал, можуть купувати один в одного
і продавати один одному свої продукти (частини свого додаткового
продукту). При нормальному перебігу справ гроші, авансовані на циркуляцію
додаткового продукту, pro tanto повертаються назад до різних
В, В' і т. д, в такій самій пропорції, в якій кожен з них авансував
ці гроші на циркуляцію своїх відповідних товарів. Коли гроші циркулюють
як виплатний засіб, то тут доводиться виплачувати лише ріжницю,
оскільки взаємні купівлі й продажі не покривають одна одну. Але важливо
всюди, як ми це робимо тут, припустити спочатку металеву циркуляцію
в її найпростішій, найпервіснішій формі, бо тоді приплив
і відплив грошей, вирівнювання ріжниць, коротко кажучи, всі моменти,
які з’являються в кредитовій системі, як свідомо урегульовані
процеси, виступлять як наявні незалежно від кредитової системи, і
вся справа виявиться в своїй природній формі, а не в пізнішій,
відображеній.

\subsubsection{Додатковий змінний капітал}

Що до цього часу мова йшла тільки про додатковий сталий капітал,
то тепер маємо перейти до розгляду додаткового змінного
капіталу.

В книзі І\footnote*{
Див. „Капітал“, т. I, розділ 23, § 3. — \emph{Ред.}
} докладно з’ясовано, як на основі капіталістичної продукції
завжди є в запасі робоча сила і як, в разі потреби, можна пустити
в рух більше праці, не збільшуючи числа вживаних робітників або маси
робочої сили. Тому покищо не треба далі зупинятись на цьому, навпаки
\parbreak{}  %% абзац продовжується на наступній сторінці
