\parcont{}  %% абзац починається на попередній сторінці
\index{ii}{0383}  %% посилання на сторінку оригінального видання
не дали б нам, крім можливости пояснити, яким чином може одночасно
утворюватись повсюдно скарб, і щоб при цьому сама репродукція, за
винятком репродукції у золотопромисловців, не посунулась ані на
крок далі.

Раніше, ніж розв’язати ці позірні труднощі, ми повинні відрізняти:
акумуляцію в підрозділі І (продукція засобів продукції) і акумуляцію
в підрозділі II (продукція засобів споживання). Ми почнемо з підрозділу
І,

І. Акумуляція в підрозділі І

1. Утворення скарбу

Очевидно, що так капіталовкладення в численних галузях промисловости,
з яких складається кляса І, як і різні індивідуальні капіталовкладення
в кожній з цих галузей промисловости, залежно від їхнього віку,
тобто від протягу їхнього минулого вже функціонування, — ми цілком
лишаємо осторонь їхню величину, технічні умови, ринкові відносини
і т. ін., — перебувають на різних ступенях процесу послідовного перетворення
додаткової вартости на потенціяльний грошовий капітал, все одно
для якої з двох форм поширення продукції має служити цей грошовий
капітал: чи для збільшення діющого капіталу, чи для закладення нових підприємств.
Тому одна частина капіталістів постійно перетворює свій потенціяльний
грошовий капітал, вирослий до відповідної величини, на продуктивний
капітал, тобто за гроші, нагромаджені через перетворення на
золото додаткової вартости, купує засоби продукції, додаткові елементи
сталого капіталу, тимчасом як друга частина капіталістів ще нагромаджує
свій потенціяльний грошовий капітал. Отже, капіталісти цих двох категорій
протистоять один одному: одні---як покупці, інші — як продавці, і
кожний з цих двох категорій виступає виключно в одній з цих ролей.

Наприклад, А продає В (що може репрезентувати й кількох покупців)
600 (= 400 с + $100 v + 100 m$). Він продав товарів на 600, за
600 грішми, що з них 100 репрезентують додаткову вартість; ці
100 він вилучає з циркуляції, нагромаджує їх як гроші; але ці 100 грішми
є лише грошова форма додаткового продукту, що був носієм вартости
величиною в 100. Утворення скарбу взагалі не є продукція, отже, певна
річ, і не приріст продукції. Діяльність капіталіста сходить при цьому
лише на те, що він вилучає з циркуляції, затримує в себе й зберігає
гроші, здобуті через продаж додаткового продукту в 100. Ця операція
відбувається не тільки на боці А, вона відбувається в численних пунктах
на периферії циркуляції в інших капіталістів А', А", А"', і всі вони однаково
старанно працюють коло такого утворення скарбу. Ці численні
пункти, де гроші вилучається з циркуляції та нагромаджуються в численні
індивідуальні скарби, зглядно потенціяльно грошові капітали, являють
так само численні перешкоди для циркуляції, бо вони унерухомлюють
золото й на більш або менш довгий час позбавляють їх циркуляційної
здібности. Але треба взяти на увагу, що скарби утворюються-
\parbreak{}  %% абзац продовжується на наступній сторінці
