засобів продукції, що на них мусить перетворитись грошовий капітал,
підвищується понад той рівень, що був на початку кругобігу, то скарб,
що функціонує як акумуляційний фонд, можна вжити на те, щоб він
заступив грошовий капітал або його частину. Отже, грошовий фонд
акумуляції править за резервний фонд для того, щоб вирівнювати
порушення кругобігу.

Як такий резервний фонд, він відрізняється від фонду купівельних і
виплатних засобів, розгляненого в кругобігу П... П. Ці засоби є
частина діющого грошового капіталу (отже — це форми буття
частини капітальної вартости, що взагалі діє в процесі), що його частини
починають функціонувати лише одна по одній, в різні моменти. В безперервному
перебігу продукційного процесу постійно утворюється
резервний грошовий капітал, бо сьогодні, прим., надійшли виплати, а
робити виплати доведеться лише пізніше, сьогодні продано багато товару,
а купувати багато товару доведеться тільки пізніше; отже, в ці переміжки
частина обігового капіталу *) існує постійно в грошовій формі.
Навпаки, резервний фонд є складова частина не діющого капіталу,
точніш грошового капіталу, а капіталу, що перебуває на підготовчій
стадії своєї акумуляції, на стадії додаткової вартости, ще не
перетвореної на активний капітал. А, проте, само собою зрозуміло, що
капіталіст під скрутний час зовсім не запитує, які певні функції мають
гроші, що перебувають в його руках, а просто вживає те, що має,
аби тільки підтримати процес кругобігу свого капіталу. Прим., в нашому
прикладі Г = 422 ф. стерл., Г' = 500 ф. стерл. Коли частина капіталу
п 422 ф. стерл. існує як фонд виплатних і купівельних засобів, як грошовий
запас, то його розраховано на те, що він, за незмінних обставин,
цілком входить у кругобіг; але його на це й вистачає. Але резервний
фонд є частина 78 ф. стерл. додаткової вартости; він може
ввійти в процес кругобігу капіталу вартістю в 422 ф. стерл. лише тоді,
коли цей кругобіг відбувається в обставинах, що не лишаються незмінні;
бо він є частина фонду акумуляції та фігурує в даному разі,
не поширюючи маштабу репродукції.

Грошовий фонд акумуляції вже є буття лятентного грошового капіталу,
отже, — перетворення грошей на грошовий капітал.

Загальна формула кругобігу продуктивного капіталу, що охоплює
просту репродукцію і репродукцію в поширеному маштабі, така:
П… 1 Т' — Г'. 2 Г — Т — Р Зп…П (П')

*) Обіговий капітал (zirkulierendes Kapital) тут Маркс вживає в тому самому
розумінні, в якому він далі, в першій частині книги III (дивись книга третя,
частина перша, розділ XVI) вживає термін „капітал циркуляції“ (Zirkulationskapital),
тобто в розумінні капіталу, що перебуває в циркуляції, тимчасом як взагалі термін
„обіговий капітал“ Маркс вживає на позначення поточної частини продуктивного
капіталу у відміну від основної частики продуктивного капіталу. (Дивись далі
розділ VIII). Ред.
