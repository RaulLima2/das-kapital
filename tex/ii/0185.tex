дійські та китайські узанції\footnote*{
Узанції (Usance) — строки оплати векселів, що визначаються згідно з місцевими
купецькими звичаями. Ред.
} (для часу, потрібного на подорож векселів
між цими країнами та Европою) з десятьох місяців по написанні до
шістьох місяців по поданні; тепер, по двадцятьох роках, коли прискорено
зв’язки й заведено телеграф, постала потреба в дальшому скороченні
з шости місяців по поданні до чотирьох місяців по написанні як
перший крок до чотирьох місяців по поданні. Плавба вітрильного судна
з Калькути до Лондону повз ріг Доброї Надії триває пересічно мало
не 90 днів. Узанція в чотири місяці по поданні дорівнювала б приблизно
150 дням плавби. А теперішня узанція в шість місяців по поданні
дорівнює 210 дням дороги“. („London Economist“, 16 червня 1866). —
Навпаки, „Бразільська узанція все ще визначається в два й три місяці
по поданні, векселі з Антверпену (на Лондон) видається на 3 місяці по
написанні й навіть Менчестер і Бредфорд видають векселі на Лондон на
три місяці й довший час. За мовчазною згодою купцеві дається достатню
змогу реалізувати свій товар, якщо й не раніше, то все ж
близько того часу, коли кінчається строк виданих за товар векселів.
Тому узанція індійських векселів не надмірна. Індійські продукти, що
їх продається в Лондоні здебільша строком платежу через три місяці,
не можна реалізувати, коли зарахувати сюди деякий час на продаж, за
значно коротший час, ніж п’ять місяців, тимчасом як ще п’ять місяців
пересічно минає між закупом їх в Індії та здачею на англійські склади.
Ми маємо тут період в десять місяців, тимчасом як строк виданих за
товар векселів не перевищує семи місяців“. (Там же, 30 червня 1866).
„2 липня 1866 п’ять великих лондонських банків, що переважно мають
зв’язок з Індією та Китаєм, а також паризька Comptoir d’Escompte заявили,
що з 1 січня 1867 року їхні філії та агентства на Сході будуть
купувати й продавати лише векселі, видані не більш як на чотири місяці
по поданні“. (Там же 7 липня 1865 р.). Це зниження однак не мало
успіху й довелось його скасувати (з того часу Суецький канал революціонізував
усе це).

Зрозуміло, коли довшає час обігу товарів, то більшає ризик, що
зміняться ціни на ринку продажу, бо довшає період, що протягом нього
можуть змінитись ціни.

Ріжниця в часі обігу — почасти індивідуальна між різними поодинокими
капіталами тієї самої галузі підприємств, почасти між різними галузями
підприємств залежно від різних узанцій, там, де не виплачується
одразу готівкою, — випливає з різних строків виплати при купівлі та
продажу. Ми не будемо тут докладніше зупинятись на цьому пункті,
важливому для кредитової справи.

Від розміру угод на поставки, а він зростає разом із зростом розмірів
і маштабу капіталістичної продукції, залежать також і ріжниці в часі
обороту. Угода на поставку, як оборудка між продавцем і покупцем, є
операція, що належить до ринку, до сфери циркуляції. Ріжниці, що ви-