\parcont{}  %% абзац починається на попередній сторінці
\index{ii}{0009}  %% посилання на сторінку оригінального видання
не в наслідок функції грошей як засобу виплати. Гроші можуть витрачатись
у такій формі лише тому, що робоча сила перебуває в стані відокремлення
від її засобів продукції (включаючи і засоби існування, що являють
засоби продукції самої робочої сили), і тому, що це відокремлення
знімається (aufgehoben wird) лише через те, що робочу силу продається
власникові засобів продукції; що, отже, покупцеві належить і приведення
робочої сили до стану поточности, що межі її ніяк не зливаються з
межами тієї кількости праці, яка потрібна для репродукції власної ціни
робочої сили. Капіталістичне відношення протягом процесу продукції
виявляється лише тому, що воно вже існує само по собі
в акті циркуляції, в тих відмінних основних економічних умовах, що в
них виступають один проти одного продавець і покупець, в їхньому клясовому
відношенні. Не природою грошей дано це відношення; навпаки,
наявне буття цього відношення перетворює просту функцію грошей на
функцію капіталу.

Щодо розуміння грошового капіталу (а лише з ним поки маємо
справу в межах тієї визначеної функції, що в ній він тут виступає), то
звичайно тут трапляються або переплітаються дві помилки. По-перше,
функції, що їх виконує капітальна вартість як грошовий капітал, і що
їх вона може виконувати саме тому, що вона перебуває в грошовій
формі, помилково висновується з її характеру як капіталу, тимчасом як
вони випливають лише з грошового стану капітальної вартости, з того,
що вона існує у формі грошей. І, по-друге, навпаки, той специфічний
зміст функції грошей, що разом з тим перетворює цю функцію на функцію
капіталу, висновується з природи грошей (отже, гроші сплутуються
з капіталом), тимчасом як ця функція має собі за передумову
суспільні умови, як от тут при сповненні акту $Г — Р$, які зовсім не
дані в простій товаровій циркуляції та відповідній грошовій циркуляції.

Купівля та продаж рабів своєю формою теж є купівля і продаж
товарів, але, коли немає рабства, гроші не можуть виконувати
цієї функції. Коли рабство є, тоді можна витратити гроші на
купівлю рабів. Навпаки, наявности грошей в руках покупця зовсім не
досить, щоб уможливити рабство.

Те, що продаж власної робочої сили (у формі продажу власної праці
або у формі заробітної плати) виявляється не як ізольоване явище,
а як міродайна для суспільства передумова товарової продукції, отже,
те, що грошовий капітал виконує розглядувану тут функцію $Г — Т\splitfrac{Р}{Зп}$
в суспільному маштабі, — це має за свою передумову історичні процеси,
що в наслідок їх зруйнувалось первісне поєднання засобів продукції з
робочою силою; процеси, що в наслідок їх маса народу, робітники як
не-власники, і не-робітники як власники цих засобів продукції, протистоять
одні другим. При цьому не впливає на суть справи, чи це поєднання
до його розпаду мало таку форму, що сам робітник належав як засіб
продукції до інших засобів продукції, чи був власником їх.
