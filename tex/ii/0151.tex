потрібен основний капітал, машини тощо, бо продуктивний капітал завжди
включає засоби праці, але він не завжди включає матеріял праці. Сам
Сміс каже безпосередньо по цьому: „Землі, копальні та рибні промисли
потребують так основного, як і обігового капіталу, щоб їх розробляти“
(отже, він згоджується з тим, що не лише поточний, а й основний капітал
потрібен для продукції сировинного матеріялу, „і“ (тут нова помилка)
„їхній продукт покриває з зиском не лише ці капітали, але й усі
інші, що є в суспільстві“\footnote*{
„Lands, mines, and fisheries, require all both a fixed and circulating capital
to cultivate them; and their produce replaces with a profit, not only those capitals,
but all the others in society“ (p. 188).
}. Це зовсім неправильно. Їхній продукт
дає сировинні матеріяли, допоміжні матеріяли тощо для всіх інших галузей
промисловости. Але їхня вартість не покриває вартости всіх інших
суспільних капіталів; вона покриває лише свою власну капітальну вартість
(+ додаткова вартість). В цьому в А. Сміса знову виявляється
вплив фізіократів.

З суспільного погляду правильно, що частина товарового капіталу,
яка складається з продуктів, що можуть служити лише засобами праці,
раніше або пізніше — якщо тільки не спродуковано їх взагалі марно,
якщо вони не лишаються непродані — функціонуватимуть як засоби праці;
інакше кажучи, на основі капіталістичної продукції такі продукти,
переставши бути товарами, справді мусять стати згідно з своїм призначенням
елементами основної частини суспільного продуктивного капіталу.

Тут перед нами ріжниця, що випливає з натуральної форми продукту.

Напр., прядільна машина немає споживної вартости, коли її не вживається
на прядіння, отже, коли вона не функціонує як елемент продукції,
отже, з капіталістичного погляду як основна частина продуктивного
капіталу. Але прядільна машина рухома. Її можна вивезти з країни, де
її випродукувано, і продати в іншій країні в обмін, безпосередньо або
посередньо, на сировинний матеріял тощо або на шампанське. В країні,
де її випродукувано, вона функціонує тоді лише як товаровий капітал,
але зовсім не функціонує — навіть після її продажу — як основний капітал.

Навпаки, продукти, що через прикріплення їх до ґрунту є льокалізовані,
і які, отже, можна використовувати лише на місці, напр., фабричні
будівлі, залізниці, мости, тунелі, доки й т. ін., меліорації тощо —
всі такі продукти не можна вивезти матеріяльно, так, як вони є. Вони
нерухомі. Або їх марно спродуковано, або, якщо їх продано, вони мусять
функціонувати як основний капітал, — у тій країні, де їх випродукувано.
Для капіталістичного продуцента, що за для спекуляції, маючи
на меті продаж, будує фабрики або поліпшує ґрунт, ці речі мають форму
його товарового капіталу, отже, за А. Смісом, форму обігового капіталу.
Але, з суспільного погляду, ці речі, щоб не лишитись некорисними,
мусять, кінець-кінцем, функціонувати у власній країні як основний
капітал у процесі продукції, фіксованому в місці їхнього перебування. Відси
ні в якому разі не випливає, що нерухомі речі, як такі, вже самі
собою є основний капітал; вони, як, напр., житлові будинки тощо,