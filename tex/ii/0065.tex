вання в часі. Напр., коли товари не можна продати, коли рух Т' — Г'
припиняється для однієї частини, то кругобіг цієї частини переривається
і її не покривається засобами її продукції; зміна функцій наступних
частин, що виходять як Т' з процесу продукції, затримується
попередніми. Коли це триває деякий час, то продукція обмежується, і
цілий процес припиняється. Кожна зупинка в послідовному русі частин
порушує послідовність їх у просторі, кожна зупинка на одній стадії
зумовлює більшу або меншу зупинку в цілому кругобігу не лише тієї
частини капіталу, що її рух спинився, але також і цілого індивідуального
капіталу.

Найближча форма, що в ній виявляється процес, є така послідовність
фаз, при якій перехід капіталу в нову фазу зумовлюється його виходом
з іншої фази. Тому кожний поодинокий кругобіг має також за вихідний
пункт і за пункт повороту одну з функціональних форм капіталу.
З другого боку цілий процес є в дійсності єдність трьох кругобігів, що є
різні форми, в яких виражається безперервність процесу. Для кожної
функціональної форми капіталу цілий кругобіг є її специфічний кругобіг,
і при цьому кожен з цих кругобігів зумовлює безперервність цілого процесу;
круговий рух однієї функціональної форми зумовлює круговий рух
інших форм. Для сукупного процесу продукції, особливо для суспільного
капіталу, є та неодмінна умова, що процес продукції є разом з
тим і процес репродукції, а значить, і процес кругобігу кожного з його
моментів. Різні частини капіталу послідовно перебігають різні стадії та
функціональні форми. В наслідок цього кожна функціональна форма, хоч
вона і репрезентує раз-у-раз іншу частину капіталу, пророблює одночасно
з іншими свій власний кругобіг. Одна частина капіталу, але
така, що завжди міняється, завжди репродукується, існує як товаровий
капітал, який перетворюється на гроші; друга частина існує як грошовий
капітал, який перетворюється на продуктивний; третя частина —
як продуктивний капітал, що перетворюється на товаровий капітал.
Постійну наявність усіх трьох форм упосереднює кругобіг цілого капіталу,
що саме й переходить ці три фази.

Отже, капітал, як ціле, одночасно перебуває в своїх різних фазах,
послідовно розміщених у просторі. Але кожна частина завжди переходить
по черзі з однієї фази, з однієї функціональної форми в іншу,
і так функціонує по черзі в усіх формах. Ці форми є такі поточні форми,
що їхню одночасність упосереднює їхня послідовність. Кожна форма
йде по другій й передує їй, так що поворот однієї частини капіталу
до однієї форми зумовлено поворотом другої частини до другої форми.
Кожна частина безупинно пророблює свій власний обіг, але в цій
формі завжди перебуває інша частина капіталу, і ці осібні обіги становлять
лише одночасні й послідовні моменти сукупного руху.

Лише в єдності трьох кругобігів здійснюється безперервність сукупного
процесу замість змальованої вище переривчастости. Сукупний
суспільний капітал завжди має цю безперервність, і його процес завжди
є єдність трьох кругобігів.
