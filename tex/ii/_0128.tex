\parcont{}  %% абзац починається на попередній сторінці
\index{ii}{0128}  %% посилання на сторінку оригінального видання
як засіб циркуляції, а потім знову як скарб відокремлюється від маси
грошей, що циркулюють. З розвитком кредитової системи, — а він неминуче
відбувається рівнобіжно з розвитком великої промисловости й капіталістичної
продукції, — гроші функціонують уже не як скарб, а як капітал,
однак в руках не їхнього власника, а другого капіталіста, що йому
їх передається в розпорядження.

\section{Цілий оборот авансованого капіталу. Цикли
оборотів}

Ми бачили, що основні й поточні складові частини продуктивного
капіталу різним способом і в різні періоди обертаються, і що різні складові
частини основного капіталу в тому самому підприємстві знову таки
мають різні періоди обороту залежно від різного часу їхнього життя, а,
значить, і репродукції. (Про справжні або позірні відмінності в обороті
різних складових частин поточного капіталу в тому самому підприємстві
див. наприкінці цього розділу під цифрою 6).

1. Цілий оборот авансованого капіталу є пересічний оборот його різних
складових частин; спосіб обчислення подається нижче. Оскільки
йдеться лише про різні періоди часу, немає, звичайно, нічого простішого,
як обчислити з них пересічне, але:

2. Тут маємо не лише кількісні, а й якісні відмінності.

Поточний капітал, що входить в процес продукції, переносить на
продукт усю свою вартість, а тому, щоб продукційний процес відбувався
безупинно, він мусить завжди заміщуватись in natura через продаж
продукту. Основний капітал, що входить у процес продукції, переносить
на продукт лише частину своєї вартости (зношування) і, не зважаючи на
зношування, і далі функціонує в продукційному процесі; тому лише через
коротші або довші переміжки, в усякому разі не так часто, як поточний
капітал, треба його заміщувати in natura. Ця потреба в заміщенні, строк
репродукції, не лише кількісно різна для різних складових частин капіталу,
але, як ми бачили вище, частина довготривалішого, багатолітнього
капіталу може бути заміщена і долучена in natura до старого основного
капіталу щорічно або навіть через коротші переміжки часу; щождо основного
капіталу іншої властивости, то його заміщення може статися лише
одним заходом наприкінці його життя.

Ось чому й треба звести особливі обороти різних частин основного
капіталу до однорідної форми обороту, так щоб вони відрізнялись один
від одного лише кількісно, триванням обороту.

Цієї якісної тотожности немає, коли ми візьмемо за вихідний пункт
$П\dots{} П$, — форму безперервного продукційного процесу. Бо певні елементи
П мусять заміщуватись in natura, а інші ні. Але форма $Г\dots{} Г'$ дає безперечно
цю тотожність обороту. Коли ми візьмемо, напр., машину вартістю
10.000 ф. стерл., яка живе 10 років, то тоді щороку знов пере-
\parbreak{}  %% абзац продовжується на наступній сторінці
