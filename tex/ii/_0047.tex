\parcont{}  %% абзац починається на попередній сторінці
\index{ii}{0047}  %% посилання на сторінку оригінального видання
продукційний процес виступає тут як продуктивна функція промислового
капіталу, так само гроші і товар виступають як форми циркуляції
того самого промислового капіталу, а, значить, і функції грошей і
товару виступають як функції циркуляції цього капіталу, які або
підготовляють функції продуктивного капіталу, або походять з них. Лише
в наслідок їхнього зв’язку як форм тих функцій, що їх повинен виконувати
промисловий капітал на різних стадіях процесу свого кругобігу, функція
грошей і функція товару є тут разом з тим функція грошового капіталу
й товарового капіталу. Отже, було б помилково специфічні властивості
й функції, що характеризують гроші як гроші і товар як товар, висновувати
з їхніх властивостей як капіталу, і так само було б помилкою,
навпаки, висновувати властивості продуктивного капіталу з його способу
існування в засобах продукції.

Скоро $Г'$ або $Т'$ фіксуються як $Г + г$, $Т + т$, тобто як відношення
капітальної вартости до додаткової вартости як до свого нащадка, то це
відношення виражається і в тому і в другому, одного разу в грошовій
формі, другого разу в товаровій, що зовсім не змінює справи. Отже, це
відношення випливає не з особливостей і функцій, що властиві грошам
як таким або товарові як такому. В обох випадках властивість, що характеризує
капітал, а саме те, що він є вартість, яка вилуплює вартість, виражається
лише як результат. $Т'$ завжди є продукт функціонування $П$, а $Г'$ завжди
є лише форма $Т'$, перетворена в кругобігу промислового капіталу. Тому,
скоро реалізований грошовий капітал знову починає функціонувати як
грошовий капітал, він перестає бути виразом того капіталістичного відношення,
що є в $Г' = Г + г$. Коли відбувся акт $Г\dots{} Г'$ і $Г'$ знову починає
кругобіг, $Г'$ фігурує вже не як $Г'$, а як $Г$, навіть і тоді, коли б ціла
додаткова вартість, що є в $Г'$, капіталізувалась. У нашому випадку другий
кругобіг починається грошовим капіталом в 500 ф. стерл., а не
422 ф. стерл., як перший. Грошовий капітал, що починає кругобіг, на
78 ф. стерл. більший, ніж був раніше; ця ріжниця існує в порівнянні
одного кругобігу з другим; але цього порівняння не може бути всередині
кожного поодинокого кругобігу. Авансовані як грошовий капітал 500 ф.
стерл., що з них 78 ф. стерл. існували раніше як додаткова вартість,
відіграють таку саму ролю, як і 500 ф. стерл., що ними другий капіталіст
починає свій перший кругобіг. Так само і в кругобігу продуктивного
капіталу. Збільшене $П'$ при відновленні кругобігу виступає як $П$,
подібно до $П$ в простій репродукції $П\dots{} П$.

У стадії $Г' — Т' \splitfrac{Р}{Зп}$ вирослу величину позначено лише $Т'$, але не
$Р'$ і $Зп'$. А що $Т$ є сума $Р$ і $Зп$ показує, то вже $Т'$ показує, що сума $Р$ і $Зп$,
 які містяться в ньому, більша, ніж була в первісному $П$. Але, подруге,
позначення $Р'$ і $Зп'$ було б помилкове, бо ми знаємо, що із зростанням
капіталу сполучається зміна його вартісного складу; з розвитком останнього
вартість $Зп$ зростає, а вартість $Р$ завжди, зменшується відносно, а часто
абсолютно.
