\parcont{}  %% абзац починається на попередній сторінці
\index{ii}{0304}  %% посилання на сторінку оригінального видання
і на обіговий сталий капітал: матеріяли продукції, як от сировинні й допоміжні
матеріяли, напівфабрикати й т. ін.

Вартість цілого річного продукту, виробленого в кожному з цих
двох підрозділів за допомогою цього капіталу розпадається на частину
вартости, яка репрезентує сталий капітал c, зужиткований у продукції і в
своїй вартості лише перенесений на продукт, і на частину вартости, долучену
всією річною працею. Остання частина знову таки розпадається
на заміщення авансованого змінного капіталу v і на надлишок над ним,
який становить додаткову вартість m. Отже, подібно до вартости кожного
поодинокого товару, і вартість цілого річного продукту кожного
підрозділу розпадається на c + v + m.

Частина вартости с, яка репрезентує сталий капітал, зужиткований
у продукції, не збігається з вартістю сталого капіталу, застосованого
в продукції. Правда, продукційні матеріяли зужитковано цілком, а тому
їхню вартість цілком перенесено на продукт. Але лише частину застосованого
основного капіталу зужитковано цілком, і значить, лише вартість
цієї частини переходить на продукт. Друга частина основного капіталу,
машини, будівлі тощо, існує й функціонує далі, як і раніше, хоч
вартість її й зменшилась в наслідок річного зношування. Коли ми розглядаємо
вартість продукту, то цієї далі діючої частини основного капіталу
для нас не існує. Вона становить частину капітальної вартости, незалежну
від цієї новоспродукованої товарової вартости, і існує поряд неї. Це
виявилося уже при розгляді вартости продукту поодинокого капіталу
(кн. І, розд. VI). Тут ми мусимо, однак, покищо абстрагуватися від застосованого
там способу розгляду. Розглядаючи вартість продукту поодинокого
капіталу, ми бачили, що вартість, втрачувана основним капіталом
в наслідок зношування, переноситься на товаровий продукт,
вироблений підчас зношування — все одно, чи заміщується протягом цього
часу частину основного капіталу in natura з цієї перенесеної вартости,
чи ні. Навпаки, тут, розглядаючи ввесь суспільний продукт і його вартість,
ми мусимо, принаймні на деякий час, абстрагуватись від частини
вартости, перенесеної протягом року в наслідок зношування основного
капіталу на річний продукт, якщо тільки цей основний капітал протягом
року не заміщується знову in natura.

В дальшому відділі цього розділу ми розглянемо цей пункт окремо.

Для нашого досліду простої репродукції ми візьмемо за основу таку схему,
де с = сталому капіталові, v = змінному капіталові, m = додатковій
вартості, а відношення зростання вартости m/v береться в 100\%. Числа
можуть значити мільйони марок, франків або фунтів стерлінґів.

І. Продукція засобів продукції:

Капітал... 4000с + 1000v = 5000.

Товаровий продукт... 4000с + 1000v + 1000m = 6000, що існує
в засобах продукції.
\parbreak{}  %% абзац продовжується на наступній сторінці
