\parcont{}  %% абзац починається на попередній сторінці
\index{ii}{0013}  %% посилання на сторінку оригінального видання
або просто на продажу надлишкового продукту як товару. Спочатку
вона надає загальности товаровій продукції, а потім поступінно перетворює
всю товарову продукцію на капіталістичну\footnote{
До цього місця рукопис VII. Звідси рукопис VІ.
}.

Хоч би які були суспільні форми продукції, робітники та засоби продукції
завжди лишаються її чинниками. Але в стані відокремлення одних
від одних і ті і другі є її чинники лише в можливості. Щоб взагалі
продукувати, вони мусять з’єднатись. Той особливий характер і спосіб,
що ним здійснюється це з’єднання, відрізняє різні економічні епохи соціяльної
структури. В досліджуваному випадку відокремлення вільного
робітника від його засобів продукції є даний вихідний пункт, і ми
бачили, як та при яких умовах обоє вони з’єднуються в руках капіталіста —
а саме з’єднуються як продуктивна форма буття його капіталу. Тому
дійсний процес, що в нього входять з’єднані таким чином особові та
речові товаротворчі елементи, — процес продукції, сам стає функцією капіталу,
— капіталістичним процесом продукції, що його природу докладно
описано в першій книзі цієї праці. Всяке зайняття товаровою продукцією
стає разом з тим зайняттям експлуатацією робочої сили; але тільки капіталістична
товарова продукція стає таким епохальним способом експлуатації,
який, у своєму поступовому історичному розвитку через організацію
процесу праці й велетенський розвиток техніки робить переворот
у цілій економічній структурі суспільства і до незрівняности лишає позаду
себе всі попередні епохи.

У наслідок різних ролей, що їх засоби продукції та робоча сила відіграють
протягом процесу продукції при утворенні вартости, отже, і
при утворенні додаткової вартости, вони відрізняються як сталий і
змінний капітал, оскільки вони є форми існування авансованої капітальної
вартости\footnote*{
Див. кн. І, розд. VI. \emph{Ред.}
}. Як різні складові частини продуктивного капіталу, відрізняються
вони далі тим, що перші, коли ними володіє капіталіст, лишаються
його капіталом і поза продукційним процесом, тимчасом як робоча сила
лише в ньому стає формою буття індивідуального капіталу. Коли робоча
сила є товар лише в руках її продавця, найманого робітника, то, навпаки,
капіталом вона стає лише в руках її покупця, капіталіста, що
йому припадає тимчасове користування з неї. Самі засоби продукції
стають речовими формами продуктивного капіталу, або продуктивним
капіталом, лише з того моменту, коли постає можливість долучати до
них робочу силу, як особову форму буття того капіталу. Отже,
як людська робоча сила з природи не є капітал, так само і засоби
продукції з природи не є капітал. Вони набувають цього специфічного
суспільного характеру лише в певних, історично розвинутих умовах,
так само як лише за таких умов благородним металям надається характер
грошей, або навіть грошам характер грошового капіталу.

Продуктивний капітал функціонуючи споживає свої власні складові
частини, щоб перетворити їх на масу продуктів вищої вартости. Що
\parbreak{}  %% абзац продовжується на наступній сторінці
