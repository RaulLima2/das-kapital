of Inquiry on Caledonian Railway, передруковано в Money Market Review, 1867)\footnote*{
„Цитоване місце є в номері з 25 січня 1868 року, і взято його з статті в „Money Market Rewiew“ —
The Caledonian Railway, The Directors Reply, де йде мова про звіт капітана Фіцморіса“. Ред.
}.

Практично неможливо й недоцільно розмежовувати заміщення й підтримання основного капіталу в
хліборобстві, принаймні, оскільки воно ще не застосовує сили пари. „Коли є повний, але не надто
великий комплект реманенту (різних хліборобських та інших всякого роду знарядь праці та
господарювання), щорічне зношування та витрати на підтримання реманенту звичайно обчислюється в
15—25\% авансованого капіталу, залежно від різних наявних умов“ (Kirchhof, Handbuch der
landwirtchaftlichen Betriebsiehre. Berlin, 1862, p. 137).

Щодо рухомої частини залізниці, то зовсім не можна розмежувати ремонт і заміщення. „Ми підтримуємо
нашу рухому частину у наявних її розмірах. Скільки паровозів є в нас, стільки ми й підтримуємо. Коли
з плином часу паровіз робиться непридатний, так що вигідніше збудувати новий, то ми й будуємо його
на кошти доходів, при чому, звичайно, записуємо на дохід вартість матеріялів, що лишились від старої
машини… А лишається завжди чимало… Колеса, осі, казан тощо, коротко кажучи, лишається чимала частина
старого паровозу“. (T. Gooch, Chairman of Great Western Railway C°, R. C. № 17327—29). „Ремонтувати
значить відновлювати; для мене немає слова „заміщення“... Коли залізничне товариство купило вагон
або паротяг, то воно мусить їх так полагодити, щоб вони вічно могли служити (17784). Ми обчислюємо
витрати на паротяги в 8 1/2 пенсів на англійську милю пробігу. На ці 8 1/2 пенсів ми назавжди
підтримуємо паротяги. Ми поновлюємо наші машини. Коли ви хочете купити машину нову, ви витрачаєте
більше грошей, ніж треба... В старій машині завжди буде пара коліс, вісь або ще яка придатна
частина, і це дає змогу збудувати дешевше таку саму гарну машину, як і цілком нова (17790). Тепер я
продукую щотижня новий паротяг, тобто такий самий гарний, як новий, бо в ньому казан, циліндр і рама
нові“ (17823. Archibald Sturrock, Locomotive Superintendent of Great Northern Railway в R. C. 1867).

Це стосується й до вагонів: „З плином часу запас паротягів і вагонів постійно поновлюється; одного
разу насаджуються нові колеса, другого разу робиться нову ряму. Частини, що на них ґрунтується рух і
що найбільше зношуються, відновлюються поступінно; таким чином, машини й вагони можуть підлягати
стільком ремонтам, що в багатьох з них не лишиться й сліду старого матеріялу… Навіть, коли вони
зробляться вже зовсім непридатні для ремонту, з старих вагонів або паротягів перероблюються
поодинокі частини і таким чином вони ніколи не гинуть цілком для залізниці. Тому рухомий капітал
перебуває в стані постійної репро-