\parcont{}  %% абзац починається на попередній сторінці
\index{ii}{0382}  %% посилання на сторінку оригінального видання
що раніше в ньому були, могли вже раніше лежати як складова частина
скарбу або бути грошовою формою заробітної плати, обслуговувати
перетворення на гроші засобів продукції або інших товарів, обслуговувати
циркуляцію сталої частини капіталу або доходу капіталіста. Вони
так само не є нове багатство, як і гроші, розглядувані з погляду простої
товарової циркуляції, що являють собою носіїв лише своєї наявної
вартости, не є носії вартости вдесятеро більшої, з тієї причини, що
вони протягом дня обернулись десять разів, реалізували десять різних
вартостей. Товари існують без них, і вони сами лишаються тим, чим
вони є (або в наслідок зношування навіть зменшуються), все одно хоч
при одному обороті, хоч при десятьох. Тільки в золотопромисловості, —
оскільки золотий продукт має в собі додатковий продукт, є носій додаткової
вартости, — утворюється нове багатство (потенціяльні гроші), і
лише, оскільки ввесь новий грошовий продукт входить у циркуляцію,
він збільшує грошовий матеріял потенціяльних нових грошових капіталів.

Не являючи нового додаткового суспільного багатства, ця нагромаджена
в грошовій формі додаткова вартість являє, однак, новий, потенціяльний
грошовий капітал в наслідок тієї функції, що для неї її нагромаджується.
(Далі ми побачимо, що новий грошовий капітал може виникнути
й іншим способом, крім поступінного перетворення додаткової вартости
на золото.)

Гроші вилучається з циркуляції й нагромаджується як скарб через
продаж товару без наступної купівлі. Отже, коли уявити собі, що ця
операція відбувається повсюдно, то, здається, не можна зрозуміти, відки
візьмуться покупці, бо в цьому процесі, — а його треба уявляти собі загальним
тому, що кожен індивідуальний капітал може перебувати в стадії
акумуляції, — кожен хоче продавати, щоб нагромаджувати скарб, а ніхто
не хоче купувати.

Коли уявити собі, що процес циркуляції між різними частинами річної
репродукції перебігає ніби прямою лінією, — а це неправильно, бо за
небагатьма винятками він завжди складається з навзаєм протилежних
рухів, — то доведеться почати з золотопромисловця (зглядно сріблопромисловця),
що купує, не продаючи, і припустити, що всі інші продають
йому. В такому разі сукупний річний суспільний додатковий продукт
(носій сукупної додаткової вартости) перейшов би до нього,
а всі інші капіталісти pro rata розподілили б між собою його
додатковий продукт, то вже з природи існує в грошовій формі і є природне
втілення в золоті його додаткової вартости; бо частину продукту
золотопромисловця, що повинна замістити його діющий капітал, уже зв’язано
й її вжито. Спродукована в золоті додаткова вартість золотопромисловця
була б тоді за єдиний фонд, що з нього всі інші капіталісти
брали б матеріял для перетворення на гроші свого річного додаткового
продукту. Отже, величиною вартости вона мусіла б дорівнювати цілій
суспільній річній додатковій вартості, що мусить лише ще залялькуватись
на форму скарбу. Хоч які безглузді ці припущення, і вони нічого
\parbreak{}  %% абзац продовжується на наступній сторінці
