дукції, і належить вона до витрат циркуляції, відраховань із загального
виторгу (включаючи й саму робочу силу, що її вживають виключно на
ці функції).

Все ж є деяка ріжниця між витратами, що випливають з бухгальтерії,
зглядно непродуктивною витратою на це робочого часу, з одного боку, і витратами
часу просто на купівлю й продаж, з другого боку. Ці останні випливають
лише з певної суспільної форми продукційного процесу, з того,
що це — процес продукції товару. Ведення книг, як контроль та ідеальне
об’єднання процесу стає то доконечніше, що більше процес поширюється
до суспільних розмірів і втрачає суто-індивідуальний характер.
Отже, ведення книг доконечніше в капіталістичній продукції, ніж
в розпорошеному ремісничому й селянському виробництві, доконечніше
у колективній продукції, ніж у капіталістичній. Але витрати на ведення
книг скорочуються з концентрацією продукції і скорочуються то більше,
що більше воно перетворюється на суспільне ведення книг.

Тут ідеться лише про загальний характер витрат циркуляції, що постають тільки з простої формальної
метаморфози. І не треба тут вдаватись у всі детальні форми цих витрат. Але наскільки форми, належні
до простого перетворення форми вартости, що, отже, випливають з певної суспільної форми
продукційного процесу, — форми, які в індивідуального товаропродуцента є лише минущі й ледве помітні
моменти, що перебігають поряд його продуктивних функцій або переплітаються з ними, — наскільки ці
форми можуть вражати як масові витрати циркуляції, це ми побачимо, поглянувши на просте приймання й
видачу грошей, коли вони усамостійнюються й концентруються в широкому маштабі, як виключна функція
банків тощо, або як виключна функція скарбників в індивідуальних підприємствах. Але треба пам’ятати,
що ці витрати циркуляції, змінюючи свій вигляд, не змінюють свого характеру.

3) Гроші

Хоч продукт продукується як товар, хоч не як товар, він завжди є
речова форма багатства, споживна вартість, призначена для особистого
або продуктивного споживання. Його вартість як товару існує ідеально
в ціні, яка нічого не змінює в його справжній споживній формі. Але та
обставина, що певні товари, напр., золото й срібло, функціонують як
гроші, і як такі перебувають виключно в процесі циркуляції (навіть як
скарб, резерв і т. ін., лишаються вони, хоч і лятентно, в сфері
циркуляції) є цілком продукт певної суспільної форми продукційного
процесу, процесу продукції товарів. Що на основі капіталістичної продукції
товар стає загальною формою продукту; що переважна кількість
продуктів продукується як товар, і тому мусить вона набрати грошової
форми; що, отже, товарова маса, тобто частина суспільного багатства,
яка функціонує як товар, безупинно зростає, — то тут більшає й кількість
золота й срібла, що, функціонують як засіб циркуляції, засіб виплати,
резерв тощо. Ці товари, що функціонують як гроші, не входять ні в
особисте, ні в продуктивне споживання. Це — суспільна праця, зафіксо-
