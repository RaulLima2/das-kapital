нього дорівнює нулеві. З цього неодмінно випливає, що грошей, потрібних
для циркуляції річного доходу, досить і для циркуляції сукупного
річного продукту; що, отже, в нашому випадку, грошей, потрібних для
циркуляції засобів споживання вартістю в 3.000, досить і для циркуляції
сукупного річного продукту вартістю в 9.000. Такий справді погляд
А. Сміса, і Т. Тук повторює його. Це хибне уявлення про відношення
маси грошей, потрібної, щоб перетворити дохід на гроші, до маси грошей,
потрібної для циркуляції сукупного суспільного продукту, є неминучий
результат незрозумілого, непродуманого уявлення про спосіб, що
ним репродукуються й щороку заміщуються різні речові й вартісні елементи
сукупного річного продукту. Тому його вже й збито.

Послухаймо самого Сміса й Тука.

Сміс каже (книга II, розділ 2): „Циркуляцію кожної країни можна
розподілити на дві частини: циркуляцію між самими торговцями й циркуляцію
між торговцями й споживачами. Хоч ті самі грошові одиниці, — паперові
або металеві, — можуть застосовуватись то в одній, то в другій
циркуляції, однак, і та й друга безупинно відбуваються одночасно одна поряд
однієї, і тому кожна з них потребує певної маси грошей того або
іншого роду, щоб і далі продовжувати свій рух. Вартість товарів, що циркулюють
між різними торговцями, ніколи не може перевищити вартости
товарів, що циркулюють між торговцями й споживачами; бо, хоч що купують
торговці, вони мусять усе це, кінець-кінцем, продати споживачам.
А що циркуляція між торговцями відбувається en gros *), то вона взагалі потребує
досить великих сум для кожного поодинокого обміну. Навпаки, циркуляція
між торговцями й споживачами відбувається здебільша en détail **)
і часто потребує лише дуже незначних грошових сум; часто досить одного
шилінґа або навіть половини пенні. Але невеличкі суми циркулюють
куди швидше, ніж великі... Тому, хоч річні закупи всіх споживачів принаймні
(чудове це „принаймні“) дорівнюють вартістю закупам усіх
торговців, однак, їх звичайно можна переводити куди меншою масою
грошей“ і т. ін.

До цього місця Адама Т. Тук („Аn Inquiry into the Currency Principle.
London, 1844“, crop. 34—36 passim) зауважує: „Не викликає жодного
сумніву, що ця подана тут ріжниця в суті правильна... Обмін між торговцями
й споживачами охоплює також і виплату заробітної плати, яка
являє головний дохід (the principal means) споживачів... Всі обміни між
торговцями, тобто всі продажі, починаючи від продуцента або імпортера,
переходячи всі щаблі посередніх процесів мануфактури й т. інш. і закінчуючи
роздрібним торговцем або купцем-експортером, можна звести
до рухів переміщення капіталу. Але переміщення капіталу не мають собі
зa неодмінну передумову й на практиці справді при більшості обмінів не
призводять до того, щоб підчас переміщення справді передавалось банк-

*) Гуртом, оптом. Ред.

**) На роздріб. Ред.
