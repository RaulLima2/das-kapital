\parcont{}  %% абзац починається на попередній сторінці
\index{ii}{0241}  %% посилання на сторінку оригінального видання
свого функціонування само підприємство через капіталізацію певної частини
додаткової вартости. Для капіталіста В це не можливо. Частина
капіталу, що про неї мовиться, мусить складати в нього частину первісно
авансованого капіталу. В обох випадках ця частина капіталу фігуруватиме
в книгах капіталіста як авансований капітал — і ним вона є в
дійсності — бо, згідно з нашим припущенням, вона становить частину
продуктивного капіталу, доконечного для провадження підприємства
в даному маштабі. Але величезна ріжниця в тому, з якого фонду
її авансується. У В вона дійсно є частина первісного авансованого
капіталу або капіталу, що його треба мати в розпорядженні.
Навпаки, в А вона є частина додаткової вартости, застосованої як
капітал. Цей останній випадок показує нам як не лише акумульований
капітал, а й частина первісно авансованого капіталу може бути просто
капіталізованою додатковою вартістю.

Скоро сюди долучається розвиток кредиту, відношення первісно авансованого
капіталу й капіталізованої додаткової вартости заплутується
ще більше. Напр., А позичає в банкіра С частину продуктивного капіталу,
що з ним, він починає або продовжує справу протягом року. З
самого початку він не має власного капіталу, достатнього для провадження
справи. Банкір С позичає йому суму, що складається виключно з
додаткової вартости, покладеної до нього підприємцями D, E, F і т. ін.
З погляду А тут ще не йдеться про акумульований капітал. А в дійсності
для D, E, F і т. ін. А є не що інше, як аґент, що капіталізує привласнену
ними додаткову вартість.

В книзі І, розділ XXII, ми бачили, що акумуляція, перетворення додаткової
вартости на капітал, своїм реальним змістом є процес репродукції
в поширеному маштабі, все одно, чи виявляється таке поширення
екстенсивно у вигляді долучення нових фабрик до старих, чи в інтенсивному
поширенні попереднього маштабу підприємства.

Розмір продукції може поширюватись малими дозами, оскільки
частину додаткової вартости застосовується на такі поліпшення, що або
тільки підвищують продуктивну силу вживаної праці, або разом з тим
дають змогу і визискувати її інтенсивніше. Або ж, коли робочий день не
обмежено законом, досить додаткової витрати обігового капіталу (на матеріяли
продукції та заробітну плату), щоб поширити розміри підприємства,
не збільшуючи основного капіталу, що його денний протяг вживання
таким чином лише подовжується, тимчасом як період обороту його
відповідно скорочується. Або, за сприятливих ринкових коньюнктур, капіталізована
додаткова вартість може дати змогу спекулювати на сировинному
матеріялі, отже, переводити такі операції, що для них не вистачило
б первісно авансованого капіталу, й т. ін.

А проте, очевидно, що там, де порівняно велике число періодів обороту
зумовлює частішу реалізацію додаткової вартости протягом року,
будуть наставати періоди, коли не можна буде ні подовжувати робочий
день, ні заводити частинні поліпшення; тимчасом як, з другого боку, пропорційне
поширення цілого підприємства, зумовлене почасти загальним
\parbreak{}  %% абзац продовжується на наступній сторінці
