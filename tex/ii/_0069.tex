\parcont{}  %% абзац починається на попередній сторінці
\index{ii}{0069}  %% посилання на сторінку оригінального видання
(додаткової вартости) на фонд акумуляції; або, якщо технічні пропорції
дозволяють це, процес продукції поширюється в більшому маштабі,
ніж це взагалі сталося б; або ж назбирується більший запас сировинного
матеріялу тощо.

Навпаки буває тоді, коли підвищується вартість елементів, що заміщують
товаровий капітал. Репродукція відбувається тоді вже не в нормальному
її розмірі (напр., працюють менше часу); або, щоб можлива
була дальша репродукція в старих розмірах, мусить виступити додатковий
капітал (зв’язування капіталу); або грошовий фонд акумуляції, якщо він
є, цілком або почасти придається на те, щоб провадити процес репродукції
в старих розмірах, замість поширювати його. Це теж є зв’язування
грошового капіталу, тільки тут додатковий грошовий капітал походить
не з-зозні, не з грошового ринку, а з засобів самого промислового
капіталіста.

Але при $П... П$, $Т'... Т'$ можуть трапитись обставини, що зумовлюють
ту або іншу зміну. Напр., коли наш прядун бавовни має великий
запас бавовни (тобто значна частина його продуктивного капіталу
перебуває в вигляді запасу бавовни), то частина його продуктивного
капіталу зневартнюється, коли падають ціни на бавовну; навпаки, коли ціни
на бавовну підвищуються, то підвищується вартість цієї частини його
продуктивного капіталу. З другого боку, коли він чималі маси вартости
зв’язав у формі товарового капіталу, напр., в бавовняній пряжі, то коли
падають ціни на пряжу, зневартнюється частина його товарового капіталу,
тобто, взагалі, частина його капіталу, що перебуває в кругобігу, а коли
ціни підвищуються, то буває навпаки. Нарешті, в процесі $Т' — Г — Т Р Зп$
маємо: коли акт $Т' — Г$, реалізація товарового капіталу, відбувається до
зміни вартости елементів Т, то капітал зазнає лише такого впливу, як у першому
випадку, а саме, в другому акті циркуляції $Г — Т Р Зп$; а коли
ця зміна відбувається до здійснення процесу $Т' — Г$, то за інших незмінних
обставин, спад цін на бавовну зумовлює відповідний спад цін на пряжу,
і навпаки, підвищення цін на бавовну зумовлює підвищення цін на пряжу.
Вплив цього на різні поодинокі капітали, вкладені у ту саму галузь
продукції, може бути дуже різний, залежно від різних обставин, що в них
вони можуть перебувати. — Звільнення і зв’язування грошового капіталу
можуть так само виникати в наслідок ріжниць у триванні процесу циркуляції,
отже, і в наслідок ріжниць у швидкості циркуляції. Однак це
стосується до вивчання обороту. Тут для нас важлива лише реальна
ріжниця між $Г — Г'$ і обома іншими формами процесу кругобігу,
ріжниця, що виявляється в них при зміні вартости елементів продуктивного
капіталу.

В добу вже розвиненого, а тому й домінантного капіталістичного
способу продукції, значна частина товарів у відділі циркуляції
\parbreak{}  %% абзац продовжується на наступній сторінці
