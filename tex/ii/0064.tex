цес кругобігу капіталу є повсякчасні перерви, вихід з однієї стадії, перехід
у наступну; скидання однієї форми, буття в другій формі; кожна з цих
стадій не лише зумовлює другу, але разом з тим і виключає її.

Але безперервність є характеристична ознака капіталістичної продукції
і зумовлюється технічною основою її, хоч не завжди безумовно досяжна.
Подивімось, як у дійсності стоїть справа. В той час, коли, напр., 10.000 ф.
пряжі виступають на ринок як товаровий капітал і перетворюються на
гроші (хоч ці гроші будуть виплатним засобом, хоч купівельним
засобом, або навіть розрахунковими грішми), місце їх заступає в продукційному
процесі нова бавовна, вугілля тощо, отже, пряжа тут знову
зворотно перетворилася з грошової й товарової форми на форму продуктивного
капіталу, і в цій формі починає свою функцію; тимчасом як у
той самий час перші 10.000 ф. пряжі перетворюються на гроші, раніш спродуковані
10.000 ф. пряжі вже перебігають другу стадію своєї циркуляції
i зворотно перетворюються з грошей на елементи продуктивного капіталу.
Всі частини каіпталу по черзі пророблюють процес кругобігу, перебувають
одночасно на різних стадіях його. Таким чином промисловий капітал
у своєму безперервному кругобігу перебуває одночасно на всіх стадіях
його й у відповідних їм різних функціональних формах. Для тієї частини,
яка вперше перетворюється з товарового капіталу на гроші, починається
кругобіг Т'... Т', тимчасом як для промислового капіталу, як для цілого,
що перебуває в русі, кругобіг Т'... Т' уже перейдено. Однією рукою гроші
авансується, другою одержується; початок кругобігу Г... Г' на одному
пункті є разом з тим його поворот на другому. Те ж саме має силу
й для продуктивного капіталу.

Справжній кругобіг промислового капіталу в своїй безперервності
є тому не лише єдність процесу циркуляції та процесу продукції, але
також і єдність усіх його трьох кругобігів. Але такою єдністю може він
бути лише остільки, оскільки кожна з різних частин капіталу може по
черзі переходити послідовні фази кругобігу, переходити з однієї фази,
з однієї функціональної форми в іншу, оскільки, отже, промисловий капітал,
як сукупність цих частин, одночасно перебуває в різних фазах
і функціях і таким чином одночасно пророблює усі три кругобіги. Чергування
кожної частини в часі зумовлюється тут чергуванням частин у
просторі, тобто поділом капіталу. Напр., в розчленованій фабричній
системі продукт завжди перебуває на різних ступенях процесу свого
творення й так само завжди переходить з однієї фази продукції до іншої,
А що індивідуальний промисловий капітал являє певну величину, що
залежить від засобів капіталіста й має для кожної галузі промисловости
певну мінімальну величину, то при поділі його мусять існувати певні числові
відношення. Величина наявного капіталу зумовлює розміри продукційного
процесу, його розміри зумовлюють розмір товарового й грошового
капіталу, оскільки вони функціонують поряд процесу продукції. Чергування
в просторі, що ним зумовлюється безперервність продукції, існує, однак,
тільки завдяки рухові частин капіталу, що в ньому вони одна по одній переходять
різні стадії. Чергування в просторі саме є лише результат чергу-
