\parcont{}  %% абзац починається на попередній сторінці
\index{ii}{0035}  %% посилання на сторінку оригінального видання
додаткової вартости в формі т (1560 ф. пряжі) — г (78 ф. стерл.) = т
(предмети споживання). Але елементи вартости кожної поодинокої частини
10.000  ф. продукту-пряжі можна так само позначити в частинах продукту,
як і елементи вартости цілого продукту. Так само як цей останній, 10.000 ф. пряжі, можна поділити на
вартість сталого капіталу (с), 7.440 ф.
пряжі вартістю в 372 ф. стерл., на вартість змінного капіталу (v), 1000 ф.
пряжі вартістю в 50 ф. стерл. і додаткову вартість (m), 1560 ф. пряжі
вартістю в 78 ф. стерл., — так само і всякий фунт пряжі можна поділити
на с = 11,904 унціям вартістю в 8,928 пенсів, на v = 1,600 унціям пряжі
вартістю в 1.200 пенсів і m = 2,496 унціям пряжі вартістю в 1,872 пенсів.
Капіталіст міг би також, поступінно продаючи 10.000 ф. пряжі, поступінно
споживати елементи додаткової вартости, що є в цих поодиноких частинах,
і так само поступінно реалізувти суму с + v. Але, кінець-кінцем, ця операція
припускає також, що продається всі 10.000 ф. пряжі і що, отже, в наслідок
продажу 8.440 ф. повернено вартість с і v. (Книга І, розділ VII, 2).

Але хоч що там, а в наслідок акту Т' — Г' капітальна вартість так
само, як і додаткова вартість, які є в Т', набирають відокремленого
існування, існування в різних грошових сумах; в обох випадках Г так
само, як і г, є справді перетворена форма тієї вартости, яка спочатку
в Т' лише як ціна товару мала власний лише ідеальний вираз.

т — г — т є проста товарова циркуляція, що її перша фаза т — г входить
у циркуляцію товарового капіталу Т' — Г', отже, в кругобіг капіталу;
навпаки, її додаткова фаза г — т\footnote*{
У німецькому тексті тут стоїть; т — г. Очевидна помилка. \emph{Ред.}
} перебуває поза цим кругобігом, як
відокремлений від нього акт загальної товарової циркуляції. Циркуляція
Т і т, капітальної вартости й додаткової вартости, розщеплюється після
перетворення Т' на Г'. Відси виходить:

По-перше. Після того, як актом Т' — Г' = Т' — (Г + г) реалізовано
товаровий капітал, постає можливість розрізнити рух капітальної
вартости й додаткової вартости, що в Т' — Г' був ще спільний і мав своїм
носієм ту саму товарову масу, бо тепер обидві ці вартості набирають
самостійних форм як грошові суми.

По-друге. Коли це відокремлення постає, і при цьому г витрачається
як дохід капіталіста, тимчасом як Г, як функціональна форма капітальної
вартости, і далі перебігає свій шлях, визначений кругобігом, — то перший
акт Т' — Г' у зв’язку з наступними актами Г — Т і г — т можна позначити,
як дві різні циркуляції: Т — Г — Т і т-г — т, як два ряди, що своєю
загальною формою належать до звичайної товарової циркуляції.

А, проте, на практиці, коли мають діло з товаровими тілами, що
їх не можна поділити, складові частини вартости відокремлюються
ідеально. Напр., у лондонських будівельних підприємствах, — а вони здебільше
працюють на основі кредиту, — підприємець одержує позики,
в міру того як будування будинку переходить від однієї стадії до іншої.
Жадна з цих стадій не є будинок, а тільки реальна наявна складова частина
майбутнього будовуваного будинку; отже, не зважаючи на свою реальність,
\parbreak{}  %% абзац продовжується на наступній сторінці
