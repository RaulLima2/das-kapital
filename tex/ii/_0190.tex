\parcont{}  %% абзац починається на попередній сторінці
\index{ii}{0190}  %% посилання на сторінку оригінального видання
з 100 до 75, або на одну чверть. Ціла сума, що на неї скорочується
продуктивний капітал, який функціонує протягом дев’ятитижневого робочого
періоду, становить 9 X 25 = 225 ф. стерл., або четверту частину
900 ф. стерл. Але відношення часу обігу до періоду обороту, як і раніш,
становить \sfrac{3}{12} = \sfrac{1}{4}. З цього випливає ось що. Для того, щоб продукція
не припинялась протягом часу обігу продуктивного капіталу, перетвореного
на товаровий капітал, щоб вона однаково невпинно продовжувалась тиждень
у тиждень, коли немає для цього окремого обігового капіталу, то
цього можна досягти, лише скоротивши продукцію, зменшивши поточну
складову частину діющого продуктивного капіталу. Поточна частина капіталу,
звільнена таким чином для процесу продукції протягом часу обігу,
відноситься до цілого авансованого поточного капіталу, як час обігу до
періоду обороту. Як ми вже зауважили, це має силу тільки для тих галузей
продукції, де процес праці відбувається тиждень-у-тиждень у
тому самому маштабі, де, отже, не треба, як у хліборобстві, в різні робочі
періоди витрачати різні кількості капіталу.

Навпаки, якщо ми припустимо, що самий характер підприємства виключає
можливість скорочення маштабу продукції, а тому й розмірів щотижнево
авансовуваного поточного капіталу, то безперервности продукції
можна досягти лише додачею поточного капіталу, в наведеному вище
прикладі додачею 300 ф. стерл. Протягом дванадцятитижневого періоду
обороту послідовно авансується 1200 ф. стерл., з них 300 являють четверту
частину, як 3 тижні від 12. По скінченні 9-тижневого робочого
періоду капітальна вартість в 900 ф. стерл. перетворюється з форми
продуктивного капіталу на форму товарового капіталу. Її робочий період
закінчено, але його не можна відновити з тим самим капіталом. Протягом
трьох тижнів, поки цей капітал перебуває в сфері циркуляції, функціонуючи
як товаровий капітал, він щодо продукційного процесу перебуває
в такому самому стані, ніби його взагалі не існувало. Ми лишаємо тут
осторонь усі кредитові відносини, а тому припускаємо, що капіталіст
господарює тільки своїм власним капіталом. Але тимчасом як капітал,
авансований на перший робочий період, завершивши процес продукції,
протягом 3 тижнів перебуває в процесі циркуляції, — у цей самий час
функціонує додатково витрачений капітал в 300 ф. стерл., так що безперервність
продукції не порушується.

Тут треба зауважити ось що:

Поперше. Робочий період авансованого спочатку капіталу в 900 ф.
стерл. закінчується по 9 тижнях, але капітал припливає назад не раніш,
як по трьох тижнях, отже, лише на початку 13-го тижня. Однак новий
робочий період починається негайно за допомогою додаткового капіталу
в 300 ф. стерл. Саме в наслідок цього підтримується безперервність
продукції.

Подруге. Функції первісного капіталу в 900 ф. стерл. і новододаного
наприкінці першого дев’ятитижневого робочого періоду капіталу в 300 ф.
стерл, який відкриває другий робочий період безпосередньо по закінченні
\parbreak{}  %% абзац продовжується на наступній сторінці
