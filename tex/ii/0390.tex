лістичній основі маса товарового капіталу й величина його вартости не
лише абсолютно більша, але й зростає з куди більшою швидкістю;
3) дедалі більший змінний капітал завжди мусить перетворюватись на
грошовий капітал; 4) бо рівнобіжно з поширенням продукції утворюються
нові капітали, отже, мусить бути наявний і матеріял для їхнього нагромадження
в формі скарбу. — Якщо це має силу для першої фази капіталістичної
продукції, коли і кредитова система супроводиться переважно
металевою циркуляцією, то й для найрозвиненішої фази кредитової
системи це має силу остільки, оскільки її базою лишається металева
циркуляція. З одного боку, додаткова продукція благородних металів
може тут, оскільки вона навперемінку буває буйніша або бідніша,
викликати порушення в товарових цінах не лише протягом довгих, а й
в межах дуже коротких періодів часу; з другого боку, ввесь кредитовий
механізм постійно дбає про те, щоб всілякими операціями, методами,
технічними засобами обмежити справжню металеву циркуляцію відносно
дедалі меншим мінімумом, наслідком чого відповідно збільшується також
штучність цілого механізму й шанси на порушення нормального його
перебігу.

Різні В, В', В'' і т. д. (І), що їхній віртуальний новий грошовий
капітал вступає в операції як активний капітал, можуть купувати один в одного
і продавати один одному свої продукти (частини свого додаткового
продукту). При нормальному перебігу справ гроші, авансовані на циркуляцію
додаткового продукту, pro tanto повертаються назад до різних
В, В' і т. д, в такій самій пропорції, в якій кожен з них авансував
ці гроші на циркуляцію своїх відповідних товарів. Коли гроші циркулюють
як виплатний засіб, то тут доводиться виплачувати лише ріжницю,
оскільки взаємні купівлі й продажі не покривають одна одну. Але важливо
всюди, як ми це робимо тут, припустити спочатку металеву циркуляцію
в її найпростішій, найпервіснішій формі, бо тоді приплив
і відплив грошей, вирівнювання ріжниць, коротко кажучи, всі моменти,
які з’являються в кредитовій системі, як свідомо урегульовані
процеси, виступлять як наявні незалежно від кредитової системи, і
вся справа виявиться в своїй природній формі, а не в пізнішій,
відображеній.

3) Додатковий змінний капітал

Що до цього часу мова йшла тільки про додатковий сталий капітал,
то тепер маємо перейти до розгляду додаткового змінного
капіталу.

В книзі І *) докладно з’ясовано, як на основі капіталістичної продукції
завжди є в запасі робоча сила і як, в разі потреби, можна пустити
в рух більше праці, не збільшуючи числа вживаних робітників або маси
робочої сили. Тому покищо не треба далі зупинятись на цьому, навпаки

*) Див. „Капітал“, т. I, розділ 23, § 3. — Ред.
