все ж в річній нормі додаткової вартости капіталів А і В є ріжниця в
900%.

Правда, це явище має такий вигляд, ніби норма додаткової вартости
залежить не лише від маси та ступеня експлуатації робочої сили,
пущеної в рух змінним капіталом, але крім того, від якихось незрозумілих
впливів, що походять з процесу циркуляції; і справді, це явище
освітлювали саме таким способом; і хоч не в цій чистій, а в своїй складнішій
та прихованішій формі (в формі річної норми зиску) воно спричинило
початку 20-х років цілковите замішання в школі Рікардо.

Дивне в цьому явищі зникає одразу, скоро ми не лише позірно, а
й справді поставимо капітал А і капітал В в цілком однакові обставини.
Обставини будуть однакові тільки тоді, коли змінний капітал В в цілому
своєму об’ємі витрачається на оплату робочої сили протягом того
самого часу, що й капітал А.

5000 ф. стерл. капіталу В витрачається тоді протягом 5 тижнів, по
1000 ф. стерл. щотижня; це становить за рік витрату в 50000 ф. стерл.
Додаткова вартість буде тоді, згідно з нашим припущенням, теж = 50000
ф. стерл. Капітал, що обернувся = 50000 ф. стерл., поділений на авансований
капітал = 5000 ф. стерл., дає число оборотів = 10. Норма додаткової
вартости = 5000 m / 5000 v = 100%, помножена на число оборотів = 10, дає річну норму додаткової
вартости = 5000 m / 5000 v = 10/1 = 1000%. Отже, тепер річні норми додаткової вартости однакові для
А і для В, а саме
1000%, але маси додаткової вартости становлять: для В — 50000 ф. стерл.,
для А — 5000 ф. стерл.; маси спродукованої додаткової вартости відносяться
тепер, як авансовані капітальні вартості В і А, а саме як 5000 : 500 =
— 10 : 1. Але зате капітал В в той самий час пустив у рух удесятеро
більше робочої сили, ніж капітал А.

Тільки капітал, дійсно застосований у процесі праці, утворює додаткову
вартість і тільки для нього мають силу всі закони, що стосуються
до додаткової вартости, а значить, і той закон, що за даної норми маса
додаткової вартости визначається відносною величиною змінного капіталу.

Самий процес праці вимірюється часом. За даної довжини робочого
дня (як тут, де ми ставимо капітал А і капітал В в однакові обставини,
щоб висвітлити краще різницю в річній нормі додаткової вартости)
робочий тиждень складається з певного числа робочих днів. Або ми
можемо розглядати якийбудь робочий період, напр., в даному разі п’ятитижневий,
як суцільний робочий день, що складається з 300 годин, коли
робочий день = 10 годинам, а тиждень = 6 робочим дням. Але далі ми
мусимо помножити це число на число робітників, що їх одночасно щодня
вживається разом у тому самому процесі праці. Коли це число
було б, напр., 10, то тижневий підсумок був би = 60 х 10 = 600 годинам,
а п’ятитижневий робочий період = 600 х 5 = 3000 годинам. Отже,
при однаковій нормі додаткової вартости і при однаковій довжині ро-
