відався від Ляссаля, що є також економіст Родбертус, і потім знайшов
його „Третій соціяльний лист“ у Британському музеї.

Такі фактичні обставини. А як стоїть справа з тим змістом, що його
ніби „украв“ Маркс у Родбертуса? „Відки виникає додаткова вартість
капіталіста, — каже Родбертус, — це я показав у моєму третьому соціяльному
листі так само, як і Маркс, тільки коротше та виразніше“. Отже,
ось де центральний пункт: теорія додаткової вартости; і справді, не
можна сказати, на що інше міг би Родбертус претендувати з Маркса,
як на свою власність. Таким чином, Родбертус виголошує тут себе за
справжнього автора теорії додаткової вартости, що її викрав у нього Маркс.

Що ж каже нам третій соціяльний лист про постання додаткової
гартости? Він каже просто, що „рента“, — а Родбертус має на увазі тут
разом і земельну ренту і зиск, — постає не з „додатку вартости“ до
вартости товарів, але „в наслідок віднімання вартости, що його зазнає заробітна
плата, або, інакше кажучи, в наслідок того, що заробітна плата
являє лише частину вартости продукту“, а за достатньої продуктивности
праці „немає потреби, щоб вона дорівнювала природній міновій вартості
її продукту для того, щоб від неї лишалася ще частина на покриття
капіталу (!) й на ренту“ *. При цьому нам не кажуть, що являє собою
ця „природна мінова вартість“ продукту, що при ній нічого не залишається,
на „покриття капіталу“, а, значить, і на покриття сировинного
матеріялу та зношування знарядь праці.

На щастя, нам припадає констатувати, яке враження справило це епохальне
відкриття Родбертусове на Маркса. В рукопису „Zur Kritik“ etc., в зшитку X,
на стор. 445 і далі, читаємо ми: „Відхилення. Пан Родбертус. Нова теорія
земельної ренти“. Тільки з цього погляду розглядається тут третій соціяльний
лист. З Родбертусовою теорією додаткової вартости справу взагалі
закінчено таким іронічним зауваженням: „Пан Родбертус спочатку досліджує
стан речей у країні, де не відокремлено посідання землею від посідання
капіталом, і доходить потім важливого висновку, що рента [а її
він розуміє, як цілу додаткову вартість] просто дорівнює неоплаченій
праці або кількості продуктів, що в ній цю працю втілено.“

Капіталістичне людство вже протягом багатьох століть продукувало
додаткову вартість і помалу дійшло того, що почало замислюватись над
її постанням. Перший погляд виник із безпосередньої купецької
практики: додаткова вартість постає з додатку до вартости продукту.
Цей погляд панував серед меркантилістів, але вже Джемс Стюарт побачив,
що при цьому те, що один виграє, другий неминуче втрачає. Не зважаючи
на це, цей погляд тримався ще й далі довгий час, особливо серед
соціалістів; але з клясичної науки його витиснув А. Сміс.

У „Багатстві народів”, кн. І, розділ VI, він каже: „Скоро капітал
(stock) нагромадився в руках поодиноких осіб, деякі з них звичайно засто-

* Цитовані місця є в Родбєртуса, а саме в „Soziale Briefe an von Kirchmann,
Dritter Brief“, Berlin, 1851, S. 87. (Примітка Кавтського до Volksausgabe. 1926 p.).
Ред.
