\parcont{}  %% абзац починається на попередній сторінці
\index{ii}{0349}  %% посилання на сторінку оригінального видання
Отже, ці гроші є грошова форма частини сталої капітальної взртости,
її основної частини. Отже, це утворення скарбу саме є елемент капіталістичного
процесу репродукції, є репродукція і нагромадження — в грошовій
формі — вартости основного капіталу або його поодиноких елементів,
до того моменту, коли основний капітал відживе свій вік і, значить, передасть
свою вартість спродукованим товарам, після чого його доводиться
замістити in natura. Але ці гроші, скоро їх знову перетворено на нові
елементи основного капіталу, щоб замістити елементи, які віджили свій
вік, втрачають лише свою форму скарбу й тому лише знову активно
входять у процес репродукції капіталу, упосереднюваний циркуляцією.

Як проста товарова циркуляція не тотожня з простим обміном продуктів,
так і перетворення річного товарового продукту не можна звести на
простий, безпосередній, взаємний обмін його різних складових частин.
Гроші відіграють у ньому специфічну ролю, яка виявляється і в способі
репродукції основної капітальної вартости. (Далі треба буде дослідіти,
який це мало б інший вигляд, коли припустити, що продукція колективна
й не має форми товарової продукції).

Тепер, повертаючись до основної схеми, ми маємо для кляси II:
2000 с + 500 v + 500 m. Всі засоби споживання, спродуковані протягом
року, дорівнюють тут вартості в 3000; і кожен з різних елементів товару,
що з них складається ця сума товару, розкладається за вартістю своєю
на 2/3 с + 1/6 v + 1/6 m, або у відсотках на 66 2/3 с + 16 2/3 v + 16 2/3 m.
Різні ґатунки товарів кляси II можуть мати в собі сталий капітал у
різних пропорціях; основна частина сталого капіталу в них так само
може бути різна; так само і протяг життя основних частин капіталу, а
значить, і річне зношування або та частина вартости, яку вони pro rata
переносять на товари, вироблювані за їх допомогою. Все це тут не має
значення. Щодо суспільного процесу репродукції, то вся справа лише в
обміні між клясами II і I. II і І протистоять тут один одному лише в
їхніх суспільних масових відношеннях; тому пропорційна величина частини
вартости с товарового продукту II (а тільки вона й має міродайне
значення для розглядуваного тепер питання) є пересічне відношення, коли
зробити загальний підсумок усіх галузей продукції, що входять у II.

Таким чином, кожен з товарових ґатунків (а це здебільша ті самі
ґатунки товарів), що їхню загальну вартість підсумовано в 2000 с + 500 v +
500 m, однаково дорівнює своєю вартістю 66 2/3\% с + 16 2/3\% v +
16 2/з\% m. Це має силу для всяких 100 одиниць товарів, хоч фігурують
вони під с, хоч під v, хоч під m.

Товари, що в них втілено 2000 с, теж можна розкласти за їхньою
вартістю на:

1) 1333 1/3с + 333 1/3 v + 333 1/3m = 2000c; так само 500 v на:

2) 333 1/3 с + 83 1/3 v + 83 1/3 m = 500 v; нарешті, 500 m на:

3) 333 1/3с + 83 1/3 v + 8З 1/3 m = 500 m;

Тепер, коли ми складемо с, що є в 1, 2 і 3, то матимемо 1333 1/3 с
+ 333 1/Зс + 333 1/3 с = 2000. Так само 333 1/3v + 83 1/3v + 83 1/3v
\index{ii}{0350}  %% посилання на сторінку оригінального видання
= 500, і те саме з m; склавши всі ці величини, матимемо, як і раріше,
сукупну вартість в 3000.

Отже, вся стала капітальна вартість, що міститься в масі товарів II
вартістю в 3000, міститься в 2000 с, і ні 500 v, ні 500 m не мають
жодного атома цієї вартости. Це саме має силу також і для v, і для m.

Інакше кажучи: вся та кількість товарової маси II, яка репрезентує
сталу капітальну вартість і тому має знову бути перетворена — хоч
на її натуральну, хоч на грошову форму — існує в 2000 с. Отже, все, що
стосується до обміну сталої вартости товарів II, обмежується рухом 2000
II с; і цей обмін можливий тільки на І (1000 v + 1000 m).

Так само для кляси І все, що стосується до обміну належної йому
капітальної вартости, треба обмежити розглядом 4000 І с.

1) Заміщення в грошовій формі зношуваної частини вартости.

Тепер, коли ми візьмемо насамперед:

І. 4000 с + 1000 v + 1000 m
II..............  2000 с + 500 v + 500 m

то обмін товарів 2000 IIc на товари такої самої вартости І (1000 v +
1000 m) припускав би, що 2000 IIс in natura цілком знову перетворюється
на спродуковані підрозділом І натуральні складові частини сталого
капіталу II; але товарова вартість 2000, в якій існує цей капітал,
містить у собі елемент, що покриває втрату вартости основного капіталу,
який не одразу треба заміщати in natura, а перетворювати на гроші, поступінно
нагромаджувані в цілу суму, поки надійде час відновити
основний капітал у його натуральній формі. Кожен рік є рік смерти для
основного капіталу й доводиться його заміщувати то в цьому, то в тому
поодинокому підприємстві, або навіть то в тій, то в цій галузі промисловості;
в тому самому індивідуальному капіталі доводиться заміщувати
ту або іншу частину основного капіталу (бо частини його мають різний
протяг життя). Розглядаючи річну репродукцію — хоча б і в незміненому
маштабі, тобто абстрагуючись від усякої акумуляції — ми починаємо не
ab ovo\footnote*{
Ab ovo — латинський вираз, дослівно „від яйця“, тобто з самого виникнення,
з самого початку. Ред.
}; ми беремо один рік з ряду багатьох, не перший рік по
народженні капіталістичної продукції. Отже, різні капітали, вкладені
в різні галузі продукції кляси II, мають різний вік, і подібно до того,
як щороку вмирають люди, які функціонують у цих галузях продукції,
так само маси основних капіталів щороку доживають свого віку, й їх доводиться
відновлювати in natura з нагромадженого грошового фонду. В
цьому розумінні обмін 2000 II с на 2000 І (v + m) включає перетворення
2000 II с з його товарової форми (засобів споживання) на натуральні
елементи, що складаються не лише з сировинних та допоміжних елемен-
\parbreak{}  %% абзац продовжується на наступній сторінці
