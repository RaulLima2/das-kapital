бочого дня застосовується змінні капітали однакової величини, коли протягом
того самого переміжку часу пускається в рух однакові маси робочої
сили (обчислювані помноженням однієї робочої сили тієї самої
ціни на число цих сил).

Повернімось тепер до наших первісних прикладів. В обох випадках
А і В протягом кожного тижня року застосовується змінні капітали
однакової величини, по 100 ф. ст. щотижня. Застосовані, справді діющі
в процесі праці змінні капітали тому однакові, але авансовані змінні
капітали зовсім неоднакові. В прикладі А на кожні 5 тижнів авансовано по
500 ф. стерл., що з них щотижня застосовується 100 ф. стерл. В прикладі
В на перший п’ятитижневий період треба авансувати 5000 ф. стерл., але
з них застосовується лише по 100 ф. стерл. щотижня, отже, протягом
5 тижнів лише 500 ф. стерл. = 1/10 авансованого капіталу. Протягом
другого п’ятитижневого періоду треба авансувати 4500 ф. стерл., але застосовується
тільки 500 ф. стерл. і т. далі. Змінний капітал, авансовуваний
на певний період часу, перетворюється на застосовуваний, тобто справді
діющий і чинний змінний капітал лише тією мірою, якою він справді входить
у відділи цього періоду часу, заповнені процесом праці, якою він
дійсно функціонує в процесі праці. В переміжки, що протягом них
частину його авансовано лише для того, щоб її можна було застосувати
пізніше, ця частина мов би зовсім не існує для процесу праці, а тому
не справляє жодного впливу ні на утворення вартости, ні на утворення
додаткової вартости. Так стоїть, приміром, справа з капіталом А в
500 ф. стерл. Його авансовано на 5 тижнів, але в процес праці послідовно
входять з нього щотижня лише 100 ф. стерл. Протягом першого тижня
застосовується 1/5 його; 4/5 авансовано, але не застосовано, хоч їх і
треба мати в запасі для процесу праці наступних 4 тижнів, і тому їх
доводиться авансувати.

Обставини, що зумовлюють ріжницю у відношенні між авансованим і
застосованим капіталом, впливають на продукцію додаткової вартости —
за даної норми додаткової вартости — лише остільки й лише тим, що
вони роблять різною ту кількість змінного капіталу, яку дійсно можна
застосувати протягом певного періоду часу, напр., протягом одного тижня,
протягом п’ятьох тижнів та ін. Авансований змінний капітал функціонує
як змінний капітал лише остільки й лише протягом того часу, оскільки й
коли його справді застосовується; але не протягом того часу, коли він
лишається авансований як запас, і не застосовується його. Однак усі
обставини, що зумовлюють ріжницю у відношенні між авансованим і застосованим
змінним капіталом, сходять на ріжницю періодів обороту
(визначувану ріжницею або робочого періоду, або періоду циркуляції,
або їх обох). Закон продукції додаткової вартости, є в тому, що, при однаковій
нормі додаткової вартости, однакові маси діющого змінного капіталу
утворюють однакові маси додаткової вартости. Отже, коли з капіталів
А і В за однакові переміжки часу при однаковій нормі додаткової
вартости застосовується однакові маси змінного капіталу, то вони мусять
протягом однакових переміжків часу утворити однакові маси додаткової
