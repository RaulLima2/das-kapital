50000/2 = 25000 дол. на 10 років = 2.500 дол. на 1 рік

50000/4 = 12.500 „„    2 „= 6.250 дол. на 1    рік

50000/4 = 12.500 „„    1/2 „= 25.000 „„ „„ / На 1 рік = 33.750 дол.

Отже, пересічний час, що протягом його ввесь капітал обертається
один раз, становить 16 місяців*. Візьмімо другий приклад. Хай чверть
усього капіталу в 50.000 доларів обертається протягом 10 років; друга
чверть — протягом року; і решта — половина — двічі на рік. Тоді річні витрати
будуть такі:

12.500/10 = 1.250 долярів

12.500 = 12.500 долярів.

25.000X2 = 50.000 долярів.
Протягом 1 року обернулось = 63.750 долярів.
(Scrope „Роl. Econ.“, edit. Alonzo Potter. New-York, 1841, р. 141, 142).

6) Справжні й позірні відмінності в обороті різних частин капіталу. —
Той самий Скроп каже там само: „Капітал, що його фабрикант, сільський
господар або купець витрачає на видачу заробітної плати, циркулює якнайшвидше,
бо він, коли робітникам платиться раз на тиждень, обертається,
може, раз на тиждень в наслідок щотижневих надходжень за
продані товари або оплачені рахунки. Капітал, вкладений в сировинний
матеріял або готові запаси, циркулює з меншою швидкістю; він може
обернутись два або чотири рази на рік, залежно від того, скільки
часу минає між закупові матеріялів і продажем товарів, — ми припускаємо,
що кредит на закуп і продаж дається на однаковий термін. Капітал, вкладений
в знаряддя й машини, циркулює ще повільніше, бо він протягом
5 або 10 років пересічно, може, зробить один оборот, тобто — його зуживеться
і поновиться, хоч деякі знаряддя вже зуживеться й по
небагатьох операціях. Капітал, вкладений в споруди, напр., в фабрики,
крамниці, склади, амбари, брук, зрошувальні споруди, тощо, як здається,
взагалі не циркулює. А в дійсності й ці споруди, відіграючи
свою ролю в продукції, зношуються цілком так само, як і вище згадані,
і їх треба репродукувати, щоб продуцент міг далі продовжувати
свої операції. Ріжниця лише в тому, що їх зуживається й репродукується
повільніше, ніж інші... Вкладений в них капітал робить, може,
один оборот протягом 20 або 50 років“.

Скроп сплутує тут ту ріжницю в русі певних частин поточного
капіталу, до якої призводять — щодо поодинокого капіталіста — терміни
виплат і кредитові відносини, з тією ріжницею оборотів, яка випливає

* В обчисленні є помилка. Пересічний час, що протягом його обертається ввесь
капітал, становить не 16 місяців, а 17,16 місяців. Ред.
