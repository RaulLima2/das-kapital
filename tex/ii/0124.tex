повинно покриватися з додаткової вартости і становить одбаву з неї. Або, розглядаючи справу з
суспільного погляду: потрібна завжди перепродукція, тобто продукція в ширших розмірах, ніж треба на
звичайне заміщення та репродукцію наявного багатства; це потрібно — зовсім лишаючи осторонь приріст
людности — для того, щоб мати напоготові засоби продукції на покриття тієї незвичайної руйнації, що
постала під впливом випадковостей і природних сил.

В дійсності лише невеличка частина капіталу, потрібного для заміщення, являє грошовий резервний
фонд.

Найважливішу частину являє поширення розмірів самої продукції, а воно почасти є дійсне поширення, а
почасти належить до нормального розміру в тих галузях продукції, де продукується основний капітал.
Так, капр., машинобудівельну фабрику будують з тим розрахунком, що фабрики покупців її щороку
поширюватимуться, а також, що частина їх потребуватиме завжди цілковитої або частинної репродукції.

При визначенні суспільної пересічної як для зношування, так і для витрат на ремонт неминуче
виявляються великі відмінності навіть для рівновеликих капіталів, взагалі вкладених за однакових
обставин у ту саму галузь промисловости. На практиці в одного капіталіста машина і т. ін. триває
понад пересічний час, а в другого не так довго. Витрати на ремонт в одного вищі, в іншого нижчі від
пересічних і т. ін. Але накидка до ціни товарів, визначувана зношуванням та витратами на ремонт, є
та сама й визначається за пересічною величиною. Таким чином в наслідок цієї накидки до ціни один
одержує більш, ніж він дійсно додатково витратив, а інший — менше. Це, як і всі інші обставини, що,
не зважаючи на однакову експлуатацію робочої сили, роблять різним зиск різних капіталістів у тій
самій галузі продукції, призводить до того, що утруднюється розуміння справжньої природи додаткової
вартости.

Межа між власне ремонтом і заміщенням, між витратами на зберігання й витратами на поновлення
більш-менш нестала. Відси постійні суперечки, напр., в залізничній справі про те, чи певні витрати
являють ремонт, чи заміщення, чи треба покрити їх з поточних видатків, чи з основного капіталу.
Перенесення ремонтних витрат до рахунку капіталу замість перенесення їх до рахунку доходу є відомий
засіб, що ним управління залізниць штучно підвищують свої дивіденди. Все ж і тут досвід вже дав
посутні пункти підпори. Напр., додаткові роботи протягом першого періоду життя залізниці є „зовсім
не ремонт, і їх треба розглядати як посутню складову частину будування залізниці, отже, їх треба
залічувати на рахунок капіталу, бо вони випливають не із зношування або з нормального впливу руху, а
виникли в наслідок первісної та неминучої недосконалости залізничної будови“ (Lardner, 1. с., р.
40). „Навпаки, єдино правильний метод той, щоб на доходи кожного року покладати зневартнення,
неминуче зв’язане з тим, щоб заслужити ці доходи, все одно, чи витрачено дану суму, чи ні“ (Captain
Fitzmaurice, Committee
