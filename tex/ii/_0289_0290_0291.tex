\parcont{}  %% абзац починається на попередній сторінці
\index{ii}{0289}  %% посилання на сторінку оригінального видання
двоїстого характеру самої праці: праці, оскільки вона як витрата робочої сили утворює вартість, і
оскільки вона як конкретна корисна праця утворює предмети споживання (споживну вартість). Загальна
сума виготовлених протягом року товарів, тобто ввесь річний продукт, є продукт корисної праці, яка
діяла протягом останнього року; всі ці
товари існують лише в наслідок того, що суспільно застосовану працю витрачено в різноманітно
розгалуженій системі різних видів корисної праці; тільки тому в їхній сукупній вартості вартість
засобів продукції, зужиткована на їх продукцію, збереглася, знову з’явившись в новій натуральній
формі. Отже, ввесь річний продукт є результат корисної праці, витраченої протягом року. Але протягом
року утворюється знову лише деяка частина вартости річного продукту; ця частина є новоспродукована
річна вартість, що в ній втілено суму праці, пущеної в рух протягом даного року.

Отже, коли А. Сміс у щойно цитованому місці каже: „Річна праця кожної нації є той фонд, що первісно
дає їй усі засоби існування, які вона споживає протягом року і т. ін.“, то він однобічно стає на
погляд просто корисної праці, яка, щоправда, надала всім цим засобам існування форму придатну для
споживання. Але він забуває при цьому, що це
було б неможливо без участи засобів праці й предметів праці, переданих від минулих років, і що в
наслідок цього „річна праця“, оскільки вона утворювала вартість, ні в якому разі не утворила всієї
вартости виготовленого нею продукту; він забуває, що новопродукована вартість менша, ніж вартість
продукту.

Коли А. Смісові й не можна закинути, що в цій аналізі він ішов лише до тих меж, як і всі його
наслідувачі (хоч спробу правильно розв’язати питання маємо вже в фізіократів), то все ж треба
сказати, що далі він губиться в хаосі, головним чином, тому, що „езотеричне“ розуміння товарової
вартости в нього взагалі завжди переплітається з екзотеричним, а це останнє в нього здебільша й
переважає, тимчасом як його науковий інстинкт час від часу знову й знов приводить його до
езотеричного погляду.

\subsubsection{Капітал і дохід у А. Сміса}

Частина вартости кожного товару (а тому й річного продукту), яка становить лише еквівалент
заробітної плати, дорівнює капіталові, авансованому капіталістом на заробітну плату, тобто дорівнює
змінній складовій частині цілого авансованого ним капіталу. Цю складову частину авансованої
капітальної вартости капіталіст одержує назад через новоспродуковану складову частину вартости
товару, виробленого найманими робітниками. Хоч авансується змінний капітал в тому розумінні, що
капіталіст виплачує грішми ту частину ще неготового для продажу продукту, яка припадає робітникові,
або хоч готового, але ще не проданого капіталістом; хоч платить він робітникові грішми, вже
одержаними від продажу виробленого робітниками товару, хоч він за допомогою кредиту антиципував
\index{ii}{0290}  %% посилання на сторінку оригінального видання
ці гроші, — в усіх цих випадках капіталіст витрачає змінний капітал, що допливає робітникам
у вигляді грошей, і володіє, з другого боку, еквівалентом цієї капітальної вартости в вигляді
частини вартости своїх товарів, що в ній робітник знову спродукував ту частину цілої вартости, яка
припадає йому самому, інакше кажучи, ту, що в ній він спродукував вартість своєї власної заробітної
плати. Замість дати робітникові цю частину вартости в натуральній формі його власного продукту,
капіталіст виплачує йому її грішми. Отже, для капіталіста змінна складова частина
авансованої ним капітальної вартости існує тепер у товаровій формі, тимчасом як робітник одержав
еквівалент за продану ним робочу силу в грошовій формі.

Отже, в той час як частина авансованого капіталістом капіталу, перетворена закупом робочої сили на
змінний капітал, функціонує в самому процесі продукції як діюща робоча сила, в той час як
витрачанням цієї сили частину цю знову продукується, тобто репродукується в товаровій формі як нову
вартість, — отже, відбувається репродукція, тобто нова продукція авансованої капітальної вартости! —
робітник витрачає вартість,
зглядно ціну, своєї проданої робочої сили на засоби існування, на засоби репродукції своєї робочої
сили. Сума грошей, рівна змінному капіталові, становить його дохід, отже, дохід, що триває лише
доти, доки він може продавати свою робочу силу капіталістам.

Товар найманого робітника, — сама його робоча сила — функціонує як товар лише остільки, оскільки її
долучається до капіталу капіталіста, оскільки вона функціонує як капітал; з другого боку, капітал
капіталіста, витрачений як грошовий капітал на закуп робочої сили, функціонує як дохід в руках
продавця робочої сили в руках найманого робітника.

Тут переплітаються різні процеси циркуляції та продукції, що їх А. Сміс не розмежовує.

Поперше. Акти, що належать до процесу циркуляції: робітник продає свій товар — робочу силу —
капіталістам; гроші, що на них капіталіст купує її, є для нього гроші, вкладені для збільшення їх
вартости, отже, грошовий капітал; капітал цей не витрачено, а лише авансовано. (В цьому справжнє
значення „авансування“ — avance фізіократів — цілком незалежно від того, відки сам капіталіст бере
гроші. Для капіталіста буде авансованою кожна вартість, що її він сплачує для процесу продукції,
незалежно від того, чи буде це до чи post festum; її авансовано самому процесові продукції). Тут
відбувається лише те, що при всякому продажу товарів: продавець віддає споживну вартість (в даному
разі робочу силу) і одержує її вартість (реалізує її ціну) в грошах; покупець віддає свої гроші й
одержує натомість самий товар — в даному разі робочу силу.

Подруге. В цроцесі продукції куплена робоча сила являє тепер частину діющого капіталу, а сам
робітник функціонує тут лише як особлива натуральна форма цього капіталу, відмінна від тих його
елементів, що існують у натуральній формі засобів продукції. Протягом процесу продукції робітник до
засобів продукції, що їх він перетворює на продукт, долучає витратою своєї робочої сили вартість,
рівну вартості його робочої
\index{ii}{0291}  %% посилання на сторінку оригінального видання
сили (лишаючи осторонь додаткову вартість); отже, він репродукує капіталістові в товаровій
формі ту частину капіталу, що ЇЇ капіталіст йому авансував або має авансувати як заробітну плату;
продукує йому еквівалент цієї плати; отже, він продукує капіталістові капітал, що його капіталіст
може знову „авансувати“ на закуп робочої сили.

Потрете. При продажу товару частина його продажної ціни повертає капіталістові авансований ним
змінний капітал і цим дає йому змогу знову купувати робочу силу, а робітникові — знову продавати її.

При всіх актах купівлі й продажу товарів — оскільки розглядається лише ці оборудки — цілком байдуже,
що зробить продавець з уторгованими за свій товар грішми, і що зробить покупець з купленими
предметами споживання. Отже, оскільки розглядається лише процес циркуляції, цілком не має значення
також та обставина, що куплена капіталістом робоча сила репродукує йому капітальну вартість, і що, з
другого боку,
гроші, вторговані як купівельна ціна робочої сили, становлять дохід робітника. На величину вартости
предмета торгівлі робітника, його робочої сили, не впливає ані те, що вона становить його „дохід“,
ані те, що споживання цього його предмета торгівлі покупцем репродукує цьому покупцеві капітальну
вартість.

Через те, що вартість робочої сили, — тобто адекватна продажна ціна цього товару — визначається
кількістю праці, потрібного для її репродукції, а саму цю кількість праці визначається тут тією
кількістю праці, яка потрібна для продукції потрібних засобів існування робітника, отже, кількістю
праці, потрібного для підтримання його життя, заробітна плата стає доходом, що з нього має жити
робітник.

Цілком неправильне твердження А. Сміса (стор. 223): „Частина капіталу, витрачена на утримання
продуктивної праці, після того як вона служила йому [капіталістові] в функції капіталу\dots{} становить
їх [робітників] дохід“. Гроші, що ними капіталіст оплачує куплену ним робочу силу „служать йому в
функції капіталу“, оскільки він за допомогою їх долучав робочу силу до речових складових частин
свого капіталу і тільки цим взагалі ставить свій капітал в умови, що в них він може функціонувати як
продуктивний капітал. Треба відрізняти таке: робоча сила в руках робітника є товар, а не капітал;
вона становить для нього дохід остільки, оскільки він може постійно повторювати її продаж; вона
функціонує як капітал після продажу в руках капіталіста, підчас самого процесу продукції. Робоча
сила служить тут подвійно; в руках робітника, як товар, продаваний за його вартістю; в руках
капіталіста, що купив її, як сила, що продукує вартість і споживну вартість. Але гроші, що їх
одержує робітник від капіталіста, він одержує лише після того, як він дав йому вжиток своєї робочої
сили, після того, як вона вже реалізована в вартості продукту праці. Капіталіст має цю вартість у
своїх руках,
перш ніж оплатить її. Отже, не гроші є те, що двічі функціонує: спочатку як грошова форма змінного
капіталу, а потім як заробітна плата. Двічі функціонує робоча сила: поперше, як товар при продажу
робочої сили (при визначенні розміру заробітної плати, що її треба
\parbreak{}  %% абзац продовжується на наступній сторінці
