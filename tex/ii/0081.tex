мусить проробити. Вартість апаратів і т. ін. переноситься на продукт
відповідно всьому тому часові, що протягом його вони функціонують;
продукт вводиться в цю стадію самою працею, і вживання цих апаратів
є так само умова продукції, як і розпорошування частини
бавовни, яка не входить у продукт, але все ж    переносить на нього
свою вартість. Друга частина латентного капіталу, напр., будівлі, машини
і т. ін., тобто засоби праці, що їхню функцію переривається лише
нормальними павзами в продукційному процесі — ненормальні
перерви в наслідок скорочення продукції, криз тощо є чисті
втрати — ця друга частина лятентного капіталу додає до продукту
вартість, хоч не бере участи в утворенні продукту; сукупна вартість, що
її ця частина додає до продукту, визначається середнім часом її тривання;
вона втрачає свою вартість, бо втрачає свою споживну вартість і в той
час, коли вона функціонує, і в той час, коли вона не функціонує.

Нарешті, вартість сталої частини капіталу, яка й далі перебуває в
продукційному процесі, не зважаючи на перерву в процесі праці, знову
з’являється в наслідок продукційного процесу. Самою працею засоби
продукції поставлено тут у такі умови, що в них вони сами собою пророблюють
певні природні процеси, що в наслідок їх постає певний
корисний результат або зміна форми їхньої споживної вартости. Праця
завжди переносить вартість засобів продукції на продукт, оскільки вона
споживає їх справді доцільно, як засоби продукції. Справа ані трохи не
змінюється від того, чи мусить праця, щоб досяглось такого ефекту,
безперервно діяти на предмет праці за допомогою засобів праці, чи вона
мусить лише дати поштовх, поставивши засоби продукції в такі умови,
що в них вони без дальшого впливу праці сами з себе зазнали б
передбаченої зміни в наслідок природних процесів.

Хоч на чому ґрунтується такий надлишок часу продукції над часом
праці, — чи на тому, що засоби продукції становлять тільки лятентний
продуктивний капітал, тобто перебувають на попередньому щаблі до
справжнього процесу продукції, чи на тому, що підчас процесу продукції
їхню власну функцію переривають павзи в продукційному процесі, чи,
нарешті, на тому, що самий процес продукції зумовлює перерви в процесі
праці — в жодному з цих випадків засоби продукції не функціонують як
вбирачі праці. Коли вони не вбирають праці, то не вбирають і додаткової
праці. Тому тут не відбувається ніякого зростання вартости продуктивного
капіталу, поки він перебуває в тій частині часу своєї продукції, що є надлишок
над часом праці, хоч як би неподільно сполучалось здійснення процесу
зростання вартости з цими павзами. Очевидно, що як більше збігаються один
з одним час продукції та час праці, то більша продуктивність і зростання
вартости даного продуктивного капіталу протягом даного часу. Відси
випливає тенденція капіталістичної продукції по змозі зменшити надлишок
часу продукції над часом праці. Але хоч час продукції капіталу й може
відхилитись від його часу праці, а проте, він завжди охоплює цей час,
і надлишок першого над другим є навіть умова продукційного процесу.
Отже, час продукції є завжди той час, що протягом його капітал про-
