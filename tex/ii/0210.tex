сума вартости авансованого капіталу зменшується з 900 ф. стерл. до
800 ф. стерл.; решта 100 ф. стерл. первісно авансованої вартости виділяється
в формі грошового капіталу. Як такий, він надходить на грошовий
ринок і утворює додаткову частину діющого тут капіталу.

З цього видно, як може постати плетора (plethora) *) грошового капіталу
і не тільки в тому розумінні, що подання грошового капіталу вище
за попит на нього; це завжди є лише відносна плетора, що, приміром,
постає в „меланхолійному періоді“, який починає новий цикл по
закінченні кризи. Це плетора грошового капіталу в тому розумінні, що
для провадження сукупного суспільного процесу репродукції (який включає
і процес циркуляції) певна частина авансованої капітальної вартости
є зайва, а тому вона й виділяється в формі грошового капіталу; це плетора,
що постає при незмінному маштабі продукції та незмінних цінах,
виключно в наслідок скорочення періоду обороту. Більша чи менша маса
грошей, що перебуває в циркуляції, не справляє на це жодного впливу.

Припустімо, навпаки, що період циркуляції більшає, напр., з трьох
тижнів до п’яти. Тоді вже при наступному обороті авансований капітал
припливає назад на два тижні пізніше. Останню частину процесу продукції
цього робочого періоду не можна провадити далі механізмом обороту
самого авансованого капіталу. Коли такий стан триває порівняно довго,
то може постати скорочення процесу продукції — того розміру, що в
ньому провадять його — так само, як в попередньому випадку постало
його поширення. Але щоб продовжувати процес у попередньому розмірі,
авансований капітал треба збільшити на 2/9 його розміру, = 200 ф.
стерл., на весь час подовження періоду циркуляції. Цей додатковий капітал
можна взяти лише з грошового ринку. Коли подовження періоду
циркуляції поширюється на одну або кілька великих галузей підприємств,
то воно може в наслідок цього справити тиск на грошовий ринок, якщо
тільки цей вплив не паралізується протилежним впливом з другого боку.
І в цьому випадку ясно й очевидно, що цей тиск, як і раніш та плетора,
не має жодного чинення ні до зміни товарових цін, ні до зміни маси
наявних засобів циркуляції.

[Виготовити до друку цей розділ становило чималі труднощі. Хоч і
добре знав Маркс альґебру, але в аритметичних обчисленнях, особливо
торгових, не був він швидкий (ungeläufig), не зважаючи на те, що є грубий
пак зшитків, де він сам на багатьох прикладах проробив усі види торгового
рахівництва. Але знання окремих видів рахівництва і вправи в
щоденному практичному рахівництві купця зовсім не те саме, і тому
Маркс заплутався в обчисленні оборотів, так що поряд незакінчености
постали деякі неправильності й суперечності. В наведених вище таблицях
я залишив тільки найпростіше і аритметично правильне, головним чином,
з таких міркувань.

Неправильні результати цих клопітних обчислень спричинили те, що

*) Плетора (plethora) — грецьке слово, що йому відповідає німецьке „Überfülle“,
„Überfluss“, або українське „повнява“, „повня", „багатість“. Ред.
