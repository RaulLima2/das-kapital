\parcont{}  %% абзац починається на попередній сторінці
\index{ii}{0352}  %% посилання на сторінку оригінального видання
бувши продана, не перетворюється знову в поточному річному періоді
репродукції з грошей на натуральну форму основних складових частин
свого сталого капіталу. Отже, до II ріжниця грішми могла б допливати
лише тоді, коли б II продав І на 2000, а купив би в І менше, ніж на
2000, наприклад, тільки на 1800; тоді І повинен би був покрити
ріжницю за допомогою 200 грішми, які не повернулись би до нього, бо
ці гроші, авансовані для циркуляції, він не вилучив би знову з неї поданням
у циркуляцію товарів = 200. В такому разі для II у нас був би
грошовий фонд на рахунок зношування його основного капіталу; але на
другому боці, на боці І, ми мали б перепродукцію засобів продукції, на
суму 200, і таким чином зникла б вся основна схема, а саме репродукція
в незмінному маштабі, за якої припускається цілковита пропорційність
між різними системами продукції. Усунувши одні труднощі, ми мали б
інші, куди неприємніші.

Що ця проблема являє особливі труднощі й до цього часу її взагалі
не досліджували політикоекономи, то ми послідовно розглянемо всі можливі
(принаймні на позір можливі) розв'язання їх або радше постави
самої проблеми.

Насамперед ми щойно припустили, що II продає І підрозділові на
2000, а купує в нього товарів лише на 1800. В товаровій вартості
2000 II$с$ було 200 на заміщення зношування, що їх треба зберегти
в грошах як скарб; таким чином, вартість 2000 II$с$ розклалась на
1800, що їх треба обміняти на засоби продукції І і на 200 для заміщення
зношування, що їх (після продажу $2000с$ підрозділові І) треба
зберегти в грошах. Або шодо своєї вартости 2000 II$с$ були б $= 1800 с +
200 с (d)$, де $d = $\emph{déchet} [зношування].

В такому разі нам треба було б дослідити:
\[
обмін І. $1000 v + 1000 m$
II.    1800 с + 200 с (d).
\]
На 1000 ф. стерл., що в формі заробітної плати потрапили до робітників
як плата за їхню робочу силу, І купує засобів споживання
1000 II$с$; II на ці самі 1000 ф. стерл. купує засобів продукції 1000 І$v$.
Таким чином, до капіталістів І повертається їхній змінний капітал у
грошовій формі, і найближчого року вони можуть купити на нього робочу
силу тієї самої величини вартости, тобто можуть in natura замістити
змінну частину свого продуктивного капіталу. — Далі, II на авансовані
400 ф. стерл. купує засоби продукції І$m$, а І$m$ на ті самі
400 ф. стерл. купує засоби споживання II$с$. Ті 400 ф. стерл., що їх II
авансував для циркуляції, повернулись таким чином до капіталістів II, але
повернулись лише як еквівалент за проданий товар. І на авансовані
400 ф. стерл. купує засоби споживання; II купує в І засобів продукції
на 400 ф. стерл., в наслідок чого ці 400 ф. стерл. повертаються до І.
Отже, рахунок поки що такий:
\parbreak{}  %% абзац продовжується на наступній сторінці
