Куллоха, Джемса Мілла та інших спроба ототожнити час продукції, що відхиляється від робочого часу, з
цим останнім, — спроба, що сама й собі походить від неправильного застосування теорії вартости.

Цикл обороту, що ми розглянули вище, визначається тривалістю основного капіталу, авансованого на
процес продукції. А що цей цикл охоплює більший або менший ряд років, то охоплює він і ряд річних,
тобто повторюваних протягом кожного року оборотів основного капіталу.

В хліборобстві такий цикл обороту зумовлюється системою сівозміни. Протяг оренди в усякому разі не
повинен бути менший, ніж час обороту при заведеній сівозміні, тому при трипільному господарстві його
завжди беруть у 3, 6, 9 років. Коли заведено трипільне господарство з чистим паром, то кожне поле
протягом шости років обробляється тільки чотири рази, при цьому в ті роки, як його обробляється, на
ньому сіють озимину або ярину і, коли того потребує або дозволяє властивість ґрунту, послідовно —
пшеницю і жито, ячмінь і овес. Кожна відміна зернівців дає на тому самому ґрунті більші або менші
врожаї, ніж інші відміни, кожна має свою вартість і продається за свою ціну. Тому, коли прибуток з
поля змінюється кожного року обробки, то й за першу половину обороту (за перші три роки) він буде не
той, що за другу. Навіть пересічний прибуток за першу й другу половину часу обороту буде
неоднаковий, бо родючість залежить не лише від якости ґрунту, а також і від погоди, так само, як і
ціни залежать від багатьох умов. Коли ми обчислимо прибуток з поля, беручи на увагу середню
родючість і пересічні ціни за ввесь шестилітній період часу обороту, то знайдемо загальну цифру
щорічного прибутку і для першого й для другого періоду часу обороту. Цього однак не буде, коли ми
обчислимо прибуток лише за половину часу обороту, тобто тільки за три роки, бо тоді загальні цифри
прибутку не будуть однакові. Відси випливає, що при трипільній системі протяг оренди треба визначити
принаймні в шість років. Але куди бажаніше завжди орендареві й землевласникові, щоб час оренди
становив кількаразовий час оренди (sic!), отже, при трипільній системі замість б років — 12, 18 і
більш років, а при семипільній замість 7—14, 28 років“. (Kirchhof, S. 117, 118).

(В рукопису тут стоїть: „Англійське сівозмінне господарство. Тут зробити примітку").

Розділ чотирнадцятий

Час обігу

Всі досі розглянуті обставини, що зумовлюють ріжниці в періодах обігу1 різних капіталів, вкладених у
різні галузі підприємств, а тому

1 Тут, очевидно, термін „період обігу“ („Umlaufsperiode“) вжито в широкому розумінні слова — як
період, що охоплює час продукції та час власне обігу, тобто в розумінні періода оборогу капіталу.
Ред.
