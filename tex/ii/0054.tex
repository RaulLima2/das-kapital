100 ф. пряжі за 5,000 ф. стерл. — змінна частина капіталу тих таки
1000 ф. пряжі; отже, 844 ф. пряжі за 42,200 ф. стерл. — покриття
капітальної вартости, яка є в 1000 ф. пряжі; нарешті, 156 ф. пряжі
вартістю в 7,800 ф. стерл., які репрезентують додатковий продукт, що
є в ній, і які можуть бути спожиті як такий.

Нарешті, може він решту 1560 ф. пряжі вартістю в 78 ф. стерл.,
якщо пощастить її продати, розкласти так, що продаж 1160,640 ф.
пряжі за 58,032 ф. стерл. покриватиме вартість засобів продукції, що
містяться в 1560 ф. пряжі, а 156 ф. пряжі вартістю в 7,800 ф. стерл. —
змінну капітальну вартість; разом 1316,640 ф. пряжі = 65,832 ф. стерл.,
покривають усю капітальну вартість; нарешті, лишається додатковий
продукт 243,360 ф. пряжі = 12,168 ф. ст., що їх можна витрачати як
дохід.

Так само, як кожен елемент, що існує в пряжі — с, v, m, можна
знову розкласти на ці самі складові частини, так само можна розкласти й
кожен окремий фунт пряжі вартістю в 1 шилінґ — 12 пенсів;

с = 0,744 ф. пряжі = 8,928 пенсів
v = 0,100 „„ = 1,200 „
m = 0,156 „„ = 1,872 „
c + v + m = 1,000 ф. пряжі = 12,000 пенсів

Коли ми складемо результати трьох зазначених частинних продажів, то
результат буде такий самий, як і пои одночасному продажу 10.000 ф.
пряжі.

Сталого капіталу ми маємо:

При 1-му продажу 5535,360 ф. пряжі = 276,768 ф. стерл.
  „2-му „744,000 „„ = 37,200 „„
  „3-му „1160,640 „„ = 58,032 „„

                  Разом.  . 7440,000 ф. пряжі = 372,000    ф. стерл.

Змінного капіталу:

При 1-му продажу 744,000 ф. пряжі = 37,200 ф. стерл.
  „2-му „100,000 „„ = 5,000 „„
  „3-му „156,000 „„ = 7,800 „„

                Разом.  . 1000,000 ф. пряжі = 50,000 ф. стерл.

Додаткової вартости:

При 1-му продажу 1160,640 ф. пряжі = 58,032    ф. стерл.
  „2-му „156,000 „„ = 7,800 „„
  „3-му „243,360 „„ = 12,168 „„

                  Разом.  . 1560,000 ф. пряжі = 78,000 ф. стерл.
