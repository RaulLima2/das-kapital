\index{ii}{0261}  %% посилання на сторінку оригінального видання
Швидший грошовий обіг буває кожного разу, коли за допомогою тієї
самої кількости грошей провадиться більше оборудок. Це може бути і
при однакових періодах репродукції капіталу, в наслідок змінених технічних
пристосовань у грошовому обігу. Далі: може збільшуватися число
оборудок, що в них грошовий обіг не виражає справжнього обміну товарів
(біржові диференційні оборудки і т. ін.). З другого боку грошового
обігу може й зовсім не бути. Напр., там, де сільський господар сам є
землевласник, немає жодного грошового обігу між орендарем і землевласником;
там, де промисловий капіталіст сам є власник капіталу, немає
жодного обігу між ним і кредитором.
\pfbreak
Щодо первісного утворення грошового скарбу в країні та привлащення
його небагатьма особами, то тут немає потреби зупинятися на
цьому докладніше.

Капіталістичний спосіб продукції — а за базу його є так само наймана
праця, як і оплата робітника грішми і взагалі перетворення натуральних
відбутків на грошові — може розвиватись у ширшому й глибшому маштабі
тільки в такій країні, де є досить грошей для циркуляції та для
зумовлюваного нею утворення скарбів (резервного фонду тощо). Така є
історична передумова, хоч не треба розуміти справу так, ніби спочатку
утворюється досить скарбів, а потім починається капіталістична продукція.
Вона розвивається одночасно з розвитком умов для неї, а за одну з
таких умов є достатнє подання благородних металів. Тому посилене подання
благородних металів, починаючи з XVI століття, являє посутній
момент в історії розвитку капіталістичної продукції. Але оскільки йдеться
про потрібне дальше подання грошового матеріялу на базі капіталістичного
способу продукції, то, з одного боку, додаткову вартість
подається в циркуляцію в вигляді продукту, без грошей потрібних для
його перетворення на гроші, а з другого боку, додаткову вартість подається
в циркуляцію в вигляді золота, без попереднього перетворення
продукту на гроші.

Додаткові товари, що мають перетворитися на гроші, знаходять потрібну
грошову суму, бо на другому боці, не через обмін, а самою продукцією
подається в циркуляцію додаткове золото (і срібло), що має
перетворитись на товари.

\subsection{Акумуляція та поширена репродукція}

Оскільки акумуляція відбувається в формі репродукції в поширеному
маштабі, то очевидно, що вона не являє жодної нової проблеми щодо грошової
циркуляції.

Насамперед, щодо додаткового грошового капіталу, потрібного для
функції ростучого продуктивного капіталу, то його дає та частина реалізованої
додаткової вартости, що її капіталіст подає в циркуляцію як грошовий
капітал, а не як грошову форму доходу. Гроші вже є в руках
капіталістів. Тільки застосування їх різне.
