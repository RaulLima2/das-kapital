новий капітал II разом з частиною капіталу І становлять капітал, що функціонує
в другому періоді обороту, тимчасом як решта капіталу І звільняється
для первісної функції капіталу II. Капітал, діющий протягом часу
обігу товарового капіталу, тут не тотожній з капіталом II, первісно авансованим
для цього, але він дорівнює йому вартістю і становить таку саму
частину цілого авансованого капіталу.

Подруге. Капітал, що функціонував протягом робочого періоду,
лежить без діла протягом часу обігу. В другому прикладі капітал функціонує
протягом 5 тижнів робочого періоду й лежить без діла протягом
5 тижнів часу обігу. Отже, увесь цей час, що його капітал І тут на протязі
року лежить без діла, дорівнює половині року. На цей час тоді ввіходить
додатковий капітал II, що, отже, і собі у даному випадку теж лежить без діла
протягом півроку. Але додатковий капітал, потрібний для того, щоб підтримати
безперервну продукцію протягом часу обігу, визначається не
всією величиною, зглядно не сумою часів обігу протягом року, а лише
відношенням часу обігу до періоду обороту. (Тут, звичайно, припускається,
що всі обороти відбуваються в однакових умовах). Тому в прикладі
II додаткового капіталу треба 500 ф. стерл., а не 2500 ф. стерл.
Це пояснюється просто тим, що додатковий капітал увіходить в оборот
цілком так само, як і первісно авансований, отже, цілком так само, як
і цей останній, числом своїх оборотів заміщує свою масу.

Потрете. Коли час продукції довший, ніж робочий час, то це нічого
не змінює в розглянутих тут обставинах. В наслідок цього в усякому
разі подовшає цілий період обороту, але при такому подовшанні обороту
не треба жодного додаткового капіталу для процесу праці. Додатковий
капітал призначається тільки на те, щоб заповнити прогалини в процесі
праці, зумовлені часом обігу; отже, він повинен лише захищати
продукцію від тих порушень, що походять з часу обігу, а порушення,
що постають з власне умов продукції, вирівнюється іншим способом, що
його ми тут не будемо розглядати. Навпаки, є такі підприємства, де роблять
лише з перервами, на замовлення, де, отже, можуть бути перерви
між робочими періодами. В таких підприємствах pro tanto відпадає потреба
в додатковому капіталі. З другого боку, в більшості випадків сезоновнх
робіт дано й певну межу для часу зворотного припливу капіталу.
Ту саму роботу в наступному році не можна відновити тим самим капіталом,
коли час циркуляції цього капіталу в проміжний час не скінчився. Навпаки,
час обігу може бути й коротший від переміжку між одним періодом
продукції й наступним. В цьому випадку капітал лежить без діла,
якщо в цей проміжний час не застосується його інакше.

Почетверте. Капітал, авансований на один робочий період, напр.,
600 ф. стерл. в прикладі III, витрачається почасти на сировинні й допоміжні
матеріяли, на продуктивний запас для робочого періоду, на сталий обіговий
капітал, а почасти на змінний обіговий капітал, на оплату самої
праці. Частина, витрачена на сталий обіговий капітал, може існувати в
формі продуктивного запасу не однаково довгий час, напр., сировинний
матеріял можна запасати не на ввесь робочий період, вугілля можна
