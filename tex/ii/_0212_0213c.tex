\parcont{}  %% абзац починається на попередній сторінці
\index{ii}{0212}  %% посилання на сторінку оригінального видання
Припустімо тепер, навпаки, незмінну величину періоду обороту, незмінний
маштаб продукції, але, з другого боку, зміну цін, тобто падіння або
підвищення цін на сировинні та допоміжні матеріяли й працю, або перших
двох з цих елементів. Припустімо, що ціна сировинних та допоміжних
матеріялів, так само, як і заробітна плата, зменшилась на половину.
Тоді в нашому прикладі треба було б авансованого капіталу щотижня
50 ф. стерл. замість 100, а для дев'ятитижневого періоду обороту — 450 ф.
стерл. замість 900 ф. стерл. 450 ф. стерл. авансованої капітальної
вартости виділюється насамперед як грошовий капітал, але процес продукції
триватиме й далі в тому самому маштабі, з тим самим періодом
обороту і з тим самим поділом останнього. Річна маса продукту лишається
теж та сама, але вартість її на половину зменшилась. Цю зміну, яку
супроводить і зміна в поданні та в попиті на грошовий капітал, спричиняє
не прискорення обігу й не зміна маси грошей, що циркулюють. Навпаки.
Зниження вартости, зглядно ціни елементів продуктивного капіталу
наполовину справило б насамперед той вплив, що авансувалось би капітальну
вартість, зменшену наполовину для того, шоб вести підприємство
X у попередніх розмірах, а що підприємство X авансує цю капітальну
вартість насамперед у формі грошей, тобто як грошовий капітал,
то, значить, воно мало б викидати на ринок лише половину попередньої
кількости грошей. Маса грошей, поданих в циркуляцію, зменшилась би
тому, що знизились ціни елементів продукції. Такий був би перший вплив.

Але подруге: половина первісно авансованої капітальної вартости в
900 ф. стерл. = 450 ф. стерл., яка а) по черзі перебігала форму грошового
капіталу, продуктивного капіталу й товарового капіталу, б) яка одночасно
постійно перебувала почасти в формі грошового капіталу, почасти
в формі продуктивного капіталу, почасти в формі товарового капіталу,
в одній поряд однієї, — ця половина виділилась би з
кругобігу підприємства X і тому надійшла б як додатковий грошовий
капітал на грошовий ринок, впливаючи на нього як додаткова складова
частина. Ці звільнені гроші, 450 ф. стерл., впливають як грошовий капітал
не тому, що вони є гроші, які стали надлишкові для продовження
підприємства X, а тому, що вони є складова частина первісної капітальної
вартости, і тому повинні й далі діяти як капітал, а не витрачатись як
простий засіб циркуляції. Найближчий спосіб надати їм чинности капіталу,
це подати їх на грошовий ринок як грошовий капітал. З другого боку,
можна було б також збільшити розміри продукції, залишаючи осторонь
основний капітал, вдвоє. Авансуючи той самий капітал в 900 ф.
стерл., можна було б провадити процес продукції в подвоєному розмірі.

З другого боку, коли б ціни поточних елементів продуктивного капіталу
підвищились наполовину, то щотижня замість 100 ф. стерл. треба
було б 150 ф. стерл., отже, замість 900 ф. стерл. — 1350 ф. стерл. Щоб
провадити підприємство в тому самому маштабі, треба було б 450 ф.
стерл. додаткового капіталу, і це залежно від стану грошового ринку
справляло б на нього pro tanto більший або менший тиск. Коли б на
ввесь вільний капітал, що є та ринку, ставилося вже попит, то це при
\index{ii}{0213}  %% посилання на сторінку оригінального видання
звело б до підвищеної конкуренції за вільний капітал. Коли б деяка частина
його лежала без діла, те її pro tanto покликали б до діяльности.

Але, потретє, за даних розмірів продукції, за незмінної швидкости
обороту та незмінних цін елементів поточного продуктивного капіталу,
ціна продуктів підприємства X може знизитись або підвищитись. Коли
ціна товарів, що їх подає підприємство X, знижується, то спадає ціна
його товарового капіталу з 600 ф. стерл., що їх воно завжди подавало
в циркуляцію, напр., до 500 ф. стерл. Отже, шоста частина вартости авансованого
капіталу не припливає назад з процесу циркуляції (додаткову вартість,
що є в товаровому капіталі, тут не береться на увагу); вона пропадає
марно в цьому процесі. Але що вартість, зглядно ціна елементів продукції
лишається та сама, то цих 500 ф. стерл., які приплили назад, вистачить
лише на те, щоб замістити \sfrac{5}{6} капіталу в 600 ф. стерл, ввесь час
занятого в процесі продукції. Отже, для того, щоб і далі провадити підприємство
в тому самому маштабі, довелось би витратити 100 ф. стерл.
додаткового грошового капіталу.

Навпаки: коли ціна продуктів підприємства X підвищиться, то підвищиться
й ціна товарового капіталу з 600 ф. стерл., напр., до 700 ф.
стерл. Сьома частина його ціни, рівна 100 ф. стерл., приходить не з
процесу продукції, не була авансована на нього, а припливає сюди з
процесу циркуляції. Однак, для заміщення продуктивних елементів треба
лише 600 ф. стерл.; отже, 100 ф. стерл. звільняються.

Дослідження причин, чому в першому випадку період обороту
скорочується або подовжується, в другому випадку ціни на сировинний
матеріял та працю, і в третьому ціни поданих продуктів підвищуються
або падають, — дослідження цих причин не входить у межі цього досліду.

Але ось що входить у межі його:

\textbf{I випадок.} \emph{Незмінний маштаб продукції, незмінні ціни елементів
продукції та продуктів, зміна в періоді циркуляції, а значить, і в періоді
обороту.}

Згідно з припущенням у нашому прикладі, в наслідок скорочення періоду
циркуляції, треба авансувати всього капіталу менше на \sfrac{1}{9}; тому
капітал цей зменшується з 900 до 800 ф. стерл., і виділюється грошовий
капітал в 100 ф. стерл.

Як і раніш, підприємство X дає той самий шеститижневий продукт такої
самої вартости в 600 ф. стерл., а що роблять цілий рік безперервно, то
воно протягом 51-го тижня дає ту саму масу продукту, вартістю в 5100 ф.
стерл. Отже, в масі та ціні продукту, що його подає підприємство в
циркуляцію, немає жодної зміни, немає її і в тих строках, що в них підприємство
подає продукт на ринок. Але виділилось 100 ф. стерл., бо
через скорочення періоду циркуляції процес насичено авансуванням капіталу
лише в 800 ф. стерл. замість попередніх 900 ф. стерл. Ці 100 ф.
стерл. виділеного капіталу існують у формі грошового капіталу. Але вони
зовсім не репрезентують тієї частини авансованого капіталу, що постійно
мусить функціонувати в формі грошового капіталу. Припустімо, що з
авансованого поточного капіталу I = 600 ф. стерл. \sfrac{4}{5} = 480 ф. стерл.
\parbreak{}  %% абзац продовжується на наступній сторінці
