ється, виклад подекуди має прогалини, і особливо наприкінці він цілком
уривчастий. Але те, що Маркс хотів сказати, так або інакше тут сказано.

Такий матеріял для II книги, що з нього я, як сказав Маркс не задовгий
час до своєї смерти своїй дочці Елеонорі, повинен був „дещо зробити“.
Це доручення я зрозумів у найвужчому його значенні; де лише
можна було, я обмежив мою роботу простим вибором між різними редакціями,
і саме так, що в основу завжди покладав останню з даних редакцій,
порівнявши її з попередніми. Справжні, тобто не лише технічні
труднощі являв при цьому тільки перший і третій відділи, але ці труднощі
були не абиякі. Я дбав про те, щоб розв’язати їх виключно в
авторовому дусі.

Цитати в тексті я здебільша перекладав там, де їх наведено на потвердження
фактів, або там, де оригінал є до послуг кожного, хто хоче
докладно обізнатися з питанням, прим., у цитатах з А. Сміса. Тільки в
розділі X не можна було зробити цього, бо тут безпосередньо критикується
англійський текст. У цитатах з І тому посилання зроблено на
сторінки другого видання його, останнього, яке вийшло за життя
Маркса.

Для III книги, крім першої обробки в рукопису „Zur Kritik“, згаданих
частин рукопису III і деяких коротеньких приміток, зроблених подекуди
в записних зшитках, маємо лише ось що: зазначений вище рукопис
in folio від 1864—1865 р., розроблений майже так само повно, як
і рукопис II книги II, і, нарешті, зшиток з 1875 р.: відношення норми
додаткової вартости до норми зиску, викладене математично (в рівняннях).
Підготовка цієї книги до друку йде швидким темпом. За думкою,
що в мене склалась до цього часу, вона являтиме, головним чином, технічні
труднощі, за винятком, звичайно, деяких дуже важливих відділів.

Тут буде до речі розбити те обвинувачення проти Маркса, що його
поширювали спочатку потихеньку й поодинці, а тепер, після смерти його
проголосили за безперечний факт німецькі катедерсоціялісти й державні
соціялісти та їхні прихильники, — обвинувачення, ніби Маркс учинив
пляґіят у Родбертуса. В іншому місці\footnote{
У передмові до „Das Elend der Philosophie. Antwort auf Proudhons Philosophie
des Elends von Karl Marx. Deutsch von E. Bernstein und K. Kautsky.
Stuttgart 1885“ (К. Маркс. „Злидні філософії“. Відповідь на „Філософію злиднів“
Прудона).
} я вже сказав усе найпосутніше
з цього приводу, але лише тут можу навести рішучі докази.

Обвинувачення це, оскільки я знаю, вперше трапляється в „Emanzipationskampf
des vierten Standes“ P. Maєpa, стор. 43: „З цих оголошених
друком праць“ (праць Родбертуса, датованих до останньої половини
тридцятих років) „Маркс, як це можна довести, почерпнув більшу частину
своєї критики“. Поки не було дальших доказів, я, звичайно, міг
припускати, що вся „довідність“ цього твердження сходить на те, що
Родбертус упевнив п. Маєра в цьому. — 1879 року виступае на кін сам Родбер-