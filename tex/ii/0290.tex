ципував ці гроші, — в усіх цих випадках капіталіст витрачає змінний капітал, що допливає робітникам
у вигляді грошей, і володіє, з другого боку, еквівалентом цієї капітальної вартости в вигляді
частини вартости своїх товарів, що в ній робітник знову спродукував ту частину цілої вартости, яка
припадає йому самому, інакше кажучи, ту, що в ній він спродукував вартість своєї власної заробітної
плати. Замість дати робітникові цю частину вартости в натуральній формі його власного продукту,
капіталіст виплачує йому її грішми. Отже, для капіталіста змінна складова частина
авансованої ним капітальної вартости існує тепер у товаровій формі, тимчасом як робітник одержав
еквівалент за продану ним робочу силу в грошовій формі.

Отже, в той час як частина авансованого капіталістом капіталу, перетворена закупом робочої сили на
змінний капітал, функціонує в самому процесі продукції як діюща робоча сила, в той час як
витрачанням цієї сили частину цю знову продукується, тобто репродукується в товаровій формі як нову
вартість, — отже, відбувається репродукція, тобто нова продукція авансованої капітальної вартости! —
робітник витрачає вартість,
зглядно ціну, своєї проданої робочої сили на засоби існування, на засоби репродукції своєї робочої
сили. Сума грошей, рівна змінному капіталові, становить його дохід, отже, дохід, що триває лише
доти, доки він може продавати свою робочу силу капіталістам.

Товар найманого робітника, — сама його робоча сила — функціонує як товар лише остільки, оскільки її
долучається до капіталу капіталіста, оскільки вона функціонує як капітал; з другого боку, капітал
капіталіста, витрачений як грошовий капітал на закуп робочої сили, функціонує як дохід в руках
продавця робочої сили в руках найманого робітника.

Тут переплітаються різні процеси циркуляції та продукції, що їх А. Сміс не розмежовує.

Поперше. Акти, що належать до процесу циркуляції: робітник продає свій товар — робочу силу —
капіталістам; гроші, що на них капіталіст купує її, є для нього гроші, вкладені для збільшення їх
вартости, отже, грошовий капітал; капітал цей не витрачено, а лише авансовано. (В цьому справжнє
значення „авансування“ — avance фізіократів — цілком незалежно від того, відки сам капіталіст бере
гроші. Для капіталіста буде авансованою кожна вартість, що її він сплачує для процесу продукції,
незалежно від того, чи буде це до чи post festum; її авансовано самому процесові продукції). Тут
відбувається лише те, що при всякому продажу товарів: продавець віддає споживну вартість (в даному
разі робочу силу) і одержує її вартість (реалізує її ціну) в грошах; покупець віддає свої гроші й
одержує натомість самий товар — в даному разі робочу силу.

Подруге. В цроцесі продукції куплена робоча сила являє тепер частину діющого капіталу, а сам
робітник функціонує тут лише як особлива натуральна форма цього капіталу, відмінна від тих його
елементів, що існують у натуральній формі засобів продукції. Протягом процесу продукції робітник до
засобів продукції, що їх він перетворює на продукт, долучає витратою своєї робочої сили вартість,
рівну вартості його ро-
