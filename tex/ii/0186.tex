пливають відси, в часі обороту, випливають, отже, з сфери циркуляції,
але вони безпосередньо відбиваються на сфері продукції, і до того ж
незалежно від строків виплат і кредитових відносин, а значить, і при виплаті
готівкою. Напр., вугілля, бавовна, пряжа, тощо е продукти подільні.
Кожний день дає певну кількість готового продукту. Але коли прядун
або власник копалень береться поставити таку масу продуктів, що для
неї потрібен, приміром, чотиритижневий або шеститижневий період послідовних
робочих днів, то відносно до протягу часу, що на нього треба
авансувати капітал, це все одно, якби в цьому процесі праці був заведений
безперервний робочий період в чотири або шість тижнів. Тут звичайно
припускається, що всю замовлену масу продуктів треба доставити
одним заходом, або що її оплатиться лише після того, як її всю доставиться.
Отже, кожен день, взятий окремо, дав свою певну кількість готового
продукту. Але ця готова маса завжди є лише частина тієї маси,
що її треба доставити згідно з угодою. В цьому разі, якщо виготовлена
вже частина замовленого товару не перебуває в процесі продукції, то
вона в усякому разі лежить на складі лише як потенціяльний капітал.

Перейдімо тепер до другого відділу часу обігу — до часу купівлі або
до періоду, що протягом його капітал з грошової форми знову перетворюється
на елементи продуктивного капіталу. Протягом цього періоду
він мусить довший або коротший час лежати в стані грошового капіталу,
а значить, певна частина цілого авансованого капіталу має перебувати
безупинно в стані грошового капіталу, хоч ця частина складається з елементів,
що постійно змінюються. В якомубудь певному підприємстві з
усього авансованого капіталу мусить бути в формі грошового капіталу,
прим., 100 ф. стерл. × n, і тимчасом як усі складові частини цих 100 ф.
стерл. × n безупинно перетворюються на продуктивний капітал, ця сума
все ж так само завжди знову поповнюється припливом із циркуляції, з реалізованого
товарового капіталу. Отже, певна частина вартости авансованого
капіталу постійно перебуває в стані грошового капіталу, отже, в формі,
що належить не до сфери його продукції, а до сфери його циркуляції.

Ми вже бачили, що подовження часу, зумовлене віддаленістю ринку,
подовження, що протягом його капітал є зв’язаний в формі товарового
капіталу, безпосередньо призводить до запізнення зворотного припливу
грошей, отже, затримує перетворення капіталу з грошового капіталу на
продуктивний.

Далі, щодо закупу товарів, ми бачили (розд. VI), як час купівлі,
більша або менша віддаленість від головних джерел придбання сировинного
матеріялу примушує купувати сировинний матеріял на довші періоди
й зберігати його придатним до вжитку у формі продуктивного запасу, латентного
або потенціяльного продуктивного капіталу; отже, що така
віддаленість, при тому самому зрештою маштабі продукції, збільшує масу
капіталу, що його доводиться авансувати одним заходом, і час, що на
нього доводиться авансувати його.

Подібно впливають в різних галузях підприємства періоди — більш
або менш протяжні — що в них на ринок подається чималі маси сиро-
