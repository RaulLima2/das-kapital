\parcont{}  %% абзац починається на попередній сторінці
\index{ii}{0093}  %% посилання на сторінку оригінального видання
треба витрачати додатковий капітал, почасти на засоби праці, в зречевленій
формі, почасти на робочу силу11).

Отже, існування капіталу в його формі товарового капіталу, а тому
і в формі товарового запасу спричиняє витрати, які, тому що вони не належать
до сфери продукції, залічукиься до витрат циркуляції. Ці витрати
циркуляції відрізняються від наведених під І тим, що вони до певної
міри входять у вартість товарів і, значить, удорожчують, товари. За всяких
обставин капітал і робоча сила, що придаються для підтримання і
зберігання товарових запасів, відтягується від безпосереднього процесу
продукції. З другого боку, вжиті тут капітали, разом з робочою силою,
як складовою частиною капіталу, мусять бути покриті із суспільного
продукту. Витрата їх впливає так само, як зменшення продуктивної сили
праці, так що треба більшої кількости капіталу й праці, щоб дійти певного
корисного ефекту. Це — затрати (Unkosten).

Оскільки витрати циркуляції, зумовлені утворенням товарового запасу,
випливають лише з того періоду, що протягом його наявні вартості
перетворюються з товарової форми на грошову форму, отже, оскільки
вони випливають лише з певної суспільної форми продукційного процесу
(тільки з того, що продукт продукується як товар і тому мусить також
перетворитись на гроші), остільки вони своїм характером збігаються з витратами
циркуляції, переліченими під І. З другого боку, вартість товарів тут зберігається,
зглядно збільшується лише тому, що споживна вартість, самий
продукт, ставиться в певні речові умови, що коштують витрат капіталу й
підлягають операціям, за допомогою яких на споживні вартості діє
додаткова праця. Навпаки, обчислення товарових вартостей, ведення книг
щодо цього процесу, операції купівлі та продажу не впливають на споживну
вартість, що в ній існує товарова вартість. Ці витрати мають
чинення лише до її форми. А тому, хоч у припущеному випадку
затрати (Unkosten) утворення запасу (а воно в цьому разі вимушене)
випливають лише з затримки в перетворенні форми і з доконечности її,
все ж вони відрізняються від затрат (Unkosten), розглянутих під І, тим,
що мета їхня не перетворення форми вартости, а зберігання вартости,
що існує в товарі, як у продукті, як у споживній вартості, і тому

14 Corbet 1841 р. обчислював витрати на зберігання пшениці протягом дев'ятимісячного
сезону: J/2\% втрати її кількости, 3\% на проценти під ціну пшениці,
2\% на наймання складів, 1\% на ссипання й провізну плату, 1/г\% на вивантаження,
а разом 7\% а о, при ціні пшениці в 51) шил, — 3 шил. 6 пенсів на
квартер. (The Corbet. „Ап Inquiry into the Causes and Modes of the Wealth of
Individuals etc“. London 1841) За свідченням ліверпулських купців пер. д залізничною
комісією, (чисті) затрати на зберігання збіжжя доходили 1865 р. 2 пенсів
на квартер, або 9--10 пенсів на тонну щомісяця. (Royal Commission on Railways.
1867. hvidence, p. 19, № 33* \footnote*{
Точна назва цього видання така: „Reports from commissioners Session“,
1867, vul. XXXVIII; зазначене місце є в другому відділі першої частини, що має
назву: „Royal Commission on Railways. Minutes of Evidence taken before the commissioners
March 1855 to Mev 1866. Presented to both Houses of Parlament by
command oi her Majesty“. London, printed 1867. \emph{Ред.}
}.
\parbreak{}  %% абзац продовжується на наступній сторінці
