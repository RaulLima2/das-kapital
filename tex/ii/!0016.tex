тілом, і таким чином він перший поставив на ноги всю хемію, яка в
своїй флогістичній формі стояла на голові. І коли навіть він не відкрив
кисню одночасно з іншими й незалежно від них, як це він стверджував
пізніше, то все ж таки власно він і відкрив кисень, а не ті двоє, що
лише описали його, не здогадуючись при цьому, що власне вони
описували.

Таке само відношення, як між Лявуазьє і Прістлеєм та Шеле, маємо
ми між Марксом і його попередниками в питанні про додаткову вартість.
Що існує та частина вартости продукту, яку ми тепер звемо додатковою
вартістю, це встановлено за довгий час до Маркса; так само більшменш
виразно було висловлено, з чого вона складається, а саме —
з продукту тієї праці, що за неї привлащувач не заплатив жодного еквіваленту.
Але далі цього не пішли. Деякі — клясичні буржуазні економісти —
щонайбільше досліджували кількісне відношення, що в ньому розподіляється
продукт праці між робітником і власником засобів продукції.
Інші — соціалісти — вважали такий розподіл за несправедливий і шукали
утопічних засобів усунути цю несправедливість. І ті і інші були в полоні
економічних категорій, що їх знайшли вони у своїх попередників.

Але тут виступив Маркс. І до того ж, гостро розходячись із своїми
попередниками. Там, де вони вбачали розв’язання, він вбачав лише
проблему. Він бачив, що перед ним було не дефлогістоване повітря і не
вогнеповітря, але кисень, — що тут справа була не в простому констатуванні
економічного факту і не в суперечності між цим фактом і вічною
справедливістю та істинною мораллю, а в факті, що йому судилось
перевернути всю політичну економію й дати ключ до розуміння цілої
капіталістичної продукції — тому, хто зумів би скористатися з цього
факту. Керуючись цим фактом, він дослідив усі попередні категорії так
само, як Лявуазьє досліджував попередні категорії флогістичної хемії, керуючись
киснем. Щоб дізнатись, що таке додаткова вартість, Маркс мусив
був дізнатись, що таке вартість. Насамперед довелось критично переглянути
саме теорію вартости Рікардо. Отже, Маркс дослідив працю щодо її
властивості утворювати вартість і перший встановив, яка праця, чому та
як утворює вартість, встановив, що вартість взагалі є не що інше, як
скристалізована праця цього роду, — пункт, що його до останнього дня не
розумів Родбертус. Маркс досліджував потім відношення між товаром і
грішми й довів, як і чому — в наслідок іманентної йому властивости вартости —
товар і товаровий обмін повинні породжувати протилежність між товаром
і грішми; його на цьому побудована теорія грошей є перша вичерпна
теорія грошей, і тепер на неї мовчазно пристають усі. Він досліджував
перетворення грошей на капітал і довів, що воно ґрунтується на купівлі
й продажу робочої сили. Поставивши робочу силу, властивість утворювати
вартість, на місце праці, він одним ударом розв’язав дещо з тих
труднощів, що на них скрахувала школа Рікардо: неможливість узгодити
взаємний обмін капіталу й праці з Рікардовим законом визначення вартости
працею. Встановивши розподіл капіталу на сталий і змінний, він
перший дійшов того, що зобразив до подробиць і разом з тим пояснив
