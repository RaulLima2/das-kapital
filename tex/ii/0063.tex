мальна або навіть суто-суб’єктивна, як ріжниця, що існує лише для спостерігача.
Оскільки кожен з цих кругобігів розглядається як особливу форму
руху, що в ній перебувають різні індивідуальні промислові капітали, остільки
й ця ріжниця завжди існує лише як індивідуальна ріжниця. Але в дійсності
кожний індивідуальний промисловий капітал одночасно перебуває
в усіх трьох кругобігах. Три кругобіги, форми репродукції трьох відмін
капіталу, безупинно відбуваються один поряд одного. Напр., одна частина
капітальної вартости, яка функціонує тепер як товаровий капітал, перетворюється
на грошовий капітал, але одночасно друга частина виходить
з продукційного процесу в циркуляцію як новий товаровий капітал. Так
завжди перебігає кругова форма Т'... Т', так само і обидві інші. Репродукція
капіталу в кожній з його форм і на кожній з його стадій
є так само безперервна, як і метаморфоза цих форм і послідовний перебіг
цих трьох стадій. Отже, ввесь кругобіг є тут дійсна єдність усіх трьох
його форм.

У нашому досліді припускалось, що вся капітальна вартість у цілому розмірі
своєму виступає як грошовий капітал, або як продуктивний капітал, або як
товаровий капітал. Напр., 422 ф. стерл. було в нас спочатку як грошовий
капітал, потім, знову таки в цілому розмірі, перетворились вони на
продуктивний капітал і, нарешті, на товаровий капітал: пряжу, вартістю
в 500 ф. стерл. (з них 78 ф. стерл. додаткової вартости). Тут ці різні
стадії становлять стільки ж перерв. Напр., поки 422 ф. стерл. лишаються
в грошовій формі, тобто поки не відбулись купівлі Г — Т (Р + Зп),
сукупний капітал існує і функціонує лише як грошовий капітал. А скоро
він перетворюється на продуктивний капітал, він уже не функціонує ні
як грошовий капітал, ні як товаровий капітал. Увесь процес його циркуляції
переривається, так само, як, з другого боку, переривається ввесь
процес його продукції, коли він починає функціонувати в одній з двох
стадій циркуляції, чи то як Г, чи то як Т'. Таким чином, кругобіг П... П
являв би собою тоді не лише періодичне відновлення продуктивного
капіталу, але так само й перерву його функції, процесу продукції, перерву,
що тривала, б доки буде закінчений процес циркуляції; замість відбуватись
безперервно, продукція відбувалась би скоками й поновлювалась би лише
по переміжках невизначеного часу, залежно від того, оскільки швидка
або повільно здійснюються обидві стадії процесу циркуляції. Так, напр.,
стоїть справа в китайського ремісника, що робить тільки на приватних
замовників і припиняє процес продукції, поки не буде нових замовлень.

У дійсності це має силу й для кожної окремої частини капіталу,
що є в русі, і всі частини капіталу послідовно пророблюють цей рух.
Напр., 10.000 ф. пряжі є тижневий продукт прядуна. Ці 10.000 ф.
пряжі цілком виходять із сфери продукції в сферу циркуляції; капітальна
вартість, що міститься в них, мусить цілком перетворитисьна грошовий капітал,
і поки вона лишається в формі грошового капіталу, вона не може ввійти
знову в продукційний процес; вона мусить спочатку ввійти в цируляцію
і знову перетворитись на елементи продуктивного капіталу Р + Зп, Про-
