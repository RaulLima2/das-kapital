Коротко кажучи, коли б при простій репродукції та інших незмінних
обставинах, отже, при незмінних продуктивній силі, загальній масі та
інтенсивності праці, — ми припустили нестале відношення між відмерлим
(що потребує відновлення) і далі діющим в старій натуральній формі
(що просто долучає вартість до продуктів на заміщення свого зношування)
основним капіталом, то в одному випадку маса обігових складових частин,
що їх треба репродукувати, лишилась би та сама, але збільшилась би
маса основних складових частин, що їх треба репродукувати; отже, вся
продукція І мусила б збільшитись або, навіть лишаючи осторонь грошові
відношення, постав би дефіцит в репродукції.

В другому випадку: коли б відносна величина основного капіталу II,
що його треба репродукувати in natura, зменшилась, а тому збільшилась
би в такому ж відношенні та складова частина основного капіталу II,
яку покищо треба замістити лише в грошах, то маса обігових складових
частин сталого капіталу II, репродукованих І, лишилась би незмінна, а
маса основних частин, що їх треба репродукувати, навпаки, зменшилась
би. Отже, або зменшення всієї продукції І, або надлишок (як раніш був
дефіцит) і до того надлишок, що його не сила перетворити на гроші.

Правда, в першому випадку та сама праця при збільшені продуктивності,
протягу та інтенсивності могла б дати більший продукт, і таким
чином можна було б покрити дефіцит у першому випадку; але така
зміна не могла б статись без переміщення праці й капіталу з однієї галузі
продукції І в іншу, а всяке таке переміщення одразу ж викликало б
розлади. А подруге, І підрозділові довелось би (оскільки збільшуються
протяг та інтенсифікація праці) обміняти більшу вартість на
меншу вартість II, отже, сталось би знецінення продукту І.

Зворотне було б у другому випадку, де підрозділ І мусить скорочувати
свою продукцію, а це означає кризу для зайнятих у ньому робітників
і капіталістів, або він дає надлишок, а це знову таки є криза.
Самі по собі такі надлишки є не лихо, а вигода, але при капіталістичній
продукції вони є лихо.

Зовнішня торговля могла б допомогти в обох випадках; в першому
випадку, — щоб товар І, утримуваний в грошовій формі, перетворити на
засоби споживання; в другому випадку, — щоб збути товаровий надлишок.
Але зовнішня торговля, оскільки вона не просто заміщує елементи (також
і за вартістю), лише відсуває суперечності в ширшу сферу, відкриває їм
більший простір.

Коли усунути капіталістичну форму репродукції, то справа сходить
на те, що розмір частини основного капіталу, яка відмирає й тому
повинна замішуватись in natura (тут капіталу, що функціонує в продукції
засобів споживання), змінюється в різні послідовні роки. Коли одного
року ця частина дуже велика (перевищує середню смертність, як це
буває з смертністю людей), то в наступному році вона, певно, буде
настільки ж менша.

Але від цього маса сировинних матеріялів, напівфабрикатів і допоміжних
матеріялів, потрібна для річної продукції засобів споживання, — при-
