= 500, і те саме з m; склавши всі ці величини, матимемо, як і раріше,
сукупну вартість в 3000.

Отже, вся стала капітальна вартість, що міститься в масі товарів II
вартістю в 3000, міститься в 2000 с, і ні 500 v, ні 500 m не мають
жодного атома цієї вартости. Це саме має силу також і для v, і для m.

Інакше кажучи: вся та кількість товарової маси II, яка репрезентує
сталу капітальну вартість і тому має знову бути перетворена — хоч
на її натуральну, хоч на грошову форму — існує в 2000 с. Отже, все, що
стосується до обміну сталої вартости товарів II, обмежується рухом 2000
II с; і цей обмін можливий тільки на І (1000 v + 1000 m).

Так само для кляси І все, що стосується до обміну належної йому
капітальної вартости, треба обмежити розглядом 4000 І с.

1) Заміщення в грошовій формі зношуваної частини вартости.

Тепер, коли ми візьмемо насамперед:

І. 4000 с + 1000 v + 1000 m
II..............  2000 с + 500 v + 500 m

то обмін товарів 2000 IIc на товари такої самої вартости І (1000 v +
1000 m) припускав би, що 2000 IIс in natura цілком знову перетворюється
на спродуковані підрозділом І натуральні складові частини сталого
капіталу II; але товарова вартість 2000, в якій існує цей капітал,
містить у собі елемент, що покриває втрату вартости основного капіталу,
який не одразу треба заміщати in natura, а перетворювати на гроші, поступінно
нагромаджувані в цілу суму, поки надійде час відновити
основний капітал у його натуральній формі. Кожен рік є рік смерти для
основного капіталу й доводиться його заміщувати то в цьому, то в тому
поодинокому підприємстві, або навіть то в тій, то в цій галузі промисловості;
в тому самому індивідуальному капіталі доводиться заміщувати
ту або іншу частину основного капіталу (бо частини його мають різний
протяг життя). Розглядаючи річну репродукцію — хоча б і в незміненому
маштабі, тобто абстрагуючись від усякої акумуляції — ми починаємо не
ab ovo *); ми беремо один рік з ряду багатьох, не перший рік по
народженні капіталістичної продукції. Отже, різні капітали, вкладені
в різні галузі продукції кляси II, мають різний вік, і подібно до того,
як щороку вмирають люди, які функціонують у цих галузях продукції,
так само маси основних капіталів щороку доживають свого віку, й їх доводиться
відновлювати in natura з нагромадженого грошового фонду. В
цьому розумінні обмін 2000 II с на 2000 І (v + m) включає перетворення
2000 II с з його товарової форми (засобів споживання) на натуральні
елементи, що складаються не лише з сировинних та допоміжних елемен-

*) Ab ovo — латинський вираз, дослівно „від яйця“, тобто з самого виникнення,
з самого початку. Ред.
