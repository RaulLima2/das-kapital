\parcont{}  %% абзац починається на попередній сторінці
\index{ii}{0229}  %% посилання на сторінку оригінального видання
Далі з цього випливає: річна норма додаткової вартости завжди =
m'n, тобто дорівнює справжній нормі додаткової вартости, спродукованої
в один період обороту змінним капіталом, зужитим протягом цього
періоду, помноженій на число оборотів цього змінного капіталу протягом
цього року, або помноженій (що те саме) на обернений дріб його часу
обороту, обчисленого на рік, що береться за одиницю. (Коли
змінний капітал обертається 10 разів на рік, то час його обороту = 1/10 року;
отже, обернений дріб його часу обороту = 10/1 = 10).

Далі з наведеного випливає: М' = m', коли n = 1. М' більше за m',
коли n більше за 1, тобто, коли авансований капітал обертається більше
як один раз на рік, або коли капітал, що обернувся, більший, ніж капітал
авансований.

Нарешті, М' менше за m', коли n менше за 1, тобто коли капітал,
що обернувся протягом року, є лише частина авансованого капіталу, отже,
коли період обороту триває більш як рік.

Зупинімось трохи на цьому останньому випадку.

Ми зберігаємо всі припущення нашого попереднього прикладу, хай
тільки період обороту продовжиться до 55 тижнів. Процес праці потребує
щотижня 100 ф. стерл. змінного капіталу, отже, 5500 ф. стерл. для періоду
обороту, і продукує щотижня $100 m$; отже, m, як і перше, = 100\%.
Число оборотів n дорівнює тут 55: 50 = 10: 11, бо час обороту (беручи рік
50 тижнів) = 1 + 1/10 року = 11: 10 року. М' = 100\%X5500Х10/11: 5500 =
100\%Х10/11 = 1000: 11\% = 90Х10/11\%, отже, менше, ніж 100\%. Справді,
коли б річна норма додаткової вартости була 100\%, то $5500 v$ протягом
року мусили б випродукувати $5500 m$, тимчасом як для цього треба 11: 10
року. Ці $5500 v$ продукують протягом року лише $5000 m$, отже, річна
норма додаткової вартости = $5000 m$: $5000 v$ = 10: 11 = 90X10: 11\%.

Тому річна норма додаткової вартости, або відношення між додатковою
вартістю, спродукованою протягом року, і взагалі авансованим
змінним капіталом (на відміну від змінного капіталу, що обернувся
протягом року), не є просте суб’єктивне відношення, а самий
справжній рух капіталу викликає це зіставлення. Наприкінці року до
власника капіталу А повернувся авансований ним змінний капітал, рівний
500 ф. стерл., і крім того 5000 ф. стерл. додаткової вартости. Величину
авансованого ним капіталу виражає не та маса капіталу, що її він застосував
протягом року, а та, що періодично до нього повертається. В
розглядуваному питанні не має жодного значення, чи існує капітал наприкінці
року почасти як продукційний запас, чи почасти як товаровий
або грошовий капітал, і в якому відношенні розподіляється він на ці
різні частини. Для власника капіталу В повернулись 5000 ф. стерл., авансований
\index{ii}{0230}  %% посилання на сторінку оригінального видання
ним капітал, та ще 5000 ф. стерл. додаткової вартости. Для власника
капіталу C (останнього розглянутого нами капіталу в 5500 ф. стерл.)
спродуковано протягом року 5000 ф. стерл. додаткової вартости (при витраті
5000 ф. стерл. і нормі додаткової вартости в 100\%), але авансований
ним капітал, а так само і спродукована додаткова вартість ще не повернулись
до нього.

М' = m'n виражає, що норма додаткової вартости, яка має силу
для змінного капіталу, застосованого протягом одного періоду обороту:

маса додаткової вартости, створена протягом одного періоду обороту:
змінний капітал, застосований протягом одного періоду обороту

має бути помножена на число періодів обороту або на число періодів
репродукції авансованого змінного капіталу — на число періодів, що протягом
їх він відновлює свій кругобіг.

В книзі І, розділ IV („Перетворення грошей на капітал“), а потім у
книзі І, розділ XXI („Проста репродукція“) ми бачили вже, що капітальну
вартість взагалі авансується, а не витрачається, бо ця вартість,
проробивши різні фази свого кругобігу, знову повертається до свого
вихідного пункту, та ще й збагачена додатковою вартістю. Це характеризує
її, як авансовану вартість. Час, що минає від її вихідного пункту
до пункту її повороту, і є той час, що на нього авансується її. Ввесь
кругобіг, що його перебігає капітальна вартість, вимірюваний часом від її
авансування де її повороту, становить її оборот, а час тривання цього обороту
становить період обороту. Коли цей період закінчився і кругобіг
вивершено, то та сама капітальна вартість може знову почати той самий
кругобіг, отже, і знову зростати в своїй вартості, утворювати додаткову
вартість. Коли змінний капітал, як в А, обертається десять разів на рік,
то протягом року тим самим авансованим капіталом десять разів утворюється
таку масу додаткової вартости, яка відповідає одному періодові
обороту.

Треба з’ясувати природу авансування з погляду капіталістичного суспільства.
Капітал А, що обертається десять разів протягом року, авансується
протягом року десять разів. На кожний новий період обороту його авансується
знову. Але разом з тим протягом року А ніколи не авансує нічого
більшого, ніж ту саму капітальну вартість в 500 ф. стерл., і дійсно,
для розглядуваного нами продукційного процесу він ніколи не має в
своєму розпорядженні нічого більшого понад ці 500 ф. стерл. Скоро тільки
ці 500 ф. стерл. закінчують один кругобіг, капіталіст А повертає їх
знову на такий самий кругобіг: капітал з природи своєї зберігає характер
капіталу лише тому, що він завжди в повторюваних процесах продукції
функціонує як капітал. Його тут ніколи не авансується на довший час,
ніж 5 тижнів. Коли оборот триватиме довший час, то капіталу не вистачить.
Коли час обороту скорочується, то частина капіталу стає надлишковою.
Тут авансується не десять капіталів по 500 ф. стерл., а один
\parbreak{}  %% абзац продовжується на наступній сторінці
