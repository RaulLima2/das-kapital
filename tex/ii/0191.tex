першого, ці функції за першого періоду обороту точно відмежовані одна
від однієї, або принаймні їх можна точно відмежувати, тимчасом як протягом
другого періоду обороту вони, навпаки, переплітаються одна з
однією.

Уявімо собі справу наочніше:

Перший період обороту триває 12 тижнів. Перший робочий період —
9 тижнів; оборот авансованого на нього капіталу закінчується на початку
13-го тижня. Протягом останніх 3 тижнів функціонує додатковий капітал
в 300 ф. стерл., який починає другий дев’ятитижневий робочий,
період.

Другий період обороту. На початку 13-го тижня 900 ф. стерл. припливають
назад і можуть почати новий оборот. Але другий робочий
період уже на десятому тижні почато за допомогою додаткових 300 ф.
стерл.; на початку 13-го тижня за допомогою тих самих 300 ф. стерл.
уже закінчено третину робочого періоду, 300 ф. стерл. з продуктивного
капіталу перетворено на продукт. А що для закінчення другого робочого
періоду треба ще лише 6 тижнів, то в процес продукції другого робочого
періоду можуть ввійти лише дві третини капіталу в 900 ф. стерл.,
який повернувся назад, а саме 600 ф. стерл. З первісних 900 ф. стерл.
звільнилося 300 ф. стерл., щоб відігравати ту саму ролю, яку відігравав
у першому робочому періоді додатковий капітал в 300 ф. стерл. Наприкінці
6-го тижня другого періоду обороту закінчено другий робочий
період. Витрачений на нього капітал в 900 ф. стерл. повертається за три
тижні, отже, наприкінці 9-го тижня другого дванадцятитижневого періоду
обороту. Протягом 3 тижнів його часу обігу ввіходить у робочий період
звільнений капітал в 300 ф. стерл. З ним починається на 7-й тиждень
другого періоду обороту або на 19-й тиждень року третій робочий
період капіталу в 900 ф. стерл.

Третій період обороту. Наприкінці 9-го тижня другого періоду обороту
знову зворотно припливають 900 ф. стерл. Але третій робочий
період почався вже на сьомому тижні попереднього періоду обороту й
6 тижнів його вже минуло. Тому він триває тільки три тижні. Отже,
з 900 ф. стерл., що повернулись назад, у процес продукції ввіходять
лише 300 ф. стерл. Четвертий робочий період заповнює дев’ятитижневу
решту цього періоду обороту, і таким чином з 37-го тижня року починається
одночасно четвертий період обороту й п’ятий робочий період.

Щоб спростити обчислення, ми припустимо робочий період в 5 тижнів,
час обігу в 5 тижнів, отже, період обороту в 10 тижнів; рік рахуватимемо
в 50 тижнів, а щотижневу витрату капіталу рахуватимемо в 100 ф.
стерл. Отже, робочий період потребує поточного капіталу в 500 ф. стерл.,
а час обігу потребує додаткового капіталу — нових'500 ф. стерл. Робочі
періоди й періоди оборотів позначиться тоді так:

1-й робочий період: тижні 1—5 (500 ф. стерл. товару повертаються
наприкінці 10 тижня).

2-й робочий період: тижні 6—10 (500 ф. стерл. товару повертаються
наприкінці 15 тижня).
