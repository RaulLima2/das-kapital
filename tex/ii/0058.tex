форма Г', П, Т' завжди є перетворена форма якоїбудь попередньої функціональної форми кругобігу, що
не є первісна форма.

Так Г' в I є перетворена форма Т', кінцеве П в II — перетворена форма Г (і в I і в II цього
перетворення досягається простим актом товарової циркуляції, в наслідок формального переміщення
товару й грошей); Т' в III є перетворена форма П, продуктивного капіталу. Але тут, у III,
перетворення стосується, поперше, не лише до функціональної форми капіталу, але також і до величини
його вартости; подруге, перетворення є результат не просто формального переміщення, властивого
процесові циркуляції, але дійсного перетворення, що його проробили в процесі продукції споживна
форма і вартість товарових складових частин продуктивного капіталу.

Форму початкового члена Г, П, Т' наперед дається для кожного кругобігу — I, II, III; форму, що знову
повторюється в кінцевому члені, дано, а, значить, і зумовлено рядом метаморфоз самого кругобігу. Т',
як кінцевий пункт кругобігу індивідуального промислового капіталу, має собі за передумову лише
неналежну до циркуляції форму П того самого промислового капіталу, що продукт його є Т'; П, як
кінцевий пункт в I, як перетворена форма Т' (Т' — Г'), має собі за передумову, що Г є в руках
покупця, існує поза кругобігом Г... Г' і лише в наслідок продажу Т' втягується в цей кругобіг і стає
його кінцевою формою. Таким чином, в II кінцеве П має собі за передумову Р і Зп (Т), як наявні поза
кругобігом і введені в кругобіг як його кінцева форма через Г — Т. Але коли облишити осторонь
останній крайній член, то ні кругобіг індивідуального грошового капіталу не припускає у своєму
кругобігу буття грошового капіталу взагалі, ні кругобіг індивідуального
продуктивного капіталу не припускає в своєму кругобігу буття продуктивного капіталу. В I може Г бути
першим грошовим капіталом, в II може П бути першим продуктивним капіталом, що виступає на кін
історії, але в III

Т' Т — — Г' т — Г — Т Р Зп... П... Т' г — т

припускається, що Т двічі існує поза кругобігом. Одного разу в кругобігу Т' — Г' — Т Р Зп. Це Т,
оскільки воно складається з Зп, є товар у руках продавця; воно саме є товаровий капітал, оскільки
воно є продукт капіталістичного продукційного процесу; а коли навіть і ні, то воно з’являється як
товаровий капітал у руках купця. Другого разу, в т — г — т в другому т, яке, щоб його можна було
купити, так само мусить бути наявне як товар. У всякому разі Р і Зп, хоч вони є това-
