для кожного індивідуального капіталу, розглядуваного окремо, отже, ніж
вона виступає з погляду кожного поодинокого капіталіста. Для останнього
товарова вартість розкладається на 1) сталий елемент (четвертий,
як каже Сміс) і 2) на суму заробітної плати і додаткової вартости, зглядно
заробітної плати, зиску і земельної ренти. Навпаки, з суспільного погляду,
четвертий елемент Сміса, стала капітальна вартість зникає.

5) Резюме

Безглузда формула, що згідно з нею три відміни доходів, заробітна
плата, зиск, рента, становлять три складові частини товарової вартости,
випливає в А. Сміса з правдоподібнішої формули, за якою товарова
вартість rеsolves itself, розкладається на ці три складові частини. Однак
і це неправильно, навіть коли припустити, що товарову вартість можна
розподілити лише на еквівалент зужитої робочої сили й на утворену
нею додаткову вартість. Але й ця помилка й собі ґрунтується тут на
глибшій, правильній засаді. Капіталістична продукція ґрунтується на тому,
що продуктивний робітник продає свою власну робочу силу, як свій
товар, капіталістові, в чиїх руках вона потім функціонує просто як елемент
його продуктивного капіталу. Ця, належна до сфери циркуляції,
оборудка — продаж і купівля робочої сили — не лише є вступ до процесу
продукції, але вона й визначає implicite *) його специфічний характер.
Продукція споживної вартости і навіть продукція товару (бо її можуть
провадити і не залежні продуктивні робітники) тут є лише засіб для продукції
абсолютної та відносної додаткової вартости для капіталістів. Тому, аналізуючи
процес продукції, ми бачили, як продукцію абсолютної та відносної
додаткової вартости визначає: 1) протяг щоденного процесу праці,
2) ввесь суспільний і технічний устрій капіталістичного процесу продукції.
В ньому самому здійснюється ріжниця між простим збереженням вартости
(сталої капітальної вартости), справжньою репродукцією авансованої вартости
(еквіваленту робочої сили) і продукцією додаткової вартости, тобто
вартости, що за неї капіталіст не авансував жодного еквіваленту раніше,
ані авансує його post festum.

Хоч привласнення додаткової вартости — вартости, яка являє надлишок
над еквівалентом авансованої капіталістом вартости — підготовляється
купівлею й продажем робочої сили, однак воно є акт, що відбувається
в самому процесі продукції й становить істотний елемент його.

Вступний акт, що є акт циркуляції — купівля й продаж робочої сили — і
собі ґрунтується на розподілі елементів продукції, що відбувся перед
розподілом суспільних продуктів і був передумовою його, а саме на
відокремленні робочої сили як товару робітника від засобів продукції
як власности не-робітників.

Але разом з тим це привласнення додаткової вартости або це роз-

*) Implicite — приховано в собі. Ред.
