продукції, а 3000 — вартість засобів споживання. Отже, вартість суспільного
доходу (v + m) становить тільки 1/3 вартости сукупного продукту,
і сукупність споживачів — робітники і капіталісти — лише на суму вартости
цієї третини можуть брати з цілого суспільного продукту товари,
продукти, і заводити їх у фонд свого споживання. Навпаки, 6000 = 2/3
вартости продукту є вартість сталого капіталу, що його треба замістити
in natura. Отже, засоби продукції на таку суму треба знову ввести в
продукційний фонд. Неминучість цього бачив уже Шторх, хоч і не міг
довести цього: „Очевидно, що вартість річного продукту розкладається
почасти на капітал, почасти на зиск, і що кожна з цих частин вартости річного
продукту регулярно купує продукти, потрібні нації так для підтримання
свого капіталу, як і для відновлення свого споживного фонду... Продукти,
що становлять капітал нації, не можуть споживатись“.*) (Storch:
„Considérations sur la nature du revenu national". Paris. 1824, p. 150).

Однак A. Сміс подав цю казкову догму, — а їй і досі йметься
віру — не тільки у тій вже вище згаданій формі, що згідно з нею сукупна
вартість суспільного продукту розкладається на дохід, на заробітну
плату плюс додаткова вартість, або — як він каже — на заробітну плату
плюс зиск (процент) плюс земельна рента. Він подав її ще в популярнішій
формі, ніби споживачі, кінець-кінцем (ultimately), мусять оплатити
продуцентам усю вартість продукту. Це й досі лишається
одним з на віру прийнятих загальників або навіть однією з вічних істин
для так званої науки політичної економії. Цю думку хочуть унаочнити таким
на позір правдоподібним способом. Візьмімо якийбудь предмет, напр,
полотняні сорочки. Насамперед прядун лянної пряжі повинен оплатити
льонівникові всю вартість льону, тобто насіння, добрива, корму для робочої
худоби і т. ін., а також ту частину вартости, що її основний
капітал льонівника, як от будівлі, сільсько-господарський реманент і т. ін.,
„передає продуктові; заробітну плату, виплачену протягом продукції
льону; додаткову вартість (зиск, земельну ренту), яка міститься в льоні;
нарешті, витрати на перевіз льону від місця його продукції до прядільні.
Потім ткач повинен повернути прядунові лянної пряжі не лише цю
ціну льону, а й ту частину вартости машин, будівель тощо, коротко,
основного капіталу, що її перенесено на льон; далі, всі зужитковані
в процесі прядіння допоміжні матеріяли, заробітну плату прядунів, додаткову
вартість і т. ін. — і так само далі стоїть справа з білильником, з витратами
на транспорт готового полотна, нарешті, з фабрикантом сорочок,
який оплатив усю ціну всіх попередніх продуцентів, які дали йому те,
що для нього є лише сировинний матеріял. В його руках далі відбувається
долучення нової вартости: почасти вартости сталого капіталу, зужитко-

*) „Il est clair que la valeur du produit annuel se distribue partie en capitaux
et partie en profits, et que chacune de ces portions de la valeur du produit annuel
va régulièrement acheter les produits dont la nation a besoin, tant pour entretenir
son capital que pour renouveler son fond consommable... les produits qui constituent
le capital d’une nation, ne sont point consommables“.
