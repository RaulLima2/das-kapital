бере участи в процесі зростання вартости, лишається грошовою сумою,
яка зростає лише тому, що гроші, які існують поза її участю, складається
до тієї самої скрині.

Форма скарбу є лише форма грошей, що не перебувають в циркуляції,
грошей, що їхню циркуляцію перервано, і тому їх зберігається
в їхній грошовій формі. Щодо самого процесу утворення скарбу, то
він властивий усякій товаровій продукції і відіграє ролю як самоціль
лише в нерозвинених передкапіталістичних формах товарової продукції.
Але в нашому прикладі скарб виступає як форма грошового капіталу, а
утворення скарбу — як процес, що тимчасово супроводить акумуляцію
капіталу, тому і остільки, що і оскільки гроші фігурують тут як
лятентний грошовий капітал; бо утворення скарбу, перебування в стані
скарбу тієї додаткової вартосте, яка є в грошовій формі, являє тут функціонально
визначену підготовчу стадію для перетворення додаткової
вартосте на справді діющий капітал, — стадію, що відбувається поза
кругобігом капіталу. Отже, своїм призначенням це є лятентний грошовий
капітал; тому й розміри, що їх мусить дійти скарб, щоб увійти в
процес, визначається кожного разу вартісним складом продуктивного
капіталу. Але поки гроші лишаються в стані скарбу, вони ще не функціонують
як грошовий капітал, вони ще є бездіяльний капітал, не тому,
що, як раніше, їхню функцію перервано, а тому, що вони ще не здатні
до своєї функції.

Ми беремо тут нагромадження грошей в його первісній реальній
формі, як дійсний грошовий скарб. Але воно може існувати і в
формі простих боргових документів, боргових вимог капіталіста,
що продав Т'. Щодо інших форм, коли цей лятентний грошовий
капітал і в проміжний період існує в формі грошей, що вилуплюють
гроші, напр., як вклади в який-будь банк на проценти, в векселях
або цінних паперах будь-якого ґатунку, то ці форми сюди не стосуються.
Додаткова вартість, реалізована в грошах, виконує тоді особливі функції
капіталу поза кругобігом того промислового капіталу, що з нього
вона походить; функції, що, поперше, не мають нічого спільного з кругобігом
цього капіталу як таким і, подруге, припускають функції капіталу,
відмінні від функцій промислового капіталу, які тут ще не досліджені.

IV. Резервний фонд

Скарб у щойно розгляненій формі, як формі існування додаткової
вартости, є грошовий фонд акумуляції, грошова форма, що її тимчасово
має акумуляція капіталу, і яка остільки сама є умова акумуляції.
Але цей акумуляційний фонд може виконувати і особливі побічні послуги,
тобто входити в процес кругобігу капіталу, не набираючи форми П... П',
а, значить, не поширюючи розмірів капіталістичної репродукції.

Коли процес Т' — Г' триває понад свій нормальний протяг, коли, отже,
перетворення товарового капіталу на грошову форму ненормально затримується;
або коли, після того як це перетворення відбулося, ціна, напр.,
