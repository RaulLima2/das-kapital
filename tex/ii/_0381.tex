\parcont{}  %% абзац починається на попередній сторінці
\index{ii}{0381}  %% посилання на сторінку оригінального видання
при індивідуальному капіталі — послідовне осаджування його зужиткованої основної складової частини б
формі грошей, що їх нагромаджується як скарб, виражається і в річній суспільній репродукції.

Коли індивідуальний капітал = $400с + 100v$, а річна додаткова вартість — 100, то товаровий продукт —
$400с + 160v + 100m$. Ці 600 перетворюється на гроші. З цих грошей 400с знову перетворюється на
натуральну форму сталого капіталу, $100v$ — на робочу силу і, крім того, — якщо додаткова вартість
нагромаджується, — $100m$ перетворюється на додатковий сталий капітал через обмін на натуральні
елементи продуктивного капіталу. При цьому припускається: 1) що в даних технічних умовах цієї суми
досить або для збільшення діющого сталого капіталу, або для відкриття нового промислового
підприємства. Але може бути й таке, то треба протягом куди довшого часу перетворювати додаткову
вартість на гроші й нагромаджувати ці гроші як скарб, раніш ніж стає можливий цей процес, отже,
раніш ніж може настати справжня акумуляція, поширення продукції. 2) Припускається, що в дійсності
продукція в поширеному маштабі настала вже раніше, бо для того, щоб гроші (нагромаджену в грошах
додаткову вартість) можна було перетворити на елементи продуктивного капіталу, треба, щоб ці
елементи вже можна було купити на ринку як товари; при цьому не має ніякого значення, коли їх не
купується як готові товари, а виготовляється на замовлення. Їх оплачується лише після того, як вони
вже є в наявності, і в усякому разі після того, як відносно них вже дійсно постала репродукція в
поширеному маштабі, постало поширення нормальної до того часу продукції. Вони мусіли існувати
потенціяльно, тобто в своїх елементах, бо для того, щоб їх в дійсності виробилось, треба було лише
поштовху — замовлення, тобто попередньої купівлі й актиципованого продажу товару до його буття. Тоді
гроші на одному боці покликають до життя поширену репродукцію на другому боці тому, що можливість
для неї є вже й без грошей, бо гроші самі собою не є елемент справжньої репродукції.

Коли, напр., капіталіст А протягом одного року або кількох років продає послідовно продуковані ним
маси товарового продукту, то разом з тим він послідовно перетворює на гроші й ту частину товарового
продукту, яка є носій додаткової вартосте, — додатковий продукт, — отже, саму додаткову вартість,
сиродуковану ким в товаровій формі, — помалу нагромаджує ці гроші, і таким чином утворюється новий
потенціяльний грошовий капітал; потенціальний, бо він здібний і призначений до перетворення на
елементи продуктивного капіталу. А в дійсності капіталіст нагромаджує лише простий скарб, що не є
елемент справжньої репродукції. При цьому його діяльність насамперед сходить на послідовне вилучення
з циркуляції обігових грошей, при чому звичайно не виключається, що ці обігові гроші, які він таким
чином тримає замкнені в скрині, перш ніж увійти в циркуляцію, сами були частиною другого скарбу. Цей
скарб капіталіста А, що потенціяльно є новий грошовий капітал, так само не є додаткове суспільне
багатство, як коли б його витратилось на засоби споживання. Але гроші, вилучені з обігу, отже,
гроші,
\parbreak{}  %% абзац продовжується на наступній сторінці
