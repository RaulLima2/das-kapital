\parcont{}  %% абзац починається на попередній сторінці
\index{ii}{0327}  %% посилання на сторінку оригінального видання
капітальної вяртости за своєю натуральною формою складається почасти
з засобів продукції кожної окремої сфери продукції, почасти з робочої
сили, потрібної для цієї сфери продукції й відповідно кваліфікованої і
модифікованої в наслідок поділу праці, залежно від специфічного роду
праці, що його вона має виконувати в кожній окремій сфері продукції.
Частина суспільного капіталу, вкладена в кожну окрему галузь продукції,
теж складається з суми вкладених у неї самостійно діющих поодиноких
капіталів. Звичайно, це однаково має силу для обох підрозділів —
і І і II.

Щождо сталої капітальної вартости, яка знову з’являється в І підрозділі
в формі його товарового продукту, то вона також почасти знову
входить як засоби продукції в ту особливу сферу продукції (або навіть у те
індивідуальне підприємство), що з нього вона вийшла як продукт; напр.,
зерно в зерновій продукції, вугілля в вугільній продукції, залізо в формі
машин у продукції заліза тощо.

Однак, оскільки частинні продукти, що з них складається стала капітальна
вартість І підрозділу безпосередньо не входять знову в свою
особливу або індивідуальну сферу продукції, вони лише змінюють своє
місце. Вони входять у натуральній формі в другу сферу продукції підрозділу
І, тимчасом як продукт інших сфер продукції підрозділу І заміщує
ix in natura. Це — просте переміщення цих продуктів. Замість однієї
групи вони всі знову входять у другу групу підрозділу І як чинники,
що заміщують сталий капітал в І. Оскільки тут відбувається обмін між
поодинокими капіталістами І, він є обмін однієї натуральної форми сталого
капіталу на другу натуральну форму сталого капіталу, одного сорту засобів
продукції на інший сорт засобів продукції. Це — взаємний обмін
різних індивідуальних частин сталого капіталу І. Продукти, оскільки вони
безпосередньо не служать як засоби продукції в своїй власній галузі продукції,
пересуваються з місця своєї продукції в інше й таким чином
заміщують один одного. Інакше кажучи (так само, як то є в II з додатковою
вартістю): кожен капіталіст І пропорційно до того, оскільки він є
співвласник цього сталого капіталу в 4000, вилучає з цієї товарової маси
потрібні йому відповідні засоби продукції. Коли б продукція була суспільна,
а не капіталістична, то очевидно, що ці продукти підрозділу І
для завдань репродукції не менш постійно розподілялось би знову як
засоби продукції між галузями продукції цього підрозділу: одна частина
безпосередньо лишалась би в тій сфері продукції, відки вона вийшла як
продукт, а друга, навпаки, пересувалася б в інші місця продукції, і таким
чином між різними місцями продукції цього підрозділу був би постійний
рух у протилежних напрямках.

VII. Змінний капітал і додаткова вартість в обох підрозділах

Отже, вся вартість спродукованих протягом року засобів споживання
дорівнює репродукованій протягом року змінній капітальній вартості II
плюс новоспродуковака додаткова вартість II (тобто дорівнює ваотості,
\parbreak{}  %% абзац продовжується на наступній сторінці
