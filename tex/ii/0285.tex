дохід, кінець-кінцем, походить від одного з них“ (кн. І, розд. 6,
стор. 48), — то в цих словах зібрано до купи всілякі qui pro quo.

1) Всі члени суспільства, що безпосередньо — хоч працюючи, хоч
без праці — не функціонують у процесі репродукції, можуть одержати
свою пайку річного товарового продукту, тобто засоби свого споживання,
насамперед лише з рук тих кляс, що їм у першу чергу дістається
продукт: з рук продуктивних робітників, промислових капіталістів і
землевласників. В цьому розумінні їхні доходи матеріяльно походять із
заробітної плати (продуктивних робітників), зиску й земельної ренти, а
тому вони протистоять цим первинним доходам, як доходи похідні.
Однак, з другого боку, одержувачі цих похідних в такому розумінні доходів
здобувають їх в наслідок своєї суспільної функції як королі,
попи, професори, повії, вояки тощо; це дає їм змогу вбачати в цих
своїх функціях первинні джерела їхніх доходів.

2) І тут доходить кульмінаційного пункту чудна помилка А. Сміса:
почавши з правильного визначення складових частин вартости товару і
суми тих новоспродукованих вартостей, що втілені в цих частинах; з’ясувавши
потім, як ці складові частини утворюють стільки ж різних
джерел доходу\footnote{
Я подаю це речення буквально, як воно є в рукопису, хоч у даному зв’язку
воно ніби суперечить і попередньому й безпосередньо дальшому. Цю позірну
суперечність розв’язується далі під цифрою 4: „Капітал і дохід в А. Сміса“. — Ф. Е.
}, виснувавши таким чином доходи з вартости, він іде
потім зворотним напрямком — і це лишається в нього домінантним уявленням
— і перетворює доходи з „складових частин“ (component parts)
на „первісні джерела всякої мінової вартости“, розкриваючи цим
широко двері вульґарній економії. (Див. нашого Рошера).

3) Стала частина капіталу

Подивімось тепер, яким чаклуванням намагається А. Сміс винищити
в товаровій вартості сталу частину вартости капіталу.

„Частина ціни зерна, напр., оплачує ренту землевласника“. Походження
цієї складової частини вартости так само не має чинення до тієї
обставини, що цю частину виплачується землевласникові, і що вона для
нього становить дохід у формі ренти, як походження інших складових
частин вартости не має чинення до того, що вони як зиск і заробітна
плата становлять джерела доходу.

„Друга частина оплачує заробітну плату й утримання робітників“
(і робочої худоби! — додає він до цього), „що були зайняті в продукції
зерна, а третя частина оплачує зиск фармера. Ці три частини, як здається
(seem, в дійсності так здається), „або безпосередньо, або кінець-кінцем
становлять усю ціну зерна“\footnote{
Ми вже зовсім не кажемо про те, що Адамові особливо не пощастило з
його прикладом. Вартість зерна тільки тому розкладається на заробітну плату,
зиск і ренту, що корм, спожитий робочою худобою, подано як заробітну плату
робочої худоби, а саму робочу худобу — як найманих робітників, а тому й найманого
робітника — як робочу худобу. (Додаток з рукопису II).
}. Вся ця ціна, тобто визначення її