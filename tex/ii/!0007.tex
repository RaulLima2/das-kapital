одне більш-менш викінчене оброблення книги II, і датовано його
1870 роком. У примітках для остаточної редакції, що про них мова
буде зараз, зазначено виразно: „В основу треба покласти друге оброблення“.
Після 1870 року знову постала перерва, зумовлена, головним чином,
хоробливим станом Маркса. Своїм звичаєм Маркс присвятив цей час
студіям. Агрономія, американські та особливо російські земельні відносини,
грошовий ринок і банківська справа, нарешті, природничі науки:
геологія і фізіологія, а особливо самостійні математичні роботи становлять
зміст численних Марксових записних зшитків цього часу. На початку
1877 року він почував себе так добре, що знову міг узятись
до своєї справжньої роботи. Кінцем березня 1877 р. датовано посилання
й замітки з чотирьох вищезгаданих рукописів, які являли основу
того нового перероблення II книги, що початок його є в рукопису V
(56 сторінок in folio). Він охоплює перші чотири розділи й ще мало
оброблений; посутні пункти розроблено в примітках під текстом; матеріял
скорше зібрано, ніж просіяно, але це останній викінчений виклад цієї найважливішої
частини першого відділу. — Першу спробу зробити з цього
рукопис, готовий до друку, являє рукопис VI (написаний після жовтня
1877 року й до липня 1878 року); в ньому лише 17 сторінок чвертьаркушевих,
які охоплюють більшу частину першого розділу; а другу — і
останню — спробу являє рукопис VII, датований „2 липня 1878 р.“, що
має лише 7 сторінок in folio.

Того часу Маркс, здається, зрозумів, що, як не буде ґрунтовного перевороту
в стані його здоров’я, він ніколи не матиме змоги закінчити оброблення
другої й третьої книги так, щоб це задовольняло його самого. І справді
на рукописах V—VIII відбилась у багатьох місцях надмірно напружена
боротьба з пригнітною недугою. Найскладнішу частину першого відділу
було перероблено наново в рукопису V, решта першого відділу і цілий
другий відділ (за винятком розділу сімнадцятого) не являли жодних
значних теоретичних труднощів; навпаки, третій відділ, — репродукція
та циркуляція суспільного капіталу, — на його думку, конче треба
було переробити. А саме, в рукопису II розглядалось репродукцію
спочатку без зв’язку з грошовою циркуляцією, що її упосереднює, а потім
ще раз у зв’язку з нею. Це треба було усунути і взагалі так переробити
цілий відділ, щоб він відповідав поширеному кругозорові автора.
Так постав рукопис VIII, зшиток, що має лише 70 сторінок чвертьаркушевих;
але скільки Маркс зумів утиснути в ці сторінки, це доводить
порівняння відділу III у друкованому вигляді, по вилученні з нього
місць, узятих з рукопису II.

І цей рукопис подає лише попереднє трактування предмету, при чому
насамперед малось на увазі визначити й розвинути новоздобуті, порівняно
з манускриптом II, погляди, залишаючи осторонь пункти, що про них
не можна було сказати нічого нового. Чималу частину розділу XVII
другого відділу, якій, окрім того, деяким чином стосується третього відділу,
внову перероблено й поширено. Логічна послідовність часто урива-
