\parcont{}  %% абзац починається на попередній сторінці
\index{ii}{0322}  %% посилання на сторінку оригінального видання
у циркуляцію тільки 1000 ф. стерл., він витяг з неї вдвоє більше. Звичайно,
це m, що перетворилось на гроші Г, одразу зникає в інших руках
(II) через те, що ці гроші витрачається на засоби споживання. Капіталісти I
вилучили лише стільки в грошах, скільки подали в циркуляцію вартості
в товарах; те, що ця вартість є додаткова вартість, тобто нічого не коштує
капіталістам, це абсолютно нічого не змінює в самій вартості цього товару;
отже, оскільки ходить про перетворення вартости в товаровій циркуляції,
ця обставина не має жодного значення. Перебування товарової вартости
в грошовій формі, звичайно минуще, так само, як минущі всі інші форми,
що їх перебігає в своїх перетвореннях авансований капітал. Воно триває
саме стільки часу, скільки минає від перетворення товару I на гроші до
наступного перетворення грошей I на товар II.

Коли б ми припустили коротші обороти, — або, розглядаючи справу
з погляду простої товарової циркуляції, швидший обіг грошей в циркуляції
— то ще меншої кількости грошей було б досить для того, щоб
пустити в циркуляцію обмінювані товарові вартості; сума грошей — за
даного числа послідовних обмінів — завжди визначається сумою цін, зглядно
сумою вартости товарів, що циркулюють. При цьому цілком байдуже,
в якій пропорції ця сума вартостей складається з додаткової вартости,
з одного боку, і капітальної вартости — з другого боку.

Коли б у нашому прикладі заробітну плату в І сплачувалось чотири
рази на рік, то 4 × 250 = 1000. Отже, 250 ф. стерл. грішми було б
досить для циркуляції І v — \sfrac{1}{2} II с і для циркуляції між змінним капіталом
І v і робочою силою І. Так само, коли б циркуляція між І m і
II c відбувалась чотирма оборотами, то для цього треба було б лише
250 ф. стерл. — отже, потрібна була б загальна грошова сума або грошовий
капітал в 500 ф. стерл. для циркуляції товарів на суму в 5000 ф. стерл.
Тоді додаткова вартість перетворювалась би на гроші не за два рази
послідовно половинами, а за чотири рази послідовно чвертями.

Коли в обміні під № 4 як покупець виступає не II, а І і, отже, витрачає
500 ф. стерл. грішми на засоби споживання такої самої вартости, то II
в обміні під № 5 купує на ті самі 500 ф. стерл. засоби продукції.
6) І на ті самі 500 ф. стерл. купує засоби споживання; 7) II на ці самі
500 ф. стерл. купує засоби продукції, отже, 500 ф. стерл., кінець-кінцем,
повертаються до І, як раніш повертались вони до II. Додаткова вартість
перетворюється тут на гроші, за допомогою грошей, що їх сам капіталістичний
продуцент цієї додаткової вартости витрачає на своє особисте
споживання — грошей, що репрезентують антиципований дохід, антициповане
надходження з додаткової вартости, яка є в товарі, що його треба
ще продати. Перетворення додаткової вартости на гроші відбувається не
через зворотний приплив 500 ф. стерл., бо І, крім 1000 ф. стерл.
в товарах І v, наприкінці обміну під № 4 подав у циркуляцію 500 ф.
стерл. грішми, і ці гроші є додаткові гроші, оскільки нам відомо, — а не
вторговані від продажу товару. Коли ці гроші припливають назад до 1,
то в них він повертає собі лише свої додаткові гроші, а не перетворює
на гроші свою додаткову вартість. Перетворення на гроші додаткової
\parbreak{}  %% абзац продовжується на наступній сторінці
