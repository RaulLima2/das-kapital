таліста, проте, таку, що відбувається поза циркуляцією його індивідуального
капіталу.

У нашому прикладі мали ми товаровий капітал Т' в 10.000 ф.
пряжі вартістю в 500 ф. стерл.; з них 422 ф. стерл. є вартість продуктивного
капіталу; як грошова форма 8440 ф. пряжі вони й далі
продовжують циркуляцію капіталу, почату Т', тимчасом як додаткова
вартість в 78 ф. стерл., грошова форма 1560 ф. пряжі, надлишкової
частини товарового продукту, виходить із цієї циркуляції і чинить
свій окремий шлях у межах загальної товарової циркуляції.

"схема"

г — т є ряд купівель на гроші, що їх капіталіст витрачає або на власне
товари, або на послуги для своєї поважної особи, або для сім’ї. Ці
купівлі розпорошені, відбуваються в різний час. Отже, гроші існують
тимчасово в формі певного грошового запасу або скарбу,
призначеного на поточне споживання, бо гроші, що їхня циркуляція
перервалась, перебувають у формі скарбу. Їхнє функціонування як
засобу циркуляції — а такі вони є і в своїй тимчасовій формі скарбу — не
входить у циркуляцію капіталу в його грошовій формі Г. Гроші тут
не авансується, а витрачається.

Ми припускали, що ввесь авансований капітал завжди цілком переходить
з однієї його фази до іншої; так само й тут ми припускаємо, що товаровий
продукт П має в собі всю вартість продуктивного капіталу
П = 422 ф. стерл. плюс додаткова вартість = 78 ф. стерл., утворена
протягом продукційного процесу. В нашому прикладі, де ми маємо справу
з подільним товаровим продуктом, додаткова вартість існує в формі
1560 ф. пряжі, цілком так само, як обчислена на 1 ф. пряжі, вона існує
у формі 2,496 унцій пряжі. Коли б, навпаки, товаровий продукт був, прим., машиною
в 500 ф. стерл. і такого самого складу щодо вартости, то хоча б
одна частина вартости цієї машини була рівна 78 ф. стерл. додаткової
вартости, все ж ці 78 ф. стерл. існували б лише в машині як цілому;
машину не можна поділити на капітальну вартість і додаткову вартість,
не розбиваючи її на куски й не знищуючи таким чином разом з її споживною
вартістю і її вартість. Отже, обидві складові частини вартости
можна лише ідеально уявляти собі як складові частини товарового тіла,
але не можна їх визначати як самостійні елементи товару Т', подібно до
кожного фунту пряжі, що його можна визначити як віддільний, самостійний
товаровий елемент 10000 ф. пряжі. У першому випадку цілий товар, товаровий
капітал, машина мусить бути цілком продана, раніше, ніж г зможе розпочати
свою окрему циркуляцію. Навпаки, коли капіталіст продає 8440 ф.,
то продаж дальших 1560 ф. являє цілком відокремлену циркуляцію
