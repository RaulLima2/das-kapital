III. Акумуляція грошей

Чи може г, додаткова вартість, перетворена на золото, відразу знову
прилучитись до капітальної вартости, що процесує, і таким чином разом
з капіталом Г як величина Г' увійти в процес кругобігу, — це залежить
від обставин, незалежних від простої наявности г. Коли г повинно правити
за грошовий капітал у другому, самостійному підприємстві, що його
мають закласти поряд першого, то, очевидно, г можна вжити на це
лише тоді, коли воно має мінімальну величину, потрібну для такого підприємства.
Коли г треба вжити на поширення початкового підприємства,
то знову таки відношення речевих чинників П і відношення їхньої
вартости потребують певної мінімальної величини г. Між усіма засобами
продукції, що діють у цьому підприємстві, є не лише певне якісне, але
й певне кількісне відношення, певна пропорційність розмірів. Ці
речеві відношення і відношення вартости, що їх носіями є речеві відношення
тих чинників, які входять у продуктивний капітал, визначають
мінімальний розмір, що його мусить мати г для того, щоб його можна
було перетворити як приріст продуктивного капіталу на додаткові засоби
продукції та додаткову робочу силу, або лише на перші. Напр., прядун
не може збільшити числа своїх веретен, коли він не набуде одночасно
відповідне число чухральних і тіпальних верстатів, не кажучи вже про
збільшені витрати на бавовну та заробітну плату, що їх зумовлює таке
поширення підприємства. А щоб перевести це поширення в життя, додаткова
вартість мусить уже становити чималу суму (звичайно нові
витрати обчислюються в 1 ф. стерл. на веретено). Доки г не має цього
мінімального розміру, кругобіг капіталу мусить кілька разів повторитись,
поки сума послідозно ним спороджених г матиме змогу функціонувати
разом з Г, тобто Г' — Т' — Р Зп. Навіть прості зміни деталів, напр., у прядильних машинах, оскільки
вони роблять їх продуктивнішими, потребують
збільшених витрат прядильного матеріялу, поширення машин, що
обробляють бавовну до прядіння тощо. Отже, у проміжний період г
нагромаджується, і його нагромадження є не його власна функція, а
результат повторних П... П. Його власна функція є його перебування в
грошовому стані, доки воно з повторних кругобігів, що дають приріст
вартости, тобто із-зовні, матиме достатній приріст, щоб досягти мінімальної
величини, потрібної для його активного функціонування, — величини,
що, тільки мавши її, воно як грошовий капітал, у даному разі, як акумульована
частина грошового діющого капіталу Г, дійсно може ввійти у функціонування
цього останнього. У проміжний період г нагромаджується й
існує лише в формі скарбу, що перебуває в процесі свого утворення, зростання.
Отже, грошова акумуляція, утворення скарбу виступає тут як процес,
що тимчасово супроводить справжню акумуляцію, поширення того маштабу,
що в ньому діє промисловий капітал. Тимчасово супроводить, бо
поки скарб перебуває в стані скарбу, він не функціонує як капітал, не
