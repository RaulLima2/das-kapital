циркуляцію, а вилучає з циркуляції 6000 ф. стерл.: 5000 ф. стерл. капіталу й
1000 ф. стерл. додаткової вартости. Ці 1000 ф. стерл. перетворились на
гроші за допомогою тих грошей, що їх вона сама пустила в циркуляцію
не як капіталіст, а як споживач, що їх вона не авансувала, а витратила.
Тепер вони повертаються до неї знову як грошова форма спродукованої
нею додаткової вартости. І з цього часу ця операція повторюється
щороку. Але, починаючи з другого року, 1000 ф. стерл., витрачувані нею,
завжди є вже перетворена форма, грошова форма спродукованої нею
додаткової вартости. Вона витрачає їх щороку і щороку ж вони повертаються
до неї назад.

Коли б капітал цього капіталіста обертався протягом року частіше,
то справа від цього ані трохи не змінилась би, але, звичайно, змінився
б протяг часу, а тому й величина тієї грошової суми, що її капіталістові
понад авансований ним грошовий капітал довелось би пускати в
циркуляцію на своє особисте споживання.

Ці гроші капіталіст пускає в циркуляцію не як капітал. Але, зрозуміло,
така вже властивість капіталіста, що доки до нього повернеться з
циркуляції додаткова вартість, він може існувати на ті засоби, які є в
його посіданні.

В цьому випадку ми припускали, що грошова сума, яку капіталіст
пускає в циркуляцію на покриття свого особистого споживання до першого
зворотного припливу його капіталу, точно дорівнює спродукованій ним
додатковій вартості, і тому має бути перетворена на гроші. Очевидно,
що таке припущення відносно поодинокого капіталіста довільне. Але
воно мусить бути правильне для цілої кляси капіталістів, якщо ми
припускаємо просту репродукцію. Воно виражає лише те, що є в цьому
припущенні, а саме, що всю додаткову вартість, але і тільки її, споживається
непродуктивно; що, отже, жодної частини первісного капіталу
не споживається непродуктивно.

Ми вище припускали, що цілої продукції благородних металів
(припустімо = 500 ф. стерл.) досить лише для того, щоб покрити зношування
грошей.

Капіталісти, що продукують золото, мають увесь свій продукт у
формі золота, так ті частини його, що покривають сталий і змінний
капітал, як і ту частину його, яка складається з додаткової вартости.
Отже, частина суспільної додаткової вартости складається з золота, а
не з такого продукту, що перетворюється на золото лише в циркуляції.
Вона з самого початку складається з золота, й пускається її в циркуляцію
для того, щоб вилучити з циркуляції продукти. Це саме тут стосується
до заробітної плати, змінного капіталу і до покриття авансованого
сталого капіталу. Отже, коли одна частина кляси капіталістів пускає в
циркуляцію товарову вартість більшу (на величину додаткової вартости),
ніж авансований ними грошовий капітал, то друга частина капіталістів
пускає в циркуляцію більшу грошову вартість (більшу на додаткову
вартість), ніж товарова вартість, що її вони постійно вилучають з циркуляції
для продукції золота. Якщо частина капіталістів постійно випом-
