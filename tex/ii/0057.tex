роблювати циркуляцію т — г — т, почасти функціонувати як елемент акумуляції капіталу.

У формі Т'... Т' споживання сукупного товарового продукту припускається, як умова нормального
перебігу кругобігу самого капіталу. Особисте споживання робітника та особисте споживання тієї
частини додаткової вартости, яку не акумулюється, охоплює все особисте споживання. Отже, споживання
в цілому — і особисте і продуктивне — входить як умова в кругобіг Т'. Продуктивне споживання (куди
посутньо входить і особисте споживання робітника, бо робоча сила в певних межах є постійний продукт
особистого споживання робітника) відбувається безпосередньо за допомогою кожного індивідуального
капіталу.
Особисте споживання остільки, оскільки воно потрібне для існування
індивідуального капіталіста — припускається тільки як суспільний акт, ні в якому разі як акт
індивідуального капіталіста.

У формах I і II ввесь рух зображено, як рух авансованої капітальної вартости. У формі III вирослий у
своїй вартості капітал, у вигляді цілого товарового продукту, становить вихідний пункт, і має форму
капіталу, що рухається, товарового капіталу. Лише після його перетворення на гроші, цей рух
розгалужується на рух капіталу й на рух доходу. В цій формі в кругобіг капіталу включено і розподіл
цілого суспільного продукту і особливий розподіл продукту для кожного індивідуального товарового
капіталу — розподіл, з одного боку, на фонд особистого споживання, а з другого — на фонд
репродукції.

Г... Г' містить у собі поширення кругобігу залежно від величини г, яке
входить у поновлений кругобіг.

П в П... П може почати новий кругобіг з тією самою вартістю, а може і з меншою — і все ж воно може
являти репродукцію в поширених розмірах; так, напр., тоді, коли елементи товару подешевшають у
наслідок підвищеної продуктивности праці. Навпаки, в протилежному випадку, вирослий своєю вартістю
продуктивний капітал може являти репродукцію в речево вужчих розмірах, коли, напр., елементи
продукції дорожчають. Те саме має силу й для Т'... Т'.

В Т'... Т' капітал у товаровій формі є передумова продукції; і знову таки як передумова повертається
він у межах цього кругобігу в другому Т. Якщо це Т ще не спродуковано або не репродуковано, то
кругобіг зупиняється: це Т мусить бути репродуковано здебільша як Т' якогось іншого промислового
капіталу. В цьому кругобігу Т' існує як вихідний пункт, переходовий пункт і кінцевий пункт руху, — і
воно тому завжди є наявне. Воно є постійна умова процесу репродукції.

Т'... Т' відрізняється від форм I і II ще й іншим моментом. У всіх трьох кругобігів спільне те, що
форма, в якій капітал починає процес свого кругобігу, є також і форма, що в ній він закінчує його, а
тому знову набирає початкової форми, в якій він знову починає той самий кругобіг. Початкова форма Г,
П, Т' є завжди та форма, що в ній авансується капітальна вартість (у III формі з прирослою до неї
додатковою вартістю) — отже, її первісна форма щодо кругобігу; кінцева
