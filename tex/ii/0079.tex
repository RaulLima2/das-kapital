припущенням, що за спонукальний мотив є особисте споживання, а не
саме збагачення, знищується саму основу капіталізму.

Але воно неможливе також і технічно. Капіталіст не лише мусить
утворити запасний капітал, щоб забезпечити себе проти коливань цін і
мати змогу чекати на сприятливу коньюнктуру для купівлі і продажу;
він мусить акумулювати капітал, щоб поширювати продукцію і вводити
технічні поліпшення в свій продуктивний організм.

Щоб акумулювати капітал, він мусить насамперед деяку частину додаткової
вартости, що до нього допливає з циркуляції, вилучати з циркуляції
в грошовій формі і збільшувати її як скарб доти, доки вона
дійде розмірів, потрібних для того, щоб поширити старе підприємство або
відкрити нове поряд старого. Поки триває скарботворення, попит капіталіста
не збільшується; гроші імобілізовано; вони не вилучають з товарового
ринку жодного товарового еквівалента за грошовий еквівалент, вилучений
з ринку за поданий товар.

Кредит ми лишаємо тут осторонь, а до кредитових відносин належить,
напр., те, що капіталіст, у міру нагромадження грошей, кладе їх у банк
на біжучий рахунок за проценти.

Розділ п’ятий

Час обігу8)

Рух капіталу через сферу продукції та дві фази сфери циркуляції
відбувається, як ми бачили, послідовно в часі. Протяг його перебування в
сфері продукції становить час його продукції, протяг перебування в
сфері циркуляції — час його циркуляції або час його обігу. Ввесь час, що
протягом його капітал робить свій кругобіг, дорівнює сумі часу продукції
та часу обігу.

Час продукції природно охоплює період процесу праці, але цей
останній не охоплює цілого часу продукції. Насамперед пригадаймо, що
одна частина сталого капіталу існує в засобах праці, як от машини,
будівлі тощо, які до останнього дня свого існування придаються в тих
самих, знову й знову повторюваних, процесах праці. Періодична перерва
процесу праці, напр., вночі, хоч і є перерва у функціонуванні цих засобів
праці, але не перерва у перебуванні їх на місці продукції. Вони
належать продукції не тільки, поки функціонують, а й тоді, коли не функціонують.
З другого боку, капіталіст мусить мати напоготові певний запас
сировинного матеріялу та допоміжних матеріялів, щоб процес продукції
протягом більш або менш довгого часу відбувався в заздалегідь визначених
розмірах, незалежно від випадковостей щоденного подання товарів
на ринку. Цей запас сировинного матеріялу тощо споживається продуктивно
лише поступінно. Відси постає ріжниця між його часом про-

8) Відси рукопис IV.
