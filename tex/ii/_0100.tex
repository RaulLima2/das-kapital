\parcont{}  %% абзац починається на попередній сторінці
\index{ii}{0100}  %% посилання на сторінку оригінального видання
і безперервність процесу циркуляції, а тому й процесу репродукції, що
має в собі і процес циркуляції.

Треба згадати, що Т' — Г' може вже відбутись для продуцента Т,
хоч Т все ще перебуває на ринку. Коли б сам продуцент захотів тримати
свій власний товар у себе на складах, доки його продасться остаточному
споживачеві, то він мусів би пустити в рух подвійний капітал:
один — як продуцент товару, другий — як купець. Для самого товару,
хоч розглядати його як поодинокий товар, хоч як складову частину
суспільного капіталу, справа зовсім не змінюється від того, чи витрати
на утворення запасу припадають на продуцента товару, чи на ряд купців,
від А до Z.

Оскільки товаровий запас є не щось інше, як товарова форма запасу,
що при даному маштабі суспільної продукції, коли б він не існував у
формі товарового запасу, існував би або як продуктивний запас (латентний
фонд продукції), або як споживний фонд (резерв засобів споживання),
остільки й витрати, що їх потребує зберігання запасу, отже, витрати на
утворення запасу, — тобто вжита для цього зречевлена або жива праця —
є лише витрати зберігання хоч суспільного продукційного фонду,
хоч суспільного фонду споживання. Підвищення вартости товару,
зумовлене ними, розподіляє ці витрати лише pro rata між різними
товарами, бо вони різні для різних сортів товару. Як і раніш,
витрати на утворення запасу лишаються одбавою із суспільного багатства,
хоч вони є умова його існування.

Лише оскільки товаровий запас є умова циркуляції товарів і навіть
форма, що доконечно постала в товаровій циркуляції, оскільки, отже,
цей позірний застій є форма самого руху, цілком так само, як утворення
грошового резерву є умова грошової циркуляції, — лише остільки
він є нормальний. Навпаки, скоро товари, що затрималися в резервуарах
циркуляції, не звільняють місця для наступної хвилі продукції,
скоро, отже, резервуари переповнюються, товаровий запас збільшується
в наслідок застою в циркуляції, цілком так само, як зростають
скарби, коли затримується грошова циркуляція. При цьому
байдуже, чи цей застій постає в амбарах промислових капіталістів, чи
на складах купця. Товаровий запас тоді є вже не умова безперервного
продажу, а наслідок того, що товари не сила продати. Витрати лишаються
ті самі, але що вони тепер випливають виключно з форми, а саме з
доконечности перетворити товари на гроші, та з труднощів у цій метаморфозі,
то вони не входять у вартість товару, а становлять одбаву, втрату
вартости при реалізації вартости. Що нормальна і анормальна форма
запасу не відрізняється щодо форми, і обидві являють застій циркуляції,
то явища можна сплутати, та й самі агенти продукції допускаються
цієї помилки тим легше, що для продуцента процес циркуляції
його капіталу може перебігати, хоч процес циркуляції його товарів,
які перейшли в руки купців, зупинився. Коли більшають розміри
продукції та споживання, то, за інших незмінних обставин, збільшується
й товаровий запас. Він відновлюється й поглинається так само швидко,
\parbreak{}  %% абзац продовжується на наступній сторінці
