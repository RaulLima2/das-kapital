робітну плату. Навпаки, увесь капітал в 2500 ф. стерл. обертається
25000/2500 = 10 разів.

Витрачений протягом продукції змінний обіговий капітал може знову
функціонувати в процесі продукції лише остільки, оскільки продукт, що
в ньому репродуковано його вартість, продано, перетворено з товарового
капіталу на грошовий капітал, щоб знову витрачуватись на оплату робочої
сили. Але так само стоїть справа і з витраченим на продукцію сталим
обіговим капіталом (матеріялами продукції), що його вартість знову з’являється
в продукції як частина вартости продукту. Що ці обидві частини —
змінна та стала частина обігового капіталу — мають спільного, і що відрізняє
їх від основного капіталу, так це не те, що їхня вартість, перенесена
на продукт, циркулює за допомогою товарового капіталу, тобто
через циркуляцію продукту як товару. Деяка частина вартости продукту,
а тому й продукту, що циркулює як товар, тобто деяка частина товарового
капіталу завжди складається з зношуваної частини основного капіталу,
тобто з частини вартости основного капіталу, перенесеної на продукт
в процесі продукції. Але ріжниця ось у чому: основний капітал і
далі функціонує в процесі продукції в своїй старій споживній формі протягом
більш-менш довгого циклу періодів обороту обігового капіталу
(= обіговому сталому + обіговий змінний капітал); тимчасом як кожен
поодинокий оборот має собі за умову заміщення цілого обігового капіталу,
що ввійшов — у вигляді товарового капіталу — із сфери продукції в
сферу циркуляції. Перша фаза циркуляції Т' — Г' є спільна для поточного
сталого й поточного змінного капіталу. В другій фазі вони відокремлюються.
Гроші, що на них перетворився товар, перетворюються деякою
частиною на продукційний запас (обіговий сталий капітал). Відповідно до
різних термінів купівель складових частин цього запасу одна частина
його може утворитись через перетворення грошей на матеріяли продукції
раніше, друга пізніше, але, кінець-кінцем, так складається ввесь запас продукційних
матеріялів. Друга частина грошей, одержаних з продажу товарів,
лишається лежати як грошовий запас, щоб помалу витрачатись на
оплату робочої сили, введеної в процес продукції. Вона становить обіговий
змінний капітал. А проте, повне заміщення тієї або другої частини капіталу
є кожного разу наслідок обороту капіталу, його перетворення на продукт,
з продукту на товар, з товару на гроші. Саме це є причина того,
що в попередньому розділі ми розглядали оборот обігового капіталу — сталого
й змінного — окремо і разом, не звертаючи уваги на основний капітал.
В питанні, що його нам треба тепер дослідити, ми мусимо зробити ще
один крок далі й розглядати змінну частину обігового капіталу так, ніби
тільки вона одна й становить обіговий капітал; інакше кажучи, ми залишаємо
осторонь сталий обіговий капітал, що обертається разом з нею.

Авансовано 2500 ф. стерл., і вартість річного продукту = 25000 ф.
стерл. Але змінна частина обігового капіталу становить 500 ф. стерл.,
тому змінний капітал, що міститься в 25000 фунт, стерлінґів, дорівнює
