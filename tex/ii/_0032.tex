\parcont{}  %% абзац починається на попередній сторінці
\index{ii}{0032}  %% посилання на сторінку оригінального видання
капіталу, оскільки припускається за передумову капіталістичний спосіб
продукції, отже, за стану суспільства, визначеного капіталістичною продукцією.
Тому капіталістичний процес продукції припускається як prius\footnote*{
Щось попереднє, наперед дане. \emph{Ред.}
},
коли не в першому кругобігу грошового капіталу якогось нововкладеного
промислового капіталу, — то за межами цього кругобігу; постійна наявність
цього процесу продукції має собі за передумову постійно поновлюваний кругобіг $П... П$. У межах першої
стадії $Г-Т-Р$, Зп виступає вже ця передумова, бо цей акт припускає, з одного боку, наявність кляси
найманих робітників; а, з другого боку, тому, що те, що для покупця
засобів продуції є перша стадія $Г — Т$, для їхнього продавця є $Т' — Г'$;
отже, $Т'$ має собі за передумову товаровий капітал, а тому й самий
товар як наслідок капіталістичної продукції і тим самим має воно за
передумову і функціонування продуктивного капіталу.

Розділ другий
Кругобіг продуктивного капіталу

Загальна формула кругобігу продуктивного капіталу така: $П... Т' —
Г' — Т... П$. Цей кругобіг означає періодично відновлюване функціонування
продуктивного капіталу, отже, репродукцію, або його процес
продукції як процес репродукції щодо зростання вартости; не лише
продукцію, але періодичну репродукцію додаткової вартости; функціонування
промислового капіталу, що перебуває в своїй продуктивній формі,
не як одноразове, а як періодично повторюване функціонування, так що
відновлення його дано вже самим вихідним пунктом. Частина $Т'$ може
безпосередньо (в деяких випадках, в деяких галузях приміщення промислового
капіталу) в формі засобів продукції знову ввійти в той самий
процес праці, що з нього вона вийшла як товар; у наслідок цього усувається
лише перетворення її вартости на справжні гроші або грошові
знаки, або вона набирає самостійного виразу лише в формі розрахункових
грошей. Ця частина вартости не входить у циркуляцію. Таким чином
входять у процес продукції вартості, що не входять у процес циркуляції.
Це саме має силу й що до тієї частини $Т'$, що її капіталіст споживає
in natura як частину додаткової вартости. Однак ця частина не має значіння
для капіталістичної продукції; вона може мати значіння щонайбільш
у рільництві.

У цій формі впадають на очі дві обставини:

По-перше. Тимчасом, як у першій формі $Г — Г'$, циркуляція грошового
капіталу переривається функцією П, продукційним процесом, що виступає
лише як посередник між обома його фазами $Г — Т$ і $Т' — Г'$, тут цілий
\parbreak{}  %% абзац продовжується на наступній сторінці
