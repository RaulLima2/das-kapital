Капітал ІI

Періоди обороту    Робочі періоди    Авансовано    Періоди циркуляції
І. Тижні 4 1/2—13 1/2  Тижні 4 1/2—9    450 ф. ст. Тижні 10—13 1/2
II. „ 13 1/2—22 1/2 „ 13 1/2—18    450 „ „ „ 19—22 1/2
III. „ 22 1/2 — З1 1/2 „ 22 1/2—27    450 „ „ „ 28—31 1/2
IV. „ 31 1/2—40 1/2 „ 31  1/2—36    450 „ „ „ 37—40 1/2
V. „ 40 1/2—49 1/2 „ 40 1/2—45    450 „ „ „ 46—49 1/2
VI. „ 49 1/2 — [58 1/2]  „ 49 1/2 — [54] 450 „ „ [55 *) — 58 1/2]

Протягом 50 тижнів, що їх ми тут беремо за рік, капітал І закінчив
шість повних робочих періодів, отже, випродукував товарів на 450 × 6
= 2700 ф. стерл., а капітал II в п’ять повних робочих періодів — на
450 × 5 = 2250 ф. стерл. Крім того, капітал II в останні 1 1/2 тижні року
(з середини 50-го до кінця 51-го тижня**) випродукував ще на 150 ф.
стерл. — всього продукту за 51 тиждень випродукувано на 5100 ф. ст.
Отже, щодо безпосередньої продукції додаткової вартости — а її продукується
лише протягом робочого періоду — цілий капітал в 900 ф. стерл. обернувся
б 5 2/3 раза (900 × 5 2/3 = 5100 ф. стерл.). Але коли ми розглянемо
справжній оборот, то побачимо, що капітал І обернувся 5 2/3 раза, бо
наприкінці 51 тижня йому треба ще протягом 3 тижнів закінчувати
свій шостий період обороту; 450 × 5 2/3 = 2550 ф. стерл.; а капітал
II обернувся 5 1/6 раза, бо він проробив тільки 1 1/2 тижні свого
шостого періоду обороту, значить, ще 7 1/2 тижнів його припадуть на
наступний рік; 450 × 5 1/6 = 2325 ф. стерл., ввесь дійсний оборот дорівнює
4875 ф. стерл.

Розгляньмо капітал І й капітал II, як два цілком самостійні один проти
одного капітали. В своїх рухах вони цілком самостійні; ці рухи доповнюють
один одного тільки тому, що їхні робочі періоди та періоди
циркуляції безпосередньо чергуються один по одному. Їх можна розглядати
як два цілком незалежні капітали, що належать різним капіталістам.

Капітал І проробив п’ять повних періодів обороту і дві третини свого
шостого періоду обороту. Наприкінці року він перебуває в формі товарового
капіталу, що йому треба ще 3 тижні для своєї нормальної реалізації.
Протягом цього часу він не може ввійти в процес продукції.
Він функціонує як товаровий капітал: він циркулює. З свого останнього
періоду обороту він проробив лише 2/3. Це можна висловити так: він
обернувся лише 2/3 раза, лише 2/3 цілої вартости його зробили повний оборот.
Ми кажемо: 450 ф. стерл. пророблюють свій оборот у 9 тижнів, отже,
300 ф. стерл. — у 6 тижнів. При такому способі вислову нехтується органічні
відношення між обома специфічно різними складовими частинами часу обо-

*) В нім. тексті тут, очевидно, помилково стоїть „54“. Ред.

**) Дальше обчислення побудовано на припущенні 51 тижня в році. Ред.
