Рікардо забуває при цьому будинок, де живе робітник, його меблі,
знаряддя його споживання, напр., ножі, виделки, посуд і т. ін., що всі
своєю довготривалістю мають той самий характер, як і засоби праці. Ті
самі речі, ті самі кляси речей виступають тут як засоби споживання,
там — як засоби праці.

Ріжниця, як її висловлює Рікардо, ось у чому: „Відповідно до того,
чи зношується капітал швидко й потребує частої репродукції, чи зуживається
його повільно, його клясифікують як обіговий, або як основний
капітал“ 28).

До цього він робить помітку: „Розподіл непосутній, що в ньому, крім
того, немає змоги точно провести розмежувальну лінію“ 29).

Таким чином ми щасливо дійшли знову до фізіократів, що в них
ріжниця між avances annuelles і avances primitives була ріжницею в часі
споживання, а, значить, і в часі репродукції ужитого капіталу. Тільки те,
що в них виражає важливий для суспільної продукції феномен і в Tableau
économique подано також у зв’язку з процесом циркуляції, тут стає
суб’єктивним і, як каже сам Рікардо, зайвим відрізненням.

Якщо частина капіталу, витрачена на працю, відрізняється від частини
капіталу, витраченої на засоби праці, лише періодом своєї репродукції,
а тому й часом своєї циркуляції; якщо одна частина складається з засобів
існування так само, як друга з засобів праці, так що останні відрізняються
від перших лише ступенем швидкости зношування й при цьому
перші й собі мають різні ступені тривалости, — коли це так, то differentia
specifica\footnote*{
Характеристичні, відзначні риси. Ред.
} між капіталом, витраченим на робочу силу, і капіталом, витраченим
на засоби продукції, звичайно, стирається.

Це цілком суперечить Рікардовій теорії вартости так само, як і його
теорії зиску, що фактично є теорія додаткової вартости. Він розглядає
ріжпицю між основним і обіговим капіталом взагалі лише остільки
оскільки різні пропорції обох, при рівновеликих капіталах, впливають
в різних галузях підприємств на закон вартости, а саме, він розглядає,
як, в наслідок цих обставин, підвищення або зниження заробітної плати
впливає на ціни. Але навіть в обмежених рямцях цього досліду він, сплутуючи
основний та обіговий капітал із сталим та змінним, робить величезні
помилки і справді будує свій дослід на цілком хибній основі. А
саме: 1) оскільки частину капітальної вартости, витрачену на робочу
силу, підводиться під рубрику обігового капіталу, неправильно висновується
визначення самого обігового капіталу і особливо ті обставини,
що підводять частину капіталу, витрачену на працю, під цю рубрику;\footnote{
S) „According as capital is rapidly perishable and requires to be frequently
reproduced, or is of slow consumption it is classed under the heads of circulating,
or fixed capital“ (Ricardo, 1. c.).
} сплутується те визначення, що, згідно з ним, частина капіталу, витра-

33) „А division not essential, and in which the line ot demarcation cannot b'
accurately drawn“ (Ricardo, 1. c.).