\parcont{}  %% абзац починається на попередній сторінці
\index{ii}{0287}  %% посилання на сторінку оригінального видання
Але це й усе, що вони дають і можуть дати. Те, що має силу для одноденної промислової праці, має
силу й для тієї праці, що її вся кляса капіталістів пускає в рух протягом року. Тому, сукупну масу
новоспродукованої суспільної річної вартости можна розкласти лише на $v + m$, на еквівалент, що ним
робітники покривають капітальну вартість, витрачену на їхню власну купівельну ціну, і на додаткову
вартість, що її вони поверх цього мусять дати тому, хто застосовує їхню працю. Але ці обидва
елементи вартости товарів становлять разом із тим джерела доходу різних кляс, що беруть участь у
репродукції: перший — заробітну плату, дохід робітників; другий — додаткову вартість, що з неї
промисловий капіталіст залишає собі одну частину в формі зиску, а другу віддає як ренту, як дохід
землевласника. Отже, звідки могла б постати ще одна складова частина вартости, коли новоспродукована
річна вартість не має жодних інших елементів, крім $v + m$? Ми стоїмо тут на ґрунті простої
репродукції. Коли вся річна сума праці розкладається на працю, потрібну для репродукції капітальної
вартости, витраченої на робочу силу, і на працю, потрібну для утворення додаткової вартости, то
відки ще взагалі могла б постати праця для продукції капітальної вартости, витраченої не на робочу
силу?

Справа стоїть ось як:

1) А. Сміс визначає вартість товару тією масою праці, що її найманий робітник додає (adds) до
предмету праці. Він каже буквально „до матеріялів“, бо в нього мова мовиться про мануфактуру, яка
вже сама переробляє продукти праці; але це совсім не змінює справи. Вартість, що її робітник додає
(і це „adds“ є вислів Адама) до предмету, зовсім
не залежить від того, чи мав уже до цього долучення самий предмет, що до нього долучається вартість,
власну вартість, чи ні. Отже, робітник утворює нову вартість у товаровій формі; за А. Смісом частина
цієї вартости є еквівалент заробітної плати робітника, і цю частину, отже, визначається розміром
вартости його заробітної плати; щоб випродукувати або репродукувати вартість, рівну його заробітній
платі, йому
доводиться додавати більшу або меншу кількість праці, залежно від того, оскільки велика або мала
його заробітна плата. Але, з другого боку, робітник, понад визначувану таким чином межу, додає
дальшу працю, що утворює додаткову вартість капіталістові, що застосовує його. Чи лишається ця
додаткова вартість цілком у руках капіталіста, чи доводиться йому частину її віддати третім особам,
це зовсім нічого не змінює ні в якісному (що це взагалі є додаткова вартість), ні в кількісному
(щодо величини) визначенні додаткової вартости, долученої найманим робітником. Це — вартість, як і
всяка інша частина вартости продукту, але вона відрізняється тим, що робітник не одержав за неї
жодного еквіваленту й потім не одержить його; навпаки, цю вартість капіталіст привласнює без
еквіваленту. Ціла вартість товару визначається кількістю
праці, витраченої робітником на його продукцію; частину цієї цілої вартости визначено тим, що Бона
дорівнює вартості заробітної плати, отже, є її еквівалент. Тому другу частину, додаткову вартість,
неодмінно теж
\parbreak{}  %% абзац продовжується на наступній сторінці
