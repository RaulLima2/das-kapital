редерська школа від часів фізіократів і Адама Сміса. Ми знаємо, що основний
капітал, після того як витрату на нього вже раз зроблено, не відновлюється
протягом усього часу свого функціонування, а функціонує й
далі в старій формі, тммчасом як вартість його поступінно осаджується
в формі грошей. Ми бачили, що періодичне відновлення основного капіталу
II с [а вся капітальна вартість II с обмінюється на елементи вартістю
в І (v + m)] має за передумову, з одного боку, просту купівлю
основної частини II с, яка зворотно перетворюється з грошової форми
на натуральну, при чому цій купівлі відповідає простий продаж І m;
з другого боку, воно має за передумову простий продаж з боку
II с, продаж тієї основної (зношеної) частини його вартости, яка осаджується
в формі грошей, при чому цьому продажеві відповідає проста
купівля І m. Для того, щоб обмін відбувався тут нормально, треба припустити,
що проста купівля з боку II с величиною вартости дорівнює
простому продажеві з боку II с, і так само, що простий продаж І m
1-ій частині II с дорівнює простій купівлі II с, частини 2 (стор. 360).
Інакше просту репродукцію порушиться; проста купівля тут мусить
покриватись простим продажем там. Так само тут треба припустити,
що простий продаж частини І m, яка утворює скарб для А,
А', А", урівноважується простою купівлею частини І m з боку В, В', В''
і т. д., які перетворюють свій скарб на елементи додаткового продуктивного
капіталу.

Оскільки рівновага відновлюється через те, що покупець потім виступає
як продавець на таку саму суму вартости, і навпаки, остільки відбувається
зворотний приплив грошей до тієї сторони, яка авансувала їх
підчас купівлі, яка продала раніше, ніж знову купила. Але дійсна рівновага,
щодо самого товарового обміну, обміну різних частин річного продукту,
зумовлюється рівністю величини вартости обмінюваних один проти
одного товарів.

Але оскільки відбуваються просто однобічні перетворення, прості
купівлі, з одного боку, і прості продажі, з другого, — а ми бачили, що
нормальне перетворення річного продукту на капіталістичній основі зумовлює
такі однобічні метаморфози, — остільки рівновага буде лише при
тому припущенні, що сума вартости однобічних купівель і сума вартости
однобічних продажів покривають одна одну. Та обставина, що товарова
продукція є загальна форма капіталістичної продукції, включає вже й ту
ролю, що її відіграють у ній гроші не лише як засіб циркуляції, а й
як грошовий капітал, і утворює певні, властиві цьому способові продукції
умови нормального обміну, отже, нормального перебігу репродукції,
усе одно, чи в простому, чи в поширеному маштабі, — умови,
що перетворюються на так само численні умови ненормального перебігу
репродукції, на так само численні можливості криз, бо рівновага за стихійного
ладу (naturwüchsigen Gestaltung) цієї продукції — сама є випадок.

Так само ми бачили, що при обміні І v на відповідну суму вартости
II с, саме для II с, кінець-кінцем, відбувається заміщення товару II
рівною сумою вартости товару І, отже, що з боку збірного капіта-
