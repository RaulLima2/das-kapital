ків. Отже, тут справа така: певну суму авансується на певний рід основного капіталу, напр., на
десять років. Цю витрату роблять одним заходом. Але певну частину цього основного капіталу, що його
вартість увійшла в вартість продукту й разом з ним перетворилась на гроші, щороку заміщується in
natura, тимчасом як друга частина існує і далі в своїй первісній натуральній формі. Ось оця-о
одноразова витрата і лише
частинна репродукція в натуральній формі й відрізняє цей капітал як основний від поточного.

Інші елементи основного капіталу складаються з неоднорідних частин, що зношуються протягом
неоднакового часу, а тому й мусять вони поновлюватись неодноразово. Саме так справа стоїть з
машинами. Те, що ми щойно зазначили щодо різної життьової тривалости різних складових частин
основного капіталу, має тут силу й щодо життьової тривалости різних складових частин тієї самої
машини, що фігурує як елемент цього
основного капіталу.

Щодо поступінного поширення підприємства з перебігом частинного поновлення, то ми зауважуємо таке.
Хоч, як ми бачили, основний капітал in natura й далі діє в продукційному процесі, однак частина його
вартости, залежно від пересічного зношування, циркулює разом з продуктом, перетворюється на гроші й
становить елемент грошового резервного фонду на заміщення капіталу, коли надходить час для його
репродукції in natura. Ця частина вартости основного капіталу, перетворена таким
чином на гроші, може придатися на те, щоб поширити підприємство або вробити поліпшення в машинах, що
збільшать їхню діяльність. Таким чином відбувається через більші або менші переміжки репродукція і
саме — розглядаючи з суспільного погляду — репродукція в поширеному маштабі: екстенсивно — коли
поширюється поле продукції; інтенсивно, коли засоби продукції робляться ефективніші. Ця репродукція
в поширеному маштабі випливає не з акумуляції — перетворення додаткової вартости на капітал, — а із
зворотного перетворення вартости, яка, відгалузившись, відокремившись у грошовій формі від тіла
основного капіталу, перетворилась на новий, або додатковий, або ефективніший, основний капітал того
самого роду. Звичайно, залежить почасти від специфічного характеру даного підприємства, чи може воно
та оскільки так поступінно поширюватись; отже, цей характер також визначає, в яких розмірах треба
нагромаджувати резервний фонд, щоб його можна було таким чином знову вкласти в це підприємство, та в
які переміжки часу це можна зробити. З другого боку, щодо спроможности запроваджувати детальні
поліпшення в наявних машинах, то це залежить, звичайно, від характеру цих поліпшень і конструкції
самої машини. Але до якої значної міри, напр., у залізничних спорудах доводиться з самого початку
звертати увагу на цю обставину, це доводить Адамс: „Вся конструкція повинна будуватись на тому
самому принципі, що панує в вулику: на здібності необмежено поширюватись. Всі надто солідні й
особливо симетричні будови являють зло, коли їх доводиться розбирати в разі поширення“ (р. 123).
