табі, ми тим самим припускаємо, що зовнішня торговля лише заміщує
тубільні предмети предметами іншої споживної або натуральної форми, не
впливаючи при цьому на відношення вартости, а значить, і на ті відношення
вартости, що в них обмінюються одна на одну дві категорії:
засоби продукції та засоби споживання, і так само не впливаючи на
відношення між сталим капіталом, змінним капіталом та додатковою
вартістю, що на них можна розкласти вартість продукту кожної з цих
двох категорій. Отже, притягнення зовнішньої торговлі до аналізи щорічно
репродукованої вартости продукту може лише заплутати справу, не даючи
жодного нового моменту ні для проблеми, ні для її розв’язання. Отже,
тут треба цілком абстрагуватись від неї; тому золото треба вважати тут
за безпосередній елемент річної репродукції, а не за довожуваний з-зовні
в наслідок обміну товаровий елемент.

Продукція золота, як і взагалі продукція металів, належить до кляси І,
до категорії, яка охоплює продукцію засобів продукції. Припустімо, що
річна продукція золота = 30 (для зручности; а дійсно цифра ця дуже
висока порівняно з числами нашої схеми); хай ця вартість розпадається
на 20с + 5v + 5m; 20с треба обміняти на інші елементи Іс, і це ми розглянемо
потім; a 5v + 5m (І) треба обміняти на елементи ІІс, тобто на
засоби споживання.

Щодо 5v, то кожне підприємство, яке продукує золото, починає з
закупу робочої сили: не на золото, спродуковане в самому цьому підприємстві,
а на деяку масу грошей, наявних у країні. На ці 5v робітники
купують засоби споживання в II, а цей на ці гроші купує засоби продукції
в І. Коли II купує, скажімо, на 2v І золото як товаровий
матеріял і т. ін. (складову частину свого сталого капіталу), то до продуцента
золота І повертаються 2v в грошах, що вже раніше належали
циркуляції. Коли II не купує в І далі матеріялу, то І купує в II, подаючи
своє золото як гроші в циркуляцію, бо на золото можна купити всякий
товар. Ріжниця тільки в тому, що І виступає тут не як продавець, а
лише як покупець. Золотопромисловці І можуть завжди збути свій товар;
він завжди є в такій формі, що його можна безпосередньо обміняти.

Припустімо, що прядун заплатив своїм робітникам 5v, а вони дають
йому за це, — лишаючи осторонь додаткову вартість, — пряжу в продукті = 5;
робітники на 5 купують у IIс, останній купує на 5 грішми пряжу в І, і
таким чином 5v грішми повертаються назад до прядуна. Навпаки, в
щойно припущеному випадку І з (так ми позначатимемо продуцента
золота) авансує своїм робітникам 5v грішми, що вже раніш належали
циркуляції; робітники витрачають ці гроші на засоби існування; але з
5 тільки 2 повертаються від II до І з. Однак І з цілком так само, як і
прядун, може знову почати процес репродукції; бо його робітники дали
йому золотом 5, що з них він продав 2, а решту 3 має в формі золота, —
отже, йому доводиться тільки карбувати з них монету\footnote{
„Значну кількість золотих зливків (gold bullion) приставляють продуценти
золота безпосередньо до карбівниці в Сан-Франціско“. — Reports of Н. М. Secretaries
of Embassy and Legation. 1879. Part III, p. 337.
} або перетво-