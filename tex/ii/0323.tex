вартости І відбувається лише через продаж товарів І m, що в них
міститься ця додаткова вартість, і перебування додаткової вартости у
формі грошей триває кожного разу лише доти, доки гроші, вторговані
від продажу товару, не витратиться знову на засоби споживання.

Підрозділ І на додаткові гроші (500 ф. стерл.) купує в II засоби
споживання; ці гроші І витратив і одержав за них еквівалент в товарах II;
першого разу гроші припливають назад тому, що II купує в І товару
на 500 ф. стерл.; отже, вони повертаються як еквівалент товару, проданого
цим І, але цей товар нічого не коштує для І, отже, становить додаткову
вартість для І, і таким чином гроші, що їх він сам
подає в циркуляцію, перетворюють на гроші його
власну додаткову вартість; так само при своїй другій купівлі
(№ 6) І одержує еквівалент в товарах II. Коли припустити, що II не
купує в І засобів продукції (№ 7), то І справді заплатив би за 1000 ф.
стерл. засобів споживання — спожив би всю свою додаткову вартість як
дохід, — а саме 500 своїми товарами І (засобами продукції) і 500 грішми;
при цьому в нього не лишилось би на складах 500 ф. стерл. у товарах І
(в засобах продукції), і він витратив би грішми 500 ф. стерл.

Навпаки, II перетворив би лише три чверті свого сталого капіталу
з форми товарового капіталу знову на продуктивний капітал; а четверту
частину — на форму грошового капіталу (500 ф. стерл.), і в дійсності на
гроші, що лежать без діла, або на гроші, що припинили свою функцію
й вичікують. Коли б такий стан тривав довго, то II мусів би скоротити
на одну чверть маштаб репродукції. Але ті 500 в засобах продукції,
які лишаються на шиї в І, не є додаткова вартість, що існує в товаровій
формі; вони з’явились замість авансованих 500 ф. стерл. грішми, що І мав
поряд своїх 1000 ф. стерл. додаткової вартости в товаровій формі. Як
гроші вони перебувають у формі, що в ній їх завжди можна реалізувати;
як товар їх у даний момент не сила продати. Відси ясно, що проста
репродукція — а за неї кожен елемент продуктивного капіталу мусить бути
заміщений так в II, як і в І, — тут можлива й далі тільки тоді, коли 500 золотих
птахів повернуться до того під розділу І, що спочатку випустив їх.

Коли капіталіст (тут ми все ще маємо перед собою промислового капіталіста,
що є разом з тим представник усіх інших) витратить гроші на
засоби споживання, то ці гроші для нього остаточно зникли, вони пішли
шляхом усього живого. Коли вони знову повертаються до нього, то це
може постати лише в тому разі, якщо він з циркуляції виловить їх за допомогою
товарів, тобто за допомогою свого товарового капіталу. Так само
як вартість його цілого річного товарового продукту (а він для нього = товаровому
капіталові), так і вартість кожного елемента цього останнього, тобто
вартість кожного поодинокого товару, розпадається для нього на сталу капітальну
вартість, змінну капітальну вартість і додаткову вартість. Отже,
перетворення на гроші одиниці з товарів (що з них як з елементів складається
товаровий продукт) є разом з тим перетворення на гроші певної частини
додаткової вартости, яка міститься в цілому товаровому продукті. Отже,
для даного випадку цілком правильно, що капіталіст сам подав у цир-
