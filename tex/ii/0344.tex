Але ті 500 в грошах, що повернулися до капіталіста II, є разом з
тим відновлений потенціяльний змінний капітал у грошовій формі. Чому
це? Гроші, отже, і грошовий капітал є потенціяльний змінний капітал
лише тому й остільки, що й оскільки їх можна перетворити на робочу
силу. Поворот цих 500 ф. стерл грішми до капіталіста II супроводиться
поворотом робочої сили II на ринок. Поворот грошей і робочої сили на
протилежні полюси — а значить, і з’явлення знову цих 500 в грошовій
формі, не лише як грошей, а також і як змінного капіталу в грошовій
формі — зумовлено тією самою процедурою. Гроші = 500 повертаються до
капіталіста II тому, що він продав робітникові II засобів споживання на
суму 500, отже, тому, що робітник витратив свою заробітну плату й таким
чином дістав змогу утримувати себе й родину, а тим самим і свою
робочу силу. Щоб йому можна було й далі існувати, і далі виступати
покупцем товарів, він мусить знову продати свою робочу силу. Отже,
поворот до капіталіста II цих 500 грішми є разом з тим поворот, зглядно
збереження, робочої сили як товару, що його можна купити на ці 500
грішми, а тому це є поворот цих 500 грішми як потенціяльного змінного
капіталу.

Щодо категорії IIЬ, яка продукує речі розкошів, то з її v —
(II Ь) — справа така сама, як і з Iv. Гроші, що відновлюють капіталістам
IIЬ їхній змінний капітал в грошовій формі, припливають до них
обкружним шляхом, через руки капіталістів II а. А проте, є ріжниця в
тому, чи купують робітники засоби свого існування безпосередньо у тих
капіталістичних продуцентів, що їм вони продають свою робочу силу, чи
купують їх у другої категорії капіталістів, за посередництвом яких гроші
повертаються до перших лише обкружним шляхом. А що робітнича кляса
живе з дня на день, то вона купує, поки може купувати. Інша справа
з капіталістом, прим., при обміні 1000 II с на 1000 І v. Капіталіст живе
не з дня на день. Рушійний мотив для нього — якомога значніше збільшення
вартости його капіталу. Тому, коли постають якісь обставини, що,
зважаючи на них, капіталістові II здається вигідніше, замість відновити
безпосередньо свій сталий капітал, хоча б почасти затримати його в
грошовій формі на більш-менш довгий час, то зворотний приплив цих
1000 ІІс (в грошах) до І уповільнюється; уповільнюється, отже, і відновлення
1000 v в грошовій формі, і капіталіст І може провадити далі
роботу в попередньому маштабі лише тоді, коли в його розпорядженні
є запасні гроші, як і взагалі потрібен запасний капітал в грошовій формі
для того, щоб можна було безперервно провадити роботу незалежно від
швидшого або повільнішого зворотного припливу змінної капітальної вартости
в грошах.

Коли треба дослідити обмін різних елементів поточної річної репродукції,
то при цьому треба дослідити й результат минулої річної праці,
праці вже закінченого року. Продукційний процес, що його результат є
цей річний продукт, лежить позад нас, минув, злився з своїм продуктом;
отже, то більше це має силу для процесу циркуляції, що передує
процесові продукції або відбувається рівнобіжно з ним, — для перетво-
