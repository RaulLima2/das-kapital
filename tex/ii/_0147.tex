\parcont{}  %% абзац починається на попередній сторінці
\index{ii}{0147}  %% посилання на сторінку оригінального видання
вони не становлять жодного елементу продуктивного капіталу, хоч яке
буде їхнє остаточне призначення, тобто чи кінець-кінцем входять вони
відповідно до свого призначення (своєї споживної вартості!) в сферу особистого,
чи продуктивного споживання. В пункті 2 ці продукти є засоби
харчування, в пункті 4 — всі інші готові продукти, що, отже, знову таки
складаються з готових засобів праці або готових засобів споживання
(інші, ніж засоби харчування, зазначені в пункті 2).

Що А. Сміс при цьому каже і про торговця, це виявляє його плутанину.
Оскільки продуцент продав торговцеві свій продукт, то цей останній
вже взагалі не становить жодної форми його капіталу. З погляду
суспільства це, правда, все ще товаровий капітал, хоч він перебуває в
інших руках, ніж руки його продуцента; але саме тому, що це капітал
товаровий, він не може бути ні основним, ні обіговим капіталом.

В кожній продукції, що не має на меті безпосередньо задовольняти
власні потреби, продукт мусить циркулювати як товар, тобто його
треба продати не для того, щоб одержати таким чином зиск, а для того,
щоб взагалі продуцент міг існувати. За капіталістичної продукції до
цього долучається та обставина, що під час продажу товару реалізується
й додаткову вартість, що міститься в ньому. Продукт виходить з процесу
продукції як товар, а тому він не є ні основний, ні обіговий елемент
цього процесу.

А проте, А. Сміс тут сам себе збиває. Всі готові продукти, хоча
яка буде їхня речова форма або їхня споживна вартість, їхній корисний
ефект, є тут товаровий капітал, тобто капітал в формі, належній до процесу
циркуляції. Перебуваючи в цій формі, вони зовсім не становлять
складових частин продуктивного капіталу їхнього власника; це ні в якому
разі не заважає тому, що скоро тільки їх продано, вони стають в
руках покупця складовими частинами продуктивного капіталу, все одно —
обігового, чи основного. Тут виявляється, що ті самі речі, що деякий
час виступали на ринку як товаровий капітал протилежно до продуктивного,
пізніше, скоро тільки їх взято з ринку, можуть функціонувати або
не функціонувати, як поточна або основна складова частина продуктивного
капіталу.

Продукт бавовнопрядника — пряжа — є товарова форма його капіталу,
товаровий капітал для нього. Пряжа не може функціонувати знову, як
складова частина його продуктивного капіталу, ні як матеріял праці, ні
як засіб праці. Але в руках ткача, що її купив, вона входить в його
продуктивний капітал, як одна з поточних складових частин його. Але
для прядуна пряжа є носій вартости частини його капіталу — так основного,
як і поточного (додаткову вартість ми лишаємо осторонь). Так
машина, як продукт фабриканта машин, є товарова форма його капіталу,
товаровий капітал для нього, і доки вона зберігає цю форму, вона не є
ні поточний, ні основний капітал. Продана одному з фабрикантів, що
вживають її, вона стає основною складовою частиною продуктивного капіталу.
Навіть тоді, коли продукт має таку споживну форму, що почасти
може, як засіб продукції, ввійти знову в той самий процес, що з нього
\parbreak{}  %% абзац продовжується на наступній сторінці
