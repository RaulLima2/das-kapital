само метаморфоза індивідуального капіталу, його оборот, є ланка в кругобігу
суспільного капіталу.

Цей сукупний процес охоплює так само і продуктивне споживання
(безпосередній процес продукції) разом з перетвореннями форми (обмінами,
коли розглядати справу з речового боку), що упосереднюють його,
і особисте споживання з перетвореннями форми або обмінами, що упосереднюють
це споживання. Він охоплює, з одного боку, перетворення
змінного капіталу на робочу силу, а значить, і введення робочої сили в
капіталістичний процес продукції. Тут робітник виступає як продавець
свого товару, робочої сили, а капіталіст як покупець її. Але, з другого
боку, продаж товару включає й купівлю його робітничою клясою, отже, її
особисте споживання. Тут робітнича кляса виступає як покупець, а капіталісти
— як продавці товарів робітникам.

Циркуляція товарового капіталу включає й циркуляцію додаткової
вартости, а значить, і купівлі й продажі, що ними капіталісти упосереднюють
своє особисте споживання, споживання додаткової вартости.

Кругобіг індивідуальних капіталів, розглядуваних у їхньому з’єднанні
в суспільний капітал, отже, кругобіг, розглядуваний в його цілості, охоплює
не лише циркуляцію капіталу, а й загальну товарову циркуляцію.
Ця остання може первісно складатись лише з двох складових частин:
1) власне кругобігу капіталу і 2) кругобігу товарів, що входять в особисте
споживання, отже, товарів, що на них робітник витрачає свою заробітну
плату, а капіталіст — свою додаткову вартість (або частину своєї
додаткової вартости). В усякому разі кругобіг капіталу охоплює й циркуляцію
додаткової вартости, оскільки вона становить частину товарового
капіталу, а також охоплює і перетворення змінного капіталу на робочу
силу, виплату заробітної плати. Але витрачання цієї додаткової вартості
та заробітної плати на товари не становить жодної ланки циркуляції
капіталу, не зважаючи на те, що принаймні витрачання заробітної плати
зумовлює цю циркуляцію.

В І книзі проаналізовано капіталістичний процес продукції і як окремий
процес і як процес репродукції: продукцію додаткової вартости
і продукцію самого капіталу. Зміни форми та речовин, що їх проробляє
капітал у сфері циркуляції, ми припустили як передумову, що на
ній не зупинялись далі. Отже, ми припускали, що капіталіст, з одного
боку, продає продукт за його вартістю, а з другого, знаходить у сфері
циркуляції речові засоби продукції, потрібні для того, щоб відновити
процес або безупинно провадити його. Єдиним актом у сфері циркуляції,
що на ньому нам довелось там зупинитись, був акт купівлі та продажу
робочої сили як основної умови капіталістичної продукції.

В першому відділі цієї II книги ми розглядали різні форми, що їх
набирає капітал у своєму кругобігу, та різні форми самого цього кругобігу.
До робочого часу, розглянутого в І книзі, тепер долучається час
циркуляції.

В другому відділі ми розглядали кругобіг капіталу як періодичний
процес, тобто як оборот капіталу. Ми показали, з одного боку, як різні
