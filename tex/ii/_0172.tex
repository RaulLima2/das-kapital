\index{ii}{0172}  %% посилання на сторінку оригінального видання
В тих галузях продукції, де робочі періоди — хоч будуть вони безперервні,
хоч перервні — залежать від певних природних умов, не можна
скоротити їх вищезазначеними засобами. „Вислів: швидший оборот — не
можна прикласти до збору зерна, бо тут можливий лише один оборот
на рік. Щодо скотарства, то ми хотіли б довідатись, як можна прискорити
оборот дволітніх і трилітніх овець або чотирилітніх і п’ятилітніх
биків?“ (W. Walter Good: Political, Agricultural and Commercial Fallacies.
London, 1866, p. 325.).

Доконечність мати якнайшвидше вільні гроші (напр., для якихось сталих
виплат, як от податки, земельна рента тощо) розв’язує це питання
так, що, напр., худобу, на велику шкоду хліборобству, продається й
ріжеться раніш, ніж вона досягне економічного нормального віку; це
призводить, кінець-кінцем, до підвищення цін на м’ясо. Люди, що раніше
переважно відгодовували худобу, щоб використовувати улітку пасовиська
Midland countries\footnote*{
Графства середньої Англії. \emph{Ред.}
}, а взимку стійла східніх графств..... в наслідок коливань
і зниження цін на збіжжя дійшли того, що тепер радіють, коли
можуть мати користь з високих цін на масло та сир; вони виносять масло
щотижня на базар, щоб покрити поточні видатки: а під сир вони беруть
аванси від скупника, а останній збирає продукт, скоро його можна перевозити,
і, звичайно, назначає свою ціну. В наслідок цього й тому, що
в сільському господарстві панують закони політичної економії, телят, що
їх раніше з молочарських округ виганялось на південь відгодовувати,
ріжеться масами, часто на 8—9-й день віку, по різницях Бірмінгема, Менчестера,
Ліверпуля та інших сусідніх великих міст. Коли б, навпаки,
солод не обкладалось податком, то не лише фармери одержували б більше
зиску і таким чином могли б тримати в себе молоду худобу, поки вона
підросте й набуде ваги, але й люди, що не мають корів, могли б вживати
замість молока солод на годівлю телят, і теперішню жахну недостачу
молодої худоби здебільше усовувалось би. Однак тепер, коли цим
дрібним господарям радять відгодовувати телят, вони кажуть: Ми дуже
добре знаємо, що відгодування молоком оплатилось би, але, поперше,
нам довелось би для цього витратити гроші, а ми цього не можемо зробити,
і подруге, нам довелось би довго чекати, поки ми повернемо собі
наші гроші, тимчасом як у молочному господарстві ми їх повертаємо
одразу“ (ibid., стор. 11).

Коли подовшання обороту має такі наслідки навіть для дрібних англійських
фармерів, то легко уявити собі, до яких розладів мусить воно
спричинитися в дрібних селян континенту.

Відповідно до протягу робочого періоду, а, значить, і періоду часу,
потрібного на те, щоб виготувати до циркуляції товар, частина вартости,
що її основний капітал шарами віддає продуктові, нагромаджується, і
зворотний приплив цієї частини уповільнюється. Але це уповільнення не
призводить до нових витрат основного капіталу. Машина функціонує й
далі в продукційному процесі, незалежно від того, чи повільніше, чи
\parbreak{}  %% абзац продовжується на наступній сторінці
