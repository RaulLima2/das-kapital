Отже, ці гроші є грошова форма частини сталої капітальної взртости,
її основної частини. Отже, це утворення скарбу саме є елемент капіталістичного
процесу репродукції, є репродукція і нагромадження — в грошовій
формі — вартости основного капіталу або його поодиноких елементів,
до того моменту, коли основний капітал відживе свій вік і, значить, передасть
свою вартість спродукованим товарам, після чого його доводиться
замістити in natura. Але ці гроші, скоро їх знову перетворено на нові
елементи основного капіталу, щоб замістити елементи, які віджили свій
вік, втрачають лише свою форму скарбу й тому лише знову активно
входять у процес репродукції капіталу, упосереднюваний циркуляцією.

Як проста товарова циркуляція не тотожня з простим обміном продуктів,
так і перетворення річного товарового продукту не можна звести на
простий, безпосередній, взаємний обмін його різних складових частин.
Гроші відіграють у ньому специфічну ролю, яка виявляється і в способі
репродукції основної капітальної вартости. (Далі треба буде дослідіти,
який це мало б інший вигляд, коли припустити, що продукція колективна
й не має форми товарової продукції).

Тепер, повертаючись до основної схеми, ми маємо для кляси II:
2000 с + 500 v + 500 m. Всі засоби споживання, спродуковані протягом
року, дорівнюють тут вартості в 3000; і кожен з різних елементів товару,
що з них складається ця сума товару, розкладається за вартістю своєю
на 2/3 с + 1/6 v + 1/6 m, або у відсотках на 66 2/3 с + 16 2/3 v + 16 2/3 m.
Різні ґатунки товарів кляси II можуть мати в собі сталий капітал у
різних пропорціях; основна частина сталого капіталу в них так само
може бути різна; так само і протяг життя основних частин капіталу, а
значить, і річне зношування або та частина вартости, яку вони pro rata
переносять на товари, вироблювані за їх допомогою. Все це тут не має
значення. Щодо суспільного процесу репродукції, то вся справа лише в
обміні між клясами II і I. II і І протистоять тут один одному лише в
їхніх суспільних масових відношеннях; тому пропорційна величина частини
вартости с товарового продукту II (а тільки вона й має міродайне
значення для розглядуваного тепер питання) є пересічне відношення, коли
зробити загальний підсумок усіх галузей продукції, що входять у II.

Таким чином, кожен з товарових ґатунків (а це здебільша ті самі
ґатунки товарів), що їхню загальну вартість підсумовано в 2000 с + 500 v +
500 m, однаково дорівнює своєю вартістю 66 2/3% с + 16 2/3% v +
16 2/з% m. Це має силу для всяких 100 одиниць товарів, хоч фігурують
вони під с, хоч під v, хоч під m.

Товари, що в них втілено 2000 с, теж можна розкласти за їхньою
вартістю на:

1) 1333 1/3с + 333 1/3 v + 333 1/3m = 2000c; так само 500 v на:

2) 333 1/3 с + 83 1/3 v + 83 1/3 m = 500 v; нарешті, 500 m на:

3) 333 1/3с + 83 1/3 v + 8З 1/3 m = 500 m;

Тепер, коли ми складемо с, що є в 1, 2 і 3, то матимемо 1333 1/3 с
+ 333 1/Зс + 333 1/3 с = 2000. Так само 333 1/3v + 83 1/3v + 83 1/3v
