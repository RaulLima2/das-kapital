шового капіталу. Таким чином, реальні елементи продуктивного капіталу
збільшуються, не потребуючи додаткового грошового капіталу. А оскільки
його треба буде на додаткові допоміжні матеріяли, то грошовий капітал,
що в ньому авансується капітальну вартість, збільшується не пропорційно
до поширення діяльности продуктивного капіталу, отже, pro
tanto зовсім не збільшується.

Ті самі засоби праці, отже, той самий основний капітал, можна використати
ефективніше так збільшуючи протяг його щоденного вживання,
як і збільшуючи інтенсивнісгь його застосування, не витрачаючи
при цьому додаткових грошей на основний капітал. В такому разі відбувається
лише швидший оборот основного капіталу, але зате елементи
його репродукції постачатиметься швидше.

Лишаючи осторонь матеріяли природи, в процес продукції можуть
заводитись, як чинники більшої або меншої ефективности, сили
природи, що нічого не коштують. Ступінь їхньої ефективности
залежить від методів та поступу науки, що нічого не коштують капіталістові.
Це саме стосується до суспільного сполучення робочої сили в продукційному
процесі та до вмілости, надбаної поодинокими робітниками.
Кері на підставі цього вважає, що власник землі ніколи не одержує досить,
бо йому оплачується не ввесь той капітал, зглядно не всю ту
працю, що її з прадавніх часів вкладалось у землю, щоб надати їй теперішньої
родючости. (Звичайно, про ту родючість, що їй відбирається,
не згадується). Але в такому разі кожен поодинокий робітник мусив
би оплачуватись відповідно до тієї праці, яку витратив увесь рід людський,
щоб перетворити дикуна на сучасного механіка. Тут слід було б
міркувати саме навпаки: коли взяти на увагу всю вкладену в землю
неоплачену, але землевласниками й капіталістами перетворену на гроші
працю, то ввесь вкладений у землю капітал повернуто багато разів та
ще з лихварським процентом, отже суспільство давно вже й багато
разів викупило земельну власність.

Підвищення продуктивних сил праці, оскільки воно не має за передумову
додаткову витрату капітальних вартостей, підвищує, правда, насамперед
лише масу продукту, а не вартість його; останню воно підвищує
лише остільки, оскільки воно дає змогу тією самою працею репродукувати
більше сталого капіталу, отже, зберегти вартість його. Але разом з
тим підвищення продуктивних сил праці утворює новий матеріял для капіталу,
тобто базу для підвищеної акумуляції капіталу.

У першій книзі вже показано, що оскільки сама організація суспільної
праці, а тому й підвищення суспільної продуктивної сили праці потребує,
щоб продукцію провадилось у широкому маштабі, отже, щоб поодинокі
капіталісти авансували великі маси грошового капіталу, — це
стається почасти через централізацію капіталу в небагатьох руках, при
цьому немає потреби в тому, щоб розмір діющих капітальних вартостей,
а тому й розмір того грошового капіталу, що в ньому їх авансується,
абсолютно зростав. Величина поодиноких капіталів може зростати в на-
