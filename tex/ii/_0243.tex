\parcont{}  %% абзац починається на попередній сторінці
\index{ii}{0243}  %% посилання на сторінку оригінального видання
спрямувати на продуктивні сили та на їхній майбутній вільний розвиток,
а не на саме лише акумульоване багатство, що впадає на очі, як це
було до цього часу. Куди більша частина так званого акумульованого
багатства є лише номінальна й складається не з справжніх речей, кораблів,
будинків, бавовняних товарів, меліорацій, а з простих юридичних
титулів, з вимог на майбутні річні продуктивні сили суспільства, з юридичних
титулів, що утворились і увічнились в наслідок засобів або
інституцій незабезпечености\dots{} Вживання таких предметів (нагромаджених
фізичних речей, або справжнього багатства) як простого
засобу, для присвоювання їхніми власниками багатства, яке лише мають
утворити майбутні продуктивні сили суспільства, таке вживання їм поступінно
відібралось би природними законами розподілу, не вживаючи
сили; за допомогою кооперованої праці (Cooperative labour) його
відібралось би їм протягом небагатьох років“. (William Thompson, „An
Inguiry into the principles of the Distribution of Wealth. London 1850, p. 453.
Ця книга вийшла першим виданням 1824 року).

„Мало хто думає, а більшість навіть і гадки не має, яка надто
незначна й масою своєю і силою свого впливу дійсна акумуляція суспільства
порівняно з продуктивними силами людства і навіть порівняно
з звичайним споживанням одного покоління протягом небагатьох лише
років. Причина очевидна, але вплив дуже шкідливий. Багатство, споживане
щороку, зникає разом із споживанням його; воно лише одну мить
навіч перед нами і справляє вражіння лише, поки з нього користуються
або поки його споживають. Але тільки повільно споживана частина
багатства, меблі, машини, будівлі, стоять перед нашими очима з нашого
дитинства й до старости, як довговічні пам’ятники людської праці.
Маючи цю сталу, довговічну, лише повільно споживану частину суспільного
багатства — землю та сировинний матеріял, що до них прикладається
працю, знаряддя, що ними працюють, будівлі, що дають притулок підчас
праці, — маючи все це, власники цих речей в своїх інтересах захоплюють
річні продуктивні сили всіх дійсно продуктивних робітників суспільства,
хоч би які незначні були ці речі порівняно з постійно відновлюваними
продуктами цієї праці. Людність Брітанії та Ірляндії дорівнює 20 мільйонам;
пересічне споживання кожної людини, — чоловіка, жінки, дитини —
становить, мабуть, щось 20 ф. стерл., увесь щорічно споживаний продукт
праці становить багатство приблизно в 400 мільйонів ф. стерл. За оцінкою,
загальна сума акумульованого капіталу в цих країнах не перевищує
1200 мільйонів, або потроєного річного продукту праці; поділивши
нарівно, маємо 60 ф. стерл. на душу; тут для нас радше має вагу відношення,
ніж більш або менш точні абсолютні підсумки сум цієї оцінки.
Процентів з цілого цього капіталу було б досить для того, щоб утримувати
всю людність при її теперішньому рівні життя приблизно два
місяці на рік, а всього акумульованого капіталу (коли б знайшлися для
нього покупці) вистачило б на утримання цієї людности протягом цілих
трьох років без якоїбудь роботи! Але потім, опинившись без будівель,
одягу й харчу, люди мусіли б загинути з голоду, або зробитись рабами
\parbreak{}  %% абзац продовжується на наступній сторінці
