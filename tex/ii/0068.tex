бігу усовуються; що більші ці перешкоди, то більший грошовий капітал
мусить мати промисловий капіталіст, щоб чекати, поки їх усунеться; а що в
розвитку капіталістичної продукції поширюються розміри кожного індивідуального
продукційного процесу, а разом з тим і мінімальна величина
авансовуваного капіталу, то ця обставина прилучається до ряду
інших, які дедалі більше перетворюють функцію промислового капіталіста
на монополію великих грошових капіталістів, поодиноких або
асоційованих.

До речі треба тут позначити, що коли постає зміна в вартості елементів
продукції, то виявляється ріжниця між формою Г... Г', з одного
боку, і формою П.. . П і Т'.. . Т', з другого боку.

В Г... Г', як формулі нововкладуваного капіталу, який спочатку
виступає як грошовий капітал, в разі що знизиться вартість засобів
продукції, напр., сировинного матеріялу, допоміжних матеріялів і т. ін., для
того, щоб одкрити підприємство певних розмірів, треба меншої витрати
грошового капіталу, ніж до цього зниження, бо розмір продукційного
процесу (за незмінного рівня розвитку продуктивної сили) залежить від
маси й розміру засобів продукції, що їх може опанувати дана кількість
робочої сили; але він не залежить ні від вартости цих засобів продукції,
ні від вартости робочої сили (остання впливає лише на величину
зростання вартости). Навпаки. Коли вартість тих елементів продукції товарів,
що є елементи продуктивного капіталу, підвищується, то треба
більше грошового капіталу, щоб закласти підприємство даних розмірів.
В обох випадках зазнає впливу лише величина нововкладуваного грошового
капіталу; в першому випадку постає надмір грошового капіталу, в
другому — зв’язується грошовий капітал, якщо в даній галузі продукції
звичайним порядком відбувається приріст нових індивідуальних промислових
капіталів.

Кругобіги П... П і Т'... Т' виявляються сами як Г... Г' лише
остільки, оскільки рух П і Т' є разом з тим акумуляція, отже, лише
остільки, оскільки додаткове г, гроші, перетворюється на грошовий капітал.
Залишаючи це осторонь, зміна вартости елементів продуктивного капіталу
відбивається на них інакше, ніж на Г... Г'; ми тут знову залишаємо
осторонь зворотний вплив, що його справляє така зміна вартости на складові
частини капіталу, які перебувають у процесі продукції. Безпосередньо
зазнає впливу тут не первісна витрата, а промисловий капітал, що перебуває
у процесі своєї репродукції, а не в своєму першому кругобігу;
отже, впливу зазнає Т'... Т – Р Зп, зворотне перетворення товарового
капіталу на елементи його продукції, оскільки ці останні складаються з товарів.
Коли падає вартість (зглядно ціна), то можливі три випадки: процес
репродукції триває далі в тих самих розмірах; тоді звільняється частина
грошового капіталу, що був досі, і нагромаджується грошовий капітал,
хоч не відбувається ані справжньої акумуляції (продукції в поширених
розмірах), ані підготовчого до неї й рівнобіжного з нею перетворення г
