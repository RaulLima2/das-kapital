дукції, потрібен, навпаки, вивіз товару II (засобів споживання). Отже, в
обох випадках потрібна зовнішня торговля.

Даймо навіть, що при вивчанні репродукції в незмінному маштабі
треба припустити, що продуктивність усіх галузей продукції, а значить,
і пропорційні відношення вартостей товарових продуктів цих галузей,
лишаються незмінні, — все ж обидва останні випадки, де ІІс (1) більше
або менше, ніж IIс (2), являли б інтерес при вивчанні продукції в поширеному
маштабі, де, безперечно, можуть настати ці випадки.

3) Результати

Щодо заміщення основного капіталу, то взагалі треба зазначити ось що.

Коли — припускаючи, що всі інші умови, а значить, не лише маштаб
продукції, а зокрема й продуктивність праці лишаються незмінні, —
поточного року відмирає більша частина основного елемента ІІс, ніж у
попередньому році, а тому й більшу частину треба відновлювати in
natura, то та частина основного капіталу, яка є лише на шляху до своєї
смерти й яку до моменту її смерти треба покищо заміщувати в грошах,
теж мусить зменшитись у такій самій пропорції, бо згідно з припущенням
сума (також і сума вартости) основної частини капіталу, діющої в II
лишається та сама. Але це тягне за собою такі обставини. Поперше. Коли
більша частина товарового капіталу І складається з елементів основного
капіталу II с, то відповідно менша частина складається з обігових складових
частин ІІс, бо вся продукція І для ІІс лишається незмінна. Коли одна
частина збільшується, то друга зменшується й навпаки. Але, з другого
боку, величина всієї продукції кляси II такожа лишається незмінна. Як
же можливо це, коли меншає в неї сировинних матеріялів, напівфабрикатів,
допоміжних матеріялів (тобто обігових елементів сталого капіталу II)?
Подруге. Більша частина основного капіталу II с, знову відновленого в
грошовій формі, припливає до І, щоб знову перетворитись з грошової
форми на натуральну форму. Отже, до І, крім грошей, що циркулюють
між І і II для простого товарного обміну, припливає більше грошей;
більше таких грошей, які правлять не за посередника у взаємному товаровому
обміні, а однобічно виступають лише в функції купівельного
засобу. Але разом з тим пропорційно зменшилась би товарова маса
II с, що є носій вартости на заміщення зношування, тобто та товарова
маса II, яка мусить бути перетворена не на товари І, а лише на гроші І.
Від II до І приплило б більше грошей як просто купівельних засобів і
було б менше товарів у II, що супроти них І мав би функціонувати як
простий покупець. Більшу частину Іm, — бо Іv уже перетворено на товари
II, — не сила було б перетворити на товари II, її довелось би затримати в
грошовій формі.

Зворотний випадок, — коли протягом року репродукція відмерлого
основного капіталу II менша і навпаки, частина на заміщення зношування
більша, — після попереднього не потребує дальшого розгляду.

І, таким чином, настала б криза — криза продукції — не зважаючи на
репродукцію в незмінному маштабі.
