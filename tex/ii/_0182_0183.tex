\parcont{}  %% абзац починається на попередній сторінці
\index{ii}{0182}  %% посилання на сторінку оригінального видання
й ріжниці в часі, що на нього треба авансувати капітал, постають в самому процесі
продукції, як ріжниця між основним і поточним капіталом, ріжниця
робочих періодів тощо. Однак час обороту капіталу дорівнює сумі часу
його продукції і часу його обігу або циркуляції. Відси зрозуміло само
собою, що різний протяг часу обігу робить різним час обороту, а значить,
і протяг періоду обороту. Найнаочніше це буде або тоді, коли
порівняти два різні капіталовкладення, при чому різні тільки часи обігу,
а всі інші обставини, що модифікують оборот, однакові, або коли взяти
певний капітал певного складу щодо основного й поточного капіталу при
певному робочому періоді і т. ін., і гіпотетично зміняти тільки час
його обігу.

Один відділ часу обігу — і порівняно найважливіший — складається з
часу продажу, з того періоду, коли капітал перебуває в стані товарового
капіталу. Відповідно до відносної величини цього періоду подовжується
або скорочується час обігу, а тому й період обороту взагалі. В наслідок
витрат на зберігання тощо може бути потрібна й додаткова витрата капіталу.
Само собою зрозуміло, що час, потрібний для продажу готових
товарів, може бути дуже різний у різних капіталістів у тій самій галузі
підприємств; отже, цей час може бути різний не лише для мас капіталів,
вкладених у різні галузі продукції, а й для різних самостійних капіталів,
що в дійсності є лише усамостійнені частини сукупного капіталу, вкладеного
в ту саму продукційну сферу. За інших незмінних обставин період
продажу для того самого індивідуального капіталу буде змінюватись
разом із загальними коливаннями ринкових відносин, або разом із
коливаннями цих відносин в поодинокій галузі продукції. На цьому ми
не будемо тут більше зупинятись. Ми лише констатуємо простий факт:
всі обставини, що взагалі зумовлюють ріжницю в періодах обороту капіталів,
вкладених у різні галузі підприємств, мають наслідком, якщо ці
обставини впливають індивідуально (коли, напр., один капіталіст має змогу
продавати швидше, ніж його конкурент, коли один більш, ніж інший,
вживає методів, що скорочують робочі періоди тощо), так само ріжницю
в обороті різних індивідуальних капіталів, що перебувають в тій самій
галузі підприємств.

Одна з причин, що завжди зумовлюють ріжницю в часі продажу, а
тому і в часі обороту взагалі, є віддаленість ринку, де продається товар,
від місця, де його виготовлюється.\footnote*{
В нім. тексті тут стоїть: „von ihrem Verkaufsplatz“, тобто: „від місця його продажу“.
Очевидна помилка. \emph{Ред.}
} Протягом цілого часу своєї подорожі
до ринку, капітал лишається зв’язаний в стані товарового капіталу;
коли товар продукують на замовлення, то — до часу здачі; коли не на замовлення,
то до часу подорожі його на ринок долучається ще той час,
що протягом його товар перебуває на ринку, чекаючи на продаж. Поліпшення
засобів зв’язку й транспорту скорочує мандрування товарів абсолютно,
але не знищує зумовлюваної цим мандруванням відносної ріжниці
в часі різних товарових капіталів або й різних частин того самого товарового
\index{ii}{0183}  %% посилання на сторінку оригінального видання
капіталу, що мандрують до різних ринків. Поліпшені вітрильні
судна та пароплави, напр., що скорочують шлях, однаково скорочують
його так до близьких, як і до далеких портів. Відносна ріжниця лишається,
хоч часто зменшена. Але відносні ріжниці можуть у наслідок розвитку
засобів транспорту й зв’язку змінюватись таким способом, який не відповідає
природним віддаленням. Напр., залізниця, що веде від місця продукції до
головного внутрішнього залюдненого центру, може зробити ближчий
унутрішній пункт, що до нього немає залізниці, абсолютно або відносно
більш віддаленим порівняно з пунктом, куди віддаленішим географічно; так
само, в наслідок цієї самої обставини, може змінюватись навіть відносна
віддаленість місць продукції від більших ринків збуту, і цим пояснюється
занепад старих і постання нових центрів продукції рівнобіжно з зміною
засобів транспорту й зв’язку. (До цього ще долучається відносно більша
дешевина транспорту на великі дистанції порівняно з невеликими). Разом
з розвитком засобів транспорту не тільки збільшується швидкість
переміщення, і в наслідок цього просторова віддаль зменшується в часі.
Розвивається не лише маса засобів комунікації, так що, напр., одночасно
багато суден виходять до того самого порту, кілька поїздів одночасно
йдуть різними залізницями між тими самими двома пунктами, але, напр.,
у різні послідовні дні тижня товарові судна виходять з Ліверпулу на
Нью-Йорк, або товарові поїзди в різні години доби йдуть з Менчестера
до Лондону. Правда, абсолютна швидкість — отже, і відповідна частина
часу обігу — в наслідок цієї останньої обставини, за даної провізної спроможности
засобів транспорту, не змінюється. Але все ж послідовні партії товарів
можна відправляти через коротші переміжки часу, що йдуть один по
одному, і таким чином вони можуть послідовно надходити на ринок, не
нагромаджуючись великими масами як потенціяльний товаровий капітал,
поки їх дійсно відправиться. Тому й зворотний приплив розподіляється
на коротші послідовні періоди часу, так що одна частина постійно перетворюється
на грошовий капітал, тимчасом як друга частина циркулює
як товаровий капітал. В наслідок такого розподілу зворотного
припливу на кілька послідовних періодів скорочується весь час обігу, а
тому скорочується й оборот. Насамперед розвивається більша чи менша
частість функціонування засобів транспорту, — напр., численність поїздів
на залізниці розвивається, з одного боку, разом із тим, як осередок
продукції продукує дедалі більше, стає більшим центром продукції, і
розвивається вона в напрямку до вже наявних ринків збуту, отже, в напрямку
до великих центрів продукції та залюднення, до вивізних портів
тощо. Але, з другого боку, ця особлива легкість сполучень і зумовлений
нею прискорений оборот капіталу (оскільки його зумовлює час обігу)
призводить, навпаки, до прискореної концентрації, з одного боку, центру
продукції, а з другого — його ринку збуту. Разом із прискореною
таким чином концентрацією в певних пунктах маси людей та капіталів,
розвивається концентрація цих мас капіталів у небагатьох руках. Разом
з тим знову пересуваються й переміщуються осередки продукції та ринки
в наслідок їх зміненого відносного положення, зумовленого зміною
\parbreak{}  %% абзац продовжується на наступній сторінці
