ліста II тут продаж власного товару доповнюється купівлею товару І на
таку саму суму вартости. Таке заміщення відбувається; але не відбувається
обміну між самими капіталістами І і II при цьому перетворенні
їхніх товарів. II с продає свої товари робітничій клясі І; ця остання
протистоїть йому однобічно як покупець товарів, а II с протистоїть робітничій
клясі І однобічно як продавець товарів; з грішми, вторгованими
таким чином, II с протистоїть збірному капіталістові І однобічно як покупець
товарів, а збірний капіталіст І однобічно протистоїть йому як продавець
товарів на суму І v. Тільки через цей продаж товарів підрозділ І, кінецькінцем,
репродукує свій змінний капітал знову в формі грошового капіталу.
Коли капітал І протистоїть капіталові II однобічно як продавець
товару на суму І v, то своїй робітничій клясі він протистоїть як покупець
товарів, що купує її робочу силу; і коли робітнича кляса І протистоїть
капіталістові II однобічно як покупець товару (а саме як покупець
засобів існування), то капіталістові І вона протистоїть однобічно як продавець
товару, а саме як продавець своєї робочої сили.

Постійне подання робочої сили з боку робітничої кляси в І, зворотне
перетворення частини товарового капіталу І на грошову форму змінного
капіталу, заміщення частини товарового капіталу II натуральними елементами
сталого капіталу II с — всі ці доконечні передумови навзаєм зумовлюють
одна одну, але їх упосереднює дуже складний процес, який
має в собі три процеси циркуляції, що перебігають незалежно один від
одного, але в той самий час переплітаються один з одним. Складність самого
цього процесу дає так само численні нагоди до ненормального перебігу.

2) Додатковий сталий капітал

Додатковий продукт, носій додаткової вартости, нічого не коштує
капіталістам І, його привлащувачам. Їм не доводиться в жодній формі
авансувати гроші або товари, щоб його одержати. Аванс (avance)
уже у фізіократів є загальна форма вартости, реалізованої в елементах
продуктивного капіталу. Вони, отже, нічого не авансують, крім свого сталого
й змінного капіталу. Своєю працею робітник не лише зберігає їм
їхній сталий капітал; він не тільки заміщує їм змінну капітальну вартість,
утворюючи відповідну нову частину вартости в формі товару; своєю додатковою
працею він, крім того, дає їм додаткову вартість, що існує
в формі додаткового продукту. Послідовно продаючи цей додатковий
продукт, воии утворюють скарб, додатковий потенціальний грошовий капітал.
В розглядуваному тут випадку цей додатковий продукт складається
з самого початку із засобів продукції засобів продукції. Тільки в руках
В, В', В" і т. д. (І) цей додатковий продукт функціонує як додатковий
сталий капітал; але віртуально він є ним раніше, ніж його продасться,
уже в руках утворювачів скарбу А, А', А" (І). Коли ми розглядаємо
тільки розмір вартости репродукції на боці І, ми перебуваємо
ще в межах простої репродукції, бо жодного додаткового капіталу
не пущено в рух, щоб утворити цей віртуальний додатковий сталий
