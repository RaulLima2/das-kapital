витрачати на медичну допомогу, ніж людині в розквіті сил. Отже, незважаючи на випадковий характер
ремонтних робіт, вони розподіляються нерівномірно на різні життьові періоди основного капіталу.

З цього, а також взагалі випадкового характеру ремонтних робіт, що їх потребує машина, випливає
таке:

З одного боку, справжня витрата на робочу силу й засоби праці для ремонтних робіт є випадкові, як
випадкові й самі обставини, що роблять потрібними ці ремонтні роботи; число потрібних полагоджень
розподіляється нерівномірно на різні життьові періоди основного капіталу. З другого боку, коли
обчислюють пересічний життьовий період основного капіталу, то припускається, що його постійно
підтримується в діяльному
стані, — почасти чищенням (сюди належить і тримання в чистоті приміщень), почасти ремонтом, що його
робиться в разі потреби. Перенесення вартости в наслідок зношування основного капіталу розраховано
на його пересічний життьовий період, але й самий цей пересічний період життя розрахований на те, що
весь час авансуватиметься додатковий капітал, потрібний на його підтримання в доброму стані.

З другого боку, так само зрозуміло, що вартість, долучувана в наслідок цієї додаткової витрати
капіталу й праці, не може входити в ціну товарів одночасно з цими витратами. Коли, напр., у
прядільника на цьому тижні поламалось колесо або розірвався пас, то він не може цього тижня
продавати свою пряжу дорожче, ніж продавав минулого. Загальні витрати прядіння ніяк не змінились в
наслідок такого нещасного випадку на одній фабриці. Тут, як і взагалі при визначенні вартости,
вирішувальне значення має пересічна величина. Досвід виявляє середнє число таких нещасних випадків і
пересічний розмір робіт на підтримання і ремонт, потрібних протягом пересічного життьового періоду
основного капіталу, вкладеного в певну галузь підприємства. Ці пересічні витрати розподіляються на
пересічний життьовий періоді відповідними аліквотними частинами їх долучається до ціни продукту, а
тому й покривається через його продаж.

Додатковий капітал, таким чином заміщуваний, належить до поточного капіталу, хоч спосіб витрат
нерегулярний. А що дуже важливо виправляти кожне ушкодження машини негайно, то при кожній великій
фабриці є, крім власне фабричних робітників, відповідний персонал інженерів, теслярів, механіків,
слюсарів і т. ін. їхня заробітна плата становить частину змінного капіталу, і вартість їхньої праці
розподіляється на
продукт. З другого боку, потрібні видатки на засоби продукції визначаються за пересічним розрахунком
і відповідно до нього ввесь час входять у продукт, як частина його вартости, хоч фактично їх
авансується нереґулярно, а, значить, нереґулярно входять вони в продукт, зглядно в основний капітал.
Цей капітал, витрачуваний власне на ремонт, з певного погляду є капітал особливого роду, що його не
можна залічити ні до поточного, ні до основного капіталу, але більше до першого, бо він належить до
категорії поточних витрат.

Система бухгальтерії, звичайно, нічого не змінює в дійсному зв’язку речей, що про них ведеться ці
книги. Але важно відзначити, що в ба-
