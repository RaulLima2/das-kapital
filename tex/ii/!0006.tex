Берліні 1859 р. під тією самою назвою. В ньому викладено на сторінках
1—220 (зшитки I—V), а потім знову на сторінках від 1159 до 1472
(зшитки XIX—XXIII) теми, досліджені в І книзі „Капіталу“, починаючи з
перетворення грошей на капітал і до кінця; це є перша наявна редакція
цих тем. На сторінках від 973 до 1158 (зшитки XVI—XVIII) мовиться
про капітал і зиск, про норму зиску, купецький капітал і грошовий капітал,
— отже, про теми, пізніше розвинуті в рукопису до III книги. Навпаки,
теми, викладені в II книзі, а також дуже багато тем, розглянутих
пізніше в III книзі, тут окремо ще не розроблені. Їх трактується мимохідь,
саме в відділі, що становить головну частину рукопису, сторінки
220—972 (зшитки VI—XV): Теорії додаткової вартости. В цьому відділі
подано докладну критичну історію центрального пункту політичної
економії, а саме теорії додаткової вартости, і разом з тим розвинуто в
формі полеміки з попередниками більшість пунктів, досліджених пізніше
окремо та в логічному зв’язку в рукопису, що стосується до II та III
книг. Я маю на думці опублікувати критичну частину цього рукопису,
викинувши з нього багато місць, докладно розглянутих у книгах II і
III, — як IV книгу „Капіталу“ *. Хоч який цінний цей рукопис, однак, у
ньому мало з чого можна було скористатися для цього видання II книги.

Дальший датою рукопис є рукопис III книги. Його написано, принаймні
більшу частину, в 1854 і 1865 році. Лише після того, як він
був готовий в основному, Маркс почав обробляти першу книгу надрукованого
1867 року першого тому. Цей рукопис III книги я обробляю
тепер до друку.

З найближчого періоду — по виданні книги І — маємо для II книги зібрання
чотирьох рукописів in folio, що їх перенумерував сам Маркс з
І до IV. З них рукопис І (150 сторінок), написаний, мабуть, 1865 або
1867 р., є перше самостійне, але більш-менш уривчасте оброблення
книги II в її теперішній побудові. І в цьому рукопису також не можна
було нічого використати. Рукопис III складається почасти з зібрання
цитат і посилань на Марксові зшитки з виписами — все це стосується переважно
до першого відділу II книги, — а почасти він є оброблення поодиноких
пунктів, а саме критики Смісових засад про основний та обіговий
капітал та про джерело зиску; далі висвітлено відношення норми
додаткової вартости до норми зиску, що стосується до III книги. Посилання
дали мало нового, бо в наслідок пізніших редакцій з них годі
було користатись для II і III книг; отже, їх теж здебільша довелось відкласти.
— Рукопис IV є готове до друку оброблення першого відділу та
першого розділу другого відділу книги II, і його тут у відповідних місцях
використано. Хоч виявилось, що цей рукопис написано раніше, ніж
рукопис II, однак, з нього можна було добре скористатись для відповідної
частини книги, бо він більш закінчений формою; досить було
зробити деякі додатки з рукопису II. — Цей останній рукопис є одним-

* Цей рукопис після смерти Енгельса виготовив до друку й видав під назвою
„Теорії додаткової вартости“ К. Каутський. Ред.
