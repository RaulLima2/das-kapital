\parcont{}  %% абзац починається на попередній сторінці
\index{ii}{0231}  %% посилання на сторінку оригінального видання
капітал в 500 ф. стерл. авансується десять разів у послідовні періоди часу.
Тому річну норму додаткової вартости обчислюється не на капітал в 500 ф.
стерл., що його авансується десять разів, або на 5000 ф. стерл., а на капітал
в 500 ф. стерл., що його авансовано один раз; цілком так само,
як один таляр, що обертається десять разів, завжди репрезентує лише
одним один таляр, що перебуває в циркуляції, хоч він виконує функцію
10 талярів. Але в тих руках, де він є при кожній оборудці, він завжди
лишається тією самою вартістю в 1 таляр.

Так само капітал А при кожному своєму повороті, а також при своєму
повороті наприкінці року показує, що його власник завжди орудує
лише тією самою капітальною вартістю в 500 ф. стерл. Тому до рук його
кожного разу повертається лише 500 ф. стерл. Тому авансований ним капітал
ніколи не перевищує 500 ф. стерл. Авансований капітал в 500 ф. стерл.
становить тому знаменика того дробу, що виражає річну норму додаткової
вартости. Формула для річної норми додаткової вартости в нас
вище була така:

М' = m'vn: $v = m$'n.

А що справжня норма додаткової вартости m' = m: v дорівнює масі додаткової
вартости, поділеній на змінний капітал, що продукує її, то в
m'n ми можемо підставити значення m', тобто m: v і тоді матимемо другу
формулу М' = mn: v.

Але в наслідок десятиразового обороту, а тому в наслідок десятиразового
поновлювання його авансування, капітал в 500 ф. стерл. виконує
функцію вдесятеро більшого капіталу, капіталу в 5000 ф. стерл., цілком так
само, як 500 монет, кожна в 1 таляр, обертаючись десять разів на рік,
виконують ту саму функцію, що й 5000 таких самих монет, які обертаються
лише один раз.

II. Кругобіг поодинокого змінного капіталу

„Хоч яка буде суспільна форма процесу продукції, процес мусить
бути безупинний або мусить періодично знову й знову перебігати ті самі
стадії... Тому всякий суспільний процес продукції, розглядуваний в
його постійному зв’язку та постійній течії його відновлення, є разом з
тим процес репродукції... Як періодичний приріст капітальної вартости
або періодичний плід капіталу, додаткова вартість набирає форми доходу,
що походить із капіталу“. (Кн. І, початок розд. XXI).

Ми маємо 10 п’ятитижневих періодів обороту капіталу А; в перший
період обороту авансується 500 ф стерл. змінного капіталу; тобто щотижня
обмінюється на робочу силу 100 ф. стерл., так що наприкінці першого
\parbreak{}  %% абзац продовжується на наступній сторінці
