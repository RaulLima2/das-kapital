кається, що капіталісти — хоч поодинокі капіталісти II а і II b, хоч відповідні
категорії капіталістів в їхній сукупності — в однаковому відношенні
розподіляють свою додаткову вартість між доконечними предметами споживання
й засобами розкошів. Один може більше витрачати на одні
предмети споживання, другий — на другі. Лишаючись на ґрунті простої
репродукції, ми припускаємо тільки, що суму вартости, рівну всій додатковій
вартості, реалізується в фонді споживання. Отже, межі тут
дано. В межах кожного підрозділу один може більше витрачати на а,
другий на b; тут можлива взаємна компенсація, так що кляси капіталістів
а і b, взяті кожна як ціле, будуть в однаковій мірі брати участь
в а і b. Але відношення вартостей — пропорційна участь в цілій вартості
продукту II обох категорій продуцентів а і b — а значить і певне кількісне
відношення між галузями продукції, що дають ці продукти — ці
відношення неодмінно є дані для кожного конкретного випадку: гіпотетичне
є лише відношення, що фігурує в прикладі; коли припустити
інше відношення, то від цього ніщо не зміниться в якісних моментах;
змінилися б лише кількісні визначення. Але коли б в наслідок тих або
інших обставин постала справжня зміна у відносних величинах а й b, то
відповідно змінились би й умови простої репродукції.

З тієї обставини, що (II b) v реалізується в еквівалентній частині
(ІІ а) m, випливає, що тією самою мірою, як більшає частина річного
продукту, яка припадає на речі розкошів, отже, тією самою мірою, як
більшає маса робочої сили, що її поглинає продукція засобів розкошів,
такою самою мірою зворотне перетворення авансованого на (II b) v
змінного капіталу в грошовий капітал, що знову функціонує як грошова
форма змінного капіталу, а в наслідок цього й існування і репродукція
частини робітничої кляси, занятої в II b — одержання цією частиною робітничої
кляси доконечних засобів споживання — зумовлюється марнотратством
кляси капіталістів, перетворенням значної частини їхньої додаткової
вартости на речі розкошів.

Кожна криза моментально зменшує споживання речей розкошів; вона
уповільнює, затримує зворотне перетворення (II b) v на грошовий капітал,
лише почасти допускає це перетворення й тим самим викидає частину робітників,
які виробляють речі розкошів, на брук, а з другого боку, саме через
це вона призводить до застою й скорочення продажу доконечних засобів
споживання. Ми залишаємо цілком осторонь звільнених разом з цим
непродуктивних робітників, які за свої послуги одержують від капіталістів
частину їхніх витрат на розкоші (самі ці робітники pro tanto є
предмети розкошів) і беруть також дуже велику участь у споживанні
доконечних засобів існування тощо. Протилежне маємо в періоди процвітання
і особливо підчас спекулятивного процвітання, коли відносна,
виражена в товарах вартість грошей, падає вже з інших причин (при
йому не відбувається дійсного перевороту в вартості), а тому ціна товарів,
незалежно від їхньої власної вартости, підвищується. При цьому
