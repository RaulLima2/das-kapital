\parcont{}  %% абзац починається на попередній сторінці
\index{ii}{0200}  %% посилання на сторінку оригінального видання
разу він потребує авансування в 600 ф. стерл. (капітал І). Період циркуляції
3 тижні; отже, період обороту, як і раніш, 9 тижнів. Капітал II
в 300 ф. стерл. ввіходить у роботу протягом тритижневого періоду циркуляції
капіталу І. Коли розглядати їх обидва, як капітали, незалежні
один від одного, то схема річного обороту матиме такий вигляд:

\begin{table}[h]
  \noindent\begin{tabularx}{\textwidth}{r c c c c c c c c}

    %\caption*{Капітал І. 600 ф. стерл.}

%    \multicolumn{3}{c}{Періоди обороту}    \multicolumn{3}{c}{Робочі періоди}    \multicolumn{3}{c}{Авансовано} & \multicolumn{2}{c}{Періоди циркуляції} \\
    І.  & Тижні         & 1 — 9     & Тижні          & 1 — 6  &  600  & ф. стерл. & Тижні & 7 — 9 \\
    II. & \ditto{Тижні} & 10 — 18   & \ditto{Тижні} & 10 — 15 &  600 & \ditto{ф.} \ditto{стерл.} & \ditto{Тижні} & 16 — 18 \\
    III.& \ditto{Тижні} & 19 — 27   & \ditto{Тижні} & 19—24 & 600   & \ditto{ф.} \ditto{стерл.} & \ditto{Тижні} & 25 — 27 \\
    IV. & \ditto{Тижні} & 28--36    & \ditto{Тижні} & 28--33 & 600 & \ditto{ф.} \ditto{стерл.}  & \ditto{Тижні} &34--36 \\
    V.  & \ditto{Тижні} & 37--45    & \ditto{Тижні} & 37--42 & 600 & \ditto{ф.} \ditto{стерл.} & \ditto{Тижні} & 43--45 \\
    VI. & \ditto{Тижні} & 45 — [54] & \ditto{Тижні} & 46--51 & 600 & \ditto{ф.} \ditto{стерл.} & \ditto{Тижні} &[52--54] \\

%    \caption*{Додатковий капітал II. 300 ф. стерл.}
%
%    Періоди обороту    Робочі періоди    Авансовано    Періоди циркуляції
%    І. Тижні 7--15    Тижні 7--9    300 ф. стерл. Тижні 10--15
%    II. „16--24 „16--18    300 „„ „19--24
%    III. „25--33 „25--27    300 „„ „28--33
%    IV. „34--42 „34--36    300 „„ „37--42
%    V. „43--51 „43--45    300 „„ „45--51
\end{tabularx}
\end{table}
Процес продукції відбувається цілий рік безперервно в однакових
розмірах. Обидва капітали І і II лишаються цілком відокремлені. Але
для того, щоб подати їх так відокремленими, нам довелось роз’єднати
їхні справжні схрещування й переплітання, а через це змінити й число
оборотів. А саме, згідно з вище наведеною таблицею, обертається:

Капітал І 600 × 5\sfrac{2}{3} = 3400 ф. стерл.

„II 300 × 5 = 1500 ф. стерл.

отже, ввесь капітал    900 × 5\sfrac{4}{9} = 4900 ф. стерл.

Але це неправильно, бо, як ми побачимо, справжні періоди продукції
та циркуляції не абсолютно збігаються з цими періодами вище наведеної
схеми, де головне було в тому, щоб подати обидва капітали, І і II, незалежними
один від одного.

В дійсності саме капітал II не має ані особливого робочого періоду, ані особливого
періоду циркуляції, відокремлених від цих періодів капіталу І. Робочий
період триває 6 тижнів, період циркуляції 3 тижні. Що капітал II дорівнює
тільки 300 ф. стерл., то він може виповнити лише частину робочого
\parbreak{}  %% абзац продовжується на наступній сторінці
