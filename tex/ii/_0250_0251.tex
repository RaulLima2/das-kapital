\parcont{}  %% абзац починається на попередній сторінці
\index{ii}{0250}  %% посилання на сторінку оригінального видання
відповідно авансованих ними продуктивних капіталів; і так само постійно
доводиться їм розподіляти між собою ту суму вартости, що її вони
з усіх боків подають у циркуляцію в товаровій формі, як відповідний
надлишок товарової вартости проти вартости її елементів продукції.

Але товаровий капітал, перш ніж він перетвориться знову на продуктивний
капітал, і перш ніж витратиться вміщену в ньому додаткову
вартість, треба перетворити на гроші. Відки беруться гроші для цього?
На перший погляд питання це видається складним, і ні Тук, ні хто інший
до цього часу не дали на нього відповіді.

Припустімо, що обіговий капітал в 500 ф. стерл. авансований у формі
грошового капіталу, — хоч який буде період його обороту, — є сукупний
обіговий капітал суспільства, тобто кляси капіталістів. Додаткова вартість
хай буде 100 ф. стерл. Яким же чином ціла кляса капіталістів може постійно
вилучати з циркуляції 600 ф. стерл., постійно подаючи в неї
лише 500 ф. стерл.?

Після того, як грошовий капітал в 500 ф. стерл. перетворився на
продуктивний капітал, цей останній у процесі продукції перетворюється
на товарову вартість в 600 ф. стерл. і таким чином в циркуляції перебуває
не лише товарова вартість в 500 ф. стерл., рівна первісно авансованому
грошовому капіталові, а й новоспродукована додаткова
вартість в 100 ф. стерл.

Цю новододану додаткову вартість в 100 ф. стерл. подано в циркуляцію
в товаровій формі. В цьому немає жодного сумніву. Але в наслідок
цієї операції не здобувається додаткових грошей для циркуляції
цієї новододаної товарової вартости.

Не слід намагатися обминати цієї трудности за допомогою зовнішньопристойних
викрутів.

Наприклад: щодо сталого обігового капіталу, то очевидно, що його
не всі витрачають одночасно. У той час, коли капіталіст А продає
свій товар, отже, коли авансований ним капітал набирає для нього грошової
форми, для покупця В його капітал, що перебуває в грошовій
формі, набирає, навпаки, форми його засобів продукції, саме тих, що їх
продукує А. Тим самим актом, що ним А знову надає грошової форми
своєму спродукованому товаровому капіталові, В знову надає продуктивної
форми своєму капіталові, перетворює його з грошової форми на засоби
продукції та робочу силу; та сама сума грошей функціонує в двобічному
процесі, як при всякій простій купівлі $Т — Г$. З другого боку, коли
А знову перетворює гроші на засоби продукції, він купує їх в С, а
цей платить тими самими грішми В і т. ін. Тоді справу з’ясувалось би. Але:

Всі закони, викладені нами (кн. І, розд. Ill) щодо кількости грошей,
які циркулюють при товаровій циркуляції, зовсім не змінюються в наслідок
капіталістичного характеру продукційного процесу.

Отже, коли кажуть, що обіговий капітал суспільства, який треба
авансувати в грошовій формі, становить 500 ф. стерл., то при цьому вже
взято на увагу, що це, з одного боку, є така сума, яку авансовано одночасно,
але що, з другого боку, сума ця пускає в рух більше продуктивного
\index{ii}{0251}  %% посилання на сторінку оригінального видання
капіталу, ніж 500 ф. стерл., бо вона по черзі править за грошовий
фонд різних продуктивних капіталів. Отже, цей спосіб пояснення
припускає, що вже є наявні ті гроші, що їх наявність він має з’ясувати.

Далі можна було б сказати так: капіталіст А продукує такі речі, що
їх капіталіст В споживає особисто, непродуктивно. Отже, гроші В перетворюють
на гроші товаровий капітал А, і таким чином та сама грошова
сума служить для перетворення на гроші додаткової вартости В і
обігового сталого капіталу А. Але тут ще пряміше припускається розв’язаним
те саме питання, що на нього треба дати відповідь. А саме, відки
В бере гроші на покриття свого доходу? Яким чином він сам перетворив
на гроші цю частину додаткової вартости свого продукту?

Потім можна було б сказати, що частина обігового змінного капіталу,
яку А постійно авансує своїм робітникам, постійно повертається до нього
з циркуляції; і тільки деяка змінна частина її завжди лишається закріплена
в нього самого для видачі заробітної плати. Однак між моментом
витрачання й моментом зворотного припливу минає деякий час, що
протягом нього гроші, витрачені на заробітну плату, можуть, між іншим,
служити і для перетворення на гроші додаткової вартости. — Але, поперше,
ми знаємо, що як довший цей час, то й більша мусить бути маса грошового
запасу, що її капіталіст А постійно мусить зберігати in petto\footnote*{
In petto — дослівно: в серці своєму, тут у значенні: з собою, при собі. \emph{Ред.}
}.
Подруге, робітник витрачає гроші, купує на них товари й тим самим
pro tanto перетворює на гроші додаткову вартість, що міститься в цих
товарах. Отже, ті самі гроші, що їх авансується в формі змінного капіталу,
pro tanto служать і для перетворення на гроші додаткової вартости.
Не вглиблюючись у це питання ще далі, тут досить лише зауважити, що
споживання цілої кляси капіталістів і залежних від неї непродуктивних
осіб відбувається рівнобіжно й одночасно з споживанням робітничої кляси;
отже, одночасно з тим, як робітники подають у циркуляцію гроші,
мусять пускати гроші в циркуляцію і капіталісти, щоб витрачати свою
додаткову вартість як дохід; отже, для цього треба вилучати гроші з
циркуляції. Таким чином, щойно наведене пояснення лише зменшувало б
кількість потрібних грошей, але не усунуло б потреби в них. —

Нарешті, можна було б сказати: однак в циркуляцію при першому
вкладенні основного капіталу постійно подається велику кількість грошей,
що їх той, хто пустив їх в циркуляцію, знову вилучає з неї лише поступінно,
частинами, протягом багатьох років. Хіба цієї суми не досить,
щоб перетворити на гроші додаткову вартість? — На це треба відповісти,
що сума в 500 ф. стерл. (яка включає й скарботворення для потрібного
резервного фонду) можливо вже включає й застосовування цієї суми як
основного капіталу, якщо не тим, хто пустив її в циркуляцію, то кимось
іншим. Крім того вже припускається, що в сумі, витрачуваній на
придбання продуктів, що служать як основний капітал, оплачено й додаткову
вартість, що міститься в цих товарах, і питання саме в тому, відки
беруться ці гроші.
