пункту в наслідок ряду переміщень. (Кн. І, розд. III, 2, б). Однак прискорений
оборот ео ipso\footnote*{
Тим самим. Ред.
} включає й прискорений обіг.

Насамперед щодо змінного капіталу: коли, напр., грошовий капітал
в 500 ф. стерл. обертається в формі змінного капіталу десять разів на
рік, то очевидно, що ця аліквотна частина грошової маси, яка циркулює,
пускає в циркуляцію вдесятеро більшу суму вартости = 5000 ф. стерл.
Вона обігає між капіталістом і робітником десять разів протягом року.
Протягом року робітника десять разів оплачується, й сам робітник платить
тією самою аліквотною частиною грошової маси циркуляції. Коли
б при однакових розмірах продукції цей змінний капітал обертався
лише один раз протягом року, то тоді відбувся б лише один обіг
в 5000 ф. стерл.

Далі, хай стала частина обігового капіталу дорівнює 1000 ф. стерл.
Коли капітал обертається десять разів, то капіталіст продає свій товар, а
значить, і сталу обігову частину його вартости десять разів на рік. Та
сама аліквотна частина грошової маси, що циркулює (1000 ф. стерл.),
десять разів на рік переходить з рук власників цієї частини до рук капіталіста.
Десять разів переміщуються ці гроші з рук у руки. Подруге,
капіталіст десять разів на рік купує засоби продукції, це знову є десять
обігів грошей з рук до рук. За допомогою грошей на суму 1000 ф. стерл.
промисловий капіталіст продає товару на 10000 ф. стерл. і знову купує
товару на 10000 ф. стерл. В наслідок двадцятиразового обігу 1000 ф. стерл.
циркулює запас товару в 20000 ф. стерл.

Нарешті, при прискореному обороті швидше циркулює й та частина
грошей, що реалізує додаткову вартість.

Навпаки, швидший обіг грошей і не включає неодмінно швидшого обороту
капіталу, а тому й швидшого обороту грошей, тобто не включає
неодмінно скорочення та швидкого поновлення процесу репродукції.

гроші, повертаючись наступного року, щоб знову зробити в неї такі ж закупи…
Отже, ви не бачите тут іншого кругобігу, крім того, де по витраті
постає репродукція, а по репродукції витрата, — кругобігу, що його перебігає
циркуляція грошей, які є міра витрати й репродукції. („Jetez les yeux sur le
Tableau Economique, vous verrez, que la classe productive d nne l’argent,
avec lequel les autres classes viennent lui acheter des productions, et qu’elles lui
rendent cet argent en revenant l’année suivante faire chez elle les mêmes achats…
Vous ne v yez donc ici d’autre cercle que celui-ci de la dépense suivie de la reproduction,
et de la réproduktion suivie de la dépense; cercle qui est parcouru par la
circulation de l’argent qui mesure la dépense et la reproduction“ — Quesnay. „Problèmes
économiques, in Daire, Physiocrates, I“, p. 208, 209).

„Саме це постійне авансування й постійний поворот капіталів треба назвати
циркуляцією грошей, тією корисною й плодотворчою циркуляцією, яка оживляє
всю працю суспільства, підтримує рух і життя в політичному організмі і яку
цілком слушно можна порівняти з кровобігом у тваринному організмі“. (C’est
cette avance et cette rentrée continuelle des capitaux qui constituent ce qu’on doit
appeller la circulation de l’argent, cette circulation utile et féconde, qui anime tous
les travaux de la société, qui entretient le mouvement et la vie dans le corps politique,
et qu’on a grande raison de comparer à la circulation du sang dans le corps animal“.
— Turgot, „Reflexions“ etc, Oeuvres, éd. Daire, I, p. 45).