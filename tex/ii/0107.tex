наприклад, коли час обороту о становить три місяці, то n = 12 / 3 = 4; капітал
робить чотири обороти на рік, або обертається чотири рази.
Коли о = 18 місяцям, то n = 12 / 18 = 2 / 3 або капітал протягом року проходить
лише 2 / 3 часу свого обороту. Коли час обороту його дорівнює кільком
рокам, то він, отже, обчислюється одним роком, повтореним кілька
разів.

Для капіталіста час обороту його капіталу є час, що протягом його він
мусить авансувати свій капітал для того, щоб він збільшився вартістю
й повернувся в своїй первісній формі.

Перш ніж перейти до ближчого розгляду того впливу, що його
оборот справляє на процес продукції та процес зростання вартости, треба
розглянути дві нові форми, що підступають до капіталу із процесу циркуляції
та впливають на форму його обороту.

Розділ восьмий

Основний капітал і обіговий капітал

І. Відмінності форми

В книзі І, розділ VI, ми бачили, що частина сталого капіталу зберігає
ту певну споживну форму, що в ній вона увіходить у процес
продукції, проти тих продуктів, що їх утворенню вона сприяє. Отже,
протягом більш або менш довгого періоду, в постійно повторюваних
процесах праці, вона завжди виконує ті самі функції. Такі, напр., майстерні,
машини і т. ін., коротко кажучи — все те, що ми об’єднуємо
під назвою засоби праці. Ця частина сталого капіталу віддає свою
вартість продуктові, в міру того, як вона разом з своєю споживною
вартістю втрачає свою мінову вартість. Цю передачу вартости або перехід
вартости таких засобів продукції на продукт, що в утворенні його
вони беруть участь, визначається за пересічним обчисленням; її вимірюється
пересічним протягом функціонування засобів продукції, від
того моменту, коли вони ввіходять в процес продукції, і до моменту
коли вони цілком зносяться, знищаться, коли їх треба буде замінити на
нові екземпляри такого ж роду, або репродукувати.

Отже, своєрідність цієї частини сталого капіталу — власне засобів
праці — ось у чому:

Частину капіталу авансується в формі сталого капіталу, тобто в формі
засобів продукції, що функціонують як чинники процесу праці, поки
зберігають ту самостійну споживну форму, що в ній вони ввіходять
у процес праці. Готовий продукт, а значить і продуктотворчі елементи,
оскільки їх перетворено на продукт, виштовхується з продукційного
процесу, щоб перейшли вони як товар з сфери продукції до сфери цир-
