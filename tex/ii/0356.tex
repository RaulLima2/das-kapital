таких капіталістів підрозділу II, які мусять не тільки замістити за допомогою
своїх товарів свої засоби продукції, що належать до обігового
капіталу, а й поновити за допомогою своїх грошей свій основний
капітал in natura, тимчасом як друга половина капіталістів II своїми
грішми заміщують  in natura тільки обігову частину свого сталого капіталу,
але не відновлюють свій основний капітал  in natura, то при такому
припущенні немає жодної суперечности в тому, що 400 ф. стерл., які зворотно
припливають (вони припливають, скоро І купує на них засоби
споживання), різно розподіляються між цими двома підрозділами капіталістів
II. Вони припливають назад до кляси II, але не повертаються в ті
самі руки, а різно розподіляються всередині цієї кляси, переходячи від
однієї частини її до іншої.

Одна частина капіталістів II, крім частини засобів продукції, заміщуваної,
кінець-кінцем, її товарами, перетворила 200 ф. стерл. грішми на
нові елементи основного капіталу  in natura. Їхні гроші, таким чином
витрачені, — як і при відкритті підприємства — повертаються до них з циркуляції
лише протягом послідовного ряду років, як заміщення зношуваної
складової частини вартости основного капіталу, що перенесена на товари,
продуковані за допомогою цього основного капіталу.

Навпаки, друга частина капіталістів II на 200 ф. стерл. не одержала
жодних товарів від І, але І платить їм тими грішми, що на них перша
частина капіталістів II купила елементи основного капіталу. Одна частина
капіталістів II знову має свою основну капітальну вартість у відновленій
натуральній формі, друга ще дбає про те, щоб нагромадити цю вартість
у грошовій формі для наступного заміщування свого основного капіталу
in natura.

Стан, що з нього нам треба виходити після попередніх, обмінів, це —
решта товарів, що їх треба обміняти з обох боків: 400 m у І підрозділі, 400 с
у II 52). Ми припускаємо, що II авансує 400 грішми для обміну цих товарів
на суму в 800. Половину цих 400 (= 200) в усякому разі мусить подати
та частина II с, що нагромадила 200 грішми як вартість зношування
і що тепер повинна знову перетворити їх на натуральну форму свого
основного капіталу.

Цілком так само, як стала капітальна вартість, змінна капітальна вартість
і додаткова вартість — що на них можна розкласти вартість товарових
капіталів так II, як і І — можуть бути виражені в окремих пропорційних
частинах самих товарів II, зглядно товарів І, — цілком так само
може бути виражена й та частина вартости самої сталої капітальної вартости,
яку ще не доводиться перетворювати на натуральну форму основного
капіталу, але треба покищо поступінно нагромаджувати в грошовій
формі як скарб. Певна кількість товарів II (отже, в нашому прикладі —
половина остачі = 200) є тут лише носій цієї вартости зношування, що
має в наслідок обміну осісти в грошовій формі. (Перша частина капі-

52) Цифри знову не відповідають попередньому припущенню. Але це тут не має
значення, бо тут мають силу тільки відношення. Ф. Е.
