нати в цей час, прим., вивіз добрива, дрів, продуктів на ринок, будівельних
матеріялів тощо“ (F. Kirchhof, Handbuch der landwirtschaftlichen
Betriebslehre. Dresden, 1852, S. 160).

Отже, що несприятливіше підсоння, то коротший робочий період у
сільському господарстві, а тому коротший і той час, коли вкладається
капітал і працю. Візьмімо, напр., Росію. Там у деяких північних місцевостях
польові роботи можливі тільки протягом 130—150 днів на рік.
Легко зрозуміти, якою втратою було б для Росії, коли б 50 мільйонів
із 65 мільйонів її європейської людности лишалось без роботи протягом
шости або восьми зимових місяців, коли мусять припинитись всі польові
роботи. Крім 200.000 селян, що роблять на 10.500 фабриках Росії, по
селах там скрізь розвинулась своя хатня промисловість. Там є села, де
всі селяни з роду в рід ткачі, чинбарі, шевці, слюсарі, ножівники і т.
ін.; особливо це стосується до губерень Московської, Владимирської, Калузької,
Костромської та Петербурзької. До речі буде зауважити, що цю
хатню промисловість дедалі більше примушується служити капіталістичній
продукції, напр., ткачам основу та піткання постачають торговці або
безпосередньо або через факторів. (Скорочено за Reports by Н. М. Secretaries
of Embassy and Legation, on the Manufactures, Commerce etc.,
№ 8, p. 86, 87). Відси видно, як розходження періоду продукції і робочого
періоду — а останній становить лише частину періоду продукції, —
утворює природну основу для об’єднання хліборобства з сільськими підсобними
промислами і як, з другого боку, ці останні дають точку опори
капіталістам, які спочатку протискуються сюди, як торговці. Рівнобіжно
з тим, як пізніше капіталістична продукція відокремлює мануфактуру від
хліборобства, сільський робітник підпадає під дедалі більшу залежність
від суто-випадкового, підсобного промисла, і його стан через це погіршується.
Для капіталу, як ми побачимо далі, всі ріжниці в обороті вирівнюються.
Для робітника вони не вирівнюються.

Тимчасом як у більшості галузей власне промисловости, як от у гірництві,
транспорті і т. ін., процес продукції відбувається рівномірно, рік-у-рік
рівномірно витрачається робочий час, і, лишаючи осторонь, як ненормальні
перерви, коливання цін, розлади в перебігу справ і т. ін., рівномірно
розподіляються витрати на капітал, що входять у повсякденний процес
циркуляції, тимчасом як за інших незмінних ринкових відносин
зворотний приплив обігового капіталу або відновлення його так само
розподіляється на рівномірні переміжки часу — в тих галузях приміщення
капіталу, де робочий час становить лише частину часу продукції, обіговий
капітал витрачається в різні періоди року дуже нерівномірно, а назад
він припливає разом, в момент, визначуваний природними умовами. Отже,
тут, при однаковому маштабі підприємства, тобто при однаковій величині
авансованого обігового капіталу, цей останній треба авансувати одним
заходом більшими масами й на довший час, ніж у підприємствах з безперервними
робочими періодами. Життьова тривалість основного капіталу
тут також більше відрізняється від того часу, що протягом його він
справді функціонує продуктивно. Звичайно, коли є ріжниця між робочим
