\parcont{}  %% абзац починається на попередній сторінці
\index{i}{0649}  %% посилання на сторінку оригінального видання
провінція Яви, що 1750~\abbr{р.} налічувала понад \num{80.000} мешканців,
1811~\abbr{р.} мала їх лише \num{8.000}. Оце вам doux commerce!\footnote*{
— лагідна торговля. \emph{Ред.}
}

Англійська східньоіндійська компанія, як відомо, крім політичної
влади у Східній Індії, здобула собі виключну монополію
на торговлю чаєм, як і взагалі на торговлю з Китаєм і на транспорт
товарів з Европи та до Европи. Але мореплавство по узбережжі
Індії й між островами, як і торговля всередині Індії,
зробилися монополією вищих службовців цієї компанії. Монополії
на сіль, опій, бетель\footnote*{
Бетель — рослина, що належить до перцевих; впливає збудно на
нервову систему. В тропічній Азії дуже поширена звичка жувати цю рослину.
\emph{Ред.}
} та інші товари були невичерпними
джерелами багатства. Службовці компанії сами визначали ціни
на товари й обдирали нещасних індусів, як сами хотіли. Генерал-губернатор
брав участь у цій приватній торговлі. Його фаворити
діставали контракти на умовах, що дозволяли їм краще за альхеміків
виробляти золото з нічого. Великі багатства виростали
як гриби після дощу, первісна акумуляція відбувалась повним
ходом без авансування жодного шилінґа. Судовий процес Воррен
Гастінґза аж кишить такими прикладами. Подаємо один із
них. Один контракт на постачання опію передано якомусь Сюлейвенові
в момент його від’їзду — з офіціяльного доручення —
до частини Індії, дуже віддаленої від районів продукції опію.
Сюлейвен продає свій контракт за \num{40.000}\pound{ фунтів стерлінґів} якомусь
Біннові; того самого дня Бінн перепродує його за \num{60.000}\pound{ фунтів
стерлінґів}, а останній покупець і виконавець контракту заявляє,
що й він після всього цього здобув величезний бариш.
За одним документом, внесеним до парляменту, компанія та її
службовці примусили індусів за час від 1757~\abbr{р.} до 1766~\abbr{р.} подарувати
їм 6 мільйонів фунтів стерлінґів! В 1769--1770~\abbr{рр.}
англійці створили голод, закупивши ввесь риж і відмовившись
перепродувати його інакше, як за казкові ціни\footnote{
1866~\abbr{р.} в самій тільки провінції Оріса померло з голоду понад
мільйон індусів. Проте намагалися збагатити індійську державну касу
за допомогою цін, за якими голодним продавали засоби існування.
}.

Поводження з тубільцями було, звичайно, найжорстокіше
на плянтаціях, призначених виключно для експортної торговлі,
як от у Західній Індії, а також у багатих і густо залюднених
країнах, що стали жертвою грабіжництва й убивства, як от Мехіко
та Східня Індія. Однак і в колоніях у власному значенні
слова виявився християнський характер первісної акумуляції
капіталу. Пуритани Нової Англії, ці тверезі віртуози протестантизму,
ухвалили в 1703~\abbr{р.} на своїй Assembly\footnote*{
— законодавчі збори. \emph{Ред.}
} видавати премію
в 40\pound{ фунтів стерлінґів} за кожний скальп індійця і за кожного
спійманого червоношкурого, в 1720~\abbr{р.} — премію в 100\pound{ фунтів
стерлінґів} за кожний скальп, а в 1744~\abbr{р.}, після того, як Massachusetts-Bay
проголосив одно плем’я бунтарським, ухвалено платити
\index{i}{0650}  %% посилання на сторінку оригінального видання
такі ціни: за чоловічий скальп 12 років і старше — 100\pound{ фунтів стерлінґів} у
новій валюті, за спійманого чоловіка — 105\pound{ фунтів стерлінґів}, за спійману
жінку або дитину — 55\pound{ фунтів стерлінґів}, за
жіночий або дитячий скальп — 50\pound{ фунтів стерлінґів}!
Кілька десятиліть пізніш колоніяльна система помстилась на нащадках
цих побожних отців-піліґримів,
що й собі стали бунтарями. З намови й за гроші англійців їх усіх
tomahawked\footnote*{
— повбивано томагавками. \emph{Ред.}
}. Брітанський парлямент
проголосив кровожерство і скальпування
за «засоби, дані йому богом і природою».

Колоніяльна система надзвичайно прискорила розвиток торговлі й мореплавства. «Монопольні товариства»
(Лютер) були могутніми підоймами концентрації капіталу. Колонії забезпечували ринок збуту для
мануфактур, що народжувались, а монополія на ринку забезпечувала їм збільшену акумуляцію капіталу.
Скарби, здобуті поза Европою безпосереднім плюндруванням, поневолюванням, грабіжництвом і
вбивствами, припливали в метрополію і тут перетворювались на капітал. Голляндія, яка перша цілком
розвинула колоніяльну систему, вже 1648~\abbr{р.} дійшла вершини своєї торговельної могутности. В її «майже
виключному посіданні була східньоіндійська торговля й засоби
комунікації поміж европейським південним заходом і північним сходом. Її рибальство, мореплавство й
мануфактури були розвиненіші, ніж у всіх інших країнах. Капітали цієї республіки були, мабуть,
значніші, ніж капітали всіх інших країн Европи разом» (Gülich: «Geschichtliche Darstellung des
Handels etc.» Iena 1830, В.~I, p. 371). Ґюліх забуває додати, що народні маси Голляндії вже в 1648~\abbr{р.} більше терпіли від надмірної праці, були більш збіднілі й пригнічені, ніж народні маси всіх інших
країн Европи разом.

За наших часів промислова перевага веде за собою торговельну перевагу. Навпаки, за власне
мануфактурного періоду торговельна перевага забезпечує перевагу промисловости. Відси та вирішальна
роля, яку в ті часи відігравала колоніяльна система. Це був той «чужий бог», що засів на вівтарі
поруч із старими божками Европи й одного чудового дня одним ударом усіх їх поскидав. Колоніяльна
система оголосила здобування зиску за останню й єдину мету людства.

Система публічного кредиту, тобто державних боргів, що її початки ми знаходимо в Ґенуї й Венеції вже
за середньовіччя, захопила цілу Европу підчас мануфактурного періоду. Колоніяльна система з її
морською торговлею і торговельними війнами була за теплицю, що прискорювала її розвиток. Так вона
вкоренилася насамперед у Голляндії. Державний борг, тобто відчуження держави — однаково, чи
деспотичної, чи конституційної, чи республіканської — накладає свою печать на капіталістичну еру.
Однісінька частина так званого національного багатства, що дійсно входить у спільне володіння
сучасних народів,
\index{i}{0651}  %% посилання на сторінку оригінального видання
— це їхні державні борги\footnoteA{
Вільям Коббет зауважує, що в Англії всі громадські установи
називаються «королівськими», але борг зате там є «національний» (national debt).
}. Тому цілком послідовна є та сучасна доктрина, що народ стає тим
багатший, чим більше він заборговується. Державний кредит стає символом віри капіталу. І з
виникненням державної заборгованости місце гріха проти
святого духа, що за нього немає прощення, заступає зламання довіри до державного боргу.

Державний борг стає за одну з найсильніших підойм первісної акумуляції. Немов доторкаючись чарівною
паличкою, він наділяє непродуктивні гроші продуктивною силою й перетворює їх таким чином на капітал,
не потребуючи при тому виставляти
їх на небезпеку та самому зазнавати турбот, нерозривно зв’язаних з вкладанням грошей у промислові
підприємства й навіть у лихварські операції. Державні кредитори в дійсності не дають нічого, бо
позичені суми перетворюються на легко переказувані
боргові посвідки, які функціонують у їхніх руках цілком так само, як коли б це була така сама сума
готівки. Але державний борг не тільки створив таким чином клясу нероб-рантьє і імпровізоване
багатство тих фінансистів, що відіграють ролю посередників
поміж урядом і нацією — а також багатство відкупників податків, купців і приватних фабрикантів, що
їм значна частина кожної державної позики робить послугу як капітал, наче з неба спалий. Державний
борг, крім того, викликав акційні товариства,
торговлю всякими цінними паперами, ажіотаж, одно слово, біржову гру й сучасну банкократію.

З самого зародження свого великі банки, прикрашені національними титулами, були лише товариствами
приватних спекулянтів, що ставали на бік урядів і, завдяки одержаним привілеям, були спроможні
позичати їм гроші. Тому для акумуляції
державного боргу немає вірнішого мірила, ніж послідовне підвищення акцій цих банків, повний розквіт
яких починається з моменту заснування Англійського банку (1694~\abbr{р.}). Англійський банк почав з того,
що позичав урядові свої гроші з 8\%; одночасно
парлямент уповноважив його карбувати гроші з того самого капіталу, ще раз позичаючи його публіці у
формі банкнот. Цими банкнотами він міг дисконтувати векселі, давати позики під товари й закуповувати
благородні металі. Минуло небагато
часу, і ці кредитові гроші, зфабриковані самим банком, стали готівкою, що нею Англійський банк
видавав позики державі і сплачував коштом держави проценти від державних позик. Мало того, що банк
давав однією рукою, щоб одержати більше другою; навіть і тоді, коли він одержував, він лишався
вічним кредитором нації на всю віддану суму до останнього шага. Помалу він став також і доконечним
сховищем металевих скарбів країни й центром тяжіння для всього торговельного кредиту.
В той самий час, коли в Англії перестали палити відьом, там почали вішати фалшівників банкнот. Яке
вражіння справила на
\parbreak{}  %% абзац продовжується на наступній сторінці
