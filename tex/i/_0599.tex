\parcont{}  %% абзац починається на попередній сторінці
\index{i}{0599}  %% посилання на сторінку оригінального видання
в околиці. Легку працю, яка потребує багато робочих рук, виконують
артілі».\footnote{
«Children's Employment Commission. Sixth Report». Evidence,
p. 37, n. 173.
} Земля потребує багато легкої польової роботи:
виполювати бур’ян, обкопувати, робити деякі операції з
гноєм, вибирати каміння й т. ін. Все це роблять артілі або організовані
ватаги, що живуть у відкритих селах.

Артіль складається з 10--40 або 50 осіб, а саме, з жінок,
підлітків обох статей (з 13 до 18 років), хоч хлопці, дійшовши
13 років, здебільша покидають ватаги, і, нарешті, з дітей обох
статей (з 6 до 13 років). На чолі стоїть Gangmaster (староста
артілі); це завжди звичайний сільський робітник, здебільша
нікчемна людина, розпусник, волоцюга і п’яниця, але з певним
духом підприємливости і savoir faire.\footnote*{
— спритности. \emph{Ред.}
} Він навербовує артіль,
яка працює під його проводом, а не під проводом фармера. З останнім
він договорюється здебільша відштучно, і його дохід, що
пересічно не дуже перевищує заробіток звичайного сільського
робітника,\footnote{
Однак деякі старости артілі доробляються до того, що стають
фармерами, які мають 500 акрів землі, або власниками цілого ряду будинків.
} майже цілком залежить від умілости, з якою він
за найкоротший час зможе добути від своєї ватаги якнайбільше
праці. Фармери відкрили, що жінки працюють як слід лише під
диктатурою чоловіків, але що, з другого боку, жінки й діти,
скоро вони вже почали працювати, витрачають свої життєві
сили з справжньою загарливістю, — це знав уже Фур’є, — тимчасом
як дорослий робітник-чоловік хитрує, щоб якомога заощадити
свої сили. Староста ватаги переходить від одного маєтку
до іншого, і таким чином його ватага працює 6--8 місяців на
рік. Тим то він для робітничої родини дохідніший і певніший
наймач, аніж окремий фармер, що тільки принагідно вживає
дітей до праці. Ця обставина так зміцнює його вплив по відкритих
селах, що здебільша дітей можна найняти лише за його посередництвом.
Наймання фармерам дітей поодинці, поза артіллю, —
це його побічне заняття.

«Темний бік» цієї системи — це надмірна праця дітей і підлітків,
величезні переходи, які їм щодня доводиться робити туди
й назад до маєтків, віддалених на 5, 6 а іноді й 7 миль, і, нарешті,
деморалізація «артілі». Хоч староста артілі, що його в деяких
місцевостях звуть «the driver» (підганяч), і озброєний довгою
палицею, проте він дуже рідко вживає її, і скарги на брутальне
поводження є виняток. Він — демократичний імператор або щось
наче щуролов із Гамельну.\footnote*{
Німецька легенда про щуролова з Гамельну (старовинне місто
над Везером у провінції (Ганновер), що своєю грою на сопілці заманив
усіх пацюків із міста, а потім і дітей, яких він завів у підземелля. \emph{Ред.}
} Отже, він потребує популярности
серед своїх підданих і прив’язує їх до себе циганськими звичаями,
що процвітають під його опікою. Груба невгамовність, весела
\parbreak{}  %% абзац продовжується на наступній сторінці
