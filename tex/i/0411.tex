бітку... Діти й підлітки мають право на те, щоб закон захищав їх
проти зловживання батьківської влади, яке передчасно нищить
їхню фізичну силу й понижує їхній моральний та інтелектуальний
рівень».311 Однак не зловживання батьківською владою
створило цю безпосередню або посередню експлуатацію недозрілих
робочих сил капіталом; навпаки, капіталістичний спосіб
експлуатації, знищивши економічну основу, що відповідала батьківській
владі, викликав зловживання цією владою. Хоч і яким
страшним і огидливим з’являється розклад старої родини всередині
капіталістичної системи, а все ж велика промисловість,
призначаючи жінкам, підліткам і дітям обох статей вирішальну
ролю в суспільно-організованому процесі продукції поза сферою
хатнього господарства, створює нову економічну основу для
вищої форми родини й відносин поміж обома статями. Певна
річ, однаково абсурдно вважати за абсолютну форму родини її
християнсько-германську форму, як і староримську, або старогрецьку,
або східню, які, зрештою, становлять історичний ряд
розвитку. Так само ясно, що склад комбінованого робочого персоналу
з індивідів обох статей і найрізнішого віку, хоч він у своїй
грубій, стихійно виниклій капіталістичній формі, де робітник
існує для процесу продукції, а не процес продукції для робітника,
є отруйне джерело морального зіпсуття й рабства, — за відповідних
умов мусить, навпаки, перетворитись на джерело гуманного
розвитку.312

Доконечність перетворити фабричний закон із виняткового
закону для пряділень і ткалень, цих перших витворів машинового
виробництва, на загальний закон усієї суспільної продукції, випливає,
як ми вже бачили, з історичного ходу розвитку великої промисловости,
на задньому пляні якої зазнають цілковитого перевороту
традиційні форми мануфактури, ремества й домашньої
праці: мануфактура постійно перетворюється на фабрику, ремество
постійно перетворюється на мануфактуру, і, нарешті,
сфери ремества й домашньої праці в дивовижно короткий час
перетворюються на злиденні трущоби, де необмежено панує найшаленіша,
потворна капіталістична експлуатація. Дві обставини
відіграють кінець-кінцем вирішальну ролю: поперше, спостереження,
яке постійно повторюється, що капітал, коли він підпадає
під державний контроль лише на поодиноких пунктах
суспільної периферії, тим безмірніше відшкодовує себе в інших
пунктах; 313 подруге, волання самих капіталістів про рівність
умов конкуренції, тобто про рівні межі експлуатації праці. 314

311 «Children’s Employment Commission. 5 th Report», p. XXV,
n. 162 і 2 nd Report, p. XXXVIII, n. 285, 289, p. XXXV, n. 191.

312 «Фабрична праця могла б бути так само чиста і приємна, як домашня
праця, а то, може, навіть і більше» («Factory labour may be as
pure and as excellent as domestic labour, and perhaps more so»). («Reports
of Insp. of Fact, for 31 st October 1865», p. 127).

313 Там же, стор. 27, 32.

314 Масові приклади цього в «Reports of Insp. of Fact.».
