\parcont{}  %% абзац починається на попередній сторінці
\index{i}{0237}  %% посилання на сторінку оригінального видання
незгладиму ганьбу англійської робітничої кляси те, що на своєму
прапорі проти капіталу, що мужньо боровся, за «повну волю
праці», вона написала «рабство фабричних законів».\footnote{
Ure, французький переклад «Philosophie des Manufactures», Paris
1836, vol. II, p. 39, 40, 67, 77 etc.
}

Франція поволі плентається за Англією. Треба було Лютневої
революції, щоб з’явився на світ дванадцятигодинний закон,\footnote{
B Compte Rendu\footnote*{
— звіті. \emph{Ред.}
} «Інтернаціонального Статистичного Конґресу
в Парижі, 1855 р.» сказано, між іншим: «Французький закон,
що обмежує тривання денної праці по фабриках і майстернях 12 годинами,
не встановлює для цієї праці певних сталих годин (періодів часу); лише
для дитячої праці приписано період між 5 годиною ранку й 9 годиною вечора.
Тим то частина фабрикантів користується з права, яке дає їм
це фатальне замовчування, щоб примушувати працювати день-у-день
без перерви, хіба що за винятком неділь. Вони вживають для цього двох
різних змін робітників, що з них жодна не перебуває в майстерні більш,
ніж 12 годин, хоч продукція в підприємстві триває і вдень і вночі. Вимоги
закону задоволено, але чи задоволено також вимоги гуманности?» Окрім
«руйнаційного впливу нічної праці на людський організм», підкреслюється
також «фатальний вплив спільного перебування вночі обох статей
у тих самих скупо освітлених майстернях».
}
що мав далеко більше хиб, ніж його англійський ориґінал. А, проте,
французька революційна метода виявляє теж свої своєрідні
переваги. Одним замахом вона диктує всім майстерням і фабрикам
однаково одну й ту саму межу робочого дня, тимчасом як
англійське законодавство проти волі то на цьому, то на тому
пункті робить під натиском обставин поступки, вишукуючи найпевніший
шлях, щоб вигадати якусь нову юридичну хитромудрість.\footnote{
«Наприклад, у моїй окрузі в тому самому фабричному будинку
той самий фабрикант як білильник і фарбар підлягає «законові про білильні
й фарбарні», як перкалевиробник — «Printwork’s Act’ові» і як finisher —
«фабричному актові»\dots{} («Report of Mr. Baker» у «Reports etc. for 31 st
Oct. 1861», p. 20). Перелічивши різні постанови цих законів і ускладнення,
що з того випливають, пан Бекер каже: «Ми бачимо, як важко забезпечити
виконання цих трьох парляментських законів, коли власник фабрики
захоче обійти закон». Але панам юристам це забезпечує процеси.
}
З другого боку, французький закон проголошує як
принцип те, що в Англії виборюють лише в ім’я дітей, неповнолітніх
і жінок, і чого тільки останніми часами домагаються як
загального права.\footnote{
Так, фабричні інспектори зважуються, нарешті, сказати: «Ці
заперечення (капіталу проти законного обмеження робочого часу) повинні
впасти перед широким принципом прав праці\dots{} є пункт, після якого право
хазяїна порядкувати працею свого робітника припиняється, і сам робітник
може порядкувати своїм часом навіть тоді, коли ще немає мови про повне
виснаження його» («These objections must succumb before the broad
principle of the rights of labour\dots{} there is a time when the master’s right
in his workmann’s labour ceases and his time becomes his own, even if
there was no exhaustion in the question»). («Reports etc. for 31 st Oct.
1862», p. 54).
}

У Сполучених штатах Північної Америки всякий самостійний
робітничий рух лишався паралізованим так довго, доки
одну частину республіки згиджувало рабство. Праця білошкурих
\index{i}{0238}  %% посилання на сторінку оригінального видання
не може визволитися там, де праця чорношкурих має на собі
ганебне тавро. Але зі смерти рабства зараз же постало нове відмолоділе
життя. Першим плодом громадянської війни була аґітація
за восьмигодинний день — аґітація, що семимильними
кроками льокомотива поширилась від Атлантійського океану до
Тихого, від Нової Англії до Каліфорнії. Загальний робітничий
конґрес у Балтіморі (16 серпня 1866 р.) заявляє: «Першою й
великою потребою сучасности, щоб визволити працю цієї країни
з-під капіталістичного рабства, є видання закону, за яким вісім
годин становили б нормальний робочий день по всіх штатах Американського
Союзу. Ми вирішили напружити всі свої сили, щоб
досягти цього славетного результату».\footnote{
«Ми, робітники з Дункірку, заявляємо, що довжина робочого
дня, що її вимагають за теперішньої системи, занадто велика й не лишає
робітникові часу для відпочинку і розвитку; навпаки, вона принижує
його до стану поневолення, який небагато кращий від рабства («а condition
of servitude but little better than slavery»). Тим то ми ухвалили, що
для робочого дня досить 8 годин, і що закон мусить визнати цей час достатнім:
ми кличемо собі на допомогу пресу, цю могутню підойму\dots{} а
всіх, хто відмовить нам цієї допомоги, ми вважатимемо за ворогів реформи
праці й робітничих прав». (Постанови робітників з Дункірку, штат
Нью-Йорк, 1866 р.).
} Одночасно (початок
вересня 1866 р.) «Інтернаціональний робітничий конґрес» у Женеві
на пропозицію лондонської генеральної ради ухвалив: «Ми
заявляємо, що обмеження робочого дня є попередня умова, без
якої всі інші визвольні старання мусять розбитися\dots{} Ми пропонуємо
8 годин праці як законну межу робочого дня».

Таким чином робітничий рух, що інстинктово виріс із самих
відносин продукції по обох боках Атлантійського океану, стверджує
заяву англійського фабричного інспектора Р. Дж. Савндерса:
«Неможливо зробити дальші кроки для реформи суспільства
з якоюсь надією на успіх, якщо уперед не обмежити робочий
день і не примусити строго додержувати його меж, приписаних
законом».\footnote{
«Reports etc. for 31 st October 1848», p. 112.
}

Треба визнати, що наш робітник виходить із процесу продукції
не таким, яким увійшов до нього. На ринку він протистояв
посідачам інших товарів як посідач товару «робоча сила», тобто
як посідач товару посідачеві товару. Контракт, за яким він продавав
капіталістові свою робочу силу, так би мовити, чорним по
білому показував, що він вільно порядкує самим собою. Після того,
як торг уже закінчено, виявляється, що він не був «вільним
аґентом», що час, на який йому вільно продавати свою робочу
силу, є час, на який він примушений її продавати,\footnote{
«Ці вчинки (маневри капіталу, приміром, 1848--1850 рр.) дали,
крім того, незаперечний доказ брехливости так часто висуваного твердження,
нібито робітники не потребують охорони, і що їх треба розглядати
як аґентів, що вільно порядкують єдиною своєю власністю, тобто
працею рук своїх і потом лиця свого» («These proceedings bave afforded,
moreover, incontrovertible proof of the fallacy of the assertion so often
advanced, that operatives need no protection, but may be considered as
free agents in the disposal of the only property they possess, the labour of
} що в дійсності
\index{i}{0239}  %% посилання на сторінку оригінального видання
вампір, який висисає його, не випускає його «доти, доки для
експлуатації лишається ще хоч однісінький мускул, однісінька
жила, однісінька крапля крови».\footnote{
Friedrich Engels: «Lage der arbeitenden Klasse in England»,
стор. 5. (Ф. Енгельс: «Становище робітничої кляси в Англії», Партвидав
«Пролетар», 1932 р.).
} Щоб «боронити» себе проти
змія, який мучить їх, робітники мусять об’єднатися і, як кляса,
добитися державного закону, могутньої суспільної перешкоди,
яка самим їм заважала б шляхом добровільного договору з капіталом
продавати на смерть і рабство себе й своїх нащадків.\footnote{
Десятигодинний біл у підпорядкованих йому галузях промисловости
«врятував робітників од цілковитого виродження і захистив їхнє
фізичне здоров’я». («Reports etc. for 31 st Oct. 1859», p. 47). «Капітал
(на фабриках) ніколи не може тримати машини в русі понад обмежений
період часу без того, щоб не шкодити здоров’ю й моральності робітників,
що в нього працюють; а робітники не в стані самих себе захистити».
(Там же, стор. 8).
}
Замість бучного каталогу «невідійманнх прав людини» з’являється
скромна Magna Charta\footnote*{
— основні права. \emph{Ред.}
} обмеженого законом робочого
дня, яка, «нарешті, з’ясовує, коли кінчається час, що його робітник
продає, і коли починається той час, що належить йому самому».\footnote{
«Ще більша користь у тому, що, нарешті, виясняється ріжниця
між власним часом, що належить робітникові, та часом, що належить
його хазяїнові. Робітник знає тепер, коли кінчається той час, що його
він продає, і коли починається той час, що належить йому; знаючи це,
він має змогу наперед розподілити свої хвилини часу, відповідно до своїх
потреб». («А still greater boon is, the distinction at least made clear between
the worker’s own time and his master’s. The worker knows now when
that which he sells is ended, and when his own begins, and by possessing a sure
foreknowledge of this, is enabled to pre-arrange his own minutes for his
own purpose’s»). (Там же, стор. 52). «Зробивши робітників хазяїнами
їхнього власного часу, вони (фабричні закони) дали їм моральну силу,
яка приведе їх однієї днини до захоплення політичної влади» («Ву
making them masters of their own time, they have given them a moral energy
which is directing them to the eventual possession of political power»).
(Там же, стор. 47). Із стриманою іронією і в дуже обережних висловах
натякають фабричні інспектори на те, що теперішній десятигодинний
закон до певної міри звільнив також і капіталіста від його природної
брутальности, властивої йому просто лише як персоніфікації капіталу, та
дав йому час для деякої «освіти». Раніш «хазяїн не мав часу на щось
інше, окрім набування грошей, а робітник не мав часу на щось інше,
крім праці» («the master had no time for anything but money; the servant
had no time for anything but labour»). (Там же, стор. 48).
}
Quantum muta tus ab illo!\footnote*{
Яка велика ріжниця проти попереднього! \emph{Ред.}
}

their hands, and the sweat of their brows»). («Reports etc. for 30 th April
1850», p. 45). «Вільна праця, коли її можна так назвати, навіть у вільній
країні потребує для своєї охорони міцної руки закону» («Free Labour,
if so it may be termed, even in a free country requires the strong arm of
the law to protect it»). («Reports etc. for 31 st Oct. 1864», p. 34). «Дозволити,
а це те саме, що примусити\dots{} працювати по 14 годин денно з перервами
на їжу або без них і т. ін.» («То permit, which is tantamount to
compelling\dots{} to work 14 hours a day with or without meals etc.»). («Reports
etc. for 30 th April 1863», p. 40).
