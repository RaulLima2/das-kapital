\index{i}{0063}  %% посилання на сторінку оригінального видання
Придивившись тепер до кінцевого результату торговлі, ткач
помітить, що замість полотна він має біблію, замість свого первісного
товару — інший товар тієї самої вартости, але відмінної
корисности. Подібним чином присвоює він собі й інші засоби
існування і засоби продукції. З його погляду цілий процес
упосереднює лише обмін продукту його праці на продукт чужої
праці, обмін продуктів.

Отже, процес обміну товару відбувається у вигляді такої зміни
форми:

Товар — Гроші — Товар

$Т — Г — Т$

За своїм речовим змістом цей рух є $Т — Т$, обмін товару на
товар, обмін речовин суспільної праці, у результаті якого згасає
й самий процес.

$Т — Г$. Перша метаморфоза товару, або продаж. Перескакування
товарової вартости з тіла товару на тіло золота є, як я
це назвав в іншому місці, salto mortale* товару. Коли він не
вдасться, то обдурений буде, правда, не товар, а товаропосідач.
Суспільний поділ праці робить працю товаропосідача так само
однобічною, як його потреби різнобічними. Саме через це його
продукт служить для нього лише за мінову вартість. Загальної
суспільно визнаної еквівалентної форми його продукт набирає
лише у грошах, а гроші лежать у чужій кишені. Щоб витягти
їх звідти, товар насамперед мусить бути споживною вартістю
для посідача грошей, отже, витрачена на нього праця мусить
бути витрачена в суспільно-корисній формі, або виявити себе
як член суспільного поділу праці. Але поділ праці є природно
вирослий продукційний організм, нитки якого ткалося й далі
тчеться за спиною продуцентів товарів. Може бути, товар є продукт
нового роду праці, який має на увазі задовольнити якусь
новопосталу потребу або хоче викликати якусь нову потребу.
Якась осібна операція праці, що вчора ще була одною серед
багатьох функцій одного й того самого товаропродуцента, сьогодні,
може бути, розриває цей зв’язок, усамостійнюється й
саме через це посилає на ринок свій частинний продукт як самостійний
товар. Обставини для цього процесу відокремлення можуть
бути дозрілі або недозрілі. Сьогодні продукт задовольняє
якусь суспільну потребу. Завтра, може бути, його виштовхнуть
з його місця цілком або частинно інші подібні продукти. І коли
навіть праця якогось продуцента, приміром, нашого ткача, є
патентований член суспільного поділу праці, то це ще зовсім
не ґарантує споживної вартости саме його 20 метрів полотна.
Коли суспільну потребу в полотні, — а вона, як і все інше, має
свою міру, — задовольнили вже конкуренти нашого ткача, то
продукт нашого приятеля робиться залишнім, зайвим, а через
те й некорисним. Звичайно, дарованому коневі в зуби не дивляться,

— карколомний скік. \emph{Ред.}
\parbreak{}  %% абзац продовжується на наступній сторінці
