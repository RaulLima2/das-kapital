продукції й панування над працею, то, з другого боку, вона
виявляється як взаємне відштовхування багатьох індивідуальних
капіталів.

Цьому роздрібненню цілого суспільного капіталу на багато
індивідуальних капіталів або взаємному відштовхуванню його
частин протидіє їхнє притягання. Це вже не проста, ідентична
з акумуляцією концентрація засобів продукції й панування над
працею. Це концентрація утворених уже капіталів, знищення
їхньої індивідуальної самостійности, експропріяція капіталіста
капіталістом, перетворення багатьох дрібних капіталів на незначне
число великих капіталів. Цей процес відрізняється від
першого тим, що він має за свою передумову лише зміну в розподілі
тих капіталів, які вже існують і функціонують, отже, його
поле діяльности не обмежене абсолютним зростанням суспільного
багатства або абсолютними межами акумуляції. Капітал зростає
великими масами тут, в одних руках, бо він зникає там, з багатьох
рук. Це — централізація у власному значенні слова, відмінно
від акумуляції й концентрації.

Законів цієї централізації капіталів або притягання капіталу
капіталом ми не можемо тут досліджувати. Досить буде коротких
фактичних вказівок. Конкуренційна боротьба провадиться через
здешевлення товарів. Дешевина товарів залежить, за інших
незмінних обставин, від продуктивности праці, а ця остання
залежить від маштабу продукції. Тим то більші капітали побивають
дрібніші. Пригадаймо собі, далі, що з розвитком капіталістичного
способу продукції зростає мінімальний розмір індивідуального
капіталу, потрібного на те, щоб провадити підприємство
в нормальних умовах. Тому дрібніші капітали ринуть
у такі сфери продукції, що їх велика промисловість опановує лише
спорадично або не цілком. Конкуренція лютує тут просто пропорційно
до числа й зворотно пропорційно до величини капіталів,
що борються між собою. Вона завжди кінчаєтеся загином багатьох
дрібних капіталістів, що їхні капітали почасти переходять до
рук переможців, а почасти гинуть. Крім цього, разом з капіталістичною
продукцією постає цілком нова сила, кредит, що спочатку
потайки прокрадається як скромний помагач акумуляції,
незримими нитками стягує в руки індивідуальних або асоційованих
капіталістів грошові засоби, розпорошені більшими або
меншими масами по поверхні суспільства; але незабаром він стає
новою і страшною зброєю в конкуренційній боротьбі і, кінець-кінцем,
перетворюється на велетенський соціяльний механізм
для централізації капіталів.

Тією самою мірою, як розвивається капіталістична продукція
й акумуляція, розвиваються також конкуренція і кредит,
ці обидві наймогутніші підойми централізації. Поруч цього
проґрес акумуляції збільшує матеріял, що його можна централізувати,
тобто збільшує поодинокі капітали, тимчасом як поширення
капіталістичної продукції утворює, з одного боку,
суспільну потребу, а з другого — технічні засоби для тих потуж-
