\index{i}{0101}  %% посилання на сторінку оригінального видання
\chapter{Перетворення грошей на капітал}
\section{Перетворення грошей на капітал}
\subsection{Загальна формула капіталу}

Товарова циркуляція є вихідний пункт капіталу. Товарова
продукція й розвинена товарова циркуляція, торговля, становлять
історичні передумови, за яких він постає. Світова торговля
й світовий ринок починають у XVI віці сучасну історію
капіталу.

Коли залишимо осторонь речовий зміст циркуляції товарів,
обмін різних споживних вартостей, і розглядатимемо лише ті
економічні форми, які створює цей процес, то ми побачимо,
що його останній продукт є гроші. Цей останній продукт товарової
циркуляції є перша форма виявлення капіталу.

Історично капітал спочатку скрізь протистоїть земельній
власності у формі грошей, як грошове майно, купецький капітал
і лихварський капітал\footnote{
Протилежність поміж владою земельної власности, що спирається
на відносини особистого поневолення й панування, і безособовою владою
грошей ясно висловлено у двох французьких приказках: «Nulle terre
sans seigneur». «L’argent n’a pas de maître»\footnote*{
«Нема землі без господаря». «Гроші не мають господаря». \emph{Ред.}
}.
}. Однак не потрібно навіть звертатись до
минулости, до історії постання капіталу, щоб пізнати, що гроші
є перша форма виявлення капіталу. Та сама історія щодня відбувається
перед нашими очима. Кожний новий капітал у першій
своїй інстанції виходить на сцену, тобто на ринок, на товаровий
ринок, на робочий ринок або грошовий ринок, завжди у формі
грошей, — грошей, що через певні процеси повинні перетворитися
на капітал.

Гроші як гроші і гроші як капітал відрізняються насамперед
лише своїми різними формами циркуляції.

Безпосередня форма товарової продукції є $Т — Г — Т$, перетворення
товару на гроші і зворотне перетворення грошей на
товар, продаж задля купівлі. Але поруч цієї форми ми находимо
другу, специфічно відмінну форму, форму $Г — Т — Г$, перетворення
\index{i}{0102}  %% посилання на сторінку оригінального видання
грошей на товар і зворотне перетворення товару на гроші,
купівлю задля продажу. Гроші, що у своїм русі пророблюють
цю останню циркуляцію, перетворюються на капітал, стають
капіталом і вже за своїм призначенням є капітал.

Пригляньмося ближче до циркуляції $Г — Т — Г$. Вона перебігає,
як і проста товарова циркуляція, дві протилежні фази.
У першій фазі, $Г — Т$, в купівлі, гроші перетворюються на товар.
У другій фазі, $Т — Г$, у продажу, товар зворотно перетворюється
на гроші. Але єдність обох фаз — це сукупний рух, що
обмінює гроші на товар і цей самий товар знов на гроші, купує
товар, щоб його продати, або, якщо не зважати на формальну
ріжницю між купівлею і продажем, за гроші купує товар і за
товар гроші\footnote{
«За гроші купуємо товари, а за товари гроші» («Avec de l’argent
on achète des marchandises, et avec des marchandises on achète de l’argent).
(\emph{Mercier de la Rivière}: «L’ordre naturel et essentiel des sociétés
politiques» Physiocrates, éd. Daire, 11. Partie, p. 543).
}. Результат, що в ньому згасає цілий процес, є
обмін грошей на гроші, $Г — Г$. Коли я за 100\pound{ фунтів стерлінґів}
купую \num{2.000} фунтів бавовни і знов продаю ці \num{2.000} фунтів
бавовни за 110\pound{ фунтів стерлінґів}, то, кінець-кінцем, я
обміняв 100\pound{ фунтів стерлінґів} на 110\pound{ фунтів стерлінґів} — гроші
на гроші.

Правда, тепер очевидно, що процес циркуляції $Г — Т — Г$
був би безглуздим і беззмістовним, коли б такими манівцями
бажали обміняти певну грошову вартість на таку саму грошову
вартість, отже, приміром, 100\pound{ фунтів стерлінґів} на 100\pound{ фунтів
стерлінґів}. Куди простішою й певнішою лишалася б метода
вбирача скарбу, який затримує в себе свої 100\pound{ фунтів стерлінґів},
замість того, щоб віддати їх на небезпеку циркуляції. З другого
боку, чи купець куплену ним за 100\pound{ фунтів стерлінґів} бавовну
знову продає за 100\pound{ фунтів стерлінґів}, чи мусить він продати її за
100\pound{ фунтів стерлінґів}, а то й за 50\pound{ фунтів стерлінґів}, — за всяких
обставин його гроші пророблюють своєрідний, ориґінальний
рух, цілком відмінний од того руху, що його пророблюють
гроші у простій товаровій циркуляції, приміром, у руках селянина,
який продає збіжжя і за виручені таким чином гроші
купує одяг. Отже, насамперед потрібно схарактеризувати ріжницю
форм кругобігів $Г — Т — Г$ і $Т — Г — Т$. Разом із тим
виясниться і ріжниця в змісті, що криється за цими ріжницями
форм.

Погляньмо насамперед, що є спільного в обох цих формах.

Обидва кругобіги розпадаються на ті самі дві протилежні
фази: $Т — Г$, продаж, і $Г — Т$, купівля. В кожній з обох цих
фаз протистоять один одному ті самі два речові елементи, товар
і гроші, — і дві особи в тих самих характеристичних економічних
масках, покупець і продавець. Кожний з цих обох кругобігів
є єдність тих самих протилежних фаз, і обидва рази ця
єдність здійснюється за допомогою появи трьох контраґентів,
\parbreak{}  %% абзац продовжується на наступній сторінці
