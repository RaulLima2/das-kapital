\parcont{}  %% абзац починається на попередній сторінці
\index{i}{0030}  %% посилання на сторінку оригінального видання
40 фунтів кави = 20 метрам полотна; отже, 10 фунтів чаю =
40 фунтам кави, або в 1 фунті кави міститься лише чверть тієї
кількости субстанції вартости, праці, що міститься в 1 фунті чаю.

Загальна відносна форма вартости товарового світу надає
(drückt\dots{} auf) виключеному в цього світу еквівалентному товарові,
полотну, характер загального еквіваленту. Власна натуральна
форма полотна є тепер спільна форма вартости товарового
світу, а тому воно тепер є безпосередньо вимінне на всі інші товари.
Його тілесна форма фігурує як видиме втілення, як загальна
суспільна лялечка (Verpuppung) всякої людської праці. Ткацтво,
приватна праця, що продукує полотно, перебуває разом з тим у
загальній суспільній формі, у формі рівности з усіма іншими
працями. Незчисленні рівнання, що з них складається загальна
форма вартости, по черзі прирівнюють працю, здійснену в полотні,
до кожної праці, що міститься в інших товарах, і через те роблять
ткацтво загальною формою виявлення людської праці взагалі.
Таким чином праця, упредметнена в товаровій вартості, виявляється
не лише неґативно, як праця, абстрагована від усіх конкретних
форм і корисних властивостей дійсних праць, але й її
власна позитивна природа виразно виступає наперед. Вона —
зведення всіх дійсних праць до спільного їм характеру людської
праці, до витрати людської робочої сили.

Загальна форма вартости, яка виражає продукти праці просто
як згустки безріжницевої людської праці, свідчить своєю власною
будовою, що вона є суспільний вираз товарового світу. Таким
чином вона виявляє, що в межах цього світу загальнолюдський
характер праці становить її специфічний суспільний характер.

\paragraph{Відношення між розвитком відносної форми вартостн й розвитком
еквівалентної форми}

Ступеневі розвитку відносної форми вартости відповідає ступінь
розвитку еквівалентної форми. Однак — і це слід добре запам’ятати
— розвиток еквівалентної форми є лише вираз і результат
розвитку відносної форми вартости.

Проста або одинична відносна форма вартости якогось товару
робить інший товар одиничним еквівалентом. Розгорнута форма
відносної вартости, цей вираз вартости якогось товару в усіх
інших товарах, надає їм (prägt ihnen\dots{} auf) форми різнорідних
осібних еквівалентів. Нарешті, даний осібний рід товару набирає
загальної еквівалентної форми тому, що всі інші товари роблять
його матеріялом своєї однорідної загальної форми вартости.

Але такою самою мірою, якою розвивається форма вартости
взагалі, розвивається і протилежність між її обома полюсами —
відносною формою вартости й еквівалентною формою.

Уже перша форма — 20 метрів полотна = 1 сурдутові —
містить у собі цю протилежність, але не фіксує її. Відповідно до
того, як читати це рівнання — з лівого боку на правий чи навпаки,
кожен з обох товарових полюсів, і полотно і сурдут, рівномірно
\index{i}{0031}  %% посилання на сторінку оригінального видання
перебуватимуть то у відносній формі вартости, то в еквівалентній
формі. Тут ще важко фіксувати цю полярну протилежність.

У формі В завжди лише один якийсь рід товару може цілком
розгорнути свою відносну вартість, або сам він має розгорнуту
відносну форму вартости лише тому й остільки, що й оскільки
всі інші товари протистоять йому в еквівалентній формі. Тут уже
не можна переставити обидві частини вартостевого рівнання: 20 метрів
полотна = 1 сурдутові, або = 10 фунтам чаю, або = 1 квартерові
пшениці й т. ін., не змінюючи його загального характеру
й не перетворюючи його з повної на загальну форму вартости.

Остання форма, форма С, дає, нарешті, товаровому світові
загальносуспільну відносну форму вартости тому й остільки,
що й оскільки всі належні до нього товари, за одним-однісіньким
винятком, є виключені з загальної еквівалентної форми. Тому
один товар, полотно, перебуває у формі безпосередньої вимінности
на всі інші товари, або в безпосередньо суспільній формі, тому
й остільки, що й оскільки всі інші товари не перебувають у такій
формі.\footnote{
В дійсності на перший погляд з форми загальної безпосередньої
вимінности ніяк не пізнати, що вона є суперечна товарова форма, так
само невіддільна від протилежної форми, що в ній неможлива безпосередня
вимінність, як позитивний полюс магнету від його негативного полюса.
Тим то уявити собі, що на всі товари можна одночасно наложити печать
безпосередньої вимінности можна з таким самим успіхом, як можна
уявити собі, що всіх католиків можна одночасно поробити папами. Для
дрібного буржуа, який у товаровій продукції бачить nec plus ultra\footnote*{
— вершину. \emph{Ред}} людської волі й індивідуальної незалежности, було б, натурально, дуже
бажано позбавитись невигід, зв’язаних з цією формою, особливо ж тієї, що
товари не можуть вимінюватись безпосередньо. Розмальовування цієї філістерської
утопії становить прудонівський соціялізм, який, як я це показав
в іншому місці, не має в собі навіть нічого ориґінального і який далеко
раніш значно краще розвинули були Gray, Bray і ін. Та це не заважає
отакій мудрості ще й нині, як якійсь пошесті, ширитися в певних колах
під ім’ям «science».\footnote*{— «наука». \emph{Ред.}}
Жодна школа не панькалась так із словом «science», як прудонівська, бо
\settowidth{\versewidth}{da stellt zur rechten Zeit ein Wort sich ein»}
\begin{verse}[\versewidth]
«Wo Begriffe fehlen,\\
da stellt zur rechten Zeit ein Wort sich ein»
\end{verse}

(«Де бракує понять, там саме в пору з’являється слово»).
}

Навпаки, товар, що фігурує як загальний еквівалент, є виключений
з однорідної, а тому й загальної відносної форми вартости
товарового світу. Для того, щоб полотно, тобто будь-який
товар, що перебуває у формі загального еквіваленту, разом з тим
брав участь також і в загальній відносній формі вартости, він
мусив би служити за еквівалент для самого себе. Ми мали б тоді:
20 метрів полотна = 20 метрам полотна — тавтологію, де не виражено
ні вартости, ані величини вартости. Щоб виразити відносну
вартість загального еквіваленту, ми скорше мусимо обернути
форму С. Він не має спільної з іншими товарами відносної
форми вартости, але його вартість відносно виражається у безконечному
\index{i}{0032}  %% посилання на сторінку оригінального видання
ряді всіх інших товарових тіл. Таким чином розгорнута
відносна форма вартости, або форма В, з’являється тепер
як специфічна відносна форма вартости товару-еквіваленту.

\paragraph{Перехід від загальної форми вартости до грошової форми}

Загальна еквівалентна форма є форма вартости взагалі. Отже,
вона може належати кожному товарові. З другого боку, даний
товар перебуває в загальній еквівалентній формі (формі С) лише
тому й остільки, що й оскільки його, як еквівалент, виключають
з-поміж себе всі інші товари. І лише від того моменту, коли це
виключення остаточно обмежується на якомусь одному специфічному
роді товару, однорідна відносна форма вартости товарового
світу здобуває об’єктивну сталість і загальну суспільну значеність
(Gültigkeit).

А специфічний рід товару, що з його натуральною формою
суспільно зростається еквівалентна форма, стає грошовим товаром,
або функціонує як гроші. Відігравати в межах товарового
світу ролю загального еквіваленту — це стає його специфічною
суспільною функцією, а тому і його суспільною монополією. Це
упривілейоване місце серед товарів, що в формі В фігурують як
осібні еквіваленти полотна, а в формі С спільно виражають у
полотні свою відносну вартість, історично завоював певний товар,
а саме золото. Тому, коли ми в формі С замість товару
«полотно» поставимо товар «золото», то матимемо:

\subsubsection{Грошова форма}

\begin{equation*}
\left.\begin{aligned}
&20\text{ метрів полотна} &= \\
&1\text{ сурдут} &= \\
&10\text{ фунтів чаю} &= \\
&40\text{ фунтів кави} &= \\
&1\text{ квартер пшениці} &= \\
&\sfrac{1}{2}\text{ тонни заліза} &= \\
&х\text{ товару }А &= \\
\end{aligned}\right\rbrace
2\text{ унціям золота}
\end{equation*}

При переході від форми А до форми В й від форми В до форми
С відбуваються посутні зміни. Навпаки, форма D нічим не
відрізняється від форми С, як тільки тим, що тепер замість полотна
форму загального еквіваленту має золото. Золото лишається
в формі D тим, чим полотно було в формі С, — загальним
еквівалентом. Проґрес лише в тім, що форма безпосередньої загальної
вимінности, або загальна еквівалентна форма, в наслідок
суспільної звички тепер остаточно зрослася зі специфічною натуральною
формою товару «золото».

Золото виступає проти інших товарів як гроші лише тому, що
воно вже раніш протистояло їм як товар. Як і всі інші товари,
золото також функціонувало як еквівалент, чи то як одиничний
еквівалент в поодиноких актах обміну, чи то як осібний еквівалент
\index{i}{0033}  %% посилання на сторінку оригінального видання
поруч з іншими товаровими еквівалентами. Помалу воно
почало у вужчих або ширших сферах функціонувати як загальний
еквівалент. Скоро тільки воно завоювало собі монополію на це
місце у виразі вартостей товарового світу, воно стало грошовим
товаром, і лише від того моменту, коли воно вже стало грошовим
товаром, форма Б відрізняється від форми С, або загальна
форма вартости перетворюється на грошову форму.

Простий відносний вираз вартости якогось товару, приміром,
полотна, в товарі, що вже функціонує як грошовий товар, приміром,
у золоті, є форма ціни. Отже, «форма ціни» полотна така:\[
20\text{ метрів полотна }= 2\text{ унціям золота,}
\]

\noindent{}або, коли 2 фунти стерлінґів є монетна назва 2 унцій золота,\[
20\text{ метрів полотна }= 2\text{ фунтам стерлінґів.}
\]

Трудність у понятті грошової форми обмежується зрозумінням
загальної еквівалентної форми, отже, загальної форми вартости
взагалі, форми С. Але форма С зводиться ретроспективно
на форму В, на розгорнуту форму вартости, а конститутивним
елементом цієї останньої є форма А: 20 метрів полотна = 1 сурдутові,
або $х$ товару $А = у$ товару $В$. Тому проста товарова форма
є зародок грошової форми.

\paragraph{Фетишистичний характер товару та його таємниця}

На перший погляд товар видається цілком зрозумілою, тривіяльною
річчю. Його аналіза виявляє, що це дуже чудернацька
річ, повна метафізичних тонкощів і теологічних примх. Як споживна
вартість він не має в собі нічого загадкового, хоч розглядати
його з того погляду, що своїми властивостями він задовольняє
людські потреби, хоч з того, що самих цих властивостей він
набуває лише як продукт людської праці. Цілком зрозуміло,
що людина своєю діяльністю змінює форми природних матеріялів
так, що робить їх корисними для себе. Наприклад, змінює форму
дерева, виробляючи з нього стіл. А все ж таки стіл лишається
деревом, ординарною, сприйманою почуттям річчю. Але скоро
тільки він виступає як товар, він перетворюється в почуттєвонадпочуттєву
річ. Він не лише стоїть на землі своїми ніжками,
але проти всіх інших товарів він стає на голову, і ця його дерев’яна
голова плодить химери куди дивовижніші, ніж коли б
він почав самохіттю танцювати.\footnote{
Пригадаймо собі, що Китай і столи почали танцювати тоді, коли
ввесь інший світ, здавалося, стояв спокійно — pour encourager les autres.\footnote*{
— щоб підбадьорити інших. \emph{Ред.}
}
}

Отже, містичний характер товару випливає не з його споживної
вартости. Так само мало випливає він із змісту визначень вартости.
Бо, по-перше, хоч які різні можуть бути корисні праці або продуктивні
діяльності, все ж таки — це фізіологічна істина, що
вони є функції людського організму, і що кожна така функція,
\parbreak{}  %% абзац продовжується на наступній сторінці
