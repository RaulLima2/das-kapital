\parcont{}  %% абзац починається на попередній сторінці
\index{i}{0016}  %% посилання на сторінку оригінального видання
не дає їм жодної форми вартости, відмінної від їхніх натуральних форм. Інакше стоїть справа з
вартостевим відношенням одного товару до іншого. Вартостевий характер товару виступає тут у наслідок
його власного відношення до іншого товару.

Коли, приміром, сурдут як предмет вартости порівнюється до полотна, то тим і працю, що міститься в
першому, порівнюється до праці, що міститься в другому. Правда, кравецтво, що виробляє сурдут, є
конкретна праця іншого роду, ніж ткацтво, що виробляє полотно. Але порівняння кравецтва до ткацтва
зводить кравецтво фактично на те, що в обох працях є дійсно однакове, — на їх спільний характер
людської праці. Таким манівцем і висловлюють, що і ткацтво, оскільки воно тче вартість, не має від
кравецтва жодних відрізняльних ознак, отже є абстрактна людська праця. Лише вираз еквівалентности
різнорідних товарів виявляє специфічний характер праці, яка творить вартість, зводячи фактично
різнорідні праці, що містяться в різнорідних товарах, на те, що є спільне різнорідним працям, на
людську працю взагалі.\footnoteA{
Примітка до другого видання. Один із перших економістів, що після Вільяма Петті збагнув природу
вартости, славетний Франклін, каже: «Через те, що торговля взагалі є не що інше, як обмін однієї
праці на іншу працю, то найправильніше вартість усіх речей оцінюється працею». («The Works of В.
Franklin etc., edited by Sparks», Boston. 1836, vol. II, p. 267). Франклін не є свідомий того, що,
цінуючи вартість усіх речей «працею», він залишає осторонь ріжницю між виміняними працями, і таким
чином зводить їх на однакову людську працю. Але, не знаючи цього, він, однак, про це говорить.
Спочатку він говорить про «одну працю», далі про «іншу працю», і, нарешті, про «працю» без дальшого
означення як про субстанцію вартости всіх речей.
}

Однак недосить виразити специфічний характер тієї праці, що з неї складається вартість полотна.
Людська робоча сила в поточному стані, або людська праця творить вартість, але сама не є вартість.
Вона стає вартістю в застиглому стані, в предметній формі. Щоб виразити вартість полотна як згусток
людської праці, його треба виразити як якусь «предметність», що речово відрізняється від самого
полотна й одночасно є спільна йому з іншим товаром. Це завдання вже розв’язано.

У вартостевому відношенні полотна сурдут фігурує як якісно рівний полотну, як річ тієї самої
природи, бо він є вартість. Тому він тут має значення як річ, що в ній з’являється вартість, або як
річ, що в своїй обмацальній натуральній формі репрезентує вартість. Правда, сурдут як тіло товару
«сурдут» є лише споживна вартість. Сурдут так само не виражає вартости, як і перший-ліпший сувій
полотна. Це доводить лише, що в межах вартостевого відношення до полотна він має більше значення,
ніж поза ним, як от та або інша людина має більше значення в брузументом вишитому сурдуті, ніж без
нього.

На продукцію сурдута фактично, в формі кравецтва, затрачено людську робочу силу. Отже, в ньому
нагромаджено людську працю. З цього боку сурдут є «носій вартости», дарма що ця його
\parbreak{}  %% абзац продовжується на наступній сторінці
