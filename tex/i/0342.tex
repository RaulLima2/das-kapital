«Через те, що кількість продуктів реґулюється переважно
швидкістю машин, то інтерес фабриканта мусить бути в тому,
щоб гнати їх із якнайбільшою швидкістю, яку тільки можна
сполучити з такими умовами: зберігання машин, щоб вони не
надто скоро псувалися, зберігання якости фабрикованих продуктів
та здатність робітника встигати за рухом без більшого напруження,
ніж яке він може безупинно розвивати. Часто буває так,
що фабрикант, дуже поспішаючися, занадто прискорює рух.
Тоді полами та поганий виріб більш ніж урівноважують швидкість,
і він примушений зробити рух машин повільнішим. А що
активний та розсудливий фабрикант знаходить осяжний максимум,
то я дійшов висновку, що неможливо за 11 годин випродукувати
стільки, як за 12 годин. Крім того, я припускав, що робітник,
якому платять відштучно, напружує свої сили до якнайбільшої
міри, скільки він може безупинно витримувати такий
ступінь інтенсивности праці».166 Тим то Горнер, усупереч експериментам
Ґарднера й ін., дійшов такого висновку, що дальше скорочення
робочого дня нижче від 12 годин мусило б зменшити
кількість продукту.167 Він сам через 10 років цитує свої сумніви
з року 1845 на доказ того, як мало він ще тоді тямив в елястичності
машин та людської робочої сили, що обидві в рівній мірі
напружуються до якнайвищого ступеня через примусове скорочення
робочого дня.

А тепер перейдімо до періоду після 1847 р., від часу запровадження
закону про десятигодинний робочий день в англійських
бавовняних, вовняних, шовкових та лляних фабриках.

«Швидкість веретен на throstles зросла на 500 обертів, на
mules на 1.000 обертів на одну хвилину, тобто швидкість веретена
throstles, яка 1839 р. мала 4.500 обертів на хвилину, становить
тепер (1862 р.) 5.000, а швидкість веретена mules, яка мала

5.000 обертів, становить тепер 6.000 обертів на хвилину; отже,
швидкість збільшилася в першому випадку на 1/І0, а в другому —
на 1/5».168 Джеме Несміс, славетний цивільний інженер із
Patricroft’a коло Менчестеру, в одному листі до Леонарда Горнера
1852 р. пояснив у подробицях про поліпшення, пороблені в паровій
машині між 1848 і 1852 рр. Зауваживши, що парова кінська
сила, яка в офіціяльній фабричній статистиці все ще визначається
за ефектом її в 1828 р.,169 є лише номінальна й може бути

166 «Reports of Insp. of Fact. to 30 th April 1845», p. 20.

167    Там же, стор. 22.

168 «Reports of Insp. of Fact. for 31 st October 1862», p. 62.

169 Це змінилося від часу «Parliamentary Return» * 1862 p. Тут
виступає справжня парова кінська сила сучасних парових машин та водяних
коліс на місце номінальної (див. примітку 109а, стор. 318). І веретен
на сукання (Dublierspindeln) уже не переплутують із прядільними веретенами
у власному значенні (як у «Returns» 1839, 1850 та 1856); далі
для вовняних фабрик подано число «gigs»,** заведено ріжницю між джу-

* — парляментського звіту. Ред.

** — ворсувальних машин. Ред.
