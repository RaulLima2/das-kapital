\parcont{}  %% абзац починається на попередній сторінці
\index{i}{0123}  %% посилання на сторінку оригінального видання
шена витрата зумовлює збільшене відновлення.\footnote{
«Тому староримський villicus, що стояв як управитель на чолі
рільничих рабів, діставав «злиденніше утримання, ніж раби, бо мав
легшу працю». (\emph{Th. Mommsen}: «Römische Geschichte», 1856, стор. 810).
} Коли власник
робочої сили працював сьогодні, то він мусить мати змогу й завтра
повторити той самий процес за тих самих умов сили й здоров’я.
Отже, суми засобів існування мусить вистачати, щоб індивіда,
який працює, утримати як такого в його нормальному життєвому
стані. Сами природні потреби, як от їжа, одяг, паливо, помешкання
тощо, є різні залежно від кліматичних і інших природних
особливостей країни. З другого боку, розміри так званих доконечних
потреб, як і спосіб задоволення їх, сами є історичний
продукт і тому здебільша залежать од культурного рівня країни
і, між іншим, значною мірою і від того, серед яких умов, отже,
з якими звичками й життєвими вимогами утворилася кляса вільних
робітників.\footnote{
Порівн. «Overpopulation and its Remedy», London 1846, von W. Th.
Thornton.
} Отже, протилежно до інших товарів визначення
вартости робочої сили містить у собі історичний і моральний
елемент. Однак для певної країни і для певного періоду пересічна
кількість доконечних засобів існування є дана.

Власник робочої сили є смертний. Отже, щоб раз-у-раз
з’являтися на ринку, а це є передумова безупинного перетворювання
грошей на капітал, продавець робочої сили мусить увічнити
себе, «як увічнює себе кожний живий індивід — через
розплодження».\footnote{
Petty.
} Робочі сили, що сходять із ринку через виснаження
і смерть, мусять постійно поповнюватись, принаймні,
таким самим числом нових робочих сил. Отже, сума засобів
існування, доконечних для продукції робочої сили, обіймає й
засоби існування для заступників, тобто дітей робітників, так
що ця раса своєрідних товаропосідачів увічнюється на товаровому
ринку.\footnote{
«Її (праці) натуральна ціна\dots{} складається з такої кількости засобів
існування і комфорту, яка, залежно від природи й клімату й відповідно
до звичаїв даної країни, доконечна для утримання самого робітника
й для того, щоб дати йому змогу утримувати родину, яка спроможна
була б забезпечити незменшуване постачання праці на ринку» («Its
(labour’s) natural price\dots{} consists in such a quantity of necessaries, and
comforts of life, as, from the nature of the climate, and the habits of the
country, are necessary to support the labourer, and to enable him to rear
such a family as may preserve, in the market, an undiminished supply of
labour»). (\emph{R. Torrens}: «An Essay on the external Corn Trade», London
1815, p. 62). Слово «праця» тут неправильно поставлено замість «робоча
сила».
}

Для модифікації загальної людської натури в такий спосіб,
щоб людина набула вправности та досвіду в певній галузі праці,
щоб стала розвинутою й специфічною робочою силою, потрібно
певної освіти або виховання, яке, із свого боку, коштує більшої
або меншої суми товарових еквівалентів. Ця сума витрат виховання
змінюється залежно від більш чи менш складного характеру
\index{i}{0124}  %% посилання на сторінку оригінального видання
робочої сили. Отже, ці витрати навчання дуже малі для
звичайної робочої сили, є складова частина вартостей, витрачених
на її продукцію.

Вартість робочої сили сходить на вартість певної суми засобів
існування. Тому вона змінюється разом зі зміною вартости цих
засобів існування, тобто разом зі зміною величини робочого
часу, потрібного на продукцію їх.

Частину засобів існування, приміром, їжу, паливо тощо, споживається
день-у-день, і мусять вони день-у-день наново поповнюватись.
Інші засоби існування, як от одяг, меблі тощо,
зуживається протягом довшого часу, а тому їх треба поповнювати
протягом довшого часу. Товари одного роду мусять купуватися
або за них треба платити щодня, товари іншого роду —
щотижня, що чверть року й т. ін. Але, хоч і як поділялася б
сума цих видатків, приміром, протягом року, вона мусить покриватися
з пересічного доходу, що його день-у-день дістає робітник.
Коли б маса товарів, потрібних щоденно на продукцію
робочої сили, дорівнювала \emph{А}, маса товарів потрібних щотижня,
дорівнювала \emph{В}, маса товарів, потрібних що чверть року, дорівнювала
\emph{С} і т. д., то щоденна пересічна кількість цих товарів
дорівнювала б\[
\frac{365А + 52В + 4С + \text{і т. д.}}{365}
\]

\noindent
Коли припустимо, що в цій масі товарів, потрібній для пересічного
дня, міститься 6 годин суспільної праці, то в робочій силі
щодня упредметнюється півдня суспільної пересічної праці, тобто
потрібно пів робочого дня на щоденну продукцію робочої сили.
Ця кількість праці, потрібна на щоденну продукцію робочої
сили, становить її денну вартість, або вартість щоденно репродукованої
робочої сили. Коли півдня пересічної суспільної праці
виражається в масі золота в 3\shil{ шилінґи} або один таляр, то один
таляр є ціна, що відповідає денній вартості робочої сили. Якщо
посідач робочої сили щоденно продає її за один таляр, то її продажна
ціна дорівнює її вартості і, за нашим припущенням, посідач
грошей, що жадає перетворити свої таляри на капітал, платить
цю вартість.

Крайню або мінімальну межу вартости робочої сили становить
вартість маси тих товарів, що без її щоденного постачання носій
робочої сили, людина, не може відновляти свого життєвого
процесу, тобто вартість фізично доконечних засобів існування.
Коли ціна робочої сили падає до цього мінімуму, то вона падає
нижче від її вартости, бо в такому разі робоча сила може утримуватись
і розвиватись лише в занепалій формі. Але вартість кожного
товару визначається тим робочим часом, що його треба,
щоб постачати товар нормальної якости.

Було б надзвичайно дешевенькою сантиментальністю вважати
за грубе це визначення вартости робочої сили, яке випливає з
\parbreak{}  %% абзац продовжується на наступній сторінці
