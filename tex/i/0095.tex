договорах.\footnote{
Характер торговлі змінився в такому напрямі, що торговля від
обміну товарів на товари, від постачання й одержання, перейшла до продажу
й платежу; тепер усі операції... відбуваються як чисто грошові
операції». («The Course of Trade being thus turned, from exchanging of
goods for goods, or delivering and taking, to selling and paying, all the
bargains... are now stated upon the foot of a Price in Money»). («An Essay
upon Publick Credit», 3 rd ed. London 1710, p. 8).
} Ренти, податки й т. ін. з постачання в натурі перетворюються
на грошові платежі. Як дуже це перетворення
залежить від загальних умов процесу продукції, показує, приміром,
спроба Римської імперії стягати всі податки грішми —
спроба, яка двічі розбилася. Страшенні злидні французького
селянства за Люї XIV, злидні, що їх так красномовно ганьблять
Буаґільбер, маршал Вобан і інші, були викликані не лише висотою
податків, але й перетворенням натурального податку на
грошовий.\footnote{
«Гроші... стали загальним катом». Фінансова справа є «реторта, в
якій перетворюють на пару страшенну кількість добра й засобів існування,
щоб одержати цей фатальний осад». «Гроші проголошують війну всьому
людському родові» («L’argent... est devenu le bourreau de toutes les choses».
Die Finanzkunst ist das «alambic qui a fait évaporer une quantité
effroyable de biens et de denrées pour faire ce fatal précis». «L’argent declare
la guerre à tout le genre humain»). (Boisguillebert: «Dissertation sur la
nature des richesses, de l’argent et des tributs», ed. Daire, «Economistes
financiers», Pans 1843. vol. I, p. 413, 417, 419).
} Коли, з другого боку, натуральна форма земельної
ренти, яка в Азії є разом з тим головний елемент державних
податків, спирається там на продукційні відносини, що репродукуються
з незмінністю природних відносин, то ця форма платежу
через зворотний вплив підтримує стару форму продукції. Вона
становить одну з таємниць самозбереження турецької держави.
Коли закордонна торговля, накинута Японії Европою, потягне
за собою перетворення натуральної ренти на грошову, то це
станеться коштом загину зразкової рільничої культури Японії.
Вузькі умови економічного існування цієї культури зруйнуються.

У кожній країні встановлюється певні загальні терміни платежів.
Почасти вони, залишаючи осторонь інші циклічні пере-

Прибутки    Ф. ст.

Термінові векселі банкірів
і купців............ 533.596

Чеки на банкірів і інші
виплати пред’явникам. . 357.715

Банкноти провінціяльних
банків................ 9.627

Банкноти Англійського
банку................ 68.554

Золото................. 28.089

Срібло й мідь........... 1.486

Поштові перекази.......... 933

Разом фунтів стерлінґів 1.000.000

Видатки    Ф. ст.

Термінові векселі.... 302.674

Чеки на лондонських банкірів
............... 663.672

Банкноти Англійського
банку............... 22.743

Золото................. 9.427

Срібло й мідь.......... 1.484

Разом фунтів стерлінґів 1.000.000

(«Report from the Select Committee on the Bankacts. July 1858»,
p. LXXI).