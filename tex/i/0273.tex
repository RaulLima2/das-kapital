Цей випадковий розподіл повторюється, виявляє свої своєрідні
користі і поволі костеніє в систематичний поділ праці. З індивідуального
продукту самостійного ремісника, що виконує багато
операцій, товар перетворюється на суспільний продукт спілки
ремісників, що з них кожен постійно виконує лише одну й ту
саму частинну операцію. Ті самі операції, що зливалися одна
з одною як послідовні операції німецького цехового майстра
паперової продукції, усамостійнились у голляндській паперовій
мануфактурі в частинні операції багатьох кооперованих робітників,
операції, що відбуваються одна поруч одної. Нюрнберзький
цеховий голкар становить основний елемент англійської голкової
мануфактури. Але тимчасом, як той голкар виконував послідовно
одну по одній, може, 20 операцій, — тут, в англійській мануфактурі,
майже 20 голкарів працюють один поруч одного, виконуючи
кожний лише одну з тих 20 операцій, які в наслідок дальшого
досвіду ще більше порозділювались, відокремились та усамостійнились
у виключні функції поодиноких робітників.

Отже, спосіб виникнення мануфактури, її виростання із ремества,
є двоякий. З одного боку, вона походить із комбінації
різнорідних, самостійних реместв, які втрачають свою самостійність
та зоднобічнюються до такої міри, коли вони вже становлять
лише частинні операції в процесі продукції того самого товару —
операції, що лише одна одну доповнюють. З другого боку, мануфактура
походить із кооперації однорідних ремісників, розкладає
те саме індивідуальне ремество на його різні осібні операції,
ізолює та усамостійнює їх аж до такої міри, коли кожна з них
стає виключною функцією окремого робітника. Отже, мануфактура,
з одного боку, вводить у процес продукції поділ праці
або розвиває його далі, з другого боку — вона комбінує ремества,
що раніш були відокремлені. Але хоч який буде її окремий вихідний
пункт, її кінцева форма та сама — продукційний механізм,
що органами його є люди.

Щоб як слід зрозуміти поділ праці в мануфактурі, важно
завжди мати на увазі такі пункти: насамперед розклад процесу
продукції на його окремі фази тут геть чисто збігається з розкладом
ремісничої діяльности на її різні частинні операції. Хоч буде
ця операція складною, хоч простою, вона однаково лишається
ремісничою, а тому й залежною від сили, вправности, хуткости
та певности поодинокого робітника в орудуванні своїм інструментом.
Ремество лишається базою. Ця вузька технічна база
виключає дійсно науковий розклад процесу продукції, бо кожний
частинний процес, що його проробляє продукт, мусить бути
виконуваний як частинна реміснича праця. І саме тому, що реміснича
вправність лишається таким чином основою процесу продукції,
кожного робітника пристосовується виключно до якоїсь
однієї частинної функції і його робочу силу на цілий вік перетворюється
в орган цієї частинної функції. Нарешті, цей поділ праці
є осібний рід кооперації, а деякі її користі походять із загальної
природи кооперації, а не з цієї осібної форми її.

18 Капітал. Т. І.
