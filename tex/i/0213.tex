постанови із статутів, про які згадує Петті, мають вагу і для
«apprentices» (учнів). Але як саме стояла справа з дитячою працею
ще наприкінці XVII віку, видно з такого нарікання: «Наші
підлітки тут, в Англії, нічого не роблять аж до самого того часу,
коли вони стають учнями, а тоді, звичайно, потребують вони довгого
часу — сім років, — щоб стати досконалими ремісниками».
Навпаки, Німеччину хвалять, бо там дітей од колиски принаймні
«привчають хоч до якої-будь роботи».\footnote{
«A Discourse on the Necessity of Encouraging Mechanic Industry»,
London 1699» p. 13. Маколей, що зфалшував англійську історію в інтересах
віґів і буржуазії, деклямує: «Практика садовити передчасно
дітей за працю... панувала в XVII віці в майже неймовірних для тодішнього
стану промисловости розмірах. В Norwich’y, головному центрі
вовняної промисловости, шестилітню дитину вважали за працездатну.
Різні письменники тих часів і між ними деякі такі, що їх вважали за людей
з надзвичайно добрими намірами, згадують з «exultation» (із захопленням)
той факт, що в цьому місті праця самих хлопчаків і дівчат створює
багатство, яке понад їхнє власне утримання становило 12.000 фунтів
стерлінґів річно. Що докладніше ми досліджуємо історію минулого, то
більш находимо підстав, щоб відкинути погляди тих, хто вважає наш
вік багатим на нові соціяльні лиха... Що є нового, так це інтелігенція,
яка викриває це лихо, та гуманність, що гоїть це лихо». («History of
England», vol. I, p. 419). Маколей міг би далі розказати про те, що «amis
du commerce» XVII віку з «надзвичайно добрими намірами», з «exultation»
оповідають, як в одному домі для бідних у Голляндії примушували
працювати чотирилітню дитину, і що цей приклад «vertu mise en pratique»\footnote*{
— практичної чесноти. \emph{Ред.}
}
проходить y всіх творах гуманістів à la Маколей аж до часів
А. Сміса. Правда, разом з виникненням мануфактури, відмінно від ремества,
помічається сліди експлуатації дітей, яка до певної міри здавна
вже існувала в селян і була то розвиненіша, що важче було ярмо,
яке тяжіло над селянином. Тенденція капіталу ясна, але сами факти
мають ще такий поодинокий характер, як і поява на світ двоголових
дітей. Тим то повні передчуття «amis du commerce» з «exultation» змалювали
ці факти для сучасників і нащадків як щось варте уваги й подиву,
і рекомендували їх для наслідування. Той самий шотляндський сикофант
і красномовець-балакун Маколей каже: «Нині ми чуємо лише про реґрес,
а бачимо лише проґрес». ІЦо за очі, а особливо що за вуха!
}

Ще протягом найбільшої частини XVIII віку, аж до епохи
великої промисловости капіталові в Англії не вдалося виплатою
тижневої вартости робочої сили захопити цілий тиждень робітника;
однак рільничі робітники становлять виняток. Та обставина,
що робітники могли цілий тиждень жити на чотириденну
заробітну плату, не видавалась їм за достатню підставу для того,
щоб працювати на капіталіста й останні два дні. Англійські економісти
одного напряму, ті, що були на службі капіталу, якнайлютіше
нападали на робітників за таку впертість, а економісти
другого напряму боронили робітників. Послухаймо, приміром,
полеміку між Постлетвайтом, що його торговельний словник
мав тоді таку саму славу, яку нині мають аналогічні твори МакКуллоха
й Мак-Ґреґора, і цитованим вище автором «Essay
on Trade and Commerce».121

121 Найлютіший з усіх обвинувачів робітників є згаданий у тексті
анонімний автор «An Essay on Trade and Commerce, containing Observations
on Taxation etc.», London 1770. Вже давніш він виступив проти