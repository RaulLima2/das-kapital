\parcont{}  %% абзац починається на попередній сторінці
\index{i}{0025}  %% посилання на сторінку оригінального видання
центр ваги на якісний бік виразу вартости, отже, на еквівалентну
форму товару, що має свою закінчену форму в грошах; навпаки,
сучасні розносники ідей вільної торговлі, які за всяку ціну мусять
позбутися цього свого товару, переміщують центр ваги на
кількісний бік відносної форми вартости. Отже, для них не існує
ні вартости, ні величини вартости товару поза виразом їх через
мінове відношення, тобто вони для них існують тільки на сторінках
щоденного прайскуранта. Шотляндець Мак-Леод, що його
функцією було приоздоблювати якнайбільшою вченістю заплутані
уявлення банкірів із Льомбардстріту,\footnote*{
Вулиця в Лондоні, де містяться великі банки. \emph{Ред.}
} являє собою вдалу синтезу
забобонних меркантилістів і освічених комівояжерів вільної
торговлі.

Уважливий розгляд виразу вартости товару А, що міститься
в його вартостевому відношенні до товару В, показав нам, що
в межах цього відношення натуральна форма товару А фігурує
лише як форма споживної вартости, а натуральна форма товару
В — лише як форма вартости, або як вигляд вартости. Отже,
захована в товарі внутрішня протилежність між споживною вартістю
й вартістю виявляється через зовнішню протилежність, тобто
через відношення двох товарів, що в ньому один товар, саме той,
вартість якого повинна бути виражена, безпосередньо виступає
лише як споживна вартість, навпаки, другий товар, той, що в
ньому виражається вартість, безпосердньо виступає лише як
мінова вартість. Отже, проста форма вартости якогось товару
є проста форма виявлення протилежносте, що міститься в ньому,
протилежности між споживною вартістю й вартістю.

Продукт праці за всіх суспільних формацій (Zuständen) є
предмет споживання, але тільки одна історично визначена епоха
розвитку перетворює продукт праці на товар: це та епоха, яка
виражає працю, витрачену на продукцію предмету споживання,
як «предметну» властивість його, тобто як його вартість. З цього
випливає, що проста форма вартости товару є одночасно й проста
товарова форма продукту праці, що, отже, і розвиток товарової
форми збігається з розвитком форми вартости.

Вже на перший погляд виявляється недостатність простої
форми вартости, цієї зародкової форми, яка, лише проробивши
ряд метаморфоз, доходить до форми ціни.

Вираз вартости товару А в будь-якому товарі В відрізняє
вартість товару А лише від його власної споживної вартости й
тому ставить його у мінове відношення лише до якогось одного
товарового роду, відмінного від товару А, замість виразити його
якісну рівність з усіма іншими товарами й кількісну пропорційність
до них. Простій відносній формі вартости одного товару відповідає
одинична еквівалентна форма якогось іншого товару.
Так, сурдут у відносному виразі вартости полотна має еквівалентну
форму, або форму безпосередньої вимінности лише щодо
цього одного товарового роду «полотно».
