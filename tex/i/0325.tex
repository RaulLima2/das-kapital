особи, як незалежні посідачі товарів, один — як посідач грошей
та засобів продукції, другий — як посідач робочої сили. А тепер
капітал купує неповнолітніх або напівповнолітніх. Раніш
робітник продавав свою власну робочу силу, що нею він порядкував
як формально вільна особа. Тепер він продає жінку
й дітей. Він стає работорговцем.\footnote{
Протилежно до того великого факту, що обмеження праці жінок
та дітей по англійських фабриках одвоювали від капіталу дорослі робітники-чоловіки,
ми знаходимо ще в найновіших звітах «Children’s Employment
Commission» справді обурливі та цілком гідні работорговців риси
у батьків-робітників щодо баришування дітьми. Але капіталістичні
фарисеї, як це можна бачити з тих самих «Reports», ще й викривають цю,
ними самими утворену, увіковічнену та експлуатовану жорстокість, яку
вони в інших випадках називають «волею праці». «Дитячу працю покликано
на поміч... навіть для того, щоб діти заробляли собі свій щоденний
шматок хліба. Без сил, потрібних, щоб витримати таку надмірну
працю, без навчання, потрібного для спрямовання їхнього дальшого життя,
їх кинуто в таке становище, що руйнувало їх фізично й морально. Єврейський
історик зауважив з приводу зруйнування Єрусалиму Титом, що
немає нічого дивовижного в тому, що підчас руйнування Єрусалиму його
так страшенно сплюндрували, коли вже якась нелюдяна мати навіть пожертвувала
свого власного сина, щоб заспокоїти муки страшного голоду».
(«Infant labour has been called into aid... even to work for their own daily
bread. Without strength to endure such disproportionate toil, without
instruction to guide their future life, they have been thrown into a situation
physically and morally polluted. The Jewish historian has remarked upon
the ovethrow of Jerusalem by Titus, that is was no wonder it should have
been destroyed, with such a signal destruction, when an inhuman mother
sacrificed her own offspring to satisfy the cravings of absolute hunger»).
(«Public Economy Concentrated», Carlisle 1833, p. 66).
} Попит на дитячу працю
часто і своєю формою подібний до попиту на негрів-рабів,
про який дуже часто можна було читати в об’явах американських
газет. «Мою увагу, — каже, приміром, один англійський
фабричний інспектор, — звернула на себе об’ява в місцевій
газеті одного з найзначніших мануфактурних міст моєї округи.
Ось копія цієї об’яви: Потрібно 12—20 хлопчаків, не молодших
від такого віку, щоб їх можна було вважати за 13-літніх.
Плата — 4 шилінґи на тиждень. Спитати й т. д.».\footnote{
A. Redgrave у «Reports of Insp. of Fact, for 31 st October
1858», p. 40, 41.
} Речення:
«щоб їх можна було вважати за 13-літніх» пояснюється тим, що
за фабричним законом дітям, молодшим від 13 років, можна
працювати тільки 6 годин. Лікар, що має визнану урядом кваліфікацію
(certifying surgeon), мусить засвідчити вік. Отже,

наймати собі заступників. Роботи, яких потребує споживання родини,
наприклад, шиття, латання й т. д., доводиться заміняти купівлею готових
товарів. Отже, зменшенню витрати хатньої праці відповідає збільшення
грошових видатків. Тому витрати продукції робітничої родини зростають
та врівноважують збільшення доходу. До цього долучається ще й те, що
економія та доцільність у використовуванні й готуванні засобів існування
стають неможливі. Про ці факти, які офіціяльна політична економія
затаює, можна знайти багатий матеріял у «Reports» фабричних інспекторів,
у звітах «Children’s Employment Commission», а особливо в «Reports
on Public Health».