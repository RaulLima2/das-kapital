лише у власних пекарнях. Під кінець тижня... тобто в четвер,
праця починається тут о 10 годині вечора й триває з незначною
лише перервою до пізньої ночі під неділю».\footnote{
Там же, стор. LXXI.
80 George Read: «The History of Baking», London 1848, p. 16.
81 «First. Report etc.», Evidence. Свідчення «full priced baker’a» Чізмена,
p. 108.
82 George Read: «The History of Baking», London 1848. Наприкінці
XVII й початку XVIII віків посередників (аґентів), що протиснулись
у всі можливі галузі ремісництва, офіціально ганьбили, називаючи
їх «Public Nuisances».* Приміром, від «Grand Jury» ** підчас чвертьрічної
сесії мирових суддів у графстві Somerset подано до Палати громад «presentment»
,*** де, між іншим, сказано: «that these factors of Blackwell Hall
are a Public Nuisance and Prejudice to the Clothing Trade and ought to be
put down as a Nuisance». (Ці посередники Blackwell Hall є суспільне лихо
й перешкода торговлі одягом, і, як таку перешкоду, їх треба знищити»).
«The Case of our English Wool etc.», London 1685) p. 6, 7).
83 «First Report etc.», p. VIII.
* — суспільним лихом. Ред.
** Велике жюрі — суд присяжних. Ред.
*** — внесення. Ред.
}

Щождо «underselling masters», то й буржуазний погляд розуміє,
що «неоплачена праця підмайстрів (the unpaid labour of
the men) становить основу їхньої конкуренції».80 «Full priced
baker» (пекар, що продає за «повну ціну») виказує перед слідчою
комісією на своїх «underselling» -конкурентів як на розкрадачів
чужої праці й фалшівників. «Вони мають успіх лише
через те, що ошукують публіку, і через те, що видушують із
своїх підмайстрів 18 годин праці, оплачуючи лише 12 годин».81

Фальсифікація хліба й утворення кляси пекарів, що продають
хліб нижче від повної ціни, розвинулися в Англії на початку
XVIII сторіччя, відколи занепав цеховий характер ремества й
за спиною номінального майстра-пекаря виступив капіталіст 82
в образі мірошника або торговця борошном. Цим покладено основу
для капіталістичної продукції, для безмірного зловження робочого
дня та нічної праці, хоч остання навіть у Лондоні стала на тверді
ноги лише в 1824 р.83

Після всього попереднього зрозуміло, чому звіт комісії зачисляє
пекарських підмайстрів до тих робітників, які живуть
недовго; щасливо поминувши небезпеку стати жертвою жахливої
дитячої смертности, яка є нормальне явище для всіх категорій
робітничої кляси, вони рідко доживають до 42 року життя. А, проте,
пекарний промисел завжди переповнений кандидатами. Джерела,
що постачають для Лондону ці «робочі сили», є Шотляндія, західні
рільничі округи Англії і — Німеччина.

В 1858—1860 рр. пекарські підмайстри в Ірляндії зорганізували
власним коштом ряд великих мітинґів для аґітації проти
нічної й недільної праці. Публіка з суто ірляндським запалом
стала на їхній бік, як це було, приміром, 1860 р. на травневому
мітинґу в Дебліні. В наслідок цього руху було дійсно успішно
заведено виключну денну працю в Wexford’і, Kilkenny, Clonmel’i,