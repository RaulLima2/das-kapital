боку, конче потрібно, щоб вартість, відмінно від різноманітних
тіл товарового світу, розвинулася в цю безглузду речову, але
разом з тим просто суспільну форму.62

Ціна є грошова назва упредметненої в товарі праці. Еквівалентність
товару й тієї кількости грошей, що її назва є його ціна, є,
отже, тавтологія,63 як і взагалі відносний вираз вартости товару
завжди є вираз еквівалентности двох товарів. Але, коли ціна
як покажчик величини вартости товару є покажчик його мінового
відношення до грошей, то звідси не випливає зворотне,
а саме, що покажчик мінового відношення товару до грошей
неодмінно є покажчик величини товарової вартости. Припустімо,
що на продукцію 1 квартера пшениці витрачається таку саму
кількість суспільно доконечної праці, що і на 2 фунти стерлінґів
(близько 72 унції золота). Два фунти стерлінґів є тоді грошовий
вираз величини вартости квартера пшениці, або його ціна.
Але, коли обставини дозволять зазначити його ціну на 3 фунти
стерлінґів або примусять знизити її до 1 фунта стерлінґів, тоді
1 фунт стерлінґів є занадто малий, а 3 фунти стерлінґів занадто
великі як вирази величини вартости пшениці, а проте вони є
ціни пшениці, бо, поперше, вони є форма її вартости, гроші, а
подруге, — покажчики її мінового відношення до грошей. За
незмінних умов продукції або за незмінної продуктивної сили
праці на репродукцію 1 квартера пшениці мусить витрачатися
стільки ж суспільного робочого часу, як і раніш. Ця обставина
не залежить ані від волі продуцентів пшениці, ані від волі інших
посідачів товарів. Отже, величина вартости товару виражає доконечне,
іманентне самому процесові творення товару відношення
до суспільного робочого часу. З перетворенням величини вар-

62 Порівн. «Theorien von der Masseinheit des Geldes» в «Zur Kritik
der Politischen Oekonomie», S. 53. («Теорії про одиницю міри грошей»
в «До критики політичної економії». ДВУ, 1926 р., стор. 91). Фантазії
про підвищення або пониження «ціни монет», які сходять на те, що законні
грошові назви для фіксованих законом частин ваги золота та срібла
треба перенести державним актом на більші або дрібніші вагові частини
і відповідно до цього карбувати з \sfrac{1}{4} унції золота 40 шилінґів замість 20, —
ці фантазії, оскільки вони не є незграбними фінансовими операціями проти
державних і приватних кредиторів, а ставлять за мету досягти економічних
«цілющих ліків», вичерпно схарактеризував Детті в «Quantulumcumque
concerning Money. To the Lord Marquis of Halifax», 1682, так
що його безпосередні наступники, сер Дудлей Норт і Джон Льокк,
не кажучи вже про пізніших, могли лише вульгаризувати його
думки. «Коли б багатство народу, — каже він між іншим, — можна
було збільшити декретом вдесятеро, то дивно було б, чому такого декрета
давним-давно не видано нашими урядами» («If the wealth of a nation
could be decupled by a Proclamation, it were strange that such proclamations
have not long since been made by our Governors», стор. 3 його щойно згаданого
твору).

63 «Або доведеться визнати, що вартість мільйона в грошах більша,
ніж така сама вартість у товарах» («Ou bien, il faut consentir à dire qu’une
valeur d’un million en argent vaut plus qu’une valeur égale en marchandises»)
(Le Trosne: «De l’Intérêt Social», p. 992), отже, «що дана вартість варта
більше, ніж рівна їй інша вартість» («qu’une valeur vaut plus qu’une valeur
égale»).
