рає супроти нього форми засобів платежу, авансованих йому
від третьої особи за його працю. Зате і його неоплачена примусова
праця ніколи не набирає форми добровільної та оплаченої
праці. Коли завтра поміщик присвоїть собі поле, робочу худобу,
насіння, коротко — засоби продукції селянина-кріпака,
то цей останній відтепер муситиме продавати свою робочу силу
сеньйорові. За інших незмінних обставин він, як і раніш, працюватиме
6 днів на тиждень: 3 дні на себе, 3 дні на колишнього
сеньйора, що тепер перетворився на пана-наймача робітників.
Як і раніш, він споживатиме засоби продукції як засоби продукції
і переноситиме їхню вартість на продукт. Як і раніш, певна
частина продукту ввіходитиме в репродукцію. Але так само як
панщанна праця набирає форми найманої праці, так само й робочий
фонд, що його, як і раніш, продукує й репродукує селянин-кріпак,
набирає форми капіталу, авансованого селянинові від
колишнього сеньйора. Буржуазний економіст, що його обмежений
мозок не може відокремити форму виявлення від того, що
в ній виявляється, заплющує очі перед тим фактом, що навіть
ще й тепер на земній кулі робочий фонд лише винятково виступає
у формі капіталу.\footnote{
«Засоби існування робітників авансуються капіталістами робітникам
навіть менше, ніж на одній четвертині земної кулі». (Richard
Jones: «Textbook of Lectures on the Political Economy of Nations», Hertford
1852, p. 16).
}

Правда, змінний капітал втрачає характер вартости, авансованої
із власного фонду капіталіста,4а лише тоді, коли ми розглядатимемо
процес капіталістичної продукції в безперервній течії
його відновлення. Однак цей процес мусить десь і колись початися.
Тому, з того погляду, що його ми досі трималися, річ
імовірна, що капіталіст якогось часу за допомогою якоїсь первісної
акумуляції, незалежної від чужої неоплаченоі праці,
став посідачем грошей і тому міг виступити на ринку як
покупець робочої сили.\footnote*{
У французькому виданні замість останніх двох речень читаємо
таке: «Однак, раніше ніж відновитися, цей процес мусив був початися
й тривати певний відтинок часу, протягом якого робітник не міг ще бути
оплачений з його власного продукту, ані жити з повітря. Отже, чи не слід
було б нам припустити, що капіталістична кляса, з’явившись уперше
на ринку праці, вже нагромадила своєю власною працею та власними
заощадженнями скарби, що дали їй змогу авансувати робітникам засоби
існування у формі грошей. Погодьмось поки що на таке рішення цієї
проблеми, яку ми докладніше розглянемо в розділі про так звану первісну
акумуляцію». («Le Capital etc.», v. I, ch. XXIII, p. 248—249). Ред.
} А втім проста безперервність капіталі-

4a «Хоч мануфактурний робітник дістає свою заробітну плату як
аванс від хазяїна, проте фактично це не коштує хазяїнові ніяких витрат,
бо сума цієї заробітної плати звичайно повертається назад разом із зиском
у підвищеній вартості речі, на яку вжито цю працю» («Though the
manufacturer has his wages advanced to him by his master, he in reality
costs him no expense, the value of these wages being generally restored,
together with a profit, in the improved value of the subject ліроп
which his labour is bestowed»). (A. Smith: «Wealth of Nations», b. II,
ch. 3, p. 355).