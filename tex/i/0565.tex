Обидві статі                                                        Пересічна                       
       Пересічна
                                                                                тижнева             
                    тижнева
                                                                                кількість вуглецю   
            кількість азоту
                                                                                (ґранів)            
                      (ґранів)

П’ять міських галузей промисловости..     28.876                                      1.192
Безробітні фабричні робітники
Ланкаширу...........................................     28.211                                     
1.295
Мінімальна кількість, запропонована
для ланкашірських робітників при
рівному числі чоловіків і жінок............      28.600                                      1.330
112

Половина, 60/125, із досліджених категорій промислових робітників
зовсім не споживала пива, 28\% — молока. Пересічна
тижнева кількість рідких поживних речовин коливалася від
7 унцій на родину в швачок до 24 унцій у панчішників. Більшість
тих, що не споживали молока, складалася з лондонських
швачок. Кількість споживаного на тиждень хліба коливалася
від 7 3/4 фунтів у швачок до 11 1/4 фунтів у шевців і становила пересічно
9,9 фунта на тиждень на дорослого. Кількість цукру (сиропу
й т. ін.), коливалася від 4 унцій на тиждень у виробників
шкуряних рукавичок до 11 унцій у панчішників; ціла пересічна
кількість на тиждень для всіх категорій — 8 унцій на одного
дорослого. Загальна пересічна кількість масла (жиру й т. ін.)
на тиждень — 5 унцій на дорослого. Пересічна тижнева кількість
м’яса (сала й т. ін.) на дорослого коливалася від 7 1/4 унцій у шовкоткачів
до 18 1/4 унцій у виробників шкуряних рукавичок;
загальна пересічна кількість для різних категорій — 13,6 унцій.
Щотижнева витрата на харчі для дорослих становила такі загальні
пересічні числа: шовкоткачі — 2 шилінґи 2 1/2 пенса,
швачки — 2 шилінґи 7 пенсів, виробники шкуряних рукавичок
— 2 шилінґи 9 1/2 пенсів, шевці — 2 шилінґи 7 3/4 пенсів,
панчішники — 2 шилінґи 6 1/4 пенсів. Для шовкоткачів з Macclesfield’у
тижнева пересічна кількість становила лише 1 шилінґ
8 1/2 пенсів. Найгірше харчувалися швачки, шовкоткачі й виробники
шкуряних рукавичок.113

У своєму загальному санітарному звіті д-р Сімон каже про
цей стан харчування таке: «Кожний, хто обізнаний з медичною
практикою серед бідних або з пацієнтами шпиталів, однаково,
чи живуть вони по шпиталях, чи поза ними, потвердить, що випадки,
коли недостача харчів породжує або загострює недуги,
дуже численні... Однак із санітарного погляду сюди долучається
ще інша, дуже важлива обставина. Треба пригадати собі, що
позбавлення харчових засобів терпиться лише з великим опором
і що, звичайно, дуже недостатнє харчування є лише наслідок

112 Там же, додаток, стор. 232.

113 Там же, стор. 232. 233.
