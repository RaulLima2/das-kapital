акр, не зважаючи на те, що він залічив у цю територію і полону
ширини Темзи. Само собою зрозуміло, що всякі санітар-пополіційні
заходи, які, як це досі робилося в Лондоні, через
зламання негодящих домів виганяють робітників з одного кварталу,
служать лише для того, щоб їх щільніше напаковувати
л іншому. «Або, — каже д-р Гентер, — цілу цю процедуру,
як безглузду, треба цілком припинити, або мусить пробудитися
громадське співчуття (!) до того, що тепер, не перебільшуючи,
можна назвати національним обов’язком, а саме до того, щоб
дати притулок людям, які через брак капіталу не можуть його
сами собі придбати, але можуть дати відшкодування власникам
квартир періодичними виплатами».122 Дивовижна річ ця капіталістична
справедливість! Землевласник, домовласник, комерсант,
коли в нього що експропріюють задля «поліпшень», як от
будова залізниць, будова нових вулиць і т. д., не тільки дістає
повне відшкодування. За своє вимушене «зречення» він, крім
того, мусить за божими й людськими законами мати як нагороду
ще й не абиякий зиск. А робітників з дружинами, дітьми й пожитками
викидають на брук, і якщо вони занадто великими масами
ринуть до міських кварталів, в яких муніципалітет особливо
стежить за добропристойністю, то їх переслідує санітарна поліція!

На початку XIX віку в Англії, окрім Лондону, не було жодного
міста, що налічувало б 100.000 мешканців. Тільки п’ять
мало понад 50.000. Тепер існує 28 міст, що мають понад 50.000 мешканців.
«Результатом цієї зміни був не тільки величезний приріст
міської людности, але й те, що старі битком набиті дрібні
міста тепер стали центрами, з усіх боків позабудованими, без
якогобудь вільного допливу свіжого повітря. Через те, що ці
міста стали вже неприємними для багатих, вони їх залишають,
оселюючись на веселіших передмістях. Наступники цих багатіїв
оселюються у великих домах, при чому кожна родина, часто ще
з квартирантами, дістає по одній кімнаті. Таким чином людність
втискується в доми, не для неї призначені, до її потреб зовсім
непристосовані, в оточенні, що справді понижує дорослих і руйнує
дітей».123 Що швидше в якомусь промисловому або торговельному
місті акумулюється капітал, то швидше припливає
приступний для експлуатації людський матеріял, то злиденніші
імпровізовані житла робітників. Тим то Ньюкестл над Тайном,
як центр кам’яновугільної й гірничої округи, що чимраз дужче
розвивається, посідає після Лондону друге місце в житловому
пеклі. Не менше, як 34.000 осіб живе там по окремих комірках.
У наслідок абсолютної небезпеки для громадського здоров’я
з наказу поліції в Ньюкестлі і Ґетшеді порозвалювано недавно
чимало домів. Будування нових домів посувається дуже повільно,
а промисловість розвивається дуже швидко. Тому 1865 р.
місто було переповнене більше ніж будь-коли раніш. Ледве
чи можна було найняти хоч одну комірку. Д-р Імблтон із шпи-
