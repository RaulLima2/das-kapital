\parcont{}  %% абзац починається на попередній сторінці
\index{i}{0071}  %% посилання на сторінку оригінального видання
на боці покупця як засіб купівлі. Вони функціонують як засіб
купівлі, реалізуючи ціну товарів. Але, реалізуючи цю останню,
гроші переносять товар із рук продавця до рук покупця, тимчасом
як сами вони одночасно віддаляються від рук покупця до
рук продавця, щоб знову повторити той самий процес з якимось
іншим товаром. Що ця однобічна форма руху грошей постає з
двобічної форми руху товарів, — ця обставина є прихована. Сама
природа товарової циркуляції утворює протилежну видимість.
Перша метаморфоза товару видима не лише як рух грошей, але
і як власний рух самого товару; але друга метаморфоза товару
видима лише як рух грошей. У першій половині своєї циркуляції
товар міняється місцем із грішми. Разом із тим його споживне
тіло випадає із циркуляції та входить у споживання.\footnote{
Навіть коли товар знов і знов продається, — явище, яке тут для нас
покищо не існує, то з остаточним продажем він переходить із сфери циркуляції
у сферу споживання, щоб служити тут засобом існування, або
засобом продукції.
} Його
місце заступає форма вартости або грошова лялечка. Другу
половину циркуляції товар перебігає вже не в своїй власній
природній шкурі, а в своїй золотій шкурі. Таким чином безперервність
руху цілком припадає лише грошам, і той самий рух,
який щодо товару включає два протилежні процеси, включає як
власний рух грошей завжди той самий процес, а саме, що гроші
завжди міняються місцями з іншими товарами. Тому результат
циркуляції товарів, обмін одного товару на інший товар, з’являється
упосереднюваним не через зміну його власної форми, а
через функцію грошей як засобу циркуляції, які упосереднюють
циркуляцію товарів, що сами по собі непорушні, переносять їх
із рук, де вони є неспоживні вартості, до рук, де вони є споживні
вартості, і при тому завжди в напрямі, протилежному до свого
власного руху. Вони постійно віддаляють товари із сфери циркуляції,
постійно заступаючи їхнє місце й тим самим віддаляючись
від свого власного вихідного пункту. Тому, хоч рух грошей є
лише вираз руху товарів, циркуляція товарів, навпаки, видається
лише результатом руху грошей.\footnote{
«Вони (гроші) не мають іншого руху, а тільки той, якого їм надають
продукти» («Il (l’argent) n’a d’autre mouvement que celui qui luï
est imprimé par les productions»), (Le Trosne: «De l’Intérêt Social»,
Physiocrates, éd. Daire. Paris 1846, p. 885).
}

З другого боку, гроші мають функцію засобу циркуляції
лише тому, що вони є усамостійнена вартість товарів. Тому їхній
рух як засобу циркуляції в дійсності є лише рух власної форми
товарів. Отже, цей рух мусить почуттєво відбиватися в обігу
грошей. Так, наприклад, полотно перетворює спочатку свою
товарову форму на свою грошову форму. Останній полюс його
першої метаморфози $Т — Г$, грошова форма, стає потім першим
полюсом його останньої метаморфози $Г — Т$, його зворотного
перетворення на біблію. Але кожна з цих обох змін форми відбувається
через обмін між товаром і грішми, в наслідок їхньої
взаємної переміни місць. Ті самі монети приходять до рук продавця
\index{i}{0072}  %% посилання на сторінку оригінального видання
як преобраяїена (entäusserte) форма товару і покидають їх
як абсолютно відчужувана форма товару. Вони двічі змінюють
місце. Перша метаморфоза полотна приносить ці монети до кишені
ткача, друга витягає їх знову звідти. Отже, обидві протилежні
зміни форми того самого товару відбиваються знов у дворазовій
зміні місця грошей у протилежному напрямі.

Навпаки, коли відбуваються лише однобічні товарові метаморфози,
однаково, самі купівлі або самі продажі, тоді ті самі
гроші перемінюють місце також лише один раз. Друга їхня
переміна місця виражає завжди лише другу метаморфозу товару,
його зворотне перетворення з грошей. В частому повторюванні
зміни місць тих самих монет відбивається не лише ряд метаморфоз
одним-одного товару, але й взаємне посплітування численних
метаморфоз товарового світу взагалі. А проте само собою зрозуміло,
що все це має силу лише для розглядуваної тут форми
простої товарової циркуляції.

Кожний товар, при першому своєму кроці у процесі циркуляції,
при першій зміні своєї форми, випадає з циркуляції, в яку завжди
вступає новий товар. Навпаки, гроші як засіб циркуляції постійно
перебувають у сфері циркуляції й постійно обертаються
в ній. Отже, постає питання, скільки грошей може постійно поглинути
ця сфера.

В даній країні щодня відбуваються одночасно, отже, просторово
одна побіч однієї, численні однобічні товарові метаморфози,
або, іншими словами, лише продажі з одного боку й лише купівлі
з другого. У своїх цінах товари вже прирівняно певним уявлюваним
кількостям грошей. А що розглядувана тут безпосередня
форма циркуляції завжди тілесно протиставить одно одному
товар і гроші, — перший на полюсі продажу, другі — на протилежному
полюсі купівлі, — то масу засобів циркуляції, потрібну
для процесу циркуляції всіх товарів, вже визначено сумою цін
товарів. Справді, гроші лише реально являють собою ту суму
золота, яку ідеально вже виражено в сумі цін товарів. Отже,
рівність цих сум є очевидна сама собою. Ми знаємо, однак, що
за незмінної вартости товарів ціни їхні змінюються разом з
зміною вартости самого золота (грошового матеріялу): підвищуються
пропорційно до зменшення його вартости і зменшуються
пропорційно до її підвищення. Коли сума цін товарів,
таким чином підвищується або знижується, то й маса грошей,
що циркулюють, мусить пропорційно збільшуватись або зменшуватись.
Правда, зміна маси засобів циркуляції залежить тут від
самих грошей, але не від їхньої функції як засобу циркуляції,
а від їхньої функції як міри вартости. Ціна товарів спочатку
змінюється зворотно пропорційно до зміни вартости грошей,
і потім маса засобів циркуляції змінюється прямо пропорційно
до змін цін товарів. Цілком таке саме явище відбувалося б, коли б,
приміром, не вартість золота понизилась, а срібло заступило
його як міра вартости, або коли б не вартість срібла підвищилась,
а золото витиснуло срібло з функції міри вартости. В першому
\parbreak{}  %% абзац продовжується на наступній сторінці
