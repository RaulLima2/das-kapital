дня й інтенсифікація праці виключають одне одного, так що
здовження робочого дня можна узгодити лише з пониженням
ступеня інтенсивности праці і, навпаки, підвищення ступеня інтенсивности
— лише із скороченням робочого дня. Скоро тільки
обурення робітничої кляси, що поступінно зростало, примусило
державу силоміць скоротити робочий час та насамперед фабриці
у власному значенні подиктувати нормальний робочий день,
отже, від того моменту, коли раз назавжди поклали край збільшенню
продукції додаткової вартости через здовження робочого
дня, капітал з усієї сили та з повною свідомістю кинувся до продукції
відносної додаткової вартости за допомогою прискореного
розвитку машинової системи. Одночасно настає зміна в характері
відносної додаткової вартости. Взагалі метода продукції відносної
додаткової вартости є в тому, щоб через збільшення продуктивної
сили праці зробити робітника здатним за тієї самої витрати праці
і протягом того самого часу більше продукувати. Той самий
робочий час, як і раніш, додає до загального продукту ту саму
вартість, хоч ця незмінена мінова вартість виражається тепер у
більшій кількості споживних вартостей, наслідком чого вартість
поодинокого товару падає. Та інша справа, коли насильне скорочення
робочого дня разом із величезним поштовхом, який воно
дає розвиткові продуктивної сили й економізації умов продукції,
примушує робітника витрачати за той самий час більше праці,
напружувати більше робочу силу, щільніше заповнювати пори
робочого часу, тобто конденсувати працю до такого ступеня, якого
можна досягти лише в межах скороченого робочого дня. Цю
згущену в даний період часу більшу масу праці вважається тепер
за те, чим вона є, — за більшу кількість праці. Поряд із мірою
робочого часу як «екстенсивної величини» виступає тепер міра
ступеня його згущення.\footnote{
Певна річ, по різних галузях продукції взагалі бувають ріжниці
щодо інтенсивности прані. Ці ріжниці, як це вже показав А. Сміс, компенсуються
почасти побічними обставинами, властивими кожному родові
праці. Але робочий час як міра вартости зазнає і тут впливу лише остільки,
оскільки інтенсивні та екстенсивні величини являють собою протилежні
вирази тієї самої кількости праці, вирази, що один одного виключають.
} Інтенсивніша година десятигодинного
робочого дня містить у собі тепер стільки ж або більше праці,
тобто витраченої робочої сили, ніж пористіша година дванадцятигодинного
робочого дня. Тому її продукт має таку саму або більшу
вартість, ніж продукт пористіших 1 1/5 годин. Не кажучи вже про
збільшення відносної додаткової вартости через підвищення продуктивної
сили праці, тепер, приміром, З 1/3 години додаткової
праці на 6 2/3 години доконечної праці дають капіталістові таку
саму масу вартости, як раніш 4 години додаткової праці на 8 годин
доконечної праці.

Тепер спитаємо, яким чином інтенсифікується працю?

Перший наслідок скороченого робочого дня ґрунтується на тому
цілком очевидному законі, що працездатність робочої сили стоїть