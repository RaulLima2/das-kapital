\parcont{}  %% абзац починається на попередній сторінці
\index{i}{0307}  %% посилання на сторінку оригінального видання
часто вживали коней, як про це, окрім лементу тогочасних аґрономів,
свідчить уже й вираз механічної сили в кінських силах —
вираз, що зберігся до наших часів. Вітер був надто непостійний
і не піддавався контролеві; крім того, в Англії, цій батьківщині
великої промисловости, вживали переважно водяної сили вже
протягом мануфактурного періоду. Вже в XVII столітті були
спроби одним водяним колесом пустити в рух двоє жорен, отже, і
два млинові кола. Але зріст розміру передатного механізму дійшов
тепер конфлікту з недостатньою силою води, і це є одна з обставин,
що спонукала докладніше досліджувати закони тертя. Так
само неоднорідне діяння рушійної сили у млинах, що їх пускали
в рух за допомогою посування та тягнення коромисел, привело
до теорії та вживання махового колеса\footnote{
Faulhaber 1625, De Caus 1688.
}, що пізніше відіграє
таку важливу ролю у великій індустрії. Таким способом мануфактурний
період розвивав перші наукові й технічні елементи
великої промисловости. Тростільну прядільню Аркрайта від
самого початку гнала вода. Алеж і вживання сили води як
головної рушійної сили було сполучене з труднощами. Її не можна
було підняти до бажаної височини й не можна було зарадити її
бракові, деколи вона відмовлялась служити, а передусім вона
була суто льокальної природи\footnote{
Новітній винахід турбін звільняє промислову експлуатацію
водяної сили від багатьох колишніх обмежень.
}. Лише з винаходом другої Ваттової
машини, так званої парової машини двійного чину, знайдено
перший мотор, що, споживаючи вугілля й воду, сам витворює
свою рушійну силу; мотор, що його сила стоїть цілком під контролем
людини; мотор пересувний і засіб до пересовування; мотор
міський, а не, як водяне колесо, сільський; мотор, що дозволяє
концентрацію продукції по містах, замість, як водяне колесо,
розкидувати її по селах;\footnote{
«В перші часи існування текстильних мануфактур місце заснування
мануфактури залежало від існування потока з остільки достатнім
спадом, щоб він міг обертати водяне колесо; і хоч будування водяних
фабрик було першим ударом для системи домашньої мануфактури, однак
ці фабрики, що їх з доконечности будували над потоками та часто-густо
в значній віддалі одну від однієї, становили скорше частину сільської, аніж
міської системи; лише після введення сили пари замість сили води фабрики
можна було зосереджувати по містах і по тих місцевостях, де було досить
вугілля та води, потрібних для продукції пари. Парова машина — мати
мануфактурних міст». («In the early days of textile manufactures, the locality
of the factory depended upon the existence of a stream having a sufficient
fall to turn a water wheel; and, although the establishment of the water
mills was the commencement of the breaking up of the domestic system of
manufacture, yet the mills necessarily situated upon streams, and frequently
at considerable distances the one from the other, formed part of a rural rather
than an urban system; and it was not until the introduction of the steampower
as a substitute for the stream, that factories were congregated in towns and
localities where the coal and water required for the production of steam were
found in sufficient quantities. The steam-engine is the parent of manufacturing
towns»). (\emph{A.~Redgrave} у «Reports of Insp. of Fact, for 30 th
April 1866», p. 36).
} мотор універсальний у своєму технологічному
\index{i}{0308}  %% посилання на сторінку оригінального видання
застосуванні й порівняно мало залежний від локальних
умов щодо свого місця застосування. Великий геній Ватта
виявляється у специфікації патенту, який він узяв у квітні 1784~\abbr{р.},
специфікації, що в ній його парову машину описано не як якийсь
винахід для окремих завдань, але як універсальний чинник
великої промисловости. Він натякає тут на такі способи її застосовування,
що з них деякі, як от, приміром, паровий молот,
заведено в життя лише більше ніж півстоліття пізніше. Однак
він сумнівався в тому, чи можна буде застосувати парову машину
до мореплавства. Його наступники, Болтон та Ватт, виставили
1851~\abbr{р.} на лондонській промисловій виставці колосальнішу парову
машину для океанських пароплавів.

Лише після того, як знаряддя перетворились із знарядь людського
організму на знаряддя механічного апарату, виконавчої
машини, тільки тоді й рухова машина набула самостійної форми,
цілком емансипованої від меж людської сили. Разом з цим та
поодинока виконавча машина, яку ми досі розглядали, зводиться
на простий елемент машинової продукції. Тепер одна рухова
машина може одночасно рухати багато робочих машин. Зі збільшенням
числа одночасно пущених у рух робочих машин зростає
й рухова машина, а передатний механізм розвивається в широченний
і складний апарат.

Тут треба розрізняти дві форми: кооперацію багатьох однорідних
машин і систему машин.

В одному випадку цілий продукт виробляє та сама робоча
машина. Вона виконує всі ті різні операції, що їх виконував своїм
знаряддям ремісник, наприклад, ткач своїм ткацьким варстатом,
або ті, що їх послідовно виконували ремісники за допомогою
різних знарядь, однаково, чи були вони самостійні ремісники,
чи члени якоїсь мануфактури\footnote{
З погляду мануфактурного поділу праці ткацтво було зовсім не
проста, а скорше складна реміснича праця, і тому механічний ткацький
варстат є машина, що виконує дуже різноманітні операції. Взагалі неправильно
думати, ніби сучасні машини первісно опанували такі операції,
які мануфактурний поділ праці вже спростив. Прядіння й ткання за мануфактурного
періоду відокремлено одне від одного на нові роди, їхнє знаряддя
поліпшено та урізноріднено, але самого процесу праці ані скільки
не поділено, він лишався ремісничий. За вихідну точку для машини є не
праця, а засіб праці.
}. Приміром, у сучасній мануфактурі
поштових конвертів один робітник за допомогою фальцу
фальцював папір, другий накладав клей, третій одгинав кляпку,
на якій друкується девізу, четвертий витискував девізу й~\abbr{т. ін.},
і при кожній з цих частинних операцій кожний окремий конверт
мусив переходити з рук до рук. Одним-одна машина виготовляти
конверти виконує всі ці операції відразу й виготовляє \num{3.000} й
більше поштових конвертів за одну годину. Одна американська
машина виготовляти паперові мішечки, виставлена па лондонській
промисловій виставці 1862~\abbr{р.}, ріже папір, намазує клей,
фальцює й виготовляє 300 штук за хвилину. Цілий процес, що в
мануфактурі є поділений і виконується послідовно, тут виконує
\parbreak{}  %% абзац продовжується на наступній сторінці
