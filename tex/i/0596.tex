голодні зимою, робітники кажуть на своєму власному діалекті,
що «the parson and gentlefolks seem frit to death at them».\footnoteA{
«Піп і шляхтич, здається, заприсяглися замордувати їх на
смерть».
}

У Floore є приклади, що в спальні найменшого розміру живе
подружжя з 4, 5, 6 дітьми, або 3 дорослих з 5 дітьми, або подружжя
з дідом і 6 дітьми, хорими на скарлятину, і т. ін.; у двох
хатах з двома спальнями — 2 родини, кожна складається з 8
і 9 дорослих.

11. Wiltshire.

Stratton: Досліджено 31 хату, 8 з них мають лише одну
спальню. Pentill у тій самій парафії. Один cot винаймають за
1 шилінґ 3 пенси на тиждень; в ньому живе 4 дорослих і 4 дітей;
крім добрих стін, у ньому немає нічого доброго, починаючи від
долівки з погано обтесаного каменю й кінчаючи зігнилою солом’яною
стріхою.

12. Worcestershire.

Тут хати не так жорстоко поруйновано; однак від 1851 р. до
1861 р. число мешканців на хату збільшилося з 4,2 до 4,6.

Badsey: Тут багато котеджів і садочків. Декотрі фармери
заявляють, що cots є «а great nuisance here, because they bring
the poor» («cots — велике лихо, бо принаджують бідноту»).
Один джентлмен каже: «Бідним від цього зовсім не краще; коли
збудувати 500 cots, їх розхоплять, наче булочки; справді, що
більше їх будують, то більше їх потрібно», отже, на його погляд,
хати покликають до життя мешканців, а мешканці, ясна річ,
натискують на «засоби мешкання». — З приводу цього вислову
д-р Гентер зауважує: «Але ж ці бідняки мусять звідкілясь
приходити, а що в Badsey немає особливої приваби, як от милостиня,
то, певно, мусить існувати відштовхування їх від якогось
ще невигіднішого місця, що й жене їх сюди. Коли б кожний міг
знайти недалечко від місця своєї праці cot і клаптик землі, то
напевне віддав би йому перевагу над Badsey, де йому за свій
клаптик землі доводиться платити удвоє дорожче, ніж фармерові
за свій».

Постійна еміграція до міст, постійне «створення перелюднення»
на селі через концентрацію фарм, перетворення нив на пасовиська,
застосування машин і т. ін., ідуть пліч-о-пліч з постійним
виганянням сільської людности через руйнування котеджів.
Що рідше заселена округа, то більше її «відносне перелюднення»,
то більший тиск останнього на засоби заробітку, то більший
абсолютний надмір сільської людности проти її житлових
засобів, отже, то більші по селах місцеве перелюднення й скупченість
людей з її наслідками — пошесними недугами. Скупченість
мас людей у порозкиданих дрібних селах і торгових містечках
відповідає ґвалтовному спустошенню людности на поверхні