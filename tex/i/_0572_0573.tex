\parcont{}  %% абзац починається на попередній сторінці
\index{i}{0572}  %% посилання на сторінку оригінального видання
от будування залізниць тощо, підприємець сам здебільшого постачає
своїй армії дерев’яні курені й~\abbr{т. ін.}, імпровізовані селища
що не мають ніяких гігієнічних засобів, не підлягають ніякому
контролеві місцевої влади, але дуже вигідні для пана підприємця,
що подвійно визискує робітників — як промислових солдатів
і як квартирантів. Залежно від того, скільки дір має курінь —
одну, дві чи три, мешканцеві, тобто копальникові й~\abbr{т. ін.}, доводиться
платити на тиждень 1, 3, 4, шилінґи.\footnote{
Там же, стор. 165.
} Досить буде одного
прикладу. У вересні 1864~\abbr{р.}, — повідомляє д-р Сімон, — міністер
внутрішніх справ сер Джордж Ґрей одержав такого звіта
від голови Nuisance Removal Committee\footnote*{
Комітет у справі боротьби в антисанітарними умовами. \emph{Ред.}
} в парафії Sevenoaks:
«Іще 12 місяців тому віспа в цій парафії була цілком невідома.
Незадовго перед цим почалися роботи коло будови залізниці
від Lewisham до Tunbridge. Опріч того, що головні роботи провадились
у безпосередньому сусідстві з цим містом, тут ще й
улаштовано головне депо цілого підприємства. Тому тут працювало
багато робітників. Через те, що неможливо було помістити
їх усіх у котеджах, то підприємець, пан Джей, побудував уздовж
залізничної колії на різних пунктах курені для житла робітників.
Ці курені не мали жодної вентиляції, ані зливів на нечисть;
крім того, вони з доконечности були надмірно переповнені,
бо кожний наймач мусив приймати інших мешканців, хоч би
яка численна була його власна родина, і хоч у кожному курені
було лише дві кімнати. За лікарським звітом, що ми його одержали,
наслідок був той, що ці бідолахи мусили ночами зносити
всі муки задухи, щоб захистити себе від заразних випарів з брудних
калюж і кльозетів, що були зараз же під вікнами. Нарешті,
один лікар, що мав нагоду відвідати ці курені, передав нашому
комітетові скаргу. В якнайгіркіших висловах говорив він про
стан цих так званих помешкань і побоювався дуже серйозних
наслідків, якщо не вживеться деяких санітарних заходів. Приблизно
рік перед тим згаданий Джей зобов’язався збудувати дім,
куди негайно мали ізолювати занятих у нього робітників, скоро
вони захоріють на заразливі недуги. Наприкінці останнього
липня він повторив цю обіцянку, але не зробив найменшого
кроку, щоб виконати її, дарма що від того часу трапилось кілька
випадків віспи і двоє чоловіка від неї померло. 9 вересня лікар
Келсон повідомив мене про нові випадки віспи в цих куренях,
змальовуючи їхній стан як жахний. Для вашої (міністра) інформації
мушу я додати, що в нашій парафії є ізольований дім, так
званий пошесний дім, де ходять за парафіянами, недужими на
заразливі хороби. Цей дім ось уже кілька місяців постійно переповнений
пацієнтами. В одній родині померло п’ятеро дітей від
віспи і пропасниці. Від 1 квітня до 1 вересня цього року трапилось
не менш як 10 смертних випадків од віспи, 4 з них у згаданих
куренях, у цих джерелах пошестей. Подати число занедужань
\index{i}{0573}  %% посилання на сторінку оригінального видання
неможливо, бо родини, де вони трапляються, ховають їх
у якнайбільшій тайні».\footnote{
Там же, стор. 18, примітка. Опікун бідних у Chapei-en-le-Frith-Union
повідомляє генерального реєстратора: «В Doveholes у великому
горбі вапняного попелу пороблено багато печер. Ці печери служать
за житла для копальників та інших робітників, занятих коло будування
залізниць. Печери тісні, вогкі, без зливів на нечисть і без кльозетів.
У них немає жодного вентиляційного приладу, за винятком відтулини у
склепінні, і ця відтулина одночасно служить і за димар. Віспа лютує і
вже спричинила декілька смертних випадків (серед троглодитів). (Там же,
примітка 2).
}

Робітники на вугільних і інших шахтах належать до найкраще
оплачуваних категорій британського пролетаріату. Якою ціною
вони купують свою заробітну плату, це показано вже раніш.\footnote{
Подробиці, наведені на стор. 415 і дальших, стосуються саме до
робітників у кам'яновугільних копальнях. Про ще гірший стан у руднях
порівн. сумлінний звіт Royal Commission з року 1864.
}
Я кину тут оком на їхні житлові умови. Експлуататор шахт,
хоч буде це власник їх, хоч наймач їх, звичайно будує певне
число котеджів для своїх рук. Вони дістають котеджі й вугілля
на опал «даремно», тобто це становить частину заробітної плати,
що її видається in natura. Хто не дістає такого помешкання,
одержує замість цього 4\pound{ фунти стерлінґів} річно. Гірничі округи
швидко приманюють до себе дуже численну людність, що складається
з самих гірників, а також ремісників, крамарів тощо,
які групуються навколо гірників. Як і скрізь, де густа залюдненість,
земельна рента й тут висока. Тому гірнопромисловець
намагається на якнайтіснішому будівельному терені при вході
в шахту побудувати стільки котеджів, скільки треба, щоб понапихати
туди всі свої робочі руки разом з їхніми родинами.
Коли недалеко відкривають нові шахти або знов починають
працювати в старих, тіснота збільшується. При будуванні котеджів
переважає лише один погляд — «поздержливість» капіталіста
від усяких не абсолютно неминучих витрат готівкою.
«Помешкання шахтарів та інших робітників, що зв’язані з копальнями
Northumberland і Durham, — каже д-р Джуліян Гентер,
— у пересічному є, мабуть, чи не найгірше й найдорожче
з усього того, що в цьому напрямі дає у великому маштабі
Англія, за винятком хіба подібних округ у Monmouthshire. Найбільше
лихо у тому, що кожна кімната надто переповнена, площа
густо забудована великим числом будинків, бракує води й
немає кльозетів, часто вживають методи ставити будинки один
над одним або розділяти їх in flats (так що різні котеджі становлять
поверхи, які вертикально лежать один над одним)\dots{} Підприємець
поводиться з усією колонією так, наче вона лише стоїть
табором, а не живе постійно».\footnote{
Там же, стоp. 180, 182.
} «Виконуючи дані мені інструкції,
— каже д-р Стивенс, — я відвідав більшу частину великих
гірничих селищ Durham Union’у\dots{} За дуже небагатьма винятками,
треба сказати, що ніде не звертають найменшої уваги на
заходи охорони здоров’я мешканців\dots{} Всі гірники прикріплені
\parbreak{}  %% абзац продовжується на наступній сторінці
