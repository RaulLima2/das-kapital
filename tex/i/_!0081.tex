\index{i}{*0081}  %% посилання на сторінку оригінального видання

\nonumsection{Післямова до другого видання}{.~}{Карл Маркс}
\label{original-81}
[Читачам першого видання я повинен насамперед дати пояснення
щодо змін, зроблених у другому виданні. Впадає на очі
чіткіший поділ матеріялу в книзі. Додаткові примітки всюди
позначено як примітки до другого видання. Що ж до самого тексту,
то в ньому найважливіші зміни такі:

В розділі першому, 1, вивід вартости за допомогою аналізи
рівнань, що в них виражається кожна мінова вартість, зроблено
з більшою науковою точністю; так само тут виразніше підкреслено
в першому виданні лише намічений зв’язок між субстанцією
вартости і визначенням величини вартости через суспільно-доконечний
робочий час. Розділ перший, 3 (форма вартости), перероблено
цілком, бо цього потребував уже двоякий виклад у
першому виданні. — Зауважу, між іншим, що цей двоякий виклад
подано з пропозиції мого приятеля, д-ра Л. Куґельмана в Ганновері.
Я гостював у нього весною 1867~\abbr{р.}, коли з Гамбурґу
прийшли перші пробні аркуші, і він переконав мене, що для більшости
читачів потрібен додатковий, більш дидактичний виклад
форми вартости. — Останній відділ першого розділу «Фетишистичний
характер товару і~\abbr{т. д.}» в більшій частині змінено. Розділ
третій, 1 (міра вартости), старанно переглянуто, бо цей відділ
у першому виданні викладено недбало, з посиланням на вже даний
виклад «До критики політичної економії, Берлін 1859». Розділ
сьомий, особливо частину 2, значно перероблено.

Зайво було б казати окремо про зміни тексту, тут і там зроблені,
часто лише стилістичні. Вони проходять через усю книжку.
А проте, переглядаючи французький переклад, видаваний у Парижі,
я бачу тепер, що деякі частини німецького ориґіналу потребують
у деяких місцях ґрунтовного перероблення, а подекуди
значних стилістичних поправок або й стараннішого виправлення
випадкових недоглядів. Та для цього бракувало часу, бо лише
восени 1871~\abbr{р.}, працюючи коло інших невідкладних робіт, одержав
я повідомлення, що книгу розпродано і що друкувати друге
видання треба розпочати вже в січні 1872~\abbr{р.}

Те зрозуміння, що його «Капітал» скоро знайшов у широких
колах німецької робітничої кляси, є найкраща нагорода за мою
працю. Віденський фабрикант, пан Маєр, людина, що в царині
економії стоїть на буржуазних поглядах, яскраво довів у брошурі,
виданій підчас німецько-французької війни, що великий теоретичний
хист, який вважали за німецьке надбання, цілком зник у
\parbreak{}  %% абзац продовжується на наступній сторінці
