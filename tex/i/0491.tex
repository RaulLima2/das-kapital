і одночасно замасковує періодичне поновлення його самопродажу,
переміна його індивідуальних хазяїнів-наймачів і коливання
ринкових цін праці.\footnote{
Пригадаймо собі, що при праці дітей і т. ін. зникає навіть ця формальність
самопродажу.
}

Отже, капіталістичний процес продукції, розглядуваний в
його загальному зв’язку, або як процес репродукції, продукує
не тільки товар, не тільки додаткову вартість, — він продукує
й репродукує саме капіталістичне відношення, капіталіста на
одному боці, найманого робітника — на другому.\footnote{
«Капітал має за передумову найману працю, наймана праця має за передумову капітал. Вони взаємно
зумовлюють одне одного: вони взаємно породжують одне одного. Хіба робітник на бавовняній фабриці
продукує лише бавовняні тканини? Ні, він продукує капітал. Він продукує вартості, які знову служать
для того, щоб командувати над його працею, щоб за допомогою її створювати нові вартості». (К. Marx:
«Lohnarbeit und Kapital» у «Neue Rheinische Zeitung», № 266, 7 April 1849). Статті, опубліковані під
цим заголовком в «Neue Rheinische Zeitung», є уривки лекцій, що їх я на цю тему читав у німецькому
робітничому товаристві у Брюсселі; друкування їх перервала Лютнева
революція.\footnote*{
Статті ці з’явилися потім окремою брошурою і під тією ж назвою.
Є українське видання: Партвидав «Пролетар» 1932 р. \emph{Ред.}
}
}

Розділ двадцять другий

Перетворення додаткової вартости на капітал

1. Капіталістичний процес продукції в поширеному маштабі.
Перетворення законів власности товарової продукції на закони
капіталістичного присвоєння

Раніше нам треба було дослідити, як додаткова вартість виникає
з капіталу, тепер треба дослідити, як із додаткової вартости
виникає капітал. Вживання додаткової вартости як капіталу
або зворотне перетворення додаткової вартости на капітал, називається
акумуляцією капіталу.\footnote{
«Акумуляція капіталу: вживання частини доходу як капіталу» («Accumulation of Capital: the
employment of a portion or revenue as capital»). (Malthus: «Definitions etc.», ed. Cazenove, p. 11).
«Перетворення доходу на капітал» («Conversion of revenue into capital»).
(Malthus: «Principles of Political Economy», 2 nd. ed. London 1836, p. 320).
}

Розгляньмо цей процес насамперед з погляду поодинокого капіталіста. Припустімо, наприклад, що
прядільний фабрикант авансував капітал у 10.000 фунтів стерлінґів, з них чотири п’ятих на бавовну,
машини й т. д., і одну п’ятину на заробітну плату. Нехай він щороку продукує 240.000 фунтів пряжі
вартістю в 12.000 фунтів стерлінґів. При нормі додаткової вартости в 100\% додаткова вартість
міститься в додатковому продукті

цього bondsman’a його всевладний пан, лічачи все, зараховує до своїх побічних доходів... Фармер не
дозволяє будувати в навкольності ніяких кльозетів, крім його власних, і не терпить щодо цього
ніякого порушення своїх сюзеренних прав». («Public Health, VII th Report 1864», p.188).