плата почала падати нижче мінімуму, і її доводилося доповнювати
офіціальною допомогою для бідних. Їхньої заробітної плати,
каже він, «ледве вистачало, щоб задовольнити доконечні життєві
потреби».

Послухаймо на хвилину ще одного оборонця inclosures і противника
д-ра Прайса. «Хибний висновок, ніби відбувається
збезлюднення, якщо не видно більше людей, що марнують свою
працю у відкритому полі... Коли після перетворення дрібних
селян на людей, що мусять працювати на інших, пускається у
рух більше праці, то це ж користь, якої нація [до неї «перетворені»,
ясна річ, не належать] мусить собі бажати... Продукту
буде більше, коли їхню комбіновану працю вживатимуть на
одній фармі; таким чином утворюється додатковий продукт для
мануфактур, і через це число мануфактур, цих розсипищ золота
нашої нації, зростає пропорційно до кількости продукованого
збіжжя».212

Зразок стоїчного душевного спокою, з яким політико-економ
розглядає якнайнахабніші порушення «святого права власносте»
й найгрубіші насильства над особою, скоро тільки вони
потрібні для створення основи капіталістичного способу продукції,
показує нам, між іншим, сер Ф. М. Ідн, отой до того
ще на торійський штаб забарвлений «філантроп». Ціла низка грабунків,
жорстокостей і народніх злигоднів, що супроводили насильну
експропріяцію народу, починаючи від останньої третини
XV аж до кінця XVIII століття, приводить його тільки до
такого «утішного» кінцевого висновку: «Треба було встановити
належну (due) пропорцію між орною землею та пасовиськами.
Ще протягом цілого XIV й більшої частини XV століття
один акр пасовиська припадав на 2, 3 й навіть 4 акри орного
поля. В середині XVI століття ця пропорція змінилася так, що
2 акри пасовиська припадили на 2 акри орного поля, а пізніше
2 акри пасовиська на 1 акр орного поля, аж поки, нарешті, встановилась
пропорція — 3 акри пасовиська на 1 акр орного поля».

У XIX столітті зникла, звичайно, навіть і згадка про зв’язок
поміж рільником і громадською власністю. Не кажучи вже
про пізніші часи, чи одержало колибудь селянство хоч шеляг
відшкодування за ті 3.511.770 акрів громадської землі, які

на повне безділля, бо землею володіли багачі, які на її оброблення вживали
рабів замість вільних людей». (Арріаn: «Römische Bürgerkriege»,
1,7). Це місце стосується до часу перед законом Ліцінія. Військова служба,
що так дуже прискорила руйнування римських плебеїв, була також
головним засобом, що ним Карл Великий дуже прискорив перетворення
вільних селян на февдально залежних і кріпаків.

212 «An Inquiry into the Connection between the present Price of
Provisons etc.», p. 124, 129. Подібне, але з протилежною тенденцією, ми
читаємо в іншого автора: «Робітників проганяють з їхніх котеджів і
примушують іти до міст шукати там роботи — але це дає більше додаткового
продукту, і таким чином капітал зростає» («Working men are driven
from their cottages, and forced into the towns to seek for employment; —
but then a larger surplus is obtained, and thus Capital is augmented»).
(«The Perils of the Nation», 2 nd ed. London 1843, p. XIV).
