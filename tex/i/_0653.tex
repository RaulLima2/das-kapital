\parcont{}  %% абзац починається на попередній сторінці
\index{i}{0653}  %% посилання на сторінку оригінального видання
фіскальної системи ще більше посилюється через протекційну систему, що є одна з складових частин
фіскальної системи.

Велика роля, яку відіграють державні борги й відповідна їм фіскальна система в капіталізації
багатств та експропріяції мас, призвела цілий ряд письменників, як от Коббета, Дублдея і інших, до
того, що вони помилково шукали в цьому головну
причину злиднів сучасних народів.

Система протекціонізму була штучним засобом фабрикувати фабрикантів, експропріювати незалежних
робітників, капіталізувати національні засоби продукції та існування, насильно скорочувати перехід
від стародавнього способу продукції до сучасного. Европейські держави билися за патент на цей
винахід і, ставши раз на службу капіталістам (Plusmacher’aм), вони для цієї мети не тільки грабували
начисто свій власний народ: посередньо — через охоронні мита, безпосередньо — через експортові
премії і т. ін. В залежних від них сусідніх країнах вони силоміць
винищували всю промисловість, як наприклад, ірляндську вовняну мануфактуру, що її знищила Англія. На
европейському континенті за прикладом Кольбера цей процес зроблено ще куди простішим. Первісний
капітал промисловців припливав тут до
них почасти безпосередньо з державної скарбниці. «Нащо, — каже Мірабо, — так далеко шукати причин
розцвіту мануфактури в Саксонії перед семилітньою війною? 180 мільйонів державних боргів!».\footnote{
«Pourquoi aller chercher si loin la cause de l’éclat manufacturier de la Saxe avant la guerre?
Cent quatre-vingt millions de dettes faites par les souverains!». (Mirabeau: «De la Monarchie
Prussienne», Londres 1788, vol. VI, p. 101).
}

Колоніяльна система, державні борги, податковий тягар, протекціонізм, торговельні війни й т. ін. —
всі ці паростки власне мануфактурного періоду колосально розростаються за дитячого періоду великої
промисловости. Народження цієї останньої
відсвятковано величезним іродовим побиттям дітей. Фабрики набирали робітників, як королівська фльота
матросів, насильною рекрутацією. Хоч який байдужий сер Ф.М.Ідн щодо страхіть експропріяції землі в
сільської людности, що тривала від останньої
третини XV століття аж до його часів, до кінця XVIII століття, хоч як самозадоволено він вітає цей
процес, «доконечний», щоб «утворити» капіталістичне рільництво та «правильне відношення між орною
землею й пасовиськом», а все ж і він не виявляє такого самого економічного розуміння щодо
доконечности крадіння дітей і рабства їх для перетворення мануфактурного виробництва на фабричне та
встановлення правильного відношення між капіталом і робочою силою. Він каже: «Може, варто було б
замислитися публіці над тим, чи може будь-яка мануфактура, що для успішного свого функціонування
мусить красти з котеджів і робітних домів дітей і примушувати їх позмінними групами тяжко працювати
більшу частину ночі, позбавляючи їх відпочинку; мануфактура, яка до того ж так збиває
\parbreak{}  %% абзац продовжується на наступній сторінці
