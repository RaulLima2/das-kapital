\parcont{}  %% абзац починається на попередній сторінці
\index{i}{0386}  %% посилання на сторінку оригінального видання
Її тепер перетворено на зовнішній відділ фабрики, мануфактури
або крамниці. Окрім фабричних робітників, мануфактурних робітників
і ремісників, яких капітал просторово концентрує великими
масами та якими командує безпосередньо, він незримими
нитками пускає в рух іншу армію, армію домашніх робітників,
порозкидуваних по великих містах та по селах. Приклад: фабрика
сорочок панів Тіллей в Лондондері в Ірландії вживає \num{1.000} фабричних
робітників та \num{9.000} домашніх робітників, порозкиданих по
селах.\footnote{
«Children’s Employment Commission. 2 nd Report 1864»,
p. LXVIII, n. 415.
}

Експлуатація дешевих та незрілих робочих сил у сучасній
мануфактурі стає ще безсоромнішою, ніж на фабриці у власному
значенні, бо технічної основи, що існує на фабриці, а саме заміни
мускульної сили машинами та легкости праці, в мануфактурі
здебільшого немає; крім того, в мануфактурі жіночий або ще
незрілий організм дітей якнайбезсовісніше віддають під впливи
отруйних речовин тощо. У так званій домашній праці ця експлуатація
стає ще безсоромнішою, ніж у мануфактурі, тому що здатність
робітників до опору зменшується з розпорошенням їх;
тому що поміж власне підприємцем і робітником втискується цілий
ряд хижацьких паразитів; тому що домашня праця всюди бореться
з машиновим або, принаймні, з мануфактурним виробництвом тієї
самої галузі продукції; тому що злидні відбирають у робітника
найпотрібніші умови праці — помешкання, світло, вентиляцію
й~\abbr{т. ін.}; тому що нереґулярність праці зростає, і, нарешті, тому
що в цих останніх притулках для всіх тих, кого велика промисловість
і рільництво зробили «зайвими», конкуренція між робітниками
неминуче досягає свого максимуму. Економізування
засобів продукції, що його вперше систематично розвиває й
організує машинове виробництво, економізування, що з самого
початку є разом з тим найнещадніше марнотратство робочої
сили та грабування нормальних передумов функціонування праці,
виявляє тепер свій антагоністичний та душогубний бік то дужче,
що менше в певній галузі промисловости розвинута суспільна
продуктивна сила праці й технічна основа комбінованих процесів
праці.

\subsubsection{Сучасна мануфактура}

А тепер я поясню на декількох прикладах подані вище тези.
З розділу про робочий день читач уже справді знає безліч доказів
їх. Металеві мануфактури в Бермінґемі та околицях вживають,
здебільша для дуже важкої роботи, \num{30.000} дітей та підлітків,
поруч \num{10.000} жінок. Ми знаходимо їх тут в антигігієнічному
мосяжництві, на фабриках ґудзиків, за ґлязуруванням, ґальванізуванням
і дякуванням.\footnote{
І навіть у шліхтуванні терпугів у Шеффілді працюють діти!
} Надмірна праця для дорослих та
\parbreak{}  %% абзац продовжується на наступній сторінці
