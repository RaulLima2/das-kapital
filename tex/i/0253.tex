Тут не час розглядати, яким способом іманентні закони капіталістичної
продукції виявляються у зовнішньому русі капіталів,
набирають сили як примусові закони конкуренції, і тому доходять
до свідомости поодинокого капіталіста як рушійні мотиви
його діяльности; але в усякому разі само собою зрозуміло: наукова
аналіза конкуренції можлива лише після того, як уже пізнано
внутрішню природу капіталу, цілком так само, як видимий рух
небесних тіл зрозумілий лише тому, хто знає їхній дійсний, але
почуттєво несприйманий рух. Однак, щоб зрозуміти продукцію
відносної додаткової вартости лише на основі здобутих уже результатів,
треба зауважити ось що.

Якщо одна робоча година виражається в кількості золота
в 6 пенсів, або 1/2 шилінґа, то протягом дванадцятигодинного
робочого дня продукується вартість у 6 шилінґів. Припустімо, що
за даної продуктивности праці виготовляється 12 штук товару
протягом цих дванадцятьох робочих годин. Вартість засобів продукції,
сировинного матеріялу тощо, зужиткованих на кожну
штуку, хай буде 6 пенсів. За цих обставин кожна штука товару
коштує 1 шилінґ, або 12 пенсів, а саме 6 пенсів є вартість засобів
продукції та 6 пенсів — нова вартість, додана до них працею
підчас їхнього перероблення. Припустімо тепер, що якомусь
капіталістові пощастить подвоїти продуктивну силу праці і тому
замість 12 продукувати 24 штуки цього товару протягом дванадцятигодинного
робочого дня. За незмінної вартости засобів
продукції вартість окремої штуки товару спадає тепер до 9 пенсів,
 а саме, 6 пенсів за вартість засобів продукції, а 3 пенси становлять
нову вартість, додану останньою працею. Хоч продуктивна
сила праці й подвоїлася, робочий день, як і раніш, створює
лише нову вартість у 6 шилінґів, яка, однак, розподіляється
тепер на подвійну кількість продуктів. Отже, на кожний окремий
продукт припадає замість 1/12 лише 1/24 цієї загальної вартости,
замість 6 пенсів 3 пенси, або, що сходить на те саме, до засобів
продукції підчас їхнього перетворення на продукт додається на
кожну штуку товару тепер лише півгодини праці замість цілої,
як це було раніш. Індивідуальна вартість цього товару стоїть
тепер нижче від його суспільної вартости, тобто він коштує менше
робочого часу, аніж велика маса тих самих товарів, які випродуковано
за пересічних суспільних умов. Одна штука цього товару
коштує пересічно 1 шилінґ, або репрезентує 2 години суспільної
праці; із зміною способу продукції вона коштує лише 9 пенсів,
або містить у собі лише 1 1/2 години праці. Але дійсна вартість
якогось товару є не його індивідуальна, а його суспільна вартість,
тобто її вимірюється не тим робочим часом, що його фактично
коштує товар продуцентові в окремому випадку, а робочим часом,
суспільно потрібним на його продукцію. Отже, коли капіталіст,
що вживає нової методи, продає свій товар за його суспільною

наслідок цього має змогу дешевше зодягати робітника... і оскільки таким
чином на робітника припадає менша частина цілого продукту». («Ram-Say:
«An Essay on the Distribution of Wealth», Edinburgh 1836, p. 168,169)
