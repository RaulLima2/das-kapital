бо капітал у 2.000 фунтів стерлінґів за старого способу виробництва
вживав би не 400, а 1.200 робітників. Отже, відносне зменшення
числа вживаних робітників узгоджується з його абсолютним
збільшенням. Вище ми припускали, що із зростом цілого
капіталу склад його лишається незмінний, бо й умови продукції
не змінюються. Але ми вже знаємо, що з кожним кроком розвитку
машинової системи стала частина капіталу, яка складається з
машин, сировинного матеріялу тощо, зростає, тим часом як
змінна частина капіталу, витрачена на робочу силу, падає; разом
з тим знаємо, що ні за якого іншого способу продукції не буває
такого постійного поліпшення машин, а тому й таких змін у
складі цілого капіталу. Але цю постійну зміну так само постійно
переривають періоди спокою та просте кількісне поширення на
даній технічній основі. Тому число вживаних робітників зростає.
Так, число робітників на бавовняних, вовняних, суканої вовни,
лляних та шовкових фабриках Об’єднаного Королівства 1835 р.
становило лише 354.684, тимчасом як 1861 р. число самих ткачів
(обох статей та найрізнішого віку, від 8 років починаючи) при
парових варстатах становило 230.654. Певна річ, цей зріст буде
менш значним, коли взяти на увагу, що ще 1838 р. брітанських
ручних ткачів бавовни разом з їхніми родинами, яким вони сами
давали заняття, налічувалося 800.000 чоловіка,230 не кажучи вже
зовсім про тих ручних ткачів, що їх витиснуто в Азії та на континенті
Европи.

В тих небагатьох увагах, що треба ще зробити про цей пункт,
ми почасти торкаємося суто фактичних відносин, до яких самий
наш теоретичний виклад ще не довів нас.

Поки машинове виробництво поширюється в якійсь галузі промисловосте
коштом традиційного ремества або мануфактури, його
успіхи настільки ж певні, як, наприклад, успіх армії, озброєної
ґвинтівками, проти армії, озброєної луками. Цей перший період,
коли машина ще тільки завойовує собі сферу діяння, має вирішальне
значення через ті надзвичайно високі зиски, що їх вона
допомагає продукувати. Ці зиски не тільки сами по собі становлять
джерело прискореної акумуляції, але вони ще й притягають у
цю сприятливу сферу продукції велику частину суспільного додаткового
капіталу, що постійно знов утворюється й шукає нового
вміщення. Особливі користі того першого періоду бурі й натиску
постійно повторюються по тих галузях продукції, де
машини заводиться вперше. Але, скоро тільки фабрична система
досягає певної ширини існування та певного ступеня зрілости,
скоро тільки, особливо, її власна технічна основа, машини, про-

230 «Страждання ручних ткачів (бавовни і тканини з домішкою
бавовни) були предметом дослідження королівської комісії, але хоч їхні
злидні були визнані й оплакані, все ж поліпшення (!) їхнього становища
віддали на волю випадкові та часові, і можна сподіватися, що ці страждання
тепер (через 20 років!) майже (nearly) зникли, чому, певно, допомогло
сучасне велике поширення парових ткацьких варстатів». («Reports
of lnsp. of Fact, for 31 st October 1856», p. 15).
