вартість робочої сили. Тому один із тих шістьох дрібних майстрів
видушував би більше, другий менше від загальної норми додаткової
вартости. Нерівності вирівнювалися б для суспільства, але
не для поодинокого майстра. Отже, закон зростання вартости
взагалі реалізується для поодинокого продуцента цілком тільки
тоді, коли він продукує як капіталіст, одночасно експлуатує
багатьох робітників, отже, коли він із самого початку пускає в
рух пересічну суспільну працю.\footnote{
Пан професор Рошер сповіщає про своє відкриття, що одна швачка,
працюючи два дні на пані професоршу, постачає більше праці, аніж дві
швачки, що працюють у неї протягом одного дня. Панові професорові
личило б свої спостереження над процесом капіталістичної продукції
робити не в дитячій кімнаті й не серед обставин, де бракує головної особи —
капіталіста.
}

Навіть за незмінного способу праці одночасне вживання
значного числа робітників викликає революцію в речових умовах
процесу праці. Будинки, де працює багато робітників, склади
на сировинний матеріял і т. ін., посуд, інструмент, апарати й т. ін.,
що одночасно або навпереміну обслуговують багатьох, коротко —
частину засобів продукції споживається тепер спільно в процесі
праці. З одного боку, мінова вартість товарів, отже, і засобів
продукції, ані скільки не підвищується в наслідок збільшеної
експлуатації їхньої споживної вартости. З другого боку,
зростає маштаб спільно вживаних засобів продукції. Кімната,
де працює 20 ткачів з їхніми 20 ткацькими варстатами, мусить
бути просторіша, аніж кімната незалежного ткача з двома підмайстрами.
Але збудувати одну майстерню для 20 осіб коштує
менше праці, ніж 10 майстерень на 2 особи кожна, і таким чином
взагалі вартість спільно вживаних і масово концентрованих засобів
продукції не зростає пропорційно до їхнього обсягу та корисного
ефекту. Спільно вживані засоби продукції віддають меншу
складову частину своєї вартости поодинокому продуктові, почасти
тому, що ціла вартість, яку вони віддають, розподіляється
одночасно на більшу масу продуктів, почасти тому, що вони увіходять
у процес продукції порівняно з індивідуально вживаними
засобами продукції, щоправда, з абсолютно більшою, але, розглядаючи
сферу діяльности їхньої, з відносно меншою вартістю.
Разом з цим знижується певна складова частина вартости сталого
капіталу, отже, пропорційно до величини цієї частини знижується
й ціла вартість товару. Ефект той самий, як коли б ці засоби
продукції товарів продукувалося дешевше. Ця економія на вживанні
засобів продукції виникає лише з спільного споживання
їх у процесі праці багатьох людей. І цього характеру засоби продукції
набирають як умови суспільної праці або суспільні умови
праці, відмінно від подрібнених і відносно дорогих засобів продукції
поодиноких самостійних робітників або дрібних майстрів,
навіть коли багато цих останніх працює разом в одному будинку,
але не спільно. Частина засобів праці набирає цього суспільного
характеру раніш, ніж його набуває сам процес праці.