\parcont{}  %% абзац починається на попередній сторінці
\index{i}{0110}  %% посилання на сторінку оригінального видання
товарів, покупець або продавець, а саме, в обох рядах оборудок
виступаю проти одного контраґента лише як покупець, а
проти другого лише як продавець, проти одного — лише як
гроші, проти другого — лише як товар; ані проти одного, ані
проти другого я не виступаю як капітал або як капіталіст або
як представник чогось такого, що було б більше, ніж гроші або
товар, або могло б заподіяти інший вплив, крім того, що його
можуть справляти гроші або товар. Для мене купівля в \emph{А} і продаж
\emph{В} становлять послідовний ряд. Але зв’язок поміж цими
обома актами існує лише для мене. А немає жодного діла до моєї
оборудки з \emph{В}, а \emph{В} — до моєї оборудки з \emph{А}. Коли б я захотів
пояснити їм особливу заслугу, яку я маю перед ними, обертаючи
послідовність ряду, то вони довели б мені, що я помиляюсь щодо
самого порядку послідовности, і що вся операція почалася не
від купівлі й кінчається не продажем, а, навпаки, почалася від
продажу й завершується купівлею. Справді, мій перший акт,
купівля, з погляду \emph{А} є продаж, а мій другий акт, продаж, з
погляду \emph{В} — купівля. Не задовольнившися цим, \emph{А} й \emph{В} заявляють,
що цілий цей порядок послідовности був зайвий фокус-покус.
\emph{А} продасть товар безпосередньо \emph{В}, а \emph{В} купить його безпосередньо
в \emph{А}. Разом з тим вся операція стискується в однобічний
акт звичайної товарової циркуляції, — просто продаж з погляду
\emph{А} і просто купівлю з погляду \emph{В}. Отже, обернувши порядок послідовности,
ми не вийшли поза сферу простої товарової циркуляції,
а тому ми мусимо розглянути, чи допускає вона з своєї природи
зростання вартостей, що входять у неї, тобто чи допускає вона
творення додаткової вартости.

Візьмімо процес циркуляції у формі, в якій він виявляється
як простий обмін товарів. Це завжди буває тоді, коли обидва
посідачі товарів купують один в одного товари і в термін платежу
вирівнюють балянс своїх взаємних грошових зобов’язань. Гроші
служать тут за рахункові гроші, щоб виразити вартості товарів
у їхніх цінах, але вони не виступають проти самих товарів речово.
Ясна річ, що, оскільки йдеться про споживну вартість,
виграти можуть обидва обмінювані. Обидва відчужують товари,
які є некорисні для них як споживні вартості, і одержують товари,
що їх вони потребують для споживання. І користь од цього може
бути не лише ця одна. \emph{А}, що продає вино й купує збіжжя, продукує,
може, більше вина, ніж його зміг би випродукувати за той
самий робочий час рільник \emph{В}, а рільник \emph{В} за той самий робочий
час продукує більше збіжжя, ніж його зміг би випродукувати
винар \emph{А}. Отже, \emph{А} дістає за таку саму мінову вартість більше
вбіжжя, а \emph{В} — більше вина, ніж дістав би відповідно кожний
із них без обміну, коли б вони мусили продукувати сами для себе
вино і збіжжя. Таким чином щодо споживної вартости можна
сказати, що «обмін є оборудка, в якій виграють обидві сторони»\footnote{
«Обмін є дивна оборудка, в якій виграють обидва контраґенти —
завжди (!)» («L’échange est une transaction admirable, dans la quelle les
deux contractants gagnent — toujours (!)»). (\emph{Destutt de Tracy}: «Traité de
la Volonté et de ses effets», Paris 1826, p. 68). Ta сама книга появилася пізніше
під назвою «Traité d’Economie Politique».
}.
\index{i}{0111}  %% посилання на сторінку оригінального видання
Інша справа з міновою вартістю. «Людина, що має багато вина,
а не має збіжжя, веде торг з людиною, що в неї багато збіжжя,
але немає вина, і вони обмінюють пшеницю вартістю в 50 на
вино вартістю в 50. Цей обмін не є збільшення мінової вартости
ні для одного, ані для другого, бо вже перед обміном кожний
з них мав вартість, рівну тій, що її він здобув собі за допомогою
цієї операції»\footnote{
\emph{Mercier de la Rivière}: «L’Ordre naturel et essentiel», Physiocrates, éd.
Daire, IІ.~Partie, p. 544.
}. Справа зовсім не змінюється, коли між товарами
виступають гроші як засіб циркуляції і акт купівлі почуттєво
відокремлюється від акту продажу\footnote{
«Само по собі цілком байдуже, чи є одна з цих двох вартостей
гроші, чи обидві вони є звичайні товари» («Que l’une de ces deux valeurs
soit argent, ou qu’elles soient toutes deux marchandises usuelles, rien de
plus indifférent en soi»). (Mercier de la Rivière: «L’Ordre naturel et essentiel».
Physiocrates, éd. Daire, II.~Partie, p. 543).
}. Вартість товарів є виражена
в їхніх цінах раніш, ніж вони вступають до циркуляції, отже,
вона є передумова циркуляції, а не її результат\footnote{
«Не контраґенти визначають вартість, її визначено ще до оборудки»
(«Ce ne sont pas les contractants, qui prononcent sur la valeur; elle est
décidée avant la convention»). (Le Trosne: «De l’Intérêt Social», Physiocrates,
éd. Daire, Paris 1846, p. 906).
}.

Розглядаючи справу абстрактно, тобто залишаючи осторонь
обставини, які не випливають з іманентних законів простої товарової
циркуляції, ми побачимо, що, крім заміни однієї споживної
вартости на іншу, в ній відбувається лише метаморфоза, проста
зміна форми товару. Та сама вартість, тобто та сама кількість
упредметненої суспільної праці, лишається в руках того самого
посідача товарів спочатку в формі товару, потім у формі грошей,
на які товар перетворився, нарешті, у формі товару, на який
знову перетворилися ці гроші. Ця зміна форми не містить у собі
жодної зміни величини вартости. Зміна, якої зазнає в цьому процесі
сама вартість товару, обмежується на зміні її грошової
форми. Спочатку вона існує як ціна подаваного на продаж товару,
потім як грошова сума, що була вже однак виражена в ціні, і,
нарешті, як ціна еквівалентного товару. Ця зміна форм сама по
собі так само мало містить у собі зміну величини вартости, як
ось розмін п’ятифунтової банкноти на соверени, півсоверени й
шилінґи. Отже, оскільки циркуляція товару зумовлює лише
зміну форми його вартости, вона зумовлює, — якщо явище відбувається
в чистій формі, — обмін еквівалентів. Тому навіть вульґарна
політична економія, хоч і як мало вона тямить, що таке
вартість, кожного разу, коли вона на свій штиб хоче розглянути
явище в його чистій формі, припускає, що попит і подання урівноважуються,
тобто, що вплив їхній взагалі припиняється. Отже,
коли щодо споживної вартости обидва контраґенти можуть виграти,
то на міновій вартості вони не можуть обидва виграти. Навпаки,
\parbreak{}  %% абзац продовжується на наступній сторінці
