ників. Не можна заперечити, що у вищенаведеному випадку,
приміром, машини не тільки звільняють 50 робітників і через це
роблять їх «вільними», але разом з тим ще знищують їхній зв’язок
із засобами існування вартістю в 1.500 фунтів стерлінґів та
«звільняють» таким чином ці засоби існування. Отже, той простий
і зовсім не новий факт, що машина звільняє робітника від
засобів існування, мовою економістів означає, що машина звільняє
засоби існування для робітника або перетворює їх на капітал,
щоб уживати робітника. Як бачимо, все залежить від того, яким
способом що висловити. Nominibus mollire licet mala.\footnote*{
Можна гарними словами підсолоджувати лихо. Ред.
}

За цією теорією засоби існування вартістю в 1.500 фунтів
стерлінґів були капіталом, що збільшив свою вартість за допомогою
праці п’ятдесятьох звільнених шпалерників. Отже, цей
капітал втрачає своє заняття, скоро тільки ті п’ятдесят робітників
звільняються від роботи, та не має і хвилини спокою, поки
не знайде нове «вміщення», де названі п’ятдесят робітників
знову зможуть споживати його продуктивно. Отже, раніш або
пізніш, капітал і робітники знову мусять зійтися, і тоді матимемо
компенсацію. Отож, страждання робітників, витиснутих машинами,
так само минущі, як і багатства цього світу.

Засоби існування в сумі 1.500 фунтів стерлінґів ніколи не протистояли
звільненим робітникам як капітал. Як капітал протистояли
їм ті 1.500 фунтів стерлінґів, які перетворено тепер на
машини. Коли ближче придивитися, то ці 1.500 фунтів стерлінґів
репрезентують лише ту частину шпалер, щорічно продукованих
за допомогою звільнених робітників, яку вони одержували
від свого хазяїна як заробітну плату не in natura,\footnote*{
— продуктами. Ред.
} а в грошовій
формі. За ці шпалери, перетворені на 1.500 фунтів стерлінґів,
купували вони собі засоби існування на таку саму суму. Тому ці
останні існували для них не як капітал, а як товари, і вони сами
існували для цих товарів не як наймані робітники, а як покупці.
Та обставина, що машина «звільнила» їх від купівельних засобів,
перетворює їх з покупців на непокупців. Звідси зменшений
попит на ці товари. Voilà tout.\footnote*{
Оце й усе. Ред.
} Якщо цей зменшений попит не
компенсується збільшеним попитом з іншого боку, то ринкова
ціна цих товарів меншає. Якщо це триває довго та у великому
розмірі, то постає переміщення робітників, уживаних у продукції
цих товарів. Частину капіталу, що раніше продукувала доконечні
засоби існування, репродукується в іншій формі.\footnote*{
У французькому виданні замість останніх двох речень читаємо
таке: «Але, може, це спричиниться до того, що капітал, якого уживалося
в продукції цих засобів існування, покличе до роботи як додаткових
робітників наших звільнених шпалерників? Цілком навпаки: якщо це
зниження цін триватиме деякий час, то почнуть знижувати заробітну плату
робітників, уживаних у продукції цих засобів існування. Якщо дефіцит
у збуті доконечних засобів існування триватиме довгий час, то частина
капіталу, вживана в продукції їх, відпливе звідси й шукатиме собі іншої
сфери вміщення». («Le Capital etc.», v. I, ch. XV. p. 190). Ред.
} Підчас спа-