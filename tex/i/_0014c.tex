\index{i}{0014}  %% посилання на сторінку оригінального видання
А. проста одинична, або випадкова форма вартости

х товару А = у товару В, або х товару А варто у товару В
(20 метрів полотна = 1 сурдутові, або 20 метрів полотна варті 1 сурдута).

1. Два полюси виразу вартости: відносна форма вартости
й еквівалентна форма

Таємниця всякої форми вартости криється в цій простій формі
вартости. Тим то саме її аналіза й являє труднощі.

Два різного роду товари А і В, в нашому прикладі полотно
й сурдут, відіграють тут, очевидно, дві різні ролі. Полотно виражає
свою вартість у сурдуті, а сурдут служить за матеріял для
цього виразу вартости. Перший товар відіграє активну ролю,
другий — пасивну. Вартість першого товару репрезентовано як
відносну вартість, або він перебуває у відносній формі вартости.
Другий товар функціонує як еквівалент, або перебуває в еквівалентній
формі.

Відносна форма вартости й еквівалентна форма є належні один
до одного нероздільні моменти, що взаємно один одного зумовлюють,
але рівночасно це є крайності, що виключають одна одну,
або одна одній протилежні, тобто це полюси того самого виразу
вартости; вони завжди розподіляються між тими різними товарами,
що вираз вартости ставить їх у взаємне відношення. Я не
можу, приміром, виразити вартість полотна в полотні. 20 метрів
полотна = 20 метрам полотна не є вираз вартости. Це рівнання
каже, власне, протилежне: 20 метрів полотна є не що інше, як
20 метрів полотна, тобто певна кількість споживного предмету
«полотно». Отже, вартість полотна можна виразити лише відносно,
тобто в іншому товарі. Тому відносна форма вартости полотна
має за передумову, що якийсь інший товар перебуває в еквівалентній
проти нього формі. З другого боку, цей інший товар, який
фігурує як еквівалент, не може одночасно перебувати у відносній
формі вартости. Не він виражає свою вартість. Він тільки дає
матеріял для виразу вартости іншого товару.

Правда, вираз: 20 метрів полотна = 1 сурдутові, або 20 метрів
полотна варті 1 сурдута, включає й зворотне відношення:
1 сурдут = 20 метрам полотна, або 1 сурдут вартий 20 метрів
полотна. Але тоді я мушу обернути рівнання, щоб виразити вартість
сурдута відносно, і скоро тільки я це зроблю, полотно стає
еквівалентом замість сурдута. Отже, той самий товар у тому самому
виразі вартости не може одночасно виступати в обох формах.
Ці останні полярно виключають одна одну.

Чи якийсь товар перебуває у відносній формі вартости, чи в
протилежній еквівалентній формі, це залежить виключно від того,
яке місце в кожному випадку він посідає у виразі вартости, тобто
від того, чи є він той товар, вартість якого виражається, чи той
товар, у якому виражається вартість.
