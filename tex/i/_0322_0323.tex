\parcont{}  %% абзац починається на попередній сторінці
\index{i}{0322}  %% посилання на сторінку оригінального видання
продукту, то межу вживання їх дано тим, що їх власна продукція
коштує менше праці, ніж праця, замінювана вживанням машин.
Однак, для капіталу ця межа визначається вужче. Через те, що
він оплачує не працю, якої вжито, а вартість ужитої робочої
сили, то для нього вживання машини обмежується ріжницею
між вартістю машини й вартістю тієї робочої сили, яку машина
заміняє. А що поділ робочого дня на доконечну працю й додаткову
працю по різних країнах є різний, так само як і в тій самій
країні він різний в різні періоди або в той самий період у різних
галузях продукції; що, далі, дійсна заробітна плата робітника
то падає нижче вартости його робочої сили, то підноситься понад
неї, то ріжниця між ціною машин і ціною робочої сили, яку ці
машини мають замінити, може дуже коливатися, навіть і тоді,
коли ріжниця між кількістю праці, потрібної для продукції
машини, і загальною кількістю праці, яку вона заміняє, лишається
без зміни.\footnoteA{
Примітка до другого видання. Тим то в комуністичному суспільстві
вживання машин мало б зовсім інший обсяг, ніж у суспільстві
буржуазному.
}  Але лише перша ріжниця визначає для самого
капіталіста витрати продукції товару та впливає на нього через
примусові закони конкуренції. Тим то в Англії нині винаходять
машини, яких уживають лише в Північній Америці, як у XVI
та XVII віці Німеччина винаходила машини, що їх уживала
лише Голляндія, і як деякі французькі винаходи XVIII віку
використовувано лише в Англії. Сама машина продукує в давніше
розвинених країнах через те, що її вживають у деяких галузях
підприємства, такий надмір праці (redundancy of labour, каже
Рікардо) по інших галузях, що тут падіння заробітної плати нижче
вартости робочої сили перешкоджає вживанню машин та робить
його зайвим, а часто й неможливим з погляду капіталу, зиск
якого і без того випливає із зменшення не просто вживаної ним
праці, а лише праці, ним оплаченої. По деяких галузях англійської
вовняної мануфактури останніми роками дитяча праця дуже
зменшилася, подекуди її майже витиснено. Чому? Фабричний
закон примусив до подвійної зміни дітей, що з них одна працює
6 годин, друга 4 години, або кожна лише по 5 годин. Але батьки
не хотіли продавати half-times (робітників половинного часу)
дешевше, ніж раніш продавали full-times (робітників повного
часу). Звідси заміна half-times машинами.\footnote{
«Підприємці не стануть без доконечности тримати дві зміни дітей
молодших за тринадцять років\dots{} Справді, одна кляса фабрикантів, що
прядуть вовну, тепер рідко вживає дітей молодших за 13 років, тобто
half-times. Вони позаводили нові машини та поліпшення різного роду,
які майже усувають працю дітей (тобто дітей до 13 років). Для ілюстрації
такого зменшення числа дітей я нагадаю про один процес праці, в якому
через додаток до тодішніх машин одного апарату, так званого piecing
machine, працю шістьох або чотирьох half-times, відповідно до особливости
кожної машини, може виконати один підліток (старший за
13 років). Система половинного часу» стимулювала «винахід присукуваль-
} Перед забороною
вживати жіночої та дитячої (нижче десятирічного віку) праці по
\index{i}{0323}  %% посилання на сторінку оригінального видання
копальнях капітал вважав за остільки згідне із своїм моральним
кодексом, і особливо із своєю головною книгою, примушувати
голих жінок та дівчат, часто разом із чоловіками, працювати по
вугільних та інших копальнях, що лише після тієї заборони він
удався до машин. Янкі винайшли машини розбивати камінь.
Англійці їх не вживають, бо «нещасний» (wretch — технічний
вислів в англійській політичній економії на означення рільничих
робітників), що виконує цю працю, дістає оплату такої
незначної частини своєї праці, що машини удорожчили б цю
продукцію для капіталіста.\footnote{
«Машини\dots{} часто не можуть знайти вжитку доти, доки праця (він
має на оці заробітну плату) не піднесеться» («Machinery\dots{} can frequently
not be employed until labour rises»). (Ricardo: «Principles of
Political Economy», 3 rd. ed, London 1821, p. 479).
} В Англії замість коней іноді все
ще вживають жінок, щоб тягати барки каналами тощо,\footnote{
Див. «Report of the Social Science Congress at Edinburgh. October
1863».
} бо
праця, потрібна на продукцію коней та машин, є математично
дана кількість, тимчасом як праця, потрібна на утримання жінок
із надмірної людности, є нижча від усякого обрахунку. Тим то
ніде немає безсоромнішого марнотратства людської сили на всякі
дрібниці, як саме в Англії, в цій країні машин.

3. Безпосередні діяння машинового виробництва на робітників

За вихідний пункт великої промисловості є, як це вже
показано, революція в засобі праці, а зреволюціонізований засіб
праці набирає своєї найрозвиненішої форми в розчленованій
системі машин на фабриці. Перше ніж розглядати, як до цього
об’єктивного організму додається людський матеріял, розгляньмо
деякі загальні діяння тієї революції на самого робітника.

а) Присвоювання капіталом додаткових робочих
сил. Жіноча та дитяча праця

Оскільки машина робить мускульну силу зайвою, вона стає
засобом, щоб уживати робітників без мускульної сили або робітників
з недостатнім фізичним розвитком, але з більшою гнучкістю
членів. Тому жіноча й дитяча праця була першим словом капіталістичного
вживання машин! Таким чином цей могутній засіб

ної машини». («Employers of labour would not unnecessarily retain two sets
fo children under thirteen\dots{} In fact one class of manufacturers, the spinners
of woollen yarn, now rarely employ children under thirteen years ages, i. e.
half-times. They have introduced improved and new machinery of various
kinds which altogether supersedes the employment of children; f. і: I will
mention one process as an illustration of this diminution in the number
of children, wherein, by thy addition of an apparatus, called a piecingmachine,
to existing machines, the work of six or four half-times, according
to the peculiarity of each machine, can be performed by one young
person\dots{} the half-time system» стимулювала «the invention of the
piecing-machine»). (Reports of Insp of Fact, for 31 st October 1858»).
\parbreak{}  %% абзац продовжується на наступній сторінці
