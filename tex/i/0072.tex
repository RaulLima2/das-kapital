давця як преобраяїена (entäusserte) форма товару і покидають їх
як абсолютно відчужувана форма товару. Вони двічі змінюють
місце. Перша метаморфоза полотна приносить ці монети до кишені
ткача, друга витягає їх знову звідти. Отже, обидві протилежні
зміни форми того самого товару відбиваються знов у дворазовій
зміні місця грошей у протилежному напрямі.

Навпаки, коли відбуваються лише однобічні товарові метаморфози,
однаково, самі купівлі або самі продажі, тоді ті самі
гроші перемінюють місце також лише один раз. Друга їхня
переміна місця виражає завжди лише другу метаморфозу товару,
його зворотне перетворення з грошей. В частому повторюванні
зміни місць тих самих монет відбивається не лише ряд метаморфоз
одним-одного товару, але й взаємне посплітування численних
метаморфоз товарового світу взагалі. А проте само собою зрозуміло,
що все це має силу лише для розглядуваної тут форми
простої товарової циркуляції.

Кожний товар, при першому своєму кроці у процесі циркуляції,
при першій зміні своєї форми, випадає з циркуляції, в яку завжди
вступає новий товар. Навпаки, гроші як засіб циркуляції постійно
перебувають у сфері циркуляції й постійно обертаються
в ній. Отже, постає питання, скільки грошей може постійно поглинути
ця сфера.

В даній країні щодня відбуваються одночасно, отже, просторово
одна побіч однієї, численні однобічні товарові метаморфози,
або, іншими словами, лише продажі з одного боку й лише купівлі
з другого. У своїх цінах товари вже прирівняно певним уявлюваним
кількостям грошей. А що розглядувана тут безпосередня
форма циркуляції завжди тілесно протиставить одно одному
товар і гроші, — перший на полюсі продажу, другі — на протилежному
полюсі купівлі, — то масу засобів циркуляції, потрібну
для процесу циркуляції всіх товарів, вже визначено сумою цін
товарів. Справді, гроші лише реально являють собою ту суму
золота, яку ідеально вже виражено в сумі цін товарів. Отже,
рівність цих сум є очевидна сама собою. Ми знаємо, однак, що
за незмінної вартости товарів ціни їхні змінюються разом з
зміною вартости самого золота (грошового матеріялу): підвищуються
пропорційно до зменшення його вартости і зменшуються
пропорційно до її підвищення. Коли сума цін товарів,
таким чином підвищується або знижується, то й маса грошей,
що циркулюють, мусить пропорційно збільшуватись або зменшуватись.
Правда, зміна маси засобів циркуляції залежить тут від
самих грошей, але не від їхньої функції як засобу циркуляції,
а від їхньої функції як міри вартости. Ціна товарів спочатку
змінюється зворотно пропорційно до зміни вартости грошей,
і потім маса засобів циркуляції змінюється прямо пропорційно
до змін цін товарів. Цілком таке саме явище відбувалося б, коли б,
приміром, не вартість золота понизилась, а срібло заступило
його як міра вартости, або коли б не вартість срібла підвищилась,
а золото витиснуло срібло з функції міри вартости. В першому
