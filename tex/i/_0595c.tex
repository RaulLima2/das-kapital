\parcont{}  %% абзац починається на попередній сторінці 
\index{i}{0595}  %% посилання на сторінку оригінального видання 
2 цалі завдовжки, 9 футів 5 цалів завширшки, вся площа має
21 фут 2 цалі довжини, 9 футів 5 цалів ширини. Спальня — це
кімната на горищі, стіни звужуються до стелі, як голова цукру,
з фасаду відкривається дахове віконце. Чому він живе тут?
Садок? Надзвичайно маленький. Квартирна плата? Висока —\footnote{
1    3    5    8    1    1    3    3    6
1    1    4    3    7    1    1    3    2    5
1    1    4    4    8    1    1    2    —   2
1    1    5    4    9    1    1    2    3    5
1    1    2    2    4    1    1    3    3    6
1    1    5    3    8    1    1    2    4    6
} шилінґ 3 пенси на тиждень. Близько до місця його праці?
Ні, хата стоїть на віддалі 6 миль від місця його праці, так що
йому доводиться день-у-день маршувати по 12 миль. Він живе
тут тому, що тут здавали в найми cot, а він хотів мати cot для
себе самого, хоч і де б то було, хоч і по якій би ціні, хоч і в якому
стані. Нижченаведена таблиця подає нам статистичні відомості
про 12 хат у Langtoft’i з 12 спальнями, що в них живуть 38 дорослих
і 36 дітей:

12 хат у Langloft’i

Хати   Спальні  Число дорослих    Число дітей    Загальне число мешканців    Хати    Спальні   
Число дорослих    Число дітей    Загальне число мешканців

9. Kent.

Kennington був надзвичайно переповнений у 1859 р., коли
з’явилася дифтерія і парафіяльний лікар організував офіціальний
дослід становища найбіднішої кляси людности. Він виявив, що
в цій місцевості, де потребують багато праці, багато cots зруйновано,
а нових не збудовано. В одній окрузі стояли 4 будинки,
так звані birdcages (пташині клітки), в кожному з них було по
4 кімнати таких розмірів у футах і цалях:

Кухня............................     9,5 × 8,11 × 6,6
Комірка-полоскальня...      8,6 × 4,6 ×× 6,6
Спальня........................     8,5 × 5,\footnote{
. Northamptonshire.

Brinworth, Pickford i Floore: В цих селах зимою тиняються
по вулицях 20—30 робітників, не находячи праці. Фармери не
завжди як слід обробляли землю під збіжжя й корінняки, і лендлорд
збагнув, що йому корисніше сполучити всі свої оренди
в дві або три. Звідси недостача роботи. Тимчасом як по одному
боці рову поле потребує обробітку, по другому боці ошукані
робітники кидають на нього пожадливі погляди. Воно й не диво,
що, виснажені гарячковою надмірною працею влітку й напів-
} × 6,3
Спальня.......................      8,3 × 8,4 × 6,3
\parbreak{}  %% абзац продовжується на наступній сторінці
