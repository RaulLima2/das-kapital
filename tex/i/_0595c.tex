\parcont{}  %% абзац починається на попередній сторінці
\index{i}{0595}  %% посилання на сторінку оригінального видання
2 цалі завдовжки, 9 футів 5 цалів завширшки, вся площа має
21 фут 2 цалі довжини, 9 футів 5 цалів ширини. Спальня — це
кімната на горищі, стіни звужуються до стелі, як голова цукру,
з фасаду відкривається дахове віконце. Чому він живе тут?
Садок? Надзвичайно маленький. Квартирна плата? Висока —
1 шилінґ 3 пенси на тиждень. Близько до місця його праці?
Ні, хата стоїть на віддалі 6 миль від місця його праці, так що
йому доводиться день-у-день маршувати по 12 миль. Він живе
тут тому, що тут здавали в найми cot, а він хотів мати cot для
себе самого, хоч і де б то було, хоч і по якій би ціні, хоч і в якому
стані. Нижченаведена таблиця подає нам статистичні відомості
про 12 хат у Langtoft’i з 12 спальнями, що в них живуть 38 дорослих
і 36 дітей:
\begin{table}[h]
\caption*{12 хат у Langloft’i}

\newlength{\myheight}
\setlength{\myheight}{5em}
\newcolumntype{Y}{>{\centering\arraybackslash}X}
\noindent\begin{tabularx}{\textwidth}{m{0.5cm}m{0.5cm}m{0.5cm}m{0.5cm}Y|m{0.5cm}m{0.5cm}m{0.5cm}m{0.5cm}Y}
  \toprule
  \centering\rotatebox[origin=c]{90}{\parbox[c]{\myheight}{Хати}} &
  \centering\rotatebox[origin=c]{90}{\parbox[c]{\myheight}{Спальні}} &
  \centering\rotatebox[origin=c]{90}{\parbox[c]{\myheight}{Число дорослих}} &
  \centering\rotatebox[origin=c]{90}{\parbox[c]{\myheight}{Число дітей}} &
  \parbox[c]{\myheight}{Загальне число мешканців} &
  \rotatebox[origin=c]{90}{\parbox[c]{\myheight}{Хати}} &
  \rotatebox[origin=c]{90}{\parbox[c]{\myheight}{Спальні}} &
  \rotatebox[origin=c]{90}{\parbox[c]{\myheight}{Число дорослих}} &
  \rotatebox[origin=c]{90}{\parbox[c]{\myheight}{Число дітей}} &
  \parbox[c]{\myheight}{Загальне число мешканців}
\\
  \midrule
1  &  1  &  3  &  5  &  8  &  1  &  1  &  3  &  3  &  6\\
1  &  1  &  4  &  3 & 7  &  1  &  1  &  3 & 2  &  5\\
1  &  1  &  4  &  4  &  8  &  1  &  1  &  2  &  —  & 2\\
1  &  1  &  5  &  4 & 9  &  1  &  1  &  2 & 3  &  5\\
1  &  1  &  2  &  2  &  4  &  1  &  1  &  3  &  3  &  6\\
1  &  1  &  5  &  3  &  8  &  1  &  1  &  2  &  4  &  6\\
\end{tabularx}
\end{table}

\paragraph{Kent.}

Kennington був надзвичайно переповнений у 1859 р., коли
з’явилася дифтерія і парафіяльний лікар організував офіціальний
дослід становища найбіднішої кляси людности. Він виявив, що
в цій місцевості, де потребують багато праці, багато cots зруйновано,
а нових не збудовано. В одній окрузі стояли 4 будинки,
так звані birdcages (пташині клітки), в кожному з них було по
4 кімнати таких розмірів у футах і цалях:

\begin{table}[h]
  \centering
  \begin{tabular}{l}
    Кухня\makebox[0.3\textwidth]{\dotfill{}}9,5 × 8,11 × 6,6 \\
    Комірка-полоскальня\dotfill{}8,6 × 4,6 × 6,6 \\
    Спальня\dotfill{}8,5 × 5,10 × 6,3 \\
    Спальня\dotfill{}8,3 × 8,4 × 6,3 \\
  \end{tabular}
\end{table}

\paragraph{Northamptonshire.}
Brinworth, Pickford i Floore: В цих селах зимою тиняються
по вулицях 20 — 30 робітників, не находячи праці. Фармери не
завжди як слід обробляли землю під збіжжя й корінняки, і лендлорд
збагнув, що йому корисніше сполучити всі свої оренди
в дві або три. Звідси недостача роботи. Тимчасом як по одному
боці рову поле потребує обробітку, по другому боці ошукані
робітники кидають на нього пожадливі погляди. Воно й не диво,
що, виснажені гарячковою надмірною працею влітку й напів-

\parbreak{}  %% абзац продовжується на наступній сторінці
