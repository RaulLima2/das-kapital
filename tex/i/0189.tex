1845 р. вона швидко розвинулася в Англії і з густо заселених
частин Лондону поширилась особливо на Менчестер, Бермінґем,
Ліверпул, Брістол, Норвіч, Ньюкестл, Ґлезґо, розносячи з
собою спазматичне стиснення щелеп — недугу, яку один віденський
лікар уже в 1845 р. визнав за своєрідну хоробу сірничкарів.
Половина робітників — це діти, молодші 13 років, і
підлітки, молодші 18 років. Унаслідок шкідливого впливу на
здоров’я та гидких умов праці в цій мануфактурі, про неї пішла
така погана слава, що лише найзанепаліша частина робітничої
кляси, напівголодні вдовиці тощо постачає для неї дітей — «обідраних,
напівголодних, цілком занехаяних і невихованих дітей».\footnote{
Там же, стор. LIV.
}
З тих свідків, яких переслухував член комісії Байт (1863 р.),
270 не мали 18 років, 50 мало менш як 10 років, 10 були лише
у 8 і 5 лише в 6 році життя. Робочий день, що довжина його коливається
між 12, 14 і 15 годинами, нічна праця, нереґулярний
час для їжі, при чому їсти здебільшого доводиться в самих майстернях,
затруєних фосфором. Данте знайшов би, що ця мануфактура
перевищує всі його найжахніші фантазії про пекло.

На фабриці шпалер грубші сорти їх друкується машинами,
кращі — руками (block printing). Найжвавіший період перебігу
справ припадає на час від початку місяця жовтня до кінця
квітня. Протягом цього періоду праця триває часто від 6 години
ранку до 10 години вечора, а то й до пізньої ночі, і майже без
перерви.

Дж. Леч свідчить: «Останньої зими (1862 р.) із 19 дівчат 6
не могли працювати далі, захорівши від надмірної праці. Щоб не
дати їм заснути, я мусив на них кричати». В. Деффі: «Від утоми
діти часто не могли тримати очей розплющеними; в дійсності й
нам самим це часто ледве вдається». Дж. Ляйтбурн: «Мені
13 років... минулої зими ми працювали до 9 години вечора, а
зиму перед тим до 10 години. Останньої зими я майже щовечора
кричав від болю на поранених ногах». Ґ. Епеден: «Цього мого
хлопчика, коли йому було лише 7 років, я день-у-день носив
на плечах туди й назад по снігу, і він звичайно працював по
16 годин!.. Часто я ставав навколішки, щоб нагодувати його в той
час, як він сам стояв біля машини, бо він не смів ні відійти від
неї, ні припинити її». Сміс, компаньйон-управитель однієї
менчестерської фабрики: «Ми (він має на гадці ті його «руки»,
що працюють на «нас») працюємо без перерви на їжу, так що
десятигодинний робочий день кінчається о 4\sfrac{1}{2}  годині вечора, а
все, що далі, є надробочий час».\footnote{
Цього не можна вважати за додатковий робочий час у нашому розумінні.
Ці пани розглядають 10\sfrac{1}{2}-годинну працю як нормальний
робочий день, що містить у собі отож і нормальну додаткову працю. Опісля
починається «надробочий час», оплачуваний трохи краще. Пізніш при
пагоді ми побачимо, що вживання робочої сили протягом так званого
} (Не знати, чи той пан Сміс
дійсно нічого не їсть протягом  10\sfrac{1}{2} годин?) «Ми (той самий
Сміс) рідко коли кінчаємо працю перед 6 годиною вечора (він