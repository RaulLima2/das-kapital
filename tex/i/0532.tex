(A. Smith: «Wealth of Nation», кн. І, розд. 8). Досі ми розглядали
лише одну особливу фазу цього процесу, ту, коли приріст капіталу
відбувається за незмінного технічного складу капіталу.
Але цей процес іде поза межі цієї фази.

Скоро загальні основи капіталістичної системи вже дані, то
в перебігу акумуляції завжди постає такий момент, коли розвиток
продуктивности суспільної праці стає наймогутнішою
підоймою акумуляції. «Та сама причина, — каже А. Сміс, —
яка підвищує заробітну плату, а саме збільшення капіталу,
спонукує до піднесення продуктивних здібностей праці й дає
змогу меншій кількості праці продукувати більшу кількість
продуктів».\footnote*{
Замість останніх двох абзаців у першому німецькому виданні
тут читаємо таке:

«Розвинене в попередньому підрозділі має силу лише за тієї передумови,
що в перебігу акумуляції відношення між масою засобів продукції і
масою робочої сили, що пускає їх у рух, лишається незмінним, отже,
за передумови, що попит на працю зростає відповідно до зростання капіталу.
Це припущення фігурує в Адама Сміса в його аналізі акумуляції
як сама собою зрозуміла аксіома. [До певної міри воно завжди лишається
правильним, бо хоч і як можуть революціонізуватись технологічні умови
процесу продукції, всеж протягом коротшого або довшого періоду то в одній,
то в другій сфері продукції акумуляція капіталу, або поширення маштабу
продукції відбувається на вже даній технологічній базі. Отже, у цих межах
попит на працю зростає разом з акумуляцією. Але сама наявна база
безупинно революціонізується].\footnote*{
Заведене у прямі дужки Маркс з другого видання вилучив. Дальшу
фразу він тут починає так: «Але Адам Сміс недобачає, що в перебігу
акумуляції і т. д...». \emph{Ред.}
} В перебігу акумуляції відбувається
велика революція у відношенні між масою засобів продукції і масою робочої
сили, що пускає їх у рух. Ця революція відбивається на мінливому
складі капітальної вартости, на її поділі на сталу і змінну складові частини,
тобто на мінливому відношенні між її частинами, що перетворюються
на засоби продукції і на робочу силу». \emph{Ред.}
}

Залишаючи осторонь природні умови, як родючість ґрунту
тощо, і вправність незалежних продуцентів, що працюють ізольовано,
вправність, яка, однак, виявляється більше якісно, в якості
продукту, аніж кількісно, в масі продукту, ступінь суспільної
продуктивности праці виражається у відносній величині розміру
засобів продукції, що їх якийсь робітник протягом даного часу
перетворює на продукт з тим самим напруженням робочої сили.
Маса засобів продукції, що за їхньою допомогою він функціонує,
зростає разом з продуктивністю його праці. При цьому ці засоби
продукції відіграють двояку ролю. Зростання одних є наслідок,
зростання інших — умова зростання продуктивности праці.
Наприклад, при мануфактурному поділі праці і при застосуванні
машин за той самий час перероблюють більше сировинного матеріялу,
отже, більша маса сировинного матеріялу й допоміжних
матеріялів увіходить у процес праці. Це — наслідок зростання
продуктивности праці. З другого боку, маса застосованих машин,
робочої худоби, мінерального добрива, дренажних труб і т. д.
є умова зростання продуктивности праці. Те ж саме стосується