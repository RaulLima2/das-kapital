1 сурдутові маємо тепер: 20 метрів полотна = 1/2 сурдута. Навпаки,
коли вартість сурдута спаде наполовину, то 20 метрів полотна
= 2 сурдутам. Отже, за незмінної вартости товару А його
відносна вартість, виражена в товарі В, падає або підвищується
зворотно пропорційно до зміни вартости В.

Коли порівняємо різні випадки пунктів І і II, то ми побачимо,
що та сама зміна величин відносної вартости може випливати з
цілком протилежних причин. Так, з рівнання: 20 метрів полотна
= 1 сурдутові може зробитися: 1) рівнання: 20 метрів
полотна = 2 сурдутам або через те, що вартість полотна подвоюється,
або через те, що вартість сурдута падає наполовину, і
може зробитися 2) рівнання: 20 метрів полотна = 1/2 сурдута
або через те, що вартість полотна падає наполовину, або через те,
що вартість сурдута зростає удвоє.

III. Хай доконечні для продукції полотна й сурдута кількості
праці одночасно змінюються в тому самому напрямі і в однаковій
пропорції. В цьому випадку, як і раніш, рівнання: 20 метрів
полотна = 1 сурдутові лишається незмінне, хоч і як змінятимуться
їхні вартості. Зміну їхньої вартости ми відкриємо, скоро
порівняємо їх з якимось третім товаром, вартість якого лишається
сталою. Коли б вартості всіх товарів підвищились або впали
одночасно і в однаковій пропорції, то їхні відносні вартості лишилися
б незмінні. Дійсну зміну їхньої вартости можна було б
побачити з того, що протягом того самого робочого часу тепер
продукувалося б взагалі більшу або меншу кількість товарів,
ніж раніш.

IV. Хай робочий час, доконечний для продукції полотна і
сурдута, а тому й їхні вартості одночасно змінюються в однаковому
напрямі, алеж не в однаковій мірі, або в протилежному
напрямі й т. ін. Вплив усяких можливих подібних комбінацій
на відносну вартість товару визначається просто через застосовування
випадків І, II, III.

Отже, дійсні зміни величини вартости не відбиваються ані
ясно, ані вичерпно в їхньому відносному виразі, або у величині
відносної вартости. Відносна вартість якогось товару може змінятися,
хоч його вартість лишається стала. Його відносна вартість
може лишатися сталою, хоч його вартість змінюється і, нарешті,
одночасні зміни величини вартости й відносного виразу
цієї величини вартости не неодмінно повинні одна однією покриватися.20
20 Примітка до другого видання. Вульґарна політична економія
із звичною їй дотепністю використовує цю невідповідність (Inkongruenz)
між величиною вартости та її відносним виразом. Приміром: «Припустіть
тільки, що А понижується тому, що В, на яке воно обмінюється, підвищується,
хоч проте на А витрачається не менше праці, ніж раніш, і
ваш загальний принцип вартости розлетиться, мов прах... Коли ж припустити,
що вартість В супроти А понижується, бо вартість А супроти В
підвищується, то в нас зникає з-під ніг той ґрунт, на якому Рікардо
будує свій великий принцип, що вартість товару завжди визначається
кількістю втіленої в ньому праці, бо коли зміна у витратах на А змінює
