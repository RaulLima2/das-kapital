плати»].\footnote*{
Cherbuliez: «Riche ou Pauvre», p. 146. — Заведене у прямі дужки
ми беремо з французького видання. \emph{Ред.}
} — Нарешті, Детю де Трасі, буржуазний доктринер з
холодною, як у риби, кров’ю, брутально заявляє: «Бідні нації —
це ті, де народові добре, а багаті нації — де народ звичайно бідний».\footnote{
Destutt de Trasy: «Traité de la Volonté et de ses effets», Paris
1826, p. 231: «Les nations pauvres, c’est là où le peuple est à son aise: et
les nations riches, c’est là où il est ordinairement pauvre».
}

5. Ілюстрація загального закону капіталістичної акумуляції

а) Англія 1846—1866 рр.

Жоден із періодів сучасного суспільства не є такий сприятливий
для вивчення капіталістичної акумуляції, як період останніх
двадцяти років. Здається, наче він найшов торбу фортуни.
Але з-поміж усіх країн клясичний приклад дає знову таки
Англія, бо вона посідає перше місце на світовому ринку,
тільки в ній цілковито розвинувся капіталістичний спосіб продукції,
і, нарешті, заведення тисячолітнього царства вільної
торговлі від 1846 р., відібрало від вульґарної політичноїї економії
її останній притулок. Про титанічний проґрес продукції, що зумовив
знову таки значну перевагу останньої половини двадцятирічного
періоду над першою, вже досить зазначено в четвертому
відділі.

Хоч абсолютне зростання англійської людности за останнє півстоліття
було дуже велике, однак відносне зростання, або норма
приросту, безперервно падало, як показує ця таблиця, запозичена
з офіціяльного перепису.

Щорічний процентовий приріст людности Англії й Велзу
за десятиріччями:

1811—\footnote{
—\footnote{
—\footnote{
—\footnote{
—1861...............1,141\%

Розгляньмо тепер, з другого боку, зростання багатства. Найпевнішу
точку опори дає тут рух зисків, земельних рент і т. ін.,
що підлягають прибутковому оподаткуванню. Приріст зисків, що
підпадають оподаткуванню (фармерів і деяких інших рубрик сюди
не включено), становив для Великобританії від 1853 до 1864 р.
50,47\% (або 4,58\% пересічно за рік),\footnote{
«Tenth Report of the Commissioners of H. M’s. Inland Revenue».
London 1866, p. 38.
} приріст людности протягом
того самого періоду — приблизно 12\%. Збільшення земельних
рент, що підпадають оподаткуванню (сюди належать
будинки, залізниці, копальні, рибальство й т. ін.), становило
від 1853 до 1864 р. 38\%, або 3\sfrac{5}{12}\% річно, при чому найдужче
збільшення припадало на такі рубрики:
}............... 1,216\%
}...............  1,326\%
}............... 1,446\%
}............... 1,533\%