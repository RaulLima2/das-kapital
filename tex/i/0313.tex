розвиток автоматичної системи та щораз неминучіше вживання
матеріялу, який важко переробити, наприклад, заліза замість
дерева, — розв’язання всіх цих завдань, що виникали стихійно,
натрапляло всюди на особисті перешкоди, які навіть комбінований
робітничий персонал мануфактури міг усунути лише
до певного ступеня, але не по суті. Таких машин, як от, наприклад,
сучасний друкарський прес, сучасний паровий ткацький
варстат та сучасна чухральна машина, не могла дати
мануфактура.

Переворот у способі продукції в одній сфері промисловості
зумовлює переворот у способі продукції і в інших сферах. Це
має силу насамперед для таких галузей промисловости, які,
хоч суспільним поділом праці так ізольовані, що кожна з них
продукує якийсь самостійний товар, проте переплітаються одна
з однією як фази одного якогось цілого процесу. Так, машинове
прядіння зробило доконечним машинове ткання, а одне й друге
разом — механічно-хемічну революцію в білінні, друкуванні
та фарбуванні. Таким саме чином, з другого боку, революція в
прядінні бавовни зумовила винахід gin’a, машини до відділювання
бавовняних волокон від насіння, через що лише й зробилася
можлива продукція бавовни в потрібному тепер великому маштабі.\footnote{
Cottongin,\footnote*{
Машина, що вибирає зерно з бавовни. \emph{Ред.}
} винайдений одним янкі, Елія Вайтнеєм, до найновіших
часів у головному зазнав менше змін, ніж якабудь інша машина XVIII віку.
Лише останніми десятиліттями (перед 1867 р.) другий американець, пан
Імрі з Альбані, в Нью-Йорку, за допомогою простого й доцільного
поліпшення зробив машину Вайтнея антикварною річчю.
}
А революція у способі продукції промисловости і рільництва
зробила доконечною й революцію в загальних умовах суспільного
процесу продукції, тобто в засобах комунікації і транспорту.
Як засоби комунікації й транспорту суспільства, що його
pivot,\footnote*{
Точка, що довкола неї все обертається, стрижень. \emph{Ред.}
} уживаючи вислову Фур’є, було дрібне рільництво з його
домашньою допомічною промисловістю та міське ремество, уже
ніяк не могли задовольняти потреб продукції мануфактурного
періоду з його поширеним поділом суспільної праці, з його концентрацією
засобів праці та робітників і з його колоніальними
ринками, — а тому й дійсно зазнали перевороту, — так само й
засоби транспорту й комунікації, що перейшли від мануфактурного
періоду, перетворились незабаром у нестерпні пута для
великої промисловости з її гарячковим темпом продукції, з її масовими
розмірами, з її постійним перекидуванням мас капіталу й
робітників з однієї сфери продукції до іншої та з її новостворюваними
світовими ринковими зв’язками. Тим то, не кажучи вже

нього було дійсно таки дві ноги, які він навпереміну підносив, як кінь.
Тільки з дальшим розвитком механіки та з нагромадженням практичного
досвіду форма машини починає цілком визначатися принципами механіки
і тим то цілком емансипується від давньої форми того знаряддя, яке розвивається
на машину.