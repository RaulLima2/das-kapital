В кожній країні є якась певна пересічна інтенсивність праці;
праця, що має інтенсивність нижчу від цієї пересічної, споживає
на продукцію товару часу більше за суспільно-доконечний,
і через це не вважається за працю нормальної якости. Тільки
такий ступінь інтенсивности, що підноситься понад національну
пересічну, зміняє в даній країні міряння вартости простим триванням
часу праці. Інакше справа стоїть на світовому ринку,
що його інтеґральними частинами є поодинокі країни. Пересічна
інтенсивність праці зміняється від країни до країни: тут вона
більша, там менша. Отже, ці національні пересічні становлять
скалю, що її одиницею міри є пересічна одиниця світової праці.
Отже, інтенсивніша національна праця порівняно з менш інтенсивною
продукує за однаковий час більше вартости, яка виражається
в більшій кількості грошей.

Але закон вартости в його інтернаціональному застосуванні
ще більше модифікується через те, що на світовому ринку продуктивнішу
національну працю також вважається за інтенсивнішу,
доки продуктивнішу націю конкуренцією не примушують
знизити продажну ціну її товару до його вартости.

В тій мірі, у якій в країні розвинута капіталістична продукція,
в тій самій мірі там підносяться національна інтенсивність і
продуктивність праці понад інтернаціональний рівень.64а Отже,
різні кількості товарів того самого роду, що їх у різних країнах
продукується за однаковий робочий час, мають неоднакові інтернаціональні
вартості, які виражаються в різних цінах, тобто
в різних грошових сумах, залежно від інтернаціональних вартостей.
Отже, відносна вартість грошей буде менша в нації з
розвинутішим капіталістичним способом продукції, ніж у нації
з менш розвинутим капіталістичним способом продукції. Звідси
випливає, що номінальна заробітна плата, тобто еквівалент робочої
сили, виражений у грошах, теж буде вища в першої нації,
ніж у другої: але це зовсім не означає, що те саме має силу і
для реальної плати, тобто і для суми засобів існування, даних
у розпорядження робітникові.

Але навіть залишаючи осторонь цю відносну неоднаковість
вартости грошей у різних країнах, ми все ж часто бачитимемо,
що поденна, потижнева і т. д. заробітна плата в нації з розвинутішим
капіталістичним способом продукції вища, ніж у нації
з менш розвинутим капіталістичним способом продукції, тимчасом
як відносна ціна праці, тобто ціна праці у відношенні
так до додаткової вартости, як і до вартости продукту, у другої
нації стоїть вище, ніж у першої.65

64а В іншому місці ми дослідимо, які обставини можуть щодо продуктивности
праці зміните цей закон для поодиноких галузей продукції.

65 Джемс Андерсон у полеміці проти А. Сміса зауважує: «Варто
також взяти на увагу, що хоч позірна ціна праці звичайно нижча в бідних
країнах, де продукти землі і взагалі збіжжя дешеві, але в дійсності
реальна ціна праці там здебільша вища, ніж у інших країнах. Бо не та
плата, яку дістає робітник за день праці, становить дійсну ціну праці,
хоч вона й є її позірна ціна; справжня ціна — це те, чого фактично коштує
