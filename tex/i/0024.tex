Але того факту, що у формі вартости товарів усяку працю виражається
як рівну людську працю й тим то як рівнозначну, —
цього факту Арістотель не міг вичитати з самої форми вартости,
бо грецьке суспільство ґрунтувалося на рабській праці, і тому
мало за природну базу нерівність людей і їхніх робочих сил.
Таємницю виразу вартости, рівність і рівнозначність усіх праць,
тому що й оскільки вони є людська праця взагалі, — цю таємницю
можна розшифрувати тільки тоді, коли поняття людської рівности
має вже тривалість народнього забобону. А це можливе лише в
такому суспільстві, де товарова форма є загальна форма продукту
праці, де, отже, і відношення людей між собою як посідачів товарів
є панівне суспільне відношення. Геній Арістотеля виявляється
саме в тому, що він у виразі вартости товарів одкриває відношення
рівности. Лише історичні рамки суспільства, що в ньому він жив,
заважали йому відкрити те, в чому ж таки «справді» це відношення
рівности.

4. Проста форма вартости в цілому

Проста форма вартости товару міститься в його вартостевому
відношенні до якогось одною товару іншого роду, або в його міновому
відношенні до цього останнього. Вартість товару А якісна
виражається через властивість товару В безпосередньо обмінюватись
на товар А. Кількісно вона виражається через здатність
певної кількости товару В обмінюватись на дану кількість товару
А. Іншими словами, вартість товару є самостійно виражена, коли
вона представлена як «мінова вартість». Коли ми на початку цього
розділу згідно із звичайною термінологією казали: товар є споживна
вартість і мінова вартість, то це, точно кажучи, було неправильно.
Товар є споживна вартість, або предмет споживання, і
«вартість». Він виявляється як отака двоїстість, що нею він є,
скоро його вартість має власну форму виявлення, відмінну від
його натуральної форми, а саме форму мінової вартости; розглядуваний
ізольовано, він ніколи не має цієї форми, а має її завжди
у вартостевому відношенні, або у міновому відношенні до якогось
другого відмінного від нього товару. Однак, скоро ми це вже
знаємо, то з такими висловами не буде нам мороки, а придаватимуться
вони для скорочення.

Наша аналіза показала, що форма вартости або вираз вартості
товару випливає з природи вартости товару, а не навпаки — що
вартість і величина вартости випливають з їхнього способу виразу
як мінової вартости. Однак саме таку помилку роблять так меркантилісти
й їхні сучасні прихильники, як от Фер’є, Ґаніль і
і т. ін.,\footnote{
Примітка до другого видання. F. С. А. Ferner (sous-inspecteur
des douanes): «Du Gouvernement considéré dans ses rapports avec
le commerce», Paris 1805 і Charles Ganilh: «Des Systèmes de l’Economie
Politique», 2-eme éd. Paris 1821.
} так і їхні антиподи — сучасні комівояжери вільної торговлі,
як от Бастіа й компанія. Меркантилісти переміщують