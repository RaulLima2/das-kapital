\parcont{}  %% абзац починається на попередній сторінці
\index{i}{0135}  %% посилання на сторінку оригінального видання
продукти як засоби продукції продуктів. Але як процес праці
первісно відбувається лише між людиною й землею, що існує
без допомоги людини, так і тепер у ньому все ще функціонують
і такі засоби продукції, які дані самою природою й не являють
собою сполуки речовини природи з людською працею.

Процес праці, як ми його проаналізували в його простих і абстрактних
моментах, є доцільна діяльність для виготовлення
споживних вартостей, пристосування природних речовин до
людських потреб, загальна умова обміну речовин між людиною
й природою, вічна природна умова людського життя, і тому він
є незалежний від усякої форми цього життя, а, навпаки, однаково
спільний усім його суспільним формам. Тим-то ми не мали
потреби розглядати робітника в його відношенні до інших робітників.
Людина та її праця на одному боці, природа та її речовини
на другому — цього було досить. Як мало зі смаку пшениці
можна пізнати, хто її виростив, так само мало видно з цього процесу
праці, за яких умов він відбувається: чи під брутальним
канчуком доглядача рабів, чи під полохливим оком капіталіста,
чи виконує його Цінцінат, обробляючи свої декілька юґерів,
чи дикун, що каменем забиває звіра.\footnote{
Виходячи з цієї високологічної основи, полковник Торренс відкриває
в камені дикуна — початок капіталу. «В першому камені, що ним
дикун кидає в звіра, за яким він ганяється, в першій ломаці, яку він бере,
щоб пригнути овочі, яких він не може дістати руками, ми бачимо присвоєння
одного предмету з метою здобути інший предмет і таким чином
відкриваємо початок капіталу». (R. Torrens: «An Essay on the
Production of Wealth etc.», Edinburgh 1836, p. 70, 71). Цією першою
ломакою (stock), мабуть, і пояснюється те, чому в англійській мові stock
є синонім капіталу.
}

Вернімось до нашого капіталіста іn spе.\footnote*{
— у сподіванці. Ред.
} Ми лишили його
після того, як він на товаровому ринку купив усі доконечні
для процесу праці фактори: предметні фактори, або засоби продукції,
і особистий фактор, або робочу силу. Хитрим оком знавця
він добрав собі засоби продукції й робочу силу, придатні для
його окремого підприємства; прядіння, фабрикації чобіт і т. ін.
Отже, наш капіталіст заходжується коло споживання купленого
ним товару, робочої сили, тобто він примушує носія робочої сили,
робітника, споживати за допомогою його праці засоби продукції.
Загальний характер процесу праці, звичайно, не змінюється від
того, що робітник виконує його для капіталіста, а не для себе.
Але й той певний спосіб, яким роблять чоботи або прядуть пряжу,
так само не може відразу змінитися через втручання капіталіста.
Останній мусить спочатку взяти робочу силу такою, якою він
находить її на ринку, отже, і працю її мусить узяти такою, якою
вона склалася того часу, коли ще не було капіталістів. Зміна
самого способу продукції через підпорядкування праці капіталові
може статися тільки пізніш, і тому її слід розглянути
тільки пізніш.
