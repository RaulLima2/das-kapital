\index{i}{0478}  %% посилання на сторінку оригінального видання
\chapter{Процес акумуляції капіталу}

Перетворення певної грошової суми на засоби продукції та
робочу силу є перший рух, що його пророблює певна кількість
вартости, яка повинна функціонувати як капітал. Цей рух відбувається
на ринку, у сфері циркуляції. Друга фаза руху, процес
продукції, закінчується, скоро тільки засоби продукції
перетворено на товари, вартість яких перевищує вартість їхніх
складових частин, отже, містить у собі первісно авансований
капітал плюс додаткова вартість. Ці товари треба потім знову
кинути у сферу циркуляції. Треба їх продати, зреалізувати їхню
вартість у грошах, ці гроші знову перетворити на капітал і
так знову й знову повторювати цей процес. Цей кругобіг, що
постійно пророблює ті самі послідовні фази, становить циркуляцію
капіталу.

Перша умова акумуляції капіталу та, щоб капіталістові вдалося
продати свої товари та знову перетворити на капітал найбільшу
частину одержаних таким чином грошей. У дальшому
викладі ми припускаємо, що капітал перебігає свій процес нормальним
способом. Ближча аналіза цього процесу належить до
другої книги.\footnote*{
У другому німецькому виданні цей початок розділу викладена
так: «Ми бачили, як капітал продукує додаткову вартість у формі товару.
Лише через продаж товару реалізується вміщена в ньому додаткова
вартість разом з капітальною вартістю, авансованою на його продукцію.
Тому процес акумуляції капіталу має за передумову процес його циркуляції.
Розгляд цього останнього ми відкладаємо до наступної книги.
Реальні умови репродукції, тобто безперервної продукції з являються
почасти лише в сфері циркуляції, а почасти вони можуть бути досліджені;
лише після аналізи процесу циркуляції. Однак це ще не все». \emph{Ред.}
}

Капіталіст, що продукує додаткову вартість, тобто висисає
неоплачену працю безпосередньо з робітників та фіксує її в товарах,
є, правда, перший присвоювач, алеж він зовсім не є
останній власник цієї додаткової вартости. Він мусить потім
поділитися нею з тими капіталістами, що виконують інші функції
в цілій суспільній продукції, з земельним власником і т. ін.
Тому додаткова вартість розпадається на різні частини. Її частини
припадають різним категоріям осіб та набирають одна проти
одної різних самостійних форм, як от зиск, процент, торговельний
зиск, земельна рента і т. ін. Ці перетворені форми додаткової
вартости можна буде розглянути лише в третій книзі.
