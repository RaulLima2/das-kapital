явище, властиве всім способам продукції й що на ньому ми на
хвилину спішимося в аналізі процесу циркуляції.*

Отже, в цьому розумінні клясична політична економія має
рацію, коли вона підкреслює як характеристичний момент процесу
акумуляції те, що додатковий продукт мусить споживатись
продуктивними робітниками, а не непродуктивними. Але тут починається
й її помилка. А. Сміт завів моду малювати акумуляцію як
просте споживання додаткового продукту продуктивними робітниками,
або малювати капіталізацію додаткової вартости як просте
перетворення її на робочу силу. Послухаймо, наприклад, Рікарда:
«Треба зрозуміти, що всі продукти країни споживаються; але величезна
ріжниця, яку тільки можна собі уявити, є в тому, чи споживаються
вони тими, що репродукують якусь іншу вартість, чи тими,
що її не репродукують. Коли ми кажемо, що дохід заощаджується
й додається до капіталу, то ми розуміємо під цим, що ту частину
доходу, про яку кажуть, що її додається до капіталу, споживається
продуктивними, а не непродуктивними робітниками. Немає
більшої помилки, як припускати, що капітал збільшується через
неспоживання».30 Немає більшої помилки, як та, що її за
А. Смітом проказують Рікардо і всі пізніші економісти, а саме,
що «ту частину доходу, про яку кажуть, що її додається до капіталу,
споживається продуктивними робітниками». За цим уяввленням
вся додаткова вартість, що перетворюється на капітал,
ставала б змінним капіталом. Навпаки, вона, як і первісно авансована
вартість, поділяється на сталий капітал і змінний капітал,
на засоби продукції й робочу силу. Робоча сила є та форма, що
в ній змінний капітал існує в процесі продукції. В цьому процесі
її саму споживає капіталіст. Вона ж своєю функцією, працею,
споживає засоби продукції. Одночасно гроші, заплачені при
купівлі робочої сили, перетворюються на засоби існування, що
їх споживає не «продуктивна праця», а «продуктивні робітники».
За допомогою аналізи, цілком хибної в своїй основі, А. Сміс
доходить такого недоладного результату, що хоч кожний індивідуальний
капітал і поділяється на сталу і змінну складову
частину, все ж суспільний капітал сходить лише на змінний
капітал, або його витрачають лише на виплату заробітної плати.
Нехай, наприклад, фабрикант сукна перетворює 2.000 фунтів стерлінґів
на капітал. Одну частину цих грошей він витрачає на купівлю
ткачів; другу — на купівлю вовняної пряжі, машин і т. д.
Але люди, що в них він купує пряжу й машини, знову оплачують
частиною з тих грошей працю і т. д., поки всі 2.000 фун-

30 Ricardo: «Principles of Political Economy», 3rd, ed. London
1821, p. 163, примітка.

* У французькому виданні це речення подано так: «Звичайний спосіб
вислову сплутує також капіталістичну акумуляцію, що є процес
продукції, з двома іншими економічними явищами, а саме: з нагромадженням
у споживному фонді багатіїв дібр, які споживаються лише
повільно, та з творенням запасів споживання — явищем, властивим усім
способам продукції». («Le Capital etc.», v. I, ch. XXIV, p. 257). Ред.
