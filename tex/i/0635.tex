Яків І: особу, що тиняється й жебрачить вважається за волоцюгу.
Мирові судді в Petty Sessions* уповноважені віддавати
таких осіб на прилюдну кару батогами й замикати їх у в’язниці
на шість місяців, спіймавши їх перший раз, і на два роки, спіймавши
їх удруге. Підчас ув’язнення їх слід карати батогами так часто
й так багато, як це вважають за відповідне мирові судді... Непоправних
і небезпечних волоцюг слід таврувати, випікаючи
їм на лівому плечі літеру «R», і вживати до примусових праць,
а коли їх іще раз спіймають на жебрацтві — нещадно карати
на смерть. Ці постанови мали силу аж до початку XVIII віку,
скасовано їх тільки актом 12 Анни с. 23.

Подібні закони були й у Франції, де в середині XVII століття
утворилось у Парижі так зване «королівство волоцюг»
(royaume des truands). Ще на початку королювання ЛюдовікаХІV
(ордонанс від 13 липня 1777 р.) кожну здорову людину між 16 і
60 роками засилали на ґалери, коли вона не мала засобів існування
й певної професії. Подібні постанови є в статуті Карла V
для Нідерляндів від жовтня 1537 р., перший едикт штатів і міст
Голляндії від 19 березня 1614 р., плякат Сполучених Провінцій
від 25 червня 1649 р. і т. д.

Таким чином, сільську людність, силоміць позбавлену землі,
вигнану й перетворену на волоцюг, за допомогою жахливо терористичних
законів, батогами, тавруванням і катуванням привчили
до дисципліни, доконечної для системи найманої праці.

Мало того, що умови праці виступають на одному полюсі як
капітал, а на другому полюсі як люди, що не мають на продаж
нічого, крім своєї власної робочої сили. Недосить також і того,
що їх примушують добровільно продавати себе. З розвитком капіталістичної
продукції розвивається робітнича кляса, що в наслідок
свого виховання, традиції, звичок визнає вимоги цього способу
продукції за само собою зрозумілі закони природи. Організація
розвиненого капіталістичного процесу продукції ламає
всякий опір; постійне створювання відносного перелюднення
тримає закон попиту й подання, а тим то й заробітну плату
в межах, відповідних до потреби самозростання капіталу; німий
гніт економічних відносин закріпляє панування капіталіста над

таки примушують красти, «за королювання Генріха VIII покарано на
смерть 72.000 великих і малих злодіїв». (Hollinshed: «Description of
England», vol. I, p. 186). За часів Єлисавети волоцюг вішали цілими лавами;
звичайно не проходило року, щоб там або деінде не повісили 300
або 400 осіб». (Strype: «Annals of the Reformation and Establishment
of Religion, and other Various Occurrences in the Church of England
during Queen Elisabeth’s Happy Reign», 2 nd ed. 1725, vol. II). За тим самим
Стріпом в Сомерсетшірі за один лише рік покарано на смерть 40 осіб,
потавровано 35, покарано батогами 37, а 183 «очайдушних злочинців»
випущено на волю. Однак, каже, він, «у це велике число обвинувачених,
у наслідок недбальства мирових суддів і безглуздого співчуття
з боку народу, не входить і п’ятина гідних кари злочинців». Він додає:
«Інші графства Англії були не у кращому становищі, ніж Сомерсетшір,
а багато навіть у гіршому».

* — малих сесіях. Ред.
