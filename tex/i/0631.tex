призначений для полювання, у багатьох випадках дає куди
більш зиску, ніж пасовисько для овець... Аматор, що шукає
місця для полювання, дає за нього таку ціну, яку тільки дозволяє
йому глибина його кишені. На гірські місцевості звалилося горе
не менш жорстоке від того, що його зазнала Англія через політику
норманських королів. Для оленів і сарн приділяють дедалі
більше простору, тимчасом як людей заганяють у дедалі тісніше
коло... У народу відбирають одну вільність по одній... І поневолення
день-у-день зростає. «Очищування» й проганяння
народу власники практикують як твердий принцип, як сільськогосподарську
доконечність, цілком так само, як по диких місцевостях
Америки й Австралії викорінюють дерева та кущі, і ця
операція проходить спокійно, діловито».220

220    Robert Sommers: «Letters from the Highlands; or the Famine
of 1847», London 1848, p. 12—28 і далі. Ці листи появилися спочатку
в «Times’i». Англійські економісти, звичайно, пояснювали голод серед
ґаелів у 1847 році — перелюдненням. У всякому разі, ґаели, мовляв,
«натискували» на свої засоби існування. — «Clearing of Estates», або,
як воно називалося у Німеччині, Bauernlegen, розвинулось у Німеччині
з особливою силою після тридцятилітньої війни і ще в 1790 р. викликало
селянські повстання в саксонському курфюрстві. Воно панувало особливо
у східній Німеччині. В більшості пруських провінцій тільки Фрідріх II
забезпечив селянам право власности. Здобувши Шльонськ, він примусив
землевласників відбудувати хати, клуні тощо й забезпечити селянські
господарства худобою та знаряддям. Йому потрібні були солдати для
його армії і платники податків для його державної скарбниці. А в тім
як приємно жилося селянам за Фрідріха II з його фінансовою політикою
та системою урядування, цією мішаниною деспотизму, бюрократизму
й февдалізму, можна побачити з ось яких слів його великого прихильника
Мірабо: «Одним із головних багатств рільника північної Німеччини
є льон. Та, на нещастя для роду людського, це тільки знаряддя проти
злиднів, а не шлях до добробуту. Безпосередні податки, панщина й інші
февдальні повинності всякого роду руйнують німецького селянина, який
ще до того платить посередні податки на все, що купує... у довершення
його руйнації він не сміє продавати своїх продуктів, де і як захоче; він
не сміє купувати потрібні йому продукти в тих купців, що продавали б
їх йому за найдешевшу ціну. Всі ці причини непомітно руйнують його,
і він не був би спроможний платити в строк безпосередніх податків,
коли б не займався прядінням; останнє являє собою для нього підмогу,
бо дає корисне заняття його дружині, дітям, слугам, наймитам і йому
самому. Але яке ж це злиденне життя, навіть з отією підмогою! Влітку
підчас оранки та жнив він працює, як каторжник; він лягає о дев’ятій
годині і встає о другій, щоб тільки упоратися з своєю працею; взимку
він мусів би відживлятися, маючи більше спочинку; але йому не вистачить
збіжжя на хліб і на насіння, коли він продасть харчові продукти,
щоб сплатити податки. Отже, він мусить прясти, щоб заткати цю діру...
мусить прясти з найбільшою пильністю та запопадливістю. Таким чином
селянин узимку лягає опівночі або о першій годині і встає о п’ятій-шостій
удосвіта; або він лягає о дев’ятій і встає о другій — і так день-у-день
ціле своє життя, за винятком неділь... Це надмірно довге неспання й ця
надмірна праця виснажують організм людини; ось чому на селі чоловіки
й жінки старіються далеко швидше, ніж по містах». («Le lin fait donc
une des grandes richesses du cultivateur dans le Nord d’Allemagne. Malheureusement
pour l’espèce humaine, ce n’est qu’une ressource contre la
misère, et non un moyen de bien-être. Les impôts directs, les corvées, les
servitudes de tout genre, écrasent le cultivateur allemand, qui paie encore
les impôts indirects dans tout ce qu’il achète... et pour comble de ruine, il
