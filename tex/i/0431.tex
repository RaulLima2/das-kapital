рослинністю, а помірна смуга є батьківщина капіталу. Не абсолютна
родючість ґрунту, а його диференційованість, різноманітність
його природних продуктів становить природну основу
суспільного поділу праці та через зміну природних умов, серед
яких живе людина, спонукає її до урізноманітнення її власних
потреб, здібностей, засобів праці та способів праці. Доконечність
суспільно контролювати якусь силу природи, економно користуватися
нею, присвоювати її собі або приборкувати у великому
маштабі за допомогою споруд, зроблених людською рукою, —
ось що відіграє найвирішальнішу ролю в історії промисловости.
Наприклад, уреґулювання води в Єгипті,\footnote{
Доконечність обчислювати періоди розливу Ніла створила єгипетську
астрономію, а з нею й панування касти жерців як керівників
рільництва. «Сонцестояння — це той момент року, коли починається
розлив Ніла, і єгиптяни мусили стежити за цим сонцестоянням з особливою
увагою. Для них важливо було встановити цей тропічний рік для
того, щоб реґулювати свої рільничі роботи. Тому вони мусили шукати
на небі виразного знаку його повороту» («Le solstice est le moment de
l’année où commence la crue du Nil, et celui que les Egyptiens ont dû
observer avec le plus d’attention... C’était cette année tropique qu’il leur
importait de marquer pour se diriger dans leurs opérations agricoles. Ils
durent donc chercher dans le ciel un signe apparent de son retour»). (Cuvier:
«Discours sur les révolutions du globe». Ed. Hoefer. Paris 1863, p. 141).
} Льомбардії, Голляндії
й т. ін. Або в Індії, Персії й т. ін., де зрошування штучними
каналами постачає ґрунтові не тільки конче потрібну воду, але
разом з її намулом приставляє з гір мінеральне добриво. Каналізація
— ось у чому була таємниця розцвіту промисловости Еспанії
та Сіцілії під арабським пануванням.\footnote{
Однією з матеріяльних основ державної влади над малими, незв'язаними
між собою продукційними організмами Індії, було реґулювання
водопостачання. Мохаммеданські володарі Індії розуміли це краще,
ніж їхні англійські нащадки. Ми нагадаємо лише про голод 1866 р., що
коштував життя більш ніж мільйонові індусів в окрузі Оріссі Бенґальського
президентства.
}

Сприятливі природні умови завжди дають лише можливість
додаткової праці, алеж ніколи не дійсність додаткової праці,
отже, і додаткової вартости, або додаткового продукту. Різні
природні умови праці призводять до того, що та сама кількість
праці по різних країнах задовольняє різні маси потреб,\footnote{
«Немає двох країн, що давали б однакову кількість доконечних
засобів існування в однаковій достатності та при однакових затратах
} отже,

of London, Merchant, and now published for the common good by his son
John Mun», London 1669, p. 181, 182). «Я також не можу уявити собі
більшого прокляття для народу, як бути закинутим на клапоть землі,
де сама природа рясно продукує засоби існування, а підсоння вимагає
або дозволяє лише мало турбуватися про одяг і житло... Можлива й протилежна
крайність. Ґрунт, що з нього навіть працею не можна вирвати
ніякого продукту, так само недобрий, як і той ґрунт, що рясно родить
без ніякої праці». («Nor can I conceive a greater curse upon a body of people,
than to be thrown upon a spot of land, where the productions for subsistence
and food were, in great measure, spontaneous, and the climate required
or admitted little care for raiment and covering... there may be an extreme
on the other side. A soil incapable of produce by labour is quite as bad as
a soil that produces plentifully without any labour»). («An Inquiry into
the Present High Price of Provisions», London 1767, p. 10).