то він мусив би, за інших незмінних обставин, працювати, як і раніш, пересічно ту саму відповідну
частину дня, щоб спродукувати вартість своєї робочої сили й таким чином придбати засоби існування,
потрібні для його власного існування й постійної репродукції. Але через те, що протягом тієї частини
робочого дня,
коли він продукує денну вартість робочої сили, приміром, 3 шилінґи, він продукує лише еквівалент її
вартости, яку вже виплатив йому капіталіст,\footnoteA{
[Примітка до третього видання. Автор уживає тут загальнопоширеної економічної мови. Нагадаймо,
що на сторінці 126 показано, як у дійсності не капіталіст «авансує» робітника, а робітник
капіталіста. — Ф. Е.].
} отже, лише компенсує новоствореною вартістю
авансовану змінну капітальну вартість, то ця продукція вартости з’являється як проста репродукція.
Отже, ту частину
робочого дня, протягом якої відбувається ця репродукція, я називаю доконечним робочим часом, а
працю, витрачену
за цей час, — доконечною працею.\footnote{
Досі ми вживали в цьому творі слова «доконечний робочий час» на означення того робочого часу, що
взагалі є суспільно-доконечний для продукції якогось товару. Відтепер ми вживатимемо його й щодо
того
робочого часу, який є доконечний для продукції специфічного товару — робочої сили. Вживати тих самих
termini technici\footnote*{
— технічних термінів. \emph{Ред.}
} в різних значеннях незручно, але цілком уникнути цього не можна в жодній науці.
Порівн. приміром, вищу й нижчу математику.
} Доконечною для робітника, бо ця праця не залежить від суспільної
форми його праці.
Доконечною для капіталу й капіталістичного світу, бо база цього
світу є постійне існування робітника.

Другий період процесу праці, протягом якого робітник працює поза межами доконечної праці, щоправда,
коштує йому
праці, витрати робочої сили, але не утворює для нього жодної вартости. Цей період утворює додаткову
вартість, що всміхається до капіталіста принадою створення з нічого. Цю частину робочого
дня я називаю додатковим робочим часом, а витрачену протягом його працю — додатковою працею (surplus
labour). Так само, як для пізнання вартости взагалі має вирішальне значення розглядати її просто як
згусток робочого часу, як лише упредметнену працю, так для пізнання додаткової вартости має
вирішальне
значення розглядати її просто як згусток додаткового робочого часу, як лише упредметнену додаткову
працю. Лише та форма, що в ній цю додаткову працю витискується з безпосереднього продуцента,
робітника, відрізняє економічні суспільні формації, приміром, суспільство рабства від суспільства
найманої праці.30

30 Пан Вільгельм Тукідід Рошер із справді ґотшедівською геніяльністю відкриває, що коли утворення
додаткової вартости або додаткового продукту і сполучену з цим акумуляцію ми завдячуємо нині
«ощадності» капіталіста, який «вимагає за це, приміром, процента», то, навпаки, «на
найнижчих ступенях культури... сильніші приневолюють слабших до ощадности». («Die Grundlagen der
Nationalökonomie». 3 Auflage. 1858, S. 78). До заощаджешшя праці? чи до заощадження надлишкових
продуктів, яких немає? Поруч із справжнім неуцтвом апологетичний страх перед сумлінною аналізою
вартости й додаткової вартости, а також і перед можливістю небезпечного й непринятного для поліції
результату, — ось