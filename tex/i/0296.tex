маси, яке виникає з поділу праці, А. Сміс радить державну
організацію народньої освіти, хоч і в обережно гомеопатичних
дозах. Послідовно полемізує проти цього його французький
перекладач та коментатор Ґ. Ґарньє, що за часів першої французької
імперії, природна річ, перетворився на сенатора. На його
думку, народня освіта суперечить основним законам поділу праці;
завівши її, «ми засудили б цілу нашу суспільну систему». «Як
і всякі інші поділи праці, — каже він, — поділ між працею фізичною
та розумовою 71 стає дедалі виразнішим та рішучішим у
міру того, як багатіє суспільство (він слушно прикладає цей
вислів до капіталу, земельної власности та їхньої держави).
Подібно до всіх інших і цей поділ праці є наслідок минулого та
причина майбутнього проґресу... Невже ж уряд має право протидіяти
цьому поділові праці та спиняти його в природному його
розвитку? Невже він має право частину державних прибутків
витрачати на спробу перемішати та сплутати дві кляси праці,
що прагнуть свого розділу й відокремлення?».72

Деяке розумове й фізичне скалічення є невіддільне навіть від
поділу праці всередині суспільства як цілости. Але що мануфактурний
період проводить це суспільне розчленування галузей
праці значно далі і що, з другого боку, цей період з властивим
йому поділом праці вперше захоплює життя індивіда в самому
його корені, то він також уперше дає матеріял та поштовх до
промислової патології.73

«Ділити людину — це значить карати її на смерть, якщо вона

праці, А. Сміс цілком ясно розумів цей пункт. На початку свого твору,
де він ex professo прославляє поділ праці, він лише мимохідь указує на
нього як на джерело суспільних нерівностей. Лише у п’ятій книзі про
державні доходи він репродукує Ферґюсона. В «Misère de la Philosophie»
я вже з’ясував усе потрібне про історичне відношення між Фергюсоном,
А. Смісом, Лемонтеєм та Сеєм у їхній критиці поділу праці; там
я також уперше подав мануфактурний поділ праці як специфічну форму
капіталістичного способу продукції (стор. 122 і далі) («Злиденність філософії»
, Партвидав 1932 р., стор. 113 і далі).

71    Вже Ферґюсон каже («History of Civil Society», part IV, sect. I.
p. 281): «І саме думання в цьому віці поділу праці стає осібною професією»
(«and thinking itself, in this age of separations, may become a peculiar
craft»).

72    G. Garnier, т. V його перекладу, стор. 2—5.

73    Рамацціні, професор практичної медицини в Падуї, опублікував
1713 р. свою працю «De morbis artificum», перекладену 1781 p.
французькою мовою і знову передруковану 1841 р. в «Encyclopédie des
Sciences Médicales, 7 ème Discours: Auteurs Classiques». Період великої
промисловости, природно, дуже збільшив його каталог робітничих хороб.
Дивись, між іншим, «Hygiène physique et morale de l’ouvrier dans les
grandes villes en général, et dans la ville de Lyon en particulier. Par le
Dr. A. L. Fonterel», Paris 1858 і «Die Krankheiten, welche verschiedenen
Ständen, Altern und Geschlechtern eigentümlich sind». 6 Bände, Ulm I860.
1854 p. Society of Arts призначило слідчу комісію щодо промислової
патології. Реєстр зібраних цією комісією документів можна найти в каталозі
«Twickenham Economic Museum». Дуже важливі офіціальні «Reports
on Public Health». Див. також Eduard Reich, M. D.: «Ueber
die Entartung des Menschen», Erlangen 1868.
