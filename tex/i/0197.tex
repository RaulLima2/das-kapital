«Наші «білі раби», — вигукнув «Morning Star», орган панів
фритредерів Кобдена й Брайта, — наші білі раби запрацьовуються
на смерть і гинуть і вмирають без найменшого шуму».\footnote{
«Morning Star» з 23 липня 1863 р. «Times» скористався цим випадком
для оборони американських рабовласників проти Брайта й т. ін.
«Дуже багато з нас, — каже «Times», — гадають, що лоти, доки ми сами
вимучуємо на смерть працею наших власних молодих жінок, погрожуючи
їм ударами голоду замість свисту батога, доти ми ледве чи маємо право
йти мечем і вогнем на ті родини, що їхні члени родилися рабовласниками
та які принаймні добре годують своїх рабів і вимагають від них лише
помірної праці» («Times», а 2 липня 1863 р.). Газета торів «Standart»
розправлялась у тому самому дусі з його преподобієм Ньюмен Холлом:
«Він відлучує від церкви рабовласників, але молиться разом із порядними
людьми, що примушують працювати за собачу плату лондонських візників
та кондукторів омнібусів і т. ін. лише по 16 годин на день». Нарешті,
пролунав голос оракула, винахідника культу генія, пана Томаса Карлейля,
про якого я вже року 1850 писав: «Геній пішов к чорту, лишився культ».
В коротенькій притчі він зводить єдину величну подію сучасної історії,
американську громадянську війну, на те, що Петро з півночі з усіх сил
намагається переломити черепа Павлові з півдня, бо Петро з півночі
наймає свого робітника «поденно», а Павло з півдня — «на ціле життя».
(«Macmillan’s Magazine». Ilias Americana in nuсе. Серпневий зошит
1863 р.). Так луснув, нарешті, шумовинний пухир торійських симпатій
до міських — але ні в якому разі не до сільських! — найманих робітників.
Основа цих симпатій — це рабство!
}

«Запрацьовуватись на смерть — це є порядок дня не лише
в майстернях кравчих, але в тисячах місць, ба на кожному місці,
де справи йдуть добре... Візьмімо як приклад коваля. Як вірити
поетам, то немає в світі людини сильнішої, веселішої за коваля.
Він устає раннім ранком і викрешує іскри перед тим, як засяє
сонце; нема такої людини, що так їла б, так пила б і спала, як
він. Якщо поглянути на долю коваля чисто з фізичного боку, то,
дійсно, за помірної праці, становище його одне з найкращих.
Але ходімо за ним до міста й погляньмо на той тягар праці, який
накладають на його дужі плечі, погляньмо на місце, яке він посідає
у таблицях смертности нашої країни? У Marylebone (один із
найбільших міських кварталів Лондону) смертність ковалів становить
31 на 1000 на рік, а це на 11 перевищує пересічну смертність
дорослих чоловіків Англії. Праця ця, майже інстинктова вмілість
людини, сама по собі бездоганна, через саму лише надмірність
стає руйнаційною для людини. Людина може зробити стільки й
стільки ударів молотом на день, стільки й стільки кроків, стільки
й стільки разів дихнути, стільки й стільки зробити якоїсь роботи
й прожити пересічно, приміром, 50 років. Її примушують робити
на стільки більше вдарів, на стільки більше кроків, стільки частіш
віддихувати, а це все разом збільшує її життєве завдання на
одну четвертину на день. Вона силкується це все робити, а результат
такий, що за обмежений період вона виконує на четвертину
більшу роботу і вмирає на 37 році замість на 50».\footnote{
Dr. Richardson: «Work and Overwork» y «Social Science Review»,
18 липня 1863.
}