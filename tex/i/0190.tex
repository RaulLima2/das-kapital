має на думці: споживання «наших» живих машин, тобто робочу
силу), так що фактично ми (iterum Crispinus) працюємо надробочий
час протягом цілого року... Діти й дорослі (152 дітей і
підлітків нижче 18 років і 140 дорослих) однаково працювали
протягом останніх 18 місяців — пересічно щонайменше по 7 днів
і 5 годин тижнево, або 78\sfrac{1}{2} годин на тиждень. Для 6 тижнів,
кінчаючи 2 травня поточного року (1863), пересічний результат
був вищий — 8 днів, або 84 години на тиждень!» Однак  цей самий
пан Сміс, що відчуває таку велику симпатію до pluralis majestatis,\footnote*{
Множина величности, тобто вживання займенника «ми» замість
«я», як це робили владарі й царі. \emph{Ред.}
}
додає підсміхаючись: «Машинова праця легка». А фабриканти,
що вживають Block Printing, кажуть: «Ручна праця
здоровіша від машинової». Загалом же пани фабриканти з обуренням
висловлюються проти проекту «припиняти машини бодай
на час їжі». Пан Отлей, управитель фабрики шпалер у Боро
(у Лондоні), каже: «Закон, який дозволяв би нам робочий день
від 6 години вранці до 9 години вечора, був би нам (!) дуже до вподоби,
але робочий день від 6 години ранку до 6 години вечора,
який приписує Factory Act, нам (!) не підходить... Ми спиняємо
свою машину підчас обіду (що за великодушність!). Це припинення
не спричинює жодної вартої згадки втрати на папері й фарбах».
«Але, — додає він із співчуттям, — я добре розумію, що втрата,
сполучена з цим, не дуже приємна річ». Звіт комісії наївно гадає,
що страх деяких «видатних фірм» утратити час, тобто час, протягом
якого присвоюється чужу працю, і через те «втратити зиск»,
що цей страх не є ще «достатня основа», щоб дітей, молодших
від 13 років, і молодь до 18 років «позбавляти їжі» протягом
12—16 годин, або щоб їм постачали харч так, як засобам праці
постачають допоміжні матеріяли: машині — воду й вугілля,
вовні — мило, колесам — мастиво і т. ін., тобто підчас самого
процесу продукції.\footnote{
«Children’s Employment Commission, 1863», Evidence, p. 123,
124, 125, 140 і LIV.
}

В жодній галузі промисловости в Англії — (ми лишаємо осторонь
машинове виробництво хліба, яке тільки-но починає прокладати
собі шлях) — не зберігся донині такий стародавній
і навіть, як це можна побачити в поетів з часів Римської імперії,
дохристиянський спосіб продукції, як пекарство. Але капітал,
як зазначено раніш, є спочатку байдужий щодо технічного характеру
того процесу праці, який він опановує. Він бере його
спочатку таким, яким його находить.

Неймовірну фальсифікацію хліба, особливо в Лондоні, відкрив
уперше комітет Палати громад у справі «про фалшування
харчів» (\sfrac{1855}{56} р.) і твір д-ра Гесселя: «Adulterations de-

нормального дня оплачується нижче вартости, так що «надробочий
час» — то лише хитрощі капіталістів, щоб видушити більше «додаткової
праці»; зрештою, це лишається без зміни й тоді, коли робочу силу, вживану
протягом «нормального дня», дійсно оплачується цілком.