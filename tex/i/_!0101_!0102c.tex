\parcont{}  %% абзац починається на попередній сторінці 
\index{i}{*0101}  %% посилання на сторінку оригінального видання 
Ґледстонових слів, якого надано було йому хитромудро ізольованим цитуванням, то тоді Маркс подався
назад під приводом, що в нього немає часу!»

Так ось, значить, у чім річ! І як славно відбилась у продуктивно-кооперативній
фантазії Кембріджу анонімна кампанія пана Брентано в «Concordia». Отакий то він, той святий Юрій
спілки німецьких фабрикантів, отак то він носить свою шпагу, так «майстерно завдає нею вдари», так
що пекельний дракон Маркс «дуже швидко в передсмертних корчах» звивсь у його ніг!

Однак увесь цей малюнок битви в стилі Аріосто служить лише для того, щоб прикрити викрути нашого св.
Юрія. Тут уже й словечка немає про «прибріхування», про «фальсифікацію», а тільки про «хитромудре
ізольоване цитування» (craftily
isolated quotation). Цілу справу було пересунено, і святий Юрій із своїм кембріджським джурою дуже
добре знали чому.

Елеонора Маркс відповідала в місячнику «То-Day», лютий 1884 р., бо «Times» відмовився прийняти її
статтю, і звела полеміку на одним-один пункт: чи Маркс «прибрехав» те речення, чи ні? На це п.
Sedley Taylor відповів: «Питання про те, чи
якесь там речення було в промові Ґледстона, чи його не було», на його думку, «мало дуже другорядне
значення» в суперечці між Марксом і Брентано «порівняно з питанням, чи цитату подано з наміром
передати сенс слів Ґледстона, чи перекрутити
його». Опісля він додає, що в звіті «Times’a» «дійсно є суперечність у словах»; але, але... решта
промови, якщо її правильно зрозуміти, тобто в ліберально-ґледстонівському дусі, указує на те, що́
саме Ґледстон хотів сказати («То-Day», березень 1884 р.). Що найкомічніше тут, так це те, що наш
чоловічок із Кембріджу обстоює тепер цитування промови не за «Hansard’oм», як велить «звичай»,
згідно з анонімом Брентано, а за звітом «Times’a», що його той самий Брентано називає «безумовно
поганим». Ясна річ, аджеж фатального речення немає в «Hansard’і» !

Елеонора Маркс легко розбила дощенту цю арґументацію в тому самому числі «То-Day». Або п. Taylor
читав полеміку з 1872 р. — тоді він тепер «бреше», не лише «прибріхуючи», але й «відбріхуючись»; або
він не читав її — тоді він мусив був
мовчати. У всякому разі було сконстатовано, що він ні на хвилину не зважився підтримувати те
обвинувачення свого приятеля Брентано, що Маркс «прибрехав». Навпаки, тепер уже Маркс, мовляв, не
прибрехав, але примовчав одно важливе речення. Але це саме речення процитовано на стор. 5
inaugural-адреси кількома рядками перед тим, нібито «прибріханим». Щождо «суперечности» в промові
Ґледстона, то хіба ж не сам Маркс говорить на стор. 618 «Капіталу» (стор. 672 третього видання)
(стор. 562 цього українськоговидання), примітка 105, про «постійні кричущі суперечності в бюджетових
промовах Ґледстона з 1863 і 1864 рр.» ! Тільки що
він не зважується à la Sedley Taylor тлумачити їх у ліберальноєлейкуватому
дусі. А далі в кінцевому резюме своєї відповіді Е. Маркс каже: «Навпаки, Маркс ані затаїв нічого, що
варто
\index{i}{*0102}  %% посилання на сторінку оригінального видання 
було б наводити, ані прибрехав будь-чого. Він тільки відновив
і врятував од забуття одне речення з промови Ґледстона, яке безперечно
було сказане, але так чи сяк знайшло спосіб зникнути
геть із «Hansard’а».

І панові Sedley Taylor цього було досить. Результат цієї професорської
кампанії, що тривала протягом двох десятиліть
по двох великих країнах, був такий, що більше вже не зважувалися
нападати на літературну сумлінність Маркса; але від
того часу й п. Sedley Taylor так само мало довірятиме літературним
войовничим бюлетеням пана Брентано, як і пан Брентано
папській непогрішності «Hansard’а».

Фрідріх Енґельс

Лондон, 25 липня 1890 р.
