хоч які будуть її зміст і форма, в суті своїй є витрата людського
мозку, нервів, м’яснів, чуттєвих органів і т. ін. Подруге, що торкається
того, що лежить в основі визначення величини вартости,
часу тривання цієї витрати, або кількости праці, то кількість
праці навіть наочно відрізняється від якости праці. За всіх суспільних
формацій (Zustände, états sociaux) робочий час, що його
коштувала продукція засобів існування, мусив цікавити людей,
хоч і неоднаково на різних ступенях розвитку.\footnote{
Примітка до другого видання. У стародавніх германців величину
одного морґа землі обчислювалось працею одного дня; звідси назва морга
Tagwerk (праця одного дня), (також Tagwanne) (jurnale або jurnalis,
terra jurnalis, jumalis або diornalis), Mannwerk, Mannskraft, Mannsmaad,
Mannshauet і т. ін. Див. Georg Ludwig von Maurer: «Einleitung zur Geschichte
der Mark — Hof usw. Verfassung», München 1859, p. 129 і далі.
} Нарешті, скоро
люди в будь-який спосіб працюють один на одного, то і праця
їхня набуває суспільної форми.

Звідки ж випливає загадковий характер продукту праці,
скоро тільки він набирає форми товару? Ясна річ, із самої цієї
форми. Рівність людських праць набирає речової форми однакової
вартостевої предметности (Wertgegenständlichkeit) продуктів
праці;\footnote*{
У французькому виданні це речення зредаґовано так: «Характер
рівности людських праць набирає форми вартостевої властивости продуктів
праці» («Le caractère d’égalité des travaux acquiert la forme de
a qualité de valeur des produits du travail»).
} виміряння витрати людської робочої сили часом її тривання
набирає форми величини вартости продуктів праці, нарешті,
відносини між продуцентами, що в них виявляються ті
суспільні визначення їхніх праць, набирають форми суспільного
відношення продуктів праці.

Отже, таємність товарової форми є попросту в тому, що форма
ця відбиває людям суспільний характер їхньої власної праці як
предметний характер самих продуктів праці, як суспільні природні
властивості цих речей; тим то й суспільне відношення продуцентів
до всієї сукупної праці вона відбиває як суспільне відношення
предметів, яке існує поза ними. В наслідок такого quid
pro quo продукти праці стають товарами, почуттєво-надпочуттєвими,
або суспільними речами. Так, світловий вплив якоїсь речі
на зоровий нерв виявляється не як суб’єктивне подразнення самого
зорового нерву, а як предметна форма речі, що є поза оком.
Але при баченні світло дійсно кидається однією річчю, зовнішнім
предметом, на другу річ, на око. Це є фізичне відношення між
фізичними речами. Навпаки, товарова форма й вартостеве відношення
продуктів праці, що в ньому вона виявляється, не мають
абсолютно нічого спільного з їхньою фізичною природою і речовими
відношеннями, що з неї випливають. Це є лише певне суспільне
відношення самих людей, яке для них набирає тут фантасмагоричної
форми відношення речей. Тим то, щоб знайти для
цього аналогію, ми мусимо кинутися в туманну царину релігійного
світу. Тут продукти людської голови видаються обдарованими
власним життям, самостійними постатями, які стоять у пев-