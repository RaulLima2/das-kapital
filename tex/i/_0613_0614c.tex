\parcont{}  %% абзац починається на попередній сторінці
\index{i}{0613}  %% посилання на сторінку оригінального видання
(unsound). Лорд тримається фактів. А факт є той, що в міру того,
як меншає кількість ірляндської людности, ірляндські ренти
зростають, що збезлюднений «добродійне» для земельного власника,
отже, і для землі, отже, і для народу, що є лише приналежність
землі. Отож він заявляє, що Ірляндія все ще перелюднена,
і що потік еміґрації пливе все ще занадто поволі. Щоб бути цілком
щасливою, Ірляндія мусить позбутися принаймні ще \sfrac{1}{3} мільйона
робітників. Не думайте собі, що цей, до того всього ще й
поетичний, лорд є лікар із школи Sangrado, який завжди, коли
він не помічав у свого недужого поліпшення, приписував йому
кровоспуск, потім знову кровоспуск, поки, нарешті, в недужого
разом з його кров’ю пропадала і його хороба. Лорд Дюфрен
вимагає нового кровоспуску лише в \sfrac{1}{3} мільйона людей
замість майже 2 мільйонів, кровоспуску, без якого дійсно ніяк
неможливо завести тисячолітнього блаженного царства на Еріні.
Докази подати не важко.
\begin{table}[h]
  \caption*{Число і розмір фарм в Ірляндії 1864 р.}
  \small
  \toprule
  \noindent\begin{tabularx}{\textwidth}{X X X X X X X X}
  \multicolumn{2}{c}{1} & \multicolumn{2}{c}{2} & \multicolumn{2}{c}{3} & \multicolumn{2}{c}{4} \\
  \multicolumn{2}{>{\centering}p{2.5cm}}{\mbox{Фарми не більш} від 1 акра} &
  \multicolumn{2}{>{\centering}p{2.5cm}}{Фарми  від 2 до 5 акрів} &
  \multicolumn{2}{>{\centering}p{2.5cm}}{Фарми від 6 до 15 акрів} &
  \multicolumn{2}{>{\centering}p{2.5cm}}{Фарми від 16 до 30 акрів} \\
  %\parbox{2cm}{\centering
  %one two three four five six}
  \cmidrule(l){1-2}
  \cmidrule(l){3-4}
  \cmidrule(l){5-6}
  \cmidrule(l){7-8}
  Число & Акри & Число & Акри & Число & Акри & Число & Акри \\
  48.653 & 25.394 & 82.037 & 288.916 & 176.368 & 1.836.310 & 136.578 & 3.051.343\\
  \\
  \toprule
  \multicolumn{2}{c}{5} & \multicolumn{2}{c}{6} & \multicolumn{2}{c}{7} & \multicolumn{2}{c}{8} \\
  \multicolumn{2}{>{\centering}p{2.5cm}}{\mbox{Фарми від 31} до 50 акрів} &
  \multicolumn{2}{>{\centering}p{2.5cm}}{Фарми від 51 до 100 акрів} &
  \multicolumn{2}{>{\centering}p{2.5cm}}{Фарми понад 100 акрів} &
  \multicolumn{2}{>{\centering}p{2.5cm}}{\mbox{Загальна площа}} \\
  \cmidrule(l){1-2}
  \cmidrule(l){3-4}
  \cmidrule(l){5-6}
  \cmidrule(l){7-8}
  Число & Акри & Число & Акри & Число & Акри & \multicolumn{2}{c}{Акри} \\
  71.961 & 2.906.274 & 54.247 & 3.983.880 & 31.927 & 8.227.807 & \multicolumn{2}{c}{29.319.924\footnotemark{} % ця мітка у заголовку
 % текст примітки прямо під заголовком
}\\
  \end{tabularx}
\end{table}
\footnotetext{Загальна площа включає також торфовища й пустирі.}

Централізація знищила між 1851 і 1861 рр. переважно фарми
перших трьох категорій — нижче 1 і не вище 15 акрів. Вони
мусять зникнути передусім. Це дає 307.058 «зайвих» фармерів,
або 1.228.232 особи, коли при низькому пересічному обрахунку
покласти 4 особи на родину. При неймовірному припущенні, що
по закінченні революції в рільництві \sfrac{1}{4} з них знову знайде собі
роботу, все ж лишається 921.174 особи, що мусять еміґрувати. Категорії
4, 5 і 6, більші за 15 і не більші за 100 акрів, як це давно
відомо в Англії, занадто дрібні для капіталістичного рільництва,
а для вівчарства це зовсім незначні величини. Отже, при
тому самому припущенні, що й раніш, мусять еміґрувати ще
788.761 особа, разом 1.709.532. А що l’appétit vient en
\index{i}{0614}  %% посилання на сторінку оригінального видання
mangeant,\footnote*{
— апетит приходить під час їди. \emph{Ред.}
} то великі землевласники незабаром відкриють, що Ір-
ляндія із 3\sfrac{1}{2} мільйонами людности все ще бідна країна, а бідна,
тому що перелюднена, отже, збезлюднення її мусить піти ще
значно далі, щоб вона могла виконати своє справжнє призначення
бути за пасовисько для овець і рогатої худоби Англії.\footnoteA{
Як окремі земельні власники й англійське законодавство пляномірно
використовували голод і викликані ним обставини, щоб силоміць
провести революцію в рільництві і звести людність Ірляндії до кількости,
вигідної для лендлордів, це я покажу докладніше у третій книзі
цього твору, у відділі про земельну власність. Там я повернуся й до становища
дрібних фермерів і сільських робітників. Тут я подам лише одну
цитату. Нассау В. Сеніор у своєму посмертному творі «Journals, Conversations
and Essays relating to Ireland». 2 volumes. London 1868,
vol. II, p. 282 каже, між іншим, ось що: «Влучно зауважив д-р Ґ., що в
нас є закон про бідних, і що він є могутнє знаряддя, щоб забезпечити
перемогу лендлордам; друге знаряддя — еміґрація. Жоден друг Ірляндії
не побажає, щоб війна (між лендлордами й дрібними кельтськими
фармерами) тривала далі, — ще менш, щоб вона скінчилась перемогою
фармерів\dots{} Що швидше вона (ця війна) скінчиться, що швидше Ірляндія
перетвориться на пасовиська (grazing country) з порівняно нечисленною
людністю, якої треба для пасовиськ, то краще для всіх кляс». — Англійські
хлібні закони 1815 р. забезпечували Ірляндії монополію вільно довозити
хліб у Великобрітанію. Таким чином вони штучно сприяли рільництву.
У 1846 р. разом із скасуванням хлібних законів одразу знищено
і цю монополію. Не кажучи вже про всі інші обставини, лише цієї
події було досить, щоб надати потужного поштовху перетворенню ірляндської
орної землі на пасовиська, концентрації фарм і вигнанню дрібних
селян. Після того, як протягом 1815--1846 рр. уславляли родючість
ірляндського ґрунту і вселюдно оголосили, що з самої природи
його призначено виключно на культивування пшениці, тепер англійські
аґрономи, економісти, політики раптом зробили відкриття, що він придатний
лише для культивування кормових трав! Пан Леонс де Лявернь
поспішив повторити це по той бік каналу. Треба бути такою «серйозною»
людиною, як пан Лявернь, щоб йняти віри таким наївним теревеням.
}

Ця корисна метода, які і все гарне на цьому світі, має свій
поганий бік. Рівнобіжно з акумуляцією земельної ренти в Ірляндії
ірляндці акумулюються в Америці. Ірляндець, що його виганяють
вівці та бики, з’являється по той бік океану, як феній.
І проти старої владарки морів повстає чимраз грізніш велетенська
молода республіка.

                                      Acerba fata Romanos agunt
                                       Scelusque fraternae necis.\footnote*{
Жене римлян сувора доля і злочин братовбивства. \emph{Ред.}
}

\section{Так звана первісна акумуляція}

\subsection{Таємниця первісної акумуляції}

Ми бачили, як гроші перетворюються на капітал, як за допомогою
капіталу утворюється додаткова вартість, а з додаткової
вартости — додатковий капітал. Але акумуляція капіталу має за
передумову додаткову вартість, додаткова вартість — капіталі-
\parbreak{}  %% абзац продовжується на наступній сторінці
