ношенням між масою застосованих засобів продукції, з одного
боку, і масою праці, потрібного для застосування тих засобів —
з другого. Перший я називаю вартостевим складом капіталу, а
другий — технічним складом капіталу. Між тим і другим існує
щільне взаємовідношення. Щоб висловити це взаємовідношення,
я називаю органічним складом капіталу його вартостевий склад,
оскільки останній визначається його технічним складом і відбиває
зміни технічного складу. Де говориться просто про склад капіталу,
там треба завжди розуміти його органічний склад.

Численні поодинокі капітали, вкладені в певну галузь продукції,
більш або менш відрізняються між собою щодо складу.
Пересіччя їхніх поодиноких складів дає нам склад цілого капіталу
цієї галузі продукції. Нарешті, загальне пересіччя цих пересічних
складів усіх галузей продукції дає нам склад суспільного
капіталу якоїсь країни, і тільки про нього в останній інстанції
й буде далі мова.

Зростання капіталу включає і зростання його змінної, або
перетвореної на робочу силу складової частини. Частина додаткової
вартости, перетвореної на додатковий капітал, мусить завжди
знову перетворюватися на змінний капітал, або на додатковий
робочий фонд. Коли ми припустимо, що разом з іншими незмінними
обставинами незмінним лишається і склад капіталу, тобто,
що завжди потрібно тієї самої маси робочої сили для того, щоб
пустити в рух певну масу засобів продукції, або сталого капіталу,
то в такому випадку попит на працю й фонд засобів існування
робітників, очевидно, зростатиме пропорційно до зросту капіталу,
і то швидше, що швидше зростатиме капітал. Через те, що
капітал щорічно продукує додаткову вартість, частину якої
щороку додається до первісного капіталу, через те, що сам цей
приріст щороку зростає із збільшенням розміру капіталу, який
уже функціонує, і через те, насамкінець, що при особливому збудженні
жадоби до збагачення, як от, наприклад, при відкритті
нових ринків, нових сфер вкладення капіталу в наслідок розвитку
нових суспільних потреб і т. ін., маштаб акумуляції можна раптом
поширити самою лише зміною поділу додаткової вартости
або додаткового продукту на капітал і дохід, — через це потреби
акумуляції капіталу можуть випередити зріст робочої сили або
число робітників, попит на робітників — випередити подання
їх, і тому заробітні плати можуть підвищитися. Це, кінець-кінцем,
навіть мусить статися, якщо вищенаведені передумови й
далі існують без зміни. Через те, що кожного року вживається
більше робітників, ніж у попередньому році, то раніш або пізніш
мусить настати момент, коли потреби акумуляції починають
переростати звичайне подання праці, коли, отже, настає підвищення
заробітної плати. Нарікання на це лунають в Англії
протягом цілого XV й першої половини XVIII століття. Однак
більш або менш сприятливі обставини, при яких наймані робітники
зберігаються й розмножуються, не змінюють нічого в основному
характері капіталістичної продукції. Як проста репро-
