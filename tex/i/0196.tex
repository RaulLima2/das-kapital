плуатувала одна пані з добродушним іменням Еліза. Тут знову
викрито стару історію,88 яку часто розказують, історію про те,
як ці дівчата працюють пересічно по 16\sfrac{1}{2} годин на добу, а в час
сезону часто й 30 годин без перерви, причому їхню «робочу силу»,
що відмовлялася служити, час від часу підтримували хересом,
портвайном і кавою. А був це саме найгарячіший час сезону. Треба
було вмить виготовити для благородних леді розкішні сукні на
баль у честь свіжоімпортованої принцеси Велзької. Мері Енні
Волклей працювала без перерви 26\sfrac{1}{2} годин разом із 60 іншими
дівчатами, по 30 в одній кімнаті, яка ледве чи давала третину
потрібної кубатури повітря; спати ж їм доводилося по дві в одному
ліжку в одній такій задушній конурі, де спальні повідгороджувано
різними дощаними перегородками.89 І це була одна з кращих
модних кравецьких майстерень Лондону. Мері Енні Волклей
занедужала в п’ятницю й померла в неділю, не потурбувавшись
навіть, на велике здивовання пані Елізи, скінчити ще перед тим
останню сукню. Лікар, пан Кейз, покликаний запізно до її смертельної
постелі, такими сухими словами дав свої свідчення перед
«Coroner’s Jury»: «Мері Енні Волклей вмерла від надмірної праці
в переповненій майстерні й від того, що спала в занадто тісній,
погано провітрюваній спальні». Щоб подати лікареві лекцію,
як слід поводитися, «Coroner’s Jury» заявив на те: «Небіжчиця
вмерла від апоплексії, але є підстава побоюватися, що надмірна
праця в переповненій майстерні й т. ін. прискорила її смерть».

88 Порівн. F. Engels: «Die Lage der arbeitenden Klasse in England»,
Leipzig 1845, S. 253, 254. («Становище робітничої класи в Англії»,
Партвидав «Пролетар» 1932 р., стор. 210—212).

89    Д-р Лісбі, лікар при Board of Health, пояснював тоді: «Спальня
дорослої людини мусить мати мінімум 300 кубічних футів повітря, а
мешкальна кімната — мінімум 500 кубічних футів». Д-р Річардсон,
головний лікар однієї лондонської лікарні, каже ось що: «Різні швачки,
модистки, кравчихи й звичайні швачки страждають від потрійного лиха:
надмірної праці, недостачі повітря й недостачі харчів або вад у травленні.
Загалом, така праця за всяких обставин більше личить жінкам, ніж чоловікам.
Але нещастя цього ремества в тому, що воно монополізоване, особливо
ж у столиці, від якихбудь 26 капіталістів, які, користуючися з
усіх ресурсів влади, що випливає з капіталу (that spring from capital),
видушують із праці економію (force economy out of labour; він хотів сказати:
заощаджують на видатках, марнуючи робочу силу). їхню владу
відчуває на собі ціла ця кляса робітниць. Коли кравчисі пощастить придбати
невелике коло замовниць, то конкуренція примушує її дома запрацьовуватись
на смерть, щоб зберегти цих замовниць, і до такої надмірної
праці вона примушена з конечности приневолювати й своїх помічниць.
Коли не пощастить їй у справі, або не зможе вона влаштуватися самостійно,
то вона звертається до якогось підприємства, де працювати доводиться
не менше, але заробіток певніший. Таким чином вона потрапляє у стан
справжньої рабині, яку кидає туди й сюди кожна суспільна хвиля: то вмирає
вона від голоду вдома в маленькій кімнатці, або недалеко їй до цього,
то знов працює по 15, 16 та навіть 18 годин на добу в такому повітрі, що
його ледве можна знести, маючи таку харч, що коли б навіть була й добра,
то її не перетравлював би організм через відсутність свіжого повітря.
Такими жертвами живляться сухоти, що є не що інше, як хороба від недостачі
повітря». (Dr. Richardson: «Work and Overwork» y «Social Science
Review» 18 липня 1863).
