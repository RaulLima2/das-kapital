Розділ одинадцятий
Кооперація

Капіталістична продукція, як ми бачили, починається фактично
лише там, де той самий індивідуальний капітал одночасно експлуатує значне число робітників, отже,
лише там, де процес праці
поширює свій обсяг і постачає продукти у великому маштабі.
Праця значного числа робітників у той самий час, у тому самому
приміщенні (або, як хто хоче, на тому самому полі праці), для продукції того самого ґатунку товарів,
під командою того самого
капіталіста, становить історично й логічно вихідний пункт капіталістичної продукції. Щодо самого
способу продукції мануфактура, приміром, на початках свого розвитку ледве чи відрізняється від
цехової ремісничої промисловости чимось іншим, як
хібащо більшим числом робітників, яких одночасно експлуатує
той самий капітал. Майстерню цехового майстра тільки розширено.

Отже, спочатку ріжниця є лише кількісна. Ми бачили, що
маса додаткової вартости, яку продукує даний капітал, дорівнює
додатковій вартості, яку дає окремий робітник, помноженій на
число одночасно експлуатованих робітників. Це число само по
собі нічого не змінює в нормі додаткової вартости або в ступені
експлуатації робочої сили; щодо продукції товарової вартости
взагалі, то для неї всякі якісні зміни робочого процесу, очевидно,
не мають значення. Це випливає з природи вартости. Якщо один
дванадцятигодинний робочий день упредметнюється в 6 шилінґах, то 1.200 таких робочих днів — у 6
шилінґах X 1.200. В одному випадку в продукті втілилось 12 X 1.200 робочих годин, у другому тільки
12 робочих годин. У продукції вартости велике число
має значення завжди тільки як багато окремих одиниць. Отже, для
продукції вартости немає ніякої ріжниці, чи 1.200 робітників продукують поодиноко, чи спільно під
командою того самого капіталу.

А, проте, в певних межах тут відбувається модифікація. Праця,
упредметнена у вартості, є праця пересічної суспільної якости,
тобто виявлення якоїсь пересічної робочої сили. Але пересічна
величина існує завжди лише як пересічна багатьох різних індивідуальних величин того самого роду. В
кожній галузі промисловости індивідуальний робітник, Петро чи Павло, більш чи менш
відхиляється від пересічного робітника. Ці індивідуальні відхилення, які в математиці називаються
«помилками», компенсуються і зникають, скоро тільки взяти разом велике число робітників. Славетний
софіст і сикофант Едмунд Берк навіть запевняє
на підставі свого практичного досвіду як фармер, що всяка індивідуальна ріжниця праці зникає вже
«для такої незначної групи»,
як 5 рільничих наймитів, отже, п’ять перших-ліпших дорослих
англійських рільничих наймитів протягом того самого часу виконають разом стільки ж праці, як і
яких-будь інших п’ять англійських рільничих наймитів.8 Хоч би й що, а ясно, що сукупний

8 «Безперечно, щодо сили, вправности та сумлінної пильности, існує
велика ріжниця між вартістю праці однієї людини і вартістю праці іншої
