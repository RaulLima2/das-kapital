мешкання, подібні до маленьких комірок без опалення... Переповнення
цих закутків та зумовлена цим переповненням зіпсованість
повітря доходять часто крайнього ступеня. До цього ще
долучається шкідливий вплив стоків, клозетів, гнилі й іншого
бруду, що є звичайна річ при вході до невеликих котеджів». Щодо
помешкань, то «в одній школі мережива 18 дівчат і хазяйка,
35 кубічних футів на кожну особу; в іншій, де нестерпний сморід,
18 осіб, 24\sfrac{1}{2} кубічних фута на людину. Трапляється й таке, що
в цій промисловості вживають до праці дітей 2—2\sfrac{1}{2} років».261

Там, де в сільських графствах Букінґему та Бедфорду припиняється
плетіння мережива, починається плетіння в соломи. Воно
поширене по значній частині Гертфордширу та по західніх і північних
частинах Есексу. 1861 р. коло плетіння з соломи та коло
виготовлення солом’яних брилів працювало 40.043 особи, з них
3.815 чоловічої статі всякого віку, решта — жіночої статі, при
чому 14.913 молодші від 20 років, з них 7.000 дітей. Замість шкіл
мережива з’являються тут «straw plait schools» (школи плетіння
з соломи). Тут дітей починають учити плести з соломи звичайно
від 4 року, іноді між 3 та 4 роком життя. Виховання вони, звичайно,
не дістають ніякого. Сами діти називають початкові школи
«natural schools» (натуральними школами) відмінно від тих кровососних
установ, де їх тримають за працею просто для того, щоб
вони виготовили роботу, наказану їм від їхніх напівзголоднілих
матерів — здебільша 30 ярдів на день. Ці матері потім часто
примушують їх працювати ще вдома до 10, 11, 12 години вночі.
Солома ріже їм пучки й рот, яким вони її постійно змочують.
Згідно з загальним поглядом медичних урядовців Лондону, що
його зрезюмував д-р Беллярд, 300 кубічних футів на кожну
особу становлять мінімум об’єму для спальні або робітної
кімнати. Але в школах плетіння з соломи помешкання ще менші,
ніж у школах мережива, а саме на кожну особу в них припадає
12\sfrac{2}{3}, 17, 18\sfrac{1}{2} і менше ніж 22 кубічні фути. «Менші з цих чисел, —
каже комісар Байт, — дають менший об’єм, ніж половина того,
що його заняла б дитина, коли б її запакувати в коробку з вимірами
по 3 фути кожний». Отакі радощі життя дітей до 12 або
14 років. Бідні, занепалі батьки думають тільки про те, щоб
якомога більше видушити з дітей. Діти, вирісши, звичайно,
зовсім не дбають про своїх батьків та кидають їх. «Немає нічого
дивного в тому, що неуцтво й розпуста панують серед людности,
так вихованої... Її моральність є на щонайнижчому щаблі... Багато
жінок має нешлюбних дітей, і деякі з них у такому недозрілому
віці, що сами знавці кримінальної статистики з дива німіють
перед цим фактом».262 І батьківщина цих зразкових родин є
Англія — зразкова християнська країна Европи, як каже граф
Монталямбер, — людина, певна річ, найкомпетентніша в справах
християнства!
