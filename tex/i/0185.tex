чора, у визначені законом робочі години, їм не дається часу виковувати
такі роботи».55

«Екстра-зиск, що його дає надмірна праця понад визначений
законом час, є, здається, для багатьох фабрикантів надто
велика спокуса, щоб вони могли встояти проти неї. Вони сподіваються
на те, що їх не спіймають, і що, навіть якщо це й
викриється, то незначні грошові кари й невеличкі судові витрати
їм усе ще таки забезпечують бариш».56 «Там, де додатковий
час добувається помноженням дрібних крадіжок («a multiplication
of small thefts») протягом дня, інспектори надибують на
майже непереможні труднощі, коли хочуть подати докази порушення
закону».57 Ці «дрібні крадіжки», що їх робить капітал
коштом обіднього часу й відпочинку робітників, фабричні інспектори
називають іще «petty pilferings of minutes», крадіж хвилин,58
«snatching a few minutes», вривання хвилин59 або, вживаючи технічної
мови робітників, «nibbling and cribbling at meal times».60

Ми бачимо, що в цій атмосфері творення додаткової вартости
додатковою працею не є таємниця. «Коли б ви мені дозволили, —
казав мені якось один дуже поважний фабрикант, — заставляти
працювати щоденно лише на 10 хвилин більше понад визначений

55 «Reports etc. 31 st October 1860», p. 23. З яким фанатизмом, — за
свідченням фабрикантів на суді, — їхні фабричні руки опираються кожній
перерві фабричної праці, показує такий курйоз. На початку липня 1836 р.
суддю в Люсбері (Yorkshire) повідомили про те, що власники вісьмох великих
фабрик біля Batley порушили фабричний закон. Частину цих панів
обвинувачувано в тому, що вони примушували п’ятеро хлоп’ят 11—15 років
працювати від 6 годин ранку в п’ятницю до 4 години після полудня
в суботу, не дозволяючи їм жодного відпочинку, крім часу на їжу й однієї
години на спання опівночі. І ці діти мали виконувати без відпочинку тридцятигодинну
працю в «shoddy-hole», як воно зветься, те пекло, де розсмикується
вовняне ганчір’я й де повітря так насичене пилюгою, відпадками,
що навіть дорослі робітники примушені постійно зав’язувати собі рота
хусточками, щоб захистити свої легені! Пани обвинувачені давали запевнення
замість присяги, — як квакери, вони були занадто скрупульозно-релігійні
люди для того, щоб давати присягу, — що в своєму великому
милосерді вони були б дозволили безталанним дітям спати по 4 години,
але ці вперті діти зовсім не хотіли іти до ліжка! Панів квакерів
позасуджувано на 20 фунтів стерлінґів грошової кари. Драйден передбачав
таких квакерів:

Fox full fraught in seeming sanctity,

That feared an oath, but like the devil would lie,

That look’d like Lent, and had the holy lear,

And durst not sinl before he said his prayer!

[Лисиця повна святощів фалшивих,

Клятьби боїться, але бреше як диявол.

Як піст свята з-під ока визирає

І зроду не грішить, не помолившись].

66 «Reports etc. for 31 st October 1856», p. 34.

57 Там же, стор. 35.

58 Там же, стор. 48. у

5» Там же.

40 Там же.

• — обгризання і обкрадання часу, призначеного для їжі. Ред.
