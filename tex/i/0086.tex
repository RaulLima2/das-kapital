заставу».88 Його потреби безперестанно відновляються й безперестанно
спонукують купувати чужі товари, тимчасом як продукція
й продаж його власного товару коштують часу й залежать
од випадків. Щоб купити, не продаючи, він мусить раніш продати,
не купуючи. Ця операція, проведена в суспільному маштабі,
здається, сама собі суперечить. Однак біля джерел своєї продукції
благородні металі обмінюється безпосередньо на інші товари.
Тут відбувається продаж (з боку посідача товарів) без купівлі
(з боку посідача золота або срібла).89 І пізніші продажі без наступних
купівель упосереднюють лише дальший розподіл благородних
металів серед усіх посідачів товарів. Таким чином на
всіх пунктах циркуляції постають золоті й срібні скарби найрізнішого
розміру. Разом з можливістю затримати товар як
мінову вартість або мінову вартість як товар прокидається й
жадоба на золото. З поширенням товарової циркуляції зростає й
влада грошей, цієї завжди готової до бою абсолютно суспільної
форми багатства. «Золото — чудова річ! Хто його має, той пан
над усім, чого він бажає. За допомогою золота можна навіть
душам відчинити двері в рай» (Колюмб, у листі з Ямайки, 1503 р.).
А що по грошах не пізнати, щó на них перетворено, то на гроші
перетворюється все, — товари й не-товари. Все стає предметом
продажу й купівлі, все можна купити й продати. Циркуляція
стає величезною суспільною ретортою, в яку все впадає, щоб
знову вийти звідти у формі грошового кристалу. Проти цієї
альхемії не можуть устояти навіть мощі святих, не кажучи вже
про менш грубі res sacrosanctae, extra commercium hominum.*90
Як у грошах стирається всяка якісна ріжниця товарів, так гроші
й собі, як радикальний левелер, затирають усяку ріжницю.91

88 «Гроші є застава» («Money is a pledge»). (John Bellers: «Essays
about the Poor, Manufactures, Trade, Plantations and Immorality», London
1699, p. 13).

89 Купівля в категоричному розумінні має за свою передумову, що
золото або срібло є власне вже перетворена форма товару, або продукт
продажу.

90    Генріх III, найхристияніший король Франції, грабує в манастирів
і т. ін. їхні реліквії, щоб перетворити їх на гроші. Відомо, яку ролю
в історії Греччини відіграло пограбування скарбів дельфійського храму
фокійцями. Мешканням бога товарів у стародавніх народів були, як відомо,
храми. Вони були «святими банками». У фінікійців, народу торговельного,
par excellence, гроші вважалося за преображену форму (entäusserte
Form) всіх речей. Отже, було цілком природно, коли дівчата, що в
свята богині кохання віддавалися чужинцям, жертвували богині монету,
одержану ними в нагороду.

91    Gold! yellow, glittering precious gold!
Thus much of this, will make black white; foul, fair,
Wrong, right; base, noble; old, young; coward, valiant.
... What this, you gods! Why this
Will lug jour priests and servants from your sides;
Pluck staut men’s pillows from below their heads.
This yellow slave
Will knit and break religions; bless the accurs'd;
Make the hoar leprosy ador’d; place thieves

* Святі речі, що ними люди не торгують. Ред.
