ретворюється взагалі з безпосереднього продукту індивідуального
продуцента на суспільний, на спільний продукт колективного
робітника, тобто на продукт комбінованого робочого персоналу,
що його члени беруть ближчу або дальшу участь в обробленні
предмету праці. Тому з кооперативним характером самого процесу
праці неодмінно ширшає поняття продуктивної праці та
її носія, продуктивного робітника. Щоб працювати продуктивно,
йому тепер уже не треба самому прикладати рук, а досить бути
органом колективного робітника, виконувати одну якусь його
частинну функцію. Наведене вище первісне визначення продуктивної
праці, виведене з самої природи матеріяльної продукції,
завжди зберігає свою силу для колективного робітника, розглядуваного
як ціле. Але воно вже не має сили для кожного з його
членів, взятого окремо.

Але, з другого боку, поняття продуктивної праці вужчає.
Капіталістична продукція є не тільки продукція товару, вона
з самої суті своєї є продукція додаткової вартости. Робітник
продукує не для себе, а для капіталу. Тому вже недосить того,
що він взагалі продукує. Він мусить продукувати додаткову
вартість. Тільки той робітник продуктивний, що продукує додаткову
вартість для капіталіста, або служить для самозростання
вартости капіталу. Так, шкільний учитель, якщо можна
взяти приклад з-поза сфери матеріяльної продукції, є продуктивний
робітник тоді, коли він не тільки обробляє дитячі голови,
але й себе витрачає, щоб збагатити підприємця. Те, що останній
вклав свій капітал не у фабрику ковбас, а у фабрику навчання,
нічого не змінює в цьому відношенні. Тому поняття продуктивного
робітника ні в якому разі не містить у собі тільки відношення
між діяльністю та корисним ефектом, поміж робітником та продуктом
праці; воно містить у собі ще й специфічно-суспільне,
історично постале продукційне відношення, яке робить робітника
безпосереднім засобом зростання вартости капіталу. Тому бути
продуктивним робітником — це не щастя, а біда. У четвертій
книзі цієї праці,* що розглядає історію теорії, ми ближче побачимо,
що клясична політична економія вже віддавна зробила
продукцію додаткової вартости характеристичною вирішальною
ознакою продуктивного робітника. Тому із зміною її розуміння
природи додаткової вартости змінюється й її визначення продуктивного
робітника. Так, фізіократи заявляють, що тільки
рільнича праця продуктивна, бо тільки вона дає додаткову вартість.
Але для фізіократів додаткова вартість існує виключно
у формі земельної ренти.

Здовження робочого дня поза той пункт, коли робітник спродукував
би лише еквівалент вартости своєї робочої сили, і присвоєння
цієї додаткової праці капіталом — оце є продукція
абсолютної додаткової вартости. Продукція абсолютної додаткової

* Мова йде про теорії додаткової вартости, що їх Маркс гадав видати
як четверту книгу «Капіталу». Ред.
