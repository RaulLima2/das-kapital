у 2.000 фунтів стерлінґів, яка капіталізується. Новий капітал у
2.000 фунтів стерлінґів дає нову додаткову вартість у 400 фунтів
стерлінґів; ця остання знову капіталізується, отже, перетворюється
на другий додатковий капітал, що дає нову додаткову
вартість у 80 фунтів стерлінґів, і т. д.

Ми залишаємо тут осторонь ту частину додаткової вартости,
що її споживає капіталіст. Так само мало цікавить нас у даний
момент те, чи додаткові капітали додаються до первісного капіталу,
чи відокремлюються від нього, щоб самостійно зростати
своєю вартістю; чи використовує їх той самий капіталіст, що
їх нагромадив, чи він передає їх до інших рук. Ми мусимо лише
не забувати, що поруч новоутворених капіталів первісний капітал
і далі репродукує себе й продукує додаткову вартість, і що
те саме має силу для кожного акумульованого капіталу у відношенні
до створеного ним додаткового капіталу.

Первісний капітал утворився в наслідок авансування 10.000
фунтів стерлінґів. Звідки має їх їхній власник? Він їх добув
своєю власною працею і працею своїх предків! — відповідають
нам в один голос представники політичної економії,21с і це їхнє
припущення дійсно здається одним-єдиним, що узгоджується
з законами товарової продукції.

Цілком інакше стоїть справа з додатковим капіталом у 2.000
фунтів стерлінґів. Процес його постання ми знаємо цілком докладно.
Він є капіталізована додаткова вартість. Від самого початку
він не містить у собі жодного атома вартости, що не походив
би з неоплаченої чужої праці. Засоби продукції, до яких долучається
додаткова робоча сила, і так само засоби існування,
з яких вона себе утримує, є не що інше, як інтегральні складові
частини додаткового продукту, отієї данини, яку кляса капіталістів
щорічно вириває в робітничої кляси. Коли кляса капіталістів
за якусь частину цієї данини купує в робітничої кляси
додаткову робочу силу навіть за повну ціну, так що еквівалент
обмінюється на еквівалент, то все таки це давно відома операція
завойовника, що купує у переможених товари за їхні власні,
пограбовані в них гроші.

Якщо додатковий капітал уживає до праці свого власного
продуцента, то цей останній мусить, поперше, і далі збільшувати
вартість первісного капіталу і, крім того, відкуповувати продукт
своєї попередньої праці за більшу працю, ніж той продукт коштував.
Коли розглядати це як оборудку між клясою капіталістів
і робітничою клясою, то справа ані трохи не зміниться,
коли за допомогою неоплаченої праці занятих досі робітників
уживатимуть до праці додаткових робітників. Капіталіст, може,
перетворює додатковий капітал на машину, яка викидає продуцента
цього додаткового капіталу на брук, заміняючи його
кількома дітьми. У всіх випадках робітнича кляса своєю додатне
«Первісна праця, якій його капітал завдячує своє походження»
(«Le travail primitif auquel son capital a dû sa naissance»). (Sismondiz
«Nouveaux Principes d’Economie Politique», éd. Paris, vol. I, p. 109).
