\parcont{}  %% абзац починається на попередній сторінці
\index{i}{0439}  %% посилання на сторінку оригінального видання
продуктивної сили праці ціна робочої сили могла б постійно
спадати разом з одночасним постійним зростанням маси засобів
існування робітника. Але відносно, тобто порівняно з додатковою
вартістю, вартість робочої сили постійно спадала б, і, отже,
ширшала б прірва між життєвим становищем робітника й капіталіста.\footnote{
«Коли постає зміна в продуктивності промисловости, і певна
кількість праці й капіталу продукує більше або менше продуктів, то
відносна величина заробітної плати може значно змінятися, тим часом як
кількість продуктів, що її репрезентує ця відносна величина, лишається
та сама; абож може змінятися кількість, тимчасом як відносна
величина лишається незмінною» («When an alteration takes place in the productiveness
of industry, and that either more or less is produced by a given
quantity of labour and capital, the proportion of wages may obviously
vary, whilst the quantity, which that proportion represents, remains the
same, or the quantity may vary, whilst the proportion remains the same»).
(«Outlines of Political Economy», London 1832, p. 67).
}

Рікардо перший точно зформулював зазначені вище три закони.
Хиби його досліду такі: 1) ті особливі умови, серед яких
мають силу ці закони, він вважає за само собою зрозумілі, загальні
й виключні умови капіталістичної продукції. Він не визнає
ніякої зміни ні в довжині робочого дня, ні в інтенсивності
праці, так що в нього продуктивна сила праці сама собою стає
єдиним змінним фактором; — 2) але, — і це куди більше вносить
помилок у його аналізу, — як і всі інші економісти, він ніколи
не досліджував додаткову вартість як таку, тобто незалежно
від її осібних форм, як от зиск, земельна рента й т. ін. Тому
він закони про норму додаткової вартости безпосередньо скидає
до однієї купи із законами про норму зиску. Як уже сказано,
норма зиску є відношення додатковоі вартости до цілого
авансованого капіталу, тимчасом як норма додаткової вартости
є відношення додаткової вартости лише до змінної частини
цього капіталу. Припустімо, що капітал у 500 фунтів стерлінґів
($С$) поділяється на сировинний матеріял засоби праці й т. ін.,
разом 400 фунтів стерлінґів ($с$) і на 100 фунтів стерлінґів заробітної
плати ($v$); далі, що додаткова вартість дорівнює 100 фунтам
стерлінґів ($m$). Тоді норма додаткової вартости\[
   \frac{m}{v} = \frac{100\text{ фунтів стерлінґів}}{100\text{ фунтів стерлінґів}} = 100\%
\]

Норма ж зиску $ \frac{m}{C} = \frac{100\text{ фунтів стерлінґів}}{500\text{ фунтів стерлінґів}} = 20\%$. Крім того,
ясно, що норма зиску може залежати від обставин, які ні в якому
разі не впливають на норму додаткової вартости. Пізніше, в
третій книзі цього твору, я доведу, що та сама норма додаткової
вартости може виражатися в якнайрізніших нормах зиску, і
різні норми додаткової вартости, при певних обставинах, можуть
виражатись у тій самій нормі зиску.
