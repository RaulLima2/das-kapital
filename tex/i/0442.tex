1) Скорочення робочого дня за даних умов, тобто, коли продуктивність
та інтенсивність праці не змінюються, залишає
вартість робочої сили, а тому й доконечний робочий час незмінним.
Воно зменшує додаткову працю й додаткову вартість. З абсолютною
величиною останньої падає і її відносна величина, тобто
її величина проти незмінної величини вартости робочоі сили.
Тільки через пониження її ціни нижче від її вартости може капіталіст
триматися без втрат.

В усіх звичайних запереченнях проти скорочення робочого
дня виходять з тієї гіпотези, що це явище відбувається при припущених
тут обставинах, тимчасом як у дійсності, навпаки, зміни
продуктивности й інтенсивности праці або передують скороченню
робочого дня, або безпосередньо настають після нього.\footnote{
«Є обставини, що компенсують це... їх вивів на світ десятигодинний
фабричний закон» («There are compensating circumstances... which
the working of the Ten Hour’s Act has brought to light»). («Reports of
Insp. of Fact, for 1 st December 1848», p. 7).
}

2) Здовження робочого дня. Хай доконечний робочий час
буде 6 годин, або вартість робочої сили 3 шилінґи, так само додаткова
праця — 6 годин, або додаткова вартість 3 шилінґи.
Тоді цілий робочий день становить 12 годин і виражається у
продукті вартістю в 6 шилінґів. Якщо робочий день здовжується
на дві години, а ціна робочої сили лишається незмінна, то а
абсолютною зростає й відносна величина додаткової вартости.
Хоч величина вартости робочої сили абсолютно лишається незмінна,
відносно вона спадає. За умов пункту 1) відносна величина
вартости робочої сили не могла змінятися без зміни її абсолютної
величини. Тут, навпаки, відносна зміна величини вартости
робочої сили є результат абсолютної зміни величини додаткової
вартости.

А що новоспродукована вартість, у якій виражається робочий
день, зростає разом з його здовженням, то ціна робочої сили
і додаткова вартість можуть зростати одночасно, чи то на однакову
чи на неоднакову величину. Отже, це одночасне зростання
можливе у двох випадках — при абсолютному здовженні робочого
дня і при ростущій інтенсивності праці без такого здовження.

Із здовженням робочого дня ціна робочої сили може впасти
нижче від її вартости, хоч би номінально ця ціна й лишилася
незмінна або навіть і зросла. Адже ж денну вартість робочої сили,
як ми собі пригадуємо, оцінюється за її нормальним пересічним
триванням або за нормальним періодом життя робітника та за
відповідним, нормальним, властивим людській натурі перетворенням
життєвої субстанції на рух.\footnote{
«Кількість праці, витраченої людиною протягом 24 годин, можна
приблизно визначити, досліджуючи хемічні зміни, що відбулися в її
тілі, бо зміна форм матерії показує на попередню діяльність динамічної
сили» («The amount of labour which a man had undergone in the course
of 24 hours might approximative arrived at by an-examination of the
chemical changes which had taken place in his body, changed forms in
matter indicating the anterior exercise of dynamic force»). (Grove: «On
the Correlation of Physical Forces»).
} До певного пункту збіль-