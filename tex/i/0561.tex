рис соціяльного становища країни, — каже Ґледстон, — що одночасно
зі зменшенням споживної сили народу та збільшенням
статків і злиднів робітничої кляси відбувається постійна акумуляція
багатства у вищих клясах і постійний приріст капіталу».
102 Так говорив цей єлейний міністер у Палаті громад
13 лютого 1843 р. Двадцять років пізніше, 16 квітня 1863 р.,
подаючи на розгляд бюджет, він каже: «Від 1842 р. до 1852 р.
доходи цієї країни, що підпадають оподаткуванню, зросли на
6\%... За вісім років, від 1853 до 1861 р., вони збільшились, якщо
взяти за основу доходи 1853 р., на 20\%. Це такий дивовижний
факт, що він майже неймовірний... Це приголомшливе збільшення
багатства й сили... обмежується геть чисто на заможних
клясах, але... але воно мусить давати посередню користь і робітничій
людності, бо воно здешевлює предмети загального споживання,
— в той час, як багаті стали багатшими, бідні в усякому
разі стали менш бідними. Я не наважуся сказати, що крайності
бідности змінилися».103 Яка слабенька прикінцева частина цього
періоду!

Якщо робітнича кляса лишилася «бідною», тільки «менш
бідною» порівняно з тим «приголомшливим збільшенням багатства
й сили», яке вона спродукувала для кляси власників, то
відносно вона лишилася так само бідною, як і раніш. Якщо
крайності бідности не зменшилися, то вони збільшилися, бо
збільшилися крайності багатства. Щождо подешевшання засобів
існування, то офіціяльна статистика, наприклад, дані лондонського
сирітського дому (Orphan Asylum), показує подорожчання
на 20\% пересічно за три роки 1860—1862 порівняно з роками
1851—1853. В наступні три роки, 1863—1865, проґресивне
подорожчання м’яса, масла, молока, цукру, соли, вугілля й
сили інших доконечних засобів існування. 104 Наступна бюд-

102 Ґледстон у Палаті громад 13 лютого 1843 р.: «It is one of the
most melancholy features in the social state of this country that we see,
beyond the possibility of denial, that while there is at this moment, a decrease
in the consuming powers of the people, an increase of the pressure
of privations and distress; there is at the same time a constant accumulation
of wealth in the upper classes, an increase in the luxuriousness of their
habits, and of their means of enjoyment». («Times», 14 Februar 1843. —
«Hansard», 13 Februar).

103 «From 1842 to 1852 the taxable income of the country increased by
6 per cent... In the 8 years from 1853 to 1861, it had increased from the
basis taken in 1853, 20 per cent! The fact is so astonishing as to be almost
incredible... this intoxicating augmentation of wealth and power... entirely
confined to classes of property... must be of indirect benefit to the
labouring population, because it cheapens the commodities of general consumption
— while the rich have been growing richer the poor have been
growing less poor! at any rate, whether the extremes of poverty are less,
I do not presume to say». Ґледстон у Палаті громад 16 квітня 1863 р.
«Morning Star», 17 квітня.

104    Див. офіціяльні дані в Синій Книзі: «Miscellaneous Statistics
of the United Kingdom». Part VI, London 1866, crop. 260—273 і далі.
Замість статистики сирітських домів за докази могли б служити й деклямації
міністерських журналів, що обстоюють придане для дітей королівського
дому. Там ніколи не забувають про дорожнечу засобів існування.
