та ткач іде на ринок не подарунки робити. Але припустімо, що
продукт його виявив себе як споживна вартість і, отже, товар
притягає гроші. Та постає питання тепер, скільки саме грошей?
Правда, відповідь вже антициповано в ціні товару, в покажчику
величини його вартости. Ми залишаємо осторонь можливі суто
суб’єктивні рахункові помилки посідача товарів, які зараз же
об’єктивно виправляється на ринку. Припустімо, що він витратив
на свій продукт лише пересічний суспільно-доконечний
робочий час. Отже, ціна товару є лише грошова назва упредметненої
в ньому кількости суспільної праці. Але без дозволу нашого
ткача й за його спиною стався переворот в традиційних умовах
ткання полотна. Те, що вчора ще без сумніву було робочим часом,
суспільно-доконечним для продукції одного метра полотна,
сьогодні перестало ним бути, і посідач грошей якнайгарячіше
демонструє це, посилаючись на ціни різних конкурентів нашого
приятеля. На його нещастя, на світі є багато ткачів. Припустімо,
нарешті, що кожний сувій полотна, що є на ринку, містить у
собі лише суспільно-доконечний робочий час. Проте загальна
сума тих сувоїв може містити в собі зайво витрачений робочий
час. Коли шлункові ринку не сила поглинути всієї кількости
полотна за нормальну ціну в 2 шилінґи за метр, то це доводить
лише, що у формі ткання полотна витрачено занадто велику
частину суспільного сукупного робочого часу. Вплив цього
такий самий, як коли б кожний поодинокий ткач витратив на
свій індивідуальний продукт робочого часу більше за суспільнодоконечний.
Тут можна сказати німецькою приказкою: «разом
спіймали, разом і повісили». Все полотно на ринку фігурує як
один товар, а кожний сувій його лише як певна частина. І дійсно
таки, вартість кожного індивідуального метра й є лише матеріялізація
тієї самої суспільно визначеної кількости однорідної
людської праці.

Як бачимо, товар любить гроші, але «правдива любов ніколи
не проходить гладенько».* Кількісне розчленований суспільного
продукційного організму, який виявляє свої membra disjecta **
в системі поділу праці, є так само стихійно випадкове, як і якісне
розчленовання його. Тим то наші товаропосідачі відкривають, що
той самий поділ праці, який їх робить незалежними приватними
продуцентами, робить незалежним від них самий суспільний процес
продукції і їхні відносини в цьому процесі, — що взаємна
незалежність осіб доповнюється системою всебічної речової залежности.
Поділ праці перетворює продукт праці на товар і через те
робить доконечним перетворення його на гроші. Одночасно він
робить випадковим, чи вдасться це перевтілення. Але тут ми маємо
розглянути явище в чистому вигляді, отже, мусимо припустити,
що перебіг його відбувається нормально. А проте, коли це явище

* «The course of true love never does run smooth» (Shakespeare).

** — розрізнені члени. Ред.
