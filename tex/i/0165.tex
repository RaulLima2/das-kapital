лишається, є єдина вартість, дійсно спродукована в процесі творення товару. Коли додаткову вартість
дано, то, щоб знайти змінний капітал, ми відлічуємо її від цієї новоспродукованої вартости.
Коли дано змінний капітал і ми шукаємо додаткову вартість, то ми робимо навпаки. Коли дано і те й
друге, то треба тільки зробити кінцеву операцію — обчислити відношення додаткової вартости до
змінного капіталу, m/v.

Хоч і яка проста ця метода, однак, здається, до речі буде зазнайомити читачів на кількох прикладах
із незвичним для них
способом розуміння, що лежить в основі цієї методи.

Насамперед, приклад прядільної фабрики на 10.000 мюльних веретен, де прядеться з американської
бавовни пряжу № 32 і
щотижня продукується по 1 фунту пряжі на веретено. Відпадки дорівнюють 6\%. Отже, щотижня 10.600
фунтів бавовни переробляється на 10.000 фунтів пряжі й 600 фунтів відпадків. У квітні 1871 р. ця
бавовна коштувала 7\sfrac{3}{4} пенса за фунт, отже, 10.600 фунтів, заокругляючи суму, коштували 342 фунти
стерлінґів. Ці 10.000
веретен, включаючи машини для попереднього обробітку бавовни й парову машину, коштують 1 фунт
стерлінґів за веретено, тобто 10.000 фунтів стерлінґів. Щорічне зужитковання їх становить
10\% = 1.000 фунтів стерлінґів, або 20 фунтів стерлінґів на тиждень. Наймання фабричного будинку —
300 фунтів стерлінґів, або на тиждень 6 фунтів стерлінґів. Вугілля (4 фунти на годину
й кінську силу при 100 кінських силах [індикаторних] і 60 годинах тижнево, включаючи й опалення
будинку) 11 тонн
на тиждень, по 8 шилінґів 6 пенсів за тонну, коштує, заокругляючи суму, 4\sfrac{1}{2} фунти стерлінґів на
тиждень; газ — 1 фунт стерлінґів на тиждень, мастиво — 4\sfrac{1}{2} фунти стерлінґів на тиждень, отже, всі
допоміжні матеріяли — 10 фунтів стерлінґів на тиждень. Отже, стала частина вартости становить 378
фунтів стерлінґів
на тиждень. Заробітна плата становить 52 фунти стерлінґів на тиждень. Ціна пряжі — 12\sfrac{1}{4} пенсів за
фунт, або за 10.000 фунтів
це становить 510 фунтів стерлінґів, отже, додаткова вартість дорівнює 510 - 430 = 80 фунтам
стерлінґів. Ми припускаємо, що стала частина вартости в 378 фунтів стерлінґів дорівнює нулеві, бо
вона не бере участи в тижневому творенні вартости.

Лишається тижнева новоспродукована вартість 132 = 52v + 80m фунтів стерлінґів. Отже, норма
додаткової вартости дорівнює
\sfrac{80}{52} = 153\sfrac{11}{18}\%. За десятигодинного пересічного робочого дня це дає: доконечна праця = 3\sfrac{31}{33}
годин і додаткова праця = 6\sfrac{2}{33} годин.31

31 Примітка до другого видання. Приклад прядільної фабрики з 1860 р., наведений у першому виданні,
мав деякі фактичні помилки. Наведені в тексті цілком точні числа подав мені один менчестерський
фабрикант. — Треба зауважити, що в Англії стару кінську силу обчислювалось
на основі діяметра циліндра, а нову обчислюється на основі дійсної сили, яку показує індикатор.
