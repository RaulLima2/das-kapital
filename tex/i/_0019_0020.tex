\parcont{}  %% абзац починається на попередній сторінці
\index{i}{0019}  %% посилання на сторінку оригінального видання
1 сурдутові маємо тепер: 20 метрів полотна = \sfrac{1}{2} сурдута. Навпаки,
коли вартість сурдута спаде наполовину, то 20 метрів полотна
= 2 сурдутам. Отже, за незмінної вартости товару А його
відносна вартість, виражена в товарі В, падає або підвищується
зворотно пропорційно до зміни вартости В.

Коли порівняємо різні випадки пунктів І і II, то ми побачимо,
що та сама зміна величин відносної вартости може випливати з
цілком протилежних причин. Так, з рівнання: 20 метрів полотна
= 1 сурдутові може зробитися: 1) рівнання: 20 метрів
полотна = 2 сурдутам або через те, що вартість полотна подвоюється,
або через те, що вартість сурдута падає наполовину, і
може зробитися 2) рівнання: 20 метрів полотна = \sfrac{1}{2} сурдута
або через те, що вартість полотна падає наполовину, або через те,
що вартість сурдута зростає удвоє.

III. Хай доконечні для продукції полотна й сурдута кількості
праці одночасно змінюються в тому самому напрямі і в однаковій
пропорції. В цьому випадку, як і раніш, рівнання: 20 метрів
полотна = 1 сурдутові лишається незмінне, хоч і як змінятимуться
їхні вартості. Зміну їхньої вартости ми відкриємо, скоро
порівняємо їх з якимось третім товаром, вартість якого лишається
сталою. Коли б вартості всіх товарів підвищились або впали
одночасно і в однаковій пропорції, то їхні відносні вартості лишилися
б незмінні. Дійсну зміну їхньої вартости можна було б
побачити з того, що протягом того самого робочого часу тепер
продукувалося б взагалі більшу або меншу кількість товарів,
ніж раніш.

IV. Хай робочий час, доконечний для продукції полотна і
сурдута, а тому й їхні вартості одночасно змінюються в однаковому
напрямі, алеж не в однаковій мірі, або в протилежному
напрямі й т. ін. Вплив усяких можливих подібних комбінацій
на відносну вартість товару визначається просто через застосовування
випадків І, II, III.

Отже, дійсні зміни величини вартости не відбиваються ані
ясно, ані вичерпно в їхньому відносному виразі, або у величині
відносної вартости. Відносна вартість якогось товару може змінятися,
хоч його вартість лишається стала. Його відносна вартість
може лишатися сталою, хоч його вартість змінюється і, нарешті,
одночасні зміни величини вартости й відносного виразу
цієї величини вартости не неодмінно повинні одна однією покриватися.20
20 Примітка до другого видання. Вульґарна політична економія
із звичною їй дотепністю використовує цю невідповідність (Inkongruenz)
між величиною вартости та її відносним виразом. Приміром: «Припустіть
тільки, що А понижується тому, що В, на яке воно обмінюється, підвищується,
хоч проте на А витрачається не менше праці, ніж раніш, і
ваш загальний принцип вартости розлетиться, мов прах\dots{} Коли ж припустити,
що вартість В супроти А понижується, бо вартість А супроти В
підвищується, то в нас зникає з-під ніг той ґрунт, на якому Рікардо
будує свій великий принцип, що вартість товару завжди визначається
кількістю втіленої в ньому праці, бо коли зміна у витратах на А змінює

\index{i}{0020}  %% посилання на сторінку оригінального видання
3. Еквівалентна форма

Ми бачили, що коли якийсь товар А (полотно) виражає свою вартість у споживній вартості відмінного
від нього товару В (сурдута), то він надає (drückt\dots{} auf) цьому останньому специфічної форми
вартости, форми еквіваленту. Товар «полотно» виявляє своє власне вартостеве буття тим, що сурдут, не
набираючи будь-якої іншої форми вартости, відмінної від його тілесної форми, є рівнозначний полотну.
Отже, полотно фактично виражає своє власне вартостеве буття тим, що сурдут є безпосередньо вимінний
на нього. Отже, еквівалентна форма якогось товару є форма його безпосередньої вимінности на інший
товар.

Коли якийсь рід товару, як от сурдут, служить за еквівалент якомусь іншому родові товару, приміром,
полотну, і тому сурдути набирають характеристичної властивости перебувати у формі, безпосередньо
вимінній на полотно, то цим ще аж ніяк не дано ту пропорцію, що в ній сурдути й полотно можуть
обмінюватися між собою. А що величину вартости полотна дано, то пропорція ця залежить від величини
вартости сурдутів. Чи сурдут виражено як еквівалент, а полотно як відносну вартість, чи, навпаки,
полотно як еквівалент, а сурдут як відносну вартість, величина вартости сурдута, як і раніш,
визначається кількістю робочого часу, доконечного для його продукції, отже, вона  визначається
незалежно від його форми вартости. Але скоро тільки рід товару «сурдут» займе місце еквіваленту у
виразі вартости, то величина його вартости не набуває жодного виразу як величина вартости. У
рівнанні вартостей вона фігуруватиме скорше лише як певна кількість даної речі.

Приміром, 40 метрів полотна «варті» — чого? 2 сурдутів. Що рід товару «сурдут» відіграє тут ролю
еквіваленту, що споживна вартість «сурдут» фігурує супроти полотна як тіло вартости, то досить
певної кількости сурдутів, щоб виразити певну кількість вартости полотна. Тому 2 сурдути можуть
виразити величину вартости 40 метрів полотна, але ж ніколи не можуть вони виразити величини своєї
власної вартости, величини вартости сурдутів. Поверхове розуміння цього факту, а саме того, що в
рівнанні вартостей еквівалент завжди має лише форму простої кількости якоїсь речі, якоїсь споживної
вартости, призвело Ваіlеу’а, як і багатьох його попередників і наступників, до тієї помилки, що

не лише його власну вартість проти В, на яке воно обмінюється, а ще й вартість В відносно вартости
А, хоч не сталося ніякої зміни в кількості праці, потрібної для продукції В, тоді падає не лише
доктрина, яка запевняє, що вартість товару регулюється кількістю витраченої на нього праці, але й та
доктрина, за якою витрати продукції якогось товару регулюють його вартість». (J. Broadhurst:
«Political Economy», London 1842, p. 11, 14).

Пан Бродерст міг би так само влучно сказати: пригляньтесь до числових відношень \sfrac{10}{20}, \sfrac{10}{50},\sfrac{10}{100}
і т. д. Число 10 лишається незмінним, а проте його пропорційна величина, його величина щодо
знаменників 20, 50, 100 постійно вменшується. Отже, падає великий принцип, що величину числа,
наприклад, 10, «реґулюється» кількістю одиниць, що є в ньому.
\parbreak{}  %% абзац продовжується на наступній сторінці
