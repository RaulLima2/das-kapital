\parcont{}  %% абзац починається на попередній сторінці
\index{i}{0248}  %% посилання на сторінку оригінального видання
«голова льоно-й бавовнопрядільної фабрики Карлейл, сини й
Ко в Paisley, однієї з найстаріших і найповажніших фірм у західній
Шотляндії, яка існує від 1752~\abbr{р.} і яку з покоління на покоління
веде та сама родина», — цей надзвичайно інтеліґентний
джентлмен написав до «Glasgow Daily Mail» 25 квітня 1849~\abbr{р.}
листа\footnote{
«Reports of Insp. of Fact, for 30 th April 1894», p. 59.
} під заголовком «Система змін», де, між іншим, є таке
чудернацьке, наївне місце: «Подивімось тепер на ті нещастя,
які випливають із скорочення робочого часу з 12 годин до 10 годин\dots{}
Вони «сходять» на якнайсерйозніше ушкодження перспектив
і власности фабриканта. Коли він (тобто його «руки») працював
раніше 12 годин, а тепер його обмежили 10 годинами, то кожні
12 машин або веретен у його підприємстві поменшали до 10 («then
every 12 machines or spindles, in his establishment, shrink to 10»),
і коли б він захотів продати свою фабрику, то ці машини цінували
б лише як 10, так що через це кожна фабрика по цілій країні
втратила б шосту частину своєї вартости».\footnote{
Там же, стор. 60. Фабричний інспектор Стюарт, сам шотландець і,
протилежно до англійських фабричних інспекторів, геть чисто захоплений
капіталістичним способом думання, виразно заявляє, що цей лист,
долучений ним до його звіту, «є щонайкорисніше повідомлення, подане від
будь-якого фабриканта, що вживає системи змін, і цілком розраховане
на те, щоб усунути пересуди й сумніви щодо тієї системи».
}

Для цього буржуазного мозку з західньої Шотляндії, що
успадкував від своїх предків капіталістичні властивості, вартість
засобів продукції, веретен тощо так неподільно зливається з
їхньою капіталістичною властивістю самозростати або щоденно вбирати
певну кількість чужої дармової праці, що шеф фірми Карлейл
і К° дійсно уявляє собі, що при продажу його фабрик йому
заплатять не лише за вартість веретен, але ще поверх того й за
самозростання їхньої вартости, не лише за працю, вміщену в них
і потрібну для продукції веретен того самого роду, але й за додаткову
працю, яку за їхньою допомогою день-у-день висмоктують
із бравих західніх шотландців Paisley’a, і саме через те,
гадає він, із скороченням робочого дня на дві години продажна
ціна кожних 12 прядільних машин спадає до ціни 10 машин.
