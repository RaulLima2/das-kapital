\index{i}{0274}  %% посилання на сторінку оригінального видання
2. Частинний робітник та його знаряддя

Якщо тепер підійти ближче до деталів, то насамперед ясно,
що робітник, який цілий свій вік виконує одну й ту саму просту
операцію, перетворює ціле своє тіло на її автоматично однобічний
орган і тому витрачає на це менше часу, ніж ремісник, що
виконує навпереміну цілий ряд операцій. Але комбінований
колективний робітник, що становить живий механізм мануфактури,
складається тільки з таких однобічних частинних робітників.
Тим то тут порівняно з самостійним ремеством за коротший
час продукується більше, тобто продуктивна сила праці підвищується.\footnote{
«Що більше працю якоїсь складної мануфактурної галузі розчленовано
й поділено між різними ремісниками, то ліпше й скорше цю працю виконується,
то менше витрачається часу та праці» («The more any manufacture
of much variety shall be distributed and assigned to different atrists
the same must needs be better done and with greater expedition, with less
loss of lime and labour»), («The Advantages of the East-India Trade»,
London 1720, p. 71).
}
Крім цього, якщо частинна праця всамостійнюється
у виключну функцію однієї особи, то вдосконалюється і її метода.
Постійне повторювання тієї самої обмеженої роботи й концентрація
уваги на цій обмеженій роботі навчають через досвід досягати
наміченого корисного ефекту з якнайменшою витратою сили.
А що різні ґенерації робітників завжди одночасно живуть разом
і працюють разом у тих самих мануфактурах, то набуті таким
чином способи технічної вмілости швидко вкорінюються, нагромаджуються
та передаються від однієї ґенерації до другої.\footnote{
«Легка праця є успадкована вправність» («Easy labour is transmitted
skill»). (Th. Hodgskin: «Popular Political Economy», London 1827, p. 48).
}

Мануфактура, репродукуючи всередині майстерні й систематично
розвиваючи до крайніх меж те розмежування реместв,
яке вона знайшла у містах середньовіччя, тим самим фактично
продукує віртуозність частинних робітників. З другого боку, її
перетворення частинної праці на життєву професію людини
відповідає прагненню попередніх суспільств робити ремество спадковим,
надавати йому закам’янілої форми каст або, — якщо
певні історичні умови створювали змінливість індивідів, яка
суперечила кастовому ладу, — закостенілої форми цехів. Касти
й цехи виникають із того самого природного закону, що реґулює
поділ рослин і тварин на роди й підроди, з тією лише відміною,
що на певному ступені розвитку спадковість каст або винятковість
цехів декретується як суспільний закон.\footnote{
«Вправності також... дійшли в Єгипті належного ступеня досконалости.
Бо лише в цій одній країні ремісники ні в якому разі не сміють
встрявати до занять інших громадянських кляс, а повинні працювати
лише в тій професії, яка за законом спадково належала їхньому родові...
В інших народів ми знаходимо, що ремісники поділяють свою увагу на
надто багато об'єктів... То заходяться вони коло обробітку землі, то беруться
до торговельних справ, то займаються одночасно двома або трьома
ремествами. У вільних державах вони часто бігають на народні збори...
Навпаки, в Єгипті кожного ремісника, що встрявав до державних справ
або береться одночасно до кількох реместв, піддають тяжким карам.
} «Мусліну з Даккі
\index{i}{0275}  %% посилання на сторінку оригінального видання
щодо його тонкости, ситців і інших матерій з Короманделя щодо
пишноти та тривалости фарб ще ніколи не перевищено. А проте
їх продукують без капіталу, без машин, без поділу праці або
якогось із тих інших засобів, що дають так багато переваг європейській фабрикації. Ткач — то
ізольований індивід, що на замовлення споживача виготовлює тканину, а до того — ще й на ткацькому
варстаті якнайпростішої конструкції, варстаті, що
часом складається лише з дерев’яних грубо позбиваних брусів.
У нього немає навіть апарату для натягування основи, і тому
ткацький варстат мусить лишатися розтягнутим на цілу свою
довжину, та такий він незграбний і широкий, що не може вміститися в хаті продуцента, і тому цей
останній мусить виконувати свою працю на вільному повітрі, перериваючи її повсякчас у негоду».\footnote{
«Historical and descriptive Account of British India etc. by
Hugh Murray, James Wilson etc.», Edinburgh 1832, vol. II, p. 449, 450.
Індійський ткацький варстат дуже високий, бо основу натягується вертикально.
}
Лише ця особлива вмілість, що нагромаджувалася
від покоління до покоління та спадково переходила від батька
до сина, дає індусові, як і павукові, цю віртуозність. А проте
порівняно з більшістю мануфактурних робітників такий індійський ткач виконує дуже складну працю.

Ремісник, що виконує один по одному різні частинні процеси
в продукції якогось виробу, мусить змінювати то місце, то інструменти. Перехід від однієї операції
до іншої перериває хід його праці і становить, так би мовити, пори в його робочому дні. Ці пори
звужуються, якщо він протягом цілого дня безупинно
виконує ту саму операцію, або вони зникають у міру того, як
меншає змінливість його операцій. Збільшена продуктивність
постає тут або із збільшення витрати робочої сили протягом
даного часу, отже, із зросту інтенсивности праці, або із зменшення
непродуктивного споживання робочої сили. А саме: зайва витрата
сили, що її вимагає кожний перехід од спокою до руху, компенсується при довшому триванні осягнутої
вже нормальної швидкости
праці. З другого боку, безперервність одноманітної праці ослабляє
напруження уваги та розмах життєвого духа, який саме в зміні
діяльности знаходить свій відпочинок та принаду.

Продуктивність праці залежить не тільки від віртуозности
робітника, але й від досконалосте його знарядь. Знарядь того
самого роду, як, приміром, різальних, свердлильних, поштовхових та ударних і т. ін., вживається в
різних процесах праці, і той самий інструмент у тому самому процесі праці придається до різних
операцій. Однак, скоро тільки різні операції якогось
процесу праці відокремляться одна від одної, і кожна частинна
операція набуде в руках частинного робітника якнайвідповіднішої, а через це й виключної форми, то
постає доконечність змін

Таким чином ніщо не може їм заважати пильно працювати в своїй професії... А до того, діставши багато
правил від своїх прадідів, вони ревно
дбають про те, шоб винайти нові удосконалення». (Diodorus Siculus:
«Historische Bibliothek», Bd. I, Kap. 74, S. 117, 118).
\parbreak{}  %% абзац продовжується на наступній сторінці
