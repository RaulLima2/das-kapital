а перед ним ще мудріший Мак-Куллох, нічого не розуміли з
політичної економії та християнства. Вони, між іншим, не розуміли,
що машина — найвипробуваніший засіб до здовження робочого
дня. Вони виправдували рабство однієї людини лише як
засіб до повного людського розвитку іншої. Але, щоб проповідувати
рабство мас із метою перетворити небагато грубих або
напівосвічених вискочнів у «eminent spinners», «extensive sausage
makers» та «influential shoe black dealers»,\footnote*{
— «видатних прядунів», «великих ковбасників» і «впливових
продавців вакси». \emph{Ред.}
} для цього їм
бракувало специфічно-християнського почуття.

с) Інтенсифікація праці\footnote*{
У французькому виданні Маркс додає до цього таку примітку;
«Словом інтенсифікація праці ми позначаємо методи, що роблять працю
напруженішою» («Par le mot intensification nous designons les procedés
qui rend le travail plus intense»). \emph{Ред.}
}

Безмірне здовження робочого дня, що його продукують машини
в руках капіталу, приводить пізніше, як ми вже бачили, до
реакції з боку суспільства, життю якого воно загрожувало в
самому корені, а тим самим і до законодатно обмеженого нормального
робочого дня. На основі останнього набирає вирішальної
ваги явище, з яким ми вже раніш зустрічались, а саме інтенсифікація
праці. При аналізі абсолютної додаткової вартости йшлося
насамперед про екстенсивну величину праці, а ступінь її інтенсивности
припускалось за даний. Тепер ми маємо розглянути
перетворення екстенсивної величини на інтенсивну величину,
тобто на величину, вимірювану щодо ступеня.

Само собою зрозуміло, що разом із розвитком машин та з нагромадженням
досвіду спеціяльною клясою машинових робітників
природно зростає швидкість, а тим самим і інтенсивність
праці. Так, в Англії протягом цілого півстоліття здовження
робочого дня йде поруч зростання інтенсивности фабричної праці.
Однак зрозуміло, що при такій праці, де йдеться не про минущі
пароксизми, а про реґулярну одноманітність, що з дня на день
повторюється, мусить настати пункт, коли здовження робочого

Lasst uns leben das Leben das Väter, und lasst uns der Gaben
Arbeitslos uns freun, welche die Göttin uns schenkt».

(«Gedichte aus dem Griechischen übersetzt von
Christian Graf zu Stolberg. Hamburg. 1782»).

(«Руки свої бережіть, о млинарки, і спіте спокійно, —
Півні нехай сповіщають про ранок — для вас то байдуже.
Део роботу дівочу на німф відтепер всю поклала,
Німфи віднині на колесах легко і жваво танцюють,
І обертаються осі і крутяться спині із ними,
І перемелюють зерно важкеє на кам'яних жорнах.
Отже, живімо як предки жили, без утомної праці
І заживаймо дарів, подарованих нам від богині»).