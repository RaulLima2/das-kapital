\index{i}{0520}  %% посилання на сторінку оригінального видання
5. Так званий робочий фонд

У перебігу нашого досліду виявилося, що капітал є не стала
величина, а елястична частина суспільного багатства, частина,
що постійно змінюється залежно від поділу додаткової вартости
на дохід і додатковий капітал. Ми бачили, далі, що навіть за
даної величини капіталу, який функціонує, захоплені капіталом
робоча сила, наука й земля (під нею економічно треба розуміти
всі предмети праці, які природа дає без допомоги людини) становлять
його елястичні потенції, які в певних межах дають йому
простір, незалежний від його власної величини. При цьому ми
залишали осторонь усі відносини процесу циркуляції, що спричинюють
дуже різні ступені діяльности тих самих мас капіталу.
Через те, що ми припускали як передумову межі капіталістичної
продукції, отже, суто стихійний лад суспільного процесу продукції,
ми залишали осторонь також всяку, безпосередньо й пляномірно
здійснювану, раціональнішу комбінацію наявних засобів
продукції й робочих сил. Клясична політична економія віддавна
любила розглядати суспільний капітал як сталу величину
з сталим ступенем діяльности. Але цей забобон на рівень догми
підніс й закріпив лише прафілістер Єремія Бентам, цей тверезо-педантичний
шорстко-язикатий оракул плазовитого буржуазного
розуму XIX віку.\footnote{
Порівн. між іншим: J. Bentham: «Theorie des Peines, et des Récompenses.»,
trad. Et. Dumont. 3 éme éd. Paris 1826, vol. II, liv. 4, ch. 2.
} Бентам серед філософів є те
саме, що Мартін Туппер серед поетів. Обох їх можна було
зфабрикувати лише в Англії.\footnote{
Єремія Бентам — суто англійське явище. Ніколи й ні в одній
країні ніхто ще, — не виключаючи навіть нашого філософа Хрістіяна
Вольфа, — з таким самозадоволенням не проповідував таких доморослих
банальностей. Принцип корисности — це зовсім не винахід Бентама.
Він лише бездарно повторяв те, що талановито виклали Гельвецій та
інші французи XVIII віку. Коли ми, наприклад, хочемо знати, що корисно
для собаки, то ми мусимо пізнати природу собаки. Самої цієї природи
не можна конструювати з «принципу корисности». Якщо прикласти
цей принцип до людини, якщо ми хочемо на основі принципу корисности
розглядати всякий людський вчинок, рух, відносини тощо,
то спочатку треба вивчити людську природу взагалі, а потім людську
природу, як вона змодифікована в кожній історичній епосі. Бентам цими
дрібничками не турбується. З найнаївнішою сухістю він вважає сучасного
дрібного буржуа (Spiessbürger), спеціяльно англійського, за типічну
людину. Що корисне для цієї оригінальної типічної людини й світу її,
те корисне само по собі. Цим маштабом міряє він минуле, сучасне й майбутнє.
Наприклад, християнська релігія «корисна», бо вона з релігійного
погляду забороняє ті самі злочини, що їх карний кодекс осуджує з
юридичного. Мистецька критика «шкідлива», бо вона заважає шановним
людям зазнавати насолойи з творів Мартіна Туппера й т. д. Отаким мотлохом
наповнив гори книжок цей бравий чоловічок, що його девізою було
«nulla dies sine 1іnеа».\footnote*{
— жодного дня без рядка. \emph{Ред.}
}Коли б у мене була відвага мого приятеля Г. Гайне,
я назвав би пана Єремію генієм буржуазної дурости.
} Коли прийняти його догму, то
стають цілком незрозумілими найзвичайнісінькі явища процесу
продукції, як от, наприклад, раптові його поширення і звуження,
\parbreak{}  %% абзац продовжується на наступній сторінці
