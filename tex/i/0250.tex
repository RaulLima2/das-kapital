З другого боку, сама величина додаткової праці, очевидно,
є дана, коли дано величину робочого дня й вартість робочої
сили. Вартість робочої сили, тобто робочий час, потрібний на її
продукцію, визначається робочим часом, доконечним для репродукції
її вартости. Якщо одна робоча година виражається в
кількості золота в \sfrac{1}{2} шилінґа, або 6 пенсів, а денна вартість робочої
сили становить 5 шилінґів, то робітник мусить працювати
10 годин денно, щоб компенсувати виплачену йому капіталістом
денну вартість його робочої сили, або спродукувати еквівалент
вартости його доконечних денних засобів існування. Коли дано
вартість цих засобів існування, то тим самим дано й вартість його
робочої сили,1 а коли дано вартість його робочої сили, то тим
самим дано й величину його доконечного робочого часу. Але величину
додаткової праці ми матимемо, віднявши доконечний робочий
час від цілого робочого дня. Віднявши десять годин від дванадцятьох,
ми матимемо остачу в дві години, і не можна зрозуміти,
яким чином, за даних умов, можна здовжити додаткову
працю понад дві години. Правда, капіталіст може платити робітникові
замість 5 шилінґів тільки 4 шилінґи 6 пенсів або ще менше.
Для репродукції цієї вартости в 4 шилінґи 6 пенсів вистачило б
9 робочих годин, і тому на додаткову працю припало б із дванадцятигодинного
робочого дня 3 години замість 2, а сама додаткова
вартість підвищилася б з 1 шилінґа до 1 шилінґа 6 пенсів. Однак
цього результату можна було б досягнути лише знижуючи плату
робітника нижче від вартости його робочої сили. Маючи 4 шилінґи
6 пенсів, які він продукує за 9 годин, робітник порядкує
засобами існування на \sfrac{1}{10} менше, ніж раніш, і таким чином відбувається
лише дефективна репродукція його робочої сили. Тут
додаткову працю можна було б здовжити лише через порушення

1 «Вартість денної пересічної заробітної плати визначається тим,
чого потребує робітник, «щоб жити, працювати й множитися». (William
Petty: «Political Anatomy of Ireland», 1762, p. 64). «Ціна праці
завжди складається з цін доконечних засобів існування» («The Price of
Labour is always constituted of the price of necessaries»). Робітник не дістає
належної плати, «коли... його заробітної плати недосить, щоб утримувати,
згідно з його низьким станом і становищем як робітника, таку родину
яка часто буває у багатьох із них» («whenever... the labouring man’s
wages will not, suitably to his low rank and station, as a labouring man,
support such a family as is often the lot of many of them to have»). (J. Vanderlint:
«Money answers all Things», London 1734, p. 15). «Простий
робітник, який нічого не має, окрім своїх рук та знання ремества, дістає
лише стільки, за скільки йому пощастить продати свою працю іншим...
По всіх галузях праці мусить траплятися й дійсно трапляється те, що
заробітна плата робітника обмежується на тому, що доконечне йому
для підтримання його життя». («Le simple ouvrier, qui n’a que ses bras
et son industrie, n’a rien qu’autant qu’il parvient à vendre à d’autres sa
peine... En tout genre de travail il doit arriver et il arrive en effet, que le
salaire de l’ouvrier se borne à ce qui lui est nécessaire pour lui procurer
sa subsistance»). (Turgot: «Réflexions sur la Formation et la Distribution
des Richesses». Oeuvres, éd. Daire. vol. I, p. 10). «Ціна доконечних
засобів існування — це в дійсності витрати продукції праці» («The
price of the necessaries of life is, in fact, the cost of producing labour»).
(Malthus: «Inquiry into etc.», Rent. London 1815, p. 48, примітка).
