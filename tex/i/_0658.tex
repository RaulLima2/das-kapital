\index{i}{0658}  %% посилання на сторінку оригінального видання 
Скоро тільки цей процес перетворення в достатній мірі розклав
старе суспільство углиб і вшир, скоро тільки робітників
перетворено на пролетарів, а їхні умови праці на капітал, скоро
тільки капіталістичний спосіб продукції став на власні ноги,
дальше усуспільнення праці і дальше перетворення землі та
інших засобів продукції на суспільно-експлуатовані, отже, на
спільні засоби продукції, і тим то й дальша експропріяція приватних
власників набуває нової форми. Тепер експропріяції
підлягає вже не робітник, що сам веде самостійне господарство,
а капіталіст, що експлуатує багатьох робітників.

Ця експропріяція здійснюється наслідком гри іманентних законів
самої капіталістичної продукції, через централізацію капіталів.
Один капіталіст побиває багатьох. Пліч-о-пліч із цією
централізацією або експропріяцією багатьох капіталістів небагатьма
розвивається кооперативна форма процесу праці в
щораз ширших, більших розмірах, розвивається свідоме технічне
застосування науки, пляномірна експлуатація землі, перетворення
засобів праці на такі засоби праці, що їх можна
вживати тільки колективно, економізування всіх засобів продукції
через вживання їх як засобів продукції комбінованої,
суспільної праці, вплетіння всіх народів у сіть світового ринку,
а разом з тим інтернаціональний характер капіталістичного режиму.
Разом з постійним меншанням числа маґнатів капіталу,
що узурпують і монополізують усі вигоди цього процесу перетворення,
зростає маса злиднів, пригноблення, рабства, виродження,
експлуатації, але разом з тим і обурення робітничої
кляси, щораз більшої й більшої числом, що її навчає, об’єднує
й організує механізм самого процесу капіталістичної продукції.
Монополія капіталу стає путами того способу продукції, що
зріс за неї і під нею. Централізація засобів продукції та усуспільнення
праці досягають такого пункту, коли вони стають
несполучні з їхньою капіталістичною оболонкою. Її розривається.
Б’є остання година капіталістичної приватної власности.
Експропріяторів експропріюють.

Капіталістичний спосіб присвоєння, що випливає з капіталістичного
способу продукції, а тому й капіталістична власність
є перше заперечення індивідуальної приватної власности, основаної
на власній праці. Але капіталістична продукція з доконечністю
природного процесу породжує заперечення самої себе.
Це є заперечення заперечення. Воно відбудовує не приватну
власність, а індивідуальну власність на основі завоювань капіталістичної
ери, на основі кооперації і спільного володіння землею
й засобами продукції, спродукованих самою ж працею.

Перетворення роздрібненої приватної власности, основаної
на власній праці індивідуумів, на капіталістичну власність, є,

nous tendons à séparer toute espèce de propriété d’avec toute espèce de
travail»). («Sismondi: «Nouveaux Principes de l’Economie Politique»,
vol. II, p. 434).
\parbreak{}  %% абзац продовжується на наступній сторінці
