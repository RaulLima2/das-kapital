Товари, що їхні ціни є визначені, з’являються у формі: а товару А = х золота; b товару B = z золота;
с товару С = y золота й т. д., де а, b, c являють собою певні маси товарових родів А, В, С, а х, z,
у — певні маси золота. Таким чином товарові вартості перетворено в уявлювані кількості золота різної
величини, отже, не зважаючи на строкату різноманітність товарових тіл, товарові вартості перетворено
у величини однойменні, у величини золота. Як такі різні кількості золота вони порівнянні й
спільномірні між собою, при чому виникає технічна конечність звести їх до якоїсь фіксованої
кількости золота як до їхньої одиниці міри. Сама ця одиниця міри через дальший поділ на
аліквотні частини розвивається у маштаб. Золото, срібло, мідь ще до того, як вони стали грішми,
мають уже такі маштаби в своїх мірах ваги, так що, коли за одиницю міри служить, приміром, 1 фунт,
то, з одного боку, він поділяється знову на унції й т. ін., а з другого — складається в центнери й
т. ін.54

спроби розглядати їх як ту саму речовину завжди були даремні. Коли припустити, що той самий робочий
час мусить упредметнюватись незмінно в тій самій пропорції золота й срібла, то тим самим фактично
припускається, що срібло й золото є та сама речовина, і що певна маса менш вартісного металю,
срібла, становить незмінну частину певної маси золота. Починаючи від королювання Едварда III аж до
часів Ґеорґа II, історія англійської грошової справи являє собою раз-у-раз ряд порушень, що
поставали з колізії між законодавчо фіксованим вартостевим відношенням
золота до срібла та дійсними коливаннями їхньої вартости. То золото цінилося занадто високо, то
срібло. Металь, який цінилося занадто низько, вилучалося з циркуляції, його перетоплювали й вивозили
за кордон. Потім вартостеве відношення обох металів знову мінялось законом, але й нова номінальна
вартість незабаром входила з дійсним вартостевим відношенням у такий самий конфлікт, як і стара. —
За наших часів дуже
незначне й тимчасове падіння вартости золота проти срібла, як наслідок індійсько-китайського попиту
на срібло, викликало у Франції те саме явище в найбільшім розмірі: вивіз срібла й витиснення його з
циркуляції золотом. Протягом 1855, 1856 і 1857 рр. надмір довозу золота до Франції понад вивіз
золота з Франції становив 41.580.000 фунтів стерлінґів, тимчасом як надмір вивозу срібла понад довіз
срібла становив 14.704.000 фунтів стерлінґів. Справді, у країнах, де обидва металі є визнані законом
міри вартости, отже, де обидва повинні прийматись при оплатах, але кожний може собі платити як хоче,
— золотом чи сріблом, — у таких країнах металь, який підноситься у вартості, набуває ажіо й виміряє
свою ціну, — як і кожний інший товар, — у металі, що ціниться занадто високо, тимчасом як виключно
цей останній служить за міру вартости. Ввесь історичний досвід у цій галузі сходить просто на те, що
скрізь, де законом надано двом товарам функції міри вартости, фактично завжди лише один із них
зберігає цю функцію». (К. Marx: «Zur Kritik der Politischen Oekonomie», S. 52—53. — K. Маркс: «До
критики політичної економії», ДВУ, 1926 р., стор. 89—90).

54 Примітка до другого видання. Та дивна обставина, що в Англії унція золота як одиниця грошового
маштабу неподільна на аліквотні частини, пояснюється ось як: «Нашу монетну справу було пристосовано
спершу лише до вжитку срібла, — тому унція срібла може бути завжди поділена на певну кількість цілих
монет; а що золото пізніш заведено в монетну систему, пристосовану лише до срібла, то з унції золота
не можна
викарбувати відповідного числа монет» («Our coinage was originally adapted to the employment of
silver only — hence an ounce of silver can always be divided into a certain adequate number of
pieces of coin; but as gold
