\parcont{}  %% абзац починається на попередній сторінці
\index{i}{0423}  %% посилання на сторінку оригінального видання
машин у рільництві здебільша вільний від шкідливого фізичного
впливу, справлюваного на фабричного робітника\footnote{
Докладний опис машин, уживаних в англійському рільництві,
знаходимо в «Die landwirtschaftlichen Geräte und Maschinen Englands,
von Dr. W. Hamm». 2 Auflage 1856. У своєму нарисі про розвиток англійського
рільництва пан Гам надто некритично йде за паном Леонс де
Лявернь. [До четвертого видання. — Розуміється, тепер цей нарис застарів.
— \emph{Ф. Е}].
},
то в «утворенні зайвих» робітників вони тут діють ще інтенсивніше
і не маючи собі в цьому опору, як ми це пізніше
побачимо в подробицях. У графствах Кембрідж і Суффолк, приміром,
площа обробленої землі за останні двадцять років дуже
поширилася, тимчасом як сільська людність за той самий період
не тільки відносно, а й абсолютно зменшилася. У Сполучених
штатах Північної Америки рільничі машини заміняють робітників
покищо лише у можливості, тобто вони дозволяють продуцентові
обробляти більшу площу землі, але не проганяють дійсно
занятих робітників. В Англії й Велзі число осіб, що працювали
у фабрикації рільничих машин, становило 1861~\abbr{р.} \num{1.034}, тим часом
як число рільничих робітників, занятих коло парових і робочих
машин, становило лише \num{1.205}.

У сфері рільництва велика промисловість діє якнайбільш
революційно в тому розумінні, що вона нищить твердиню старого
суспільства, «селянина», і висуває на його місце найманого робітника.
Таким чином потреби соціяльного перевороту й соціяльні
противенства\footnote*{
У французькому виданні тут замість «соціяльні противенства»
cказано «клясова боротьба». \emph{Ред.}
} на селі доходять такого ж рівня, як і в місті.
Замість найрутиннішого і найнераціональнішого виробництва
постає свідоме технологічне застосування науки. Капіталістичний
спосіб продукції завершує розрив того первісного родинного
зв’язку рільництва з мануфактурою, який об’єднував дитинячі,
нерозвинуті форми одного і другої. Але разом із цим цей спосіб
продукції утворює матеріяльні передумови нової, вищої
синтези, а саме спілки рільництва і промисловости, на основі
їхніх антагоністично розвинених форм. Капіталістична продукція
в міру того, як перевага міської людности, яку вона стягає до
великих центрів, щораз більшає, — нагромаджує, з одного боку,
історичну силу руху суспільства, а з другого боку, перешкоджає
обмінові речовин між людиною й землею, тобто перешкоджає
повертанню ґрунтові тих його складових частин, які людина
зужила у формі харчових засобів і одягу, отже, вона порушує
вічну природну умову тривалої родючости ґрунту. Цим самим
вона одночасно руйнує фізичне здоров’я міських робітників і
інтелектуальне життя сільських робітників\footnote{
«Ви розділяєте народ на два ворожі табори: на необтесаних мужиків
і слабовитих карликів. Боже мій! Нація, що розділилася на рільничі
й торговельні інтереси, вважає себе за здорову й називає себе
навіть освіченою й цивілізованою не наперекір, а саме в наслідок цього
потворного й неприродного поділу». («You divide the people into two
hostile camps of clownish boors and emasculated dwarfs. Good heavens!
a nation divided into agricultural and commercial interests calling itself
sane, nay styling itself enlightened and civilized, not only in spite of,
but in consequence of this monstrous and unnatural division»). (\emph{David
Urquhart}: «Familiar Words», London 1855, p. 119). Це місце показує
одночасно силу і слабість такого роду критики, яка вміє обмірковувати
та ганьбити сучасність, але не вміє її зрозуміти.
}.  Але, руйнуючи
спонтанейно посталі умови цього обміну речовин, капіталістична
\index{i}{0424}  %% посилання на сторінку оригінального видання
продукція разом з тим примушує відновити його систематично
як закон, що реґулює суспільну продукцію, і у формі, адекватній
повному розвиткові людини. У рільництві, як і в мануфактурі,
капіталістичне перетворення продукційного процесу є
разом з тим мартиролог продуцентів, засіб праці є разом з тим
засіб поневолення, засіб експлуатації й засіб павперизації робітника,
суспільна комбінація процесу праці є разом з тим організоване
пригнічення індивідуальної життьової сили робітника,
його волі й самостійности. Розпорошеність сільських робітників
по великих просторах ламає одночасно силу їхнього
опору, тимчасом як концентрація міських робітників підносить
її. У сучасному рільництві, так само і в міській промисловості
підвищення продуктивної сили й більша ефективність праці
купується ціною нищення й виснаження самої робочої сили.
І кожний проґрес капіталістичного рільництва — це не тільки
проґрес у вмілості грабувати робітника, але разом з тим і у вмілості
грабувати ґрунт, кожний проґрес у піднесенні родючости
його на даний час — це разом з тим проґрес у руйнуванні тривалих
джерел цієї родючости. Що більш якась країна, як, приміром,
Сполучені штати Північної Америки, виходить від великої
промисловости як бази свого розвитку, то швидший цей
процес руйнування\footnote{
Порівн. \emph{Liebig}: «Die Chemie in ihrer Anwendung auf Agrikultur
und Physiologie». 7. Auflage 1862, особливо також «Einleitung
in die Naturgesetze des Feldbaues» у першому томі. Вияснення неґативного
боку сучасного рільництва з погляду природознавства — це одна
з невмирущих заслуг Лібіґа. Його історичні нариси з історії рільництва,
хоч вони й не без грубих помилок, також висвітлюють деякі питання.
Можна пожалкувати, що він навмання зважується висловлювати
ось які погляди: «Продовжуване далі роздрібнювання й частіше переорювання
підвищує обмін повітря всередині поруватих частин землі, збільшує
й поновлює поверхню цих частин землі, на яку має впливати повітря;
але легко зрозуміти, що додатковий здобуток із поля не може бути,
пропорційний до витраченої на поле праці, а зростає в куди меншій пропорції».
«Цей закон, — додає Лібіґ, — уперше висловив Дж. Ст. Мілл у своїм
«Principles of Political Economy», v. I, p. 17, ось так: «Те, що продукт
землі за інших рівних умов зростає в дедалі меншій пропорції порівняно
до зростання числа вживаних робітників (навіть відомий закон Рікарда
пан Мілл повторює тут у фалшивому формулюванні, бо через те, що «зменшення
числа вживаних робітників» («the decrease of the labourers employed»)
в Англії постійно відбувалося поруч із проґресом рільництва, закона,
вигаданого для Англії і в Англії, не можна було б застосувати, принаймні
в Англії), — це універсальний закон рільничої промисловости». Це, — каже
далі Лібіґ, — річ досить дивна, бо Міллові була невідома основа цього закону»
(\emph{Liebig}, там же, книга 1, стор. 143 і примітка). Не кажучи вже про
помилкове тлумачення слова «праця», під яким Лібіґ розуміє щось інше,
ніж політична економія, в усякому разі «річ досить дивна», що він із Дж.
Ст. Мілла робить першого оповісника теорії, яку Джемс Андерсон опублікував
уперше за часів А. Сміса й повторював її в різних творах аж до
початку XIX віку, теорії, яку 1815~\abbr{р.} присвоїв собі Малтуз, взагалі
майстер у пляґіятах (ціла його теорія залюднення є безсоромний пляґіят),
яку Вест розвинув одночасно з Андерсоном і незалежно від нього,
яку Рікардо 1817~\abbr{р.} зв’язав із загальною теорією вартости й яка від
того часу під ім’ям Рікарда обійшла ввесь світ, яку 1820~\abbr{р.} звульґаризував
Джемс Мілл (батько Дж. Ст. Мілла) і яку, нарешті, повторює, між
іншим, і пан Дж. Ст. Мілл як шкільну догму, що встигла вже зробитися
банальною фразою. Безперечно, Дж. Ст. Мілл завдячує свій, в усякому
разі, «дивний» авторитет майже виключно подібним qui pro quo.
}. Тому капіталістична продукція розвиває
техніку й комбінування суспільного процесу продукції, але
лише так, що вона разом з цим підриває джерела виникнення
всякого багатства: землю й робітника.

\index{i}{0425}  %% посилання на сторінку оригінального видання

\chapter{Продукція абсолютної і відносної
Додаткової вартости}

\section{Абсолютна й відносна додаткова вартість}

Спочатку ми розглядали процес праці абстрактно (див. п’ятий
розділ), незалежно від його історичних форм, як процес між
людиною і природою. Там ми казали: «Коли розглядати цілий
процес праці з погляду його результату, [продукту]\footnote*{
Заведене у прямі дужки взято з французького видання. \emph{Ред.}
}, то і засоби
праці, і предмет праці, одне й друге, з’являються як засоби
продукції, а сама праця — як продуктивна праця». У примітці
сьомій був додаток: «Цього визначення продуктивної праці,
що випливає з погляду простого процесу праці, зовсім недосить
для капіталістичного процесу продукції». Це нам треба тут
розвинути далі.

Оскільки процес праці є суто індивідуальний, той самий
робітник сполучає всі ті функції, що пізніше розділяються. В індивідуальному
присвоюванні предметів природи для своїх життєвих
цілей він контролює сам себе. Пізніше його контролюють.
Поодинока людина не може впливати на природу, не пускаючи
в рух своїх мускулів під контролем свого власного мозку. Як у
системі природи голова й руки належать одне до одного, так само
і процес праці сполучає працю голови й працю рук. Пізніше ті
праці розділяються аж до ворожої протилежности. Продукт перетворюється
\index{i}{0426}  %% посилання на сторінку оригінального видання
взагалі з безпосереднього продукту індивідуального
продуцента на суспільний, на спільний продукт колективного
робітника, тобто на продукт комбінованого робочого персоналу,
що його члени беруть ближчу або дальшу участь в обробленні
предмету праці. Тому з кооперативним характером самого процесу
праці неодмінно ширшає поняття продуктивної праці та
її носія, продуктивного робітника. Щоб працювати продуктивно,
йому тепер уже не треба самому прикладати рук, а досить бути
органом колективного робітника, виконувати одну якусь його
частинну функцію. Наведене вище первісне визначення продуктивної
праці, виведене з самої природи матеріяльної продукції,
завжди зберігає свою силу для колективного робітника, розглядуваного
як ціле. Але воно вже не має сили для кожного з його
членів, взятого окремо.

Але, з другого боку, поняття продуктивної праці вужчає.
Капіталістична продукція є не тільки продукція товару, вона
з самої суті своєї є продукція додаткової вартости. Робітник
продукує не для себе, а для капіталу. Тому вже недосить того,
що він взагалі продукує. Він мусить продукувати додаткову
вартість. Тільки той робітник продуктивний, що продукує додаткову
вартість для капіталіста, або служить для самозростання
вартости капіталу. Так, шкільний учитель, якщо можна
взяти приклад з-поза сфери матеріяльної продукції, є продуктивний
робітник тоді, коли він не тільки обробляє дитячі голови,
але й себе витрачає, щоб збагатити підприємця. Те, що останній
вклав свій капітал не у фабрику ковбас, а у фабрику навчання,
нічого не змінює в цьому відношенні. Тому поняття продуктивного
робітника ні в якому разі не містить у собі тільки відношення
між діяльністю та корисним ефектом, поміж робітником та продуктом
праці; воно містить у собі ще й специфічно-суспільне,
історично постале продукційне відношення, яке робить робітника
безпосереднім засобом зростання вартости капіталу. Тому бути
продуктивним робітником — це не щастя, а біда. У четвертій
книзі цієї праці\footnote*{
Мова йде про теорії додаткової вартости, що їх Маркс гадав видати
як четверту книгу «Капіталу». \emph{Ред.}
}, що розглядає історію теорії, ми ближче побачимо,
що клясична політична економія вже віддавна зробила
продукцію додаткової вартости характеристичною вирішальною
ознакою продуктивного робітника. Тому із зміною її розуміння
природи додаткової вартости змінюється й її визначення продуктивного
робітника. Так, фізіократи заявляють, що тільки
рільнича праця продуктивна, бо тільки вона дає додаткову вартість.
Але для фізіократів додаткова вартість існує виключно
у формі земельної ренти.

Здовження робочого дня поза той пункт, коли робітник спродукував
би лише еквівалент вартости своєї робочої сили, і присвоєння
цієї додаткової праці капіталом — оце є продукція
абсолютної додаткової вартости. Продукція абсолютної додаткової
\parbreak{}  %% абзац продовжується на наступній сторінці
