Машина продукує відносну додаткову вартість не тільки тим,
що вона безпосередньо зневартнює робочу силу та посередньо
здешевлює її, здешевлюючи товари, потрібні для її репродукції,
але ще й тим, що, при першому спорадичному заведенні її, вона
перетворює вживану посідачем машини працю на працю вищого
ступеня і більшої ефективности (potenzierte Arbeit), суспільну
вартість машинового продукту підвищує понад його індивідуальну
вартість, і таким чином дає змогу капіталістові денну вартість
робочої сили покривати меншою частиною вартости денного продукту.
Тому підчас цього переходового періоду, коли машинове
виробництво лишається своєрідною монополією, бариші є надзвичайно
великі, і капіталіст силкується якнайґрунтовніше визискати
цей «перший час молодого кохання» за допомогою якнайбільшого
здовження робочого дня. Великий бариш загострює ненажерливу
жадобу ще більшого баришу.

У міру того, як машина стає загальним явищем у тій самій
галузі продукції, суспільна вартість машинового продукту знижується
до його індивідуальної вартости та потверджується той
закон, що додаткова вартість постає не з тих робочих сил, що їх
капіталіст замістив машиною, а, навпаки, з тих робочих сил,
яких він коло неї вживає. Додаткова вартість походить тільки
із змінної частини капіталу, і ми вже бачили, що масу додаткової
вартости визначають два фактори — норма додаткової вартости
та число одночасно вживаних робітників. За даної довжини робочого
дня норма додаткової вартости визначається тим відношенням,
що в ньому робочий день розпадається на доконечну працю
й додаткову працю. Число одночасно вживаних робітників визначається,
з свого боку, відношенням змінної частини капіталу до
сталої. Тепер ясно, що хоч як машинове виробництв збільшувало б
через піднесення продуктивної сили праці додаткову працю коштом
доконечної праці, воно доходить цього результату лише тим,
що зменшує число робітників, вживаних якимось даним капіталом.
Воно перетворює на машини, отже, на сталий капітал, що не продукує
жодної додаткової вартости, частину капіталу, який раніш
був змінний, тобто перетворювався на живу робочу силу. З двох
робітників, приміром, неможливо витиснути стільки додаткової
вартости, скільки з 24. Якщо кожний із 24 робітників дає за 12 годин
праці лише одну годину додаткової праці, то разом вони дають
24 години податкової праці, тимчасом як уся праця двох робітників
становить лише 24 години. Отже, вживання машин з метою
продукції додаткової вартости містить у собі іманентну суперечність,
бо з двох факторів додаткової вартости, що її дає капітал
даної величини, машини збільшують один фактор, норму додаткової
вартости, лише тим, що зменшують другий фактор —
число робітників. Ця іманентна суперечність виявляється, скоро
тільки з загальним поширенням машин у якійсь галузі промисловости
вартість продукованого машиновим способом товару стає
реґулятивною суспільною вартістю всіх товарів того самого
роду, і це є та суперечність, яка, не доходячи до свідомости капі-
