\index{i}{0134}  %% посилання на сторінку оригінального видання
Отже, ми бачимо: чи з’являється споживна вартість як сировинний
матеріял, чи як засіб праці або продукт — це геть чисто
залежить від певної її функції у процесі праці, від місця, яке
вона займає в ньому, і зі зміною цього місця змінюються й ті
її визначення.

Тому, увіходячи як засоби продукції в нові процеси праці,
продукти втрачають характер продуктів. Вони функціонують
ще тільки як предметні фактори живої праці. Прядун ставиться
до веретена лише як до засобу, яким він пряде, до льону — лише
як до предмету, що його він пряде. Звичайно, не можна прясти
без матеріялу для прядіння й без веретен. Тому наявність цих
продуктів доводиться припустити вже з самого початку прядіння.
Але для самого цього процесу той факт, що льон і веретена є
продукти минулої праці, не має ніякого значення цілком так
само, як для акту харчування не має ніякого значення той факт,
що хліб є продукт минулої праці рільника, мірошника, пекаря
й т. ін. Навпаки, якщо засоби продукції й виявляють у процесі
праці свій характер продуктів минулої праці, то лише через їхні
вади. Ніж, що не ріже, пряжа, що раз-у-раз рветься, і т. ін.
живо нагадують ножівника А і прядуна В. На вдалому продукті
не помітно слідів минулої праці, що надала йому його споживних
властивостей.

Машина, що не функціонує в процесі праці, є некорисна.
Окрім того, вона зазнає руйнаційного впливу природного обміну
речовин. Залізо ржавіє, дерево гниє. Пряжа, що її не тчуть або
не плетуть, є зіпсована бавовна. Жива праця мусить охопити
ці речі, з мертвих зробити їх живими, перетворити їх з лише
можливих у дійсні й діяльні споживні вартості. Охоплені вогнем
праці, яка асимілює їх як своє тіло, надхнені в процесі праці на виконання
функцій, що відповідають їхній ідеї і призначенню, вони
хоч і будуть спожиті, але спожиті доцільно, як елементи утворення
нових споживних вартостей, нових продуктів, які здатні
увійти як засоби існування в особисте споживання або як засоби
продукції в новий процес праці.

Отже, коли наявні продукти є не лише результат, але й умови
існування процесу праці, то, з другого боку, вкладання їх у
процес праці, отже, їхній контакт з живою працею, є єдиний
засіб для того, щоб зберегти й реалізувати ці продукти минулої
праці як споживні вартості.

Праця споживає свої речові елементи, свій предмет і свої
васоби, з’їдає їх, отже, це є процес споживання. Це продуктивне
споживання відрізняється тим від особистого споживання, що в
останньому продукти споживається як засоби існування живого
індивіда, а в першому — як засоби існування праці, робочої
сили індивіда, яка виявляється в діяльності. Отже, продукт особистого
споживання є сам споживач, а результат продуктивного
споживання є відмінний від споживача продукт.

Оскільки засоби й предмет праці сами вже є продукти, праця
споживає продукти, щоб утворювати продукти, або застосовує
\parbreak{}  %% абзац продовжується на наступній сторінці
