винулося до розмірів справжньої галузі торговлі. Палата громад
навряд чи повірить цьому. Ця реґулярна торговля, баришування
людським м’ясом, тривала й далі, і цих людей купували манчестерські
аґенти й продавали менчестерським фабрикантам так
само регулярно, як продають негрів плянтаторам бавовни в південних
державах... Рік 1860 — кульмінаційний пункт бавовняної
промисловости... Знов бракувало робочих рук. Фабриканти
знову звернулись до аґентів продажу людського м’яса... і ці пронишпорили
всі прибережні дюни Dorset’a, верховину Devon’a
й рівнини Wilts’a, але надмір людности був уже спожитий».
«Bury Guardian» нарікав на те, що після складання англійсько-французького
торговельного договору можна було б поглинути
10.000 додаткових рук, а незабаром їм потрібно буде ще 30.000—40.000.
Після того як аґенти й субаґенти цієї торговлі м’ясом
майже безуспішно пронишпорили 1860 р. всі рільничі округи,
«депутація фабрикантів звернулась до пана Вільєрса, президента
Poor Law Board,* із проханням знов дозволити їм брати на
фабрики сиріт і дітей бідних із робітних домів. 110

110 Там же. Хоч Вільєрс і був дуже доброзичливий до фабрикантів,
а проте мусив «на основі закону» відмовити домаганням фабрикантів.
Однак ці пани досягли своєї мети завдяки прислужливості місцевої
адміністрації домів для бідних. Фабричний інспектор А. Редґрев запевняє,
що цього разу систему, за якої сиріт і дітей павперів «за законом» вважається
за учнів (apprentices), «не супроводилось колишніми зловживаннями»
(про ці зловживання див. в Енґельса: «Становище робітничої
кляси в Англії»), хоч, все ж, в одному випадку «зловжито системою щодо
дівчат і молодих жінок, яких привезено до Ланкашіру й Чешіру з рільничих
округ Шотляндії». Ця «система» полягає в тому, що фабрикант складає
з адміністрацією домів для бідних контракт на певний строк. Він харчує
дітей, одягає їх і дає їм помешкання й невеличку доплату грішми.
Дивно якось чути дальше зауваження пана Редґрева, особливо, коли взяти
на увагу, що навіть за часів розцвіту англійської бавовняної промисловости
1860 рік стоїть цілком окремо і що, крім того, заробітна плата була
дуже високою, бо надзвичайний попит на робітників збігся із знелюдненням
Ірляндії, з небувалою еміґрацією з англійських і шотляндських
рільничих округ до Австралії й Америки, з позитивним зменшенням людности
в декотрих англійських рільничих округах, що сталося почасти в
наслідок щасливо осягнутого зламу життєвої сили, а почасти в наслідок
того, що торгівці людським м’ясом ще раніше вичерпали всю зайвину
людности. І, не зважаючи на все це, пан Редґрев каже: «Однак на працю
такого роду (праця дітей із домів для бідних) є попит лише тоді, коли не
можна найти жодної іншої, бо це дорога праця (high-priced labour).
Звичайна заробітна плата підлітка від 13 років дорівнює приблизно 4 шилінґам
на тиждень: але дати помешкання, одягати і харчувати 50 або 100
таких підлітків, забезпечити їм лікарську допомогу й подбати про потрібний
догляд та, крім того, давати ще невеличку доплату грішми, — цього
не можна зробити за 4 шилінґи на душу за тиждень». («Reports of Insp.
of Fact. 30 th April 1860», p. 27). Пан Редґрев забуває сказати, яким
чином сам робітник може дати все це своїм дітям за їхню заробітну плату
в 4 шилінґи, коли фабрикантові не сила зробити цього для 50 або 100
хлопців, що живуть укупі, користуються спільним утриманням, спільним
доглядом. Щоб уникнути фалшивих висновків із тексту, я мушу тут ще
зауважити, що англійську бавовняну промисловість від того часу, як
її підпорядковано Factory Act’oвi 1850 р. що вреґулював робочий день
і т. д., треба розглядати як зразкову промисловість Англії. Англійський

* — інституції у справах бідних. Ред.
