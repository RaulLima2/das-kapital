шення багатства й сили... обмежується цілком і виключно на
заможних клясах». «Але цього речення немає ніде в Ґледстоновій
промові. В ній сказано прямо протилежне. [Жирним
шрифтом] Маркс формально й матеріяльно прибрехав це речення!»

Маркс, якому надіслано було це число «Concordia» в наступному
травні, відповів анонімові в «Volksstaat’i» з 1 червня. Через
те, що він уже не пригадував собі, з якого газетного звіту він цитував,
він обмежився на тому, що покликавсь насамперед на рівнозначну
цитату з двох англійських творів і, крім того, процитував
звіт «Times’а», що згідно з ним Ґледстон каже: «Такий є
стан нашої країни з погляду багатства. Щодо мене, то я мушу
признатися, що я майже з занепокоєнням і жалем поглядав би на
це приголомшливе збільшення багатства й сили, коли був би певний,
що воно обмежується на дійсно заможних клясах. Воно
взагалі не стосується до становища робітничої людности. Цей
зріст багатства, який я описую тут на підставі даних, на мою
думку, цілком точних, обмежується виключно на заможних
клясах». 1

Отож Ґледстон каже тут, що йому, мовляв, було б жалко,
коли б так було, але це справді так: це приголомшливе збільшення
сили й багатства обмежується цілком і виключно на заможних
клясах. А щодо quasi-офіційного «Hansard’a», то Маркс каже
далі: «Ґледстон був такий мудрий, що в своєму пізніше виправленому
виданні попідчищував тут місця, справді компромітовні
в устах англійського канцлера скарбу; а втім, це традиційна
англійська парляментська звичка, а зовсім не винахід Ляскера
проти Бебеля».

Анонім дедалі більше сердиться. У своїй відповіді в «Concordia»
4 липня він, відкидаючи джерела з другої руки, соромливо
зазначає про «звичай» цитувати парляментські промови з стенографічних
звітів; алеж, мовляв, і звіт «Times’a» (де є це «прибріхане»
речення) і звіт «Hansard’a» (де його немає) «матеріяльно
цілком збігаються»; у звіті «Times’a» так само міститься «прямо
протилежне до того славетного місця з inaugural-адреси»; при
цьому цей добродій старанно замовчує, що звіт «Times’a» поруч
із цією вигаданою «протилежністю» виразно містить саме «те
славетне місце»! А все ж анонім почуває, що він уклепався, і що
його може врятувати хіба лише новий викрут. Отже, переповнюючи
свою статтю, що, як це тільки но доведено, аж рябіє від
«нахабної брехливости», навчальними лайками — такими, як
«mala fides», «безчесність», «брехливі дані», «та брехлива цитата»,
«нахабна брехливість», «цитата, цілковито підроблена»,

1 «That is the state of the case as regards the wealth of this country.
I must say for one, I should look almost with apprehension and with pain
upon this intoxicating augmentation of wealth and power, if it were my
belief that it was confined to classes who are in easy circumstances. This
takes no cognizance at alle of the condition of the labouring population.
The augmentation I have described and which is founded, I think, upon
accurate returns, is an augmentation entirely confined to classes of property».
