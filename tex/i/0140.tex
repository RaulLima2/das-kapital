Отож, надзвичайно важливо, щоб під час процесу, тобто перетворення
бавовни на пряжу, споживано тільки суспільно-доконечний
робочий час. Коли за нормальних, тобто пересічних
суспільних умов продукції, а фунтів бавовни за одну робочу
годину мусять бути перетворені на b фунтів пряжі, то значення
12-годинного робочого дня має лише такий робочий день, який
12 х а фунтів бавовни перетворює на 12 х b фунтів пряжі, бо
при творенні вартости береться до уваги лише суспільно-доконечний
робочий час.

Так сама праця, як і сировинний матеріял і продукт з’являються
тут у зовсім іншому світлі, ніж з погляду власне процесу
праці. Сировинний матеріял має тут значення лише остільки,
оскільки він вбирає в себе певну кількість праці. Через це вбирання
він дійсно перетворюється на пряжу, бо робочу силу витрачено
й додано до нього у формі прядіння. Але продукт, пряжа,
є тепер лише мірило праці, увібраної бавовною. Коли за одну
годину випрядається 1 2/3 фунта бавовни, або її перетворюється на
1 2/3 фунта пряжі, то 10 фунтів пряжі свідчать про шість увібраних
годин праці. Певні й установлені досвідом кількості продукту
репрезентують тепер не що інше, як певні кількості праці, певні
маси застиглого робочого часу. Вони є лише матеріялізація однієї
години, двох годин, одного дня суспільної праці.

Те, що праця є саме праця прядіння, її матеріял — бавовна,
а її продукт — пряжа, тут так само не має значення, як і те, що
самий предмет праці вже є продукт, тобто сировинний матеріял.
Коли б робітник працював не в прядільні, а в копальні, то предмет
праці, вугілля, був би даний природою. Проте певна кількість
вугілля, видобутого з вугляних покладів, приміром, один
центнер, репрезентувала б певну кількість увібраної праці.

При продажу робочої сили припускалось, що її денна вартість
дорівнює 3 шилінґам, і що в останніх утілено 6 робочих годин,
отже, що ця кількість праці потрібна на те, щоб випродукувати
пересічну суму засобів існування для робітника на один день.
Коли наш прядун за 1 робочу годину перетворює 1 2/3 фунта
бавовни в 1 2/3 фунта пряжі,12 то за 6 годин він перетворить
10 фунтів бавовни в 10 фунтів пряжі. Отже, протягом процесу
прядіння бавовна вбирає в себе 6 робочих годин. Цей самий робочий
час виражається в кількості золота в 3 шилінґи. Отже, самим
прядінням додано до бавовни вартість у 3 шилінґи.

Погляньмо тепер на цілу вартість продукту, цих 10 фунтів
пряжі. У них упредметнено 2 1/2, робочих днів: 2 дні містяться в
бавовні та веретенах, 1/2 дня праці увібрано протягом процесу прядіння.
Цей самий робочий час виражається в масі золота в 15 шилінґів.
Отже, ціна цих 10 фунтів пряжі, адекватна їхній вартості,
становить 15 шилінґів, ціна 1 фунта пряжі — 1 шилінґ 6 пенсів.

Наш капіталіст збентежений. Вартість продукту дорівнює
вартості авансованого капіталу. Авансована вартість не зросла,

12 Числа тут цілком довільні.
