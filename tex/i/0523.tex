Розділ двадцять третій

Загальний закон капіталістичної акумуляції

1. Зростання попиту на робочу силу разом з акумуляцією
при незмінному складі капіталу

У цьому розділі ми розглядаємо той вплив, що його справляє
зростання капіталу на долю робітничої кляси. Найважливіший
фактор при цьому дослідженні — це склад капіталу й зміни,
що їх він зазнає в перебігу процесу акумуляції.

Склад капіталу треба розуміти в двоякому значенні. З боку
вартости він визначається тим відношенням, що в ньому капітал
поділяється на сталий капітал, або вартість засобів продукції,
і на змінний капітал, або вартість робочої сили, загальну суму
заробітних плат. З боку речовини, що функціонує в процесі
продукції, кожний капітал поділяється на засоби продукції й
живу робочу силу; з цього боку склад капіталу визначається від-

вартість певної кількости засобів існування, що розмірам їх сама природа
завжди ставить фатальні межі, які робітнича кляса марно намагається
переступити. Отже, сума, належна до розподілу між найманими робітниками,
є наперед визначена, а звідси випливає, що коли частина, яка
дістається кожному робітникові, є занадто мала, то це тому, що число
робітників занадто велике, і що бідність робітників, кінець-кінцем,
є наслідок не суспільного ладу, а природних умов.

Насамперед, межі, які капіталістична система ставить споживанню
продуцента, є «природні» лише в умовах цієї системи, цілком так
само, як батіг функціонує як «природна» спонука лише в умовах рабства.
В дійсності природі капіталістичної продукції властиве обмеження
частини продуцента тим, що є доконечне для підтримання його робочої
сили, так само, як ій властиве і захоплення додаткового продукту
капіталістом. Природі цієї системи продукції властиве також і те, що
додатковий продукт, який дістається капіталістові, поділяється ним
самим на дохід і додатковий капітал, тимчасом як робітник може лише
у виняткових випадках збільшити свій фонд споживання коштом фонду
споживання неробітників. «Багатий, — каже Сісмонді, — диктує закони
бідним... бо він сам переводить поділ річної продукції і залишає все
те, що він зве доходом, для самого себе, а все те, що він зве капіталом,
він відступає бідним, щоб вони з цього зробили для нього дохід» (Читай:
щоб вони з цього зробили для нього додатковий дохід). (Sismondi: «Nouveaux
Principes d’Economie Politique», v. I, p. 107—108)...

Отже, насамперед, економісти мусили б довести, що капіталістичний
спосіб суспільної продукції, не зважаючи на те, що він зовсім
недавно виник, все ж є незмінний і «природний» спосіб продукції. Але,
навіть припускаючи розміри капіталістичної системи за дані, неправда,
«що фонд заробітної плати» є наперед визначений величиною суспільного
багатства, або величиною суспільного капіталу.

Через те, що суспільний капітал є лише мінлива й хитка частина
суспільного багатства, фонд заробітної плати, являючи собою лише
певну частину цього капіталу, не може бути фіксованою і наперед визначеною
частиною суспільного багатства; з другого боку, відносна величина
фонду заробітної плати залежить від тієї пропорції, що в ній суспільний
капітал поділяється на капітал сталий і капітал змінний, а ця
пропорція, як ми вже бачили і як ми це докладно покажемо в дальших
розділах, не залишається незмінною протягом процесу акумуляції».
(«Le Capital etc.», ch. XXIV, § 5, p. 267—268). Ред.
