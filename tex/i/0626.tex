коло них заняття й утримання».208 Під приводом обгородження
сусідні лендлорди присвоювали собі не тільки землі, що лежали
перелогом, але часто-густо ще й землі, оброблювані або самою
громадою, або орендарями, які наймали їх у громади за певну
плату. «Я кажу тут про обгороджування відкритих полів і земель,
уже оброблених. Навіть ті письменники, що боронять
inclosures, визнають, що воно збільшує монополію великих
фарм, підвищує ціни засобів існування і продукує збезлюднення...
Навіть обгороджування гулящих земель, як це тепер практикують,
відбирає у бідняків частину їхніх засобів існування і надзвичайно
збільшує фарми, що і без того вже занадто великі».209
«Коли земля, — каже д-р Прайс, — попадає в руки небагатьох
великих фармерів, то дрібні фармери [що про них він раніш говорив
як про «масу дрібних власників і фармерів, що утримують
себе та свої родини продуктами оброблюваної ними землі, вівцями,
птицею, свиньми і т. ін., яких вони пасуть на громадській
землі, так що їм мало доводиться купувати засобів існування»]
перетворюються на людей, що примушені заробляти на життя
працею на інших і купувати на ринку все їм потрібне... Може,
тепер більше працюють, бо більше силують до праці... Міста й
мануфактури зростатимуть, бо до них зганяють більше людей,
які шукають роботи. Такий є неминучий вплив концентрації
фарм, і так вона фактично впливала в цьому королівстві протягом
багатьох років».210 Загальний вплив inclosures він резюмує
так: «У цілому становище нижчих народніх кляс майже з
кожного боку погіршало, дрібні землевласники й фармери позведені
до рівня поденників і наймитів, і в той самий час куди тяжче
стало за таких умов заробляти на життя».211 Справді, узурпація
громадської землі й революція в рільництві, що супроводила
цю узурпацію, мали такий гострий вплив на рільничих робітників,
що, як каже сам Ідн, між 1765 і 1780 рр. їхня заробітна

208    Reverend Addington: «An Inquiry into the Reasons for and against
enclosing open-fields», London 1772, p. 37—43 і далі.

209    Dr. R. Price: «Observations on Reversionary Payments 6 th ed.
By W. Morgan», London 1805, vol. II, p. 155. Прочитайте Форстера,
Едінґтона, Кента, Прайса й Джемса Андерсона й порівняйте з ними жалюгідне
сикофантське базікання Мак Куллоха у його каталогу «The
Literature of Political Economy», London 1845.

210    Там же, стор. 147.

211    Там же, стор. 159. Пригадаймо собі стародавній Рим. «Багаті
захопили у свої руки більшу частину неподілених земель. Тогочасні
обставини викликали в них упевненість, що земель у них уже не відберуть,
і тому вони поскуповували сумежні дільниці бідняків, почасти за
згодою останніх, почасти відбираючи їх силою, так що замість поодиноких
нив вони почали обробляти великі маєтки. При цьому вони для
рільництва й скотарства вживали рабів, бо вільних людей у них забрали б
від праці до військової служби. Володіння рабами давало їм ше йту
велику користь, що раби, звільнені від військової служби, могли спокійно
розмножуватись і мали багато дітей. Таким чином вельможні постягали
до своїх рук усі багатства, і ціла країна аж кишіла рабами. Навпаки,
італійців ставало раз-у-раз менше, їх нищили злидні, податки й військова
служба. Коли ж наставали мирні часи, то вони були засуджені
