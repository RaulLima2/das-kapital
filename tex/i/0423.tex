машин у рільництві здебільша вільний від шкідливого фізичного
впливу, справлюваного на фабричного робітника,\footnote{
Докладний опис машин, уживаних в англійському рільництві,
знаходимо в «Die landwirtschaftlichen Geräte und Maschinen Englands,
von Dr. W. Hamm». 2 Auflage 1856. У своєму нарисі про розвиток англійського
рільництва пан Гам надто некритично йде за паном Леонс де
Лявернь. [До четвертого видання. — Розуміється, тепер цей нарис застарів.
— Ф. Е].
}
то в «утворенні зайвих» робітників вони тут діють ще інтенсивніше
і не маючи собі в цьому опору, як ми це пізніше
побачимо в подробицях. У графствах Кембрідж і Суффолк, приміром,
площа обробленої землі за останні двадцять років дуже
поширилася, тимчасом як сільська людність за той самий період
не тільки відносно, а й абсолютно зменшилася. У Сполучених
штатах Північної Америки рільничі машини заміняють робітників
покищо лише у можливості, тобто вони дозволяють продуцентові
обробляти більшу площу землі, але не проганяють дійсно
занятих робітників. В Англії й Велзі число осіб, що працювали
у фабрикації рільничих машин, становило 1861 р. 1.034, тим часом
як число рільничих робітників, занятих коло парових і робочих
машин, становило лише 1.205.

У сфері рільництва велика промисловість діє якнайбільш
революційно в тому розумінні, що вона нищить твердиню старого
суспільства, «селянина», і висуває на його місце найманого робітника.
Таким чином потреби соціяльного перевороту й соціяльні
противенства\footnote*{
У французькому виданні тут замість «соціяльні противенства»
cказано «клясова боротьба». Ред.
} на селі доходять такого ж рівня, як і в місті.
Замість найрутиннішого і найнераціональнішого виробництва
постає свідоме технологічне застосування науки. Капіталістичний
спосіб продукції завершує розрив того первісного родинного
зв’язку рільництва з мануфактурою, який об’єднував дитинячі,
нерозвинуті форми одного і другої. Але разом із цим цей спосіб
продукції утворює матеріяльні передумови нової, вищої
синтези, а саме спілки рільництва і промисловости, на основі
їхніх антагоністично розвинених форм. Капіталістична продукція
в міру того, як перевага міської людности, яку вона стягає до
великих центрів, щораз більшає, — нагромаджує, з одного боку,
історичну силу руху суспільства, а з другого боку, перешкоджає
обмінові речовин між людиною й землею, тобто перешкоджає
повертанню ґрунтові тих його складових частин, які людина
зужила у формі харчових засобів і одягу, отже, вона порушує
вічну природну умову тривалої родючости ґрунту. Цим самим
вона одночасно руйнує фізичне здоров’я міських робітників і
інтелектуальне життя сільських робітників.\footnote{
«Ви розділяєте народ на два ворожі табори: на необтесаних мужиків
і слабовитих карликів. Боже мій! Нація, що розділилася на рільничі
й торговельні інтереси, вважає себе за здорову й називає себе
навіть освіченою й цивілізованою не наперекір, а саме в наслідок цьо-
}  Але, руйнуючи
спонтанейно посталі умови цього обміну речовин, капіталістична