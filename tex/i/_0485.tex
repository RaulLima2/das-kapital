\parcont{}  %% абзац починається на попередній сторінці
\index{i}{0485}  %% посилання на сторінку оригінального видання
на продукти вищої вартости, ніш вартість авансованого капіталу.
Це його продуктивне споживання. Одночасно воно є й
споживання його робочої сили капіталістом, що купив її. З другого
боку, робітник витрачає гроші, заплачені за його робочу
силу, на засоби існування: це його особисте споживання. Отже,
продуктивне й особисте споживання робітника цілком різні.
В першому він діє як рушійна сила капіталу і належить капіталістові;
у другому він належить собі самому й виконує життєві
функції поза процесом продукції. Результат першого є життя
капіталіста, результат другого — життя самого робітника.

Розглядаючи «робочий день» тощо, ми, між іншим, виявили,
що робітника часто примушують робити з свого особистого споживання
простий епізод процесу продукції. В цьому випадку
він додає до себе засоби існування, щоб підтримувати в русі
свою робочу силу, так само як до парової машини додають вугілля
й воду, до колеса — мастива й т. ін. В цьому випадку його засоби
споживання є лише засоби споживання одного із засобів продукції,
його особисте споживання — безпосередньо продуктивне
споживання. Однак це з’являється як зловживання, що неістотне
для капіталістичного процесу продукції.\footnote{
Россі не декламував би з приводу цього пункту з такою бундючністю.
коли б він справді збагнув таємницю «продуктивного споживання».
}

Інакше справа виглядає, коли ми розглядаємо не поодинокого
капіталіста й не поодинокого робітника, а клясу капіталістів
і клясу робітників, не ізольований процес продукції товару, а
капіталістичний процес продукції в його течії і в його суспільному
обсягу. — Коли капіталіст перетворює частину свого капіталу
на робочу силу, то тим самим він збільшує вартість цілого
свого капіталу. Він одним махом забиває двох зайців. Він має
зиск не тільки з того, що одержує від робітника, а ще й з того,
що він робітникові дає. Капітал, відчужений в обмін на робочу
силу, перетворюється на засоби існування, споживання яких
служить для того, щоб репродукувати мускули, нерви, кістки,
мозок наявних робітників і продукувати нових робітників. Тому
особисте споживання робітничої кляси в межах абсолютної
доконечности є зворотне перетворення засобів існування, відчужених
капіталом за робочу силу, на робочу силу, яку капітал
знову може експлуатувати. Воно є продукція й репродукція
найдоконечнішого для капіталіста засобу продукції — самого
робітника. Отже, особисте споживання робітника лишається
моментом продукції і репродукції капіталу, однаково, чи відбувається
воно всередині майстерні, фабрики й т. ін., чи поза ними,
в межах процесу праці чи поза ним, цілком так само, як чищення
машин незалежне від того, чи відбувається воно підчас процесу
праці, чи підчас певних перерв цього процесу. Справа ані
трохи не змінюється від того, що робітник здійснює своє особисте
споживання задля себе самого, а не задля капіталіста. Адже і
споживання робочої худоби не перестає бути доконечним момен-
\parbreak{}  %% абзац продовжується на наступній сторінці
