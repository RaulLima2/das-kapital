\parcont{}  %% абзац починається на попередній сторінці
\index{i}{0256}  %% посилання на сторінку оригінального видання
припускаючи, що вартість грошей не змінюється, — продукує
завжди ту саму нову вартість у 6 шилінґів, хоч би як ця сума
вартости поділялась між еквівалентом вартости робочої сили й
додатковою вартістю. Але якщо внаслідок підвищення продуктивної сили праці
вартість денних засобів існування, а тому й
денна вартість робочої сили падає з 5 шилінґів до 3, то додаткова
вартість зростає з 1 шилінґа до 3 шилінґів. Щоб репродукувати
вартість робочої сили раніше було потрібно 10, а тепер треба
лише 6 робочих годин. Чотири робочі години стали вільні, і їх
можна прилучити до сфери додаткової праці. Звідси іманентне
прагнення й постійна тенденція капіталу підвищувати продуктивну
силу праці, щоб здешевити товари та через здешевлення товарів
здешевити самого робітника.\footnote{
«У тій самій пропорції, в якій меншають видатки робітника, буде
зменшено і його заробітну плату, якщо тільки разом з цим промисловість
звільняють від обмежень» («In whatever proportion the expenses of a
labourer are diminished, in the same proportion will his wages be diminished,
if the restraints upon industry are at the same time taken off»).
(«Considerations concerningt taking off the Bounty on Corn exported etc.»,
London 1752, p. 7). «Інтереси промисловости вимагають, щоб хліб і
взагалі всякі харчові речі були якомога дешевші: бо те, що їх робить
дорожчими, робить дорожчою і працю\dots{} по всіх країнах, де промисловість
вільна від обмежень, ціна на предмети харчування мусить впливати
на ціну праці. Цю останню завжди понижують, коли дешевшають потрібні
засоби існування». («The interest of trade requires, that corn and all
provisions should be as cheap as possible; for whatver makes them dear, must
make labour dear also\dots{} in all countries, where industry is not restrained,
the price of provisions must affect the Price of Labour. This will always
be diminished when necessaries of life grow cheaper»). (Там же, стор. 3).
«Заробітну плату понижують в тій самій пропорції, в якій зростають
продуктивні сили. Правда, машини здешевлюють засоби існування, але
вони також і робітників роблять дешевшими». («Wages are decreased
in the same proportion as the powers of production increase. Machinery,
it is true, cheapens the necessaries of life, but it also cheapens the labourer
too»). («А Prize Essay on the comparative merits of Competition and
Cooperation», London 1834, p. 27).
}

Абсолютна вартість товару для капіталіста, що його продукує, сама по собі
байдужа. Капіталіста цікавить лише додаткова
вартість, що міститься в товарі й що її можна реалізувати в продажі. Реалізація
додаткової вартости включає й повернення авансованої вартости. А що відносна
додаткова вартість зростає просто пропорційно до розвитку продуктивної сили
праці, тимчасом як вартість падає зворотно пропорційно до того самого розвитку,
отже, що той самий ідентичний процес здешевлює товари та збільшує додаткову
вартість, яка міститься в них, то й розв’язується
та загадка, що капіталіст, який дбає лише про продукцію мінової
вартости, постійно намагається знизити мінову вартість товарів, —
суперечність, якою один з основників політичної економії, а
саме Кене, мучив своїх супротивників, що так і не дали йому на
неї відповіді. «Ви визнаєте, — каже Кене, — що чим більше
можна без шкоди для продукції заощадити витрат та зменшити
дорогі роботи при фабрикації промислових продуктів, тим корисніше
\index{i}{0257}  %% посилання на сторінку оригінального видання
це заощадження, бо воно зменшує ціну цих виробів. А, проте,
ви думаєте, що продукція багатства, яке походить з праці промисловців, полягає у
збільшенні мінової вартости їхніх виробів».\footnote{
«Іls conviennent que plus on peut, sans préjudice, épargner de frais ou
de travaux dispendieux dans la fabrication des ouvrages des artisans, plus
cette épargne est profitable par la diminution du prix de ces ouvrages.
Cependant ils croient que la production de richesse qui résulte des travaux
des artisans consiste dans l’augmentation de la valeur vénale de leurs ouvrages»).
(\emph{Quesnay}: «Dialogues sur le Commerce et sur les Travaux des
Artisans», ed. Daire, Paris 1846, p. 188, 189).
}

Отже, заощадження на праці\footnote{
«Ці спекулянти, що так багато заощаджують на праці робітників,
яку вони мусили б оплатити» («Ces spéculateurs si économes du travail
des ouvriers qu’il faudrait qu’ils payassent»). (\emph{J. N. Bidaull}: «Du
Monopole qui s’établit dans les arts industrielles et le commerce», Paris
1828, p. 13). «Підприємець завжди намагатиметься заощаджувати час і
працю» («The employer will be always on the stretch to economise
time and labour»). (\emph{Dugald Stewart}: Works ed. by Sir W. Hamilton.
Edinburgh 1885, vol. Ill, «Lectures on Political Economy», p. 318). «їхній
(капіталістів) інтерес вимагає того, шоб продуктивні сили зуживаних
ними робітників були якомога найбільші. Тому вони звертають свою
увагу майже виключно на збільшення цієї сили». («Their (the capitalists’)
interest is that the productive powers of the labourers they employ should
be the greatest possible. On promoting that power their attention is fixed
and almost exclusively fixed»). (\emph{R. Jones}: «Textbook of Lectures on
the Political Economy of Nations», Hertford 1852, Lecture III).
} внаслідок розвитку продуктивної сили праці за капіталістичної продукції
зовсім не має на меті скорочення робочого дня. Воно має на меті лише скорочення
робочого часу, доконечного для продукції певної кількости товарів.
Те, що робітник за підвищеної продуктивної сили його праці
продукує за одну годину, приміром, вдесятеро більше товарів,
ніж раніш, отже, на кожну штуку товару потребує вдесятеро
менше робочого часу, аж ніяк не заважає тому, що його тепер,
як і раніш, примушують працювати 12 годин та продукувати
протягом 12 годин 1.200 штук товару замість 120. Навіть більше:
його робочий день разом з тим може здовжуватися, так що він
тепер продукуватиме 1.400 штук за 14 годин, і т. ін. Тому в економістів
такої породи, як от Мак Кулох, Юр, Сеніор і tutti
quanti,\footnote*{
— всі, скільки їх є. \emph{Ред.}
} ми на одній сторінці читаємо, що робітник повинен дякувати капіталові за
розвиток продуктивних сил, бо цей останній
скорочує доконечний робочий час, а на другій сторінці — що
робітник мусить виявити цю вдячність, працюючи в майбутньому
15 годин замість 10. За капіталістичної продукції розвиток продуктивної сили
праці має на меті скоротити ту частину робочого
дня, протягом якої робітник мусить працювати для себе самого,
щоб саме цим здовжити другу частину робочого дня, протягом
якої він може працювати задурно на капіталіста. У якій мірі
можна досягти цього результату і не здешевлюючи товарів, це
виявиться при розгляді особливих метод продукції відносної
додаткової вартости, до чого ми тепер і переходимо.
