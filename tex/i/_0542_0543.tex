\parcont{}  %% абзац починається на попередній сторінці
\index{i}{0542}  %% посилання на сторінку оригінального видання
дуктивної сили їхньої праці, з поширенням і збільшенням усіх
джерел багатства поширюється й маштаб, в якому більше притягування
робітників капіталом зв’язане з більшим їх відштовхуванням,
зростає швидкість зміни органічного складу капіталу
та його технічної форми й ширшає коло тих сфер продукції, що
їх то одночасно, то навпереміну охоплює ця зміна. Отже, робітнича
людність, разом з продукованою нею самою акумуляцією
капіталу, продукує в щораз більшому розмірі засоби, які роблять
саму її відносно зайвою.\footnote{
Деякі видатні економісти клясичної школи більше передчували,
ніж розуміли закон прогресивного зменшення відносної величини змінного
капіталу і його вплив на становище кляси найманих робітників.
Найбільша заслуга в цій справі належить Джонові Бартону, хоч і він, як
усі інші, сплутує сталий капітал з основним, а змінний з обіговим. Він
каже: «Попит на працю залежить від зростання обігового капіталу, а
не основного. Коли б це була правда, що відношення поміж цими двома
відмінами капіталу за всяких часів і серед усяких обставин однакове,
то з цього випливало б, що число занятих робітників є пропорційне до
багатства держави. Але таке припущення не має й тіні ймовірности. Що
більше розвиваються промисли й поширюється цивілізація, то більше
й більше основний капітал переважає обіговий. Сума основного капіталу,
вживаного для продукції однієї штуки англійського мусліну, щонайменше
всотеро, а може і в тисячу разів більша, ніж основний капітал, що його
вживають на продукцію такої самої штуки індійського мусліну. А обіговий
капітал відносно в сто або тисячу разів менший\dots{} Коли б усю суму річних
заощаджень додавано до основного капіталу, то це все ж не спричинило
б жодного впливу на зростання попиту на працю». («The demand for
labout depends on the increase of circulating and not of fixed capital. Were
it true that the proportion between these two sorts of capital is the same at
all times, and in all circumstances, then, indeed, it follows that the number
of labourers employed is in proportion to the wealth of the state.
But such a proposition has not the semblance of probability. As arts are
cultivated, and civilization is extended, fixed capital bears a larger and
larger proportion to circulating capital. The amount of fixed capital employed
in the production of a piece of British muslin is at least a hundred, probably
a thousand times greater than that employed in a similar piece of
Indian muslin. And the proportion of circulating capital is a hundred or thousand times less\dots{} the whole of the annual savings, added to the fixed
capital, would have no effect in increasing the demand for labour»). (\emph{John Barton}: «Observations of the circumstances which influence the Condition
of the Labouring Classes of Society», London 1817, p. 16, 17). «Та
сама причина, в наслідок якої зростає чистий дохід країни, може одночасно
на другому боці зробити людність надмірною і погіршити становище
робітників» («The same cause which may increase the net revenue of the country
may at the same time render the population redundant, and deteriorate
the condition of the labourer»). (\emph{Ricardo}: «Principles of Political Economy»,
3 rd cd. London 1821, p. 469). Із збільшенням капіталу «попит (на
працю) відносно дедалі зменшується» («the demand (for labour) will be ina diminishing
ratio»). (Там же, стор. 480, примітка). «Сума капіталу, призначена
на утримання праці, може варіювати незалежно від якихбудь змін
у загальній сумі капіталу\dots{} Великі коливання в наявній кількості роботи
й великі страждання можуть ставати частішими в міру того, як сам капітал
зростає». (The amount of capital devoted to the maintenance of labour
may vary, independently of any changes in the whole amount of capital\dots{}
Great fluctuations in the amount of employment, and great suffering may
become more frequent as capital itself becomes more plentiful»). (\emph{Richard Jones}: «An Introductory Lecture on Political Economy», London 1833, p. 13).
«Попит (на працю) зростатиме\dots{} не пропорційно до акумуляції загального
капіталу\dots{} Тому всяке збільшення національного капіталу, призначеного
на репродукцію, з прогресом суспільства справлятиме щораз
менший вплив на становище робітника» («Demand (for labour) will rise\dots{}
not in proportion to the accumulation of the general capital\dots{} Every
augmentation, therefore to the national stock destined for reproduction,
comes, in the progress of society, to have a less and less influence upon the
condition of the labourer»). (\emph{G.Ramsay}: «An Essay on the Distribution,
of Wealth», Edinburgh 1836, p. 90, 91).

} Це є властивий капіталістичному
способові продукції закон населення, як і кожному осібному
історичному способові продукції в дійсності властиві свої осібні
закони населення, що мають історичне значення. Абстрактний
закон населення існує тільки для рослин і тварин, і то лише
остільки, оскільки вони не зазнають історичного впливу людини.

Але якщо надмірна робітнича людність є доконечний продукт
акумуляції, або розвитку багатства на капіталістичній основі,
то це перелюднення, з свого боку, стає підоймою капіталістичної
акумуляції і навіть умовою існування капіталістичного способу
продукції. Воно утворює резервну промислову армію, якою капітал
може порядкувати і яка абсолютно належить капіталові,
так, наче б він виростив її своїм власним коштом. Воно створює
для змінних потреб самозростання капіталу завжди готовий,
приступний для експлуатації людський матеріял, незалежно
від меж дійсного приросту людности. З акумуляцією й розвитком
продуктивної сили праці, що супроводить акумуляцію,
зростає сила раптового поширення капіталу не тільки через те,
\index{i}{0543}  %% посилання на сторінку оригінального видання
що зростають еластичність капіталу, який функціонує, і те абсолютне
багатство, що з нього капітал становить лише деяку
еластичну частину, не тільки через те, що кредит при кожній
особливій принаді, одразу ж віддає надзвичайну частину цього
багатства як додатковий капітал до розпорядження продукції,
— технічні умови самого процесу продукції, машини, засоби
транспорту й т. ін., в якнайбільшому маштабі уможливлюють
якнайшвидше перетворення додаткового продукту на
додаткові засоби продукції. Маса суспільного багатства, що зростає
з прогресом акумуляції, й що її можна перетворити на додатковий
капітал, несамовито рине в старі галузі продукції, що
їхній ринок раптом поширюється, або в нововідкривані галузі»
як ось залізниці й т. ін., що потреба на них випливає з розвитку
старих галузей продукції. В усіх таких випадках треба, щоб була
можливість раптом і без скорочення маштабу продукції в інших
сферах кидати великі маси людей на вирішальні пункти. Ці
маси постачає перелюднення. Характеристичний життьовий шлях
сучасної промисловости, ця форма перериваного невеликими
коливаннями десятирічного циклу періодів середнього оживлення,
продукції під високим тисненням, кризи й застою, ґрунтується
на постійному творенні, більшому або меншому поглиненні й
новоутворенні промислової резервної армії, або перелюднення.
З свого боку, мінливість фаз промислового циклу збільшує
людський матеріял для перелюднення і стає одним з найенерґійніших
факторів репродукції перелюднення.

