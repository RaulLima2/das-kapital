\parcont{}  %% абзац починається на попередній сторінці
\index{i}{0324}  %% посилання на сторінку оригінального видання
заміщувати працю та робітників перетворився відразу на засіб
збільшувати число найманих робітників, підбиваючи під безпосереднє
панування капіталу всіх членів родини без ріжниці статі
й віку. Примусова праця на капіталіста узурпувала не тільки час
дитячих забав, а ще й час вільної праці в колі домашніх, у прийнятих
звичаєм межах, для потреб самої родини.\footnote{
Підчас бавовняної кризи, що супроводила американську громадянську
війну, англійський уряд послав д-ра Едварда Сміса до Ланкашіру,
Чешіру й т. д., щоб дати звіта про стан здоров’я бавовняних робітників.
Е. Сміс, між іншим, повідомляє: «З погляду гігієни криза, крім
того, що вона витиснула робітників із фабричної атмосфери, дала чимало
й інших користей. Жінки робітників мають тепер потрібний вільний
час, щоб нагодувати груддю своїх дітей замість отруювати їх Cordial’ем
Ґодфрея (препаратом з опію). Вони тепер мають час учитися варити страви».
На нещастя, припало це куховарство на той час, коли вони не мали
чого їсти. Але ми бачимо, як капітал для свого самозростання узурпував
працю родини, потрібну для самого споживання родини. Так само кризу
використано на те, щоб по окремих школах учити дочок робітників шити.
Отже, треба було американської революції та світової кризи, щоб дівчата-робітниці,
які прядуть для цілого світу, навчилися шити.
}

Вартість робочої сили було визначено не тільки робочим
часом, потрібним, щоб утримати поодинокого дорослого робітника,
а ще й часом, потрібним, щоб утримати робітничу родину.
Викидаючи всіх членів робітничої родини на ринок праці, машини
розподіляють вартість робочої сили чоловіка на всю його родину.
Тому вони знижують вартість його робочої сили. Може бути купівля
родини, розпарцельованої на чотири робочі сили, коштує й
більше, ніж раніш коштувала купівля робочої сили голови родини,
але зате тепер чотири робочі дні стають на місце одного дня, і їхня
ціна падає пропорційно надлишкові додаткової праці чотирьох
над додатковою працею одного. Тепер для існування однієї родини
четверо мусять постачати капіталові не тільки працю, а ще й
додаткову працю. Таким чином машина з самого початку, разом
із збільшенням людського матеріялу експлуатації, цього справжнього
поля капіталістичного визиску,\footnote{
«Збільшення числа робітників було велике в наслідок дедалі
більшої заміни праці чоловіків працею жінок, а особливо праці дорослих
працею дітей. Троє дівчаток у віці 13 років із заробітною платою від 6
до 8 шилінґів на тиждень замінили дорослого чоловіка, що його плата
коливається між 18 і 45 шилінґами». («The numerical increase of labourers
has been great, through the growing substitution of female for male, and
above all of childish for adult, labour. Three girls of 13, at wages from of
6 sh. to 8 sh. a week, have replaced the one man of mature age, of wages
varying from 18 sh. to 45 sh.»). (\emph{Th. de Quincey}: «The Logic of Political
Economy», London 1844, p. 147 n.). Через те, що певних функцій родини,
як от, приміром, догляд та годування груддю дітей і т. д., не можна зовсім
усунути, то матері родин, конфісковані капіталом, мусять сяк чи так наймати собі заступників. Роботи, яких потребує споживання родини,
наприклад, шиття, латання й т. д., доводиться заміняти купівлею готових
товарів. Отже, зменшенню витрати хатньої праці відповідає збільшення
грошових видатків. Тому витрати продукції робітничої родини зростають
та врівноважують збільшення доходу. До цього долучається ще й те, що
економія та доцільність у використовуванні й готуванні засобів існування
стають неможливі. Про ці факти, які офіціяльна політична економія
затаює, можна знайти багатий матеріял у «Reports» фабричних інспекторів,
у звітах «Children’s Employment Commission», а особливо в «Reports
on Public Health».
} збільшує одночасно і ступінь експлуатації.

Машини також ґрунтовно революціонізують формальний вираз
капіталістичного відношення, контракт між робітником і капіталістом.
На основі товарового обміну першою передумовою було
те, що капіталіст і робітник протистояли один одному як вільні
\index{i}{0325}  %% посилання на сторінку оригінального видання
особи, як незалежні посідачі товарів, один — як посідач грошей
та засобів продукції, другий — як посідач робочої сили. А тепер
капітал купує неповнолітніх або напівповнолітніх. Раніш
робітник продавав свою власну робочу силу, що нею він порядкував
як формально вільна особа. Тепер він продає жінку
й дітей. Він стає работорговцем.\footnote{
Протилежно до того великого факту, що обмеження праці жінок
та дітей по англійських фабриках одвоювали від капіталу дорослі робітники-чоловіки,
ми знаходимо ще в найновіших звітах «Children’s Employment
Commission» справді обурливі та цілком гідні работорговців риси
у батьків-робітників щодо баришування дітьми. Але капіталістичні
фарисеї, як це можна бачити з тих самих «Reports», ще й викривають цю,
ними самими утворену, увіковічнену та експлуатовану жорстокість, яку
вони в інших випадках називають «волею праці». «Дитячу працю покликано
на поміч\dots{} навіть для того, щоб діти заробляли собі свій щоденний
шматок хліба. Без сил, потрібних, щоб витримати таку надмірну
працю, без навчання, потрібного для спрямовання їхнього дальшого життя,
їх кинуто в таке становище, що руйнувало їх фізично й морально. Єврейський
історик зауважив з приводу зруйнування Єрусалиму Титом, що
немає нічого дивовижного в тому, що підчас руйнування Єрусалиму його
так страшенно сплюндрували, коли вже якась нелюдяна мати навіть пожертвувала
свого власного сина, щоб заспокоїти муки страшного голоду».
(«Infant labour has been called into aid\dots{} even to work for their own daily
bread. Without strength to endure such disproportionate toil, without
instruction to guide their future life, they have been thrown into a situation
physically and morally polluted. The Jewish historian has remarked upon
the ovethrow of Jerusalem by Titus, that is was no wonder it should have
been destroyed, with such a signal destruction, when an inhuman mother
sacrificed her own offspring to satisfy the cravings of absolute hunger»).
(«Public Economy Concentrated», Carlisle 1833, p. 66).
} Попит на дитячу працю
часто і своєю формою подібний до попиту на негрів-рабів,
про який дуже часто можна було читати в об’явах американських
газет. «Мою увагу, — каже, приміром, один англійський
фабричний інспектор, — звернула на себе об’ява в місцевій
газеті одного з найзначніших мануфактурних міст моєї округи.
Ось копія цієї об’яви: Потрібно 12--20 хлопчаків, не молодших
від такого віку, щоб їх можна було вважати за 13-літніх.
Плата — 4 шилінґи на тиждень. Спитати й т. д.».\footnote{
 \emph{A.Redgrave} у «Reports of Insp. of Fact, for 31 st October
1858», p. 40, 41.
} Речення:
«щоб їх можна було вважати за 13-літніх» пояснюється тим, що
за фабричним законом дітям, молодшим від 13 років, можна
працювати тільки 6 годин. Лікар, що має визнану урядом кваліфікацію
(certifying surgeon), мусить засвідчити вік. Отже,

\parbreak{}  %% абзац продовжується на наступній сторінці
