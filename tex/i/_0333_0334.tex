\parcont{}  %% абзац починається на попередній сторінці
\index{i}{0333}  %% посилання на сторінку оригінального видання
останнє зужиткування до певної міри в зворотному відношенні
до її вжитку\footnote{
«Учинення\dots{} шкоди делікатним рухомим частинам металевого
механізму тим, що він не працює» («Occasion\dots{} injury to the delicate
moving parts of metallic mechanism by inaction»). (Ure: «Philosophy
of Manufacture», London 1835, p. 281).
}.

Але, крім матеріяльного зужиткування, машина зазнає, так би
мовити, і морального зужиткування. Вона втрачає мінову вартість
або в тій мірі, в якій машини тієї самої конструкції можуть бути
репродуковані дешевше, або в тій, в якій з нею починають конкурувати
кращі машини\footnote{
Згаданий уже раніш «Manchester Spinner» («Times», 26. Nov. 1862),
перелічуючи витрати на машини, каже: «це (а саме відрахування для покриття
зужиткування машини) призначено на повернення втрат, які постійно
постають через заміну машин, раніше ніж їх зужитковано, на інші,
нові й ліпшої конструкції» («It (allowance or deterioration of machinery)
is also intended to cover the loss which is constantly arising from the superseding
of machines before they are worn out by others of a new and better
«construction»).
}. В обох випадках, хоч і яка молода
та повна життєвої сили була б машина, її вартість визначається вже
не тим робочим часом, який фактично упредметнений у ній, а тим,
що тепер є доконечний на її репродукцію або на репродукцію
кращих машин. Тим то вона є більше або менше зневартнена.
Що коротший той період, протягом якого репродукується її цілу
вартість, то менша небезпека морального зужиткування, а що
довший робочий день, то коротший цей період. Коли машини вперше
вводяться в якусь галузь продукції, то раз-у-раз постають
нові методи дешевшої репродукції їх\footnote{
«Загалом вважають, що сконструювати одним-одну машину за
новим моделем коштує уп’ятеро дорожче, ніж реконструкція тієї самої
машини за тим самим моделем». (Babbage: «On the Economy of Machinery
and Manufactures», London 1832, p. 211).
} та поліпшення, які охоплюють
не тільки поодинокі частини або апарати, але й цілу їхню
конструкцію. Тому в перший період життя машин цей осібний
мотив до подовження робочого дня діє якнайгостріше\footnote{«Від декількох років у фабрикації тюлю пороблено такі значні та
численні поліпшення, що машину, яка добре збереглася й коштувала
первісно \num{1.200}\pound{ фунтів стерлінґів}, через декілька років продавали за
60\pound{ фунтів стерлінґів}\dots{} Поліпшення поставали одне по одному з такою
швидкістю, що машини лишалися в руках машинобудівельників незакінченими,
бо через щасливіші винаходи вони були вже застарілі». Тим то
в цей період «бурі й натиску» фабриканти тюлю незабаром збільшили первісний
восьмигодинний робочий день до 24 годин з подвійною зміною
робітників. (Там же, стор. 281).}.

За інших незмінних обставин та за даного робочого дня експлуатація
подвійного числа робітників потребує подвоєння так
тієї частини сталого капіталу, що її витрачається на машини й
будівлі, як і тієї, що її витрачається на сировинний матеріял,
допоміжні матеріяли й~\abbr{т. д.} Із здовженням робочого дня маштаб
продукції ширшає, тимчасом як частина капіталу, витрачена
на машини та будівлі, лишається незмінна\footnote{
«Само собою очевидно, що з припливами та відпливами на ринку
та за навперемінного поширення та скорочення попиту постійно трапляються випадки, коли фабрикант може вжити додаткового обігового капіталу,
не вживаючи додаткового основного капіталу\dots{} якщо додаткову
кількість сировинного матеріялу можна переробити без додаткових витрат
на будівлі та машини» («It is self-evident, that, amid the ebbings and
flowings of the market, and the alternate expansions and contractions of
demand, occasions will constantly recur, in which the manufacturer may
employ additional floating capital without employing additional fixed
capital\dots{} if additional quantities of raw material can be worked up without
incurring an additional expence for buildings and machinery»).
(R.~Torrens: «On Wages and Combination», London 1834, p. 64).
}. Тому не тільки
\index{i}{0334}  %% посилання на сторінку оригінального видання
зростає додаткова вартість, але й меншають витрати, доконечні
для того, щоб її добути. Правда, в більшій або меншій мірі це
буває завжди за всякого здовження робочого дня, але тут це
має вирішальніше значення, бо частина капіталу, перетворена
на засоби праці, тут взагалі має більше значення\footnote{
Згадану в тексті обставину я наводжу тільки для повноти, бо
норму зиску, тобто відношення додаткової вартости до цілого авансованого
капіталу, я розглядаю лише в третій книзі.
}. Дійсно,
розвиток машинового виробництва зв’язує чимраз більшу складову
частину капіталу в такій формі, в якій вона, з одного боку, може
постійно самозростати своєю вартістю, а з другого боку, втрачає
споживну вартість і мінову вартість, скоро тільки переривається
її контакт із живою працею. «Коли, — повчав пан Ешворт,
англійський бавовняний маґнат, професора Нассау В.~Сеніора, —
коли рільник кидає свій заступ, то він на цей час робить некорисним
капітал у 18\pens{ пенсів.} Коли один із наших людей (тобто
з фабричних робітників) кидає працю на фабриці, то він робить
некорисним капітал, який коштував \num{100.000}\pound{ фунтів стерлінґів}»\footnote{
«When a labourer», said Mr.~Ashworth, «lays down his spade, he
renders useless, for that period, a capital worth 18 d. When one of our people
leaves the mill, he renders useless a capital that has cost \num{100.000} pounds».
(Senior: «Letters on the Factory Act», London 1837, p. 13, 14).
}.
Подумати тільки! «Зробити некорисним», хоча б лише на хвилину,
капітал, що коштував \num{100.000}\pound{ фунтів стерлінґів}. Це — справді
жахна річ, що один із наших людей взагалі може колибудь кинути
фабрику! Чимраз більше зростання розміру машин робить «бажаним»
— визнає повчений від Ешворта Сеніор — щораз більше
й більше здовжування робочого дня\footnote{
«Велика пропорція основного капіталу проти обігового\dots{} робить
бажаним довгий робочий день» («The great proportion of fixed to circulating
capital\dots{} makes long hours of work desirable»). Із зростом розміру
машин і~\abbr{т. д.} «мотиви до здовження робочого дня дедалі посилюються,
бо це єдиний засіб, щоб зробити зисковною відносно велику масу основного
капіталу» («the motives to long hours of work will become greater,
as the only means by which a large proportion of fixed capital can be
made profitable»). (Там же, стор. 11--13). «На фабриці є різні витрати,
які лишаються постійними, незалежно від того, чи робочий час на
фабриці довший, чи коротший, наприклад, орендна плата за будівлі, місцеві
та загальні податки, убезпечення від огню, заробітна плата різним
постійним робітникам, псування машин та різні інші витрати, пропорція
яких до зиску меншає в такому самому відношенні, в якому
зростає розмір продукції». («Reports of Insp. of Fact, for 31 st October
1862», p. 19).
}.
