\parcont{}  %% абзац починається на попередній сторінці
\index{i}{*0100}  %% посилання на сторінку оригінального видання
«ця фальсифікація», «попросту мерзота» й т. ін., він тимчасом
вважає за потрібне перенести спірне питання на інше поле й тому
то обіцяє «у другій статті пояснити, яке значення надаємо ми
[не «брехливий» анонім] змістові слів Ґледстона». Наче б ця
його неавторитетна думка мала хоча б найменше відношення до
справи! Цю другу статтю надруковано в «Concordia» з 11 липня.

Маркс ще раз відповів у «Volksstaat’i» з 7 серпня, процитувавши
тепер ще й відповідне місце із звітів «Morning Star’а» і
«Morning Advertiser’a» від 17 квітня 1863 р. За обома Ґледстон
каже, що він поглядав би з занепокоєнням і т. ін. на це приголомшливе
збільшення багатства й сили, коли б думав, що воно
справді обмежується на дійсно заможних клясах (classes in essay
circumstances). Алеж це збільшення обмежується на клясах,
що мають власність (entirely confined to classes possessed of
property). Отже, і ці звіти подають дослівно те нібито «прибріхане»
речення. Далі, порівнявши тексти «Times’a» й «Hansard’а»,
Маркс ще раз констатує, що речення, яке дійсно було
таки сказане, як це потверджують однаковісінькі до слова, незалежні
один від одного, звіти трьох газет, які вийшли найближчого
ранку, — що цього речення немає в звіті «Hansard’а», переглянутому
згідно з відомим «звичаєм», що Ґледстон його,
висловлюючись словами Маркса, «поцупив заднім числом»; і
нарешті, Маркс заявляє, що в нього немає часу на дальшу полеміку
з анонімом. Цьому останньому теж було цього всього досить —
принаймні, Марксові не надсилано дальших чисел «Concordia».

Тим справу, здавалося, вбито й поховано. Правда, раз або
двічі особи, що були в зв’язку з Кембріджським університетом,
поширювали таємні чутки про невимовне літературне злочинство,
що його ніби вчинив Маркс у «Капіталі»; але, не зважаючи на
всі розпити, абсолютно не пощастило довідатися чогось певнішого.
Та ось 29 листопада 1883 р., вісім місяців після смерти
Маркса, появився в «Times’i» лист, помічений Trinity College,
Cambridge, за підписом Sedley Taylor, де цей чоловічок, що вправляється
в щонайтихішій кооперації, ні сіло, ні впало, подає нам,
нарешті, пояснення не лише кембріджських пліток, але й про
аноніма з «Concordia».

«Видається надзвичайно дивним, — каже чоловічок з Trinity
College, — що професорові Брентано (тоді у Бреславі, тепер у
Штрасбурзі) пощастило\dots{} викрити mala fides, яка, очевидно,
подиктувала цитату з промови Ґледстона в [inaugural] адресі.
П. Карл Маркс, що\dots{} силкувавсь боронити цитату, в передсмертних
корчах (deadly shifts), — а їх завдали йому дуже швидко
майстерні удари Брентано, — мав ще сміливість запевняти, що
Ґледстон зліпив звіт своєї промови для «Hansard’а» після того,
як вона появилась у «Times’i» 17 квітня 1863 р., підчистивши
місце, яке, безперечно, компромітувало його як англійського
канцлера скарбу. Коли ж Брентано, порівнявши до останніх
подробиць тексти, довів, що звіти «Times’a» й «Hansard’а» збігаються
в тому, що в них абсолютно виключений такий сенс
\parbreak{}  %% абзац продовжується на наступній сторінці
