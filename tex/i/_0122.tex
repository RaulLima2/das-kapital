\parcont{}  %% абзац починається на попередній сторінці 
\index{i}{0122}  %% посилання на сторінку оригінального видання 
такий високий рівень розвитку поділу праці всередині суспільства,
коли відокремлення між споживною вартістю й міновою
вартістю, яке за безпосередньої мінової торговлі, лише почина
ється, є вже завершене. Але такий ступінь розвитку є спільний
історично якнайрізнішим економічним суспільним формаціям.

Або, коли б ми стали розглядати гроші, то ми побачили б,
що вони мають за свою передумову певний розвиток товарового
обміну. Осібні форми грошей — просто товаровий еквівалент,
або засіб циркуляції, або засіб платежу, скарб і світові гроші —
вказують, залежно від різного обсягу і відносної переваги тієї
або іншої функції, на дуже різні ступені суспільного процесу
продукції. А проте для утворення всіх цих форм досить, як це
доводить досвід, порівняно слабо розвиненої товарової циркуляції.
Інша справа з капіталом. Історичні умови його існування зовсім
не збігаються з товаровою та грошовою циркуляцією. Капітал
постає лише там, де посідач засобів продукції й засобів існування
находить на ринку вільного робітника як продавця своєї робочої
сили, і вже одна ця історична умова обіймає цілу світову історію.
Тому капітал із самого початку свого виникнення проголошує
нову епоху суспільного процесу продукції.\footnote{
Отже, капіталістичну епоху характеризує те, що робоча сила для
самого робітника набирає форми належного йому товару, а тому його
праця набирає форми найманої праці. З другого боку, лише від цього моменту
товарова форма стає загальною для всіх продуктів праці.
}

А тепер треба ближче розглянути цей своєрідний товар, робочу
силу. Нарівні з усіма іншими товарами він має вартість.\footnote{
«Вартість або цінність людини, як і всіх інших речей, є її ціна,
тобто стільки, скільки дають за користання з її сили» («The Value or
Worth of a man is as of all other things, his price: that is to say, so much
as would be given for the use of his power»). (Th. Hobbes: «Leviathan»
in «Works ed. Moleswortli», London 1839—44, vol. Ill, p. 76).
} Як
визначається цю вартість?

Вартість робочої сили, як і вартість кожного іншого товару,
визначається робочим часом, доконечним для продукції, отже,
і репродукції цього специфічного товару. Оскільки робоча сила
є вартість, вона сама репрезентує лише певну кількість упредметненої
в ній суспільної пересічної праці. Робоча сила існує
лише як здібність живого індивіда. Отже, його існування є передумова
її продукції. Коли існування індивіда дано, то він продукує
свою робочу силу, репродукуючи або зберігаючи самого себе.
Для свого самозбереження живий індивід потребує певної суми
засобів існування. Отже, робочий час, доконечний для продукції
робочої сили, сходить на робочий час, доконечний для продукції
цих засобів існування, або вартість робочої сили є вартість засобів
існування, доконечних, щоб зберегти життя її власникові.
Однак робоча сила здійснюється лише через своє зовнішнє виявлення,
виявляється лише у праці. Але в процесі її виявлення,
в процесі праці витрачається певну кількість людських м’яснів,
нервів, мозку й т. ін., що мусять бути знову поповнені. Ця збіль-
\parbreak{}  %% абзац продовжується на наступній сторінці
