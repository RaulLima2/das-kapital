представників, згоджується на такий переворот тільки під натиском загального парляментського
закону»,\footnote{
«Цьому можна було запобігти, — каже один фабрикант, — коштом
поширення майстерень під натиском загального парляментського закону»
(«This could be obviated at the expense of an enlargement of the works
under the pressure of a General Act of Parliament»). (Там же, стор. X,
n. 38).
} який законодатним
примусом реґулює робочий день.

9. Фабричне законодавство (постанови про санітарні заходи й виховання). Загальне поширення його в
Англії

Фабричне законодавство, це перше свідоме й пляномірне реагування суспільства на спонтанейно вирослу
форму його продукційного процесу, є, як ми бачили, такий самий доконечний продукт великої
промисловости, як і бавовняна пряжа, сельфактори
і електричний телеграф. Перше ніж перейти до його загального
поширення в Англії, ми спинимося ще коротко на деяких постановах англійського фабричного закону, які
не стосуються до
числа годин робочого дня.

Постанови про санітарні заходи, не кажучи вже про їхню
редакцію, яка полегшує капіталістові обходити їх, надзвичайно
мізерні і в дійсності обмежуються приписами щодо біління стін
і деякими іншими правилами щодо чистоти, вентиляції й захисту
від небезпечних машин. У третьому томі ми повернемося до фанатичної боротьби фабрикантів проти тієї
постанови, яка октроювала їм невеличкі видатки на захист частин тіла їхніх «рук».
Тут знову блискуче стверджується догма фритредерства, ніби в
суспільстві з антагоністичними інтересами всяк сприяє загальному добру, добиваючися власної користи.
Досить буде одного
прикладу. Відомо, що за останній двадцятилітній період в Ірландії дуже зросла лляна промисловість, а
разом із нею і scutching
mills (фабрики для тіпання льону). 1864 р. там було 1.800 таких
mills. Періодично, восени й зимою, одривають від роботи на
полі переважно підлітків та жінок, синів, дочок і дружин сусідніх дрібних фармерів — людей, цілком
необізнаних з машинами,
щоб поставити їх до цих scutching mills подавати льон у вальцівні.
Нещасливі випадки по цих фабриках щодо розміру й інтенсивности зовсім не мають собі прикладів в
історії машин. Однимодна
scutching mill у Kildinan’i (коло Корку) від 1852 до 1856 р.
налічувала шість випадків смерти і 60 тяжких покалічень, при
чому всім їм можна було запобігти за допомогою якнайпростіших пристроїв, ціною в декілька шилінґів.
Др. В. Байт, certifying
surgeon\footnote*{
— лікар, що видає посвідки. Ред.
} * фабрик у Downpatrick’у, заявляє в офіціяльному
звіті з 15 грудня 1865 р.: «Нещасливі випадки на scutching
mills мають якнайжахливіший характер. У багатьох випадках одриває четверту частину тіла від тулуба.
Смерть або будучина інваліда, повна страждань — це звичайні наслідки ран.