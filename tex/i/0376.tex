дукуються знову ж таки за допомогою машин; скоро тільки відбувається
революція в добуванні вугілля й заліза, а також в обробленні
металів і в транспортовій справі; одне слово, скоро тільки
будуть створені загальні умови продукції, відповідні великій
індустрії, — з цього моменту цей спосіб виробництва набуває елястичности,
здатности до раптового стрибкуватого поширювання, що
не має інших меж, як тільки в сировинному матеріялі та в ринку
для збуту. Машини, з одного боку, безпосередньо сприяють збільшенню
сировинного матеріялу, як, наприклад, cotton gin збільшила
продукцію бавовни.\footnote{
Про інші методи, якими машини впливають на продукцію сировинного
матеріялу, буде згадано в третій книзі.
} З другого боку, дешевина машинового
продукту та переворот у засобах транспорту й комунікації є
знаряддя завоювати чужі ринки. Руйнуючи на цих ринках ремісничу
продукцію, машинове виробництво силоміць перетворює
їх на поля продукції свого сировинного матеріялу. Так, Східня
Індія була примушена продукувати для Великобрітанії бавовну,
вовну, коноплю, джут, індиґо тощо.\footnote{
Вивіз бавовни із Східньої Індії до Великобрітанії:
} Постійне «перетворювання»
робітників у «зайвих» по країнах великої промисловости примушує
до штучної еміґрації й колонізації чужих країн, які перетворюються
на місця продукції сировинного матеріялу для метрополії,
як, наприклад, Австралія перетворилася в місце продукції
вовни.\footnote{
Вивіз вовни з рогу Доброї Надії до Великобрітанії:
} Створюється новий, відповідний до розташування головних
центрів машинового виробництва, міжнародній поділ праці,
який перетворює одну частину земної кулі переважно на поле
рільничої продукції [для другої частини земної кулі, яка стає
переважно полем промислової продукції].\footnote*{
Заведене у прямі дужки ми беремо з другого німецького видання.
\emph{Ред.}
} Ця революція стоїть
у зв’язку з переворотами в рільництві, що їх ми тут ще не розглядаємо
докладніше.\footnote{
Самий економічний розвиток Сполучених штатів є продукт европейської,
особливо англійської, великої промисловости. Сполучені штати
в їхньому теперішньому вигляді (1866 р.) все ще треба розглядати як
колонію Европи. [До четвертого видання. Від того часу вони розви-
}

З ініціятиви пана Ґледстона Палата громад 17 лютого 1867 р.
наказала зібрати статистичні відомості про вивіз та довіз в
Об’єднане Королівство всякого роду збіжжя й борошна за час
від 1831 до 1866 р. Нижче я подаю зведення цих статистичних\footnote{
р. — 4.570.581 фунт, 1860 р. — 20.214.173 фунти, 1865 р. —
20.679.111 фунтів.
} р. — 34.540.143 фунти, 1860 р. — 204.141.168 фунтів, 1865 р. —
445.947.600 фунтів.

Вивіз вовни із Східньої Індії до Великобрітанії:\footnote{
р. — 21.789.346 фунтів, 1860 р. — 59.166.616 фунтів, 1865 р. —
109.734.261 фунт.
} р. — 2.958.457 фунтів, 1860 р. — 16.574.345 фунтів, 1865 р. —
29.220.623 фунти.

Вивіз вовни з Австралії до Великобрітанії: