середніх потреб самозростання капіталу, є умова існування сучасної
промисловости.

«Припустімо, — каже Г. Мерівел, раніш професор політичної
економії в Оксфорді, а потім урядовець англійського міністерства
колоній, — припустімо, що нація з нагоди якоїсь кризи напружить
свої сили, щоб за допомогою еміґрації позбутися кількох сот
тисяч зайвих бідних. Який був би з цього наслідок? Такий, що
при першому ж відновленні попиту на працю була б недостача
робітників. Хоч і як швидко відбуватиметься репродукція людей,
вона в усякому разі потребує для заміни дорослих робітників
переміжну часу однієї їенерації. Але зиски наших фабрикантів
залежать переважно від спроможности використовувати сприятливий
момент жвавого попиту й таким чином відшкодовувати
себе за часи застою. Цю спроможність фабрикантам забезпечує
тільки панува'ння над машинами й ручною працею. Для них
повинні знайтись вільні руки; вони повинні бути здібні в разі
потреби дужче напружувати або зменшувати активність своїх
операцій відповідно до стану ринку, бо інакше вони ніяк не зможуть
серед шаленої конкуренції втримати ту перевагу, на якій
основано багатство цієї країни».80 Навіть Малтуз у перелюдненні,
яке він з свого обмеженого погляду пояснює абсолютним
надмірним приростом робітничої людности, а не тим, що вона
стає відносно надмірною, визнає доконечність для сучасної промисловости.
Він каже: «Мудрі звички щодо шлюбу, доведені
до певної височини серед робітничої кляси якоїсь країни, яка
залежить головним чином від мануфактури й торговлі, були б
для цієї країни шкідливі... Відповідно до самої природи людности,
приріст робітників не може бути поданий на ринок у наслідок
особливого попиту раніше, ніж мине 16 або 18 років, а перетворення
доходу на капітал через заощадження може відбуватися
куди швидше; країні завжди загрожує, що її робочий фонд зростатиме
швидше, ніж людність».81 Оголосивши таким чином
постійну продукцію відносного перелюднення робітників доконечністю
капіталістичної акумуляції, політична економія ціл-

80    Н. Merivale: «Lectures on Colonization and Colonies», London 1841
and 1842, vol. I, p. 146.

81 «Prudential habits with regard to marriage carried to a considerable
extent among the labouring class of a country mainly depending upon manufactures
and commerce might injure it... From the natute of a population,
an increase of labourers cannot be brought into market, in consequence
of a particular demand, till after the lapse of 16 or-18 years, and the conversion
of revenue into capital, by saving, may take place much more rapidly;
a country is always liable to an increase in the quantity of the funds
for the maintenance of labour faster than the increase of population»).
(Malthus: «Principles of Political Economy», p. 254, 319, 320). У цій праці
Малтуз відкриває, нарешті, за допомогою Сісмонді, прегарну трійцю капіталістичної
продукції: перепродукцію — перелюднення — переспоживання,
три справді любісінькі почвари (three very delicate monsters, indeed)!
Порівн. F. Engels: «Umrisse zu einer Kritik der Nationalökonomie» in
Deutsch-Französische Jahrbücher, herausgegeben von Arnold Rüge
und Karl Marx, Paris 1844, S. 107 ff.
