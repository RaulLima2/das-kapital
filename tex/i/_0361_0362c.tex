\index{i}{0361}  %% посилання на сторінку оригінального видання 
Загальний результат механічних поліпшень, заведених в англійській
бавовняній промисловості під впливом американської
громадянської війни, показує оця таблиця:

Число фабрик

                                                                      1858 р.                       
1861 р.                1868 р.

Англія та Велз.......................                   2.046                          2.715        
          2.405
Шотляндія.............................                   152                               163      
                131
Ірляндія................................                     12                                    9
                        13
Об’єднане Королівство.........                 2.210                            2.887               
   2.549

Число парових ткацьких  варстатів

Англія та Велз......................              275.590                        368.125            
  344.719
Шотляндія............................               21.624                          30.110          
       31.864
Ірляндія...............................                  1.633                            1.757     
              2.746
Об’єднане Королівство.........             298.847                        399.992              
379.329

Число веретен

Англія та Велз.....................       25.818.576                    28.352.152          
30.478.228
Шотляндія..........................          2.041.129                      1.915.398             
1.397.546
Ірляндія.............................              150.512                         119.944          
      124.240
Об’єднане Королівство.......        28.010.217                    30.387.494            32.000.014

Чиcло вживаних робітників

Англія та Велз...................               341.170                          407.598            
    357.052
Шотляндія........................                  34.698                            41.237         
        39.809
Ірляндія...........................                     3.345                              2.734    
               4.203
Об’єднане Королівство.....               379.213                           451.569               
401.064

Отже, від 1861 до 1868 р. зникло 338 бавовняних фабрик, тобто
продуктивніший та більший машиновий механізм сконцентрувався
в руках меншого числа капіталістів. Число парових ткацьких
варстатів зменшилося на 20.663; але продукт їхній одночасно
збільшився, так що поліпшений ткацький варстат давав
тепер більше продукту, ніж старий. Нарешті, число веретен
зросло на 1.612.541, тимчасом як число вживаних робітників
зменшилося на 50.505. Отже, ті «тимчасові» злидні, що ними бавовняна
криза душила робітників, збільшив і зміцнив хуткий
та невпинний проґрес машинової системи.

Однак машина діє не тільки як непереможний конкурент,
який завжди напоготові зробити найманого робітника «зайвим».
Капітал голосно й тенденційно проголошує її силою, ворожою
робітникові, та саме як таку вживає її. Вона стає наймогутнішим
бойовим знаряддям придушувати періодичні робітничі повстання,
страйки і т. ін. проти автократії капіталу.\footnote{
«Відносини між хазяїнами й руками по фабриках флінтґлясу та пляшкового
скла — це хронічний страйк». Звідси швидкий розвиток мануфактури
пресованого скла, де головні операції виконуються за допомогою машин.
Одна фірма в Ньюкестлі, яка раніш продукувала 350.000 фунтів дутого
кремінного скла річно, тепер замість цієї кільцости продукує 3.000.500
фунтів пресованого скла». («Children’s Employment Commission. 4 th
Report 1865», p. 262, 263).
} За Ґаскелем,
\index{i}{0362}  %% посилання на сторінку оригінального видання 
парова машина з самого початку була антагоністом «людської
сили», що дав капіталістам змогу розбивати щораз більші
домагання робітників, які загрожували кризою фабричній системі
на самому початку її виникнення.\footnote{
Gaskell: «The Manufacturing Population of England», London
1833, p. 3, 4.
} Можна було б написати
цілу історію винаходів, які, починаючи від 1830 р., покликано
до життя лише як бойове знаряддя капіталу проти повстань робітників.
Ми нагадаємо передусім selfacting mule,\footnote*{
— автоматичну прядільну машину. Ред.
} бо нею починається
нова епоха автоматичної системи.\footnote{
Деякі дуже важливі застосування машин, щоб будувати машини,
винайшов п. Ферберн під впливом страйків на його власній фабриці.
}

У своєму свідченні перед комісією, що їй доручено було дослідити
Trades-Unions, Несміс, винахідник парового молота, повідомляє
про поліпшення в машинах, які він завів у наслідок
великого та довгого страйку машинових робітників у 1851 р.,
таке: «Характеристична риса наших сучасних механічних поліпшень
— це заведення самодіяльних виконавчих машин. Все, що
тепер має робити механічний робітник, і що може зробити всякий
підліток, — це не самому працювати, а лише наглядати за прегарною
роботою машини. Цілу клясу робітників, що залежить
виключно від своєї вмілости, тепер усунено. Раніш я на одного
механіка мав чотирьох хлопців. Завдяки цим новим механічним
комбінаціям я зменшив число дорослих чоловіків з 1.500 на 750.
Наслідком цього було значне збільшення мого зиску».

Про одну машину для друку фарбами на перкалевибійних
фабриках Юр каже: «Нарешті капіталісти почали шукати способу
визволитися з-під цієї нестерпної неволі (тобто від тяжких
для них умов контракту з робітниками), покликавши собі на допомогу
джерела науки, і незабаром їх відновили в їхніх законних
правах, правах голови над іншими частинами тіла». Про один
винахід для шліхтування основи, що його безпосередньою причиною
був страйк, він каже так: «Орда незадоволених, що, окопавшися
за старими лініями поділу праці, вважала себе за непереможну,
побачила себе таким чином оточеною з флангів, а свої
оборонні засоби знищеними сучасною механічною тактикою. Вони
мусили здатися на ласку та гнів переможців». Про винахід
selfacting mule він каже: «Вона була покликана, щоб відновити
порядок серед промислових кляс... Цей винахід потверджує
розвинуту вже нами доктрину, що капітал, примусивши науку
служити йому, завжди силує бунтівничу руку праці до покірливости».\footnote{
Ure: «Philosophy of Manufacture», стор. 367—370.
}
Хоч твір Юра з’явився 1835 р., отже, за часів порівняно
мало ще розвинутої фабричної системи, все ж він лишається клясичним
виразом духу фабрики не тільки через свій щирий цинізм,
але й через ту наївність, з якою він виказує абсурдні суперечності
капіталістичного мозку. Розвинувши, приміром, «доктрину»,
що капітал за допомогою науки, взятої ним на утримання, «завжди
\parbreak{}  %% абзац продовжується на наступній сторінці
