Ми вже показали, що додаткова вартість не може виникнути
з циркуляції, отже, при її утворенні за спиною циркуляції мусить
зчинитися щось таке, чого в ній самій не можна помітити.36 Але
чи може додаткова вартість виникнути звідкись інакше, опріч
циркуляції? Циркуляція є сума всіх товарових взаємовідносин
посідачів товарів. Поза нею посідач товарів стоїть лише у відношенні
до свого власного товару. Щождо вартости товару, то це
відношення обмежується на тім, що в товарі міститься певна
кількість власної праці товаропосідача, вимірюваної за певними
суспільними законами. Ця кількість праці виражається у величині
вартости його товару, а що величина вартости виражається
в рахункових грошах, то ця кількість праці виражається, приміром,
у ціні в 10 фунтів стерлінґів. Але його праця не виражається
у вартості товару і в лишкові понад власну вартість товару,
не виражається в ціні в 10, яка одночасно є ціна в 11, не виражається
у вартості, яка більша за себе саму. Посідач товарів
може своєю працею створювати вартості, але він не може створювати
вартості, що самозростають. Він може підвищити вартість
товару, додаючи новою працею до наявної вартости нову вартість,
приміром, виготовляючи із шкури чоботи. Той самий матеріял
має тепер більшу вартість, бо в ньому міститься більша кількість
праці. Тому чоботи мають більшу вартість, аніж шкура, алеж вартість
шкури лишилась такою, якою вона була. Вона не зросла,
не прилучила до себе додаткової вартости підчас продукції чобіт.
Отже, неможливо, щоб товаропродуцент поза сферою циркуляції,
не стикаючися з іншими посідачами товарів, збільшував вартість,
а тому й неможливо, щоб він поза сферою циркуляції перетворював
гроші або товар на капітал.

Отже, капітал не може виникнути з циркуляції і так само не
може виникнути поза циркуляцією. Він мусить виникнути одночасно
в циркуляції і не в ній.

Таким чином виявився подвійний результат.

Перетворення грошей на капітал слід розвинути на основі
законів, іманентних товаровій циркуляції, так, щоб обмін еквівалентів
правив за вихідний пункт.37 Наш посідач грошей, який

36 «Серед звичайних умов ринку обмін не створює зиску. Коли його
не було раніш, то його не створиться й після цієї оборудки». («Profit, in
the usual condition of the market, is not made by exchanging. Had it not
existed before, neither could it after that transaction»). (Ramsay: «An
Essay on the Distribution of Wealth», Edinburgh 1836, p. 184).

37  Після поданих пояснень читач розуміє, що це значить тільки ось
що: утворення капіталу мусить бути можливе й тоді, коли ціни товарів
дорівнюють їхнім вартостям. Утворення капіталу не можна пояснити відхиленням
товарових цін від товарових вартостей. Коли ціни дійсно
відхиляються від вартостей, то треба їх спочатку звести на останні, тобто
залишити цю обставину як випадкову осторонь, щоб мати перед собою в
чистій формі явище утворення капіталу на основі товарового обміну, і
щоб, спостерігаючи його, не заплутатись через побічні обставини, що
ускладнюють самий процес і є чужі для нього. Нарешті, відомо, що ця
редукція ніяким чином не є лише наукова процедура. Постійні коливання
ринкових цін, їхнє піднесення та зниження, урівноважуються,
взаємно
