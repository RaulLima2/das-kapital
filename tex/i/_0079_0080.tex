\parcont{}  %% абзац починається на попередній сторінці
\index{i}{0079}  %% посилання на сторінку оригінального видання
перших представників її в тій недоречній гіпотезі, що в процес
циркуляції товари входять без ціни, а гроші — без вартости, а
потім, у процесі циркуляції, певну частину товарової купи обмінюється
на певну частину гори металю\footnote{
Само собою зрозуміло, що кожний окремий рід товару своєю ціною
становить елемент суми цін усіх товарів, які циркулюють. Алеж зовсім
незрозуміло, як неспільномірні одна з одною споживні вартості мають
en masse\footnote*{
— всією масою. \emph{Ред.}
} обмінюватися на масу золота й срібла, що є в країні. Коли
звести товаровий світ в одним-один сукупний товар, що з нього кожний
товар становить лише певну частину, тоді вийде гарнесенький рахуночок:
сукупний товар \deq{} \emph{х} центнерам золота, товар \emph{А} \deq{} відповідній частині
сукупного товару \deq{} тій самій відповідній частині \emph{х} центнерів золота. Це
відверто висловлює Монтеск’є: «Якщо порівнюють масу золота і срібла,
що є у світі, з масою товарів, яку має світ, то цілком ясно, що кожний
окремий предмет споживання або товар можна порівняти з певного частиною
іншого. Припустімо, що на світі існує лише один предмет споживання,
або товар, або що тільки один предмет споживання, або товар, купується
на ринку й поділяється на частини так само, як гроші; тоді ця частина
цього товару відповідала б певній частині маси грошей; половина цілої
кількости першої — половині цілої кількости другої\dots{} Установлення цін
на продукти завжди суттю своєю залежить од відношення між сукупністю
продуктів і сукупністю знаків». («Si l’on compare la masse de l’or et
de l’argent qui est dans le monde, avec la somme des marchandises qui y
sont, il est certain que chaque denrée ou marchandise, en particulier, pourra
être comparée à une certaine portion de l’autre. Supposons qu’il n’y en ait
qu’une seule denrée ou marchandise dans le monde, ou qu’il n’y ait qu’une
seule qui s’achète, et qu’elle se divise comme l’argent: cette partie de cette marchandise
répondra à une partie de la masse de l’argent; la moitié du total
de l’une à la moitié du total de l’autre\dots{} l’etablissement du prix des choses
dépend toujours fondamentalement de la raison du total des choses au
total des signes».). (\emph{Montesquieu}: «Esprit des Lois», Oeuvres, London
1767 vol. 3, p. 12, 13). Про дальший розвиток цієї теорії у Рікарда
і його учнів Джемса Мілла, лорда Оверстона та інших порівн. «Zur
Kritik», стор. 140--146 і стор. 150 і далі («До критики», ДВУ 1926~\abbr{р.},
стор. 179--190 і далі). Пан Дж.~Ст.~Мілл із звичною для нього еклектичною
логікою знаходить спосіб бути однакових поглядів зі своїм
батьком, Джемсом Міллом, і одночасно протилежних. Коли порівняємо
текст його компендія «Principles of Political Economy» з передмовою
(перше видання), в якій він сам проголошує себе сучасним Ад.~Смісом,
то не знатимемо, з чого більше дивуватися, з наївности цієї людини, чи
з наївности публіки, яка на віру прийняла його за Ад.~Сміса, на якого
він схожий приблизно так само, як генерал Вільямс Карс на герцоґа
Велінґтона. Оригінальні досліди пана Дж.~Ст.~Мілла в царині політичної
економії, що не відзначаються, ані обсягом, ані змістовністю, всі вмістилися
в цілому і з подробицями в його брошурці «Some Unsettled Questions
of Political Economy», яка появилася 1884 p. Льокк просто говорить
про зв’язок між відсутністю вартости у золота й срібла і визначенням
їхньої вартости за допомогою кількости: «Люди погодились на тому, щоб
надавати золоту й сріблу уявлюваної вартости\dots{} внутрішня вартість цих
металів є не що інше, як їхня кількість» («Mankind having consented
to put an imaginary value upon gold and silver\dots{} the intrinsic value, regarded
in these metals, is nothing but the quantity»). («Some Considerations
on the Consequences ot the Lowering of Interest», 1691, Works, ed. 1777,
vol. 2, p. 15).
}.

\subsubsection{Монета. Знак вартости}

З функції грошей як засобу циркуляції випливає їхня монетна
форма. Вагова частина золота, уявлена в ціні або в грошовій
назві товарів, мусить протистати їм у циркуляції як однойменний
кусник золота або монета. Так само як і встановлення маштабу
цін, справа карбування монети припадає державі. У різних національних
мундирах, що їх носять золото й срібло як монети,
знову скидаючи їх на світовому ринку, виявляється поділ між
унутрішньою, або національною сферою товарової циркуляції
та її загальною сферою світового ринку.

\index{i}{0080}  %% посилання на сторінку оригінального видання
Отже золота монета й золото у зливках насамперед відрізняються
між собою лише виглядом, і золото завжди можна перетворювати
з однієї форми на іншу\footnote{
Звичайно, що розгляд подробиць, як от монетний податок і~\abbr{т. ін.},
лежить цілком поза межами моєї мети. Щождо романтичного сикофанта
Адама Міллера, який захоплюється «великодушною ліберальністю», з
якою «англійський уряд даром карбує монету», то я наведу лише міркування
сера Дедлей Норта: «Срібло та золото, як і інші товари, мають свої
припливи й відпливи. Після одержання певної кількости срібла з Еспанії\dots{}
його відсилається до Таверу і там карбується. Швидко після цього постає
попит на зливки для вивозу. Коли зливків немає, коли ввесь металь пішов
на карбування монет, що тоді робити? Очевидно, знов перетопити монету
на зливки. Тут не постає жодної втрати, бо карбування нічого не коштує
власникам металю. Таким чином нація має збитки й мусить викидати
свої гроші на вітер. Коли б за карбування довелося купцеві платити
(Норт сам був одним із найбільших купців за часів Карла 2), то він не
посилав би своє срібло до Таверу, не зваживши всіх обставин, і викарбована
монета завжди мала б вищу вартість, ніж срібло у зливках». («Silver
and gold, like other commodities, have their ebbings and flowings. Upon
the arrival of quantities fromm Spain\dots{} it is carried into the Tower, and
coined. Not long after there will come a demand for bullion, to be exported
again. If there is none, but all happens to be in coin, what then? Melt
it down again; there’s no loss in it, for the coining costs the owner nothing.
Thus the nation has been abused, and made to pay for the twisting of
straw, for asses to cat. If the merchant had to pay the price of the coinage,
he would not have sent his silver to the Tower without consideration; and
coined money would always keep a value above uncoined silver»). (\emph{North}:
«Discourses upon Trade», London 1691, p. 18).
}. Але шлях із монетарні є разом
із тим шлях до топильного тиґля. В обігу золоті монети стираються
одна більше, інша менше. Назва золотої монети й субстанція
золота, номінальний зміст і реальний зміст починають роз’єднуватись.
Однойменні золоті монети стають монетами неоднакової
вартости, бо вага їхня стає неоднакова. Золото як засіб циркуляції
відхиляється від золота як маштабу цін, і разом з тим перестає
бути дійсним еквівалентом товарів, що їхні ціни воно реалізує.
Історія цієї плутанини становить історію монети середньовіччя
й нових віків аж до XVIII століття. Природну тенденцію процесу
циркуляції перетворювати золоту суть (Goldsein) монети на
уявлюване золото, або монету на символ її офіціяльного металевого
змісту, признають навіть найновітніші закони, які визначають
той ступінь утрати металю, за якого золота монета стає нездатною
до обігу, або демонетизується.

Коли самий грошовий обіг відокремлює реальний зміст монети
від номінального змісту, її металеве буття від її функціонального
\parbreak{}  %% абзац продовжується на наступній сторінці
