За незмінних товарових цін маса засобів циркуляції може
зростати в наслідок того, що збільшується маса товарів, які
циркулюють, або в наслідок того, що зменшується швидкість
обігу грошей, або тому, що обидві обставини діють разом. Навпаки,
маса засобів циркуляції може меншати із зменшенням
маси товарів або із зростом швидкости циркуляції.

За загального підвищення товарових цін маса засобів циркуляції
може лишатися незмінна, коли маса товарів, що циркулюють,
меншає в тій самій пропорції, у якій зростає їхня ціна, або коли
швидкість обігу грошей збільшується так само хутко, як і зріст цін,
тимчасом як маса товарів, що циркулюють, лишається та сама.
Маса засобів циркуляції може падати тому, що маса товарів
зменшується або швидкість обігу збільшується швидше, ніж ціни.

За загального зниження товарових цін маса засобів циркуляції
може лишатися незмінна, коли маса товарів зростає в тій самій
пропорції, що в ній падає їхня ціна, або коли швидкість обігу
грошей зменшується в тій самій пропорції, що й ціни. Маса засобів
циркуляції може зростати, коли маса товарів зростає або швидкість
циркуляції зменшується швидше, ніж падають товарові ціни.

Варіяції різних факторів можуть взаємно компенсуватися,
так що наперекір їхній постійній несталості загальна сума товарових
цін, яка має бути зреалізована, отже, і маса грошей, що
циркулюють, лишається стала. Тому ми й бачимо, особливо при
розгляді порівняно довших періодів, значно сталіший пересічний
рівень маси грошей, що циркулює у кожній країні та, — за
винятком часів значних пертурбацій, що періодично виникають
із промислових і торговельних криз, рідше із зміни вартости
самих грошей, — куди менші відхилення від цього пересічного
рівня, ніж можна було сподіватися на перший погляд.

Той закон, що кількість засобів циркуляції визначається
сумою цін товарів, які є в циркуляції, і пересічною швидкістю
грошового обігу,78 можна висловити й так: за даної суми вартости
товарів і за даної пересічної швидкости їхніх метаморфоз,

78 «Щоб провадити торговлю, нація потребує грошей у певній мірі
або пропорції: більша або менша проти потрібної кількість грошей зашкодила
б торговлі. Зовсім так само, як у дрібній торговлі, потрібна певна
кількість фартинґів, шоб розміняти срібну монету або провадити такі
виплати, яких не можна перевести навіть за допомогою найдрібніших
срібних монет... І подібно до того, як пропорція числа фартинґів, потрібних
для торговлі, залежить від числа осіб, що вживають їх, або частости
розміну їх, а також — і це насамперед — від вартости щонайдрібнішої
срібної монети, таким саме чином і кількість грошей (золотих і срібних
монет), потрібних для торговлі, визначається частістю операцій і розмірами
виплат». («There is a certain measure, and proportion of money requisite
to drive the trade of a nation, more or less than which, would prejudice
the same. Just as there is a certain proportion of farthings necessary in a
small retail Trade, to change silver money, and to even such reckonings
as cannot be adjusted with the smallest silver pieces... Now as the proportion
of the number of farthings requisite in commerce is to be taken from
the number of people, the frequency of their exchanges, as also, and principally,
from the value of the smallest silver pieces of money; so in like mannerthe
proportion of money (gold and silver specie) requisite to our trade,
