\parcont{}  %% абзац починається на попередній сторінці
\index{i}{0146}  %% посилання на сторінку оригінального видання
упредметненої праці, отже, їх не береться до рахуби й не входять
вони у продукт утворення вартости.\footnote{
Це одна з обставин, які удорожчують продукцію, основану на
рабстві. Робітник, як влучно висловлювалися за старовини, відрізняється
тут лише як instrumentum vocale * від тварини як instrumentum semivocale
* * і від мертвого знаряддя праці як від instrumentum mutum. * * *
Але сам робітник дає відчути тварині і знаряддю праці, що він їм не
рівня, а що він людина. Збиткуючися з них і con amore **** руйнуючи їх,
він з самозадоволенням переконує себе самого в своїй відмінності від них.
Тому за цього способу продукції вважається за економічний принцип
вживати лише найгрубіших, найтяжчих знарядь праці, бо саме через
цю грубість і незграбність їх важко знівечити. Тому в рабовласницьких
державах, які лежать над Мехіканською затокою, до вибуху громадянської
війни, вживали плугів старокитайської конструкції, що рили землю,
як свиня або кріт, але не робили борозни, не повертали її. Порівн.
J. С. Cairns: «The Slave Power», London 1862, p. 46 і далі. У своїй праці
«Sea Bord Slave States» (p. 46, 47) Олмстед оповідає, між іншим, таке:
«Мені тут показували знаряддя, що їх у нас жодна людина із здоровим
розумом ніколи не дала б найманому робітникові, бо вони обтяжали б
його; на мою думку, їхня надзвичайна вага й незграбність збільшують
працю щонайменше на 10\% порівняно з тим знаряддям, що його звичайно
вживають у нас. І я певен, що за недбалого й грубого поводження рабів
із знаряддям праці було б неекономно дати їм легше й не таке грубе знаряддя.
А ті знаряддя, що їх ми з користю завжди даємо нашим робітникам,
не збереглися б жодного дня на хлібних полях Вірґінії, хоч ґрунт там
і легший і не такий кам’янистий, як наш. Так само, коли я спитав, чому
на всіх фармах замість коней вживають мулів, то перший арґумент, звичайно,
найдовідніший, був той, що коні не могли б витримати поводження
з боку негрів; у наслідок такого поводження коні завжди швидко нівечаться
або калічіють, тоді як мули витримують биття і брак харчів, не
зазнаючи від цього жодної матеріяльної шкоди, не перестуджуються й не
хоріють, навіть коли нехтувати ними й обтяжати їх працею. Алеж мені
досить підійти до вікна кімнати, де я пишу, щоб побачити завжди таке
поводження з худобою, за яке північний фармер негайно прогнав би погонича».
(«І am here shown tools that no man in his senses, with us, would
allow a labourer for whom he was paying wages, to be encumbered with:
and the excessive weight and clumsiness of which, I would judge, would
make work at least ten per cent greater than with those ordinarily used
with us. And I am assured that, in the careless and clumsy way they must
be used by the slaves, anything ligther or less rude could not be furnished
them with good economy, and that such tools as we constantly give our
bourers, and find our profit in giving them, would not last out a day in
Virginia cornfield — much lighter and more free from stones though it be
than ours. So, too, when I ask why mules are so universally substituted for
horses on the farm, the first reason given, and confessedly the most conclusive
one, is that horses cannot bear the treatment that they always must get
from negroes; horses are always soon foundered or crippled by them while
mules will bear cudgelling, and lose a meal or two now and then, and not
be materially injured, and they do not take cold or get sick, if neglected
or overworked. But I do not need to go further than to the window of the room
in which I am writing, to see at almost any time, treatment of cattle
that would insure the immediate discharge of the driver by almost any farmer
owning them in the North»).
}

Ми бачимо, що встановлена вже раніш аналізою товару ріжниця
між працею, оскільки вона утворює споживну вартість, і

* — знаряддя, обдароване мовою. \emph{Ред.}

** — знаряддя, обдарованого голосом. \emph{Ред.}

* * * — знаряддя німого. \emph{Ред.}

* * ** — з насолодою. \emph{Ред.}
\parbreak{}  %% абзац продовжується на наступній сторінці
