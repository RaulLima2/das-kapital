\parcont{}  %% абзац починається на попередній сторінці 
\index{i}{0418}  %% посилання на сторінку оригінального видання 
землею коло навантажування вугілля й т. ін, , тягають цебрами
вугілля до каналів і залізничних ваґонів, сортують вугілля й
т. ін. За останні 3-4 роки вживання жіночої праці дуже зросло
(№1727). Це — здебільша дружини, дочки й удови копальневих
робітників від 12 до 50 і 60 років життя (№№ 645, 1779) (№ 648).
«Що думають копальневі робітники про жіночу працю по копальнях?
— Взагалі її осуджують» (№ 649). «Чому? — Вони вважають
її за зневажливу для жіночої статі... Вони зодягнені наче в чоловічий
одяг. У багатьох випадках у них заглушено всякий сором.
Деякі жінки палять. Праця така ж брудна, як і в самих копальнях.
Серед них багато заміжніх жінок, які не мають змоги виконувати
своїх домашніх обов’язків» (№ 651 і далі) (709). «Чи
можуть удови знайти деінде такий вигідний заробіток (8-10
шилінґів на тиждень)? — Я нічого не можу сказати про це»
(№ 710). «А проте (кам’яне серце!) ви зважуєтеся відібрати в
них цей засіб існування? — Певно» (№ 1715). «Звідки такий
настрій? — Ми, копальневі робітники, надто шануємо гарну
стать, щоб бачити її засудженою на роботу в копальнях... Ця
праця здебільша дуже важка. Багато з цих дівчат підіймає
10 тонн на день» (№ 1732). «Чи не гадаєте ви, що робітниці, які
працюють у копальнях, неморальніші за тих, що працюють на
фабриках? — Процент неморальних більший серед них, ніж серед
фабричних дівчат» (№ 1733). «Алеж ви не задоволені і з стану
моральности на фабриках? — Ні» (№ 1734). «Чи не хотіли б ви
заборонити жіночу працю й на фабриках? — Ні, я цього не хочу»
(№ 1735). «Чому ні? — Вона почесніша для жіночої статі й більш
відповідає їй» (№ 1736). «А все ж ви гадаєте, що вона шкідлива
для їхньої моральности? — Ні, далеко не так, як праця в копальнях.
Зрештою, я кажу це не тільки з моральних, але й з фізичних
і соціяльних міркувань. Соціяльна деґрадація дівчат жахлива
й доходить до краю. Коли ці дівчата стають дружинами копальневих
робітників, чоловіки глибоко страждають від цієї деґрадації,
і це жене їх із дому до пияцтва» (№ 1737). «А чи не те саме буває
з жінками, що працюють на залізовиробних заводах? — Я це
можу казати про інші галузі продукції» (№ 1740). «А яка ж ріжниця
між жінками, що працюють по залізовиробних заводах,
і тими, що працюють по копальнях? — Я над цією справою не
працював» (№ 1741). «Чи можете ви знайти якусь ріжницю
між однією й другою клясою? — В цій справі я не можу сказати
нічого з певністю, але, ходячи від хати до хати, я познайомився
з ганебним станом речей у нашій окрузі» (№ 1750). «Чи не мали б
ви великої охоти усунути жіночу працю скрізь, де вона спричиняє
деґрадацію? — Так... найкращі почуття в дітей дає тільки материне
виховання» (№ 1751). «Але це стосується і до рільничої
праці жінок? — Вона триває тільки два сезони, а в нас вони працюють
всі чотири сезони року, іноді день і ніч, мокрі аж до
сорочки, їхній організм слабшає, їхнє здоров’я надломлюється»
(№ 1753). «Чи вивчали ви цю справу (саме щодо жіночої праці)
взагалі? — Я придивлявся навколо себе і можу лише сказати,
\parbreak{}  %% абзац продовжується на наступній сторінці
