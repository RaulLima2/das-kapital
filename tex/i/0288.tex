захоплює поряд економічної сфери усякі інші сфери суспільства
та всюди закладає основу тому розвиткові професіоналізму,
спеціялізації, тій парцеляції людини, яка примусила вже А. Ферґюсона,
вчителя Адама Сміса, вигукнути: «Ми є нація гелотів,
і немає серед нас вільних людей».\footnote{
A. Ferguson: «History of Civil Society», Edinburgh 1767, Part IV,
sect II, p. 285.
}

Однак, не зважаючи на численні аналогії та зв’язки між поділом
праці всередині суспільства й поділом праці всередині майстерні,
обидва вони не тільки щодо ступеня, але й суттю відмінні.
Безперечно, найяскравіше ця аналогія виступає там, де внутрішній
зв’язок сплітає різні галузі продукції. Скотар, приміром,
продукує шкури, гарбар вичиняє шкуру, швець із вичиненої
шкури робить чоботи. Кожен продукує тут частинний продукт,
а остання готова форма — це комбінований продукт їхніх окремих
праць. Сюди треба додати ще різноманітні галузі праці, які
постачають засоби продукції скотареві, гарбареві та шевцеві.
Можна собі уявити разом з А. Смісом, що цей суспільний поділ
праці відрізняється від мануфактурного лише суб’єктивно, а саме
лише для спостерігача, який тут, у мануфактурі, одним поглядом
просторово охоплює різноманітні частинні праці, тоді як там розкиданість
їх по великих просторах та велике число робітників,
занятих у кожній окремій галузі, затемнюють зв’язок.\footnote{
У справжніх мануфактурах, каже він, поділ праці видається
більшим, бо «робітники, що працюють у кожній з різних галузей праці,
часто можуть бути сполучені в тій самій майстерні, і таким чином всіх
їх одразу охоплює око спостерігача. Навпаки, у тих великих мануфактурах
(І), які мають своїм призначенням задовольняти широкі потреби
великої маси людности, кожна окрема галузь вживає такої великої кількости
робітників, що всіх їх неможливо сполучити в тій самій майстерні...
поділ праці тут далеко не так виразно впадає на очі» («... those employed
in every different branch of the work can often be collected into the
same workhouse, and placed at once under the view of the spectator. In
those great manufactures (I), on the contrary, which are destined to supply
the great wants of the great body of the people, every different branch of
the work employs so great a number of workmen, that it is impossible
to collect them all into the same workhouse... the divisions is not near
so obvious»). (A. Smith: «Wealth of Nations», b. I, ch. 1, p. 7, 17).
Знамените місце того самого розділу, яке починається словами:
«Погляньте на життєві умови простого ремісника або поденника в
цивілізованій країні, в країні, що процвітає, і т. д.» («Observe the accomodation
of the most common artificer or day labourer in a civilized and
thriving country etc.»), розділу, де змальовано далі, як безліч різноманітних
галузей промисловості спільними силами задовольняє потреби
простого робітника, — це місце майже слово в слово списано з приміток
Б. де Мандевіля до його «Fable of the Rees, or Private Vices, Publick Benefits».
(Перше видання без приміток 1706 р., друге з примітками 1714 р.).
} Але що
саме встановлює зв’язок між незалежними працями скотаря,
гарбаря, шевця? Те, що їхні продукти існують як товари. Навпаки,
що характеризує мануфактурний поділ праці? Те, що
частинний робітник не продукує жодного товару.\footnote{
«Тут уже немає нічого, що можна було б назвати природною винагородою
за індивідуальну працю. Кожен робітник продукує лише ча-
} Тільки спіль-