няним хазяїнам... Настав час, коли велика громадська думка
цих островів мусить щось зробити, щоб урятувати «цю робочу
силу» від тих, що хочуть поводитися з нею так, як вони поводяться
з вугіллям, залізом і бавовною» («to save this, «working
power» from those who would deal with it as they deal with iron
coal and cotton»).15

Стаття «Times’a» була тільки jeu d’esprit.* «Велика громадська
думка» була в дійсності така, як думка Потера, — що фабричні
робітники є рухома приналежність фабрик. Їхній еміґрації
стали на перешкоді.16 їх замкнули в «моральний робітний
дім» бавовняних округ, і вони, як і раніш, становлять «силу
(the strength) бавовняних хазяїнів Ланкашіру».

Отже, капіталістичний процес продукції самим своїм перебігом
репродукує відокремлення робочої сили від умов праці.
Тим самим він репродукує й увіковічнює умови експлуатації
робітника. Він постійно примушує робітника продавати свою
робочу силу, щоб жити, і постійно дає капіталістові змогу купувати
її, щоб багатіти.17 Тепер уже не випадок протиставить на
товаровому ринку капіталіста й робітника як покупця і продавця.
Механізм самого процесу постійно відкидає одного назад на
товаровий ринок як продавця його робочої сили і постійно перетворює
його власний продукт на купівельний засіб другого.
Фактично робітник належить капіталові раніш, ніж він продав
себе капіталістові. Його економічну підлеглість18 упосереднює

15 «Times» з 24 березня 1863 р.

16 Парламент не вотував жодного фартинга на еміґрацію, а ухвалив
тільки закони, що давали муніципалітетам можливість тримати робітників
між життям і смертю або експлуатувати їх, не платячи їм нормальної
заробітної плати. Навпаки, коли три роки пізніше спалахнула пошесть
на худобу, парлямент грубо знехтував навіть парламентською етикетою й
негайно вотував мільйони на відшкодування мільйонерам з лендлордів,
фермери яких і без того не мали ніякої шкоди завдяки піднесенню ціни
на м’ясо. Звіряче виття землевласників на відкритті парламенту 1866 р.
показало, що не треба бути індусом, щоб падати навколішки перед коровою
Сабала, ані Юпітером, щоб перетворитися на бика.

17 «Робітник вимагав засобів існування, щоб жити, підприємець
вимагав праці, щоб мати бариш» («L’ouvrier demandait de la subsistance
pour vivre, le chef demandait du travail pour gagner»). («Sismondi: «Nouveaux
Principes d’Economie Politique», vol. I, p. 91).

18    Грубо сільська форма цієї нідлеглости існує в графстві Дергем.
Це одно з тих небагатьох графств, де обставини не забезпечують фармерові
безперечного права власности на рільничих поденників. Гірнича
індустрія дозволяє їм вибирати. Тому тут, усупереч загальному правилу,
фармер бере в оренду тільки ті землі, на яких є котеджі для робітників.
Плата за наймання котеджів становить частину заробітної плати. Ці котеджі
звуться «hind’s houses**». їх винаймають робітникам з певними февдальними
зобов’язаннями, з умовою, що зветься «bondage» (кріпацька
залежність) і, наприклад, зобов’язує робітника на той час, коли він працює
деінде, посилати на працю свою дочку й т. ін. Сам робітник називається
bondsman, кріпак. Ці відносини показують нам з цілком нового
боку і особисте споживання робітника як споживання для капіталу, аоо
продуктивне споживання: «Цікаво спостерігати, що навіть екскременти

* — гра словами. Ред.

** — доми слуг. Ред.
