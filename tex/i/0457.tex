собі зиск простим шахрайством: купівлею товарів нижче й продажем
їх понад їхню вартість. Тому він не доходить до зрозуміння
того, що коли б дійсно існувала така річ, як вартість праці, і
коли б він дійсно сплатив ту вартість, то не існувало б жодного
капіталу, його гроші не перетворювались би на капітал.

До того ж дійсний рух заробітної плати показує явища, які,
як здається, доводять, що оплачується не вартість робочої
сили, а вартість її функції, вартість самої праці. Ці явища можна
звести до двох великих кляс. Поперше: зміна заробітної плати
із зміною довжини робочого дня. З таким самим правом можна б
зробити висновок, що оплачується не вартість машини, а вартість
її операцій, бо дорожче коштує найняти машину на тиждень,
аніж на день. Подруге, індивідуальна ріжниця в заробітних
платах різних робітників, що виконують ту саму функцію. Цю індивідуальну
ріжницю ми бачимо, — однак без того, щоб вона давала
нагоду до ілюзій, — і за системи рабства, де саму робочу силу
продають ділком явно й вільно, без якихбудь прикрас. Тільки за
системи рабства вигода від робочої сили, вищої за пересічну якість,
або шкода від робочої сили, нижчої за пересічну якість, припадає
рабовласникові, а за системи найманої праці — самому
робітникові, бо в останньому випадку він сам продає свою робочу
силу, в першому її продає якась третя особа.

А втім, для такої форми виявлення, як «вартість та ціна праці»
або «заробітна плата», на відміну від того посутнього відношення,
що виявляється, тобто на відміну від вартости й ціни робочої
сили, має силу те саме, що й для всіх форм виявлення та захованої
за ними їхньої основи. Перші репродукуються безпосередньо
сами собою, як найпоширеніші форми мислення, другу мусить
відкрити тільки наука. Клясична політична економія доходить
близько до правдивого стану речей, однак не формулює його
свідомо. їй і не сила зробити це, доки вона має на собі свою буржуазну
шкуру.

Розділ вісімнадцятий
Почасова плата

Сама заробітна плата знов таки набирає дуже різноманітних
форм — обставина, що про неї не можна дізнатись з економічних
підручників, які у своїй грубій заінтересованості матерією нехтують
усякі ріжниці форм. Однак з’ясування всіх цих форм
належить до спеціяльної науки про заробітну плату, отже, не
може бути завданням цього твору. А все ж дві панівні основні
форми тут треба коротко розвинути.

Як ми пригадуємо, продаж робочої сили відбувається завжди
на певні періоди часу. Тому перетворена форма, в якій безпосередньо
виражається денна вартість, тижнева вартість і т. д. робочої
сили, є форма «почасової плати», отже, поденна плата і т. д.

Насамперед треба тут зауважити, що з’ясовані в п’ятнадцятому
розділі закони про зміну величини ціни робочої сили та до-
