\parcont{}  %% абзац починається на попередній сторінці
\index{i}{0567}  %% посилання на сторінку оригінального видання
ніж колись експлуатовано копальні Потозі. Антагоністичний
характер капіталістичної акумуляції, а тому й капіталістичних
відносин власности взагалі\footnote{
«Ніде права особи не жертвовано так одверто й безсоромно на
користь праву власности, як у житлових умовах робітничої кляси. Кожне
велике місто являє собою місце людських жертов, вівтар, на якому рік-у-рік
убивають тисячі людей для Молоха ненажерливости». (\emph{S. Laing}:
«National Distress», 1844, р. 150).
} стає тут такий наочний, що навіть
офіціальні англійські звіти про цей предмет переповнені єретичними
нападами на «власність і її права». Це лихо так поширилося
з розвитком промисловости, акумуляцією капіталу, зростом
та «прикрашуванням» міст, що лише острах перед пошесними
недугами, які не щадять навіть «шановних», викликав від 1847
до 1864~\abbr{р.} не менше як десять санітарно-поліційних парляментських
актів, а перелякана буржуазія деяких міст, як от Ліверпуль,
Ґлезґо і~\abbr{т. д.}, почала втручатися в цю справу через свої
муніципалітети. Проте, д-р Сімон у своєму звіті з 1865~\abbr{р.} вигукує:
«Взагалі кажучи, цей лихий стан в Англії лишається без
контролю». З наказу Privy Council в 1864~\abbr{р.} досліджено житлові
умови сільських робітників, в 1865~\abbr{р.} — бідніших кляс по містах.
Майстерні праці д-ра Джульяна Гентера надруковано в
сьомому й восьмому звітах «Public Health». Про сільських робітників
я говоритиму пізніше. Щождо міських житлових умов,
то я насамперед наведу загальну увагу д-ра Сімона: «Хоч
мій офіціяльний погляд, — каже він, — виключно медичний, проте
звичайна гуманність не дозволяє мені не звертати уваги на другий
бік цього лиха. Дійшовши вищого ступеня, це лихо майже
неминуче зумовлює таке заперечення всякої звичайности, таке
брудне змішування тіл і фізичних функцій, таку одверту наготу
статей, що все це має звірячий, а не людський характер. Зазнавати
таких впливів — це зневага, яка стає то глибша, що довше
вона триває. Для дітей, що народилися під цим прокляттям,
воно є хрещення на ганьбу (baptism into infamy). І безнадійним
понад усяку міру є бажання, щоб люди, поставлені в такі умови,
в інших відношеннях прагнули тієї атмосфери цивілізації, що її
суть у фізичній і моральній чистоті».\footnote{
«Public Health, Eighth Report», London 1866, p. 14, примітка.
}

Перше місце щодо переповнення помешкань або й щодо абсолютної
непридатности їх для людського житла посідає Лондон.
«Дві обставини, — каже д-р Гентер, — певні: по-перше, в Лондоні
є щось з 20 великих колоній, кожна з яких має приблизно
\num{10.000}  осіб, що їхнє злиденне становище переважає все, що будь-коли
бачили деінде в Англії, і це становище є майже цілком результат
поганого стану їхнього житла; по-друге, переповненість і зруйнованість
домів по цих колоніях тепер значно гірші, ніж двадцять
років тому».\footnote{
Там же, стор. 89. Щодо дітей із цих колоній д-р Гентер каже:
«Ми не знаємо, як виховували дітей перед цією епохою тісного скупчення
бідних, і сміливим пророком був би той, хто хотів би наперед сказати,
чого можна сподіватися від дітей, які серед умов, у цій країні безприкладних,
виховуються тепер на небезпечні у майбутній своїй практиці кляси,
проводячи час до півночі з особами всякого віку, п'яними, розпусними та
лайливими». (Там же, стор. 56).
} «Не буде перебільшенням сказати, що
\index{i}{0568}  %% посилання на сторінку оригінального видання
життя в багатьох частинах Лондону й Ньюкестлю в пекло».\footnote{
Там же, стор. 62.
}

Але й тій частині робітничої кляси, що живе в кращих умовах,
а також дрібним крамарям та іншим елементам дрібної
середньої кляси, в Лондоні чимраз більше дається взнаки прокляття
цих мізерних житлових умов у міру того, як проґресують
«поліпшення», а з ними й зламання старих будинків і кварталів,
у міру того, як зростає число фабрик і наплив людей до головного
міста, нарешті, в міру того, як разом із міською земельною
рентою зростає плата за квартиру. «Плата за квартиру стала
така непомірна, що небагато робітників може оплатити більш
ніж одну кімнату».\footnote{
«Report of the Officer of Health of St. Martin’s in the Fields», 1865.
} У Лондоні немає майже жодної домовласности,
що не була б обтяжена безліччю «middlemen’ів».\footnote*{
— посередників. \emph{Ред.}
} Ціна
землі в Лондоні завжди дуже висока порівняно з річними доходами
з неї, бо кожний покупець спекулює на те, щоб раніш або
пізніше знову продати її за Jury Price (за ціну, визначену присяжними
при експропріяціях), або щоб вишахрувати надзвичайне
підвищення її вартости через близькість її до якогось великого
підприємства. Наслідок цього є реґулярна торговля контрактами
найму, що їх скуповують, коли їхній реченець наближається
до кінця. «Від джентльменів у цій справі можна сподіватися, що
вони робитимуть так, як роблять, а саме видушуватимуть з
квартирантів якомога більше, а самий дім передаватимуть своїм
наступникам у якомога злиденнішому стані».\footnote{
«Public Health. Eighth Report», London 1866. p. 91.
} Плата за квартиру
— тижнева, і ці панки нічим не ризикують. У наслідок
того, що залізниці будується в межах міста, «недавно одного
суботнього вечора у східній частині Лондону можна було бачити,
як сила родин, вигнаних із своїх старих помешкань, тинялися
з своїми злиденними пожитками на плечах, ніде не находячи
собі притулку, крім лише в робітному домі.\footnote{
Там же, стор. 88.
} Робітні
доми вже переповнені, а ухвалені парляментом «поліпшенням
почали ще тільки проводити в життя. Коли робітників проганяють
в наслідок руйнування їхніх старих домів, то вони не покидають
своєї парафії або принаймні оселяються на її межі, в найближчій
парафії. «Ясна річ, вони силкуються оселитися якомога
ближче до місця своєї праці. Наслідок той, що замість двох кімнат
родина мусить оселитися в одній кімнаті. Навіть при підвищеній
платі нове помешкання є гірше, ніж те погане, з якого їх вигнано.
Вже половині робітників на Strand’і доводиться ходити дві милі
до місця праці». Цей Strand, що його головна вулиця справляє
на чужинця імпозантне вражіння багатством Лондону, може
бути за приклад того, як напаковують людей у Лондоні. В одній
парафії цього Strand’у санітарний урядовець налічив 581 особу
\parbreak{}  %% абзац продовжується на наступній сторінці
