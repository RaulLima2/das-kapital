довгу примітку про гірничих робітників (стор. 509—515 третього
видання). Інші ж незначні зміни є суто технічного характеру.

Далі я поробив ще деякі пояснювальні додаткові примітки,
а саме там, де, як мені здавалося, цього вимагали змінені історичні
обставини. Всі ці додаткові примітки подано у прямих дужках
і позначено моїми ініціялами.

Цілковита перевірка численних цитат стала доконечною для
англійського видання, що вийшло за цей час. Для цього видання
наймолодша дочка Марксова, Елеонора, взяла на себе працю
порівняти з ориґіналами всі наведені цитати, так що цитати
з англійських джерел, які становлять переважну частину цитат,
там подаються не у зворотному перекладі з німецької мови, а
англійською мовою тексту самого оригіналу. Отже, я повинен був
узяти до уваги цей текст для четвертого видання. При цьому виявились
деякі маленькі неточності. Неправильні посилання на сторінки,
що сталися почасти підчас переписування із зшитків, а почасти
в наслідок друкарських помилок, які назбиралися протягом
трьох видань. Неправильно поставлено лапки або павзи, як це
неминуче буває за масового цитування із зшитків з витягами.
Місцями вжито не зовсім влучно обране для перекладу слово.
Окремі місця цитовано з давніх паризьких зшитків 1843—1845 рр.,
коли Маркс іще не розумів англійської мови й англійських економістів
читав у французькому перекладі; там, де через подвійний
переклад сталась легка зміна у відтінку розуміння цитат, приміром
цитати з Steuart’a, Ure’a та ін., — тепер можна було використати
англійський текст. Такі самі й інші дрібні неточності й недогляди.
Коли ми порівняємо тепер четверте видання з попередніми, то
переконаємося, що ввесь цей морочливий процес перевірки ані
трохи не змінив у книзі нічогісінько такого, що варто було б
зауважити. Лише однієї цитати не можна було знайти, а саме
з Richard’a Jones’a (стор. 562 четвертого видання, примітка 47)
(стор. 509 цього українського видання); Маркс, мабуть, помилився
в заголовку книги. Всі інші цитати зберігають свою повну
доказову силу або зміцнюють її в теперішній точній формі.

Але тут я мушу повернутися до однієї старої історії.

Мені особисто відомий один лише випадок, коли піддано під
сумнів правильність Марксової цитати. Але що цей сумнів висувалось
і після смерти Маркса, то я не можу його тут поминути.

В берлінській «Concordia», органі спілки німецьких фабрикантів,
з’явилася 7 березня 1872 р. анонімна стаття: «Як цитує
Карл Маркс». Тут із превеликим витраченням морального обурення
й непарляментськими висловами твердиться, нібито цитату
з Ґледстонової бюджетової промови від 16 квітня 1863 р.
(вміщену у відозві з приводу заснування (Inauguraladresse) інтернаціональної
асоціяції робітників у 1864 р. і повторену в першому
томі «Капіталу», стор. 617 четвертого видання, стор. 671 третього
видання, стор. 561 цього українського видання) підроблено.
У стенографічному (quasi-офіціальному) звіті Hansard’a,
мовляв, немає ані словечка з речення: «Це приголомшливе збіль-
