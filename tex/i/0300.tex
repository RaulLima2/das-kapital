поділ праці фігурує як конститутивний принцип держави, є
лише атенська ідеалізація єгипетського кастового ладу, так само
як Єгипет в інших його сучасників, наприклад, у Ізократа,\footnote{
«Він (Бузіріс) поділив усіх на окремі касти... наказав, щоб ті
самі люди завжди працювали коло тих самих справ, бо знав, що ті люди,
які змінюють свою працю, не знають ґрунтовно ніякої справи; а ті, які
завжди працюють коло тих самих справ, зможуть кожну з них виконати
якнайдосконаліше. І ми справді бачимо, що відносно вмілости та реместв
вони куди більше перевищили своїх суперників, ніж звичайно майстер
перевищує партача, а інституції, якими вони підтримують королівське
панування та інший державний лад, були в них такі досконалі, що найславетніші
філософи, яким доводилося писати про це, вихваляли державний
лад Єгипту більше, ніж інших країн». (Isocrates: «Busiris», с. 8).
83 Порівн. Diodorus Siculus.
}
вважається за зразкову промислову країну; це значення Єгипет
зберігає ще навіть для греків доби Римської імперії.83

Підчас власне мануфактурного періоду, тобто того періоду,
коли мануфактура була панівною формою капіталістичного способу
продукції, повне здійснення властивих їй тенденцій наражається
на багато різних перешкод. Хоч мануфактура, як ми
бачили, і утворює поряд ієрархічного розчленування робітників
простий поділ їх на навчених та ненавчених, однак число останніх
лишається через переважний вплив перших дуже обмеженим.
Хоч вона й пристосовує окремі операції до різних ступенів дозрілости,
сили й розвитку її живих робочих органів і тим штовхає
до продуктивного визиску жінок та дітей, все ж взагалі і в цілому
ця тенденція розбивається об звички та опір робітників-чоловіків.
Хоч розчленування ремісничої роботи понижує витрати на
навчання, отже, і вартість робітників, проте для тяжчої детальної
праці лишається потрібним довший час на навчання, і його
з запалом зберігають робітники навіть там, де він зайвий. Ми
находимо, приміром, в Англії laws of apprenticeship\footnote*{
— закони про учнів. Ред.
} з їхнім
семирічним часом навчання в повній силі аж до кінця мануфактурного
періоду, і лише велика промисловість знищила їх. Через
те, що реміснича вправність лишається основою мануфактури,

ші. «І тут немає нічого дивного, бо як і інші вмілості надто вдосконалені по
великих містах, так само й королівські страви готуються цілком своєрідно.
Бо по дрібних містах та сама людина виробляє ліжка, двері, плуги, столи;
крім цього, вона часто ще й будує доми і задоволена, коли сама находить
такі замовлення, яких вистачає, щоб підтримати своє існування.
Цілком неможливо, щоб людина, яка робить усяку всячину, робила
все добре. Але по великих містах, де кожний поодинокий продуцент
находить багато покупців, досить і одного ремества, щоб прохарчуватися.
Часто непотрібно навіть знати ціле ремество: один виробляє чоловічі
черевики, другий — жіночі. Часто один живе з того, що лише шиє, другий
— з того, що викроює черевики; один крає одяг, інший складає кусники
докупи. Неминучий наслідок цього той, що виконавець найпростішої
праці безумовно й накраще виконує її. Так само стоїть справа і
з куховарством». (Xenophon: «Cyropaedie», lib. VIII, с. 2). Тут вважається
виключно на те, як дійти доброї якости споживної вартости,
хоч уже й Ксенофонтові відома залежність маштабу поділу праці від
обсягу ринку.