якесь суспільне відношення. Навпаки є з еквівалентною формою. Аджеж вона полягає саме в тому, що
тіло якогось товару, як от, приміром, сурдут, дана річ, як така, виражає вартість, отже, з природи
має форму вартости. Правда, це має силу лише в межах того вартостевого відношення, що в ньому товар
«полотно» відноситься до товару «сурдут» як до еквіваленту.21 А що властивості
якоїсь речі не випливають з її відношення до інших речей, а, навпаки, лише виявляються у такому
відношенні, то здається, що й сурдут свою еквівалентну форму, свою властивість безпосередньо
обмінюватись має з природи, цілком так само, як і свою властивість бути важким або затримувати
тепло. Звідси та загадковість еквівалентної форми, що вражає буржуазно-грубий погляд
політико-економа лише тоді, коли ця форма виступає перед ним цілком закінченою, в грошах. Тоді він
намагається спекатися цього містичного характеру золота й срібла, підсовуючи на їхнє місце менше
сліпучі товари і все з новою й новою втіхою пробелькочуючи каталог усієї тієї товарової черні, яка
свого часу відігравала ролю товарового еквіваленту. Він і не передчуває, що вже найпростіший вираз
вартости, як, приміром, 20 метрів полотна = 1 сурдутові, дає розв’язання загадки еквівалентної
форми.

Тіло товару, що служить за еквівалент, завжди фігурує як утілення абстрактної людської праці і
завжди є продукт певної корисної, конкретної праці. Отже, ця конкретна праця стає виразом
абстрактної людської праці. Коли сурдут, приміром, фігурує лише як реалізація абстрактної людської
праці, то й кравецтво, яке фактично реалізується в ньому, фігурує лише як форма реалізації
абстрактної людської праці. У виразі вартости полотна корисність кравецтва не в тому, що воно робить
одіж, а значить, як то німці кажуть, і людей, а в тому, що воно виробляє тіло, по якому можна
пізнати, що це тіло є вартість, отже, що воно є згусток праці, яка ані трохи не відрізняється від
праці, упредметненої у вартості полотна. Щоб зробити таке дзеркало вартости, саме кравецтво не
повинно відбивати в собі нічого іншого, опріч своєї абстрактної властивости бути людською працею.

У формі кравецтва, як і у формі ткацтва, витрачається людську робочу силу. Отже, і одне і друге
мають загальну властивість людської праці, і тому в певних випадках, наприклад, у продукції
вартости, їх треба розглядати лише з цього погляду. В усьому цьому немає нічого таємничого. Але у
виразі вартости товару справу перекручується. Наприклад, щоб виразити, що ткацтво не як таке, не в
своїй конкретній формі ткацтва, утворює вартість полотна, а лише в своїй властивості людської праці
взагалі, — щоб це виразити, ткацтву протиставляється конкретну працю, що продукує еквівалент
полотна, а саме — кравецтво як

21 Такі корелятивні визначення (Reflexionsbestimmungen) взагалі являють собою щось своєрідне.
Приміром, ця людина є король тільки тому, що інші люди відносяться до неї як підданці. Вони ж
думають, навпаки, що вони підданці тому, що він є король.
