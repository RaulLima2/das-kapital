\index{i}{0437}  %% посилання на сторінку оригінального видання
Далі, вартість робочої сили не може падати, отже, додаткова
вартість не може підвищуватися без того, щоб не підвищувалася
продуктивна сила праці; наприклад, у вищенаведеному випадку
вартість робочої сили не може впасти з 3 до 2\shil{ шилінґів}, якщо
підвищена продуктивна сила праці не дозволяє за 4 години продукувати
ту саму масу засобів існування, яка раніш потребувала
для своєї продукції 6 годин. Навпаки, вартість робочої сили не
може підвищитися з 3\shil{ шилінґів} до 4 без того, щоб продуктивна
сила праці не знизилась, отже, без того, щоб не потрібно було
8 годин на продукцію тієї самої маси засобів існування, яку раніш
продукувалося за 6 годин. Звідси випливає, що збільшення продуктивности
праці знижує вартість робочої сили і тим підвищує
додаткову вартість, і, навпаки, зменшення продуктивности праці
підвищує вартість робочої сили та знижує додаткову вартість.

Формулюючи цей закон, Рікардо не помітив одного: хоч зміна
величини додаткової вартости або додаткової праці зумовлює
зворотну зміну величини вартости робочої сили або доконечної
праці, але звідси ні в якому разі не випливає, що вони змінюються
в тій самій пропорції. Вони більшають або меншають на ту саму
величину. Але пропорція, в якій кожна частина новоспродукованої
вартости або робочого дня більшає або меншає, залежить
від первісного поділу, що був перед зміною продуктивної сили
праці. Якщо вартість робочої сили була 4\shil{ шилінґи}, або доконечний
робочий час — 8 годин, додаткова вартість — 2\shil{ шилінґи},
або додаткова праця — 4 години, і якщо, в наслідок підвищення
продуктивної сили праці, вартість робочої сили падає
до 3\shil{ шилінґів}, або доконечна праця падає до 6 годин, то додаткова
вартість підвищується до 3\shil{ шилінґів}, або додаткова праця
до 6 годин. Ту саму величину, 2 години, або 1\shil{ шилінґ}, там додано,
тут однято. Але відносна зміна величин на обох сторонах
різна. Тимчасом як вартість робочої сили падає з 4\shil{ шилінґів}
до 3, отже, на \sfrac{1}{4}, або на 25\%, додаткова вартість підвищується
з 2\shil{ шилінґів} до 3, отже, на \sfrac{1}{2}, або 50\%. Звідси випливає, що
відносне збільшення або зменшення додаткової вартости, яке
постає в наслідок даної зміни продуктивної сили праці, то більше,
що менша, і то менше, що більша була первісно частина робочого
дня, яка виражається в додатковій вартості.

По-третє, збільшення або зменшення додаткової вартости є
завжди наслідок, але ніколи не причина відповідного зменшення
або збільшення вартости робочої сили\footnote{
До цього третього закону Мак Куллох зробив, між іншим, безглуздий
додаток, ніби додаткова вартість може підвищуватися й без зниження
вартости робочої сили, в наслідок скасування податків, що їх раніш мав
платити капіталіст. Скасування таких податків аніскільки не змінює
тієї кількости додаткової вартости, яку промисловий капіталіст безпосередньо
витискує з робітника. Воно змінює лише відношення між тією
частиною додаткової вартости, яку капіталіст ховає собі до кишені, і
тією частиною, що її він мусить віддати третім особам. Отже, воно нічого не
змінює у відношенні між вартістю робочої сили й додатковою вартістю.
Таким чином, виняток Мак Куллоха доводить тільки його нерозуміння
загального правила — лихо, яке з ним трапляється у вульґаризуванні
Рікарда так само часто, як і з Ж. Б. Сеєм у його вульґаризуванні А. Сміса.
}.

\index{i}{0438}  %% посилання на сторінку оригінального видання
А що робочий день є стала величина й виражається в сталій
величині вартости, що кожній зміні величини додаткової вартости
відповідає зворотна зміна величини вартости робочої сили, що
вартість робочої сили може змінятися лише разом із зміною продуктивної
сили праці, то, за цих умов, цілком ясно, що кожна
зміна величини додаткової вартости постає в наслідок зворотної
зміни величини вартости робочої сили. Тому, коли ми раніш
бачили, що неможлива жодна зміна абсолютних величин вартости
робочої сили й додаткової вартости без зміни їхніх відносних
величин, то тепер бачимо, що жодна зміна їхніх відносних
величин вартости неможлива без зміни абсолютної величини вартости
робочої сили.

За третім законом зміна величини додаткової вартости має
собі за передумову зміну вартости робочої сили, спричинену
зміною продуктивної сили праці. Межу зміни величини додаткової
вартости дано новою межею вартости робочої сили. Але навіть
у тому випадку, коли обставини дозволяють цьому законові
діяти, все ж можуть відбуватися проміжні зміни. Якщо, наприклад,
у наслідок підвищеної продуктивної сили праці вартість
робочої сили падає з 4\shil{ шилінґів} до 3, або доконечний робочий час
падає з 8 до 6 годин, то ціна робочої сили могла б спасти лише до
3\shil{ шилінґів} 8\pens{ пенсів}, 3\shil{ шилінґів} 6\pens{ пенсів}, 3\shil{ шилінґів} 2\pens{ пенсів}
і~\abbr{т. д.}, а тому додаткова вартість могла б підвищитися лише до
3\shil{ шилінґів} 4\pens{ пенсів}, 3\shil{ шилінґів} 6\pens{ пенсів}, 3\shil{ шилінґів} 10\pens{ пенсів},
і~\abbr{т. д.} Ступінь спаду, що його мінімальна межа є 3\shil{ шилінґи},
залежить од відносної ваги, яку кидають на терези натиск капіталу,
з одного боку, опір робітників — з другого.

Вартість робочої сили визначається вартістю певної кількости
засобів існування. Із зміною продуктивної сили праці змінюється
вартість цих засобів існування, а не їхня маса. Сама ця
маса, при зростанні продуктивної сили праці, може зростати для
робітника і для капіталіста одночасно і в однаковій пропорції
без якоїбудь зміни між величинами ціни робочої сили й додаткової
вартости. Коли первісна вартість робочої сили дорівнює 3\shil{ шилінґам},
а доконечний робочий час становить 6 годин, коли додаткова
вартість теж дорівнює 3\shil{ шилінґам}, або додаткова праця
становить також 6 годин, то подвоєння продуктивної сили праці,
при незмінному поділі робочого дня, лишило б незмінними ціну
робочої сили й додаткову вартість. Тільки кожна з них виражалася
б у подвійній кількості, але відповідно до цього й здешевілих
споживних вартостей. Хоча ціна робочої сили й лишалася б
незмінна, все ж вона підвищилася б понад її вартість. Коли б
ціна робочої сили впала, але не до мінімальної межі в 1\sfrac{1}{2}\shil{ шилінґа},
визначеної її новою вартістю, а до 2\shil{ шилінґів} 10\pens{ пенсів}, 2\shil{ шилінґів}
6\pens{ пенсів} і~\abbr{т. д.}, то й це падіння ціни все ще репрезентувало б
зростання маси засобів існування. Таким чином при зростанні
\parbreak{}  %% абзац продовжується на наступній сторінці
