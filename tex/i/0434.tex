Далі: «Якщо сукупність робітників якоїсь країни продукує
на 20\% більше понад суму їхніх заробітних плат, то зиски будуть
20\%, хоч який буде стан товарових цін». — Це, з одного
боку, надзвичайно вдатна тавтологія, бо якщо робітники продукують
для своїх капіталістів додаткову вартість у 20\%, то й зиски
відноситимуться до всієї заробітної плати робітників як 20: 100.
З другого боку, це абсолютна неправда, що зиски «будуть 20\%».
Вони завжди мусять бути менші, бо зиски обчислюється на всю
суму авансованого капіталу. Припустімо, наприклад, що капіталіст
авансував 500 фунтів стерлінґів, з них 400 фунтів стерлінґів
на засоби продукції, 100 фунтів на заробітну плату. Хай
норма додаткової вартости буде, як ми вже припустили, 20\%;
тоді норма зиску буде 20: 500, тобто 4\%, а не 20\%.

Далі йде блискучий зразок того, як Мілл розглядає різні
історичні форми суспільної продукції: «Я всюди припускаю
сучасний стан речей, який з деякими винятками панує всюди,
тобто припускаю, що капіталіст робить усі видатки наперед,
включаючи сюди й оплату робітника».* Дивовижна оптична
омана — бачити всюди стан речей, який досі панує на земній
кулі лише винятково! Але далі. Мілл ласкаво згоджується, що
«немає абсолютної доконечности в тому, щоб такий був стан
речей». Навпаки. «Робітник міг би зачекати виплати навіть
усього свого заробітку, поки працю цілком закінчиться, коли б
він мав засоби, потрібні на його утримання протягом цього часу.
Але в тому випадку він був би до певної міри капіталістом, що
вкладав би капітал у підприємство й постачав би частину фонду,
потрібного на його ведення». З таким самим правом Мілл міг би
сказати, що робітник, який сам собі авансує не тільки засоби
існування, а ще й засоби праці, є в дійсності свій власний найманий
робітник. Або що американський селянин є свій власний
раб який лише відбуває панщину на себе самого, а не на чужого
пана.

Показавши нам так ясно, що капіталістична продукція, навіть
і тоді, коли б її не існувало все таки існувала б, Мілл потім
настільки послідовний, що доводить, що капіталістична продукція
не існує навіть тоді, коли вона існує: «І навіть у попередньому
випадку [коли капіталіст авансує найманому робітникові всі
засоби його існування] робітника можна розглядати з того самого
погляду [тобто як капіталіста]. Бо, віддаючи свою працю нижче

* Це речення — як про це пише Маркс у своєму листі з 28 листопада
1878 р. до перекладача російського видання «Капіталу» — Даніельсона —
треба читати так: «Я всюди припускаю сучасний стан речей, який з деякими
винятками панує всюди, де робітники й капіталісти є відокремлені
кляси, тобто припускаю, що капіталіст» і т. д. А дальші два речення:
«Дивовижна оптична омана — бачити всюди стан речей, який досі
панує на земній кулі лише винятково! Але далі.» — треба викреслити, і
наступне речення читати так: «Пан Мілл, звичайно, гадає, що немає
абсолютної доконечности в тому, щоб такий був стан речей — навіть
і за економічної системи, де робітники й капіталісти є відокремлені
кляси». Ред.
