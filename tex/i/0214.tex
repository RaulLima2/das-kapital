Постлетвайт каже, між іншим: «Я не можу закінчити цих
коротких заміток, не звернувши уваги на дуже поширений тривіальний
спосіб вислову. Коли робітник (industrious poor), чуємо
ми з уст багатьох, за п’ять днів може одержати досить для того,
щоб жити, то він не захоче працювати повних шість днів. Тому
вони доходять висновку, що треба податками або якимись іншими
способами вдорожчити навіть доконечні засоби існування, щоб
примусити ремісника й мануфактурного робітника до безперервної
шестиденної праці на тиждень. Я мушу попросити дозволу мати
собі інший погляд, ніж мають ці великі політики, що ламають
списи за постійне рабство робітничої людности цього королівства
(«the perpetual slavery of the working people»); вони забувають
приказку: «all work and no play» (сама тільки праця без
забави робить дурним). Хіба ж не пишаються англійці геніальністю
і вмілістю своїх ремісників і мануфактурних робітників,
що досі здобували для брітанських товарів загальну довіру і
славу? Якій обставині завдячується це? Мабуть, нічому іншому,
а тільки тому способові, яким наша робітнича людність ориґінально-весела,
уміє розважатися. Коли б вони мусили працювати
цілий рік усі шість днів на тиждень, повторюючи з дня на день
ту саму працю, невже це не притупило б їхньої геніяльности й не
зробило б їх, жвавих і вправних, тупо-байдужими, і чи не втратили
б наші робітники через таке вічне рабство своєї слави замість
зберегти її?.. Якої майстерної вправности можна б сподіватися
від таких жорстоко мордованих тварин (hard driven animals)?..
Багато з них виконує за 4 дні таку кількість праці, яку француз
виконує за 5 або 6 днів. Але коли англійці мають бути робітниками,
що вічно обтяжені працею, то можна побоюватись, що вони
ще більш виродяться (degenerate), ніж французи. Коли наш нарід
славиться своєю хоробрістю на війні, то хіба не кажемо ми, що
цим він завдячує, з одного боку, доброму англійському ростбіфові
й пудинґові в його шлунку, а з другого — не в меншій мірі
нашому конституційному духові волі. І чому б більшу геніальність,
енергію і вправність наших ремісників і мануфактурних робітників
не завдячувалось тій волі, з якою вони на свій лад розважаються?
Я сподіваюся, що вони ніколи не втратять ні цих
привілеїв, ні доброго життя, звідки однаково випливають їхня
вмілість у праці і їхня сміливість!» 122

На це автор: «Essay on Trade and Commerce» відповідає ось як:

них у своєму творі «Considerations on Taxes», London 1765. Сюди в
першу чергу слід зарахувати й Полоній Артура Юнґа, невимовного базікала
у статистиці. Серед оборонців робітників визначаються: Jacob Vanderlint
в «Money answers all things», London 1734, Reverend Nathaniel
Forster, D. D. в «An Enquiry into the Causes of the Present Price of Provisions»,
London 1767, Dr. Price і, особливо Postlethwayt, так в одному
додатку до його «Universal Dictionary of Trade and Commerce», як і в
«Great Britain’s Commercial Interest explained and improved», 2nd ed.
London 1775. Сами факти находимо сконстатованими в багатьох інших
письменників того часу, між іншим, у Джосії Текера.

122 Postlethwayt. Там же: «Firts Preliminary Discourse», р. 14.
