Грабування церковних маєтків, шахрайське відчуження державних
земель, крадіж громадських земель, узурпаторське і з
нещадним тероризмом проведене перетворення февдальної й
кланової власности на сучасну приватну власність, — такі є
ідилічні методи первісної акумуляції. Вони завоювали поле
для капіталістичного рільництва, прилучили землю до капіталу
й утворили для міської промисловости потрібний приплив вільних
як птиці, пролетарів.

n’ose pas vendre ses productions où et comme il le veut; il n’ose pas acheter
ce dont il a besoin aux marchands qui pourraient le lui livrer au meilleur
prix. Toutes ces causes le ruinent insensiblement, et il se trouverait hors
d'état de payer les impôts directs à l’échéance sans la filerie: elle lui offre
une ressource, en occupant utilement sa femme, ses enfants, ses servants, ses valets,
et lui même: mais quelle pénible vie, même aidée de secours! En été, il
travaille comme un forçat au labourage et à la récorlte; il se couche’ à
9 heures et se lève à deux, pour suffire aux travaux; en hiver il devrait réparer
ses forces par un plus grand repos; mais il manquera de grains pour le
pain et les semailles, s’il se défait des denrées qu’il faudrait vendre pour payer
les impôts. Il faut done filer pour suppléer à ce vide... il faut y apporter
la plus grande assiduité. Aussi le paysan se couche-t-il en hiver à
minuit, une heure, et se lève à cinq ou six; ou bien il se couche à neuf, et
se lève à deux, et cela tous les jours de sa vie si ce n’est le dimanche. Cet
excès de veille et de travail usent la nature humaine, et de là vient qu’hommes
et femmes vieillissent beaucoup plutôt dans les campagnes que dans
les villes»). (Mirabeau: «De la Monarchie Prussienne», Londres 1788,
vol. III, p. 212, 222).

Додаток до другого видання. У квітні 1866 року, 18 років після
опублікування вищецитованої праці Роберта Сомерса, професор Леон
Леві прочитав у Society of Arts доповідь про перетворення овечих пасовиськ
на мисливські парки, що в ній він змалював проґрес у спустошенні
в шотляндських гірських місцевостях. Він каже, між іншим, таке: «Проганяння
людности й перетворювання орного поля на пасовиська для
овець становили найвигідніший засіб діставати доходи без витрат... Заміна
овечих пасовиськ мисливськими парками, стала звичайним явищем
по гірських місцевостях. Овець виганяють дикі звірі, як раніше виганяли
людей, щоб очистити місце для овець... У Форфаршірі можна пройти
від маєтків графа фон Делгуза до маєтків Джон Ґротс, не виходячи
зовсім з мисливських лісів. У багатьох (із цих лісів) уже давно живуть
лисиці, дикі коти, куниці, тхорі, ласки й альпійські зайці, недавно
найшли собі туди шлях кролі, білиці й пацюки. Величезні земельні простори,
які в шотляндській статистиці фігурували як винятково родючі
та обширні луки, нині стоять поза всякою культурою і поліпшеннями і
призначені виключно на мисливську забаву небагатьох осіб, тай то ця
забава триває лише недовгий період року».

Лондонський «Economist» від 2 червня 1866 р. пише «Одна шотляндська
газета між іншими новинами з останнього тижня подає й таку:
«Одну з найкращих овечих фарм у Sutherlandshire, за яку недавно, як
вийшов строк оренди, давали 1.200 фунтів стерлінґів річної ренти, перетворено
на deer forest!» Февдальні інстинкти виявляються так само...
як за тих часів, коли нормандський завойовник... зруйнував 36 сел, щоб
створити new forest... Два мільйони акрів, що охоплюють деякі з найродючіших
земель Шотляндії, перетворено на цілковиту пустелю. Природні
трави з Glen Tilt’a вважалось за найпоживнішу пашу у графстві Perth;
мисливський ліс у Ben Aulder давав найкращу траву великій окрузі
Badenoch; частина лісу Blak-Mount була найкращим у Шотляндії пасовиськом
для чорних овець. Про розміри земельної площі, спустошеної
для аматорів полювання, можна собі скласти уявлення з того факту, що
ця площа простором далеко більша, аніж ціле графство Perth. Як багато
