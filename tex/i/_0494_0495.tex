\parcont{}  %% абзац починається на попередній сторінці
\index{i}{0494}  %% посилання на сторінку оригінального видання
у \num{2.000}\pound{ фунтів стерлінґів}, яка капіталізується. Новий капітал у
\num{2.000}\pound{ фунтів стерлінґів} дає нову додаткову вартість у 400\pound{ фунтів
стерлінґів}; ця остання знову капіталізується, отже, перетворюється
на другий додатковий капітал, що дає нову додаткову
вартість у 80\pound{ фунтів стерлінґів}, і~\abbr{т. д.}

Ми залишаємо тут осторонь ту частину додаткової вартости,
що її споживає капіталіст. Так само мало цікавить нас у даний
момент те, чи додаткові капітали додаються до первісного капіталу,
чи відокремлюються від нього, щоб самостійно зростати
своєю вартістю; чи використовує їх той самий капіталіст, що
їх нагромадив, чи він передає їх до інших рук. Ми мусимо лише
не забувати, що поруч новоутворених капіталів первісний капітал
і далі репродукує себе й продукує додаткову вартість, і що
те саме має силу для кожного акумульованого капіталу у відношенні
до створеного ним додаткового капіталу.

Первісний капітал утворився в наслідок авансування \num{10.000}\pound{ фунтів стерлінґів}. Звідки має їх їхній власник? Він їх добув
своєю власною працею і працею своїх предків! — відповідають
нам в один голос представники політичної економії,\footnoteA{
«Первісна праця, якій його капітал завдячує своє походження»
(«Le travail primitif auquel son capital a dû sa naissance»). (\emph{Sismondi}:
«Nouveaux Principes d’Economie Politique», éd. Paris, vol. I, p. 109).
} і це їхнє
припущення дійсно здається одним-єдиним, що узгоджується
з законами товарової продукції.

Цілком інакше стоїть справа з додатковим капіталом у \num{2.000}\pound{ фунтів стерлінґів}. Процес його постання ми знаємо цілком докладно.
Він є капіталізована додаткова вартість. Від самого початку
він не містить у собі жодного атома вартости, що не походив
би з неоплаченої чужої праці. Засоби продукції, до яких долучається
додаткова робоча сила, і так само засоби існування,
з яких вона себе утримує, є не що інше, як інтеґральні складові
частини додаткового продукту, отієї данини, яку кляса капіталістів
щорічно вириває в робітничої кляси. Коли кляса капіталістів
за якусь частину цієї данини купує в робітничої кляси
додаткову робочу силу навіть за повну ціну, так що еквівалент
обмінюється на еквівалент, то все таки це давно відома операція
завойовника, що купує у переможених товари за їхні власні,
пограбовані в них гроші.

Якщо додатковий капітал уживає до праці свого власного
продуцента, то цей останній мусить, поперше, і далі збільшувати
вартість первісного капіталу і, крім того, відкуповувати продукт
своєї попередньої праці за більшу працю, ніж той продукт коштував.
Коли розглядати це як оборудку між клясою капіталістів
і робітничою клясою, то справа ані трохи не зміниться,
коли за допомогою неоплаченої праці занятих досі робітників
уживатимуть до праці додаткових робітників. Капіталіст, може,
перетворює додатковий капітал на машину, яка викидає продуцента
цього додаткового капіталу на брук, заміняючи його
кількома дітьми. У всіх випадках робітнича кляса своєю додатковою
\index{i}{0495}  %% посилання на сторінку оригінального видання
працею протягом даного року створила капітал, що наступного
року вживатиме до праці додаткових робітників.\footnote{
«Праця створює капітал, раніш ніж капітал починає вживати
праці» («Labour creates capital, before capital employs labour»). (\emph{E. G.
Wakefield}: «England and America», London 1833, vol. II, p. 110).
} Оце й є те, що зветься: «утворювати капітал капіталом».

За передумову акумуляції першого додаткового капіталу в
\num{2.000}\pound{ фунтів стерлінґів} була сума вартости в \num{10.000}\pound{ фунтів стерлінґів},
авансована капіталістом і належна йому силою його
«первісної праці». Навпаки, передумова другого додаткового
капіталу в 400\pound{ фунтів стерлінґів} є не що інше, як попередня акумуляція
першого, акумуляція тих \num{2.000}\pound{ фунтів стерлінґів}, що
їхньою капіталізованою додатковою вартістю є цей додатковий
капітал у 400\pound{ фунтів стерлінґів}. Власність на минулу неоплачену
працю з’являється тепер одним-однією умовою теперішнього
присвоєння живої неоплаченої праці в щораз більшому й
більшому розмірі. Що більше капіталіст акумулював, то більше
може він акумулювати.

Оскільки додаткова вартість, що з неї складається додатковий
капітал № І, була результатом купівлі робочої сили за частину
первісного капіталу, купівлі, що відповідала законам товарового
обміну і з юридичного погляду припускає лише вільне порядкування
на боці робітника його власними здібностями, а на боці
власника грошей або товарів — належними йому вартостями;
оскільки додатковий капітал № II і~\abbr{т. ін.} є лише результат додаткового
капіталу № І, отже, наслідок цього першого відношення;
оскільки кожна поодинока оборудка завжди відповідає
законові товарового обміну, отже, капіталіст завжди купує
робочу силу, а робітник завжди продає її, припустімо, навіть
за її дійсною вартістю, остільки ясно, що закон присвоєння, або
закон приватної власности, що ґрунтується на товаровій продукції
й товаровій циркуляції, перетворюється через свою власну,
внутрішню, неминучу діялектику на свою пряму протилежність.
Обмін еквівалентів, що виступав як первісна операція, зазнав
таких змін, що тепер лише на позір відбувається обмін, бо, по-перше,
частина капіталу, обміняна на робочу силу, сама є лише
частина продукту чужої праці, присвоєного без еквіваленту, і,
по-друге, її продуцент, робітник, мусить не тільки замістити її,
а замістити її ще й з новим додатком. Відношення обміну між
капіталістом і робітником стає таким чином тільки позірністю,
властивою процесові циркуляції, лише формою, що є чужа самому
змістові й тільки містифікує його. Постійна купівля й продаж
робочої сили — це форма. Зміст є той, що капіталіст частину
упредметненої вже чужої праці, яку він безупинно присвоює
собі, не даючи за неї жодного еквіваленту, постійно знову обмінює
на більшу кількість живої чужої праці. Спочатку право
власности здавалося нам основаним на власній праці. Ми мусили,
принаймні, визнати це припущення, бо лише рівноправні власники
товарів протистоять один одному, а засіб до присвоєння чужого
\parbreak{}  %% абзац продовжується на наступній сторінці
