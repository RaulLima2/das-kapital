реження й репродукцію, а з другого боку, знищуючи засоби
існування, воно дбає про те, щоб вони постійно знову й знов
з’являлися на ринку праці. Римський раб був прикований
кайданами, а найманий робітник прив’язаний незримими нитками
до свого власника. Видимість його незалежности підтримує
постійна зміна індивідуальних панів-наймачів і юридична фікція
контракту.

Колись капітал, де це йому здавалося потрібним, здійснював
своє право власности на вільного робітника за допомогою примусового
закону. Так, наприклад, до 1815 р. еміграцію машинобудівельних
робітників в Англії було заборонено під загрозою
тяжкої кари.

Репродукція робітничої кляси включає також передачу і
нагромаджування вправности від одного покоління до другого.\footnote{
«Єдина річ, про яку можна сказати, що її нагромаджують і заздалегідь
підготовляють, — це вправність робітника... Акумуляція і нагромадження
вправної праці, ця найважливіша операція, провадиться щодо
великої маси робітників без жодного капіталу». (Hodgskin: «Labour
Defended etc.» p. 13).
}
До якої міри капіталіст вважає існування такої вправної
робітничої кляси за одну з належних йому умов продукції, розглядає
її в дійсності як реальне існування свого змінного капіталу,
виявляється тоді, коли криза загрожує йому її втратою.
Як відомо, в наслідок американської громадянської війни й
бавовняного голоду, що її супроводив, було викинуто на брук
більшість бавовняних робітників у Ланкашірі й т. ін. З надр
самої робітничої кляси, як і з інших верств суспільства, залунав
заклик до державної допомоги та добровільних національних
пожертов, щоб уможливити еміґрацію «зайвих» робітників до
англійських колоній або до Сполучених штатів. Тоді «Times»
(24 березня 1863 р.) опублікував листа Едмунда Потера, колишнього
президента менчестерської торговельної палати. В Палаті
громад його лист цілком справедливо названо «маніфестом
фабрикантів».\footnote{
«Цей лист можна розглядати як маніфест фабрикантів» («Thal
letter might be looked upon as the manifesto of the manufacturers»).
(Ferrand: «Подання з приводу бавовняного голоду, засідання Палати
громад з 27 квітня 1863 р.»).
} Ми подаємо тут із нього деякі характеристичні
місця, де без прикрас говориться про право власности капіталу
на робочу силу.

«Бавовняним робітникам можуть сказати, що їх забагато на
ринку праці... що їх, може, треба б зменшити на одну третину,
і тоді настане нормальний попит на останні дві третини... Громадська
думка наполягає на еміграції. Хазяїн (тобто бавовняний
фабрикант) не може добровільно згодитися на зменшення подання
праці; він вважає, що це було б так само несправедливо, як
і неправильно... Якщо еміграцію підтримують із громадських
фондів, то він має право вимагати, щоб його вислухали, а може
і протестувати». Цей Потер пояснює далі, яка корисна бавовняна
промисловість, як «вона, безперечно, відтягла людність з Ірлян-