є рішучий проґресист, проте він фабрикує чоботи не задля них
самих. Споживна вартість за товарової продукції взагалі не є
та річ, «qu’on aime pour lui-même».\footnote*{
— «яку люблять задля неї самої». Ред.
} Споживні вартості взагалі
продукується тут лише тому й остільки, що й оскільки вони є
матеріяльний субстрат, носії мінової вартости. І наш капіталіст
дбає про дві речі. Поперше, він хоче продукувати споживну
вартість, що має мінову вартість, предмет, призначений на продаж,
товар. А, подруге, він хоче продукувати товар, вартість
якого вища від суми вартости товарів, потрібних для його продукції,
тобто засобів продукції й робочої сили, на які він авансував
на товаровому ринку свої власні гроші. Він хоче продукувати
не тільки споживну вартість, але й товар, не тільки споживну
вартість, але й вартість, і не тільки вартість, а ще й додаткову
вартість.

А що тут мова йде про товарову продукцію, то ми на ділі,
очевидно, розглядали досі лише один бік процесу. Як сам товар
є єдність споживної вартости й вартости, так і процес продукції
товару мусить бути єдністю процесу праці й процесу творення
вартости.

Розгляньмо тепер процес продукції ще й як процес творення
вартости.

Ми знаємо, що вартість кожного товару визначається кількістю
праці, зматеріялізованої в його споживній вартості, робочим
часом, суспільно-доконечним для його продукції. Це має силу
і для продукту, що його наш капіталіст дістав як результат процесу
праці. Отже, слід насамперед обчислити працю, упредметнену
в цьому продукті.

Нехай це буде, приміром, пряжа.

Для виготовлення пряжі потрібно було насамперед сировинного
матеріялу, приміром, 10 фунтів бавовни. Яка вартість бавовни,
цього тепер не доводиться досліджувати, бо капіталіст
купив її на ринку за її вартістю, приміром, за 10 шилінґів. У ціні
бавовни потрібну для її продукції працю вже виражено як загальносуспільну
працю. Припустімо далі, що зужиткована на
перероблення бавовни кількість веретен, які для нас репрезентують
усі інші вжиті засоби праці, — має вартість у 2 шилінґи.
Коли маса золота в 12 шилінґів є продукт 24 робочих годин, або
2    робочих днів, то з цього насамперед випливає, що у пряжі упредметнено
2 робочі дні.

Та обставина, що бавовна змінила свою форму, а зужиткована
кількість веретен зовсім зникла, не повинна спантеличувати
нас. За загальним законом вартости, 10 фунтів пряжі, приміром,
є еквівалент 10 фунтів бавовни й чверти веретена, коли вартість
40 фунтів пряжі дорівнює вартості 40 фунтів бавовни плюс вартість
одного цілого веретена, тобто, коли потрібно однакового
робочого часу, щоб випродукувати одну й другу сторони цього
рівнання. У цьому випадку той самий робочий час репрезенту-