рення грошей на товар і зворотне перетворення товару на гроші,
купівлю задля продажу. Гроші, що у своїм русі пророблюють
цю останню циркуляцію, перетворюються на капітал, стають
капіталом і вже за своїм призначенням є капітал.

Пригляньмося ближче до циркуляції Г — Т — Г. Вона перебігає,
як і проста товарова циркуляція, дві протилежні фази.
У першій фазі, Г — Т, в купівлі, гроші перетворюються на товар.
У другій фазі, Т — Г, у продажу, товар зворотно перетворюється
на гроші. Але єдність обох фаз — це сукупний рух, що
обмінює гроші на товар і цей самий товар знов на гроші, купує
товар, щоб його продати, або, якщо не зважати на формальну
ріжницю між купівлею і продажем, за гроші купує товар і за
товар гроші.2 Результат, що в ньому згасає цілий процес, є
обмін грошей на гроші, Г — Г. Коли я за 100 фунтів стерлінґів
купую 2.000 фунтів бавовни і знов продаю ці 2.000 фунтів
бавовни за 110 фунтів стерлінґів, то, кінець-кінцем, я
обміняв 100 фунтів стерлінґів на 110 фунтів стерлінґів — гроші
на гроші.

Правда, тепер очевидно, що процес циркуляції Г — Т — Г
був би безглуздим і беззмістовним, коли б такими манівцями
бажали обміняти певну грошову вартість на таку саму грошову
вартість, отже, приміром, 100 фунтів стерлінґів на 100 фунтів
стерлінґів. Куди простішою й певнішою лишалася б метода
вбирача скарбу, який затримує в себе свої 100 фунтів стерлінґів,
замість того, щоб віддати їх на небезпеку циркуляції. З другого
боку, чи купець куплену ним за 100 фунтів стерлінґів бавовну
знову продає за 100 фунтів стерлінґів, чи мусить він продати її за
100 фунтів стерлінґів, а то й за 50 фунтів стерлінґів, — за всяких
обставин його гроші пророблюють своєрідний, ориґінальний
рух, цілком відмінний од того руху, що його пророблюють
гроші у простій товаровій циркуляції, приміром, у руках селянина,
який продає збіжжя і за виручені таким чином гроші
купує одяг. Отже, насамперед потрібно схарактеризувати ріжницю
форм кругобігів Г — Т — Г і Т — Г — Т. Разом із тим
виясниться і ріжниця в змісті, що криється за цими ріжницями
форм.

Погляньмо насамперед, що є спільного в обох цих формах.

Обидва кругобіги розпадаються на ті самі дві протилежні
фази: Т — Г, продаж, і Г — Т, купівля. В кожній з обох цих
фаз протистоять один одному ті самі два речові елементи, товар
і гроші, — і дві особи в тих самих характеристичних економічних
масках, покупець і продавець. Кожний з цих обох кругобігів
є єдність тих самих протилежних фаз, і обидва рази ця
єдність здійснюється за допомогою появи трьох контраґентів,

2 «За гроші купуємо товари, а за товари гроші» («Avec de l’argent
on achète des marchandises, et avec des marchandises on achète de l’argent).
(Mercier de la Rivière: «L’ordre naturel et essentiel des sociétés
politiques» Physiocrates, éd. Daire, 11. Partie, p. 543).
