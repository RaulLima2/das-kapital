\parcont{}  %% абзац починається на попередній сторінці
\index{i}{0584}  %% посилання на сторінку оригінального видання
різного віку, то число їх зменшилося з 1.241.269 в 1851 р. до
1.163.217 в 1861 р.\footnote{
Число пастухів овець збільшилося з 12.517 до 25.559.
} Тим-то, коли генеральний англійський
реєстратор справедливо зауважує: «Приріст числа фармерів і
сільських робітників од 1801 р. не стоїть ні в якій пропорції
до приросту рільничого продукту»,\footnote{
«Census etc.», vol. II, р. 36.
} то ця диспропорція ще
куди більше стосується до останнього періоду, коли позитивне
зменшення сільської робітничої людности відбувалось рівнобіжно
із збільшенням оброблюваної площі, інтенсифікацією культури,
нечуваною акумуляцією капіталу, вкладеного в землю і призначеного
для її оброблювання, нечуваним в історії англійської
аґрономії зростанням кількости рільничого продукту, буйним
підвищенням ренти землевласників і надзвичайним зростанням
багатства капіталістичних фармерів. Коли до цього всього додати
ще безперервне швидке поширення міських ринків збуту й
панування вільної торговлі, то сільського робітника post tot
discrimina rerum,\footnote*{
— після стількох різних пригод. \emph{Ред.}
} нарешті, поставлено в такі умови, які secundum
artem\footnote*{
— згідно з теорією. \emph{Ред.}
} мусіли були зробити його безмежно щасливим.

Навпаки, професор Роджерс доходить такого висновку, що
становище сільського робітника за наших часів надзвичайно
погіршало не лише порівняно з становищем його попередника
в останній половині XIV століття і в XV столітті — про це й
казати годі — але й порівняно з становищем його попередника
періоду від 1770 р. до 1780 р., що «він знов став кріпаком», і
до того ж кріпаком погано годованим і погано забезпеченим щодо
житла.\footnote{
Rogers: «A History of Agriculture and Prices in England», Oxford
1866, vol. I, p. 693, «The peasant has again become a serf». Там же, стор. 10.
П. Роджерс належить до ліберальної школи, він особистий приятель
Кобдена і Брайта, отже, зовсім не laudator temporis acti.\footnote*{
— хвалій минулих часів. \emph{Ред.}
}
} Д-р Джуліян Гентер у своєму епохальному звіті про
житлові умови сільських робітників каже: «Кошти існування
hind’a (назва сільського робітника за часів кріпацтва) фіксовано
в розмірі якнайнижчої суми, що на неї він може жити\dots{} його
заробітна плата й притулок не стоять ні в якій пропорції до зиску,
що мають видушити з нього. Він є нуль у рахунках фармера.\footnote{
«Public Health. Seventh Report», London 1865, p. 242. «The cost of
the hind is fiked at the lowest possible amount on which he can live\dots{} the
supplies of wages or shelter are not calculated on the profit to be derived
from him. He is a zero in farming calculations». Тим-то немає нічого незвичайного
в тім, що або домовласник збільшує квартирну плату для робітника,
коли він почує, що робітник заробляє дещо більше, або фармер
зменшує робітникові заробітну плату, «бо його жінка найшла собі
роботу». (Там же).
}
Засоби його існування завжди розглядають як сталу величину»,\footnote{
Там же, стор. 135.
}
«Щождо дальшого зменшування його доходу, то він може сказати:
nihil habeo, nihil curo.\footnote*{
Нічого не маю, ні про що не дбаю. \emph{Ред.}
} Йому будучина не страшна, бо
\parbreak{}  %% абзац продовжується на наступній сторінці
