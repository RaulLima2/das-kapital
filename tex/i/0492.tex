або в чистому продукті в 40.000 фунтів пряжі, тобто в шостині
гуртового продукту вартістю в 2.000 фунтів стерлінґів, яка
реалізується в його продажу. [Якщо ці 2.000 фунтів стерлінґів
знову авансуються як капітал, то первісний капітал зростає
з 10.000 фунтів стерлінґів до 12.000 фунтів стерлінґів, тобто
відбулася акумуляція. Насамперед не має значення, чи цей додатковий
капітал додається до старого, чи він самостійно зростає
своєю вартістю].\footnote*{
Заведене у прямі дужки ми беремо з другого німецького видання.
\emph{Ред.}
} Сума вартости в 2.000 фунтів стерлінґів є сума
вартости в 2.000 фунтів стерлінґів. По цих грошах не чуєш і не
бачиш, що вони є додаткова вартість. Характер вартости як додаткової
вартости показує нам, яким чином вона дісталася до
рук свого власника, але нічого не змінює в природі вартости
або грошей.

Отже, щоб перетворити новоприбулу до нього суму в 2.000 фунтів
стерлінґів на капітал, прядільний фабрикант, за інших незмінних
обставин, авансує з цієї суми чотири п’ядинина купівлю
бавовни й т. д. і однуп’ятину на купівлю нових робітників-прядунів,
які знайдуть на ринку ті засоби існування, що їхню вартість
він їм авансував. Тоді цей новий капітал 2.000 фунтів
стерлінґів, функціонуватиме у прядільництві та з свого боку
приноситиме додаткову вартість у 400 фунтів стерлінґів.

Капітальна вартість була первісно авансована в грошовій
формі; навпаки, додаткова вартість з самого початку існує як
вартість якоїсь певної частини гуртового продукту. Якщо цей
продукт продається, перетворюється на гроші, то капітальна
вартість знову набирає своєї первісної форми, а додаткова вартість
змінює свій первісний спосіб буття. Однак від цього моменту
капітальна вартість і додаткова вартість, одна і друга, є грошові
суми, і їхнє зворотне перетворення на капітал відбувається цілком
тим самим способом. І одну і другу капіталіст витрачає на
купівлю товарів, що дають йому змогу знову розпочати виготовлювати
свій продукт, і до того ж цього разу в поширеному
маштабі. Але, щоб купити ці товари, він мусить знайти їх на ринку.

Його власна пряжа циркулює лише тому, що він виносить
свій річний продукт на ринок, як це роблять із своїми товарами
також і всі інші капіталісти. Але перш, ніж ці товари з’явилися
на ринку, вони вже містилися в фонді річної продукції, тобто в
загальній масі предметів усякого роду, на які перетворюється
протягом року ціла сума поодиноких капіталів або цілий суспільний
капітал, що з нього кожний поодинокий капіталіст має
у своїх руках лише певну частину. Процеси, що відбуваються на
ринку, лише переміщують поодинокі складові частини річної
продукції, пересилають їх із рук у руки, але не можуть ні збільшити
цілу суму річної продукції, ані змінити природу випродукованих
предметів. Отже, той ужиток, що його можна зробити
з цілого річного продукту, залежить од власного складу цього
продукту, але ж зовсім не від циркуляції.