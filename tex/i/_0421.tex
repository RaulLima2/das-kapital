\parcont{}  %% абзац починається на попередній сторінці
\index{i}{0421}  %% посилання на сторінку оригінального видання
місця, що підносило б їхні рішення понад небезсторонність, якої
ви боїтесь? — Я відмовляюсь відповідати на питання про особистий
характер цих людей. Я переконаний, що в багатьох випадках
вони поводяться дуже небезсторонньо, і що треба відібрати
в них цю владу там, де на карту ставиться людське життя».
В того самого буржуа вистачає безстидства спитати: «Чи не гадаєте
ви, що й власники копалень мають утрати в наслідок вибухів?»
— І, нарешті, ще (№ 1042): «Чи не можете ви, робітники,
сами захистити свої власні інтереси, не вдаючися по допомогу
до уряду? — Ні». — 1865 р. у Великобрітанії було 3.2І7 кам’яновугільних
копалень і — 12 інспекторів. Один власник копалень
у Йоркшірі («Times», з 26 січня 1867 р.) сам обчислює, що
інспектори, коли навіть залишити осторонь їхні суто бюрократичні
справи, які забирають у них увесь час, могли б відвідати
кожну копальню лише один раз на десять років. Не диво, що
останніми роками (особливо ж і 1866 і 1867 рр.) катастрофи проґресивно
збільшувалися числом і своїми розмірами (іноді число
жертов становило 200—300 робітників). Оце вам краса «вільної»
капіталістичної продукції!

У всякому разі закон 1872 р., хоч які в нього хиби, є перший
закон, що реґулює години праці дітей, вживаних по копальнях,
і до певної міри робить відповідальними за так звані нещасливі
випадки експлуататорів і власників копалень.

Королівська комісія 1867 р. для розсліду умов праці дітей,
підлітків та жінок у рільництві опублікувала декілька дуже
важливих звітів. Зроблено різні спроби застосувати принципи
фабричного законодавства, у змодифікованій формі, до рільництва,
але всі вони досі кінчалися повною невдачею. Але на одно
я маю тут звернути увагу, а саме на існування непереможної
тенденції до загального застосування цих принципів.

Якщо, з одного боку, загальне поширення фабричного законодавства
як засобу фізичного й духовного захисту робітничої
кляси стало неминучим, то, з другого боку, як це вже зазначено,
воно повсюдно поширює й прискорює перетворення розпорошених
процесів праці карликового маштабу на комбіновані процеси
праці великого суспільного маштабу, отже, повсюдно поширює
й прискорює концентрацію капіталу й самовладне панування
фабричного режиму. Воно руйнує всі старовинні й переходові
форми, за якими панування капіталу почасти ще ховається і
замінює їх прямим, незахованим пануванням капіталу. Цим
воно повсюдно поширює й безпосередню боротьбу проти цього
панування. Тимчасом як в індивідуальних майстернях воно примушує
до одноманітности, реґулярности, порядку й економії, —
воно, в наслідок величезного поштовху, що його дають розвиткові
техніки обмеження й реґулювання робочого дня, збільшує
анархію й катастрофи капіталістичної продукції в цілому, збільшує
інтенсивність праці й конкуренцію машин з робітником. Разом
із сферами дрібного виробництва та домашньої праці воно знищує
останні притулки «зайвих» робітників, а тим самим і на-
\parbreak{}  %% абзац продовжується на наступній сторінці
