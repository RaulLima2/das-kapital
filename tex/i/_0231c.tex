\parcont{}  %% абзац починається на попередній сторінці
\index{i}{0231}  %% посилання на сторінку оригінального видання
пішовши на фабрику.\footnote{
«The delicate texture of the fabric in which they were employed
requiring a lightness of touch, only to be acquired by their early introduction
to these factories». (Там же, стор. 20).
} Задля делікатних пальців геть вирізували
дітей, як на півдні Росії ріжуть рогату худобу задля шкури
й лою. Нарешті, 1850 р. привілей, даний законом 1844 р., залишено
тільки відділам сукання й розмотування шовку, але, щоб
відшкодувати позбавлений «волі» капітал, робочий час для дітей
11--13 років тут підвищено з 10 до 10\sfrac{1}{2} годин. Привід: «Праця
на фабриках шовку легша, ніж на інших, і ні в якому разі не така
шкідлива для здоров’я».\footnote{
«Reports etc. for 31 st October 1861», p. 26.
} Офіціяльні лікарські розсліди показали
згодом, що, навпаки, «пересічний відсоток смертности в
округах шовкової промисловости винятково високий, а серед жіночої
частини людности навіть вищий, ніж в округах бавовняної
промисловости Ланкашіру».\footnote{
Там же, стор. 27. Загалом фізичний стан робітничої людности,
що підлягала фабричному законові, дуже поліпшився. Всі лікарські
свідчення погоджуються в цьому, і мої особисті спостереження різного
часу переконали мене цього. А все ж, навіть залишаючи осторонь неймовірно
високу смертність дітей у перших роках життя, офіціяльні
звіти д-ра Ґрінхова свідчать про несприятливий стан здоров’я у фабричних
округах порівняно з «рільничими округами з нормальним здоров'ям».
На доказ подаю, між іншим, оцю таблицю з його звіту від 1861 р.:

\newlength{\myheight}
\setlength{\myheight}{10em}

\noindent\begin{tabularx}{\textwidth}{Xccccc}
  \toprule
  \multicolumn{1}{c}{Назва округ} &
  \rotatebox[origin=c]{90}{\parbox[c]{\myheight}{Відсоток дорослих чоловіків, що працюють у мануфактурі}} &
  \rotatebox[origin=c]{90}{\parbox[c]{\myheight}{Смертність від нездужання на легені на кожні 100.000 чоловік.}} &
  \rotatebox[origin=c]{90}{\parbox[c]{\myheight}{Відсоток дорослих жінок, що працюють у мануфактурі}} &
  \rotatebox[origin=c]{90}{\parbox[c]{\myheight}{Смертність від нездужання на легені на кожні 100.000 жінок}} &
  Рід праці жінок \\

  \midrule
  Wigam\dotfill{}            & 14,9 & 598 & 18,0 & 644 & бавовна \\
  Blackburn\dotfill{}        & 42,6 & 708 & 34,9 & 734 & \ditto{бавовна} \\
  Halifax\dotfill{}          & 37,3 & 547 & 20,4 & 564 & вовна \\
  Bradford\dotfill{}         & 41,9 & 611 & 30,0 & 603 & \ditto{вовна} \\
  Macclesfield\dotfill{}     & 31,0 & 691 & 26,0 & 804 & шовк \\
  Leck\dotfill{}             & 14,9 & 588 & 17,9 & 705 & \ditto{шовк} \\
  Stoke-upon-Trent\dotfill{} & 36,6 & 721 & 19,3 & 665 & ганчарство \\
  Woolstanton\dotfill{}      & 30,4 & 726 & 13,9 & 727 & \ditto{ганчарство} \\
  \noindent\parbox[b]{\hsize}{Вісім здорових рільничих округ\dotfill{}}
                             &  --- & 305 &  --- & 340 & --- \\
\end{tabularx}

} Незважаючи на протести фабричних
інспекторів, які повторювалися щопівроку, це неподобство
триває й досі.\footnote{
Відомо, з яким опором англійські «фритредери» відмовили шовковій
мануфактурі у заведенні для неї охоронного мита. Замість оборони
проти французького довозу маємо тепер безоборонність англійських фабричних
дітей.
}
