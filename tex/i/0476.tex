Пересічне число веретен на одну фабрику

Англія.................12.600
Швайцарія............8.000
Австрія.................7.000
Саксонія...............4.500
Бельґія..................4.000
Франція................1.500
Прусія...................1.500

Пересічне число веретен на одну особу

Франція............................14
Росія.................................28
Прусія...............................37
Баварія..............................46
Австрія..............................49
Бельґія..............................50
Саксонія........................... 50
Дрібні німецькі держави...55
Швайцарія.........................55
Великобрітанія.................74

«Це порівняння, — каже пан Редґрев, — крім інших причин, ще й тому особливо несприятливе для
Великобрітанії, що в ній є дуже багато фабрик, де машинове ткання сполучене з прядінням, тимчасом як
у цьому обчисленні не виключено жодного з ткачів. Навпаки, закордонні фабрики здебільша лише
прядільні. Коли б ми могли точно порівнювати рівне з рівним, то я міг би налічити в моїй окрузі
багато пряділень бавовни, де за мюлями з 2.200 веретенами наглядають лише одним-один робітник
(minder) з двома помічницями, які щодня продукують 220 фунтів пряжі 400 (англійських) миль
завдовжки». («Reports of Insp. of Fact. 31 st October 1866», p. 31—37 і далі).

Відомо, що в Східній Европі, так само як і в Азії, англійські компанії взялися будувати залізниці, і
при цьому вони побіч тубільних робітників вживали й певне число англійських робітників. Примушені
практичною доконечністю брати таким чином
на увагу національні ріжниці в інтенсивності праці, вони від того не зазнали ніякої шкоди. Їхній
досвід навчає, що коли розмір заробітної плати й відповідає більше або менше пересічній
інтенсивності праці, то відносна ціна праці (у відношенні до продукту) взагалі рухається у
протилежному напрямі.

У своєму «Дослідженні про норму заробітної плати», 66 в одному із своїх найраніших економічних
творів, Г. Кері силку-

the day, is much lower in Scotland than in England... Labour by the piece
is generally cheaper in England»). (James Anderson: «Observations on
the means of exciting a spirit of National Industry etc.», Edinburgh 1777,
p. 350, 351). — Навпаки, низька заробітна плата з свого боку спричинюється
до подорожчання праці. «Праця дорожча в Ірляндії, ніж в Англії...
бо заробітна плата там відповідно нижча» («Labour being dearer in Ireland
than it is in England... because the wages are so much lower»).
(N. 2074 in «Royal Commission on Railways, Minutes. 1867»).

66 «Essay on the Rate of Wages: with an Examination of the Causes
of the Differences in the Conditions of the Labouring Population throughout
the World», Philadelphia 1835.
