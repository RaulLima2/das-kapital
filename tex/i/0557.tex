в світі створили бог і природа». 90 Коли венеціянський чернець
у вироку долі, що увіковічнював злидні, вбачав виправдання
існування християнської добродійности, безженства, манастирів
і побожних установ, то протестантський парох, навпаки, находить
у цьому привід до того, щоб осуджувати закони, які давали бідному
право на мізерну громадську допомогу. — «Проґрес суспільного
багатства, — каже Шторх, — породжує ту корисну клясу суспільства...
яка виконує найнудніші, найпаскудніші й найогидливіші
роботи, одне слово, звалює собі на плечі все, що тільки є в житті
неприємного і принизливого для людини, і саме цим дає іншим
клясам вільний час, веселість духу й умовну (c’est bon! *) гідність
характеру й т. д».91 Шторх питає себе, в чому ж, власне,
перевага цієї капіталістичної цивілізації з її злиднями й деґрадацією
мас над варварством. Він находить лише одну відповідь:
у безпеці! — «Через проґрес промисловости й науки, каже Сісмонді,
кожний робітник може продукувати щодня куди більше,
аніж йому треба для власного споживання. Але тим часом, як
його праця продукує багатство, багатство зробило б його малопридатним
до праці, коли б він мав сам його споживати». На його
думку, «люди [тобто нероби], певно, відмовилися б від усяких
удосконалень на полі мистецтва й від усяких приємностей, які
нам дає промисловість, коли б вони все це мусили купувати
такою впертою працею, як праця робітника... За наших часів
напруження відокремлено від нагороди за нього; не та сама людина,
що спочатку працює, потім відпочиває: навпаки, саме
тому, що одна людина працює, інша мусить спочивати... Отже,
безмежне помноження продуктивних сил праці не може мати
ніякого іншого результату, як тільки збільшення розкошів і
насолод багатих нероб».92 [Шербюльє, учень Сісмонді, доповнює
його, кажучи: «Сами робітники... сприяючи своєю працею акумулювати
продуктивні капітали, допомагають тому результатові,
що рано або пізно відбирає в них частину їхньої заробітної

90 «А Dissertation on the Poor Laws. By a Wellwisher of Mankind
(The Rev. Mr. J. Townsend) 1786», republished London 1817, p. 15, 39,
41. Цей «делікатний» піп, що з його щойно наведеного твору, а також із
його опису подорожі по Еспанії Малтуз пасто списує цілі сторінки, сам
запозичив більшу частину своєї доктрини в сера Дж. Стюарта, якого він
однак перекручує. Наприклад, коли Стюарт каже: «Тут, за системи рабства,
існувала насильницька метода робити людство працьовитим (для
нероб). Тоді примушували людей до праці (тобто задурно працювати на
інших), бо вони були рабами інших; тепер людей примушують до праці
(тобто задурно працювати для нероб), бо вони раби своїх власних потреб»,
то з цього він не робить, як той товстий парох, висновку, що наймані
робітники повинні завжди голодувати. Навпаки, Стюарт хоче збільшити
їхні потреби і зробити для них щораз більше число їхніх потреб разом
із тим стимулом їхньої праці на «делікатніших».

91    Storch: «Cours d’Economie Politique», éd. Pétersbourg 1815, vol. III,
p. 223.

92    Sismondi: «Nouveaux Principes d’Economie Politique», I, p. 78,
79, 81.

* — чудесно! Ред.
