\parcont{}  %% абзац починається на попередній сторінці 
\index{i}{0154}  %% посилання на сторінку оригінального видання 
ливарню, щоб на другий день знову повернутися відти в формі
масивного заліза на фабрику.

Засоби продукції переносять вартість на нову форму продукту
лише остільки, оскільки вони протягом процесу праці втрачають
вартість у формі їхніх старих споживних вартостей. Максимум
втрати вартости, що її вони можуть зазнати в процесі праці, очевидно,
обмежується на первісній величині вартости, з якою вони
входять у процес праці, або на робочому часі, потрібному для
їхньої власної продукції. Тому засоби продукції ніколи не можуть
додати до продукту більше вартости, ніж вони мають, незалежно
від процесу праці, що його вони обслуговують. Хоч би й який
корисний був якийсь матеріял, якась машина, якийсь засіб продукції,
але все ж, коли він коштує 150 фунтів стерлінґів, приміром,
500 робочих днів, то він ніколи не додає більше, ніж 150 фунтів
стерлінґів до цілого продукту, що його утворення він обслуговує.
Вартість його визначається не тим процесом праці, в який
він увіходить як засіб продукції, а тим процесом праці, з якого
він виходить як продукт. У процесі праці він служить лише за
споживну вартість, за річ з корисними властивостями, а тому він
не віддавав би продуктові жодної вартости, коли б не мав вартости
раніше, ніж увійшов до процесу.\footnote{
Тому легко зрозуміти безглуздість банального Ж. Б. Сея, який додаткову
вартість (процент, зиск, ренту) хоче вивести з «services productifs»
(продуктивних послуг), що їх засоби продукції: земля, інструмент
шкура й т. ін., роблять через свої споживні вартості в процесі праці.
Пан Вільгельм Рошер, який ніколи не проминає нагоду чорним на білому
зареєструвати хитромудрі апологетичні вигадки, вигукує: Ж. Б. Сей
у «Traité», vol. І, ch. 4 дуже слушно зауважує: «Утворена олійнею вартість
після відлічення всіх витрат — це ж щось нове, по суті відмінне від
пралі, якою була створена сама олійня» («Die Grundlagen der Nationalökonomie»,
3. Auflage, 1858, S. 82, примітка). Цілком слушно! «Олія»,
виготовлена олійнею — це щось дуже відмінне від праці, якої коштувала
будова олійні. А під «вартістю» пан Рошер розуміє таку річ, як «олія»,
бо «олія» має вартість; але що «в природі» є кам’яна олія (нафта), хоч
порівняно її й не «дуже багато», то він, певне, з цього приводу робить
другу заувагу: «Вона (природа!) майже зовсім не витворює мінових
вартостей». З природою Рошера щодо мінової вартости трапилося те саме,
що тій дурненькій дівчині з дитиною, яка лише «була цілком маленька».
Той самий «учений» (savant sérieux) зауважує ще з зазначеного вище
приводу: «Школа Рікарда звичайно підводить під поняття праці й капітал
як «заощаджену працю». Це незручно (!), бо (!) аджеж (!) власник капіталу
(!) все ж (!) зробив щось більш (!), ніж просте (?!) витворення (?)
і (??) зберігання його (чого?): а саме (?!?), він здержувався від власної
втіхи, за що він, приміром (!!!), вимагає процентів» (там же). Яка «зручна»
(!) ця «анатомічно-фізіологічна метода» політичної економії, що з
простого «вимагання» розвиває навіть «вартість».
}

В той час, коли продуктивна праця перетворює засоби продукції
на елементи творення нового продукту, їхня вартість зазнає
щось подібне до переселення душ. Із спожитого тіла вартість
переходить у новозформоване тіло. Але це переселення душ відбувається
немов поза спиною дійсної праці. Робітник не може
додати нової праці, отже, не може створити нової вартости, не
зберігаючи старої вартости, бо він мусить додавати працю завжди
\parbreak{}  %% абзац продовжується на наступній сторінці
