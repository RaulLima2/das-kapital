\parcont{}  %% абзац починається на попередній сторінці
\index{i}{0194}  %% посилання на сторінку оригінального видання
Waterford’i й т. ін. «В Limerick’y, де, як відомо, страждання
найманих підмайстрів переходять усяку міру, цей рух розбився
об опір пекарів-хазяїнів, особливо ж пекарів-мірошників. Приклад
Limerick’a призвів до назаднього руху в Eniss’i та Tripperary.
В Cork’y, де громадське обурення виявилось у найжвавішій
формі, хазяї розбили рух, використавши свою силу викинути підмайстрів
на вулицю. В Дебліні хазяї виявили якнайрішучіший
опір і, переслідуючи тих підмайстрів, що стояли на чолі аґітації,
примусили решту поступитись і згодитись на нічну та недільну
працю».\footnote{
«Report of Committee on the Baking Trade in Ireland for 1861».
85 Там же.
86 Публічний мітинґ рільничих робітників у Lasswade біля Glasgow
5 січня 1866 р. (див. «Workman’s Advocate» з 13 січня I860 р.). — Утворення
наприкінці 1865 р. тред-юньойону рільничих робітників, передучім
у Шотландії, є історична подія. В одній з найпригнобленіших рільничих
округ Англії, в Buckingamshire, наймані робітники влаштували
в березні 1867 р. величезний страйк з метою вибороти підвищення тижневої
заробітної плати з 9--10 шилінґів до 12 шилінґів. (Із попереднього
видно, що рух англійського рільничого пролетаріяту, геть чисто зламаний
від часів придушення його енерґійних демонстрацій після року 1830
і особливо після заведення нового закону про бідних, знову починається
в шістдесятих роках, доки нарешті в році 1782 стає епохальним. У ІІ томі
я повертаюсь до цього питання, а так само й до Синіх Книг про становище
англійського рільничого робітника, що з’явилися після 1867 р. — Додаток
до третього видання).
} Комісія англійського уряду, озброєного в Ірландії
з ніг до голови, жалібно нарікає, немов голосільниця, на невблаганних
пекарів-хазяїнів Дебліну, Корку й т. ін. «Комітет гадає,
що робочий час обмежено природними законами, яких не можна
порушувати безкарно. Тримаючи своїх робітників під загрозою
звільнення, хазяїни примушують їх порушувати їхні релігійні
переконання, не слухатися законів країни і зневажати громадську
думку (все це останнє стосується до недільної праці), вони сіють
ворожнечу між капіталом і працею та дають приклад, небезпечний
для релігії, моральности й суспільного ладу\dots{} Комітет гадає,
що здовження робочого дня понад 12 годин є узурпаторське втручання
у родинне й приватне життя робітника, і через встрявання
в родинний побут людини і у виконання нею своїх родинних обов’язків
як сина, брата, чоловіка й батька воно призводить до лихих
моральних результатів. Праця понад 12 годин має тенденцію підточувати
здоров’я робітника, викликає передчасну старість і
ранню смерть, а тому й нещастя робітничих родин, позбавляючи
(«are deprived») їх опіки й підпори голови родини саме в такий
час, коли це їм якнайпотрібніше».85

Ми тільки що побували в Ірландії. По тому боці каналу, в
Шотляндії, рільничий робітник, робітник плуга, нарікає на свою
13--14-годинну працю в найсуворішому кліматі, з чотиригодинною
додатковою працею в неділю (це в країні святкувальників
суботи!),86 одночасно з цим перед лондонським Grand Jury стало
троє залізничників: пасажирний кондуктор, машиніст і сиґнальник.
Велика залізнична катастрофа відправила сотні пасажирів
\parbreak{}  %% абзац продовжується на наступній сторінці
