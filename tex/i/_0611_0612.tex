\parcont{}  %% абзац починається на попередній сторінці
\index{i}{0611}  %% посилання на сторінку оригінального видання
на своїх сеньйорів за заробітну плату взагалі нижчу від заробітної
плати звичайних поденників, не кажучи вже про ті невигоди
і втрати, які постають для них у наслідок того, що в критичну
пору сівби або жнив вони мусять занедбувати свої власні поля».187і

Отже, незабезпеченість та іреґулярність заняття, часті й довготривалі
перерви у роботі — всі ці симптоми відносного перелюднення
фігурують у звітах інспекторів адміністрації в справах
про бідних, як так само численні тяготи ірляндського рільничого
пролетаріату. Ми пригадуємо собі, що подібні явища ми бачили
й серед англійського сільського пролетаріату. Але ріжниця
в тому, що в Англії, промисловій країні, промислову резервну
армію рекрутують на селі, тимчасом як в Ірляндії, у рільничій
країні, рільничу резервну армію рекрутують у містах, притулках
вигнаних сільських робітників. В Англії зайві сільські робітники
перетворюються на фабричних, в Ірляндії ж, загнані в міста,
вони, справляючи тиск на заробітну плату по містах, все ж залишаються
сільськими робітниками і примушені завжди вертатися
назад у села, щоб знайти собі роботу.

Автори офіціяльних звітів резюмують свої висновки про
матеріяльний стан рільничих поденників так: «Хоч вони живуть
надзвичайно ощадно, проте їхньої заробітної плати ледве вистачає
на те, щоб здобути харчі для себе й своєї родини і заплатити
за своє житло; на одяг вони потребують додаткових доходів...
Атмосфера їхніх мешкань разом з іншими злиднями робить цю
клясу часто здобиччю тифу і сухот».\footnoteA{
Там же, стор. 21, 13.
} Після цього немає чого
дивуватися, що, за одноголосним свідченням авторів звітів,
хмура незадоволеність охоплює ряди цієї кляси, що вона сумує
за минулим, ненавидить теперішнє, зневірюється в будучині,
«піддається згубним впливам демагогів» і охоплена лише однією
idée fixe — еміґрувати до Америки. Така та блаженна країна,
на яку збезлюднення, велика малтузіянська панацея, перетворило
зелений Ерін!

Щоб побачити, як благоденствують ірляндські мануфактурні
робітники, досить одного прикладу:

«Під час моєї останньої інспекторської подорожі на півночі
Ірляндії, — каже англійський фабричний інспектор Роберт Бекер,
— мене вразило, як один вправний ірляндський робітник
силкувався із своїх злиденних коштів дати освіту своїм дітям.
Я точно передаю його оповідання, як я чув його з власних його
уст. Що він справді вправний фабричний робітник, видно з того,
що його вживали до виробу товарів на менчестерський ринок.
Джонсон: З професії я beefier, і працюю від 6 години ранку до
11 години вночі, від понеділка до п’ятниці; суботами ми кінчаємо
о 6 годині вечора і маємо 3 години на обід і відпочинок. В мене
п’ятеро дітей. За цю працю я дістаю 10 шилинґів 6 пенсів на тиждень;
моя дружина теж працює і заробляє на тиждень 5 шилін-

187i Там же, стор. 30.
\index{i}{0612}  %% посилання на сторінку оригінального видання
ґів. Старша дочка, дванадцяти років, доглядає хати. Вона наша
куховарка й однісінька помічниця. Вона підготовляє менших
до школи. Моя дружина встає разом зі мною і йде теж зі мною.
Одна дівчина, що проходить повз нашої хати, будить мене о 5 1/2
годині ранку. Перед тим, як іти на роботу, ми нічого не їмо.
Вдень дванадцятилітня дочка доглядає менших дітей. Снідаємо
о 8 годині і для цього приходимо додому. Чай п’ємо раз на тиждень;
звичайно ми їмо юшку (stirabout), іноді з вівсяного борошна,
іноді з кукурудзяного, залежно від того, що можемо дістати.
Взимку до кукурудзяного борошна додаємо трохи цукру й води.
Влітку копаємо потроху картоплю, що сами садимо на клаптику
землі, а коли картопля кінчається, знову вертаємося до юшки.
Так іде з дня на день, у неділю і будні, цілий рік. Скінчивши
роботу, я ввечорі завжди почуваю надзвичайну втому. Трошки
м’яса нам винятково доводиться бачити, але дуже рідко. Троє
з наших дітей ходять до школи, і за те ми платимо за кожне по
1 пенсу на тиждень. Наша квартирна плата становить на тиждень
9 пенсів, торф і опалення коштують щонайменше 1 шилінґ 6 пенсів
на два тижні».\footnote{
«Reports of Insp. of Fact, for 31st October 1866», p. 96.
} Така ірляндська заробітна плата, таке
ірляндське життя!

Справді, злидні Ірляндії знову стали в Англії темою дня.
Наприкінці 1866 й на початку 1867 р. один з ірляндських земельних
маґнатів, лорд Дюфрен, взявся на сторінках «Times’a» за
розв’язання цього питання. «Яка гуманність з боку такого великого
пана!»

Із таблиці Е видно, що в 1864 р. із загального зиску в 4.368.610
фунтів стерлінґів троє тільки капіталістів (Plusmacher) поклали до
своєї кишені 262.610 фунтів стерлінґів, а в 1865 р. тих самих троє
віртуозів «поздержливости» з 4.669.979 фунтів стерлінґів загального
зиску дістали 274.448 фунтів стерлінґів; в 1864 р. 26 капіталістів
дістали 646.377 фунтів стерлінґів; в 1865 р. 28 капіталістів —
736.448 фунтів стерлінґів; в 1864 р. 121 капіталіст — 1.066.912
фунтів стерлінґів в 1865 р. 186 капіталістів — 1.320.996 фунтів
стерлінґів; в 1864 р. 1.131 капіталіст — 2.150.818 фунтів стерлінґів,
майже половину загального річного зиску; в 1865 р.
1.194 капіталісти дістали 2.418.933 фунти стерлінґів — більше, ніж
половину загального річного зиску. Але та левина частина,
яку проковтує з усієї річної суми земельних рент зовсім мале
число земельних маґнатів Англії, Шотляндії та Ірляндії, така
потворно велика, що англійська державна мудрість вважає за
доцільне не давати про розподіл земельної ренти такого самого
статистичного матеріялу, як про розподіл зиску. Лорд Дюфрен
— один із цих земельних маґнатів. Гадати, що ренти й зиски
колибудь можуть бути «надмірні», або що плетора (plethora)\footnote*{
— повнява. Ред.
}
ренти і зисків є в якомусь зв’язку з плеторою народніх злиднів,
це, звичайно, є уявлення так само «непочтиве», як і «нездорове»
\parbreak{}  %% абзац продовжується на наступній сторінці
