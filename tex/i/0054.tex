яке, так би мовити, існує тільки в їхніх головах. Тим-то посідач
товарів мусить позичити їм свій язик або понавішувати на них
ярлички, щоб повідомити зовнішній світ про їхні ціни.\footnote{
Дикун і напівдикун вживає при цьому свого язика інакше. Капітан
Перрі зауважує, приміром, про жителів західнього узбережжя затоки
Бафіна ось що: «В цьому випадку (при обміні продуктів) вони лижуть
її (подану їм річ) двічі язиком, після чого, здасться, вважають торг за задовільно
закінчений» («In this case.... they licked it (the thing represented
to them) twice to their tongues, after which they seemed to consider
the bargain satisfactorily concluded»). У східніх ескімосів обмінювач
також кожний раз облизував одержувану при обміні річ. Коли на півночі
язик є таким чином за орган присвоєння, то нічого дивного, що на
півдні живіт вважається за орган нагромадженої власности; так, кафр
цінує багатство людини з її опасистости. Дійсно, кафри дуже розумні
люди, бо тимчасом як офіційний англійський санітарний звіт з 1864 р.
скаржиться на недостачу жиротворних субстанцій у більшої частини робітничої
кляси, лікар Гарвей, драма що він і не відкрив кровобігу, того
самого року зробив собі кар’єру шахрайськими рецептами, що обіцяли
буржуазії й аристократії увільнити її від тягару надмірного жиру.
} А що
вираз товарових вартостей у грошах є ідеальний, то для цієї
операції можна вживати також тільки уявлюваного, або ідеального
золота. Кожний товаропосідач знає, що, надавши вартості
своїх товарів форми ціни або форми уявлюваного золота, він
ще далеко не перетворив на золото свої товари, і що йому не
треба ані крихітки реального золота, щоб оцінити мільйони
товарових вартостей у золоті. Отже, в їхній функції міри вартостей
гроші служать тільки як уявлювані, або ідеальні гроші.
Ця обставина породила якнайбезглуздіші теорії.\footnote{
Див. К. Marx: «Zur Kritik der Politischen Oekonomie». — «Theorien
von der Messeinheit des Geldes», S. 53 ff. (K. Маркс: «До критики
політичної економії». — «Теорії про одиницю міри грошей», ДВУ, 1926 р.,
стор. 91 і далі).
} Хоч функцію
міри вартостей виконують лише уявлювані гроші, все ж ціни
цілком залежать від реального грошового матеріялу. Вартість,
тобто кількість людської праці, що міститься, наприклад, в
одній тонні заліза, виражається в уявлюваній кількості грошового
товару, яка містить у собі рівно стільки ж праці. Отже,
залежно від того, чи золото, срібло або мідь служать за міру
вартости, вартість тонни заліза набирає цілком різних виразів
ціни, або репрезентується в цілком різних кількостях золота,
срібла або міді.

Тому, коли два різні товари, приміром, золото й срібло,
служать одночасно за міру вартости, то всі товари мають два
різні вирази для своїх цін — золоті ціни й срібні ціни, які спокійнісінько
існують одні побіч одних, доки вартостеве відношення
між золотом й сріблом лишається незмінне, приміром,
1: 15. Але всяка зміна цього вартостевого відношення порушує
відношення між золотими й срібними цінами товарів, і таким
чином доводить фактично, що подвійність міри вартости суперечить
її функції.\footnote{
Примітка до другого видання. «Там, де золото й срібло на підставі
закону функціонують одне поряд одного як гроші, тобто як міра вартости,
}