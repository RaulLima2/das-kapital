Передмова до третього видання

Марксові не судилося самому виготовити до друку це третє
видання. Могутній мислитель, що перед величністю його схиляють
голови тепер навіть його супротивники, помер 14 березня 1883 р.

На мене, що в особі Маркса втратив людину, яка сорок років
була моїм найкращим й найнезламнішим другом, другом, якому
я завдячую більше, ніж це можна сказати словами, — спав
тепер обов’язок подбати так за це третє видання першого тому,
як і за другий том, що його він залишив у рукопису. Як я виконав
першу частину цього обов’язку, — про це я повинен дати тут
читачеві звіт.

Спочатку Маркс мав на думці переробити більшу частину
тексту першого тому, гостріше зформулювати деякі теоретичні
пункти, додати нові пункти, доповнити історичний і статистичний
матеріял аж до найновішого часу. Його хворість і бажання
закінчити редакцію другого тому примусили його зректися цього.
Малося змінити тільки щонайпотрібніше, додати тільки те, що
вже містилося у французькому виданні («Le Capital», par Karl
Marx. Paris, Lachâtre 1873), яке за той час вийшло.

Серед спадщини Маркса знайшовся також і німецький примірник
«Капіталу», подекуди виправлений ним та з посиланнями
на французьке видання; знайшлось і французьке видання з
відзначеними докладно місцями, якими він хотів був скористуватися.
Ці зміни й доповнення обмежуються, за небагатьма винятками,
останньою частиною книги, відділом: «Процес акумуляції
капіталу». Текст цього відділу більш, ніж інші, збігається з первісним
нарисом, тимчасом як попередні відділи були вже ґрунтовно
перероблені. Тому стиль був тут жвавіший, одностайніший,
але й недбайливіший, траплялись англіцизми, місцями неясності;
розвиток думки виявляв подекуди прогалини, бо окремі важливі
моменти були лише намічені.

Щодо стилю, то Маркс сам ґрунтовно перевірив багато підвідділів,
і цим, як і численними усними вказівками, зазначив мені
міру, як далеко мені можна піти, усуваючи англійські технічні
вислови й інші англіцизми. Додатки й доповнення Маркс, безперечно,
був би ще переробив і гладеньку французьку мову був би
замінив своєю власною, стислою німецькою; я мусив задовольнитися
тим, що переніс їх на відповідні місця, додивляючися, щоб
вони по змозі пасували до первісного тексту.

Отже, в цьому третьому виданні не змінено жодного слова,
про яке напевно не знав би я, що автор сам би був його змінив.
Мені не могло спасти на думку заводити до «Капіталу» загальновживаний
жарґон, яким звичайно висловлюються німецькі еко-
