дуктивної сили їхньої праці, з поширенням і збільшенням усіх
джерел багатства поширюється й маштаб, в якому більше притягування
робітників капіталом зв’язане з більшим їх відштовхуванням,
зростає швидкість зміни органічного складу капіталу
та його технічної форми й ширшає коло тих сфер продукції, що
їх то одночасно, то навпереміну охоплює ця зміна. Отже, робітнича
людність, разом з продукованою нею самою акумуляцією
капіталу, продукує в щораз більшому розмірі засоби, які роблять
саму її відносно зайвою.79 Це є властивий капіталістичному
способові продукції закон населення, як і кожному осібному
історичному способові продукції в дійсності властиві свої осібні
закони населення, що мають історичне значення. Абстрактний
закон населення існує тільки для рослин і тварин, і то лише
остільки, оскільки вони не зазнають історичного впливу людини.

Але якщо надмірна робітнича людність є доконечний продукт
акумуляції, або розвитку багатства на капіталістичній основі,
то це перелюднення, з свого боку, стає підоймою капіталістичної
акумуляції і навіть умовою існування капіталістичного способу
продукції. Воно утворює резервну промислову армію, якою капітал
може порядкувати і яка абсолютно належить капіталові,
так, наче б він виростив її своїм власним коштом. Воно створює
для змінних потреб самозростання капіталу завжди готовий,
приступний для експлуатації людський матеріял, незалежно
від меж дійсного приросту людности. З акумуляцією й розвитком
продуктивної сили праці, що супроводить акумуляцію,
зростає сила раптового поширення капіталу не тільки через те,

79    Деякі видатні економісти клясичної школи більше передчували,
ніж розуміли закон прогресивного зменшення відносної величини змінного
капіталу і його вплив на становище кляси найманих робітників.
Найбільша заслуга в цій справі належить Джонові Бартону, хоч і він, як
усі інші, сплутує сталий капітал з основним, а змінний з обіговим. Вік
каже: «Попит на працю залежить від зростання обігового капіталу, а
не основного. Коли б це була правда, що відношення поміж цими двома
відмінами капіталу за всяких часів і серед усяких обставин однакове,
то з цього випливало б, що число занятих робітників є пропорційне до
багатства держави. Але таке припущення не має й тіні ймовірности. Що
більше розвиваються промисли й поширюється цивілізація, то більше
й більше основний капітал переважає обіговий. Сума основного капіталу,
вживаного для продукції однієї штуки англійського мусліну, щонайменше
всотеро, а може і в тисячу разів більша, ніж основний капітал, що його
вживають на продукцію такої самої штуки індійського мусліну. А обіговий
капітал відносно в сто або тисячу разів менший... Коли б усю суму річних
заощаджень додавано до основного капіталу, то це все ж не спричинило
б жодного впливу на зростання попиту на працю». («The demand for
labout depends on the increase of circulating and not of fixed capital. Were
it true that the proportion between these two sorts of capital is the sameat
all times, and in all circumstances, then, indeed, it follows that the number
of labourers employed is in proportion to the wealth of the state.
But such a proposition has not the semblance of probability. As arts are
cultivated, and civilization is extended, fixed capital bears a larger and
larger proportion to circulating capital. The amount of fixed capital employed
in the production of a piece of British muslin is at least a hundred, probably
a thousand times greater than that employed in a similar piece of
Indian muslin. And the proportion of circulating capital is a hundred or
