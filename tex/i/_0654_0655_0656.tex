\parcont{}  %% абзац починається на попередній сторінці 
\index{i}{0654}  %% посилання на сторінку оригінального видання 
докупи осіб обох статей різного віку й різних нахилів, що заразливість прикладу мусить привести до
зіпсованости й розпусти, — чи може така мануфактура збільшити суму національного й індивідуального
щастя?».\footnote{
Eden: «The State of the Poor», v. II, ch. I, p. 421.
} «В Дербішірі, Нотінґемшірі й особливо
Ланкашірі, — каже Фелден, — ужито недавно винайдені машини на великих фабриках, побудованих над
річками, що могли пускати в рух водяне колесо. Одразу постала потреба в тисячах робочих рук у цих
місцевостях, віддалених від міст; і особливо
Ланкашір, до того часу порівняно мало залюднений і неродючий, потребував тепер перш за все людности.
Потрібні були насамперед маленькі й меткі дитячі руки. Одразу ж повівся звичай набирати учнів (!) із
лондонських, бірмінґемських та інших парафіяльних робітних домів. Таким чином багато-багато тисяч
цих маленьких безпорадних істот від 7 до 13 або 14 років життя вивезено на північ. У хазяїна (тобто
крадія дітей) повівся звичай одягати, годувати й приміщувати своїх учнів у «будинку для учнів»
близько фабрик. Він наймав доглядачів, що мали стежити за їхньою працею. В інтересах цих доглядачів
за рабами було примушувати дітей працювати понад усяку міру, бо їхня плата залежала від кількости
продукту, що її можна було видушити
з дітей. Природний наслідок цього була жорстокість... По багатьох фабричних округах, особливо ж у
Ланкашірі, цих невинних і беззахисних істот, відданих на волю фабрикантів, катували з надзвичайною
жорстокістю. Їх замордовували до смерти надмірною працею... їх били батогами, заковували в кайдани й
катували з найвишуканішою витонченістю й жорстокістю; у багатьох випадках їх виснажували голодом до
шкури-кости, і все ж батогом примушували до праці... В деяких випадках їх доводили навіть до
самогубства!.. Чудові й романтичні долини Дербішіру, Нотінґемшіру та Ланкашіру, заховані від
громадського ока, стали жахливим місцем катувань і — часто вбивства!... Зиски фабрикантів були
величезні. Це лише розпалювало їхню вовчу ненажерливість. Вони почали заводити нічну працю, тобто
тримали напоготові для нічної праці групу робітників, що заступала другу групу робітників,
знесилених денною працею; денна йшла до ліжок, які тільки но покинула нічна група, і навпаки. У
Ланкашірі є народній переказ, що ці ліжка ніколи не простигали».\footnote{
John Fielden: «The Curse of the Factory System», London 1836, p. 5, 6. Про всі ті гидоти, які
творилися від початку фабричної системи порівн. Dr. Aikin: «Description of the Country from thirty
to forty miles round Manchester», London 1795, p. 129 та Gisborne: «Enquiry into the duties of men»,
1795, vol II. — Через те, що парова машина перенесла
фабрики від сільських водоспадів до центру міст, то «прихильний до поздержливости» капіталіст
(Plusmacher) находив дитячий матеріял під рукою, так що не треба було насильно транспортувати рабів
із робітних домів. — Коли сер Роберт Піл (батько «міністра уважливости») запропонував у 1815 р. біл
для охорони дітей, Ф. Горнер (світило «Bul-
}

\index{i}{0655}  %% посилання на сторінку оригінального видання 
З розвитком капіталістичної продукції протягом мануфактурного
періоду громадська думка Європи позбулася останніх решток
сорому й сумління. Нації цинічно пишались кожною підлотою,
що була засобом для акумуляції капіталу. Прочитайте, приміром,
наївні торговельні анали, складені щирим А. Андерсоном.

Тут, як тріумф англійської державної мудрости, розхвалюється
той факт, що Англія за Утрехтським миром на основі
угоди асієнто\footnote*{
— угода щодо торговлі рабами. Ред.
} вимусила від Еспанії привілей, що давав їй
право на торговлю неграми, яку вона досі вела лише між Африкою
й англійською Західньою Індією, вести й між Африкою та
еспанською Америкою. Англія здобула право аж до 1743 р. постачати
еспанській Америці щорічно 4.800 негрів. Це забезпечувало
їй разом з тим офіціяльне прикриття для контрабанди.
Ліверпул виріс як велике місто на ґрунті торговлі рабами.
Вона становить його методу первісної акумуляції. І ще й по сей
день «поважні» ліверпулські громадяни лишилися Піндарами
работорговлі, яка — порівн. вже цитований вище твір д-ра Ейкіна
з 1795 р. — «підносить дух комерційної підприємливости аж до
пристрасти, створює славних моряків і приносить колосальні
гроші». В 1730 р. в Ліверпулі коло торговлі рабами працювало
15 кораблів, в 1751р. — 53 кораблі, в 1760 р. — 74, в 1770 р. —
96 і в 1792 р. — 132 кораблі.

Бавовняна промисловість, завівши в Англії рабство дітей,
дала разом з тим поштовх до перетворення рабовласницького
господарства Сполучених штатів, доти більш або менш патріярхального,
на комерційну систему експлуатації. Взагалі приховане
рабство найманих робітників в Европі потребувало, як основи,
рабства sans phrase\footnote*{
— попросту. Ред.
} у Новому Світі.\footnote{
«В 1790 р. в англійській Західній Індії 10 рабів припадало на
одну вільну людину, у французькій — 14 на одну, в голляндській — 23
на одну». (Henry Barougham: «An Inquiry into the Colonial Policy of
the European Powers», Edinburgh 1803, vol. II, p. 74).
}

Tantae molis erat\footnote*{
Стільки праці коштувало. Ред.
} розв’язати «вічні природні закони»
капіталістичного способу продукції, вивершити процес відо-

lion Komitees\footnote*{
— комітет у справах зливків. Ред.
} та інтимний приятель Рікарда) заявив у палаті громад:
«Загальновідомий факт, що разом з цінними речами одного банкрута
призначено на продаж і продано з авкціона, як частину його майна, банду
фабричних дітей, коли можна вжити такого слова. Два роки тому (року
1813) перед King’s Bench\footnote*{
— найвищим судом. Ред.
} розглядувано огидний випадок. Ішлося про
групу хлопчиків. Одна лондонська парафія віддала їх якомусь фабрикантові,
що від себе знову передав їх якомусь іншому. Нарешті, декілька філантропів
знайшло їх у стані повного виснаження від голоду («absolute
famine»). З другим випадком, ще огиднішим, його познайомили, як члена
парламентської слідчої комісії. Декілька років тому одна лондонська
парафія склала контракт з одним ланкашірським фабрикантом, за яким
він зобов’язувався на двадцять здорових купованих ним дітей приймати
одного ідіота».
\index{i}{0656}  %% посилання на сторінку оригінального видання 
кремлення робітників від умов праці, перетворити на одному
полюсі суспільні засоби продукції та засоби існування на капітал,
а на протилежному — народню масу на найманих робітників,
на вільних «працюючих бідняків», — цей витвір мистецтва
сучасної історії.\footnote{
Вислів «labouring poor»\footnote*{
— працюючі бідняки. Peд.
} подибуємо в англійських законах від
часу, коли кляса найманих робітників стає помітна. Labouring poor протистоять,
з одного боку, «idle poor»,\footnote*{
— біднякам-неробам. Peд.
} жебракам і т. ін., з другого боку,
тим робітникам, що не є ще цілком обскубані кури, а є ще власники
своїх засобів праці. Із законів вислів «labouring poor» перейшов
до політичної економії, починаючи від Колпепера, Дж. Чайлда й ін.,
аж до А. Сміса й Ідна. По цьому можна судити, яка bonne foi\footnote*{
— сумлінність. Peд.
} в Едмунда
Берка, цього «execrable political cantononger»\footnote*{
— огидливого політичного крамаря. Peд.
}, коли він вислів
«labouring poor» зве «execrable political cant».\footnote*{
— огидливим політичним перекрученням. Ред.
} Цей сикофант, що,
бувши на утриманні англійської олігархії, відігравав ролю романтика проти
французької революції, так само, як на початку заворушень в Америці
він, бувши на утриманні північно-американських колоній, відігравав
ролю ліберала проти англійської олігархії, в дійсності був наскрізь ординарним
буржуа: «Закони торговлі є закони природи, отже, і закони самого
бога» (Е. Burke: «Thoughts and Details on Scarcity», ed. London
1800, p. 31,32). He диво, що він, вірний законам бога й природи, завжди
продавав себе самого на найкращому ринку! У творах панотця Тукера —
Тукер був піп і торі, але зрештою цілком пристойна людина й путящий
політико-економ — можна знайти дуже гарну характеристику цього
Едмунда Берка за його ліберальних часів. При тій огидливій безхарактерності,
яка панує тепер і побожно вірить у «закони торговлі», треба
знову й знов таврувати таких Берків, що від своїх наступників відрізняються
лише одним — талантом!
} Коли гроші, як каже Ож’є, «родяться на
світ із природними кривавими плямами лише на одній щоці»,\footnote{
Marie Augier: «Du Crédit Public», Paris 1842, p. 265.
}
то капітал, що родиться на світ, прискає кров’ю й брудом від
голови до ніг із усіх своїх пор.\footnote{
«Капітал, — каже «Quarterly Reviewer», — уникає заколотів
і сварок і з природи своєї боязкий. Це цілковита правда, алеж не вся
правда. Капітал боїться відсутности зиску або дуже малого зиску, як
природа боїться порожнечі. При відповідному зиску капітал стає відважним.
При певних 10 процентах його можна вживати повсюди; при 20 процентах
він стає жвавим; при 50 процентах він абсолютно готовий ризикувати;
за 100 процентів він топче ногами всі людські закони; 300 процентів
— і немає такого злочину, що на нього він не ризикнув би, навіть
під загрозою шибениці. Коли заколоти й сварки дають зиск, він заохочує
і до заколотів і до сварок. Докази — контрабанда і торговля рабами».
(Т. J. Dunning: «Trades-Unions and Strikes», London 1860, p. 36).
}

7. Історична тенденція капіталістичної акумуляції

На що ж сходить первісна акумуляція капіталу, тобто його
історична генеза? Оскільки вона не є безпосереднє перетворення
рабів і кріпаків на найманих робітників, отже, не є проста
зміна форми, вона означає лише експропріяцію безпосередніх
продуцентів, тобто розклад приватної власности, основаної на
власній праці.
