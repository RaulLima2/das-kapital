не вистачило б для провадження національної продукції в її
теперішньому маштабі. Велика більшість «непродуктивних»
тепер робітників мусила б перетворитись на «продуктивних».

Взагалі і в цілому загальні коливання заробітної плати реґулюються
виключно поширенням і зменшенням промислової резервної
армії, що відповідають зміні періодів промислового циклу.
Отже, вони визначаються не рухом абсолютного числа робітничої
людности, а тим змінним відношенням, що в ньому робітнича кляса
розпадається на активну й резервну армію, тобто збільшенням
та зменшенням відносних розмірів перелюднення, ступенем, у
якому перелюднення то поглинається, то знову звільняється.
Для сучасної промисловости з її десятилітнім циклом і його періодичними
фазами, які, крім того, з проґресом акумуляції перериваються
неправильними коливаннями, що чимраз швидше йдуть
одне по одному, — це справді був би прегарний закон, що робив
би рух капіталу залежним від абсолютного руху маси людности
замість, навпаки, реґулювати попит і подання праці поширенням
і скороченням капіталу, тобто відповідно до його кожноразових
потреб самозростання, отже, реґулювати таким чином,
що ринок праці видається то відносно неповним, у наслідок
поширення капіталу, то знову переповненим, у наслідок скорочення
капіталу. Однак така є догма політичної економії. За цією
догмою в наслідок акумуляції капіталу зростає заробітна плата.
Підвищена заробітна плата стимулює швидше розмноження робітничої
людности, і це розмноження триває доти, доки ринок
праці переповнюється, отже, триває доти, доки капітал стане
відносно недостатнім проти подання праці. Заробітна плата падає,
і тепер ми маємо зворотний бік медалі. В наслідок падання
заробітної плати робітнича людність поволі рідшає, так що проти
неї капітал знову стає надмірний, абож, як це пояснюють інші,
падання заробітної плати й відповідне підвищення експлуатації
робітника знову прискорює акумуляцію, тимчасом як нижча
заробітна плата одночасно затримує зростання робітничої кляси.
Таким чином знову постає таке відношення, коли подання праці
нижче від попиту на працю, заробітна плата зростає й т. д.
Яка прегарна метода руху для розвинутої капіталістичної продукції!
Поки в наслідок підвищення заробітної плати міг би
настати будь-який позитивний зріст дійсно працездатної людности,
декілька разів минув би той час, що протягом його треба
провести промислову кампанію та вирішити справу в бою.

Між 1849 і 1859 рр., одночасно з падінням цін на збіжжя,
практично сталось лише номінальне підвищення заробітної
плати в англійських рільничих округах: наприклад, у Wiltshire
тижнева плата зросла з 7 до 8 шилінґів, у Dorsetshire — з 7 або
8 шилінґів до 9 шилінґів і т. ін. Це був наслідок надзвичайного
відпливу надмірної рільничої людности, спричиненого потребами
війни, масовим поширенням будування залізниць, фабрик, гірничих
підприємств і т. ін. Що нижча заробітна плата, то вищі процентові
числа, що в них виражається всяке, хоч би й яке незначне
