цента «Morning Star», що з початком 1867 р. відвідав головні
центри злиднів. «У східній частині Лондону, в округах Poplar
Millwall, Greenwich, Deptford, Limehouse і Canning Town щонайменше
15.000 робітників разом із своїми родинами живуть у
якнайтяжчих злиднях, поміж ними понад 3.000 навчених механіків.
їхні резервні фонди вичерпано в наслідок шести-восьмимісячного
безробіття... Багато зусиль коштувало мені протиснутись
до воріт робітного дому (в Роріаг’і), бо його облягла зголодніла
юрба. Вона чекала на хлібні картки, але час роздавання
їх ще не настав. Подвір’я являє собою великий квардрат із
піддашшям навколо мурів. Кучугури снігу густо вкривали
кам’яний брук на середині подвір’я. Тут деякі невеличкі площі
були загороджені івовим тином, наче кошари для овець, де гарної
години працюють чоловіки. В день моїх відвідин кошари так
були позасипувані снігом, що ніхто не міг у них сидіти. Однак
під захистом підашшя чоловіки розбивали брукняк. Кожний з
них сидів на великому бруковому камені і тяжким молотом бив
по обмерзлому ґраніту. доки набивав з нього 5 бушлів. Тоді його
денна робота кінчалась, і він діставав 3 пенси і хлібну картку.
У другій частині подвір’я стояла злиденна дерев’яна хатина.
Відчинивши двері, ми побачили, що вона була повна чоловіків,
які тулилися один до одного, щоб зігрітись. Вони дерли клоччя
з корабельної линви і сперечалися між собою, хто з них при мінімумі
харчів може найдовше працювати, бо витривалість була
тут point d’honneur.* В цьому одному лише робітному домі діставало
допомогу 7.000 осіб, серед них сотні таких, що 6 або 8 місяців
тому заробляли вправною працею найвищу в цій країні
заробітну плату. Число їх було б удвоє більше, коли б не те,
що багато з них, навіть вичерпавши всі свої грошові заощадження,
все-таки не наважуються вдаватися по допомогу до парафії,
доки в них іще лишається що-будь заставляти... Покинувши робітний
дім, я пішов вулицями здебільша з одноповерховими будинками,
що їх так багато в Роріаг’і. Моїм поводирем був член
комітету безробітних. Перший дім, куди ми зайшли, був дім
залізничника, що 27 тижнів був уже без роботи. Я найшов його
з цілою його родиною в задній кімнатці. В кімнатці були ще
деякі меблі, її ще опалювали. Це конче треба було, щоб захистити
голі ноги маленьких дітей від холоду, бо день був страшенно
зимний. На тарілці проти вогню лежало клоччя, і його жінка
й діти дерли у відплату за хліб з робітного дому. Чоловік працював
в одному з вищеописаних подвір’їв за хлібну картку й
З пенси на день. Тепер він прийшов додому обідати, дуже зголоднілий,
як сказав він нам з гіркою посмішкою, а його обід
складався з кількох шматків хліба з смальцем і склянки чаю
без молока... Дальші двері, куди ми постукали, відчинила жінка
середнього віку, яка, не сказавши й слова, провела нас у малюсіньку
задню кімнатку, де мовчки сиділа ціла її родина, втупивши

— справою чести. Ред.
