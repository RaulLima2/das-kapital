\index{i}{0555}  %% посилання на сторінку оригінального видання
Закон, що за ним щораз більшу масу засобів продукції, у
наслідок проґресу продуктивности суспільної праці, можна пускати
в рух із щораз меншою витратою людської сили, — цей
закон на капіталістичній основі, де не робітник уживає засобів
праці, а засоби праці вживають робітника, виражається в тому,
що, чим вища продуктивна сила праці, тим більший тиск робітників
на засоби їхньої роботи, отже, тим непевніша умова їхнього
існування: продаж власної сили для збільшування чужого багатства
або для самозростання капіталу. Отже, швидше зростання
засобів продукції і продуктивности праці, швидше, ніж
зростання продуктивної людности, виражається за капіталізму,
навпаки, в тому, що робітнича людність завжди зростає швидше,
ніж потреби самозростання капіталу.

У четвертому відділі, аналізуючи продукцію відносної додаткової
вартости, ми бачили, що за капіталістичної системи всі
методи підвищення суспільної продуктивної сили праці відбуваються
коштом індивідуального робітника; всі засоби для розвитку
продукції перетворюються на засоби поневолення й експлуатації
продуцента, калічать робітника, роблячи з нього несповналюдину,
принижують його до стану додатку до машини, з муками
його праці знищують і її зміст, відчужують від робітника духовні
сили процесу праці в тій самій мірі, в якій наука сполучається
з цим останнім як самостійна сила; вони спотворюють умови, серед
яких працює робітник, підбивають його підчас процесу праці
під якнайдріб’язковішу, ненависну деспотію, ціле його життя
перетворюють на робочий час, його жінку й дітей кидають під
джеґґернавтові колеса капіталу. Але всі методи продукції додаткової
вартости є разом з тим методи акумуляції, і всяке поширення
акумуляції стає, навпаки, засобом розвитку цих метод.
Звідси випливає, що в міру того, як акумулюється капітал,
становище робітника мусить гіршати, хоч яка б була його плата —
висока чи низька. Нарешті, той закон, що завжди тримає відносне
перелюднення, або промислову резервну армію, в рівновазі
з розмірами й енерґією акумуляції, приковує робітника до капіталу
міцніше, аніж молот Ґефеста прикував Прометея до скелі.
Цей закон зумовлює акумуляцію злиднів, що відповідає акумуляції
капіталу. Отже, акумуляція багатства на одному полюсі
є разом з тим акумуляція злиднів, мук праці, рабства, неуцтва,
здичавіння й моральної деґрадації на протилежному полюсі,
тобто на боці тієї кляси, що продукує свій власний продукт як
капітал.

Цей антагоністичний характер капіталістичної акумуляції\footnote{
«З дня на день стає ясніше, що відносини продукції, в яких
рухається буржуазія, мають не однорідний, простий характер, а двоїстий
характер; що в тій самій пропорції, в якій продукується багатство,
продукуються і злидні; що в тій самій пропорції в якій відбувається розвиток
продуктивних сил, розвивається й сила поневолення; що ці відносини
продукують буржуазне багатство, тобто багатство буржузної кляси, лише
постійно знищуючи багатство поодиноких членів цієї кляси і створюючи
пролетаріят, що дедалі більше зростає» («De jour en jour il devient donc
plus clair que les rappotrs de production dans lesquels se meut la bourgeoisie
n’ont pas un caractère un, un caractère simple, mais un caractère de duplicité;
que dans les mêmes rapports dans lesquels se produit la richesse, la
misère se produit aussi: que dans les mêmes rapports dans lesquels il y a
développement des forces productives, il y a une force productive de répression;
que ces rapports ne produisent la richesse bourgeoise, c’est à dire
la richesse de la classe bourgeoise, qu’en anéantissant continuellement la
richesse des membres intégrants de cette classe et en produisant un prolétariat
toujours croissant». (\emph{K.~Marx}: «Misère de la Philosophie», p. 116.
— \emph{K.~Маркс}: «Злиденність філософії», Партвидав 1932, стор. 110).
}
зазначали в різних формах політико-економи, хоч вони почасти
\index{i}{0556}  %% посилання на сторінку оригінального видання
сплутують з ним аналогічні, але посутньо відмінні явища передкапіталістичних
способів продукції.

Венеціанський чернець Ортес, один із найбільших письменників"=економістів
XVIII віку, розглядає антагонізм капіталістичної
продукції як загальний природний закон суспільного
багатства. «Економічне добро й економічне зло в якійсь нації
завжди зрівноважуються (il bene ed il male economico in una
nazione sempre all’istessa misura), повнява дібр в одних є завжди
недостача дібр в інших (la copia dei beni in alcuni sempre eguale
alla mancanza di esse is altri). Велике багатство небагатьох завжди
супроводиться абсолютним грабуванням доконечного в далеко
більшого числа інших. Багатство якоїсь нації відповідає її людності,
а злидні її відповідають її багатству. Працьовитість одних
вимушує ледарство інших. Бідні й нероби є неминучий продукт
багатих і працьовитих» і~\abbr{т. д.}\footnote{
\emph{G.~Ortes}: «Delia Economia Nazionale libri sei», 1777, y Custodi.
Parte Moderna, vol. XXI, p. 6, 9, 22, 25 etc. Ортес каже (там же, стор. 32):
«Замість вигадувати нікчемні системи, як зробити народи щасливими,
я хочу обмежитися на розсліді причин їхнього нещастя» («In luoco di
progettar sistemi inutili per la felicità de popoli, mi limiterô a investigare
la ragione delia loro infelicità»).
} Через якихось десять років
після Ортеса англікансько-протестантський піп Тавнсенд цілком
грубим способом розхвалював злидні як доконечну умову
багатства. «Законодатний примус до праці є пов’язаний із чималими
труднощами, насильством і шумом, тимчасом як голод не
тільки є мирний, мовчазний, безупинний натиск, але, являючи
собою якнайприроднішу спонуку до промисловости і праці, викликає
якнайдужче напруження». Отже, все сходить на те, щоб
для робітничої кляси зробити голод перманентним, і про це, за
Тавнсендом, дбає принцип залюднення, який є особливо активний
серед бідних. «Це є, здається, природний закон, що бідні
до певної міри легкодумні (improvident) (а саме так легкодумні,
що приходять на світ без золотої ложки в роті), так що завжди
знаходяться люди (that there always may be some) для виконання
найнижчих, найбрудніших і найпаскудніших функцій у суспільстві.
Запас людського щастя (the fund of human happiness) через
те дуже збільшується, делікатніші люди (the more delicate) увільнені
від мук праці й можуть без перешкод іти за своїм вищим
покликанням і т. д\dots{} Закон про бідних має тенденцію зруйнувати
гармонію і красу, симетрію й порядок цієї системи, що її
\parbreak{}  %% абзац продовжується на наступній сторінці
