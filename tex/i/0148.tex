приміром, один день вищої праці на х днів простої праці.19 [Коли
фахівці економісти обурюються проти цього «довільного твердження»,
то про них безперечно можна сказати, відповідно до
німецької приказки, що за деревами вони не бачать лісу. Те, що
вони кваліфікують як теоретичні хитрощі, на ділі є лише процедура,
що її щодня практикують по всіх краях світу. Повсюди
вартості найрізноманітніших товарів однаковісінько виражені
в грошах, тобто в певній кількості золота або срібла. Цим самим
різні роди праці, репрезентовані цими вартостями, зведені у
різних пропорціях на певні суми одного і того самого роду звичайної
праці, праці, що продукує золото або срібло].* Отже,
ми заощадимо собі зайву операцію і зробимо простішою аналізу,
припускаючи, що робітник, якого вживає капітал, виконує
просту суспільну пересічну працю.

Розділ шостий

Сталий капітал і змінний капітал

Різні фактори процесу праці беруть різну участь у творенні
вартости продукту.

Робітник додає до предмету праці нову вартість, долучаючи
до нього певну кількість праці, хоч і який був би конкретний
вміст, мета й технічний характер його праці. З другого боку, ми
знову находимо вартості зужиткованих засобів продукції як
складові частини вартости продукту, приміром, вартості бавовни
й веретен — у вартості пряжі. Отже, вартість засобів продукції
зберігається через перенесення її на продукт. Це перенесення
відбувається підчас перетворення засобів продукції на продукт,
у процесі праці. Праця упосереднює це перенесення. Але яким
чином?

Робітник не працює подвійно в той самий час: раз для того,
щоб своєю працею додати до бавовни вартість, а другий раз для
того, щоб зберегти стару вартість бавовни, або, що те саме, для
того, щоб на продукт, на пряжу, перенести вартість бавовни, яку
він обробляє, і вартість веретен, якими він працює. Він зберігає
стару вартість лише тим, що додає до неї нову вартість. Але що

згаданих 11 мільйонів. (S. Laing: «National Distress etc.», London
1844, стор. 51, 52). «Велика кляса, що нічого не може дати за харчі,
крім простої праці, становить головну масу народу». («The great class,
who have nothing to give for food but ordinary labour, are the great body
of the people»). (James Mill in Art. «Colony». Supplement to the. Encyclopedia
Britannica, 1831, p. 8).

19 «Коли кажуть про працю як про міру вартости, то неодмінно мають
на думці працю певного роду... відношення її до інших родів праці можна
легко визначити». («Where reference is made to labour as a measure of
value, it necessarily implies labour of one particular kind... the proportion
which the other kinds bear to it being easily ascertained»). («Outlines of
Political Economy», London 1832, p. 22, 23).

* Заведене у прямі дужки ми беремо з французького видання. («Le Саpital
etc.», v. І, ch. VII, p. 84). Peд.
