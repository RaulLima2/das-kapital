них відносинах одні до одних і до людей. Так само стоїть справа
в товаровому світі з продуктами людських рук. Це я зву фетишизмом,
що пристає до продуктів праці, скоро їх продукується
як товари, і що, отже, є невіддільний від товарової продукції.

Цей фетишистичний характер товарового світу випливає, як
показала вже попередня аналіза, із своєрідного суспільного
характеру праці, що продукує товари.

Предмети споживання стають взагалі товарами лише тому, що
вони є продукти приватних праць, виконуваних незалежно одна
від однієї. Комплекс цих приватних праць становить сукупну
працю суспільства. А що продуценти ввіходять у суспільний контакт
лише через обмін продуктів своєї праці, то й специфічні суспільні
характери їхніх приватних праць виявляються лише в
межах цього обміну. Інакше кажучи, приватні праці фактично
виявляються як члени сукупної праці суспільства лише через ті
відносини, в які обмін ставить продукти праці, а за допомогою цих
продуктів і продуцентів. Тим то для продуцентів суспільні відносини
їхніх приватних праць з’являються тим, чим саме вони
й є, тобто не безпосередньо суспільними відносинами осіб у самій
їхній праці, а, навпаки, речовими відносинами осіб і суспільними
відносинами речей.

Лише в межах обміну їх продукти праці набирають суспільно
однакової вартостевої предметносте (Wertgegenständlichkeit), відокремленої
від їхніх почуттєво різних споживних предметностей
(Gebrauchsgegenstandlichkeit). * Це розщеплення продукту праці
на корисний предмет і предмет вартости реалізується на практиці
лише тоді, коли обмін набирає вже достатнього поширення й значення
для того, щоб корисні речі продукувалося для обміну, коли,
отже, вартостевий характер речей береться на увагу вже за самої
продукції їх. Від цього моменту приватні праці продуцентів
дійсно набувають двоїстого суспільного характеру. З одного боку,
вони мусять як певні корисні праці задовольняти певні суспільні
потреби й таким чином довести, що вони є члени сукупної праці,
члени природно вирослої системи суспільного поділу праці. З другого
боку, вони задовольняють лише різноманітні потреби своїх
власних продуцентів, оскільки кожна окрема корисна приватна
праця є вимінна на кожний інший рід корисної приватної праці,
тобто є йому рівнозначна. Рівність toto coelo** різних праць може
бути лише в абстрагуванні від їхньої справжньої нерівности, у
зведенні до спільного характеру, що його вони мають як витрати
людської робочої сили, абстрактної людської праці. Мозок приватних
продуцентів відбиває двоїстий суспільний характер їхніх
приватних праць тільки в тих формах, які з’являються у прак-

* У французькому виданні це місце зредаґовано так: «Лише в процесі
обміну їх продукти праці набирають як вартості тотожного й одноманітного
суспільного існування, що є відмінне від їхнього матеріяльного
й різноманітного існування як предметів споживання. («Le Capital
etc.», v. I., ch. I., p. 29). Ред.

** — всіма сторонами. Ред.
