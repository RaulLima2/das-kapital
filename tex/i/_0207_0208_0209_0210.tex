\parcont{}  %% абзац починається на попередній сторінці
\index{i}{0207}  %% посилання на сторінку оригінального видання
той час, поки воно триває. Тому основний принцип рабовласницького
господарства по тих країнах, куди ввозять рабів, такий:
найуспішніша економія в тому, щоб із людської худоби (human
chattle) видушити якнайбільшу кількість праці протягом якнайменшого
часу. Саме в країнах з тропічною культурою, де річний
зиск часто дорівнює цілому капіталові плантацій, життя негрів
якнайнещадніше кидається в жертву. Саме рільництво Західньої
Індії, вікова колиска казкового багатства, пожерло мільйони
людей африканської раси. Саме тепер на Кубі, де доходи рахуються
на мільйони, де плянтатори є князі, ми бачимо, що кляса
рабів не тільки дістає щонайгірше харчування і зазнає якнайвиснажувальнішої
й безперестанної муки, але й чимала частина
її рік-у-рік просто винищується через повільне катування
надмірною працею й недостачу сну й відпочинку»\footnote{
\emph{J.~С.~Cairns}: «The Slave Power», London 1862, p. 110, 111.
}.

Mutato nomine de te fabula narratur\footnote*{
Під іншим іменем тут про тебе мова. \emph{Ред.}
}.  Замість работорговлі
читай — робочий ринок, замість Кентукі й Вірґінія — Ірляндія
й рільничі округи Англії, Шотляндії й Велзу, замість Африка —
Німеччина! Ми чули, як надмірна праця спустошує лави лондонських
пекарів, а, проте, лондонський робочий ринок завжди
переповнений німецькими й іншими кандидатами на смерть по
пекарнях. Ганчарні, як ми бачили, є одна з галузей промисловости,
де життя робітників якнайкоротше. Та чи бракує від того ганчарів?
Джосія Веґвуд, винахідник сучасної ганчарні, сам із роду
звичайний робітник, заявив 1785~\abbr{р.} перед Палатою громад,
що в цілій цій мануфактурі працює від 15 до 20 тисяч осіб\footnote{
\emph{John Ward}: «History of the Borough of Stoke-upon-Trent», London
1834, p. 42.
}.
Року 1861 людність тільки міських центрів цієї промисловости
Великобрітанії становила \num{101.302} чол. «Бавовняна промисловість
існує вже 90 років. За час трьох поколінь англійської раси
вона пожерла дев’ять поколінь бавовняних робітників»\footnote{
Промова Феранда в Палаті громад 27 квітня 1863~\abbr{р.}
}.
Певна річ, підчас окремих епох гарячкового розцвіту робочий
ринок виявляв чималі прогалини. Так було, приміром, року 1834.
Але тут панове фабриканти запропонували Poor Law Commissioners\footnote*{
— комісії у справах бідних. \emph{Ред.}
}
відправляти «надмір людности» рільничих округ на північ,
заявивши, що «фабриканти його поглинули б і спожили б»\footnote{
«That the manufacturers would absorb it and use it up. Those were
the very words used by the cotton manufacturers». (Там же).
}.
Це були їхні власні слова. «За згодою Poor Law Commissioners
призначено аґентів до Менчестеру. Виготовлено й вручено цим
аґентам списки рільничих робітників. Фабриканти кинулися до
бюр, і після того, як вони повибирали все, що їм було потрібно,
цілі родини повисилано з півдня Англії. Ці людські вантажі із
значками, як паки товарів, постачувано каналом і вантажевими
возами; декотрі пошкандибали за ними пішки, а багато збилося
з шляху й напівголодні блукали по промислових округах. Це розвинулося
\index{i}{0208}  %% посилання на сторінку оригінального видання
до розмірів справжньої галузі торговлі. Палата громад
навряд чи повірить цьому. Ця реґулярна торговля, баришування
людським м’ясом, тривала й далі, і цих людей купували манчестерські
аґенти й продавали менчестерським фабрикантам так
само регулярно, як продають негрів плянтаторам бавовни в південних
державах\dots{} Рік 1860 — кульмінаційний пункт бавовняної
промисловости\dots{} Знов бракувало робочих рук. Фабриканти
знову звернулись до аґентів продажу людського м’яса\dots{} і ці пронишпорили
всі прибережні дюни Dorset’a, верховину Devon’a
й рівнини Wilts’a, але надмір людности був уже спожитий».
«Bury Guardian» нарікав на те, що після складання англійсько-французького
торговельного договору можна було б поглинути
\num{10.000} додаткових рук, а незабаром їм потрібно буде ще \num{30.000}--\num{40.000}.
Після того як аґенти й субаґенти цієї торговлі м’ясом
майже безуспішно пронишпорили 1860~\abbr{р.} всі рільничі округи,
«депутація фабрикантів звернулась до пана Вільєрса, президента
Poor Law Board\footnote*{
— інституції у справах бідних. \emph{Ред.}
}, із проханням знов дозволити їм брати на
фабрики сиріт і дітей бідних із робітних домів\footnote{
Там же. Хоч Вільєрс і був дуже доброзичливий до фабрикантів,
а проте мусив «на основі закону» відмовити домаганням фабрикантів.
Однак ці пани досягли своєї мети завдяки прислужливості місцевої
адміністрації домів для бідних. Фабричний інспектор А.~Редґрев запевняє,
що цього разу систему, за якої сиріт і дітей павперів «за законом» вважається
за учнів (apprentices), «не супроводилось колишніми зловживаннями»
(про ці зловживання див. в Енґельса: «Становище робітничої
кляси в Англії»), хоч, все ж, в одному випадку «зловжито системою щодо
дівчат і молодих жінок, яких привезено до Ланкашіру й Чешіру з рільничих
округ Шотляндії». Ця «система» полягає в тому, що фабрикант складає
з адміністрацією домів для бідних контракт на певний строк. Він харчує
дітей, одягає їх і дає їм помешкання й невеличку доплату грішми.
Дивно якось чути дальше зауваження пана Редґрева, особливо, коли взяти
на увагу, що навіть за часів розцвіту англійської бавовняної промисловости
1860 рік стоїть цілком окремо і що, крім того, заробітна плата була
дуже високою, бо надзвичайний попит на робітників збігся із знелюдненням
Ірляндії, з небувалою еміґрацією з англійських і шотляндських
рільничих округ до Австралії й Америки, з позитивним зменшенням людности
в декотрих англійських рільничих округах, що сталося почасти в
наслідок щасливо осягнутого зламу життєвої сили, а почасти в наслідок
того, що торгівці людським м’ясом ще раніше вичерпали всю зайвину
людности. І, не зважаючи на все це, пан Редґрев каже: «Однак на працю
такого роду (праця дітей із домів для бідних) є попит лише тоді, коли не
можна найти жодної іншої, бо це дорога праця (high-priced labour).
Звичайна заробітна плата підлітка від 13 років дорівнює приблизно 4\shil{ шилінґам}
на тиждень: але дати помешкання, одягати і харчувати 50 або 100
таких підлітків, забезпечити їм лікарську допомогу й подбати про потрібний
догляд та, крім того, давати ще невеличку доплату грішми, — цього
не можна зробити за 4\shil{ шилінґи} на душу за тиждень». («Reports of Insp.
of Fact. 30 th April 1860», p. 27). Пан Редґрев забуває сказати, яким
чином сам робітник може дати все це своїм дітям за їхню заробітну плату
в 4\shil{ шилінґи}, коли фабрикантові не сила зробити цього для 50 або 100
хлопців, що живуть укупі, користуються спільним утриманням, спільним
доглядом. Щоб уникнути фалшивих висновків із тексту, я мушу тут ще
зауважити, що англійську бавовняну промисловість від того часу, як
її підпорядковано Factory Act’oвi 1850~\abbr{р.} що вреґулював робочий день
і~\abbr{т. д.}, треба розглядати як зразкову промисловість Англії. Англійський
бавовняний робітник стоїть з кожного погляду вище, ніж його товаришу
недолі на континенті. «Пруський фабричний робітник працює щонайменше
10 годин на тиждень більше, ніж його англійський суперник, а коли він
працює в себе вдома на власному ткацькому варстаті, то відпадає навіть
і ця межа його додаткових робочих годин». («Reports of Insp. of Fact.
31 st October 1855», p. 103). Згаданий вище фабричний інспектор Редґрев
після промислової виставки 1851~\abbr{р.} поїхав на континент, спеціально до
Франції й Прусії, щоб вивчити порядки на фабриках у цих країнах.
Ось що каже він про пруського фабричного робітника: «Він дістає заробітну
плату, якої вистачає лише на ті прості харчі й маленький комфорт, до
якого він звик і яким задовольняється\dots{} Він живе гірше й працює важче,
ніж його англійський суперник». («Reports of Insp. of Fact. 31st October
1853», p. 85).
}.

\index{i}{0209}  %% посилання на сторінку оригінального видання
Взагалі, досвід показує капіталістові, що існує постійне
перелюднення, тобто перелюднення проти наявної в даний момент
потреби капіталу самозростати своєю вартістю, хоч це перелюднення
і утворюється з людських зниділих поколінь, — поколінь,
що швидко вимирають і витискують одне одного, поколінь, так
би мовити, позриваних у недостиглому стані\footnote{
«Люди, що надмірно працюють, вмирають навдивовижу швидко,
але місця тих, що гинуть, зараз же поповнюються знов, і часта зміна
осіб не зумовлює жодної зміни на сцені». «England and America»,
London 1833, vol. I, p. 55. (Автор: \emph{E.~G.~Wakefield}).
}. Щоправда,
тямущому спостерігачеві досвід показує також, як швидко
й глибоко капіталістична продукція, що, висловлюючись історично,
виникла лише вчорашнього дня, підкопала вже до
самого коріння життя народню силу, як виродження промислової
людности уповільнюється тільки тим, що постійно вбирається
природно-вирослі життєві елементи села, і як навіть сільські
робітники починають уже вимирати, не вважаючи на свіже повітря
й таке могутнє панування серед них principle of natural selection\footnote*{
— принципу природного добору. \emph{Ред.}
},
який дозволяє виростати лише найсильнішим індивідам\footnote{
Див. «Public Health. Sixth Report of the Medical officer of the
Privy Council. 1863». Опубліковано в Лондоні 1864 p. В цьому звіті
мова йде саме про рільничих робітників. «Графство Sutherland змальовували
таким, наче б у ньому пороблено значні поліпшення, але недавні
розсліди показали, що в округах нього графства, яке колись так славилося
красою чоловіків і сміливістю солдатів, людність виродилася в
хиряву й зниділу расу. В найздоровіших місцевостях, на приморських
узбіччях гір обличчя дітей такі худі й бліді, як вони можуть бути
тільки в гнилій атмосфері якогось лондонського заулка». (\emph{Thornton}:
«Overpopulation and its Remedy», London 1846, p. 74, 75). Вони дійсно
скидаються на тих \num{30.000} «gallant Highlanders»\footnote*{
— бравих горян. \emph{Ред.}
}, що їх Ґлезґо збирає
до купи з проститутками й злодіями по своїх wynds і closes\footnote*{
— заулках і вертепах. \emph{Ред.}
}.
}.
Капітал, що має такі «достатні причини» заперечувати страждання
покоління робітників, яке оточує його, у своєму практичному
русі так само мало керується перспективою майбутнього загнивання
людства, отже, кінець-кінцем, неминучого знелюднення, —
як тією перспективою, що земля може впасти на сонце. За кожної
спекуляції акціями всякий знає, що колись та мусить ударити
грім, але всякий сподівається, що він спаде на голову його ближнього,
\index{i}{0210}  %% посилання на сторінку оригінального видання
і саме після того, як йому самому пощастить зібрати золотий
дощ і заховати його в безпечне місце. Après moi le déluge!\footnote*{
Після мене хоч потоп! \emph{Ред.}
} —
оце гасло кожного капіталіста й кожної капіталістичної нації.
Тому капітал не звертає найменшої уваги на здоров’я та життя
робітника там, де суспільство не примушує його звертати на це
увагу\footnote{
«Хоч здоров'я людности і є такий важливий елемент національного
капіталу, ми боїмося, що доведеться визнати, що капіталісти зовсім
не мають нахилу берегти й цінити цей скарб\dots{} Фабрикантів примушено
звертати увагу на здоров'я робітників». («Times», 5 листопада 1861).
«Чоловіки West Riding’a стали сукноробами для цілого людства\dots{} здоров’я
робітничої людности було віддано на жертву і протягом декількох
поколінь раса була б виродилася, але наступила реакція. Обмежено
години дитячої праці й~\abbr{т. ін.}» («Report of the Registrar General for October
1861»).
}. На скарги на фізичний і інтелектуальний занепад,
передчасну смерть, катування надмірою працею він відповідає:
невже мають боліти нам ті муки, коли вони збільшують нашу
втіху (зиск)? Але взагалі і в цілому це навіть не залежить від
доброї або злої волі поодинокого капіталіста. Вільна конкуренція
накидає поодинокому капіталістові іманентні закони капіталістичної
продукції як зовнішній примусовий закон\footnote{
\label{footnote-114}Тим-то ми бачимо, наприклад, що на початку 1863~\abbr{р.} 26 фірм,
які посідають величезні ганчарні в Staffordshire, між ними й Дж.~Веґвуд
із синами, прохають у поданому меморіялі «насильного втручання держави».
«Конкуренція з іншими капіталістами» не дозволяє, мовляв, їм
зробити жодного «добровільного» обмеження робочого часу дітей і~\abbr{т. ін.}
«Тим то, хоч би ми й як нарікали на вищезгадане лихо, його ніяк не можна
було б усунути шляхом якоїсь згоди поміж фабрикантами\dots{} Зважаючи на
всі ці пункти, ми прийшли до того переконання, що потрібен примусовий
закон». («Children’s Employment, Commission. 1st Report 1863»,
p. 322).

Додаток до примітки \ref{footnote-114}. Ще яскравіший приклад дає нам найближче
минуле. Високі ціни на бавовну за часів гарячкового розвитку справ спонукали
власників ткалень бавовни у Blackburn’i за взаємною згодою між
собою на якийсь певний реченець скоротити на своїх фабриках робочий
час. Реченець цей скінчився приблизно з кінцем листопада (1871~\abbr{р.})
Тимчасом багатші фабриканти, в яких прядіння було сполучене з тканням,
використали скорочення продукції, зумовлене цією згодою, на те, щоб
поширити свої власні підприємства й таким чином коштом дрібних підприємців
здобути великі зиски. Опинившись у скрутному становищі,
останні звернулися до фабричних робітників, закликаючи їх серйозно
заходитися коло аґітації за дев’ятигодинний робочий день, і обіцяли їм
грошову допомогу на цю справу.
}.

Установлення нормального робочого дня є результат багатовікової
боротьби між капіталістом і робітником. Але історія цієї
боротьби виявляє дві протилежні течії. Порівняймо, приміром,
англійське фабричне законодавство нашого часу з англійськими
робітничими статутами від ХІV аж далеко до половини XVIII
століття\footnote{Ці робітничі статути, які ми знаходимо одночасно і у Франції,
Нідерляндах і~\abbr{т. д.}, формально скасовано в Англії лише 1813~\abbr{р.}, вже після
того, як їх давно були усунули сами продукційні відносини.}. Тоді як сучасний фабричний закон насильно скорочує
робочий день, ці статути намагаються насильно його здовжити.
Певна річ, домагання капіталу в ембріональному стані,
\parbreak{}  %% абзац продовжується на наступній сторінці
