жетова промова Ґледстона, з 7 квітня 1864 р., — це піндарівський
дитирамб на проґрес у збагаченні й на стримуване «злиднями»
щастя народу. Він каже про маси, що стоять «на краю
павперизму», про галузі продукції, «де заробітна плата не
зросла», і наприкінці резюмує щастя робітничої кляси в таких
словах: «Людське життя в дев’ятьох із десятьох випадків це
просто боротьба за існування».105 Професор Фавсет, не зв’язаний,
як Ґледстон, офіціяльними міркуваннями, прямо заявляє:
«Я, звичайно, не заперечую, що грошова плата підвищилася із
цим збільшенням капіталу [останніми десятиріччями], але ця
позірна користь у значних розмірах знову пропадає через те,
що багато потрібних засобів існування щораз дорожчає на його
думку, через падіння вартости благородних металів]... Багаті
швидко стають ще багатшими (the rich grow rapidly richer),
тимчасом як у побуті робітничих кляс не помітно ніякого поліпшення...
Робітники стають майже рабами крамарів, що в них
вони позаборговувались».106

У розділах про робочий день і машини ми розкрили ті обставини,
в яких брітанська робітнича кляса створила «приголомшливе
збільшення багатства й сили» для маєтних кляс. Однак
нас тоді цікавив переважно робітник підчас його суспільної
функції. Щоб цілком висвітлити закони капіталістичної акумуляції,
треба також на хвилину зупинитись на становищі робітника
поза майстернею, на тому, яке його харчове й житлове становище.
Рамки книги цієї примушують нас насамперед взяти тут на увагу
найгірш оплачувану частину промислового пролетаріату і рільничих
робітників, тобто більшість робітничої кляси.

105 «Think of those who are on the border of that region (pauperism)»,
«wages... in others not increased... human life is but, in nine cases out of ten,
a struggle for existence» (Ґледстон у Палаті громад 7 квітня 1864 р.).
«Hansard» дає таку версію цього резюме: «Висловлюючи це в загальнішій
формі: «Що таке людське життя в більшості випадків, як не боротьба
за існування» («Again, and yet more at large, what is human life but,
in the majority of cases, a struggle for existence»). — Постійні кричущі суперечності
в бюджетових промовах Ґледстона з 1863 і 1864 рр. один англійський
письменник характеризує такою цитатою з Мольєра:

«Voilà l’homme en effet. Il va du blanc au noir.
Il condamne au matin ses sentiments du soir.
Importun à tout autre, à soi même incommode,
Il change à tous moments d’esprit comme de mode».

(«Така людина: зараз біле, далі чорне.
Що ввечорі хвалила, засуджує уранці.
Усім набридла і самій собі як тягар,
І щохвилини настрій змінює як моду»).

(«Th. Theory of Exchanges etc.»,
London 1864, p. 135).

106    H. Fawcett: «The Economie Position of the British Labourer»,
London 1865. p. 67, 82. Щождо дедалі більшої залежності! робітників од
крамарів, то це є наслідок щораз частіших коливань і перерв у їхній
занятості.
