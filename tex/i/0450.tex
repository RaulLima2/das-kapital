Відділ шостий

Заробітна плата

Розділ сімнадцятий

Перетворення вартости, зглядно ціни робочої
сили на заробітну плату

На поверхні буржуазного суспільства плата робітника здається
ціною праці, певного кількістю грошей, що її платять за
певну кількість праці. При цьому говорять про вартість праці
і грошовий вираз її вартости називають доконечною або природною
ціною праці. З другого боку, говорять про ринкові ціни
праці, тобто про ціни, що коливаються вище або нижче її доконечної
ціни.

Але що таке вартість якогось товару? Предметна форма витраченої
на його продукцію суспільної праці. А чим міряємо
ми величину його вартости? Величиною праці, що міститься
в ньому. Отже, чим можна б визначити вартість, наприклад,
дванадцятигодинного робочого дня? Дванадцятьма годинами
праці, що містяться у дванадцятигодинному робочому дні, а це
є нісенітна тавтологія.21

Для того, щоб бути проданою на ринку як товар, праця мусила
б у всякому разі існувати ще до її продажу. Але коли б

21 «Пан Рікардо дуже дотепно уникає труднощів, які на перший
погляд загрожують звалити його доктрину, що вартість залежить від
кількости праці, витраченої на продукцію. Якщо суворо додержувати
цього принципу, то з нього випливає, що вартість праці залежить від
кількости праці, зужитої на її продукцію, а це очевидний абсурд. Тому
за допомогою мудрого виверту пан Рікардо ставить вартість праці у
залежність від кількости праці, потрібної на продукцію заробітної плати,
або, виражаючись його власними словами, він каже, що вартість праці
оцінюється за кількістю праці, потрібної на продукцію заробітної плати;
під цим він розуміє кількість праці, потрібної на продукцію грошей або
товарів, що їх дають робітникові. Це те саме, що сказати: вартість сукна
оцінюється не за кількістю праці, витраченої на його продукцію, а за
кількістю праці, витраченої на продукцію того срібла, на яке сукно
обмінюється». («Мг. Ricardo, ingeniously enough, avoids a difficulty
which, on a first view, threatens to encumber his doctrine, that value depends
on the quantity of labour employed in production. If this principle
is rigidly adhered to, it follows that the value of labour depends on the
quantity of labour employed in producing it — which is evidently absurd.
By a dextrous turn, therefore, Mr. Ricardo make the value of labour depend
on the quantity of labour required to produce wages; or, to give him the
oenefit of his own language, he maintains, that the value of labour is to
be estimated by the quantity of labour required to produce wages; by which
