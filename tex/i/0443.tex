шене зужитковання робочої сили, невіддільне від здовження
робочого дня, можна компенсувати збільшеним відживленням її.
Поза цим пунктом зужитковування зростає в геометричній прогресії
і одночасно руйнуються всі нормальні умови репродукції
та функціонування робочої сили. Ціна робочої сили і ступінь
її експлуатації перестають бути спільномірними величинами.

IV. Одночасні зміни тривання праці, продуктивної сили прапі
та інтенсивносте праці

Тут, очевидно, можливе велике число комбінацій. Можуть
змінятися два фактори, а один лишатися сталим, або всі три фактори
можуть одночасно змінятися. Вони можуть змінятися в
однаковій або неоднаковій мірі, в тому самому або в протилежному
напрямі, і їхні зміни можуть тому почасти або цілком
навзаєм компенсуватися. А втім, аналіза всіх можливих випадків,
після висновків, поданих у пунктах І, II та III, легка. Результат
кожної можливої комбінації можна знайти, коли розглядати
почережно кожний з факторів як змінний, а інші як сталі.
Тому ми тут коротко зазначимо лише два важливі випадки.

1) Падуща продуктивна сила праці при одночасному здовженні
робочого дня.

Коли ми тут говоримо про падущу продуктивну силу праці,
то йдеться про галузі праці, що їхні продукти визначають вартість
робочої сили, отже, наприклад, про падущу продуктивну силу
праці в наслідок чимраз більшої неродючосте ґрунту та відповідного
подорожчання продуктів землі. Припустімо, що робочий
день триває 12 годин, новоспродукована протягом нього вартість
становить 6 шилінґів, з чого половина покриває вартість робочої
сили, а друга половина становить додаткову вартість. Отже,
робочий день розпадається на 6 годин доконечної праці і 6 годин
додаткової праці. Хай у наслідок подорожчання продуктів землі
вартість робочої сили підвищується з 3 шилінґів до 4 шилінґів,
отже, доконечний робочий час — з 6 до 8 годин. Якщо робочий
день лишається незмінний, то додаткова праця спадає з б до 4 годин,
а додаткова вартість з 3 до 2 шилінґів. Якщо робочий день
здовжується на 2 години, тобто з 12 до 14 годин, то додаткова
праця лишається 6 годин, а додаткова вартість 3 шилінґи, але
величина цієї останньої падає порівняно з вартістю робочої сили,
вимірюваною доконечною працею. Якщо робочий день здовжується
на 4 години — з 12 до 16 годин, то відносні величини додаткової
вартости й вартости робочої сили, додаткової праці й
доконечної праці, лишаються незмінні, але абсолютна величина
додаткової вартости зростає з 3 до 4 шилінґів, абсолютна величина
додаткової праці — з 6 до 8 робочих годин, отже, на 1/3,
або 33 1/3\%. Отже, при зменшенні продуктивної сили праці
і одночасному здовженні робочого дня абсолютна величина
додаткової вартости може лишатись незмінна, тимчасом як її
відносна величина падає; її відносна величина може лишатись
