лися мертвою буквою. «Це факт, що перед законом 1833 р. дітей
і підлітків мучено працею («were worked») цілу ніч, цілий день
або день і ніч ad libitum».\footnote*{
— досхочу. Peд.
} \footnote{
«Reports of Insp. of Fact, for 30 th April 1860», p. 51.
}

Лише від часу фабричного закону 1833 р. — він поширюється
на фабрики бавовни, вовни, льону й шовку — починається нормальний
робочий день для сучасної промисловости. Ніщо так
добре не характеризує дух капіталу, як історія англійського
фабричного законодавства від 1833 до 1864 року!

Закон 1833 р. проголошує, що звичайний робочий день на
фабриці має починатись о пів на шосту ранку й кінчатись о пів
на дев’яту вечора. В межах цього 15-годинного періоду закон
дозволяє вживати праці підлітків (тобто осіб між 13 і 18 роками
життя) у будь-який час дня, припускаючи завжди, що одна й
та сама молода особа працює не більш як 12 годин протягом одного
дня, за винятком деяких спеціяльно передбачених випадків. Відділ
шостий закону постановляє, «що протягом кожного дня кожній
особі, що її робочий час обмежено, мусить призначатися щонайменше
1 1/2 години на їжу». Заборонялося вживати праці дітей
до 9 років, за одним винятком, про який треба буде згадати пізніш;
працю дітей од 9 до 13 років обмежено на 8 годин денно.
Нічна праця, тобто, за цим законом, праця між пів на дев’яту
вечора й пів на шосту ранку, була заборонена для всіх осіб між
9 і 18 роком життя.

Законодавці були такі далекі від бажання посягнути на волю
капіталу висисати дорослу робочу силу або, як вони це називали,
на «волю праці», що вигадали осібну систему, щоб запобігти
таким жахним наслідкам фабричного закону.

«Велике лихо фабричної системи, як її зорганізовано в теперішній
час, — сказано в першому звіті центральної ради комісії
з 25 червня 1833 р., — є в тому, що вона створює доконечність
поширити дитячу працю до крайніх меж робочого дня дорослих.
Однісінький лік на це лихо, не обмежуючи праці дорослих, звідки
могло б постати ще більше лихо, ніж те, що йому треба запобігти,
є, здається, плян завести подвійні зміни дітей». Тим то плян цей
і здійснено під назвою «Relaissystem» («Sustemof Relays»; Relay
по-англійському, як і по-французькому, означає змінювання
поштових коней на різних станціях), так що одну зміну дітей
од 9 до 13 року життя запрягають до роботи, наприклад, від пів
на шосту зранку до пів на другу по півдні, другу — від пів на другу
по півдні до пів на дев’яту вечора й т. д.

Але в нагороду за те, що панове фабриканти якнайнахабніше

щодо його виконання полишено на добру волю «amis du commerce», і це
в країні, де кожна миша є під оком поліції. Лише від 1853 р. в одинодному
департаменті, Département du Nord, бачимо ми оплачуваного урядового
інспектора. Не менш характеристичний для розвитку французького
суспільства взагалі є той факт, що закон Люї-Філіпа до революції 1848 р.
був однісіньким законом у тій французькій законодавчій фабриці, що
своєю сіткою оплітає геть усе.