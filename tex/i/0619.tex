них дружин, які, як слушно зауважує Джемс Стюарт, «всюди
марно наповняли будинки й двори». Хоч королівська влада,
сама продукт буржуазного розвитку, у своєму прагненні абсолютного
суверенітету силоміць прискорювала розпуск тих дружин,
проте вона зовсім не була його єдиною причиною. Сами
великі февдали, що стояли в якнайгострішій опозиції до королівської
влади й парламенту, створили куди численніший пролетаріят,
силоміць зганяючи селян із тієї землі, на яку селяни
мали таке саме февдальне право, як і сами великі февдали, і
узурпуючи їхні громадські землі. Безпосередній поштовх до цього
в Англії дав головним чином розквіт фляндрійської вовняної
мануфактури й відповідний зріст цін на вовну. Стару февдальну
шляхту поглинули февдальиі війни, нова ж була дитиною свого
часу, для якого гроші були силою над силами. Тому гаслом її
стало перетворення орного поля на пасовиська для овець. Геррісон
у своїх «Description of England. Prefixed to Holinshed’s
Chronicles» описує, як експропріація дрібних селян руйнує
країну. «What care our great incroachers!» (Що нашим великим
узурпаторам до того!). Селянські житла й робітничі котеджі
силоміць поруйновано або засуджено на руїну. «Коли порівняти, —
каже Геррісон, — давніші інвентарі кожного лицарського маєтку,
то виявиться, що безліч хат і дрібних селянських господарств
зникло, що земля годує тепер далеко менше людей, що багато
міст занепало, хоч деякі нові міста проквітають... Я міг би багато
чого розповісти про міста й села, які поруйновано задля
овечих пасовиськ і в яких позалишалися самі тільки панські
замки». Нарікання таких старих хронік завжди трохи прибільшені,
— проте вони точно змальовують вражіння, яке революція
в продукційних відносинах справила на самих сучасників.
Порівнюючи твори канцлера Фортескю й Томаса Мора, ми виразно
бачимо ту безодню, що відділяє XV і XVI століття. Із свого
золотого віку англійська робітнича кляса, як слушно каже Торнтон,
без ніяких переходових ступенів попала в залізний.

Законодавство злякалося цього перевороту. Воно ще не
стояло на тій височині цивілізації, де «Wealth of the Nation»,\footnote*{
— національне багатство. \emph{Ред.}
}
тобто утворення капіталу й нещадна експлуатація та павперизація
народньої маси, вважається за ultima Thule\footnote*{
— вершину. \emph{Ред.}
} всякої державної
мудрости. У своїй історії Генріха VII Бекон каже: «Того
часу (1489) збільшилися нарікання на перетворення орного поля
в пасовиська (для овець тощо), за якими легко може доглядати
кілька чабанів; а фарми, що здавалися в доживотну оренду, на декілька
років або на рік (з чого жила велика частина yeomen’ів\footnote*{
— вільних рільників. \emph{Ред.}
})
перетворено на панські маєтки. Це привело до занепаду народу,
а через це й до занепаду міст, церков, десятин... У лікуванні
цього лиха король і парлямент виявили в той час мудрість, гідну
подиву... Вони вжили заходів проти цієї узурпації громадських