\index{i}{0586}  %% посилання на сторінку оригінального видання 
Загальний результат досліджень лікарської комісії 1863 р.
про стан харчування гірше харчованих кляс народу вже відомий
читачеві. Читач пригадує собі, що харчі більшої частини родин
сільських робітників стоять нижче мінімальної міри, потрібної,
«щоб забезпечити себе від хороб у наслідок голоду». Так стоїть
справа особливо по всіх суто рільничих округах — Cornwall, Devon
Somerset, Witts, Stafford, Oxford, Berks і Herts. «Кількість
харчів, що їх дістає сільський робітник, — каже д-р Сімон, —
більша, ніж показує пересічна кількість, бо сам він дістає куди
більшу частину засобів існування, доконечну для його праці,
аніж інші члени його родини; в найбідніших округах він дістає
майже все м’ясо або сало. Та кількість харчів, що припадає на
жінку, а так само й на дітей у період їхнього швидкого зросту,
в багатьох випадках і майже по всіх графствах недостатня, особливо
щодо азоту».\footnote{
«Public Health. Sixth Report 1863», p. 238, 249, 261,262.
} Наймити й наймички, що живуть у самих
фармерів, харчуються добре. Число їх із 288.277 в 1851 р. спало
до 204.962 в 1861 р. «Праця жінок на полі, — каже д-р Сміс, —
хоч би й якими взагалі шкідливими наслідками вона супроводилася,
за сучасних обставин є дуже корисна для родини, бо дає
їй засоби на взуття, одяг, квартирну плату, і таким чином змогу
краще харчуватись».\footnote{
Там же, стор. 262.
} Одним із найвизначніших результатів
цього дослідження було виявлення того факту, що сільський
робітник в Англії харчується куди гірше, ніж в інших частинах
Сполученого королівства («is considerably the worst fed»), як
це видно з нижченаведеної таблиці.

Тижневе споживання вуглецю й азоту пересічно
на одного сільського робітника
                                                                                Вуглецю             
             Азоту
                                                                                (грани)             
               (грани)
Англія.....................                                           40,673                        
     1,594
Велз........................                                           48,354                       
       2,031
Шотландія..............                                            48,980                           
   2,348
Ірландія.................                                             43,366                        
     2,439 \footnote{
Там же, стор. 17. Англійський сільський робітник дістає лише
четвертину тієї кількости молока й лише половину тієї кількости хліба,
яку дістає ірляндський сільський робітник. Кращі умови харчування
ірландського сільського робітника відзначив уже А. Юнґ на початку
цього століття в своїй «Tour through Ireland». Причина цього та, що бідний
ірляндський фармер куди гуманніший, ніж багатий англійський.
А щодо Велзу, то наведені в тексті дані не стосуються до його південнозахідньої
частини. «Всі тамошні лікарі згоджуються на тому, що збільшення
проценту смертности від туберкульози, золотухи й т. ін. інтенсивно
вростає з погіршанням фізичного стану людности, і це погіршання всі
приписують злидням. Денне утримання сільського робітника обчислюють
там у 5 пенсів, у багатьох округах фармер (що й сам бідує) платить іще
менше. Шматок засоленого м’яса, сухий, як тверде червоне дерево, і ледве
}
\index{i}{0587}  %% посилання на сторінку оригінального видання 
«Кожна сторінка звіту д-ра Гентера, — каже д-р Сімон
у своєму офіціяльному санітарному звіті, — свідчить про недостатню
кількість і злиденну якість помешкань нашого сільського
робітника. І ось уже багато років, як стан його з цього боку проґресивно
гіршає. Тепер сільському робітникові куди важче підшукати
помешкання, а коли й підшукає, то воно куди менше
відповідає його потребам, аніж це було, може, декілька століть
тому... Особливо швидко зростає це лихо протягом останніх 30
або 20 років, і житлові умови селянина тепер надзвичайно сумні.
Він тут цілком безпорадний, хібащо ті, кого він збагачує своєю
працею, захочуть завдати собі клопоту поводитися з ним із певного
роду жалісливістю та ласкавістю. Чи найде він житло на
тій землі, яку обробляє, чи буде те житло придатне для людей
чи лише для свиней, чи буде при ньому невеличкий садок, що так
полегшує гніт злиднів, — усе це залежить не від його готовости

чи вартий важкого процесу травлення, або шматок сала є за приправу до
великої кількости юшки з борошна й цибулі або до вівсянки, і це деньу-день
становить обід сільського робітника... Проґрес промисловости
для нього мав такі наслідки, що дебеле домотканне сукно витиснено в
цьому суворому й вогкому підсонні дешевими бавовняними тканинами,
а міцніші напої — «номінальним» чаєм... Після багатьох годин перебування
на вітрі й дощі рільник вертається до свого котеджу, щоб присісти
біля печі, де горить торф або кавалки, збиті з глини й покидьків кам’яного
вугілля, що, згораючи, виділюють цілі хмари вуглекислоти й сульфатної
кислоти. Стіни хатини пороблено з глини й каменю, долівка —
гола земля, що була тут і перед будуванням хатини, дах — маса понакидуваної,
непошитої соломи. Кожну щілину заткнуто, щоб не виходило
тепло, і в цій атмосфері диявольського смороду, на брудній землі, часто
висушуючи на своєму тілі свою однісіньку одіж, він сідає вечеряти а
дружиною й дітьми. Акушери, примушені проводити частину ночі в цих
хатах, описували, як їхні ноги грузли у брудній земляній долівці і як
їм доводилося — легенька собі справа! — продовбувати дірку в стіні, щоб
здобути собі хоч трохи свіжого повітря. Численні свідки різного ранґу
свідчать, що недосить харчований (underfed) селянин кожної ночі зазнає
цих і інших шкідливих для його здоров’я впливів; результат цього —
квола й золотушна людність, про це справді маємо більш ніж досить
доказів... Повідомлення парафіяльних урядовців у Caermarthenshire і Cardiganshire
виразно потверджують такий самий стан речей. Сюди треба
додати ще більше лихо — поширення ідіотизму. А тепер ще декілька слів
про кліматичні умови. Буйні південно-західні вітри пронизують усю
країну вісім-дев’ять місяців на рік, вони навівають страшні зливи,
що спадають переважно на західніх схилах гір. Дерева трапляються
рідко, хіба лише по затишних місцях; там, де вони не захищені, їх нищить
вітер. Хатини туляться під гірськими терасами, часто по ярах або
каменярнях; лише найдрібніша порода овець і місцева рогата худоба
можуть жити на таких пасовиськах... Молодь еміґрує до східніх гірничих
округ Glamorgan і Monmouth... Caermarthenshire — це розсадник
шахтарів і їхній інвалідний дім... Людність ледве-ледве підтримує
свою чисельність на тому самому рівні. Так, у Cardiganshire було:

                                                                               1851р.               
        1861р.

Чоловічої статі.......                                          45.155                       44.446

Жіночої статі.........                                           22.459                      52.955

                                                                                                    
  97.614 97.401

(Звіт д-ра Гентера в «Public Health. Seventh Report 1864», London
1865, p. 498—502 passim.).
\parbreak{}  %% абзац продовжується на наступній сторінці
