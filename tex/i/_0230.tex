\parcont{}  %% абзац починається на попередній сторінці
\index{i}{0230}  %% посилання на сторінку оригінального видання
клясовий антагонізм дійшов неймовірного напруження. Навіть
частина фабрикантів ремствувала: «Через суперечні судові
вироки панує цілком ненормальний анархічний стан. Один закон
має силу в Йоркшірі, другий — у Ланкашірі, третій — у якійсь
парафії Ланкашіру, четвертий — у його безпосереднім сусідстві.
Фабрикант у великих містах має змогу обійти закон, фабрикант
по містечках не находить персоналу, потрібного для системи змін,
а ще менше для перекидування робітників з однієї фабрики на
одну й т. ін.». А рівність в експлуатації робочої сили є перше
людське право капіталу.

За цих обставин стався компроміс між фабрикантами й робітниками,
який набув санкції парляменту в новому додатковому
фабричному законі з 5 серпня 1850 р. Робочий день «підлітків
і жінок» збільшено для перших 5 днів тижня з 10 до 10\sfrac{1}{2} годин
і обмежено 7\sfrac{1}{2} годинами у суботу. Праця мусіла відбуватися від
6 години ранку до 6 години вечора\footnote{
Взимку цей період можна було заміняти періодом від 7 години
ранку по 7 години вечора.
} з півторагодинними перервами
на їжу, які треба давати одночасно, згідно з постановами
1844 р. і т. ін. Таким чином раз назавжди покладено край
Relaissystem’i.\footnote{
«Теперішній закон (з 1850 р.) був компромісом: робітники зреклися
вигід закону про десятигодинну працю в заміну за вигоду одночасного
початку й закінчення роботи тих, що їх праця була обмежена» («То
present law (of 1850) was a compromise whereby the employed surrendered
the benefit of the Ten Hours’ Act for the adventage of one uniform period
for the commencement and termination of the labour of those whose labour
is restricted»). («Reports etc. for 30 th April 1852», p. 14).
} Для дитячої праці лишився в силі закон з
1844 р.

Одна категорія фабрикантів забезпечила собі цього разу,
як і раніш, осібні сеньйоріяльні права на пролетарські діти. Це
були власники фабрик шовку. 1833 р. вони погрозливо рикали,
що «коли віднімуть у них волю виснажати дітей усякого віку
протягом десяти годин на день, то тим спинять їхні фабрики»
(«if the liberty of working children of any age for 10 hours a day
was taken away, it mould stop their works»). Вони, мовляв, не
мають змоги купити достатнього числа дітей, старших за 13 років.
Вони вимусили для себе бажаний привілей. Ця причіпка за пізнішого
розсліду виявилася чистою брехнею,\footnote{
«Reports etc. for 30 th Sept. 1844», p. 13.
} та це не заважало
їм ціле десятиліття по 10 годин денно прясти шовк з крови
маленьких дітей, що їх для виконування їхньої роботи доводилося
ставити на стільці.\footnote{
Там же.
} Хоч закон року 1844 і «відняв» у них
«волю» примушувати дітей молодших від 11 років працювати
довше ніж 6\sfrac{1}{2} годин денно, але зате він забезпечив їм привілей
експлуатувати дітей 10—13 років по 10 годин денно і скасував
примусове шкільне навчання, приписане для інших фабричних
дітей. Цим разом висунуто іншу причіпку: «Делікатність тканини
потребує тендітности пальців, що її можна досягти, тільки змалку
\parbreak{}  %% абзац продовжується на наступній сторінці
