вання залізниць. Навпаки, централізація за допомогою акційних
товариств досягла цього наче одним махом руки. Збільшуючи
і прискорюючи таким чином діяння акумуляції, централізація
одночасно поширює і прискорює ті перевороти в технічному
складі капіталу, що збільшують його сталу частину коштом його
змінної частини й тим зменшують відносний попит на працю.

Маси капіталу, що їх миттю збиває до купи централізація,
репродукуються та збільшуються так само, як і інші капітали,
тільки швидше, і таким чином вони стають новими могутніми
підоймами суспільної акумуляції. Отже, коли говорять про прогрес
суспільної акумуляції, то під нею за наших часів мовчки
розуміють і діяння централізацій.

Додаткові капітали, утворені в перебігу нормальної акумуляції
(див. розділ XXII, І), служать переважно як засоби
експлуатації нових винаходів, відкрить тощо, одним словом,
промислових удосконалень. Але з часом і для старого капіталу
приходить момент відновлення його голови й членів, момент,
коли він змінює свою шкуру й теж відроджується в такій удосконаленій
технічній формі, коли досить меншої маси праці,
щоб пускати в рух більшу масу машин і сировинних матеріялів.
Абсолютне зменшення попиту на працю, що звідси неминуче
випливає, буде, ясна річ, то більше, що більше в наслідок руху
централізації є вже нагромаджені великими масами капітали,
які пророблюють цей процес відновлення.*

Отже, з одного боку, додатковий капітал, що утворився в
розвитку акумуляції, притягує порівняно з своєю величиною

* Наводимо тут переклад цього абзацу за другим німецьким виданням,
де його подано повніше: «Зростання розміру індивідуальних мас
капіталу стає матеріяльною базою постійного перевороту в самому способі
продукції. Капіталістичний спосіб продукції безупинно завойовує
такі галузі праці, що зовсім ще не підпорядковані йому або підпорядковані
лише спорадично або лише формально. Поруч цього на ґрунті
цього ж способу продукції виростають нові галузі праці, що а самого початку
належать до нього. Нарешті, в галузях праці, проваджуваних
уже капіталістично, продуктивна сила праці виростає, наче в теплиці.
В усіх цих випадках число робітників знижується порівняно до маси
оброблюваних ними засобів продукції. Чимраз більша частина капіталу
перетворюється на засоби продукції, чимраз менша — на робочу силу.
Разом із збільшенням розмірів, концентрації і технічного діяння
засобів продукції, проґресивно зменшується їхня роля як засобів, що
дають заняття робітникам. Паровий плуг куди ефективніший засіб продукції,
ніж звичайний плуг, але вмішена в ньому капітальна вартість
куди меншою мірою є засіб, що дає заняття робітникам, ніж коли б її
було зреалізовано в звичайному плузі. Спочатку якраз долучення нового
капіталу до старого дозволяє поширити і технічно зреволюціонізувати
речові умови процесу продукції. Але незабаром зміна складу і технічна
перебудова більшою або меншою мірою охоплює ввесь старий капітал,
для якого настав строк репродукції і який тому наново репродукується.
Ця метаморфоза старого капіталу до певної міри так само не залежить від
абсолютного зростання суспільного капіталу, як і централізація. Але ця
остання, що лише інакше розподіляє наявний суспільний капітал і з’єднує
багато старих капіталів в один капітал, і собі діє як потужний чинник
у цій метаморфозі старого капіталу». Ред.
