\parcont{}  %% абзац починається на попередній сторінці
\index{i}{0400}  %% посилання на сторінку оригінального видання
звички самих робітників, особливо там, де панує відштучна плата
і де прогул частини дня або тижня можна поповнити пізнішою
надмірною або нічною працею — метода, що затуркує дорослого
робітника й цілком виснажує сили його товаришів, підлітків та
жінок\footnote{
При домнах, наприклад, «наприкінці тижня праця звичайно
триває значно довше через звичку робітників понеділкувати, а іноді не
працювати цілий вівторок або частину його» («\dots{} work towards the
end of the week is generally much increased in duration, in consequence
of the habit of the men of idling on Monday and occasionally during a part
or the whole of Tuesday also»). («Children’s Employment Commission.
3 rd Report», p. VI). «Дрібні майстрі, звичайно, працюють дуже нерегулярно.
Вони прогаюють 2--3 дні, а потім, щоб надолужити це, працюють
цілими ночами\dots{} Вони завжди вживають до праці своїх дітей, якщо вони
є в них» («The little masters generally have very irregular hours. They
lose 2 or 3 days, and then work all night to make it up\dots{} They always employ
their own children if they have any»). (Там же, стор. VII). «Нерегулярності
в починанні роботи сприяє можливість і звичка надолужувати прогаяне,
збільшивши години праці» («The want of regularity in coming to work,
encouraged by the possibility and practice of making up for this by working
longer hours»). (Там же, стор. XVIII). «Величезна втрата часу в Бермінгемі\dots{} даремне гайнування
однієї частини часу, праця до знемоги — протягом другої» («Enormous loss of time in Birmingham\dots{}
idling part of
the time, slaving the rest»). (Там же, стор. XI).
}. Хоч ця безладність у витрачуванні робочої сили є
природна груба реакція проти нудоти монотонної болісної
праці, однак, у незрівнянно більшій мірі вона випливає з тієї
анархії самої продукції, що з свого боку знов таки має за передумову необмежену експлуатацію робочої
сили капіталом. Поряд
загальних періодичних змін промислового циклу й окремих ринкових коливань у кожній галузі продукції
є ще так званий сезон — усе одно, чи залежить він від періодичности сприятливих
для судноходства пір року, чи від моди — і раптові великі замовлення, що їх треба виконати в
якнайкоротший час. Звичка до
таких замовлень поширюється разом із залізничою й телеграфною справою. «Поширення залізничної
системи, — каже, наприклад, один лондонський фабрикант, — по всій країні дуже сприяло
звичці робити короткотермінові замовлення; покупці приїжджають тепер із Ґлезґо, Менчестеру й
Едінбурґу раз на 14 день
для гуртового закупу у великих крамницях Сіті, куди ми постачаємо товари. Замість купувати на
складах, як це звичайно робилося раніш, вони дають замовлення, які треба негайно виконати.
Попередніми роками ми завжди мали змогу протягом часу
слабого попиту працювати наперед для задоволення попиту найближчого сезону, але тепер ніхто не може
заздалегідь сказати,
на що саме тоді буде попит»\footnote{
«Children’s Employment Commission, 4 th Report», p. XXXII,
XXXIII. «Поширення залізничної системи дуже, кажуть, сприяло цій
звичці робити короткотермінові замовлення, а це призвело до похапливости в роботі, занедбування часу
їжі та до надмірної нічної праці»
(«The extension of the railway system is said to have contributed greatly
to this custom of giving sudden orders, and the consequent hurry, neglect
of mealtimes, and late hours of the workpeople»). (Там же, стор. XXXI).
}.
