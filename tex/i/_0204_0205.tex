\parcont{}  %% абзац починається на попередній сторінці
\index{i}{0204}  %% посилання на сторінку оригінального видання
тільки переривається їхня функція вбирати працю. «Але тоді
була б утрата на дуже коштовних машинах, що половину часу
стояли б без діла, і ми мусили б подвоїти розмір будівель і число
машин для того, щоб виробити таку масу продуктів, яку ми можемо
виробити за теперішньої системи, а це подвоїло б видатки». Але
чому саме ці Сандерсони вимагають привілеїв супроти інших
капіталістів, яким дозволено працювати лише вдень і в яких,
отже, будівлі, машини, сировинний матеріял вночі «лежать без
діла»? «Це правда, — відповідає Е. Ф. Сандерсон за всіх Сандерсонів,
— це правда, що ця втрата від припинення вночі машин
стається по всіх мануфактурах, де працюють лише вдень. Але
вживання топильних печей спричинило б у нашому випадку
екстраординарну втрату. Коли тримати їх у русі напоготові, тоді
марнується паливо (замість життя робітників, яке марнується тепер),
коли ж припинити їхній рух, тоді марнується час на розпалювання
й добування потрібної температури (тимчасом як втрата
часу, потрібного на сон, навіть у восьмилітніх дітей є виграш
часу для всієї сандерсонівської братії), та й сами печі зазнали
б шкоди від зміни температури» (тимчасом як ті самі печі
нічого не терплять від денної й нічної зміни праці).\footnote{
«Children’s Employment Commission. 4 th Report etc.», 1865,
p. 84. Подібним делікатним способом пани фабриканти скла твердили, що
призначення «реґулярного часу на обід» для дітей є неможливе, бо це
призвело б до «чистої втрати» й «марнотратства» певної кількости тепла,
яке дають печі; на ці міркування слідчий комісар Вайт, зовсім не схожий
на Юра, Сеніора і інших та їхніх обмежених німецьких підбрехачів à la
Рошер та іншіх, що зворушені «поздержливістю», «самовідреченням» і
«ощадністю» капіталістів при витраті своїх грошей і їхнім тімур-тамерланським
«марнотратством» людського життя, — відповідає так: «Можливо,
що в наслідок забезпечення регулярного часу для їжі і витрачатиметься
марно певну кількість тепла проти теперішньої, але ця витрата,
навіть виражена в грошовій вартості, є ніщо порівняно з марнотратством
життєвої сили («the waste of animal power»), що його зазнає тепер королівство
через те, що діти, які працюють на гутах і перебувають у періоді
зросту, не мають вільного часу, щоб спокійно попоїсти й перетравити
їжу». (Там же, p. XLV.). І це в «рік проґресу», в рік 1865! Залишаючи
осторонь витрату сили при підійманні й переношуванні тягара, така
дитина мусить, на гутах, де виробляється пляшки й кремінне скло (Flintglas),
підчас своєї безперервної праці зробити протягом 6 годин 15-16
миль (англійських)! А праця часто триває 14-15 годин! По багатьох
таких гутах, як от і на московських прядільнях, панує система шестигодинних
змін. «Протягом тижневого робочого часу найдовший безперервний
відпочинок триває 6 годин, але звідси треба відлічити час, потрібний на
те, щоб дійти до фабрики й назад, щоб умитися, одягтися, поїсти, на що
теж треба часу. Таким чином у дійсності лишається на відпочинок тільки
якнайкоротший час. Немає часу погратися, подихати свіжим повітрям,
хіба що коштом сну, що так дуже потрібний дітям, які за такої спеки
виконують таку напружену працю\dots{} Навіть цей короткий сон, і той переривається,
бо дитина мусить сама просипатися вночі або прокидатися вдень
від зовнішнього гуркоту». Пан Вайт наводить випадки, коли один підліток
працював 36 годин без якої-будь перерви, коли дванадцятилітні хлоп’ята
мучаються до 2 години вночі, а потім сплять на гуті до 5 години ранку
(3 години!), щоб знову розпочати денну працю! «Кількість праці, — кажуть
редактори загального звіту Тременгір і Тефнель, — що її виконують
хлопчаки, дівчата й жінки протягом денної або нічної зміни праці («spell of
labour»), просто неймовірна». (Там же, стор. XLIII і XLIV). Тимчасом
«повний самовідречення» капітал скляної промисловості, повертаючись
пізно вночі з клюбу додому й похитуючись від портвайну, по-ідіотичному
мугикає собі під носом: «Britons never, never shall be slaves!»\footnote*{
Ніколи, ніколи британці не будуть рабами. \emph{Ред.}}
}

\index{i}{0205}  %% посилання на сторінку оригінального видання
\section*{5. Боротьба за нормальний робочий день. Примусові закони про
здовження робочого дня від середини XIV до кінця XVII віку}

«Що таке робочий день?» Який великий той час, що протягом
його капітал може споживати робочу силу, денну вартість якої
він оплачує? Як далеко можна здовжувати робочий день поза
межі робочого часу, доконечного для репродукції самої робочої
сили? На ці запити, як ми бачили, капітал відповідає: робочий
день налічує повних 24 години на добу, за винятком небагатьох
годин відпочинку, без якого робоча сила абсолютно не в стані
відновити свою службу. Насамперед само собою зрозуміло, що
робітник ціле своє життя є не що інше, як тільки робоча сила,
і що тому цілий вільний його час з природи й на основі права є
робочий час, отже, належить процесові самозростання капіталу.
Час, потрібний для освіти людини, для її інтелектуального розвитку,
для виконання соціальних функцій, для стосунків з приятелями,
для вільної гри фізичних і інтелектуальних життєвих
сил, навіть для святкування неділі, хоча б і в країні святкувальників
суботи\footnote{
Приміром, в Англії ще й тепер подекуди засуджують на ув’язнення
робітника за те, що він, працюючи в садку перед своїм домом, порушує
святість суботи. Цей самий робітник дістає кару за зламання контракту,
коли не піде в неділю, хоча б і з релігійних мотивів, на фабрику металю,
паперу або скла. Ортодоксальний парлямент глухий на зневажання святости
суботи, коли воно трапляється у «процесі зростання вартости»
капіталу. В одному меморіялі (серпень 1863 р.), де лондонські поденники
з крамниць, що торгували рибою і птицею, вимагають скасувати недільну
працю, сказано, що їхня праця триває протягом перших 6 днів тижня
пересічно по 15 годин щоденно, а в неділю — 8--10 годин. Із того ж меморіялу
видно, що до цієї «недільної праці» заохочує саме вибагливо
витончена обжерливість аристократичних лицемірів із Exeter Hall.
Ці «святуни», такі ревні «in cute curanda»,\footnote*{
— в піклуванні про себе. \emph{Ред.}
} виявляють свою християнську
душу в тій покорі, з якою зносять надмірну працю, злидні й голод
третіх осіб. Obsequium ventris istis (робітникам) perniciosus est.\footnote*{
Догоджати череву є для них (робітників) згубна річ. \emph{Ред.}
}
} — все це є чиста нісенітниця! Але в своєму безмірному,
сліпому прагненні, у своїй вовчій ненажерливій жадобі
до додаткової праці, капітал переступає не лише моральні,
але й суто фізичні максимальні межі робочого дня. Він узурпує
час, потрібний для зросту, розвитку й здорового збереження
тіла. Він грабує час, потрібний на споживання свіжого повітря
й сонячного світла. Він зменшує час на їжу і по змозі приєднує
його до самого процесу продукції, так що харч додається робітникові
як простому засобові продукції, як паровому казанові —
вугілля і машинам — мастиво або олію. Здоровий сон, потрібний,
щоб акумулювати, поновити й відсвіжити життєву силу, капітал
зводить на стільки годин заціпеніння, скільки неодмінно потрібно,
щоб оживити абсолютно виснажений організм. Не нормальне

\parbreak{}  %% абзац продовжується на наступній сторінці
