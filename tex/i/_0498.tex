\parcont{}  %% абзац починається на попередній сторінці
\index{i}{0498}  %% посилання на сторінку оригінального видання
за свій товар справедливу ціну, всю до останнього шага, і, по-друге,
ця справа взагалі ані трохи не обходить робітника В. Чого вимагає
В, і що має він право вимагати, так це те, щоб капіталіст
платив йому вартість його робочої сили. «І один і другий ще
виграли: робітник — тому, що йому авансовано продукти його
праці [слід би сказати: неоплаченої праці інших робітників]
раніше, ніж ту працю було виконано [слід би сказати: раніше,
ніж його праця створила продукт]; хазяїн — тому, що праця
цього робітника була варта більше, ніж його заробітна плата
[слід би сказати: випродукувала більше вартости, ніж вартість
його заробітної плати]».\footnote*{
«Tous deux gagnaient encore: l'ouvrier parce qu’on lui avançait les
fruits de son travail [слід би сказати: du travail gratuit d’autres ouvriers]
avant qu’il fût fait [слід би сказати: avant que le sien ait porté de fruit],
le maître, parce que le travail de cet ouvrier valait plus que son salaire
[слід би сказати: produisait plus de valeur que celle de son salaire]».
(\emph{Sismondi}, 1. c., p. 135).
}

Правда, справа виглядає цілком інакше, коли ми розглядаємо
капіталістичну продукцію в безперервній течії її поновлення
й замість поодинокого капіталіста й поодинокого робітника
беремо на увагу сукупність, клясу капіталістів і проти
неї клясу робітників. Але тим самим ми приклали б маштаб,
цілком невластивий товаровій продукції.

У товаровій продукції протистоять один одному лише незалежні
один від одного продавець і покупець. їхні взаємовідносини
кінчаються тоді, коли минає строк складеної між ними угоди.
Якщо оборудка відновлюється, то вже в наслідок нової угоди,
що не має нічого спільного з попередньою й лише випадково
зводить знову того самого покупця з тим самим продавцем.

Отже, якщо судити про товарову продукцію або якийсь до
неї належний процес за її власними економічними законами,
то треба кожен акт обміну розглядати сам по собі, поза всяким
зв’язком його з попереднім і наступним актами обміну. А що акти
купівлі й продажу відбуваються лише поміж поодинокими індивідами,
то неприпустимо шукати в них відносин між цілими
суспільними клясами.

Хоч який довгий є ряд послідовних періодичних репродукцій
і попередніх акумуляцій, які проробив той капітал, що сьогодні
функціонує, він завжди зберігає свою первісну незайманість.
Поки в кожному акті обміну, — беручи кожний акт відокремлено, —
зберігаються закони обміну, спосіб присвоювання може зазнати
цілковитого перевороту, не порушуючи ніяк права власности,
відповідного товаровій продукції. Те ж саме право власности
має силу так напочатку, коли продукт належить продуцентові
і коли останній, обмінюючи еквівалент на еквівалент, може багатіти
лише через свою власну працю, як і в капіталістичний період,
коли суспільне багатство в чимраз більшій і більшій мірі
стає власністю тих, що мають змогу постійно знову й знову присвоювати
собі неоплачену працю інших.
