теру робочої сили. Отже, ці витрати навчання дуже малі для
звичайної робочої сили, є складова частина вартостей, витрачених
на її продукцію.

Вартість робочої сили сходить на вартість певної суми засобів
існування. Тому вона змінюється разом зі зміною вартости цих
засобів існування, тобто разом зі зміною величини робочого
часу, потрібного на продукцію їх.

Частину засобів існування, приміром, їжу, паливо тощо, споживається
день-у-день, і мусять вони день-у-день наново поповнюватись.
Інші засоби існування, як от одяг, меблі тощо,
зуживається протягом довшого часу, а тому їх треба поповнювати
протягом довшого часу. Товари одного роду мусять купуватися
або за них треба платити щодня, товари іншого роду —
щотижня, що чверть року й т. ін. Але, хоч і як поділялася б
сума цих видатків, приміром, протягом року, вона мусить покриватися
з пересічного доходу, що його день-у-день дістає робітник.
Коли б маса товарів, потрібних щоденно на продукцію
робочої сили, дорівнювала А, маса товарів потрібних щотижня,
дорівнювала В, маса товарів, потрібних що чверть року, дорівнювала
С і т. д., то щоденна пересічна кількість цих товарів
дорівнювала б

(365А + 52В + 4С + і т. д.) : 365

Коли припустимо, що в цій масі товарів, потрібній для пересічного
дня, міститься 6 годин суспільної праці, то в робочій силі
щодня упредметнюється півдня суспільної пересічної праці, тобто
потрібно пів робочого дня на щоденну продукцію робочої сили.
Ця кількість праці, потрібна на щоденну продукцію робочої
сили, становить її денну вартість, або вартість щоденно репродукованої
робочої сили. Коли півдня пересічної суспільної праці
виражається в масі золота в 3 шилінґи або один таляр, то один
таляр є ціна, що відповідає денній вартості робочої сили. Якщо
посідач робочої сили щоденно продає її за один таляр, то її продажна
ціна дорівнює її вартості і, за нашим припущенням, посідач
грошей, що жадає перетворити свої таляри на капітал, платить
цю вартість.

Крайню або мінімальну межу вартости робочої сили становить
вартість маси тих товарів, що без її щоденного постачання носій
робочої сили, людина, не може відновляти свого життєвого
процесу, тобто вартість фізично доконечних засобів існування.
Коли ціна робочої сили падає до цього мінімуму, то вона падає
нижче від її вартости, бо в такому разі робоча сила може утримуватись
і розвиватись лише в занепалій формі. Але вартість кожного
товару визначається тим робочим часом, що його треба,
щоб постачати товар нормальної якости.

Було б надзвичайно дешевенькою сантиментальністю вважати
за грубе це визначення вартости робочої сили, яке випливає з
