Розвиток грошей як засобу платежу примушує нагромаджувати
гроші, потрібні на терміни платежу позичених сум. Тимчасом
як скарботворення, як самостійна форма збагачення, зникає
разом із проґресом буржуазного суспільства, воно, навпаки,
зростає разом з ним у формі резервного фонду засобів платежу.

с) Світові гроші

Виходячи поза межі внутрішньої сфери циркуляції, гроші
знову скидають із себе придбані там місцеві форми маштабу
цін, монети, розмінної монети і знаків вартости — і повертаються
назад до своєї первісної форми зливків благородних
металів. У світовій торговлі товари розгортають універсально
свою вартість. Тому тут їхня самостійна форма вартости і протистоїть
їм як світові гроші. Лише на світовому ринку гроші функціонують
у повному обсягу як товар, натуральна форма якого
є разом з тим безпосередня суспільна форма здійснення людської
праці in abstracto. Спосіб буття їх стає адекватним їх поняттю.

У сфері внутрішньої циркуляції лише один товар може служити
за міру вартости, отже, і за гроші. На світовому ринку
панує подвійна міра вартости — золото й срібло. 108

108 Звідси зрозуміле безглуздя кожного законодавства, що приписує
національним банкам нагромаджувати лише той благородний металь,
що всередині країни функціонує як гроші. Відомі, наприклад, ті «любі
перешкоди», які таким чином Англійський Банк створив собі сам. Про
великі історичні епохи у зміні відносної вартости золота й срібла див.
Карл Маркс: «Zur Kritik der Politischen Oekonomie», S. 136 і далі. («До
критики і т. д.», ДВУ, 1926 р. стор. 170 і далі). — Додаток до другого видання:
Сер Роберт Пілл у своєму банковому акті з 1844 р. силкувався зарадити
цьому лихові тим, що дозволив Англійському Банкові випустити банкноти
під забезпечення срібними зливками, однак так, щоб запас срібла
не був ніколи більший за четвертину запасу золота. При тому вартість
срібла цінувалося за його ринковою ціною (в золоті) на лондонському
ринку. [До четвертого видання. — Ми живемо знову в добу сильної зміни
відносної вартости золота й срібла. Приблизно перед 25 роками вартостеве
відношення золота до срібла рівнялось 15 1/2: 1, тепер* воно приблизно
дорівнює 22: 1, і срібло ще далі падає проти золота. Це є, головним чином,
наслідок перевороту в способах продукції обох металів. Раніш золото
видобувалось майже виключно через промивання алювіальних
верств, що мали в собі золото, тобто через промивання продуктів вивітрювання
золотодайних мінералів. Тепер цієї методи вже не вистачає,
і цю методу відсунуло на задній плян розроблення самих золотодайних
жил кварцу, що, правда, вже було добре відоме за старовини (Діодор, III,
12—14), але посідало раніше лише другорядне місце. З другого боку, не
тільки відкрито величезні нові поклади срібла в Скелястих горах Західньої
Америки, але проведено до них і до мехіканських копалень срібла
залізниці, уможливлено довіз туди новітніх машин і палива, а через це
й видобування срібла в найширшому маштабі і з меншими витратами.
Але є велика ріжниця в тому, як обидва металі трапляються в рудних
жилах. Золото здебільшого трапляється в суцільному вигляді, але зате воно
порозкидане у кварці в незначнісіньких кількостях; тому цілу жилову
породу треба потовкти й золото вимити або витягти за допомогою живого
срібла. На 1 мільйон грамів кварцу часто припадає тоді ледве 1—3,
* 1890 р. — рік виготовлення 4 видання. Ред.
