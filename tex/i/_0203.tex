\parcont{}  %% абзац починається на попередній сторінці
\index{i}{0203}  %% посилання на сторінку оригінального видання
собою заміна дитячої праці працею дорослих. Скільки це становило
б, я не можу сказати, але, можливо, не так багато, щоб фабрикант
зміг підвищити ціну сталі; отже, втрата спала б на нього,
бо робітники (який упертий нарід!) звичайно, відмовилися б
понести ці втрати». Пан Сандерсон не знає, скільки платить він
дітям, але, «мабуть, це становить від 4 до 5\shil{ шилінґів} на душу за
тиждень\dots{} Хлоп’яча робота є така, що для неї взагалі («generaly»,
звичайно, «в окремих випадках» не завжди) цілком досить
сили підлітків, отже, з більшої сили дорослих не було б жодної
вигоди, щоб компенсувати втрати, або була б така вигода тільки
не в багатьох випадках, коли дуже важкий металь. Дорослим
було б теж не дуже приємно не мати під собою хлопчаків, бо дорослі
менш слухняні. Крім того, підлітки мусять починати працю
в молодому віці, щоб вивчитись ремества. Коли б працю підлітків
обмежити лише денною працею, то цієї мети осягти не можна
було б». Але чому так? Чому підлітки не могли б учитися свого
ремества вдень? Ваші докази? «Тому, що в наслідок цього дорослі
робітники, які працюють навпереміну один тиждень удень,
другий уночі, відокремлені протягом цього часу від підлітків
своєї зміни, втратили б половину тієї вигоди, яку вони з них
мають. Бож навчання, яке вони дають підліткам, вважається за
частину заробітної плати тих підлітків, і це дає дорослим змогу
дешевше діставати працю підлітків. Кожний дорослий робітник
утратив би половину свого зиску». (Іншими словами, панове
Сандерсон мусили б оплачувати якусь частину заробітної плати
дорослих робітників із власної кишені замість оплачувати її
нічною працею підлітків. Зиск панів Сандерсонів за тієї нагоди
трохи зменшився б, а це саме й є для Сандерсонів достатня підстава,
чому підлітки не могли б вивчитися свого ремества вдень)\footnote{
«За наших часів, багатих на рефлекси й резонування, недалеко
піде той, хто не зможе навести достатніх причин на все, навіть на найгірше
й найнедоречніше. Все, що на світі зіпсоване, зіпсувалося з достатніх
причин». (\emph{Hegel}: «Enzyklopädie. Erster Teil: Die Logik», Berlin
1840, S. 249).
}.
Крім того, це навантажило б реґулярну нічну працю на плечі
дорослих, яких тепер зміняють підлітки, і вони не витримали б
цього. Коротко кажучи, труднощі були б такі великі, що призвели б
мабуть, до цілковитого припинення нічної праці. «Щождо продукції
самої сталі, — каже Е.~Ф.~Сандерсон, — то це не становило б,
ані і найменшої ріжниці, але!». Але панове Сандерсони мають
щось більше робити, ніж виробляти сталь. Продукція сталі, —
лише привід для продукції додаткової вартости. Домни, вальцювальні
тощо, будівлі, машини, залізо, вугілля й~\abbr{т. ін.} мають
щось більше робити, ніж перетворюватись у сталь. Вони існують
для того, щоб вбирати в себе додаткову працю, і вбирають вони
її, звичайно, за 24 години більш, ніж за 12. Вони на основі божих
і людських законів дійсно дають Сандерсонам право на робочий
час певного числа рук протягом повних 24 годин і втрачають свій
характер капіталу, тобто є чиста втрата для Сандерсонів, скоро
\parbreak{}  %% абзац продовжується на наступній сторінці
