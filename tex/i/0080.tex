Отже золота монета й золото у зливках насамперед відрізняються
між собою лише виглядом, і золото завжди можна перетворювати
з однієї форми на іншу.81 Але шлях із монетарні є разом
із тим шлях до топильного тиґля. В обігу золоті монети стираються
одна більше, інша менше. Назва золотої монети й субстанція
золота, номінальний зміст і реальний зміст починають роз’єднуватись.
Однойменні золоті монети стають монетами неоднакової
вартости, бо вага їхня стає неоднакова. Золото як засіб циркуляції
відхиляється від золота як маштабу цін, і разом з тим перестає
бути дійсним еквівалентом товарів, що їхні ціни воно реалізує.
Історія цієї плутанини становить історію монети середньовіччя
й нових віків аж до XVIII століття. Природну тенденцію процесу
циркуляції перетворювати золоту суть (Goldsein) монети на
уявлюване золото, або монету на символ її офіціяльного металевого
змісту, признають навіть найновітніші закони, які визначають
той ступінь утрати металю, за якого золота монета стає нездатною
до обігу, або демонетизується.

Коли самий грошовий обіг відокремлює реальний зміст монети
від номінального змісту, її металеве буття від її функціонального

економії, що не відзначаються, ані обсягом, ані змістовністю, всі вмістилися
в цілому і з подробицями в його брошурці «Some Unsettled Questions
of Political Economy», яка появилася 1884 p. Льокк просто говорить
про зв’язок між відсутністю вартости у золота й срібла і визначенням
їхньої вартости за допомогою кількости: «Люди погодились на тому, щоб
надавати золоту й сріблу уявлюваної вартости... внутрішня вартість цих
металів є не що інше, як їхня кількість» («Mankind having consented
to put an imaginary value upon gold and silver... the intrinsic value, regarded
in these metals, is nothing but the quantity»). («Some Considerations
on the Consequences ot the Lowering of Interest», 1691, Works, ed. 1777,
vol. 2, p. 15).

81    Звичайно, що розгляд подробиць, як от монетний податок і т. ін.,
лежить цілком поза межами моєї мети. Щождо романтичного сикофанта
Адама Міллера, який захоплюється «великодушною ліберальністю», з
якою «англійський уряд даром карбує монету», то я наведу лише міркування
сера Дедлей Норта: «Срібло та золото, як і інші товари, мають свої
припливи й відпливи. Після одержання певної кількости срібла з Еспанії...
його відсилається до Таверу і там карбується. Швидко після цього постає
попит на зливки для вивозу. Коли зливків немає, коли ввесь металь пішов
на карбування монет, що тоді робити? Очевидно, знов перетопити монету
на зливки. Тут не постає жодної втрати, бо карбування нічого не коштує
власникам металю. Таким чином нація має збитки й мусить викидати
свої гроші на вітер. Коли б за карбування довелося купцеві платити
(Норт сам був одним із найбільших купців за часів Карла 2), то він не
посилав би своє срібло до Таверу, не зваживши всіх обставин, і викарбована
монета завжди мала б вищу вартість, ніж срібло у зливках». («Silver
and gold, like other commodities, have their ebbings and flowings. Upon
the arrival of quantities fromm Spain.... it is carried into the Tower, and
coined. Not long after there will come a demand for bullion, to be exported
again. If there is none, but all happens to be in coin, what then? Melt
it down again; there’s no loss in it, for the coining costs the owner nothing.
Thus the nation has been abused, and made to pay for the twisting of
straw, for asses to cat. If the merchant had to pay the price of the coinage,
he would not have sent his silver to the Tower without consideration; and
coined money would always keep a value above uncoined silver»). (North:
«Discourses upon Trade», London 1691, p. 18).
