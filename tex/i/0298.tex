of Nations» і т. ін. Він не тільки розвиває суспільну продуктивну
силу праці для капіталіста замість для робітника, але й розвиває
її через покалічення індивідуального робітника. Він створює
нові умови панування капіталу над працею. Тим-то, якщо цей
поділ праці, з одного боку, з’являється як історичний проґрес
та доконечний момент розвитку процесу економічного формування
суспільства, то, з другого боку, він з’являється як засіб
цивілізованої та рафінованої експлуатації.

Політична економія, що стає спеціяльною наукою лише за
мануфактурного періоду, розглядає суспільний поділ праці взагалі
лише з погляду мануфактурного поділу праці,\footnote{
Давніші письменники, як от Петті, анонімний автор «Advantages
of the East-India Trade» і т. д., зрозуміли капіталістичний характер
мануфактурного поділу праці ясніше, ніж А. Сміс.
} лише як
засіб з тією самою кількістю праці продукувати більше товару,
отже, здешевити товари та прискорити акумуляцію капіталу.
В гострій протилежності до цього підкреслювання кількости та
мінової вартости письменники клясичної старовини звертають
увагу виключно на якість та на споживну вартість.\footnote{
Виняток серед сучасних письменників становлять лише деякі
автори XVІІІ віку, як от Беккарія та Джемс Гарріс, які щодо поділу
праці майже виключно наслідують давніх. Напр., Беккарія каже: «Кожен
знає з досвіду, що, прикладаючи рук і розуму завжди до однорідної праці
та до виготовлювання тих самих продуктів, можна з більшою легкістю
досягти значніших та ліпших успіхів, ніж у тому випадку, коли б кожний
ізольовано сам для себе виготовляв усі потрібні йому речі... Люди поділяються
таким чином на різні кляси й стани в інтересах спільної та індивідуальної
користи». («Ciascuno prova coll’esperienza, che applicando
la mano e l’ingegno sempre allo stesso genere di opere e di produtti, egli
più facili, più abbondanti e migliori ne traca resultati, di quello che se
ciascuno isolatamente le cose tutte a se necessarie soltanto facesse... Dividendosi
in tal maniera per la comune e privata utilita gli uomini
in varie classe e condizioni»). (Cesare Beccaria: «Elementi di Economia
Publica», ed. Custodi, Parte Moderna, vol. XI, p. 28). Джеме
Гарріс, пізніш граф Малмсберійський, відомий своїми «Diaries» про
своє перебування послом у Петербурзі, сам каже в одній примітці до
свого «Dialogue concerning Happiness», London 1741, пізніше знов передрукованого
в «Three Treatises etc.», З rd ed. London 1772: «Всі докази
природности суспільства (а саме докази, засновані на «поділі занять»)
я взяв із другої книги «Республіки» Платона. (The whole argument,
to prove society natural is taken from the second book of Plato’s
republic»).
} У наслідок
роз’єднання суспільних галузей продукції товари виробляються
ліпше, різні нахили й таланти людей вибирають собі відповідні
сфери діяльности,\footnote{
Так, в «Одісеї», XIV, 228 говориться: «Ἄλλoς γὰρ τ’ἄλλοισιν ἀνὴρ ἐπιτέρπεται ἔργοις» * a
Архілох y Секста Емпірика каже: «Ἄλλος ἄλλῳ ἐπ’ἔργῳ καρδίην ἰαίνεται * *
} а без обмеження ніде не можна зробити нічого
значного. 79 Отже продукт і продуцент удосконалюються через

78 «Поλλ’ ἠπίστατο ἔργα, κακῶς δ’ ἠπίστατο πάντα»\footnote*{
Багато знав він справ, та всі погано знав. \emph{Ред.}
} — Атенець, як товаропродуцент,
почував свою перевагу над спартанцем, бо цей останній міг

* Одні люди люблять одне, інші — інше. \emph{Ред.}

** Одне тішить серце одного, інше — іншого. \emph{Ред.}