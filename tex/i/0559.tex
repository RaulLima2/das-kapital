Приріст річного доходу 1864 р. проти 1853 р. Збільшення за рік
Від    будинків....                           38,60\%                                                
      3,50\%
» каменярень....                             84,76\%                                                 
    7,70\%
» копалень....                                 68,85\%                                               
 6,26\%
» чавуноливарень....                      39,92\%                                               3,63\%
» рибальства....                              57,37\%                                               
5,21\%
» газівень....                                 126,02\%                                              
 11,45\%
» залізниць....                                    83,29\%                                           
  7,57\% 95

Якщо порівняти між собою щочотирирічки періоду 1853 —
1864 рр., то ступінь збільшення доходів невпинно зростає. Приміром,
для доходів, що походять із зиску, він 1853—1857 рр.
становить 1,73\% на рік, 1857—1861 рр. — 2,74\% на рік і 1861 —
1864рр. 9,30\% на рік. Загальна сума доходів, що підпадають прибутковому
оподаткуванню, становила в Об’єднаному королівстві
1856 р. 307.068.898 фунтів стерлінґів, 1859 р. — 328.127.416 фунтів
стерлінґів, 1862 р. — 351.745.241 фунт стерлінґів, 1863 р. —
359.142.897 фунтів стерлінґів, 1864 р. — 362.462.279 фунтів стерлінґів,
1865 р. — 385.530.020 фунтів стерлінґів.96

Акумуляцію капіталу одночасно супроводили його концентрація
й централізація. Хоч для Англії не існувало офіціяльної
рільничої статистики (в Ірляндії вона існує), проте 10 графств подали
її з власної волі. Тут виявився з неї такий результат, що від
1851 до 1861 р. число фарм, нижчих за 100 акрів, зменшилося з
13.583 до 26.567, отже 5.016 фарм сполучилося з більшими фармами.
97  Від 1815 до 1825 р. з рухомого майна, що підпадало спадщинному
податкові, не було жодного понад 1 мільйон фунтів стерлінґів;
навпаки, від 1825 до 1835 р. їх було 8, від 1856 до червня
1859 р., тобто за 4\sfrac{1}{2} роки, — 4.98 Однак централізацію найкраще
можна побачити з короткої аналізи прибуткового податку
в рубриці D (зиски, за винятком фармерських і т. ін.) за роки
1864 і 1865. Спочатку зауважу, що доходи з цього джерела,
які розміром не нижчі за 60 фунтів стерлінґів, підпадають
income tax.* Ці доходи, що підпадають оподаткуванню, становили
в Англії, Велзі й Шотляндії 1864 р. 95.844.222 фунтів стер-

95    Там же.

98 Цих чисел для порівняння досить, але коли розглядати їх абсолютно,
то вони фалшиві, бо щорічно «затаюється» може більше ніж
100 мільйонів фунтів стерлінґів доходів. Нарікання Commissioners of Irland
Revenue на систематичне шахрайство, особливо з боку купців і промисловців,
повторюються в кожному їхньому звіті. Приміром, читаємо:
«Одно акційне товариство показало свій належний до оподаткування зиск
у 6.000 фунтів стерлінґів, таксатор визначив його у 88.000 фунтів стерлінґів,
і, кінець-кінцем, податок виплачено з цієї суми. Друга компанія
показала зиск у 190.000 фунтів стерлінґів і мусила признатися, що
дійсна сума є 250.000 фунтів стерлінґів». (Там же, стор. 42).

97 «Census etc.», vol. III, p. 29. Твердження Джона Брайта, що 150
землевласникам належить половина англійської землі, а 12 — землевласникам
половина шотляндської, не збито.

98 «Fourth Report etc. of Inland Revenue», London 1860, p. 17.

* — прибутковому податкові. Ред.
