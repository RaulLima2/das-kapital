\parcont{}  %% абзац починається на попередній сторінці
\index{i}{0236}  %% посилання на сторінку оригінального видання
значенні слова, але що й мануфактури з більш-менш застарілими
методами виробництва, як от ганчарні, гути тощо, і старомодні
ремества, як от пекарство, і, нарешті, навіть розпорошене
так зване домашнє виробництво, як от цвяхоробство й т. ін.,\footnote{
Про стан цієї так званої домашньої промисловости надзвичайно
багатий матеріял дають останні звіти «Children’s Employment Commission».
}
так само вже давно попали під капіталістичну експлуатацію,
як і фабрика. Тому законодавство було примушене поволі
позбутися свого виняткового характеру або — там, де воно чинить
за методою римської казуїстики, як от в Англії, — на свою вподобу
проголосити фабрикою (factory) кожний дім, де працюють.\footnote{
«Акти останньої сесії (1864 р.)... охоплюють багато різних виробництв,
що в них методи дуже різні; вживання механічної сили, щоб пускати
машини в рух, не є вже, як це було раніш, конечна умова для того, щоб
підприємство вважалося за фабрику в законному значенні слова» («The
Acts of last Session (1864)... embrace a diversity of occupations the customs
in which differ greatly, and the use of mechanical power to give motion to
machinery is no longer one of the elements necessary, as formerly, to constitute
in legal phrase a Factory»). («Reports etc. for 31 st Oct. 1864», p. 8).
}

Подруге, історія реґулювання робочого дня в деяких галузях
продукції, а в інших — боротьба за це реґулювання, що триває
ще й досі, наочно доводять, що поодинокий робітник, робітник як
«вільний» продавець своєї робочої сили, на певному ступені достиглости
капіталістичної продукції, неспроможний ставити опір.
Тому створення нормального робочого дня — це продукт довгочасної
більш-менш прихованої громадянської війни між клясою
капіталістів і клясою робітників. А що боротьба розпочалася
у сфері сучасної промисловости, то вона й вибухає насамперед
на її батьківщині, в Англії.\footnote{
Бельґія, цей рай лібералізму на континенті, не виявляє й сліду
цього руху. Навіть у її копальнях вугілля й металів споживають робітників
обох статей і всякого віку з повного «волею» на який завгодно час і
коли завгодно. На кожну тисячу осіб, що там працюють, припадає 733 чоловіків,
88 жінок, 135 підлітків і 44 дівчаток молодших за 16 років;
у домнах і т. д. на кожну тисячу осіб припадає 668 чоловіків, 149 жінок,
98 підлітків і 85 дівчаток, молодших за 16 років. Сюди треба ще додати
низьку заробітну плату при величезній експлуатації дозрілих і недозрілих
робочих сил, плату пересічно в 2 шилінґи 8 пенсів денно для чоловіків,
1 шилінґ 8 пенсів для жінок і 1 шилінґ 2 1/2 пенсів для підлітків.
Але зате Бельґія року 1863 майже подвоїла проти 1850 р. кількість і
вартість свого вивозу вугілля, заліза й т. д.
} Англійські фабричні робітники
були першими борцями не лише англійської, а й сучасної робітничої
кляси взагалі, так само, як їхні теоретики перші атакували
теорію капіталу.\footnote{
Коли Роберт Оуен з початком другого десятиліття цього віку не
лише теоретично обстоював доконечність обмеження робочого дня, але
й насправжки завів десятигодинний робочий день на своїй фабриці в
Нью-Ленарку, то цю новизну висміяно як комуністичну утопію, цілком
так само, як і його «сполуку продуктивної праці з вихованням дітей»,
цілком так само, як і кооперативні підприємства робітників, що їх він покликав
до життя. Тепер перша утопія стала фабричним законом, друга
фігурує як офіціяльна фраза в усіх «Factory Acts», а третя служить
навіть прикриттям для реакційних шахрайств.
} Тому філософ фабрики Юр плямує як
\parbreak{}  %% абзац продовжується на наступній сторінці
