\parcont{}  %% абзац починається на попередній сторінці
\index{i}{*0079}  %% посилання на сторінку оригінального видання
фабричні інспектори, як англійські лікарі, що складають звіти
про «Public Health» (громадське здоров’я), як англійські комісари
для дослідження умов експлуатації жінок і дітей, житлових
умов, умов харчування і~\abbr{т. ін.} Персеєві треба було шапки-невидимки,
щоб переслідувати потвор. А ми глибоко натягаємо на очі
й вуха шапку-невидимку, щоб мати змогу заперечувати існування
потвор.

Нема чого творити собі ілюзій. Як у XVIII віці американська
війна за незалежність вдарила на сполох для середньої кляси
Европи, так само американська громадянська війна в XIX віці
вдарила на сполох для робітничої кляси Европи. В Англії процес
перевороту відчувається цілком виразно. На певному пункті свого
розвитку він мусить перекинутися на континент. Тут він набуватиме
в своєму перебігу брутальніших або гуманніших форм, залежно
від рівня розвитку самої робітничої кляси. Отже, незалежно
від вищих мотивів, власний інтерес панівних тепер кляс вимагає
усунути всі приступні законодавчому контролеві перешкоди, що
гальмують розвиток робітничої кляси. Тим то я дав у цьому
томі так багато місця, між іншим, історії, змістові й результатам
англійського фабричного законодавства. Одна нація повинна й
може вчитися від іншої. Навіть тоді, коли суспільство напало на
слід природного закону свого руху, — а остаточна мета цього
твору — розкрити закон економічного руху сучасного суспільства,
— навіть тоді воно не може ні переплигнути через природні
фази розвитку, ані усунути їх декретами. Але воно може скоротити
й полегшити муки родива.

Ще кілька слів, щоб уникнути можливих непорозумінь. Постаті
капіталіста й землевласника я змальовую зовсім не в рожевому
світлі. Але про особи тут ідеться лише остільки, оскільки вони є
персоніфікація економічних категорій, носії певних клясових
відносин і інтересів. З мого погляду, — а я розумію розвиток економічної
формації суспільства як природно-історичний процес, —
менше, ніж з усякого іншого, можна окрему особу робити відповідальною
за відносини, що їхнім продуктом у соціяльному значенні
вона лишається, хоч би й як високо вона підносилася над
ними суб’єктивно.

В царині політичної економії вільне наукове дослідження
зустрічає не лише того самого ворога, що й в інших царинах.
Своєрідна природа матеріялу, який вона досліджує, викликає на
боротьбу проти неї найлютіші, найдріб’язковіші й наймерзотніші
людські пристрасті, фурії приватного інтересу. Англіканська
церква, приміром, вибачить скоріш напади на 38 із 39 членів
її символу віри, ніж напади на \sfrac{1}{39} її грошових доходів. За нинішніх
часів навіть атеїзм є culpa levis\footnote*{
— легенький гріх. \emph{Ред.}
} порівняно з критикою
традиційних відносин власности. Однак і тут є очевидний проґрес.
Зазначу, приміром, опубліковану останніми тижнями Синю
Книгу «Correspondence with Her Majesty’s Mission Abroad, regarding
\index{i}{*0080}  %% посилання на сторінку оригінального видання
Industrial Questions and Trade’s Unions». Закордонні
представники англійської корони зазначають тут без околичностей,
що в Німеччині, Франції, одно слово, по всіх культурних
державах європейського континенту, переворот у наявних відносинах
між капіталом і працею так само помітний і так само неминучий,
як і в Англії. Одночасно, по той бік Атлантійського океану
пан Вейд, віце-президент Сполучених штатів Північної Америки,
заявив на публічних мітинґах: «По скасуванні рабства
на порядок дня стає переворот у відносинах капіталістичної й
земельної власності!» Це — ознаки часу, яких не сховаєш уже
ні під пурпуровою мантією, ні під чорною рясою. Це не значить,
що завтра станеться чудо. Вони показують, що навіть серед панівних
кляс прокидається прочуття, що теперішнє суспільство —
це не твердий кристал, а організм, який здатний до перетворень
і який постійно перебуває в процесі перетворення.

Другий том цієї праці розглядатиме процес циркуляції капіталу
(книга II) і форми (Gestaltungen) капіталістичного процесу
в цілому (книга III), останній, третій том (книга IV) — історію
теорій.

Кожний присуд наукової критики я радо вітатиму. Щождо
забобонів так званої громадської думки, перед якою я ніколи не
поступався, то за своє гасло вважатиму я, як і раніш, слова великого
фльорентійця: Segui il tuo corso, е lascia dir le genti!\footnote*{
Прямуй своїм шляхом, а люди кажуть хай, що хочуть! \emph{Ред.}
}

\begin{flushright}
  \emph{Карл Маркс}
\end{flushright}

{\small Лондон, 25 липня 1867~\abbr{р.}}
