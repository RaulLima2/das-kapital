\index{i}{0506}  %% посилання на сторінку оригінального видання
«Промисловість Манчестеру, — читаємо в одному творі, опублікованому
доктором Ейкіном 1795 р., — можна поділити на
чотири періоди. В першому періоді фабриканти були змушені
вперто працювати, щоб підтримати своє життя». Особливо збагачувались
вони, обкрадаючи батьків, що віддавали їм своїх дітей
як учнів і мусили дорого платити за це, тимчасом як учнів цих
капіталісти мучили голодом. З другого боку, пересічні зиски
були низькі, і акумуляція потребувала великої ощадности.
Вони жили як збирачі скарбів, і далеко не споживали навіть
процентів од свого капіталу. «У другому періоді вони почали
набувати дрібні маєтки, але працювали так само вперто, як і
раніш», бо безпосередня експлуатація праці коштує праці, як
це знає всякий погонич рабів, «і жили в тому самому скромному
стилі, як і раніш\dots{} У третьому періоді почалися розкоші, підприємства
поширювалися через посилання в кожне торговельне
місто королівства верхівців (кінних комівояжерів) по замовлення.
Мабуть, що перед 1690 р. було небагато, а то й зовсім не було,
капіталів від 3.000 до 4.000 фунтів стерлійґів, набутих у промисловості.
Однак близько того часу або трохи пізніше промисловці
вже нагромадили грошей і почали будувати собі кам’яні
будинки замість будинків із дерева й глини\dots{} Ще в перші десятиліття
XVIII віку один менчестерський фабрикант, що почастував
своїх гостей пінтою чужоземного вина, викликав пересуди
своїх сусід, що докірливо кивали головою». Перед появою машин
фабриканти, сходячись увечорі в шинках, ніколи не споживали
більш, як склянку пуншу за 6 пенсів і жмут тютюну за 1 пенс.
Лише 1758 р. — і це становить епоху — побачили «особу, справді
заняту в промисловості, у своєму власному екіпажі»! Четвертий
період», остання третина XVIII віку, «відзначається великими
розкошами і марнотратством, що спиралися на поширення
справ».\footnote{
Dr. Aikin: «Description of the Country from 30 to 40 miles round
Manchester», London 1795, p. 182 і далі.
} Що сказав би сердечний доктор Ейкін, коли б він
воскрес і поглянув би тепер на Менчестер!

Акумулюйте, акумулюйте! В цьому Мойсей і пророки! «Промисловість
постачає матеріял, що його акумулює ощадність».\footnote{
A. Smith: «Wealth of Nations», b. II. ch. ІІІ, p. 367.
}
Отже, заощаджуйте, заощаджуйте, тобто перетворюйте якомога
більшу частину додаткової вартости або додаткового продукту
знову на капітал! Акумуляція задля акумуляції, продукція
задля продукції — в цій формулі клясична політична економія
висловила історичну місію буржуазного періоду. Вона навіть ні
на одну хвилину не чинила собі ілюзії щодо тих мук, у яких родиться
багатство,\footnote{
Навіть Ж. В. Сей каже: «Заощадження багатих постають коштом
бідних» («Les épargnes des riches se font aux dépens des pauvres»). «Римський
пролетар жив майже цілком коштом суспільства\dots{} Можна б майже
сказати, що сучасне суспільство живе коштом пролетарів, коштом тієї
частини, яку воно відбирає в них із винагороди за працю». (Sismondi:
«Etudes etc.», vol. I, p. 24).
} але яка користь із нарікання на історичну
\parbreak{}  %% абзац продовжується на наступній сторінці
