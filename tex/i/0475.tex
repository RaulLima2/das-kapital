Дж. В. Коуелл, член фабричної комісії 1833 р., дійшов після
старанного розсліду прядільництва того висновку, що «в Англії
заробітна плата в суті справи для фабрикантів нижча, ніж на
континенті, хоч для робітників може вона бути й вища». (Ure:
«Philosophy of Manufacture», p. 314). Англійський фабричний
інспектор Александер Редґрев у фабричному звіті з 31 жовтня
1866 р., порівнюючи статистику Англії з континентальними державами,
доводить, що, не зважаючи на нижчу заробітну плату
й далеко довший робочий час, континентальна праця у відношенні
до продукту дорожча, ніж англійська. Один англійський
директор (menager) бавовняної фабрики в Ольденбурзі заявляв,
що там робочий час триває від 5 години 30 хвилин ранку до
8 години вечора, не виключаючи й суботи, і що тамошні робітники,
працюючи під доглядом англійців, дають протягом того часу
менше продукту, ніж англійці за 10 годин, а працюючи під
доглядом німців, ще куди менше. Заробітна плата там далеко
нижча, ніж в Англії, в багатьох випадках на 50%, але число
рук у відношенні до машин далеко більше, в різних відділах
відношення рук до машин є 5: 3. А. Редґрев подає дуже докладні
деталі про російські бавовняні фабрики. Ці відомості дав йому
один англійський директор, що недавно ще там працював. На
цьому російському ґрунті, такому родючому на всяку підлість,
розцвітають якнайповнішим квітом і старі страхіття з дитячого
періоду англійської фабрики. Директори, ясна річ, — англійці,
бо тубільний російський капіталіст нездатний до фабричного
підприємства. Не зважаючи на всю надмірну працю, невпинну
працю вдень і вночі та мізерну оплату робітників, російські фабрики
животіють лише завдяки забороні довозити закордонні фабрикати.
— Наприкінці я додаю ще порівняльний огляд п. Редґрева
щодо пересічного числа веретен на 1 фабрику й на 1 прядуна
по різних країнах Европи. Сам п. Редґрев зауважує, що ці
числа він зібрав перед кількома роками й що від того часу в Англії
зросли й розміри фабрик і число веретен, яке припадає на 1 робітника.
Але він припускає, що в перелічених країнах континенту
відбувався порівняно такий самий проґрес, так що подані
числа зберегли своє значення для порівняння.

підприємцеві певна кількість виготовлених продуктів, і розглянута з
цього погляду праця майже в усіх випадках є дешевша в багатших-країнах,
ніж у бідніших, хоч ціна збіжжя й інших засобів існування в останніх
звичайно значно нижча, ніж у перших... Праця, вимірювана поденно,
далеко дешевша в Шотландії, ніж в Англії... Відштучна праця звичайно
дешевша в Англії». («It deserves likewise to be remarked, that although the
apparent price of labour is usually lower in poor countries, where the
produce of the soil, and grain in general, is cheap; yet it is in fact for the
most part really higher than in other countries. For it is not the wages that
is given to the labourer per day that constitutes the real price of labour,
although it is its apparent price. The real price is that which a certain
quantity of work performed actually costs the employer; and considered
in this light, labour is in almost all cases cheaper in rich countries than in
those that are poorer, although the price of grain, and other provisions, is
usually much lower in the last than in the first... Labour estimated by
