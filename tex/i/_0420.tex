\parcont{}  %% абзац починається на попередній сторінці
\index{i}{0420}  %% посилання на сторінку оригінального видання
саме по тих місцях, де ми працюємо. Коли ж якийсь робітник
надішле скаргу з приводу вентиляції до інспектора, то його звільняють,
і він стає вже «поміченим» робітником, що й деінде не
знайде собі ніякої праці. «Mining inspecting Act» 1860 р. —
це просто клапоть паперу. Інспектор — число їх занадто вже
мале — мабуть лише один раз на сім років формально відвідує
копальню. Наш інспектор — це цілком неспроможна сімдесятлітня
людина, і він має інспектувати більш ніж 130 кам’яновугільних
копалень. Опріч більшого числа інспекторів, нам треба
субінспекторів» (№ 280). «Тоді уряд повинен тримати таку армію
інспекторів, щоб вони сами, без інформації самих робітників,
могли робити все те, чого ви вимагаєте? — Це неможливо, але
вони повинні сами приходити в копальні по інформації» (№ 285).
«Чи не гадаєте ви, що наслідок цього був би такий, що відповідальність
(!) за вентиляцію й т. ін. з власників копалень спала б
на державних урядовців? — Зовсім ні; їхній обов’язок був би
примушувати виконувати наявні вже закони» (№ 294). «Коли ви
говорите про субінспекторів, то чи не маєте ви на думці людей
з меншою платою й нижчої категорії, аніж сучасні інспектори? —
Я зовсім не бажаю нижчих людей, тоді як ви можете дати кращих»
(№ 295). «Чи хочете ви більшого числа інспекторів або
людей нижчої кляси ніж інспектори? — Нам треба людей, які
сами товклися б по копальнях, людей, що не тремтіли б за свою
шкуру» (№ 296). «Коли б ваше бажання мати інспекторів нижчого
сорту здійснилося, то чи не постане небезпека з того, що
вони недосить здібні? — Ні, це є справа уряду призначити
здібних людей». Нарешті, цей спосіб допиту стає навіть для
президента слідчої комісії надто безглуздим: «Ви хочете, — питає
він, встряваючи, — людей практичних, що сами оглядали б
копальні й повідомляли б інспектора, який потім міг би використати
свої ширші знання» (№ 531). «Чи не спричинить вентиляція
по всіх цих старих копальнях багато видатків? — Так,
затрати, може, і зростуть, але людське життя буде захищене»
(№ 581). Якийсь копальневий робітник протестує проти 17 відділу
закону 1860 р.: «Тепер, коли інспектор копалень знаходить
якусь частину копальні в непридатному для праці стані, він
мусить про це повідомити власника копальні й міністра внутрішніх
справ. Після цього власник копальні має двадцять днів на
роздум; наприкінці цих 20 днів він може відмовитися зробити
будь-які зміни. Коли він відмовляється зробити зміни, то він
мусить писати до міністра внутрішніх справ і запропонувати
йому п’ять гірничих інженерів, з-поміж яких міністер мусить
призначати третейських суддів. Ми запевняємо, що в цьому випадку
сам власник копальні має можливість призначати своїх
власних суддів» (№ 586). Буржуа-екзамінатор, сам власник
копалень: «Це — суто спекулятивне заперечення» (№ 588). «Отже,
ви дуже невисокої думки про чесність гірничих інженерів? —
Я кажу, що це велика кривда й велика несправедливість»
(№ 589). «Чи не займають гірничі інженери такого офіціяльного
\parbreak{}  %% абзац продовжується на наступній сторінці
