\parcont{}  %% абзац починається на попередній сторінці
\index{i}{0390}  %% посилання на сторінку оригінального видання
при головному диспансері (General Dispensary) в Нотінґемі. На
кожні 686 пацієнток, мережівниць, здебільша від 17 до 24 року
життя, сухітниць було:

1852 р.   1    на    45    1857 р.   1   на 13
1853  »   1     »      28    1858   »  1   »    15
1854  »   1     »      17    1859   »  1   »      9
1855  »   1     »      18    1860   »  1   »      8
1856  »   1     »      15    1861   »  1   »      8 \footnote{
«Children’s Employment Commission. 2nd. Report» p. XXII, n. 166.
}

Цей проґрес у поширенні сухот мусить задовольнити найоптимістичніших
проґресистів та найбрехливіших німецьких комівояжерів
вільної торговлі.

Фабричний закон 1861 р. реґулює виробництво мережива
у власному значенні, оскільки в ньому продукують машинами,
а це в Англії є звичайне явище. Галузі, що їх ми тут розглядаємо
коротко, — і саме щодо так званих домашніх робітників, а не
тих, що сконцентровані по мануфактурах, товарових домах і
т. ін., — розпадаються на: 1) finishing (остаточне оброблення
мережива, що фабрикується машиновим способом, — категорія,
що знову таки розпадається на численні підвідділи); 2) плетіння
мережива.

Lace finishing робиться як домашню роботу або по так званих
«mistresses houses»,\footnote*{
— домах хазяйок. \emph{Ред.}
} або в приватних помешканнях жінок, що
працюють сами чи з своїми дітьми. Жінки, які тримають «mistresses
houses», сами є бідні. Майстерня становить частину їхнього
приватного помешкання. Вони дістають замовлення від фабрикантів,
власників крамниць і т. ін. та вживають до праці жінок,
дівчат і малих дітей відповідно до розміру їхньої кімнати та коливань
попиту в цій галузі промисловості. Число занятих робітниць
змінюється по деяких з тих майстерень від 20 до 40, по інших від
10 до 20. Пересічний мінімальний вік, від якого діти починають
працювати, — 6 років, але декотрі мають менше ніж 5 років. Звичайно
робочий час триває від 8 години ранку до 8 години вечора,
з перервою для їжі на 1\sfrac{1}{2} години, при чому їдять нереґулярно
і часто в тих самих вонючих норах, де працюють. Коли справи
йдуть добре, то праця часто триває від 8 години (іноді від 6 години)
ранку до 10, 11 або й 12 години ночі. В англійських казармах
приписаний для кожного солдата об’єм становить 500—600 кубічних
футів, по військових лазаретах — 1.200. А по тих норах
для праці на кожну особу припадає від 67 до 100 кубічних
футів. Крім того, газове світло знищує кисень повітря. Щоб
мереживо тримати чистим, діти мусять часто скидати черевики,
навіть зимою, хоч долівка зроблена з кам’яних плит або з цегли.
«В Нотінґемі немає нічого незвичайного в тому, що в одну невеличку
кімнату, може, не більшу за 12 футів у квадраті, напихають
\parbreak{}  %% абзац продовжується на наступній сторінці
