\parcont{}  %% абзац починається на попередній сторінці
\index{i}{0385}  %% посилання на сторінку оригінального видання
десятиліття, як от виробництво ковертів, сталевих пер тощо, звичайно
проходили спочатку ремісниче виробництво, а потім мануфактурне
виробництво як короткі переходові фази до фабричного
виробництва. Найтяжчою ця метаморфоза лишається там, де мануфактурна
продукція виробу являє собою не послідовний ряд процесів
виготовлення його, а багато відокремлених процесів. Це становило,
приміром, велику перешкоду для фабрикації сталевих пер.
Однак уже перед якимись п’ятнадцятьма роками винайдено автомат,
що воднораз виконує шість окремих процесів. 1820 р. ремество
дало перші 12 тузенів сталевих пер за 7\pound{ фунтів стерлінґів} 4 шилінґи,
мануфактура дала їх 1830 р. за 8 шилінґів, а фабрика в
наші часи постачає їх гуртовим торгівцям за 2--6 пенсів.\footnote{
Пан Джілло влаштував у Бермінґемі першу мануфактуру сталевих
пер у великому маштабі. Вже 1851 р. вона дала понад 180 мільйонів пер
та споживала річно 120 тонн сталевої бляхи. Бермінґем, що монополізував
цю промисловість в Об’єднаному Королівстві, продукує тепер річно
мільярди сталевих пер. Число осіб, занятих у цьому виробництві, за
переписом 1861 р. становило 1.428; з них 1.268 робітниць, що почали
працювати з п’ятьох років.
}

\subsubsection{Зворотний вплив фабричної системи
на мануфактуру й домашню працю}

З розвитком фабричної системи та переворотом у рільництві, що
супроводить цей розвиток, не тільки поширюється маштаб продукції
по всіх інших галузях промисловости, але змінюється й характер
їхній. Принцип машинового виробництва: розкладати продукційний
процес на його складові фази та розв’язувати проблеми,
що постають таким чином, за допомогою застосування механіки,
хемії і т. ін., коротко кажучи, за допомогою природничих наук, —
цей принцип усюди набуває вирішального значення. Тому машина
протискується в мануфактури, захоплюючи то той, то інший
частинний процес. Отже, тривала кристалізація організації мануфактури,
що походить із старого поділу праці, розпадається
та зазнає безперестанних змін. Крім цього, склад колективного
робітника або комбінованого робочого персоналу зазнає ґрунтовного
перевороту. Протилежно до мануфактурного періоду
плян поділу праці ґрунтується тепер на вживанні жіночої праці,
праці дітей усякого віку, ненавчених робітників, де це ще
можливо, словом, — на вживанні «cheap labour», дешевої праці,
як її характеристично звуть англійці. Це має силу не тільки для
всякої комбінованої у великому маштабі продукції — незалежно
від того, чи вживає вона машин, чи ні, — але й для так званої
домашньої промисловости, все одно, чи працюють робітники в
своїх приватних мешканнях, чи в невеличких майстернях. Ця
так звана сучасна домашня промисловість, крім назви, не має
нічого спільного із стародавньою домашньою промисловістю,
яка має за передумову незалежне міське ремество, самостійне
селянське господарство та передусім хату для робітничої родини
\parbreak{}  %% абзац продовжується на наступній сторінці
