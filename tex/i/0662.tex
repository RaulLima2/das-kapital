trat social із царства мрій у царство дійсности. Але навіщо тоді взагалі «систематична колонізація»
протилежно до природної колонізації? Але, але: «сумнівно, чи в північних штатах американського союзу
хоч десята частина людности належить до категорії найманих робітників... В Англії... велика маса
народу складається з найманих робітників».\footnote{
Там же, стор. 42, 43, 44.
} В дійсності нахилу до самоекспропріяції на славу
капіталові в трудящого людства так небагато, що рабство, навіть за Векфілдом, є єдина природна
основа колоніяльного багатства. Його систематична колонізація є просто pis aller,\footnote*{
Pis aller — французький вираз: щось, до чого вдаються, коли немає нічого кращого. \emph{Ред.}
} бо ж йому
доводиться мати справу з вільними людьми, а не з рабами. «Перші еспанські поселенці на Сан-Домінґо
не діставали робітників із Еспанії. Але без робітників [тобто без рабства] капітал був би загинув
або принаймні скоротився б до таких дрібних розмірів, що всякий індивід міг би застосувати його
своїми власними руками. Так воно в дійсності й сталося в останній заснованій англійцями колонії, де
великий капітал у насінні, худобі й знарядді загинув через недостачу найманих робітників, і де жоден
поселенець не має капіталу більше, ніж він може застосувати своїми власними руками».\footnote{
Там же, т. II, стор. 5.
}

Ми бачили: експропріяція землі в народніх мас становить основу капіталістичного способу продукції.
Навпаки, суть вільних колоній у тому, що маса землі є ще народня власність, і тому кожний поселенець
може частину її перетворити на свою приватну власність і на свій індивідуальний засіб продукції, не
перешкоджаючи цим пізнішому поселенцеві зробити те саме.\footnote{
«Щоб стати елементом колонізації, земля не лише повинна бути необробленою, але й бути
громадською власністю, яку можна перетворити на приватну власність». (Там же, т. II, стор. 125).
} В цьому таємниця так процвітання колоній
як і їхніх болячок — їхнього опору проти вселення капіталу. «Де земля дуже дешева й усі люди вільні,
де кожний може з свого бажання дістати шматок землі для самого себе, там праця не лише дуже дорога,
беручи до уваги ту пайку, що припадає робітникові з його продукту, але й взагалі важко хоч за
якубудь ціну дістати комбіновану працю».\footnote{
Там же, т. І, стор. 247.
} А що в колоніях немає ще відокремлення робітника від
умов праці й їхньої основи, від землі, або відокремлення таке існує лише спорадично або на занадто
обмеженому просторі, то там ще немає й відокремлення рільництва від промисловости і не знищена ще
сільська домашня промисловість. Але звідки ж тоді там узятися внутрішньому ринкові для капіталу? «За
винятком рабів та їхніх хазяїнів, що скомбіновують капітал і працю для великих підприємств, жодна
частина людности Америки не працює виключно коло ріль-