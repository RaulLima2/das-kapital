із засобами праці і засоби існування\footnote{
«You take my life, when you do take the means whereby I livre»
(«Життя мені візьмете, як засоби візьмете до життя») (Шекспір).
} і разом із його частинною
функцією зробити і його самого зайвим; як ця суперечність дає
вихід своїй лютості в безперестанному заколенні на жертву
робітничої кляси, безмірному марнуванні робочих сил і спустошеннях
суспільної анархії, [яка всякий економічний проґрес
перетворює на суспільне лихо].\footnote*{
Заведене y прямі дужки беремо з французького видання. Ред.
}  Це негативний бік. Але коли
переміна праці пробиває собі тепер шлях лише як всепереможний
природний закон та із сліпо руйнаційною силою природного закону,
який всюди наражається на перешкоди,\footnote{
Один французький робітник, повернувшись із Сан-Франсіско,
пише: «Я ніколи не подумав би, щоб я був здатний працювати в усіх
тих промислах, якими я займавсь у Каліфорнії. Я був твердо переконаний,
що я ні до чого іншого не здатний, крім друкування книжок... Однак,
впинившися серед цього світу авантурників, що змінюють своє ремество
легше ніж сорочку, — це щира правда, — я поводив себе так, як поводилися
й інші. Через те, що праця по копальнях виявилася недосить вигідною,
я покинув її та подався до міста, де я почережно був друкарем, дахарем,
ливарником тошо. В наслідок цього досвіду, який показав, що я здатний
до всякого роду праці, я почуваю себе менше молюском і більше людиною».
(A. Corbon: «De l’enseignement professionnel», 2 ème éd., p. 50).
} то велика промисловість
самими своїми катастрофами робить питанням життя або
смерти — визнати переміну праці, а тому й найбільшу різнобічність
робітників за загальний суспільний закон продукції і
пристосувати відносини до нормального здійснення цього закону.
Вона робить питанням життя або смерти — замінити потворність
нещасної робітничої людности, що її тримають у резерві завжди
напоготові для змінних експлуататорських потреб капіталу,
абсолютною придатністю людини для змінних потреб праці;
замінити частинного індивіда, простого носія певної суспільної
детальної функції, цілком розвинутим індивідом, що для нього
різні суспільні функції є почережно змінювані способи діяльности.
Один момент цього процесу перевороту, момент, що спонтанейно
розвинувся на основі великої промисловости, є політехнічні й
агрономічні школи, другий — «écoles d’enseignement professionel»,\footnote*{
— професійно-технічні школи. Ред.
}
де дітей робітників навчають дечого з технології і практично
застосовувати різні знаряддя продукції. Якщо фабричне законодавство,
ця перша ледве видерта в жорстокій боротьбі у капіталу
поступка, сполучає лише елементарне навчання з фабричною працею,
то не підлягає ніякому сумнівові, що неминуче завоювання
політичної влади робітничою клясою завоює також відповідне
місце в робітничих школах і для технологічної освіти, теоретичної
і практичної. Так само не підлягає ніякому сумнівові, що
капіталістична форма продукції та відповідні їй економічні відносини
робітників\footnote*{
У французькому виданні тут сказано: «... Капіталістична форма
продукції і економічні умови, в які вона ставить робітників, стоять
і т. д.». Ред.
} стоять у найдіяметральнішій суперечності