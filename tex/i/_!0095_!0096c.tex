\parcont{}  %% абзац починається на попередній сторінці
\index{i}{*0095}  %% посилання на сторінку оригінального видання
документальні докази поданих у тексті тверджень. Але в багатьох
випадках наводиться місця з письменників-економістів, щоб показати,
коли, де та хто вперше ясно висловив той або інший
погляд. Це робиться в тих випадках, коли наведений погляд має
значення як більш-менш адекватний вираз панівних для певного
часу умов суспільної продукції та обміну і цілком незалежно від
того, чи його Маркс визнає, чи він має загальне значення. Такі
цитати дають текстові побіжні коментарі, запозичені з історії
науки.

Наш переклад охоплює лише першу книгу твору. Але — ця
перша книга великою мірою становить сама по собі єдине ціле і
протягом двадцятьох років її вважалось за самостійний твір. Друга
книга, що я її видав німецькою мовою в 1855 році, є цілком
неповна без третьої, що її не можна буде опублікувати до кінця
1887 року. Після того як третю книгу видасться в німецькому
оригіналі, доведеться подумати про підготову англійського видання
обох книг.

«Капітал» часто звуть на континенті «біблією робітничої
кляси». Що висновки, здобуті в цій книзі, дедалі більше стають
основними принципами великого руху робітничої кляси не тільки
в Німеччині та Швайцарії, але також у Франції, Голляндії і
Бельґії, в Америці і навіть в Італії та Еспанії; що всюди робітнича
кляса дедалі більше вбачає в цих висновках якнайвідповідніший
вираз свого становища і своїх прагнень, — цього не буде
заперечувати жодна людина, з цим рухом обізнана. І в Англії також
теорії Маркса справляють саме тепер могутній вплив на соціалістичний
рух, що шириться в рядах «освічених верств» не менше,
ніж в рядах робітничої кляси. Та це не все. Швидко наближається
той час, коли ґрунтовна аналіза економічного стану Англії стане
неминуча як непереборна національна доконечність. Розвиток
промислової системи Англії, неможливий без постійного та швидкого
поширення продукції, а значить і ринків, дійшов до застою.
Вільна торговля вичерпала свої ресурси. Навіть Менчестер узяв
під сумнів цю свою колишню економічну євангелію.\footnote{
На квартальному засіданні торговельної палати Менчестеру, що
відбулося сьогодні в пообідні години, постала жвава дискусія з приводу
вільної торговлі. Подано резолюцію такого змісту, що після того, як
«даремно чекали 40 років, що інші нації наслідують англійський приклад
вільної торговлі, палата вважає, що надійшов нині час змінити цю позицію».
Резолюцію відхилено більшістю одного голоса, а співвідношення
голосів було 21 за і 22 проти («Evening Standard», 1 листопада 1886 р.).
} Чужоземна
промисловість, швидко розвиваючись, всюди протистоїть англійській
продукції, не тільки на захищених митом, але також і на
невтральних ринках, і навіть по цей бік каналу. Тим часом, як
продуктивна сила зростає в геометричній проґресії, ринки поширюються
в кращому разі в аритметичній. Десятирічний цикл
застою, розквіту, перепродукції та криз, що від 1825 р. до 1867 р.
завжди повертався, здається справді пройшов свою путь; але
лише для того, щоб кинути нас у багно сумнівів довгочасної та
\index{i}{*0096}  %% посилання на сторінку оригінального видання
хронічної депресії. Сподіваний період розквіту не хоче надходити;
і кожного разу, коли нам здається, що ми бачимо провісні
симптоми, вони одразу ж розвіюються як дим. Тим часом  кожна
наступна зима знову ставить питання: «Що робити з безробітними?»
Але тим часом, як число безробітних рік-у-рік більшає,
немає нікого, хто відповів би на це питання; і ми могли б майже
обчислити час, коли цим безробітним урветься терпець і вони
візьмуть свою долю у свої власні руки. В такий момент, звичайно,
треба, щоб почувся голос людини, що її вся теорія є наслідок
ціложиттьового вивчання економічної історії та стану Англії,
людини, яку це вивчання довело до висновку, що, принаймні в
Европі, Англія є єдина країна, де неминуча соціальна революція
може бути переведена цілком мирними і леґальними засобами.
Звичайно, ця людина ніколи не забувала додати, що вона навряд
сподівається, щоб англійська панівна кляса підкорилась цій
мирній і леґальній революції без «proslavery rebellion» (повстання
на оборону рабства).

5 листопада 1886 р.

Фрідріх Енґельс
