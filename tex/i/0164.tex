Через те, що вартість змінного капіталу дорівнює вартості купленої ним робочої сили, через те, що
вартість цієї робочої
сили визначає доконечну частину робочого дня, а додаткову вартість із свого боку визначає надлишкова
частина робочого дня, то звідси випливає: додаткова вартість відноситься до змінного капіталу, як
додаткова праця до доконечної, абож норма додаткової вартости m/υ = додаткова праця/доконечна праця.
Обидві пропорції виражають те саме відношення в різній формі: раз у формі упредметнецої
праці, другий — у формі текучої праці.

Тому норма додаткової вартости є точний вираз ступеня експлуатації робочої сили капіталом, або
робітника капіталістом.30а

За нашим припущенням, вартість продукту дорівнювала 410 фунтам стерлінґів + 90 фунтів стерлінґів +
90 фунтів стерлінґів,
авансований капітал дорівнював 500 фунтам стерлінґів. Через те, що додаткова вартість дорівнює 90 і
авансований капітал 500, то на основі звичайного способу обчислення вийшло б, що норма
додаткової вартости (яку плутають із нормою зиску) дорівнює 18\%, — процент, низький рівень якого міг
би зворушити пана Кері й інших гармоністів. В дійсності ж норма додаткової вартости є не m/С, або
m/с + υ, отже не 90/500, а 90/90 = 100\%, більш ніж уп’ятеро проти того, що на позір становить
ступінь експлуатації. Хоч у даному разі ми не знаємо ні абсолютної величини робочого дня, ні періоду
процесу праці (день, тиждень і т. д.), ні, нарешті, числа робітників, що їх одночасно пускає в рух
змінний капітал у 90 фунтів стерлінґів, все ж норма додаткової вартости m/υ тим, що вона може бути
перетворена на формулу
додаткова праця / доконечна праця, докладно показує нам взаємне відношення обох складових частин
робочого дня. Воно дорівнює 100\%. Отже, робітник одну половину дня працює на себе, другу на
капіталіста. Отже, метода обчислення норми додаткової вартости, коротко
кажучи, така: ми беремо цілу вартість продукту й припускаємо, що стала капітальна вартість, яка лише
знов у з’являється у вартості продукту, дорівнює нулеві. Сума вартости, яка після цього

що приневолює Рошера й К° більш або менш імовірні мотиви, що ними капіталіст виправдує присвоєння
наявної додаткової вартости, перекручувати на причини її постання.

30а Примітка до другого видання. Хоч норма додаткової вартости є точний вираз ступеня експлуатації
робочої сили, вона однак не є вираз абсолютної величини експлуатації. Приміром, коли доконечна праця
= 5 годинам і додаткова праця = 5 годинам, то ступінь експлуатації дорівнює
100\%. Величину експлуатації вимірюється тут 5 годинами. Коли ж доконечна праця дорівнює 6 годинам і
додаткова праця 6 годинам, то ступінь експлуатації в 100\% лишається незмінний, тимчасом як величина
експлуатації зростає на 20\%, з 5 на 6 годин.
