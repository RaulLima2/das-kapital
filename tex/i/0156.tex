кується, так це лише нову споживну вартість, що в ній знову
з’являється стара мінова вартість.25

Інакше стоїть справа з суб’єктивним фактором процесу праці,
з діющою робочою силою. Тимчасом як праця в наслідок своєї
доцільної форми переносить вартість засобів продукції на продукт
і зберігає її, кожний момент руху праці творить новододавану вартість,
нову вартість. Припустімо, що процес продукції припиняється
на тому пункті, коли робітник випродукував еквівалент
вартости своєї власної робочої сили, коли він, приміром, шестигодинною
працею додав вартість у 3 шилінґи. Ця вартість становить
надлишок вартости продукту понад ті її складові частини,
що своє постання завдячують вартості засобів продукції. Вона є
однісінька нова вартість, що постала в межах цього процесу,
однісінька частина вартости продукту, випродукована самим цим
процесом. Певна річ, вона лише компенсує ті гроші, що їх авансував
капіталіст підчас купівлі робочої сили та які сам робітник
витратив на засоби існування. Щодо цих витрачених 3 шилінґів
нова вартість у 3 шилінґи з’являється лише як репродукція їх.
Але вона дійсно репродукована, а не лише на позір, як от вартість
засобів продукції. Заміщення однієї вартости іншою тут упосереднюється
творенням нової вартости.

Однак ми вже знаємо, що процес праці триває далі поза той
пункт, коли репродукується й прилучається до предмету праці
лише еквівалент вартости робочої сили. Замість 6 годин, яких
для цього було досить, процес триває, приміром, 12 годин. Отже,
в наслідок діяння робочої сили не тільки репродукується її власну
вартість, але й продукується ще надлишок вартости. Ця додаткова
вартість становить надлишок вартости продукту понад

25 В одному північно-американському стислому підручнику, що витримав,
може, 20 видань, ми читаємо: «Не має жодного значення те, в
якій формі капітал з’являється знов» («It matters not in what form capital
reappears»). Після багатомовного переліку всіх можливих складових
частин продукції, що їх вартість знов з’являється в продукті, наприкінці
сказано: «Різні ґатунки харчу, одягу й житла, доконечні для існування
й комфорту людини, теж зазнають змін. Їх час від часу споживається, і
вартість їхня знову з’являється в новій фізичній і розумовій силі людини, що
являє собою новий капітал, який можна знову вжити на продукцію» («The
various kinds of food, clothing and shelter, necessary for the existence and
comfort of the human being, are also changed. They are consumed from time
to time, and their-value re-appears, in that new vigour im arted to his
body and mind, forming fresh capital, to be employed again in the work
of production»). (F. Wayland: «The Elements of Political Economy»,
Boston 1853, p. 31, 32). Залишаючи осторонь всі інші чудноти, зауважимо,
що, приміром, не ціна хліба знов з’являється у відновленій силі,
а його кровотворні елементи. Навпаки, те, що знову з’являється як вартість
сили, є не засоби існування, а їхня вартість. Ті самі засоби існування,
якщо вони коштують лише половину, випродукують цілком стільки ж
мускулів, костей і т. ін., словом, таку саму силу, але силу не тієї самої
вартости. Ця плутанина, перетворення «вартости» на «силу», і вся ця
фарисейська невизначеність криють у собі спробу — звичайно, даремну —
викрутами вивести додаткову вартість із простого повернення авансованих
вартостей.
