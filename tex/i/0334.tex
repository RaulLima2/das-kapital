зростає додаткова вартість, але й меншають витрати, доконечні
для того, щоб її добути. Правда, в більшій або меншій мірі це
буває завжди за всякого здовження робочого дня, але тут це
має вирішальніше значення, бо частина капіталу, перетворена
на засоби праці, тут взагалі має більше значення.150 Дійсно,
розвиток машинового виробництва зв’язує чимраз більшу складову
частину капіталу в такій формі, в якій вона, з одного боку, може
постійно самозростати своєю вартістю, а з другого боку, втрачає
споживну вартість і мінову вартість, скоро тільки переривається
її контакт із живою працею. «Коли, — повчав пан Ешворт,
англійський бавовняний маґнат, професора Нассау В. Сеніора, —
коли рільник кидає свій заступ, то він на цей час робить некорисним
капітал у 18 пенсів. Коли один із наших людей (тобто
з фабричних робітників) кидає працю на фабриці, то він робить
некорисним капітал, який коштував 100.000 фунтів стерлінґів».151
Подумати тільки! «Зробити некорисним», хоча б лише на хвилину,
капітал, що коштував 100.000 фунтів стерлінґів. Це — справді
жахна річ, що один із наших людей взагалі може колибудь кинути
фабрику! Чимраз більше зростання розміру машин робить «бажаним»
— визнає повчений від Ешворта Сеніор — щораз більше
й більше здовжування робочого дня.152

ються випадки, коли фабрикант може вжити додаткового обігового капіталу,
не вживаючи додаткового основного капіталу... якщо додаткову
кількість сировинного матеріялу можна переробити без додаткових витрат
на будівлі та машини» («It is self-evident, that, amid the ebbings and
flowings of the market, and the alternate expansions and contractions of
demand, occasions will constantly recur, in which the manufacturer may
employ additional floating capital without employing additional fixed
capital... if additional quantities of raw material can be worked up without
incurring an additional expence for buildings and machinery»).
(R. Torrens: «On Wages and Combination», London 1834, p. 64).

150    Згадану в тексті обставину я наводжу тільки для повноти, бо
норму зиску, тобто відношення додаткової вартости до цілого авансованого
капіталу, я розглядаю лише в третій книзі.

151 «When a labourer», said Mr. Ashworth, «lays down his spade, he
renders useless, for that period, a capital worth 18 d. When one of our people
leaves the mill, he renders useless a capital that has cost 100.000 pounds».
(Senior: «Letters on the Factory Act», London 1837, p. 13, 14).

152 «Велика пропорція основного капіталу проти обігового... робить
бажаним довгий робочий день» («The great proportion of fixed to circulating
capital... makes long hours of work desirable»). Із зростом розміру
машин і т. д. «мотиви до здовження робочого дня дедалі посилюються,
бо це єдиний засіб, щоб зробити зисковною відносно велику масу основного
капіталу» («the motives to long hours of work will become greater,
as the only means by which a large proportion of fixed capital can be
made profitable»). (Там же, стор. 11—13). «На фабриці є різні витрати,
які лишаються постійними, незалежно від того, чи робочий час на
фабриці довший, чи коротший, наприклад, орендна плата за будівлі, місцеві
та загальні податки, убезпечення від огню, заробітна плата різним
постійним робітникам, псування машин та різні інші витрати, пропорція
яких до зиску меншає в такому самому відношенні, в якому
зростає розмір продукції». («Reports of Insp. of Fact, for 31 st October
1862», p. 19).
