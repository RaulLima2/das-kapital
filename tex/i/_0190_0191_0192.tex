\parcont{}  %% абзац починається на попередній сторінці
\index{i}{0190}  %% посилання на сторінку оригінального видання
має на думці: споживання «наших» живих машин, тобто робочу
силу), так що фактично ми (iterum Crispinus) працюємо надробочий
час протягом цілого року... Діти й дорослі (152 дітей і
підлітків нижче 18 років і 140 дорослих) однаково працювали
протягом останніх 18 місяців — пересічно щонайменше по 7 днів
і 5 годин тижнево, або 78 1/2 годин на тиждень. Для 6 тижнів,
кінчаючи 2 травня поточного року (1863), пересічний результат
був вищий — 8 днів, або 84 години на тиждень!» Однак  цей самий
пан Сміс, що відчуває таку велику симпатію до pluralis majestatis,\footnote*{
Множина величности, тобто вживання займенника «ми» замість
«я», як це робили владарі й царі. Ред.
}
додає підсміхаючись: «Машинова праця легка». А фабриканти,
що вживають Block Printing, кажуть: «Ручна праця
здоровіша від машинової». Загалом же пани фабриканти з обуренням
висловлюються проти проекту «припиняти машини бодай
на час їжі». Пан Отлей, управитель фабрики шпалер у Боро
(у Лондоні), каже: «Закон, який дозволяв би нам робочий день
від 6 години вранці до 9 години вечора, був би нам (!) дуже до вподоби,
але робочий день від 6 години ранку до 6 години вечора,
який приписує Factory Act, нам (!) не підходить... Ми спиняємо
свою машину підчас обіду (що за великодушність!). Це припинення
не спричинює жодної вартої згадки втрати на папері й фарбах».
«Але, — додає він із співчуттям, — я добре розумію, що втрата,
сполучена з цим, не дуже приємна річ». Звіт комісії наївно гадає,
що страх деяких «видатних фірм» утратити час, тобто час, протягом
якого присвоюється чужу працю, і через те «втратити зиск»,
що цей страх не є ще «достатня основа», щоб дітей, молодших
від 13 років, і молодь до 18 років «позбавляти їжі» протягом
12—16 годин, або щоб їм постачали харч так, як засобам праці
постачають допоміжні матеріяли: машині — воду й вугілля,
вовні — мило, колесам — мастиво і т. ін., тобто підчас самого
процесу продукції.\footnote{
«Children’s Employment Commission, 1863», Evidence, p. 123,
124, 125, 140 і LIV.
}

В жодній галузі промисловости в Англії — (ми лишаємо осторонь
машинове виробництво хліба, яке тільки-но починає прокладати
собі шлях) — не зберігся донині такий стародавній
і навіть, як це можна побачити в поетів з часів Римської імперії,
дохристиянський спосіб продукції, як пекарство. Але капітал,
як зазначено раніш, є спочатку байдужий щодо технічного характеру
того процесу праці, який він опановує. Він бере його
спочатку таким, яким його находить.

Неймовірну фальсифікацію хліба, особливо в Лондоні, відкрив
уперше комітет Палати громад у справі «про фалшування
харчів» (1855/56 р.) і твір д-ра Гесселя: «Adulterations de-

нормального дня оплачується нижче вартости, так що «надробочий
час» — то лише хитрощі капіталістів, щоб видушити більше «додаткової
праці»; зрештою, це лишається без зміни й тоді, коли робочу силу, вживану
протягом «нормального дня», дійсно оплачується цілком.
\index{i}{0191}  %% посилання на сторінку оригінального видання
tected».\footnote{
Галун, дрібно розтертий, або змішаний із сіллю, є нормальний
предмет торговлі, що має характеристичну назву «baker’s stuff» («порошок
пекарів»).
} Як наслідок того відкриття оголошено закона з 6 серпня
I860 р.: «for preveting the adulteration of articles of food
and drink»,\footnote*{
Про заходи, запобіжні проти фальсифікації харчів і напоїв. Ред.
** — заробити копійку чесною працею. Ред.
*** Друзі торговлі. Ред.
**** — наочно. Ред.
} закона без ніякої сили, бо ж він, розуміється, з
надзвичайною делікатністю ставиться до кожного фритредера,
що має намір через купівлю та продаж фальшованих товарів «to
turn an honest penny**».\footnote{
Сажа є, як відомо, дуже енерґійна форма вуглеця і служить за
добриво, яке капіталістичні коминярі продають англійським фармерам.
1862 р. в одному судовому процесі брітанському «Juryman» (судді) довелося
вирішувати, чи така сажа, що до неї без відома покупця домішано
90\% пилу й піску, є «правдива» сажа в «комерційному» значенні слова,
чи вона є «фальсифікована» сажа в «законному» значенні. «Amis du
commerce» *** вирішили, що це «правдива» комерційна сажа, і відкинули
скаргу орендаря, який, крім того, мусив ще заплатити судові витрати.
} Сам комітет сформулював більш-менш
наївно своє переконання, що свобода торговлі означає по суті
торговлю фальшованим, або, як це дотепно кажуть англійці, «софістикованими
продуктами». Дійсно, такого роду «софістика»
вміє краще за Протагора робити з білого чорне і з чорного біле,
і краще за елеатів демонструвати ad oculos**** тільки подобу всього
реального.\footnote{
Французький хемік Шевальє у розвідці про «софістикацію» товарів
налічує для багатьох із 600 продуктів, що їх він розглядає, до 10, 20,
30 різних способів фалшування. Він додає, що не знає всіх способів, і
згадує не про всі, які він знає. Для цукру він наводить 6 способів фальсифікації,
для маслинової олії — 9, для масла — 10, для соли — 12, молока
— 19, хліба — 20, для горілки — 23, для борошна — 24, для шоколяди
— 28, для вина — 30, для кави — 32 і т. ін. Навіть милосердого господа-бога
не минула ця доля. Див. Ronard de Card: «De la falsification
des substances sacramentelles», Paris 1856.
}

В кожному разі комітет звернув увагу громадянства на його
«хліб насущний», а тим самим і на пекарство. Одночасно на публічних
мітинґах і в петиціях, звернених до парляменту, залунав
крик лондонських пекарських підмайстрів про надмірну працю
й т. ін. Крик зробився такий настирливий, що пана Г. С. Тріменгіра,
члена не раз уже згадуваної комісії 1863 р.,\footnote{
«Report etc. relating to the Grievances complained of by the
Journeymen Bakers etc.», London 1862 і «Second Report etc.», London
1863.
} призначено
було на королівського слідчого комісара. Його звіт разом із виказами
свідків зворушив публіку, не серце її, а її шлунок. Правда,
начитаний у біблії англієць знав, що людину, якщо вона не є з
ласки божої ні капіталіст, ні лендлорд і не має синекури, призначено
на те, щоб у поті чола свого їсти хліб свій, та він не знав
того, що він сам мусить день-у-день з’їдати в своєму хлібі
певну кількість людського поту, змішаного з виділенням гнійних
ґуль, павутинням, трупами тарганів, з гнилими німецькими дріжджами,
\index{i}{0192}  %% посилання на сторінку оригінального видання
не кажучи вже про галун, пісок і інші приємні мінеральні
домішки. Тому, не зважаючи на її святість, «вільна торговля»,
«вільне» перед тим пекарство піддано під догляд державних
інспекторів (наприкінці парляментської сесії 1863 р.), і той самий
парляментський акт заборонив пекарським підмайстрам, молодшим
за 18 років, працювати між дев’ятою годиною вечора й п’ятою
годиною ранку. Останній додатковий пункт свідчить красномовніш,
ніж цілі томи, про надмірну працю в цій галузі промисловости,
що від неї віє такою патріярхальністю.

«Праця лондонського пекарського підмайстра починається
звичайно об 11 годині ночі. В цей час він робить тісто, дуже втомний
процес, що триває від 1/2 до 3/4 години залежно від величини
та якости печива. Потім він лягає на місильну дошку, що разом
з тим служить і за покришку діжі, де виробляється тісто, і засинає
на декілька годин, підклавши один лантух з-під борошна під
голову й накрившися другим. Після того починається швидка й
безупинна чотиригодинна праця: викидають із діжі тісто, важать
його, формують, садовлять до печі, виймають із печі й т. ін. Температура
пекарні сягає 75 і навіть 90 градусів,\footnote*{
За Фаренгайтом; за Цельсієм це становить 24—32 градуси, за
Реомюром — 19—26 градусів. Ред.
} а в невеличких пекарнях
вона скорше буває більша, аніж менша. Коли справу печення
хліба, булок тощо закінчено, починається розподілювання хліба;
значна частина робітників, скінчивши тількищо описану важку
нічну працю, протягом дня розносить хліб у кошах або розвозить
його на візках від одного дому до другого, а в переміжках іноді працює
і в пекарні. Залежно від пори року та розміру підприємства
праця кінчається між першою й шостою годиною по півдні, тоді коли
інша частина підмайстрів працює в пекарні до пізнього вечора».\footnote{
Там же. «First Report etc.», р. VI.
}
«Підчас лондонського сезону підмайстри в пекарів Вестенда, що
продають хліб за «повну» ціну, реґулярно починають працювати
об 11 годині вночі і працюють коло печення хліба з однією або
двома часто дуже короткими перервами до 8 години найближчого
ранку. Потім до 4, 5, 6, а то навіть і до 7 години вони розносять
хліб або печуть бісквіти в пекарні. Після закінчення праці вони
відпочивають, засипаючи на 6 годин, часто лише на 5 або 4 години.
У п’ятницю праця завжди починається раніш, так щось о 10 годині
вечора, і триває безперестанку, чи то при виготовленні чи розношуванні
хліба, до 8 години вечора наступної суботи, але ж здебільшого
до 4 або 5 години в ніч під неділю. І в першорядних
пекарнях, що продають хліб за «повну ціну», неділями знов таки
доводиться працювати протягом 4—5 годин, щоб підготовити
роботу наступного дня... Ще довший робочий день пекарських
підмайстрів у «underselling masters» (що продають хліб нижче
від повної ціни), а ці останні становлять, як це вже раніш зазначено,
більш ніж 3/4 лондонських пекарів, але праця їхня
майже цілком обмежена пекарнею, бо їхні майстри-хазяї, за
винятком постачання хліба до дрібних крамниць, продають хліб
\parbreak{}  %% абзац продовжується на наступній сторінці
