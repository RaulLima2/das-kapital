[А що ми досі визначили лише субстанцію вартости й величину
вартости, то перейдімо тепер до аналізи форми вартости].\footnote*{
Заведене у прямі дужки ми беремо з першого нім. видання. \emph{Ред.}
}

3. Форма вартости або мінова вартість

Товари з’являються на світ у формі споживних вартостей, або
товарових тіл, як залізо, полотно, пшениця тощо. Це їхня доморосла
натуральна форма. Однак товари вони є лише через свою
двоїстість, лише через те, що вони є предмети споживання й одночасно
носії вартости. Тому вони являють собою товари або мають
форму товарів лише остільки, оскільки мають двоїсту форму:
форму натуральну й форму вартости.

Предметність вартости (Wertgegenständlichkeit) товарів тим
відрізняється від удовиці Квіклі,\footnote*{
Дієва особа в комедії ІІІекспіра «Веселі віндзорські молодиці».
\emph{Ред.}
} що невідомо, де її можна
знайти. Цілком протилежно до почуттєво-грубої предметности
товарових тіл жоден атом природної речовини не входить у їхню
вартостеву предметність. Тому можна крутити й вивертати поодинокий
товар на всі боки — як предмет вартости (Wertding)
він лишається несхопним. Однак, коли ми пригадаємо, що товари
мають предметність вартости лише остільки, оскільки вони є
вирази однакової суспільної одиниці, людської праці, і що, отже,
їхня вартостева предметність є суто суспільна, тоді стане також
само собою зрозуміло, що вона може виявитись тільки в суспільному
відношенні одного товару до іншого. Справді, ми виходили

визначення вартости кількістю праці, витраченої на продукцію товару
з визначенням товарової вартости вартістю праці, і тому він силкується
довести, що рівні кількості праці завжди мають однакову вартість. З другого
боку, він передчуває, що праця, оскільки вона виражається у вартості
товарів, є лише витрата робочої сили, але цю витрату він знов таки
бере тільки як жертву спокоєм, волею та щастям, а не як нормальну життєву
діяльність. Правда, він має на оці тільки сучасного найманого робітника.
— Куди влучніше міркує цитований вже в дев’ятій примітці анонімний
попередник А. Сміса: «Одна людина затратила тиждень на виготовлення
певної речі, потрібної для життя... і той, хто дає їй на обмін якусь
іншу річ, може найкраще оцінити, яка кількість її становить еквівалент
першої речі, лише вирахувавши, що саме йому коштувало точно стільки ж
часу й праці; це в дійсності сходить на те, що працю однієї людини, витрачену
на продукцію якоїсь речі протягом якогось часу, обмінюється
на працю іншої людини, витрачену протягом того самого часу на продукцію
іншої речі» («One man has employed himself a week in providing this
necessary of life... and he that gives him some other in exchange, cannot
make a better estimate of what is a proper equivalent, than by computing what
cost him just as much labour and time: which in effect is no more tran exchanging
one man’s labour in one thing for a time certain for another an’s labour
in another thing for the same time»). («Some Thoughts on the Interestofmoney
in general etc.», p. 39). — [«До 4 видання: Англійська мова має
ту перевагу, що в ній є два різні слова для означення цих двох різних аспектів
праці. Якісно визначену працю, що утворює споживні вартості,
називається Work протилежно до Labour; працю, що творить вартості
й виміряється лише кількісно, називається Labour протилежно до Work.
Див. примітку до англійського перекладу, стор. 14. — Ф. Е].