\parcont{}  %% абзац починається на попередній сторінці 
\index{i}{0399}  %% посилання на сторінку оригінального видання 
досягти до робітника.\footnote{
Заведення цієї й інших машин на фабриці сірників замінило в
одному відділі  230 паідлітків 32 підлітками й дівчатами від 14 до 17 років.
Цю економію на робітниках 1865 р. проведено ще далі за допомогою застосування
парової сили.
} Так само ще й тепер твердять у тих
галузях мануфактури мережив, які ще не підпорядковані фабричному
законові, що, мовляв, не можна встановити реґулярного
часу на їжу через те, що різні матеріяли для мережив потребують
для сушіння неоднакового часу, що коливається від трьох хвилин
до однієї години й більше. На це комісари «Children’s Employment
Commission» відповідали: «Обставини тут такі ж самі, що
й у виробництві шпалер. Декотрі з головних фабрикантів з цієї
галузі енергійно доводили, що природа вживаних матеріялів і
різнорідність процесів, які вони проходять, не дозволяють, без великої
втрати, раптом припиняти роботу для їжі... За пунктом 6
відділу VI Factory Act’s Extension Act\footnote*{
— закону про поширення сфери чинности фабричного закону. Ред.
} (1864 p.) їм дано від
часу видання закону вісімнадцятимісячний строк, після скінчення
якого вони мусили прийняти перерви для відпочинку,
приписані фабричним актом».\footnote{
«Children’s Employaient Commission. 2nd Report 1864», p. IX,
n. 50.
} Ледве закон дістав санкцію
парляменту, як пани фабриканти вже зробили відкриття: «Той
лихий стан, якого ми сподівалися від заведення фабричного закону,
не настав. Ми не бачимо, щоб продукція будь-як ослабла.
В дійсності ми продукуємо більше протягом того самого часу».\footnote{
«Reports of Insp. oî Fact. for 31 st October 1865», p. 22.
}
Отже, ми бачимо, що англійський парлямент, якому, певно,
ніхто не закине геніяльности, через досвід дійшов зрозуміння
того, що примусовий закон простими приписами може усунути
всі так звані природні перешкоди продукції щодо обмеження
й регулювання робочого дня. Тому при заведенні фабричного
закону в якійсь галузі промисловості визначається строк від 6
до 18 місяців, протягом якого справа фабрикантів є усунути
технічні перешкоди. Фраза Мірабо: «Impossible? Ne me dites
jamais ce bête de mot!»\footnote*{
Неможливо? Не говоріть мені ніколи цього дурного слова! Ред.
} стосується особливо до сучасної технології.
Але, активізуючи таким чином, мов у теплиці, розвиток
матеріяльних елементів, доконечних для перетворення мануфактурного
виробництва на фабричне, фабричний закон разом з
цим через доконечність збільшених витрат капіталу прискорює
загибіль дрібніших майстрів та концентрацію капіталу.\footnote{
«Потрібних поліпшень... не можна завести в багатьох старих
мануфактурах без таких витрат капіталу, які перевищують засоби багатьох
сучасних власників... Заведення фабричних законів неодмінно
супроводиться переходовою дезорганізацією. Розмір цієї дезорганізації
прямо пропорційний до величини того лиха, якому треба зарадити».
(Там же, стор. 96, 97).
}

Залишаючи осторонь суто технічні й технічно усовувані
перешкоди, реґулювання робочого дня наражається на безладні
\parbreak{}  %% абзац продовжується на наступній сторінці
