підручних на фабриці, то її можна заміняти почасти машинами,\footnote{
Приклад: різні механічні апарати, які позаводжувано від часів
закону 1844 р. по вовняних фабриках для заміни дитячої праці. Скоро
тільки дітям самих панів фабрикантів доведеться проходити «їхню школу»
як підручним на фабриці, то ця майже незаймана сфера механіки одразу
розвинеться до дивовижних розмірів. «Ледве чи є ще така небезпечна
машина, як от selfacting-mule. Більша частина нещасливих випадків
трапляється з малими дітьми й саме через те, що вони підлазять під мюлі
тоді, як вони в русі, щоб замести долівку. Багатьох «minders» (робітників
при мюігях) притягли (фабричні інспектори) до судової відповідальности
та засудили на грошові кари за ці провини, але без будь-якої загальної
користи. Коли б будівники машин винайшли хоч одну машину
для замітання долівки, вживання якої звільнило б цих малих дітей від
потреби лазити під машини, то це було б щасливим додатком до наших
охоронних заходів». («Reports of Insp. of Factories for 31 st October
1866», p. 63).
}
а почасти вона дозволяє — через те, що вона зовсім проста —
хутко й постійно зміняти людей, обтяжених цими муками.

Хоч машина технічно знищує стару систему поділу праці,
все ж таки остання животіє на фабриці й далі, спочатку за звичкою,
як традиція мануфактури, а потім капітал систематично
репродукує та закріпляє її в ще огидливішій формі як засіб
експлуатації робочої сили. Довічна спеціальність орудувати
частинним знаряддям стає довічною спеціяльністю служити частинній
машині. Машинами зловживають, щоб самого робітника
від дитячих років перетворювати на частину частинної машини.\footnote{
Після цього можна оцінити неймовірну вигадку Прудона, який
«конструює» машини не як синтезу засобів праці, а як синтезу частинних
праць для самих робітників. [Крім того, він робить остільки ж історичне,
як і дивовижне відкриття, що «машиновий період відзначається специфічним
характером, а саме найманою працею»].\footnote*{
Подане у прямих дужках ми беремо з французького видання. Ред.
}
}
Таким чином не тільки значно зменшуються витрати, потрібні
для його власної репродукції, але й одночасно завершується його
безпорадна залежність від фабрики, як цілости, отже, від капіталіста.
Тут, як і всюди, треба розрізняти збільшення продуктивности,
що його зумовлює розвиток суспільного процесу продукції,
від того збільшення продуктивности, що його зумовлює
капіталістичний визиск цього процесу.

В мануфактурі і в реместві знаряддя служить робітникові,
на фабриці робітник служить машині. Там рух знаряддя праці
виходить від нього, тут — він має йти за його рухом. У мануфактурі
робітники є члени живого механізму. На фабриці існує
мертвий механізм незалежно від них, а їх додають до нього як
живі додатки. «Сумна одноманітність безконечної муки праці, з
якою той самий механічний процес знову й знову повторюється, є
подібна до муки Сізіфової; тягар праці немов скеля знову й знову
спадає на знесилених робітників».\footnote{
«F. Engels: «Die Lage der arbeitenden Klasse in England», Leipzig
1845, S. 217 (Ф. Енґельс: «Становище робітничої кляси в Англії»,
Партвидав «Пролетар», 1932 р., стор. 187). Навіть цілком ординарний,
«оптимістичний фритредер, пан Молінарі, „зауважує: «Людина, догля-
} Виснажаючи до крайности