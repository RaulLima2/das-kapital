ництва. Вільні американці, що сами обробляють землю, мають одночасно ще багато інших занять. Частину
вживаних ними меблів і знарядь вони звичайно виготовлюють сами. Вони часто будують свої власні
будинки й постачають продукти своєї власної
промисловости на якнайдальші ринки. Вони одночасно прядуни й ткачі, вони фабрикують мило й свічки,
взуття і одяг для свого власного вжитку. В Америці рільництво часто є побічне заняття коваля,
мірошника або крамаря».\footnote{
Там же, стор. 21, 22.
} Де ж тут лишається серед таких чудаків поле для «поздержливости»
капіталіста?

Велика принадність капіталістичної продукції в тому, що вона не лише постійно репродукує найманого
робітника як найманого робітника, але й пропорційно до акумуляції капіталу завжди продукує відносне
перелюднення найманих робітників.
Таким чином закон попиту й подання праці утримується в належній колії, коливання заробітної плати
вганяється у межі, вигідні для капіталістичної експлуатації, і, нарешті, ґарантується стільки
доконечну соціяльну залежність робітника від
капіталіста, те відношення абсолютної залежности, що його політико-економ може у себе дома, в
метрополії, пишномовно перебріхувати на вільне договірне відношення між покупцем і продавцем, між
двома однаково незалежними посідачами товарів,
посідачем товару капітал і посідачем товару праця. Але в колоніях ця чудова ілюзія зникає. Абсолютна
кількість людности тут зростає куди швидше, ніж у метрополії, бо багато робітників приходить тут на
світ уже дорослими, і все ж ринок праці тут завжди неповний. Закон попиту й подання праці тут цілком
крахує. З одного боку, старий світ постійно вкидає туди капітал, жаждущий експлуатації, охоплений
потребою в
поздержливості; з другого боку, реґулярна репродукція найманих робітників як найманих робітників
наражається на якнайнеприємніші й почасти непереможні перешкоди. Де ж тут думати про продукцію
зайвих найманих робітників пропорційно
до акумуляції капіталу! Сьогоднішній найманий робітник на завтра стає незалежним, самостійно
господарюючим селянином або ремісником. Він зникає з ринку праці, та тільки не в робітний дім. Це
постійне перетворювання найманих робітників на незалежних продуцентів, що працюють не на капітал, а
на самих себе, і збагачують не пана капіталіста, а самих себе, із свого боку надзвичайно шкідливо
впливає на стан
ринку праці. Не тільки ступінь експлуатації найманого робітника лишається до непристойности низький.
Найманий робітник, крім цього, втрачає разом із своєю залежністю й почуття залежности від
поздержливого капіталіста. Відси всі ті прикрості,
що їх так відважно, так пишномовно й так зворушливо змальовує нам Е. Ґ Векфілд.

Подання найманої праці, — скаржиться він, — і непостійне, і нерівномірне, і недостатнє. Воно «не
лише завжди занадто