\index{i}{0479}  %% посилання на сторінку оригінального видання
Отже, ми припускаємо тут, з одного боку, що капіталіст,
який продукує товар, продає його за його вартістю, і не будемо
далі спинятися на його повороті до товарового ринку, ні на
нових формах, яких капітал набирає в сфері циркуляції, ані на
захованих у них конкретних умовах репродукції. З другого боку,
ми розглядаємо капіталістичного продуцента як власника цілої
додаткової вартости або, коли хочете, як представника всіх учасників
у тій здобичі. Отже, ми розглянемо насамперед акумуляцію
абстрактно, тобто просто як момент безпосереднього процесу
продукції.

А втім, оскільки відбувається акумуляція, остільки капіталістові
вдається продати випродукований товар та вторговані
від цього продажу гроші перетворити знову на капітал. Далі,
розпад додаткової вартости на різні частини не змінює нічого в
її природі, ні в тих доконечних умовах, що в них вона стає елементом
акумуляції. Хоч у якій пропорції капіталістичний продуцент
затримуватиме додаткову вартість для себе самого, або
відступатиме її іншим, первісно він завжди присвоює її собі.
Отже, те, що ми припускаємо при нашому розгляді акумуляції,
припускається самим справжнім перебігом акумуляції. З другого
боку, розпад додаткової вартости та упосереднювальний рух циркуляції
затемнюють просту основну форму процесу акумуляції.
Тому чиста аналіза процесу акумуляції вимагає залишити покищо
осторонь всі ті явища, які приховують внутрішню гру його
механізму.

Розділ двадцять перший
Проста репродукція

Хоч яка буде суспільна форма процесу продукції, він мусить
бути безперервний, тобто мусить періодично знову й знов перебігати
ті самі стадії. Суспільство не може перестати продукувати,
як не може воно перестати споживати. Тому всякий процес
суспільної продукції, розглядуваний в його постійному зв’язку
та постійній течії його відновлення, є разом з тим процес репродукції.
Умови продукції є разом з тим і умови репродукції. Ніяке
суспільство не може безупинно продукувати, тобто репродукувати,
не перетворюючи безупинно певної частини своїх продуктів
знову на засоби продукції або на елементи нової продукції.
За інших незмінних обставин воно може репродукувати своє багатство
в тому самому маштабі або зберігати його лише тоді, коли
воно зужитковані, приміром, протягом року засоби продукції,
тобто, засоби праці, сировинні матеріяли та допоміжні матеріяли,
заміщує in natura однаковою кількістю нових екземплярів того
самого роду, що їх воно відділяє від річної маси продукту та
знову заводить у процес продукції. Отже, певна кількість річного
продукту належить до продукції. Призначена з самого початку
для продуктивного споживання, та кількість існує здебільша
\parbreak{}  %% абзац продовжується на наступній сторінці
