\parcont{}  %% абзац починається на попередній сторінці
\index{i}{0008}  %% посилання на сторінку оригінального видання
суспільно поділена, але продукти в наслідок цього не стають товарами.
Або, щоб узяти ближчий приклад, на кожній фабриці
праця є систематично поділена, але цей поділ постає не з того,
що робітники обмінюються своїми індивідуальними продуктами.
Тільки продукти самостійних і незалежних одна від однієї приватних
праць протистоять один одному як товари.

Отже, ми бачили, що в споживній вартості кожного товару
міститься якась певна доцільна продуктивна діяльність, або корисна
праця. Споживні вартості не можуть протистояти одна
одній як товари, коли в них не містяться якісно різні корисні
праці. У такому суспільстві, продукти якого загально набирають
форми товарів, тобто в суспільстві товаропродуцентів, ця якісна
ріжниця корисних праць, які виконується незалежно одну від
однієї як приватні справи самостійних продуцентів, розвивається
в багаточленову систему, в суспільний поділ праці.

Сурдутові зрештою байдуже, чи його носитиме кравець, чи
той, хто його замовив. В обох випадках він функціонує як споживна
вартість. Так само мало змінюється само по собі відношення
між сурдутом і тією працею, яка його витворила, в наслідок
того, що кравецтво стає особливою професією, самостійним
членом суспільного поділу праці. Там, де потреба в одежі примушувала
людину кравцювати, вона кравцювала цілі тисячоліття,
раніш ніж з людини зробився кравець. Але буття сурдута,
полотна, всякого елементу речового багатства, що його природа
не дає в готовому вигляді, завжди мусило бути наслідком спеціяльної,
доцільної продуктивної діяльности, яка пристосовує окремі
природні матеріяли до окремих людських потреб. Тому праця,
як творець споживних вартостей, як корисна праця, є незалежна
від усіх суспільних форм умова існування людини, вічна природна
доконечність, що упосереднює обмін речовин між людиною і природою,
тобто людське життя.

Споживні вартості: сурдут, полотно тощо, коротко — товарові
тіла, є сполуки двох елементів: природної речовини і праці. Коли
відлічити загальну суму всіх різних корисних праць, що містяться
в сурдуті, полотні й ін., то завжди лишатиметься матеріяльний
субстрат, даний природою без участи людини. Людина може оперувати
у своїй продукції лише так, як сама природа, тобто може
лише змінювати форми речовин\footnote{
«Всі світові явища, хоч походять вони від людини, хоч від самих
загальних фізичних законів, не дають нам поняття про дійсну творчість,
а є лише перетворення речовин. Сполука й розділ — ось ті однісінькі
елементи, що їх людський розум завжди находить, аналізуючи ідею репродукції;
так стоїть справа при репродукції вартости (споживної вартости,
хоч Verri в цій своїй полеміці з фізіократами сам не знає гаразд, про яку
саме вартість він говорить) і багатства, коли земля, повітря й вода на
ниві перетворюються на збіжжя, а перероблені людською рукою клейкі
слизоти деяких комах перетворюються на шовк, або поодинокі кусні
металю сполучаються докупи й стають годинником». («Tutti і fenomeni
dell' universo, sieno essi prodotti dell’uomo, ovvero delle universali leggi
della fisica, non ci danno idea di attuale creazione, ma unicamente di una
modificazione della maieria. Accostare e separare sono gli unici elementiche
l’ingegno umano ritrova analizzando l’idea della riproduzione; e tanto
è riproduzione di valore e di ricchezze se la terra, l’aria e l’acqua ne’campi
si trasmutino in grano, come se colla mano dell'uomo il glutine di un insetto
si trasmuti in velluto ovvero alcuni pezzetti di metallo si organizzino a formare
una ripetizione»). (\emph{Pietro Verri}: «Meditazioni sulla Economie Politica»,
вперше надруковано року 1773 — y виданні італійських економістів
von Custodi, Parte Modema, t. XV, p. 22).
}. Навіть більше: в самій цій праці
\index{i}{0009}  %% посилання на сторінку оригінального видання
формування людину постійно підтримують сили природи. Отже,
праця не є однісіньке джерело продукованих нею споживних вартостей,
речового багатства. Праця є його батько, як каже Вільям
Петті, а земля — його мати.

Перейдімо тепер від товару як предмету споживання до товарової
вартости.

Згідно з нашим припущенням, сурдут має удвоє більшу вартість,
ніж полотно. Та це лише кількісна ріжниця, яка нас поки
не цікавить. Тому ми нагадуємо, що коли вартість одного
сурдута удвоє більша від вартости 10 метрів полотна, то 20 метрів
полотна мають вартість такої самої величини, як один сурдут. Як
вартості, сурдут і полотно — речі однакової субстанції, об’єктивні
вирази однорідної праці. Але кравецтво й ткацтво є якісно
різні праці. Однак, бувають такі суспільні обставини, коли та
сама людина займається навперемінку то кравецтвом, то ткацтвом,
і тому ці дві різні форми праці є лише модифікації праці
того самого індивіда, а не є ще осібні сталі функції різних індивідів,
цілком так само, як сурдут, пошитий нашим кравцем сьогодні,
і штани, які він пошиє завтра, є лише відміни тієї самої
індивідуальної праці. Наочний досвід навчає нас далі, що в нашому
капіталістичному суспільстві, залежно від змін напряму
в попиті на працю, певна кількість людської праці постачається
навперемінку то у формі кравецтва, то у формі ткацтва. Ця зміна
форм праці, звичайно, не відбувається без тертя, але вона мусить
відбуватись. Коли залишити осторонь визначеність продуктивної
діяльности, а тому й корисний характер праці, то в ній зостається
те, що вона є затрата людської робочої сили. Хоч кравецтво й
ткацтво є якісно різні форми продуктивної діяльности, а проте
обидва вони є продуктивна затрата людського мозку, мускулів,
нервів, рук тощо, і в цьому розумінні вони обидва є людська праця.
Це тільки дві різні форми затрати людської робочої сили. Певна
річ, сама людська робоча сила мусить бути більш або менш розвинена,
щоб витрачатись у тій чи іншій формі. Але вартість товару
репрезентує просто людську працю, затрату людської праці
взагалі. Як у буржуазному суспільстві генерал або банкір відіграють
велику, а просто людина, навпаки, дуже мізерну ролю\footnote{
Порівн. \emph{Hegel}: «Philosophie des Rechts», Berlin 1840, cтop. 250,
§ 190.
},
так само стоїть тут справа і з людською працею. Вона є затрата
простої робочої сили, що її пересічно має в своєму тілесному
організмі кожна звичайна людина без особливої підготови. Сама
\emph{проста пересічна праця}, хоч вона й змінює свій характер у різних
країнах і в різні культурні епохи, проте вона є визначена в кожному
\index{i}{0010}  %% посилання на сторінку оригінального видання
даному суспільстві. Складна праця — це тільки \emph{піднесена
до ступеня}, або ж, скорше, \emph{помножена} проста праця, так що менша
кількість складної праці дорівнює більшій кількості простої праці.
Що таке зведення складної праці до простої відбувається постійно,
— це показує досвід. Один товар може бути продуктом якнайскладнішої
праці, його \emph{вартість} робить його рівним продуктові
простої праці, і тому вона сама репрезентує лише певну кількість
простої праці\footnote{
Читач мусить мати на увазі, що тут мова не про заробітну плату
або вартість, яку робітник дістає, приміром, за один робочий день, а про
вартість товарів, що в ній його робочий день упредметнюється. Категорія
заробітної плати взагалі ще не існує на цьому ступені нашого досліду.
}. Різні пропорції, що в них різні роди праці зводяться
до простої праці як одиниці їхньої міри, установлюються
через суспільний процес за спиною продуцентів і тому видаються
їм даними традицією. Для спрощення ми далі вважатимемо всякий
рід робочої сили безпосередньо за просту робочу силу, а це
тільки заощадить нам роботу з цією редукцією.

Отже, як у вартостях «сурдут» і «полотно» залишено осторонь
ріжницю їхніх споживних вартостей, так само і в працях,
репрезентованих цими вартостями, залишено осторонь ріжницю
між їхніми корисними формами — між кравецтвом і ткацтвом.
Так само, як споживні вартості, «сурдут» і «полотно» є сполуки
певних доцільних продуктивних діяльностей із сукном і пряжею,
а вартості «сурдут» і «полотно», навпаки — тільки згустки
однорідної праці, так само і вміщені в цих вартостях праці мають
силу не в наслідок їхнього продуктивного відношення до сукна
й пряжі, а тільки як затрати людської робочої сили. Кравецтво
й ткацтво є елементи, що творять споживні вартості «сурдут»
і «полотно» саме в наслідок їхніх різних якостей; субстанцією
вартостей сурдута й полотна вони є лише остільки, оскільки залишається
осторонь їхні осібні якості й оскільки обидва вони мають
однакову якість, якість людської праці.

Але сурдут і полотно — не тільки вартості взагалі, а вартості
визначеної величини, і, за нашим припущенням, сурдут вартий
удвоє більше, ніж 10 метрів полотна. Звідки ця ріжниця у величинах
їхньої вартости? Від того, що 10 метрів полотна містять у
собі тільки половину тієї праці, що її має сурдут, так що на продукцію
цього останнього мусять витрачати робочої сили протягом
удвоє більшого часу, ніж на продукцію першого.

Отже, коли щодо споживної вартости вміщена в товарі праця
має значення лише своєю якістю, то щодо величини вартости вона
має значення лише своєю кількістю, скоро її вже зведено на людську
працю без особливої якости. Там ідеться про те, як витрачається
праця і що вона продукує, тут — скільки її витрачено,
протягом якого часу. А що величина вартости якогось товару
виражає лише кількість праці, що міститься в ньому, то товари
в певній пропорції мусять бути завжди рівновеликими вартостями.

Коли продуктивна сила, приміром, усіх корисних праць, потрібних
на продукцію сурдута, лишається незмінна, то величина
\parbreak{}  %% абзац продовжується на наступній сторінці
