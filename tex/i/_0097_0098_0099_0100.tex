\index{i}{0097}  %% посилання на сторінку оригінального видання
Розвиток грошей як засобу платежу примушує нагромаджувати
гроші, потрібні на терміни платежу позичених сум. Тимчасом
як скарботворення, як самостійна форма збагачення, зникає
разом із проґресом буржуазного суспільства, воно, навпаки,
зростає разом з ним у формі резервного фонду засобів платежу.

\subsubsection{Світові гроші}

Виходячи поза межі внутрішньої сфери циркуляції, гроші
знову скидають із себе придбані там місцеві форми маштабу
цін, монети, розмінної монети і знаків вартости — і повертаються
назад до своєї первісної форми зливків благородних
металів. У світовій торговлі товари розгортають універсально
свою вартість. Тому тут їхня самостійна форма вартости і протистоїть
їм як світові гроші. Лише на світовому ринку гроші функціонують
у повному обсягу як товар, натуральна форма якого
є разом з тим безпосередня суспільна форма здійснення людської
праці in abstracto. Спосіб буття їх стає адекватним їх поняттю.

У сфері внутрішньої циркуляції лише один товар може служити
за міру вартости, отже, і за гроші. На світовому ринку
панує подвійна міра вартости — золото й срібло.\footnote{
Звідси зрозуміле безглуздя кожного законодавства, що приписує
національним банкам нагромаджувати лише той благородний металь,
що всередині країни функціонує як гроші. Відомі, наприклад, ті «любі
перешкоди», які таким чином Англійський Банк створив собі сам. Про
великі історичні епохи у зміні відносної вартости золота й срібла див.
\emph{Карл Маркс}: «Zur Kritik der Politischen Oekonomie», S. 136 і далі. («До
критики і т. д.», ДВУ, 1926 р. стор. 170 і далі). — Додаток до другого видання:
Сер Роберт Пілл у своєму банковому акті з 1844 р. силкувався зарадити
цьому лихові тим, що дозволив Англійському Банкові випустити банкноти
під забезпечення срібними зливками, однак так, щоб запас срібла
не був ніколи більший за четвертину запасу золота. При тому вартість
срібла цінувалося за його ринковою ціною (в золоті) на лондонському
ринку. [До четвертого видання. — Ми живемо знову в добу сильної зміни
відносної вартости золота й срібла. Приблизно перед 25 роками вартостеве
відношення золота до срібла рівнялось $15\sfrac{1}{2} : 1$, тепер\footnote*{
1890 р. — рік виготовлення 4 видання. \emph{Ред.}
} воно приблизно
дорівнює $22 : 1$, і срібло ще далі падає проти золота. Це є, головним чином,
наслідок перевороту в способах продукції обох металів. Раніш золото
видобувалось майже виключно через промивання алювіальних
верств, що мали в собі золото, тобто через промивання продуктів вивітрювання
золотодайних мінералів. Тепер цієї методи вже не вистачає,
і цю методу відсунуло на задній плян розроблення самих золотодайних
жил кварцу, що, правда, вже було добре відоме за старовини (Діодор, III,
12--14), але посідало раніше лише другорядне місце. З другого боку, не
тільки відкрито величезні нові поклади срібла в Скелястих горах Західньої
Америки, але проведено до них і до мехіканських копалень срібла
залізниці, уможливлено довіз туди новітніх машин і палива, а через це
й видобування срібла в найширшому маштабі і з меншими витратами.
Але є велика ріжниця в тому, як обидва металі трапляються в рудних
жилах. Золото здебільшого трапляється в суцільному вигляді, але зате воно
порозкидане у кварці в незначнісіньких кількостях; тому цілу жилову
породу треба потовкти й золото вимити або витягти за допомогою живого
срібла. На 1 мільйон грамів кварцу часто припадає тоді ледве 1--3,
дуже рідко 30--60 грамів золота. Срібло рідко трапляється в суцільному
вигляді, але зате в осібних рудах, що порівняно легко відділяються від жилової
породи й мають здебільшого від 40 до 90\% срібла; або воно міститься
в менших кількостях у мідяних, оливових рудах тощо, які вже сами собою
варті розроблення. Вже звідси виходить, що тимчасом як праця,
витрачувана на видобування золота, скорше збільшилась, праця, витралювана
на видобування срібла значно зменшилась, отже, і зниження
вартости срібла пояснюється цілком природно. Це зменшення вартости
виразилося б іще в більшому спаді цін, якби ціни на срібло ще й тепер
не підтримувано на високому рівні штучними засобами. Але срібні поклади
Америки покищо приступні до розроблення ще в малій частині, і таким
чином є всі надії на те, що вартість срібла ще довгий час падатиме. Падінню
вартости срібла ще більше мусить сприяти відносне зменшення попиту
на срібло для предметів споживання й розкошів, заміна срібла товарами
з накладного срібла, алюмінієм тощо. З цього можна оцінити утопізм
біметалевих уявлень, що, мовляв, міжнародній примусовий курс
знов піднесе срібло до старого відношення вартости $1 : 15\sfrac{1}{2}$. Скорше
срібло дедалі більш і більш утрачатиме свою грошову функцію на світовому
ринку. —\emph{ Ф. Е.}].
}

\index{i}{0098}  %% посилання на сторінку оригінального видання
Світові гроші функціонують як загальний засіб платежу,
загальний засіб купівлі і абсолютно суспільна матеріялізація
багатства взагалі (universal wealth). Функція засобу платежу
для вирівнювання інтернаціональних балянсів,переважає. Звідси
гасло меркантильної системи — торговельний балянс!\footnote{
Противники меркантильної системи, яка за мету світової торговлі
вважала вирівнювання торговельного балянсу золотом'і сріблом, з свого
боку зовсім не розуміли функції світових грошей. Як неправильне розуміння
законів, що регулюють масу засобів циркуляції, лише відбивається
у неправильному розумінні міжнароднього руху благородних металів,
це я докладно показав на прикладі Рікарда («Zur Kritik der Politischen
Oekonomie», S. 150 і далі. — «До критики політичної економії», ДВУ
1926 р., стор. 186 і далі). Його помилкову догму: «Несприятливий торговельний
балянс завжди постає лише в наслідок надлишку засобів циркуляції..
Вивіз монети викликає її дешевина, і він є не наслідок, а причина
несприятливого балянсу» («An unfavourable balance of trade never
arises but from a redundant currency\dots{} The exportation of the coin is caused
by its cheapness, and is not the effect, but the cause of an unfavourable
balance»). (\emph{Ricardo}: «The high Price of Bullion etc.», p. 11, 12, 14)
находимо ми вже в Барбона: «Торговельний балянс, коли такий є, не
є причина вивозу грошей за кордон з якоїсь країни, вивіз грошей відбувається
в наслідок ріжниці вартостей грошового металю в різних країнах»
(«The Balance of Trade, if there be one, is not the cause of sendingaway
the money out of a nation; but that proceeds from the difference of the value
of Bullion in every country»). (\emph{N. Bardon}: «A Discourse on coining
the new money lighter», London 1696, p. 59). \emph{Мак Куллох} «The Literature
of Political Economy, a classified catalogue», London 1845, хвалить
Барбона за цю антиципацію теорії Рікарда, алеж розсудливо оминає
навіть згадувати про ті наївні форми, що в них у Барбона ще виявляються
абсурдні припущення «currency principle». Безкритичність і навіть нечесність
цього каталога доходить апогею в розділах про історію теорії
грошей, бо Мак Куллох тут крутить хвостом як сикофант лорда Оберстона
(ех-банкіра Лойда), що його він називав «facile, princeps argentariorum.\footnote*{
— безперечним головою банкірів. \emph{Ред.}
}
} Золото й срібло служать за міжнародній засіб купівлі головним чином
тоді, коли раптом порушиться звичну рівновагу обміну речовин
між різними націями. Нарешті, вони функціонують як абсолютна
суспільна матеріялізація багатства тоді, коли йдеться не
про купівлю і не про платіж, а про перенесення багатства з однієї
\index{i}{0099}  %% посилання на сторінку оригінального видання
країни до іншої, і коли це перенесення в товаровій формі неможливе
або в наслідок коньюнктур товарового ринку, або через
саму мету, яку воно має досягти.\footnote{
Наприклад, при субсидіях, грошових позиках на ведення війни
або на віднову банкових платежів готівкою тощо вартість може бути потрібна
саме у грошовій формі.
}

Кожна країна потребує певного резервного фонду для циркуляції
на світовому ринку, так само, як для своєї внутрішньої
циркуляції. Отже, функції скарбів виникають почасти з функції
грошей як внутрішнього засобу циркуляції й платежу, почасти
з їхньої функції як світових грошей.\footnoteA{
Примітка до другого видання. «Справді, на мою думку, ледве чи
можна бажати яснішого доказу здатности механізму резервних фондів
у країнах а металевою циркуляцією покривати всі доконечні міжнародні
платежі без якоїсь помітної підтримки з боку загальної циркуляції, ніж
та легкість, з якою Франція, ледве очунявши від руйнаційного ворожого
нападу, виплатила протягом 27 місяців приблизно 20 мільйонів фунтів
стерлінґів контрибуції, що наклали на неї союзні держави; до того значну
частину цієї контрибуції вона виплатила дзвінкою монетою без жодного
помітного скорочення або порушення свого внутрішнього обігу, навіть
без будь-яких занепокійливих коливань її векселевого курсу». (I would
desire, indeed, no more convincing evidence of the competency of the machinery
of the hoards in specie-paying countries to perform every necessary
office of international adjustment, without any sensible aid from the general
circulation, than the facility with which France, when but just recovering
from the shock of a destructive foreign invasion, completed within the
space of 27 month the payment of her forced contribution of nearly 20 millions
to the allied powers, and a considerable proportion of that sum in
specie, without perceptible contraction of derangement of her domestic currency,
or even any alarming fluctuation of her exchange»). (\emph{Fullarton}:
«Regulation of Currencies». 2 nd ed. London 1845, p. 191). [До четвертого
видання. — Ще разючіший приклад маємо в тій легкості, з якою та сама
Франція в 1871--73 рр. мала силу заплатити протягом 30 місяців більш
ніж десятикратне воєнне відшкодування так само значною мірою в металевих
грошах. — \emph{Ф. Е.}].
} Для цієї останньої
ролі завжди потрібен дійсний грошовий товар — тіло золота
й срібла, через що Джемс Стюарт виразно характеризує золото
й срібло, на відміну від їхніх лише місцевих заступників, як
money of the world.\footnote*{
— світові гроші. \emph{Ред.}
}

Рух потоку золота й срібла є подвійний. З одного боку, він
шириться від своїх джерел через цілий світовий ринок, де його
в різному обсягу вловлюють різні національні сфери циркуляції,
щоб він увіходив у їхні внутрішні канали циркуляції, заміняв
стерті золоті й срібні монети, постачав матеріял для люксусових
товарів і застигав у формі скарбів.\footnote{
«Гроші поділяються поміж націями відповідно до їхньої потреби
на гроші\dots{} завжди притягувані продуктами» («L’argent se partage
entre les nations relativement au besoin qu’elles en ont\dots{} étant toujours
attiré par les productions»), (\emph{Le Trosne}: «De l’Intérêt Social». Physiocrates,
éd. Daire, Paris. 1846, p. 916). «Копальні, що безперервно
постачають золото й срібло, дають їх досить для того, щоб дати таку
доконечну кількість кожній нації» («The mines which are continually giving
gold and silver, do give sufficient to supply such a needful balance to every
nation». (\emph{I. Vanderlint}: «Money answers all Things», London 1734, p. 40).
} Цей перший рух упосереднюється
безпосереднім обміном національних праць, зреалізованих
\index{i}{0100}  %% посилання на сторінку оригінального видання
у товарах, на зреалізовану у благородних металях працю
країн, що продукують золото й срібло. З другого боку, золото
й срібло безупинно пливуть туди й сюди поміж різними національними
сферами циркуляції, — рух, що йде слідом за безнастанними
коливаннями векселевого курсу.\footnote{
«Векселеві курси підносяться й падають щотижня, в певні моменти
року вони особливо сприятливі для однієї нації, в інші моменти
остільки ж сприятливі для іншої нації» («Exchanges rise and fall every
week, and at some particular times in the year run high against a nation,
and at other times run as high on the contrary»). (\emph{N. Barbon}: «A Discourseconcerning
coining the new money lighter», London 1696, p. 39).
}

Країни з розвиненою буржуазною продукцією обмежують
скарби, що масами сконцентровані по банкових резервуарах,
потрібним для їхніх специфічних функцій мінімумом.\footnote{
Ці різні функції можуть дійти небезпечного конфлікту, скоро
тільки до них долучиться функція розмінного фонду для банкнот.
} За деяким винятком, надзвичайне переповнення скарбових резервуарів
понад їхній пересічний рівень свідчить про застій товарової
циркуляції або про перерив течії товарової метаморфози.\footnote{
«Кількість грошей, що перевищує абсолютно потрібну кількість
для внутрішньої торговлі, є мертвий капітал і не приносить країні, що
ними володіє, жодного зиску, хіба тільки тоді, коли їх ввозиться
або вивозиться через торговельні операції» («What money is more than
of absolute necessity for a Home Trade, is dead stock, and brings no profit
to that country it’s kept in, but as it is transported in Trade, as well as
imported»). (\emph{John Bellers}: «Essays about the Poor», London 1669, p. 12).
«Що маємо робити, коли в нас є забагато монет? Ми можемо тоді найваговитіші
з них знов перетопити й перетворити в люксусові полумиски,
посуд і начиння з золота й срібла; або надіслати їх як товари туди, де їх
потребують і є на них попит; або позичити їх на проценти туди, де рівень
проценту високий» («What if we have too much coin? We may melt down
the heaviest and turn it into the splendour of plate, vessels or ustensils of
gold and silver; or send it out, as a commodity, where the same is wanted
or desired; or let it out at interest, where interest is high»). (\emph{W. Petty}:
«Quantulumcunque concerning Money», 1682, p. 7). «Гроші — це не
більш, як жир політичного тіла; тому надлишок їх часто заважає його
рухливості, а недостача наводить на нього недугу; як жир мастить мускули
і полегшує їх рух, відживляє за недостачі їжі, заповнює порожнечу й
прикрашає тіло, так само і гроші прискорюють діяльність державного
тіла, відживляють чужоземним продуктом підчас неврожаю в себе дома,
вирівнюють рахунки\dots{} і оздоблюють ціле. А проте, — іронічно закінчує
автор, — останнє стосується переважно до тих осіб, що мають забагато
грошей». («Money is but the fat of the Body Politick, whereof too much
does as often hinder its agility, as too little makes it sick\dots{} as fat lubricates
the motion of the muscles, feeds in want of victuals, fills up uneven cavities,
and beautifies the body; so doth money in the state quicken its actions,
feeds from abroad in time of dearth at home; evens accounts\dots{} and beautifies
the whole; although», ironisch abschlissend, «more especially the particular
persons that have it in plenty»). (\emph{W. Petty}: «Political Anatomy
of Ireland, 1672». Ed. London, p. 14).
}

