між 1801 і 1831 рр. загарбовано в нього і парляментськими актами подаровано лендлордами лендлордам?

Нарешті, останнім великим процесом експропріяції земель у рільників був так званий Clearing of
Estates («очищення маєтків», у дійсності очищення їх від людей). «Очищення» — це кульмінаційний
пункт усіх розглянутих досі англійських метод експропріяції. Як ми бачили в попередньому відділі,
при розгляді сучасних відносин, тепер, коли вже немає незалежних селян, яких треба було б проганяти,
справа доходить до «очищення» землі від котеджів, так що рільничі робітники на оброблюваній ними
землі навіть не находять уже місця, потрібного для їхніх жител. А що означає «Clearing of Estates» у
власному значенні слова, це ми можемо пізнати лише в горішній Шотляндії, цій обітованій країні
сучасної літератури романів. Там цей процес
визначається своїм систематичним характером, тим широким маштабом, що в ньому він воднораз
відбувається (в Ірляндії землевласники за одним разом очищають землю від декількох сел; у горішній
Шотляндії йдеться про очищення земельних просторів
завбільшки як німецькі герцоґства) — і, нарешті, особливою формою загарбовуваної земельної
власности.

Кельти горішньої Шотляндії складалися з кланів, з яких кожний був власником заселеної ним землі.
Представник клану, його голова, або «великий чоловік», був тільки номінальним власником цієї землі,
цілком так само, як англійська королева є номінальна власниця всієї національної території. Коли
англійському урядові пощастило придушити внутрішні війни поміж цими «великими чоловіками» і їхні
постійні напади на рівнини долішньої Шотляндії, то голови кланів зовсім не покинули свого старого
розбійницького ремества; вони тільки змінили його форму. Своєю власною владою вони перетворили своє
номінальне право власности на приватне право власности, а що вони при цьому наражались на опір з
боку членів клану, то вони вирішили просто прогнати їх силою. «Якийбудь англійський король міг би з
таким самим правом претендувати на те, щоб позаганяти своїх підданих у море» — каже професор
Ньюмен.213 Цю революцію що почалась у Шотляндії після останнього повстання претендента, можна
простежити в її перших фазах у творах сера Джемса Стюарта214 й Джемса Андерсона.215 У XVIII сто-

213 «А king of England migth as well claim to drive all his subjects into the sea». (F. W. Newman:
«Lectures on Political Economy», London 1851, p. 132).

214 Стюарт каже: «Ренти з цих земель (він помилково переносить цю економічну категорію на данину, що
її taskmen* платить голові клану) є цілком незначні порівняно з розмірами останніх; щождо числа
осіб, які живуть з оренди, то мабуть виявиться, що шматок землі в гірських місцевостях Шотляндії
прогодовує вдесятеро більше людей, аніж земля такої самої вартости в найбагатших провінціях».
(«Works etc., ed. by General Sir James Steuart, his son», London 1801, vol. I, ch. 16, p. 104).

215 James Anderson: «Observation on the means of exciting a spirit of National Indusrty etc.»,
Edinburgh 1777.

* — васаль. Ред.
