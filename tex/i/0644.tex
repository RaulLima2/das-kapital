великих фармах зовсім не пізнати, що вони постали із об’єднання
багатьох дрібних майстерень і через експропріацію багатьох
дрібних незалежних продуцентів. Однак безстороннього глядача
це не введе в обман. За часів Мірабо, цього лева революції, великі
мануфактури ще називалися manufactures réunies, тобто об’єднаними
майстернями, як ми говоримо про об’єднані поля. «Звертають
увагу, — каже Мірабо, — лише на великі мануфактури,
що в них сотні людей працюють під керівництвом одного директора
і що їх звичайно звуть об’єднаними мануфактурами (manufactures
réunies). Навпаки, на ті майстерні, де працює дуже багато
робітників порізно і кожний на власну руку, навряд чи
хто й оком скине. їх зовсім відсувають на задній плян. Цедуже
велика помилка, бо лише вони становлять дійсно важливу складову
частину народнього багатства... Об’єднана фабрика (fabrique
réunie) на диво збагачує одного або двох підприємців, але
робітники — це лише краще або гірше оплачувані поденники,
що не беруть ніякої участи в добробуті підприємця. Навпаки,
роз’єднана фабрика (fabrique séparée) нікого не збагачує, але
зате маса робітників живе в добробуті... Число працьовитих і
хазяйновитих робітників зростатиме, бо в розумному способі
життя, в працьовитості вони бачать засіб, щоб посутньо поліпшити
своє становище, замість здобувати собі невеличке підвищення
заробітної плати, що ніколи не може мати важливого
значення для будучини і в найкращому разі тільки дозволить
робітникам трохи краще жити з дня на день. Лише роз’єднані
індивідуальні мануфактури, сполучені здебільша з дрібним сільським
господарством, є вільні».233 Експропріяція і зганяння
частини сільської людности не тільки звільняють разом із самими
робітниками їхні засоби існування і їхній матеріял праці для
промислового капіталу, а й створюють внутрішній ринок.

Справді, ті події, що перетворюють дрібних селян на найманих
робітників, а засоби їхнього існування і праці — на речові
елементи капіталу, створюють разом із цим для цього останнього
внутрішній ринок. Раніш селянська родина продукувала
і обробляла засоби існування й сировинні матеріяли, що їх вона
потім здебільшого сама ж і споживала. Ці сировинні матеріяли
й засоби існування тепер стали товарами; великий фармер продає
їх; мануфактури є його ринок. Пряжа, полотно, грубі вовняні
вироби — речі, що для них сировинний матеріял був у руках
кожної селянської родини, при чому кожна родина пряла і ткала
їх для власного споживання — перетворюються тепер на ману-

vous ayez l’honneur de me servir, à condition que vousme donnez le peu
qui vous reste pour la peine que je prends de vous commander»). (J. J-Rousseau:
«Discours sur l’Economie Politique». Geneve 1760, p. 70).

233 Mirabeau: «De la Monarchie Prussienne», Londres, 1788, vol. ni,
p. 20—109 і далі. Коли Мірабо гадає, що роз’єднані майстерні в
також економічніші й продуктивніші, ніж майстерні «об’єднані», і Щ
останні вважає лише за штучні, тепличні рослини, що виросли під охороною
уряду, то це пояснюється тодішнім станом більшої частини континентальних
мануфактур.
