\parcont{}  %% абзац починається на попередній сторінці
\index{i}{0623}  %% посилання на сторінку оригінального видання
панським «полюбовницям». Навіть наймані сільські робітники
були ще тоді співпосідачами громадських земель. Десь близько
1750~\abbr{р.} yeomanry зникли\footnote{
«А Letter to Sir Т. С. Bunbury, Brt.: On the High Price of Provisions.
By a Suffolk Gentleman», Ipswich 1795 p. 4. Навіть фанатичний
оборонець великого фармерства, автор «Inquiry into the Connection
betveen the present Price of provisions and the size of Farms», London
1773, p. 139, каже: «Я дуже жалкую за тим, що позникали наші yeomanry,
ця кляса людей, що дійсно підтримувала незалежність нашої нації;
з сумом бачу, що їхні землі опинились тепер у руках монополістів-лордів
і позаорендовані дрібними фармерами на умовах, мало чим кращих від
тих умов, у яких були васалі, завжди готові відгукнутися на заклик при
всякій капосній справі» («І most truly lament the loss of our yeomanry, that
set of men who really kept up the independence of this nation; and sorry I am
to see their land now in the hands of monopolizing lords, tenanted out to
small farmers, who hold their leases on such conditions as to be little better
than vassals ready to attend a summons on every mischievous occasion»).
}, а останніми десятиліттями XVIII
століття зник і останній слід громадської власности рільників.
Ми залишаємо тут осторонь суто економічні пружини революції
у рільництві. Нас цікавлять її насильні підойми\footnote*{
У французькому виданні останні два речення подано так:
«Залишаючи осторонь суто економічні впливи, що підготували експропріяцію
рільників, ми переходимо тут до підойм, вжитих, щоб насильно
прискорити цю експропріяцію». («Le Capital etc.», ch. XXVII,
p. 319). \emph{Ред.}
}.

За реставрації Стюартів землевласники законодатним шляхом
добилися тієї узурпації, яка повсюди на континенті відбулася й
без законодатних церемоній. Вони знищили февдальний земельний
лад, тобто скинули з себе всякі повинності щодо держави,
«відшкодували» державу податками, накинутими на селянство
й решту народньої маси, присвоїли собі сучасне право приватної
власности на маєтки, на які вони мали лише февдальні права,
і, нарешті, октроювали ті закони про оселення (law of settlement),
які, mutatis mutandis, мали такий самий вплив на англійських
селян рільників, як едикт татарина Бориса Годунова на російське
селянство.

«Glorious Revolution» (славетна революція) привела до влади
разом з Вільгельмом III Оранським\footnote{
Про особисту мораль цього буржуазного героя можна судити,
між іншим, ось із чого: «Обдарування леді Оркней великими маєтками
в Ірляндії у 1695~\abbr{р.} — це явний доказ королівської прихильности й впливу
леді\dots{} Милі послуги леді Окрней були, кажуть, нечисті послуги устами».
(«The large grant of lands in Ireland to Lady Orkney, in 1695, is a public
instance of the king’s affection, and the lady’s influence\dots{} Lady Orkney’s
endearing offices are supposed to have been-foeda labiorum ministeria»).
(У Sloane Manuscript Collection, у Брітанському музеї, № 4224.
Манускрипт називається: «The character and behaviour of King William,
Sunderland etc. as represented in Original Letters to the Duke of
Shrewsbury from Somers, Halifax, Oxford, Secretary Vernon etc.». У ньому
повно курйозів).
} земельних і капіталістичних
присвоювачів додаткової вартости (Plusmacher). Вони освятили
нову еру, практикуючи в колосальних розмірах крадіж
державних маєтків, що перед тим мав лише помірні розміри.
Державні землі роздаровувано, продавано за безцінь або приєднувано
\index{i}{0624}  %% посилання на сторінку оригінального видання
до приватних маєтків простою узурпацією\footnote{
«Незаконне відчуження коронних земель, почасти через продаж,
почасти через дарування, становить скандальний розділ в англійській
історії\dots{} величезне ошукання нації (gigantic fraud on the nation)».
(\emph{F. W. Newman}: «Lectures on Political Economy», London 1851, p. 129,
130). — [Подробиці про те, як сучасні англійські великі землевласники
придбали свої маєтки, див. в «Our old Nobility. By Noblesse Oblige», London
1879. — \emph{Ф. E.}].
}. Все це
робилося без найменшого додержання етикету законности. Присвоєне
таким шахрайським способом державне майно разом з
понаграбовуваним церковним майном, оскільки останнє не втрачено
підчас республіканської революції, становить основу сучасних
княжих доменів англійської олігархії\footnote{
Див., наприклад, памфлет E. Burke про герцоґську родину
Бедфордів, нащадком якої є Джон Рессел, «the tomtit of liberalism».
}. Буржуазні
капіталісти сприяли цій операції, між іншим, для того, щоб
перетворити землю на предмет вільної торговлі, поширити сферу
великої рільничої продукції, збільшити приплив із села вільних,
як птиці, пролетарів і~\abbr{т. д.} До того ж нова земельна аристократія
була природною союзницею нової банкократії, цієї фінансової
шляхти, що тільки но вилупилася з яйця, і великих мануфактуристів,
що тоді спирались на охоронні мита. Англійська буржуазія
так само правильно чинила в своїх інтересах, як шведські
міщани, що, навпаки, спільно з своєю економічною твердинею
— селянством, підтримували королів, що силоміць відбирали
від олігархії коронні землі (починаючи від 1604~\abbr{р.} й
пізніше, за Карла X й Карла XI).

Громадська власність — цілком відмінна від щойно розглянутої
державної власности — була старогерманською інституцією,
що й далі існувала під покровом февдалізму. Ми бачили
вже, як насильна узурпація цієї громадської власности, що її
здебільшого супроводило перетворення орної землі на пасовиська,
почалася наприкінці XV століття й тривала далі в XVI столітті.
Але тоді цей процес відбувався лише як індивідуальний
насильний акт, проти якого законодавство даремно боролося протягом
150 років. Проґрес XVIII століття виявляється в тому,
що тепер сам закон стає знаряддям грабування народньої землі,
хоч поруч із цим великі фармери вшивають і своїх дрібних незалежних
приватних метод\footnote{
«Фармери забороняють cottager’aм (бурлакам) тримати якубудь
живу істоту, крім самих себе, під тією причіпкою, що коли вони триматимуть
худобу або дробину, то крастимуть корм з клунь. Вони кажуть
також: тримайте cottager’iв у біді, якщо ви хочете, щоб вони були працьовитими.
Але справжній факт той, що фармери таким способом узурпують
усі права на громадські землі». («A Political Enquiry into the Consequences
of enclosing Waste Lands», London 1785, p. 75).
}. Парляментська форма цього грабування
є «Bills for Inclosures of Commons» (закони про обгороджування
громадських земель), інакше кажучи, декрети, за допомогою
яких землевласники сами собі дарують народню землю
у приватну власність, — декрети експропріяції народу. Сер Ф. М.
Ідн у своїй хитромудрій оборонній адвокатській промові
\parbreak{}  %% абзац продовжується на наступній сторінці
