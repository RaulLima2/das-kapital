Узурпація громадських випасів і т. ін. дозволяє фармерові значно
збільшити кількість своєї худоби майже без витрат, а худоба дає
йому багато угноєння для землі.

В ХУІ столітті сюди долучається ще один вирішально-важливий
момент. Того часу орендні контракти були довготермінові,
часто на 99 років. Безперервне падіння вартости благородних
металів, а тому й грошей дало фармерам золоті плоди. Передусім
воно, не кажучи вже про всі інші вищерозглянуті обставини,
понизило заробітну плату. Частину цієї заробітної плати долучувано
до фармерського зиску. Невпинний зріст цін на збіжжя,
вовну, м’ясо, одне слово, на всі рільничі продукти, збільшував
грошовий капітал фермера без жодної його участи, тимчасом
як земельна рента, яку він мусив платити, зменшувалась у наслідок
зневартнення грошей підчас тривання контракту.\footnote{
Про вплив зневартнення грошей у XVI столітті на різні кляси
суспільства див. «А Compendious or Briefe Examination of Certayne Ordinary
Complaints of Diverse of our Countrymen in these our Days. By
W. S., Gentleman», London 1581. Діялогічпа форма цього твору сприяла
тому, що його довго приписували Шекспірові, і ше року 1751 цей твір
знову видано під його йменням. Автор цього твору є Вільям Стафорд.
В одному місці лицар (knight) резонує так;

Лицар: «Ви, мій сусіде, рільнику, ви, пане крамарю, і ви, добродій
котлярю, так само і всі інші ремісники, ви можете гаразд зарадити
собі. Бо наскільки всі речі стали тепер дорожчі ніж були раніш, настільки
ви підвищуєте ціни своїх товарів і продукти своєї праці, які продаєте.
А ми не маємо нічого, що могли б продавати, ми не маємо нічого, на що
могли б підвищити ціну, щоб відшкодувати себе за ті речі, які ми мусимо
купувати». В іншому місці лицар так запитує доктора: «Скажіть мені,
будь ласка, яких саме людей ви маєте на думці? І насамперед хто, на
ваш погляд, не терпить при цьому жодних втрат?» Доктор: «Я маю на
увазі всіх тих, що живуть із купівлі й продажу; бо коли вони дорого
купують, то потім так само дорого і продають». Лицар: «А хто ті, що,
як ви кажете, виграють на цьому?» Доктор: «Звичайно, всі ті, що господарюють
на маєтках або фармах, платячи стару ренту; бо, платячи за старою
нормою, вони продають за новою, тобто вони платять за свою землю
дуже дешево, а всі продукти, що виростають на ній, продають дорого...»
Лицар: «Ну, а що ж то за люди ті, що, як ви кажете, втрачають на цьому
більше, ніж ті виграли». Доктор: «Вся шляхта, джентлмени та всі інші,
що живуть з фіксованої ренти або з фіксованого утримання, або сами
не господарюють на своїй землі, або не займаються купівлею і продажем».
(Knight: «You, my neighbour, the husbandman, you Maister Mercer,
and you Goodman Copper, with other artificers, may save yourselves
metely well. For as much as all things are deerer than they were, so much
do you arise in the pryce of your wares and occupations that yee sell agayne.
But we have nothing to sell where by we might advance ye pryce there of,
to countervaile those things that we must buy agayne». — «I pray you, what
be those sorts that ye meane. And, first, of those that yee thinke should
have no losse hereby?» — Doktor: «I meane all these that live by buying
and selling, for, as they buy deare, they sell thereafter». — Knight: «What
is the next sorte that yee say would win by it?* — Doktor: «Marry, all
such as have takings or fearmes in their owne manurance (тобто cultivation)
at the old rent, for where they pay after the olde rate, they sell after
} Таким
чином фармер багатів одночасно коштом своїх найманих

40, 50, 100 фунтів стерлінґів і вважають, що вони зробили невигідну операцію,
коли по закінченні орендного контракту вони не зможуть відкласти
для себе суми, що дорівнювала б ренті за шість-сім років».