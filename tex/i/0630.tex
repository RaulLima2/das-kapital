робилися амфібіями й шили, як каже один англійський письменник,
наполовину на землі й наполовину на воді, але і з того
і з другого вони жили тільки наполовину.\footnote{
Коли нинішня герцоґиня Сотерлендська з великою пишністю
приймала в Лондоні пані Бічер-Стов, авторку «Хижини дядька Тома»,
щоб показати свою симпатію до рабів-негрів американської республіки, —
підчас громадянської війни, коли кожне «благородне» англійське серце
співчувало рабовласникам, вона разом з іншими аристократками благорозумно
забула про цю симпатію, — я змалював у «New-York Tribune»
становище сотерлендських рабів. (Частину моєї статті Кері подав у витягах
у «The Slave Trade», London 1853, p. 202, 203). Мою статтю передруковано
в одній шотляндській газеті, і вона викликала «ввічливу» полеміку
поміж цією газетою і сикофантами Сотерлендів.
}

Але бравим ґаелам довелося ще тяжче спокутувати своє гірсько-романтичне
ідолопоклонство перед «великими людьми»
клану. Запах риби лоскотав великим людям у носі. Вони занюхали
тут щось зисковне і заорендували узмор’я великим риботорговцям
з Лондону. Ґаелів прогнано вдруге.219

Але кінець-кінцем частину овечих пасовиськ перетворено
на мисливські парки. Як відомо, в Англії нема справжніх лісів.
Дичина по парках вельмож — це конституційна домашня худоба,
гладка, як лондонський aldermen.\footnote*{
— член міської ради. Ред.
} Тим то Шотляндія є останнє
пристановище цієї «благородної пристрасти». «У гірських місцевостях,
— каже Сомерс в 1848 р., — лісова площа значно
поширилась. Тут, по цей бік Gaick’a, ви бачите новий ліс Glenfeshie,
а там, по другий бік, новий ліс Ardverikie. Там же ви маєте
й Blak-Mount, величезну пущу, нещодавно тільки заведену.
Із сходу на захід, від околиць Aberdeen’a й до скель Oban’а,
тягнеться тепер безперервна смуга лісів, тимчасом як по інших
частинах гірського краю стоять нові ліси Loch Archaig, Glengarry,
Glenmoriston і ін. Перетворення земель ґаелів на пасовиська...
загнало їх на неродючі землі. Тепер олені й сарни (Rotwild)
починають витискувати овець, кидаючи цим ґаелів у ще
гірші злидні... Мисливські парки\footnoteA{
У шотляндських «deer forests» (мисливських парках) немає жодного
дерева. Овець виганяють геть, на їхнє місце в голі гори приганяють
оленів, і це називають «deer forest». Отже, тут немає навіть лісової культури.
} й народ не можуть існувати
одне побіч одного. В усякому разі хтобудь із них мусить очистити
місце. Якщо місця для полювання протягом найближчої чверти
віку зростатимуть щодо кількости й простору, як і минулої
чверти, то жодного ґаела не залишиться на його рідній землі.
Цей рух серед землевласників гірських місцевостей спричинено
почасти модою, аристократичними примхами, мисливським запалом
тощо, а почасти землевласники торгують дичиною, маючи
на меті виключно зиск. Бо це факт, що шматок гірської землі,

219    Цікаві подробиці про цю торговлю рибою подає пан Давид Уркварт
«Portfolio, New Series». — Н. В. Сеніор y своєму вищецитованому посмертному
творі «Journals, Conversations and Essays relating to Ireland»,
London 1868, кваліфікує «процедуру в Sutherlandshire, як одне з найдобродійніших
очищень (clearings), що їх люди пам’ятають».