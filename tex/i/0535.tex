вростанням продуктивносте праці не тільки зростає розмір споживаних
нею засобів продукції, але й вартість їхня знижується
порівняно з їхнім розміром. Отже, їхня вартість зростає абсолютно,
але не пропорційно до їхнього розміру. Тому зростання
ріжниці між сталим і змінним капіталом далеко менше, ніж зростання
ріжниці між масою засобів продукції, що на них перетворюється
сталий капітал, і масою робочої сили, що на неї перетворюється
змінний капітал. Перша ріжниця більшає разом з останньою,
але в меншій мірі, ніж остання.

А втім, якщо проґрес акумуляції зменшує відносну величину
змінної частини капіталу, то цим він зовсім не виключає збільшення
її абсолютної величини. Припустімо, що капітальна вартість спочатку
розпадається на 50% сталого й 50% змінного капіталу,
а потім — на 80% сталого й 20% змінного. Коли тимчасом первісний
капітал, приміром, 6.000 фунтів стерлінґів, зріс до 18.000
фунтів стерлінґів, то його змінна складова частина також зросла
на п’ятину. Раніш вона становила 3.000 фунтів стерлінґів, тепер
вона становить 3.600 фунтів стерлінґів. Але, якщо раніш
досить було б зрости капіталові на 20%, щоб підвищити попит
на працю на 20%, то тепер для цього потрібне потроєння первісного
капіталу.

У четвертому відділі було показано, що розвиток суспільної
продуктивної сили праці має своєю передумовою кооперацію у
великому маштабі, що лише за цієї передумови можна організувати
поділ і комбінацію праці, економізувати засоби продукції
через масову концентрацію, покликати до життя такі засоби
праці, що їх уже з самої речової природи їхньої можна вживати
лише спільно, наприклад, систему машин і т. д., примусити величезні
сили природи служити продукції і перетворити процес продукції
на технологічне застосування науки. На основі товарової
продукції, де засоби продукції є власність приватних осіб і де,
отже, ручний робітник продукує товари або ізольовано й самостійно,
або продає свою робочу силу як товар, бо в нього немає
засобів до самостійної продукції, та передумова реалізується
лише через зростання індивідуальних капіталів, або в тій мірі,
в якій суспільні засоби продукції і засоби існування перетворюються
на приватну власність капіталістів. На ґрунті товарової
продукції продукція у великому маштабі може існувати лише в
капіталістичній формі. Тому певна акумуляція капіталу в руках
індивідуальних товаропродуцентів становить передумову специфічно
капіталістичного способу продукції. Тим то ми й мусили
припустити таку акумуляцію при переході від ремества до капіталістичної
продукції. Її можна назвати первісною акумуляцією,
бо вона є не історичний результат, а історична основа специфічно
капіталістичної продукції. Тут нам ще не треба досліджувати,
як виникає сама ця акумуляція. Досить сказати, що вона становить
вихідний пункт. Але всі методи підвищення суспільної
продуктивної сили праці, що виростають на цій основі, є разом
з тим методи збільшення продукції додаткової вартости або
