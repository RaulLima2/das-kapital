в бавовняних округах спеціальні комісари, керуючись спеціальними
законодавчими нормами і застосовуючи примусової праці,
щоб підтримати моральну цінність одержувачів милостині...
Чи може бути щось гірше для земельних власників або для хазяїнів
(«сап anything be worse for landowners or masters»), ніж
позбутися своїх найкращих робітників і здеморалізувати та
збентежити решту через широку спустошливу еміґрацію і спустошення
цілої провінції щодо вартости й капіталу?»

Потер, цей вибраний адвокат бавовняних фабрикантів, розрізняє
дві групи «машин», при чому і ті і другі належать капіталістові,
тільки одні стоять у його фабриці, а другі вночі й
неділями перебувають поза фабрикою в котеджах. Одні мертві,
другі живі. Мертві машини не тільки щодня погіршуються та
зневартнюються, але через невпинний технічний проґрес значна
частина з наявної маси їх постійно так дуже старіється, що їх
з вигодою і протягом небагатьох місяців можна замінити на нові.
Живі машини, навпаки, поліпшуються, що довше вони функціонують,
що більше вони від покоління до покоління нагромаджують
вправности. «Times», між іншим, так відповів цьому
фабричному маґнатові:

«Пан Е. Потер так пройнявся почуттям надзвичайної й абсолютної
ваги бавовняних хазяїнів, що для збереження цієї кляси
й увіковічнення її промислу хотів би замкнути півмільйона робітничої
кляси проти її волі у великий моральний робітний дім.
Чи варта ця промисловість того, щоб її підтримувати? — питає
Потер. Певна річ, всіма чесними засобами, — відповідаємо ми.
Чи варто тримати машини в порядку? — знову питає Потер.
Тут ми збентежені. Під машинами Потер розуміє людські машини,
бо він запевняє, що не має на думці розглядати їх як абсолютну
власність. Ми мусимо признатися, що ми не вважаємо «за варте»,
а то навіть і за можливе тримати ці людські машини в порядку,
тобто замикати їх і мастити, поки їх не потребуватимуть. Людська
машина має властивість іржавіти від бездіяльности, хоч
як дуже ви її маститимете та чиститимете. А до того людська
машина, як ми це бачимо, може сама з себе видавати пару та вибухати
або шаліти по наших великих містах у танку св. Вітта.
Можливо, як це каже Потер, і потрібен довший час на репродукцію
робітників, але, маючи машиністів і гроші, ми завжди
знайдемо заповзятих, загартованих промислових людей, щоб
зробити з них більше фабричних хазяїнів, аніж ми зможемо їх
використати... Пан Потер базікає про нове пожвавлення промисловосте
через 1, 2, 3 роки й вимагає від нас, щоб ми не заохочували
або не дозволяли еміґрації робочої сили! На його думку,
це природна річ, що робітники хочуть еміґрувати, але він вважає,
що нація мусить цих півмільйона робітників разом із 700.000 тих,
що з ними зв’язані, замкнути проти їхнього бажання в бавовняних
округах і — неминучий наслідок — придушити силою їхнє
незадоволення та підтримувати їх самих милостинею, і все це,
сподіваючись, що якоїсь днини, може, їх знову треба буде бавов-
