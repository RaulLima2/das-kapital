лем, парова машина з самого початку була антагоністом «людської
сили», що дав капіталістам змогу розбивати щораз більші
домагання робітників, які загрожували кризою фабричній системі
на самому початку її виникнення.\footnote{
Gaskell: «The Manufacturing Population of England», London
1833, p. 3, 4.
} Можна було б написати
цілу історію винаходів, які, починаючи від 1830 р., покликано
до життя лише як бойове знаряддя капіталу проти повстань робітників.
Ми нагадаємо передусім selfacting mule,\footnote*{
— автоматичну прядільну машину. Ред.
} бо нею починається
нова епоха автоматичної системи.\footnote{
Деякі дуже важливі застосування машин, щоб будувати машини,
винайшов п. Ферберн під впливом страйків на його власній фабриці.
}

У своєму свідченні перед комісією, що їй доручено було дослідити
Trades-Unions, Несміс, винахідник парового молота, повідомляє
про поліпшення в машинах, які він завів у наслідок
великого та довгого страйку машинових робітників у 1851 р.,
таке: «Характеристична риса наших сучасних механічних поліпшень
— це заведення самодіяльних виконавчих машин. Все, що
тепер має робити механічний робітник, і що може зробити всякий
підліток, — це не самому працювати, а лише наглядати за прегарною
роботою машини. Цілу клясу робітників, що залежить
виключно від своєї вмілости, тепер усунено. Раніш я на одного
механіка мав чотирьох хлопців. Завдяки цим новим механічним
комбінаціям я зменшив число дорослих чоловіків з 1.500 на 750.
Наслідком цього було значне збільшення мого зиску».

Про одну машину для друку фарбами на перкалевибійних
фабриках Юр каже: «Нарешті капіталісти почали шукати способу
визволитися з-під цієї нестерпної неволі (тобто від тяжких
для них умов контракту з робітниками), покликавши собі на допомогу
джерела науки, і незабаром їх відновили в їхніх законних
правах, правах голови над іншими частинами тіла». Про один
винахід для шліхтування основи, що його безпосередньою причиною
був страйк, він каже так: «Орда незадоволених, що, окопавшися
за старими лініями поділу праці, вважала себе за непереможну,
побачила себе таким чином оточеною з флангів, а свої
оборонні засоби знищеними сучасною механічною тактикою. Вони
мусили здатися на ласку та гнів переможців». Про винахід
selfacting mule він каже: «Вона була покликана, щоб відновити
порядок серед промислових кляс... Цей винахід потверджує
розвинуту вже нами доктрину, що капітал, примусивши науку
служити йому, завжди силує бунтівничу руку праці до покірливости».\footnote{
Ure: «Philosophy of Manufacture», стор. 367—370.
}
Хоч твір Юра з’явився 1835 р., отже, за часів порівняно
мало ще розвинутої фабричної системи, все ж він лишається клясичним
виразом духу фабрики не тільки через свій щирий цинізм,
але й через ту наївність, з якою він виказує абсурдні суперечності
капіталістичного мозку. Розвинувши, приміром, «доктрину»,
що капітал за допомогою науки, взятої ним на утримання, «завжди