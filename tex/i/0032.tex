конечному ряді всіх інших товарових тіл. Таким чином розгорнута
відносна форма вартости, або форма В, з’являється тепер
як специфічна відносна форма вартости товару-еквіваленту.

3. Перехід від загальної форми вартости до грошової форми

Загальна еквівалентна форма є форма вартости взагалі. Отже,
вона може належати кожному товарові. З другого боку, даний
товар перебуває в загальній еквівалентній формі (формі С) лише
тому й остільки, що й оскільки його, як еквівалент, виключають
з-поміж себе всі інші товари. І лише від того моменту, коли це
виключення остаточно обмежується на якомусь одному специфічному
роді товару, однорідна відносна форма вартости товарового
світу здобуває об’єктивну сталість і загальну суспільну значеність
(Gültigkeit).

А специфічний рід товару, що з його натуральною формою
суспільно зростається еквівалентна форма, стає грошовим товаром,
або функціонує як гроші. Відігравати в межах товарового
світу ролю загального еквіваленту — це стає його специфічною
суспільною функцією, а тому і його суспільною монополією. Це
упривілейоване місце серед товарів, що в формі В фігурують як
осібні еквіваленти полотна, а в формі С спільно виражають у
полотні свою відносну вартість, історично завоював певний товар,
а саме золото. Тому, коли ми в формі С замість товару
«полотно» поставимо товар «золото», то матимемо:

D. грошова форма

20 метрів полотна =
1 сурдут =
10 фунтів чаю =
40 фунтів кави =
1 квартер пшениці =
1/2  тонни заліза =
х товару А —

2 унціям золота

При переході від форми А до форми В й від форми В до форми
С відбуваються посутні зміни. Навпаки, форма D нічим не
відрізняється від форми С, як тільки тим, що тепер замість полотна
форму загального еквіваленту має золото. Золото лишається
в формі D тим, чим полотно було в формі С, — загальним
еквівалентом. Проґрес лише в тім, що форма безпосередньої загальної
вимінности, або загальна еквівалентна форма, в наслідок
суспільної звички тепер остаточно зрослася зі специфічною натуральною
формою товару «золото».

Золото виступає проти інших товарів як гроші лише тому, що
воно вже раніш протистояло їм як товар. Як і всі інші товари,
золото також функціонувало як еквівалент, чи то як одиничний
еквівалент в поодиноких актах обміну, чи то як осібний еквіва-
