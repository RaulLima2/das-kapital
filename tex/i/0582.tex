латі лордів. Е. Дж. Векфілд, найвидатніший економіст того періоду,
каже: «Сільський робітник південної Англії і не раб
і не вільна людина, — він павпер».\footnote{
«England and America», London 1833, vol. I, p. 47.
}

Час, що безпосередньо передував скасуванню хлібних законів,
по-новому освітлив становище сільських робітників. З одного
боку, в інтересі буржуазних агітаторів було показати, як
мало ті охоронні закони захищають дійсних продуцентів хліба.
З другого боку, промислова буржуазія кипіла гнівом з приводу
того, що земельні аристократи викривали умови фабричної роботи,
з приводу того, що ці наскрізь зіпсовані, безсердечні і
знатні нероби виявляли вдаване співчуття до страждань фабричного
робітника та «дипломатичний запал» до фабричного законодавства.
Є давня англійська приказка, що коли два злодії
чублять один одного, то з цього завжди буде якась користь.
І дійсно, галаслива, пристрасна суперечка поміж двома фракціями
панівної кляси про те, яка з них якнайбезсоромніше експлуатує
робітника, допомогла і справа і зліва вияснити правду. Граф
Шефтсбері, інакше лорд Ешлі, стояв на чолі аристократичного
філантропічного походу проти фабрик. Тим то в 1844 й
1845 рр. він був улюбленою темою для «Morning Chronicle», що
викривав становище рільничих робітників. Ця газета, найзначніший
тодішній ліберальний орган, надіслала до селянських округ
власних комісарів, які зовсім не задовольнилися загальним
описом і статистикою, а опублікували імена так тих робітничих
родин, що їх становище вони дослідили, як і їхніх панів-землевласників.
Нижченаведена таблиця подає заробітну плату,
яку платять у трьох селах, у сусідстві Blanford’a, Wimbourne
і Poole. Села ці — власність містера Дж. Бенкса і графа Шефтсбері.
Треба зауважити, що цей папа «low church»,\footnote*{
— низької церкви. Ред.
} цей голова англійських
пієтистів, так само як і згаданий Бенкс, із злиденної
заробітної плати робітників відбирав ще в них значну частину
під приводом плати за квартиру.

а) Дітей    b) Членів родин    с) Тижнева зароб. плата чоловіків    d) Тижнева заробітна плата дітей
   е) Тижневий дохід цілої родини    f) Тижнева квартирна плата
g) Загальний тижневий заробіток з відрахуванням квартирної плати    її) Тижневий заробіток на людину

Перше село
2    4    8 шил. —                8 шил. —     2 шил. —       6 шил.   —     1 шил. 6      п.
3    5    8    »     —                 8   »      —     1    »      6 п.   6    »       6 п.  1   « 
    З \sfrac{1}{2} »
2    4    8    »     —                 8   »      —     1    »      —      7    »       —      1   «
     9       »
2    4    8    »     —                 8   »      —     1    »      —      7    »       —      1   «
     9       »
6    8    7    »    1-1 ш. 6 п.  10 »      6п.   2    »     —       8    »       6 п.   1  «        
\sfrac{1}{4} »
3    5    7    »    1-2  » —       7   »      —     1    »      4 п.   5    »       8 «     1  «    
 1\sfrac{1}{2}»