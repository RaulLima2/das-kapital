ний спосіб продукції ґрунтувався на громадській власності, але
не на громадській власності в слов’янській або навіть індійській
формі. На одній частині землі самостійно господарювали члени
громад, як на вільній приватній власності, другу частину — ager
publicus\footnote*{
— громадську землю. Ред.
} — обробляли вони спільно. Продукти цієї спільної праці
почасти служили за резервний фонд на випадок неврожаю й інших
несподіванок, а почасти були державним фондом для покриття
коштів війни, релігійних і інших громадських видатків. З часом
військові й церковні достойники захопили разом з громадською
власністю і зв’язані з нею повинності. Праця вільних селян на
їхній громадській землі перетворилась на панщинну працю для
розкрадачів громадської землі. Відси одночасно з цим розвинулися
кріпацькі відносини, однак тільки фактично, а не юридично,
аж поки всесвітня визвольниця Росія під приводом скасування
кріпацтва піднесла їх до рівня закону. Кодекс панщинної праці,
оголошений 1831 р. від російського генерала Кісельова, звичайно,
подиктували сами бояри. Таким чином Росія одним ударом завоювала
собі маґнатів дунайських князівств і похвальні оплески
ліберальних кретинів цілої Европи.

За «Réglement organique» — так називається той кодекс панщинної
праці — кожний волоський селянин, окрім безлічі докладно
перелічених натуральних повинностей, має ще супроти так
званого земельного власника такі обов’язки: 1) дванадцять робочих
днів взагалі, 2) один день працювати на полі та 3) один день
возити дрова, — отже, разом 14 днів на рік. Однак з глибоким розумінням
політичної економії робочий день узято не в його звичайному
значенні, а як робочий день, доконечний на виготовлення
пересічного денного продукту, а пересічний денний продукт так
хитро визначено, що й жоден циклоп не впорався б з ним за
24 години. Тому сам реґлямент сухими словами з справжньою
руською іронією пояснює, що під 12 робочими днями треба розуміти
продукт 36 днів ручної праці, під одним днем роботи на полі —
три дні, під одним днем возіння дров — теж три дні, разом це
42 панщинні дні. Та сюди ще додано так звану «Jobagie» — службові
повинності, які треба виконувати на користь землевласника
в надзвичайних випадках, що їх висувають потреби продукції.
Кожне село мусить щорічно постачати для «Jobagie» певний
континґент робітників відповідно до кількости своєї людности.
Ця додаткова панщинна праця для кожного волоського селянина
становить 14 днів. Таким чином, обов’язкова панщинна праця
становить річно 56 днів. Але рільничий рік у Волощині через
поганий клімат має лише 210 днів, з яких припадає 40 днів на
неділі й свята, 30 днів пересічно — на негоду, а разом відпадає
70 днів. Лишається 140 робочих днів. Відношення панщинної
праці до доконечної праці, або \sfrac{56}{84}, або 66\sfrac{2}{3}\%, виражає норму додаткової
вартости, куди меншу за ту, що реґулює працю англій-