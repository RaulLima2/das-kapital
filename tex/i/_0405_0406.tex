\index{i}{0405}  %% посилання на сторінку оригінального видання 
Хоч і які мізерні в цілому постанови фабричного закону про
виховання, все ж вони оголосили початкове навчання за обов’язкову
умову праці.\footnote{
За англійським фабричним законом батьки не можуть посилати
дітей молодших від 14 років до «контрольованих» фабрик, не даючи
їм одночасно початкової освіти. Фабрикант відповідає за недодержання
закону. «Навчання при фабриках обов'язкове та є умова праці» («Factory
education is compulsory, and it is a condition of labour»). («Reports of
Insp. of Fact, for 31 st. October 1863», p. 111).
} Їхній успіх вперше довів, що можна сполучати
навчання і гімнастику\footnote{
Про найкращі наслідки сполучення гімнастики (а для хлопців
і військових вправ) з обов’язковим навчанням дітей по фабриках і школах
для бідних дивись промову Н. В. Сеніора на сьомому, щорічному конґресі
«National Association for the Promotion of Social Science» в «Report
of Procedings etc.», London 1863, p. 63, 64, а також звіти фабричних інспекторів
з 31 жовтня 1865 р., стор. 118, 119, 120, 126 і далі.
} з ручною працею, отже, і ручну
працю з навчанням і гімнастикою. З свідчень учителів фабричні
інспектори незабаром виявили, що фабричні діти, хоч їх навчають
удвоє менше, ніж звичайних школярів, здобувають стільки ж
знання, а часто й більше, ніж ті останні. «Справа проста. Ті, що
проводять у школі тільки половину дня, завжди мають свіжу
голову й майже завжди здатні й охочі вчитися. Система поперемінного
чергування праці й навчання робить одне заняття відпочинком
від другого, отже, вона значно відповідніша для дитини,
ніж безперервність одного з цих двох заняттів. Хлопчина, що від
самого ранку сидить у школі, особливо ж у спеку, ніяк не може
змагатися з якимось іншим, що жвавий і втямливий приходить до
школи від своєї праці».\footnote{
«Reports of Insp. of Fact, for 31 st October 1865», p. 118. Один
наївний фабрикант шовку заявив слідчому комісарові «Children’s Employment
Commission» ось що: «Я цілком переконаний, що справжній секрет
продукувати вправних робітників знайдено у сполученні праці з навчанням,
починаючи від дитинства. Звичайно, праця не повинна бути ні
надто напружена, ні осоружна, ані нездорова. Я бажав би, щоб мої
власні діти мали працю й забави як відпочинок від школи». («Children’s
Employment Commission. 5 th Report», p. 82, n. 36).
} Дальші докази можна знайти в промові
Сеніора на соціологічному конґресі в Едінбурзі 1863 р. Сеніор
зазначає тут, між іншим, і те, що однобічний непродуктивний і
довгий шкільний день дітей вищих і середніх кляс без користи
збільшує працю вчителя, «тимчасом як він не тільки даремно,
але й з абсолютною шкодою для дітей забирає їм час, виснажує
їхнє здоров’я й енергію».\footnote{
Сеніор, там само, стор. 66. Яким чином велика промисловість на
певному ступені розвитку через переворот у способі матеріяльної продукції
і в суспільних продукційних відносинах робить переворот і в головах,
показує яскраво порівняння промови Н. В. Сеніора з 1863 р. з його філіппікою
проти фабричного закону 1833 р., або порівняння поглядів згаданого
конгресу з тим фактом, що в певних сільських частинах Англії бідним
батькам ще й досі заборонено під загрозою голодної смерти навчати
своїх дітей. Так, наприклад, пан Снелл повідомляє як про звичайну практику
в Сомерсетшірі, що коли бідна людина подається до парафії
по допомогу, то її примушують забрати своїх дітей із школи. Так, пан
Воллестоп, піп з Feltham’y, оповідає про випадки, коли деяким родинам
} Із фабричної системи, як можна простежити
\index{i}{0406}  %% посилання на сторінку оригінального видання 
в деталях у Роберта Овена, виріс зародок виховання
майбутности, яке для всіх дітей певного віку сполучатиме продуктивну
працю з навчанням і гімнастикою, і не тільки як методу
піднесення суспільної продукції, але і як єдину методу творити
всебічно розвинутих людей.

Ми бачили, що велика промисловість технічно знищує (aufhebt)
мануфактурний поділ праці, який на все життя прив’язував
геть усю людину до однієї детальної операції, але разом з цим
капіталістична форма великої промисловости репродукує той
поділ праці в ще потворнішому вигляді: на фабриці у власному
значенні — через перетворення робітника в свідомий додаток
до частинної машини, повсюди — почасти через спорадичне вживання
машини і машинової праці,\footnote{
Де ремісничі машини, що їх пускає в рух людська сила, безпосередньо
або посередньо конкурують із розвинутими машинами, отже, з
машинами, які мають своєю передумовою механічну рушійну силу, там
постає велика переміна щодо робітника, який пускає машину в рух.
Первісно парова машина заміняла цього робітника, тепер він повинен
заміняти парову машину. Тому напруження й витрата його робочої сили
досягають величезних розмірів, особливо для підлітків, що засуджені
на такі тортури! Так, комісар Лендж бачив у Ковентрі й околицях 10 —
15-літніх хлопців, що їх уживають крутити стьожковий варстат, не
кажучи вже про ще молодших дітей, які мусили крутити варстати менших
розмірів. «Це надзвичайно тяжка праця. Хлопчики просто заміняють
парову силу» («The boy is a mere substitute for steam power»),
(Children’s Employment Commission. 5 th Report 1866», p. 114, n. 6). Про
вбійчі наслідки «цієї системи рабства», як називає її офіціяльний звіт,
там же і далі.
} а почасти через заведення
жіночої, дитячої і некваліфікованої праці як нової основи поділу
праці. Суперечність поміж мануфактурним поділом праці і суттю
великої промисловости виявляється ґвалтовно. Вона виявляється,
між іншим, у тому жахливому факті, що велику частину дітей,
уживаних по сучасних фабриках та мануфактурах і від наймолодшого
віку прикованих до найпростіших маніпуляцій, визискують
цілими роками, при чому вони не навчаються ніякої праці, яка
згодом зробила б їх придатними хоча б у тій самій мануфактурі
або фабриці. Наприклад, в англійських друкарнях раніше практикували
перехід учнів од легших до змістовніших робіт, перехід,
що відповідав системі давньої мануфактури й ремества. Учні
проходили науку, аж поки ставали навченими друкарями. Вміти
читати й писати — це було доконечною вимогою від усіх тих,
хто бажав стати ремісником. Але все це змінилося з того часу,
як заведено друкарську машину. Вона вживає робітників двох
категорій — дорослого робітника, доглядача машини, і друкарських
хлопчаків, здебільша від 11 до 17 років, що їхня праця
виключно в тому, щоб подавати аркуші паперу в машину абож
забирати з неї надруковані аркуші. Вони виконують, особливо
в Лондоні, цю тяжку працю в деякі дні тижня по 14, 15, 16 годин
без перерви, а часто 36 годин уряд лише з двома годинами перерви

відмовлено всякої допомоги за те, що «вони посилали своїх дітей до
школи»!
\parbreak{}  %% абзац продовжується на наступній сторінці
