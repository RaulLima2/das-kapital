\index{i}{*0077}  %% посилання на сторінку оригінального видання
\section*{Передмова до першого видання}

Твір, що його перший том я подаю публіці, є продовження
моєї праці «Zur Kritik der Politischen Ökonomie», опублікованої
року 1859. Довгу павзу між початком і продовженням спричинила
довголітня хороба, що знову й знов переривала мою працю.

Зміст того попереднього твору зрезюмовано в першому розділі
цього тому. Це зроблено не тільки заради зв’язку й повноти. Поліпшено
самий виклад. Оскільки це якось дозволяла суть справи,
чимало пунктів, раніше тільки намічених, тут розвинуто докладно,
тимчасом як те, що там розвинуто докладно, тут подано лише
коротко. Відділи про історію теорії вартости й грошей тут,
звичайно, цілком виключено. Однак, читач попередньої праці
знайде в примітках до першого розділу нові джерела до історії
тих теорій.

Всякий початок тяжкий — це має силу для кожної науки.
Тим то найбільші труднощі для зрозуміння являє перший розділ,
особливо той відділ, що містить у собі аналізу товару. Щождо
самої аналізи субстанції вартости й величини вартости, то я по
змозі її спопуляризував.\footnote{
Це здавалось то потрібнішим, що значні непорозуміння є навіть
у творі Ф. Ляссаля проти Шульце-Деліча, у тому відділі, де, як він
заявляє, подано «духовну квінтесенцію» моїх думок про цей предмет.
En passant.\footnote*{
— між іншим. \emph{Ред.}
} Коли Ф. Ляссаль усі загальні теоретичні тези своїх економічних
праць, приміром, про історичний характер капіталу, про зв’язок
між продукційними відносинами й способом продукції і т. ін. і т. ін.,
майже дослівно, аж до створеної мною термінології, запозичив із моїх
творів, а до того ще й не подаючи джерел, то це пояснюється, звичайно,
цілями пропаганди. Я не говорю, розуміється, про окремі деталі його
теорії та її практичні застосування, що до них я не маю ніякого чинення.
} Форма вартости, що її закінчений
вигляд (fertige Gestalt) є грошова форма, дуже беззмістовна і
проста. Та все ж людський розум протягом більш ніж 2.000 років
даремно намагався збагнути її, тимчасом як, з другого боку, осягнуто,
принаймні приблизно, аналізу форм, багатших на зміст і
складніших. Чому це так? Тому що зформоване тіло легше вивчати,
ніж клітину тіла. Крім того, аналізуючи економічні
форми, не можна користуватися ні з мікроскопа, ні з хемічних
реактивів. І одно і друге мусить заступити сила абстракції. Але
для буржуазного суспільства товарова форма продукту праці або
вартостева форма товару є форма економічної клітини. Людині
неосвіченій аналізи цієї форми здається чимось тільки хитромудрим.
І тут справді йдеться про щось хитромудре, але лише
так, як це буває в мікрологічній анатомії.
