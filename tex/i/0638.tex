акт 13 Ґеорґа III с. 68 уповноважував мирових суддів реґулювати
заробітну плату шовкоткачів; ще 1796 р. треба було двох
присудів вищих судових інстанцій, щоб вирішити, чи судові
накази мирових суддів про заробітну плату мають силу і для
нерільничих робітників; ще 1799 р. один парляментський акт
потвердив, що плату копальневих робітників Шотляндії має
реґулювати статут Єлисавети і два шотляндські акти 1661 і 1671 рр
Але якого радикального перевороту зазнали за той час економічні
обставини, показав один нечуваний в англійській палаті
громад випадок. Тут, де більш ніж протягом 400 років фабриковано
закони про той максимум, що його ні в якому разі не
могла перевищувати заробітна плата, Вайтбред запропонував
у 1796 р. встановити законодатно мінімум заробітної плати для
рільничих робітників. Піт спротивився цьому, згоджуючись
однак, що «становище бідних жахливе» (cruel). Нарешті, в
1813 році закони про реґулюваннн заробітної плати скасовано.
Вони стали смішною аномалією від того часу, коли капіталіст
почав реґулювати працю на фабриці своїм приватним законодавством,
а плату сільського робітника почали доповнювати до доконечного
мінімуму податком на користь бідних. Але постанови
робочих статутів щодо контрактів між хазяїнами й робітниками,
щодо строків звільнення й т. ін., постанови, за якими хазяїна,
що зламав контракт, можна позивати лише до цивільного суду,
а робітника, що зламав контракт, до карного суду, — мають ще
й тепер повну силу.

Жорстокі закони проти об’єднань впали в 1825 р. в наслідок
грізної позиції, що її заняв пролетаріят. А проте вони впали
лише частково. Деякі прекрасні рештки старих статутів зникли
лише в 1859 році. Нарешті, парляментський акт з 29 червня
1871 р. через законодатне визнання тред-юньойонів претендував
усунути останні сліди цього клясового законодавства. Але
інший парляментський акт, виданий того самого дня (An act
to amend the criminal law relating to violence, threats and molestation),
фактично відновлював попередній стан у новій формі.
Цими парляментськими викрутасами всі засоби, що ними могли
користатися робітники підчас страйку або льокавту (страйк
об’єднаних фабрикантів, що одночасно закривають фабрики),
виключено з загального права і підведено під виняткові карні
закони, що їхня інтерпретація належала самим фабрикантам
у їхній ролі мирових суддів. Два роки перед тим та сама палата

(«Kaiserliche Privilegien und Sanctiones für Schlesien», I, 125). Цілих
сто років у наказах князів не вгавають гіркі нарікання поміщиків на
злісну й непокірливу голоту, що не хоче коритися суворому режимові,
не хоче задовольнятися приписаною законом платою; поодиноким поміщикам
заборонено було давати вищу плату, ніж установлює такса, вироблена
властями округи. А проте умови служби були після війни
часом кращі, ніж сто років пізніше; у Шльонську челядь ще в 1652 р.
діставала м’ясо двічі на тиждень, а в нашому столітті там по деяких
округах челядь дістає м’ясо лише тричі на рік. І заробітна плата була
після війни вища, ніж у наступних століттях». (G. Freitag).
