джами, не кажучи вже про галун, пісок і інші приємні мінеральні
домішки. Тому, не зважаючи на її святість, «вільна торговля»,
«вільне» перед тим пекарство піддано під догляд державних
інспекторів (наприкінці парляментської сесії 1863 р.), і той самий
парляментський акт заборонив пекарським підмайстрам, молодшим
за 18 років, працювати між дев’ятою годиною вечора й п’ятою
годиною ранку. Останній додатковий пункт свідчить красномовніш,
ніж цілі томи, про надмірну працю в цій галузі промисловости,
що від неї віє такою патріярхальністю.

«Праця лондонського пекарського підмайстра починається
звичайно об 11 годині ночі. В цей час він робить тісто, дуже втомний
процес, що триває від \sfrac{1}{2} до \sfrac{3}{4} години залежно від величини
та якости печива. Потім він лягає на місильну дошку, що разом
з тим служить і за покришку діжі, де виробляється тісто, і засинає
на декілька годин, підклавши один лантух з-під борошна під
голову й накрившися другим. Після того починається швидка й
безупинна чотиригодинна праця: викидають із діжі тісто, важать
його, формують, садовлять до печі, виймають із печі й т. ін. Температура
пекарні сягає 75 і навіть 90 градусів,* а в невеличких пекарнях
вона скорше буває більша, аніж менша. Коли справу печення
хліба, булок тощо закінчено, починається розподілювання хліба;
значна частина робітників, скінчивши тількищо описану важку
нічну працю, протягом дня розносить хліб у кошах або розвозить
його на візках від одного дому до другого, а в переміжках іноді працює
і в пекарні. Залежно від пори року та розміру підприємства
праця кінчається між першою й шостою годиною по півдні, тоді коли
інша частина підмайстрів працює в пекарні до пізнього вечора».78
«Підчас лондонського сезону підмайстри в пекарів Вестенда, що
продають хліб за «повну» ціну, реґулярно починають працювати
об 11 годині вночі і працюють коло печення хліба з однією або
двома часто дуже короткими перервами до 8 години найближчого
ранку. Потім до 4, 5, 6, а то навіть і до 7 години вони розносять
хліб або печуть бісквіти в пекарні. Після закінчення праці вони
відпочивають, засипаючи на 6 годин, часто лише на 5 або 4 години.
У п’ятницю праця завжди починається раніш, так щось о 10 годині
вечора, і триває безперестанку, чи то при виготовленні чи розношуванні
хліба, до 8 години вечора наступної суботи, але ж здебільшого
до 4 або 5 години в ніч під неділю. І в першорядних
пекарнях, що продають хліб за «повну ціну», неділями знов таки
доводиться працювати протягом 4—5 годин, щоб підготовити
роботу наступного дня... Ще довший робочий день пекарських
підмайстрів у «underselling masters» (що продають хліб нижче
від повної ціни), а ці останні становлять, як це вже раніш зазначено,
більш ніж \sfrac{3}{4} лондонських пекарів, але праця їхня
майже цілком обмежена пекарнею, бо їхні майстри-хазяї, за
винятком постачання хліба до дрібних крамниць, продають хліб

78 Там же. «First Report etc.», р. VI.

* За Фаренгайтом; за Цельсієм це становить 24—32 градуси, за
Реомюром — 19—26 градусів. Ред.
