(unsound). Лорд тримається фактів. А факт є той, що в міру того,
як меншає кількість ірляндської людности, ірляндські ренти
зростають, що збезлюднений «добродійне» для земельного власника,
отже, і для землі, отже, і для народу, що є лише приналежність
землі. Отож він заявляє, що Ірляндія все ще перелюднена,
і що потік еміґрації пливе все ще занадто поволі. Щоб бути цілком
щасливою, Ірляндія мусить позбутися принаймні ще 1/3 мільйона
робітників. Не думайте собі, що цей, до того всього ще й
поетичний, лорд є лікар із школи Sangrado, який завжди, коли
він не помічав у свого недужого поліпшення, приписував йому
кровоспуск, потім знову кровоспуск, поки, нарешті, в недужого
разом з його кров’ю пропадала і його хороба. Лорд Дюфрен
вимагає нового кровоспуску лише в 1/3 мільйона людей
замість майже 2 мільйонів, кровоспуску, без якого дійсно ніяк
неможливо завести тисячолітнього блаженного царства на Еріні.
Докази подати не важко.

Число і розмір фарм в Ірляндії 1864 р.

             1                                2                           3                         
               4
Фарми не більш        Фарми  від 2      Фарми від 6                 Фарми від 16
      від 1 акра                 до 5 акрів          до 15 акрів                   до 30 акрів
Число      Акри           Число    Акри      Число     Акри             Число      Акри
 48.653    25.394          82.037  288.916    176.368   1.836.310  136.578   3.051.343

          5                                   6                                7                    
              8
Фарми від 31              Фарми від 51              Фарми понад           Загальна площа
   до 50 акрів                   до 100 акрів               100 акрів
Число    Акри               Число    Акри              Число   Акри                 Акри 188а
71.961  2.906.274          54.247   3.983.880       31.927 8.227.807              29.319.924

Централізація знищила між 1851 і 1861 рр. переважно фарми
перших трьох категорій — нижче 1 і не вище 15 акрів. Вони
мусять зникнути передусім. Це дає 307.058 «зайвих» фармерів,
або 1.228.232 особи, коли при низькому пересічному обрахунку
покласти 4 особи на родину. При неймовірному припущенні, що
по закінченні революції в рільництві 1/4 з них знову знайде собі
роботу, все ж лишається 921.174 особи, що мусять еміґрувати. Категорії
4, 5 і 6, більші за 15 і не більші за 100 акрів, як це давно
відомо в Англії, занадто дрібні для капіталістичного рільництва,
а для вівчарства це зовсім незначні величини. Отже, при
тому самому припущенні, що й раніш, мусять еміґрувати ще
788.761 особа, разом 1.709.532. А що l’appétit vient en man-

188a Загальна площа включає також торфовища й пустирі.
