аніж тоді».223 Закон установив тариф заробітної плати для міста
й села, для відштучної й поденної праці. Сільські робітники
повинні найматися на рік, міські ж «на вільному ринку». Під
загрозою ув’язнення забороняється платити заробітну плату
вишу, ніж її приписує статут, але тих, що беруть таку вищу
плату, карають важче, ніж тих, що платять її. Так, ще статут
Єлисавети про учнів, в розділах 18 і 19, визначає десятиденне
ув’язнення для того, хто заплатить вищу заробітну плату, і
тритижневе для того, хто її бере. Статут із р. 1360 посилив ці
кари й навіть уповноважував хазяїна шляхом фізичного примусу
приневолювати робітників працювати на основі законного тарифу.
Всі спілки, угоди, присяги й т. ін., що ними зобов’язалися поміж
собою мулярі й теслярі, проголошено за недійсні. Об’єднання
робітників розглядається як важкий злочин, починаючи від
XIV віку аж до 1825 р., коли скасовано закони проти об’єднань.
Дух робітничого статуту з р. 1349 і всіх пізніших яскраво виявляється
в тому, що держава, щоправда, встановлювала максимум
заробітної плати, але аж ніяк не мінімум її.

В XVI столітті становище робітників, як це відомо, дуже
погіршало. Грошова плата підвищилася, але зовсім не пропорційно
до зневартнення грошей і відповідного зросту товарових цін.
Отже, на ділі заробітна плата спала. Проте закони, що мали на
меті зменшення заробітної плати, все ще мали силу; все ще відрізували
вуха й таврували тих, «кого ніхто не хотів брати на
службу». Статут про учнів 5, Єлисавети с. З, уповноважує мирових
суддів встановлювати певну заробітну плату і змінювати
її відповідно до пори року й зміни товарових цін. Яків І поширив
це реґулювання праці також і на ткачів, прядунів та на всякі
інші категорії робітників,224 а Ґеорґ II поширив закони проти
робітничих об’єднань на всі мануфактури.

За власне мануфактурного періоду капіталістичний спосіб
продукції настільки зміцнився, що міг зробити законодатне
реґулювання заробітної плати так само можливим, як і зайвим,
а все ж, про всякий випадок, хотілося мати напоготові зброю
з старого арсеналу. Ще акт 8 Ґеорґа II забороняв давати кравцям
підмайстрам Лондону й околиць більш ніж 2 шилінґи 7 1/2 пенса
поденної плати, за винятком випадків загального трауру; ще

223 Sophisms of Free Trade. By a Barrister», London 1850 p., p. 206).
Він лукаво додає: «Ми завжди готові були виступити в обороні підприємців.
Невже ж таки нічого не можна зробити для робітників?»

224 З одного параграфу статуту 2, Якова І, с. 6, видно, що деякі
фабриканти сукна, які були разом з тим і мировими суддями, дозволяли
собі офіціяльно визначати тариф заробітної плати у своїх власних майстернях.
У Німеччині дуже часто видавали статути для пониження
заробітної плати, особливо після тридцятирічної війни. «Поміщикам
дуже дошкуляв брак челяді й робітників у збезлюднених місцевостях.
Всім мешканцям села заборонено було винаймати кімнати неодруженим
чоловікам і жінкам; про всіх таких осіб треба було доносити урядові
й замикати їх до в’язниці, якщо вони не хотіли бути слугами, навіть
і тоді, коли вони утримували себе з якоїсь іншої роботи, працюючи на
полі у селян за поденну плату або навіть торгуючи грішми і збіжжям.
