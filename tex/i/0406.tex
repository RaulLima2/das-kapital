стежити в деталях у Роберта Овена, виріс зародок виховання
майбутности, яке для всіх дітей певного віку сполучатиме продуктивну
працю з навчанням і гімнастикою, і не тільки як методу
піднесення суспільної продукції, але і як єдину методу творити
всебічно розвинутих людей.

Ми бачили, що велика промисловість технічно знищує (aufhebt)
мануфактурний поділ праці, який на все життя прив’язував
геть усю людину до однієї детальної операції, але разом з цим
капіталістична форма великої промисловости репродукує той
поділ праці в ще потворнішому вигляді: на фабриці у власному
значенні — через перетворення робітника в свідомий додаток
до частинної машини, повсюди — почасти через спорадичне вживання
машини і машинової праці,301 а почасти через заведення
жіночої, дитячої і некваліфікованої праці як нової основи поділу
праці. Суперечність поміж мануфактурним поділом праці і суттю
великої промисловости виявляється ґвалтовно. Вона виявляється,
між іншим, у тому жахливому факті, що велику частину дітей,
уживаних по сучасних фабриках та мануфактурах і від наймолодшого
віку прикованих до найпростіших маніпуляцій, визискують
цілими роками, при чому вони не навчаються ніякої праці, яка
згодом зробила б їх придатними хоча б у тій самій мануфактурі
або фабриці. Наприклад, в англійських друкарнях раніше практикували
перехід учнів од легших до змістовніших робіт, перехід,
що відповідав системі давньої мануфактури й ремества. Учні
проходили науку, аж поки ставали навченими друкарями. Вміти
читати й писати — це було доконечною вимогою від усіх тих,
хто бажав стати ремісником. Але все це змінилося з того часу,
як заведено друкарську машину. Вона вживає робітників двох
категорій — дорослого робітника, доглядача машини, і друкарських
хлопчаків, здебільша від 11 до 17 років, що їхня праця
виключно в тому, щоб подавати аркуші паперу в машину абож
забирати з неї надруковані аркуші. Вони виконують, особливо
в Лондоні, цю тяжку працю в деякі дні тижня по 14, 15, 16 годин
без перерви, а часто 36 годин уряд лише з двома годинами перерви

відмовлено всякої допомоги за те, що «вони посилали своїх дітей до
школи»!

301    Де ремісничі машини, що їх пускає в рух людська сила, безпосередньо
або посередньо конкурують із розвинутими машинами, отже, з
машинами, які мають своєю передумовою механічну рушійну силу, там
постає велика переміна щодо робітника, який пускає машину в рух.
Первісно парова машина заміняла цього робітника, тепер він повинен
заміняти парову машину. Тому напруження й витрата його робочої сили
досягають величезних розмірів, особливо для підлітків, що засуджені
на такі тортури! Так, комісар Лендж бачив у Ковентрі й околицях 10 —
15-літніх хлопців, що їх уживають крутити стьожковий варстат, не
кажучи вже про ще молодших дітей, які мусили крутити варстати менших
розмірів. «Це надзвичайно тяжка праця. Хлопчики просто заміняють
парову силу» («The boy is a mere substitute for steam power»),
(Children’s Employment Commission. 5 th Report 1866», p. 114, n. 6). Про
вбійчі наслідки «цієї системи рабства», як називає її офіціяльний звіт,
там же і далі.
