14—20 дітей і примушують їх 15 годин на добу займатися такою
роботою, яка сама по собі виснажує людину своєю нудністю та
монотонністю та до того ще й виконується в умовах, що якнайдужче
руйнують здоров’я. Навіть наймолодші діти працюють з
напруженою увагою та дивовижною швидкістю, майже ніколи
не дозволяючи своїм пальцям відпочити або рухатися повільніше.
Коли до них звертаються з запитаннями, то вони не підводять очей
від роботи, боячися втратити хоча б один момент». «Довга палиця»
служить для «mistresses» за засіб спонукати дітей до праці то
більше, що більше здовжується робочий час. «Діти поступінно
втомлюються та стають неспокійні, як птиці, під кінець того довгого
часу, протягом якого їх прив’язано до роботи, монотонної, шкідливої
для очей, виснажливої через одноманітність позиції тіла.
Це справжня рабська праця» («Their work is like slavery»).\footnote{
«Children’s Employment Commission. 2 nd. Report 1864» p. XIX.
XX XXI.
}
Там, де жінки працюють разом із своїми власними дітьми вдома
в сучасному значенні цього слова, тобто в найманій кімнаті,
часто в якійсь халупці на горищі, це становище ще гірше, якщо
це тільки можливо. Таку роботу роздають на 80 миль навколо
Нотінґему. Коли дитина, що працює в крамниці, виходить із неї
о дев’ятій або десятій годині вечора, то часто їй дають на дорогу
ще цілий клунок для того, щоб вона закінчила роботу вдома.
Капіталістичний фарисей, що його репрезентує один з його наймитів,
робить це, звичайно, зворушливо приказуючи: «це для
матері», алеж сам дуже добре знає, що нещасній дитині доведеться
самій присісти та допомагати матері.\footnote{
Там же, стор. XXI, XXVI.
}

Промисловість плетіння мережива поширена головним чином
у двох рільничих округах Англії: в мереживній окрузі Honiton,
20—30 миль поздовж південного берега Девонширу, включаючи
й небагато місць Північного Девону, та в другій окрузі, що
охоплює більшу частину графств Букінґему, Бедфорду, Нортґемптону
та сусідні частини Оксфордширу та Гетінґдонширу.
Котеджі рільничих поденників звичайно являють собою й приміщення
для праці. Деякі мануфактуристи вживають понад
3.000 таких домашніх робітників, головне, дітей та підлітків,
виключно жіночої статі. Тут повторюються ті умови, що ми їх
описали при розгляді lace finishing. Тільки замість «mistresses
houses» тут виступають так звані «lace schools» (школи мережива),
що їх у своїх халупках тримають бідні жінки. Починаючи від
п’ятого року, а іноді й раніше, і до 12 або 15 року, працюють
діти по цих школах; протягом першого року наймолодші працюють
від 4 до 8 години, а потім від 6 години ранку до 8 та 10 години
вечора. «Загалом кажучи, кімнати — це звичайні комірки невеличких
котеджів, камін у них забито, щоб не було протягу, мешканці
іноді й зимою огріваються лише теплом свого власного
тіла. В інших випадках ці так звані шкільні кімнати — це по-