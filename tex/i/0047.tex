Так стає він грішми. «Вони мають одну думку й передають свою
силу й владу звірові. І щоб ніхто не посмів ні купувати, ані
продавати, як хіба той, хто має характер або ім’я звіра, або число
імені його».\footnote*{
«Illi unum consilium habent et virtutem et potestatem suam bestiae
tradunt. Et ne quis possit emere aut vendere, nisi qui habet characterem
aut nomen bestiae, aut numerum nominis ejus». (Apocalypse).
}

Грошовий кристаль є доконечний продукт процесу обміну,
що в ньому різнорідні продукти праці фактично прирівнюється
один одному і тому фактично перетворюється на товари. Історичний
процес поширення й поглиблення обміну розвиває дрімотну
в природі товару суперечність між споживною вартістю й вартістю.
Потреба надати для обороту зовнішній вираз цій суперечності
примушує прагнути самостійної форми для товарової вартости
й ні на хвилиночку не дає заспокоїтися доти, доки остаточно
не осягнуто такої форми через роздвоєння товару на товар і
гроші. Отже, тією самою мірою, якою здійснюється перетворення
продуктів праці на товари, здійснюється й перетворення товару
на гроші.\footnote{
3 того можемо оцінити дотепність дрібнобуржуазного соціялізму,
який бажає увіковічнити товарову продукцію й одночасно усунути «суперечність
між грішми й товаром», отже, усунути й самі гроші, бо вони
існують лише в цій суперечності. З таким самим успіхом можна було б
усунути папу залишити католицтво. Ближче про це див. мою працю «Zur
Kritik der Politischen Oekonomie», стор. 61 і далі («До критики політичної
економії», ДВУ 1926 р., стор. 100).
}

Безпосередній обмін продуктів, з одного боку, має форму
простого виразу вартости, а з другого — ще не має її. Та форма
була: х товару A = y товару B. Форма безпосереднього обміну
товарів така: х предмету споживання A = y предмету споживання
В.\footnote{
Поки ще обмінюється не два різні предмети споживання, а, як
ми це часто спостерігаємо в дикунів, пропонується хаотичну масу речей
як еквівалент за щось третє, доти сам безпосередній обмін продуктів є
ще тільки в зародку.
} Речі A й B не є тут товари перед обміном, але стають ними
лише через обмін. Перша умова, за якої предмет споживання,
є в можливості мінова вартість, це — його буття як неспоживної
вартости, як кількости споживної вартости, що перевищує
безпосередні потреби свого посідача. Речі сами по собі супроти
людини зовнішні, а тому й можна їх відчужувати. Для того,
щоб це відчужування стало взаємним, потрібно, щоб люди тільки
мовчки виступали один проти одного як приватні власники тих
відчужуваних речей і саме через те — як особи одні від одних
незалежні. Однак таке відношення взаємної відчужености не
існує для членів примітивної громади, хоч матиме вона форму
патріярхальної родини, хоч староіндійської громади, хоч держави
інків і т. ін. Обмін товарів починається там, де кінчається
громада, в пунктах її контакту з чужими громадами або членами
чужих громад. Але скоро тільки речі стали вже товарами поза
громадою, то в наслідок зворотного впливу вони стають това-