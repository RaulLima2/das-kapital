ному даному суспільстві. Складна праця — це тільки піднесена
до ступеня, або ж, скорше, помножена проста праця, так що менша
кількість складної праці дорівнює більшій кількості простої праці.
Що таке зведення складної праці до простої відбувається постійно
— це показує досвід. Один товар може бути продуктом якнайскладнішої
праці, його вартість робить його рівним продуктові
простої праці, і тому вона сама репрезентує лише певну кількість
простої праці.15 Різні пропорції, що в них різні роди праці зводяться
до простої праці як одиниці їхньої міри, установлюються
через суспільний процес за спиною продуцентів і тому видаються
їм даними традицією. Для спрощення ми далі вважатимемо всякий
рід робочої сили безпосередньо за просту робочу силу, а це
тільки заощадить нам роботу з цією редукцією.

Отже, як у вартостях «сурдут» і «полотно» залишено осторонь
ріжницю їхніх споживних вартостей, так само і в працях,
репрезентованих цими вартостями, залишено осторонь ріжницю
між їхніми корисними формами — між кравецтвом і ткацтвом.
Так само, як споживні вартості, «сурдут» і «полотно» є сполуки
певних доцільних продуктивних діяльностей із сукном і пряжею,
а вартості «сурдут» і «полотно», навпаки — тільки згустки
однорідної праці, так само і вміщені в цих вартостях праці мають
силу не в наслідок їхнього продуктивного відношення до сукна
й пряжі, а тільки як затрати людської робочої сили. Кравецтво
й ткацтво є елементи, що творять споживні вартості «сурдут»
і «полотно» саме в наслідок їхніх різних якостей; субстанцією
вартостей сурдута й полотна вони є лише остільки, оскільки залишається
осторонь їхні осібні якості й оскільки обидва вони мають
однакову якість, якість людської праці.

Але сурдут і полотно — не тільки вартості взагалі, а вартості
визначеної величини, і, за нашим припущенням, сурдут вартий
удвоє більше, ніж 10 метрів полотна. Звідки ця ріжниця у величинах
їхньої вартости? Від того, що 10 метрів полотна містять у
собі тільки половину тієї праці, що її має сурдут, так що на продукцію
цього останнього мусять витрачати робочої сили протягом
удвоє більшого часу, ніж на продукцію першого.

Отже, коли щодо споживної вартости вміщена в товарі праця
має значення лише своєю якістю, то щодо величини вартости вона
має значення лише своєю кількістю, скоро її вже зведено на людську
працю без особливої якости. Там ідеться про те, як витрачається
праця і що вона продукує, тут — скільки її витрачено,
протягом якого часу. А що величина вартости якогось товару
виражає лише кількість праці, що міститься в ньому, то товари
в певній пропорціїмусять бути завжди рівновеликими вартостями.

Коли продуктивна сила, приміром, усіх корисних праць, потрібних
на продукцію сурдута, лишається незмінна, то величина

15 Читач мусить мати на увазі, що тут мова не про заробітну плату
або вартість, яку робітник дістає, приміром, за один робочий день, а про
вартість товарів, що в ній його робочий день упредметнюється. Категорія
заробітної плати взагалі ще не існує на цьому ступені нашого досліду.
