\index{i}{0578}  %% посилання на сторінку оригінального видання
Через те, що серед англійських капіталістів повелася мода
змальовувати Бельґію як рай для робітників, бо «воля праці»
або, що те саме, «воля капіталу» там не обмежена ні деспотизмом
тред-юньйонів, ані фабричними законами, скажімо тут кілька
слів про «щастя» бельґійського робітника [що його пригноблюють
лише духовенство, земельна аристократія, ліберальна буржуазія
і бюрократія, але аж ніяк не тред-юньйони і не фабричні закони].\footnote*{
Заведене у прямі дужки беремо з другого німецького видання. \emph{Ред.}
}
Напевно, ніхто не був глибше посвячений у всі таємниці
цього щастя, ніж небіжчик пан Дюкпетіо, головний інспектор
бельґійських в’язниць та добродійних установ і член бельгійської
центральної статистичної комісії. Загляньмо до його твору
«Budgets économiques des classes ouvrières en Belgique», Bruxelles
1855. Тут ми знаходимо, між іншим, бельгійську середню
робітничу родину, що її щорічні видатки й доходи обчислено
на основі дуже докладних даних і що її умови харчування потім
порівнюється з умовами харчування солдата, флотського матроса
й арештанта. Родина «складається з батька, матері й чотирьох
дітей». Із цих шістьох осіб «четверо можуть цілий рік займатися
корисною працею»; припускається, «що між ними немає ані
хорих, ані нездатних до праці», що вони не роблять ані «видатків
на релігійні, моральні й інтелектуальні потреби, за винятком
невеличкої оплати за місце в церкві», ані «вкладок до ощадних
кас і до кас забезпечення на старість», ані «видатків на предмети
розкошів або інших зайвих видатків». Однак батько і старший
син палять і заходять неділями до шинку, а на це їм треба цілих
86 сантимів на тиждень. «Із загального зіставлення заробітної
плати, що її дістають робітники різних галузей промисловости,
випливає\dots{} що найвища пересічна денна заробітна плата становить
1 франк 56 сантимів для чоловіків, 89 сантимів для жінок,
56 сантимів для хлопців і 55 сантимів для дівчат. За таким обчисленням
доходи родини становили б щонайбільше 1.068 франків
на рік\dots{} У родині, що її ми визнали за типову, ми підрахували
загальну суму всяких можливих доходів. Але, якщо припустимо,
що й мати одержує заробітну плату, то ми тим самим
лишаємо хатнє господарство без керівництва; хто тоді піклуватиметься
про домівку, про маленькі діти? Хто тоді варитиме, пратиме,
лататиме? Ця дилема щодня стає перед очима робітників».

Отже, бюджет родини такий:

\begin{center}
\noindent\begin{tabularx}{0.8\textwidth}{X@{}l@{ }l@{ }r@{ }l}
    Батько\dotfill{} & 300 робочих &
    днів по 1,56 франка & $=$ 468 & франків \\
    
    Мати\dotfill{}   & 300\ditto{робочих} &
    \ditto{днів} \ditto{по} 0,89\ditto{франка} & $=$ 267 & \ditto{франків} \\

    Син\dotfill{}    & 300\ditto{робочих} &
    \ditto{днів} \ditto{по} 0,56\ditto{франка} & $=$ 168 & \ditto{франків} \\

    Дочка\dotfill{}  & 300\ditto{робочих} &
    \ditto{днів} \ditto{по} 0,55\ditto{франка} & $=$ 165 & \ditto{франків} \\

    \cmidrule{3-5}
    & & Разом\dotfill{} & 1.068 & франків
\end{tabularx}
\end{center}

Річні видатки родини і її дефіцит становили б, коли б робітник
діставав харчі:

\begin{center}
\noindent\begin{tabular}{l@{ }l}

    Флотського матроса & 1.828 франків \textemdash{} дефіцит 760 франків \\

    Солдата\dotfill{}  & 1.473 \ditto{франків} \ditto{\textemdash{}} \ditto{дефіцит}
     405  \ditto{франків} \\

    Арештанта\dotfill{}& 1.112 \ditto{франків} \ditto{\textemdash{}} \ditto{дефіцит}
     \phantom{0}44  \ditto{франків} \\
\end{tabular}
\end{center}
