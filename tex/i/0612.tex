ґів. Старша дочка, дванадцяти років, доглядає хати. Вона наша
куховарка й однісінька помічниця. Вона підготовляє менших
до школи. Моя дружина встає разом зі мною і йде теж зі мною.
Одна дівчина, що проходить повз нашої хати, будить мене о 5\sfrac{1}{2}
годині ранку. Перед тим, як іти на роботу, ми нічого не їмо.
Вдень дванадцятилітня дочка доглядає менших дітей. Снідаємо
о 8 годині і для цього приходимо додому. Чай п’ємо раз на тиждень;
звичайно ми їмо юшку (stirabout), іноді з вівсяного борошна,
іноді з кукурудзяного, залежно від того, що можемо дістати.
Взимку до кукурудзяного борошна додаємо трохи цукру й води.
Влітку копаємо потроху картоплю, що сами садимо на клаптику
землі, а коли картопля кінчається, знову вертаємося до юшки.
Так іде з дня на день, у неділю і будні, цілий рік. Скінчивши
роботу, я ввечорі завжди почуваю надзвичайну втому. Трошки
м’яса нам винятково доводиться бачити, але дуже рідко. Троє
з наших дітей ходять до школи, і за те ми платимо за кожне по
1 пенсу на тиждень. Наша квартирна плата становить на тиждень
9 пенсів, торф і опалення коштують щонайменше 1 шилінґ 6 пенсів
на два тижні».188 Така ірляндська заробітна плата, таке
ірляндське життя!

Справді, злидні Ірляндії знову стали в Англії темою дня.
Наприкінці 1866 й на початку 1867 р. один з ірляндських земельних
маґнатів, лорд Дюфрен, взявся на сторінках «Times’a» за
розв’язання цього питання. «Яка гуманність з боку такого великого
пана!»

Із таблиці Е видно, що в 1864 р. із загального зиску в 4.368.610
фунтів стерлінґів троє тільки капіталістів (Plusmacher) поклали до
своєї кишені 262.610 фунтів стерлінґів, а в 1865 р. тих самих троє
віртуозів «поздержливости» з 4.669.979 фунтів стерлінґів загального
зиску дістали 274.448 фунтів стерлінґів; в 1864 р. 26 капіталістів
дістали 646.377 фунтів стерлінґів; в 1865 р. 28 капіталістів —
736.448 фунтів стерлінґів; в 1864 р. 121 капіталіст — 1.066.912
фунтів стерлінґів в 1865 р. 186 капіталістів — 1.320.996 фунтів
стерлінґів; в 1864 р. 1.131 капіталіст — 2.150.818 фунтів стерлінґів,
майже половину загального річного зиску; в 1865 р.
1.194 капіталісти дістали 2.418.933 фунти стерлінґів — більше, ніж
половину загального річного зиску. Але та левина частина,
яку проковтує з усієї річної суми земельних рент зовсім мале
число земельних маґнатів Англії, Шотляндії та Ірляндії, така
потворно велика, що англійська державна мудрість вважає за
доцільне не давати про розподіл земельної ренти такого самого
статистичного матеріялу, як про розподіл зиску. Лорд Дюфрен
— один із цих земельних маґнатів. Гадати, що ренти й зиски
колибудь можуть бути «надмірні», або що плетора (plethora)*
ренти і зисків є в якомусь зв’язку з плеторою народніх злиднів,
це, звичайно, є уявлення так само «непочтиве», як і «нездорове»

188 «Reports of Insp. of Fact, for 31st October 1866», p. 96.

* — повнява. Ред.
