\parcont{}  %% абзац починається на попередній сторінці
\index{i}{0126}  %% посилання на сторінку оригінального видання
він дозволяє покупцеві споживати її раніш, ніж останній заплатив
її ціну, отже, повсюди робітник кредитує капіталіста. Що це
кредитування не є пуста химера, показує не лише втрата кредитованої
заробітної плати на випадок банкрутства капіталіста,\footnote{
«Робітник позичає свою працю» — каже Шторх, але додає єхидно:
«він нічим не ризикує», хіба що «втратою свого заробітку\dots{} робітник не
додає нічого матеріяльного». («L’ouvrier prête son industrie\dots{} il ne risque
rien, exepté de perdre son salaire\dots{} l’ouvrier ne transmet rien de matériel»).
(\emph{Storch}: «Cours d’Economie Politique», Pétersbourg 1815, vol. 2, p. 37).
}
але й низка затяжніших наслідків.\footnote{
Приклад. У Лондоні є два сорти пекарів: «full priced», що продають
хліб за його повною вартістю, і «undersellers», що продають його
нижче його вартости. Остання кляса становить понад три четвертини
загального числа всіх пекарів. (P. 32 в «Report» урядового комісара
H. S. Tremenheere про «Grievances complained of by the journemeymen
bakers etc.», London 1862). Ці undersellers продають, майже без винятку,
хліб, фальсифікований домішкою галуну, мила, поташу, вапна, дербішірської
кам’яної муки та інших подібних приємних, поживних і здорових
інґредієнтів. (Див. цитовану вище Синю Книгу, а також звіт «Committee
of 1855 on the Adulteration of Bread» і твір доктора Hassall’a: «Adulterations
Detected», 2-nd ed. London 1862). Сер Джон Ґордон заявив
перед комітетом 1855 р., що «в наслідок такої фальсифікації бідний, який
живе з двох фунтів хліба на день, в дійсності не дістає тепер і четвертини
поживного матеріялу, не кажучи вже про шкідливий вплив на його здоров’я».
Як причину того, що «дуже велика частина робітничої кляси»,
хоч і добре знає про фальсифікацію, все ж купує галун, кам’яну муку
й т. ін., Tremenheere наводить те (1. с. p. 48), що для них «є неминуче
брати такий хліб у свого пекаря або в крамаря, який цьому останньому
забажається їм дати». Через те, що плату вони дістають лише наприкінці
тижня, вони можуть «за спожитий їхньою родиною протягом
тижня хліб заплатити лише наприкінці тижня»; і Tremenheere додає,
покликаючись на свідків: «загальновідома річ, що хліб із такими домішками
виготовляється виключно для цього роду покупців» («it is notorius
that bread composed of those mixtures, is made expressly for sale in this
manner»). «У багатьох рільничих округах Англії (а ще більш у шотляндських)
заробітну плату видається раз на два тижні й навіть раз на місяць.
Через такі довгі реченці платежу рільничий робітник примушений купувати
собі товари на кредит\dots{} Він мусить платити вищі ціни й фактично
прикований до тієї крамниці, що його висмоктує. Так, приміром, у Ногningsham
in Wilts, де заробітна плата щомісячна, таке саме борошно, за
яке в іншому місці він заплатив би 1\shil{ шилінґ} 10\pens{ пенсів}, йому коштує 2\shil{ шилінґи}
4\pens{ пенси} за stone». («Sixth Report» on «Public Health» by «The
Medical Officer of the Privy Council etc.», 1864, p. 264). «Робітники перкалево-вибійчаних
майстерень у Песлі й Кільмарноці (Західня Шотляндія)
страйком примусили в 1853 р. скоротити реченець платежу з одного
місяця на два тижні». («Reports of the Inspectors of Faktories for 31 st
October 1853», p. 34). Як приклад дальшого чемного розвитку кредиту,
що його дає робітник капіталістові, можна розглядати методу багатьох
англійських посідачів кам’яновугільних копалень, що платять робітникові
лише наприкінці місяця, а в проміжний час робітник дістає від капіталіста
аванси, часто товарами, які він мусить оплачувати понад їхню
ринкову ціну (Trucksystem).\footnote*{
Початок цієї фрази у французькому виданні зредаґовано так: «Як
приклад експлуатації робітника, що постає з того кредиту, який він дає
капіталістові, можна розглядати\dots{}». — «Comme exemple de l’exploitation,
qui resulte pour l’ouvrier du credit qu’il donne au capitaliste\dots{}». \emph{Peд.}
}
«Серед власників кам’яновугільних копалень
стало звичаєм платити робітникам раз на місяць і давати їм позики
наприкінці кожного проміжного тижня, позику видається в крамниці
(а саме в tommy shop, тобто в крамниці, що належить самому хазяїнові);
робітники отримують гроші в одному куті й зараз же віддають їх у другому»
(«It is a common practice with the coal masters to pay once a month,
and advance cash to their workmen at the end of each intermediate week.
The cash is given in the shop; the men take it on one side and lay it out on
the other»). («Children’s Employment Commission, 3 rd Report», London
1864, p. 38, n. 192).
}
Проте природа самого товарового
обміну аніскільки не змінюється від того, як функціонують
гроші: як засіб купівлі, чи як засіб платежу. Ціну робочої сили
встановлено контрактом, хоч реалізується вона лише пізніш,
як і наймова ціна за будинок. Робочу силу вже продано, хоч
плату за неї буде виплачено лише пізніше. Однак для ясного розуміння
цього відношення корисно покищо тимчасово припустити,
що посідач робочої сили щоразу одночасно з продажем її дістає
зразу ж і умовлену в контракті ціну.
\index{i}{0127}  %% посилання на сторінку оригінального видання

Ми знаємо тепер спосіб визначення вартости, яку посідач
грошей виплачує посідачеві цього своєрідного товару, робочої
сили. Споживна вартість, яку посідач грошей отримує собі в
обмін, виявляється лише в дійсному вживанні, у процесі споживання
робочої сили. Всі потрібні для цього процесу речі, як от
сировинний матеріял і т. ін., посідач грошей купує на товаровому
ринку й платить за них повну ціну. Процес споживання
робочої сили є разом з тим процес продукції товару й додаткової
вартости. Споживання робочої сили, як і споживання всякого
іншого товару, відбувається поза межами ринку або сфери циркуляції.
Тим то ми, разом із посідачем грошей і посідачем робочої
сили, полишаємо цю шумну сферу, де все відбувається на поверхні
та перед очима всіх і кожного, щоб піти за ними обома в таємні
місця продукції, що на їх порозі написано: «No admittance except
on business.\footnote*{
Увіходити дозволяється лише y справах. \emph{Ред.}
} Тут виявиться не лише те, як капітал продукує,
алеж і те, як його самого продукують, як продукують капітал.
Таємниця продукції додаткової вартости (Plusmacherei) мусить,
нарешті, відкритися.

Сфера циркуляції, або обміну товарів, у межах якої рухається
купівля та продаж робочої сили, була дійсно за правдивий
едем природжених прав людини. Тут панує лише воля, рівність,
власність і Бентам. Воля! Бо покупець і продавець товару, приміром,
робочої сили, керуються лише своєю свободною волею.
Вони складають умови як вільні, юридично рівноправні особи.
Контракт є кінцевий результат, у якому їхні волі знаходять собі
спільний юридичний вираз. Рівність! Бо вони відносяться один
до одного лише як посідачі товарів і обмінюють еквівалент на
еквівалент. Власність! Бо кожний порядкує лише своїм. Бентам!
Бо кожний з обох дбає лише про себе самого. Однісінька
сила, що ставить їх у зв’язок і взаємне відношення, — це сила
їхньої власної користи, їхньої власної вигоди, їхніх приватних
інтересів. Але саме через те, що кожний дбає лише про себе й
ніхто не дбає про іншого, всі вони в наслідок наперед установленої
\index{i}{0128}  %% посилання на сторінку оригінального видання
гармонії речей, або під проводом всемудрого провидіння,
здійснюють лише справу своєї взаємної вигоди, спільної користи,
спільних інтересів.

У той момент, коли ми розстаємося з цією сферою простої
циркуляції, або товарового обміну, з якої фритредер vulgaris
запозичає свої погляди, поняття й маштаб для своїх думок про
суспільство капіталу й найманої праці, ми помічаємо, що вже де
в чому зміняється, здається, фізіономія наших dramatis personae.
Колишній посідач грошей іде попереду як капіталіст, посідач
робочої сили йде слідком за ним як його робітник; один іде,
многозначно усміхаючися й жадаючи взятися до справи; другий —
боязко, опираючись, як людина, що винесла на ринок свою власну
шкуру й не має вже нічого іншого сподіватись, як тільки того,
що цю шкуру будуть гарбарювати.
