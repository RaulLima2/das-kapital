вона знову та знову становить вихідний пункт таких революцій
щоразу, коли ремісниче або мануфактурне виробництво переходить
у машинове.

Коли придивитися ближче до виконавчої машини, або власне
робочої машини, то взагалі і в цілому знову побачимо в ній, хоч
часто і в дуже змодифікованій формі, ті самі апарати й знаряддя,
що ними працює ремісник та мануфактурний робітник, але
вже не як знаряддя людини, а як знаряддя якогось механізму,
або як механічні знаряддя. Або ціла машина є лише більш
чи менш змінене механічне видання колишніх ремісницьких
інструментів, як от у механічному ткацькому варстаті,\footnote{
Особливо в первісній формі механізованого ткацького варстату
можна з першого ж погляду пізнати давній ткацький варстат. У своїй
сучасній формі він посутньо змінений.
} або
розміщені на кістяку робочої машини чинні органи є давні
знайомі, як от веретена у прядільній машині, дроти в машині
на плетіння панчіх, пили у тартаку, ножі в різальній машині
й т. д. Відміна цих знарядь від власне тіла робочої машини
починається вже від самих їхніх народин, а саме: ці знаряддя
здебільшого все ще продукуються ремісничим або мануфактурним
способом, і лише потім прикріплюють їх до кістяка
робочої машини, спродукованої машиновим способом.\footnote{
Тільки приблизно від 1850 р. дедалі більшу частину знарядь робочих
машин починають фабрикувати в Англії машиновим способом, хоч
фабрикують їх не ті самі фабриканти, що виробляють сами машини.
Машинами для фабрикації такого механічного знаряддя є, наприклад,
automatic bobbin-making engine, card-setting engine — машини виготовляти
ткальні самопряди, машини виготовляти веретена mule і throstle.
} Отже,
виконавча машина це — такий механізм, що, одержавши відповідний
рух, виконує своїм знаряддям ті самі операції, що їх раніш
виконував робітник подібним знаряддям. Чи рушійна сила виходить
від людини, чи навіть знову таки від якоїсь машини, це
нічого не змінює в суті справи. Після перенесення власне знаряддя
від людини до механізму машина стає на місце простого знаряддя.
Ріжниця відразу впадає на очі, навіть і тоді, коли сама людина
все ще лишається першим мотором. Число робочих інструментів,
якими людина може орудувати одночасно, обмежене кількістю
її природних знарядь продукції, кількістю органів її власного
тіла. В Німеччині пробували були примусити прядуна рухати
два прядільні колеса, тобто працювати одночасно обома руками
й обома ногами. Це була надто напружена робота. Пізніше вигадали
ножаний самопряд з двома веретенами, але такі віртуози-прядуни,
які могли одночасно прясти дві нитки, траплялися майже
так рідко, як двоголові люди. Навпаки, машина jenny вже з
самого початку свого з’явлення пряде 12—18 веретенами, машина
на плетіння панчіх плете одночасно кількома тисячами дротів і
т. д. Число знарядь, що ними та сама виконавча машина одночасно
робить, від самого початку емансиповане від тієї органічної
межі, що її не могло переступити ручне знаряддя робітника.