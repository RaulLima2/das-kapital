\parcont{}  %% абзац починається на попередній сторінці
\index{i}{0625}  %% посилання на сторінку оригінального видання
силкується виставити громадську власність так, ніби вона є
приватна власність великих землевласників, які заступили місце
февдалів; але він сам збиває цю свою оборонну промову, вимагаючи
«загального парляментського акту про обгородження громадських
земель», отже, визнаючи, що для перетворення їх на
приватну власність потрібен парляментський державний переворот,
і домагаючися, з другого боку, від законодавства «відшкодування»
для експропрійованих бідняків.\footnote{
Eden: «The State of the Poor», передмова.
}

Тим часом як замість незалежних уеomen’ів появилися tenants-at-will,
— дрібні фармери з річним строком оренди, юрба
людей, принижених і залежних від сваволі лендлордів, — систематичний
крадіж громадської власности, поряд із грабуванням
державних маєтків, особливо сприяв величезному зростанню
тих великих фарм, що їх у XVIII столітті називали «капітальними
фармами»\footnote{
«Capital-farms». («Two Letters on the Flour Trade and the Dearness
of Corn. By a Person in Business», London 1767, p. 19, 20).
} або «купецькими фармами»\footnote{
«Merchant-farms». («An Enquiry into the Present High Price
of Provisions», London 1767, p. 11, примітка). Автор цієї гарної праці,
виданої анонімно, є панотець Натаніел Форстер.
}, а також і перетворенню
сільської людности на пролетаріят, «звільнений» для
промисловости.

Однак у XVIII столітті ще не розуміли в такій мірі, як у
XIX, що національне багатство тотожне з народніми злиднями.
Звідси така надзвичайно завзята полеміка в економічній літературі
того часу з приводу обгороджування громадських земель
(«inclosure of commons»). З тієї маси матеріялу, що в мене під
рукою, я наводжу тут деякі місця, які яскраво унаочнюють тогочасні
обставини.

«У багатьох парафіях Гертфордшіру, — пише одне обурене
перо, — 24 фарми, пересічно в 50--150 акрів кожна, злито в
3 фарми».\footnote{
Thomas Wright: «A short address to the Public on the Monopoly
of large farms», 1779, p. 2, 3.
} «У Нортгемптоншірі й Лінколншірі обгороджування
громадських земель відбувалося в широких розмірах, і
більшість нових лордств, що постали з цього обгороджування,
перетворено на пасовиська; у наслідок того багато лордств, які
раніше мали під плугом до 1.500 акрів, не мають тепер і 50 акрів\dots{}
Руїни колишніх домів, клунь, стаєнь тощо» — оце єдині
сліди колишніх мешканців. «Подекуди сотні домів та родин
зменшено\dots{} до 8 або 10\dots{} У більшості парафій, де обгороджування
почалося лише 15 або 20 років тому, залишилось тільки дуже
невелике число землевласників, порівняно з числом тих, що
обробляли землю тоді, коли вона ще не була обгороджена. Досить
часто можна бачити, що 4 або 5 багатих скотарів узурпують
собі великі, недавно обгороджені лордства, які раніш були в
руках 20--30 фармерів і такого ж числа дрібних власників та
селян. Усіх цих людей з їхніми родинами прогнано з їхньої
власности, разом з ними прогнано й багато інших родин, що мали
\parbreak{}  %% абзац продовжується на наступній сторінці
