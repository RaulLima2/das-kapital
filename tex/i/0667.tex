ляти нових робітників у колонії». 273 Октройована державою ціна землі мусить, звичайно, бути
«достатня» (sufficient price), тобто така висока, «щоб перешкоджати робітикам ставати незалежними
селянами доти, доки не з’являться інші, щоб заступити їхнє місце на ринку найманої праці». 274 Ця
«достатня ціна землі» є не що інше, як пом’якшене означення викупних грошей, які робітник платить
капіталістові за дозвіл покинути ринок найманої праці й заходитися коло обробітку землі. Спочатку
робітник мусить створити панові капіталістові «капітал», щоб він міг експлуатувати більше число
робітників, а потім він мусить приставити на ринок праці «заступника», якого його коштом уряд
транспортує із-за моря для його колишнього пана капіталіста.

Надзвичайно характеристично, що англійський уряд протягом багатьох років запроваджував цю методу
«первісної акумуляції капіталу», рекомендовану паном Векфілдом для вжитку спеціяльно по колоніяльних
країнах. Фіяско було, звичайно, таке саме ганебне, як фіяско з банковим актом Піла. Потік еміґрації
лише повернувся від англійських колоній до Сполучених штатів. Тимчасом проґрес капіталістичної
продукції в Европі, супроводжуваний дедалі більшим утиском з боку уряду, зробив рецепт Векфілда
зайвим. З одного боку, величезний і невпинний потік людей, що рік-у-рік тече до Америки, залишає на
сході Сполучених штатів застійні осади, бо хвиля еміґрації з Европи кидає туди людей на робітничий
ринок швидше, ніж друга хвиля еміґрації встигає занести їх на захід. З другого боку, американська
громадянська війна потягла за собою колосальний національний борг, а разом з ним податковий тиск,
народження найпідлішої фінансової аристократії, роздаровування величезної частини громадських земель
товариствам спекулянтів для експлуатації залізниць, копалень і т. ін., — одно слово, вона потягла за
собою якнайшвидшу централізацію капіталу. Отже, велика республіка перестала бути обітованою землею
для робітників-еміґрантів. Капіталістична продукція йде там велетенськими кроками вперед, хоч спад
заробітної плати й залежність найманого робітника далеко ще не зведені до европейського нормального
рівня. Безсоромне марнотратне роздаровування англійським урядом необроблених колоніяльних земель
аристократам і капіталістам, яке сам Векфілд голосно засуджує, разом із потоком людей, що їх
притягають копальні золота, і з конкуренцією, яку імпорт англійських товарів створює навіть
найдрібнішому ремісникові, породили, особливо в Австралії,275 достатнє «відносне перелюд-

273 Wakefield. Там же, т. II, стор. 192.

274 Там же, стор. 45.

275 Скоро Австралія стала своєю власною законодавицею, вона звичайно, видала закони, сприятливі для
переселенців, але марнотратство земель, що його перевели вже англійці, стоїть на перешкоді. «Перша й
найважливіша мета, яку ставить новий земельний закон з року
