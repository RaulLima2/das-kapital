\parcont{}  %% абзац починається на попередній сторінці
\index{i}{0088}  %% посилання на сторінку оригінального видання
товару міряє ступінь притяжної сили його щодо всіх елементів
речового багатства, отже, міряє суспільне багатство свого посідача.
Для варварсько-примітивного посідача товарів, навіть
для західньоевропейського селянина, вартість є невіддільна від
форми вартости, а тому збільшення золотого й срібного скарбу
є для нього збільшення вартости. Правда, вартість грошей змінюється,
чи то в наслідок зміни їхньої власної вартости, чи то
в наслідок зміни вартости товарів. Але це, з одного боку, не
заважає тому, що 200 унцій золота як і раніш, мають більше
вартости, ніж 100, 300 — більше, ніж 200 і т. д., з другого, —
що натуральна металева форма цієї речі лишається загальною
еквівалентною формою всіх товарів, безпосередньо суспільним
втіленням усякої людської праці. Потяг до скарботворення з
самої природи своєї безмірний. Якісно, або своєю формою, гроші
не мають меж, тобто вони є загальний представник речового
багатства, бо вони безпосередньо можуть перетворюватись на
кожний товар. Але разом з тим кожна реальна грошова сума є
кількісно обмежена, отже, вона є купівельний засіб з обмеженою
силою. Ця суперечність поміж кількісною обмеженістю і якісною
безмежністю грошей завжди спонукує збирача скарбів до сізіфової
праці акумуляції. З ним справа стоїть так само, як з
тим завойовником світу, що з кожною новою країною здобуває
лише якийсь новий кордон.

Щоб затримати в себе золото як гроші, а тому і як елемент
скарботворення, треба перешкодити йому циркулювати, або розпускатись
як засобу купівлі в засобах споживання. Тому збирач
скарбів жертвує золотому фетишеві бажання свого тіла.
Він навсправжки приймає євангелію відречення. Але, з другого
боку, він може витягти з циркуляції в грошах лише те, що він
дав їй у товарах. Що більш він продукує, то більш може він
продати. Тим то працьовитість, ощадність і скупість становлять
його кардинальні чесноти; багато продавати й мало купувати —
в цьому вся його політична економія.\footnote{
«Збільшити якомога число продавців усіх товарів, зменшити якомога
число покупців їх — ось основні пункти, на які сходять усі заходи
політичної економії» («Accrescere quanto più si puó il numero de’ venditori
d’ogni merce, diminuire quanto piú si puó il numero dei compratori, questi
sono і cardini sui quali si raggirano tutte le operazioni di economia politica»).
(Verri: «Meditazioni sulla Economia Politica», Custodi, Parte \emph{}Moderna,
vol. XV, p. 52).
}

Поруч із безпосередньою формою скарбу йде його естетична
форма, володіння золотими й срібними товарами. Воно зростає
разом із багатством буржуазного суспільства. «Soyons riches
ou paraissons riches»\footnote*{
Будьмо багаті або видаваймося багатими. \emph{Ред.}
} (Дідро). Таким чином утворюється почасти
чимраз поширеніший ринок для золота й срібла, незалежно від
їхніх грошових функцій, почасти — лятентне джерело постачання
грошей, що особливо швидко тече за часів суспільних бур.

Скарботворення виконує різні функції в економії металевої
циркуляції. Найближча його функція виникає з умов обігу золотої
\index{i}{0089}  %% посилання на сторінку оригінального видання
або срібної монети. Ми бачили, як з постійними коливаннями
товарової циркуляції щодо розміру, цін і швидкости безперестанно
припливає й відпливає маса грошей, що є в обігу. Отже, вона
мусить бути здатна до звуження та розширення. Часом гроші
мусять притягатися як монета, часом монета відштовхуватися
як гроші. Щоб маса грошей, які дійсно є в обігу, завжди відповідала
ступеневі насичення сфери циркуляції, кількість золота
або срібла, що є у країні, мусить бути більша, ніж та, що виконує
монетну функцію. Ця умова виконується через скарбову
форму грошей. Резервуари скарбів одночасно служать за відпливні
й припливні канали для грошей, що циркулюють, тим то
гроші ніколи не переповнюють каналів свого обігу.\footnote{«Щоб нація могла вести свою торгівлю, потрібна певна сума
готівки, яка може змінятись, то збільшуючись, то зменшуючись залежно
від обставин\dots{} Ці припливи й відпливи грошей урівноважуються
сами собою без жодної допомоги з боку політиків\dots{} Відра працюють почережно:
коли мало грошей, то карбують монету, а коли мало грошового
металю — перетоплюють монету знов у зливки». («There is required for
carrying on the trade of the nation, a determinate sum of specific money,
which varies, and is sometimes more, sometimes less, as the circumstances
we are in require\dots{} This ebbing and flowing of money, supplies and accommodates
itself, without any aid of Politicians\dots{} The buckets work alternately;
when money is scarce, bullion is coined; when bullion is scarce, money
is metled»). (Sir D. North: «Discourses upon Trade», London 1691, p. 22).
Джон Стюарт Мілл, довгий час урядовець східньо-індійської компанії,
потверджує, що в Індії срібна оздоба все ще функціонує безпосередньо як
скарб: «Срібні оздоби витягуються із сховищ і перекарбовуються в монету,
коли норма проценту висока, і знов повертаються назад, коли норма проценту
падає» («Silver ornaments are brought out and coined when there
is a high rate of interest, and go back again when the rate of interest fales»).
(J. St. Mill’s: «Evidence Reports on Bankacts», 1857, p. 2084). За одним
парламентським документом з 1864 р. про імпорт золота й срібла і експорт
до Індії, імпорт золота й срібла в 1863 р. перевищив експорт на 19.367.764
фунти стерлінґів. За вісім останніх років перед 1864 р. лишок імпорту
благородних металів над експортом їх становив 109.652.917 фунтів стерлінґів.
Протягом цього століття в Індії викарбовано монети далеко понад
200.000.000 фунтів стерлінґів.}

\subsubsection{Засіб платежу}

В розглянутій досі безпосередній формі товарової циркуляції
та сама величина вартости завжди з’являлась двоїсто: як товар —
на одному полюсі, як гроші — на протилежному полюсі. Тому
посідачі товарів увіходили в контакт лише як представники
наявних взаємних еквівалентів. Однак з розвитком товарової
циркуляції розвиваються обставини, в наслідок яких відчуження
товарів відокремлюється у часі від реалізації їхніх цін. Тут
досить буде зазначити найпростіші з цих обставин. Один рід
товару потребує для своєї продукції довшого, інший коротшого
часу. Продукція різних товарів пов’язана з різними сезонами.
Один товар родиться на своєму ринку, інший мусить мандрувати
на далекий ринок. Тим то один посідач товарів може виступити
як продавець раніш, ніж інший як покупець. За постійного повто-
\index{i}{0090}  %% посилання на сторінку оригінального видання
рювання тих самих оборудок між тими самими особами умови
продажу товарів реґулюється умовами їхньої продукції. З другого
боку, використовування деяких родів товарів, наприклад,
будинку, продається на якийсь певний час. Лише по скінченні
терміну покупець фактично дістає споживну вартість товару.
Тому він купує товар раніш, ніж сплатить за нього. Один посідач
товарів продає наявний товар, а другий купує лише як представник
грошей або як представник майбутніх грошей. Продавець
стає кредитором, покупець — винуватцем. А що метаморфоза товару
або розвиток його форми вартости тут змінюється, то й гроші
набувають іншої функції. Вони стають засобом платежу.\footnote{
Лютер відрізняє гроші як засіб купівлі й гроші як засіб платежу:
«Ти робиш мені подвійну шкоду тим, що в одному місці не можу платити,
а в другому не можу купити» («Machest mir einen Zwilling aus dem Schadewacht,
das ich hier nicht bezahlen und dort nicht kauffen kann»). (Martin Luther:
«An die Pfarrherrn, wider den Wucher zu predigen», Wittenberg 1540).
}

Те, що діяч циркуляції стає кредитором або винуватцем, випливає
тут із простої товарової циркуляції. Зміна її форми накладає ці
нові печаті на продавця й покупця. Отже, спочатку це ролі так
само минущі й так само навперемінки виконувані тими самими аґентами
циркуляції, як і ролі покупця й продавця. Однак, тепер ця
протилежність вже від самого початку виглядає менш добродушною,
і вона здатна до більшої кристалізації.\footnote{
Про відносини між кредитором і винуватцем серед англійських
купців на початку XVIII віку: «Серед людей торговлі панує тут, в Англії,
такий дух жорстокости, якого не зустрінеш ні в якому іншому суспільстві
або іншому королівстві світу» («Such a spirit of cruelty reigns here
in England among the men of trade, that is not to be met with, in any other
society of men, nor in any other kingdom of the world»). («An Essay оn Credit and the Bankrupt Act.
London», 1707, p. 2).
} Але ці самі відносини
можуть виникати й незалежно від товарової циркуляції.
Приміром, клясова боротьба в античному світі рухається, головне,
у формі боротьби між кредитором і винуватцем і кінчається в
Римі загином винуватця-плебея, місце якого заступає раб. У середні
віки боротьба ця кінчається занепадом винуватця-февдала,
що втрачає свою політичну владу разом з її економічною
базою. Однак за цих епох грошова форма — а відношення між
кредитором і винуватцем має форму грошового відношення — лише
відбиває антагонізм глибших економічних умов життя.

Вернімось до сфери товарової циркуляції. Одночасного з’явлення
еквівалентів, товару і грошей, на обох полюсах процесу
продажу вже не відбувається. Гроші функціонують тепер, поперше,
як міра вартости при визначенні ціни продаваного товару.
Установлена контрактом ціна його міряє зобов’язання покупця,
тобто ту грошову суму, яку він винен на певний термін. Вони
функціонують, подруге, як ідеальний засіб купівлі. Хоч вони
існують лише як обіцянка покупця сплатити гроші, все ж вони
зумовлюють перехід товарів із рук до рук. Лише коли настає
платіжний термін, засіб платежу дійсно вступає в циркуляцію,
тобто з рук покупця переходить до рук продавця. Засіб циркуляції
\index{i}{0091}  %% посилання на сторінку оригінального видання
перетворився на скарб, бо процес циркуляції перервався
на першій фазі, тобто тому, що перетворену форму товару витягнуто
з циркуляції. Засіб платежу вступає в циркуляцію, але
лише після того, як товар уже вийшов із неї. Гроші вже не упосереднюють
процес. Вони самостійно замикають його, як абсолютне
буття мінової вартости, або загальний товар. Продавець
перетворив товар на гроші, щоб грішми задовольнити якусь
потребу, збирач скарбів — щоб зберегти товар у грошовій
формі, винуватець-покупець — щоб мати змогу заплатити. Коли
він не заплатить, то відбудеться примусовий продаж його майна.
Отже, в наслідок суспільної конечности, що випливає з відносин
самого процесу циркуляції, форма вартости товару, гроші, стають
тепер самостійною метою продажу.

Покупець зворотно перетворює гроші на товар раніш, ніж
він перетворив товар на гроші, або провадить другу метаморфозу
товару поперед першої. Товар продавця циркулює, але реалізує
свою ціну лише в приватноправному титулі на одержання
грошей. Він перетворюється на споживну вартість раніш, ніж
перетворився на гроші. Здійснення його першої метаморфози
настає тільки пізніш. \footnote*{Примітка до другого видання. З дальшої цитати, запозиченої з
моєї праці, що з’явилася 1859 р., видно, чому я в тексті не звертаю жодної
уваги на протилежну форму: «Навпаки, у процесі $Г — Т г$роші можуть
бути відчужені як дійсний засіб купівлі, і таким чином ціну товару реалізується
раніш, ніж реалізується споживну вартість грошей або відчужується
товар. Це відбувається, приміром, у звичайній формі передплати.
Або у формі, якої вживає англійський уряд, купуючи опій в індійських
райотів\dots{} Однак у цих випадках гроші функціонують лише в знайомій
уже нам формі засобу купівлі. Капітал звичайно теж авансується у формі
грошей\dots{} Але цей погляд постає лише за горизонтом простої циркуляції»
(«Zur Kritik der Politischen Oekonomie», Berlin 1859, S. 119, 120. — «До
критики політичної економії», ДВУ, 1926 р., стор. 151).}\footnote{У французькому виданні цей абзац зредаговано так: «Припустімо,
що селянин купує у ткача 20 метрів полотна ціною в 2 фунти стерлінґів,
які становлять також ціну одного квартера пшениці, і сплачує ці 2 фунти
стерлінґів лише через місяць. Селянин перетворює свою пшеницю на
полотно раніш, ніж він перетворив її на гроші. Отже, він провадить останню
метаморфозу свого товару раніше за першу. Потім він продає
пшеницю за 2 фунти стерлінґів, які він передає ткачеві в умовлений термін.
Реальні гроші для нього тут уже не служать більше за посередника
для заміщення пшениці полотном. Це вже зроблено. Навпаки, для нього
гроші є останній акт оборудки, оскільки гроші є абсолютна форма вартости,
яку він мусить віддати, загальний товар. Що ж до ткача, то його
товар циркулював і зреалізував свою ціну, але лише за посередництвом
титулу, що ґрунтується на приватному праві. Його товар увійшов у сферу
споживання інших осіб раніш, ніж він перетворився на гроші. Отже,
першу метаморфозу його полотна відкладено, вона здійснюється лише
пізніш, коли настає платіжний термін для селянина». («Le Capital etc.», vol.
I, ch. Ill, p. 56). \emph{Ред.}}

Протягом кожного певного відтинку часу процесу циркуляції
платіжні зобов’язання, що їм настав термін платежу, репрезентують
суму цін тих товарів, що їх продаж викликав ці зобов’язання.
Маса грошей, потрібна для реалізації цієї суми цін,
залежить насамперед від швидкости обігу засобів платежу.
\parbreak{}  %% абзац продовжується на наступній сторінці
