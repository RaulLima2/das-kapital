Законодавче обмеження робочого дня покладає кінець цій втісі.\footnote{
Див. «Reports of Insp. of Fact, for 30 th April 1863», p. 5. Цілком
справедливо критикуючи цей стан речей, лондонські будівельні робітники
заявили підчас великого страйку та льокавту 1860 р., що вони згодяться
на погодинну заробітну плату лише на двох умовах: 1) щоб разом
із ціною робочої години було установлено нормальний робочий день на
9 або 10 годин та щоб ціна за годину десятигодинного робочого дня була
більша, ніж ціна за годину дев’ятигодинного; 2) щоб за кояшу годину
понад нормальний день платилося, як за наднормовий час, відповідно вище.
}

Загальновідомий факт, що чим довший у якійсь галузі промисловосте
робочий день, тим нижча заробітна плата.\footnote{
«Це дуже дивна річ, що там, де довгий робочий день є загальне
правило, загальне правило є й низька заробітна плата» («It is a very
notable thing, too, that where long hours are the rule, small wages are also
so»). («Reports of Insp. of Fact, for 31 st October 1863», p. 9). «Праця, що
добуває мізерну кількість засобів існування, звичайно є надмірно подовжена
» («The work which obtains the scanty pittance of food is for the most
part excessively prolonged»). («Public Health, Sixth Report 1864», p. 15).
} Фабричний
інспектор А. Редґрев ілюструє це порівняльним оглядом
двадцятирічного періоду від 1839 до 1859 р., з якого видно, що
на фабриках, підлеглих законові про десятигодинний робочий
день, заробітна плата підвищилася, тимчасом як на фабриках,
де працювали від 14 до 15 годин на день, вона зменшилась.\footnote{
«Reports of Insp. of Fact. for 30 th April 1860», p. 31, 32.
}

Зі закону, що «при даній ціні праці поденна або потижнева
плата залежить від кількосте постачуваної праці», випливає
насамперед, що чим нижча ціна праці, тим більша мусить бути
кількість праці, бо тим довший мусить бути робочий день, щоб
робітник забезпечив собі хоча б мізерну пересічну заробітну плату.
Низька ціна праці діє тут як спонука здовжувати робочий час.\footnote{
Так, наприклад, в Англії ручні робітники-цвяхарі в наслідок
низької ціни праці мусили працювати 15 годинна день, щоб одержати якнаймізернішу
тижневу плату. «Багато, багато годин на день та в усякий.
}

Але і, навпаки, здовження робочого часу, з свого боку, викликає
зниження ціни праці, а разом з ним і зниження поденної
або потижневої плати.

Визначення ціни праці дробом

денна вартість робочої сили/робочий час даного числа годин

показує, що просте здовження робочого дня знижує ціну праці,
якщо при цьому немає ніякої компенсації. Але ті самі обставини,

за наднормовий час — це спокуса, якій робітники не можуть протистояти».
(«Reports of Insp. of Fact, of 30 th. April 1848», p. 5). Палітурня
в лондонському Сіті вживає до роботи дуже багато молодих дівчат
14—15 років, і саме на підставі контракту для учнів, що приписує певні
робочі години. А проте вони працюють останній тиждень кожного місяця
до 10, 11, 12 і навіть до 1 години вночі разом із старшими робітниками
в дуже мішаному товаристві. «Хазяїни спокушають (tempt) їх підвищеною
платою та грішми на добру вечерю», яку вони беруть у сусідніх
шинках. Велика розпуста, поширена таким способом серед цієї
young immortals\footnote*{
— безсмертної молоді. \emph{Ред.}
} («Child. Empl. Comm.» V Rep., p. 44, n. 191), компенсується
тим, що вони оправляють, між іншим, і багато біблій та інших
душоспасенних книг.