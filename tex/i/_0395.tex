\parcont{}  %% абзац починається на попередній сторінці 
\index{i}{0395}  %% посилання на сторінку оригінального видання 
нові галузі продукції. Дітей наймолодшого віку усувається. Заробітна
плата машинових робітників зростає порівняно з платою
домашніх робітників, з-поміж яких багато належать до
«найбідніших із бідних» («the poorest of the poor»). Заробітна плата
поставлених у кращі умови ремісників, що з ними конкурує
машина, падає. Нові машинові робітники — це виключно дівчата
й молоді жінки. За допомогою механічної сили вони знищують
монополію чоловічої праці в тяжчій роботі й витискують із сфери
легшої праці маси старих жінок і дітей-недолітків. Всемогутня
конкуренція вбиває найслабших робітників ручної праці. Жахливий
зріст числа випадків голодної смерти (death from starvation)
у Лондоні за останнє десятиліття йде паралельно з поширенням
машинового шиття.\footnote{
Приклад. 26 лютого 1864 р. тижневий збіт про смертність Registrar
General реєструє п’ять випадків голодної смерти. Того самого
дня «Times» повідомляє про новий випадок голодної смерти. Шість жертов
голодної смерти за один тиждень!
} Нові робітниці коло швацької машини,
що її вони пускають у рух рукою й ногою або тільки рукою,
сидьма й навстоячки, залежно від важкости, величини й спеціяльности
машини, витрачають багато робочої сили. їхнє заняття стає
шкідливим для здоров’я в наслідок тривалости процесу, хоч він
здебільша і коротший, аніж за старої системи. Всюди, де швацькі
машини, як ось у виробництві чобіт, корсетів, капелюхів тощо,
заводиться в майстерні, що й без того були тісні й переповнені,
вони збільшують антигігієнічні впливи. «Вражіння, — каже комісар
Лорд, — при вході до низьких робітних приміщень, де працює
разом 30—40 машинових робітників, нестерпне... Спека, почасти
від газових пічок для огрівання прасок, жахлива... Навіть тоді,
коли в таких приміщеннях переважають так звані помірні робочі
години, тобто від 8 години ранку до 6 години вечора, — то все ж
день-у-день зомліває звичайно 3 або 4 особи».\footnote{
«Children’s Employment Commission. 2 nd. Report 1864», p. LXVII,
n. 406—409, p. 84, n. 124, p. LXXIII, n. 441, p. 66, n. 6, p. 84, n. 126,
p. 78, n. 85, p. 76, n. 69, p. LXXII, n. 483.
}

Переворот у суспільному способі продукції, цей неминучий
продукт перетворень у засобі продукції, відбувається в строкатому
хаосі переходових форм. Вони змінюються залежно від того,
в якому обсязі та протягом якого часу швацька машина вже захопила
ту або іншу галузь промисловости, змінюються залежно
від того стану, в якому перебували робітники до заведення
машини, — залежно від переваги мануфактурного, ремісничого або
домашнього виробництва, наймової плати за робітні приміщення\footnote{
«Орендна плата за робітні приміщення є, здається, той елемент,
що відіграє вирішальну ролю, а тому в столиці стара система роздавати
роботу дрібним підприємцям і родинам трималася найдовше; так само
до неї там і поверталися найшвидше» («The rental of premises required
for work rooms seems the element which ultimately determines the point,
and consequently it is in the metropolis, that the old system of giving work
out to small employers and families has been longest retained, and earliest
returned to»). (Там же, стор. 84, n. 123). Остання фраза стосується виключно
до шевства.
}
\parbreak{}  %% абзац продовжується на наступній сторінці
