Послухаймо тепер, як сам капітал малює цю 24-годинну
систему. Він, звичайно, поминає мовчанням ексцеси цієї системи,
зловживання нею задля «жорстокого й неймовірного» здовжування
робочого дня. Він говорить лише про систему в її «нормальній»
формі.

«Панове Нейлор і Вікерс, фабриканти сталі, які вживають
600—700 робітників, поміж ними лише 10% молодші за 18 років
життя, а з числа останніх лише 20 хлопчаків працює вночі,
висловлюються ось як: «Хлопці зовсім не страждають від спеки.
Температура щось до 86—90°... У кузнях і вальцювальнях
руки працюють вдень і вночі навпереміну, навпаки, вся інша
праця виконується вдень, від 6 години ранку до 6 години вечора.
В кузні працюють від 12 години до 12. Деякі руки постійно працюють
вночі, без зміни денної і нічної праці... Ми не вважаємо,
щоб денна або нічна праця різно впливали на здоров’я (добродіїв
Нейлора й Вікерса?), певно, люди сплять краще, коли користуються
з відпочинку в той самий час, ніж як його зміняти...
Щось із двадцять хлопців, молодших за вісімнадцять років,
працюють разом з нічною зміною. Ми не можемо обійтись (not
well do) без нічної праці підлітків, молодших за 18 років. Наше
заперечення — збільшення витрат продукції. Умілі руки й
керівників відділів знайти не легко, хлопців же можна мати
скільки схочете... Певна річ, беручи до уваги незначний процент
підлітків, що їх ми вживаємо, обмеження нічної праці мало б
для нас невелику вагу або інтерес».99

Пан Дж. Елліс з фірми панів Дж. Бравн і К°, з фабрики
сталі й заліза, які вживають 3.000 чоловіків і підлітків,
і саме для частини важких сталевих і залізних робіт, «змінами,
вдень і вночі», заявляє, що у важких сталевих відділах на двох

тиждень, звичайно від неділі вночі до 12 години ночі наступної суботи.
Робочий персонал, що працює в денній зміні, робить п’ять днів на тиждень
по 12 годин і один день 18 годин, а ті, що працюють у нічній зміні, — п’ять
ночей по 12 годин і одну ніч 6 годин. В інших випадках кожна зміна працює
по 24 години одна за однією навпереміну. Одна зміна працює 6 годин у
понеділок і 18 годин у суботу, щоб було повних 24 години. В інших випадках
заведено проміжну систему, за якої всі робітники, що працюють
коло машин для вироблення паперу, працюють протягом цілого тижня
щодня 15—16 годин. Ця система, — каже член слідчої комісії Лорд, —
сполучає, здається, в собі всі злигодні 12-годинної і 24-годинної системи
змін. Діти менші за 13 років, підлітки молодші за 18 і жінки працюють
за такої нічної системи. Інколи за дванадцятигодинної системи вони
мусили працювати подвійну зміну — 24 години, — щоб заступити відсутніх.
Свідчення свідків доводять, що хлопчаки і дівчата дуже часто працюють
наднормовий час, що частенько триває без перестанку 24 і навіть 36 годин.
В «безперервному й незмінному процесі» ґлянсування можна побачити
дванадцятирічних дівчаток, які працюють цілий місяць по 14 годин на
день «без жодного регулярного відпочинку або перерви, крім двох або
щонайбільше трьох півгодинних павз для їжі». По Деяких фабриках, де
зовсім немає регулярної нічної праці, працюють страшенно багато в наднормовий
час, і «це часто у найбрудніших, наймонотонніших процесах
серед неймовірної спеки». («Children’s Employment Commission. 4 th
Report, 1865», p. XXXVIII and XXXIX).

99 Fourth Report etc., 1865, 79, p. XVI.
