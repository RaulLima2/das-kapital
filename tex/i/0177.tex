частиною, засобами продукції, якомога більшу кількість додаткової
праці.\footnote{
«Завдання капіталіста — за допомогою витраченого капіталу здобути
якомога більшу суму праці» («D’obtenir du capital dépensé la
plus forte somme de travail possible»), (I. G. Courcelle-Seneuil: «Traité
théorique et pratique des entreprises industrielles», 2 ème éd. Paris 1857,
p. 62).
} Капітал є вмерла праця, що, як вампір, оживає,
лише вбираючи в себе живу працю, і то довше живе, що більше
вона її вбирає. Час, протягом якого робітник працює, — це є
той час, протягом якого капіталіст споживає куплену ним робочу
силу.\footnote{
«Втрата однієї години праці на день є велика шкода для торговельної
держави». «Споживання предметів розкоші робітничою біднотою цього
королівства, а особливо мануфактурною людністю, є дуже велике; на це
вона споживає свій час, а це є найфатальніша з усіх форм споживання»
(«An Hour’s Labour lost in a day is a prodigious injury to a commercial
state». «There is a very great consumption of luxuries among the labouring
poor of this kingdom; particularly among the manufacturing populace; by
which they also consume their time, the most fatal of consumptions»). (An
Essay on Trade and Commerce», London 1770, p. 47, 153).
} Коли робітник споживає для себе самого той час, що ним
він порядкує, то він обкрадає капіталіста.\footnote{
«Коли вільний робітник, відпочине на хвилинку, зажерлива економія,
яка турботно стежить за ним, запевняє, що він обкрадає її» («Si le
manouvrier libre prend un instant de repos» l’économie sordide qui le suit
des yeux avec inquiétude, prétend qu’il la vole».). (N. Linguet; «Théorie
des Loix Civiles etc.», London 1767, vol. II, p. 466).
}

Отже, капіталіст покликається на закон товарового обміну.
Він, як і кожний інший покупець, силкується здобути якомога
більшу користь із споживної вартости свого товару. Але раптом
чути голос робітника, голос, що занімів був серед шуму й гуркоту
процесу продукції:

Товар, що його я тобі продав, відрізняється від решти товарової
черні тим, що споживання його створює вартість, і вартість
більшу, ніж сам він коштує. Оце була причина, з якої ти його
купив. Те, що тобі видається зростанням капіталу, для мене є
надмірна трата робочої сили. Ти і я знаємо на ринку лише один
закон — закон обміну товарів. Споживання товару належить
не продавцеві, що його відчужує, а покупцеві, що здобуває його.
Тому тобі належить споживання моєї денної робочої сили. Але
за допомогою її денної продажної ціни я мушу щоденно репродукувати
її, щоб мати змогу її знову продавати. Залишаючи осторонь
природне виснажування в наслідок старости й т. ін., я мушу
бути спроможний працювати назавтра з таким самим нормальним
станом сили, здоров’я й свіжости, що й сьогодні. Ти постійно
проповідуєш мені євангелію «ощадности» й «поздержливости».
Гаразд! Як розумний, ощадний господар я хочу зберегти своє
єдине майно, робочу силу, і здержуватися від усякого дикого
марнотратства її. Я буду щоденно приводити в рухомий стан,
перетворювати на рух, на працю лише таку її кількість, яка
погоджується з її нормальним триванням і її здоровим розвитком.
Безмірним подовженням робочого дня ти можеш за один день
пустити в рух більшу кількість моєї робочої сили, ніж я можу