її ринкової ціни (!), він начебто авансує ріжницю своєму підприємцеві
(?) і т. ін.».9а В дійсності робітник авансує даром
капіталістові свою працю протягом тижня і т. д., з тим, щоб
наприкінці тижня і т. ін. одержати її ринкову ціну; за Міллом,
це перетворює робітника на капіталіста! На пласкій рівнині
й грудка землі видається горбом; пласкість нашої сучасної буржуазії
можна зміряти калібром її «великих мислителів».

Розділ п’ятнадцятий

Зміна величини ціни робочої сили та додаткової
вартости

[У третьому відділі, сьомий розділ, ми аналізували норму
додаткової вартости, але лише з погляду продукції абсолютної
додаткової вартости. У четвертому відділі ми знайшли додаткові
визначення. Тут нам треба зрезюмувати все посутнє про це].*

Вартість робочої сили визначається вартістю звичних доконечних
засобів існування пересічного робітника. Маса цих засобів
існування, хоч форма їхня і може змінятись, для даної епохи
й даного суспільства є дана, а тому її треба розглядати як сталу
величину. Змінюється лише вартість цієї маси. Ще два інші
фактори, входять у визначення вартости робочої сили. З одного
боку, витрати на її розвиток, які змінюються із зміною способу
продукції, з другого боку, природні ріжниці робочої сили, тобто,
чи є вона чоловіча або жіноча, дорослих робітників або підлітків.
Споживання цих різних робочих сил, знову ж таки зумовлюване
способом продукції, створює велику ріжницю у витратах репродукції
робітничої родини та у вартості дорослого робітникачоловіка.
Однак у дальшому досліді обидва ці фактори не береться
на увагу. 9b

Ми припускаємо: 1) що товари продається за їхньою вартістю;
2) що ціна робочої сили може іноді піднестися понад свою вартість,
але ніколи не падає нижче від неї.

Зробивши такі припущення, ми виявили, що відносні величини
ціни робочої сили й додаткової вартости визначаються
трьома обставинами: 1) довжиною робочого дня, або екстенсивною
величиною праці; 2) нормальною інтенсивністю, або її інтенсивною
величиною, тобто тією обставиною, що певну кількість
праці витрачається за певний час; 3) нарешті, продуктивною
силою праці, тобто тією обставиною, що та сама кількість праці
за той самий час дає, залежно від ступеня розвитку умов про-

9a J. St. Mill: «Principles of Political Economy», London 1868,
p. 252, 253 passim. (Вищенаведені місця перекладено з французького
видання «Капіталу». — Ф. Е.).

9b Розглянутий на стор. 253—254 випадок, тут, природно, також
виключено. (Примітка до третього видання — Ф. Е.).

* Заведене у прямі дужки ми беремо з другого німецького видання.
Ред.
