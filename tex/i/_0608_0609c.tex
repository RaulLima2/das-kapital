\parcont{}  %% абзац починається на попередній сторінці
\index{i}{0608}  %% посилання на сторінку оригінального видання
частина пустирів і торфовищ, що їх раніш не використовували,
служить тепер для поширення скотарства. Дрібні й середні
фармери — я залічую сюди всіх тих, що обробляють не більше
як 100 акрів землі — все ще становлять приблизно \sfrac{8}{10} із загального
числа. 186c  Конкуренція капіталістичної рільничої продукції
щораз більше й більше душить їх, і тим то вони постійно постачають
клясі найманих робітників нових рекрутів. Однісінька
велика промисловість Ірляндії, фабрикація полотна, потребує
порівняно мало дорослих робітників-чоловіків і, не зважаючи
на її поширення після подорожчання бавовни в 1861--1866 рр.,
вона взагалі вживає лише порівняно незначну частину людности.
Як усяка інша велика промисловість, вона постійними коливаннями
у своїй власній сфері постійно продукує відносне перелюднення,
навіть і за абсолютного зростання маси робітників, яку
вона поглинає. Злидні сільської людности є за п’єдестал для велетенських
фабрик сорочок тощо, робітнича армія яких розпорошена
здебільшого по селах. Тут ми знову таки бачимо змальовану
раніш систему домашньої праці — систему, де недостатня
плата за роботу і надмірна праця служать за методичні засоби
продукувати «зайвих» робітників. Нарешті, хоч зменшення людности
не має тут таких руйнаційних наслідків, як у країні з розвиненою
капіталістичною продукцією, проте й тут воно відбувається
не без постійного зворотного впливу на внутрішній ринок.
[Еміґрація лишає по собі не лише порожні будинки, але й зруйнованих
квартироздавців].* * Та прогалина, що її створює тут
еміґрація, не лише зменшує місцевий попит на працю, але також
і доходи дрібних крамарів, ремісників, взагалі дрібних промисловців.
[Кожне нове виселення перетворює частину дрібної середньої
кляси на пролетарів].* Звідси зменшення доходів між
60 і 10 фунтами стерлінґів у таблиці Е.

Ясну картину становища сільських поденників в Ірляндії
ми маємо у звітах інспекторів ірляндської адміністрації в справах
про бідних (1870).\footnoteA{
«Reports from the Poor Law Inspectors on the wages of Agricultural
Labourers in Dublin, 1870». Порівн. також «Agricultural Labourers
(Ireland) Return etc. dated 8th March 1861», London, 1862.
} Урядовці такого уряду, що тримається
лише за допомогою баґнетів і стану облоги, то явного, то прихованого,
мусять бути обережними у висловах, чим їхні колеґи
в Англії нехтують; а проте вони не дозволяють своєму урядові
уколисувати себе ілюзіями. За їхніми відомостями, рівень
заробітної плати на селі, і досі все ще дуже низький, все ж за
останні двадцять років підвищився на 50--60\% і становить
тепер пересічно 6--9 шилінґів на тиждень. Але за цим позір-

186с Примітка до другого видання. Згідно з однією таблицею в Murphy:
«Ireland, Industrial, Political and Social», 1870, 94,6\% усіх земель
є фарми, менші від 100 акрів кожна і 5,4\% — фарми понад 100 акрів.

* Заведене у прямі дужки ми беремо з другого німецького видання.
\emph{Ред.}
\index{i}{0609}  %% посилання на сторінку оригінального видання
ним підвищенням криється реальне зниження заробітної плати,
бо воно навіть не урівноважує того підвищення цін на доконечні
засоби існування, що сталося за той час. Доказ — нижченаведений
витяг з офіціяльних звітів одного ірляндського робітного
дому.

Пересічні тижневі витрати на утриманця однієї людини

Роки    Харчі    Одяг    Разом
Від 29 вересня 1848 р.
до 29 вересня 1849 р.    1 шилінґ З \sfrac{1}{4} пенса    3 пенси    1 шил. 6\sfrac{1}{4} пенса

Від 29 вересня 1868 р.
до 29 вересня 1869 р.    2 шилінґи 7\sfrac{1}{4}пенса    6 пенсів    3 шил. \sfrac{1}{4} пенса

Отже, ціна доконечних засобів існування підскочила майже
вдвоє, а ціна одягу рівно вдвоє, аніж перед двадцятьма роками.

Навіть коли залишити осторонь цю диспропорцію, то саме
порівняння заробітних плат, визначених у грошах, далеко ще
не дає правдивого висновку. Перед голодом велику частину заробітної
плати на селі видавали in natura, грішми виплачували
лише дуже невеличку частину; нині грошова виплата стала загальним
правилом. Вже з цього випливає, що, хоч який буде
рух реальної заробітної плати, її грошовий рівень мусив підвищитися.
«Перед голодом сільський поденник мав шматок
землі, де він культивував картоплю і відгодовував свиней та
дробину. Нині він мусить не тільки купувати собі всі засоби
існування, але він втрачає й ті доходи, що він мав із продажу
свиней, дробини і яєць».\footnote{
Там же, стор. 291.
} Справді, раніше сільські робітники
зливалися з дрібними фармерами і здебільша становили
лише ар’єрґард середніх і великих фарм, де вони находили для
себе заняття. Лише від часу катастрофи 1846 р. вони почали становити
частину кляси власне найманих робітників, окрему верству,
зв’язану із своїми панами-наймачами лише грошовими відносинами.
Ми вже знаємо, який був їхній житловий стан перед 1846 р.
Від того часу він ще більше погіршав. Деяка частина сільських
поденників, що, зрештою, з дня на день меншає, живе ще на
землях фармерів у переповнених хатинах, що їхній огидний стан
далеко перевищує все те найгірше, що виявили нам у цьому відношенні
англійські рільничі округи. І такий є стан речей повсюди,
за винятком деяких округ в Ulster’i, на півдні у графствах
Cork, Limerick, Kilkenny та інших; на сході у Wicklow’i
Wexford’i і т. д.; у центрі в King’s і Queen’s County, Dublin’i
і т. д.; на півночі в Down’i, Antrim’y, Tyrone і т. д.; на заході
в Sligo, Roscommon’i, Mayo, Galway і т. д. «Це, — вигукує один
\parbreak{}  %% абзац продовжується на наступній сторінці
