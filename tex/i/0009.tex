формування людину постійно підтримують сили природи. Отже,
праця не є однісіньке джерело продукованих нею споживних вартостей,
речового багатства. Праця є його батько, як каже Вільям
Петті, а земля — його мати.

Перейдімо тепер від товару як предмету споживання до товарової
вартости.

Згідно з нашим припущенням, сурдут має удвоє більшу вартість,
ніж полотно. Та це лише кількісна ріжниця, яка нас поки
не цікавить. Тому ми нагадуємо, що коли вартість одного
сурдута удвоє більша від вартости 10 метрів полотна, то 20 метрів
полотна мають вартість такої самої величини, як один сурдут. Як
вартості, сурдут і полотно — речі однакової субстанції, об’єктивні
вирази однорідної праці. Але кравецтво й ткацтво є якісно
різні праці. Однак, бувають такі суспільні обставини, коли та
сама людина займається навперемінку то кравецтвом, то ткацтвом,
і тому ці дві різні форми праці є лише модифікації праці
того самого індивіда, а не є ще осібні сталі функції різних індивідів,
цілком так само, як сурдут, пошитий нашим кравцем сьогодні,
і штани, які він пошиє завтра, є лише відміни тієї самої
індивідуальної праці. Наочний досвід навчає нас далі, що в нашому
капіталістичному суспільстві, залежно від змін напряму
в попиті на працю, певна кількість людської праці постачається
навперемінку то у формі кравецтва, то у формі ткацтва. Ця зміна
форм праці, звичайно, не відбувається без тертя, але вона мусить
відбуватись. Коли залишити осторонь визначеність продуктивної
діяльности, а тому й корисний характер праці, то в ній зостається
те, що вона є затрата людської робочої сили. Хоч кравецтво й
ткацтво є якісно різні форми продуктивної діяльности, а проте
обидва вони є продуктивна затрата людського мозку, мускулів,
нервів, рук тощо, і в цьому розумінні вони обидва є людська праця.
Це тільки дві різні форми затрати людської робочої сили. Певна
річ, сама людська робоча сила мусить бути більш або менш розвинена,
щоб витрачатись у тій чи іншій формі. Але вартість товару
репрезентує просто людську працю, затрату людської праці
взагалі. Як у буржуазному суспільстві генерал або банкір відіграють
велику, а просто людина, навпаки, дуже мізерну ролю, 14
так само стоїть тут справа і з людською працею. Вона є затрата
простої робочої сили, що її пересічно має в своєму тілесному
організмі кожна звичайна людина без особливої підготови. Сама
проста пересічна праця, хоч вона й змінює свій характер у різних
країнах і в різні культурні епохи, проте вона є визначена в кож-

che l’ingegno umano ritrova analizzando l’idea della riproduzione; e tanto
è riproduzione di valore e di ricchezze se la terra, l’aria e l’acqua ne’campi
si trasmutino in grano, come se colla mano dell'uomo il glutine di un insetto
si trasmuti in velluto ovvero alcuni pezzetti di metallo si organizzino a formare
una ripetizione»). (Pietro Verri: «Meditazioni sulla Economie Politica»,
вперше надруковано року 1773 — y виданні італійських економістів
von Custodi, Parte Modema, t. XV, p. 22).

14 Порівн. Hegel: «Philosophie des Rechts», Berlin 1840, cтop. 250,
§ 190.
