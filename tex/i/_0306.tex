\parcont{}  %% абзац починається на попередній сторінці
\index{i}{0306}  %% посилання на сторінку оригінального видання
заступити їхнє місце. Це, природно, не виключає того, що така
заміна часто зумовлює великі технічні переміни в механізмі,
сконструйованому первісно виключно для людської рушійної
сили. Нині всі машини, яким ще тільки доводиться прокладати
собі шлях, як, приміром, машина до шиття, машини виготовляти
хліб і~\abbr{т. д.}, якщо вони з самого початку за своїм призначенням не
виключають малого маштабу, конструюються одночасно і для людської
і для суто механічної рушійної сили.

Машина, яка є вихідний пункт промислової революції, заміняє
робітника, що орудує одним-однісіньким знаряддям, таким механізмом,
який одночасно оперує масою однакових або однорідних
знарядь, і що його рухає одним-одна рушійна сила, хоч би яка
була її форма\footnote{
«Сполучення всіх цих простих інструментів, що їх пускає в рух
одним-один мотор, становить машину». (\emph{Babbage}: «On the Economy
of Machinery», London 1832).
}. Тут ми маємо машину, але ще тільки як простий
елемент машинової продукції.

Збільшення розміру робочої машини та числа її знарядь,
що одночасно функціонують, потребує більшого рушійного механізму,
а цей механізм потребує дужчої за людську рушійної сили,
щоб перебороти свій власний опір, не кажучи вже про те, що
людина є дуже недосконалий інструмент для продукції однорідного
й безупинного руху. Коли припустити, що людина діє
вже лише як проста рушійна сила, отже, що місце її знаряддя
заступила виконавча машина, то сили природи можуть тепер
заступити її і як рушійну силу. З-поміж усіх великих рушійних
сил, що перейшли з мануфактурного періоду, найгірша була
кінська сила, почасти тому, що кінь має свою власну голову,
а почасти тому, що він дорогий і може вживатись по фабриках
лише обмежено\footnote{
У січні 1861 p. Джон К.~Мортон прочитав у Society of Arts реферата
про «сили, що їх уживають у рільництві». В ньому сказано, між
іншим: «Кожне поліпшення, що призводить до одноманітности ґрунту,
збільшує можливість уживати парову машину, щоб добувати суто механічну
силу\dots{} Сила коней потрібна там, де покручені горожі й інші перешкоди
утруднюють одностайний рух. З кожним днем ці перешкоди щораз
більше зникають. У таких операціях, які потребують більше волі й менше
дійсної сили, єдино придатна є сила, якою щохвилини керує людський
розум, отже, людська сила». Потім Мортон зводить парову силу, кінську
силу та людську силу на одиницю міри, прийняту в парових машинах,
а саме на силу, що підіймає \num{33.000} фунтів угору на один фут за одну хвилину,
та обчислює витрати на одну парову кінську силу: при паровій
машині в 3\pens{ пенси}, а з конем у 5\sfrac{1}{2}\pens{ пенсів} за годину. Далі, коня — коли
дбають цілком зберегти його здоров’я — можна вживати щодня лише
8 годин. Обробляючи землю силою пари, можна протягом року з кожних
7 коней заощадити щонайменш працю 3, при чому витрати будуть не
більші, ніж витрати на ці звільнені коні протягом тих 3 або 4 місяців,
коли їх саме й можна дійсно використовувати. Нарешті, в тих рільничих
операціях, де можна вживати силу пари, вона, порівняно з силою коней,
поліпшує якість продукту. Щоб виконати роботу парової машини, довелося
б ужити 66 робітників із загальною платою 15\shil{ шилінґів} за годину,
а щоб виконати роботу коня — 32 осіб із загальною платою в 8\shil{ шилінґів}
за годину.
}. А все ж у дитячий вік великої промисловости
\parbreak{}  %% абзац продовжується на наступній сторінці
