нець-кінцем визначається рівнанням: вартість денної праці =
денній вартості робочої сили. Отже, відштучна плата — це лише
змодифікована форма почасової плати.

Розгляньмо тепер трохи ближче характеристичні особливості
відштучної плати.

Якість праці контролюється тут через самий продукт, який
мусить бути пересічної добротности, щоб одержати повну відштучну
ціну. З цього боку відштучна плата стає найстрашнішим
джерелом відраховань із заробітної плати й капіталістичного
шахрайства.

Вона дає капіталістові цілком визначену міру інтенсивности
праці. Лише той робочий час, що втілюється у заздалегідь визначеній
і досвідом встановленій кількості товарів, уважається за
суспільно-доконечний робочий час та оплачується як такий.
Тому по великих кравецьких майстернях Лондону певна штука
продукту праці, наприклад, жилет тощо, зветься годиною,
півгодиною й т. ін., а годину рахується по 6 пенсів. З практики
відомо, скільки пересічно продукту виробляється за годину.
При зміні моди або полагодженнях тощо постає суперечка між
підприємцем та робітником, чи певна штука дорівнює одній годині
і т. д., аж поки й тут досвід вирішить справу. Те саме бачимо й
по лондонських майстернях меблів і т. ін. Коли робітник не має
пересічної дієздатности й тому не може дати певний мінімум денного
продукту, то його звільняють.\footnote{
«Йому (прядунові) видають певну кількість фунтів бавовни, і за
певний час він мусить дати натомість стільки й стільки фунтів ниток
або пряжі певної тонини, і він одержує стільки й стільки плати за кожний
даний ним фунт продукту. Коли продукт щодо якости має хиби, то
на нього накладають пеню; коли ж кількість менша за мінімум, визначений
для даного часу, то його звільняють та замінюють умілішим робітником».
(«So much weight of cotton is delivered to him (the spinner),
and he has to return by a certain time, in lieu of it, a given weight of
twist or yarn, of a certain degree of fineness, and he is paid so much per
pound for all that he so returns. If his work is defective in quality, the
penalty falls on him; if less in quantity than the minimum fixed for a
given time, he is dismissed and an abler operative procured»). (C7re: «Philosophy
of Manufacture», p. 316).
}

А що якість та інтенсивність праці тут контролюються через
саму форму заробітної плати, то догляд за працею у значній мірі
стає зайвий. Тому ця форма становить основу так змальованої
вже раніш сучасної домашньої праці, як і ієрархічно організованої
системи експлуатації та пригнічення. Ця остання система
має дві основні форми. Відштучна плата, з одного боку, полегшує
паразитам всуватися поміж капіталіста й найманого робітника,
перепродавати працю (subletting of labour). Бариш посередників
випливає виключно з ріжниці між ціною праці, яку платить
капіталіст, і тією частиною цієї ціни, яку посередники дійсно
платять робітникові.80 Ця система має в Англії характеристичну

50 «Коли праця переходить через багато рук і кожна особа хоче
мати свою частину зиску, тимчасом як лише остання з них дійсно працює,
тоді плата, яку дістає робітниця, є надто, мізерна» («It is when work
