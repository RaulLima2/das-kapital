точні постанови закону з 1844 р. про час для їжі дають робітникам
лише дозвіл їсти й пити перед їхнім приходом на фабрику
й після їхнього виходу з фабрики, отже, в себе вдома! І чому б
таки робітникам не обідати перед 9 годиною ранку? Однак коронні
юристи вирішили, що приписаний законом час на їжу «треба
давати в перервах дійсного робочого дня, і що це протизаконно
примушувати працювати без перерви одна по одній 10 годин
від 9 години ранку до 7 години вечора».149

Після цих добродушних демонстрацій капітал, щоб, підготовити
бунт, вдався до такого кроку, що відповідав букві закону
з 1844 р., отже, був леґальний.

Щоправда, закон з 1844 р. забороняв вживати праці дітей 8 —
13-річного віку, які працювали перед 12 годиною дня, знову після
1 години по півдні. Але він ніяким чином не реґулював 6\sfrac{1}{2}-годинної
праці дітей, що їхній робочий час починався о 12 годині
дня або пізніш! Отже, восьмилітніх дітей, якщо вони починали
роботу о 12 годині дня, можна було вживати до праці від 12 до
1 години, тобто на одну годину, від 2 до 4 години, тобто на 2 години,
і від 5 до пів на 9 годину вечора, тобто на 3\sfrac{1}{2} години; разом
6\sfrac{1}{2} годин, визначених законом! Або ще краще. Щоб пристосувати
вживання праці дітей до праці дорослих робітників-чоловіків, які
працювали до пів на 9 годину вечора, фабрикантам треба було не
давати дітям жодної роботи перед 2 годиною по півдні, а потім вони
могли їх тримати на фабриці без жодних перерв до пів на 9 годину
вечора! «А тепер уже виразно визнають, що останніми часами,
внаслідок ненажерливости фабрикантів та їхнього бажання тримати
машини в русі більш, ніж 10 годин на добу, в Англії крадькома
встановилась практика примушувати 8—13-літніх дітей
обох статей працювати після того, як підуть усі підлітки й жінки,
з самими дорослими чоловіками до пів на дев’яту вечора».150
Робітники й фабричні інспектори протестували з гігієнічних і
моральних причин. Але капітал відповідав:

«На голову мою хай вчинки
Падуть мої! Я вимагаю
Лиш права власного! Пені
Та векселя мого оплати!» *

Справді, за поданими до Палати громад 26 липня 1850 р.
статистичними даними, на 15 липня 1850 р. було, не зважаючи
на всі протести, 3.742 дітей на 275 фабриках, підданих під цю
«практику».151 Але цього ще не досить! Капітал своїм оком
рися відкрив, що хоч закон 1844 р. і не дозволяє п’ятигодинної
праці в передобідній час без принаймні 30-хвилинної перерви
для відпочинку, але не приписує нічого подібного щодо післяобідньої
праці. Тому він домагався й добився тієї приємности,

149 «Reports etc. for 31 st October 1848», p. 130.

150 «Reports etc.», там же, стор. 42.

151 «Reports etc. for 31 st October 1850», p. 5, 6.

* Цю цитату, як і дальшу, взято з Шекспірового «Шейлока». Ред.
