\parcont{}  %% абзац починається на попередній сторінці
\index{i}{0284}  %% посилання на сторінку оригінального видання
до якоїсь однобічної функції та зв’язується з нею на цілий його
вік, то, з другого боку, різні операції праці в такій самій мірі
пристосовується до тієї ієрархії природних і придбаних здібностей.\footnote{
Доктор Юр у своїй апотеозі великої промисловости гостріше
схоплює специфічний характер мануфактури, ніж попередні економісти,
що не мали його полемічного інтересу, і навіть ніж його сучасники,
як от Беббедж, який хоч і перевищує його як математик і механік, а проте
розглядає велику промисловість, власне кажучи, лише з погляду мануфактури.
Юр зауважує: «Пристосування робітника до кожної окремої
операції становить суть поділу праці». З другого боку, цей поділ він називає
«пристосуванням праць до різних індивідуальних здібностей» і характеризує
нарешті цілу мануфактурну систему як «систему градацій
відповідно до ступеня вправности», як «поділ праці за різними ступенями
вправности» й т. д. (Ure: «Philosophy of Manufacture», p. 19 —
23 і далі).
} Тим часом кожний процес продукції потребує певних
простих маніпуляцій, до яких здатна кожна перша-ліпша людина.
І ці маніпуляції звільняються тепер від їхнього рухомого
зв’язку з змістовнішими моментами діяльности й кам’яніють у
виключні функції.

Тим-то мануфактура утворює в кожному реместві, яке вона
захоплює, клясу так званих ненавчених (ungeschickter) робітників,
яких ремісниче виробництво строго виключало. Розвиваючи
до віртуозности одну якусь цілком зоднобічнену спеціяльність
коштом загальної працездатности, вона вже й самий брак усякого
розвитку починає робити спеціяльністю. Поруч ієрархічних
щаблів постає простий поділ робітників на навчених та ненавчених
(geschickte und ungeschickte). Для останніх витрати
на навчання відпадають цілком, для перших вони нижчі порівняно
з ремісником, бо функції їхні простіші. В обох випадках вартість
робочої сили спадає.\footnote{
«Кожний професійний робітник\dots{} дістаючи, в наслідок того,
що він постійно виконує ту саму працю, змогу вдосконалюватися, стає
дешевший» («Each handicraftsman, being\dots{} enabled to perfect to himself
by practice in one point, became\dots{} a cheaper workman»). (Ure:
«Philosophy of Manufacture», p. 19).
} Винятки бувають лише остільки, оскільки
розчленування робочого процесу створює нові складні функції,
яких у ремісничому виробництві або зовсім не було, або якщо вони
й були, то не в такому самому розмірі. Відносне зневартнення
робочої сили, яке виникає в наслідок зникнення або зменшення
витрат на навчання, безпосередньо спричинюється до підвищеного
зростання вартости капіталу, бо все, що скорочує час, доконечний
для репродукції робочої сили, поширює сферу додаткової
праці.

4. Поділ праці всередині мануфактури та поділ праці
всередині суспільства

Ми розглянули спочатку походження мануфактури, потім
її прості елементи — частинного робітника і його знаряддя, —
нарешті, її цілий механізм. Тепер розгляньмо коротко відношення
\parbreak{}  %% абзац продовжується на наступній сторінці
