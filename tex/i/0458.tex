даткової вартости перетворюються через просту зміну форми
на закони заробітної плати. Так само ріжниця між міновою вартістю
робочої сили й масою засобів існування, на які обмінюється
ця вартість, з’являється тепер як ріжниця між номінальною та
реальною заробітною платою. Марно було б повторювати щодо
форми виявлення те, що вже розвинуто щодо посутньої форми.
Тому ми обмежуємось тими небагатьма пунктами, що характеризують
почасову плату.

Та грошова сума, 30 що її робітник одержує за свою денну,
тижневу й так далі працю, становить суму його номінальної
заробітної плати, або заробітної плати, оцінюваної за її вартістю.
Але ясно, що, залежно від довжини робочого дня, отже,
залежно від кількост і праці, яку він дає за день, та сама денна,
тижнева й так далі заробітна плата може репрезентувати дуже
різну ціну праці, тобто дуже різні грошові сум і за ту саму кількість
праці.31 Отже, при почасовій платі треба знову таки відрізняти
цілу суму заробітної плати, поденної, потижневої і так
далі, від ціни праці. Як же знайти цю ціну, тобто грошову вартість
даної кількости праці? Пересічну ціну праці ми добудемо,
поділивши пересічну денну вартість робочої сили на число годин
пересічного робочого дня. Якщо, приміром, денна вартість
робочої сили становить 3 шилінґи, вартість, спродуковану протягом
6 робочих годин, а робочий день має 12 годин, то ціна однієї
робочої години дорівнює 3 шилінґам/12 = 3 пенсам. Знайдена
таким чином ціна робочої години служить за одиницю міри для
ціни праці.

Звідси випливає, що денна, тижнева і так далі заробітна плата
може лишатися однакова, хоч ціна праці постійно падатиме.
Якщо, наприклад, звичайний робочий день має 10 годин, а денна
вартість робочої сили рівна 3 шилінґам, то ціна робочої години
становить З\sfrac{3}{5} пенсів; вона спадає до 3 пенсів, якщо робочий день
зростає до 12 годин, та до \sfrac{22}{5} пенсів, коли він зростає до 15 годин.
Денна або тижнева заробітна плата, не зважаючи на це,
лишаються незмінні. Навпаки, денна або тижнева заробітна
плата може підвищитися, хоч ціна праці лишатимеї'ься стала або
навіть падатиме. Кол і, приміром, робочий день має десять годин,
а денна вартість робочої сили становить 3 шилінґи, то ціна однієї
робочої години дорівнює \sfrac{33}{5} пенсів. Якщо робітник у наслідок
збільшення роботи працює при незмінній ціні праці 12 годин,
то його денна плата зростає до 3 шилінґів 7\sfrac{1}{5} пенсів без зміни
ціни праці. Той самий результат міг би постати, коли б замість

30 Вартість самих грошей тут завжди припускається за сталу.

31 «Ціна праці є сума, виплачена за певну кількість праці» - («The
price of labour is the sum paid for a given quantity of labour»). (. Sir Edward
West: «Price of Corn and Wages of Labour», London 1826, p. 67).
Вест є автор анонімного твору, епохального в історії політичної економії:
«Essay on the Application of Capital to Land. By a Fellow of the
University College of Oxford», London 1815.
