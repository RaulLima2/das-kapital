Економію на засобах продукції треба взагалі розглядати з
подвійного погляду. По-перше, оскільки вона здешевлює товари
й тим знижує вартість робочої сили. По-друге, оскільки вона змінює
відношення додаткової вартості до цілого авансованого капіталу,
тобто до суми вартостей його сталих та змінних складових
частин. Цей останній пункт ми розглядаємо лише у першому
відділі третьої книги цього твору, куди задля зв’язку ми відносимо
й дещо інше, що можна б уже тут розглянути. Хід аналізу
вимагає саме так розбити тему, і це, зрештою, відповідає духові
капіталістичної продукції. Саме тому, що тут умови праці протистоять
робітникові як самостійні, то й економія на них виступає
як осібна операція, яка аніскільки не обходить робітника і
тому відокремлена від методів, що підвищують його особисту продуктивність.

Та форма праці, коли багато осіб планомірно й спільно, один
поруч одного, працюють у тому самому процесі продукції або в різних,
але зв’язаних між собою процесах продукції, називається
кооперацією. 10

Подібно до того, як сила нападу ескадрону кавалерії або сила
опору полку піхоти посутньо відмінні від суми сил нападу й
опору кожного поодинокого кавалериста й піхотинця, так і механічна
сума сил поодиноких робітників відмінна від тієї суспільної
сили, яка розвивається, коли багато рук одночасно спільно працює
над тією самою неподільною операцією, приміром, коли треба
підняти тягар, покрутити корбою, забрати із шляху якусь перешкоду. 11
За таких обставин цього результату комбінованої
праці або зовсім не можна було б досягти поодинокими силами,
або, якщо й можна було б, то тільки протягом довшого часу або
лише в карликовому маштабі. Тут справа не тільки в підвищенні
індивідуальної продуктивної сили через кооперацію, але й у
створенні продуктивної сили, яка сама по собі мусить бути масовою
силою. *11а

10 «Concours de forces» («сполучення сил»). (Destutt de Tracy: «Traité
de la Volonté et de ses effets», Paris 1826, p. 78).

11 «Є безліч таких простих операцій, що їх не можна поділити на
частки, і все ж не можна виконати їх без кооперації багатьох рук. Так,
наприклад, підняти велику колоду на віз... коротко, всяка праця, що
її не можна виконати без співробітництва багатьох рук, які одночасно
помагають одна одній у тому самому неподільному процесі праці». («There
are numerous operations of so simple a kind as not to admit a division into
parts, which cannot be performed without the cooperation of many pairs
of hands. For instance the lifting og a large tree on a wain... every thing
in short, which cannot be done unless a great many pairs ef hands help each
other in the same undivided employment, and at the same time»). (E. G.
Wakefield: «А View of the Art of Colonization», London 1849, p. 168).

11а «Якщо одна людина зовсім не може, а десятеро людей можуть
тільки з найбільшою напругою всіх своїх сил підняти тягар вагою в тонну,
то сто людей посягнуть цього, працюючи кожен лише одним пальцем» («As

* У французькому виданні це речення подано так: «Справа не тільки
в підвищенні індивідуальних продуктивних сил, але й у створенні за допомогою
кооперації нової сили, яка функціонує лише як колективна
сила» Ред.
