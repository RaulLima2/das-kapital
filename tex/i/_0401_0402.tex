\index{i}{0401}  %% посилання на сторінку оригінального видання
На фабриках і мануфактурах, не підданих ще фабричному
законові, панує якнайстрашніша надмірна праця: періодично,
підчас так званого сезону, і спорадично — в наслідок раптових замовлень. У зовнішніх відділах
фабрики, мануфактури й
крамниць, тобто у сфері домашньої праці, і без того цілком нерегулярної, цілком залежної щодо
сировинного матеріялу й замовлень
від примх капіталіста, який тут не є обмежений міркуваннями
про експлуатацію будівель, машин тощо і не ризикує нічим, хіба
тільки шкурою самих робітників, — у цій сфері таким чином систематично вирощують промислову резервну
армію, якою завжди
можна порядкувати і яку протягом однієї частини року винищують якнайжорстокішим примусом до праці, а
протягом другої
частини зводять до стану голоти через брак праці. «Підприємці, — каже «Children’s Employment
Commission», — визискують
звичну нереґулярність домашньої праці, щоб у ті часи, коли
потрібна нагальна праця, здовжувати її до 11, 12, 2 години
вночі, а в дійсності, як каже поширена там фраза, здовжувати
її до «всякої години» та ще до того в помешканнях, «де самого
лише смороду досить, щоб вас збити з ніг (the stench is enough
to knock you down). Ви дійдете, може, до дверей і відчините їх,
але вас пройме жах, і ви не підете далі».\footnote{
«Children’s Employment Commission, 4th Report», p. XXXV,
n. 235 і 237.
} «Чудаки оті наші
підприємці, — каже один із вислуханих свідків, швець, — вони
гадають, нібито дитині не шкодить, коли протягом однієї половини року її на смерть виснажують
працею, а протягом другої
половини майже примушують тинятися без роботи».\footnote{
Там же, стор. 127, n. 56.
}

Як про технічні перешкоди, так само й про ці так звані «промислові звички» («usages which have grown
with the growth of
trade»\footnote*{
— звички, що розвивалися разом з розвитком промислів. \emph{Ред.}
}) заінтересовані капіталісти твердили й твердять, ніби
вони є «природні межі» продукції — улюблений лемент бавовняних лордів тих часів, коли вперше почав
їм загрожувати фабричний закон. Хоч їхня промисловість більше ніж усяка інша
спирається на світовий ринок, а тому й на судноходство, однак
досвід спростував їхню брехню. Від того часу англійські фабричні
інспектори дивляться на кожну таку «промислову перешкоду»
як на просту викрутку.\footnote{
«Щодо втрат, що їх зазнає торговля в наслідок невчасного виконання замовлень на товари, які
доводиться перевозити морем, то я пригадую
собі, що це був улюблений арґумент панів фабрикантів у 1832 та 1833 рр.
Все, що можна тепер сказати з цього приводу, не має такої ваги, як тоді,
коли пара не скоротила ще наполовину всіх дистанцій і не утворила нових
умов для перевозу. І в той час цей арґумент не витримував критики, а
тепер він зовсім не витримує її». («With respect to the loss of trade by
the non-completion of shipping orders in time, I remember that this was
the pet argument of the factory masters in 1832 and 1833. Nothing that
can be advanced now on this subject could have the force that it had then,
before steam had halved all distances and established new regulations
for transit. It quite failed at that time of proof when put to the test, and
again it will certainly fail should it have to be tried»). («Reports of Insp.
of Fact, for 31 st October 1862», p. 54, 55).
} Ґрунтовні й сумлінні досліди «Children’s
\index{i}{0402}  %% посилання на сторінку оригінального видання
Employment Commission» справді доводять, що в деяких
галузях промисловости регулювання робочого дня тільки рівномірніше розподілило б на цілий рік ту
масу праці, якої вживається вже в цих галузях;\footnote{
«Children’s Employment Commission, 4 th Report», p. XVIII,
n. 118.
} що це реґулювання є перше раціональне
обмеження людовбивчих, нісенітних, пустотливих примх моди,\footnote{
Джон Беллерс уже 1699 р. зауважує: «Непостійність мод збільшує
число бідних. Вона спричиняє два великі лиха: 1) робітники бідують
узимку від недостачі праці, бо торговці матеріями і ткачі-хазяїни не
наважуються витрачати свої капітали, щоб дати робітникам роботу, поки
настане весна й виявиться, яка буде мода; 2) по весні бракує робітників,
і ткачі-хазяїни мусять брати багато учнів, щоб задовольнити потреби
королівства протягом кварталу або півроку; це відбирає руки від рільництва, позбавляє село робочих
сил і здебільша переповнює міста жебраками; а ті, що соромляться жебракувати, зимою помирають з
голоду».
(«The uncertainty of fashions does increase necessitous Poor. It has two
great mischiefs in it: 1st) The journeymen are miserable in winter for want
of work, the mercers and master-weavers not daring to layout their stocks
to kepp the journeymen imployed before the spring comes and they know
what the fashion will then be; 2dly) In the spring the journeymen are
not sufficient, but the master-weavers must draw in many prentices, that
they may supply the trade of the kingdom in a quarter or half a year, which
robs the plow of hands, drains the country of labourers, and in a great part
stocks the city with beggars, and starves some in winter that are ashamed
to beg»). («Essays about the Poor, Manufactures etc.», p. 9).
}
які сами по собі не відповідають системі великої промисловости;
що розвиток океанського судноходства й засобів комунікації
взагалі усунув власне технічну підставу сезонової праці;\footnote{
«Children’s Employment Commission. 5 th Report», p. 171, n. 31.
}
що всі інші обставини, яких нібито не можна контролювати,
усувається поширенням будівель, додатковими машинами, збільшенням числа одночасно занятих робітників\footnote{
Так, наприклад, у свідченнях бретфордських торговців-експортерів читаємо: «За цих обставин
ясно, що немає потреби примушувати
дітей працювати по крамницях довше, ніж від 8 години ранку до 7—7\sfrac{1}{2} годин вечора. Це — справа лише
додаткових видатків і додаткових рук.
[Дітям не треба було б працювати до пізньої ночі, коли б деякі підприємці не були такі жадні на
бариші: додаткова машина коштує лише
16 або 18 фунтів стерлінґів]\dots{} Всі труднощі випливають із недостатнього
устаткування та недостатнього помешкання». (Там же, стор. 171, n. 31,
36 і 38).
} і зворотним впливом, що його всі ці зміни справляють на систему великої торгівлі.\footnote{
Там же. Один лондонський фабрикант, який, зрештою, розглядає
примусове реґулювання робочого дня як засіб захисту робітників проти
фабрикантів і самих фабрикантів проти великої торговлі, свідчить: «Нашу
промисловість притискують торговці-експортери, які, приміром, відсилаючи товари вітрильним кораблем,
хочуть до початку певного сезону
бути вже на місці і разом з тим сховати собі до кишені ріжницю між
фрахтом вітрильного корабля й пароплава; абож із двох пароплавів вони
хочуть вибрати собі той, що відпливає раніш, щоб з’явитися на закордонному ринку попереду своїх
конкурентів».
}  Проте
капітал, як він це не раз заявляв устами своїх

\parbreak{}  %% абзац продовжується на наступній сторінці
