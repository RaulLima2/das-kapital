Дуже великої ваги набрало спорадичне вживання машин у
XVII віці, бо це давало великим математикам тих часів практичні
пункти опори та стимул до створення сучасної механіки.

Специфічною машиною мануфактурного періоду лишається
сам колективний робітник, скомбінований із багатьох частинних
робітників. Різні операції, які продуцент якогось товару виконує
навпереміну і які переплітаються в цілості його робочого процесу,
вимагають від нього напруження його різних здібностей.
В одній він мусить розвивати більше сили, у другій — більше
вправности, у третій — більше уважности й т. ін., а той самий
індивід посідає ці властивості не в однаковій мірі. Після відокремлення,
усамостійнення та ізоляції різних операцій робітники
поділяються, класифікуються та групуються за тими властивостями,
які в кожного з них переважають. Якщо їхні природні
особливості становлять основу, на якій виростає поділ праці,
то мануфактура, скоро її вже заведено, розвиває робочі сили, які
з природи придатні лише до однобічних окремих функцій. Тепер
колективний робітник має всі продуктивні властивості в однаково
високій мірі віртуозности та витрачає їх якнайощадніше, вживаючи
всіх своїх органів, індивідуалізованих в осібних робітниках
або робітничих групах, виключно на їхні специфічні функції.45
Однобічність і навіть недосконалість частинного робітника стають
досконалістю його як члена колективного робітника.46 Звичка
до якоїсь однобічної функції перетворює його на орган цієї
функції, який діє з природною певністю, тимчасом як зв’язок
сукупного механізму примушує його функціонувати з реґулярністю
якоїсь частини машини.47

А що різні функції колективного робітника можуть бути
простіші або складніші, нижчі або вищі, то й органи його, індивідуальні
робочі сили, потребують освіти в неоднаковому ступені,
і тому мають неоднакову вартість. Отже, мануфактура розвиває
ієрархію робочих сил, якій відповідає скаля заробітних плат.
Коли, з одного боку, індивідуального робітника пристосовується

45 «Поділяючи вироблення продукту на декілька різних операцій,
з яких кожна вимагає різних ступенів вправности й сили, власник мануфактури
може придбати собі саме таку кількість сили та вправности,
що точно відповідала б кожній операції. Навпаки, коли б цілий продукт
мав виробляти один робітник, то той самий індивід мусив би мати досить
вправности для якнайделікатніших і досить сили для якнафтяжчих
операцій». (Ch. Babbage; «On the Economy of Machinery», London 1832,
ch. XVIII).

46    Наприклад, однобічний розвиток мускулів, покривлення костей
і т. ін.

47    Пан В. Маршал, головний управитель однієї мануфактури скла,
дуже влучно відповів на запити слідчого комісара, як підтримувати
працьовитість серед робітників-підлітків: «Вони ніяк не можуть нехтувати
свою працю; почавши її, вони мусять і далі працювати; вони
є не що інше, як частини машини» («They cannot well neglect their work;
when they once begin, they must go on; they are just the same as parts of
a machine»). («Children’s Employment Commission. Fourth Report 1865»,
p. 247).
