телеграфію, фотографію, пароплавство та залізниці. Перепис
1861 р. (для Англії та Велзу) подає для газової промисловости
(газові заводи, продукція механічних апаратів, аґенти газових
компаній і т. ін.) 15.211 осіб, для телеграфії — 2.399, для фотографії
— 2.366, для служби на пароплавах — 3.570 та для залізниць —
70.599, куди входять приблизно 28.000 більш-менш постійно
занятих «ненавчених» землекопів і цілий адміністративний і комерційний
персонал. Отже, загальне число індивідів у цих п’ятьох
нових галузях промисловости — 94.145.

Нарешті, надзвичайно підвищена продуктивна сила в сферах
великої промисловости, супроводжувана, як ми це спостерігаємо,
інтенсивним та екстенсивним збільшенням визиску робочої
сили по всіх інших сферах продукції, дає змогу непродуктивно
вживати щораз більшу й більшу частину робітничої кляси й таким
чином репродукувати щораз більшими масами стародавніх домашніх
рабів під назвою «кляси слуг», як от слуг, покоївок, льокаїв
і т. ін. За переписом 1861 р. вся людність Англії й Велзу налічувала
20.066.244 особи, з того 9.776.259 чоловіків та 10.289.965 жінок.
Якщо від цього відлічити всіх тих, що застарі або замолоді
для праці, всіх «непродуктивних» жінок, підлітків і дітей,
далі «ідеологічні» професії, як от урядовців, попів, юристів,
військових тощо, потім усіх тих, що їхнє виключне заняття є
споживання чужої праці в формі земельної ренти, процентів і
т. ін., насамкінець, павперів, волоцюг, злочинців і т. ін., то залишається
приблизно 8 мільйонів осіб обох статей та найрізнішого
віку, залічуючи сюди й усіх капіталістів, що так або інакше
функціонують у продукції, торговлі, фінансах тощо. З цих 8 мільйонів
припадає на:

Рільничих робітників (залічуючи сюди пастухів та
наймитів і наймичок, що живуть у фармерів).................1.908.261 осіб
Всіх, що працюють на бавовняних, вовняних, напіввовняних,
лляних, конопляних, шовкових
і джутових фабриках, на механічних в’язальнях
панчіх та коло фабрикації мережива............................... 642.607 223»
Всіх, що працюють по копальнях та руднях............................565.835»
Всіх, що працюють па металюрґійних заводах
(домни, вальцювальні тощо) та металевих мануфактурах усякого
роду............................................................................ 396.998 224»
Клясу слуг.................................................................... 1.208.648 225»

223 З того чоловіків, старших від 13 років, лише 177.596.

224 З того жінок 30.501.

225 З того чоловіків 137.447. З цього числа в 1.208.648 виключено ввесь
персонал, що служить не у приватних осіб.

Додаток до другого видання. Від 1861 р. до 1870 р. число слуг-чоловіків
майже подвоїлося. Воно зросло до 267.671. 1847 р. сторожів дичини
було 2.694 (в аристократичних мисливських парках), а 1869 р. — 4.291.
— Молодих дівчат, що служать у лондонських дрібних буржуа, народньою
мовою називають «little slaveys» — маленькі рабині.*

* Тут у власному Марксовому примірнику 1 німецького видання є
така цитата з «Evening Star» від 11 вересня 1868 р.: «Як виснажують над-
