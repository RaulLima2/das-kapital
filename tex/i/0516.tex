техніки, — то продуктивніші і, розглядувані щодо розміру
їхньої дієздатносте, дешевші машини, знаряддя, апарати і т. ін.
стають на місце старих. Старий капітал репродукується в продуктивнішій
формі, не кажучи вже про невпинні зміни деталів
у наявних засобах праці. Друга частина сталого капіталу, сировинний
матеріял і допоміжний матеріял, репродукується невпинно
протягом року, а якщо він походить із рільництва, то здебільшого
щорічно. Отже, кожне заведення ліпшої методи й т. ін.
діє тут майже одночасно і на додатковий капітал і на той капітал,
що вже функціонує. Кожний проґрес на полі хемії не тільки урізноманітнює
число корисних речовин і застосування вже відомих,
поширюючи тим самим разом із зростанням капіталу сферу його
прикладання. Одночасно він навчає повертати екскременти процесу
продукції і споживання назад у кругобіг процесу репродукції,
отже, він створює нові матеріяли для капіталу без попередньої
витрати капіталу. Подібно до того, як через просте підвищення
напруження робочої сили збільшується експлуатація
природних багатств, так само наука й техніка створює для капіталу,
що функціонує, незалежну від даної його величини силу
поширюватись. Вони одночасно впливають і на ту частину первісного
капіталу, що увійшла в стадію свого відновлення. У своїй
новій формі капітал захоплює для себе задурно той суспільний
проґрес, що відбувся за спиною його старої форми. Правда, цей
розвиток продуктивної сили супроводиться частинним зневартненням
капіталів, що функціонують. Оскільки це зневартнення
дає себе гостро відчувати через конкуренцію, головний тягар
його спадає на робітника: капіталіст намагається поповнити свої
втрати через підвищену експлуатацію робітника.

Праця переносить на продукт вартість спожитих нею засобів
продукції. З другого боку, вартість і маса засобів продукції,
що їх пускає в рух дана кількість праці, зростає пропорційно
до того, як праця стає продуктивнішою. Отже, хоч та сама кількість
праці й додає до своїх продуктів завжди лише ту саму суму
нової вартосте, а все ж із зростом продуктивности праці зростає
та стара капітальна вартість, яку вона одночасно переносить
на продукти.

Наприклад, коли англійський і китайський прядун працюватимуть
однакове число годин і з однаковою інтенсивністю, то за
тиждень вони обидва вироблять рівні вартості. Не зважаючи
на цю рівність, існує величезна ріжниця між вартістю тижневого
продукту англійця, що працює за допомогою потужного
автомата, і китайця, що має лише самопряд. За той самий час,
за який китаєць випрядає один фунт бавовни, англієць випрядає
декілька сот фунтів. У кілька сот разів більша сума старих вартостей
збільшує вартість продукту англійця, в якому ті старі
вартості зберігаються в новій корисній формі, і таким чином
можуть знову функціонувати як капітал. «1782 р., — повідомляє
Ф. Енґельс, — увесь збір вовни за три попередні роки (в Англії)
лежав ще необроблений через брак робітників і мусив би ще ле-
