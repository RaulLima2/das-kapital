\index{i}{0167}  %% посилання на сторінку оригінального видання 
вартість пряжі в 30 шилінґів = 24 шилінґи + 3v шилінґи + З m шилінґи. А що вся ця вартість
виражається в цілому продукті, в 20 фунтів пряжі, то й різні елементи вартости безперечно можуть
бути
виражені у відносних частинах продукту.

Коли вартість в 30 шилінґів існує в 20 фунтах пряжі, то 8/10 цієї вартости, або її стала частина в
24 шилінґи, існує у 8/10 продукту, або в 16 фунтах пряжі. З них 13 1/3 фунтів репрезентують вартість
сировинного матеріялу, перепряденої бавовни на 20 шилінґів, а 2 2/3 фунта репрезентують вартість
зужиткованих допоміжних матеріялів і засобів праці, веретен тощо на 4 шилінґи.

Отже, 13 1/3 фунтів пряжі репрезентують усю бавовну, що з неї випрядено цілий продукт, 20 фунтів
пряжі, тобто репрезентують сировинний матеріял цілого продукту і нічого більше. У них міститься,
щоправда, лише 13 1/3 фунтів пряжі вартістю в 13 1/3 шилінґів, але додана до них вартість у 6 2/3
шилінґів становить еквівалент бавовни, що з неї випрядено ті 6 2/3 фунтів пряжі, які ще залишились.
Це так, немов би з цих 6 2/3 фунтів пряжі вискублено бавовну і всю бавовну цілого продукту впхнуто в
13 1/3 фунтів пряжі. Але зате ці 13 1/3 фунтів пряжі не містять тепер у собі жодного атому ні
вартости спожитих допоміжних матеріялів і засобів праці, ні нової вартости, створеної в процесі
прядіння.

Так само й дальші 2 2/3 фунтів пряжі, що в них міститься решта сталого капіталу (= 4 шилінґи),
репрезентують не що інше, як
тільки вартість допоміжних матеріялів і засобів праці, спожитих на цілий продукт — 20 фунтів пряжі.

Отже, 8/10 продукту, або 16 фунтів пряжі, розглядувані в їхній тілесній формі, як споживна вартість,
як пряжа, хоч і
є так само витвір праці прядіння, як і решта частин продукту, однак, у даному зв’язку вони не
містять у собі жодної праці прядіння, жодної праці, увібраної підчас самого процесу прядіння. Це
так, начебто вони без прядіння перетворились на пряжу і начебто їхня форма пряжі є чиста омана.
Дійсно, коли капіталіст продасть їх за 24 шилінґи і за ці гроші знов купить свої засоби продукції,
то виявляється, що 16 фунтів пряжі — це лише бавовна, веретена, вугілля і т. д., які тільки змінили
свій одяг.

Навпаки, ті 2/10 продукту, що лишаються ще, або 4 фунти пряжі, репрезентують тепер не що інше, як
тільки нову вартість
у 6 шилінґів, спродуковану протягом дванадцятигодинного процесу прядіння. Щождо вартости
зужиткованого сировинного матеріялу й засобів праці, які були вміщені в них, то та вартість була вже
вительбушена з них і увійшла у склад перших 16 фунтів пряжі. Втілену в 20 фунтів пряжі працю
прядіння сконцентровано
в 2/10 продукту. Це так, немов би прядун випряв 4 фунти пряжі з повітря або з такої бавовни й з
такими веретенами, які
існують із самої природи без допомоги людської праці та не додають до продукту жодної вартости.
