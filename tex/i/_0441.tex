\parcont{}  %% абзац починається на попередній сторінці
\index{i}{0441}  %% посилання на сторінку оригінального видання
Однак навіть і тоді пересічні ступені інтенсивности праці лишалися
б у різних націй різні і тому модифікували б застосування
закону вартости до робочих днів різних націй. Інтенсивніший
робочий день однієї якоїсь нації виражається в більшій сумі
грошей, ніж менш інтенсивний робочий день іншої нації.\footnote{
«За інших однакових умов англійський фабрикант може за даний
час пустити в рух значно більшу суму праці, ніж чужоземний фабрикант,
так що це урівноважує ріжницю між 60-годинним тижнем у нас
і 72-або 80-годинним по інших країнах» («All things being equal,
the English manufacturer can turn out a considerably larger amount of
work in a given time than a foreign manufacturer, so much as to counterbalance
the difference of the working days, between 60 hours a week here
and 72 or 80 elsewhere»). («Reports of Insp. of Fact, for 31 st October 1885»,
p. 65). Більше законодавче скорочення робочого дня в континентальних
фабриках було б найпевнішим засобом зменшити цю ріжницю між робочою
годиною на континенті і в Англії.
}

\subsection{Продуктивна сила та інтенсивність праці сталі, робочий
день змінюється}

Робочий день може змінятися в двох напрямах: він може
скорочуватись або здовжуватись. [При наших нових даних ми
маємо такі закони:

Робочий день втілюється прямо пропорціонально своїй довжині
в більшій або меншій вартості; остання, отже, є величина
змінна, а не стала.

Всяка зміна у відношенні величин додаткової вартости й вартости
робочої сили випливає із зміни абсолютної величини додаткової
праці, отже, і додаткової вартости.

Абсолютна вартість робочої сили може змінятися лише в
наслідок зворотної дії, що її справляє здовження додаткової
праці на ступінь зужитковування робочої сили. Отже, всяка зміна
її абсолютної вартости є наслідок, але ніколи не причина зміни
величини додаткової вартости.

У цьому розділі, як і в наступних, ми припускатимемо завжди,
що робочий день, який первісно становить 12 годин, — шість
годин доконечної праці і шість годин додаткової праці — продукує
вартість у 6 шилінґів, що з неї одна половина дістається
робітникові, а друга — капіталістові.

Розгляньмо спочатку скорочення робочого дня, приміром,
а 12 годин до 10. Тепер він дає лише вартість у 5 шилінґів. Додаткова
праця спадає з 6 годин до 4, додаткова вартість — з 3 шилінґів
до 2 шилінґів. Це зменшення її абсолютної величини
призводить до зменшення її відносної величини. Вона відносилась
до вартости робочої сили як 3: 3, а тепер — лише як 2: 3. Зате
вартість робочої сили, хоч вона й лишається та сама, зростає
в своїй відносній величині; вона відноситься тепер до додаткової
вартости як 3: 2, а не як 3: 3].\footnote*{
Заведене у прямі дужки ми беремо з французького видання («Le
Capital etc.», v. І, ch. XVII, p. 226--227). \emph{Ред.}
}
\parbreak{}  %% абзац продовжується на наступній сторінці
