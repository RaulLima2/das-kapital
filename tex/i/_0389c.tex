\parcont{}  %% абзац починається на попередній сторінці
\index{i}{0389}  %% посилання на сторінку оригінального видання
цього свого права на здоров’я, вони не можуть одержати жодної
ефективної допомоги і від спеціяльних урядовців санітарної
поліції\dots{} Життя багатьох тисяч робітників та робітниць тепер
без усякої потреби нівечиться та скорочується через безмежні
фізичні страждання, що їх породжує сама лише їхня праця».\footnote{
«Public Health. 6 th Report», London 1864, p. 31.
}
Для ілюстрації впливу робітних приміщень на стан здоров’я
робітників д-р Сімон подає таку таблицю смертности:

\noindent\begin{tabularx}{\textwidth}{Xccccc@{}l@{}}
  \toprule 
	\multirowcell{2}[2.5ex]{Порівняння галузей \\ промисловости щодо \\ їхнього впливу \\ на здоров'я} &
  	\multirowcell{2}[2.5ex]{Число осіб усякого \\ віку, занятих \\ у відповідних галузях  \\ промисловости} &
	\multicolumn{4}{c}{\makecell{
		Норма смертности на 100.000 \\
        чоловіка у відповідних \\
        галузях промисловости \\
        за віком
	}} \\
  \cmidrule(rl){3-5}
  	& & 25\textendash{}35 р. & 35\textendash{}45 р. & 45\textendash{}55 р. \\

  \midrule
	\makehangcell{Рільництво в Англії та Велзі\dotfill{}} & 958.265 & 743 & 805 & 1.145 \\

	Лондонські кравці\dotfill{} &\
	$\left\{
	\begin{array}{l}
	  \text{22.301 чоловіків}\\ 
	  \text{12.379 жінок}
	\end{array} 
	\right\}$& 958 & 1.262 & 2.093 \\
                                              

  	Лондонські друкарі\dotfill{} &\phantom{0}13.803 & 894 & 1.747 
  	& 2.367\footnote{
Там же, стор. 30. Д-р Сімон зауважує, що смертність лондонських
кравців і друкарів на 25--30 році життя в дійсності куди більша,
бо їхні лондонські підприємці одержують із села велике число молодих
людей до 30 років як «учнів» та «improvers» (які хочуть досконало вивчити
своє ремество). В перепису вони фігурують як лондонці, і через те
збільшують число людей, на яке обчислюється смертність у Лондоні,
хоч вони дають відносно менше число смертних випадків, ніж лондонці.
Велика частина з них, особливо у випадках тяжких недуг, повертається
на село. (Там же).
}

\end{tabularx}

  	
             
         
              


\subsubsection{Сучасна домашня праця}

Тепер я звертаюсь до так званої домашньої праці. Щоб скласти
собі уявлення про цю сферу капіталістичної експлуатації, що
являє собою задній плян великої промисловости, та про її страхіття,
треба розглянути, приміром, цілком ідилічне на позір виробництво
цвяхів, що його провадять по деяких глухих селах Англії.\footnote{
Тут мова про ковані цвяхи, відмінно від різаних, що їх фабрикують
машиновим способом. Див. «Children’s Employment Commission
З rd Report», p. XI, p. XIX, n. 125--130, p. 53, n. 11, p. 114, n. 487,
p. 137, n. 674.
}
Тут досить кількох прикладів із галузей продукції мережив та
плетіння з соломи, які зовсім ще не вживають машин або конкурують
з машиновим та мануфактурним виробництвом.

Із тих 150.000 осіб, що працюють у виробництві мережива в
Англії, приблизно 10.000 підлягають фабричному законові 1861 р.
Величезна більшість решти 140.000 осіб складається з жінок,
підлітків та дітей обох статей, хоч чоловіча стать репрезентована
дуже слабо. Стан здоров’я цього «дешевого» матеріялу для експлуатації
видно з такого зіставлення д-ра Трюймена, лікаря
\parbreak{}  %% абзац продовжується на наступній сторінці
