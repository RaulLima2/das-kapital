в певній корисній формі, а він не може додавати її в корисній
формі, не роблячи продукти засобами продукції нового продукту
й не переносячи тим самим їхньої вартости на новий продукт.
Отже, зберігати вартість, додаючи вартість, — це є дар природи,
діющої робочої сили, живої праці, дар природи, який робітникові
нічого не коштує, а капіталістові дає багато, — зберігає наявну
капітальну вартість.22а  Доки справи йдуть добре, капіталіст
надто заглиблений у фабрикацію додаткової вартости, щоб помічати
цей дармовий дар праці. Ґвалтовні перерви процесу праці,
кризи, роблять його для капіталіста дошкульно помітним.23

Що взагалі споживається в засобах продукції, так це лише
їхня споживна вартість, споживанням якої праця творить продукти.
їхню вартість у дійсності не споживається,24 отже, і
не може вона бути репродукована. Вона зберігається, але не
тому, що з нею самою відбувається якась операція в процесі
праці, а тому, що та споживна вартість, у формі якої вона первісно
існувала, хоч і зникає, але лише в іншій споживній вартості.*
Тому вартість засобів продукції з’являється знов у вартості
продукту, але її, строго кажучи, не репродукується. Що проду-

22а «З усіх знарядь фармерської продукції людська праця... є таке
знаряддя, від якого фармер більше за все може сподіватися, що його капітал
назад повернеться. Інші два: робоча худоба і... вози, плуги, лопати
тощо, нічого не варті без певної частини першого» («Of all the instruments
of the farmer’s trade, the labour of man... is that on which he is most to
rely for the re-payment of his capital. The other two — the working stock
of the cattle, and the... carts, ploughs, spades, and so forth — without a given
portion of the first, are nothing at all»). (Edmund Burke: «Thoughts and
Details on Scarcity, originally presented to the Right Hon. W. Pitt in the
Month of November 1795», ed. London 1800, p. 10).

23    У «Times’i» з 26 листопада 1862 p. один фабрикант, що на його
прядільні працює 800 робітників і щотижня споживається пересічно 150 пак
східньо-індійської або щось із 130 пак американської бавовни, нарікає
перед публікою на ті витрати, які викликає щорічне припинення його
фабрики. Він цінує їх на 6.000 фунтів стерлінґів. Серед цих втратних
витрат (Unkosten) є багато таких, які нас тут не цікавлять, як от земельна
рента, податки, страхові премії, плата найнятим на рік робітникам, управителеві,
бухгалтерові, інженерові й т. ін. Але далі він залічує сюди
150 фунтів стерлінґів за вугілля, щоб час від часу опалювати фабрику та
вряди-годи пускати в рух парову машину, крім того, заробітну плату
робітникам, що своєю тимчасовою працею підтримують «напоготові»
цілий механізм. Нарешті, 1.200 фунтів стерлінґів на те, що машини псуються,
бо «силу погоди й руйнаційних природних впливів не спиняється
тим, що парові машини перестали рухатися» («the weather and the natural
principle of decay do not suspend their operations because the steamengine
ceases to revolve»). Він виразно зауважує, що ця сума в 1.200 фунтів
стерлінґів така незначна через те, що машини вже значно зужитковані.

24 «Продуктивне споживання: коли споживання товару становить
частину процесу продукції... В цих випадках немає споживання вартости»
(«Productive Consumption: where the consumption of a commodity is a
part of the process of production... In these instances there is no consumption
of value»). (S. P. Newman: «Elements of Political Economy» Andover
and New-Jork 1835, p. 296).

* Кінець цього речення у французькому виданні подано так: «... але
зникає лише на те, щоб набрати нової корисної форми». Ред.
