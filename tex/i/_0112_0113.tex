\parcont{}  %% абзац починається на попередній сторінці
\index{i}{0112}  %% посилання на сторінку оригінального видання
тут можна сказати: «де є рівність, там немає баришу».\footnote{
«Dove è egualità, non è lucro». (\emph{Galiani}: «Délla Moneta», y Custodi:
Parte Moderna, t. IV, p. 244).
} Правда,
товари можуть продаватись за цінами, що відхиляються від їхніх
вартостей, але це відхилення являє собою порушення закону
обміну товарів.\footnote{
«Обмін стає некорисним для однієї з сторін, коли якась побічна
обставина зменшує або збільшує ціну: тоді рівність порушується, але це
порушення постає з цієї причини, а не через обмін» («L’échange devient
désavantageux pour l’une des parties, lorsque quelque chose étrangère vient
diminuer ou exagérer le prix: alors l’égalité est blessée, mais la lésion procede
de cette cause et non de l’échange»). (\emph{Le Trosne}: «De l’Intérêt Social»,
Physiocrates, éd. Daire, Paris 1846, p. 904).
} У своїй чистій формі обмін товарів є обмін
еквівалентів, отже, він не є засіб збагачуватися через збільшення
вартости.\footnote{
«Обмін з природи своєї є договір рівности, коли за одну вартість
дають таку саму вартість, Отже, це не засіб для збагачення, бо тут
дають рівно стільки, скільки одержують». («L’échange est de sa nature
un contract d’égalité qui se fait de valeur pour valeur égale. Il n’est
donc pas un moyen de s’enrichir, puisque l’on donne autant que l’on
reçoit»). (\emph{Le Trosne}: «De l’Intérêt Social», Physiocrates, éd. Daire, Paris
1846, p. 903).
}

Тому за спробами виставити товарову циркуляцію як джерело
додаткової вартости криється здебільшого quid pro quo, переплутування
споживної й мінової вартости. Приміром, у Кондільяка:
«Це неправда, що при обміні товарів рівна вартість обмінюється
на рівну вартість. Навпаки, кожний з обох контраґентів завжди
віддає меншу вартість за більшу. Коли б дійсно люди завжди
обмінювали рівні вартості, то не було б жодного виграшу ні для
одного з контраґентів. Однак, обидва вони виграють абож повинні
вигравати. Але чому? Тому, що вартість речей полягає
лише в їхньому відношенні до наших потреб. Що для одного є
більше, для іншого є менше, і навпаки\dots{} Не можна припустити,
щоб ми подавали на продаж речі, доконечні для нашого споживання\dots{}
Ми хочемо віддати некорисну для нас річ, щоб одержати
доконечну нам річ; ми хочемо віддати менше за більше\dots{} Природно
було дійти до висновку, що при обміні рівну вартість віддається
за рівну вартість кожного разу, коли кожна вимінювана
річ за вартістю дорівнювала тій самій кількості грошей\dots{} Але
треба взяти до уваги ще й інший погляд: постає питання, чи не
обмінюємо ми обидва якийсь надлишок на дещо, доконечне для
нас».\footnote{
\emph{Condillac}: «Le commerce et le Gouvernement (1776). Edit. Daire
et Molinari y «Mélanges d’Economie Politique», Paris 1817, p. 267.
} Ми бачимо, як Кондільяк не лише сплутує споживну
вартість з міновою вартістю, але й справді по-дитячому підсуває
суспільству з розвинутою товаровою продукцією такий стан речей,
за якого продуцент сам продукує свої засоби існування й подає
в циркуляцію тільки надлишок, надмір, що лишається після
задоволення власних потреб.\footnote{
Тому ле Трон цілком слушно відповідає своєму приятелеві Кондільякові:
«У сформованому суспільстві не існує жодного надміру»
(«Dans la société formée il n’y a pas de surabondant en aucun genre»).
Одночасно він дратує його, зауважуючи, що «коли обидва обмінювачі
одержують однаково більше за однаково менше, кожний з них дістає
рівно стільки, скільки і другий». Через те, що Кондільяк ще нічогісінько
не тямить у справі природи мінової вартости, то він є дуже добрий порадник
для пана проф. Вільгельма Рошера в його власних дитячих поглядах.
Див. його: «Die Grundlagen der Nationalökonomie», 3 Auflage,
1858.
} А проте арґумент Кондільяка
\index{i}{0113}  %% посилання на сторінку оригінального видання
часто повторюють сучасні економісти, а саме тоді, коли вони
намагаються довести, що розвинена форма обміну товарів, торговля,
є джерело додаткової вартости. «Торговля, — кажуть вони,
наприклад, — додає продуктам вартости, бо ті самі продукти
мають більше вартости в руках споживачів, ніж у руках продуцентів;
отже, її треба розглядати, точно кажучи (strictly), як
акт продукції».\footnote{
\emph{S. P. Newman}: «Elements of Political Economy», Andover and
New-York. 1835, p. 175.
} Але за товари не платять двічі, раз за їхню
споживну вартість, другий раз за їхню вартість. І коли споживна
вартість товару корисніша покупцеві, ніж продавцеві, то його
грошова форма корисніша продавцеві, ніж покупцеві. А то ж
хіба б він продавав товар? І таким чином можна б так само сказати,
що покупець, точно кажучи (strictly), виконує «акт продукції»,
перетворюючи, наприклад, панчохи купця на гроші.

Коли обмінюється товари або товари й гроші рівної мінової
вартости, отже, еквіваленти, то ясна річ, що ніхто не витягає
з циркуляції більше вартости, ніж подає до неї. Тоді не відбувається
жодного творення додаткової вартости. У своїй чистій
формі процес циркуляції товарів зумовлює обмін еквівалентів.
Однак у дійсності процеси не відбуваються в чистій формі. Припустімо,
отже, що обмінюється не-еквіваленти.

У всякому разі на товаровому ринку протистоять один одному
лише посідачі товарів, і влада, яку вони використовують один
проти одного, — це лише влада їхніх товарів. Речова відмінність
товарів є речовий мотив обміну. Ця відмінність робить посідачів
товарів одного від одного залежними, бо жоден із них не має в
своїх руках предмету своєї власної потреби, а кожний із них
має в себе предмет потреби іншого. Поза цією речовою відмінністю
споживних вартостей товарів існує ще лише одна ріжниця
між ними, ріжниця між їхньою натуральною формою і їхньою
перетвореною формою, між товаром і грішми. І таким чином
посідачі товарів відрізняються між собою лише як продавці —
посідачі товару, і як покупці — посідачі грошей.

Припустімо тепер, що продавець через якийсь нез’ясовний
привілей може продавати товари понад їхню вартість, за 110,
коли вони варті 100, отже, з номінальним додатком 10\% до ціни.
Отже, продавець одержує додаткову вартість у 10. Але після
того, як він був продавцем, він стає покупцем. Третій посідач
товарів зустрічається тепер з ним як продавець і собі користується
привілеєм продавати товар на 10\% дорожче. Таким чином
наш посідач товарів як продавець виграв 10, щоб як покупець
\parbreak{}  %% абзац продовжується на наступній сторінці
