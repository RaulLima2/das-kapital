\parcont{}  %% абзац починається на попередній сторінці
\index{i}{0404}  %% посилання на сторінку оригінального видання
Збільшення фабрик у цій країні збільшить, певна річ, число цих
жахливих результатів. Я переконаний, що коли б держава відповідно наглядала за
scutching mills, то можна уникнути більшої частини цих жертов у здоров’ї й
житті людей».\footnote{
Там же, cтop. XV. n. 72 і далі.
} Що могло б краще характеризувати капіталістичний спосіб продукції, ніж
ця доконечність примусовим державним законом накинути йому
найпростіші санітарні й захисні заходи? «Фабричний закон 1864 р.
побілив і почистив по ганчарнях понад 200 майстерень після
того, як вони двадцять років, а то й зовсім здержувалися від
усякої такої операції (оце та «поздержливість» капіталу!). У цих
підприємствах працює 27.800 робітників, що досі підчас надмірної денної, а
часто й нічної праці, вдихали отруйне повітря,
яке цю взагалі порівняно нешкідливу галузь праці насичує
хоробами та смертю. Закон значно збільшив кількість засобів
вентиляції».\footnote{
«Reports of Insp. of Fact, for 31st October 1865», p. 127.
}

Одночасно ця частина фабричного закону яскраво показує, як
капіталістичний спосіб продукції з самої суті своєї виключає
поза якоюсь певного межею всяке раціональне поліпшення. Ми
вже не раз зазначали, що англійські лікарі в один голос вважають
500 кубічних футів повітря на особу за ледве достатній мінімум
при безупинній праці. Гаразд! Якщо фабричний закон усіма
своїми примусовими заходами посередньо прискорює перетворення дрібніших
майстерень на фабрики, а тому посередньо
зазіхає на право власности дрібніших капіталістів і забезпечує
монополію великим, то закон про забезпечення по майстернях
для кожного робітника потрібної кількости повітря одним махом
безпосередньо експропріював би тисячі дрібних капіталістів!
Це підтяло б самий корінь капіталістичного способу продукції,
тобто самозростання капіталу — малого чи великого — за допомогою «вільної»
купівлі й споживання робочої сили. Тим то
перед цими 500 кубічними футами повітря фабричному законодавству перехопило
дух. Санітарні установи, комісії для досліду промисловости, фабричні інспектори
знову й знов повторюють про доконечність 500
кубічних футів і неможливість накинути їх капіталові.
Таким чином вони фактично проголошують
сухоти й інші недуги легенів у робітництва за умову існування
капіталу.\footnote{
Досвідом установлено, що середній здоровий індивід за кожного
вдихання середньої інтенсивносте споживає приблизно 25 кубічних
цалів  повітря, і що він робить приблизно 20 таких вдихань на хвилину. Згідно з
цим, один індивід за 24 години мав би споживати повітря приблизно 720.000
кубічних цалів, або 416 кубічних футів. Але
відомо, що вдихане повітря вже не може більше служити для того самого
процесу, поки воно не очиститься у великій майстерні природи. Згідно
з експериментами Валентіна і Бруннера, здорова людина, здається, видихує
приблизно 1.300 кубічних цалів вуглекислоти за годину; звідси
випливає, що легені за 24 години викидають приблизно 8 унцій твердого
вугілля! «Для кожної людини потрібно щонайменше 800 кубічних
футів». (Гекслі).
}
