Ми знаємо тепер спосіб визначення вартости, яку посідач
грошей виплачує посідачеві цього своєрідного товару, робочої
сили. Споживна вартість, яку посідач грошей отримує собі в
обмін, виявляється лише в дійсному вживанні, у процесі споживання
робочої сили. Всі потрібні для цього процесу речі, як от
сировинний матеріял і т. ін., посідач грошей купує на товаровому
ринку й платить за них повну ціну. Процес споживання
робочої сили є разом з тим процес продукції товару й додаткової
вартости. Споживання робочої сили, як і споживання всякого
іншого товару, відбувається поза межами ринку або сфери циркуляції.
Тим то ми, разом із посідачем грошей і посідачем робочої
сили, полишаємо цю шумну сферу, де все відбувається на поверхні
та перед очима всіх і кожного, щоб піти за ними обома в таємні
місця продукції, що на їх порозі написано: «No admittance except
on business.\footnote*{
Увіходити дозволяється лише y справах. Ред.
} Тут виявиться не лише те, як капітал продукує,
алеж і те, як його самого продукують, як продукують капітал.
Таємниця продукції додаткової вартости (Plusmacherei) мусить,
нарешті, відкритися.

Сфера циркуляції, або обміну товарів, у межах якої рухається
купівля та продаж робочої сили, була дійсно за правдивий
едем природжених прав людини. Тут панує лише воля, рівність,
власність і Бентам. Воля! Бо покупець і продавець товару, приміром,
робочої сили, керуються лише своєю свободною волею.
Вони складають умови як вільні, юридично рівноправні особи.
Контракт є кінцевий результат, у якому їхні волі знаходять собі
спільний юридичний вираз. Рівність! Бо вони відносяться один
до одного лише як посідачі товарів і обмінюють еквівалент на
еквівалент. Власність! Бо кожний порядкує лише своїм. Бентам!
Бо кожний з обох дбає лише про себе самого. Однісінька
сила, що ставить їх у зв’язок і взаємне відношення, — це сила
їхньої власної користи, їхньої власної вигоди, їхніх приватних
інтересів. Але саме через те, що кожний дбає лише про себе й
ніхто не дбає про іншого, всі вони в наслідок наперед установле-

англійських посідачів кам’яновугільних копалень, що платять робітникові
лише наприкінці місяця, а в проміжний час робітник дістає від капіталіста
аванси, часто товарами, які він мусить оплачувати понад їхню
ринкову ціну (Trucksystem).\footnote*{
Початок цієї фрази у французькому виданні зредаґовапо так: «Як
приклад експлуатації робітника, що постає з того кредиту, який він дає
капіталістові, можна розглядати...». — «Comme exemple de l’exploitation,
qui resulte pour l’ouvrier du credit qu’il donne au capitaliste...». Peд.
} «Серед власників кам’яновугільних копалень
стало звичаєм платити робітникам раз на місяць і давати їм позики
наприкінці кожного проміжного тижня, позику видається в крамниці
(а саме в tommy shop, тобто в крамниці, що належить самому хазяїнові);
робітники отримують гроші в одному куті й зараз же віддають їх у другому»
(«It is a common practice with the coal masters to pay once a month,
and advance cash to their workmen at the end of each intermediate week.
The cash is given in the shop; the men take it on one side and lay it out on
the other»). («Children’s Employment Commission, 3 rd Report», London
1864, p. 38, n. 192).