\parcont{}  %% абзац починається на попередній сторінці
\index{i}{0626}  %% посилання на сторінку оригінального видання
коло них заняття й утримання».\footnote{
Reverend Addington: «An Inquiry into the Reasons for and against
enclosing open-fields», London 1772, p. 37--43 і далі.
} Під приводом обгородження
сусідні лендлорди присвоювали собі не тільки землі, що лежали
перелогом, але часто-густо ще й землі, оброблювані або самою
громадою, або орендарями, які наймали їх у громади за певну
плату. «Я кажу тут про обгороджування відкритих полів і земель,
уже оброблених. Навіть ті письменники, що боронять
inclosures, визнають, що воно збільшує монополію великих
фарм, підвищує ціни засобів існування і продукує збезлюднення\dots{}
Навіть обгороджування гулящих земель, як це тепер практикують,
відбирає у бідняків частину їхніх засобів існування і надзвичайно
збільшує фарми, що і без того вже занадто великі».\footnote{
Dr. R. Price: «Observations on Reversionary Payments 6 th ed.
By W. Morgan», London 1805, vol. II, p. 155. Прочитайте Форстера,
Едінґтона, Кента, Прайса й Джемса Андерсона й порівняйте з ними жалюгідне
сикофантське базікання Мак Куллоха у його каталогу «The
Literature of Political Economy», London 1845.
}
«Коли земля, — каже д-р Прайс, — попадає в руки небагатьох
великих фармерів, то дрібні фармери [що про них він раніш говорив
як про «масу дрібних власників і фармерів, що утримують
себе та свої родини продуктами оброблюваної ними землі, вівцями,
птицею, свиньми і т. ін., яких вони пасуть на громадській
землі, так що їм мало доводиться купувати засобів існування»]
перетворюються на людей, що примушені заробляти на життя
працею на інших і купувати на ринку все їм потрібне\dots{} Може,
тепер більше працюють, бо більше силують до праці\dots{} Міста й
мануфактури зростатимуть, бо до них зганяють більше людей,
які шукають роботи. Такий є неминучий вплив концентрації
фарм, і так вона фактично впливала в цьому королівстві протягом
багатьох років».\footnote{
Там же, стор. 147.
} Загальний вплив inclosures він резюмує
так: «У цілому становище нижчих народніх кляс майже з
кожного боку погіршало, дрібні землевласники й фармери позведені
до рівня поденників і наймитів, і в той самий час куди тяжче
стало за таких умов заробляти на життя».\footnote{
Там же, стор. 159. Пригадаймо собі стародавній Рим. «Багаті
захопили у свої руки більшу частину неподілених земель. Тогочасні
обставини викликали в них упевненість, що земель у них уже не відберуть,
і тому вони поскуповували сумежні дільниці бідняків, почасти за
згодою останніх, почасти відбираючи їх силою, так що замість поодиноких
нив вони почали обробляти великі маєтки. При цьому вони для
рільництва й скотарства вживали рабів, бо вільних людей у них забрали б
від праці до військової служби. Володіння рабами давало їм ше йту
велику користь, що раби, звільнені від військової служби, могли спокійно
розмножуватись і мали багато дітей. Таким чином вельможні постягали
до своїх рук усі багатства, і ціла країна аж кишіла рабами. Навпаки,
італійців ставало раз-у-раз менше, їх нищили злидні, податки й військова
служба. Коли ж наставали мирні часи, то вони були засуджені
} Справді, узурпація
громадської землі й революція в рільництві, що супроводила
цю узурпацію, мали такий гострий вплив на рільничих робітників,
що, як каже сам Ідн, між 1765 і 1780 рр. їхня заробітна
\index{i}{0627}  %% посилання на сторінку оригінального видання
плата почала падати нижче мінімуму, і її доводилося доповнювати
офіціальною допомогою для бідних. Їхньої заробітної плати,
каже він, «ледве вистачало, щоб задовольнити доконечні життєві
потреби».

Послухаймо на хвилину ще одного оборонця inclosures і противника
д-ра Прайса. «Хибний висновок, ніби відбувається
збезлюднення, якщо не видно більше людей, що марнують свою
працю у відкритому полі\dots{} Коли після перетворення дрібних
селян на людей, що мусять працювати на інших, пускається у
рух більше праці, то це ж користь, якої нація [до неї «перетворені»,
ясна річ, не належать] мусить собі бажати\dots{} Продукту
буде більше, коли їхню комбіновану працю вживатимуть на
одній фармі; таким чином утворюється додатковий продукт для
мануфактур, і через це число мануфактур, цих розсипищ золота
нашої нації, зростає пропорційно до кількости продукованого
збіжжя».\footnote{
«An Inquiry into the Connection between the present Price of
Provisons etc.», p. 124, 129. Подібне, але з протилежною тенденцією, ми
читаємо в іншого автора: «Робітників проганяють з їхніх котеджів і
примушують іти до міст шукати там роботи — але це дає більше додаткового
продукту, і таким чином капітал зростає» («Working men are driven
from their cottages, and forced into the towns to seek for employment; —
but then a larger surplus is obtained, and thus Capital is augmented»).
(«The Perils of the Nation», 2 nd ed. London 1843, p. XIV).
}

Зразок стоїчного душевного спокою, з яким політико-економ
розглядає якнайнахабніші порушення «святого права власносте»
й найгрубіші насильства над особою, скоро тільки вони
потрібні для створення основи капіталістичного способу продукції,
показує нам, між іншим, сер Ф. М. Ідн, отой до того
ще на торійський штаб забарвлений «філантроп». Ціла низка грабунків,
жорстокостей і народніх злигоднів, що супроводили насильну
експропріяцію народу, починаючи від останньої третини
XV аж до кінця XVIII століття, приводить його тільки до
такого «утішного» кінцевого висновку: «Треба було встановити
належну (due) пропорцію між орною землею та пасовиськами.
Ще протягом цілого XIV й більшої частини XV століття
один акр пасовиська припадав на 2, 3 й навіть 4 акри орного
поля. В середині XVI століття ця пропорція змінилася так, що
2 акри пасовиська припадили на 2 акри орного поля, а пізніше
2 акри пасовиська на 1 акр орного поля, аж поки, нарешті, встановилась
пропорція — 3 акри пасовиська на 1 акр орного поля».

У XIX столітті зникла, звичайно, навіть і згадка про зв’язок
поміж рільником і громадською власністю. Не кажучи вже
про пізніші часи, чи одержало колибудь селянство хоч шеляг
відшкодування за ті 3.511.770 акрів громадської землі, які

на повне безділля, бо землею володіли багачі, які на її оброблення вживали
рабів замість вільних людей». (Арріаn: «Römische Bürgerkriege»,
1,7). Це місце стосується до часу перед законом Ліцінія. Військова служба,
що так дуже прискорила руйнування римських плебеїв, була також
головним засобом, що ним Карл Великий дуже прискорив перетворення
вільних селян на февдально залежних і кріпаків.
\parbreak{}  %% абзац продовжується на наступній сторінці
