tected».\footnote{
Галун, дрібно розтертий, або змішаний із сіллю, є нормальний
предмет торговлі, що має характеристичну назву «baker’s stuff» («порошок
пекарів»).
} Як наслідок того відкриття оголошено закона з 6 серпня
I860 р.: «for preveting the adulteration of articles of food
and drink»,\footnote*{
Про заходи, запобіжні проти фальсифікації харчів і напоїв. \emph{Ред.}
** — заробити копійку чесною працею. \emph{Ред.}
*** Друзі торговлі. \emph{Ред.}
**** — наочно. \emph{Ред.}
} закона без ніякої сили, бо ж він, розуміється, з
надзвичайною делікатністю ставиться до кожного фритредера,
що має намір через купівлю та продаж фальшованих товарів «to
turn an honest penny**».\footnote{
Сажа є, як відомо, дуже енерґійна форма вуглеця і служить за
добриво, яке капіталістичні коминярі продають англійським фармерам.
1862 р. в одному судовому процесі брітанському «Juryman» (судді) довелося
вирішувати, чи така сажа, що до неї без відома покупця домішано
90\% пилу й піску, є «правдива» сажа в «комерційному» значенні слова,
чи вона є «фальсифікована» сажа в «законному» значенні. «Amis du
commerce» *** вирішили, що це «правдива» комерційна сажа, і відкинули
скаргу орендаря, який, крім того, мусив ще заплатити судові витрати.
} Сам комітет сформулював більш-менш
наївно своє переконання, що свобода торговлі означає по суті
торговлю фальшованим, або, як це дотепно кажуть англійці, «софістикованими
продуктами». Дійсно, такого роду «софістика»
вміє краще за Протагора робити з білого чорне і з чорного біле,
і краще за елеатів демонструвати ad oculos**** тільки подобу всього
реального.\footnote{
Французький хемік Шевальє у розвідці про «софістикацію» товарів
налічує для багатьох із 600 продуктів, що їх він розглядає, до 10, 20,
30 різних способів фалшування. Він додає, що не знає всіх способів, і
згадує не про всі, які він знає. Для цукру він наводить 6 способів фальсифікації,
для маслинової олії — 9, для масла — 10, для соли — 12, молока
— 19, хліба — 20, для горілки — 23, для борошна — 24, для шоколяди
— 28, для вина — 30, для кави — 32 і т. ін. Навіть милосердого господа-бога
не минула ця доля. Див. Ronard de Card: «De la falsification
des substances sacramentelles», Paris 1856.
}

В кожному разі комітет звернув увагу громадянства на його
«хліб насущний», а тим самим і на пекарство. Одночасно на публічних
мітинґах і в петиціях, звернених до парляменту, залунав
крик лондонських пекарських підмайстрів про надмірну працю
й т. ін. Крик зробився такий настирливий, що пана Г. С. Тріменгіра,
члена не раз уже згадуваної комісії 1863 р.,\footnote{
«Report etc. relating to the Grievances complained of by the
Journeymen Bakers etc.», London 1862 і «Second Report etc.», London
1863.
} призначено
було на королівського слідчого комісара. Його звіт разом із виказами
свідків зворушив публіку, не серце її, а її шлунок. Правда,
начитаний у біблії англієць знав, що людину, якщо вона не є з
ласки божої ні капіталіст, ні лендлорд і не має синекури, призначено
на те, щоб у поті чола свого їсти хліб свій, та він не знав
того, що він сам мусить день-у-день з’їдати в своєму хлібі
певну кількість людського поту, змішаного з виділенням гнійних
ґуль, павутинням, трупами тарганів, з гнилими німецькими дріж-