Її визначають дві обставини: посплітування відносин між кредиторами й винуватцями, так що A,
одержавши гроші від свого винуватця B, виплачує їх своєму кредиторові C і т. ін., — і протяг часу
між різними термінами платежів. Ланцюг послідовних платежів або перших метаморфоз, що відбуваються
пізніш як доповнення, істотно відрізняється від раніш розглянутого посплітування рядів метаморфоз. В
обігу засобів циркуляції зв’язок між продавцями й покупцями не просто тільки виражається. Самий цей
зв’язок постає лише у грошовому обігу й разом із ним. Навпаки, рух засобів платежу виражає певний
суспільний зв’язок, що існував уже перед цим рухом.

Якщо продажі йдуть одночасно й один поруч одного, то це обмежує заміну маси монет швидкістю обігу.
Навпаки, ці обставини становлять нову підойму в справі економії засобів платежу. З концентрацією
платежів у тому самому місці природно розвиваються осібні установи й методи для їхнього
вирівнювання, як от, приміром, virements * у середньовічному Ліоні. Треба лише порівняти між собою
боргові зобов’язання А до В, В до С, С до А і т. д., щоб до певної межі скасувати їх навзаєм як
позитивні й неґативні величини. Таким чином доведеться виплатити тільки балянс боргів. Що масовіша
концентрація платежів, то відносно менший балянс, отже, то менша й маса засобів платежу, що
циркулюють.

Функція грошей як засобу платежу містить у собі безпосередню суперечність. Оскільки платежі
вирівнюються, гроші функціонують лише ідеально як рахункові гроші, або міра вартостей. Оскільки ж
доводиться робити справжні платежі, гроші
виступають не як засіб циркуляції, не як лише минуща й упосереднювальна форма обміну речовин, а як
індивідуальне втілення суспільної праці, самостійне буття мінової вартости, абсолютний товар. Ця
суперечність вибухає у той момент промислових і торговельних криз, який зветься грошовою кризою. 99
Вона постає лише там, де ланцюг послідовних платежів і штучна система вирівнювання їх є цілком
розвинені. За загальних порушень цього механізму, хоч звідки вони виникли б, гроші раптом і
безпосередньо перетворюються з лише ідеальної форми рахункових грошей на дзвінку монету. Їх уже не
може заступити звичайний товар. Споживна вартість товару стає нічого не вартою, і вартість його
зникає перед його власного формою вартости. Ще напередодні буржуа, сп’янілий від розцвіту й
пишаючись своєю вченістю, вважав гроші за порожню мрію. — Лише товар є гроші! — кричав він. — Лише
гроші є товар! — лунає тепер по цілому

99 Цю грошову кризу, що в тексті її визначається як осібну фазу кожної загальної промислової й
торговельної кризи, слід відрізняти від спеціяльного ґатунку кризи, що теж зветься грошовою кризою,
але може виникати самостійно, так, що вона справляє на промисловість і торговлю лише відбитий вплив.
Це такі кризи, що їхній центр руху є капітал-гроші, отже, безпосередня сфера їхня є банк, біржа,
фінанси. (Примітка Маркса до третього видання).

* — установи, де провадять взаємне вирівнювання платежів. Ред.
