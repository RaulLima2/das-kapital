\parcont{}  %% абзац починається на попередній сторінці
\index{i}{0158}  %% посилання на сторінку оригінального видання
зміниться робочий час, суспільно-потрібний на продукцію товару, — а та сама кількість бавовни,
приміром, за несприятливого врожаю репрезентує більшу кількість праці, ніж за сприятливого,
— то це справляє відбитий вплив на старий товар, що його завжди рахується лише за окремий екземпляр
свого ґатунку,\footnote{«Всі продукти того самого роду становлять, власне, одну масу, ціну якої визначається загалом
незалежно від особливих обставин». («Toutes les productions d’un même gepre ne forment proprement
qu’une masse, dont le prix se détermine en général et sans égard aux circonstances particulières»),
(\emph{Le Trosne}: «De l’Intérêt Social». Physiocrates, éd. Daire,
Paris 1846, p. 893).
}
вартість якого завжди вимірюється суспільно-доконечною працею, тобто працею, завжди доконечною за
наявних суспільних умов.

Вартість засобів праці, що вже функціонують у процесі продукції,
машин і т. ін., отже, і та частина вартости, яку вони віддають продуктові, може змінюватися так
само, як вартість сировинного матеріялу. Коли, приміром, у наслідок якогось нового винаходу машина
того самого роду репродукується із зменшеною витратою праці, то старі машини більш або менш
зневартнюються
й тому переносять на продукт порівняно менше вартости. Але й тут зміна вартости постає поза тим
процесом продукції,
в якому машина функціонує як засіб продукції. В цьому процесі вона ніколи не віддає більше вартости,
ніж та, яку вона має незалежно від цього процесу.

Так само, як зміна вартости засобів продукції, хоч вона й справляє на них відбитий вплив навіть
після вступу їх до процесу продукції, не змінює їхнього характеру як сталого капіталу, так і зміна в
пропорції між сталим і змінним капіталом не порушує їхньої функціональної ріжниці. Приміром,
технічні умови процесу праці можуть змінитися так, що там, де раніш 10 робітників із 10 знаряддями
невеликої вартости обробляли порівняно малу кількість сировинного матеріялу, тепер 1 робітник
дорогою машиною обробляє в сто раз більшу кількість сировинного матеріялу. В цьому випадку сталий
капітал, тобто маса вартости застосованих
засобів продукції, дуже зросла б, а змінна частина капіталу, частина капіталу, авансована на робочу
силу, дуже спала б. Однак ця зміна переінакшує лише відношення між величиною сталого й змінного капіталу, або пропорцію,
в якій увесь капітал розпадається на сталі та змінні складові частини, але вона не
порушує самої ріжниці між сталим і змінним капіталом.

\section{Норма додаткової вартости}
\subsection{Ступінь експлуатації робочої сили}
Додаткова вартість, яку авансований капітал С створив у процесі продукції, або зростання авансованої
капітальної вартости С виступає перед нами насамперед як надлишок вартости
продукту понад суму вартости елементів його продукції.
