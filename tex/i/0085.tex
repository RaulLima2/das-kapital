bile циркуляції. Гроші імобілізуються, або перетворюються, як
каже Буаґільбер, із meuble на immeuble,* з монети на гроші,
скоро тільки переривається ряд метаморфоз, скоро тільки продаж
не доповнюється наступною купівлею.

Уже з самого початку розвитку товарової циркуляції розвивається
конечність і жагуче бажання затримати в себе продукт
першої метаморфози, перетворену форму товару, або її золоту
лялечку.86 Товар продається не на те, щоб купити інший товар,
але щоб замінити товарову форму на грошову. З простого посереднього
моменту обміну речовин ця зміна форми стає за самоціль.
Преображеній формі (entäusserte Gestalt) товару заважається
функціонувати як його абсолютно відчужуваній формі,
або як його лише минущій грошовій формі. Разом із цим гроші
кам’яніють у скарб, а продавець товарів стає збирачем скарбів.

Саме на початках товарової циркуляції на гроші перетворюється
лише зайвина споживної вартости. Таким чином золото
й срібло сами собою стають суспільними виразами надлишку або
багатства. Ця наївна форма скарботворення увіковічнюється в
тих народів, де традиційному й розрахованому на власне споживання
способові продукції відповідає стало замкнене коло потреб,
приміром, в азійців, особливо в індусів. Вандерлінт, який уявляє
собі, що товарові ціни визначається масою золота й срібла, що
є в країні, запитує себе, чому індійські товари такі дешеві.
Відповідь: тому, що індуси закопують гроші в землю. Від 1602
до 1734 рр., зауважує він, вони закопали 150 мільйонів фунтів
стерлінґів срібла, яке первісно привезено було з Америки до
Европи.87 Від 1856 до 1866 рр. отже, за одне десятиліття, Англія
вивезла до Індії й Китаю (експортований до Китаю металь іде в
значній частині знову таки до Індії) 120 мільйонів фунтів стерлінґів
срібла, що раніш було виміняне на австралійське золото.

За розвиненішої товарової продукції кожний товаропродуцент
мусить забезпечити собі nexus rerum, «суспільну рухому

1 як засіб циркуляції. Першу функцію він виконує як ідеальні гроші;
в другій функції він може бути заміщений символами. Але є такі функції,
що в них він мусить з’являтись у своїй металевій тілесності, як реальний
еквівалент товарів, або як грошовий товар. Але є ще інша функція, яку він
може виконувати або власною особою або через заступників, але в якій
він завжди протистоїть споживним товарам як однісіньке адекватне
втілення їхньої вартости. В усіх цих випадках ми кажемо, що він функціонує
як гроші у власному значенні в протилежність його функціям міри
вартостей і монети». («Le Capital etc.», v. I, ch. III , p. 53) Ред.

86 «Багатство на гроші є не що інше, як... багатство на продукти,
перетворені на гроші» («Une richesse en argent n’est que... richesse en
productions, converties en argent»). (Mercier de lа Rivière: «L’Ordre naturel
et essentiel des sociétés politiques». Physiocrates, ed. Daire, I. Partie, Paris.
1848, p. 573). «Вартість продуктів... зміняє тут лише форму» («Une valeur
en productions... n’a fait que changer de forme»). (Там же, стор. 486).

87 «Саме завдяки цьому звичаєві вони тримають свої товари й мануфактурні
вироби на такому низькому рівні цін» («Tis by this practice
they keep all their goods and manufactures at such low rates»). (Vanderlint:
«Money answers all Things», London 1734, p. 95, 96).

* — з рухомих на нерухомі. Ред.
