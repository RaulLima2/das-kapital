2. Частинний робітник та його знаряддя

Якщо тепер підійти ближче до деталів, то насамперед ясно,
що робітник, який цілий свій вік виконує одну й ту саму просту
операцію, перетворює ціле своє тіло на її автоматично однобічний
орган і тому витрачає на це менше часу, ніж ремісник, що
виконує навпереміну цілий ряд операцій. Але комбінований
колективний робітник, що становить живий механізм мануфактури,
складається тільки з таких однобічних частинних робітників.
Тим то тут порівняно з самостійним ремеством за коротший
час продукується більше, тобто продуктивна сила праці підвищується.\footnote{
«Що більше працю якоїсь складної мануфактурної галузі розчленовано
й поділено між різними ремісниками, то ліпше й скорше цю працю виконується,
то менше витрачається часу та праці» («The more any manufacture
of much variety shall be distributed and assigned to different atrists
the same must needs be better done and with greater expedition, with less
loss of lime and labour»), («The Advantages of the East-India Trade»,
London 1720, p. 71).
}
Крім цього, якщо частинна праця всамостійнюється
у виключну функцію однієї особи, то вдосконалюється і її метода.
Постійне повторювання тієї самої обмеженої роботи й концентрація
уваги на цій обмеженій роботі навчають через досвід досягати
наміченого корисного ефекту з якнайменшою витратою сили.
А що різні ґенерації робітників завжди одночасно живуть разом
і працюють разом у тих самих мануфактурах, то набуті таким
чином способи технічної вмілости швидко вкорінюються, нагромаджуються
та передаються від однієї ґенерації до другої.\footnote{
«Легка праця є успадкована вправність» («Easy labour is transmitted
skill»). (Th. Hodgskin: «Popular Political Economy», London 1827, p. 48).
}

Мануфактура, репродукуючи всередині майстерні й систематично
розвиваючи до крайніх меж те розмежування реместв,
яке вона знайшла у містах середньовіччя, тим самим фактично
продукує віртуозність частинних робітників. З другого боку, її
перетворення частинної праці на життєву професію людини
відповідає прагненню попередніх суспільств робити ремество спадковим,
надавати йому закам’янілої форми каст або, — якщо
певні історичні умови створювали змінливість індивідів, яка
суперечила кастовому ладу, — закостенілої форми цехів. Касти
й цехи виникають із того самого природного закону, що реґулює
поділ рослин і тварин на роди й підроди, з тією лише відміною,
що на певному ступені розвитку спадковість каст або винятковість
цехів декретується як суспільний закон.\footnote{
«Вправності також... дійшли в Єгипті належного ступеня досконалости.
Бо лише в цій одній країні ремісники ні в якому разі не сміють
встрявати до занять інших громадянських кляс, а повинні працювати
лише в тій професії, яка за законом спадково належала їхньому родові...
В інших народів ми знаходимо, що ремісники поділяють свою увагу на
надто багато об'єктів... То заходяться вони коло обробітку землі, то беруться
до торговельних справ, то займаються одночасно двома або трьома
ремествами. У вільних державах вони часто бігають на народні збори...
Навпаки, в Єгипті кожного ремісника, що встрявав до державних справ
або береться одночасно до кількох реместв, піддають тяжким карам.
} «Мусліну з Даккі