ються 8 шилінґів як грошовий вираз вартости, що в ній утілюється робочий день. Цей грошовий вираз
вищий, ніж грошовий
вираз пересічної суспільної праці того самого роду, що з неї
12 годин праці виражаються лише в 6 шилінґах. Праця виняткової продуктивної сили функціонує як
складна, тобто піднесена
до ступеня (potenzierte) праця, або творить протягом однакового
часу вищі вартості, аніж пересічна суспільна праця того самого
роду. Але наш капіталіст, як і раніш, платить за денну вартість
робочої сили лише 5 шилінґів. Отже, робітник замість колишніх
10 потребує тепер лише 7\sfrac{1}{2} годин для репродукції цієї вартости.
Тому його додаткова праця зростає на 2\sfrac{1}{2} години, а спродукована
ним додаткова вартість — з 1 до 3 шилінґів. Отже, капіталіст,
що вживає поліпшеного способу продукції, присвоює собі як додаткову працю більшу частину робочого
дня, ніж усі інші капіталісти
з тієї самої галузі промисловости. Він робить індивідуально те,
що при продукції відносної додаткової вартости капітал робить
взагалі і в цілому. Але, з другого боку, та понаддодаткова вартість
зникає, скоро тільки новий спосіб продукції узагальнюється, і
разом з тим зникає ріжниця між індивідуальною вартістю дешевше
продукованих товарів та їхньою суспільною вартістю. Той самий
закон визначення вартости робочим часом, який капіталістові
з новою методою продукції дається взнаки в такій формі, що він
мусить продавати свій товар нижче від його суспільної вартости, — цей самий закон як примусовий
закон конкуренції спонукає його конкурентів вводити в себе новий спосіб продукції.4
Отже, цілий цей процес зачіпає загальну норму додаткової вартости кінець-кінцем лише тоді, коли
підвищення продуктивної
сили праці охоплює такі галузі продукції, отже, понижує ціни
на такі товари, що увіходять у коло доконечних засобів існування,
і тому становлять елементи вартости робочої сили.

Вартість товарів стоїть у зворотному відношенні до продуктивної сили праці, так само й вартість
робочої сили, бо вона
визначається вартістю товарів. Навпаки, відносна додаткова
вартість стоїть у простому відношенні до продуктивної сили праці.
Вона зростає із зростом і падає із спадом продуктивної сили праці.
Пересічний суспільний дванадцятигодинний робочий день, —

4 «Якщо мій сусіда, продукуючи багато з невеликою кількістю праці,
може продавати дешево, то і я мушу старатися продавати так само дешево,
як він. Так, усякий винахід, усяка метода чи машина — все, що дає змогу
обходитися з меншою кількістю рук і тим то продукувати дешевше, доконечно викликає в інших змагання
або застосувати той самий винахід,
ту саму методу чи машину, або ж винайти щось їм подібне, так, щоб
усі були в однакових умовах, і ніхто не міг би продавати дешевше за свого
сусіди». («If my neighbour by doing much with little labour, can sell
cheap, I must contrive to sell as cheap as he. So that every art, trade, or
engine, doing work with labour of fewer hands, and consequently cheaper,
begets in others a kind of necessity and emulation, either of using the same
art, trade, or engine or of inventing something like it, that every man
may be upon the square, that no man may be able to undersell his neighbour»). («The Advantages of
the East-India Trade to England», London
1720, p. 67).
