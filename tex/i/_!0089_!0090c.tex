\index{i}{*0089}  %% посилання на сторінку оригінального видання
\section*{Передмова й післямова до французького
видання}

\begin{flushright}
Лондон, 18 березня 1872
\end{flushright}

\begin{center}
Громадянинові Морісові Ля-Шатр.
\end{center}

\hspace{\parindent}\hspace{\parindent}
Шановний громадянине!

Я вітаю вашу думку опублікувати переклад «Капіталу» періодичними
випусками. В такій формі праця буде приступніша
робітничій клясі, а для мене це міркування важливіше, ніж
усі інші.

Це гарний бік вашої медалі, але тут є і зворотний бік: метода
аналізи, що я її вжив і що її не застосовували ще до економічних
проблем, робить досить важким читання перших розділів і можна
побоюватись, що французька публіка, завжди жадібна до висновків,
бажаючи знати зв’язок основних принципів із безпосередніми
питаннями, що її хвилюють, відстрашиться, не маючи одразу ж
змоги перейти до дальшого викладу.

Це є незручність, що проти неї я нічого не можу зробити, хіба
лише попередити й перестерегти читачів, що дбають про істину.
Для науки немає проторованих шляхів, і тільки ті мають надію
досягти до її осяйних вершин, хто не боїться потомитися, здираючись
по стрімких стежках.

\hspace{\parindent}\hspace{\parindent}
Прийміть, шановний громадянине,

\hspace{\parindent}\hspace{\parindent}
\hspace{\parindent}\hspace{\parindent}\hspace{\parindent}
запевнення в моїй відданості.

\begin{flushright}
\emph{Карл Маркс}
\end{flushright}

\section*{До читача}

Пан Ж. Руа взявся дати переклад по змозі точний і навіть
дослівний; він сумлінно виконав своє завдання. Але саме його
сумлінність примусила мене змінити редакцію, щоб зробити її
приступнішою читачеві. Ці перероблення, переведені за різних
часів, бо книгу видавалось окремими випусками, зроблено з
неоднаковою уважністю, і це мусило призвести до стилістичних
розходжень.

Взявши на себе цей перегляд, я вважав за потрібне застосувати
його також до покладеного в основу тексту ориґіналу (до другого
німецького видання), спростити деякі частини викладу, поширити
\index{i}{*0090}  %% посилання на сторінку оригінального видання
інші, дати додаткові історичні або статистичні матеріяли, додати
критичні зауваження і т. ін. Хоч які є літературні недосконалості
цього французького видання, все ж воно має наукову цінність,
незалежну від оригіналу і до нього навіть повинні вдаватись
читачі, обізнані з німецькою мовою.

Я подаю далі ті місця післямови другого німецького видання,
що стосуються до розвитку політичної економії в Німеччині і до
методи, вжитої в цій праці.\footnote*{
% TODO виправити посилання на сторінку
Див. післямову до другого німецького видання, стор. 81*--88* \emph{Ред.}
}

\begin{flushright}
\emph{Карл Маркс}
\end{flushright}

Лондон, 28 квітня 1875 р.
