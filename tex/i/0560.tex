лінґів і 1865 р. — 105.435.579 фунтів стерлінґів,\footnote{
Це — чисті доходи, отже, після того, коли вже відлічено з них
певні суми, визначені законом.
} число оподаткованих
осіб 1864 р. — 308.416 на загальну кількість людности в
23.891.009, 1865 р. — 332.431 особа на загальну кількість людности
в 24.127.003. Про розподіл цих доходів за обидва роки
довідуємося з цієї таблиці:

    Рік, що кінчається 5 квітня 1864 р. Рік, що кінчається 5 квітня 1865 р.
    Доходи від зиску (фунтів стерлінґів) Число  осіб    Доходи від зиску (фунтів стерлінґів) Число 
осіб
Загальний дохід....  95.844.222   308.416   105.435.738   332.431
З того....................  57.028.289   23.334      64.554.297     24.265
»»...................  36.415.225   3.619        42.535.576     4.021
»»...................  22.809.781   832           27.555.313     973
»»...................  8.744.762      91             11.077.238     107

1855 р. в Об’єднаному королівстві випродуковано 61.453.079
тонн вугілля вартістю в 16.113.167 фунтів стерлінґів, 1864 р. —
92.787.873 тонни вартістю в 23.197.968 фунтів стерлінґів; 1855 р. —
3.218.154 тонни чавуну вартістю в 8.045.385 фунтів стерлінґів,
1864 р. — 4.767.951 тонну вартістю в 11.919.877 фунтів стерлінґів.
1854 р. довжина залізниць, експлуатованих в Об’єднаному
королівстві, становила 8.054 милі з капіталовкладенням
у 286.068.794 фунти стерлінґів, 1864 р. довжина у милях становила
12.789, а вкладений капітал — 425.719.613 фунтів стерлінґів.
Загальний експорт і імпорт Об’єднаного королівства
становив 1854 р. 268.210.145 фунтів стерлінґів, 1865 р. —
489.923.285. Нижченаведена таблиця показує рух експорту:

Роки    Фунтів стерлінґів

1846................ 58.842.377

1849. . . .......... 63.596.052

1856................ 115.826.948

1860.................. 135.842.817

1865................ 165.862.402

1866................ 188.917.563 \footnote{
В цей момент, березень 1867 р., індійсько-китайський ринок
уже знову переповнений комісійними товарами брітанських бавовняних
фабрикантів. 1866 р. почалося зниження заробітної плати бавовняних
робітників на 5\%, в 1867 р. в наслідок подібних операцій стався страйк
20.000 робітників у Престоні. [Це був пролог кризи, що вибухла одразу
після того. — Ф. Е.].
}

Після цих небагатьох даних стає зрозумілим тріюмфальний
крик генерального реєстратора брітанського народу: «Хоч і
як швидко зростала людність, вона не встигала за проґресом промисловости
й багатства».\footnote{
«Census etc.», там же, стор. 11.
} А тепер вдаймося до безпосередніх
аґентів цієї промисловости або до продуцентів цього багатства,
до робітничої кляси. «Це одна з найсумніших характеристичних