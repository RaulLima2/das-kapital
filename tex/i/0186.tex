законом час, ви вкладали б мені до кишені річно 1.000 фунтів
стерлінґів».\footnote{
Там же, стор. 48.
} «Атоми часу є елементи баришу».\footnote{
«Moments are the element of profit». («Reports of the Insp. etc.
30 th April 1860», p. 56).
}

З цього боку немає нічого характеристичнішого, як назва
«full times»\footnote*{
— повний час. \emph{Ред.}
} для робітників, що працюють повний час, і «half
times»\footnote*{
— половина часу. \emph{Ред.}
} для дітей до тринадцятилітнього віку, яким дозволяється
працювати лише по 6 годин.\footnote{
Цей вислів має офіціальне право громадянства так на фабриці,
як і у фабричних звітах.
} Робітник є тут не що більше,
як персоніфікований робочий час. Усі індивідуальні ріжниці
сходять на ріжницю між «Vollzeitler» і «Halbzeitler».\footnote*{
— робітником повночасним і робітником півчасним. \emph{Ред.}
}

3. Галузі англійської промисловостн без законодавчих меж

експлуатації

Досі ми розглядали прагнення здовжувати робочий день,
ненажерливий вовчий голод за додатковою працею, на такому
полі, де безмірні зловживання, не перевищені і навіть — як каже
один буржуазний англійський економіст, — жорстокостями еспанців
проти червоношкурих Америки,\footnote{
«Ненажерливість власників фабрик призводить до того, що в погоні
за баришем вони допускаються таких жорстокостей, яких ледве чи
перевищили жорстокості еспанців підчас завойовування Америки в гонитві
за золотом» («The cupidity of mill-owners, whose cruelties in pursuit
of gain, have hardly been exceeded by those perpetrated by the Spaniards
on the conquest of America in the pursuit of gold»). (John Wade:
«History of the Middle and Working Classes», 3 id ed. London 1835, p-114).
Теоретична частина цієї книги, свого роду нарис політичної економії,
містить у собі дещо оригінальне для свого часу, приміром, про торговельні
кризи. Щождо історичної частини, то вона є безсоромний пляґіят із Sir
М. Eden: «History of the Poor», London 1799.
} спричинилися, нарешті,
до того, що капітал закували у ланцюги законодавчого регулювання.
А тепер киньмо оком на деякі галузі промисловости, де
висисання робочої сили або ще й тепер вільне від тих законодавчих
пут, або було таким ще зовсім недавно.

«Пан Бровтон, суддя графства, як голова мітингу, який
відбувся в нотінгемському міському будинку 14 січня 1860 р.,
заявив, що серед частини міської людности, занятої виробництвом
мережива, панують такі страшні злидні й нужда, що решта цивілізованого
світу ще таких не знає... О 2, 3, 4 годині ранку 9 —
10-літніх дітей виривають із їхніх брудних ліжок і примушують
тільки за мізерний харч працювати до 10,11,12 години вночі,
в наслідок чого нидіють їхні члени, корчиться тіло, тупіють риси
їх обличчя, і їхнє ціле людське єство дубіє в німій нерухомості, на
яку навіть глянути страшно. Це для нас не диво, що пан Малет
і інші фабриканти виступили з протестом проти всякої дискусії.
Система, як її описав панотець Монтегіо Вальпі, — це система без-