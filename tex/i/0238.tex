рих не може визволитися там, де праця чорношкурих має на собі
ганебне тавро. Але зі смерти рабства зараз же постало нове відмолоділе
життя. Першим плодом громадянської війни була аґітація
за восьмигодинний день — аґітація, що семимильними
кроками льокомотива поширилась від Атлантійського океану до
Тихого, від Нової Англії до Каліфорнії. Загальний робітничий
конґрес у Балтіморі (16 серпня 1866 р.) заявляє: «Першою й
великою потребою сучасности, щоб визволити працю цієї країни
з-під капіталістичного рабства, є видання закону, за яким вісім
годин становили б нормальний робочий день по всіх штатах Американського
Союзу. Ми вирішили напружити всі свої сили, щоб
досягти цього славетного результату».\footnote{
«Ми, робітники з Дункірку, заявляємо, що довжина робочого
дня, що її вимагають за теперішньої системи, занадто велика й не лишає
робітникові часу для відпочинку і розвитку; навпаки, вона принижує
його до стану поневолення, який небагато кращий від рабства («а condition
of servitude but little better than slavery»). Тим то ми ухвалили, що
для робочого дня досить 8 годин, і що закон мусить визнати цей час достатнім:
ми кличемо собі на допомогу пресу, цю могутню підойму... а
всіх, хто відмовить нам цієї допомоги, ми вважатимемо за ворогів реформи
праці й робітничих прав». (Постанови робітників з Дункірку, штат
Нью-Йорк, 1866 р.).
} Одночасно (початок
вересня 1866 р.) «Інтернаціональний робітничий конґрес» у Женеві
на пропозицію лондонської генеральної ради ухвалив: «Ми
заявляємо, що обмеження робочого дня є попередня умова, без
якої всі інші визвольні старання мусять розбитися... Ми пропонуємо
8 годин праці як законну межу робочого дня».

Таким чином робітничий рух, що інстинктово виріс із самих
відносин продукції по обох боках Атлантійського океану, стверджує
заяву англійського фабричного інспектора Р. Дж. Савндерса:
«Неможливо зробити дальші кроки для реформи суспільства
з якоюсь надією на успіх, якщо уперед не обмежити робочий
день і не примусити строго додержувати його меж, приписаних
законом».\footnote{
«Reports etc. for 31 st October 1848», p. 112.
}

Треба визнати, що наш робітник виходить із процесу продукції
не таким, яким увійшов до нього. На ринку він протистояв
посідачам інших товарів як посідач товару «робоча сила», тобто
як посідач товару посідачеві товару. Контракт, за яким він продавав
капіталістові свою робочу силу, так би мовити, чорним по
білому показував, що він вільно порядкує самим собою. Після того,
як торг уже закінчено, виявляється, що він не був «вільним
аґентом», що час, на який йому вільно продавати свою робочу
силу, є час, на який він примушений її продавати,\footnote{
«Ці вчинки (маневри капіталу, приміром, 1848—1850 рр.) дали,
крім того, незаперечний доказ брехливости так часто висуваного твердження,
нібито робітники не потребують охорони, і що їх треба розглядати
як аґентів, що вільно порядкують єдиною своєю власністю, тобто
працею рук своїх і потом лиця свого» («These proceedings bave afforded,
moreover, incontrovertible proof of the fallacy of the assertion so often
advanced, that operatives need no protection, but may be considered as
free agents in the disposal of the only property they possess, the labour of
} що в дійс-