як продавець своєї власної робочої сили, той стан, коли людська
праця не визволилась була ще із своєї первинної інстинктивної
форми, відходить у глибини праісторичних часів. Ми припускаємо
працю в такій формі, що в ній вона належить виключно людині.
Павук виконує операції, що подібні до операцій ткача, а бджола
будовою своїх воскових комірочок засоромить не одного архітектора-людину.
Але найгіршого архітектора від найкращої
бджоли з самого ж початку відрізняє те, що він, раніш ніж збудувати
комірочку з воску, вже збудував її у своїй голові. Наприкінці
процесу праці виходить такий результат, який на початку
цього процесу вже існував в уяві робітника, отже, вже
існував ідеально. Він не лише змінює форму того, що дала природа;
в тому, що дала природа, він здійснює одночасно й свою
свідому мету, яка як закон визначає спосіб і характер його діяння
і якій він мусить підпорядкувати свою волю. І це підпорядковання
не є відокремлений акт.\footnote*{
У французькому виданні це речення подано так: «І це підпорядковання
не є короткочасне» («Et cette subordination n’est pas momen
tanée»). Peд.
} Окрім напруження органів, що працюють,
потрібна на цілий час праці доцільна воля, яка виявляється
в увазі, і потрібна вона то більше, що менше праця захоплює
робітника своїм змістом, способом та характером її виконання,
тобто, що менше він тішиться нею як грою своїх власних
фізичних і інтелектуальних сил.

Прості моменти процесу праці є: 1) доцільна діяльність, або
сама праця, 2) предмет праці й 3) засоби праці.

Земля (економічно розуміємо під нею й воду), що первісно
постачає1 людині харчі, готові засоби існування, існує без будь-якої
допомоги людини як загальний предмет людської праці.
Всі речі, які праця відриває лише від їхнього безпосереднього
зв’язку з усесвітом, є предмети праці, дані природою, приміром,
риба, яку ловлять, відривають од її життєвої стихії, води, дерево,
що рубають у пралісі, руда, яку видобувають із її жил. Навпаки, коли
сам предмет праці є вже, так би мовити, профільтрований
попередньою працею, то ми звемо його сировинним матеріялом,
як ось видобута вже руда, що її потім промивають. Усякий сировинний
матеріял є предмет праці, та не всякий предмет праці є
сировинний матеріял. Предмет праці є сировинний матеріял тільки
тоді, коли він уже зазнав зміни, спричиненої працею.

Засіб праці є якась річ або комплекс речей, що їх робітник
ставить поміж собою й предметом праці й що служать за провід-

1 «Здається, — та воно так і справді є, — що природні продукти
землі, яких є лише обмежена кількість і які існують цілком незалежно
від людини, є дані природою цілком так само, як дають юнакові невеличку
суму грошей, щоб справити його на шлях якоїсь діяльности й дати йому
спроможність надбати собі статки» («The earth’s spontaneous productions
being in small quantity, and quite independent of man, appear,
as it were, to be furnished by nature, in the same way as a small sum is given
to a young man, in order to put him in a way of industry, and of making
his fortune»). (James Steuart: «Principles of Political Economy», Ed.
Dublin. 1770» vol. I, p. 116).