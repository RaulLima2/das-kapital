\parcont{}  %% абзац починається на попередній сторінці
\index{i}{0288}  %% посилання на сторінку оригінального видання
захоплює поряд економічної сфери усякі інші сфери суспільства
та всюди закладає основу тому розвиткові професіоналізму,
спеціялізації, тій парцеляції людини, яка примусила вже А. Ферґюсона,
вчителя Адама Сміса, вигукнути: «Ми є нація гелотів,
і немає серед нас вільних людей».\footnote{
A. Ferguson: «History of Civil Society», Edinburgh 1767, Part IV,
sect II, p. 285.
}

Однак, не зважаючи на численні аналогії та зв’язки між поділом
праці всередині суспільства й поділом праці всередині майстерні,
обидва вони не тільки щодо ступеня, але й суттю відмінні.
Безперечно, найяскравіше ця аналогія виступає там, де внутрішній
зв’язок сплітає різні галузі продукції. Скотар, приміром,
продукує шкури, гарбар вичиняє шкуру, швець із вичиненої
шкури робить чоботи. Кожен продукує тут частинний продукт,
а остання готова форма — це комбінований продукт їхніх окремих
праць. Сюди треба додати ще різноманітні галузі праці, які
постачають засоби продукції скотареві, гарбареві та шевцеві.
Можна собі уявити разом з А. Смісом, що цей суспільний поділ
праці відрізняється від мануфактурного лише суб’єктивно, а саме
лише для спостерігача, який тут, у мануфактурі, одним поглядом
просторово охоплює різноманітні частинні праці, тоді як там розкиданість
їх по великих просторах та велике число робітників,
занятих у кожній окремій галузі, затемнюють зв’язок.\footnote{
У справжніх мануфактурах, каже він, поділ праці видається
більшим, бо «робітники, що працюють у кожній з різних галузей праці,
часто можуть бути сполучені в тій самій майстерні, і таким чином всіх
їх одразу охоплює око спостерігача. Навпаки, у тих великих мануфактурах
(І), які мають своїм призначенням задовольняти широкі потреби
великої маси людности, кожна окрема галузь вживає такої великої кількости
робітників, що всіх їх неможливо сполучити в тій самій майстерні...
поділ праці тут далеко не так виразно впадає на очі» («... those employed
in every different branch of the work can often be collected into the
same workhouse, and placed at once under the view of the spectator. In
those great manufactures (I), on the contrary, which are destined to supply
the great wants of the great body of the people, every different branch of
the work employs so great a number of workmen, that it is impossible
to collect them all into the same workhouse... the divisions is not near
so obvious»). (A. Smith: «Wealth of Nations», b. I, ch. 1, p. 7, 17).
Знамените місце того самого розділу, яке починається словами:
«Погляньте на життєві умови простого ремісника або поденника в
цивілізованій країні, в країні, що процвітає, і т. д.» («Observe the accomodation
of the most common artificer or day labourer in a civilized and
thriving country etc.»), розділу, де змальовано далі, як безліч різноманітних
галузей промисловості спільними силами задовольняє потреби
простого робітника, — це місце майже слово в слово списано з приміток
Б. де Мандевіля до його «Fable of the Rees, or Private Vices, Publick Benefits».
(Перше видання без приміток 1706 р., друге з примітками 1714 р.).
} Але що
саме встановлює зв’язок між незалежними працями скотаря,
гарбаря, шевця? Те, що їхні продукти існують як товари. Навпаки,
що характеризує мануфактурний поділ праці? Те, що
частинний робітник не продукує жодного товару.\footnote{
«Тут уже немає нічого, що можна було б назвати природною винагородою
за індивідуальну працю. Кожен робітник продукує лише ча-
} Тільки спільний
\index{i}{0289}  %% посилання на сторінку оригінального видання
продукт частинних робітників перетворюється на товар.58а
Поділ праці всередині суспільства упосереднюється купівлею та
продажем продуктів різних галузей праці, зв’язок між частинними
робітниками в мануфактурі — продажем різних робочих
сил тому самому капіталістові, який вживає їх як одну комбіновану
робочу силу. Мануфактурний поділ праці має собі за передумову
концентрацію засобів продукції в руках одного капіталіста,
суспільний поділ праці — розпорошення засобів продукції
між багатьма один від одного незалежними продуцентами товарів.
Тимчасом як у мануфактурі залізний закон пропорційного числа
або пропорційности реґулює (subsumiert) розподіл певних робочих
мас між певними функціями, випадок і сваволя ведуть свою
вередливу гру в поділі товаропродуцентів і їхніх засобів продукції
поміж різними суспільними галузями праці. Правда, різні сфери
продукції постійно намагаються дійти рівноваги, бо, з одного
боку, кожний товаропродуцент мусить виробляти споживну вартість,
отже, задовольняти осібну суспільну потребу, але обсяг
цих потреб кількісно є різний і внутрішній зв’язок сполучає
різні маси потреб в одну природно вирослу систему: з другого ж
боку, закон вартости товарів визначає, скільки з усього робочого
часу, який суспільство має в своєму розпорядженні, може воно
витратити на продукцію кожного окремого роду товару. Але ця
постійна тенденція різних сфер продукції дійти рівноваги виявляється
лише як реакція проти постійного нищення (Aufhebung)
цієї рівноваги. Норма, що її а priori і пляномірно дотримуються

cтину цілого, а через те, що кожна частина сама по собі не має ніякої
вартости або корисности, то тут немає нічого такого, що робітник міг би
взяти і сказати: це мій продукт, це я залишаю собі» («There is no longer
anything which we can call the natural reward of individual labour.
Each labourer produces only some part of a whole, and each part, having no
value or utility of itself, there is nothing on which the labourer can seize,
and say: it is my product, this I will keep for myself»). («Labour defended
against the claims of Capital», London 1825, p. 25). Автором цієї
прегарної праці є цитований вище Т. Годжскін.

58a Примітка до другого видання. — Янкі практично зілюстрували
цю ріжницю між суспільним і мануфактурним поділом праці. Одним
з нових податків, вигаданих у Вашинґтоні за часів громадянської війни,
був акциз у 6\% на «всі промислові продукти». Питання: що таке промисловий
продукт? Законодавець відповідає: Кожна річ є продукт,
«якщо вона зроблена» (when it is made), а вона є зроблена, якщо готова
для продажу. Ось один із багатьох прикладів. Мануфактури Нью-Йорку
і Філадельфії за старих часів «робили» парасолі з усіма їхніми причандалами.
Але що парасоля є mixtum compositum\footnote*{
— складне сполучення. \emph{Ред.}
} цілком різнорідних
складових частин, то ці останні поволі поробилися продуктами незалежних
одна від одної й розкиданих по різних місцях галузей продукції,
їхні частинні продукти входили як самостійні товари в парасольну
мануфактуру, яка лише сполучає їх в одну цілість. Янкі охристили такого
роду продукти «assembled articles» (збірними продуктами) — назва,
яку вони заслужили собі саме як «збирачі» податків. Так, парасоля «збирала»
спочатку 6\% акцизу з ціни кожного із своїх елементів, а потім
знову 6\% з ціни цілого продукту.
\parbreak{}  %% абзац продовжується на наступній сторінці
