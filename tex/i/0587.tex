«Кожна сторінка звіту д-ра Гентера, — каже д-р Сімон
у своєму офіціяльному санітарному звіті, — свідчить про недостатню
кількість і злиденну якість помешкань нашого сільського
робітника. І ось уже багато років, як стан його з цього боку проґресивно
гіршає. Тепер сільському робітникові куди важче підшукати
помешкання, а коли й підшукає, то воно куди менше
відповідає його потребам, аніж це було, може, декілька століть
тому... Особливо швидко зростає це лихо протягом останніх 30
або 20 років, і житлові умови селянина тепер надзвичайно сумні.
Він тут цілком безпорадний, хібащо ті, кого він збагачує своєю
працею, захочуть завдати собі клопоту поводитися з ним із певного
роду жалісливістю та ласкавістю. Чи найде він житло на
тій землі, яку обробляє, чи буде те житло придатне для людей
чи лише для свиней, чи буде при ньому невеличкий садок, що так
полегшує гніт злиднів, — усе це залежить не від його готовости

чи вартий важкого процесу травлення, або шматок сала є за приправу до
великої кількости юшки з борошна й цибулі або до вівсянки, і це деньу-день
становить обід сільського робітника... Проґрес промисловости
для нього мав такі наслідки, що дебеле домотканне сукно витиснено в
цьому суворому й вогкому підсонні дешевими бавовняними тканинами,
а міцніші напої — «номінальним» чаєм... Після багатьох годин перебування
на вітрі й дощі рільник вертається до свого котеджу, щоб присісти
біля печі, де горить торф або кавалки, збиті з глини й покидьків кам’яного
вугілля, що, згораючи, виділюють цілі хмари вуглекислоти й сульфатної
кислоти. Стіни хатини пороблено з глини й каменю, долівка —
гола земля, що була тут і перед будуванням хатини, дах — маса понакидуваної,
непошитої соломи. Кожну щілину заткнуто, щоб не виходило
тепло, і в цій атмосфері диявольського смороду, на брудній землі, часто
висушуючи на своєму тілі свою однісіньку одіж, він сідає вечеряти а
дружиною й дітьми. Акушери, примушені проводити частину ночі в цих
хатах, описували, як їхні ноги грузли у брудній земляній долівці і як
їм доводилося — легенька собі справа! — продовбувати дірку в стіні, щоб
здобути собі хоч трохи свіжого повітря. Численні свідки різного ранґу
свідчать, що недосить харчований (underfed) селянин кожної ночі зазнає
цих і інших шкідливих для його здоров’я впливів; результат цього —
квола й золотушна людність, про це справді маємо більш ніж досить
доказів... Повідомлення парафіяльних урядовців у Caermarthenshire і Cardiganshire
виразно потверджують такий самий стан речей. Сюди треба
додати ще більше лихо — поширення ідіотизму. А тепер ще декілька слів
про кліматичні умови. Буйні південно-західні вітри пронизують усю
країну вісім-дев’ять місяців на рік, вони навівають страшні зливи,
що спадають переважно на західніх схилах гір. Дерева трапляються
рідко, хіба лише по затишних місцях; там, де вони не захищені, їх нищить
вітер. Хатини туляться під гірськими терасами, часто по ярах або
каменярнях; лише найдрібніша порода овець і місцева рогата худоба
можуть жити на таких пасовиськах... Молодь еміґрує до східніх гірничих
округ Glamorgan і Monmouth... Caermarthenshire — це розсадник
шахтарів і їхній інвалідний дім... Людність ледве-ледве підтримує
свою чисельність на тому самому рівні. Так, у Cardiganshire було:

                                                                               1851р.               
        1861р.

Чоловічої статі.......                                          45.155                       44.446

Жіночої статі.........                                           22.459                      52.955

                                                                                                    
  97.614 97.401

(Звіт д-ра Гентера в «Public Health. Seventh Report 1864», London
1865, p. 498—502 passim.).
