\parcont{}  %% абзац починається на попередній сторінці
\index{i}{0622}  %% посилання на сторінку оригінального видання
в 1834 р. йому дійсно надано нової гострішої форми.\footnote{
Протестантський «дух» можна, між іншим, побачити ось із чого.
У південній Англії кілька землевласників і заможних фермерів, таємно
зібравшись, склали десять запитань про правильне тлумачення закону
Єлисавети про бідних і подали їх на розгляд славетному юристові того
часу, Серджент Сніджеві (згодом, за Якова І, був суддею). «Дев’яте запитання:
Деякі багаті фармери парафії вигадали мудрий плян, за допомогою
якого можна уникнути всяких непорядків при проведенні закону.
Вони радять збудувати в парафії в’язницю. Кожному бідному, що не
згодиться, щоб його замкнули в ту в’язницю, буде відмовлено в допомозі.
Далі, по околицях треба оголосити, що коли будь-яка особа забажає
заорендувати бідних з цієї парафії, то вона якогось певного дня
має подати запечатану заяву з зазначенням найнижчої ціни, за яку вона
схоче їх у нас взяти. Автори цього пляну гадають, що в сусідніх графствах
є люди, які не хочуть працювати й не мають ані маєтности, ані
кредиту, щоб здобути собі фарму або корабель, і таким чином жити,
не працюючи («so as to live without labour»). Такі особи будуть ладні
зробити парафії дуже вигідні пропозиції. А якщо подекуди бідні і гинутимуть
під опікою такого наймача, то цей гріх спав би на його голову,
бо парафія виконала свій обов’язок проти тих бідних. Однак ми побоюємося,
що даний закон не допускає подібних мудрих заходів («prudential
measure»); але ви мусите знати, що всі інші freeholders\footnote*{
— вільні дрібні господарства. \emph{Ред.}
} нашого та
сусідніх графств приєднаються до нас, щоб спонукати своїх представників
у палаті громад внести закон, який дозволяв би замикати бідних у
в’язницю й віддавати їх на примусову працю, так щоб ні одна людина,
яка не згодиться дати себе замкнути у в’язницю, не мала права ні на яку
допомогу. Це, ми сподіваємося, стримуватиме бідних від того, щоб вимагати
допомогу» («will prevent persons in distress from wanting relief»),
(R. Blakey: «The History of Political Literature from the earliest times»,
London 1855, vol. II, p. 84, 85). — В Шотляндії кріпацтво скасовано
кілька століть пізніше, ніж в Англії. Ще в 1698 р. Флетчер із Селтовну
заявив у шотляндському парламенті: «У Шотляндії є не менш як 200.000
жебраків. Єдина порада, яку можу дати я, принципіяльний республіканець,
— це відновити колишній стан кріпацтва й поробити рабами всіх
тих, що неспроможні піклуватися про своє власне існування». У Eden’а,
«The State of the Poor», vol. I, ch. 1, p. 60,61 читаємо: «Павперизм почався
відтоді, коли селяни стали вільними\dots{} Мануфактури й торговля —
це справжні батьки наших бідних». Ідн, як і згаданий шотляндський
принципіяльний республіканець, помиляється лише в тому, що не скасування
кріпацтва, а скасування права власности селянина на землю
зробило його пролетарем або павпером. — Англійським законам про
бідних відповідає у Франції, де експропріація відбулась іншим способом,
ордонанс Муленса з 1571 р. та едикт з 1656 р.
} Ці безпосередні
наслідки реформації не були найважливішими. Церковна
маєтність становила релігійну твердиню старовинних відносин
земельної власности. І коли вона впала, не могли триматись
далі й ці відносини.\footnote{
Пан Роджерс, хоч він був тоді професором політичної економії в
Оксфордському університеті, в цьому осередку протестантської ортодоксії,
підкреслює в своїй передмові до «History of Agriculture» павпериза-цію
народніх мас у наслідок реформації.
}

Ще останніми десятиліттями XVII століття yeomanry, незалежне
селянство, було численніше, ніж кляса фермерів. Воно
становило головну силу Кромвела і, як визнає навіть Маколей,
являло собою вигідний контраст проти п’янюг-дворянчиків та
їхніх прислужників, сільських попів, які мусили давати шлюб
\parbreak{}  %% абзац продовжується на наступній сторінці
