бітників оподатковувати проґрес промисловости і просто заявляє,
що продуктивність праці взагалі зовсім не обходить робітника.68

Розділ двадцятий

Національні відмінності в заробітній платі

У п’ятнадцятому розділі ми розглядали різноманітні комбінації,
що їх може спричинити зміна абсолютної або відносної
(тобто порівняно з додатковою вартістю) величини вартости робочої
сили, тим часом як, з другого боку, кількість засобів існування,
в яких реалізується ціна робочої сили, і собі може пророблювати
рухи, незалежні\footnote{
«Не точно було б сказати, що заробітна плата (мова тут іде про
її ціну) зросла, якщо вона дає змогу купити більшу кількість якогось
дешевшого продукту» («Ц is not accurate to say that wages are increased,
because they purchase more of a cheaper article»). (David Buchanan
у його виданні «Wealth of Nations» А. Сміса, 1814, т. I, стop. 417,
примітка).
} або відмінні від зміни цієї ціни. Як уже зауважено,
через просте перетворення вартости, зглядно ціни робочої
сили на екзотеричну форму заробітної плати, всі подані там
закони перетворюються на закони руху заробітної плати. Те,
що в межах цього руху являє собою змінні комбінац ї, може для
різних країн являти собою одночасні відмінності в національних
заробітних платах. Отже, при порівнянні національних заробітних
плат треба брати на увагу всі моменти, що визначають зміну
у величині вартости робочої сили: ціну і обсяг природних та
історично розвинутих доконечних життєвих потреб, витрати
на виховання робітника, ролю жіночої й дитячої праці, продуктивність
праці, її екстенсивну й інтенсивну величину. Навіть
найповерховіше порівняння потребує насамперед зведення пересічної
денної плати для однакових виробництв у різних країнах
до однаково великих робочих днів. Після такого вирівнепня денних
заробітних плат почасову плату знову треба перевести на
відштучну, бо тільки ця остання є мірило так продуктивности,
як і інтенсивної величини праці.

цілком такими, як і раніш. Але це було номінальне зниження, про яке
робітників — як запевняють — не було заздалегідь попереджено». («Bright’s
partners had introduced new machinery which would turn out 240 yards
of carpet in the time and with the labour (!) previously required to produce
160 yards. The workmen had no claim whatewer to share in the profits
made by the investment of their employer’s capital in mechanical improvements.
Accordingly, Messrs. Bright proposed to lower the rate of pay from
1\sfrac{1}{2} d. per yard to 1 d., leaving the earnings of the men exactly the. same
as before for the same labour. But there was a nominal reduction, of which
the operatives, it is asserted, had not fair warning before hand»).

63    Тред-юньойни, силкуючись утримати заробітну плату на певному
рівні, домагаються участи в зиску від поліпшених машин! (Який
жах!)... Вони вимагають вищої заробітної плати на тій підставі, що праця
скорочена... інакше кажучи, вони прагнуть накласти податки на промислові
поліпшення». («On Combination of Trades». New Edit, London
1834, p. 42).