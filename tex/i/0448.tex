залежно від того, як потім далі розділяється та здобич між капіталістом
і земельним власником і т. ін. За цим обчисленням
додаткова праця англійського сільського робітника відноситься
до його доконечної праці як 3 : 1, процентова норма експлуатації
дорівнює 300\%.

Шкільна метода розглядати робочий день як сталу величину
зміцнилася в наслідок застосування формул II, бо тут додаткову
працю завжди порівнюється з робочим днем даної величини.
Те саме буде, коли звертати увагу виключно на поділ новоспродукованої
вартости. Робочий день, що вже упредметнився в якійсь
спродукованій вартості, є завжди робочий день даної величини.

Вираз додаткової вартости й вартости робочої сили в частинах
новоспродукованої вартости — спосіб виразу, що, зрештою,
сам виростає з капіталістичного способу продукції і що його
значення з’ясується пізніше — приховує специфічний характер
капіталістичного відношення, а саме обмін змінного капіталу
на живу робочу силу та відповідне усунення робітника від продукту.
Замість того постає фалшива видимість відносин товариства,
де робітник і капіталіст ділять між собою продукт пропорційно
до різних факторів, що утворюють його.19

А втім формули II можна завжди обернути назад у формули І.

Якщо ми, наприклад, маємо: додаткова праця в 6 годин /
робочий день у 12 годин то доконечний
робочий час дорівнює робочому дневі в 12 годин мінус
додаткова праця в 6 годин.

Отже, маємо:
додаткова праця в 6 годин/доконечна праця в 6 годин = \sfrac{100}{100}.
Третя формула, яку я, забігаючи наперед, принагідно вже
наводив, така:

III : додаткова вартість/вартість робочої сили =
додаткова праця/доконечна праця =
неоплачена праця/оплачена праця.

Те непорозуміння, що до нього могла призвести формула
неоплачена праця/оплачена праця, непорозуміння, наче капіталіст оплачує працю,
а не робочу силу, відпадає після поданих вище міркувань.

19    А що всі розвинуті форми капіталістичного процесу продукції
є форми кооперації, то немає, природно, нічого легшого, як абстрагуватися
від їхнього специфічного антагоністичного характеру та одним
махом перетворити їх на форми вільної асоціяції, як то зробив граф
А. де Ляборд у своїй книзі «De l’Esprit de l’Association dans tous les
interêts de la Communauté», Paris 1818. Янкі Г. Керей з таким самим
успіхом принагідно витворяє такі фіґлі навіть щодо відносин системи
рабства.
