своєї метрополії, він намагається силоміць усунути з свого шляху спосіб продукції й присвоєння,
оснований на власній праці. Той самий інтерес, що в метрополії штовхає сикофанта капіталу,
політико-економа, проголошувати теоретичну тотожність капіталістичного способу продукції з його
власною протилежністю, — той самий інтерес спонукає його тут «to make a clean breast of it»\footnote*{
— очистити своє сумління. Ред.
} і гучно
проголосити протилежність цих двох способів продукції. З цією метою він доводить, що розвиток
суспільної продуктивної сили праці, кооперація, поділ праці, вживання машин у великому маштабі й т.
ін. неможливі без експропріяції робітників та відповідного перетворення засобів продукції на
капітал. В інтересах так званого національного багатства він шукає штучних засобів створення
народньої бідности. Його апологетичний панцер кришиться тут на шматочки, як трухлява губка.

Велика заслуга Е. Ґ. Векфілда не в тому, що він сказав щось нове про колонії,\footnote{
Небагато променів світла, що їх Векфілд кинув на суть самих колоній, цілковито передбачили
Мірабо-батько, фізіократи й ще багато раніш англійські економісти.
} а в тому, що в
колоніях він розкрив правду про капіталістичні відносини в метрополії. Як протекційна система на
своїх початках\footnote{
Пізніше вона стає тимчасовою доконечністю в міжнародній конкуренційній боротьбі. Але хоч і які
були б її мотиви, наслідки її лишаються ті самі.
} намагалася фабрикувати капіталістів
у метрополії, так теорія колонізації Векфілда, що її Англія довгий час силкувалася здійснити
законодавчим способом, намагається фабрикувати найманих робітників у колоніях. Це він називає
«systematic colonization», систематичною колонізацією.

Насамперед Векфілд відкрив у колоніях, що володіння грішми, засобами існування, машинами й іншими
засобами продукції ще не робить із людини капіталіста, коли бракує такого додатку, як найманий
робітник, другої людини, що примушена добровільно сама себе продавати. Він відкрив, що капітал не є
річ, а суспільне відношення між людьми, упосереднене речами.\footnote{
«Негр є негр... Лише за певних відносин він стає рабом. Бавовнопрядна машина є машина для
прядіння бавовни. Лише за певних відносин вона стає капіталом. Вирвана з цих відносин, вона так само
не є капітал, як золото само по собі не є гроші або цукор — ціна цукру... Капітал є суспільне
продукційне відношення. Він — історичне продукційне відношення». (Karl Marx: «Lohnarbeit und
Kapital», «Neue Rheinische Zeitung», № 226 з 7 квітня 1849 p. — Карл Маркс: «Наймана праця і
капітал». Партвидав «Пролетар» 1932 р., стор. 22, 23).
} Пан Піл, — скаржиться він нам, —
взяв із собою з Англії на Лебединий берег у Новій Голляндії засобів існування та засобів продукції
на суму 50.000 фунтів стерлінґів. Пан Піл був такий передбачливий, що, крім того, взяв із собою
3.000 осіб із робітничої кляси — чоловіків, жінок і дітей. Але, прибувши на місце призначення, Піл
залишився без жодного