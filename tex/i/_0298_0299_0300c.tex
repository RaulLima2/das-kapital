\parcont{}  %% абзац починається на попередній сторінці
\index{i}{0298}  %% посилання на сторінку оригінального видання
of Nations» і~\abbr{т. ін.} Він не тільки розвиває суспільну продуктивну
силу праці для капіталіста замість для робітника, але й розвиває
її через покалічення індивідуального робітника. Він створює
нові умови панування капіталу над працею. Тим-то, якщо цей
поділ праці, з одного боку, з’являється як історичний проґрес
та доконечний момент розвитку процесу економічного формування
суспільства, то, з другого боку, він з’являється як засіб
цивілізованої та рафінованої експлуатації.

Політична економія, що стає спеціяльною наукою лише за
мануфактурного періоду, розглядає суспільний поділ праці взагалі
лише з погляду мануфактурного поділу праці\footnote{
Давніші письменники, як от Петті, анонімний автор «Advantages
of the East-India Trade» і~\abbr{т. д.}, зрозуміли капіталістичний характер
мануфактурного поділу праці ясніше, ніж А.~Сміс.
}, лише як
засіб з тією самою кількістю праці продукувати більше товару,
отже, здешевити товари та прискорити акумуляцію капіталу.
В гострій протилежності до цього підкреслювання кількости та
мінової вартости письменники клясичної старовини звертають
увагу виключно на якість та на споживну вартість\footnote{
Виняток серед сучасних письменників становлять лише деякі
автори XVІІІ віку, як от Беккарія та Джемс Гарріс, які щодо поділу
праці майже виключно наслідують давніх. Напр., Беккарія каже: «Кожен
знає з досвіду, що, прикладаючи рук і розуму завжди до однорідної праці
та до виготовлювання тих самих продуктів, можна з більшою легкістю
досягти значніших та ліпших успіхів, ніж у тому випадку, коли б кожний
ізольовано сам для себе виготовляв усі потрібні йому речі\dots{} Люди поділяються
таким чином на різні кляси й стани в інтересах спільної та індивідуальної
користи». («Ciascuno prova coll’esperienza, che applicando
la mano e l’ingegno sempre allo stesso genere di opere e di produtti, egli
più facili, più abbondanti e migliori ne traca resultati, di quello che se
ciascuno isolatamente le cose tutte a se necessarie soltanto facesse\dots{} Dividendosi
in tal maniera per la comune e privata utilita gli uomini
in varie classe e condizioni»). (\emph{Cesare Beccaria}: «Elementi di Economia
Publica», ed. Custodi, Parte Moderna, vol. XI, p. 28). Джеме
Гарріс, пізніш граф Малмсберійський, відомий своїми «Diaries» про
своє перебування послом у Петербурзі, сам каже в одній примітці до
свого «Dialogue concerning Happiness», London 1741, пізніше знов передрукованого
в «Three Treatises etc.», З rd ed. London 1772: «Всі докази
природности суспільства (а саме докази, засновані на «поділі занять»)
я взяв із другої книги «Республіки» Платона. (The whole argument,
to prove society natural is taken from the second book of Plato’s
republic»).
}. У наслідок
роз’єднання суспільних галузей продукції товари виробляються
ліпше, різні нахили й таланти людей вибирають собі відповідні
сфери діяльности\footnote{
Так, в «Одісеї», XIV, 228 говориться: «\textgreek{Ἄλλoς γὰρ τ’ἄλλοισιν ἀνὴρ ἐπιτέρπεται ἔργοις}»\footnote*{
Одні люди люблять одне, інші — інше. \emph{Ред.}
} a Архілох y Секста Емпірика каже: «\textgreek{Ἄλλος ἄλλῳ ἐπ’ἔργῳ καρδίην ἰαίνεται}»\footnote*{
Одне тішить серце одного, інше — іншого. \emph{Ред.}
}}, а без обмеження ніде не можна зробити нічого
значного\footnote{
«\textgreek{Поλλ’ ἠπίστατο ἔργα, κακῶς δ’ ἠπίστατο πάντα}»\footnote*{
Багато знав він справ, та всі погано знав. \emph{Ред.}
} — Атенець, як товаропродуцент,
почував свою перевагу над спартанцем, бо цей останній міг
порядкувати у війні людьми, але не грішми, — як це Тукідід вкладав
в уста Перікла у промові, в якій він підцьковує атенців до пелопонеської
війни: «\textgreek{Σώμασί τε ἐτοιμότεροι οἱ αὐτουργοὶ τῶν ἀνθρώπων ἤ χρήμασι πολεμεῖν}»\footnote*{
Люди, що працюють для задоволення власних потреб, радше
віддадуть на війну свої тіла, ніж гроші. \emph{Ред.}
}. (\emph{Thucydides}:
«Geschichte des Peloponnesischen Krieges», книга перша, відділ
141). А проте їхнім ідеалом, навіть у матеріяльній продукції, була
\textgreek{αυταρχεια}\footnote*{
— автаркія. \emph{Ред.}
}, що протиставляється поділові праці, бо «\textgreek{παρ’ ὧν γὰρ τὸ εὖ, παρὰ τούτων καὶ τὸ
αὐτάρκες}»\footnote*{
«з цього постає благо, а з того і незалежність». \emph{Ред.}
}. Треба при цьому зважити, що за часів упадку
30 тиранів не було ще й \num{5.000} атенців без земельної власности.
}. Отже продукт і продуцент удосконалюються через
\index{i}{0299}  %% посилання на сторінку оригінального видання
поділ праці. Якщо ці письменники принагідно згадують і про
зріст маси продуктів, то лише щодо більшої повноти споживних
вартостей. Про мінову вартість, про подешевшання товарів вони
зовсім не згадують. Цей погляд споживної вартости панує так
у Платона\footnote{
Платон виводить поділ праці всередині громади з багатобічности
потреб та однобічности здібностей індивідів. Його головний погляд є в
тому, що робітник мусить пристосовуватися до справи, а не справа до
робітника, як воно неминуче буває тоді, коли він разом працює в різних
ремествах, отже, в тому чи тому реместві працює як у побічному. «Бо
справі ніколи чекати на вільний час продуцента, а треба, щоб продуцент
виконував свою справу пильно і не між іншим. — Треба. — Аджеж кожна
річ продукується легше й ліпше і в більшій кількості, коли людина робить
лише одну річ, що відповідає її здібності, та в належний час, вільний
від усякої іншої роботи». («\textgreek{Οὐ γὰρ ἐθέλει τὸ πραττόμενον τὴν τοῦ πράττοντος σχολὴν περιμένειν, ἀλλ’
ἀνάγκη τὸν πράττοντα τῷ πραττομένῷ ἐπακολουθεῖν μὴ ἐν παρέργου μέρει. \textemdash{} Ἀνάγκη. Ἐκ δὴ τούτων πλείω
τε ἕκαστα γίγνεται καὶ κάλλιον καὶ ῥᾷον, ὅταν εἷς ἓν κατὰ φύσιν καὶ ἐν καιρῷ, σχολὴν τῶν ἄλλων ἄγων,
πράττῃ»}). («Respublica», lib. II, c. 12,
ed. Baiter, Orelli etc.). Подібні думки ми маємо в Тукідіда: «Geschichte
des Peloponnesischen Krieges», книга перша, відділ 142: «Морська справа
є така ж умілість, як і будь-що інше, і не можна коло неї працювати принагідно,
як коло якоїсь побічної справи, навпаки, морська справа не
дозволяє працювати коло чогось іншого навіть як побічної справи».
Якщо справа мусить чекати на робітника, каже Платон, то часто ґавиться
критичний момент продукції і продукт псується, «\textgreek{ἔργου καιρὸν διόλλυται}». Цю
саму платонівську ідею подибуємо знов у протесті англійських власників
білилень проти того застереження фабричного закону, яке встановлює визначену
годину на їжу для всіх робітників. Їхнє підприємство, мовляв, не
може пристосовуватися до робітників, бо «в різних операціях опалювання,
промивання, біління, качання, прасування та фарбування не можна
перервати роботу у наперед визначений момент без небезпеки заподіяти
шкоду\dots{} встановлення для всіх робітників тієї самої перерви на їжу —
це значило б у певних випадках кинути коштовні продукти на небезпеку,
що вони попсуються через незакінчені операції» («in the various operations
of singeing, washing, bleaching, mangling, calendering, and dycing,
none of them can be stopped at a given moment without risk of damage\dots{}
to enforce the same dinner hour for all the workpeople might occasionally
subject valuable goods to the risk of danger by incomplete operations»).
Le platonisme où va-t-il se nicher!\footnote*{
Куди ще може продертись платонізм! \emph{Ред.}
}.
}, що розглядав поділ праці як основу поділу суспільства
на стани, як і в Ксенофонта\footnote{
Ксенофонт оповідає, що не тільки велика честь діставати страви
зі столу перського короля, але що й ці страви куди смачніші, ніж інші.
«І тут немає нічого дивного, бо як і інші вмілості надто вдосконалені по
великих містах, так само й королівські страви готуються цілком своєрідно.
Бо по дрібних містах та сама людина виробляє ліжка, двері, плуги, столи;
крім цього, вона часто ще й будує доми і задоволена, коли сама находить
такі замовлення, яких вистачає, щоб підтримати своє існування.
Цілком неможливо, щоб людина, яка робить усяку всячину, робила
все добре. Але по великих містах, де кожний поодинокий продуцент
находить багато покупців, досить і одного ремества, щоб прохарчуватися.
Часто непотрібно навіть знати ціле ремество: один виробляє чоловічі
черевики, другий — жіночі. Часто один живе з того, що лише шиє, другий
— з того, що викроює черевики; один крає одяг, інший складає кусники
докупи. Неминучий наслідок цього той, що виконавець найпростішої
праці безумовно й накраще виконує її. Так само стоїть справа і
з куховарством». (\emph{Xenophon}: «Cyropaedie», lib. VIII, с. 2). Тут вважається
виключно на те, як дійти доброї якости споживної вартости,
хоч уже й Ксенофонтові відома залежність маштабу поділу праці від
обсягу ринку.
}, який з характеристичним для
нього буржуазним інстинктом уже ближче підходить до поділу
праці всередині майстерні. Платонова республіка, оскільки в ній
\index{i}{0300}  %% посилання на сторінку оригінального видання
поділ праці фігурує як конститутивний принцип держави, є
лише атенська ідеалізація єгипетського кастового ладу, так само
як Єгипет в інших його сучасників, наприклад, у Ізократа\footnote{
«Він (Бузіріс) поділив усіх на окремі касти\dots{} наказав, щоб ті
самі люди завжди працювали коло тих самих справ, бо знав, що ті люди,
які змінюють свою працю, не знають ґрунтовно ніякої справи; а ті, які
завжди працюють коло тих самих справ, зможуть кожну з них виконати
якнайдосконаліше. І ми справді бачимо, що відносно вмілости та реместв
вони куди більше перевищили своїх суперників, ніж звичайно майстер
перевищує партача, а інституції, якими вони підтримують королівське
панування та інший державний лад, були в них такі досконалі, що найславетніші
філософи, яким доводилося писати про це, вихваляли державний
лад Єгипту більше, ніж інших країн». (\emph{Isocrates}: «Busiris», с. 8).
},
вважається за зразкову промислову країну; це значення Єгипет
зберігає ще навіть для греків доби Римської імперії\footnote{
Порівн. \emph{Diodorus Siculus.}
}.

Підчас власне мануфактурного періоду, тобто того періоду,
коли мануфактура була панівною формою капіталістичного способу
продукції, повне здійснення властивих їй тенденцій наражається
на багато різних перешкод. Хоч мануфактура, як ми
бачили, і утворює поряд ієрархічного розчленування робітників
простий поділ їх на навчених та ненавчених, однак число останніх
лишається через переважний вплив перших дуже обмеженим.
Хоч вона й пристосовує окремі операції до різних ступенів дозрілости,
сили й розвитку її живих робочих органів і тим штовхає
до продуктивного визиску жінок та дітей, все ж взагалі і в цілому
ця тенденція розбивається об звички та опір робітників-чоловіків.
Хоч розчленування ремісничої роботи понижує витрати на
навчання, отже, і вартість робітників, проте для тяжчої детальної
праці лишається потрібним довший час на навчання, і його
з запалом зберігають робітники навіть там, де він зайвий. Ми
находимо, приміром, в Англії laws of apprenticeship\footnote*{
— закони про учнів. \emph{Ред.}
} з їхнім
семирічним часом навчання в повній силі аж до кінця мануфактурного
періоду, і лише велика промисловість знищила їх. Через
те, що реміснича вправність лишається основою мануфактури,
\parbreak{}  %% абзац продовжується на наступній сторінці
