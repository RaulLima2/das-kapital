\parcont{}  %% абзац починається на попередній сторінці
\index{i}{0046}  %% посилання на сторінку оригінального видання
він хоче відчужити його, обмінюючи на такі товари, що їхня
споживна вартість задовольняє його потреби. Всі товари є неспоживні
вартості для їхніх посідачів, і вони є споживні вартості
для їхніх непосідачів. Отже, вони мусять всебічно переходити
з рук до рук. Але цей перехід становить їхній обмін, а їхній обмін
ставить їх у взаємне відношення як вартості і реалізує їх як
вартості. Тому товари мусять реалізуватись як вартості, раніш
ніж вони зможуть реалізуватись як споживні вартості.

З    другого боку, вони мусять довести, що вони мають споживну
вартість, раніш ніж вони зможуть реалізуватись як вартості,
бо витрачену на них людську працю рахується лише остільки,
оскільки її витрачено в якійсь формі, корисній іншим. Але лише
обмін їх може показати, чи ця праця корисна для інших, отже,
чи продукт її задовольняє чужі потреби.

Кожний посідач товарів бажає відчужити свій власний товар,
обмінюючи його лише на такі інші товари, що їхня споживна
вартість задовольняє його потреби. В цьому розумінні обмін
є для нього лише індивідуальний процес. З другого боку, він
бажає зреалізувати свій товар як вартість, отже, зреалізувати
його в усякому іншому товарі такої самої вартости, незалежно
від того, чи має його товар споживну вартість для посідача іншого
товару чи ні. В цьому розумінні обмін є для нього загальносуспільний
процес. Але той самий процес не може бути для всіх
посідачів товарів одночасно лише індивідуальним і разом із тим
лише загальносуспільним.

Придивившися ближче, ми побачимо, що для кожного посідача
товарів кожний чужий товар має значення осібного еквіваленту
його товару, отже, його власний товар має значення
загального еквіваленту всіх інших товарів. А що всі посідачі
товарів роблять те саме, то жодний товар не є загальний еквівалент,
а тому товари не мають також жодної загальної відносної
форми вартости, що в ній їх ототожнювалося б як вартості й
порівнювалося б як величини вартости. Отже, вони взагалі
протистоять один одному не як товари, а лише як продукти, або
споживні вартості.

У своєму скрутному становищі наші посідачі товарів думають
як Фавст: «На початку було діло». Отже, вони вже зробили
діло, раніш ніж почали думати. Закони товарової природи здійснюються
через природний інстинкт посідачів товарів. Вони можуть
відносити свої товари один до одного як вартості, а тому
і як товари лише тоді, коли протиставлять їх до якогось цілком
іншого товару як загального еквіваленту. Це показала аналіза
товару. Але лише суспільний акт може зробити якийсь певний
товар загальним еквівалентом. Тому суспільна акція всіх інших
товарів вилучає якийсь певний товар, що в ньому вони всебічно
виражають свої вартості. Через це натуральна форма цього
товару стає суспільно-визнаною еквівалентною формою. Бути
загальним еквівалентом,— це стає, за допомогою суспільного
процесу, специфічною суспільною функцією вилученого товару.
\parbreak{}  %% абзац продовжується на наступній сторінці
