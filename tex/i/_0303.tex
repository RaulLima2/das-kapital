\parcont{}  %% абзац починається на попередній сторінці
\index{i}{0303}  %% посилання на сторінку оригінального видання
був би машиною, а Клавзенів Circular Loom,\footnote*{
— коловий варстат. \emph{Ред.}
} що його рухає
рука одного робітника й що виробляє 96.000 петель за одну хвилину,
був би простим знаряддям. Навіть більше, той самий Loom
був би знаряддям, коли його урухомлювати рукою, і машиною —
коли урухомлювати його парою. А що вживання сили тварин є
один із найдавніших винаходів людства, то виходило б фактично,
що машинова продукція передувала ремісничій продукції. Коли
Джон Вайт 1735 р. оповістив про свою прядільну машину, а
разом з нею і про промислову революцію XVIII віку, він і словечка
не згадав, що цю машину, замість людини, рухає осел, а
проте ця роля припала ослові. Машина «для того, щоб прясти
без пальців» — така була його програма.\footnote{
Вже й перед ним уживали прядільних машин, хоч і дуже недосконалих,
щось найпевніш, насамперед в Італії. Критична історія технології
взагалі показала б, як мало будь-який технічний винахід XVIII століття
належить якомусь одному індивідові. Досі такої праці ще немає. Дарвін
скерував інтерес на історію природної технології, тобто на формування
рослинних та тваринних органів як знарядь продукції для життя рослин і
тварин. Невже не заслуговує такої самої уваги й історія утворення
продуктивних органів суспільної людини, матеріяльної бази кожної
осібної суспільної організації? І чи не легше було б її дати, бо, як каже
Віко, історія людства відрізняється від історії природи тим, що першу
робили ми, а другу — не ми? Технологія розкриває активне ставлення
людини до природи, безпосередній процес продукції життя людини, а
разом з тим і суспільних відносин її життя та інтелектуальних уявлень,
що мають своє джерело в цих відносинах. Навіть усяка історія релігії,
що абстрагується від цієї матеріяльної бази, є некритична. Дійсно,
куди легше за допомогою аналізи знайти земне зерно туманних релігійних
уявлень, аніж навпаки, з наявних реальних життєвих відносин розвинути
унебеснені форми цих відносин. Остання метода — це єдина матеріялістична,
отже, і наукова метода. Хиби абстрактного природознавчого
матеріялізму, який виключає історичний процес, видно вже з абстрактних
та ідеологічних уявлень його представників, скоро тільки вони зважуються
вийти поза межі своєї спеціяльности.
}

Всяка розвинена машина складається з трьох посутньо відмінних
частин: із рухової машини, передатного механізму, нарешті,
виконавчої машини, або робочої машини. Рухова машина
діє як рушійна сила цілого механізму. Вона або породжує свою
власну рушійну силу, як от парова машина, кальорійна машина,
електромагнетична машина й т. д., або дістає поштовх з-поза
себе від якоїсь готової вже сили природи, як от водяне колесо — від
водоспаду, вітряк — від вітру й т. д. Передатний механізм,
що складається з маховиків, рухливих валів, зубчастих коліс,
ексцентриків, стрижнів, шнурів, пасів, проміжних пристроїв
та знарядь найрізноманітнішого роду, реґулює рух, змінює, де
це потрібно, його форму, приміром, з простовисного на коловий,
розподіляє його та переносить на виконавчі машини. Обидві ці
частини механізму існують лише на те, щоб дати виконавчій машині
рух, що за його допомогою вона схоплює предмет праці та
доцільно зміняє його. Промислова революція XVIII віку виходить
саме від цієї частини машини, від виконавчої машини. І ще тепер
\parbreak{}  %% абзац продовжується на наступній сторінці
