що робітник, який живе навіть на оброблюваній ним землі, знаходить
собі таке помешкання, що на нього заслуговує його продуктивне
життя. Навіть у князівських маєтках робітничі котеджі
часто мають якнайзлиденніший характер, Є лендлорди, що вважають
і стайню за досить добре житло для своїх робітників
і їхніх родин, а проте не соромляться видушувати якнайбільше
грошей з винаймання таких помешкань.166 Хай це буде лише
напівзавалена халупа з однією кімнатою для спання, без печі,
без кльозета, з вікнами, що не відчиняються, без водопостачання,
крім якогось рівчака, без садка, — робітник безпорадний проти
такої несправедливости. А наші санітарно-поліційні закони
(The Nuisances Removal Acts) — це мертва буква. Бо ж проводити
їх доручено тим саме власникам, що здають у найми
такі діри... Виняткові веселіші картини не повинні засліплювати
нас та закривати перед нами величезну силу фактів, що є ганьба
англійської цивілізації. Дійсно, жахний мусить бути стан речей,
коли, не зважаючи на очевидну потворність теперішніх
помешкань, компетентні спостерігачі одноголосно доходять такого
висновку, що навіть ці повсюдно нікчемні помешкання є ще
безмірно менше лихо, ніж просто кількісний брак помешкань.
Вже віддавна переповнення помешкань сільських робітників

і помешкання. Тоді поряд із фармером він є другий пан для сільського
робітника. Останній мусить бути одночасно і його покупцем. «З 10 шилінґами
на тиждень мінус 4 фунти квартирної плати на рік він зобов’язаний
купувати чай, цукор, борошно, мило, свічки й пиво по цінах,
які сподобається визначити крамареві» (Там же, стор. 134). Ці відкриті
села — це дійсно «карні колонії» англійського рільничого пролетаріату.
Багато з цих котеджів — це чисті постоялі двори, через які проходить
уся бродяча наволоч з околиці. Селянин і його родина, що серед,
найбрудніших обставин часто справді навдивовижу зберегли путящість
і чистоту характеру, тут цілком гинуть. Серед знатних Шейлоків це,
звичайно, мода по-фарисейському знизувати плечима на адресу будівельних
спекулянтів, дрібних власників і відкритих сел. Вони дуже
добре знають, що їхні «закриті села й показні села» є місце народження
«відкритих сел» і не могли б існувати без цих останніх. «Без дрібних
власників одкритих сел найбільша частина сільських робітників мусіла б
спати під деревами тих маєтків, де вони працюють» (Там же, стор. 135).
Система «відкритих» і «закритих» сел панує по всій середній і всій
східній Англії.

166 Винаймач помешкання (фармер або лендлорд) безпосередньо або
посередньо збагачується з праці людини, що їй він платить 10 шилінґів
на тиждень, а потім знову відбирає в цього бідолахи 4 або 5 фунтів
стерлінґів річної квартирної плати за хати, що на вільному ринку не
варті й 20 фунтів стерлінґів, але зберігають свою штучну ціну, бо власник
має силу сказати: «Бери мою хату, або йди геть звідси і шукай собі,
не мавши від мене атестації, якесь інше пристановище»... Коли людина
хоче поліпшити своє становище й піде на залізницю укладати шини або
на каменярню, то та сама сила знову каже йому: «Працюй в мене за що
низьку плату, або йди геть за тиждень після попередження; забирай свою
свиню, коли вона в тебе є, і поміркуй про те, що ти дістанеш за картоплю,
яка росте на твоєму городі». Коли ж проганяти робітника не в інтересах
власника (або фармера), то він у таких випадках іноді вважає за краще
підвищити квартирну плату, щоб покарати робітника, за те, що він
покинув у нього служити» (Д-р Гентер, там же, стор. 132).
