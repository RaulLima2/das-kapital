\parcont{}  %% абзац починається на попередній сторінці
\index{i}{0059}  %% посилання на сторінку оригінального видання
боку, конче потрібно, щоб вартість, відмінно від різноманітних
тіл товарового світу, розвинулася в цю безглузду речову, але
разом з тим просто суспільну форму.\footnote{
Порівн. «Theorien von der Masseinheit des Geldes» в «Zur Kritik
der Politischen Oekonomie», S. 53. («Теорії про одиницю міри грошей»
в «До критики політичної економії». ДВУ, 1926 р., стор. 91). Фантазії
про підвищення або пониження «ціни монет», які сходять на те, що законні
грошові назви для фіксованих законом частин ваги золота та срібла
треба перенести державним актом на більші або дрібніші вагові частини
і відповідно до цього карбувати з 1/4 унції золота 40 шилінґів замість 20, —
ці фантазії, оскільки вони не є незграбними фінансовими операціями проти
державних і приватних кредиторів, а ставлять за мету досягти економічних
«цілющих ліків», вичерпно схарактеризував Детті в «Quantulumcumque
concerning Money. To the Lord Marquis of Halifax», 1682, так
що його безпосередні наступники, сер Дудлей Норт і Джон Льокк,
не кажучи вже про пізніших, могли лише вульгаризувати його
думки. «Коли б багатство народу, — каже він між іншим, — можна
було збільшити декретом вдесятеро, то дивно було б, чому такого декрета
давним-давно не видано нашими урядами» («If the wealth of a nation
could be decupled by a Proclamation, it were strange that such proclamations
have not long since been made by our Governors», стор. 3 його щойно згаданого
твору).
}

Ціна є грошова назва упредметненої в товарі праці. Еквівалентність
товару й тієї кількости грошей, що її назва є його ціна, є,
отже, тавтологія,\footnote{
«Або доведеться визнати, що вартість мільйона в грошах більша,
ніж така сама вартість у товарах» («Ou bien, il faut consentir à dire qu’une
valeur d’un million en argent vaut plus qu’une valeur égale en marchandises»)
(Le Trosne: «De l’Intérêt Social», p. 992), отже, «що дана вартість варта
більше, ніж рівна їй інша вартість» («qu’une valeur vaut plus qu’une valeur
égale»).
} як і взагалі відносний вираз вартости товару
завжди є вираз еквівалентности двох товарів. Але, коли ціна
як покажчик величини вартости товару є покажчик його мінового
відношення до грошей, то звідси не випливає зворотне,
а саме, що покажчик мінового відношення товару до грошей
неодмінно є покажчик величини товарової вартости. Припустімо,
що на продукцію 1 квартера пшениці витрачається таку саму
кількість суспільно доконечної праці, що і на 2 фунти стерлінґів
(близько 72 унції золота). Два фунти стерлінґів є тоді грошовий
вираз величини вартости квартера пшениці, або його ціна.
Але, коли обставини дозволять зазначити його ціну на 3 фунти
стерлінґів або примусять знизити її до 1 фунта стерлінґів, тоді
1 фунт стерлінґів є занадто малий, а 3 фунти стерлінґів занадто
великі як вирази величини вартости пшениці, а проте вони є
ціни пшениці, бо, поперше, вони є форма її вартости, гроші, а
подруге, — покажчики її мінового відношення до грошей. За
незмінних умов продукції або за незмінної продуктивної сили
праці на репродукцію 1 квартера пшениці мусить витрачатися
стільки ж суспільного робочого часу, як і раніш. Ця обставина
не залежить ані від волі продуцентів пшениці, ані від волі інших
посідачів товарів. Отже, величина вартости товару виражає доконечне,
іманентне самому процесові творення товару відношення
до суспільного робочого часу. З перетворенням величини вартости
\index{i}{0060}  %% посилання на сторінку оригінального видання
на ціну це доконечне відношення з’являється як мінове
відношення даного товару до грошового товару, що існує поза
ним. Але в цьому відношенні так само добре може бути виражена
величина вартости товару, як і той плюс або мінус, з яким його
за даних обставин відчужується. Отже, можливість кількісного
незбігу ціни з величиною вартости, або відхилення ціни від величини
вартости, є в самій формі ціни. Це не вада цієї форми, а
навпаки, це робить її адекватною формою способу продукції,
при якому правило може пробивати собі шлях лише як сліпо
діючий закон пересіччя іреґулярностей (Durchschnittsgesetz der
Regellosigkeit).\footnote*{
У французькому виданні кінець цього речення зформульовано
так: «... способу продукції, при якому правило здійснюється як закон
лише через сліпу гру іреґулярностей, що, у пересічному, компенсують,
паралізують і нищать одна одну». («Le Capital etc», ch. Ill, p. 43). \emph{Ред.}
}

Однак форма ціни не лише допускає можливість кількісного
незбігу величини вартости з ціною, тобто величини вартости
з її власним грошовим виразом, але може ховати в собі ще й
якісну суперечність, так що ціна взагалі перестає бути виразом
вартости, дарма що гроші є лише форма вартости товарів. Речі,
які сами по собі не є товари, приміром, сумління, честь і т. ін.,
можуть бути продажними для своїх посідачів за гроші і таким
чином через свою ціну набути товарової форми. Отже, річ формально
може мати ціну, не маючи вартости. Вираз ціни стає тут
уявлюваним так само, як і певні величини в математиці. З другого
боку, і уявлювана форма ціни, — як, приміром, ціна некультивованого
ґрунту, що не має вартости, бо в ньому не упредметнено
жодної людської праці, — може ховати в собі дійсне вартостеве
відношення або відношення, вивідне з цього останнього.

Ціна, як і відносна форма вартости взагалі, виражає вартість
якогось товару, приміром, тонни заліза, тим що певна кількість
еквіваленту, наприклад, унція золота, завжди є безпосередньо
вимінна на залізо, тимчасом як залізо, навпаки, аж ніяк не в
безпосередньо вимінне на золото. Отже, щоб практично впливати
як мінова вартість, товар мусить скинути з себе своє природне
тіло, перетворитися з лише уявлюваного золота на дійсне
золото, хоч би таке перевтілення було для нього «прикріше»,
ніж для геґелівського «поняття» перехід від доконечности до
свободи, або для морського рака скинути з себе свою шкаралупу,
абож для св. Єроніма скинути з себе старого Адама.\footnote{
Коли Єронімові замолоду доводилося багато боротися із своєю
матеріяльною плоттю, на що вказує його боротьба в пустелі з образами
гарних жінок, то на старість йому довелось боротися з духовною плоттю.
«Мені здавалось, — каже він, приміром, — що я стою перед суддею світу».
«Хто ти?» — запитав голос. «Я — християнин». — «Брешеш, — гримнув
на мене суддя світу, — ти лише ціцероніянець».
} Поруч
із своєю реальною формою, наприклад, заліза, товар може мати
в ціні ідеальну форму вартости або уявлювану форму золота,
але він не може бути одночасно дійсним залізом і дійсним золотом.
Для того, щоб дати йому ціну, досить прирівняти його до
\parbreak{}  %% абзац продовжується на наступній сторінці
