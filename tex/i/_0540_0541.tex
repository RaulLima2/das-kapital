\parcont{}  %% абзац починається на попередній сторінці
\index{i}{0540}  %% посилання на сторінку оригінального видання
дедалі менше й менше робітників. З другого боку, старий капітал,
що періодично репродукується в новому складі, відштовхує
дедалі більше й більше робітників, що їх він раніше вживав.

\subsection[Прогресивне утворення відносного перелюднення,
або промислової резервної армії]{Прогресивне утворення відносного перелюднення,
або промислової резервної армії\footnotemarkZ{}}
\footnotetextZ{У додатках подаємо уривок з цього параграфа за французьким
виданням. Див. стор. 736. \emph{Ред.}}

Акумуляція капіталу, що первісно виступала лише як кількісне
поширення, відбувається, як ми бачили, при безперервній
якісній зміні його складу, при постійному збільшенні його сталої
складової частини коштом змінної\footnoteA{
Примітка до третього видання. — У власному примірнику Маркса
тут на берегах книги є така увага: «Тут для пізнішого треба зауважити:
коли поширення є лише кількісне, то за більшого або меншого капіталу
в тій самій галузі промисловости зиски відносяться один до одного як
величини авансованих капіталів. Якщо кількісне поширення діє і якісно,
то разом з тим підноситься норма зиску для більшого капіталу». [\emph{Ф. Е.}].
}.

Специфічно капіталістичний спосіб продукції, відповідний
до нього розвиток продуктивної сили праці, спричинена ним
зміна в органічному складі капіталу не тільки йдуть пліч-о-пліч
з розвитком акумуляції або з зростанням суспільного багатства.
Вони йдуть уперед куди швидше, бо проста акумуляція,
або абсолютне поширення цілого капіталу супроводиться централізацією
його індивідуальних елементів, а технічний переворот
у додатковому капіталі супроводиться технічним переворотом
у первісному капіталі. Отже, з проґресом акумуляції
відношення сталої частини капіталу до змінної, коли воно було
первісно 1: 1, змінюється в 2: 1, 3: 1, 4: 1, 5: 1, 7: 1 і~\abbr{т. д.},
так що із зростанням капіталу на робочу силу замість \sfrac{1}{2} його
загальної вартости перетворюється проґресивно лише \sfrac{1}{3}, \sfrac{1}{4},
\sfrac{1}{5} \sfrac{1}{6} \sfrac{1}{8} і~\abbr{т. д.}, а на засоби продукції, навпаки, — \sfrac{2}{3}, \sfrac{3}{4}, \sfrac{4}{5},
\sfrac{5}{6}, \sfrac{7}{8} і~\abbr{т. д.} Отже, через те, що попит на працю визначається не
розміром цілого капіталу, а розміром його змінної складової
частини, то із зростанням цілого капіталу попит на працю проґресивно
падає, замість, як це раніше припускалося, більшати
пропорційно до цього зростання. Він падає відносно проти величини
цілого капіталу і в щораз швидшій проґресії із зростанням
цієї величини. Щоправда, із зростанням цілого капіталу зростає
і його змінна складова частина, або додавана до нього робоча
сила, але зростає вона в щораз меншій пропорції. Павзи, що протягом
їх акумуляція діє як просте поширення продукції на
даній технічній основі, скорочуються. Але мало того, що прискорена
в чимраз більшій проґресії акумуляція цілого капіталу
потрібна, щоб поглинути певне додаткове число робітників або
навіть щоб дати заняття тим робітникам, які вже функціонують
та в наслідок постійної метаморфози старого капіталу втрачають
роботу. Це щораз більше зростання акумуляції й централізації
ще й собі перетворюється на джерело нових змін у складі капіталу,
\index{i}{0541}  %% посилання на сторінку оригінального видання
або нового прискореного зменшення його змінної складової
частини порівняно із сталою. Це відносне зменшення змінної
складової частини капіталу, що прискорюється із зростанням
цілого капіталу і до того ж куди швидше, ніж його власне зростання,
видається на другому боці, навпаки, щораз швидшим
абсолютним зростанням робітничої людности, щораз швидшим,
ніж зростання змінного капіталу або засобів для праці цієї людности.
У дійсності ж капіталістична акумуляція постійно продукує,
і саме пропорційно до своєї енерґії і свого розміру, відносну,
тобто для середніх потреб самозростання капіталу надмірну,
а тому й зайву, або додаткову робітничу людність.

Розглядаючи цілий суспільний капітал, ми бачимо, що рух
його акумуляції то викликає періодичні зміни, то моменти цього
руху одночасно розподіляються між різними сферами продукції.
У деяких сферах відбувається зміна в складі капіталу без зростання
його абсолютної величини, в наслідок простої концентрації;
у деяких сферах абсолютне зростання капіталу зв’язане з абсолютним
зменшенням його змінної складової частини, або вбируваної
ним робочої сили; у деяких сферах то капітал і далі зростає
на своїй даній технічній основі і пропорційно до свого зростання
притягує додаткову робочу силу, то постає органічна зміна капіталу
і зменшується його змінна складова частина; в усіх сферах
зростання змінної частини капіталу, а тому й числа занятих
робітників, завжди зв’язане з великими коливаннями й тимчасовим
утворенням перелюднення, однаково, чи набирає воно
помітнішу форму відштовхування занятих уже робітників, чи
менш помітну, але не менш дійову форму утрудненого поглинення
додаткової робітничої людности її звичайними відвідними каналами\footnote{
Перепис в Англії та Велзі показує, між іншим:

Всіх осіб, що працювали в рільництві (залічуючи сюди власників,
фармерів, садівників, пастухів і~\abbr{т. д.}), було 1851~\abbr{р.} \num{2.011.447}, 1861~\abbr{р.} —
\num{1.924.110}, зменшення — \num{87.337}. Вовняна мануфактура: 1851~\abbr{р.} — \num{102.714}
осіб, 1861~\abbr{р.} — \num{79.242}; шовкові фабрики: 1851~\abbr{р.} — \num{111.940}; 1861~\abbr{р.} —
\num{101.678}; ситцевибійники: 1851~\abbr{р.} — \num{12.098}, 1861~\abbr{р.} — \num{12.556}; це незначне
збільшення, поруч величезного поширення підприємства, означає
велике відносне зменшення числа занятих робітників. Капелюшники:
1851~\abbr{р.} — \num{15.957}, 1861~\abbr{р.} — \num{18.814}: виробники солом’яних і жіночих капелюхів:
1851~\abbr{р.} — \num{20.393}, 1861~\abbr{р.} — \num{18.176}; солодівники: 1851~\abbr{р.} —
\num{10.566}, 1861~\abbr{р.} — \num{10.677}; виробники свічок: 1851~\abbr{р.} — \num{4.949} осіб, 1861~\abbr{р.} —
\num{4.686}. Це зменшення є, між іншим, наслідок поширення газового освітлення.
Гребінники: 1851~\abbr{р.} — \num{2.038}, 1861~\abbr{р.} — \num{1.478}; пильщики 1851~\abbr{р.} —
\num{30.552}, 1861~\abbr{р.} — \num{31.647}, невеличке збільшення в наслідок розвитку
машин до пиляння; голкарі: 1851~\abbr{р.} — \num{26.940}, 1861~\abbr{р.} — \num{26.130}, зменшення
в наслідок машинової конкуренції; робітники по копальнях цинку й
міді: 1851~\abbr{р.} — \num{31.360}, 1861~\abbr{р.} — \num{32.041}. Навпаки, бавовнопрядні та бавовноткальні:
1851~\abbr{р.} — \num{371.777}, 1861~\abbr{р.} — \num{456.646}; кам’яновугляні копальні:
1851~\abbr{р.} — \num{183.389}, 1861~\abbr{р.} — \num{246.613}. «Взагалі збільшення числа робітників
після 1851~\abbr{р.} найбільше в таких галузях, де досі ще не застосовувано
машин з успіхом». («Census of England and Wales for 1861», vol. III.
London 1863, p. 35--39).
}.
Разом з величиною суспільного капіталу, що вже функціонує,
і з ступенем його зростання, з поширенням маштабу
продукції й маси робітників, пущених у рух, з розвитком про-
\parbreak{}  %% абзац продовжується на наступній сторінці
