\parcont{}  %% абзац починається на попередній сторінці
\index{i}{0510}  %% посилання на сторінку оригінального видання
політичної економії у східньоіндійському коледжі в Haileybury,
влучно пояснює це на двох великих фактах. Через те, що
найбільша частина індійського народу самостійні господарі-селяни,
то їхній продукт, їхні засоби праці й засоби існування
ніколи не існують «у формі («in the shape») фонду, який заощаджується
з чужого доходу («saved from Revenue»), а тому
й не перебігають вони попереднього процесу акумуляції» («а
previous process of accumulation».\footnote{
Там же, стор. 36. [До 4 вид. — Це, певно, недогляд, цього місця
не можна було знайти. — \emph{Ф. Е.}].
} З другого боку, в тих провінціях,
де англійське панування найменше зруйнувало стару
систему, нерільничих робітників вживають до праці безпосередньо
вельможі, до яких припливає частина сільського додаткового продукту
в формі данини або земельної ренти. Частину цього продукту
вельможі споживають у натуральній формі, другу частину
їхні робітники перетворюють для них на предмети розкошів
і всякі інші засоби споживання, а решта становить заробітну
плату робітників, які є власники своїх знарядь праці. Продукція
й репродукція в поширеному маштабі відбуваються тут без
жодного втручання того дивовижного святого, того лицаря сумної
постаті, — «нездержливого» капіталіста.

\subsection{Обставини, що визначають розмір акумуляції незалежно від
тієї пропорції, в якій додаткова вартість поділяється на капітал
і дохід: Ступінь експлуатації робочої сили. — Продуктивна
сила праці. — Зростання ріжниці між застосовуваним і споживаним
капіталом. — Величина авансованого капіталу}

Якщо припустити те відношення, що в ньому додаткова вартість
розпадається на капітал і дохід, за дане, то величина акумульованого
капіталу залежить, очевидно, від абсолютної величини
додаткової вартости. Коли припустити, що 80\% капіталізується,
а 20\% з’їдається, то акумульований капітал становитиме
\num{2.400}\pound{ фунтів стерлінґів} або \num{1.200}\pound{ фунтів стерлінґів} залежно
від того, чи становить ціла сума додаткової вартости \num{3.000}\pound{ фунтів
стерлінґів}, чи \num{1.500}\pound{ фунтів стерлінґів}. Тому при визначенні
величини акумуляції діють всі ті обставини, що визначають масу
додаткової вартости. [Обставини, що визначають величину додаткової
вартости, ми докладно розвинули в розділах про продукцію
додаткової вартости].\footnote*{
Заведене у прямі дужки ми беремо з другого німецького видання.
\emph{Ред.}
} Ми ще раз розглядаємо їх тут
разом, але лише остільки, оскільки вони щодо акумуляції дають
нам нові погляди.

Як ми собі пригадуємо, норма додаткової вартости залежить
насамперед від ступеня експлуатації робочої сили. Політична
економія так високо цінує цю ролю, що вона принагідно ідентифікує
прискорення акумуляції через збільшення продуктивної
сили праці з прискоренням її через збільшення експлуатації
\parbreak{}  %% абзац продовжується на наступній сторінці
