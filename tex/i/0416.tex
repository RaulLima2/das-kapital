ними запитами намагається спантеличити свідка й перекрутити
його слова. Тут у ролі адвокатів виступають сами члени парламентської
слідчої комісії, між ними власники й експлуататори копалень;
свідки тут робітники копалень, здебільша кам’яновугільних.
Ця ціла фарса надто характеристична для духу капіталу, так що
не можна не подати тут декілька витягів. Для легшого огляду
я подаю результати слідства й т. ін. за рубриками. Нагадую, що
питання і обов’язкові відповіді в англійських Синіх Книгах
нумеровані, і що свідки, чиї свідчення тут цитується, є робітники
кам’яновугільних копалень.

1. Праця дітей від 10 років по копальнях. Праця разом з
неминучим ходінням від і до копалень триває звичайно 14 —
15 годин, винятково довше, від 3, 4, 5 години ранку до 4—5 години
вечора (№№ 6, 452, 83). Дорослі робітники працюють двома
змінами, або по 8 годин, але для підлітків, щоб заощадити на
видатках, такої зміни нема (№№ 80, 203, 204). Малих дітей уживають
головно щоб відчиняти й зачиняти двері в різних відділах
копальні, а старших дітей — до тяжкоїроботи: перевозити вугілля
й т. ін. (№№ 122, 739, 1747). Довгий робочий день під землею
триває до 18 або 22 року життя, коли відбувається перехід до
власне копальневої праці (№ 161). Дітей і підлітків тепер тяжче
мордують працею, ніж колибудь у попередні часи (№№ 1663 —
67). Копальневі робітники майже одноголосно вимагають парляментського
закону про заборону копальневої праці для дітей,
молодших за 14 років. Але ось Гессей Вівіян (сам експлуататор
копальні) питає: «Чи не залежить ця вимога від більших або
менших злиднів батьків?» — А містер Брюс: «Чи це не жорстоко,
якщо батько помер або покалічений тощо, відбирати в родини
цей ресурс? Адже ця заборона мусить мати силу, як загальне
правило. Чи хочете ви підземну працю дітей до 14 років заборонити
в усіх випадках?» Відповідь: «В усіх випадках» (№№ 107
до 110). Вівіян: «А якщо працю дітей до 14 років по копальнях
заборонять, то чи не посилатимуть батьки дітей на фабрики
тощо? — Як правило, ні» (№ 174). Робітник: «Відчиняти й зачиняти
двері, здається, легко. Але це виснажна праця. Не кажучи
вже про постійний протяг, дитина сидить там немов у в’язниці,
цілком так, наче в темній тюремній камері». Буржуа Вівіян:
«А не може дитина, вартуючи при дверях, читати, якщо вона
матиме світло? — ІІоперше, вона мусила б купити собі свічку.
Але, крім того, їй цього і не дозволили б. Її поставили, щоб пильнувала
справи, вона має виконувати певний обов’язок. Я ніколи
не бачив, щоб якабудь дитина читала в копальні» (№№ 141—160).

2. Виховання. Копальневі робітники вимагають закона про
обов’язкове навчання дітей, як на фабриках. Вони заявляють,
що той пункт закону 1860 р., який вимагає шкільної посвідки для
того, щоб вживати до праці дітей 10—12 років, є чисто ілюзоричний.
«Педантична» процедура допитування капіталістичними
слідчими стає тут справді забавною. (№ 115). «Чи цей закон більше
потрібний проти підприємців, чи проти батьків? — Проти тих
