\parcont{}  %% абзац починається на попередній сторінці 
\index{i}{0407}  %% посилання на сторінку оригінального видання 
на їжу й сон!\footnote{
Там же, стор. З, п. 24.
} Велика частина з них не вміє читати, і звичайно
це — цілком здичавілі, анормальні істоти. «Щоб зробити їх
придатними до їхньої праці, не треба ніякого інтелектуального
виховання; в них мало нагоди, щоб надбати вправности, і ще
менше, щоб вправляти свій розум; їхня заробітна плата, хоч вона
до певної міри й висока для хлопчаків, не зростає в міру того, як
вони сами стають дорослими, і велика більшість із них не має
ніякої надії на дохідніше й відповідальніше місце доглядача
машини, бо на кожну машину припадає лише 1 доглядач і часто
аж 4 хлопчаки».\footnote{
Там же, стор.7, п. 60.
} Коли вони стають задорослі для своєї
дитячої праці, отже, мають щось близько 17 років, їх звільняють
із друкарні. Вони стають кандидатами на злочинців. Деякі спроби
дати їм деінше якусь роботу розбивалися об їхнє неуцтво, грубість,
фізичний та інтелектуальний занепад.

Те, що має силу щодо мануфактурного поділу праці всередині
майстерні, має силу й щодо поділу праці всередині суспільства.
Поки ремество й мануфактура становлять загальну основу суспільної
продукції, доти підпорядкування продуцента виключно
одній галузі продукції, зруйнування первісної різноманітности
його праці 305 є доконечний момент розвитку. На цій основі
кожна окрема галузь продукції емпірично знаходить відповідну
їй технічну форму, повільно вдосконалює її і швидко кристалізує,
скоро тільки досягнуто певного ступеня зрілости. Час від
часу стаються зміни, що їх викликає, поруч нового матеріялу
праці, постачуваного торговлею, також і поступінна зміна інструменту
праці. Але скоро тільки емпірично знайдено відповідну
форму робочого інструменту, то й він костеніє, як це доводить
його перехід, часто протягом тисячоліття, із рук однієї ґенерації
до другої. Характеристично, що до самого XVIII віку окремі
ремества називалися містеріями (mysteries, mystères),305 і їхніх
таємниць міг дійти лише той, хто був у них емпірично і професійно

304 «За Statistical Account у деяких частинах Горішньої Шотландії...
багатьох пастухів і cotters\footnote*{
— селян. Ред.
} з жінками й дітьми можна було бачити
в чоботях, пошитих ними з шкури власного виробу, в убраннях, що до
них не торкалася ніяка інша рука, крім їхньої власної, і що матеріял
для них вони сами настригли з овець, а льон для них сами обробляли.
Для виготовлення одежі ледве чи вживали вони будь-яких купованих
речей, за винятком шила, голки, наперстка й дуже небагатьох частин
залізного знаряддя, вживаного у ткацтві. Фарби добували сами жінки з
дерев, кущів, трав і т. ін.». (Dugald Stewart: «Works, ed. Hamilton»,
vol. VIII, p. 327—328).

305    У славнозвісній «Livre des métiers» Етьєна Буальо є, між іншим,
припис, щоб підмайстер, коли його приймали до майстрів, присягався
«любити по-братерському своїх братів по реместву, допомагати їм —
кожний у своєму реместві, тобто не виказувати добровільно таємниць
ремества і навіть — в інтересах усієї корпорації — не звертати уваги
покупців на вади у виробах інших ремісників з метою рекомендації свого
власного товару».
\parbreak{}  %% абзац продовжується на наступній сторінці
