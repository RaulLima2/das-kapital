\parcont{}  %% абзац починається на попередній сторінці
\index{i}{0374}  %% посилання на сторінку оригінального видання
попередніх періодів».\footnote{
«Reports etc. for 31 st October 1862», p. 79.

Додаток до другого видання. Наприкінці грудня 1871~\abbr{р.} фабричний
інспектор А. Редґрев в одній доповіді, яку він прочитав у Бредфорді, в
«New Mechanics’ Institution», сказав: «Що мене від деякого часу почало
вражати, так це зміна вигляду вовняних фабрик. Раніш вони були переповнені
жінками й дітьми, тепер здається, що машина виконує всю працю.
На мій запит один фабрикант дав мені таке пояснення: за старої системи в
мене працювало 63 особи; після заведення поліпшених машин я зменшив
число своїх рук до 33, а недавно, в наслідок нових великих змін, я зміг
зменшити їх з 33 до 13».
} У деяких випадках збільшення числа
занятих робітників часто є лише позірне, тобто воно завдячує не
поширенню фабрик, що вже ґрунтуються на машиновому виробництві,
а поступінному прилучуванню до них побічних галузей.
Наприклад, «збільшення числа механічних ткацьких варстатів
і фабричних робітників, занятих коло них, від 1838 до 1858~\abbr{рр.}
в (британській) бавовняній фабриці завдячує просто поширенню
цієї галузі промисловости; навпаки, по інших фабриках воно
завдячує тому, що там до варстатів на ткання килимів, стрічок,
полотна тощо, які до того часу пускали у рух силою людських
мускулів, почали прикладати парову силу».\footnote{
«Reports etc. for 31 st October 1856», p. 16.
} Отже, збільшення
числа цих фабричних робітників було просто виразом зменшення
загального числа вживаних робітників. Нарешті, ми тут цілком
залишаємо осторонь те, що скрізь, за винятком металевих фабрик,
підлітки (молодші за 18 років), жінки та діти становлять геть
переважну частину фабричного персоналу.

Однак зрозуміло, що, незважаючи на масу робітників, яких фактично
витискує та в можливості заступає машинове виробництво,
число фабричних робітників із зростанням самого машинового виробництва,
вираженим у збільшеному числі фабрик того самого
роду або в поширеному розмірі наявних фабрик, кінець-кінцем,
може бути більше за число витиснутих ними мануфактурних робітників
та ремісників. Припустімо, що тижнево вживаний капітал
у 500\pound{ фунтів стерлінґів} складається, приміром, за старого способу
продукції з \sfrac{2}{5} сталої та \sfrac{3}{5} змінної складової частини, тобто 200\pound{ фунтів стерлінґів} витрачається на засоби продукції, 300\pound{ фунтів
стерлінґів} — на робочу силу, скажімо, по 1\pound{ фунту стерлінґів}
на робітника. Зі заведенням машинового виробництва склад цілого
капіталу змінюється. Він розпадається тепер, наприклад, на \sfrac{4}{5}
сталої та \sfrac{1}{5} змінної складових частин, або на робочу силу витрачається
лише 100\pound{ фунтів стерлінґів}. Отже, дві третини вживаних
раніш робітників звільняється. Якщо це фабричне виробництво
поширюється і цілий ужитий капітал за інших незмінних умов
продукції зростає з 500 до \num{1.500}, дотепер уживатиметься 300 робітників
— стільки ж, скільки й перед промисловою революцією,
Якщо вжитий капітал зростає й далі до \num{2.000}, то вживатиметься
400 робітників, отже, на \sfrac{1}{3} більше, ніж за старого способу виробництва.
Число вжитих робітників абсолютно збільшилося на 100,
а відносно, тобто проти цілого авансованого капіталу, впало на 800,
\parbreak{}  %% абзац продовжується на наступній сторінці
