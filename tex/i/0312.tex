частина була з’єднана в мануфактурах, де, як уже раніш згадувано,
поділ праці панував з особливою точністю. Із збільшенням числа
винаходів і зростом попиту на нововинайдені машини щораз
більше розвивався, з одного боку, розклад фабрикації машин на
різноманітні самостійні галузі, а з другого боку — поділ праці
всередині машинобудівельних мануфактур. Отже, ми тут вбачаємо
в мануфактурі безпосередню технічну основу великої промисловости.
Мануфактура продукувала машини, за допомогою яких
велика промисловість знищила ремісниче та мануфактурне виробництво
в тих галузях продукції, які вона насамперед охопила.
Отже, машинове виробництво виросло стихійно на невідповідній
йому матеріяльній основі. На певному ступені розвитку машинове
виробництво само мусило зробити переворот у цій основі, яку воно
спочатку застало готового і потім далі виробляло в її старій формі,
та створити для себе нову базу, відповідну його власному способові
продукції. Як поодинока машина лишається карликовою, поки
її пускає в рух лише людина, як система машин не могла вільно
розвиватися, поки на місце рушійних сил, які вона застала, —
худоби, вітру, а то й води — виступила парова машина, так само
й велика промисловість була паралізована в цілому своєму розвитку
доти, доки характеристичний для неї засіб продукції,
сама машина, завдячувала своє існування особистій силі та особистій
вправності, а значить, залежала від розвитку мускулів,
гостроти зору та віртуозности рук, що з ними частинний робітник
у мануфактурі й ремісник поза нею орудували своїм карликовим
інструментом. Залишаючи осторонь подорожчання машин у
наслідок такого способу виникнення їх, — обставина, що, як свідомий
мотив, панує над капіталом, — поширення промисловости,
яка провадилася вже машиновим способом, та проходження машин
у нові галузі продукції лишалися таким чином цілком залежними
від зросту тієї категорії робітників, яка через напівмистецький
характер своєї праці могла збільшуватися тільки поступінно,
а не скоками. Але на якомусь певному ступені розвитку велика
промисловість стає і технічно в суперечність із своєю ремісничою
та мануфактурною основою. Збільшення розміру рухових
машин, передатного механізму та виконавчих машин, збільшення
складности, різноманітности і точної правильности складових
частин виконавчої машини в міру того, як вона відривається від
того ремісничого зразка, що спочатку цілком визначає її будову, і
дістає вільну форму, яку визначає тільки її механічне завдання,\footnote{
Механічний ткацький варстат у своїй першій формі складається
переважно з дерева, поліпшений, сучасний — із заліза. До якої міри стара
форма засобу продукції напочатку опановує його нову форму, показує,
між іншим, найповерховіше порівняння сучасного парового ткацького
варстату з давнім, або сучасних роздмухових пристроїв на ливарнях
з першим безпорадним механічним відродженням звичайного ковальського
міха, і, може, ще влучніше, ніж усе інше, перший льокомотив, що його
пробували збудувати ще перед винаходом теперішніх льокомотивів: у
}