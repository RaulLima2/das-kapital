вартість мусить бути її носієм. По-друге, припускається, що витрачено
лише доконечний за даних суспільних обставин продукції
робочий час. Отже, коли б було треба лише 1 фунта бавовни
на те, щоб випрясти 1 фунт пряжі, то на утворення 1 фунта пряжі
треба зужити лише 1 фунт бавовни. Так само стоїть справа і з
веретенами. Коли б капіталістові прийшла фантазія вживати
золотих веретен замість залізних, то до вартости пряжі зарахувалося
б усе ж лише суспільно-доконечну працю, тобто робочий
час, доконечний для продукції залізних веретен.

Тепер ми знаємо, яку частину вартости пряжі становлять
засоби продукції, бавовна й веретена; вона дорівнює 12 шилінґам,
або матеріялізації двох робочих днів. Отже, тепер ідеться про
ту частину вартости, яку праця самого прядуна додає до бавовни.

Ми маємо тепер розглянути цю працю зовсім з іншого погляду,
ніж підчас процесу праці. Там справа йшла про доцільну діяльність,
про перетворення бавовни на пряжу. Що доцільніша праця,
то краща пряжа, припускаючи, що всі інші умови лишаються
ті самі. Праця прядуна була специфічно відмінна від інших
родів продуктивної праці, і ця відмінність виявлялася суб’єктивно
й об’єктивно, в осібній меті прядіння, в осібному характері його
операцій, в осібній природі його засобів продукції, в осібній споживній
вартості його продукту. Бавовна й веретена служать
за засоби існування праці прядіння, але ж ними не можна зробити
нарізних гармат. Навпаки, оскільки праця прядуна є вартостетворча,
тобто становить джерело вартости, вона зовсім не відрізняється
від праці свердлівника гармат або, що в даному випадку
ближче до нас, від праці плянтатора бавовни й продуцента
веретен, яка зреалізована в засобах продукції пряжі. Лише через
цю тотожність продукція веретен, вирощування бавовни і прядіння
можуть становити тільки кількісно відмінні частини тієї
самої сукупної вартости, вартости пряжі. Тут вже йдеться не
про якість, не про властивості та зміст праці, а тільки про її
кількість. Цю останню вже дуже просто облічити. Ми припускаємо,
що праця прядіння є проста праця, пересічна суспільна
праця. Пізніш ми побачимо, що протилежне припущення ані
трохи не змінює справи.

Підчас процесу праці праця безупинно переходить із форми
неспокою (Unruhe) у форму буття, з форми руху у форму предметности
(Gegenständlichkeit). * Наприкінці однієї години рух
прядіння репрезентується в певній кількості пряжі, отже, певна
кількість праці, одна робоча година, є упредметнена в бавовні.
Ми кажемо: робоча година, тобто витрачання життєвої сили прядуна
протягом однієї години, бо праця прядіння має значення
тут лише остільки, оскільки вона є витрачання робочої сили,
а не тому, що вона є специфічна праця прядіння.

* У французькому виданні це речення подано так: «Підчас
процесу продукції праця безупинно переходить з динамічної форми в статичну
форму» («Pendant le procès de la production le travail passe sans
cesse de la forme dynamique à la forme statique»). Peд.
