\parcont{}  %% абзац починається на попередній сторінці
\index{i}{0228}  %% посилання на сторінку оригінального видання
себе збити з пантелику. «Після того, — оповідає Леонард Горнер, —
як я спробував примусити виконувати закон, розпочавши 10 процесів
у 7 різних судових округах, і лише в одному випадку найшов
підтримку в суддів\dots{} я вважаю за некорисні дальші переслідування
за оминання закону. Та частина закону, що її укладено
з метою створити одностайність у робочих годинах\dots{} вже не існує
більше в Ланкашірі. Так само я абсолютно не маю, як і мої помічники,
ніяких засобів, щоб упевнитися, що по тих фабриках, де
панує так звана Relaissystem, підлітків і жінок не примушують
працювати більш як 10 годин. Наприкінці квітня 1849 р. вже
114 фабрик у моїй окрузі працювали за цією методою, і число
їх останніми часами швидко зростає. Загалом же вони працюють
тепер 13\sfrac{1}{2} годин, від шостої години ранку до пів на восьму вечора;
в деяких випадках вони працюють 15 годин, від пів на шосту
ранкy до пів на дев’яту вечора».\footnote{
«Reports etc. for 30 th April 1849», p. 5.
} Вже у грудні 1848 р. Леонард
Горнер мав список 65 фабрикантів і 29 фабричних доглядачів,
які одноголосно заявляли, що жодна система контролю не може
за такої системи змін перешкодити поширенню якнайінтенсивнішої
надмірної праці.\footnote{
«Reports etc. for 31 st October 1849», p. 6.
} То тих самих дітей і підлітків переводять
із прядільні до ткальні й т. д., то протягом 15 годин їх
кидають (shifted) з однієї фабрики до однієї.\footnote{
«Reports etc. for 30 th April 1849», p. 21.
} Як можна контролювати
таку систему змін, «яка зловживає словом зміна, щоб із
безмежною різноманітністю перемішувати робочі руки, як карти,
і день-у-день так пересовувати години праці й відпочинку
різних осіб, що один і той самий повний асортимент рук ніколи
не працює разом на тому самому місці в той самий час»!\footnote{
«Reports etc. for 1 st December 1848», p. 95.
}

Але й залишаючи цілком осторонь дійсну надмірну працю,
ця так звана система змін була таким витвором фантазії капіталу,
що його ніколи не перевищив Фур’є у своїх гумористичних
нарисах «courtes séances»,\footnote*{
— коротких сеансів. \emph{Ред.}
} з тією лише ріжницею, що притягання
праці тут перетворилося на притягання капіталу. Подивімось
на ці схеми, утворені фабрикантами і прославлені добрячою
пресою як зразок того, «що можна зробити з розумною мірою
дбайливости й методичности» («what a reasonable degree of care
and method can accomplish»). Робочий персонал розділювано
іноді на 12--15 категорій, що знову раз-у-раз зміняли свої
складові частини. Протягом п’ятнадцятигодинного періоду фабричного
дня капітал притягав робітника то на 30 хвилин, то на
годину, потім знову відштовхував його, щоб знову притягти
його на фабрику й знов одштовхнути, ганяючи його то туди, то сюди
розрізненими шматками часу, але постійно не випускаючи його
із своїх рук, доки десятигодинну працю не буде цілком закінчено.
Як на театральній сцені, мали виступати ті самі особи навпереміну
в різних явах різних дій. Але, як актор належить до
\parbreak{}  %% абзац продовжується на наступній сторінці
