\parcont{}  %% абзац починається на попередній сторінці
\index{i}{0277}  %% посилання на сторінку оригінального видання
пружин, виготівник циферблятів, виготівник спіральних пружин,
робітник, що робить дірки для каменя та вставляє рубін, виготівник
стрілок, виготівник коробки для годинника, виготівник
шрубів, позолотник з багатьма підрозділами, як от, приміром,
колісник (виріб мідяних та сталевих коліщат знову поділено),
Triebmacher, Zeigenverkmacher, acheveur de pignon (закріплює
коліщата в належних місцях, полірує facettes і т. ін.), Zapfenmacher,
planteur de finissage (вставляє в механізм різні коліщата
та пружини), finisseur de barillet (вирізує зубці, поширює дірочки
до належного розміру та закріплює установку), Hemmungmacher
і, як підрозділ цієї галузі, виготівник циліндрів, виготівник
трибків, виготівник маятників, Raquettemacher (тобто виготівник
механізму, що реґулює годинник), planteur d’échappement
(Hemmungmacher у власному значенні); далі: repasseur de barillet
(виготовлює коробку для пружини та закріпляє її установку),
ґлянсувальник сталі, ґлянсувальник коліщат, ґлянсувальник
шруб, маляр цифр, Blattmacher (покриває мідь емалем),
fabricant de pendants (виготовлює лише кільця до годинникової
коробки), finisseur de charnière (вставляє мосяжевий штифт
всередину коробки й т. ін.), faiseur de secret (виготовлює пружину,
що відкриває кришку годинника), ґравер, ciseleur, полірувальник
годинникової коробки і т. д. і т. д., нарешті, repasseur, що
складає окремі частини годинника докупи та пускає годинника
в рух. Лише небагато частин годинника переходить через різні
руки, і всі ці membra disjecta збираються лише в руках того,
хто, кінець-кінцем, сполучає їх в один цілий механізм. Це
зовнішнє відношення готового продукту до його різнорідних
елементів тут, як і в подібних роботах, лишає комбінацію частинних
робітників у тій самій майстерні випадковою. Самі
частинні праці знов таки можуть провадитись як незалежні
одне від одного ремества, як от у кантоні Ваадт та Невшатель,
тимчасом як у Женеві, приміром, існують великі мануфактури
годинників, тобто існує безпосередня кооперація частинних робітників
під командою одного капіталу. І в останньому випадку
цифербляти, пружини й коробки рідко виготовлюють у самій
мануфактурі. Комбіноване мануфактурне виробництво зисковне
тут лише за виняткових умов, бо конкуренція поміж робітниками,
що хочуть працювати вдома, надзвичайно велика, роздрібнення
продукції на масу гетерогенних процесів мало дає змоги застосовувати
спільні засоби праці; крім того, при роздрібненій фабрикації
капіталіст заощаджує собі видатки на робітні приміщення
й т. д.\footnote{
Женева в 1854 р. випродукувала \num{80.000} годинників, що не складає
навіть і п’ятої частини продукції годинників кантону Невшатель.
Chaux-de-Fonds, що його можна розглядати як єдину мануфактуру годинників,
сам щорічно дає удвоє більше, ніж Женева. Від 1850 і до 1861 р.
Женева постачила \num{750.000} годинників. Див. «Report from Geneva on
the watch Trade» в «Reports by H. M. ’s Secretaries of Embassy and Legation
on the Manufactures, Commerce etc.». №6 1863. Якщо відсутність
зв’язку між процесами, на які розпадається продукція складних продуктів, сама по собі дуже утруднює перетворення таких мануфактур на
машинове виробництво великої промисловости, то при продукції годинників
сюди долучаються ще дві інші перешкоди: дрібність і тендітність
їхніх елементів та їхній люксусовий характер, отже і різноманітність їх,
наприклад, ліпші лондонські фірми протягом цілого року ледве чи виробляють
тузінь годинників, які були б подібні один до одного. Фабрика
годинників Vacheron and Constantin, що з успіхом вживає машин,
дає щонайбільше три-чотири відміни годинників, різних щодо величини
й форми.
} Однак становище й цих частинних робітників, які працюють
\index{i}{0278}  %% посилання на сторінку оригінального видання
хоч і вдома, але на капіталіста (Fabrikant, établisseur),
геть цілком відмінне від становища самостійного ремісника, що
працює для своїх власних клієнтів.\footnote{
В годинникарстві, в цьому клясичному прикладі гетерогенної
мануфактури, можна дуже докладно вивчити згадані вище диференціяцію
та спеціялізацію робочих інструментів, що випливають із розчленування
ремісничої праці.
}

Другий рід мануфактури, її закінчена форма, продукує
вироби, що перебігають зв’язані між собою фази розвитку, певний
ряд послідовних процесів, як от, приміром, дріт у голчаній
мануфактурі, який проходить через руки 72, а то й 92 специфічних
частинних робітників.

Оскільки така мануфактура комбінує ремества первісно розпорошені,
вона зменшує просторове відокремлення між окремими
фазами продукції виробу. Час на перехід його з однієї
стадії до одної скорочується, і так само зменшується праця, що
упосереднює ці переходи.\footnote{
«За такого тісного співжиття людей на транспортування неодмінно
мусить витрачатись менше часу» («In so close a cohabitation
of the People, the carriage must needs be less»). («The Advantages of the
East-India Trade», London 1720, p. 106).
} Порівняно з ремеством, таким способом
досягається вищої продуктивної сили, і цей виграш виникає
саме з загального кооперативного характеру мануфактури.
З другого боку, властивий мануфактурі принцип поділу праці
зумовлює ізоляцію різних фаз продукції, що усамостійнюються
одна проти одної як відповідна кількість частинних праць ремісничого
характеру. Встановлення і зберігання зв’язку поміж
ізольованими функціями вимагає постійного транспортування
виробу з одних рук до одних та з одного процесу до одного.
З погляду великої промисловости ця обставина виступає як характеристична
та іманентна принципові мануфактури обмеженість,
що удорожчує продукцію.\footnote{
«Ізоляція різних стадій мануфактури, що постає в наслідок вживання
ручної праці, надзвичайно збільшує витрати продукції, при чому
втрата виникає, головне, з самого лише переходу від одного процесу
до одного» («The isolation of the different stages of manufacture consequent
upon the employment of the manual labour adds immensely to
the cost of production, the loss mainly arising from the mere removals from
one process to another»). («The Industry of Nations», London 1855,
Part. II, p. 200).
}

Коли подивимось на певну кількість сировинного матеріялу,
приміром, ганчірок у паперовій мануфактурі або дроту в голчаній
мануфактурі, то побачимо, що цей матеріял перебігає в руках
різних частинних робітників почерговий щодо часу ряд фаз
\parbreak{}  %% абзац продовжується на наступній сторінці
