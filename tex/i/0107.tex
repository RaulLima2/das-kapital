Як свідомий носій цього руху посідач грошей стає капіталістом.
Його особа, або, скорше, його кишеня, є пункт, звідки виходять
і куди повертаються гроші. Об’єктивний зміст тієї циркуляції
— зростання вартости — є його суб’єктивна мета, і лише
остільки, оскільки дедалі більше присвоєння абстрактного багатства
є єдиним движним мотивом його операцій, — лише остільки
він функціонує як капіталіст або персоніфікований, наділений
волею й свідомістю капітал. Отже, споживну вартість ніколи не
можна розглядати як безпосередню мету капіталіста.\footnote{
«Товари (тут у значенні споживних вартостей) не є кінцева мета
капіталіста-купця... його кінцева мета є гроші» («Commodities are
not the terminating object of the trading capitalist... money is his terminating
object»). (Th. Chalmers: «On Political Economy etc.» 2 nd ed. London
1832, p. 165, 166).
} І не окремий
бариш, а лише безупинний рух одержування баришу є його
мета.8 Це абсолютне прагнення до збагачення, ця пристрастна
гонитва за вартістю9 є спільне в капіталіста і збирача скарбу,
але тимчасом як збирач скарбу є лише збожеволілий капіталіст,
капіталіст є раціональний збирач скарбу. Того безперестанного
зростання вартости, якого прагне збирач скарбу, рятуючи 10  гроші
від циркуляції, розумніший капіталіст досягає, віддаючи їх знову
й знов до циркуляції.\footnoteA{
«Ту безмежність, якої речі не осягають, рухаючись в однім напрямі,
вони осягають шляхом кругобігу» («Quell’ infinito che le cose
non hanno nella progressione, la hanno nel giro»). (Galiani, p. 156).
}

Ті самостійні форми, грошові форми, які вартість товарів

межне. Як кожна вмілість, що її мета має для неї значення не засобу,
а останньої кінцевої мети, є безмежна у своєму прагненні, бо старається
дедалі більше наблизитись до неї, тимчасом як умілості, що мають на оці
лише засоби для здійснення мети, не є безмежні, бо сама мета ставить їм
межі, так само для цієї хрематистики немає жодних меж у її меті, бо її
мета — це абсолютне збагачення. Економіка, а не хрематистика має
межу... перша має на меті щось відмінне від самих грошей, друга — лише
збільшення їх... Сплутування обох форм, що переходять одна в одну, дає
привід декому дивитись на збереження й безмірне збільшення грошей,
як на останню мету економіки». (Aristoteles: «De Republica», ed. Bekker,
lib. I, c. 8 und 9 passim).

8 «Хоч купець і не вважає за ніщо вже одержаний бариш, а все ж
завжди має на меті майбутній бариш» («Il mercante non conta quasi per
niente il lucro fatto, ma mira sempre al futuro»). (A. Genovesi: «Lezioni
di Economia Civile», 1765. Видання італійських економістів von Custodi,
Parte Moderna, vol. VIII, p. 139).

9 «Непогасна жага баришу, auri sacra fames, завжди характеризує
капіталіста». (Мас Culloch: «The Principles of Political Economy»,
London 1830, p. 179). Цей погляд, звичайно, не заважає тому самому
Мак Куллохові й К° при теоретичних труднощах, приміром, при аналізі
перепродукції, перетворити того самого капіталіста на доброго буржуа,
що для нього справа ходить лише про споживну вартість і що в нього
навіть розвивається дійсно вовчий апетит до чобіт, капелюхів, яєць,
перкалю й інших в якнайвищій мірі звичайних сортів споживної вартости.
10 «Σώζειν» (рятувати) — це один з найхарактеристичніших висловів
греків на означення скарботворення. Так само англійське «to save» значить
одночасно і «рятувати» і «зберігати».