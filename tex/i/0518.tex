того капіталу, до якого долучено працю, — цілком так само,
як суспільні продуктивні сили праці здаються його властивостями,
і так само, як постійне присвоювання додаткової праці
капіталістом здається постійним самозростанням вартости капіталу.
Всі сили праці здаються силами капіталу, як усі форми
вартости товару — формами грошей.

Зі зростанням капіталу зростає ріжниця між застосованим
і спожитим капіталом. Інакше кажучи, зростає вартість і речова
маса засобів праці, як от будівлі, машини, дренажні труби, робоча
худоба, апарати всякого роду, — засобів праці, що протягом
довшого або коротшого періоду функціонують в постійно повторюваних
процесах продукції, або служать, щоб досягти певних
корисних ефектів, у повному своєму обсягу, тимчасом як зужитковуються
вони лише поступінно і тому втрачають свою вартість
лише частинами, отже, і лише частинами переносять її на продукт.
У тій самій мірі, в якій ці засоби праці служать як продуктотворці,
не додаючи до продукту вартости, отже, застосовуються
цілком, а споживаються лише частинно, в цій самій мірі,

ном» праці й засобів продукції на продукт — і стане ясно як день, що
ми дістанемо тим більше мінової вартости, чим більше споживних вартостей
дає нам продукція. Іншими словами, що більше споживних вартостей,
наприклад, панчіх, дає один робочий день фабрикантові панчіх,
то багатший він на панчохи. Однак раптом Сеєві спадає на думку, що «зі
збільшенням кількости» панчіх їхня «ціна» (яка, природно, не має нічого
спільного з міновою вартістю) падає, «бо конкуренція примушує їх (продуцентів)
віддавати продукти за стільки, скільки вони їм коштують»
(«parce que la concurrence les (les producteurs) oblige à donner les produits
pour ce qu’ils leur coûtent»). Звідки ж береться зиск, коли капіталіст
продає товари за цінами, що їх вони йому коштують? Але облишмо
це. Сей заявляє, що в наслідок підвищеної продуктивности кожен дістав
тепер в обмін за той самий еквівалент дві пари панчіх замість однієї, як
це було раніш, і т. д. Результат, до якого він доходить, є саме та теза
Рікарда, яку він хотів був збити. Після такої величезної напруги думки
він, тріюмфуючи, звертається до Малтуза з такими словами: «Така, мій
пане, є добре пов’язана доктрина, що без неї, я це заявляю, неможливо
пояснити якнайбільші труднощі в політичній економії й особливо питання,
яким чином можливо, щоб нація стала багатшою тоді, коли вартість її
продуктів меншає хоч багатство і складається з вартостей» («Telle est,
monsieur, la doctrine bien liée sans laquelle il est impossible, le je déclare,
d’expliquer les plus grandes difficultés de l’économie politique et notamment,
comment il se peut qu’une nation soit plus riche lorsque ses produits
diminuent de valeur, quoique la richesse soit de la valeur»). (Там же, стор.
170). Один англійський економіст зауважує з приводу подібних фокусів
у «Листах» Сея: «Ці афектовані манери патякати («those affected ways
of talking») становлять y цілому те, що пан Сей залюбки називає своєю
доктриною і що він радить Малтузові викладати в Гертфорді, як це вже
робиться «в багатьох місцях Европи». Він каже: «Коли в усіх цих тезах
найдете дещо парадоксальним, то погляньте на ті речі, що їх ці тези виражають,
і я смію сподіватися, що вони здаватимуться вам дуже простими
й дуже розумними» («Si vous trouvez une physionomie de paradoxe â
toutes ces propositions, voyez les choses qu’elles expriment, et j’ose croire
qu’elles vous paraîtront fort simples et fort raisonnables»). Безсумнівно,
але в результаті того самого процесу вони видаватимуться всім, чим хочете,
та тільки не оригінальним або важливим». («An Inquiry mto those
Principles respecting the Nature of Demand etc.», p. 116. 110).
