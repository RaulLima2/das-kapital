тут можна сказати: «де є рівність, там немає баришу».18 Правда,
товари можуть продаватись за цінами, що відхиляються від їхніх
вартостей, але це відхилення являє собою порушення закону
обміну товарів.19 У своїй чистій формі обмін товарів є обмін
еквівалентів, отже, він не є засіб збагачуватися через збільшення
вартости.20

Тому за спробами виставити товарову циркуляцію як джерело
додаткової вартости криється здебільшого quid pro quo, переплутування
споживної й мінової вартости. Приміром, у Кондільяка:
«Це неправда, що при обміні товарів рівна вартість обмінюється
на рівну вартість. Навпаки, кожний з обох контраґентів завжди
віддає меншу вартість за більшу. Коли б дійсно люди завжди
обмінювали рівні вартості, то не було б жодного виграшу ні для
одного з контраґентів. Однак, обидва вони виграють абож повинні
вигравати. Але чому? Тому, що вартість речей полягає
лише в їхньому відношенні до наших потреб. Що для одного є
більше, для іншого є менше, і навпаки... Не можна припустити,
щоб ми подавали на продаж речі, доконечні для нашого споживання...
Ми хочемо віддати некорисну для нас річ, щоб одержати
доконечну нам річ; ми хочемо віддати менше за більше... Природно
було дійти до висновку, що при обміні рівну вартість віддається
за рівну вартість кожного разу, коли кожна вимінювана
річ за вартістю дорівнювала тій самій кількості грошей... Але
треба взяти до уваги ще й інший погляд: постає питання, чи не
обмінюємо ми обидва якийсь надлишок на дещо, доконечне для
нас».21 Ми бачимо, як Кондільяк не лише сплутує споживну
вартість з міновою вартістю, але й справді по-дитячому підсуває
суспільству з розвинутою товаровою продукцією такий стан речей,
за якого продуцент сам продукує свої засоби існування й подає
в циркуляцію тільки надлишок, надмір, що лишається після
задоволення власних потреб.22 А проте арґумент Кондільяка

18 «Dove è egualità, non è lucro». (Galiani: «Délla Moneta», y Custodi:
Parte Moderna, t. IV, p. 244).

19 «Обмін стає некорисним для однієї з сторін, коли якась побічна
обставина зменшує або збільшує ціну: тоді рівність порушується, але це
порушення постає з цієї причини, а не через обмін» («L’échange devient
désavantageux pour l’une des parties, lorsque quelque chose étrangère vient
diminuer ou exagérer le prix: alors l’égalité est blessée, mais la lésion procede
de cette cause et non de l’échange»). (Le Trosne: «De l’Intérêt Social»,
Physiocrates, éd. Daire, Paris 1846, p. 904).

20 «Обмін з природи своєї є договір рівности, коли за одну вартість
дають таку саму вартість, Отже, це не засіб для збагачення, бо тут
дають рівно стільки, скільки одержують». («L’échange est de sa nature
un contract d’égalité qui se fait de valeur pour valeur égale. Il n’est
donc pas un moyen de s’enrichir, puisque l’on donne autant que l’on
reçoit»). (Le Trosne: «De l’Intérêt Social», Physiocrates, éd. Daire, Paris
1846, p. 903).

21  Condillac: «Le commerce et le Gouvernement (1776). Edit. Daire
et Molinari y «Mélanges d’Economie Politique», Paris 1817, p. 267.

22  Тому ле Трон цілком слушно відповідає своєму приятелеві Кондільякові:
«У сформованому суспільстві не існує жодного надміру»
