світовому ринку. Як олень кричить за свіжою водою, так душа буржуа кричить за грішми, цим єдиним
багатством.\footnote{
«Цей раптовий перехід від кредитової системи до монетарної системи додає теоретичного страху до
практичної паніки: аґенти циркуляції здригаються перед нерозгадною тайною своїх власних відносин».
(К. Marx: «Zur Kritik der Politischen Oekonomie», Berlin 1859, S. 148. — K. Маркс: «До критики і т.
д.», ДВУ, 1926 р., стор. 157). «Бідний сидить без роботи, бо багатий не має грошей, щоб ужити його
до праці, хоч він має ту саму землю і ті самі робочі руки, щоб виробляти засоби існування й одежу,
як і раніш; але саме це й становить дійсне багатство нації, а зовсім не гроші» («The Poor stand
still, because the Rich have no Money to employ them, though they have the same land and hands to
provide victuals and cloaths, as ever they had; which is the true Riches of a Nation, and not the
Money»). (John Bellers: «Proposals for raising a Colledge of
Industry», London 1696, p. 3).
} Підчас кризи протилежність між товаром і формою його вартости, грішми, підноситься до
абсолютної суперечности. Тим то і форма виявлення грошей не має тут значення. Грошовий голод
залишається той самий, незалежно від того, чи треба платити золотом, чи кредитовими грішми,
приміром, банкнотами.\footnote{
Ось як використовують подібні моменти «amis du commerce»:\footnote*{
— друзі торговлі. Ред.
} «В одному з таких випадків (1839)
старий зажерливий банкір (із Сіті), піднявши верх бюрка, за яким сидів у своєму кабінеті, показав
своєму приятелеві жмути банкнот, заявивши з незвичайною радістю, що там 600.000 фунтів стерлінґів,
які він тримав у себе, щоб загострити потребу в грошах, але сьогодні після трьох годин він пустить
їх в обіг» («On one occasion (1839) an old grasping banker (der City) in his private room raised the
lid of the desk he sat over, and displayed to a friend rolls of banknotes, saying with intense glee
there were 600 000 £ of them, they were held to make money tight, and would all be let out after
three o'clock on the same day»). («The Theory of the Exchanges. The Bank Charter Act of 1814» ,
London 1864 p. 81). Напівофіціяльний орган «The Observer» з 24 квітня 1864 p. зауважує: «Поширюється
ряд дуже курйозних чуток про ті засоби, з яких користувалися, щоб утворити недостачу банкнот. Хоч
дуже сумнівна річ, чи справді вжито подібних трюків, проте зазначені чутки були остільки поширені,
що дійсно заслуговують на згадку». («Some very curious rumours are current of the means which have
been resorted to in order to create a scarcity of Banknotes... Questionable as it would seem, to
suppose that any trick of the kind would be adopted, the report has been so universal that really
deserves mention»).
}

Коли ми тепер розглянемо загальну суму грошей, які циркулюють протягом якогось даного часу, то
побачимо, що вона за даної швидкости обігу засобів циркуляції та засобів платежу дорівнює сумі тих
товарових цін, що їх треба зреалізувати, плюс сума платежів, що їм настав термін платежу, мінус сума
платежів, що урівноважується, мінус, нарешті, число обігів, у яких та сама монета функціонує
навпереміну то як засіб циркуляції, то як засіб платежу. Приміром, селянин продає своє збіжжя за 2
фунти стерлінґів, які таким чином функціонують як засіб циркуляції. Коли настає платіжний термін,
він віддає їх, щоб сплатити борг за полотно, яке постачив йому ткач. Тепер ці самі 2 фунти
стерлінґів функціонують як засіб платежу. Далі ткач купує біблію за готівку ; ці 2 фунти стерлінґів
тепер знов функціо-