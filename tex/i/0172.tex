ніяких чарів, коли нова вартість, яку він спродукував за 5\sfrac{3}{4} годин прядіння, дорівнює вартості
продукту однієї годинии прядіння. Але ви цілком помиляєтесь, думаючи, що він втрачає хоч один атом
часу свого робочого дня на репродукцію, або на «покриття» вартостей бавовни, машин і т. ін. Через
те, що його праця виробляє з бавовни й веретен пряжу, через те, що він пряде, вартість бавовни й
веретен сама собою переходить на пряжу.
Це завдячує якості його праці, а не її кількості. Певна річ, за одну годину він перенесе на пряжу
більшу вартість бавовни й т. ін., ніж за півгодини, але лише тому, що за одну годину він випряде
бавовни більше, ніж за півгодини. Отже, зрозумійте: коли ви кажете, що робітник за передостанню
годину продукує вартість своєї заробітної плати, а за останню годину — чистий прибуток, то це
означає лише те, що у пряжі, продукті двох годин його робочого дня, — однаково, чи є вони перші
години, чи останні, — втілено 11\sfrac{1}{2} робочих годин, тобто, саме стільки годин, скільки має цілий
його робочий день. А коли ви кажете, що він за перші 5\sfrac{3}{4} годин продукує свою заробітну плату, а за
останні 5\sfrac{3}{4} годин ваш чистий прибуток, то це означає знову лише те, що за перші 5\sfrac{3}{4} годин ви
платите, а за останні 5\sfrac{3}{4} годин
ви не платите. Я кажу про оплату праці замість казати про оплату робочої сили, для того, щоб
висловлюватись вашим жарґоном. Тепер, панове, коли ви порівняєте відношення того робочого
часу, який ви оплачуєте, до того робочого часу, який ви не оплачуєте, то побачите, що воно дорівнює
відношенню половини дня до половини дня, тобто 100\%, що безперечно є чималий рівень процента. І
немає найменшого сумніву, що, коли ви примусите ваші «руки» працювати 13 годин замість 11\sfrac{1}{2} і — що
до вас подібне, як дві краплини води між собою, — надмірні 1\sfrac{1}{2} години просто додасте до даткової
праці, то остання зросте з 5\sfrac{3}{4} годин до 7\sfrac{1}{4} годин, а тому норма додаткової вартости зросте з
100\% до 126\sfrac{2}{23}\%. Але ви занадто шалені санґвініки, коли сподівається, що через додаток 1\sfrac{1}{2} годин
вона піднесеться з 100 до 200\% і навіть більш ніж до 200\%, тобто «більш ніж подвоїться». З другого
боку, — серце людини є дивовижна річ, особливо коли людина носить своє серце в гаманці, — ви занадто
безглузді песимісти, коли боїтеся, що із скороченням робочого
дня з 11\sfrac{1}{2} на 10\sfrac{1}{2} годин піде за вітром увесь ваш чистий прибуток. Ні в якому разі. Припускаючи
інші обставини за незмінні,
додаткова праця спаде з 5\sfrac{3}{4} на 4\sfrac{3}{4} годин, а це дає все ще дуже значну норму додаткової вартости,
а саме 82\sfrac{14}{23}\%. Але та фатальна «остання година», про яку ви понарозповідали більше байок, ніж
хіліясти про кінець світу, є «all bosh».* Втрата її не відбере ані у вас «чистого прибутку», ані в
дітей обох статей, яких ви примушуєте працювати, «душевної чистоти».32а

32а Якщо Сеніор довів, що від «останньої робочої години» залежить чистий прибуток фабрикантів,
існування англійської бавовняної про-

* — цілковита дурниця. Ред.
