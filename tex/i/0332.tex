Продуктивність машин, як ми вже бачили, стоїть у зворотному
відношенні до величини тієї складової частини вартости, яку вона
переносить на продукт. Що довший той період, протягом якого
машина функціонує, то більша маса продуктів, на яку розділяється
додана нею вартість, і то менша та частина вартости
яку вона долучає до кожного окремого товару. Період активного
життя машини, очевидно, визначається довжиною робочого дня
або триванням денного процесу праці, помноженого на число
днів, у які цей процес повторюється.

Зужиткування машини зовсім не відповідає з математичною
точністю часові користування нею.\footnote*{
На берегах свого власного екземпляра першого видання Маркс
тут дає таку примітку: «Це має силу й щодо інших витрат, пов’язаних із
машинами. Наприклад: «Кожний фабрикант знає, що коли треба розігріти
парову машину, то здобути пару на 3 години коштує стільки ж саме,
скільки і на 4 години... Звідси постає (для залізниць) маленька економія
на паливі, коли перебіг робиться на велику віддаль». («Royal Commission
on Railways», London 1867. Evidence, p. 175). \emph{Ред.}
} Та навіть, коли припустити
таку відповідність, то й тоді машина, яка служить протягом
7\sfrac{1}{2} років по 16 годин щоденно, охоплює такий самий великий
період продукції та додає до загального продукту не більше
вартости, ніж та сама машина, що протягом 15 років служить
лише по 8 годин на день. Але в першому випадку вартість машини
була б репродукована удвоє швидше, ніж в останньому, і капіталіст
у першому випадку за допомогою цієї машини проковтнув би
за 7\sfrac{1}{2}  років стільки ж додаткової праці, скільки в другому випадку
за 15 років.

Матеріяльне зужиткування машини є двояке. Одно випливає
з її уживання — так само, як монети стираються від циркуляції,
друге — з невживання її, як от меч без ужитку ржавіє в піхвах,
В останньому випадку вона стає здобиччю стихій. Зужиткування
першого роду стоїть більше або менше у прямому відношенні,

часу на фабриках те велике іродське грабіжництво дітей, що його капітал
на податках фабричної системи практикував по домах для бідних та
сиріт, і за допомогою якого він здобув собі цілком безвільний людський
матеріял. Так, наприклад, Філден, сам англійський фабрикант, каже:
«Очевидно, робочий день здовжувала та обставина, що велика численність
безпритульних дітей, яких приводили з різних частин країни, унезалежнювала
підприємців від робочих рук, і вони, завівши за допомогою такого
нужденного, так добутого матеріялу, звичай довгої праці, дуже легко
могли накинути це і своїм сусідам» («It is evident that the long hours of
work were brought about by the circumstance of so great a number of destitute
children being supplied from different parts of the country, that the
masters were independent of the hands, and that, having once established
the custom by means of the miserable materials which they had procured in
this way, they could impose it on their neighbours with the greater facility»).
(J. Fielden: «The Curse of the Factory System», London 1836, p. 11).
Щодо жіночої праці фабричний інспектор Савндерс каже у фабричному
звіті за 1844 рік: «Серед робітниць є жінки, які багато тижнів один по
одному, за винятком лише небагатьох днів, працюють від 6 години ранку
до 12 години ночі, маючи менше ніж 2 годин на їжу, так що 5 днів на тиждень
у них із 24 годин лишається тільки 6 годин на те, щоб дійти додому
й назад та відпочити в ліжку».