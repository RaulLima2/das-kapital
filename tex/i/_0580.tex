\index{i}{0580}  %% посилання на сторінку оригінального видання
Коли візьмемо дані Артура Юнґа, точного спостерігача, але
поверхового мислителя, — коли візьмемо його дані про сільських
робітників року 1771, то побачимо, що ці останні відіграють дуже
мізерну ролю порівняно із своїми попередниками кінця XIV в.,
«коли вони могли шити в достатках і акумулювати багатство»,
не кажучи вже зовсім про XV століття, «золотий вік англійських
робітників міста й села». Однак ми не маємо потреби
вертатися так далеко назад. В одному дуже змістовному творі
з 1777 р. ми читаємо: «Великий фармер піднісся майже до рівня
джентельмена, тимчасом як бідний сільський робітник придушений
майже до землі... Його безталанне становище виявляється
цілком ясно, коли порівняти його становище тепер і сорок років
тому... Земельний власник і фармер спільно діють, щоб пригнобити
робітника».\footnote{
Reasons for the late Increase of the Poorrates; or, a comparative
view of the price of labour and provisions», London 1777, p. 5, 11.
} Далі там докладно показано, що реальна
заробітна плата на селі за час від 1737 до 1777 р. зменшилась
майже на \sfrac{1}{4},  або на 25\%. «Сучасна політика, — каже в той самий
час д-р Річард Прайс, — сприяє вищим клясам народу; наслідок
буде такий, що раніш або пізніш усе королівство складатиметься
лише з джентельменів і жебраків, ґрандів і рабів».\footnote{
Dr. Richard Price: «Observations cn Reversionary Payments»,
6 th ed. By W. Morgan. London 1805, vol. II, p. 158, 159. Ha crop. 159
Прайс зауважує: «Номінальна ціна робочого дня тепер хіба лише вчетверо
або, щонайбільше, вп’ятеро більша, ніж у 1514 р. Але ціна збіжжя тепер
усемеро більша, а м’яса й одягу — майже вп’ятнадцятеро. Отже, ціна
праці збільшувалась так непропорційно до зростання видатків на доконечні
засоби існування, що тепер робітник, мабуть, не може зробити й
половини тих витрат, що їх він робив раніш». («The nominal price of
day labour is at present no more than about four times, or at most five times
higher than it was in the year 1514. But the price of corn is seven times,
and of flesh-meat and raiment about fifteen times higher. So far, therefore,
has price of labour been even from advancing in proportion to the
increase in the expences of living, that it does not appear that it bears now
half the proportion to those expences that it did bear»).
}

А проте становище англійського сільського робітника років
1770—1780 так щодо його харчів і житлових умов, як і щодо
його почуття власної гідности, розваг тощо, є ідеал, якого пізніш
ніколи не досягнуто.

Його пересічна заробітна плата, виражена в пінтах пшениці,
становила в 1770—1771 рр. 90 пінт, за часів Ідна (1797 р.)
вже лише 65, а року 1808 лише 60 пінт.\footnote{
Barton: «Observations on the circumstances which influence the
condition of the Labouring Classes of Society», London 1817», p. 26. Для
кінця XVIII століття порівн. Eden: «The State of the Poor».
}

138 James Е. Th. Rogers, Prof, of Political Economy in the University
of Oxford: «А History of Agriculture and Prices in England», Oxford
1866, vol. I, p. 690. Цей старанно опрацьований твір у досі виданих двох
перших томах обіймає лише період віз 1259 до 1400 р. Другий том містить
лише статистичний матеріял. Це перша автентична «History of Prices»,\footnote*{
— історія цін. \emph{Ред.}
}
яку ми маємо для того часу.
