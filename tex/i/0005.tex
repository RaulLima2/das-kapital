чіша або невміліша людина, то вищу вартість матиме її товар,
бо то більше часу потребує вона для виготовлення товару. Але
праця, що становить субстанцію вартостей, є однакова людська
праця, затрата тієї самої людської робочої сили. Сукупна робоча
сила суспільства, що виражається у вартостях товарового світу,
має тут значення однієї і тієї самої людської робочої сили, дарма
що вона складається з безлічі індивідуальних робочих сил. Кожна
з цих індивідуальних робочих сил є така сама людська робоча
сила, як і всяка інша, оскільки вона має характер суспільної
пересічної робочої сили і функціонує як така суспільна пересічна
робоча сила, отже, оскільки вона на продукцію якогось товару
потребує лише пересічно доконечного або суспільно-доконечного
робочого часу. Суспільно-доконечний робочий час є робочий
час, потрібний на те, щоб виготовити якусь споживну вартість
при даних суспільно-нормальних умовах продукції і суспільному
пересічному ступені вмілости та інтенсивности праці. Приміром,
в Англії після заведення парового ткацького варстату,
щоб перетворити певну кількість пряжі в тканину, досить було,
може, вдвоє менше праці, ніж раніш. Англійський ручний ткач
на ділі потребував на це перетворення того самого робочого часу,
що й раніш, але продукт його індивідуальної робочої години
репрезентував тоді лише половину суспільної робочої години, і
через це його вартість зменшилась удвоє проти колишньої.

Отже, величина вартости якоїсь споживної вартости визначається
лише кількістю суспільно-доконечної праці або кількістю
робочого часу, суспільно-доконечного, щоб виготовити
її. 9 Кожен поодинокий товар вважається тут взагалі за пересічний
екземпляр свого роду. 10 Тому всі товари, що в них міститься
однакова кількість праці, або що їх можна виготовити протягом
однакового робочого часу, мають однакову величину вартости.
Вартість одного товару відноситься до вартости всякого іншого
товару, як робочий час, доконечний на те, щоб випродукувати
один товар, відноситься до робочого часу, доконечного на те,
щоб випродукувати всякий інший товар. «Як вартості, всі товари
є тільки певна маса застиглого робочого часу». 11

9 Примітка до другого видання: «Вартість засобів споживання, коли
їх обмінюється один на один, визначається кількістю праці, доконечно потрібної
і звичайно витрачуваної на продукцію їх» («The value of them
(the necessaries of life) when they are exchanged the one for another, is
regulated by the quantity of labour necessarily required, and commonly
taken in producing them»). («Some Thoughts on the Interest of Money in
general, and particulary in the Public Funds etc.», London, p. 36). Цей
вартий уваги анонімний твір минулого століття не датований. Однак
з його змісту видно, що він вийшов у світ за часів Ґеорґа II, приблизно
року 1739 або 1740.

10 «Всі продукти того самого роду становлять, власне кажучи, одну
масу, що її ціну визначається загально і не зважаючи на поодинокі обставини»
(«Toutes les productions d’un même genre ne forment proprement
qu’une masse, dont le prix se détermine en général et sans égard aux circonstances
particulières»). (Le Trosne: «De l’Intérêt Social», p. 893).

11    K. Marx: «Zur Kritik der Politischen Ökonomie», S. 6. (K. Маркс.
«До критики політичної економії». ДВУ, 1926 р., стор. 48).
