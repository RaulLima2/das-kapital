ється протилежне, під сталим капіталом, авансованим на продукцію
вартости, ми завжди розуміємо лише вартість зужиткованих
у продукції засобів продукції.

Припустивши таку передумову, повернімось до формули

С = с + υ, яка перетворюється на С' = с + υ + m, а саме через це
перетворення й С перетворюється на С'. Відомо, що вартість сталого
капіталу лише знов з’являється в продукті. Отже, вартість,
дійсно новоспродукована в процесі, є відмінна від цілої вартости
продукту, добутої в процесі, тим-то вона становить не
c + υ + m, або 410 фунтів стерлінґів + 90 фунтів стерлінґiв + 90 фунтів
стерлінґів, як то здається на перший погляд, а υ + m, або
90 фунтів стерлінґів + 90 фунтів стерлінґів, тобто не 590 фунтів
стерлінґів, а 180 фунтів стерлінґів. Коли б с, сталий капітал,
дорівнював 0, іншими словами, коли б існували такі галузі промисловости,
де капіталіст не мав би застосовувати жодних спродукованих
засобів продукції — ані сировинного матеріялу, ані
допоміжних матеріялів, ні знаряддя праці, а мав би застосовувати
тільки такі матеріяли, які існують з природи, і робочу
силу, тоді на продукт не переносилося б жодної частини сталої
вартости. Тоді цей елемент вартости продукту, в нашому прикладі
410 фунтів стерлінґів, відпав би, але новоспродукована вартість
у 180 фунтів, що містить у собі 90 фунтів стерлінґів додаткової
вартости, лишалася б цілком такого самого розміру, як коли б
с являло собою найбільшу суму вартости. Ми мали б С = 0 + υ = υ,
і С', вирослий у своїй вартості капітал, дорівнював би υ + m, а
С' мінус С, як і раніш, дорівнювало б m. Навпаки, коли б m дорівнювало
0, іншими словами, коли б робоча сила, вартість якої
авансується у змінному капіталі, продукувала лише еквівалент,
тоді б С = с + υ і С' (вартість продукту) = с + υ + 0, а тому С = С'.
Авансований капітал не зріс би своєю вартістю.

В дійсності ми вже знаємо, що додаткова вартість є лише наслідок
тієї зміни вартости, яка відбувається з υ, з частиною капіталу,
перетвореною на робочу силу, що, отже, υ + m = Δυ (υ плюс
приріст υ). Але дійсна зміна вартости й відношення, в якому
змінюється вартість, затемнюються тим, що в наслідок зростання
складової змінної частини капіталу зростає також і ввесь авансований
капітал. Він був 500, а стає 590. Отже, аналіза процесу в його
чистій формі вимагає, щоб ми цілком абстрагувалися від тієї
частини вартости продукту, в якій лише знов з’являється стала
капітальна вартість, тобто припустили, що сталий капітал с
дорівнює нулеві, і таким чином застосували той закон математики,
що ним вона оперує змінними й сталими величинами, коли
стала величина тільки через додавання або віднімання зв’язана
зі змінною.
