\index{i}{0474}  %% посилання на сторінку оригінального видання
В кожній країні є якась певна пересічна інтенсивність праці;
праця, що має інтенсивність нижчу від цієї пересічної, споживає
на продукцію товару часу більше за суспільно-доконечний,
і через це не вважається за працю нормальної якости. Тільки
такий ступінь інтенсивности, що підноситься понад національну
пересічну, зміняє в даній країні міряння вартости простим триванням
часу праці. Інакше справа стоїть на світовому ринку,
що його інтеґральними частинами є поодинокі країни. Пересічна
інтенсивність праці зміняється від країни до країни: тут вона
більша, там менша. Отже, ці національні пересічні становлять
скалю, що її одиницею міри є пересічна одиниця світової праці.
Отже, інтенсивніша національна праця порівняно з менш інтенсивною
продукує за однаковий час більше вартости, яка виражається
в більшій кількості грошей.

Але закон вартости в його інтернаціональному застосуванні
ще більше модифікується через те, що на світовому ринку продуктивнішу
національну працю також вважається за інтенсивнішу,
доки продуктивнішу націю конкуренцією не примушують
знизити продажну ціну її товару до його вартости.

В тій мірі, у якій в країні розвинута капіталістична продукція,
в тій самій мірі там підносяться національна інтенсивність і
продуктивність праці понад інтернаціональний рівень.\footnoteA{
В іншому місці ми дослідимо, які обставини можуть щодо продуктивности
праці зміните цей закон для поодиноких галузей продукції.
} Отже,
різні кількості товарів того самого роду, що їх у різних країнах
продукується за однаковий робочий час, мають неоднакові інтернаціональні
вартості, які виражаються в різних цінах, тобто
в різних грошових сумах, залежно від інтернаціональних вартостей.
Отже, відносна вартість грошей буде менша в нації з
розвинутішим капіталістичним способом продукції, ніж у нації
з менш розвинутим капіталістичним способом продукції. Звідси
випливає, що номінальна заробітна плата, тобто еквівалент робочої
сили, виражений у грошах, теж буде вища в першої нації,
ніж у другої: але це зовсім не означає, що те саме має силу і
для реальної плати, тобто і для суми засобів існування, даних
у розпорядження робітникові.

Але навіть залишаючи осторонь цю відносну неоднаковість
вартости грошей у різних країнах, ми все ж часто бачитимемо,
що поденна, потижнева і т. д. заробітна плата в нації з розвинутішим
капіталістичним способом продукції вища, ніж у нації
з менш розвинутим капіталістичним способом продукції, тимчасом
як відносна ціна праці, тобто ціна праці у відношенні
так до додаткової вартости, як і до вартости продукту, у другої
нації стоїть вище, ніж у першої.\footnote{
Джемс Андерсон у полеміці проти А. Сміса зауважує: «Варто
також взяти на увагу, що хоч позірна ціна праці звичайно нижча в бідних
країнах, де продукти землі і взагалі збіжжя дешеві, але в дійсності
реальна ціна праці там здебільша вища, ніж у інших країнах. Бо не та
плата, яку дістає робітник за день праці, становить дійсну ціну праці,
хоч вона й є її позірна ціна; справжня ціна — це те, чого фактично коштує
}

\index{i}{0475}  %% посилання на сторінку оригінального видання
Дж. В. Коуелл, член фабричної комісії 1833 р., дійшов після
старанного розсліду прядільництва того висновку, що «в Англії
заробітна плата в суті справи для фабрикантів нижча, ніж на
континенті, хоч для робітників може вона бути й вища». (Ure:
«Philosophy of Manufacture», p. 314). Англійський фабричний
інспектор Александер Редґрев у фабричному звіті з 31 жовтня
1866 р., порівнюючи статистику Англії з континентальними державами,
доводить, що, не зважаючи на нижчу заробітну плату
й далеко довший робочий час, континентальна праця у відношенні
до продукту дорожча, ніж англійська. Один англійський
директор (menager) бавовняної фабрики в Ольденбурзі заявляв,
що там робочий час триває від 5 години 30 хвилин ранку до
8 години вечора, не виключаючи й суботи, і що тамошні робітники,
працюючи під доглядом англійців, дають протягом того часу
менше продукту, ніж англійці за 10 годин, а працюючи під
доглядом німців, ще куди менше. Заробітна плата там далеко
нижча, ніж в Англії, в багатьох випадках на 50\%, але число
рук у відношенні до машин далеко більше, в різних відділах
відношення рук до машин є 5: 3. А. Редґрев подає дуже докладні
деталі про російські бавовняні фабрики. Ці відомості дав йому
один англійський директор, що недавно ще там працював. На
цьому російському ґрунті, такому родючому на всяку підлість,
розцвітають якнайповнішим квітом і старі страхіття з дитячого
періоду англійської фабрики. Директори, ясна річ, — англійці,
бо тубільний російський капіталіст нездатний до фабричного
підприємства. Не зважаючи на всю надмірну працю, невпинну
працю вдень і вночі та мізерну оплату робітників, російські фабрики
животіють лише завдяки забороні довозити закордонні фабрикати.
— Наприкінці я додаю ще порівняльний огляд п. Редґрева
щодо пересічного числа веретен на 1 фабрику й на 1 прядуна
по різних країнах Европи. Сам п. Редґрев зауважує, що ці
числа він зібрав перед кількома роками й що від того часу в Англії
зросли й розміри фабрик і число веретен, яке припадає на 1 робітника.
Але він припускає, що в перелічених країнах континенту
відбувався порівняно такий самий проґрес, так що подані
числа зберегли своє значення для порівняння.

підприємцеві певна кількість виготовлених продуктів, і розглянута з
цього погляду праця майже в усіх випадках є дешевша в багатших-країнах,
ніж у бідніших, хоч ціна збіжжя й інших засобів існування в останніх
звичайно значно нижча, ніж у перших\dots{} Праця, вимірювана поденно,
далеко дешевша в Шотландії, ніж в Англії\dots{} Відштучна праця звичайно
дешевша в Англії». («It deserves likewise to be remarked, that although the
apparent price of labour is usually lower in poor countries, where the
produce of the soil, and grain in general, is cheap; yet it is in fact for the
most part really higher than in other countries. For it is not the wages that
is given to the labourer per day that constitutes the real price of labour,
although it is its apparent price. The real price is that which a certain
quantity of work performed actually costs the employer; and considered
in this light, labour is in almost all cases cheaper in rich countries than in
those that are poorer, although the price of grain, and other provisions, is
usually much lower in the last than in the first\dots{} Labour estimated by

\index{i}{0476}  %% посилання на сторінку оригінального видання
Пересічне число веретен на одну фабрику

Англія\dotfill12.600
Швайцарія\dotfill8.000
Австрія\dotfill7.000
Саксонія\dotfill4.500
Бельґія\dotfill4.000
Франція\dotfill1.500
Прусія\dotfill1.500

Пересічне число веретен на одну особу

Франція\dotfill14
Росія\dotfill28
Прусія\dotfill37
Баварія\dotfill46
Австрія\dotfill49
Бельґія\dotfill50
Саксонія\dotfill 50
Дрібні німецькі держави\dots{}55
Швайцарія\dotfill55
Великобрітанія\dotfill74

«Це порівняння, — каже пан Редґрев, — крім інших причин, ще й тому особливо несприятливе для
Великобрітанії, що в ній є дуже багато фабрик, де машинове ткання сполучене з прядінням, тимчасом як
у цьому обчисленні не виключено жодного з ткачів. Навпаки, закордонні фабрики здебільша лише
прядільні. Коли б ми могли точно порівнювати рівне з рівним, то я міг би налічити в моїй окрузі
багато пряділень бавовни, де за мюлями з 2.200 веретенами наглядають лише одним-один робітник
(minder) з двома помічницями, які щодня продукують 220 фунтів пряжі 400 (англійських) миль
завдовжки». («Reports of Insp. of Fact. 31 st October 1866», p. 31--37 і далі).

Відомо, що в Східній Европі, так само як і в Азії, англійські компанії взялися будувати залізниці, і
при цьому вони побіч тубільних робітників вживали й певне число англійських робітників. Примушені
практичною доконечністю брати таким чином
на увагу національні ріжниці в інтенсивності праці, вони від того не зазнали ніякої шкоди. Їхній
досвід навчає, що коли розмір заробітної плати й відповідає більше або менше пересічній
інтенсивності праці, то відносна ціна праці (у відношенні до продукту) взагалі рухається у
протилежному напрямі.

У своєму «Дослідженні про норму заробітної плати»,\footnote{
«Essay on the Rate of Wages: with an Examination of the Causes
of the Differences in the Conditions of the Labouring Population throughout
the World», Philadelphia 1835.
} в одному із своїх найраніших економічних
творів, Г. Кері силку-

the day, is much lower in Scotland than in England\dots{} Labour by the piece
is generally cheaper in England»). (James Anderson: «Observations on
the means of exciting a spirit of National Industry etc.», Edinburgh 1777,
p. 350, 351). — Навпаки, низька заробітна плата з свого боку спричинюється
до подорожчання праці. «Праця дорожча в Ірляндії, ніж в Англії\dots{}
бо заробітна плата там відповідно нижча» («Labour being dearer in Ireland
than it is in England\dots{} because the wages are so much lower»).
(N. 2074 in «Royal Commission on Railways, Minutes. 1867»).
\index{i}{0477}  %% посилання на сторінку оригінального видання
ється довести, що різні національні заробітні плати просто пропорційні
ступеням продуктивности національних робочих днів,
щоб із цього інтернаціонального відношення зробити такий висновок,
що заробітна плата взагалі зростає й падає пропорційно
до продуктивности праці. Ціла наша аналіза продукції додаткової
вартости доводить безглуздість цього висновку, навіть тоді,
коли б Кері довів свій засновок, замість своїм звичаєм скидати
як попало до однієї купи некритично й поверхово назбираний
статистичний матеріял. Найкраще з усього є його твердження,
що справа в дійсності стоїть не так, як це повинно б бути на
основі теорії. Держава саме своїм втручанням перекрутила це природне
економічне відношення. Тому національні заробітні плати
треба обчисляти так, наче та частина їх, що припадає державі
у формі податків, припадала б самому робітникові. Чи не повинен
був би п. Кері далі подумати над тим, чи ці «державні витрати»
не є також «природні плоди» капіталістичного розвитку? Це міркування
цілком гідне людини, яка спершу проголосила капіталістичні
продукційні відносини за вічні закони природи й розуму,
що їхню вільну гармонійну гру порушує лише втручання держави,
а потім зробила відкриття, що диявольський вплив Англії
на світовому ринку, — вплив, що, як здається, не випливає з
природних законів капіталістичної продукції, — робить доконечним
втручання держави, тобто державний захист тих ваконів
природи й розуму, інакше кажучи, робить доконечним запровадження
протекційної системи. Далі він відкрив, що теореми
Рікарда й інших, у яких зформульовані існуючі суспільні протилежності
й суперечності, не є ідеальний продукт дійсного економічного
руху, а що, навпаки, дійсні протилежності капіталістичної
продукції в Англії та інших країнах є результат теорії
Рікарда й інших! Нарешті, він відкрив, що, кінець-кінцем,
торговля нищить природну красу й гармонію капіталістичної
продукції. Ще один крок далі, і він, можливо, зробить відкриття,
що єдине лихо капіталістичної продукдії — це сам капітал.
Тільки людина з такою жахливою некритичністю й такою фалшивою
(de faux aloi) вченістю могла заслужити собі, не вважаючи
на свою протекціоністичну єресь, того, щоб стати таємним джерелом
гармонійної премудрости Бастія й усіх інших сучасних
фритредерів-оптимістів.\footnote*{
У другому німецькому виданні тут є така примітка: «У четвертій
книзі я докладніше доведу поверховність його вчености». \emph{Ред.}
}
