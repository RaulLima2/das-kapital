нів азоту, — для жінок приблизно стільки поживних речовин,
скільки їх є у двох фунтах доброго пшеничного хліба, для чоловіка — на 1/9 більше; для дорослих
жінок і чоловіків — щонайменше
28.600 ґранів вуглецю і 1.330 ґранів азоту пересічно на тиждень.
Практика на диво ствердила його обчислення: воно цілком погоджувалося
з тією мізерною кількістю харчів, на яку нужденний
стан звів споживання бавовняних робітників. Вони одержували
в грудні 1862 р. 29.211 ґранів вуглецю і 1.295 ґранів азоту
на тиждень.

1863 р. Privy Council наказав розслідувати нужденний стан
тієї частини англійської робітничої кляси, що харчувалася найгірше.
Д-р Сімон, медичний урядовець Privy Council, вибрав
для цієї праці вищезгаданого доктора Сміса. Розслід його охоплює,
з одного боку, рільничих робітників, а з другого — шовкоткачів,
швачок, виробників шкуряних рукавичок, панчішників,
рукавичників і шевців. Останні категорії, за винятком панчішників,
є виключно міські робітники. При дослідженні дотримували
правила — вибирати в кожній категорії якнайздоровіші
й відносно забезпеченіші родини.

Загальний результат дослідження був той, що «тільки в
одній із досліджених кляс міських робітників кількість споживаного
азоту трохи перевищувала ту абсолютну мінімальну
міру, нижче якої постають недуги від голоду; у двох клясах
була недостача, а в одній з них навіть дуже велика недостача
у споживанні так азотових, як і вуглецевих харчів; з досліджених
рільничих родин більш як п’ятина одержувала вуглецевих
харчів менше, ніж доконечно, більш як 1/3 одержувала
азотових харчів менш, ніж доконечно, а в трьох графствах
(Berkshire, Oxfordshire і Somersetshire) пересічно панувала недостача
мінімальної кількости азотових харчів».\footnote{
«Public Health. 6 th Report etc. for 1863», London 1864, p. 13.
} Серед рільничих
робітників найгірші харчі діставали робітники Англії —
найбагатшої частини Об’єднаного королівства.\footnote{
Там же. стор, 17.
} Серед сільських
робітників недостатнє харчування припадало переважно
на жінок і дітей, бо «чоловік мусить їсти, щоб виконувати свою
роботу». Ще більша нужда лютувала серед досліджених міських
категорій робітників. «Вони харчуються так погано, що випадки
жорстокої нужди, яка руйнує здоров’я, мусять траплятися дуже
часто [все це є «поздержливість» капіталіста! а саме, він здержується
від того, щоб платити за засоби існування, доконечні
просто для животіння його «рук»!].\footnote{
Так же, стор. 13.
}

Нижченаведена таблиця дає можливість порівняти стан харчування
згаданих вище суто міських категорій робітників з
тим харчовим мінімумом, що його визначив д-р Сміс, і з
кількістю харчів бавовняних робітників за часів їхньої найбільшої
нужди: