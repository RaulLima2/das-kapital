Ми вже раніш відзначали становище сільських робітників
наприкінці антиякобінської війни, протягом якої так надзвичайно
позбагачувалися земельні аристократи, фармери, фабриканти,
купці, банкіри, біржові лицарі, постачальники до армії
й т.ін. Номінальна заробітна плата підвищилась почасти в наслідок
знецінення банкнот, почасти в наслідок, незалежного від
цього, зросту цін на найдоконечніші засоби існування. Але дійсний
рух заробітної плати можна сконстатувати дуже простим
способом, не вдаючись у непотрібні тут подробиці. Закон про
бідних і відповідна адміністрація були в 1814 р. ті самі, що і в
1795 р. Пригадаймо собі, як цей закон застосовувано на селі:
у формі милостині парафія доповняла номінальну заробітну
плату до номінальної суми, потрібної для простого животіння
робітника. Відношення між заробітною платою, що її платить
фармер, і тим дефіцитом її, що його поповнює парафія, показує
нам таке: поперше, наскільки заробітна плата впала нижче її
мінімуму, подруге, міру, в якій сільський робітник складався
з найманого робітника й павпера, або міру, в якій його перетворювано
на кріпака його парафії. Ми виберемо графство, яке
репрезентує пересічні умови всіх інших графств. 1795 р. пересічна
тижнева заробітна плата в Northamptonshire становила
7 шилінґів 6 пенсів, загальна сума річних видатків родини з
6 осіб — 36 фунтів стерлінґів 12 шилінґів 5 пенсів, загальна сума
її доходів — 29 фунтів стерлінґів 18 шилінґів, дефіцит, поповнюваний
парафією, — 6 фунтів стерлінґів 14 шилінґів 5 пенсів.
У тому самому графстві тижнева заробітна плата становила 1814 р.
12 шил. 2 пенси, загальна сума річних видатків родини з 5 осіб —
54 фунти стерлінґів 18 шилінґів 4 пенси, загальна сума її доходів
— 36 фунтів стерлінґів 2 шилінґи, дефіцит, поповнюваний
парафією, — 18 фунтів стерлінґів 6 шилінґів 4 пенси, 142 в 1795 р.
дефіцит становив менш ніж четвертину заробітної плати, в
1814 р. — більше, ніж половину. Само собою зрозуміло, що за
таких обставин зник 1814 р. і той невеличкий комфорт, що його
бачив Ідн у котеджі сільського робітника.143 З усіх тварин,
що їх тримає фармер, робітник, цей instrumentum vocale,* лишився
відтепер тією, яку мучать якнайбільше, годують якнайгірше і
з  якою поводяться щонайбрутальніше.

Такий стан речей спокійно тривав далі, доки «бурхливі повстання
1830 р. виявили нам (тобто панівним клясам) при світлі
підпалених скирт хліба, що під поверхнею рільничої Англії
злидні й глухе бунтівниче незадоволення палають так само буйно,
як і під поверхнею промислової Англії».144 Седлер охристив тоді
в палаті громад сільських робітників «білими рабами» («white
slaves»); з уст якогось єпископа пролунав цей самий епітет у па-

142 Parry: «The Question of the Necessity of the existing Cornlaws
considered», London 1816, p. 86.

143 Там же, стор. 213.

144 S. Laing: «National Distress», 1844, p. 62.

* — говорюще знаряддя. Ред.
