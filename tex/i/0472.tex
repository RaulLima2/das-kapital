понижують у тому самому відношенні, в якому зростає число
штук, випродукованих протягом того самого часу,\footnote{
«Продуктивну силу його прядільної машини точно вимірюють,
і норму плати за працю, виконану за допомогою машини, понижують
із зростом її продкутивної сили, хоч не в такій самій пропорції» («The
productive power of his spinning machine is accurately measured, and the
rate of pay for work done with it decreases with, though not as the increase
of its productive power»). (Ure: «Philosophy of Manufacture», p. 317).
Останній апологетичний зворот Юр сам знову касує. Він визнає, що при
здовженні мюлі jenny, наприклад, деяке збільшення Праці випливає
з того здовження. Отже, праця меншає не в тій самій мірі, в якій зростає
її продуктивність. Далі: «Через таке збільшення продуктивна сила машини
зростає на одну п’яту. В цьому випадку прядун уже не одержує
за свою працю тієї плати, яку він одержував раніш; але через те, що його
плата зменшується не в тій самій пропорції, тобто не на п’ятину, то це
вдосконалення машини збільшує його дохід за те саме число годин його
роботи» — але, але — «вищесказане потребує деякого виправлення...
прядун із своїх додаткових шістьох пенсів має зробити додаткові витрати
на збільшення числа малолітніх помічників, яке супроводиться витисненням
частини дорослих робітників» («By this increase the productive power
of the machine will be augmented one-fifth. When this event happens,
the spinner will not be paid at the same rate for work done as he was before»
but as that rate will not be diminished in the ratio of one-fifth, the improvement
will augment his money-earnings for any given membero! hours’work
— the foregoing statement requires a certain modification... the spinner
has to pay something additional for juvenile aid out of his additional
sixpence, accompanied by displacing a portion of adults») (там же, стор. 320,
321), а це ніяким чином не має тенденції підвищувати заробітну плату.
} отже, в тому
самому відношенні, в якому меншає робочий час, витрачуваний
на ту саму штуку. Ця зміна відштучної плати, хоч вона є суто
номінальна, викликає постійні бої між капіталістом і робітником:
або тому, що капіталіст використовує це як привід»,
щоб дійсно знизити ціну праці, або тому, що підвищення продуктивної
сили праці супроводиться підвищенням її інтенсивности,
абож тому, що зовнішню видимість відштучної плати
робітник приймає серйозно, вважаючи, що йому платять за його
продукт, а не за його робочу силу, і тому опирається такому
зниженню заробітної плати, якому не відповідає зниження продажної
ціни товару. «Робітники пильно стежать за ціною сировинного
матеріялу й ціною фабрикованих продуктів і таким чином
можуть точно визначити зиск своїх хазяїнів».\footnote{
Н. Fawcell: «The Economic Position of the British Labourer».
Cambridge and London 1865, p. 178.
} Такі претенсії
капітал по праву відкидає як грубу помилку щодо природи
заробітної плати.\footnote{
У лондонському «Standard’i» з 26 жовтня 1861 p. находимо звіт
про процес фірми Джон Брайт і К° перед рочдельським магістратом
«проти1 представників тред-юньйону килимарів за залякуіання»
(«to prosecute for intimidation the agents of the Carpet Weavers Trades
Union»). «Фірма Брайт завела нові машини, що мали виготовляти 240 ярдів
килимів за той самий час і з тією самою кількістю праці (І), яких
раніш потрібно було на виготовлення 160 ярдів. Робітники не мали жодного
права вимагати будь-яку частину того зиску, який створювався в наслідок
витрати капіталу їхніх підприємців на механічні поліпшення. Тому
пани Брайт запропонували понизити заробітну плату з 1 пенса на
1 пенс за ярд, при чому доходи робітників за виконану працю вони лишали
} Він обурюється проти цього домагання ро-