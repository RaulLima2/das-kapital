ту саму просторінь за коротший час. З другого боку, комбінацію
праці маємо й тоді, коли, приміром, будівлю розпочинають одночасно
з різних боків, хоч би кооперовані робітники робили те
саме або однорідне. Комбінований робочий день в 144 години,
який охоплює предмет праці з багатьох боків у простороні, бо
комбінований робітник або робітник колективний має очі й руки
спереду й ззаду і є до певної міри всюдисущий, — той робочий
день посуває наперед виготовлення цілого продукту швидше,
ніж 12 дванадцятигодинних робочих днів більш або менш відокремлених
робітників, які мусять братися до своєї праці однобічніше.
Просторово різні частини продукту таким чином вистигають
у той самий час.

Ми підкреслювали, що багато робітників, які один одного
доповнюють, роблять те саме або однорідне, бо ця найпростіша
форма спільної праці відіграє чималу ролю і в найрозвиненішій
формі кооперації. Якщо процес праці складний, то вже сама
маса тих, що спільно працюють, дозволяє розподіляти різні
операції поміж різних робітників, отже, і виконувати їх одночасно
та через це скорочувати робочий час, потрібний, щоб виготовити
цілий продукт.15

У багатьох галузях продукції бувають критичні моменти,
тобто визначувані самою природою робочого процесу періоди
часу, протягом яких мусять бути досягнені певні результати
праці. Коли, приміром, треба постригти ватагу овець або зжати
та звезти хліб із певного числа морґів, то кількість і якість продукту
залежить від того, щоб операція почалася в певний час

усі ту саму роботу, а все ж між ними є якийсь рід поділу праці, який полягає
в тому, що кожний з них переносить цеглу на певну віддаль, і що всі
вони приставляють її на місце призначення далеко швидше, ніж це було б,
якби кожний з них сам носив свою цеглу на те високе риштовання»
(«On doit encore remarquer que cette division partielle de travail peut se
faire quand même les ouvriers sont occupés d’une même besogne. Des maçons,
par exemple, occupés de faire passer de mains en mains des briques à un
échafaudage supérieur, font tous la même besogne, et pourtant il existe
parmi eux une espèce de division de travail, qui consiste en ce que chacun
d’eux fait passer la brique par un espace donné, et que tous ensemble la
font parvenir beaucoup plus promptement à l’endroit marqué qu’ils ne le
feraient si chacun d’eux portait sa brique séparement jusqu’à l’échafaudage
supérieur»). (F. Skarbek: «Théorie des richesses sociales», 2 éme éd.
Paris 1840, vol. I., p. 97, 98).

15 «Коли треба виконати складну працю, різні справи треба виконувати
одночасно. Один робить одне, тимчасом як другий робить друге,
і всі разом допомагають досягти результату, якого зовсім не могла б здійснити
одна людина. Один гребе, тимчасом як другий кермує стерном, а
третій закидає невід або б'є рибу бодцем, — і влови риби дають такий
результат, що був би неможливий без такого співробітництва». («Est-il
question d’exécuter un travail compliqué, plusieurs choses doivent être
faites simultanément. L’un en fait une pendant que l’autre en fait une
autre, et tous contribuent à l’effet qu’un seul homme n'aurait pu produire.
L’un rame pendant que l’autre tient le gouvernail, et qu’un troisième jette
le filet, ou harponne le poisson, et la pêche a un succès impossible sans
ce concours»). (Destutt de Tracy: «Traité de la Volonté et de ses effets»,
Paris 1826, p. 78).
