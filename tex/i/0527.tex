а не своїм власним здібностям, які ані трохи не кращі, ніж в
інших; не володіння землею і грішми, а панування над працею
(«the command of labour») відрізняє багатих від бідних... Бідним
відповідає не стан занедбаности або рабства, а догідний і
ліберальний стан залежности («а state of easy and liberal dependence»),
а людям, що мають власність, відповідає достатній вплив
і авторитет над тими, що на них працюють. Такий стан залежности,
як це знає кожен знавець людської натури, доконечний
для вигоди самих робітників».\footnote{
Eden: «The State of the Poor, or an History of the Labouring Glasses
in England», vol. I, b. 1, ch. 1, p. 1, 2 і передмова, p. XX.
} Сер Ф. М. Еден, до речі зауважити,
є єдиний учень Адама Сміса, що протягом XVIIІ століття
зробив дещо важливе.75

75    Якщо читач нагадає нам Малтуза, що його «Essay on Population»
появився 1798 р., то я нагадаю, що ця праця у своїй першій формі є не
що інше, як по-школярському поверховий і по-попівському пишномовний
пляґіят з деФо, Сера Джемса Стюарта, Тавнсенда, Франкліна, Уоллеса
та інших і не має в собі ані однісінької самостійно продуманої тези. Велика
сенсація, яку викликав цей памфлет, пояснюється виключно партійними
інтересами. Французька революція знайшла в Брітанському королівстві
палких оборонців; «принцип залюднення», що повільно вироблявся у
XVIII віці та що його потім підчас великої соціальної кризи під звуки
сурм і барабанний бій проголосили як непомильну протиотруту супроти
теорій Кондорсе й інших, англійська олігархія вітала з великою радістю,
вбачавши в ньому великого гасителя всіх прагнень до дальшого розвитку
людства. Малтуз, надзвичайно здивований своїм успіхом, заходився тоді
коло того, щоб стару схему заповнити поверхово скомпільованим матеріялом
і додати до нього новий, не Малтузом відкритий, а ним лише
присвоєний. — До речі зауважимо тут, що хоч Малтуз був попом англіканської
церкви, а все ж він дав чернечу обітницю на безженство.
Саме це — одна з умов членства (fellowship) в протестантському кембріджському
університеті. «Ми не дозволяємо, щоб члени колегій були
жонаті. Хто ожениться, той повинен зараз вийти з членів колегії» («Socios
collegiorum maritos esse non permittimus, sed statim postquam quis
uxorem duxerit, socius collegii desinat esse»). («Reports of Cambridge University
Commission», p. 172). Ця обставина вигідно відрізняє Малтуза
від інших протестантських попів, що відкинули католицьку заповідь
попівського безженства і в такій мірі засвоїли заповідь «плодіться й
розмножуйтеся» як свою специфічну біблійну місію, що вони повсюди
в справді непристойних розмірах допомагають збільшувати людність,
тимчасом як робітникам вони проповідують «принцип залюднення».
Характеристично, що економічна пародія гріхопадіння, адамове яблуко,
«непереможне бажання», «перепони, що силкуються притупити стріли
Купідона» («urgent appetite», «the checks which tend to blunt the shafts
of Cupid»), як весело каже піп Тавнсенд, що цей дражливий пункт монополізували
й тепер монополізують пани представники протестантської
теології або, точніше, церкви. За винятком венеціянського ченця Ортеса,
оригінального й талановитого письменника, більшість проповідників
принципу залюднення — це протестантські попи. Такий, наприклад,
Брукер, що в йоги творі «Théorie du Système animal», Ley de 1767 вичерпано
всю сучасну теорію залюднення, до якої подала ідеї короткочасна
суперечка між Кене і його учнем Мірабо-батьком на цю саму тему,
потім ідуть піп Уоллес, піп Тавнсенд, піп Малтуз і його учень, архіпіп
Т. Чалмерс, не кажучи вже про дрібніших попів-писак того самого напряму.
Первісно над політичною економією працювали філософи, такі,
як Гоббс, Лок, Юм, комерційні й державні люди, як от Томас Мор, Тімпл,
Сюллі, деВітт, Норт, Ло „Вандерлінт, Кантільйон, Франклін, а особливо
над теорією її з великим успіхом працювали медики, як от Петті, Барбон,
