дів, — це їхні державні борги.\footnoteA{
Вільям Коббет зауважує, що в Англії всі громадські установи
називаються «королівськими», але борг зате там є «національний» (national debt).
} Тому цілком послідовна є та сучасна доктрина, що народ стає тим
багатший, чим більше він заборговується. Державний кредит стає символом віри капіталу. І з
виникненням державної заборгованости місце гріха проти
святого духа, що за нього немає прощення, заступає зламання довіри до державного боргу.

Державний борг стає за одну з найсильніших підойм первісної акумуляції. Немов доторкаючись чарівною
паличкою, він наділяє непродуктивні гроші продуктивною силою й перетворює їх таким чином на капітал,
не потребуючи при тому виставляти
їх на небезпеку та самому зазнавати турбот, нерозривно зв’язаних з вкладанням грошей у промислові
підприємства й навіть у лихварські операції. Державні кредитори в дійсності не дають нічого, бо
позичені суми перетворюються на легко переказувані
боргові посвідки, які функціонують у їхніх руках цілком так само, як коли б це була така сама сума
готівки. Але державний борг не тільки створив таким чином клясу нероб-рантьє і імпровізоване
багатство тих фінансистів, що відіграють ролю посередників
поміж урядом і нацією — а також багатство відкупників податків, купців і приватних фабрикантів, що
їм значна частина кожної державної позики робить послугу як капітал, наче з неба спалий. Державний
борг, крім того, викликав акційні товариства,
торговлю всякими цінними паперами, ажіотаж, одно слово, біржову гру й сучасну банкократію.

З самого зародження свого великі банки, прикрашені національними титулами, були лише товариствами
приватних спекулянтів, що ставали на бік урядів і, завдяки одержаним привілеям, були спроможні
позичати їм гроші. Тому для акумуляції
державного боргу немає вірнішого мірила, ніж послідовне підвищення акцій цих банків, повний розквіт
яких починається з моменту заснування Англійського банку (1694 р.). Англійський банк почав з того,
що позичав урядові свої гроші з 8\%; одночасно
парлямент уповноважив його карбувати гроші з того самого капіталу, ще раз позичаючи його публіці у
формі банкнот. Цими банкнотами він міг дисконтувати векселі, давати позики під товари й закуповувати
благородні металі. Минуло небагато
часу, і ці кредитові гроші, зфабриковані самим банком, стали готівкою, що нею Англійський банк
видавав позики державі і сплачував коштом держави проценти від державних позик. Мало того, що банк
давав однією рукою, щоб одержати більше другою; навіть і тоді, коли він одержував, він лишався
вічним кредитором нації на всю віддану суму до останнього шага. Помалу він став також і доконечним
сховищем металевих скарбів країни й центром тяжіння для всього торговельного кредиту.
В той самий час, коли в Англії перестали палити відьом, там почали вішати фалшівників банкнот. Яке
вражіння справила на