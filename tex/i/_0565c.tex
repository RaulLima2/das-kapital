
\index{i}{0565}  %% посилання на сторінку оригінального видання
\begin{center}
\noindent\begin{small}
\begin{tabularx}{\textwidth}{@{}Xrrr@{}}
    \toprule
    \makecell[l]{Обидві статі} &
    \makecell[r]{Пересічна \\ тижнева \\к-ть вуглецю \\ (ґранів)} &
    \makecell[r]{Пересічна \\ тижнева \\к-ть азоту \\ (ґранів)} \\
    \midrule

    П’ять міських галузей промисловости\dotfill{} & \num{28.876} & \num{1.192} \\
    Безробітні фабричні робітники Ланкашіру\dotfill{} & \num{28.211} & \num{1.295} \\
    \makehangcell{Мінімальна кількість, запропонована
    для ланкашірських робітників при
    рівному числі чоловіків і жінок\dotfill{}}
    & \num{28.600}  & \num{1.330}\hang{l}{\footnote{Там же, додаток, стор. 232.}}
\end{tabularx}
\end{small}
\end{center}

\noindent{}Половина, \sfrac{60}{125}, із досліджених категорій промислових робітників
зовсім не споживала пива, 28\% — молока. Пересічна
тижнева кількість рідких поживних речовин коливалася від
7 унцій на родину в швачок до 24 унцій у панчішників. Більшість
тих, що не споживали молока, складалася з лондонських
швачок. Кількість споживаного на тиждень хліба коливалася
від 7\sfrac{3}{4} фунтів у швачок до 11\sfrac{1}{4} фунтів у шевців і становила пересічно
9,9 фунта на тиждень на дорослого. Кількість цукру (сиропу
й~\abbr{т. ін.}), коливалася від 4 унцій на тиждень у виробників
шкуряних рукавичок до 11 унцій у панчішників; ціла пересічна
кількість на тиждень для всіх категорій — 8 унцій на одного
дорослого. Загальна пересічна кількість масла (жиру й~\abbr{т. ін.})
на тиждень — 5 унцій на дорослого. Пересічна тижнева кількість
м’яса (сала й~\abbr{т. ін.}) на дорослого коливалася від 7\sfrac{1}{4} унцій у шовкоткачів
до 18\sfrac{1}{4} унцій у виробників шкуряних рукавичок;
загальна пересічна кількість для різних категорій — 13,6 унцій.
Щотижнева витрата на харчі для дорослих становила такі загальні
пересічні числа: шовкоткачі — 2\shil{ шилінґи} 2\sfrac{1}{2}\pens{ пенса},
швачки — 2\shil{ шилінґи} 7\pens{ пенсів}, виробники шкуряних рукавичок
— 2\shil{ шилінґи} 9\sfrac{1}{2}\pens{ пенсів}, шевці — 2\shil{ шилінґи} 7\sfrac{3}{4}\pens{ пенсів},
панчішники — 2\shil{ шилінґи} 6\sfrac{1}{4}\pens{ пенсів.} Для шовкоткачів з Macclesfield’у
тижнева пересічна кількість становила лише 1\shil{ шилінґ}
8\sfrac{1}{2}\pens{ пенсів.} Найгірше харчувалися швачки, шовкоткачі й виробники
шкуряних рукавичок\footnote{
Там же, стор. 232. 233.
}.

У своєму загальному санітарному звіті д-р Сімон каже про
цей стан харчування таке: «Кожний, хто обізнаний з медичною
практикою серед бідних або з пацієнтами шпиталів, однаково,
чи живуть вони по шпиталях, чи поза ними, потвердить, що випадки,
коли недостача харчів породжує або загострює недуги,
дуже численні\dots{} Однак із санітарного погляду сюди долучається
ще інша, дуже важлива обставина. Треба пригадати собі, що
позбавлення харчових засобів терпиться лише з великим опором
і що, звичайно, дуже недостатнє харчування є лише наслідок
\parbreak{}  %% абзац продовжується на наступній сторінці
