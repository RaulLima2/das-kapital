існування, отже, родючість ґрунту, багаті рибою води й т. ін.,
і природне багатство на засоби праці, як от водоспади, судноплавні
річки, дерево, металі, вугілля й т. ін. На початках культури
має переважне значення перша форма природного багатства,
на вищих ступенях розвитку — друга форма. Порівняйте,
наприклад, Англію з Індією, або — в античному світі — Атени
й Корінт з країнами на узбережжі Чорного моря.

Що менше число природних потреб, які абсолютно треба
задовольняти, і що більша природна родючість ґрунту та сприятливість
підсоння, то менший робочий час, доконечний для утримання
й репродукції продуцента. Отже, то більший може бути й
надлишок його праці на інших супроти його праці на самого себе.
Так, уже Діодор зауважує про давніх єгиптян: «Просто неймовірно,
як мало праці й витрат коштує їм виховання їхніх дітей.
Вони варять для них першу-ліпшу просту страву; дають їм
їсти й долішню частину папіруса, яку можна присмажити на
вогні, та коріння й стебла болотяних рослин, почасти сирі, почасти
варені й печені. Діти здебільша ходять без взуття й одягу,
бо повітря там дуже м’яке. Тому дитина коштує своїм батькам,
доки виросте, в цілому не більше, як двадцять драхм. Цим головне
й можна пояснити, що в Єгипті така численна людність,
у наслідок чого й можна було збудувати там такі великі споруди».\footnote{
Diodorus Siculus: «Bibliotheca historica», lib. I, c. 80.
}
Однак великі споруди давнього Єгипту своє існування
завдячують менше чисельності його людности, ніж тій обставині,
що відносно великою частиною людности можна було порядкувати
для цієї справи. Як індивідуальний робітник може давати
тим більше додаткової праці, чим менший його доконечний робочий
час, цілком так само чим менша частина робітничої людности,
потрібна для продукції доконечних засобів існування,
тим більша та її частина, якою можна порядкувати для іншої
справи.

Скоро капіталістичну продукцію дано як передумову, то, за
інших незмінних обставин і за даної довжини робочого дня, величина
додаткової праці буде змінюватися залежно від природних
умов праці, особливо ж залежно від родючости ґрунту. Але звідси
ні в якому разі не випливає протилежне, а саме те, що найродючіший
ґрунт є найвідповідніший для зростання капіталістичного
способу продукції. Цей спосіб припускає панування людини над
природою. Занадто марнотратна природа «водить людину, як
дитину, на мотузочку». Вона не робить власний розвиток людини
природною доконечністю.\footnote{
«Перше (природні багатства), будучи найсприятливішим і найкориснішим,
робить народ безтурботним, чванливим та схильним до всяких
надмірностей, тимчасом як друге приневолює до дбайливости, науки,
мистецтва та розумної політики» («The first (natural wealth), as it
is most noble and advantageous, so doth it make the people careless, proud,
and given to all excesses; whereas the second enforceth vigilancy, literature,
arts and policy»). («England’s Treasure by Foreign Trade. Or the Balance
of our Foreign Trade is the Rule of our Treasure. Written by Thomas Mun,
} Не тропічне підсоння з його буйною