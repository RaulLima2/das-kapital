продукту, то межу вживання їх дано тим, що їх власна продукція
коштує менше праці, ніж праця, замінювана вживанням машин.
Однак, для капіталу ця межа визначається вужче. Через те, що
він оплачує не працю, якої вжито, а вартість ужитої робочої
сили, то для нього вживання машини обмежується ріжницею
між вартістю машини й вартістю тієї робочої сили, яку машина
заміняє. А що поділ робочого дня на доконечну працю й додаткову
працю по різних країнах є різний, так само як і в тій самій
країні він різний в різні періоди або в той самий період у різних
галузях продукції; що, далі, дійсна заробітна плата робітника
то падає нижче вартости його робочої сили, то підноситься понад
неї, то ріжниця між ціною машин і ціною робочої сили, яку ці
машини мають замінити, може дуже коливатися, навіть і тоді,
коли ріжниця між кількістю праці, потрібної для продукції
машини, і загальною кількістю праці, яку вона заміняє, лишається
без зміни.116а  Але лише перша ріжниця визначає для самого
капіталіста витрати продукції товару та впливає на нього через
примусові закони конкуренції. Тим то в Англії нині винаходять
машини, яких уживають лише в Північній Америці, як у XVI
та XVII віці Німеччина винаходила машини, що їх уживала
лише Голляндія, і як деякі французькі винаходи XVIII віку
використовувано лише в Англії. Сама машина продукує в давніше
розвинених країнах через те, що її вживають у деяких галузях
підприємства, такий надмір праці (redundancy of labour, каже
Рікардо) по інших галузях, що тут падіння заробітної плати нижче
вартости робочої сили перешкоджає вживанню машин та робить
його зайвим, а часто й неможливим з погляду капіталу, зиск
якого і без того випливає із зменшення не просто вживаної ним
праці, а лише праці, ним оплаченої. По деяких галузях англійської
вовняної мануфактури останніми роками дитяча праця дуже
зменшилася, подекуди її майже витиснено. Чому? Фабричний
закон примусив до подвійної зміни дітей, що з них одна працює
6 годин, друга 4 години, або кожна лише по 5 годин. Але батьки
не хотіли продавати half-times (робітників половинного часу)
дешевше, ніж раніш продавали full-times (робітників повного
часу). Звідси заміна half-times машинами.117 Перед забороною
вживати жіночої та дитячої (нижче десятирічного віку) праці по

116а  Примітка до другого видання. Тим то в комуністичному суспільстві
вживання машин мало б зовсім інший обсяг, ніж у суспільстві
буржуазному.

117 «Підприємці не стануть без доконечности тримати дві зміни дітей
молодших за тринадцять років... Справді, одна кляса фабрикантів, що
прядуть вовну, тепер рідко вживає дітей молодших за 13 років, тобто
half-times. Вони позаводили нові машини та поліпшення різного роду,
які майже усувають працю дітей (тобто дітей до 13 років). Для ілюстрації
такого зменшення числа дітей я нагадаю про один процес праці, в якому
через додаток до тодішніх машин одного апарату, так званого piecing
machine, працю шістьох або чотирьох half-times, відповідно до особливости
кожної машини, може виконати один підліток (старший за
13 років). Система половинного часу» стимулювала «винахід присукуваль-
