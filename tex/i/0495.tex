новою працею протягом даного року створила капітал, що наступного
року вживатиме до праці додаткових робітників. Оце
й є те, що зветься: «утворювати капітал капіталом».

За передумову акумуляції першого додаткового капіталу в
2.000 фунтів стерлінґів була сума вартости в 10.000 фунтів стерлінґів,
авансована капіталістом і належна йому силою його
«первісної праці». Навпаки, передумова другого додаткового
капіталу в 400 фунтів стерлійґів є не що інше, як попередня акумуляція
першого, акумуляція тих 2.000 фунтів стерлінґів, що
їхньою капіталізованою додатковою вартістю є цей додатковий
капітал у 400 фунтів стерлінґів. Власність на минулу неоплачену
працю з’являється тепер одним-однією умовою теперішнього
присвоєння живої неоплаченої праці в щораз більшому й
більшому розмірі. Що більше капіталіст акумулював, то більше
може він акумулювати.

Оскільки додаткова вартість, що з неї складається додатковий
капітал № І, була результатом купівлі робочої сили за частину
первісного капіталу, купівлі, що відповідала законам товарового
обміну і з юридичного погляду припускає лише вільне порядкування
на боці робітника його власними здібностями, а на боці
власника грошей або товарів — належними йому вартостями;
оскільки додатковий капітал № II і т. ін. є лише результат додаткового
капіталу № І, отже, наслідок цього першого відношення;
оскільки кожна поодинока оборудка завжди відповідає
законові товарового обміну, отже, капіталіст завжди купує
робочу силу, а робітник завжди продає її, припустімо, навіть
за її дійсною вартістю, остільки ясно, що закон присвоєння, або
закон приватної власности, що ґрунтується на товаровій продукції
й товаровій циркуляції, перетворюється через свою власну,
внутрішню, неминучу діялектику на свою пряму протилежність.
Обмін еквівалентів, що виступав як первісна операція, зазнав
таких змін, що тепер лише на позір відбувається обмін, бо, по-перше,
частина капіталу, обміняна на робочу силу, сама є лише
частина продукту чужої праці, присвоєного без еквіваленту, і,
по-друге, її продуцент, робітник, мусить не тільки замістити її,
а замістити її ще й з новим додатком. Відношення обміну між
капіталістом і робітником стає таким чином тільки позірністю,
властивою процесові циркуляції, лише формою, що є чужа самому
змістові й тільки містифікує його. Постійна купівля й продаж
робочої сили — це форма. Змісте той, що капіталіст частину
упредметненої вже чужої праці, яку він безупинно присвоює
собі, не даючи за неї жодного еквіваленту, постійно знову обмінює
на більшу кількість живої чужої праці. Спочатку право
власности здавалося нам основаним на власній праці. Ми мусили,
принаймні, визнати це припущення, бо лише рівноправні власники
товарів протистоять один одному, а засіб до присвоєння чужого

22 «Праця створює капітал, раніш ніж капітал починає вживати
праці» («Labour creates capital, before capital employs labour»). (E. G.
Wakefield: «England and America», London 1833, vol. II, p. 110).
