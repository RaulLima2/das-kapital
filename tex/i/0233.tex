А все ж принцип,\footnote*{
Мається на увазі принцип законодавчого втручання в промислові
справи. \emph{Ред.}
} перемігши у великих галузях промисловости,
які є найспецифічніший витвір сучасного способу продукції,
переміг остаточно. Дивовижний розвиток цих галузей промисловости
на протязі часу від 1853 р. до 1860 р., який відбувався
поруч фізичного й морального відродження фабричних робітників,
розкрив очі найдурнішим. Сами фабриканти, що в них у
півстолітній громадянській війні крок за кроком відвойовано
законодавчі обмеження й реґулювання робочого дня, хвальковито
вказували на контраст між цими галузями промисловости
й тими сферами експлуатації, що лишилися ще «вільними».\footnote{
Так, приміром, висловлюється E. Поттер у листі до «Times’y»
з 24 березня 1863 р. «Times» нагадав йому про бунт фабрикантів проти
десятигодинного закону.
}
Фарисеї «політичної економії» проголосили тепер переконання
про доконечність законодавчим шляхом реґулювати робочий день
новим характеристичним здобутком їхньої «науки».\footnote{
Так, між іншим, висловлюється пан В. Ньюмарч, співробітник
і видавець «History of Prices» Тука. Невже ж це науковий проґрес —
робити боягузливі поступки громадській думці?
} Легко
зрозуміти, що після того, як фабричні маґнати скорилися перед
неминучим і примирилися з ним, сила опору капіталу помалу
слабшала, тоді як у той самий час сила наступу робітничої кляси
зростала разом із зростом числа її спільників серед суспільних
верств, безпосередньо не заінтересованих. Цим то й пояснюється
порівняно швидкий проґрес від 1860 р.

Фарбарні й білильні\footnote{
Виданий 1860 р. закон про білильні та фарбарні установляє,
що робочий день від 1 серпня 1861 р. тимчасово скорочується до 12, а
від 1 серпня 1862 р. остаточно до 10 годин, тобто до 10\sfrac{1}{2} годин у робочі
дні та 7\sfrac{1}{2} годин суботами. Але ось настав лихий 1862 р., і повторився
старий фарс. Пани фабриканти звернулись до парляменту з петицією
стерпіти ще один-однісінький рік дванадцятигодинну працю підлітків
і жінок... «За сучасного стану справ (підчас бавовняного голоду) було б
дуже корисно для робітників, коли б їм дозволили працювати по 12 годин
щодня й діставати по змозі якнайбільшу заробітну плату... Вже було
пощастило внести до парляменту біл у цьому дусі, але він провалився
через аґітацію робітників у білильнях Шотляндії». («Reports etc. for
31 st October 1862», p. 14, 15). Капітал, побитий таким чином тими самими
робітниками, що іменем їхнім він претендував говорити, відкрив тепер
за допомогою юридичних окулярів, що закон з 1860 р., подібно до всіх
парляментських законів про «охорону праці», складений поплутаними,
покрученими словами, дає привід не поширювати його на категорії робітників
«calenderers»\footnote*{
— пресувальники сукна. \emph{Ред.}
} і «finishers».\footnote*{
— апретери. \emph{Ред.}
} Англійська юрисдикція, завжди
вірний наймит капіталу, санкціонувала це закарлюцтво постановою так
званого «Common Pleas».\footnote*{
— цивільний суд. \emph{Ред.}
} «Це викликало велике незадоволення серед
робітників, і дуже шкода, що ясні наміри законодавства нищиться під
приводом хибного окреслення слів». (Там же, стор. 18).
} підведено під фабричний закон 1850 р.
ще 1860 р., а мереживні й панчішні — 1861 р. Внаслідок першого

Printworks Act is admitted to be a failure, both with reference to its educational
and protective provisions»). («Reports etc. for 31 st Oct. 1862», p. 52).