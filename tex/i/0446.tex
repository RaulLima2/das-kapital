Розділ шістнадцятий

Різні формули норми додаткової вартости

Ми бачили, що норма додаткової вартости виражається в таких
формулах:

I. додаткова вартість/змінний капітал (m/v) =
додаткова вартість/вартість робочої сили =
додаткова праця/доконечна праця

Дві перші формули виражають у формі відношення вартостей
те саме, що третя виражає у формі відношення відтинків часу,
що протягом їх ці вартості продукується. Ці формули, що одна
одну доповнюють, є строго логічні. Тим то ми находимо їх у клясичній
політичній економії, правда, щодо суті, але виробленими
несвідомо. Зате ми бачимо там такі вивідні формули:

II. додаткова праця*/робочий день =
додаткова вартість/вартість продукту =
додатковий продукт/сукупний продукт

Ту саму пропорцію виражено тут навпереміну то у формі
робочих часів, то у формі вартостей, що в них вони втілюються,
то у формі продуктів, що в них існують ці вартості. Звичайно,
припускається, що під вартістю продукту треба розуміти лише
вартість, новоспродуковану протягом робочого дня, а сталу частину
вартости продукту виключено.

У всіх цих формулах дійсний ступінь експлуатації праці, або
норму додаткової вартости, виражено неправильно. Хай робочий
день буде 12 годин. Якщо інші припущення нашого попереднього
прикладу лишаються незмінні, то в цьому випадку дійсний
ступінь експлуатації праці виразиться в таких пропорціях:

6    годин додаткової праці/6 годин доконечної праці =
додаткова вартість у 3 шилінґи/змінний капітал у 3 шилінґи = 100\%.

Навпаки, за формулою II ми маємо:

6 годин додаткової праці/робочий день у 12 годин =
додаткова вартість у 3 шилінґи/новоспродукована вартість у 6 шилінґів =
50\%.

Ці вивідні формули в дійсності виражають ту пропорцію,
що в ній робочий день або новоспродукована протягом нього

* У французькому виданні Маркс заводить цю формулу в дужки
і дає до цього таку примітку: «Ми заводимо першу формулу в дужки,
бо ясно вираженого поняття додаткової праці ми не знаходимо в буржуазній
політичній економії». Ред.
