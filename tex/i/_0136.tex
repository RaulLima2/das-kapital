
\index{i}{0136}  %% посилання на сторінку оригінального видання
Процес праці, оскільки він відбувається як процес споживання
капіталістом робочої сипи, виявляє два характеристичні феномени.

Робітник працює під контролем капіталіста, якому належить
його праця. Капіталіст доглядає за тим, щоб працю виконувано
як слід і щоб засоби продукції вживано доцільно, отже, щоб не
марнувалося сировинного матеріялу та щоб із знаряддям праці
поводились ощадно, тобто, щоб його зужитковувалось лише
стільки, скільки цього вимагає вживання його в праці.

А, подруге, продукт є власність капіталіста, а не безпосереднього
продуцента, не робітника. Капіталіст виплачує, приміром,
денну вартість робочої сили. Отже, споживання її, як і кожного
іншого товару, приміром, коня, що його капіталіст найняв на
один день, належить на цей день йому. Покупцеві товару належить
споживання товару, і посідач робочої сили, віддаючи свою
працю, віддає в дійсності лише продану ним споживну вартість.
З того моменту, як він увійшов до майстерні капіталіста, споживна
вартість його робочої сили, тобто споживання її, праця, належить
капіталістові. Купівлею робочої сили капіталіст долучив саму
працю як живий фермент до мертвих, йому ж таки належних
елементів творення продукту. З його погляду процес праці є
лише споживання купленого ним товару, робочої сили, яку він
може однак споживати, лише долучивши до неї засоби продукції.
Процес праці є процес поміж речами, що їх купив капіталіст,
поміж належними йому речами. Тим то й продукт цього процесу
належить йому цілком так само, як і продукт процесу ферментації
в його льоху для вина\footnote{
«Продукти присвоюється раніш, ніж вони перетворюються на
капітал; це перетворення не звільняє їх од того присвоєння» (\emph{Cherbuliez}:
«Riche ou Pauvre», éd. Paris 1841, p. 53, 54). «Продаючи свою
працю за певну кількість засобів існування (approvisionement), пролетар
геть чисто відмовляється від усякої участи в продукті. Присвоєння продуктів
лишається таким самим, як і раніш; його ні в якому разі не змінює
згаданий договір. Продукт належить виключно капіталістові, який постачав
сировинний матеріял і засоби існування. Це — строгий висновок
із закону присвоєння, що його основним принципом було, навпаки, виключне
право власности кожного робітника на його продукт» (там же,
стор. 58). Джемc Мілл: «Elements of Political Economy etc.», London
1821, p. 70: «Коли робітники працюють за заробітну плату, то капіталіст
є власник не лише капіталу [тут розуміється засоби продукції], але
й праці (of the labour also). Коли в поняття капіталу, як це звичайно
робиться, включають і те, що виплачується як заробітна плата, то було б
нісенітницею говорити про працю окремо від капіталу. Слово капітал у
цьому значенні включає одне й друге: капітал і працю».
}.

\subsection[Процес зростання вартости]{Процес зростання вартости\footnotemarkZ{}}
\footnotetextZ{
У французькому виданні заголовок цей подано так: «Production
de la plus-value» — «Продукція додаткової вартоcти». \emph{Ред.}
}

Продукт — власність капіталіста — є якась споживна вартість:
пряжа, чоботи й~\abbr{т. ін.} Але хоч чоботи, приміром, до деякої
міри становлять основу суспільного проґресу, а наш капіталіст
\parbreak{}  %% абзац продовжується на наступній сторінці
