знайомства з вами в кращому світі. Addio!..\footnote{
Однак пан професор мав деяку користь із своєї подорожі до Менчестеру. В «Letters on the
Factory Act» увесь чистий прибуток, «зиск» і «процент» і навіть «something more»,\footnote*{
— щось більше. Ред,
} залежить від
однієї неоплаченої
робочої години робітника! Роком раніш у своїх «Outlines of Political Economy», складених для
насолоди оксфордських студентів і освічених філістерів, Сеніор, полемізуючи проти Рікардового
визначення вартости робочим часом, «відкрив», що зиск постає з праці капіталіста, а процент з його
аскетичности, з його «поздержливости». Сама побрехенька була стара» але слово «поздержливість»
(«Abstinenz») було нове. Пан Рошер правильно переклав його німецькою мовою словом «Enthaltung»
(«поздержливість»). А його компатріоти, менше биті в латині, Вірти, Шульци
й інші Міхелі, переклали його на чорнече «самовідречення» («Entsagung»).
} Сиґнал «останньої години», що її винайшов Сеніор 1836
р., наново протрубив був 15 квітня 1848 р. в «London Economist» Джеме Вілсон, один з головних
мандаринів економічної науки, у своїй полеміці проти
закону про десятигодинний робочий день.

4. Додатковий продукт

Ту частину продукту (\sfrac{1}{10} від 20 фунтів пряжі, або 2 фунти пряжі, у прикладі §2), яка репрезентує
додаткову вартість, ми
називаємо додатковим продуктом (surplus produce, produit net). Як норму додаткової вартости визначає
відношення додаткової
вартости не до цілої суми капіталу, а лише до його змінної складової частини, так і рівень
додаткового продукту визначає відношення останнього не до решти цілого продукту, а до тієї частини
його, яка репрезентує доконечну працю. Як продукція додаткової вартости є визначальна мета
капіталістичної продукції, так і ступінь багатства вимірюється не абсолютною величиною
продукту, а відносною величиною додаткового продукту.34

34 «Для індивіда, що має капітал у 20.000 фунтів стерлінґів, і що його зиски становлять 2.000 фунтів
стерлінґів на рік, було б цілком байдуже, чи його капітал вживає 100 чи 1.000 робітників, чи
випродуковані товари продається за 10.000 фунтів стерлінґів чи за 20.000 фунтів стерлінґів, аби лише
його зиски в усіх цих випадках не падали нижче як 2.000 фунтів
стерлінґів. Хіба реальний інтерес націй не такий самий? Коли припустити, що реальний чистий прибуток
нації, її ренти й зиски лишаються однакові, то не має найменшої ваги, чи нація складається з 10 чи
12 мільйонів людности». (Ricardo: «The Principles of Political Economy», 3 rd. ed, London 1821, p.
416). Задовго перед Рікардом Артур Юнґ, фанатик додаткового продукту, взагалі язикатий, неспроможний
на будь-яку критику письменник, що його слава стоїть у зворотному відношенні
до його заслуг, сказав, між іншим: «Що за користь була б для сучасного королівства з якоїсь цілої
провінції, що в ній землю обробляли б на староримський лад дрібні незалежні селяни, про мене хоч би
й як і найкраще? Яка мета була б у цьому, крім одним-однієї мети продукувати людей
(«the mere purpose of breeding men»), а це саме по собі не має ніякої мети» (is a most useless
purpose»). (Arthur Young: «Political Arit hmetic etc.», 1774, p. 47).

Додаток до примітки 34. Дивний є «великий нахил малювати чистий прибуток корисним для робітничої
кляси... та проте ясно, що це стається не через те, що він чистий» («the strong inclination to
represent net wealth as beneficial to the labouring class... though it is evidently not on account
of being net»). (Th. Hopkins: «On Rent of Land etс.», London 1823, p. 126).