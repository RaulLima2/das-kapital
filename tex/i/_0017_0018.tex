\parcont{}  %% абзац починається на попередній сторінці
\index{i}{0017}  %% посилання на сторінку оригінального видання
властивість не прозирає крізь його тканину, хоч і як вона буде
прозориста. І у вартостевому відношенні до полотна він має значення
лише з цього боку, отже, як утілена вартість, як тіло вартости.
Не вважаючи на те, що сурдут з’являється застебнений на
всі ґудзики, полотно пізнає в ньому споріднену собі прекрасну
душу вартости. Однак, сурдут не може супроти полотна репрезентувати
вартість без того, щоб вартість одночасно не набрала
для полотна форми сурдута. Так особа $А$ не може поставитися до
особи $B$ як до якоїсь ясновельможности без того, щоб ця ясновельможність
одночасно не набрала для $А$ тілесного вигляду $B$;
тому то риси обличчя, волосся й ще дещо інше міняється кожного
разу разом зі зміною володаря країни.

Отже, в тому вартостевому відношенні, що в ньому сурдут
становить еквівалент полотна, форма сурдута фігурує як форма
вартости. Тому вартість товару «полотно» виражається в тілі
товару «сурдут», вартість одного товару — у споживній вартості
іншого. Як споживна вартість, «полотно» є річ, почуттєво відмінна
від сурдута, як вартість — воно є «сурдуторівне» й тому
виглядає як сурдут. Таким чином воно набирає форми вартости,
відмінної від його натуральної форми. Його вартостеве буття
виявляється в його рівності сурдутові так само, як овеча натура
християнина виявляється в тому, що він подібний агнцеві божому.

Ми бачимо, що все те, що нам сказала раніш аналіза товарової
вартости, каже caмé полотно, скоро воно вступає в стосунки з
іншим товаром, з сурдутом. Воно лише висловлює свої думки єдино
приступною йому мовою — мовою товарів. Щоб сказати, що
праця в її абстрактній властивості людської праці утворює його
власну вартість, полотно каже, що сурдут, оскільки він йому
рівнозначний, тобто оскільки він є вартість, складається з тієї
самої праці, як і воно, полотно. Щоб висловити, що його виспренна
предметність вартости відмінна від його жорсткого полотняного
тіла, воно каже, що вартість має вигляд сурдута, і тому
воно саме як предмет вартости схоже на сурдут, як дві краплі
води. Зауважмо до речі, що товарова мова, опріч єврейської, має;
багато інших більш або менш точних говірок. Німецька «Wertsein»,
наприклад, менше влучно, ніж романське дієслово valere,
valer, valoir, висловлює, що порівняння товару $B$ з товаром $А$ є
вираз власної вартости товару $А$. Paris vaut bien une messe.\footnote*{
Париж таки вартий служби божої. \emph{Ред.}
}

Отже, за допомогою вартостевого відношення натуральна форма
товару $B$ стає формою вартости товару $А$ або тіло товару $B$
стає дзеркалом вартости товару $А$.\footnote{
У деякому відношенні з людиною справа стоїть так, як із товаром.
Через те, що вона родиться на світ ані з дзеркалом, ані як фіхтівський
філософ: «Я є я», то людина спершу видивляється в іншу людину, як у
дзеркало. Лише через відношення до людини Павла як до подібного до
себе, людина Петро відноситься й до себе самої як до людини. Але тим
самим і Павло з шкурою й волоссям, у його Павловій тілесності, стає для
нього за форму виявлення роду «людина».
} Товар $А$, відносячись до товару
$B$ як до тіла вартости, як до матеріялізації людської праці, робить
\index{i}{0018}  %% посилання на сторінку оригінального видання
споживну вартість $B$ матеріялом виразу своєї власної вартости.
Вартість товару $А$, виражена таким чином у споживній
вартості товару $B$, має форму відносної вартости.

\subsubsection{Кількісна визначеність відносної форми вартости}

Кожний товар, що його вартість має бути виражена, являє собою
дану кількість якогось предмету споживання — 15 шефлів
пшениці, 100 фунтів кави тощо. Ця дана кількість товару містить
у собі певну кількість людської праці. Отже, форма вартости має
виразити не лише вартість взагалі, але й кількісно визначену
вартість, або величину вартости. Тим то у вартостевому відношенні
товару $А$ до товару $B$, полотна до сурдута, рід товару «сурдут»
не лише якісно прирівнюється до полотна, як тіло вартости
взагалі, але й до певної кількости полотна, наприклад, до 20 метрів
полотна прирівнюється певну кількість тіла вартости, або
еквіваленту, приміром, 1 сурдут.

Рівнання: «20 метрів полотна = 1 сурдутові, або: 20 метрів
полотна варті 1 сурдута» має за передумову, що в 1 сурдуті міститься
рівно стільки субстанції вартости, як і в 20 метрах полотна,
що, отже, обидві кількості товарів коштують рівну кількість
праці, або рівну кількість робочого часу. Але робочий час,
доконечний для продукції 20 метрів полотна або 1 сурдута, змінюється
з кожною зміною в продуктивній силі ткацтва або кравецтва.
Вплив таких змін на відносний вираз величини вартости
треба тепер розглянути докладніше.

I. Хай вартість полотна змінюється,\footnote{
Вислову «вартість» уживається тут, як і в деяких місцях раніш,
для позначення кількісно визначеної вартости, тобто величини вартости.
} тимчасом як вартість
сурдута лишається стала. Коли робочий час, доконечний для продукції
полотна, подвоюється, приміром, у наслідок того, що зменшується
родючість землі, яка родить льон, то подвоюється і його
вартість. Замість рівнання: 20 метрів полотна = 1 сурдутові ми
мали б: 20 метрів полотна = 2 сурдутам, бо 1 сурдут містить у
собі тепер лише половину того робочого часу, що міститься в
20 метрах полотна. Навпаки, коли доконечний для продукції
полотна робочий час зменшиться наполовину, приміром, у наслідок
поліпшення ткацьких варстатів, то й вартість полотна спаде
наполовину; отже, відповідно до цього ми мали б тепер: 20 метрів
полотна = \sfrac{1}{2} сурдута. Отже, за незмінної вартости товару $B$
відносна вартість товару $А$, тобто його вартість, виражена в товарі
$B$, підвищується й падає просто пропорційно до вартости
товару $А$.

II. Хай вартість полотна лишається стала, тимчасом як вартість
сурдута змінюється. Коли за цих обставин доконечний для
продукції сурдута робочий час подвоюється, приміром, у наслідок
недостатнього збору вовни, то ми замість: 20 метрів полотна =
\parbreak{}  %% абзац продовжується на наступній сторінці
