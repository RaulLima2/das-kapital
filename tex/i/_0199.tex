\index{i}{0199}  %% посилання на сторінку оригінального видання
Не кажучи вже про загальні шкідливі впливи нічної праці\footnote{
«Цілком природно, — зауважує один фабрикант сталі, який
уживає до нічної праці дітей, — що молодь, яка працює вночі, не має
змоги спати вдень і не може користуватися путнім відпочинком, а лише
без перестанку тиняється на другий день («Children’s Employment Commission.
Fourth Report», 63, p. XIII). Про значення сонячного світла для
збереження й розвитку організму один лікар каже, між іншим: «Світло
безпосередньо впливає і на тканини тіла, яким дає міць і елястичність.
Мускули тварин, позбавлених нормальної кількости світла, стають як губка
і втрачають свою елястичність, сила нервів через недостачу побудливих
спонук втрачає свій тонус, і розвиток усього, що перебуває в процесі
зростання, занепадає\dots{} Щождо дітей, то для їхнього здоров’я є вельми
важливий постійний рясний приплив денного світла й безпосередній вплив
сонячного проміння протягом якоїсь частини дня. Світло помагає перетворювати
харч у добру плястичну кров і зміцнює новоутворені фібри. Воно
побуджує й органи зору і через те викликає інтенсивнішу діяльність різних
мозкових функцій». Пан В. Стрендж, старший лікар «General Hospital»
у Worcester’i, що з його твору «Про здоров’я» (1864) запозичено це
місце, пише в одному листі до члена слідчої комісії пана Вайта: «Я мав
давніше нагоду стежити в Ланкашірі за впливом нічної праці на фабричних
дітей і, всупереч улюбленому запевненню деяких працедавців, я
рішуче заявляю, що така праця швидко підтинає здоров’я дітей». («Children’s
Employment Commission. 4 th Report», 284, p. 55). Що такі речі
можуть взагалі бути предметом серйозних суперечок, найкраще показує
те, як впливає капіталістична продукція на «мозкові функції» капіталістів
та їхніх retainers\footnote*{
— прихильників. \emph{Ред.}
}.
},
безперервний процес продукції, що триває протягом двадцяти чотирьох
годин, дає незвичайно бажану нагоду для того, щоб переступати
межі номінального робочого дня. Приміром, у згаданих
вище галузях промисловости, де працюють з дуже великим напруженням,
офіціяльний робочий день становить для кожного робітника
здебільшого 12 годин нічних або денних. Але наднормова
праця, яка виходить поза ці межі, в багатьох випадках, уживаючи
слів англійського офіційного звіту, «справді повна жаху»
(«truly fearful»)\footnote{
Там же, 57, p. XII.
}. Ніякий людський розум, — каже звіт, — не
може уявити собі тієї маси праці, яку, за даними свідків, виконують
хлопці 9--12 років, і не дійти при цьому неминуче
до висновку, що такого зловживання владою батьків та працедавців
надалі не можна дозволяти»\footnote{
Там же (4 th Report, 1865), 58, p. XII.
}.

«Вже та метода, що хлопчаків взагалі примушують працювати
навпереміну то вдень то вночі, — вже це приводить так підчас
оживлення справ, як і за звичайного стану речей до ганебного здовження
робочого дня. Це здовження в багатьох випадках є не лише
жорстоке, але просто неймовірне. Часто-густо буває, що з тієї
або іншої причини іноді не прийде якийсь із хлопчаків на зміну.
Тоді один або декілька з присутніх хлопчаків, що вже скінчили
свій робочий день, мусять заступити відсутнього. Ця система
така загальновідома, що управитель однієї вальцювальні на мій запит,
як заповнюється місця відсутніх хлопчаків, відповів: «Аджеж
я знаю, що вам це так само добре відомо, як і мені», — і, ні трохи
\parbreak{}  %% абзац продовжується на наступній сторінці
