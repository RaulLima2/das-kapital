у відношення як товари, хоронителі товарів мусять ставитись
одні до одних як особи, воля яких панує в тих речах, так що один
лише з волі другого, отже, кожний лише за допомогою акту волі,
спільного одному й другому, присвоює собі чужий товар, відчужуючи
свій власний. Тому вони мусять визнавати себе взаємно
за приватних власників. Це правне відношення, що його формою
є договір, законно чи незаконно складений, є вольове відношення,
що в ньому відбивається економічне відношення. Зміст цього
правного або вольового відношення дається самим економічним
відношенням.38 Особи існують тут одна для однієї лише як представники
товарів, і тим то як посідачі товарів. У дальшому ході
викладу ми взагалі побачимо, що характеристичні економічні
маски осіб є лише персоніфікації економічних відносин, носіями
яких ці особи протистоять одна одній.

Що саме відрізняє посідача товарів від товару, так це та обставина,
що для товару кожне інше товарове тіло є лише форма виявлення
його власної вартости. Левелер з роду та цинік, товар
завжди готовий обміняти не лише душу, але й тіло на всякий
інший товар, хоч би останній був огидливіший, ніж Маріторна.**Цю
відсутню у товару здібність схоплювати конкретні
властивості товарових тіл посідач товарів доповнює своїми п’ятьма
й більше чуттями. Його товар не має для нього жодної безпосередньої
споживної вартости. А то він не виніс би його на ринок.
Він має споживну вартість для інших. Для посідача товарів
товар безпосередньо має лише споживну вартість бути носієм
мінової вартости й таким чином засобом обміну.39 Через те

з-поміж товарів, що були на ринку в Landit, налічує поруч з матеріями
для одягу, черевиками, шкурами, сільськогосподарським знаряддя і т. ін.
також і «femmes folles de leur corps».*

8 Прудон черпає свій ідеал справедливосте, justice éternelle (вічної
справедливости) з правних відносин, відповідних товаровій продукції,
даючи також тим, до речі сказавши, дуже втішний для всіх філістерів
доказ, що форма товарової продукції така сама вічна, як і справедливість.
Потім, навпаки, він хоче реформувати за цим ідеалом дійсну товарову
продукцію й відповідне їй дійсне право. Що подумали б про хеміка, який,
замість студіювати дійсні закони обміну речовин і розв’язувати на їхній
основі певні завдання, захотів би реформувати обмін речовин за «вічними
ідеями», «naturalité» й «affinité» (властивости й споріднености)? Коли
нам кажуть, що «лихвар» суперечить «justice éternelle», «équité éternelle
» (вічній правді), «mutualité éternelle» (вічній взаємності) та іншим
«vérités éternelles» (вічним правдам), то невже ж ми дізнаємося про лихваря
щось більше, ніж знали про нього отці церкви, коли вони казали
що він суперечить «grâce éternelle» (вічному милосердю), «foi éternelle»
(вічній вірі), «volonté éternelle de dieu» (вічній волі божій)?

39 «Бо вжиток кожного добра є подвійний. — Один вжиток, властивий
речі як такій, другий — ні; так, сандали можуть бути, щоб взувати
на ноги, і для обміну. Обидва є споживні вартості сандалів, бо й той, хто
обмінює сандали на щось, чого йому бракує, наприклад, на харчі, користується
сандалами як сандалами. Але це не є їхній природний спосіб
ужитку, бо вони існують не для обміну». (Aristoteles: «De Republica»,
lib. I, c. 9).

* — повій. Peд.
** Дієва особа з роману Сервантеса «Дон-Кіхот». Ред.
