виходить із готової їхньої форми, щоб потім її аналізувати. Процес
обміну дає товарові, що його він перетворює на гроші, не його
вартість, а його специфічну форму вартости. Переплутування
цих двох визначень приводило до того, що вартість золота й
срібла вважалося за уявлювану.\footnote{
«Золото й срібло мають свою вартість як металі раніш, ніж вони
стають грішми» («L’oro е l’argento hanno valore come metalli anteriore
all’essere moneta»). (Galiani: «Della Moneta»). Льокк каже: «Загальна
згода людей надає сріблу задля його властивостей, які роблять
його придатним на ролю грошей, уявлюваної вартости». Навпаки, Ло
каже: «Як різні нації могли б надати одній речі уявлювану вартість...
або як могла б утриматись ця уявлювана вартість?» Але як мало він сам
розумів справу, видно з цих його слів: «Срібло обмінювалося відповідно
до тієї споживної вартости, яку воно мало, отже, за своєю дійсною вартістю;
в наслідок свого призначення бути... грішми воно набуло додатково
ще однієї вартости (une valeur additioneile)». (Jean Law: «Considérations
sur le numéraire et le commerce» y виданні E. Daire: «Economistes
Financiers du XVIII siècle», p. 469, 470).
} Через те, що гроші в певних
своїх функціях можуть бути заміщені простими знаками грошей,
виникла інша помилка, що гроші є лише знаки. З другого боку,
тут містилося передчуття того, що грошова форма речі супроти
самої речі зовнішня й є лише форма виявлення захованих за нею
людських відносин. У цьому розумінні кожний товар був би
знаком, бо як вартість він є лише речова оболонка витраченої
на нього людської праці.\footnote{
«Гроші є їх (харчових продуктів) знак» («L’argent en (des denrées)
est le signe»). (V. de Forbonnais: «Eléments du Commerce»,
Nouv. Edit. Leyde 1766, vol. II, p. 143). «Як знак вони притягаються
харчовими продуктами» («Comme signe il est attiré par les denrées»). (Там
само, p. 155). «Гроші — знак речі, і вони репрезентують річ» («L’argent
est un signe d’une chose et la représente»). (Montesquieu: «Esprit des
Lois», Oeuvres, London 1767, vol. III, p. 2). «Гроші не простий знак,
бо вони сами є багатство: вони не репрезентують вартостей, а еквівалентні
їм» («L’argent n’est pas simple signe, car il est lui-même richesse; il
ne représente pas les valeurs, il les équivaut»). (Le Trosne: «De l’Intérêt
Social», p. 910). «Коли розглядають поняття вартости, то саму річ
вважають лише за знак, і вона має значення не сама по собі, а лише як
те, чого вона варта». (Hegel: «Philosophie des Rechtes», S. 100). Задовго
до економістів уявлення про золото як простий знак та про уявлювану
вартість благородних металів пустили в хід юристи; лакеї й сикофанти
королівської влади, вони протягом цілого середньовіччя обґрунтовували
традиціями Римської імперії й поняттями про гроші з пандектів
право цієї влади фалшувати монету. «Ніхто не може й не сміє сумніватися,
— каже вірний учень юристів, Філіп Валюа, в одному декреті з 1346 р., —
що лише нам і нашій королівській ясновельможності належить... справа
виготовлення, постачання грошей і всякі розпорядження щодо грошей,
} Але, проголошуючи ті суспільні

карбовані, хоч їх уживається як міру для всіх інших речей, все ж таки
являють собою товари не менш, ніж вино, масло, тютюн, тканини або матерії»
(«Silver and gold, coined or uncoined, tho’they are used fora measure
of all other things, are no less a commodity than wine, oyl, tobacco,
cloth or stuffs»). («А Discourse concerning Trade, and that in particular
of the East-Indies etc.», London 1689, p. 2). «Капітал і багатство королівства
не обмежується лише на самих грошах, так само золото й срібло
не можна вилучити з числа предметів торговлі» («The stock and riches
of the kingdom cannot properly be confined to money, nor ought gold and
silver to de excluded from being merchandize»). («The East-India Trade
a most Profitable Trade», London 1677, p. 4).