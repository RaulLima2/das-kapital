ніше це заощадження, бо воно зменшує ціну цих виробів. А, проте,
ви думаєте, що продукція багатства, яке походить з праці промисловців, полягає у збільшенні мінової
вартости їхніх виробів».\footnote{
«Ils conviennent que plus on peut, sans préjudice, épargner de frais ou
de travaux dispendieux dans la fabrication des ouvrages des artisans, plus
cette épargne est profitable par la diminution du prix de ces ouvrages.
Cependant ils croient que la production de richesse qui résulte des travaux
des artisans consiste dans l’augmentation de la valeur vénale de leurs ouvrages»).
(Quesnay: «Dialogues sur le Commerce et sur les Travaux des
Artisans», ed. Daire, Paris 1846, p. 188, 189).
}

Отже, заощадження на праці\footnote{
«Ці спекулянти, що так багато заощаджують на праці робітників,
яку вони мусили б оплатити» («Ces spéculateurs si économes du travail
des ouvriers qu’il faudrait qu’ils payassent»). (J. N. Bidaull: «Du
Monopole qui s’établit dans les arts industrielles et le commerce», Paris
1828, p. 13). «Підприємець завжди намагатиметься заощаджувати час і
працю» («The employer will be always on the stretch to economise
time and labour»). (Dugald Stewart: Works ed. by Sir W. Hamilton.
Edinburgh 1885, vol. Ill, «Lectures on Political Economy», p. 318). «їхній (капіталістів) інтерес
вимагає того, шоб продуктивні сили зуживаних
ними робітників були якомога найбільші. Тому вони звертають свою
увагу майже виключно на збільшення цієї сили». («Their (the capitalists’)
interest is that the productive powers of the labourers they employ should
be the greatest possible. On promoting that power their attention is fixed
and almost exclusively fixed»). (R. Jones: «Textbook of Lectures on
the Political Economy of Nations», Hertford 1852, Lecture III).
} внаслідок розвитку продуктивної сили праці за капіталістичної продукції
зовсім не має на меті
скорочення робочого дня. Воно має на меті лише скорочення робочого часу, доконечного для продукції
певної кількости товарів.
Те, що робітник за підвищеної продуктивної сили його праці
продукує за одну годину, приміром, вдесятеро більше товарів,
ніж раніш, отже, на кожну штуку товару потребує вдесятеро
менше робочого часу, аж ніяк не заважає тому, що його тепер,
як і раніш, примушують працювати 12 годин та продукувати
протягом 12 годин 1.200 штук товару замість 120. Навіть більше:
його робочий день разом з тим може здовжуватися, так що він
тепер продукуватиме 1.400 штук за 14 годин, і т. ін. Тому в економістів
такої породи, як от Мак Кулох, Юр, Сеніор і tutti
quanti,\footnote*{
— всі, скільки їх є. \emph{Ред.}
} ми на одній сторінці читаємо, що робітник повинен дякувати капіталові за розвиток
продуктивних сил, бо цей останній
скорочує доконечний робочий час, а на другій сторінці — що
робітник мусить виявити цю вдячність, працюючи в майбутньому
15 годин замість 10. За капіталістичної продукції розвиток продуктивної сили праці має на меті
скоротити ту частину робочого
дня, протягом якої робітник мусить працювати для себе самого,
щоб саме цим здовжити другу частину робочого дня, протягом
якої він може працювати задурно на капіталіста. У якій мірі
можна досягти цього результату і не здешевлюючи товарів, це
виявиться при розгляді особливих метод продукції відносної
додаткової вартости, до чого ми тепер і переходимо.