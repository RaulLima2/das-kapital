\parcont{}  %% абзац починається на попередній сторінці
\index{i}{0165}  %% посилання на сторінку оригінального видання
лишається, є єдина вартість, дійсно спродукована в процесі творення товару. Коли додаткову вартість
дано, то, щоб знайти змінний капітал, ми відлічуємо її від цієї новоспродукованої вартости.
Коли дано змінний капітал і ми шукаємо додаткову вартість, то ми робимо навпаки. Коли дано і те й
друге, то треба тільки зробити кінцеву операцію — обчислити відношення додаткової вартости до
змінного капіталу, $\frac{m}{v}$.

Хоч і яка проста ця метода, однак, здається, до речі буде зазнайомити читачів на кількох прикладах
із незвичним для них
способом розуміння, що лежить в основі цієї методи.

Насамперед, приклад прядільної фабрики на \num{10.000} мюльних веретен, де прядеться з американської
бавовни пряжу № 32 і
щотижня продукується по 1 фунту пряжі на веретено. Відпадки дорівнюють 6\%. Отже, щотижня \num{10.600}
фунтів бавовни переробляється на \num{10.000} фунтів пряжі й 600 фунтів відпадків. У квітні 1871~\abbr{р.} ця
бавовна коштувала 7\sfrac{3}{4}\pens{ пенса} за фунт, отже, \num{10.600} фунтів, заокругляючи суму, коштували 342\pound{ фунти
стерлінґів}. Ці \num{10.000}
веретен, включаючи машини для попереднього обробітку бавовни й парову машину, коштують 1\pound{ фунт
стерлінґів} за веретено, тобто \num{10.000}\pound{ фунтів стерлінґів}. Щорічне зужитковання їх становить
10\% \deq{} \num{1.000}\pound{ фунтів стерлінґів}, або 20\pound{ фунтів стерлінґів} на тиждень. Наймання фабричного будинку —
300\pound{ фунтів стерлінґів}, або на тиждень 6\pound{ фунтів стерлінґів}. Вугілля (4 фунти на годину
й кінську силу при 100 кінських силах [індикаторних] і 60 годинах тижнево, включаючи й опалення
будинку) 11 тонн
на тиждень, по 8\shil{ шилінґів} 6\pens{ пенсів} за тонну, коштує, заокругляючи суму, 4\sfrac{1}{2}\pound{ фунти стерлінґів} на
тиждень; газ — 1\pound{ фунт стерлінґів} на тиждень, мастиво — 4\sfrac{1}{2}\pound{ фунти стерлінґів} на тиждень, отже, всі
допоміжні матеріяли — 10\pound{ фунтів стерлінґів} на тиждень. Отже, стала частина вартости становить 378\pound{ фунтів стерлінґів}
на тиждень. Заробітна плата становить 52\pound{ фунти стерлінґів} на тиждень. Ціна пряжі — 12\sfrac{1}{4}\pens{ пенсів} за
фунт, або за \num{10.000} фунтів
це становить 510\pound{ фунтів стерлінґів}, отже, додаткова вартість дорівнює $510 - 430 \deq{} 80$\pound{ фунтам
стерлінґів}. Ми припускаємо, що стала частина вартости в 378\pound{ фунтів стерлінґів} дорівнює нулеві, бо
вона не бере участи в тижневому творенні вартости.
Лишається тижнева новоспродукована вартість $132 \deq{} \oversetn{v}{52} \dplus{} \oversetn{m}{80}\text{\pound{ фунтів стерлінґів}}$. Отже, норма
додаткової вартости дорівнює
$\frac{80}{52} \deq{} 153\sfrac{11}{18}\%$. За десятигодинного пересічного робочого дня це дає: доконечна праця \deq{} 3\sfrac{31}{33}
годин і додаткова праця \deq{} 6\sfrac{2}{33} годин\footnote{
Примітка до другого видання. Приклад прядільної фабрики з 1860~\abbr{р.}, наведений у першому виданні,
мав деякі фактичні помилки. Наведені в тексті цілком точні числа подав мені один менчестерський
фабрикант. — Треба зауважити, що в Англії стару кінську силу обчислювалось
на основі діяметра циліндра, а нову обчислюється на основі дійсної сили, яку показує індикатор.
}.
