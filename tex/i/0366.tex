дання ринкових цін та переміщення капіталу робітники, уживані
у продукції доконечних засобів існування, також «звільняються»
від якоїсь частини їхньої заробітної плати. Отже, замість
довести, що машини, звільняючи робітників од засобів існування,
одночасно перетворюють ці останні на капітал, щоб уживати
перших, пан апологет із своїм випробуваним законом попиту й
подання доводить, навпаки, що машини не тільки в тій галузі
продукції, де їх заводять, але й у тих галузях продукції, де їх
не заведено, викидають робітників на брук.

Дійсні факти, перекручені економічним оптимізмом, такі.
Витиснутих машинами робітників викидають із майстерні на
ринок праці, і вони збільшують там число робочих сил, що ними
можна порядкувати для капіталістичної експлуатації. В сьомому
відділі ми побачимо, що цей вплив машин, який нам тут змальовано
як компенсацію для робітничої кляси, спадає, навпаки, як
найстрашніша кара на робітника. Тут зауважимо лише ось
що: робітники, викинуті з однієї галузі промисловости, можуть,
щоправда, шукати заняття в якійсь іншій. Якщо вони знаходять
собі заняття, і таким чином відновлюється зв’язок між ними й
засобами існування, які були звільнені разом з ними, то це
стається за допомогою нового додаткового капіталу, що шукає
вміщення, а зовсім не того капіталу, що вже раніш функціонував
і тепер перетворений на машини. Але навіть і в такому випадку,
— які мізерні їхні перспективи! Скалічені через поділ праці,
ці бідолахи так мало чого варті поза своєю колишньою сферою
праці, що вони знаходять собі доступ лише до небагатьох нижчих,
і тому завжди переповнених та низько оплачуваних, галузей
праці.\footnote{
Один рікардіянець зауважує з приводу цього, заперечуючи проти
нісенітниць Ж. Б. Сея: «За розвиненого поділу праці вмілість робітників
може придатись тільки в тій окремій галузі, де вони навчилися її; вони
сами є своєрідні машини. Тим то абсолютно не поможе, коли верзти, як
папуга, що речі мають тенденцію знаходити свій рівень. Нам треба лише
поглянути навколо себе, і ми побачимо, що вони довгий час не можуть
знайти свого рівня, а якщо і знайдуть його, то цей рівень нижчий, ніж
він був на початку процесу». («An Inquiry into those Principles respecting
the Nature of Demand etc.», London 1821, p. 72).
} Далі, кожна галузь промисловости притягає щороку
новий потік людей, який дає їй континґент для регулярної заміни
та зросту.\footnote*{
У французькому виданні кінець цього речення подано так: «... який
дає їй континґент для заміни спрацьованої робочої сили та для того поповнення,
що його вимагає реґулярний розвиток цієї галузі». Ред.
} Скоро тільки машини звільняють частину робітників,
що досі працювали в певній галузі промисловости, то й новий
потік промислових рекрутів перерозподіляється, і його вбирають
інші галузі праці, тимчасом як первісні жертви за переходовий
час здебільша занепадають і гинуть.

Безперечний факт, що машини сами по собі не винні у «звільненні»
робітників од засобів існування. Вони здешевлюють і
збільшують продукт у тій галузі, яку захоплюють, та спочатку
лишають ту масу засобів існування, що її продукується по інших