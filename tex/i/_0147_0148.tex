\parcont{}  %% абзац починається на попередній сторінці
\index{i}{0147}  %% посилання на сторінку оригінального видання
тією самою працею, оскільки вона утворює вартість, виявилась
тепер як ріжниця між різними сторонами процесу продукції.

Як єдність процесу праці й процесу утворення вартости процес
продукції є процес продукції товарів; як єдність процесу
праці й процесу зростання вартости він є капіталістичний процес
продукції, капіталістична форма товарової продукції.

Вже раніш зазначено, що для процесу зростання вартости
цілком байдуже, чи є присвоєна капіталістом праця проста, суспільна
пересічна праця, чи складніша праця, праця вищої специфічної
ваги. Праця, що є виша, складніша супроти суспільної пересічної
праці, є виявлення такої робочої сили, що на її освіту витрачено
більше коштів, продукція якої коштує більше робочого
часу і яка тому має вищу вартість, ніж проста робоча сила.
Коли вартість цієї сили вища, то й виявляється вона у вищій
праці й тим то упредметнюється за той самий час у відповідно
вищих вартостях. Але хоч і яка була б ріжниця щодо ступеня
між працею прядуна й працею ювеліра, та частка праці, якою
ювелірний робітник повертає тільки вартість своєї власної робочої
сили, якісно аж ніяк не відрізняється від тієї додаткової
частки праці, що нею він утворює додаткову вартість. Тут, як і
раніш, додаткову вартість утворюється лише в наслідок кількісного
надлишку праці, в наслідок здовженого тривання того самого
процесу праці: в одному випадку — процесу продукції пряжі,
в другому випадку — процесу ювелірної продукції.\footnote{
Ріжниця між складною і простою працею, між «skilled» і «unskilled
labour», ґрунтується почасти просто на ілюзіях, або, принаймні,
на ріжницях, які давним-давно перестали бути реальними і існують далі
лише як традиційні умовності, а почасти на безпораднішому становищі
певних верств робітничої кляси, через яке їм більш, ніж іншим, не сила
вибороти оплату своєї робочої сили за її вартістю. Випадкові обставини
відіграють при цьому таку велику ролю, що того самого роду праці міняють
своє місце. Там, наприклад, де фізична сила (Substanz) робітничої кляси
ослаблена й порівняно вичерпана, як от по всіх країнах розвиненої капіталістичної
продукції, грубі роботи, що потребують багато мускульної сили,
взагалі набирають вищого характеру порівняно з делікатнішими роботами,
які спадають до ступеня простої праці; приміром, праця bricklayer’à (муляра)
в Англії має значно вищий ступінь, ніж праця ткача дамасту.
З другого боку, праця fustian cutter (робітника, що стриже менчестер),
хоч вона й коштує більшого фізичного напруження і, опріч того, дуже нездорова,
фігурує як «проста» праця. Зрештою, не треба уявляти собі, що
так звана «skilled labour» має кількісно значні розміри в національній
праці. Лен обчислює, що в Англії (і у Велзі) існування більш ніж
11 мільйонів людей спирається на просту працю. Відлічивши один мільйон
аристократів і півтора мільйона павперів, волоцюг, злочинців, тих, що
живуть із проституції і т. ін., з 18 мільйонів загальної кількости людности
за часів писання його твору, лишається 4.650.000 осіб середньої
кляси, залічуючи сюди дрібніших рантьє, урядовців, письменників,
артистів, учителів і т. д. Щоб одержати цих 4\sfrac{2}{3} мільйона, він залічує до
працівної частини середньої кляси, крім банкірів, тощо всіх краще оплачуваних
«фабричних робітників»! Навіть bricklayer’і (мулярі) опинилися
серед «кваліфікованих робітників». Після цього лишається в нього
}

З    другого боку, в кожному процесі утворення вартости вища
праця завжди мусить зводитися на суспільну пересічну працю,
\index{i}{0148}  %% посилання на сторінку оригінального видання
приміром, один день вищої праці на х днів простої праці.\footnote{
«Коли кажуть про працю як про міру вартости, то неодмінно мають
на думці працю певного роду\dots{} відношення її до інших родів праці можна
легко визначити». («Where reference is made to labour as a measure of
value, it necessarily implies labour of one particular kind\dots{} the proportion
which the other kinds bear to it being easily ascertained»). («Outlines of
Political Economy», London 1832, p. 22, 23).
} [Коли
фахівці економісти обурюються проти цього «довільного твердження»,
то про них безперечно можна сказати, відповідно до
німецької приказки, що за деревами вони не бачать лісу. Те, що
вони кваліфікують як теоретичні хитрощі, на ділі є лише процедура,
що її щодня практикують по всіх краях світу. Повсюди
вартості найрізноманітніших товарів однаковісінько виражені
в грошах, тобто в певній кількості золота або срібла. Цим самим
різні роди праці, репрезентовані цими вартостями, зведені у
різних пропорціях на певні суми одного і того самого роду звичайної
праці, праці, що продукує золото або срібло].\footnote*{
Заведене у прямі дужки ми беремо з французького видання. («Le Саpital
etc.», v. І, ch. VII, p. 84). Peд.
} Отже,
ми заощадимо собі зайву операцію і зробимо простішою аналізу,
припускаючи, що робітник, якого вживає капітал, виконує
просту суспільну пересічну працю.

Розділ шостий

Сталий капітал і змінний капітал

Різні фактори процесу праці беруть різну участь у творенні
вартости продукту.

Робітник додає до предмету праці нову вартість, долучаючи
до нього певну кількість праці, хоч і який був би конкретний
вміст, мета й технічний характер його праці. З другого боку, ми
знову находимо вартості зужиткованих засобів продукції як
складові частини вартости продукту, приміром, вартості бавовни
й веретен — у вартості пряжі. Отже, вартість засобів продукції
зберігається через перенесення її на продукт. Це перенесення
відбувається підчас перетворення засобів продукції на продукт,
у процесі праці. Праця упосереднює це перенесення. Але яким
чином?

Робітник не працює подвійно в той самий час: раз для того,
щоб своєю працею додати до бавовни вартість, а другий раз для
того, щоб зберегти стару вартість бавовни, або, що те саме, для
того, щоб на продукт, на пряжу, перенести вартість бавовни, яку
він обробляє, і вартість веретен, якими він працює. Він зберігає
стару вартість лише тим, що додає до неї нову вартість. Але що

згаданих 11 мільйонів. (S. Laing: «National Distress etc.», London
1844, стор. 51, 52). «Велика кляса, що нічого не може дати за харчі,
крім простої праці, становить головну масу народу». («The great class,
who have nothing to give for food but ordinary labour, are the great body
of the people»). (James Mill in Art. «Colony». Supplement to the. Encyclopedia
Britannica, 1831, p. 8).
\parbreak{}  %% абзац продовжується на наступній сторінці
