\index{i}{0559}  %% посилання на сторінку оригінального видання

Приріст річного доходу 1864 р. проти 1853 р. 
Збільшення за рік 

\noindent\begin{center}
\begin{tabularx}{0.8\textwidth}{XD{,}{,}{-2}D{,}{,}{-2}}
Від будинків\dotfill{} & 38,60\% & 3,50\% \\
\ditto{Від} каменярень\dotfill{} & 84,76\% & 7,70\% \\
\ditto{Від} копалень\dotfill{} & 68,85\% & 6,26\% \\
\ditto{Від} чавуноливарень\dotfill{} &  39,92\%  & 3,63\% \\
\ditto{Від} рибальства\dotfill{} &  57,37\% & 5,21\% \\
\ditto{Від} газівень\dotfill{} & 126,02\% & 11,45\%\\
\ditto{Від} залізниць\dotfill{} & 83,29\% & 7,57\% \footnote{Там же.}
\\
\end{tabularx}
\end{center}


Якщо порівняти між собою щочотирирічки періоду 1853--1864 рр.,
то ступінь збільшення доходів невпинно зростає. Приміром,
для доходів, що походять із зиску, він 1853--1857 рр.
становить 1,73\% на рік, 1857--1861 рр. — 2,74\% на рік і 1861--1864рр. 9,30\% на рік.
Загальна сума доходів, що підпадають прибутковому
оподаткуванню, становила в Об’єднаному королівстві
1856 р. 307.068.898 фунтів стерлінґів, 1859 р. — 328.127.416 фунтів
стерлінґів, 1862 р. — 351.745.241 фунт стерлінґів, 1863 р. —
359.142.897 фунтів стерлінґів, 1864 р. — 362.462.279 фунтів стерлінґів,
1865 р. — 385.530.020 фунтів стерлінґів.\footnote{
Цих чисел для порівняння досить, але коли розглядати їх абсолютно,
то вони фалшиві, бо щорічно «затаюється» може більше ніж
100 мільйонів фунтів стерлінґів доходів. Нарікання Commissioners of Irland
Revenue на систематичне шахрайство, особливо з боку купців і промисловців,
повторюються в кожному їхньому звіті. Приміром, читаємо:
«Одно акційне товариство показало свій належний до оподаткування зиск
у 6.000 фунтів стерлінґів, таксатор визначив його у 88.000 фунтів стерлінґів,
і, кінець-кінцем, податок виплачено з цієї суми. Друга компанія
показала зиск у 190.000 фунтів стерлінґів і мусила признатися, що
дійсна сума є 250.000 фунтів стерлінґів». (Там же, стор. 42).
}

Акумуляцію капіталу одночасно супроводили його концентрація
й централізація. Хоч для Англії не існувало офіціяльної
рільничої статистики (в Ірляндії вона існує), проте 10 графств подали
її з власної волі. Тут виявився з неї такий результат, що від
1851 до 1861 р. число фарм, нижчих за 100 акрів, зменшилося з
13.583 до 26.567, отже 5.016 фарм сполучилося з більшими фармами.\footnote{
«Census etc.», vol. III, p. 29. Твердження Джона Брайта, що 150
землевласникам належить половина англійської землі, а 12 — землевласникам
половина шотляндської, не збито.
}  Від 1815 до 1825 р. з рухомого майна, що підпадало спадщинному
податкові, не було жодного понад 1 мільйон фунтів стерлінґів;
навпаки, від 1825 до 1835 р. їх було 8, від 1856 до червня
1859 р., тобто за 4\sfrac{1}{2} роки, — 4.\footnote{
«Fourth Report etc. of Inland Revenue», London 1860, p. 17.
} Однак централізацію найкраще
можна побачити з короткої аналізи прибуткового податку
в рубриці D (зиски, за винятком фармерських і т. ін.) за роки
1864 і 1865. Спочатку зауважу, що доходи з цього джерела,
які розміром не нижчі за 60 фунтів стерлінґів, підпадають
income tax.\footnote*{
— прибутковому податкові. \emph{Ред.}
} Ці доходи, що підпадають оподаткуванню, становили
в Англії, Велзі й Шотляндії 1864 р. 95.844.222 фунтів стерлінґів
\index{i}{0560}  %% посилання на сторінку оригінального видання
і 1865 р. — 105.435.579 фунтів стерлінґів,\footnote{
Це — чисті доходи, отже, після того, коли вже відлічено з них
певні суми, визначені законом.
} число оподаткованих
осіб 1864 р. — 308.416 на загальну кількість людности в
23.891.009, 1865 р. — 332.431 особа на загальну кількість людности
в 24.127.003. Про розподіл цих доходів за обидва роки
довідуємося з цієї таблиці:

\noindent\begin{small}
\begin{tabularx}{\textwidth}{Xrrrr}
    \toprule
    & \multicolumn{2}{c}{\makecell{Рік, що кінчається \\ 5 квітня 1864 р.}}
    & \multicolumn{2}{c}{\makecell{Рік, що кінчається \\ 5 квітня 1865 р.}} \\
    \cmidrule{2-5}
    & \makecell{Доходи від \\ зиску (фунтів \\  стерлінґів)}
    & \makecell{Число \\ осіб}
    & \makecell{Доходи від \\ зиску (фунтів \\ стерлінґів)}
    & \makecell{Число \\ осіб} \\
    \midrule
    Загальний дохід\dotfill{} & 95.844.222 &  308.416 &  105.435.738 &  332.431 \\
    З того\dotfill{} &  57.028.289  & 23.334  &    64.554.297   &  24.265 \\
    \ditto{З}\ditto{того}\dotfill{} &  36.415.225 &   3.619    &    42.535.576   &  4.021 \\
    \ditto{З}\ditto{того}\dotfill{} &   22.809.781 &  832     &      27.555.313 &    973\\
    \ditto{З}\ditto{того}\dotfill{} &    8.744.762   &   91      &       11.077.238   &  107 \\
\end{tabularx}
\end{small}


1855 р. в Об’єднаному королівстві випродуковано 61.453.079
тонн вугілля вартістю в 16.113.167 фунтів стерлінґів, 1864 р. —
92.787.873 тонни вартістю в 23.197.968 фунтів стерлінґів; 1855 р. —
3.218.154 тонни чавуну вартістю в 8.045.385 фунтів стерлінґів,
1864 р. — 4.767.951 тонну вартістю в 11.919.877 фунтів стерлінґів.
1854 р. довжина залізниць, експлуатованих в Об’єднаному
королівстві, становила 8.054 милі з капіталовкладенням
у 286.068.794 фунти стерлінґів, 1864 р. довжина у милях становила
12.789, а вкладений капітал — 425.719.613 фунтів стерлінґів.
Загальний експорт і імпорт Об’єднаного королівства
становив 1854 р. 268.210.145 фунтів стерлінґів, 1865 р. —
489.923.285. Нижченаведена таблиця показує рух експорту:

Роки    Фунтів стерлінґів

1846\dotfill 58.842.377

1849. . . \dotfill 63.596.052

1856\dotfill 115.826.948

1860\dotfill 135.842.817

1865\dotfill 165.862.402

1866\dotfill 188.917.563 \footnote{
В цей момент, березень 1867 р., індійсько-китайський ринок
уже знову переповнений комісійними товарами брітанських бавовняних
фабрикантів. 1866 р. почалося зниження заробітної плати бавовняних
робітників на 5\%, в 1867 р. в наслідок подібних операцій стався страйк
20.000 робітників у Престоні. [Це був пролог кризи, що вибухла одразу
після того. — Ф. Е.].
}

Після цих небагатьох даних стає зрозумілим тріюмфальний
крик генерального реєстратора брітанського народу: «Хоч і
як швидко зростала людність, вона не встигала за проґресом промисловости
й багатства».\footnote{
«Census etc.», там же, стор. 11.
} А тепер вдаймося до безпосередніх
аґентів цієї промисловости або до продуцентів цього багатства,
до робітничої кляси. «Це одна з найсумніших характеристичних
\parbreak{}  %% абзац продовжується на наступній сторінці
