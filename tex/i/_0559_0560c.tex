
\index{i}{0559}  %% посилання на сторінку оригінального видання
\begin{table}[H]
\centering

\noindent\begin{tabular}{lrr}
& \makecell[r]{Приріст річного\\ доходу 1864~\abbr{р.}\\ проти 1853~\abbr{р.}} &
 \makecell[r]{Збільшення \\ за рік} \\

Від будинків\dotfill{} & 38,60\% & 3,50\% \\

\ditto{Від} каменярень\dotfill{} & 84,76\% & 7,70\% \\
\ditto{Від} копалень\dotfill{} & 68,85\% & 6,26\% \\
\ditto{Від} чавуноливарень\dotfill{} & 39,92\% & 3,63\% \\
\ditto{Від} рибальства\dotfill{} & 57,37\% & 5,21\% \\
\ditto{Від} газівень\dotfill{} & \makebox[0pt][r]{1}26,02\% & \hang{r}{1}1,45\% \\
\ditto{Від} залізниць\dotfill{} & 83,29\% & 7,57\%\hang{l}{\footnote{Там же.}} \\
\end{tabular}
\end{table}
 
\noindent{}Якщо порівняти між собою щочотирирічки періоду 1853--1864~\abbr{рр.},
то ступінь збільшення доходів невпинно зростає. Приміром,
для доходів, що походять із зиску, він 1853--1857~\abbr{рр.}
становить 1,73\% на рік, 1857--1861~\abbr{рр.} — 2,74\% на рік і 1861--1864~\abbr{рр.}
9,30\% на рік. Загальна сума доходів, що підпадають прибутковому
оподаткуванню, становила в Об’єднаному королівстві
1856~\abbr{р.} \num{307.068.898}\pound{ фунтів стерлінґів}, 1859~\abbr{р.} — \num{328.127.416}\pound{ фунтів
стерлінґів}, 1862~\abbr{р.} — \num{351.745.241}\pound{ фунт стерлінґів}, 1863~\abbr{р.} —
\num{359.142.897}\pound{ фунтів стерлінґів}, 1864~\abbr{р.} — \num{362.462.279}\pound{ фунтів стерлінґів},
1865~\abbr{р.} — \num{385.530.020}\pound{ фунтів стерлінґів}\footnote{
Цих чисел для порівняння досить, але коли розглядати їх абсолютно,
то вони фалшиві, бо щорічно «затаюється» може більше ніж
100 мільйонів фунтів стерлінґів доходів. Нарікання Commissioners of Irland
Revenue на систематичне шахрайство, особливо з боку купців і промисловців,
повторюються в кожному їхньому звіті. Приміром, читаємо:
«Одно акційне товариство показало свій належний до оподаткування зиск
у \num{6.000}\pound{ фунтів стерлінґів}, таксатор визначив його у \num{88.000}\pound{ фунтів стерлінґів},
і, кінець-кінцем, податок виплачено з цієї суми. Друга компанія
показала зиск у \num{190.000}\pound{ фунтів стерлінґів} і мусила признатися, що
дійсна сума є \num{250.000}\pound{ фунтів стерлінґів}». (Там же, стор. 42).
}.

Акумуляцію капіталу одночасно супроводили його концентрація
й централізація. Хоч для Англії не існувало офіціяльної
рільничої статистики (в Ірляндії вона існує), проте 10 графств подали
її з власної волі. Тут виявився з неї такий результат, що від
1851 до 1861~\abbr{р.} число фарм, нижчих за 100 акрів, зменшилося з
\num{13.583} до \num{26.567}, отже \num{5.016} фарм сполучилося з більшими фармами\footnote{
«Census etc.», vol. III, p. 29. Твердження Джона Брайта, що 150
землевласникам належить половина англійської землі, а 12 — землевласникам
половина шотляндської, не збито.
}.  Від 1815 до 1825~\abbr{р.} з рухомого майна, що підпадало спадщинному
податкові, не було жодного понад 1 мільйон фунтів стерлінґів;
навпаки, від 1825 до 1835~\abbr{р.} їх було 8, від 1856 до червня
1859~\abbr{р.}, тобто за 4\sfrac{1}{2} роки, — 4\footnote{
«Fourth Report etc. of Inland Revenue», London 1860, p. 17.
}. Однак централізацію найкраще
можна побачити з короткої аналізи прибуткового податку
в рубриці D (зиски, за винятком фармерських і~\abbr{т. ін.}) за роки
1864 і 1865. Спочатку зауважу, що доходи з цього джерела,
які розміром не нижчі за 60\pound{ фунтів стерлінґів}, підпадають
income tax\footnote*{
— прибутковому податкові. \emph{Ред.}
}. Ці доходи, що підпадають оподаткуванню, становили
в Англії, Велзі й Шотляндії 1864~\abbr{р.} \num{95.844.222}\pound{ фунтів стерлінґів}
\index{i}{0560}  %% посилання на сторінку оригінального видання
і 1865~\abbr{р.} — \num{105.435.579}\pound{ фунтів стерлінґів}\footnote{
Це — чисті доходи, отже, після того, коли вже відлічено з них
певні суми, визначені законом.
}, число оподаткованих
осіб 1864~\abbr{р.} — \num{308.416} на загальну кількість людности в
\num{23.891.009}, 1865~\abbr{р.} — \num{332.431} особа на загальну кількість людности
в \num{24.127.003}. Про розподіл цих доходів за обидва роки
довідуємося з цієї таблиці:
\begin{table}[H]
\centering
\noindent\begin{tabular}{lrrrr}
\toprule
& \multicolumn{2}{c}{\makecell{Рік, що кінчається \\ 5 квітня 1864~\abbr{р.}}} & 
\multicolumn{2}{c}{\makecell{Рік, що кінчається \\ 5 квітня 1865~\abbr{р.}}} \\
\cmidrule(rl){2-3}
\cmidrule(rl){4-5}
& \makecell[r]{Доходи від \\ зиску,\pound{(фунтів стерлінґів)}}&
\makecell[r]{Число \\ осіб} &
\makecell[r]{Доходи від \\  зиску ,\pound{(фунтів стерлінґів)}} &
\makecell[r]{Число \\ осіб} \\
\midrule
Загальний дохід\dotfill{} & \num{95.844.222} & \num{308.416} & \hang{r}{1}\num{05.435.738} & \num{332.431} \\
З того\dotfill{} & \num{57.028.289}  & \num{23.334}  & \num{64.554.297}  & \num{24.265} \\
\ditto{З} \ditto{того}\dotfill{}& \num{36.415.225} & \num{3.619}  &  \num{42.535.576} &  \num{4.021} \\
\ditto{З} \ditto{того}\dotfill{} & \num{22.809.781} &  832  &  \num{27.555.313}  &     973 \\
\ditto{З} \ditto{того}\dotfill{} & \phantom{0}\num{8.744.762}  &  91   &   \num{11.077.238}  &     107 
\end{tabular}
\end{table}
\noindent{}1855~\abbr{р.} в Об’єднаному королівстві випродуковано \num{61.453.079}
тонн вугілля вартістю в \num{16.113.167}\pound{ фунтів стерлінґів}, 1864~\abbr{р.} —
\num{92.787.873} тонни вартістю в \num{23.197.968}\pound{ фунтів стерлінґів}; 1855~\abbr{р.} —
\num{3.218.154} тонни чавуну вартістю в \num{8.045.385}\pound{ фунтів стерлінґів},
1864~\abbr{р.} — \num{4.767.951} тонну вартістю в \num{11.919.877}\pound{ фунтів стерлінґів}.
1854~\abbr{р.} довжина залізниць, експлуатованих в Об’єднаному
королівстві, становила \num{8.054} милі з капіталовкладенням
у \num{286.068.794}\pound{ фунти стерлінґів}, 1864~\abbr{р.} довжина у милях становила
\num{12.789}, а вкладений капітал — \num{425.719.613}\pound{ фунтів стерлінґів}.
Загальний експорт і імпорт Об’єднаного королівства
становив 1854~\abbr{р.} \num{268.210.145}\pound{ фунтів стерлінґів}, 1865~\abbr{р.} —
\num{489.923.285}. Нижченаведена таблиця показує рух експорту:

\begin{table}[H]
\centering
\noindent\begin{tabular}{lr}

\emph{Роки} & \emph{\pound{Фунтів стерлінґів}} \\

1846\dotfill{} & \num{58.842.377} \\

1849\dotfill{} & \num{63.596.052} \\

1856\dotfill{} & \num{115.826.948} \\

1860\dotfill{} & \num{135.842.817} \\

1865\dotfill{} & \num{165.862.402} \\

1866\dotfill{} & \num{188.917.563}\hang{l}{\footnote{
В цей момент, березень 1867~\abbr{р.}, індійсько-китайський ринок
уже знову переповнений комісійними товарами брітанських бавовняних
фабрикантів. 1866~\abbr{р.} почалося зниження заробітної плати бавовняних
робітників на 5\%, в 1867~\abbr{р.} в наслідок подібних операцій стався страйк
\num{20.000} робітників у Престоні. [Це був пролог кризи, що вибухла одразу
після того. — \emph{Ф.~Е.}].
}} \\
\end{tabular}
\end{table}

\noindent{}Після цих небагатьох даних стає зрозумілим тріюмфальний
крик генерального реєстратора брітанського народу: «Хоч і
як швидко зростала людність, вона не встигала за проґресом промисловости
й багатства»\footnote{
«Census etc.», там же, стор. 11.
}. А тепер вдаймося до безпосередніх
аґентів цієї промисловости або до продуцентів цього багатства,
до робітничої кляси. «Це одна з найсумніших характеристичних
\parbreak{}  %% абзац продовжується на наступній сторінці
