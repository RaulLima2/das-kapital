\parcont{}  %% абзац починається на попередній сторінці
\index{i}{0011}  %% посилання на сторінку оригінального видання
вартости сурдутів підвищується разом з їхньою кількістю. Коли
1 сурдут репрезентує $х$, то 2 сурдути репрезентують $2х$ робочих
днів і т. д. Але припустімо, що праця, доконечна для продукції
одного сурдута, збільшується удвоє або зменшується наполовину.
В першому випадку один сурдут має стільки вартости, скільки
раніше мали два сурдути, в останньому випадку два сурдути
мають лише стільки вартости, скільки раніше мав один сурдут,
хоч в обох випадках один сурдут виконує ту саму службу, що й
раніш, і вміщена в ньому корисна праця має ту саму якість, що
й раніш. Але кількість праці, витрачена на його продукцію,
змінилась.

Більша кількість споживної вартости становить сама по собі
більше речове багатство: два сурдути більше, ніж один; двома
сурдутами можна одягти двох людей, одним сурдутом тільки одну
людину й т. д. Проте, збільшенню маси речового багатства може
відповідати одночасне зменшення величини його вартости. Цей
протилежний рух походить з двоїстого характеру праці. Продуктивна
сила є, звичайно, завжди продуктивна сила корисної, конкретної
праці, і в дійсності вона визначає тільки ступінь діяльности
доцільної продуктивної праці в даний період часу. Тому
корисна праця стає багатшим або біднішим джерелом продуктів
просто пропорційно до підвищення або падіння її продуктивної
сили. Навпаки, зміна продуктивної сили сама по собі зовсім не
зачіпає праці, репрезентованої вартістю. А що продуктивна сила належить
до конкретної корисної форми праці, то вона, природно, не
може вже більше торкатися праці, скоро тільки ми абстрагувались
від її конкретної корисної форми. Тому та сама праця дає в однакові
періоди часу завжди однакову величину вартости, хоч і як
змінюватиметься продуктивна сила. Але вона дає в однаковий
період часу різні кількості споживних вартостей: більше — коли
продуктивна сила підвищується, і менше коли вона падає. Отже,
та сама зміна продуктивної сили, що збільшує видатність праці,
а тим самим і масу даваних нею споживних вартостей, зменшує
величину вартости цієї збільшеної загальної маси, якщо вона
скорочує суму робочого часу, доконечного для її продукції. І так
само навпаки.

Всяка праця є, з одного боку, затрата людської робочої сили
У фізіологічному розумінні, і в цій властивості однакової людської
або абстрактної людської праці вона творить (bildet) товарову
вартість. Всяка праця є, з другого боку, затрата людської робочої
сили в осібній доцільно-визначеній формі, і в цій властивості
конкретної корисної праці вона продукує споживні вартості.\footnote{
Примітка до другого видання. Щоб довести, «що сама тільки праця є
остаточна й реальна міра, якою можна цінувати й порівнювати вартість
усіх товарів у всі часи», \emph{А. Сміс} каже: «Рівні кількості праці мусять в
усі часи й по всіх місцях мати для самого робітника однакову вартість.
При нормальному стані свого здоров'я, сили й діяльности і пересічному
ступені вправности, яку він може мати, він мусить завжди віддавати однакову
частину свого спокою, своєї волі й свого щастя». («Wealth of Nations»,
b. I, ch. V, p. 104, 105). З одного боку, А. Сміс плутає тут (не скрізь)
визначення вартости кількістю праці, витраченої на продукцію товару
з визначенням товарової вартости вартістю праці, і тому він силкується
довести, що рівні кількості праці завжди мають однакову вартість. З другого
боку, він передчуває, що праця, оскільки вона виражається у вартості
товарів, є лише витрата робочої сили, але цю витрату він знов таки
бере тільки як жертву спокоєм, волею та щастям, а не як нормальну життєву
діяльність. Правда, він має на оці тільки сучасного найманого робітника.
— Куди влучніше міркує цитований вже в дев’ятій примітці анонімний
попередник А. Сміса: «Одна людина затратила тиждень на виготовлення
певної речі, потрібної для життя\dots{} і той, хто дає їй на обмін якусь
іншу річ, може найкраще оцінити, яка кількість її становить еквівалент
першої речі, лише вирахувавши, що саме йому коштувало точно стільки ж
часу й праці; це в дійсності сходить на те, що працю однієї людини, витрачену
на продукцію якоїсь речі протягом якогось часу, обмінюється
на працю іншої людини, витрачену протягом того самого часу на продукцію
іншої речі» («One man has employed himself a week in providing this
necessary of life\dots{} and he that gives him some other in exchange, cannot
make a better estimate of what is a proper equivalent, than by computing what
cost him just as much labour and time: which in effect is no more tran exchanging
one man’s labour in one thing for a time certain for another an’s labour
in another thing for the same time»). («Some Thoughts on the Interestofmoney
in general etc.», p. 39). — [«До 4 видання: Англійська мова має
ту перевагу, що в ній є два різні слова для означення цих двох різних аспектів
праці. Якісно визначену працю, що утворює споживні вартості,
називається Work протилежно до Labour; працю, що творить вартості
й виміряється лише кількісно, називається Labour протилежно до Work.
Див. примітку до англійського перекладу, стор. 14. — \emph{Ф. Е}].}
\index{i}{0012}  %% посилання на сторінку оригінального видання
[А що ми досі визначили лише субстанцію вартости й величину
вартости, то перейдімо тепер до аналізи форми вартости].\footnote*{
Заведене у прямі дужки ми беремо з першого нім. видання. \emph{Ред.}
}

\subsubsection{Форма вартости або мінова вартість}

Товари з’являються на світ у формі споживних вартостей, або
товарових тіл, як залізо, полотно, пшениця тощо. Це їхня доморосла
натуральна форма. Однак товари вони є лише через свою
двоїстість, лише через те, що вони є предмети споживання й одночасно
носії вартости. Тому вони являють собою товари або мають
форму товарів лише остільки, оскільки мають двоїсту форму:
форму натуральну й форму вартости.

Предметність вартости (Wertgegenständlichkeit) товарів тим
відрізняється від удовиці Квіклі,\footnote*{Дієва особа в комедії
Шекспіра «Веселі віндзорські молодиці».
\emph{Ред.}
} що невідомо, де її можна
знайти. Цілком протилежно до почуттєво-грубої предметности
товарових тіл жоден атом природної речовини не входить у їхню
вартостеву предметність. Тому можна крутити й вивертати поодинокий
товар на всі боки — як предмет вартости (Wertding)
він лишається несхопним. Однак, коли ми пригадаємо, що товари
мають предметність вартости лише остільки, оскільки вони є
вирази однакової суспільної одиниці, людської праці, і що, отже,
їхня вартостева предметність є суто суспільна, тоді стане також
само собою зрозуміло, що вона може виявитись тільки в суспільному
відношенні одного товару до іншого. Справді, ми виходили
\parbreak{} (¬ абзац продовжується на наступній сторінці)
