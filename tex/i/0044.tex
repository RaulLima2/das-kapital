цікавить людей. Нас же вона, як речей, не стосується. Що нас,
власне, стосується як речей, так це наша вартість. Це показують
наші власні взаємовідносини як речей-товарів. Ми відносимось
один до одного лише як мінові вартості. А послухаймо
тепер, як економіст висловлює думки товарової душі: «Вартість
(мінова вартість) є властивість речей, багатство (споживна вартість)
є властивість людей. У цьому розумінні вартість неодмінно
включає обмін, багатство ж — ні».\footnote{
«Value is a property of things, riches of man. Value, in this sense,
necessarily implies exchanges, riches do not». («Observations on some verbal
disputes in Pol. Econ., particularly relating to value and to supply and
demand», London 1821, p. 16).
} «Багатство (споживна вартість) —
це атрибут людини, вартість — атрибут товарів. Людина
або суспільство багаті; перла або діямант мають вартість...
Перла або діямант мають вартість як перла або діямант».\footnote{
«Riches are the attribute of man, value is the attribute of commodities.
A man or community is rich, a pearl or a diamond is valuable...
A pearl or a diamond is valuable as a pearl or diamond». (S. Bailey:
«А critical Dissertation on the Nature etc. of value», p. 165).
}
Досі ще жоден хемік не винайшов у перлі або в діяманті мінової
вартости. Але економісти-винахідники цієї хемічної субстанції,
які претендують на особливу глибину критичної думки, вважають,
що споживна вартість речей незалежна від їхніх речових властивостей,
а вартість, навпаки, властива їм як речам. Їх запевняє
в цьому та дивна обставина, що споживна вартість речей реалізується
для людини без обміну, отже, в безпосередньому відношенні
між річчю й людиною, тимчасом як їхня вартість, навпаки,
реалізується лише в обміні, тобто в певному суспільному процесі.
Хто не пригадає собі тут сердечного Доґбері, який навчає нічного
сторожа Сіколя: «Бути приємною людиною, це — дар обставин,
а вміння читати й писати дає природа».\footnote{
Автор «Observations» і L. Bailey обвинувачують Рікарда в тому,
що він перетворив мінову вартість з чогось-то лише відносного на щось
абсолютне. Навпаки. Позірну відносність, яку мають ці речі, приміром,
діямант і перла, як мінові вартості, він звів до захованого за цією позірністю
дійсного відношення — до їхньої відносности як простих виразів
людської праці. Коли рікардіянці відповіли Ваіlеу’єві грубо, але невлучно,
то лише через те, що в самого Рікарда вони не знайшли жодного
пояснення щодо внутрішнього зв’язку між вартістю й формою вартости,
або міновою вартістю.
}

Розділ другий
Процес обміну

Товари сами не можуть ходити на ринок і сами не можуть
обмінюватися. Отже, ми мусимо звернутися до їхніх хоронителів —
до посідачів товарів. Товари є речі й тим то не можуть вони чинити
опору людям. Як вони не йдуть з доброї волі, то людина може
вжити сили, тобто взяти їх.\footnote{
В XII віці, так уславленому побожністю, серед цих товарів часто
траплялися дуже делікатні речі. Так, один французький поет того часу
} Щоб ці речі ставити одну до однієї