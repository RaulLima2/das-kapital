Відповідно до збільшення маси сировинного матеріялу, півфабрикатів,
робочих інструментів і т. ін., що їх дає машинове
виробництво при порівняно меншому числі робітників,
оброблення цього сировинного матеріялу та півфабрикатів розподіляється
на численні підроди, отже, зростає різноманітність
галузей суспільної продукції. Машинове виробництво розвивав
суспільний поділ праці незрівнянно більше, ніж мануфактура,
бо воно в незрівнянно вищій мірі збільшує продуктивну силу
захоплених ним галузей промисловості.

Найближчий результат машинової системи в тому, що вона
збільшує додаткову вартість, а разом з тим масу продуктів, у
якій вона втілена, отже, з тією субстанцією, з якої живиться
кляса капіталістів та її похлібці, вона збільшує самі ці суспільні
верстви. Зріст багатства цих останніх та постійне відносне меншання
числа робітників, потрібних для продукції доконечних
засобів існування, породжують разом із новими потребами в розкошах
і нові засоби задоволення їх. Більша частина суспільного
продукту перетворюється на додатковий продукт, більша частина
додаткового продукту репродукується і споживається в щораз
витонченіших та різноманітніших формах. Інакше кажучи: продукція
предметів розкошів зростає.\footnote{
Ф. Енґельс у «Lage der arbeitenden Klasse in England» («Становище
робітничої кляси в Англії») відзначає злиденний стан великої частини
саме цих продуцентів предметів розкошів. Про численні нові докази
цього див. у звітах «Children’s Employmeni Commission».
} Витонченість та різноманітність
продуктів випливає так само з нових відносин світового
ринку, створюваних великою промисловістю. Не тільки
більшу кількість закордонних засобів споживання вимінюється
на тубільний продукт, але й більша маса чужих сировинних матеріялів,
складових елементів, півфабрикатів тощо ввіходить
як засіб продукції в тубільну промисловість. З розвитком цих
відносин світового ринку зростає попит на працю в транспортовій
промисловості, і остання розпадається на численні нові
підроди.\footnote{
1861 p. в Англії та Велзі в торговельній фльоті працювало 94.665
моряків.
}

Збільшення засобів продукції й засобів існування при відносному
меншанні числа робітників спонукає поширювати працю в
таких галузях промисловости, продукти яких, як от канали,
товарові доки, тунелі, мости й т. д., дають плоди у далекій будучині.
Безпосередньо на основі машинової системи або на основі
відповідного їй загального промислового перевороту, постають
цілком нові галузі продукції, а тому й нові сфери праці. Однак
участь їхня в загальній продукції навіть у найрозвиненіших
країнах незначна. Число вживаних ними робітників більшає
прямо пропорційно до того, наскільки репродукується доконечність
щонайгрубішої ручної праці. Головними галузями промисловости
цього роду в наші часи можна вважати газові заводи,