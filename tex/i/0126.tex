він дозволяє покупцеві споживати її раніш, ніж останній заплатив
її ціну, отже, повсюди робітник кредитує капіталіста. Що це
кредитування не є пуста химера, показує не лише втрата кредитованої
заробітної плати на випадок банкрутства капіталіста,50
але й низка затяжніших наслідків.51 Проте природа самого товарового
обміну аніскільки не змінюється від того, як функціонують
гроші: як засіб купівлі, чи як засіб платежу. Ціну робочої сили
встановлено контрактом, хоч реалізується вона лише пізніш,
як і наймова ціна за будинок. Робочу силу вже продано, хоч
плату за неї буде виплачено лише пізніше. Однак для ясного розуміння
цього відношення корисно покищо тимчасово припустити,
що посідач робочої сили щоразу одночасно з продажем її дістає
зразу ж і умовлену в контракті ціну.

50 «Робітник позичає свою працю» — каже Шторх, але додає єхидно:
«він нічим не ризикує», хіба що «втратою свого заробітку... робітник не
додає нічого матеріяльного». («L’ouvrier prête son industrie... il ne risque
rien, exepté de perdre son salaire... l’ouvrier ne transmet rien de matériel»).
(Storch: «Cours d’Economie Politique», Pétersbourg 1815, vol. 2, p. 37).

51 Приклад. У Лондоні є два сорти пекарів: «full priced», що продають
хліб за його повного вартістю, і «undersellers», що продають його
нижче його вартости. Остання кляса становить понад три четвертини
загального числа всіх пекарів. (P. 32 в «Report» урядового комісара
H. S. Tremenheere про «Grievances complained of by the journemeymen
bakers etc.», London 1862). Ці undersellers продають, майже без винятку,
хліб, фальсифікований домішкою галуну, мила, поташу, вапна, дербішірської
кам’яної муки та інших подібних приємних, поживних і здорових
інґредієнтів. (Див. цитовану вище Синю Книгу, а також звіт «Committee
of 1855 on the Adulteration of Bread» і твір доктора Hassall’a: «Adulterations
Detected», 2-nd ed. London 1862). Сер Джон Ґордон заявив
перед комітетом 1855 р., що «в наслідок такої фальсифікації бідний, який
живе з двох фунтів хліба на день, в дійсності не дістає тепер і четвертини
поживного матеріялу, не кажучи вже про шкідливий вплив на його здоров’я».
Як причину того, що «дуже велика частина робітничої кляси»,
хоч і добре знає про фальсифікацію, все ж купує галун, кам’яну муку
й т. ін., Tremenheere наводить те (1. с. p. 48), що для них «є неминуче
брати такий хліб у свого пекаря або в крамаря, який цьому останньому
забажається їм дати». Через те, що плату вони дістають лише наприкінці
тижня, вони можуть «за спожитий їхньою родиною протягом
тижня хліб заплатити лише наприкінці тижня»; і Tremenheere додає,
покликаючись на свідків: «загальновідома річ, що хліб із такими домішками
виготовляється виключно для цього роду покупців» («it is notorius
that bread composed of those mixtures, is made expressly for sale in this
manner»). «У багатьох рільничих округах Англії (а ще більш у шотляндських)
заробітну плату видається раз на два тижні й навіть раз на місяць.
Через такі довгі реченці платежу рільничий робітник примушений купувати
собі товари на кредит... Він мусить платити вищі ціни й фактично
прикований до тієї крамниці, що його висмоктує. Так, приміром, у Ногningsham
in Wilts, де заробітна плата щомісячна, таке саме борошно, за
яке в іншому місці він заплатив би 1 шилінґ 10 пенсів, йому коштує 2 шилінґи
4 пенси за stone». («Sixth Report» on «Public Health» by «The
Medical Officer of the Privy Council etc.», 1864, p. 264). «Робітники перкалево-вибійчаних
майстерень у Песлі й Кільмарноці (Західня Шотляндія)
страйком примусили в 1853 р. скоротити реченець платежу з одного
місяця на два тижні». («Reports of the Inspectors of Faktories for 31 st
October 1853», p. 34). Як приклад дальшого чемного розвитку кредиту,
що його дає робітник капіталістові, можна розглядати методу багатьох
