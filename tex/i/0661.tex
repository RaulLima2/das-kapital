слуги, який міг би постелити йому постіль або набрати води з річки.257 Безталанний пан Піл! Він усе
передбачив, та забув лише експортувати англійські продукційні відносини на Лебединий берег.

Щоб зрозуміти дальші відкриття Векфілда, потрібні два попередні зауваження. Ми знаємо, що коли
засоби продукції та засоби існування є власність безпосереднього продуцента, то вони не є капітал.
Вони стають капіталом лише за таких умов, коли вони разом з тим служать за засоби експлуатації та за
засоби упідлеглення робітника. Але ця їхня капіталістична душа в голові політико-економа з’єднана
таким тісним подружнім зв’язком із їхньою речовою субстанцією, що він за всяких обставин називає їх
капіталом, навіть і тоді, коли вони є сáме протилежність капіталу. Так стоїть справа й у Векфілда.
Далі: роздрібнення засобів продукції, як індивідуальної власности багатьох незалежних один від
одного, самостійно господарюючих робітників, він називає рівним поділом капіталу. З
політико-економом трапляється те саме, що і з февдальним юристом. Цей останній і на суто грошові
відносини наклеює свої февдальні правні етикетки.

«Коли б, — каже Векфілд, — капітал був поділений поміж усіма членами суспільства рівними пайками, то
жодна людина не була б заінтересована в тому, щоб акумулювати капіталу більш, ніж вона може
застосувати своїми власними руками. Так до певної міри стоїть справа в нових американських колоніях,
де жадоба до земельної власности перешкоджає існуванню кляси найманих робітників».258 Отже, поки
робітник має змогу акумулювати для себе самого, — а це він може робити, поки він лишається власником
своїх засобів продукції, — доти капіталістична акумуляція й капіталістичний спосіб продукції
неможливі. Бракує доконечної для цього кляси найманих робітників. Але як же тоді в старій Европі
здійснено експропріяцію в робітника його умов праці, яким чином, отже, створено там капітал і
найману працю? За допомогою contrat social* дуже ориґінального характеру. «Людство... засвоїло собі
просту методу активізувати акумуляцію капіталу», яка, звичайно, від часів Адама здавалась йому
останньою й єдиною метою його буття: «воно поділилось на власників капіталу і власників праці... цей
поділ був результатом добровільного порозуміння та погодження» (Kombination).259 Одне слово, маса
людства сама себе експропріювала на славу «акумуляції капіталу». А тепер треба б думати, що інстинкт
цього самовідданого фанатизму мусив би вільно виявитися саме в колоніях, де тільки й існують люди й
умови, які могли б перенести con-

257 Е. G. Wakefield: «England and America». London 1833, vol. II, p. 33.
258 Там же, т. І, стор. 17, 18.
259 Там же, стор. 18.
* — суспільного договору. Ред.
