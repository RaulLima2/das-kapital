\parcont{}  %% абзац починається на попередній сторінці 
\index{i}{0547}  %% посилання на сторінку оригінального видання 
ніж відповідне відносне зменшення змінної частини капіталу
проти сталої частини. Якщо засоби продукції, в міру того як
вони збільшуються щодо розміру й сили діяльности, у щораз
меншому ступені стають засобами, які дають заняття робітникам,
то саме це відношення знов таки модифікується тим, що в міру
того, як зростає продуктивна сила праці, капітал швидше збільшує
своє подання праці, ніж свій попит на робітників. Надмірна
праця занятої частини робітничої кляси збільшує її резервні
ряди, тимчасом як, навпаки, збільшений тиск, який резервні
ряди справляють своєю конкуренцією на занятих робітників,
примушує останніх надмірно працювати й коритися наказам
капіталу. Те, що одна частина робітничої кляси в наслідок надмірної
праці другої її частини засуджена на примусове безділля
і навпаки, — це стає засобом збагачення для поодинокого капіталіста\footnote{
Навіть за часів бавовняного голоду 1863 р. в одному памфлеті
прядунів бавовни з Blackburn’а ми находимо гострі нарікання на надмірну
працю, що силою фабричного закону припадала, звичайно, лише
на дорослих робітників-чоловіків. «Від дорослих робітників цієї фабрики
вимагали щодня дванадцяти-тринадцятигодинної праці, тимчасом
як сотні робітників примушені лишатися без роботи, хоч охоче згодились
би працювати неповний час, аби тільки утримати свої родини і врятувати
своїх товарішив од передчасної смерти через надмірну працю» («The
adult operatives at this mill have been asked to work from 12 to 13 hours
per day, while there are hundreds who are compelled to be idle who would
willingly work partial time, in order to maintain their families and save
their brethren from a premature grave through being overworked»). «Ми, —
сказано далі, — хотіли б запитати, чи можливі стерпні відносини поміж
господарем і «слугами» за цієї практики працювати наднормовий час?
Жертви надмірної праці так само відчувають кривду, як і засуджені
нею на примусове безділля (condemned to forced idleness). Коли б працю
розподіляли справедливо, то в цій окрузі її вистачило б, щоб дати всім
частинну роботу. Ми вимагаємо лише права, вимагаючи від хазяїнів загального
скорочення робочого часу, принаймні на той час, доки триває
тепершній стан речей, замість примушувати одну частину до надмірної
праці, тимчасом як друга через брак праці примушена животіти коштом
добродійності!». («Reports of lnsp. of Fact. 31st October 1863», p.8).
Автор «Essay on Trade and Commerce» із своїм звичним непомильним
інстинктом буржуа розуміє вплив відносного перелюднення на занятих
робітників. «Друга причина ледарства (idleness) в цьому королівстві —
це брак достатнього числа робочих рук. Скоро тільки в наслідок якогось
незвичайного попиту на фабрикати маса праці стає недостатньою, робітники
відчувають свою власну вагу й хочуть дати відчути її також своїм
хазяїнам; це дивна річ; алеж голови цих суб’єктів такі зіпсовані, що в цих
випадках групи робітників з’єднуються між собою, щоб завдати клопоту
своїм хазяїнам, б’ючи байдики цілісінький день». («An Essay on Trade
and Commerce», London 1770, p. 27, 28). Ці суб’єкти домагалися власне
підвищення заробітної плати.
} й разом з тим прискорює продукцію промислової резервної
армії в маштабі, що відповідає проґресові суспільної
акумуляції. Яку вагу має цей момент у творенні відносного перелюднення,
доводить, наприклад, Англія. Її технічні засоби до
«заощаджування» праці колосальні. А проте, коли б завтра
повсюди обмежити працю раціональним розміром і розподілити
її між різними верствами робітничої кляси за ґрадаціями відповідно
до віку й статі, то наявної робітничої людности абсолютно
\parbreak{}  %% абзац продовжується на наступній сторінці
