номірно перебуватимуть то у відносній формі вартости, то в еквівалентній
формі. Тут ще важко фіксувати цю полярну протилежність.
У формі В завжди лише один якийсь рід товару може цілком
розгорнути свою відносну вартість, або сам він має розгорнуту
відносну форму вартости лише тому й остільки, що й оскільки
всі інші товари протистоять йому в еквівалентній формі. Тут уже
не можна переставити обидві частини вартостевого рівнання: 20 метрів
полотна = 1 сурдутові, або = 10 фунтам чаю, або = 1 квартерові
пшениці й т. ін., не змінюючи його загального характеру
й не перетворюючи його з повної на загальну форму вартости.

Остання форма, форма С, дає, нарешті, товаровому світові
загальносуспільну відносну форму вартости тому й остільки,
що й оскільки всі належні до нього товари, за одним-однісіньким
винятком, є виключені з загальної еквівалентної форми. Тому
один товар, полотно, перебуває у формі безпосередньої вимінности
на всі інші товари, або в безпосередньо суспільній формі, тому
й остільки, що й оскільки всі інші товари не перебувають у такій
формі.24

Навпаки, товар, що фігурує як загальний еквівалент, є виключений
з однорідної, а тому й загальної відносної форми вартости
товарового світу. Для того, щоб полотно, тобто будь-який
товар, що перебуває у формі загального еквіваленту, разом з тим
брав участь також і в загальній відносній формі вартости, він
мусив би служити за еквівалент для самого себе. Ми мали б тоді:
20 метрів полотна = 20 метрам полотна — тавтологію, де не виражено
ні вартости, ані величини вартости. Щоб виразити відносну
вартість загального еквіваленту, ми скорше мусимо обернути
форму С. Він не має спільної з іншими товарами відносної
форми вартости, але його вартість відносно виражається у без-

24 В дійсності на перший погляд з форми загальної безпосередньої
вимінности ніяк не пізнати, що вона є суперечна товарова форма, так
само невіддільна від протилежної форми, що в ній неможлива безпосередня
вимінність, як позитивний полюс магнету від його негативного полюса.
Тим то уявити собі, що на всі товари можна одночасно наложити печать
безпосередньої вимінности можна з таким самим успіхом, як можна
уявити собі, що всіх католиків можна одночасно поробити папами. Для
дрібного буржуа, який у товаровій продукції бачить nec plus ultra *
людської волі й індивідуальної незалежности, було б, натурально, дуже
бажано позбавитись невигід, зв’язаних з цією формою, особливо ж тієї, що
товари не можуть вимінюватись безпосередньо. Розмальовування цієї філістерської
утопії становить прудонівський соціялізм, який, як я це показав
в іншому місці, не має в собі навіть нічого ориґінального і який далеко
раніш значно краще розвинули були Gray, Bray і ін. Та це не заважає
отакій мудрості ще й нині, як якійсь пошесті, ширитися в певних колах
під ім’ям «science».** Жодна школа не панькалась так із словом «science»,
як прудонівська, бо

«Wo Begriffe fehlen,

da stellt zur rechten Zeit ein Wort sich ein»

(«Де бракує понять, там саме в пору з’являється слово»).

* — вершину. Ред
** — «наука». Ред.
