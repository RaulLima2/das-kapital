\parcont{}  %% абзац починається на попередній сторінці 
\index{i}{0428}  %% посилання на сторінку оригінального видання 
форми репродукується подекуди і на базі великої промисловости,
хоч і з цілком зміненою фізіономією.

Коли, з одного боку, для продукції абсолютної додаткової
вартости досить лише формальної підпорядкованости праці капіталові,
досить, наприклад, того, щоб ремісники, які раніш працювали
на самих себе абож як підмайстри цехового майстра,
стали тепер як наймані робітники під безпосередній контроль
капіталіста, то, з другого боку, виявилося, що методи продукції
відносної додаткової вартости є разом з тим методи продукції
абсолютної додаткової вартости. Аджеж безмірне здовження
робочого дня виявилось як найхарактеристичніший продукт великої
промисловости. Взагалі специфічно-капіталістичний спосіб
продукції перестає бути простим засобом продукувати відносну
додаткову вартість, скоро тільки він опановує цілу галузь
продукції, і ще більше — скоро він опановує всі вирішальні
галузі. Він стає тепер загальною, суспільно-панівною формою
подукційного процесу. Як осібна метода продукувати відносну
додаткову вартість, він діє ще лише остільки, оскільки, поперше,
захоплює галузі продукції, досі лише формально підпорядковані
капіталові, отже, поширюючи сферу свого впливу; подруге,
остільки, оскільки галузі промисловости, що підпали вже під
його руку, постійно революціонізуються через зміну метод продукції.

З певного погляду ріжниця між абсолютною і додатковою вартістю
видається взагалі ілюзорною. Відносна додаткова вартість
є абсолютна, бо вона зумовлює абсолютне здовження робочого
дня поза робочий час, доконечний для існування самого робітника.
Абсолютна додаткова вартість є відносна, бо вона зумовлює:
розвиток такої продуктивности праці, що дозволяє обмежити
доконечний робочий час певною частиною робочого дня. Але
коли звернути увагу на рух додаткової вартости, то ця позірна
тотожність зникає. Скоро тільки капіталістичний спосіб продукції
виник і став загальним способом продукції, ріжниця між абсолютною
й відносною додатковою вартістю стає відчутною, коли
йдеться про підвищення норми додаткової вартости взагалі.
Коли припустити, що робочу силу оплачується за її вартістю,
то ми опиняємось перед такою альтернативою: при даній продуктивності
праці й нормальному ступені її інтенсивности норму
додаткової вартости можна підвищити тільки через абсолютне
здовження робочого дня; з другого боку, при даних межах робочого
дня норму додаткової вартости можна підвищити лише через
зміну відносних величин його складових частин, доконечної
праці й додаткової праці, а це, з свого боку, має собі за передумову
зміну продуктивности або інтенсивности праці, якщо
заробітна плата не повинна впасти нижче вартости робочої сили.

Якщо робітник потребує всього свого часу на те, щоб продукувати
засоби існування, потрібні для утримання його самого і
його родини, то йому не лишається вже часу задурно працювати
на третіх осіб. Без певного ступеня продуктивности праці в робітника
\index{i}{0429}  %% посилання на сторінку оригінального видання 
не може бути такого вільного часу, без такого надлишкового
часу не може бути додаткової праці, а тому й капіталістів,
але також не може бути й рабовласників, февдальних баронів,
одне слово, жодної кляси великих власників.\footnote{
«Саме існування капіталістичних підприємців як осібної кляси
залежить від продуктивности праці» («The very existence of the master-capitalists
as a distinct class is dependent on the productiveness
of industry») - (Ramsay: «An Essay on the Distribution of Wealth»,
Edinburgh 1836, p. 206). «Коли б праці кожної людини вистачало лише
для продукції її власних засобів існування, то не могло б бути й власности»
(«If each man’s labour were but enough to produce his own food,
there could be no property»). (Ravenstone: «Thoughts on the Funding
System», London 1824, p. 14, 15).
}

Таким чином можна говорити про природну базу додаткової
вартости, але лише в тому цілком загальному розумінні, що в
природі немає жодної абсолютної перешкоди, яка б не дозволяла
одній людині звалювати з себе на іншу людину працю, потрібну
для її власного існування, наприклад, так само, як у природі
не існує жодних абсолютних перешкод для того, щоб одна людина
вживала для харчування м’яса іншої.\footnoteA{
На основі нещодавно зробленого обчислення лише в досліджених
уже частинах землі живе ще щонайменше чотири мільйони канібалів.
} Ні в якому разі не
слід, як це іноді робилося, сполучати містичні уявлення з
цим стихійним розвитком продуктивности праці. Тільки тоді,
коли люди тяжкою працею вибилися з свого первісного тваринного
стану, отже, коли сама їхня праця до деякої міри вже є
усуспільнена, — лише тоді постають відносини, за яких додаткова
праця однієї людини стає умовою існування іншої. На початках
культури здобуті продуктивні сили праці незначні, але так само
незначні й потреби, що розвиваються разом з розвитком засобів
для задоволення тих потреб та залежно від цього розвитку. Далі,
на тих початках культури частина суспільства, що живе з чужої
праці, є величина зникомо мала супроти маси безпосередніх продуцентів.
З проґресом суспільної продуктивної сили праці ця
частина зростає абсолютно й відносно.2 Зрештою, капіталістичне
відношення постає на економічному ґрунті, який є продукт
довгого процесу розвитку. Наявна продуктивність праці,
з якої воно виходить як з основи, не є дар природи, а дар історії,
яка охоплює тисячі століть.

Якщо абстрагуватися від більш або менш розвиненої форми
суспільної продукції, то залишається, що продуктивність праці
зв’язана з природними умовами. Всі ці умови можна звести до
природи самої людини, як от раса й т. ін., та до природи, що оточує
людину. Зовнішні природні умови розпадаються з економічного
погляду на дві великі кляси: природне багатство на засоби

* У диких індіян Америки мало не все належить робітникові, 99\%
продуктів припадає робітникові; в Англії на робітника не припадає й
двох третин» («Among the wild Indians in America, almost every thing
is the labourer’s, 99 parts of an hundred are to be put upon the account of
Labour; In England, perhaps the labourer has not 2/3»). («The Advantages
of the East-India Trade etc.», London 1720, p. 73).
\parbreak{}  %% абзац продовжується на наступній сторінці
