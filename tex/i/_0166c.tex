\index{i}{0166}  %% посилання на сторінку оригінального видання
Жакоб, припускаючи ціну пшениці в 80 шилінґів за квартер і пересічний урожай в 22 бушлі з одного
акра, так що один акр приносить 11\pound{ фунтів стерлінґів}, наводить для 1815 р. обрахунок, який через те,
що в ньому вже переведено компенсацію різних
пунктів, дуже хибний, але все ж для нашої мети придатний.

\begin{table}[h]
\caption*{Продукція вартости на 1 акр}
\noindent\begin{tabularx}{\textwidth}{X*{4}{@{~}r}X*{4}{@{~}r}}
Насіння (пшениця)\dotfill{} & 1 & ф. ст. & 9 &шил. &
Десятини, податки\dotfill{} & 1 & ф. ст. & 1 &шил. \\
Добриво\dotfill{} & 2 & \dittomark{} & 10 & \dittomark{} &
Рента\dotfill{} & 1 & \dittomark{} &  8 & \dittomark{} \\

Заробітна плата\dotfill{} &  3  & \dittomark{} &  10 & \dittomark{} &
Зиск фармера й проц & 1 & \dittomark{} & 2 & \dittomark{} \\

\cmidrule(rl){1-5}  \cmidrule(rl){6-10} 

Разом\dotfill{} & 7 & ф. ст. & 9 & шил. &
Разом\dotfill{} & З & ф. ст. & 11 & шил. 

\end{tabularx}
\end{table}



Додаткова вартість, припускаючи завжди, що ціна продукту дорівнює його вартості, розподіляється тут
між різними рубриками: зиск, процент, десятина й т. ін. Ці рубрики для нас не мають значення. Ми
складаємо їх і як результат маємо додаткову вартість у 3\pound{ фунти стерлінґів} 11 шилінґів. Ті 3\pound{ фунти
стерлінґів} 19 шилінґів, що коштують насіння і добриво, ми, як сталу частину капіталу, прирівнюємо
нулеві. Лишається авансований змінний капітал у 3\pound{ фунти стерлінґів} 10 шилінґів, замість якого
спродуковано нову вартість у 3\pound{ фунти стерлінґів} 10 шилінґів + 3\pound{ фунти стерлінґів} 11 шилінґів. Отже,
$\frac{m}{v} = \frac{3\text{ фунти ст. }11\text{ шилінґів}}{3\text{ фунти ст. }10\text{ шилінґів}}$ становить більше, ніж 100\%. Робітник більш
ніж половину свого робочого дня вживає на продукцію додаткової вартости, яку різні особи під різними
приводами  розподіляють проміж себе.\footnoteA{
Наведені обчислення мають значення лише як ілюстрація. Справді, ми припускаємо, що ціни
дорівнюють вартостям. У третій книзі ми побачимо,
що це прирівняння робиться не так просто навіть для пересічних цін.
}

\subsection{Вираз вартости продукту у відносних частинах продукту}

Вернімось тепер до того прикладу, що показав нам, як капіталіст із грошей робить капітал. Доконечна
праця його прядуна
становила 6 годин, додаткова праця — стільки ж, отже, ступінь експлуатації робочої сили — 100\%.

Продукт дванадцятигодинного робочого дня є 20 фунтів пряжі вартістю в 30 шилінґів. Не менше як \sfrac{8}{10}
вартости цієї пряжі (24 шилінґи) становить вартість зужиткованих засобів продукції, що лише знову
з’являється (20 фунтів бавовни на 20 шилінґів, веретена й т. ін. на 4 шилінґи), або, інакше кажучи,
складається з сталого капіталу. Решта, \sfrac{2}{10}, є нова вартість у 6 шилінґів, яка постала підчас
процесу прядіння, що з них половина компенсує авансовану денну вартість робочої сили, або змінний
капітал, а друга половина становить додаткову вартість у 3 шилінґи. Отже, сукупна вартість цих 20
фунтів пряжі складаєься ось як:
