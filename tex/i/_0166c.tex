\index{i}{0166}  %% посилання на сторінку оригінального видання
Жакоб, припускаючи ціну пшениці в 80\shil{ шилінґів} за квартер і пересічний урожай в 22 бушлі з одного
акра, так що один акр приносить 11\pound{ фунтів стерлінґів}, наводить для 1815~\abbr{р.} обрахунок, який через те,
що в ньому вже переведено компенсацію різних
пунктів, дуже хибний, але все ж для нашої мети придатний.

\begin{table}[H]
\caption*{Продукція вартости на 1 акр}
\noindent\begin{tabularx}{\textwidth}{@{}X*{4}{@{~}r}@{\hspace{5em}}X*{4}{@{~}r}@{}}
Насіння (пшениця)\dotfill{} & 1 & \pound{ф. ст.} & 9 &\shil{шил.} &
Десятини, податки\dotfill{} & 1 & \pound{ф. ст.} & 1 &\shil{шил.} \\
Добриво\dotfill{} & 2 & \dittomark{} & 10 & \dittomark{} &
Рента\dotfill{} & 1 & \dittomark{} &  8 & \dittomark{} \\

Заробітна плата\dotfill{} &  3  & \dittomark{} &  10 & \dittomark{} &
Зиск фармера й проц.\dotfill{}& 1 & \dittomark{} & 2 & \dittomark{} \\

\cmidrule(r{5em}){1-5}  \cmidrule{6-10} 

Разом\dotfill{} & 7 & \pound{ф. ст.} & 9 & \shil{шил.} &
Разом\dotfill{} & З & \pound{ф. ст.} & 11 & \shil{шил.}
\end{tabularx}
\end{table}

\noindent{}Додаткова вартість, припускаючи завжди, що ціна продукту дорівнює його вартості, розподіляється тут
між різними рубриками: зиск, процент, десятина й~\abbr{т. ін.} Ці рубрики для нас не мають значення. Ми
складаємо їх і як результат маємо додаткову вартість у 3\pound{ фунти стерлінґів} 11\shil{ шилінґів.} Ті 3\pound{ фунти
стерлінґів} 19\shil{ шилінґів}, що коштують насіння і добриво, ми, як сталу частину капіталу, прирівнюємо
нулеві. Лишається авансований змінний капітал у 3\pound{ фунти стерлінґів} 10\shil{ шилінґів}, замість якого
спродуковано нову вартість у 3\pound{ фунти стерлінґів} 10\shil{ шилінґів} + 3\pound{ фунти стерлінґів} 11\shil{ шилінґів.} Отже,
$\frac{m}{v} = \frac{3\text{\pound{ фунти ст.} }11\text{\shil{ шилінґів}}}{3\text{\pound{ фунти ст.} }10\text{\shil{ шилінґів}}}$ становить більше, ніж 100\%. Робітник більш
ніж половину свого робочого дня вживає на продукцію додаткової вартости, яку різні особи під різними
приводами  розподіляють проміж себе\footnoteA{
Наведені обчислення мають значення лише як ілюстрація. Справді, ми припускаємо, що ціни
дорівнюють вартостям. У третій книзі ми побачимо,
що це прирівняння робиться не так просто навіть для пересічних цін.
}.

\subsection{Вираз вартости продукту у відносних частинах продукту}

Вернімось тепер до того прикладу, що показав нам, як капіталіст із грошей робить капітал. Доконечна
праця його прядуна
становила 6 годин, додаткова праця — стільки ж, отже, ступінь експлуатації робочої сили — 100\%.

Продукт дванадцятигодинного робочого дня є 20 фунтів пряжі вартістю в 30\shil{ шилінґів.} Не менше як \sfrac{8}{10}
вартости цієї пряжі (24\shil{ шилінґи}) становить вартість зужиткованих засобів продукції, що лише знову
з’являється (20 фунтів бавовни на 20\shil{ шилінґів}, веретена й~\abbr{т. ін.} на 4\shil{ шилінґи}), або, інакше кажучи,
складається з сталого капіталу. Решта, \sfrac{2}{10}, є нова вартість у 6\shil{ шилінґів}, яка постала підчас
процесу прядіння, що з них половина компенсує авансовану денну вартість робочої сили, або змінний
капітал, а друга половина становить додаткову вартість у 3\shil{ шилінґи.} Отже, сукупна вартість цих 20
фунтів пряжі складаєься ось як:
