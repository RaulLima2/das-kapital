\parcont{}  %% абзац починається на попередній сторінці
\index{i}{0512}  %% посилання на сторінку оригінального видання
ніш в Англії, бо французька біднота тяжко працює та дуже
ощадна щодо харчу й одягу; вони споживають головно хліб,
овочі, городину, корінці та сушену рибу; вони дуже рідко їдять
м’ясо, і коли пшениця дорога, то й дуже мало їдять хліба»\footnote{
Фабрикант із Нортгемптоншіру чинить тут ріа fraus\footnote*{
— благочестивий обман. \emph{Ред.}
}, який йому
можна вибачити, бо це є порив серця. Він порівнює нібито життя англійських
і французьких мануфактурних робітників, алеж у вищецитованих
словах, як він і сам пізніше признається в замішанні, змальовує він
життя французьких рільничих робітників!
}.
«До того ще, — каже далі наш есеїст, — треба додати, що вони
п’ють лише воду або подібні неміцні напитки, так що вони дійсно
надзвичайно мало витрачають грошей\dots{} Подібного стану речей,
безперечно, дуже тяжко добитися, але його можна досягти, як
це виразно доводить наявність його так у Франції, як і в Голляндії»\footnote{
Там же, стор. 70, 71. Примітка до третього видання. Нині, завдяки
конкуренції на світовому ринку, що склалася з того часу, ми значно
посунулися наперед. «Якщо Китай, — заявляє своїм виборцям член парляменту
Степлтон, — стане великою промисловою країною, то я не бачу,
як робітнича людність Европи могла б витримати боротьбу, не знижуючись
до рівня своїх конкурентів» («Times», 3 вересня 1873~\abbr{р.}). Отже,
не континентальні, а китайські заробітні плати є вже тепер бажана
мета англійського капіталу.
}. Два десятиріччя пізніш один американський шахрай,
янкі, що дістав титул барона, Бенжамен Томсон (інакше граф
Румфорд), з великим успіхом розвивав перед богом і людьми ті
самі філантропічні ідеї. Його «Essays» — це куховарська книга
з рецептами всякого роду, як дорогі нормальні страви робітників
заміняти на суроґати. Ось особливо вдатний рецепт цього
дивовижного «філософа»: «П’ять фунтів ячменю, п’ять фунтів
кукурудзи, на 3\pens{ пенси} оселедців, на 1\pens{ пенс} соли, на 1\pens{ пенс} оцту,
на 2\pens{ пенси} перцю й городини — разом на суму в 20\sfrac{3}{4}\pens{ пенсів}
маємо юшку для 64 осіб; за пересічних цін на хліб ці витрати
можна навіть знизити до \sfrac{1}{4}\pens{ пенса} на людину (менше ніж 3 пфеніґи)»\footnote{
Benjamin Thomson: «Essays, political, economical and philosopical
etc.», 3 volumes, London 1796--1802, vol. I, p. 288. У своєму «The
State of the Poor, or an History of the Labouring Classes in England etc.»
cep Ф.~M.~Еден дуже рекомендує злиденну румфордову юшку начальникам
робітних домів і докірливо нагадує англійським робітникам,
що, мовляв, «у Шотляндії є багато родин, які замість пшениці, жита й
м’яса цілі місяці живуть вівсяними крупами та ячним борошном, перемішаним
лише з водою й сіллю, і все таки живуть дуже комфортабельно»
(«and that very comfortably too»). (Там же, книга II, розд. 2, стор. 503).
Подібні «вказівки» ми мали і в XIX віці. «Англійські рільничі робітники,
— читаємо, наприклад, — не хочуть їсти мішанини з гірших сортів
жита. В Шотляндії, де виховання краще, цей забобон, мабуть, невідомий».
(\emph{Charles Н.~Parry, M.~D.}: «The Question of the Necessity of the
existing Cornlaws considered», London 1816, p. 69). Однак той самий
Пері нарікає, що англійський робітник тепер (1815~\abbr{р.}) дуже підупав
порівняно з часами Едена (1797~\abbr{р.}).
}.
З проґресом капіталістичної продукції фальсифікація
товарів зробила зайвими ідеали Томсона\footnote{
Зі звітів останньої парляментської слідчої комісії у справі фальсифікації
засобів існування бачимо, що навіть фальсифікація ліків в
Англії є не виняток, а правило. Наприклад, аналіза 34 проб опію, купленого
в 34 різних лондонських аптеках, виявила, що 31 проба були фальсифіковані
домішкою макових головок, пшеничного борошна, ґуми,
глини, піску й~\abbr{т. ін.} Багато з них не мали й атома морфіну.
}.

\index{i}{0513}  %% посилання на сторінку оригінального видання
Наприкінці XVIII століття і протягом перших десятиліть
XIX століття англійські фармери й лендлорди примусом добилися
абсолютно мінімальної заробітної плати, виплачуючи рільничим
поденникам у формі заробітної плати менше ніш мінімум,
і додаючи їм решту у формі допомоги від парафій. Ось приклад
тих фарсів, що їх витворяли англійські dogberries при «легальному»
встановленні тарифу заробітної плати: «Коли сквайри
1795~\abbr{р.} встановлювали заробітну плату для Speenhamland’y,
вони саме тоді обідали, але, очевидно, гадали, що робітники
чогось такого не потребують\dots{} Вони вирішили, що тижнева
плата має бути 3\shil{ шилінґи} на людину поки буханець хліба
у 8 фунтів 11 унцій коштує 1\shil{ шилінґ} і має рівномірно зростати
доти, доки буханець коштуватиме 1\shil{ шилінґ} 5\pens{ пенсів.}
Коли ціна хліба піднесеться ще вище, то заробітна плата
має пропорційно меншати, поки ціна буханця дійде 2\shil{ шилінґів},
і тоді харчі робітника будуть на \sfrac{1}{5} менші, ніш раніш»\footnote{
\emph{G.~В.~Newnham} (barrister at law): «А Review of the Evidence
before the Committees of the two Houses of Parliament on the Cornlaws»,
London 1815, p. 28 n.
}.
1814 року в слідчому комітеті палати лордів запитали якогось
А.~Беннета, великого фармера, суддю, адміністратора дому для
бідних і таксатора заробітної плати: «Чи додержується якоїсь
пропорції між вартістю денної праці й допомогою робітникам
від парафій?» Відповідь: «Так. Тижневий дохід кожної родини
доповнюється поверх її номінального заробітку настільки, щоб
можна було купити буханець вагою в 1 ґальон (8 фунтів 11 унцій)
і мати ще 3\pens{ пенси} на людину\dots{} Ми припускаємо, що буханця
вагою в 1 ґальон досить на утримання кожної особи в родині
протягом тижня; а 3\pens{ пенси} — то на одяг; коли парафія захоче
сама постачати одяг, то ці три пенси вона відраховує. Ця практика
панує не тільки всюди на захід від Вілтшіру, але, на мою
думку, і в цілій країні»\footnote{
Там же, стор. 19, 20.
}. «Таким чином, — вигукує один буржуазний
письменник того часу, — фармери протягом багатьох років
спричинювали деґрадацію поважної кляси своїх земляків, примушуючи
їх шукати собі притулку в робітних домах\dots{} Фармер
збільшив свій власний дохід тим, що перешкоджав акумуляції
фонду найпотрібнішого споживання на боці робітників»\footnote{
\emph{Ch.~H.~Parry}: «The Question of the Necessity of the existing Cornlaws
considered», London 1816, p. 77, 69. Панове лендлорди з свого боку
не тільки «винагородили» себе за антиякобінську війну, яку вони вели
від імени Англії, а ще й надзвичайно збагатіли. «За вісімнадцять років
їхні ренти збільшились удвоє, утроє, вчетверо, а у виняткових випадках
навіть ушестеро». (Там же, стор. 100, 101).
}. Яку
ролю в утворенні додаткової вартости, а тому і в утворенні
фонду акумуляції капіталу відіграє за наших днів безпосереднє
грабування із доконечного фонду споживання робітника, показала
\parbreak{}  %% абзац продовжується на наступній сторінці
