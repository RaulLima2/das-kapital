аналізи грошей. Труднощі не в тому, щоб зрозуміти, що гроші
є товар, а в тому, як, чому й через що товар є гроші.\footnote{
Пан професор Рошер спочатку повчає нас: «Фальшиві визначення
грошей можна поділити на дві головні групи: такі, що вважають гроші
за щось більше, і такі, що вважають їх за щось менше, ніж товар». Потім
іде безладний каталог творів про гроші, що в ньому немає жодного
розуміння дійсної історії теорії, і, нарешті, мораль: «Не можна заперечувати,
що більшість новітніх економістів звертає недосить уваги
на властивості, які відрізняють гроші від інших товарів (отже, гроші є
щось більше або менше, ніж товар?)... В цьому розумінні напівмеркантильна
реакція Ґаніля... не зовсім необґрунтована». (Wilhelm Roscher:
«Die Grundlagen der Nationaloekonomie», 3 Auflage, 1858, S. 207—210).
Більш-менш — недосить — в цьому розумінні — не зовсім! Ось вам визначення
понять! І таке еклектичне професорське базікання пан Рошер
скромно охрищує ім’ям «анатомічно-фізіологічна метода» політичної
економії! Однак, одне відкриття ми завдячуємо йому, а саме, що гроші
є «приємний товар».
}

Ми бачили, як уже в найпростішому виразі вартости: х товару
А = у товару В, річ, що в ній виражається величина вартости
іншої речі, має, як здається, свою еквівалентну форму незалежно
від цього відношення, як суспільну природну властивість. Ми
простежили, як міцнішає ця фальшива видимість. Вона завершується,
скоро тільки форма загального еквіваленту зростається
з натуральною формою якогось осібного товарового роду або
кристалізується в грошову форму. Здається, що даний товар став
грішми не тому, що всі інші товари виражають у ньому всебічно
свої вартості, а, навпаки, здається, що вони взагалі виражають
свої вартості в ньому тому, що він є гроші. Упосереднювальний
рух зникає у своєму власному результаті й не лишає по собі
жодного сліду. Без жодної його допомоги товари находять свою
власну форму вартости готовою, як таке товарове тіло, що існує
поза ними й поруч із ними. Ці речі, золото й срібло — такі,
якими їх видобувається з надр землі, одночасно є безпосереднє
втілення усякої людської праці. Звідси магічний характер грошей.
Суто атомістичні відносини між людьми в їхньому процесі
суспільної продукції, а тому й речова форма їхніх власних продукційних
відносин, незалежна від їхнього контролю та від
їхньої свідомої індивідуальної діяльности, виявляються насамперед
у тому, що продукти їхньої праці взагалі набирають товарову
форму. Отже, загадка грошового фетишу є не що інше, як
вагадка товарового фетишу, яка лише вражає нас своєю сліпучою
формою.

Розділ третій

Гроші або циркуляція товарів

1. Міра вартостей

У цій праці я заради простоти припускаю скрізь, що грошовий
товар це — золото.

Перша функція золота в тому, щоб дати товаровому світові
матеріял для виразу його вартостей, або виразити вартості това-