\parcont{}  %% абзац починається на попередній сторінці
\index{i}{0003}  %% посилання на сторінку оригінального видання
пшениця має не одним-одну, а різноманітні мінові вартості.
А що \emph{х} вакси до чобіт, так само \emph{у} шовку, так само \emph{z} золота
й т. ін. є мінова вартість одного квартера пшениці, то \emph{х} вакси до чобіт,
\emph{у} шовку, \emph{z} золота й т. ін. мусять бути замінні одне на одне, або
являти собою рівновеликі мінові вартості. Звідси випливає, поперше,
що мінові вартості, на які обмінюється той самий товар,
виражають щось однакове, а, подруге, що мінова вартість взагалі
може бути тільки способом виразу, «формою виявлення» якогось
відмінного від неї змісту.

Візьмімо далі два товари, приміром, пшеницю й залізо. Хоч
яке буде їхнє мінове відношення, його завжди можна виразити
в рівнанні, де якусь дану кількість пшениці буде прирівняно до
якоїсь кількости заліза, приміром: 1 квартер пшениці = \emph{а} центнерам
заліза. Що каже це рівнання? Що в двох різних речах,
в 1 квартері пшениці і так само в \emph{а} центнерах заліза, існує щось
спільне однакової величини. Отже, обидві ці речі рівні чомусь
третьому, яке само по собі не є ані перша, ані друга річ. Отже,
кожна з цих двох речей, оскільки вона є мінова вартість, мусить
бути зведена до цього третього.

Пояснімо це простим геометричним прикладом. Щоб вимірювати
й порівнювати поверхні всіх простолінійних фігур, їх розкладають
на трикутники. Самий трикутник зводять до виразу,
цілком одмінного від його видимої фігури, — до половини здобутку
його основи та його висоти. Так само треба звести мінові
вартості товарів до чогось спільного для них, з чого вони репрезентують
більшу або меншу кількість.

Цим чимось спільним для них не може бути геометрична, фізична,
хемічна або якабудь інша природна властивість товарів,
їхні тілесні властивості взагалі розглядається лише остільки,
оскільки вони роблять їх корисними, отже, споживними вартостями.
Але, з другого боку, якраз абстрагування від їхніх споживних
вартостей і є те, що виразно характеризує мінове відношення
товарів.\footnote*{
У французькому виданні це речення подано так: «Але, з другого
боку, очевидно, що коли обмінюються товарами, то абстрагуються від
споживної вартости, і що саме цією абстракцією характеризується всяке
мінове відношення». («Le Capital» par K. Marx, traduction de M. I. Roy,
entièrement revisée par l’auteur. Paris 1875, v. I, ch. 1, p. 14). \emph{Ред.}
} У межах мінового відношення товарів одна споживна
вартість має таке саме значення, як і інша, коли тільки
вона є в належній пропорції. Або, як каже старий Барбон: «Один
рід товару є такий самий добрий, як і інший, коли їхні мінові
вартості мають однакову величину. Між речами однакової мінової
вартости нема ніякої ріжниці або відмінности».\footnote{
«One sort of wares are as good as another, if the value be equall.
There is no difference or distinction in things of equal value». І далі Барбон
додає: «Кількість заліза або олива на сто фунтів стерлінґів мають таку
саму мінову вартість, які кількість срібла або золота на сто фунтів стерлінґів»
(«One hundred pounds worth of lead or iron, is of as great a value
as one hundred pounds worth of silver and gold»). (\emph{N. Barbon:}  «A
Discourse cocerning coining the new money lighter», p. 53 u. 7).
} Як споживні
\parbreak{}  %% абзац продовжується на наступній сторінці
