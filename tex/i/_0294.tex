\parcont{}  %% абзац починається на попередній сторінці 
\index{i}{0294}  %% посилання на сторінку оригінального видання 
незмінним спосіб праці поодиноких осіб, мануфактура до самої
основи революціонізує його та захоплює індивідуальну робочу
силу в самому її корені. Вона калічить робітника, робить із нього
якусь потвору, штучно активізуючи в ньому розвиток лише
якоїсь однієї частинної вмілости через пригнічення цілого світу
продуктивних хистів і інстинктів, так само як, наприклад, у
штатах Ля Пляти вбивають тварину, щоб добути з неї шкуру
або лій. Не тільки окремі частинні праці розподіляються між
різних індивідів, але й сам індивід розділяється, перетворюється
на автоматичне знаряддя якоїсь частинної праці, і таким чином
здійснюється безглузда байка Мененія Аґріппи, яка змальовує
людину лише як простий фраґмент її ж власного тіла. Коли
первісно робітник продає свою робочу силу капіталові, бо в нього
немає матеріяльних засобів для продукції товару, то тепер сама
його індивідуальна робоча сила відмовляється служити, якщо
її не продано капіталові. Вона функціонує лише в певному сполученні,
яке існує тільки після її продажу, в майстерні капіталіста.
Зробившися через свої природні властивості нездатним робити
щось самостійне, мануфактурний робітник розвиває продуктивну
діяльність тільки як приналежність до майстерні капіталіста.
Як на чолі вибраного народу було написано, що він є власність
Єгови, так само поділ праці накладає на мануфактурного робітника
печать, що таврує його на власність капіталу.

Ті знання, розум і воля, що їх, хоч і в невеликому маштабі,
розвиває в собі самостійний селянин або ремісник — подібно до
того, як дикун усю військову вмілість практикує у формі особистих
хитрощів, — все це тепер потрібне лише для майстерні як цілости.
Духовні потенції продукції поширюють свій маштаб на одному
боці, бо на багатьох вони зникають. Те, що втрачають частинні
робітники, концентрується супроти них у капіталі. Мануфактур-
\parbreak{}  %% абзац продовжується на наступній сторінці
