\parcont{}  %% абзац починається на попередній сторінці 
\index{i}{0132}  %% посилання на сторінку оригінального видання 
служать лише як сховища предметів праці й що їхню сукупність
взагалі можна назвати судинною системою продукції, як от
труби, бочки, коші, жбани тощо. Лише в хемічній фабрикації
вони відіграють значну ролю.\footnoteA{
Примітка до другого видання. Хоч і як мало дотеперішня історична
наука знає розвиток матеріяльної продукції, основу всього суспільного
життя, отже, і всієї дійсної історії, проте принаймні передісторичні часи
поділено — на основі природничо-наукових, а не так званих історичних,
дослідів — за матеріялом знарядь праці і зброєю на кам’яний вік,
бронзовий вік і залізний вік.
}

Крім тих речей, що упосереднюють діяння праці на предмет
праці і тому так або інакше служать за провідників діяльности,
процес праці залічує до своїх засобів у ширшому розумінні всі
речові умови, які взагалі потрібні, щоб відбувся процес. Вони
не входять у нього безпосередньо, але без них він або зовсім
не може відбуватися або відбувається лише в недосконалій формі.
За загальний засіб праці цього роду є знов таки сама земля, бо
вона дає робітникові locus standi,\footnote*{
— місце, на якому він стоїть. Ред.
} а його процесові — поле
діяльности (field of employment). За засоби праці цього роду,
але такі, що вже оброблялись працею, приміром, є робочі будинки,
канали, шляхи й т. ін.

Отже, у процесі праці діяльність людини спричинює за допомогою
засобів праці зміну в предметі праці, яку вона собі заздалегідь
поставила за мету. Процес згасає в продукті. Продукт його є
споживна вартість, речовина природи, через зміну форми пристосована
до людських потреб. Праця сполучилась із своїм предметом
праці. Вона упредметнена, а предмет праці оброблено. Те, що
на боці робітника з’являлось у формі неспокою (Unruhe), з’являється
тепер на боці продукту як властивість спокою (ruhende
Eigenschaft), у формі буття. Робітник пряв, і продукт є прядиво.

Коли розглядати цілий процес з погляду його результату,
продукту, то засоби праці й предмет праці, обидва, з’являються
як засоби продукції,\footnote{
Видається парадоксом, коли, приміром, рибу, якої ще не впіймано,
називати засобом продукції для рибальства. Але досі ще не винайдено
вмілости ловити рибу у водах, де її немає.
} а сама праця — як продуктивна праця.\footnote{
Цього визначення продуктивної праці, що випливає з погляду простого
процесу праці, зовсім недосить для капіталістичного процесу продукції.
}

Коли якась споживна вартість виходить із процесу праці як
продукт, то інші споживні вартості, продукти попередніх процесів
праці, входять у нього як засоби продукції. Та сама споживна
вартість, що є продукт одного процесу праці, становить засіб
продукції для іншого процесу праці. Тому продукти є не лише
результат, але разом з тим і умова процесу праці.

За винятком добувальної промисловости, що знаходить свій
предмет праці в самій природі, як от гірництво, полювання,
рибальство й т. ін. (рільництво лише остільки, оскільки воно в
першу чергу обробляє саму цілину), всі інші галузі промисловости
обробляють предмет, що є сировинний матеріял, тобто предмет
\index{i}{0133}  %% посилання на сторінку оригінального видання 
праці, що його вже профільтровано через процес праці й що
сам вже є продукт праці, приміром, насіння в рільництві. Тварини
й рослини, що їх звикли вважати за продукти природи, є не
лише продукти праці, може, попереднього року, але в їхніх теперішніх
формах вони є продукти змін, що склалися протягом
багатьох поколінь під контролем людини, за допомогою людської
праці. Щождо засобів праці зокрема, то величезна більшість їх
навіть при найповерховішому розгляді показує сліди минулої
праці.

Сировинний матеріял може становити головну субстанцію
продукту або ввіходити в процес його утворення лише як допоміжний
матеріял. Допоміжний матеріял споживають засоби праці,
як от вугілля споживає парова машина, олію — колесо, сіно —
робочий кінь, або його додають до сировинного матеріялу, щоб
спричинити в ньому речову зміну, як от хлор — до небіленого
полотна, вугілля — до заліза, фарбу — до вовни, або він допомагає
виконувати саму працю, як от, приміром, матеріяли, що
ними користуються для освітлення й опалення робітних майстерень.
У власне хемічній фабрикації ріжниця між головним матеріялом
і допоміжним матеріялом зникає, бо жоден з уживаних
сировинних матеріялів не появляється знов як субстанція продукту.\footnote{
Шторх одрізняє власне сировинний матеріял як «matière» від
допоміжних матеріялів як «matériaux»; Шербулье називає допоміжні
матеріяли «matières instrumentales».
}
А що всяка річ має багато різних властивостей і тому придатна
для різних способів використовування, то той самий продукт
може бути сировинним матеріялом для дуже різних процесів
праці. Наприклад, зерно є сировинний матеріял для мірошника,
для фабриканта крохмалю, гуральника, скотаря й т. ін. Як
насіння воно стає сировинним матеріялом для своєї власної продукції.
Так само й вугілля як продукт виходить із гірничої промисловости,
а як засіб продукції — увіходить до неї.

Той самий продукт у тому самому процесі праці може служити
за засіб праці й за сировинний матеріял. Приміром, худоба,
що її годують, є оброблюваний сировинний матеріял і разом
з тим засіб для виготовлення гною.

Продукт, що існує в готовій для споживання формі, може знов
стати сировинним матеріялом для іншого продукту, як ось виноград
— сировинним матеріялом для вина. Або праця випускає
свій продукт у формах, у яких він знов може бути вжитий лише
як сировинний матеріял. Сировинний матеріял у такому стані
зветься півфабрикатом або краще звати його ступневим фабрикатом
(Stufenfabrikat), як от бавовна, нитки, пряжа тощо.
Первісний сировинний матеріял, хоч і сам він вже є продукт,
повинен, однак, перейти ще цілий ряд різних процесів, де він
у завжди змінливій формі знову й знов функціонує як сировинний
матеріял аж до останнього процесу праці, який відштовхує
його від себе як готовий засіб існування або як готовий засіб праці.
