\parcont{}  %% абзац починається на попередній сторінці
\index{i}{0514}  %% посилання на сторінку оригінального видання
вже, наприклад, так звана домашня праця (див. розділ XIII,
8, с.). Дальші факти про це ми подаємо далі в цьому відділі.

Хоч у всіх галузях промисловости тієї частини сталого капіталу,
яка складається із засобів праці, мусить вистачати для
певного числа робітників, визначуваного величиною вкладеного
капіталу, проте ця частина зовсім не мусить завжди зростати в
тій самій пропорції, в якій зростає число занятих робітників.
Припустімо, що на якійсь фабриці 100 робітників за восьмигодинної
праці дають 800 робочих годин. Коли капіталіст схоче
збільшити цю суму наполовину, то він може найняти 50 нових
робітників; але тоді він мусить авансувати й новий капітал,
не тільки для заробітної плати, а й на засоби праці. Однак, він
також може примусити попередніх 100 робітників працювати
12 годин замість 8, і тоді вистачить наявних уже засобів праці,
які тоді лише швидше зужитковуватимуться. Таким чином новододавана
праця, створена через більше напруження робочої сили,
може збільшити додатковий продукт і додаткову вартість, субстанцію
акумуляції, без відповідного збільшення сталої частини
капіталу.

У добувальній промисловості, наприклад, у копальнях, сировинні
матеріяли не становлять складової частини авансованого
капіталу. Тут предмет праці не є продукт попередньої праці, а
є gratis\footnote*{
— безплатно. \emph{Ред.}
} дарований природою. Наприклад, металева руда, мінерали,
кам’яне вугілля, каміння й т. ін. Тут сталий капітал складається
майже виключно із засобів праці, що дуже добре можуть
служити і при збільшеній кількості праці (наприклад, при денних
і нічних змінах робітників). Однак, припускаючи всі інші
обставини за незмінні, маса й вартість продукту зростатимуть
прямо пропорційно до вжитої праці. Як і першої днини продукції,
тут ідуть пліч-о-пліч первісні творці продукту, а тому й творці
речових елементів капіталу: людина й природа. Завдяки елястичності
робочої сили сфера акумуляції поширюється без попереднього
збільшення сталого капіталу.

У рільництві не можна поширити площу оброблюваної землі,
не авансуючи додаткового насіння і добрива. Але скоро це вже
авансовано, то самий лише механічний обробіток поля дивовижно
впливає на збільшення маси продукту. Більша кількість
праці, давана попередньою кількістю робітників, збільшує таким
чином родючість, не потребуючи нових авансувань на засоби
праці. Це знову таки безпосереднє діяння людини на природу,
яке стає безпосереднім джерелом збільшеної акумуляції без
участи якогось нового капіталу.

Насамкінець, у промисловості у власному значенні слова
кожна додаткова витрата на працю має своєю передумовою відповідну
додаткову витрату на сировинні матеріяли, але не неодмінно
і на засоби праці. А що добувальна промисловість і рільництво
постачають фабричній промисловості її сировинні матеріяли
\parbreak{}  %% абзац продовжується на наступній сторінці
