звіту «комісії про працю дітей» (1863 р.) тієї самої долі зазнали
всі мануфактури глиняних товарів (не лише ганчарні), мануфактура
сірників, капсулів, патронів, фабрики шпалер, фабрики
обрізків плюшу (fustian cutting) і численні процеси, об’єднані
під назвою «finishing» (остаточна апретура). 1863 р. «білильні
на свіжому повітрі»\footnote{
«Білильники на свіжому повітрі» ухилилися від додержування
закону 1860 р. про білильні брехнею, нібито в них жінки вночі не працюють.
Брехню викрили фабричні інспектори, але одночасно й петиції
робітників розбили уявлення парляменту про працю у «білильнях на
вільному повітрі», як про працю на прохолодних запашних луках. В цих
білильнях на повітрі користуються сушарнями з температурою в 90—100°
Фаренгайта, і там працюють переважно дівчата. Є навіть технічний вислів
«cooling» (прохолоджування), що ним зветься принагідне вибігання з
сушарні на свіже повітря. «П’ятнадцять дівчат у сушарні. Температура
від 80 до 90° для полотна, 100° і вище для батисту. Дванадцять дівчат прасують
і складають (батист тощо) у маленькій кімнатці якихось приблизно
10 квадратових футів із щільно закритою піччю посередині. Дівчата стоять
колом круг печі, від якої пашить жахливою жарою і яка швидко висушує
батист для прасувальниць. Число годин для цих рук необмежене. Коли
праці багато, вони працюють багато днів уряд до 9 або до 12 годин вночі».
(«Reports etc. for 31 st Oct. 1862», p. 56). Один лікар заявляє: «Осібних
годин для прохолоджування не дозволяється, але, коли температура стає
надто нестерпна, або коли руки робітниць забруднюються від поту, їм
дозволяється вийти на декілька хвилин... Мій досвід у лікуванні недуг
цих робітниць примушує мене констатувати, що стан їхнього здоров'я
багато гірший від здоров’я бавовнопрядних робітниць (а капітал у своїх
петиціях до парляменту розмальовував їх у стилі Рубенса, нібито від них
пашить здоров’ям!). Найбільш поширено серед них такі недуги: сухоти,
бронхіт, недуги уразу, гістерія в якнайжахнішій формі та ревматизм.
На мою думку, всі ці недуги походять, посередньо або безпосередньо,
від перегрітого повітря їхніх майстерень і недостачі досить теплого одягу,
який міг би захистити їх при повороті додому від вогкої й холодної атмосфери
зимових місяців». (Там же, стор. 56, 57). Фабричні інспектори зауважують
щодо закону з 1863 р., додатково відвойованого в життєрадісних
власників «білилень на свіжому повітрі»: «Цей закон не лише не забезпечує
охорони робітникам, яку він, як здається, їм забезпечує... його й
зформульовано так, що ця охорона починається лише тоді, коли дітей
і жінок спіймано за працею після 8 годин вечора, але навіть і тоді приписаний
законом спосіб доказу такий заплутаний, що ледве чи можна
покарати винних». (Там же, стор. 52). «Як закон із гуманними й виховними
цілями, він цілком невдалий. Ледве чи можна назвати гуманним
дозволяти жінкам і дітям або, що сходить на те саме, примушувати їх
працювати по 14 годин денно, а може й більше, з перервами на їжу або
й без них, як доведеться, без обмежень щодо віку, без ріжниці статі
і не звертаючи уваги на суспільні звички родин із тих сусідніх околиць,
де лежать ці білильні»). («Reports etc. for 30 th April 1863», р. 40).
} й пекарні підведено під осібні закони,
з яких перший, між іншим, забороняє працювати дітям, підліткам
та жінкам у нічний час (від 8 години вечора до 6 години
ранку), а другий — вживати праці пекарських підмайстрів, молодших
за 18 років, між 9 годиною вечора й 5 годиною ранку.
До пізніших пропозицій згаданої комісії, які загрожують позбавленням
«волі» всім важливим галузям англійської промисловости,
за винятком рільництва, копалень і транспорту, ми ще
повернемось.\footnoteA{
(Примітка до другого видання). Від року 1866, коли я писав те,
що є в тексті, знову надійшла реакція. [Капіталісти тих галузей про-
}