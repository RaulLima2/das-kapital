ського рільничого або фабричного робітника. Однак це лише
законом приписана панщинна праця. A «Réglement organique»
ще в «ліберальнішому дусі», ніж англійське фабричне законодавство,
зумів улегшити собі способи нехтувати себе самого.
Зробивши з 12 днів 54, він знову визначає номінальну денну працю
кожного з тих 54 панщинних днів так, що на дальші дні мусить
припадати якийсь додаток праці. Приміром, за один день слід
виполоти якийсь шмат поля — операція, яка, особливо на кукурудзяному
полі, потребує удвоє більше часу. Установлену законом
денну працю для деяких рільничих робіт можна тлумачити
так, що вона повинна починатися у травні, а кінчатися у жовтні.
Для Молдавії постанови ще суворіші. «Дванадцять панщинних
днів, що їх приписує «Réglement organique» — гукнув один
сп’янілий від перемоги боярин. — складають 365 днів на рік!».\footnote{
Дальші подробиці можна найти у Е. Régnault: «Histoire politique
et sociale des Principautés Danubiennes», Paris 1855.
}

Коли «Réglement organique» дунайських князівств був позитивним
висловом ненажерливої жадоби до додаткової праці, яку
леґалізує кожний його параграф, то англійські Factory-Acts
є неґативні вислови тієї самої ненажерливої жадоби. Ці закони
загнуздують прагнення капіталу до безмірного висмоктування
робочої сили, встановлюючи примусове обмеження робочого дня
державою, і до того ж державою, в якій панує капіталіст і лендлорд.
Не кажучи вже про робітничий рух, що з дня на день загрозливіше
зростав, обмеження фабричної праці було подиктоване
тією самою доконечністю, яка примусила виливати ґуано
на англійські поля. Те саме сліпе хижацтво, що в одному випадку
виснажило землю, в другому випадку підривало життєві сили
нації в самому корінні. Періодичні епідемії тут свідчать про це
так само виразно, як і зменшення міри зросту в солдатів у Німеччині
і у Франції.\footnote{
«Загалом, зріст органічних істот поза пересічну міру свого роду
свідчить, у певних межах, про їхній розцвіт. Розмір тіла в людини меншає,
коли її добробутові шкодять фізичні або соціяльні обставини. У всіх європейських
країнах, де існує конскрипція, від часу, як її заведено, середній
зріст дорослої людини й загалом її здатність до військової служби зменшились.
Перед революцією (1789 р.) мінімум для піхотинця у Франції
становив 165 сантиметрів, у 1818 р. (закон з 10 березня) — 157 сантиметрів,
за законом з 21 березня 1852 р. — 156 сантиметрів; пересічно у Франції
більш ніж половину рекрутів визнають за нездатних через недостатній
зріст та фізичні хиби. В Саксонії військова міра була 1780 р. 178 сантиметрів,
тепер — 155 сантиметрів. У Прусії тепер — 157 сантиметрів. За даними
д-ра Маєра в «Bayerischen Zeitung» від 9 травня 1862 р. виходить,
що в Прусії за дев’ятилітній період пересічно із 1.000 рекрутів 716 визнають
за нездатних до військової служби: 317 — через недостатній зріст і
399 — через фізичні хиби... В 1858 р. Берлін не міг виставити відповідного
контингенту рекрутів — бракувало 156 осіб». (J. ù. Liebig: «Die
Chemie in ihrer Anwendung auf Agrikultur und Physiologie», 7. Auflage.
1862. Bd. I, S. 117, 118).
}

Factory Act 1850 р., що має тепер (1867) силу, дозволяє пересічний
тижневий день у 10 годин, а саме для перших п’ятьох
днів тижня по 12 годин, від шостої години ранку до шостої го-