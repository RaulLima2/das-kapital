\parcont{}  %% абзац починається на попередній сторінці 
\index{i}{0526}  %% посилання на сторінку оригінального видання 
для кожної приватної особи, для кожної окремої родини в суспільстві
— це жити ощадно; але інтерес усіх багатих націй у
тому, щоб якнайбільша частина бідних ніколи не була без праці,
і щоб вони, однак, завжди витрачали все те, що одержують...
Ті, що підтримують своє життя своєю щоденною працею, не
мають нічого такого, що спонукало б їх до праці, крім своїх
нужд, які розумно було б полегшувати, але безглуздо зціляти.
Одним-одна річ, яка може зробити робітника працьовитим, —
це помірна заробітна плата. Занадто мала заробітна плата робить
його залежно від його темпераменту малодушним або доводить
до розпачу, занадто велика робить його нахабним та ледащим...
З усього досі сказаного випливає, що для вільної нації, де рабство
не дозволено, найпевніше багатство — це маса працьовитих бідняків.
Крім того, що вони є невичерпне джерело постачання для
фльоти й армії, без них не було б ніяких насолод і не можна
було б використати продукт жодної країни. Щоб зробити суспільство
(яке, звичайно, складається з неробітників) щасливим, а
народ задоволеним навіть у нужденному стані, для цього треба,
щоб велика більшість лишалася так в неуцтві, як і в бідності.
Знання поширює й помножує наші бажання, а що менше людина
бажає, то легше можна задовольнити її потреби».\footnote{
В. de Mandeville: «The Fable of the Bees», 5th ed., London 1728,
Remarks, p. 212, 213, 328. — «Помірне життя й постійна праця — це для
бідних шлях до матеріяльного щастя (під яким він розуміє якомога довший
робочий день і якомога менше засобів існування) і до багатства для
держави (а саме для землевласників, капіталістів та їхніх політичних
сановників і аґентів)». («An Assay on Trade and Commerce», London
1770, p. 54).
} Мандевіль,
чесна людина і ясна голова, ще не розуміє того, що самий механізм
процесу акумуляції разом із збільшенням капіталу збільшує
і масу «працьовитих бідняків», тобто найманих робітників,
які мусять перетворювати свою робочу силу на щораз, більшу
силу збільшувати вартість капіталу, який чимраз більше зростає,
і саме цим увічнювати відносини своєї залежности від
свого власного продукту, персоніфікованого в капіталісті. З приводу
цих відносин залежности сер Ф. М. Еден зауважує в своєму
творі «Становище бідних, або історія робітничої кляси Англії»:
«Наш клімат вимагає праці для задоволення потреб, і тому
принаймні частина суспільства мусить невтомно працювати...
Декотрі, що не працюють, усе ж мають у своєму розпорядженні
продукти працьовитосте. Але це ці власники завдячують лише
цивілізації і порядкові; вони — чисті креатури громадянських
інституцій.\footnote{
Еден повинен був би спитати, чиї ж то креатури ті «громадянські
інституції». З погляду юридичних ілюзій він розглядає закон не як
продукт матеріяльних відносин продукції, а, навпаки, — відносини продукції
як продукт закону. Ленґе одним словом збив ілюзорний «Дух
законів» («Esprit des lois») Монтеск’є: «Дух законів — це власність»
(«L’esprit des lois, c’est la propriété»).
} Бо ці останні визнали, що плоди праці можна
присвоювати й іншим способом, ніж працею. Люди з незалежним
майном майже цілком завдячують своє майно праці інших,
\parbreak{}  %% абзац продовжується на наступній сторінці
