\index{i}{0266}  %% посилання на сторінку оригінального видання


Коли робітники взагалі не можуть безпосередньо співробітничати,
не бувши згуртовані, отже, коли згуртовання їх у певному
місці є умова їхньої кооперації, то наймані робітники не можуть
кооперувати без того, щоб той самий капітал, той самий капіталіст
не вживав їх одночасно, отже, і не купував одночасно їхні робочі
сили. Тому сукупна вартість цих робочих сил, або сума заробітної
плати робітників за день, тиждень і~\abbr{т. д.}, мусить вже бути нагромаджена
в кишені капіталіста раніш, ніж сами робочі сили будуть
сполучені у продукційному процесі. Заплатити 300 робітникам
відразу навіть хоч би й лише за один день — це вимагає
більшого авансування капіталу, аніж платити кільком робітникам
тиждень за тижнем протягом цілого року. Отже, число
кооперованих робітників, або маштаб кооперації, залежить насамперед
від величини того капіталу, що його поодинокий капіталіст
може авансувати на купівлю робочої сили, тобто від того обсягу,
в якому кожен поодинокий капіталіст порядкує засобами
існування багатьох робітників.

І щодо сталого капіталу справа стоїть так само, як і щодо
змінного. Приміром, видатки на сировинний матеріял для одного
капіталіста, що вживає 300 робітників, у тридцять разів більші,
ніж для кожного з тих 30 капіталістів, що кожний з них вживає
10 робітників. Правда, розмір вартости й маса матеріялу спільно
вживаних засобів праці зростають не в такій пропорції, як число
вживаних робітників, але все ж вони зростають дуже значно.
Отже, концентрація більших мас засобів продукції в руках поодиноких
капіталістів є матеріяльна умова кооперації найманих
робітників, а розмір кооперації, або маштаб продукції залежить
від розміру цієї концентрації.

Первісно певна мінімальна величина індивідуального капіталу
виступала як доконечна для того, щоб кількости одночасно
визискуваних робітників, а тому й маси продукованої додаткової
вартости вистачило для звільнення самого визискувача від ручної
праці, для перетворення дрібного майстра на капіталіста, отже,
і для того, щоб формально створити капіталістичне відношення.
Тепер вона виступає як матеріяльна умова для перетворення
багатьох розпорошених і один від одного незалежних індивідуальних
процесів праці на один комбінований суспільний процес
праці.

Так само командування капіталу над працею первісно виступало
лише як формальний наслідок того, що робітник, замість
працювати на себе, працює на капіталіста, а тому й під доглядом
капіталіста. З розвитком кооперації багатьох найманих робітників
командування капіталу розвивається на доконечність для
виконання самого процесу праці, на дійсну умову продукції.
Наказ капіталіста на полі продукції стає тепер так само доконечний,
як наказ генерала на полі бою.

Всяка безпосередньо суспільна або спільна праця у великому
маштабі потребує в більшій або меншій мірі керування, яке упосереднює
гармонію між індивідуальними діями та виконує загальні
\index{i}{0267}  %% посилання на сторінку оригінального видання
функції, що виникають із руху цілого продуктивного тіла
відмінно від руху його самостійних органів. Окремий скрипаль
дириґує собі сам, оркестра потребує дириґента. Ця функція
керування, догляду та упосереднення стає функцією капіталу,
скоро тільки підпорядкована йому праця стає кооперативною.
Як специфічна функція капіталу функція керування набирає
специфічних характеристичних ознак.

Насамперед рушійним мотивом і визначальною метою капіталістичного
процесу продукції є якомога більше самозростання
капіталу\footnote{
«Зиск\dots{} однісінька мета продукції» («Profits\dots{} is the sole end
of trade»). (\emph{J.~Vanderlint}: «Money answers all Things», London 1734,
p. 11)
}, тобто якомога більша продукція додаткової вартости,
отже, якомога більший визиск робочої сили капіталом. Із зростом
маси одночасно експлуатованих робітників зростає і їхній
опір, а тому неминуче зростає і гніт капіталу, щоб перебороти
той опір. Керування капіталіста є не тільки осібна функція, що
виникає з природи суспільного процесу праці й належить до нього,
воно є одночасно й функція визиску суспільного процесу праці
і тому зумовлюється неминучим антагонізмом між визискувачем
і сировинним матеріялом його визиску. Так само із зростом розміру
засобів продукції, що протистоять найманому робітникові
як чужа власність, зростає й доконечність контролю над доцільним
уживанням цих засобів\footnote{
Часопис англійських філістерів «Spectator» сповіщає в числі
з 3 червня 1866~\abbr{р.}, що після заведення чогось на зразок товариського підприємства
між капіталістом та робітником у «Wirework company of Manchester»
«першим результатом було те, що раптом зменшилося псування
матеріялу, бо робітники зрозуміли, що їм, як і всім іншим власникам,
нема нащо псувати своє власне майно, а псування знаряддя та матеріялу
є, може, найбільше, після легкодушних боргів, джерело втрат у промисловості»
(«the first result was a sudden decrease in waste, the men not seeing
why they should waste their own property any more than any other master’s,
and waste is perhaps, next to bad debts, the greatest source of manufacturing
loss»). Той самий часопис викриває ось яку основну хибу в рочдельських кооперативних
спробах: «Вони показали, що робітничі асоціяції можуть
успішно порядкувати крамницями, фабриками та майже всіма формами
промисловости, і що вони надзвичайно поліпшили становище самих робітників,
але! але в такому випадку вони зовсім не залишали якогось виразного
місця для капіталіста» («They showed that associations of workmen
could manage shops, mills, and almost all forms of industry with success,
and they immensely improved the condition of the men, but then they did
not leave a clear place for masters»). Quelle horreur!\footnote*{
Який жах! \emph{Ред.}
}
}. Далі, кооперація найманих робітників
є лише результат діяння капіталу, який їх одночасно вживає.
Зв’язок їхніх функцій та їхня єдність як продуктивного цілого
тіла лежать поза ними, в капіталі, що їх згуртовує та тримає
вкупі. Тим-то зв’язок їхніх праць протистоїть їм ідеально як плян,
практично — як авторитет капіталіста, як сила чужої волі, що
підпорядковує їхню діяльність своїй меті.

Тому, якщо капіталістичне керування своїм змістом є двоїсте
внаслідок двоїстости самого продукційного процесу, що ним
\parbreak{}  %% абзац продовжується на наступній сторінці
