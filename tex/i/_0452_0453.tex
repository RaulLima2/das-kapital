\index{i}{0452}  %% посилання на сторінку оригінального видання 
Тут нічого не зарадить і пояснення обміну більшої кількости
праці на меншу ріжницею форми, тим, що в одному випадку праця
є упредметнена, в другому — жива.\footnote{
«Треба було згодитися (ще одна відміна «contrat social»), що кожного
разу, коли обмінюватиметься працю виконану на працю, яка ще
має бути виконана, останній (капіталіст) матиме вартість вищу, ніж перший
(робітник)» («Il a fallu convenir que toutes les fois qu’il échangerait du
travail fait contre du travail à faire, le dernier (le capitaliste) aurait une valeur
supérieure au premier (le travailleur)». Simonde de Sismondi: «De
la Richesse Commerciale», Genève 1803, vol. I, p. 37).
} Це тим недоладніша річ,
що вартість товару визначається не кількістю дійсно упредметненої
в ньому праці, а кількістю живої праці, доконечної для
його продукції. Нехай якийсь товар репрезентує 6 робочих годин.
Якщо пороблять такі винахода, що через них той товар можна
буде випродукувати протягом 3 годин, то й вартість випродукованого
вже товару спаде наполовину. Тепер він репрезентує
замість попередніх 6 годин тільки 3 години доконечної суспільної
праці. Отже, величину вартости товару визначає кількість праці,
потрібна на його продукцію, а не предметна форма цієї праці.

Фактично на товаровому ринку власникові грошей безпосередньо
протистоїть не праця, а робітник. Те, що останній продає,
є його робоча сила. Скоро тільки дійсно починається його
праця, вона вже не належить йому, отже, він її не може вже
більше продати. Праця є субстанція й іманентна міра вартостей,
але сама вона не має вартости.26

У вислові «вартість праці» поняття вартости не тільки цілком
погашено, але ще й перетворено на свою протилежність. Це такий
самий іраціональний вислів, як от, наприклад, вартість землі.
Однак такі іраціональні вислови випливають із самих продукційних
відносин. Це — категорії для форм виявлення посутніх
відносин. Що речі в своєму виявленні часто з’являються
покрученими, це досить відомо в усіх науках, крім політичної
економії.26

25 «Праця, виключна міра вартости... творець усякого багатства,
не є товар» («Labour, the exclusive standard of value... the creator of
ail wealth, no commodity»). (Th. Hodaskin: «Popular Political Economy»,
p. 186).

26 Навпаки, поясняти такі вислови просто як licentia poetica\footnote*{
— поетичні вільності. Ред.
} — це
свідчить лише про безсилля аналізи. Тим то на фразу Прудона: «Про
працю кажуть, що вона має вартість не як про власне товар, а маючи
на оці ті вартості, що їх припускається за потенціяльно вміщені в ньому.
Вартість праці — вираз фігуральний і т. ін.» («Le travail est dit valoir,
non pas en tant que marchandise lui-même, mais en vue de valeurs qu’on
suppose renfermées puissanciellement en lui. La valeur du travail est une
expression figurée etc.»), — я зауважую: «В праці-товарі, який має
жахливу реальність, він убачає тільки граматичну еліпсу. Отже,
виходить, що ціле сучасне суспільство, яке ґрунтується на товарі-праці,
відтепер ґрунтується на поетичній вільності, на фігуральному вислові.
Якщо суспільство захоче «усунути всі недоладності», що його мучать
що ж І — хай воно усуне лише недоброзвучні, вислови, змінить мову, а для
цього досить лише звернутися до академії з вимогою випустити нове
видання її словника» («Dans le travail-marchandise, qui est d une réalité

\index{i}{0453}  %% посилання на сторінку оригінального видання 
Клясична політична економія запозичила без дальшої критики
з щоденного життя категорію «ціна праці», щоб потім поставити
собі питання, як визначається цю ціну? Вона пізнала
незабаром, що зміна відношення між попитом і поданням не
пояснює нічого в ціні праці, як і в ціні всякого іншого товару,
крім її зміни, тобто коливання ринкових цін понад або нижче
деякої величини. Якщо попит і подання взаємно покриваються,
то, за інших незмінних умов, коливання в цінах припиняється.
Але тоді попит і подання перестають щобудь пояснювати. Ціна
праці, коли попит і подання покриваються, є її природна ціна,
визначена незалежно від відношення між попитом і поданням,
природна ціна, що її таким чином і знайдено як справжній предмет
аналізи. Або брали довший період коливань ринкової ціни,
наприклад, один рік, і тоді відкривали, що її зростання або
спадання вирівнюється в якусь середню пересічну величину,
сталу величину. Само собою зрозуміло, цю пересічну величину
треба визначати інакше, а не тими відхиленнями від неї самої,
що взаємно покриваються. Ця ціна праці, що панує над випадковими
ринковими цінами праці та їх реґулює, — ця «доконечна
піна» (фізіократи), або «природна ціна» праці (Адам Сміт) може
бути, як і при інших товарах, лише її вартістю, вираженою в
грошах. Цим способом політична економія гадала від випадкових
цін праці пробитися до її вартости. Як і для інших товарів,
цю вартість визначали потім витратами продукції. Але що таке
витрати продукції, витрати продукції робітника, тобто витрати,
потрібні на те, щоб спродукувати або репродукувати самого
робітника? Цим питанням політична економія несвідомо підмінила
первісне питання, бо, досліджуючи витрати продукції
праці як такої, вона крутилася мов у колі, і ніяк не могла рз’їнити
з місця. Отже, те, що вона називає вартістю праці (value of labour),
є в дійсності вартість робочої сили, яка існує в особі
робітника і так само відмінна від своєї функції, від праці, як
відмінна машина від її операцій. Захоплена ріжницею між рин-

effrayante, il ne voit qu’une ellipse gramaticale. Donc toute la société actuelle,
fondée sur le travail marchandise, est désormais fondée sur une licence
poétique, sur une expression figurée. La société veut-elle «éliminer tous
les inconvénients», qui la travaillent, eh bienl qu’elle élimine les termes
malsonnants, qu’elle change de langage, et pour cela elle, n’a qu’a s’adresser
à l’Académie pour lui demander une nouvelle édition de son dictionnaire»).
(K. Marx: «Misère de la Philosophie», p. 34, 35. — K. Маркс: «Злиденність
філософії», Партвидав 1932 р., стор.55). Природно, ще зручніше під «вартістю»
зовсім нічого не розуміти. Тоді можна залежно від обставин підводити
під цю категорію все. Так робить, наприклад, Ж. Б. Сей. Що
таке «вартість» («valeur»)? Відповідь: «Те, чого варта дана річ» («C’est
ce qu’une chose vaut»). А що таке ціна (prix)? Відповідь: «Вартість даної
речі, виражена в грошах» («La valeur d’une chose exprimée en monnaie».
A чому «праця землі має... вартість»? («le travail de la terre... une
valeur»?) «Тому, що за неї дають певну ціну» («parce qu’on у met un
prix»). Отже вартість, є те, чого варта річ, а земля має «вартість» тому,
що вартість її «виражають у грошах». У всякому разі це дуже проста
метода поясняти «чому і як» щодо речей.
\parbreak{}  %% абзац продовжується на наступній сторінці
