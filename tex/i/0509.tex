мість з’їдати парові машини, бавовну, залізниці, добриво, робочих
коней тощо або, як це собі по-дитячому уявляє вульгарний
економіст, протринькати «їхню вартість» на розкоші
й інші засоби споживання.\footnote{
«Нестатки, що їх бере на себе капіталіст, позичаючи (цього
евфемізму вжито на те, щоб за випробованою манерою вульґарних економістів
ідентифікувати найманого робітника, визискуваного промисловим
капіталістом, із самим промисловим капіталістом, що позичає гроші
в капіталістів-кредиторів!) свої знаряддя продукції робітникові, замість
присвятити їхню вартість своєму власному споживанню, перетворивши
їх на предмети споживання або втіх» («La privation que s’impose le
capitaliste, en prêtant ses instruments de production au travailleur au lieu
d’en consacrer la valeur à son propre usage, en la transformant en objets
d'utilité ou d’agrément»). (G. de Molinari: «Etudes Economiques», Paris
1846, p. 36).
} Як саме кляса капіталістів має
це зробити, — це таємниця, що її досі вперто зберігає вульгарна
економія. Досить. Світ живе лише з самокатування капіталіста,
цього сучасного покаянного поклонника Вішну. Не тільки
акумуляція, а й просте «збереження капіталу потребує постійного
напруження, щоб устояти проти спокуси з’їсти його».\footnote{
«La conservation d’un capital exige... un effort... constant pour
résister à la tentation de le consommer». (Cuurcelle Seneuil: «Traité théorique
et pratique des entreprises industrielles», 2 ème éd. Paris 1857, p. 20).
}
Отже, проста гуманність, очевидно, вимагає визволити капіталіста
від цього мучеництва і спокуси, визволити його таким самим
способом, яким недавно через скасування рабства визволено
ґеорґійського рабовласника від тяжкої дилеми — чи прогуляти
геть чисто на шампанське додатковий продукт, видушений із
негрів-рабів, чи знову перетворити його частково на додаткову
кількість негрів і землі.

В найрізніших суспільно-економічних формаціях відбувається
не тільки проста репродукція, але ще й — правда, в різних
розмірах — репродукція в поширеному маштабі. Щораз більше
продукують і більше споживають, отже, і більше продукту перетворюють
на засоби продукції. Але цей процес не є акумуляція
капіталу, а тому й не є він функція капіталіста, доки засоби продукції
робітника, а тому і його продукт і його засоби існування
не протистоять ще йому в формі капіталу.\footnote{
«Осібні кляси доходу, що найбільше сприяють прогресові національного
капіталу, змінюються на різних стадіях розвитку, а тому вони
цілком різні в націй, що стоять на різних ступенях розвитку... Зиск...
на давніших стадіях суспільного розвитку... є незначне джерело акумуляції
по івняно із заробітною платою й рентою... Коли сили національної
праці до певної міри зростають, то відносне значення зиску як
джерела акумуляції зростає». («The particular classes of income which
yield the most abundantly to the progress of national capital, change at
different stages of their progress, and are therefore entirely different in
nations occupying different positions in that progress... Profits... unimportant
source of accumulation, compared with wages and rents, in the
earlier stages of society... When a considerable advance in the powers of
national industry has actually taken place, profits rise into comparative
importance as a source of accumulation»). (Richard Jones: «Textbook
etc.», p. 16, 21).
} Померлий перед
кількома роками Річард Джонс, наступник Малтуза на катедр