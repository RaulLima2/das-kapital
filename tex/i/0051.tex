риси, що їх набирають речі, або речові риси, що їх набирають
суспільні визначення праці на основі певного способу продукції,
простими знаками, їх одночасно проголошується за самовільний
рефлективний продукт людини. Це й була улюблена манера
пояснення у XVIII столітті, яку вживано, щоб із загадкових
форм людських відносин, що їхнього процесу постання ще не
могли розшифрувати, принаймні покищо зняти подобу чогось
незрозумілого.\footnote*{
У французькому виданні цю фразу подано так: «Це була улюблена
манера пояснення у XVIII столітті; через те, що тоді ще не могли розшифрувати
ні постання, ані розвитку загадкових форм суспільних відносин,
їх загадковість намагалися відкинути тим, що їх проголошувано
людським винаходом, чимось таким, що не впало з неба». («Le Capital
etc.», v. І, ch. II, p. 39). Ред.
}

Раніш ми вже зазначали, що еквівалентна форма товару не
містить у собі кількісного визначення величини його вартости.
Коли відомо, що золото є гроші, а тому й безпосередньо вимінне
на всі інші товари, то з того ще невідомо, чого варті, приміром,
10 фунтів золота. Як і кожний інший товар, золото може виражати
величину своєї власної вартости лише відносно в інших товарах.
Його власна вартість визначається робочим часом, потрібним
на його продукцію, і виражається в такій кількості кожного
іншого товару, в якій скристалізовано стільки ж робочого часу.\footnote{
«Якщо людина може доставити до Лондону унцію срібла з перуанських
копалень за такий самий час, який був би їй потрібний на продукцію
четверика хліба, то перший з цих продуктів є природна ціна другого;
а якщо вона з відкриттям нових і багатших копалень може добути дві
унції срібла так само легко, як раніш одну, то хліб, за інших незмінних
умов, буде такий самий дешевий при ціні в 10 шилінґів за четверик, як
раніш при ціні в 5 шилінґів за четверик» («If a man can bring to London
an ounce of silver out of the earth in Peru, in the same time that he can
produce a bushel of corn, then one the natural price of the other; now if
by reason of new and more easier mines a man can get two ounces of silver
as easily as he formerly did one, the corn will be as cheap at 10 shillings
the bushel, as it was before at 5 shillings, caeteris paribus»). (William
Petty: «А Treatise of Taxes and Contributions», London 1667, p. 31).
}
Це фіксування відносної величини вартости золота відбувається
біля джерел його продукції в безпосередній міновій торговлі.
Скоро тільки воно входить як гроші в циркуляцію, то вартість
його вже дано. І коли вже в останні десятиліття XVII віку
добре знали, що гроші є товар, то все ж таки це був лише початок

призначати їм такий курс і таку ціну, яку завгодно нам і яку ми визнали
за добре» («Qu’aucum puisse ni doive faire doute, que à nous et à notre
majesté royale n’appartienne seulement... le mestier, le fait, l’état, la provision
et toute l’ordonnance des monnaies, de donner tel cours, et pour tel
prix comme il nous plait et bon nous semble»). Декретування вартости
грошей імператором було догмою римського права. Було виразно заборонено
трактувати гроші як товар. «Грішми не вільно нікому торгувати,
бо, призначені для загального користування, вони не повинні бути товаром»
(«Pecunias vero nullі emere fas erit, nam in usu publico constitutas,
oportet non esse mercem»). Досконалі пояснення про це див. у G. F. Pagnin:
«Saggio sopra il giusto pregio delle cose», 1751 y Custodi, Parte
Moderna, vol. II. Особливо полемізує Паньїні з панами юристами у другій
частині своєї праці.