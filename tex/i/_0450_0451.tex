\index{i}{0450}  %% посилання на сторінку оригінального видання
Відділ шостий

Заробітна плата

Розділ сімнадцятий

Перетворення вартости, зглядно ціни робочої
сили на заробітну плату

На поверхні буржуазного суспільства плата робітника здається
ціною праці, певного кількістю грошей, що її платять за
певну кількість праці. При цьому говорять про вартість праці
і грошовий вираз її вартости називають доконечною або природною
ціною праці. З другого боку, говорять про ринкові ціни
праці, тобто про ціни, що коливаються вище або нижче її доконечної
ціни.

Але що таке вартість якогось товару? Предметна форма витраченої
на його продукцію суспільної праці. А чим міряємо
ми величину його вартости? Величиною праці, що міститься
в ньому. Отже, чим можна б визначити вартість, наприклад,
дванадцятигодинного робочого дня? Дванадцятьма годинами
праці, що містяться у дванадцятигодинному робочому дні, а це
є нісенітна тавтологія.\footnote{
«Пан Рікардо дуже дотепно уникає труднощів, які на перший
погляд загрожують звалити його доктрину, що вартість залежить від
кількости праці, витраченої на продукцію. Якщо суворо додержувати
цього принципу, то з нього випливає, що вартість праці залежить від
кількости праці, зужитої на її продукцію, а це очевидний абсурд. Тому
за допомогою мудрого виверту пан Рікардо ставить вартість праці у
залежність від кількости праці, потрібної на продукцію заробітної плати,
або, виражаючись його власними словами, він каже, що вартість праці
оцінюється за кількістю праці, потрібної на продукцію заробітної плати;
під цим він розуміє кількість праці, потрібної на продукцію грошей або
товарів, що їх дають робітникові. Це те саме, що сказати: вартість сукна
оцінюється не за кількістю праці, витраченої на його продукцію, а за
кількістю праці, витраченої на продукцію того срібла, на яке сукно
обмінюється». («Мг. Ricardo, ingeniously enough, avoids a difficulty
which, on a first view, threatens to encumber his doctrine, that value depends
on the quantity of labour employed in production. If this principle
is rigidly adhered to, it follows that the value of labour depends on the
quantity of labour employed in producing it — which is evidently absurd.
By a dextrous turn, therefore, Mr. Ricardo make the value of labour depend
on the quantity of labour required to produce wages; or, to give him the
oenefit of his own language, he maintains, that the value of labour is to
be estimated by the quantity of labour required to produce wages; by which
}

Для того, щоб бути проданою на ринку як товар, праця мусила
б у всякому разі існувати ще до її продажу. Але коли б
\index{i}{0451}  %% посилання на сторінку оригінального видання
робітник міг дати їй самостійне існування, то він продавав би
товар, а не працю.\footnote{
«Якщо ви звете працю товаром, то все ж вона не подібна до товару,
що його спочатку продукують на обмін, а потім виносять на ринок, де
його мусять обміняти на інші товари, відповідно до кількостей кожного
з них, що можуть одночасно бути на ринку. Праця утворюється в той
момент, коли її подають на ринок, або ще точніше, її подають на ринок,
раніш, ніж її утворено». («If you call labour a commodity, it is not like a
commodity which is first produced in order to exchange, and then brought
to market where it must exchange with other commodities according to the
respective quantities of each which there may be at the market in the
time; labour is created at themoment it is brought to market; nay, it is
brought to market before it is created»). («Observations, on some verbal
disputes etc.», p. 75, 76).
}

Залишаючи осторонь ці суперечності, безпосередній обмін грошей,
тобто упредметненої праці, на живу працю або знищив би
закон вартости, що вільно розвивається саме лише на основі
капіталістичної продукції, або знищив би саму капіталістичну
продукцію, що ґрунтується саме на найманій праці. Нехай дванадцятигодинний
робочий день виражається, наприклад, у грошовій
вартості в 6 шилінґів. Або обмінюються еквіваленти, і тоді
робітник за дванадцятигодинну працю дістає 6 шилінґів. Ціна
його праці дорівнювала б ціні його продукту. В цьому разі він
не продукував би жодної додаткової вартости для покупця його
праці, ці 6 шилінґів не перетворилися б на капітал, основа капіталістичної
продукції зникла б, алеж саме на цій основі він
продає свою працю, саме на цій основі його праця є наймана
праця. Абож робітник дістає за 12 годин праці менше ніж 6 шилінґів,
тобто менше ніж 12 годин праці. Дванадцять годин праці
обмінюється на 10, 6 і т. д. годин праці. Це урівняння нерівних
величин не тільки знищує всяке визначення вартости. Таку
суперечність, що сама себе знищує, не можна взагалі навіть висловити
або формулювати як закон.\footnote{
«Якщо розглядати працю як товар, а капітал, продукт праці,
як інший товар, то тоді, якщо вартості цих двох товарів визначаються
однаковими кількостями праці, дану суму праці\dots{} обмінюватиметься
на таку кількість капіталу, що була спродукована такою самою кількістю
праці; минулу працю обмінюватиметься\dots{} на таку саму суму теперішньої
праці. Але вартість праці у відношенні до інших товарів\dots{} не визначається
однаковими кількостями праці». («Treating Labour as a commodity,
and Capital, the produce of labour, as another, then, if the values
of those two commodities were regulated by equal quantities of labour,
a given amount of labour would\dots{} exchange for that quantity of capital
which had been produced by the same amount of labour; antecedent labour
would\dots{} exchange for the same amount as present labour. But the value
of labour, in relation to other commodities\dots{} is determined not by equall
quantities of labour»). (E. G. Wakefield у його виданні А. Сміса «Wealth
of Nations», London 1836, т. I, crop. 231, примітка).
}

he means the quantity of labour required to produce the money or commodities
given to the labourer. This is similar to saying, that the value of cloth
is estimated, not by the quantity of labour bestowed on its production,
but by the quantity of labour bestowed on the production of the silver,
for which the cloth is exchanged»). («А Critical Dissertation on the Nature
etc. of Value», p. 50, 51).
