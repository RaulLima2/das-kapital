цей капітал на число робітничої людности».\footnote{
Я нагадую тут читачеві, що категорії змінний капітал і сталий
капітал вперше почав уживати я. Політична економія, починаючи від
А. Сміса, сплутує визначення, що містяться в них, з тими ріжницями
форм основного і обігового капіталу, що постають із процесу циркуляції.
Докладніше про це в другій книзі, другий відділ.
} Отже, це значить,
що спочатку ми підсумовуємо дійсно заплачені індивідуальні
заробітні плати, а потім уже твердимо, що результат цього додавання
становить суму вартости «робочого фонду», дарованого
богом і природою. Нарешті, визначену таким способом суму
ділимо на число робітників, щоб знову найти, скільки пересічно
може припасти кожному робітникові індивідуально. Це надзвичайно
хитромудра процедура. Вона не заважає панові Фавсетові
за одним духом сказати: «Ціле багатство, що його щорічно
акумулюється в Англії, поділяється на дві частини. Одну
частину застосовують в Англії для підтримання нашої власної
промисловости. Другу його частину експортують до інших
країн... Та частина, що її застосовують у нашій промисловості,
становить лише незначну частину багатства, що його щорічно
акумулюється в цій країні».\footnote{
Н. Fawsett, там же, стор. 123, 122.
} Отже, більшу частину додаткового
продукту, який щорічно наростає і який відбирають в англійського
робітника без еквіваленту, капіталізується не в Англії,
а в чужих краях. Алеж разом з експортованим таким чином
додатковим капіталом експортують і частину «робочого фонду»,
вигаданого богом і Бентамом».69*

69    Можна було б сказати, що не тільки капітал, а й робітників щороку
вивозять з Англії у формі еміграції. Однак, у тексті немає навіть і згадки
про майно переселенців, які здебільшого неробітники. Велика частина
їх сини фармерів. Англійський додатковий капітал, що його вивозять
щороку за кордон для того, щоб мати з нього проценти, становить далеко
більшу величину порівняно з щорічною акумуляцією, аніж щорічна
еміграція порівняно з щорічним приростом людности.

* Що у французькому виданні теорію «робочого фонду» і критику
її подано повніше, наводимо тут відповідні уривки з цього видання:

«Капіталісти, їхні співвласники, їхні слуги та їхній уряд щороку
прогулюють чималу частину додаткового продукту. Крім того, вони
затримують у своїх споживних запасах багато повільно споживаних
предметів, придатних для репродукції, і перетворюють на непродуктивні
багато робочих сил, уживаючи їх для своїх особистих послуг.
Отже, частина багатства, що капіталізується, ніколи не досягає тієї
величини, якої вона могла б досягти. Її відношення до сукупного суспільного
багатства змінюється при всякій зміні в поділі додаткової вартости
на особистий дохід і додатковий капітал, а пропорція, в якій відбувається
цей поділ, постійно варіює під впливом коньюнктур, що на них ми
тут не зупиняємось. Для нас досить сконстатувати, що капітал зовсім
не є наперед визначена і фіксована частина суспільного багатства, а,
навпаки, змінна і хитка його частина...

Догма, ніби суспільний капітал завжди є певна фіксована величина,
не лише суперечить найзвичайнішим явищам продукції, як от рухам її
поширення і звуження, але вона робить незрозумілою і саму акумуляцію...
Цю догму Бентам і його прихильники — Мак-Куллох, Мілл та інші —
застосовували переважно до тієї частини капіталу, що обмінюється на
робочу силу і що її вони називають «фондом заробітної плати», або «робочим
фондом». На їх погляд це є осібна частина суспільного багатства
