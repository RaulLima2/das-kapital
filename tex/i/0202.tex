дорослих припадає один або двоє підлітків. Їхнє підприємство нараховує
500 підлітків молодших за 18 років, з них біля третини,
або 170, молодші за 13 років. Щодо проектованої зміни закону
пан Елліс гадає: «Я не думаю, щоб це заслуговувало на велику
догану (very objectionable), якби не дозволяли особам молодшим
за 18 років працювати понад 12 годин на добу. Але я не думаю,
щоб можна було провести якусь демаркаційну лінію, щоб перешкодити
нам застосовувати вночі молодь понад 12 років. Ми навіть
радше ухвалили б закон, що забороняв би взагалі вживати
підлітків, які не мають 13, а то й 15 років, ніж заборону використовувати
вночі підлітків, які вже в нас працюють. Підлітки,
що працюють у денній зміні, мусять навпереміну працювати і
в нічній зміні, бо дорослі робітники не можуть постійно працювати
вночі; це зруйнувало б їхнє здоров’я. Однак, ми гадаємо,
що нічна праця, чергуючись тиждень-у-тиждень з денною,
не шкодить здоров’ю. (Панове Нейлор і Вікерс, у згоді з інтересами
свого підприємства, гадали, навпаки, що не постійна, а
саме періодична змінна нічна праця може шкодити). Ми гадаємо,
що люди, які навпереміну працюють то вночі, то вдень, так само
здорові, як і ті, які працюють лише вдень... Наші заперечення
проти заборони нічної праці підлітків молодших за 18 років
випливали б з того, що це збільшило б видатки, але це і є однісінька
підстава. (Що за цинічна наївність!) Ми вважаємо, що це
збільшення видатків перевищило б те, що його могло б витримати
підприємство (the trade), як звертати належну увагу на його
успіхи. (As the trade with due regard to etc. could fairly bear!
Яка кашкувата фразеологія!). Тут праця рідка, а за такого реґулювання
могла б стати й недостатня» (тобто Елліс, Бравн і К°
могли б попасти в таке фатальне становище, що змушені були б
оплачувати повну вартість робочої сили).100

«Циклопічні фабрики сталі й заліза» панів Кеммел і К°
працюють у такому самому великому маштабі, як і підприємства
згаданих Джон Бравн і К°. Директор, керівник підприємства,
подав урядовому комісарові Байтові свої писані свідчення, але
пізніш вважав за потрібне затаїти рукопис, повернений йому на
перегляд. Однак у пана Байта гарна пам’ять. Він пригадує собі
дуже докладно, що для цих панів циклопів заборона нічної
праці для дітей і підлітків є «неможлива річ; це було б усе одно,
що припинити їхню фабрику», а проте їхнє підприємство нараховує
трохи більше за 6\% підлітків молодших за 18 років і лише
1\% молодших за 13 років! 101

Пан Е. Ф. Сандерсон від фірми Сандерсон, Брос і К°, фабрики
сталі, вальцювалень і кузень в Attercliffe, висловлюється на ту
саму тему так: «Із заборони вживати нічної праці підлітків молодших
за 18 років постали б великі труднощі; головна трудність
постала б із збільшення витрат, що його неодмінно потягла б за

100 Там же, 80, стор. XVI.
101 Там же, 82, стор. XVII.
