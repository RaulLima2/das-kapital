талю для хорих на тиф у Ньюкестлі каже: «Безперечно, причина
тривання й поширення тифу є переповнення помешкань людьми
та нечистота цих помешкань. Доми, де звичайно живуть робітники,
стоять у глухих заулках і подвір’ях. Щодо світла, повітря,
простору й чистоти це є справді зразки недостатности й негігієнічности,
ганьба для кожної цивілізованої країни. Там чоловіки
жінки й діти лежать ночами вкупі, поперемішувані як попало.
Щодо чоловіків, то нічна зміна невпинною течією йде по
денній, так що ліжка ледве встигають прохолонути. Доми погано
забезпечено водою і ще гірше кльозетами, вони страшенно нечисті,
не вентилюються, поширюють заразу». 124 Тижнева плата
за такі діри становить від 8 пенсів до 3 шилінґів. «Ньюкестл
над Тайном, каже д-р Гентер, являє собою приклад того, як
одне з найкращих племен поміж нашими земляками через зовнішні
умови, а саме через стан помешкань і вулиць, занепадає
часто майже до стану дикого виродження».124

У наслідок постійного припливу й відпливу капіталу й праці
житловий стан якогось промислового міста може бути сьогодні
стерпний, а на завтра стає огидний. Або ж міська влада, нарешті,
отямлюється і починає усувати щонайгірші непорядки. Але на
завтра ж хмарою сарани насувають обідрані ірляндці або занепаді
англійські рільничі робітники. Їх запихають у льохи й
комори, або порядний колись дім для робітників перетворюють
на помешкання, де персонал змінюється так швидко, як солдатські
постої підчас тридцятирічної війни. Приклад: Bradford.
Саме там муніципальні філістери заходилися коло міської реформи.
Крім того, там 1861 р. було ще 1.751 незаселений дім.
Аж ось справи пішли добре, як про це нещодавно так любо
розводився солодкувато-ліберальний професор Форстер, приятель
негрів. Певна річ, з поліпшенням справ постає повідь
від хвиль «резервної армії», або «відносного перелюднення»,
що постійно хвилюється. Найгидкіші мешкання по льохах
та коморах, зареєстрованих у списку,\footnote{
Список, що його склав аґент одного товариства для забезпечення
робітників у Bradford’i:

Vulcanstreet. Nr. 122..............1 кімната 16 осіб
Lumleystreet. Nr. 13...............1       »        11 осіб
Bowerstreet. Nr. 41.................1        »       11 осіб
Portlandstreet. Nr. 112...........1        »       10 осіб
Hardystreet. Nr. 17. ...............1        »        10 осіб
Northstreet. Nr. 18.................1        »       16 осіб
         ditto Nr.. 17..................1       »       13 осіб
Wymerstreet. Nr. 19................1       »        8 дорослих
Jowettstreet. Nr. 56.................1       »        12 осіб
Georgestreet. Nr. 150..............1       »         3 родини
Rifle-Court, Marygate. Nr. 11...1       »         11 осіб
Marshallstreet. Nr. 28.............1        »         10 осіб
                ditto Nr. 49................1       »         3 родини
Georgestreet. Nr. 128..............1       »         18 осіб
                 ditto Nr. 130.............1       »         16 осіб
} що його одержав
д-р Гентер від аґента одного страхового товариства, займали
здебільшого добре оплачувані робітники. Вони заявляли, що

124 Там же, стор. 149.

125 Там же, стор. 50.