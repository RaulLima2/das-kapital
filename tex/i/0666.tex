мірі перевищує найдоконечніші засоби існування робітника?.. Без сумніву, робочі коні в сільському
господарстві мають в Англії далеко кращий корм, аніж англійські рільничі робітники, бож коні є цінне
майно».\footnote{
Там же, т. І, стор. 47, 246.
} Але never mind,* адже національне багатство з природи тотожнє з народніми злиднями.

Але як же вилікувати колонії від цієї антикапіталістичної болячки? Коли б хто хотів за одним махом
перетворити всю землю з народньої власности на приватну власність, то цим би він, правда, знищив
корінь зла, але разом з тим — і колонії. Майстерність у тім, щоб одним пострілом убити двох зайців.
Треба, щоб уряд надав незайманій землі штучну ціну, незалежну від закону попиту й подання, ціну, що
примусить еміґранта
працювати довший час найманим робітником, доки він заробить досить грошей, щоб купити собі землю\footnote{
«Ви кажете, що завдяки присвоєнню землі й капіталів, людина, яка не має нічого, крім своїх рук,
находить собі роботу та створює собі дохід... навпаки, лише завдяки індивідуальному присвоєнню
землі, стається те, що є люди, які не мають нічого, крім своїх рук. Ставлячи людину в безповітряний
простір, ви захоплюєте собі атмосферу. Те саме ви робите, захоплюючи собі землю... Це все одно, що
кинути людину в простір, де немає багатств, щоб зробити її життя залежним від вашої волі». («C’est,
ajoutez-vous, grâce à l’appropriation du sol et des capitaux que l’homme, qui n’a que ses bras,
trouve de l’occupation, et se fait un revenu... c’est au contraire, grâce à l’appropriation
individuelle du
sol qu’il se trouve des hommes n’ayant que leurs bras... Quand vous mettez un homme dans le vide,
vous vous emparez de l’atmosphère. Ainsi faites-vous, quand vous vous emparez du sol. C’est le
mettre dans le vide de richesse, pour ne le laisser vivre qu’à votre volonté»). (Colins: «L’Economie
Politique, Source des Révolutions et des Utopies prétendues Sосіаlistes», Paris 1857, vol. III, p.
267—271 passim.).
* — що з того. Ред.
** Все буде якнайкраще в цьому найкращому із світів. Ред.
}
й перетворитись на незалежного селянина. З другого боку, фонду, створеного через продаж земель по
ціні, порівняно неприступній для найманого робітника, отже, цього грошового фонду, видушеного із
заробітної плати через порушення святого закону попиту й подання, уряд повинен уживати в міру його
зростання на те, щоб імпортувати з Европи в колонії голоту і підтримувати таким
чином для пана капіталіста ринок найманої праці повним. За таких обставин tout sera pour le mieux
dans le meilleur des mondes possibles.** Оце — велика таємниця «систематичної колонізації». «За цим
пляном, — вигукує тріюмфуючи Векфілд, —
подання праці мусить бути стале й реґулярне; бо, поперше, через те, що жоден робітник не має змоги
купити собі землі доти, доки він не попрацює певний час за гроші, всі еміґранти-робітники, працюючи
комбінованими групами як наймані робітники, продукували б своєму підприємцеві капітал для вживання
ще більшої кількости праці; подруге, кожний, що кинув би найману працю і став би земельним
власником, саме через купівлю землі забезпечував би певний фонд, щоб пристав-