людини людиною. Однак лицарі промисловости витиснули лицарів
меча лише тим, що вони скористалися з подій, які сталися
без жодної їхньої участи. Вони піднеслися за допомогою таких
самих ницих засобів, якими колись користувалися римські пущені
з рабства, щоб стати владарями своїх патронів.

Вихідним пунктом розвитку, що утворив так найманого робітника,
як і капіталіста, було рабство робітника. Проґрес полягав
у зміні форми цього рабства, у перетворенні февдальної експлуатації
на капіталістичну експлуатацію. Щоб зрозуміти перебіг
цього процесу, нам зовсім не треба заглиблюватись у дуже
далеке минуле. Хоч перші початки капіталістичної продукції
спорадично можна спостерігати ще в XIV і XV століттях по
деяких містах на побережжі Середземного моря, однак капіталістична
ера починається лише з XVI століття. Там, де вона настає,
вже давно знищено кріпацтво і з давнього часу почався
занепад краси й гордощів середньовіччя — вільних міст.

В історії первісної акумуляції епохальне значення мають всі
ті перевороти, які служили за підойму піднесення кляси капіталістів,
що формувалася; але надто важливе значення мають
ті моменти, коли величезні маси людей раптом і силоміць відривано
від засобів їхнього існування й викидувано на ринок праці
як вільних, мов птахи, пролетарів. Експропріяція землі в сільського
продуцента, селянина, становить основу цілого процесу.
[Тому ми повинні розглянути її насамперед].* Історія її в
різних країнах набирає різного забарвле ння, перебігає різні
фази в різній послідовності і в різні історичні епохи. Клясичну
форму вона має лише в Англії, яку ми тому й беремо як приклад.189

2. Експропріяція землі в сільської людности

В Англії кріпацтво зникло фактично наприкінці XIV століття.
Величезна більшість людности 190 складалась тоді — і ще
більше в XV столітті — з вільних селян, які господарювали
самостійно, хоч за якими февдальними вивісками ховалася їхня
власність. У великих панських маєтках вільний фармер витиснув

189 В Італії, де капіталістична продукція розвинулась найраніше,
найраніше відбувся і розклад кріпацьких відносин. Кріпака тут визволено
раніш, ніж він устиг забезпечити собі яке-небудь право давности на
землю. Тому визволення відразу перетворило його на вільного, мов
птах, пролетаря, який до того ж находить нових панів у тих містах, що
здебільша збереглися ще від римської епохи. Коли революція на світовому
ринку, що почалася з кінця XV століття, знищила торговельну
перевагу північної Італії, то постав рух зворотного напряму. Робітників
масами витискували з міст на село, де відтоді дрібна культура, організована
за типом садівництва, набула нечуваного розквіту.

190 «Дрібні землевласники, які обробляли власні поля власного працею
і тішилися скромними достатками... становили тоді куди більшу
частину нації, ніж тепер... Не менш, як 160.000 землевласників, що разом
із своїми родинами становили більше однієї сьомої частини всієї людности,
жили з того, що обробляли свої дрібні Freehold-дільниці [Freehold —

* Заведене у прямі дужки беремо з другого німецького видання. Ред.
