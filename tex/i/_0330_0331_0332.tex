\parcont{}  %% абзац починається на попередній сторінці 
\index{i}{0330}  %% посилання на сторінку оригінального видання 
вплив задушливого та гидкого повітря на бідні діти... Я побував
у багатьох таких школах, там я бачив цілі лави дітей, що абсолютно
нічого не робили; і ці діти мали посвідки, що ходили до
школи, а в офіціяльній статистиці вони фігурують як такі, що
дістали освіту (educated)».\footnote{
Leonhard Horner у «Reports etc. for 31 st October 1857», р. 17, 18.
} В Шотляндії фабриканти силкуються
по змозі не приймати на роботу дітей, що мусять ходити до школи.
«Цього досить, щоб показати велику неприхильність фабрикантів
до пунктів закону про виховання дітей»\footnote{
Sir J. Kincaid у «Reports etc. for 31 st October 1856», p. 66.
}. У неймовірно
жахливій формі виявляється це по перкалевибійних фабриках та
інших вибійнях, що підведені під осібний фабричний закон.
За постановами цього закону, «кожна дитина, раніш ніж її можна
прийняти до такої вибійні на роботу, мусить відвідувати школу
щонайменше 30 днів та не менш як 150 годин протягом 6 місяців
безпосередньо перед тим днем, коли вона вперше починає працювати.
Протягом того часу, коли дитина працює на цих вибійних
фабриках, вона так само мусить що шість місяців на рік ходити
до школи по 30 день, або 150 годин... Відвідувати школу треба
між 8 годиною ранку й 6 годиною по півдні. Відвідування школи,
що триває менше ніж 2 1/2  години або більше як 5 годин на день,
не може вважатися за частину тих 150 годин. За звичайних обставин
діти відвідують школу вранці й по півдні протягом 30 днів
по 5 годин на день, а після цих 30 днів, дійшовши встановленої
статутами повної суми в 150 годин, скінчивши, як вони сами
висловлюються, свою книжку, вони повертаються до вибійні
й лишаються там знову 6 місяців, доки знову прийде строк іти
до школи; тоді вони знову лишаються в школі доти, доки знову
скінчать свою книжку... Дуже багато підлітків, які відвідували
школу протягом приписаних 150 годин, вертаючись до неї після
шестимісячного перебування на фабриці, знають не більше, ніж
вони знали з самого початку... Певна річ, вони знову позабували
все, чого набралися раніш, відвідуючи школу. По інших перкалевибійних
фабриках відвідування школи геть чисто залежить від
потреб фабрики. Потрібне число годин протягом кожного півріччя
поповнюється зарахуванням 3—5-годинних відвідувань, що порозкидувані,
може, і по цілому півроці. Приміром, одного дня
школу відвідують від 8 до 11 години ранку, іншого дня — від
1 до 4 години по півдні, і після того, як дитина потім знову декілька
днів не відвідувала школу, вона раптом знову приходить
від 3 до 6 години по півдні; потім, може бути, вона ходить 3 або 4 дні
або й цілий тиждень підряд, а далі знову зникає на 3 тижні або
на цілий місяць та вертається на декілька годин у вільні дні,
коли підприємець випадково її не потребує; отак дитину, так би
мовити, кидають туди та сюди (buffeted), із школи до фабрики,
з фабрики до школи, поки нараховується сума в 150 годин».\footnote{
A. Redgrave у «Reports of Insp. of Fact. for 31 st October
1857», p. 41, 42. По тих галузях англійської промисловости, де від давні-
}
\index{i}{0331}  %% посилання на сторінку оригінального видання 
Додаючи переважну кількість дітей та жінок до комбінованого
робочого персоналу, машина, кінець-кінцем, ламає опір, що
його чоловік-робітник у мануфактурі ставив ще деспотизмові
капіталу.\footnote{
«Пан E., фабрикант, повідомив мене, що коло своїх механічних
ткацьких варстатів він вживає виключно жіночої праці; він дає перевагу
заміжнім жінкам, особливо жінкам, що мають дома родину, утримання
якої залежить від них; вони куди уважніші та слухняніші, ніж незаміжні,
та мусять до крайности напружувати свої сили, щоб добувати собі доконечні
засоби існування. Таким чином чесноти, властиві жіночому характерові,
повертаються їм на шкоду, — таким чином усе моральне й ніжне
їхньої природи стає засобом їхнього поневолення та джерелом їхнього
страждання». («Ten Hours Factory Bill. The Speech of Lord Ashley, 15 th
March», London 1844, p. 20).
}

b) Здовження робочого дня

Якщо машина є якнаймогутніший засіб збільшувати продуктивну
силу праці, тобто скорочувати робочий час, потрібний для
продукції товару, то, як носій капіталу, стає вона насамперед
у безпосередньо охоплених нею галузях промисловости якнаймогутнішим
засобом здовжувати робочий день поза всяку природну
межу. Вона створює, з одного боку, нові умови, що дозволяють
капіталові давати повну волю цій своїй постійній тенденції,
з другого боку, вона створює нові мотиви до загострення його
ненажерливої жадоби чужої праці.

Насамперед рух та функціонування засобу праці в машині
усамостійнюється проти робітника. Засіб праці стає сам по собі
промисловим perpetuum mobile, яке продукувало б безнастанно,
коли б воно не натрапляло на певні природні межі у своїх помічниках
— людях: на слабощі їхнього тіла й на їхню сваволю.
Тому, як капітал, — а, як такий, автомат має в особі капіталіста
свою свідомість і волю, — засіб праці є надхнений прагненням
звести людські природні межі, що ставлять йому опір, але є елястичні,
до мінімуму опору.\footnote{
«З того часу, як повсюди заведено коштовні машини, людину
примусили працювати далеко більше, ніж їй пересічно під силу» («Sіnce
the general introduction of expensive machinery, human nature has been
forced far beyond its average strength»). (Robert Owen: «Observations
on the effects of the manufacturing system», 2 nd ed. London 1817).
} Але й без того цей опір зменшується
через позірну легкість праці коло машини та більшу покірливість
і гнучкість жіночого й дитячого елементу».\footnote{
Англійці, які охоче розглядають першу емпіричну форму виявлення
речі, як її причину, часто вважають за причину довгого робочого
}

шого часу панує фабричний закон у власному значенні (не Print Work's
Act, що його ми щойно навели в тексті), перешкоди проти пунктів про
виховання за останні роки до певної міри переборено. А в тих галузях
промисловости, які не підведені під фабричний закон, ще й досі цілком
панують погляди фабриканта скла, Дж. Ґедса, який так навчав члена слідчої
комісії Вайта: «Оскільки я розумію, більша освіта, яку дістала останніми
роками певна частина робітничої кляси, є лихо. Вона небезпечна,
бо робить робітників надто незалежними». («Children’s Employment Commission.
4 th Report», London 1865, p. 253).

\index{i}{0332}  %% посилання на сторінку оригінального видання 
Продуктивність машин, як ми вже бачили, стоїть у зворотному
відношенні до величини тієї складової частини вартости, яку вона
переносить на продукт. Що довший той період, протягом якого
машина функціонує, то більша маса продуктів, на яку розділяється
додана нею вартість, і то менша та частина вартости
яку вона долучає до кожного окремого товару. Період активного
життя машини, очевидно, визначається довжиною робочого дня
або триванням денного процесу праці, помноженого на число
днів, у які цей процес повторюється.

Зужиткування машини зовсім не відповідає з математичною
точністю часові користування нею.\footnote*{
На берегах свого власного екземпляра першого видання Маркс
тут дає таку примітку: «Це має силу й щодо інших витрат, пов’язаних із
машинами. Наприклад: «Кожний фабрикант знає, що коли треба розігріти
парову машину, то здобути пару на 3 години коштує стільки ж саме,
скільки і на 4 години... Звідси постає (для залізниць) маленька економія
на паливі, коли перебіг робиться на велику віддаль». («Royal Commission
on Railways», London 1867. Evidence, p. 175). Ред.
} Та навіть, коли припустити
таку відповідність, то й тоді машина, яка служить протягом
7 1/2 років по 16 годин щоденно, охоплює такий самий великий
період продукції та додає до загального продукту не більше
вартости, ніж та сама машина, що протягом 15 років служить
лише по 8 годин на день. Але в першому випадку вартість машини
була б репродукована удвоє швидше, ніж в останньому, і капіталіст
у першому випадку за допомогою цієї машини проковтнув би
за 7 1/2  років стільки ж додаткової праці, скільки в другому випадку
за 15 років.

Матеріяльне зужиткування машини є двояке. Одно випливає
з її уживання — так само, як монети стираються від циркуляції,
друге — з невживання її, як от меч без ужитку ржавіє в піхвах,
В останньому випадку вона стає здобиччю стихій. Зужиткування
першого роду стоїть більше або менше у прямому відношенні,

часу на фабриках те велике іродське грабіжництво дітей, що його капітал
на податках фабричної системи практикував по домах для бідних та
сиріт, і за допомогою якого він здобув собі цілком безвільний людський
матеріял. Так, наприклад, Філден, сам англійський фабрикант, каже:
«Очевидно, робочий день здовжувала та обставина, що велика численність
безпритульних дітей, яких приводили з різних частин країни, унезалежнювала
підприємців від робочих рук, і вони, завівши за допомогою такого
нужденного, так добутого матеріялу, звичай довгої праці, дуже легко
могли накинути це і своїм сусідам» («It is evident that the long hours of
work were brought about by the circumstance of so great a number of destitute
children being supplied from different parts of the country, that the
masters were independent of the hands, and that, having once established
the custom by means of the miserable materials which they had procured in
this way, they could impose it on their neighbours with the greater facility»).
(J. Fielden: «The Curse of the Factory System», London 1836, p. 11).
Щодо жіночої праці фабричний інспектор Савндерс каже у фабричному
звіті за 1844 рік: «Серед робітниць є жінки, які багато тижнів один по
одному, за винятком лише небагатьох днів, працюють від 6 години ранку
до 12 години ночі, маючи менше ніж 2 годин на їжу, так що 5 днів на тиждень
у них із 24 годин лишається тільки 6 годин на те, щоб дійти додому
й назад та відпочити в ліжку».
\parbreak{}  %% абзац продовжується на наступній сторінці
