\parcont{}  %% абзац починається на попередній сторінці
\index{i}{0533}  %% посилання на сторінку оригінального видання
і до маси засобів продукції, сконцентрованих у будівлях, велетенських
домнах, засобах транспорту й т. ін. Але чи є зростання
величини розміру засобів продукції порівняно з долученою до
них робочою силою умова чи наслідок — усе одно воно виражає
зростання продуктивности праці. Отже, збільшення продуктивности
праці виявляється у зменшенні маси праці порівняно до
тієї маси засобів продукції, що її пускає в рух ця праця, або у
зменшенні величини суб’єктивного фактора процесу праці порівняно
з його об’єктивними факторами.

[З постанням великої промисловости в Англії винайдено
спосіб перетворювати чавун з коксом на ковке залізо. Цей
спосіб, який звуть пудлінґуванням, і який полягає в тому, що
в печах особливої конструкції чавун очищають од вуглецю,
спричинив величезне поширення домен, вживання повітронагрівних
апаратів і т. ін., коротко кажучи, таке збільшення знарядь
праці і матеріялів праці, урухомлюваних тією самою кількістю
праці, що незабаром можна було постачати залізо у досить великій
кількості і досить дешево для того, щоб у багатьох випадках
витиснити з ужитку камінь і дерево. А що залізо й вугілля є
великі підойми сучасної промисловости, то значення цього новозаведеного
способу ніяк не можна перебільшити.

Однак пудлінґувач, робітник, занятий очищенням чавуну,
виконує ручну операцію; отож величина печей, що їх він може
обслуговувати, обмежена його особистими здібностями, і саме
ця межа затримує тепер те дивовижне піднесення металюрґійної
промисловости, що почалося 1780 року, року винаходу пудлінґування.
«Факт той, — вигукує «The Engineering», один з органів
англійських інженерів, — що застарілий спосіб ручного пудлінґування
є не що інше, як рештки від варварства (the fact is that the old
process of hand-puddling is little better than a barbarism)\dots{}
Сучасна тенденція нашої промисловости в тому, щоб на різних
ступенях фабрикації обробляти чимраз більші маси матеріялу.
Тим то ми бачимо, що майже щороку постають чимраз
більші домни, чимраз важніші парові молоти, чимраз могутніші
вальцівні варстати і велетенські знаряддя, що їх застосовують
у багатьох галузях металюрґії. Серед цього загального зросту —
зросту засобів продукції проти уживаної праці — спосіб пудлінґування
залишився майже незмінним і ставить тепер нестерпні
межі промисловому розвиткові\dots{} Тим то по всіх великих заводах
починають замінювати ручний спосіб пудлінґування на
печі з автоматичним перемішуванням, що дає можливість колосально
навантажувати печі цілком незалежно від меж ручної
праці». («The Engineering», 13 June, 1874).

Отже, пудлінґування, після того, як воно революціонізувало
металюрґійну промисловість і викликало величезне збільшення
засобів праці і матеріялів праці, оброблюваних певною кількістю
праці, стало в перебігу акумуляції економічним гальмом. Від
цього гальма промисловість тепер намагається визволитися новими
\parbreak{}  %% абзац продовжується на наступній сторінці
