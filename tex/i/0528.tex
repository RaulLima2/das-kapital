За тих найсприятливіших для робітників умов акумуляції,
які ми досі припускали, відношення залежности робітників од
капіталу прибирається у зносні, або, як каже Еден, у «приємні
й ліберальні» форми. Замість ставати із зростом капіталу інтенсивнішим,
воно стає тільки екстенсивнішим, тобто сфера експлуатації
й, панування капіталу лише поширюється разом із збіль-

Мандевіль, Кене. Ще в середині XVIII віку піп Текер, видатний економіст
свого часу, прохає вибачення за те, що він займався мамоною. Пізніше,
а саме одночасно з «принципом залюднення», настав час протестантських
попів. Неначе передчуваючи це партацтво, Петті, що вважає людність
за базу багатства і в, так само як і Адам Сміс, непримиренний
ворог попів, каже: «Релігія найкраще процвітає там, де священники найбільше
підпадають усмирению плоті, так само як право найкраще процвітає
там, де адвокати вмирають з голоду». Тому він радить протестантським
попам, якщо вони не хочуть іти за прикладом апостола Павла
і «умерщвляти свою плоть» безженством, «принаймні не плодити більше
попів («not to breed more Churchmen»), аніж їх могли б поглинути наявні
парафії (benefices); тобто коли в Англії й Велзі існує лише 12.000 парафій,
то нерозумно гіаплоджувати 24.000 попів («it will not be safe to breed
24.000 ministers»), бо 12.000 незабезпечених завжди намагатимуться
здобути собі засоби існування, а як можуть вони найлегше досягти цього,
як не ходячи серед народу та переконуючи його в тому, що ті 12.000 попів,
що мають парафії, отруюють душі, заморюють їх голодом та вказують
їм неправдивий шлях до неба?» (Petty: «A Treatise on Taxes and Contributions»
, London 1667, p. 57). Ставлення Адама Сміса до протестантського
попівства його часів характеризується ось чим. В «А Letter to A. Smith,
L. L. D. On the Life, Death and Philosophy of his Friend David Hume.
By One of the People called Christians», 4th ed. Oxford 1784 англіканський
єпископ д-р Херн із Норвіча докоряє А. Смісові за те, що він в одному
відкритому листі до пана Стрехена «бальзамує свого приятеля Давіда»
(тобто Юма), що він оповідає публіці, як «Юм на своєму смертному ліжку
розважував себе Лукіяном і Вайстом», і що він навіть мав нахабство написати:
«Я завжди вважав Юма так за його життя, як і після його смерти
таким близьким до ідеалу цілком мудрої й доброчесної людини, як це тільки
дозволяють слабощі людської натури». Єпископ з обуренням вигукує:
«Чи то воно гаразд з вашого боку, мій пане, змальовувати нам як цілком
мудрий і доброчесний характер і побут людини, що була пройнята невигойною
антипатією до всього того, що зветься релігією, і напружувала
кожний свій нерв, щоб, оскільки це від неї залежало, стерти з людської
пам’яті навіть назву релігія?» (Там же, стор. 8). «Але не журіться ви,
приятелі правди, атеїзмові недовго жити» (стор. 17). Адам Сміс — «є
гидкий нечестивець («the atrocious wickedness»), він пропагує у країні
атеїзм (саме своєю «Theory of moral sentiments»)... Ми знаємо ваші хитрощі,
пане докторе 1 Ви добрий задум мали, але цим разом ви рахували без господаря.
На прикладі високошановного Давіда Юма ви хочете напоумити
нас, що атеїзм — це єдиний живлющий лік («cordial») для занепалого
духу і єдина протиотрута супроти страху перед смертю... Глузуйте ж
собі з руїн Вавилону та вітайте озвірілого лиходія Фараона»! (Там же,
стор. 21, 22). Один з ортодоксальних слухачів колегії, де навчав А. Сміс,
пише після його смерти: «Приятелювання Сміса з Юмом... перешкоджало
йому бути християнином... Він вірив Юмові в усьому на слово.
Коли б Юм сказав йому, що місяць — зелений сир, він був би йому повірив.
Тим то він вірив йому, що немає бога й чудес... Своїми політичними
принципами він наближався до республіканізму». («The Bee». By James
Anderson. 18 volumes. Edinburgh 1791—93, vol. Ill, p. 166, 165). Піц
T. Чалмерс запідозрює А. Сміса в-тому, що він вигадав категорію «непродуктивних
робітників» просто із злости спеціально для протестантських
попів, не зважаючи на їхню благословенну працю в божому вертограді.
