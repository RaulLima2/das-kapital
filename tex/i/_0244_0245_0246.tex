\parcont{}  %% абзац починається на попередній сторінці
\index{i}{0244}  %% посилання на сторінку оригінального видання
звичайно, більше сировинного матеріялу, веретен і~\abbr{т. ін.}, ніж
для того, щоб уживати 100 прядунів. Але вартість цих додаткових
засобів продукції може збільшуватися, падати, лишатися незмінною,
бути великою або малою, — це не має значення; вона однаково
лишається без якогобудь впливу на процес зростання вартости,
зумовлюваний тими робочими силами, що пускають ці
засоби в рух. Отже, закон, констатований вище, набирає такої
форми: продуковані різними капіталами маси вартости й додаткової
вартости за даної вартости й однакового ступеня експлуатації
робочої сили є просто пропорційні до величин змінних складових
частин цих капіталів, тобто їхніх складових частин, перетворених
на живу робочу силу.

Цей закон явно суперечить усякому досвідові, побудованому
на видимості. Кожний знає, що прядун бавовни, який, коли
обчислити складові частини всього застосованого капіталу в процентах,
вживає відносно багато сталого й мало змінного капіталу,
з цієї причини добуває собі від нього не менший зиск або не меншу
додаткову вартість, як пекар, що пускає в рух відносно багато
змінного й мало сталого капіталу. Щоб розв’язати цю позірну
суперечність, треба ще багатьох посередніх ланок, як от в елементарній
альґебрі треба багатьох посередніх ланок, щоб зрозуміти,
що \frac{0}{0} може репрезентувати дійсну величину. Хоч клясична політична
економія ніколи не формулювала цього закону, але вона
інстинктово цупко тримається його, бо він є доконечний наслідок
закону вартости взагалі. Насильною абстракцією вона силкується
врятувати його від суперечностей виявлення. Пізніш\footnote{
Докладніше про це буде в «Четвертій книзі»\footnote*{
Мова йде про «Теорії додаткової вартости», що їх Маркс мав на
думці видати як «Четверту книгу Капіталу». \emph{Ред.}
}.
}
ми побачимо, як школа Рікарда спіткнулася на цій перешкоді.
Вульґарна економія, яка «справді таки нічого не навчилась»
тут, як і всюди, чваниться позірністю явища, заперечуючи закон
явища. Вона думає, всупереч Спінозі, що «неуцтво є достатня
підстава».

Працю, яку день-у-день пускає в рух сукупний капітал
якогось суспільства, можна розглядати як одним-один робочий
день. Коли, приміром, число робітників — 1 мільйон, а пересічний
робочий день одного робітника становить 10 годин, то суспільний
робочий день складається з 10 мільйонів годин. За даної
довжини цього робочого дня, — хоч межі його визначаються фізичними,
хоч соціяльними умовами, — маса додаткової вартости може
бути збільшена лише через збільшення числа робітників, тобто
робітничої людности. Зріст людности становить тут математичну
межу для продукції додаткової вартости сукупним суспільним
капіталом. Навпаки, за даної величини людности цю межу становить
можливе здовження робочого дня\footnote{
«Праця, що є вжитий на господарювання час суспільства, являє
собою величину дану, приміром, 10 годин денно на кожного з мільйона
людности, або 10 мільйонів годин\dots{} Капітал має свої межі зростання.
Кожного даного періоду цих меж можна досягти, застосовуючи ввесь
час, який є в розпорядженні для господарювання». («The labour, that
is the economic time of society, is a given portion, say ten hours a day of
a million of people or ten millions hours\dots{} Capital has its boundary of
increase. The boundary may, at any given period, be attained in the actual
extent of economic time employed»). («An Essay on the Political Economy
of Nations», London 1821, p. 47, 49).
}. В найближчому розділі
\index{i}{0245}  %% посилання на сторінку оригінального видання
побачимо, що цей закон має силу лише щодо розглянутої досі
форми додаткової вартости.

З попереднього розгляду продукції додаткової вартости випливає,
що не кожну довільну суму грошей або вартости можна
перетворити на капітал; навпаки, передумовою такого перетворення
є певний мінімум грошей, або мінової вартости, в руках
поодинокого посідача грошей або товарів. Мінімум змінного капіталу
є ціна витрат (Kostenpreis)\footnote*{
У французькому виданні тут замість слова «Kostenpreis» маємо
«prix moyen», що значить: «пересічна ціна». \emph{Ред.}
} на одну робочу силу, яку протягом
цілого року день-у-день споживають, щоб здобути додаткову
вартість. Коли б у цього робітника були свої власні засоби
продукції та коли б він задовольнявся життям робітника,
то йому досить було б робочого часу, доконечного для репродукції
його засобів існування, приміром, 8 годин денно. Отже, і
засобів продукції він потребував би лише на 8 робочих годин.
Навпаки, капіталіст, який, крім цих 8 годин, примушує його
працювати ще, приміром, 4 години додаткової праці, потребує
додаткової грошової суми на придбання додаткових засобів продукції.
Однак за нашого припущення він мусів би вже вживати
двох робітників для того, щоб на додаткову вартість, яку він
щодня присвоює, жити як робітник, тобто мати змогу задовольняти
свої доконечні потреби. В цьому випадку метою його продукції
було б тільки підтримання життя, а не збільшення багатства,
а саме останнє й має на меті капіталістична продукція.
Щоб жити лише удвоє краще від звичайного робітника й половину
спродукованої додаткової вартости знову перетворювати
на капітал, він мусів би разом з числом робітників увосьмеро
збільшити мінімум авансованого капіталу. Певна річ, він сам,
подібно до того як його робітник, може безпосередньо прикладати
свої руки до процесу продукції, але тоді він є лише щось середнє
між капіталістом і робітником — «дрібний майстер». На певному
рівні розвитку капіталістичної продукції потрібно, щоб капіталіст
цілий час, протягом якого він функціонує як капіталіст,
тобто як персоніфікований капітал, міг уживати на присвоєння,
а тому й на контроль чужої праці, і на продаж продуктів цієї
праці\footnote{
«Фармер не може покладатись на свою власну працю, а коли
він це робить, то, на мою думку, він од того втрачає. Його функція — це
доглядати за всім: він мусить стежити за молотником, а то плата за останнім
пропаде, а хліб не вимолотиться; він мусить доглядати за своїми
косарями, женцями й~\abbr{т. ін.}: мусить завжди обходити своє господарство;
мусить додивлятися, щоб не було жодного недбальства, алеж воно неминуче,
якщо він буде прикутий до якогось одного місця». («The farmer
cannot rely on his own labour; and if he does, I will maintain that he is
a loser by it. His employement should be, a general attention to the whole:
his thrasher must be watched, or he will soon lose his wages in corn not
thrashed out; his mowers, reapers etc. must be looked after; he must constantly
go round his fences: he must see there is no neglect; which would
be the case if he was confined to any one spont»). («An Enquiry into Connection
between the Price o Provisions, and the Size of Farms etc. By a
Farmer», London 1773, p. 12). Цей твір дуже цікавий. На ньому можна
вивчати генезу «фармера-капіталіста» або «merchant farmer»\footnote*{
— фармера-купця. \emph{Ред.}
}, як виразно
його тут названо, і почути, як той фармер пишається перед «small
farmer»\footnote*{
— дрібним фармером. \emph{Ред.}
}, для якого справа по суті сходить на засоби існування. «Кляса
капіталістів, спочатку частинно і кінець-кінцем цілком визволяється
від доконечности ручної праці». (\emph{Richard Jones}: «Textbook of Lectures on
the Political Economy of Nation», Hertford 1852, Lecture III, p. 39).
}. Середньовічні цехи намагалися силоміць перешкодити
\index{i}{0246}  %% посилання на сторінку оригінального видання
ремісникові-майстрові перетворюватись на капіталіста, обмежуючи
дуже незначним максимумом число робітників, яких дозволяли
вживати поодинокому майстрові. Посідач грошей або товарів
тільки тоді дійсно перетворюється на капіталіста, коли авансована
на продукцію мінімальна сума далеко перевищує середньовічний
максимум. Тут, як і в природознавстві, потверджується
правдивість закону, відкритого Геґелем в його «Логіці»,
що прості кількісні зміни на певному пункті перетворюються
на якісні ріжниці\footnoteA{
Молекулярна теорія, застосована в сучасній хемії, вперше науково
розвинута в Льорана й Жерара, спирається саме на цей закон. [Додаток
до 3 видання]. — Для пояснення цієї примітки, досить темної для осіб,
незнайомих із хемією, ми зауважимо, що автор говорить тут про вуглеводневі
сполуки, названі Жераром в 1843~\abbr{р.} спочатку «гомологічними
рядами», що з них кожний має власну альґебричну формулу сполуки. Наприклад,
ряд парафіни: Сn, Н2n + 2; ряд нормальних алькоголів:
Сn, Н2n + 2, O; ряд нормальних жирних кислот; Сn, Н2n, O2 і багато інших.
У вищенаведених прикладах за допомогою простого кількісного додавання
СН2 до молекулярної формули кожного разу твориться якісно відмінне
тіло. Щождо Марксової переоцінки участи Льорана й Жерара в установленні
цього важного факту порівн. \emph{Kopp}: «Entwicklung der Chemie»,
München 1873, S. 709 і 716, та \emph{Schorlemmer}: «Rise and Progress of
Organic Chemistry», London 1879, p. 54. — \emph{Ф.~E.}
}.

Та мінімальна сума вартости, якою мусить порядкувати поодинокий
посідач грошей або товарів, щоб перетворитися на капіталіста,
змінюється на різних ступенях розвитку капіталістичної
продукції і, за даного ступеня розвитку, вона в різних сферах
продукції різна залежно від їхніх осібних технічних умов. Певні
галузі промисловости вже на початках капіталістичної продукції
потребують такого мінімуму капіталу, якого ще немає в руках
поодиноких індивідів. Це призводить почасти до державних субсидій
для приватних осіб, як от у Франції за часів Кольбера, а в
деяких німецьких державах геть аж і донині, а почасти до утворення
товариств із законною монополією на ведення деяких
галузей промисловости й торговлі\footnote{
Такі установи Мартин Лютер називає «Die Gesellschaft Monopolia»,
тобто монопольними товариствами.
} — предтеч сучасних акційних товариств.
