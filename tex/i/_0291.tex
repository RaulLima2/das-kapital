\parcont{}  %% абзац починається на попередній сторінці
\index{i}{0291}  %% посилання на сторінку оригінального видання
незмінному поділі праці, який підчас закладання нових громад
дає готовий плян та схему продукції. Кожна така громада становить
продукційну цілість, що сама себе задовольняє, а її поле
продукції коливається між 100 й кількома тисячами акрів. Головну
масу продукту продукується для безпосередніх власних
потреб громади, а не як товар, і тому сама продукція не залежить
від того поділу праці в цілому індійському суспільстві, що упосереднюється
обміном товарів. Тільки лишок продуктів перетворюється
на товар, знову ж таки почасти тільки в руках держави,
до якої від непам’ятних часів припливає певна кількість продукту
як натуральна рента. Різні частини Індії мають різні
форми громад. У своїй найпростішій формі громада обробляє
землю спільно та поділяє свої продукти поміж членів громади,
тимчасом як кожна родина займається прядінням, тканням і т. ін.
як домашнім побічним промислом. Поряд цієї маси, що робить
однорідну роботу, находимо ми: «голову громади» (Haupteinwohner),
що сполучає в своїй одній особі суддю, поліцая і збирача
податків; бухгалтера, що веде рахунки рільництва та кадаструє
й реєструє все, що сюди належить; третього урядовця, що переслідує
злочинців, захищає чужинців-мандрівників та проводить
їх від села до села; прикордонного вартового, що охороняє межі
громади від сусідніх громад; доглядача води, який розподіляє
для потреб рільництва воду з громадських резервуарів; брагмана,
що виконує функції релігійного культу; вчителя, який учить
дітей громади писати й читати на піску; брагмана-календарника,
що як астролог подає відомості про час засіву, жнив та про годину
чи негоду для всіх окремих рільничих праць; одного коваля й
одного тесляра, що виготовляють та лагодять усе сільськогосподарське
знаряддя; ганчаря, що виробляє ввесь посуд для
села; голяра; прача, що чистить одяг; срібляра; подекуди поета,
який по деяких громадах заступає срібляра, а по інших шкільного
вчителя. Ця дванадцятка осіб утримується коштом цілої
громади. Якщо людність зростає, то на необробленій землі закладається
нову громаду на зразок старої. Механізм громади виявляє
пляномірний поділ праці, але мануфактурний поділ праці в ній
неможливий, бо ринок для коваля, тесляра й т. д. лишається незмінний,
і щонайбільше, залежно від величини селищ, буває
замість одного два або три ковалі, ганчарі тощо.\footnote{
Lieut. Col. \emph{Mark Wilks}: «Historical Sketches of the South
of India», London 1810--17, vol. I, p. 118--120. Добре порівняння
різних форм індійських громад можна найти в \emph{George Campbell’а}:
«Modern India», London 1852.
} Закон, що
реґулює поділ праці громади, діє тут а незламною силою (Autorität)
закону природи, тимчасом як кожний окремий ремісник,
як от коваль тощо, виконує всі належні до його фаху операції
традиційним способом, але самостійно і не визнаючи жодного
авторитету в своїй майстерні. Простий продуктивний механізм
цих громад, що сами себе задовольняють, що постійно репродукують
себе в тій самій формі та, бувши випадково зруйновані,
\parbreak{}  %% абзац продовжується на наступній сторінці
