ває з конкуренції машин, зміни в якості вживаних робітників,
частинних та загальних криз.

При зростанні поденної або потижневої заробітної плати ціна
праці може номінально лишитися сталою і все ж впасти нижче
від свого нормального рівня. Це буває щоразу тоді, коли за сталої
ціни праці, зглядно робочої години, робочий день здовжується
поза звичайний час його тривання. Коли у дробу
денна вартість робочої сили/робочий день зростає
знаменник, то ще швидше зростає чисельник. Вартість робочої сили, в наслідок зростання її
зужитковування, зростає разом із триванням
її функціонування, та ще й у швидшій пропорції, ніж приріст тривання її
функціонування. Тому в багатьох галузях продукції, де панує
почасова заробітна плата без законодавчих обмежень робочого
часу, сама собою (naturwüchsig) розвинулася звичка вважати
робочий день за нормальний лише до певного пункту, наприклад,
до скінчення десятої години («normal working day», «the day’s
work», «the regular hours of work»).* Поза цією межею робочий
час становить наднормовий час (overtime), і за той час, беручи
за одиницю міри годину, платять більше (extra рау), хоч часто
в пропорції до смішного малій.35 Нормальний робочий день
існує тут як дріб дійсного робочого дня, і цей останній часто протягом
цілого року є довший від нормального.36 Зростання ціни
праці із здовженням робочого дня поза певну нормальну межу
в різних галузях брітанської промисловости призводить до того
результату, що низька ціна праці в так званий нормальний час змушує
робітника працювати наднормовий час, за який краще платять,
якщо він взагалі хоче одержати достатню заробітну плату.37

36 «Норма плати за наднормовий час (у виробництві мережива) така
мала, півпенні тощо за годину, що вона стоїть у гострому контрасті до
того масового лиха, якого зазнає від неї здоров’я та життєва сила робітників...
Крім того, здобутий таким чином невеликий надлишок до заробітної
плати робітникові доводиться часто витрачати на різні засоби,
що підживляють його сили». («Children’s Employment Commission. 2nd
Report», p. XVI, n. 117).

36    Наприклад, y виробництві шпалер перед недавнім заведенням
фабричного закону. «Ми працюємо без перерв на їжу, так що 10 У2-годинний
робочий день кінчається близько пів на п’яту по півдні, а все
дальше є наднормовий час, що рідко закінчується перед восьмою годиною
вечора, і ми в дійсності протягом цілого року працюємо наднормовий
час». (Мг. Smith’s Evidence у «Children’s Employment Commission.
1 st Report», p. 125).

37    Наприклад, y шотландських білильнях. «У деяких частинах
Шотландії цю промисловість провадилося (перед заведенням фабричного
закону 1862 р.) за системою наднормового часу, тобто 10 годин вважали
за нормальний робочий день. Заце робітник діставав Ішилінґ 2 пенси.
Але до цього треба ще додати щодня наднормовий час у 3—4 години,
за що платили по 3 пенси за годину. Наслідки цієї системи: робітник,
що працював тільки нормальний час, міг заробити лише 8 шилінґів на
тиждень. Без наднормового часу цієї заробітної плати не вистачало».
(«Reports of Insp. of Fact. for 30 th April 1863», p. 10). «Підвищена плата

* — «нормальний робочий день», «денна праця», «регулярний робочий
час». Ред.
