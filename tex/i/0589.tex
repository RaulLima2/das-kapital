ної кількости й де ніхто не сміє жити, крім чабанів, садівників
і сторожів дичини, — цих постійних слуг, з якими шановне панство
поводиться з ласкою, звичайною для цієї кляси слуг.\footnote{
Таке показне село має дуже принадний вигляд, але воно таке
саме нереальне, як і села, що їх бачила Катерина II підчас своєї подорожі
до Криму. Останніми часами навіть і чабана часто виганяють з цих showvillages.
Напр., біля Market Harborough є пасовисько для овець, що
займає майже 500 акрів, де потрібна праця лише однієї людини. Для скорочення
далеких переходів цими просторими рівнинами, гарними пасовиськами
Leicester’a й Northampton’a, чабанові звичайно давали котедж
на фармі. Тепер йому дають тринадцятий шилінґ на помешкання, що його
він мусить собі шукати далеко у відкритому селі.
}
дле земля потребує оброблення, і ми бачимо, що робітники,
заняті на ній, живуть не в земельного власника, а приходять з
відкритого села, що лежить, може, на віддалі трьох миль, де
їх пустили до себе численні дрібні домовласники після того, як
по закритих селах зруйнували котеджі робітників. Там, де справи
наближаються до цього результату, котеджі своїм злиденним
виглядом звичайно свідчать про долю, що на неї вони засуджені.
Вони перебувають на різних щаблях природної руїни. Поки
дах тримається, робітникові дозволяють платити за помешкання
ренту, і він часто дуже радий, що йому це дозволяють, хоча б
йому доводилося платити таку ціну, як і за добре помешкання.
Але жодного ремонту, жодних полагоджень, крім таких, що їх
зможе зробити сам пребідний мешканець. Коли ж котедж стає,
нарешті, цілком непридатний для житла, то це означає лише,
що число зруйнованих котеджів збільшилось на один і остільки
менше доведеться надалі платити податку на бідних. Тимчасом
як великі землевласники таким способом звалюють із себе податок
на користь бідним через вилюднення належної їм землі, найближче
містечко або відкрите село приймає до себе викинутих
робітників; я кажу найближче, але оце «найближче» може лежати
три-чотири милі від фарми, де робітник має день-удень
тяжко працювати. Таким чином, до його денної праці, так, наче
це була б дурничка, долучається потреба день-у-день проходити
7—8 миль, щоб заробити собі на щоденний хліб. Всі сільські
роботи, що їх виконують його дружина й діти, відбуваються
тепер серед таких самих важких обставин. Але й це ще не все
лихо, що його спричиняє йому віддаленість від місця роботи.
В одкритому селі будівельні спекулянти скуповують шматки землі
і якомога густіше забудовують їх найдешевшими халупами
всякого роду. І в таких злиденних житлах, які, навіть тоді,
коли вони виходять на чисте поле, мають найжахливіші характеристичні
риси найпоганіших міських жител, туляться рільничі
робітники Англії...\footnote{
«Доми робітників (у відкритих селах, які, звісно, завжди переповнені)
звичайно збудовані рядами, задніми стінами до крайньої лінії
того шматка землі, що його будівельний спекулянт називає своїм. Через
те світло й повітря може проходити до них лише з фасаду». (Звіт д-ра
Гентера в «Public Health. Seventh Report 1864», London 1865, p. 135).
Дуже часто власник пивниці або сільський крамар здає також у найми
} З другого боку, не треба собі уявляти,