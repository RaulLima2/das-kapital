\parcont{}  %% абзац починається на попередній сторінці
\index{i}{0345}  %% посилання на сторінку оригінального видання
таку інтенсивність праці, що руйнує здоров’я робітників, отже,
руйнує саму робочу силу. «В більшості бавовняних фабрик, фабрик
суканої вовни та шовкових фабрик той запал, що виснажує
робітника і є потрібний для праці коло машин, рух яких останніми
роками так надзвичайно прискорено, є, здається, однією
з причин тієї надмірної смертности від недуг на легені, яку виявив
д-р Ґрінхов у своєму найновішому вартому уваги звіті»\footnote{
«Reports of Insp. of Fact, for 31 st October 1861», p. 25, 26.
}.
Немає найменшого сумніву, що тенденція капіталу, скоро тільки
закон раз назавжди покладе край здовжуванню робочого дня,
а саме тенденція відшкодовувати себе систематичним підвищенням
ступеня інтенсивности праці та перетворенням усякого поліпшення
машин на засіб до більшого висисання робочої сили, мусить
незабаром знову привести до того поворотного пункту, де знов
стає неминучим скорочення робочого дня\footnote{
Тепер (1867 p.) у Ланкашірі почалася серед фабричних робітників
аґітація за восьмигодинний робочий день.
}. З другого боку,
швидкий поступ англійської промисловости за час від 1848~\abbr{р.} до
наших часів, тобто за періоду десятигодинного робочого дня, ще
дужче перевищує розвиток її за час від 1838 до 1847~\abbr{р.}, тобто за
періоду дванадцятигодинного робочого дня, ніж цей останній
перевищує розвиток промисловости протягом півстоліття від часу
заведення фабричної системи, тобто за періоду необмеженого
робочого дня\footnote{
Декілька нижченаведених чисел показують поступ власне «фабрик»
в Об’єднаному Королівстві, починаючи від 1848~\abbr{р.}:

\setlength{\tabcolsep}{3pt}
\begin{footnotesize}
\noindent\begin{tabularx}{\textwidth}{@{}Xrrrr@{}}
  \addlinespace
  \toprule
  & \multicolumn{4}{c}{\emph{Розмір експорту по роках}} \\
  \cmidrule{2-5}
  & 1848 & 1851 & 1860 & 1865\\
  \midrule

  \addlinespace
  \emph{Бавовняні фабрики} \\
  Бавовняна пряжа\dotfill{} & \num{135.831.162} \samewidth{ярд.}{фун.} & \num{143.966.106} \samewidth{ярд.}{фун.} & \num{197.343.655} \samewidth{ярд.}{фун.} & \num{103.751.455} \samewidth{ярд.}{фун.} \\
  Нитки до шиття\dotfill{} & \textemdash & \num{4.392.176} \samewidth{ярд.}{фун.} & \num{6.287.554} \samewidth{ярд.}{фун.} & \num{4.648.611} \samewidth{ярд.}{фун.} \\
  Бавовняні тканини\dotfill{} & \num{1.091.373.930} ярд. & \num{1.543.161.789} ярд.
     & \num{2.776.218.427} ярд.   & \num{2.015.237.851} ярд. \\

  \addlinespace
  \makecell[l]{\emph{Льнопрядні та}\\\emph{коноплепрядні фабрики}} \\
  Пряжа\dotfill{} & \num{11.722.182} \samewidth{ярд.}{фун.} & \num{18.841.326} \samewidth{ярд.}{фун.} &  \num{31.210.612} \samewidth{ярд.}{фун.}  &  \num{36.777.334} \samewidth{ярд.}{фун.} \\
  Тканини\dotfill{} &  \num{88.901.519} ярд. &   \num{129.106.753} ярд.
    &   \num{143.996.773} ярд.  &  \num{247.012.329} ярд. \\

  \addlinespace
  \emph{Шовкові фабрики} \\
  Пряжа й нитки\dotfill{} &  \num{194.815} \samewidth{ярд.}{фун.} &   \num{462.513} \samewidth{ярд.}{фун.}  &    \num{897.402} \samewidth{ярд.}{фун.}  & \num{812.589} \samewidth{ярд.}{фун.} \\
  Тканини\dotfill{}       & \textemdash & \num{1.181.455} ярд. & \num{1.307.293} ярд.
     & \num{2.869.837} ярд. \\

  \addlinespace
  \emph{Вовняні фабрики} \\
  Пряжа\dotfill{}   & \textemdash & \num{14.670.880} \samewidth{ярд.}{фун.}  &  \num{27.533.968} \samewidth{ярд.}{фун.} &\num{31.669.267} \samewidth{ярд.}{фун.} \\
  Тканини\dotfill{} & \textemdash & \num{151.231.153} ярд.  & \num{190.371.537} ярд. &\num{278.837.418} ярд. \\
\end{tabularx}
\end{footnotesize}

\setlength{\tabcolsep}{\tabcolsepdef}
\begin{footnotesize}
\centering
\noindent\begin{tabularx}{\textwidth}{@{}Xrrrr@{}}
  \addlinespace
  \toprule
  & \multicolumn{4}{c}{\emph{Вартість експорту по роках,\pound{ (фун. стерл.)}} } \\
  \cmidrule{2-5}
  & 1848 & 1851 &
    1860 & 1865 \\
  \midrule

  \addlinespace
  \emph{Бавовняні фабрики} \\
  Бавовняна пряжа\dotfill{} & \num{5.927.831} &  \num{6.634.026} & \num{9.870.875} & \num{10.351.049} \\
  Бавовняні тканини\dotfill{} & \num{16.753.369}  & \num{23.454.810}  & \num{42.141.505} & \num{46.903.796}\\

  \addlinespace
  \emph{Льнопрядні та коноплепрядні фабрики} \\
  Пряжа\dotfill{} & \num{493.449}  & \num{951.426} & \num{1.801.272} & \num{2.505.497} \\
  Тканини\dotfill{} & \num{2.802.789}  &  \num{4.107.396} & \num{4.804.803} & \num{9.155.358} \\

  \addlinespace
  \emph{Шовкові фабрики} \\
  Пряжа й нитки\dotfill{} &  \num{77.789} &  \num{196.380} & \num{826.107} & \num{768.064} \\
  Тканини\dotfill{}       & \num{510.328} & \num{1.130.398} & \num{1.587.303} & \num{1.409.221} \\

  \addlinespace
  \emph{Вовняні фабрики} \\
  Пряжа\dotfill{}   &  \num{776.975} & \num{1.484.544} & \num{3.843.450} & \num{5.424.047} \\
  Тканини\dotfill{} & \num{5.733.829} & \num{8.377.183} & \num{12.156.998} & \num{20.102.259} \\

\end{tabularx}
\end{footnotesize}


\noindent{}(Див. Сині Книги: «Statistical Abstract for the United Kingdom»,
№ 8 і № 13, London 1861 і 1866).

\noindent{}В Ланкашірі число фабрик збільшилося між 1839 і 1850~\abbr{рр.} лише на
4\%, між 1850 і 1856 — на 19\%, між 1856 і 1862 — на 33\%, тимчасом
як число занятих осіб за обидва одинадцятилітні періоди абсолютно підвищилось,
відносно ж зменшилось. Див. «Reports of Insp. of Fact. for
31 st October 1862», p. 63. В Ланкашірі мають перевагу бавовняні фабрики.
А яке відносно велике місце мають вони взагалі у фабрикації пряжі й
тканини, це видно з того, що з загального числа всіх фабрик такого роду
в Англії, Велзі, Шотляндії та Ірляндії на них самих припадає 45,2\%, а із
загального числа веретен — 83,3\%, із загального числа парових ткацьких
варстатів — 81,4\%, із загального числа парових кінських сил, що дають
рух тим фабрикам, — 72,6\%, а із загального числа занятих осіб — 58,2\%.
(Там же, стор. 62, 63).
}.

\index{i}{0346}  %% посилання на сторінку оригінального видання
\subsection{Фабрика}

На початку цього розділу ми розглядали тіло фабрики, тобто
організовану систему машин. Далі ми бачили, як машини, присвоюючи
собі жіночу й дитячу працю, збільшують людський
матеріял, що його експлуатує капітал, як вони, безмірно здовжуючи
робочий день, забирають увесь час життя робітника та,
нарешті, як їхній поступ, що дозволяє продукувати велетенські,
чимраз більші маси продукту в щораз коротший час, служить
систематичним засобом, щоб кожного моменту пускати в рух
більше праці, тобто раз-у-раз інтенсивніше визискувати робочу
силу. Тепер ми перейдемо до фабрики як до цілости, і саме в її
найрозвиненішій формі.

Д-р Юр, Піндар автоматичної фабрики описує її, з одного
боку, як «кооперацію різних кляс робітників, дорослих і недорослих,
які з вправністю і пильністю доглядають систему продуктивних
\index{i}{0347}  %% посилання на сторінку оригінального видання
машин, що її безнастанно пускає в рух центральна
сила (перший мотор)», а з другого боку, «як велетенський автомат,
складений з безлічі механічних та самосвідомих органів,
які оперують у згоді й без перерви, щоб продукувати той самий
предмет, так що всі ці органи підпорядковані одній рушійній
силі, яка сама собою рухається». Ці обидва визначення зовсім
не є ідентичні. В першому комбінований колективний робітник
або суспільне робоче тіло з’являється як домінантний суб’єкт,
а механічний автомат — як об’єкт; у другому сам автомат є
суб’єкт, а робітники як свідомі органи лише додані до його
несвідомих органів і разом з цими останніми підпорядковані
центральній рушійній силі. Перше визначення можна прикласти
до кожного можливого вживання машин у великому маштабі,
друге характеризує капіталістичне вживання машин, а тому й
сучасну фабричну систему. Тим то Юр любить також змальовувати
центральну машину, що від неї виходить рух, не тільки як
автомата, але і як автократа. «По цих величезних майстернях
благодатна сила пари збирає довкола себе міріяди своїх підданців»\footnote{
«\emph{Ure}: «Philosophy of Manufacture», р. 18.
}.

Разом із робочим знаряддям і віртуозність керувати ним
переходить від робітника до машини. Видатність знаряддя звільняється
від особистих рамок людської робочої сили. Тим самим
усунуто (aufgehoben) ту технічну основу, на якій ґрунтується
поділ праці в мануфактурі. Тому замість тієї ієрархії спеціялізованих
робітників, що характеризує мануфактуру, на автоматичній
фабриці виступає тенденція зрівняти, або знівелювати ті
праці, які мають виконувати помічники машин\footnote{
Там же, стор. 20. Порівн. \emph{K. Marx}: «Misère de la Philosophie»,
Paris 1847. p. 140, 141. (\emph{K. Маркс}: «Злиденність філософії», Партвидав
«Пролетар», 1932~\abbr{р.}, стор. 126, 127).
}, замість штучно
витворених ріжниць між частинними робітниками виступають
переважно природні ріжниці віку та статі.

Оскільки поділ праці відроджується на автоматичній фабриці,
він є насамперед розподіл робітників між різними спеціялізованими
машинами та розподіл мас робітників, які однак не являють
собою організованих груп, між різними відділами фабрики, де
вони працюють коло однорідних виконавчих машин, що стоять
рядом одна побіч однієї; отжеж, серед них існує тільки проста
кооперація. Розчленовану групу мануфактури тут замінено
зв’язком головного робітника з небагатьма помічниками. Посутня
ріжниця є поміж робітниками, які справді працюють коло виконавчих
машин (сюди належать декілька робітників, що доглядають
рушійної машини та опалюють її), і простими підручними
(майже виключно діти) цих машинових робітників. До підручних
у більшій або меншій мірі залічується всіх «feeders» (які
лише подають машинам матеріял праці). Побіч цих головних кляс
виступає чисельно незначний персонал, що доглядає всіх машин
\parbreak{}  %% абзац продовжується на наступній сторінці
