гнані водою, лише насилу перемагали в Англії народній опір,
що його підтримував парлямент. Коли 1758 р. Іврі збудував
першу машину стригти овець, гнану водою, її спалили кілька сот
людей, які опинилися без праці. Проти scribbling mills * та чухральних
машин Аркрайта 50.000 робітників, які досі жили з
чухрання вовни, звернулися з петицією до парляменту. Масове
руйнування машин в англійських мануфактурних округах протягом
перших 15 років XIX віку, зумовлене вживанням парових
варстатів і відоме під назвою руху луддітів, дало антиякобінському
урядові Сідмавта, Castlereagh’a та інших привід до
якнайреакційніших насильницьких кроків. Треба часу й досвіду,
щоб робітник навчився відрізняти машину від капіталістичного
вживання її, а тому й переносити свої напади з самих матеріяльних
засобів продукції на суспільну форму експлуатації їх.195

Боротьба за заробітну плату в мануфактурі припускає наявність
мануфактури й зовсім не скерована проти її існування. Якщо
хто і боровся проти утворення мануфактур, то робили це не наймані
робітники, а цехові майстри й упривілейовані міста. Тому письменники
мануфактурного періоду розуміють поділ праці переважно
як засіб заміняти робітників у можливості, а не в дійсності витискувати
з мануфактур робітників.** Ця ріжниця сама собою

лоти серед брузументників примусили магістрат спочатку заборонити її;
генеральні штати своїми різними постановами з 1623, 1639 рр. і т. д.
мали обмежити її вживання; нарешті, її дозволено під певними умовами
постановою 15 грудня 1661 р. «У цьому місті, — каже Боксхерн («Inst.
Рої. 1663») про заведення стьожкової машини в Ляйдені, — якихось 20 років
тому винайдено ткацький варстат, на якому один робітник міг легше й
більше виробляти тканин, аніж багато робітників за той самий час без
варстату. Але це спричинилося до скарг та заколотів серед ткачів, і магістрат,
нарешті, заборонив уживати машини» («In hac urbe ante hos viginti
circiter annos instrumentum quidam invenerunt textorium, quo solus quis
plus panni et facilius conficere poterat, quam plures aequali tempore. Hinc
turbae ortae et querulae textorum, tandemque usus hujus instrumenti a
magistratu prohibitus est»). (Boxhorn: «Institutiones politicae», Leyden
1663). Ту саму машину 1676 p. заборонено в Кельні, тимчасом як введення
її в Англії в той самий час спричинилось до заколотів серед робітників.
Королівським ециктом з 19 лютого 1685 р. заборонено вживати її по всій
Німеччині. В Гамбурзі з наказу магістрату її прилюдно спалили. Карл VI
відновив 9 лютого 1719 р. едикт з 1685 р., а в саксонському курфюрстві
загальний вжиток її дозволено лише 1765 р. Ця машина, що наробила в
світі стільки шуму, була в дійсності попередницею прядільних і ткальних
машин, отже, і промислової революції XVIII віку. Вона робила цілком
недосвідченого у ткацтві підлітка здатним пускати в рух цілий
варстат з усіма його човниками, через саме лише совання рушієм туди
й назад; у своїй поліпшеній формі вона давала воднораз 40—50 сувоїв
тканини.

195 У старомодних мануфактурах ще й за наших часів повторюються
інколи грубі форми обурення робітників проти машин. Так, наприклад,
у виробництві напилків у Шеффілді 1865 р.

* — машин для першого, грубого чесання вовни. Ред.

** У французькому виданні це речення подано так: «Письменники
мануфактурного періоду в поділі праці вбачають можливий засіб поповнювати
недостачу в робітниках, а не витискувати з мануфактур робітників,
що вже працюють». Ред.
