земель, що зменшувала людність (depopulating inclosures), і
проти пасовищного господарства, що йшло слідом за цією узурпацією
і теж зменшувало людність (depopulating pasture)».
Актом Генріха VII, 1489, розд. 19, заборонено руйнувати всі
селянські хати, до яких належало принаймні 20 акрів землі.
Акт 25 Генріха VIII відновлює той самий закон. У ньому, між
іншим, сказано, що «багато фарм і великі череди, особливо
овець, скупчуються в небагатьох руках, і через те дуже зросли
земельні ренти й дуже занепало рільництво (tillage), церкви й
хати поруйновано, а величезні маси людей утратили змогу утримувати
себе й свої родини». Тому закон наказує відбудувати
занепалі селянські садиби, визначає відношення між полем та
пасовиськами й т. д. Один акт з року 1533 нарікає, що деякі
власники мають по 24.000 овець, і обмежує їхнє число двома тисячами.193
Нарікання народу й закони проти експропріяції дрібних
фармерів та селян, видавані протягом 150 років, починаючи від
часів Генріха VII, були однаково даремні. Таємницю їхньої
безрезультатности викриває нам Бекон, сам цього не знаючи.
«Акт Генріха VII, — каже він у своїх «Essays, civil and moral»,
Sect. 20, — був глибокий і гідний подиву: він створив рільничі
господарства й садиби певних нормальних розмірів, тобто забезпечив
за ними таку кількість землі, при якій вони могли народжувати
на світ підданих, досить багатих, не принижених рабською залежністю,
що мали змогу орудувати плугом як власники, а не
як наймити» («to keep the plough in the hand of the owners and
not hirelings».193а Але капіталістична система потребувала,
навпаки, стану рабської залежности народньої маси, перетворення
її на наймитів і перетворення її засобів праці на капітал.
Під час цього переходового періоду законодавство намагалося

193    У своїй «Утопії» Томас Мор оповідає про дивну країну, де
«вівці пожирають людей» («Utopia», transl. Robinson, ed. Arbor, London
1869, p. 41).

193а Бекон пояснює зв’язок між вільним заможним селянством і
доброю піхотою. «Для могутности й мужности королівства надзвичайно
важливо було мати фарми достатніх розмірів, щоб забезпечити діяльних
людей від злиднів і закріпити більшу частину земель королівства в посіданні
yeomanry, або людей середнього стану між шляхтою і бурлаками
(cottagers) та наймитами... Всі бо найвидатніші знавці військової справи
тієї думки... що головна сила армії в інфантерії, або піхоті. Але, щоб
утворити добру піхоту, треба мати людей, які виросли не в рабському
пониженні та злиднях, а на волі і в стані певної заможности. Тим-то,
коли в державі головну ролю відіграють шляхта й вище панство, тим часом
як рільники й плугатарі — це тільки поденники й наймити, або
бурлаки, тобто жебраки, що мають хату, то тоді, може, і можна мати
добру кінноту, але ніколи не можна мати доброї витривалої піхоти...
Ми бачимо це у Франції та Італії й по деяких інших чужих країнах, де
ціла людність справді складається із шляхти й бідних селян... до такої
міри, що ці країни змушені користуватися для своїх батальйонів піхоти
найманими ватагами швайцарців тощо; звідси той факт, що ці нації
мають багато людности, але мало солдатів». («The Reign of Henry VІІ
etc. Verbatim Reprint from Kennet’s England, ed. 1719», London 1870,
p. 308).
