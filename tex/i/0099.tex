країни до іншої, і коли це перенесення в товаровій формі неможливе
або в наслідок коньюнктур товарового ринку, або через
саму мету, яку воно має досягти. 110

Кожна країна потребує певного резервного фонду для циркуляції
на світовому ринку, так само, як для своєї внутрішньої
циркуляції. Отже, функції скарбів виникають почасти з функції
грошей як внутрішнього засобу циркуляції й платежу, почасти
з їхньої функції як світових грошей.\footnoteA{
Примітка до другого видання. «Справді, на мою думку, ледве чи
можна бажати яснішого доказу здатности механізму резервних фондів
у країнах а металевою циркуляцією покривати всі доконечні міжнародні
платежі без якоїсь помітної підтримки з боку загальної циркуляції, ніж
та легкість, з якою Франція, ледве очунявши від руйнаційного ворожого
нападу, виплатила протягом 27 місяців приблизно 20 мільйонів фунтів
стерлінґів контрибуції, що наклали на неї союзні держави; до того значну
частину цієї контрибуції вона виплатила дзвінкою монетою без жодного
помітного скорочення або порушення свого внутрішнього обігу, навіть
без будь-яких занепокійливих коливань її векселевого курсу». (I would
desire, indeed, no more convincing evidence of the competency of the machinery
of the hoards in specie-paying countries to perform every necessary
office of international adjustment, without any sensible aid from the general
circulation, than the facility with which France, when but just recovering
from the shock of a destructive foreign invasion, completed within the
space of 27 month the payment of her forced contribution of nearly 20 millions
to the allied powers, and a considerable proportion of that sum in
specie, without perceptible contraction of derangement of her domestic currency,
or even any alarming fluctuation of her exchange»). (Fullarton:
«Regulation of Currencies». 2 nd ed. London 1845, p. 191). [До четвертого
видання. — Ще разючіший приклад маємо в тій легкості, з якою та сама
Франція в 1871—73 рр. мала силу заплатити протягом 30 місяців більш
ніж десятикратне воєнне відшкодування так само значною мірою в металевих
грошах. — Ф. Е.].
} Для цієї останньої
ролі завжди потрібен дійсний грошовий товар — тіло золота
й срібла, через що Джемс Стюарт виразно характеризує золото
й срібло, на відміну від їхніх лише місцевих заступників, як
money of the world.\footnote*{
— світові гроші. Ред.
}

Рух потоку золота й срібла є подвійний. З одного боку, він
шириться від своїх джерел через цілий світовий ринок, де його
в різному обсягу вловлюють різні національні сфери циркуляції,
щоб він увіходив у їхні внутрішні канали циркуляції, заміняв
стерті золоті й срібні монети, постачав матеріял для люксусових
товарів і застигав у формі скарбів.\footnote{
«Гроші поділяються поміж націями відповідно до їхньої потреби
на гроші... завжди притягувані продуктами» («L’argent se partage
entre les nations relativement au besoin qu’elles en ont... étant toujours
attiré par les productions»), (Le Trosne: «De l’Intérêt Social». Physiocrates,
éd. Daire, Paris. 1846, p. 916). «Копальні, що безперервно
} Цей перший рух упосереднюється
безпосереднім обміном національних праць, зреалізо-

стона (ех-банкіра Лойда), що його він називав «facile, princeps argentariorum.\footnote*{
— безперечним головою банкірів. Ред.
}

110    Наприклад, при субсидіях, грошових позиках на ведення війни
або на віднову банкових платежів готівкою тощо вартість може бути потрібна
саме у грошовій формі.