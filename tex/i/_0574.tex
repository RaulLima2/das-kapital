\parcont{}  %% абзац починається на попередній сторінці
\index{i}{0574}  %% посилання на сторінку оригінального видання
(«bound» — вислів, що, як і bondage, походить із часів епохи
кріпацтва), — на дванадцять місяців до орендаря («lessee») або
власника копалень. Коли ж вони виявляють своє незадоволення
або якимсь іншим способом дошкуляють доглядачеві («viewer»),
то він робить у своїй книжечці значок або помітку біля їхнього
ймення і звільняє їх при відновленні річного контракту\dots{} Мені
здається, що жодна з форм trucksystem’n не може бути гіршою,
ніж та, що панує в цих густо позаселюваних округах. Робітник
примушений одержувати як частину своєї заробітної плати дім
у заразливому оточенні. Він не може сам собі допомогти. Він
усіма сторонами — кріпак (he is to all intents and purposes a
serf). Ще питання, щоб хто міг допомогти йому, окрім його власника,
але цей власник радиться насамперед із своїм балянсом,
і результат цього майже безпомилковий. Власник постачає робітникові
також і воду. Хоч добра вона, хоч погана, хоч дістає
її робітник, хоч ні, він мусить за неї платити або, точніше, дозволити
вивертати за неї з заробітної плати».\footnote{
Там же, стор. 515, 517.
}

У конфлікті з «громадською думкою» або навіть із санітарною
поліцією капітал зовсім не соромиться «виправдувати»
ті почасти небезпечні, почасти ганебні умови, в які він заганяє
працю й домашнє життя робітника, тим, що це потрібно для того,
щоб вигідніше визискувати його. Так стоїть справа, коли він
«поздержується» від пристроїв для охорони від небезпечних машин
на фабриці, від вентиляційних і убезпечливих засобів на шахтах
і~\abbr{т. ін.} Так само тут стоїть справа і з житлами для гірників. «Як
виправдання ганебних житлових умов, — каже в своєму офіціальному
звіті д-р Сімон, лікарський урядовець Privy Council, —
наводять те, що копальні звичайно експлуатують, беручи їх в
оренду, що строк орендного контракту (в копальнях здебільша
21 рік) занадто короткий, щоб орендарям варто було влаштовувати
як слід помешкання для робітників, ремісників і~\abbr{т. ін.},
яких притягає до себе підприємство; коли б у нього навіть і був
намір бути щедрішим з цього боку, то в цьому йому став би на
перешкоді власник. А саме цей останній одразу почав би вимагати
надзвичайно високої додаткової ренти за привілей будувати
на його ґрунті порядне й пристойне селище для робітників,
які добувають його підземну власність. Ця заборонна ціна, якщо
не пряма заборона, залякує і тих, що за інших умов хотіли б
будувати селища\dots{} Я не хочу докладніше досліджувати гідність
цього виправдання, і так само не хочу досліджувати, на кого
остаточно спали б додаткові видатки за будову порядних помешкань
— на землевласника, на орендаря копалень, на робітника,
чи на суспільство\dots{} Але, зважаючи на такі ганебні факти, як
ось ті, що їх викривають долучені звіти [д-ра Гентера, Стівенса
й~\abbr{т. ін.}], треба вжити заходів, щоб усунути їх. Права
на земельну власність використовують так, що учиняють велику
суспільну несправедливість. Як власник копальні землевласник
\parbreak{}  %% абзац продовжується на наступній сторінці
