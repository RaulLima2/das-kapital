субстанції; на одному боці посідач засобів продукції і засобів
існування, на другому — посідач нічого іншого, а тільки робочої
еили. Отже, відокремлення продукту праці від самої праці, об’єктивних
умов праці від суб’єктивної робочої сили, було фактично
даною основою, вихідним пунктом капіталістичного процесу
продукції.

Але те, що спочатку було вихідним пунктом, те, в наслідок
простої безперервности процесу, в наслідок простої репродукції,
завжди знову продукується й увіковічнюється як власний результат
капіталістичної продукції. З одного боку, процес продукції
постійно перетворює матеріяльне багатство на капітал,
на засоби зростання вартости й засоби споживання для капіталіста.
З другого боку, робітник постійно виходить із цього процесу
таким, яким він у нього увійшов — як особисте джерело
багатства, але позбавлений усіх засобів до того, щоб здійснити
це багатство для себе. А що перед його вступом у цей процес
його власну працю в нього самого відчужено, присвоєно капіталістом
і долучено до капіталу, то ця праця протягом цього процесу
постійно упредметнюється в чужому продукті. А що процес
продукції є разом з тим і процес споживання робочої сили капіталістом,
то продукт робітника постійно перетворюється не
тільки на товар, а й на капітал, на вартість, що висисає вартостетворчу
силу, на засоби існування, що купують людей, на засоби
продукції, які порядкують продуцентом.\footnote{
«В цьому особливо дивовижна властивість продуктивного споживання.
Що споживається продуктивно, те є капітал, і воно стає капіталом
через споживання». (James Mill: «Eléments d’Economie Politique», 1823,
p. 242). Однак Дж. Мілл не натрапив на слід «цієї особливо дивовижної
властивости».
} Тому сам робітник
постійно продукує об’єктивне багатство як капітал, як чужу
йому силу, що панує над ним і його визискує, а капіталіст так
само постійно продукує робочу силу як суб’єктивне джерело
багатства, відокремлене від засобів його власного упредметнення
і здійснення, абстрактне джерело, що існує в простій тілесності
робітника, коротко — продукує робітника як найманого робітника.\footnote{
«Це дійсно правда, що кожна наново заведена мануфактура вживає
до праці багато бідних, але вони не перестають бути бідними; дальше
існування мануфактури створює багато нових бідняків» («It is true indeed
that the first introducing a manufacture employes many poor, but
they cease not to be so, and the continuance of it makes many»). («Reasons
for a limited Exportation of Wool», London 1677, p. 19). «Тепер фармер
запевняє цілком безглуздо, що він утримує бідних. В дійсності їх тримають
у злиднях». («The farmer now absurdly asserts, that he keeps the
poor. They are indeed kept in misery»). («Reasons for the late Increase
of the Poor Rates: or a comparative view of the prices of labour and provisions»,
London 1777, p. 37).
}
Ця постійна репродукція, або увіковічнення робітника
є sine qua non\footnote*{
— неодмінна умова. \emph{Ред.}
} капіталістичної продукції.

Споживання робітника є двоякого роду. В самій продукції
він споживає своєю працею засоби продукції й перетворює їх