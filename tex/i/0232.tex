Закон з 1850 р. перетворив лише для «підлітків і жінок»
п’ятнадцятигодинний період від пів на шосту ранку до пів на
дев’яту вечора на дванадцятигодинний період від шостої години
ранку й до шостої години вечора. Отже, не для дітей, яких усе
ще можна було експлуатувати 1/2 години перед початком і 2 1/2
години по скінченні цього періоду, хоч загальний час тривання
їхньої праці не повинен був перевищувати 6 1/2 годин. Під час обговорення
закону фабричні інспектори подали до парляменту статистичні
дані про ганебні зловживання, до яких приводила ця
аномалія. Але все це даремно. Потайний намір був у тому, щоб
за допомогою дітей у роки розцвіту знов догнати робочий день
дорослих до 15 годин. Досвід дальших трьох років довів, що
така спроба мусіла б розбитись об опір дорослих робітниківчоловіків.177
Тим то закон з 1850 р. й доповнено, нарешті, 1853 р.
забороною «вживати праці дітей ранком перед початком і ввечері
по скінченні праці підлітків і жінок». Починаючи з цього часу,
фабричний закон 1850 р. реґулював, за деякими винятками, в
підпорядкованих йому галузях промисловости робочий день усіх
робітників.178 Від часу оголошення першого фабричного закону
проминуло тепер півстоліття.179

Поза свою первісну сферу законодавство вийшло вперше через
«Printworks’ Act» (закон про перкалеві фабрики тощо), виданий
1845 р. Нехіть, з якою капітал допустив цю нову «екстраваґантність»,
промовляє з кожного рядка закону! Він обмежує
робочий день дітей 8—13 років і жінок 16 годинами — від шостої
години ранку до десятої години вечора, не призначаючи жодної
взаконеної перерви на їжу. Він дозволяє примушувати до праці
робітників-чоловіків старших від 13 років довільно цілий
день і цілу ніч.180 Це — парляментський викидень.181

177 «Reports etc. for 30 th April 1853», p. 31.

178    За часів найвищого розквіту англійської бавовняної промисловости,
в роках 1859—1860, деякі фабриканти приманою високої заробітної
плати за наднормовий час пробували підохотити дорослих прядунів до
здовження робочого дня. Прядуни на ручних варстатах і на сельфакторах
поклали кінець цій спробі, подавши меморіял своїм підприємцям, де, між
іншим, зазначено: «Сказати по правді, наше життя є тягар для нас, і
поки ми прикуті до фабрики майже на 2 дні (20 годин) у тижні
більше, ніж інші робітники, то ми почуваємо себе в країні гелотами й
сами собі докоряємо за те, що увіковічнюємо таку систему, яка фізично
й морально шкодить нам самим і нашим нащадкам... Тим то з повною
пошаною доводимо до вашого відома, що від першого дня нового року
не працюватимемо й хвилини довше понад 60 годин тижнево, від шостої
години до шостої години, відлічуючи законом призначені перерви на
1 1/2 години». («Reports etc. for 30 th April 1860», p. 30).

179 Про засоби порушувати цей закон, що їх дає редакція цього закону,
див. Parliamentary Return: «Factory Regulations Acts» (6 серпня 1859 р.) і
там само Leonhard Horner: «Suggestions for Amending the Factory Acts to
enable the Inspectors to prevent illegal working, now become very prevalent».

180 «За останнє півріччя (1857) у моїй окрузі дітей 8 років і старших
справді катують від 6 години ранку й аж до 9 години вечора». («Reports
etc. for 31 st October 1857», p. 39).

181 «Закон про перкалеві фабрики вважається за невдалий так шодо його
постанов про навчання, як і щодо його постанов про охорону праці» («The-
