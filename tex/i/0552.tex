є суперечність самого руху капіталу. Він потребує більших мас робітників молодшого віку, менших мас
— дорослого. Ця суперечність не більш кричуща за ту другу, що нарікають на брак робочих рук у той
самий час, коли багато тисяч викинуто на
брук через те, що поділ праці приковує їх до якоїсь певної галузі продукції.85 До того ж, капітал
споживає робочу силу так швидко, що здебільшого робітник середнього віку є вже більше або менше
виснажений. Він попадає в ряди зайвих або його витискують
із вищого щабля на нижчий щабель. Саме в робітників великої промисловости ми натрапляємо на
найкоротший протяг життя. «Д-р. Лі, санітарний урядовець Менчестеру, сконстатував, що в тому місті
пересічний протяг життя заможної кляси
38 років, а робітничої кляси — лише 17 років. У Ліверпулі він становить для першої кляси 35 років,
для другої — 15. Отже, з цього випливає, що упривілейована кляса має асиґнату на життя (have a lease
of life) понад удвоє більшу, ніж її менш щасливі співгромадяни».85а За цих обставин для абсолютного
зростання цієї частини пролетаріяту потрібна така форма, при якій чисельність її зростала б, не
зважаючи на швидке виснажування її елементів. Отже, потрібна швидка зміна поколінь робітників.
(Цей закон не має сили для решти кляс людности).
Цю суспільну потребу задовольняється ранніми шлюбами, — неминучий наслідок відносин, серед яких
живуть робітники великої промисловости, — і тією премією, яку дає експлуатація дітей робітників за
продукцію їх.

Скоро тільки капіталістична продукція опановує рільництво, або в міру того, як вона опановує
рільництво, попит на сільську робітничу людність абсолютно меншає з акумуляцією капіталу, що тут
функціонує, так що відштовхування робітничої людности
тут не доповнюється, як у нерільничій промисловості, більшим притяганням. Тому частина сільської
людности завжди готова перейти в ряди міського або мануфактурного пролетаріяту і лише вичікує
сприятливих умов для цього перетворення. (Слова мануфактура тут уживається в розумінні всякої
нерільничої
промисловости).86 Отже, це джерело відносного перелюднення

85    Тимчасом як протягом останнього півріччя 1866 р. в Лондоні лишилося без роботи 80.000—90.000
робітників, у фабричному звіті про це саме півріччя читаємо: «Здається, не зовсім правда, що попит
створює подання саме в ту хвилину, коли це потрібно. Щодо праці справа стояла інакше, бо протягом
останнього року багато машин не працювало через брак рук». («It does not appear absolutely true to
say that demand
will always produce supply just at the moment when it is needed. It has not done so with labour, for
much machinery had been idle last year for want of hands»). («Report of Insp. of Fact. for 31st
October 1866», p. 81).

85a Промова, що її виголосив на відкритті санітарної конференції в Бермінґемі 15 січня 1875 р. Дж.
Чемберлен, тодішній мер міста, теперішній (1883) міністер торговлі.

86 За переписом 1861 р. в Англії і Велзі налічувалося «781 місто з 10.960.998 жителями, тимчасом як
по селах і сільських парафіях налічувалося лише 9.105.226 жителів... У перепису 1851 р. фігурувало
580 міст, що їх людність приблизно дорівнювала людності прилеглих до них
