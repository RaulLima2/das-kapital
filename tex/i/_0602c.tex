\index{i}{0602}  %% посилання на сторінку оригінального видання
Людність Ірляндії зросла в 1841 р. до 8.222.664 осіб, в 1851 р.
вона зменшилась до 6.623.985, в 1861 р. до 5.850.309, в 1866 р. —
до 5\sfrac{1}{2} мільйонів, тобто приблизно до свого рівня 1801 р. Зменшення
почалося з 1846 голодного року, так що менше ніж за
20 років Ірляндія втратила більше як \sfrac{5}{16} своєї людности.\footnote{
Людність Ірляндії: в 1801 р. — 5.319.867 осіб, в 1811 р. — 6.084.996,
в 1821 р. — 6.869.544, в 1831 р. — 7.828.347, в 1841 р. — 8.222.664 особи.
}
Загальне число еміґрантів від травня 1851 р. до липня 1865 р.
становило 1.591.487 осіб, число еміґрантів за останні п’ять років,
від 1861 до 1865 р., становило більш ніж \sfrac{1}{2} мільйона осіб. Число
заселених будинків зменшилось від 1851 р. до 1861 р. на 52.990.
Від 1851 р. до 1861 р. число фарм розміром від 15 до 30 акрів зросло
на 61.000, число фарм понад 30 акрів — на 109.000, тимчасом
як загальне число всіх фарм зменшилось на 120.000, отже зменшення,
спричинене виключно знищенням фарм нижче 15 акрів,
тобто централізацією їх.

Зменшення людности, певна річ взагалі і в цілому супроводилось
зменшенням маси продуктів. Для нашої мети досить
розглянути п’ятиріччя 1861--1865, протягом якого еміґрувало
понад \sfrac{1}{2} мільйона і абсолютна кількість людности спала
більш ніж на \sfrac{1}{3} мільйона (див. таблицю А).

\begin{table}[h]
  \begin{flushright}
    \emph{Таблиця А}
  \end{flushright}
\caption*{Худоба}
  \newlength{\myheight}
  \hangindent=1em
  \setlength{\myheight}{10em}
  \newcolumntype{Y}{>{\centering\arraybackslash}X}
  \noindent\begin{tabularx}{\textwidth}{Y Y Y Y Y Y}
    \toprule
      Роки & \multicolumn{2}{c}{Коні} & \multicolumn{3}{c}{Рогата худоба}\\
    \cmidrule(rl){2-3}
    \cmidrule(l){4-6}
    &
    \mbox{Загальна} \mbox{кількість} &
    Змен\-шення &
    \mbox{Загальна} \mbox{кількість} &
    Змен\-шення &
    Збільшення
    \\
    \midrule
      1860 & 619.811 & — & 3.606.374 & — & — \\
      1861 & 614.232 & 5.993 & 3.471.688 & 138.316 & — \\
      1862 & 602.894 & 11.338 & 3.254.890 & 216.798 & — \\
      1863 & 579.978 & 22.916 & 3.144.231 & 110.695 & — \\
      1864 & 562.158 & 17.820 & 3.262.294 & — & 118.063 \\
      1865 & 547.867 & 14.291 & 4.493.414 & — & 231.120 \\
  \end{tabularx}
\end{table}
\begin{table}[h]

  \newlength{\myheight}
  \hangindent=1em
  \setlength{\myheight}{10em}
  \newcolumntype{Y}{>{\centering\arraybackslash}X}
  \noindent\begin{tabularx}{\textwidth}{Y Y Y Y Y Y Y}
  \toprule
    Роки & \multicolumn{3}{c}{Вівці} & \multicolumn{3}{c}{Свині}\\
  \cmidrule(rl){2-4}
  \cmidrule(l){5-7}
  &
  \mbox{Загальна} \mbox{кількість} &
  Змен\-шення &
  Збільшення &
  \mbox{Загальна} \mbox{кількість} &
  Змен\-шення &
  Збільшення
  \\
  \midrule
    1860 & 3.542.080 & — & — & 1.271.072 & — & — \\
    1861 & 3.556.050 & — & 13.970 & 1.102.042 & 169.030 & — \\
    1862 & 3.456.132 & 99.918 & — & 1.154.324 & — & 52.282 \\
    1863 & 3.308.204 & 147.982 & — & 1.067.458 & 86.866 & — \\
    1864 & 3.366.941 & — & 58.737 & 1.058.480 & 8.978 & — \\
    1865 & 3.688.742 & — & 321.801 & 1.299.893 & — & 241.413 \\
  \end{tabularx}
\end {table}
