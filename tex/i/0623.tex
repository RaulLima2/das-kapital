панським «полюбовницям». Навіть наймані сільські робітники
були ще тоді співпосідачами громадських земель. Десь близько
1750 р. yeomanry зникли,199 а останніми десятиліттями XVIII
століття зник і останній слід громадської власности рільників.
Ми залишаємо тут осторонь суто економічні пружини революції
у рільництві. Нас цікавлять її насильні підойми.*

За реставрації Стюартів землевласники законодатним шляхом
добилися тієї узурпації, яка повсюди на континенті відбулася й
без законодатних церемоній. Вони знищили февдальний земельний
лад, тобто скинули з себе всякі повинності щодо держави,
«відшкодували» державу податками, накинутими на селянство
й решту народньої маси, присвоїли собі сучасне право приватної
власности на маєтки, на які вони мали лише февдальні права,
і, нарешті, октроювали ті закони про оселення (law of settlement),
які, mutatis mutandis, мали такий самий вплив на англійських
селян рільників, як едикт татарина Бориса Годунова на російське
селянство.

«Glorious Revolution» (славетна революція) привела до влади
разом з Вільгельмом III Оранським200 земельних і капіталістичних
присвоювачів додаткової вартости (Plusmacher). Вони освятили
нову еру, практикуючи в колосальних розмірах крадіж
державних маєтків, що перед тим мав лише помірні розміри.
Державні землі роздаровувано, продавано за безцінь або приєд-

199 «А Letter to Sir Т. С. Bunbury, Brt.: On the High Price of Provisions.
By a Suffolk Gentleman», Ipswich 1795 p. 4. Навіть фанатичний
оборонець великого фармерства, автор «Inquiry into the Connection
betveen the present Price of provisions and the size of Farms», London
1773, p. 139, каже: «Я дуже жалкую за тим, що позвикали наші yeomanry,
ця кляса людей, що дійсно підтримувала незалежність нашої нації;
з сумом бачу, що їхні землі опинились тепер у руках монополістів-лордів
і позаорендовані дрібними фармерами на умовах, мало чим кращих від
тих умов, у яких були васалі, завжди готові відгукнутися на заклик при
всякій капосній справі» («І most truly lament the loss of our yeomanry, that
set of men who really kept up the independence of this nation; and sorry I am
to see their land now in the hands of monopolizing lords, tenanted out to
small farmers, who hold their leases on such conditions as to be little better
than vassals ready to attend a summons on every mischievous occasion»).

200    Про особисту мораль цього буржуазного героя можна судити,
між іншим, ось із чого: «Обдарування леді Оркней великими маєтками
в Ірляндії у 1695 р. — це явний доказ королівської прихильности й впливу
леді... Милі послуги леді Окрней були, кажуть, нечисті послуги устами».
(«The large grant of lands in Ireland to Lady Orkney, in 1695, is a public
instance of the king’s affection, and the lady’s influence... Lady Orkney’s
endearing offices are supposed to have been-foeda labiorum ministeria»).
(У Sloane Manuscript Collection, у Брітанському музеї, № 4224.
Манускрипт називається: «The character and behaviour of King William,
Sunderland etc. as represented in Original Letters to the Duke of
Shrewsbury from Somers, Halifax, Oxford, Secretary Vernon etc.». У ньому
повно курйозів).

* У французькому виданні останні два речення подано так:
«Залишаючи осторонь суто економічні впливи, що підготували експропріацію
рільників, ми переходимо тут до підойм, вжитих, щоб насильно
прискорити цю експропріяцію». («Le Capital etc.», ch. XXVII,
p. 319). Ред.
