\parcont{}  %% абзац починається на попередній сторінці
\index{i}{0145}  %% посилання на сторінку оригінального видання
до того пункту, де оплачену капіталом вартість робочої сили заміщується
новим еквівалентом, то це є простий процес утворення
вартости. Коли процес утворення вартости триває поза цей пункт,
то він стає процесом зростання вартости.\footnote*{
Вислів «процес зростання вартости» («Verwertungsprozess») у французькому
виданні тут, як і всюди далі, подано: «Production de plus
value», тобто «продукція додаткової вартости». \emph{Ред.}
}

Далі, коли ми порівняємо процес утворення вартости й процес
праці, то побачимо, що останній полягає в корисній праці, яка
продукує споживні вартості. Рух розглядається тут якісно,
з боку його особливого характеру, мети й змісту. А в процесі
утворення вартости той самий процес праці з’являється перед
нами лише з кількісного боку. Тут ідеться тільки про час, що
його потребує праця для своєї операції, або тільки про час, що
протягом його витрачається робочу силу з користю. І товари,
що входять у процес праці, мають тут значення вже не як функціонально
визначені речові фактори доцільно діющої робочої
сили. Їх береться до уваги лише як певні кількості упредметненої
праці. Чи міститься праця в засобах продукції, чи додана
вона робочою силою, тепер її береться до рахуби лише щодо
кількости часу її тривання. Вона становить стільки то годин,
днів і т. д.

Однак її береться до рахуби лише остільки, оскільки час,
витрачений на продукцію споживної вартости, є суспільнодоконечний.
А в цьому містяться різні умови. Робоча сила мусить
функціонувати за нормальних умов. Коли в суспільстві прядільна
машина є панівний засіб праці для прядіння, то робітникові
не можна давати прядки до рук. Він мусить діставати бавовну
нормальної якости, а не дрантя, яке щохвилини рветься. В противному
разі йому довелося б в обох випадках на продукцію одного
фунта пряжі витрачати робочого часу більше за суспільнодоконечний,
але цей надлишковий час не утворив би вартости
або грошей. Однак, нормальний характер речових факторів праці
залежить не від робітника, а від капіталіста. Дальшою умовою
є нормальний характер самої робочої сили. В тому фаху, в якому
її вживається, вона мусить володіти звичайним пересічним ступенем
вправности, вмілости й швидкости. Але наш капіталіст
купив на ринку робочу силу нормальної якости. Ця сила мусить
витрачатись із звичайною пересічною мірою напруги, з суспільнозвичайним
ступенем інтенсивности. Капіталіст стежить так само
дбайливо за цим, як і за тим, щоб жодна хвилина не пропадала
марно без праці. Він купив робочу силу на певний реченець.
Він бажає одержати своє. Він не хоче бути окраденим. Нарешті —
і для цього цей самий пан має свій власний code pénal** — не
повинно бути жодного недоцільного споживання сировинного
матеріялу й засобів праці, бо змарнований матеріял або змарновані
засоби праці являють собою зайво витрачені кількості

* * — карний кодекс. \emph{Ред.}
\parbreak{}  %% абзац продовжується на наступній сторінці
