Технічне підпорядковання робітника одноманітній ході засобу
праці і своєрідний склад робочого тіла з індивідів обох статей
і якнайрізнішого віку створюють казармову дисципліну, яка
розвивається на вивершений фабричний режим та цілком розвиває
раніш уже згадану працю нагляду, отже, разом з тим і
поділ робітників на ручних робітників та наглядачів за працею,
на рядових промислових солдатів, та промислових унтер-офіцерів.
«Головні труднощі в автоматичній фабриці... були... в дисципліні,
доконечній, щоб примусити людей одмовитися від їхньої
звички до нереґулярности в роботі та пристосувати їх до незмінної
реґулярности великого автомату. Але ж винайти та з успіхом
провести в життя дисциплінарний кодекс, що відповідав би
потребам та швидкості автоматичної системи — ця робота, гідна
Геркулеса, була благородним ділом Аркрайта! Навіть за наших
днів, коли цю систему зорганізовано в цілій її повноті, майже
неможливо знайти серед робітників, що вилюдніли вже в дозрілих
людей... корисних помічників для автоматичної системи».\footnote{
Ure: «Philosophy of Manufacture», стор. 15. Хто знає біографію
Аркрайта, тому ніколи не спаде на думку назвати цього геніяльного
голяра «благородним». Поміж усіх великих винахідників XVIII віку
він безперечно був найбільший крадій чужих винаходів та наймерзенніший
суб’єкт.
}
Фабричний кодекс, що в ньому капітал формулює свою автократію
над своїми робітниками приватноправним шляхом та самовладно,
без поділу влади, взагалі такого любого буржуазії, і
без ще улюбленішої репрезентативної системи, — цей кодекс є
лише капіталістична карикатура того суспільного реґулювання
процесу праці, яке стає потрібне при кооперації у великому маштабі
та при вживанні спільних засобів праці, особливо машин.
Місце батога в наглядача за рабами заступає карна книга наглядача.
Всі кари природно сходять до грошової кари й відраховань
із заробітної плати, а законодавча бистродумність фабричних
Лікурґів робить порушування їхніх законів, коли можливо,
ще прибутковішим для них, ніж додержування їх.\footnote{
«Рабство, в кайданах якого буржуазія тримає пролетаріят, ніде
так ясно не виявляється, як у фабричній системі. Тут настає кінець
усякій свободі, і юридично, і фактично. Вранці о пів на шосту робітник
мусить бути на фабриці; якщо він спізниться на декілька хвилин, — на
нього накладають кари; якщо він спізниться на 10 хвилин — його зовсім
не впускають до фабрики до кінця сніданку, і він втрачає плату за
чверть дня. Він мусить на команду їсти, пити та спати... Деспотичний
дзвінок підіймає його з ліжка, відриває його від сніданку та обіду. А як
воно на фабриці? Тут фабрикант — абсолютний законодавець. Він видає
фабричні правила, як йому забажається; змінює та робить додатки
до свого кодексу, які йому захочеться; і хоч які безглузді ці зміни та
додатки, суди все ж таки кажуть робітникові: через те, що ви з доброї
волі згодилися на цей контракт, то мусите тепер його виконувати... Ці
}

a common labourer саn learn». («The Master Spinners’and Manufacturer,
Defence Fund. Report of the Committee», Manchester 1854, p. 17). Пізніш
побачимо, що цей «хазяїн» співає іншої пісеньки, коли йому загрожує
небезпека втратити свої «живі» автомати.