й т. ін. Наприклад, у виробництві модного вбрання, де праця
здебільша вже була зорганізована, переважно у формі простої
кооперації, швацька машина спочатку становить лише новий
фактор мануфактурного виробництва. В кравецтві, виробництві
сорочок, у шевстві й т. ін. перехрещуються всі форми. Тут —
власне фабричне виробництво. Там — посередники дістають сировинний
матеріял від капіталіста en chef та гуртують коло швацьких
машин по «комірках» і «горищах» 10—50, а то й більше найманих
робітників. Нарешті, як то буває при всіх машинах, які не
являють собою розчленованої системи й які можна застосовувати
в карликовому розмірі, ремісники або домашні робітники із
своєю власного родиною або за участю небагатьох чужих робітників
уживають швацьких машин, що їм самим і належать.270
В Англії тепер фактично переважає така система, що капіталіст
концентрує значне число машин у своїх будівлях і потім розподіляє
машиновий продукт для дальшого оброблення серед армії
домашніх робітників.271 Однак строкатість переходових форм
не заховує тенденції до перетворення на фабричне виробництво
у власному значенні. Цю тенденцію живить сам характер швацької
машини, що її придатність до різноманітного вжитку штовхає
до сполучення відокремлених раніш галузей продукції в
тому самому будинку й під командою того самого капіталу; її
живить та обставина, що попереднє зшивання й деякі інші операції
найдоцільніше виконувати там, де стоїть машина; нарешті,
її живить неминуча експропріяція ремісників і домашніх робітників,
що продукують власними машинами. Ця доля спіткала
їх уже почасти тепер. Невпинний зріст маси капіталу, вкладеного
у швацькі машини,272 спонукує збільшувати продукцію та спричинює
на ринку застої, що дає сигнал домашнім робітникам продавати
швацькі машини. Перепродукція самих таких машин примушує
продуцентів, які потребують збуту, віддавати їх на прокат
за тижневу плату, і цим самим вона утворює вбивчу конкуренцію
для дрібних власників машин.273 Постійні зміни конструкції,
що все ще тривають далі, та здешевлення машин так само постійно
знецінюють старі екземпляри машин і дають змогу вживати їх
із зиском тільки великим капіталістам, що купують їх масами
по неймовірно низьких цінах. Нарешті, тут, як і в усіх подібних
процесах перевороту, вирішальне значення має заміна людини
паровою машиною. Вживання парової сили натрапляє спочатку
на суто технічні перешкоди, як от двигіт машин, труднощі щодо

270    Цього немає в рукавичництві й т. ін., де становище робітників
ледве можна відрізнити від становища павперів.

271 «Children’s Employment Commission. 2 nd Report 1864», p. 2,
n. 122.

272    В Лейчестері в самому лише чобітно-шевському виробництві,
що продукувало на гуртовий продаж, уже 1864 р. вживалося 800 швацьких
машин.

273 «Children’s Employment Commission. 2nd Report 1864», p. 84,
n. 124.
