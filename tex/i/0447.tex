вартість поділяються між капіталістом і робітником. Тому, якщо
розглядати їх як безпосередні вирази ступеня самозростання
капіталу, то дійшлося б такого неправильного закону: додаткова
праця або додаткова вартість ніколи не може досягти 100\%.17
А що додаткова праця завжди може становити лише певну частину
робочого дня, або додаткова вартість — лише певну частину
новоспродукованої вартости, то додаткова праця завжди є неодмінно
менша, ніж робочий день, або додаткова вартість завжди
є менша, ніж новоспродукована вартість. Але для того, щоб
відноситися одна до однієї як \sfrac{100}{100} вони мусили б бути між собою
рівні. Щоб додаткова праця забрала цілий робочий день (тут
мова йде про пересічний день робочого тижня, робочого року
й т. ін.), доконечна праця мусила б упасти до нуля. Але коли
зникає доконечна праця, то зникає й додаткова праця, бо остання
є лише функція першої. Отже, пропорція додаткова праця/робочий день =
додаткова вартість/новоспродукована вартість ніколи не може досягти межі
\sfrac{100}{100}, а ще менше — підвищитися до (100 + х)/100. Але це цілком можлива річ
для норми додаткової вартости, або дійсного ступеня експлуатації
праці. Візьмімо, приміром, обчислення пана Л. де Ляверня,
що за ним англійський рільничий робітник дістає лише чвертину,
а капіталіст (фармер) — три чверті продукту18 або його вартости, не-

17 Див., наприклад, «Dritter Brief an v. Kirchmann von Rodbertus.
Widerlegung der Ricardoschen Theorie von der Grundrente und Begründung
einer neuen Rententheorie», Berlin 1854. Я пізніше повернуся
до цього твору, який, не вважаючи на його хибну теорію земельної ренти,
доходить суті капіталістичної продукції. — [Додаток до третього видання.
— Ми бачимо тут, як доброзичливо цінував Маркс своїх попередників,
коли находив у них якийсь справжній крок наперед, якусь вірну
нову думку. Тимчасом опубліковані листи Родбертуса до Руд. Маєра
обмежують до певної міри вищенаведене визнання. Там читаємо: «Треба
врятувати капітал не тільки від праці, але й від себе самого, а цього в
дійсності можна найкраще досягти, якщо розглядати діяльність капіталіста-підприємця як народньо- й
державногосподарську функцію,
покладену на нього капіталістичною власністю, а його дохід — як певну
форму утримання, бо ми ще не знаємо ніякої іншої соціяльної організації.
Але утримання повинні бути вреґульовані і знижені, коли вони
забагато відбирають від заробітної плати. Таким способом слід також
відбити напад Маркса на суспільство — так я назвав би його книгу...
Взагалі Марксова книга — це не так дослід про капітал, як полеміка
проти теперішньої форми капіталу, яку він сплутує з самим поняттям
капіталу, з чого саме й постають його помилки». («Briefe usw. von Dr.
Rodbertus-Jagetzow, herausgegeben von Dr. Rud. Meyer», Berlin 1881,
Bd. I, S. 111, 48. Brief von Rodbertus). У таких ідеологічних банальностях
зникають справді сміливі напади Родбертусових «соціяльних листів».
— Ф. Е.].

18 Ту частину продукту, яка тільки покриває витрачений сталий
капітал, само собою зрозуміло, з цього обчислення виключено. — Пан
Л. де Лявернь, сліпий прихильник Англії, подає скорше надто низьке,
ніж високе відношення.
