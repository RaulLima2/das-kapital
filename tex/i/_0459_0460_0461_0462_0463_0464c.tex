\parcont{}  %% абзац починається на попередній сторінці
\index{i}{0459}  %% посилання на сторінку оригінального видання
екстенсивної величини праці зросла її інтенсивна величина.\footnote{
«Заробітні плати залежать від ціни праці та від кількости виконаної
роботи. Збільшення плати за працю не означає ще неодмінно зросту
ціни праці. Із збільшенням робочого часу і при більшому напруженні
плата за працю може значно зрости, тимчасом як ціна праці може лишитися
незмінною». («The wages of labour depend upon the price of labour
and the quantity of labour performed\dots{} An increase in the wages of
labour does not necessarily imply an enhancement of the price of labour.
From fuller employment, and greater exertions, the wages of labour may
be considerably increased, while the price of labour may continue the same»).
(West: «Price of Corn and Wages of Labour», London. 1826, p. 67, 68, 112).
Головного питання: як визначається «ціну праці», Вест, між іншим,
збувається банальними фразами.
}
Тому підвищення номінальної поденної або потижневої заробітної
плати може супроводитися ціною праці, що лишається
незмінною або падає. Те саме має силу і щодо доходу робітничої
родини, скоро тільки кількість праці, давана головою сем’ї,
збільшиться працею членів родини. Отже, існують методи понижати
ціну праці, незалежні від зменшення номінальної поденної
або потижневої заробітної плати.\footnote{
Це правильно відчуває фанатичний представник промислової
буржуазії XVIII віку, не раз цитований нами автор «Essay on Trade
and Commerce», хоч він з’ясовує справу плутано: «Кількість праці,
а не ціна її (під останньою він розуміє номінальну поденну або потижневу
заробітну плату) визначається ціною харчових та інших доконечних засобів
існування. Коли ви дуже понизите ціну засобів існування, то ви
тим самим зменшите у відповідній пропорції кількість праці\dots{} Фабриканти
знають, що існують різні шляхи підвищувати та знижувати ціну
праці, незалежно від зміни її номінального розміру». («It is the quantity
of labour and not the price of it, that is determined by the price of provisions
and other necessaries: reduce the price of necessaries very low, and
of course you reduce the quantity of labour in proportion\dots{} Master-manufacturers
know, that there are various ways of raising and felling the
price of labour, besides that of altering its nominal amount»). (Там же,
стор. 48 і 61). Н. В. Сеніор у своїх «Three Lectures on the Rate of Wages»,
London 1830, де він використовує твір Веста, не покликаючись на
нього, між іншим, каже: «Робітник головним чином заінтересований
у височині заробітної плати» («The labourer is principally interested
in the amount of wages») (p. 14). Отже, робітник заінтересований головним
чином у тому, що він одержує — в номінальній сумі заробітної плати,
а не в тому, що він дає — не в кількості праці!
}

Але звідси випливає такий загальний закон: якщо дано кількість
денної, тижневої і так далі праці, то денна або тижнева
заробітна плата залежить від ціни праці, яка сама змінюється
або із зміною вартости робочої сили, або з відхиленням ціни робочої
сили від її вартости. Навпаки, якщо дано ціну праці, то денна
або тижнева заробітна плата залежить від кількости денної або
тижневої праці.

Одиниця міри почасової плати, ціна робочої години, є результат
від ділення денної вартости робочої сили на число годин звичайного
робочого дня. Припустімо, що робочий день становить
12 годин, денна вартість робочої сили — 3 шилінґи, тобто вартість,
продуковану за 6 робочих годин. Ціна робочої години за
цих умов є 3 пенси, спродукована протягом неї вартість — 6 пенсів.
\index{i}{0460}  %% посилання на сторінку оригінального видання
Якщо ж робітника вживатимуть до праці менше ніж 12 годин
на день (або менше ніж 6 днів на тиждень), наприклад, лише
6 або 8 годин, тоді він одержить за цієї ціни праці лише 2 або
1\sfrac{1}{2} шилінґи денної плати.\footnote{
Вплив такої ненормальної недостачі роботи цілком відрізняється
від впливу загального примусового законодавчого скорочення робочого
дня. Перша не має нічого спільного з абсолютною довжиною робочого
дня й може постати так само при п’ятнадцятигодинному, як і при шестигодинному
робочому дні. Нормальну ціну праці в першому випадку обчислюється
з того, що робітник працює пересічно 15 годин, у другому — з
того, що він працює пересічно 6 годин денно. Тому вплив лишається той
самий, якщо робітник працює в одному випадку лише 7\sfrac{1}{2}, а в другому
лише 3 години.
} А що, за нашим припущенням, він
мусить працювати пересічно 6 годин на день, щоб випродукувати
лише денну плату, відповідну вартості його робочої сили, що,
за тим самим припущенням, він працює лише половину з кожної
години на себе самого, а половину на капіталіста, то ясно, що
він не може одержати для себе вартости, випродукованої протягом
6 годин, коли працюватиме менше ніж 12 годин. Якщо раніше
ми бачили руйнаційні наслідки надмірної праці, то тут ми відкриваємо
джерела тих страждань, які постають для робітника від
недостачі роботи.

Коли погодинну заробітну плату встановлюється так, що
капіталіст не зобов’язується платити якусь поденну або потижневу
заробітну плату, а зобов’язується платити тільки за ті
робочі години, протягом яких йому забажається вживати робітника,
то він може вживати його на час, менший від того, що його
первісно покладено в основу визначення погодинної заробітної
плати або одиниці міри для ціни праці. А що ця одиниця міри
визначається відношенням

денна вартість робочої сили/робочий день даного числа годин,

то вона, природно, втрачає всякий сенс, скоро тільки робочий
день перестає мати в собі певне число годин. Знищується зв’язок
між оплаченою й неоплаченою працею. Тепер капіталіст може
видушити з робітника певну кількість додаткової праці, не даючи
йому для роботи робочого часу, доконечного для його власного
утримання. Він може знищити всяку реґулярність праці й цілком
залежно від своєї вигоди, сваволі та інтересів моменту, навпереміну
то примушувати робітника до неймовірної надмірної
праці, то залишати його почасти або й зовсім без роботи. Він
може з того приводу, що він нібито платить «нормальну ціну
праці», ненормально здовжувати робочий день без будь-якої відповідної
компенсації для робітника. Звідси цілком раціональне
повстання лондонських будівельних робітників (1860 р.) проти
спроби капіталістів накинути їм таку погодинну заробітну плату.
Законодавче обмеження робочого дня покладає кінець цій неподобності,
хоч, природно, не знищує недостачі роботи, що випливає
\index{i}{0461}  %% посилання на сторінку оригінального видання
з конкуренції машин, зміни в якості вживаних робітників,
частинних та загальних криз.

При зростанні поденної або потижневої заробітної плати ціна
праці може номінально лишитися сталою і все ж впасти нижче
від свого нормального рівня. Це буває щоразу тоді, коли за сталої
ціни праці, зглядно робочої години, робочий день здовжується
поза звичайний час його тривання. Коли у дробу
денна вартість робочої сили/робочий день зростає
знаменник, то ще швидше зростає чисельник. Вартість робочої сили, в наслідок зростання її
зужитковування, зростає разом із триванням
її функціонування, та ще й у швидшій пропорції, ніж приріст тривання її
функціонування. Тому в багатьох галузях продукції, де панує
почасова заробітна плата без законодавчих обмежень робочого
часу, сама собою (naturwüchsig) розвинулася звичка вважати
робочий день за нормальний лише до певного пункту, наприклад,
до скінчення десятої години («normal working day», «the day’s
work», «the regular hours of work»).\footnote*{
— «нормальний робочий день», «денна праця», «регулярний робочий
час». \emph{Ред.}
} Поза цією межею робочий
час становить наднормовий час (overtime), і за той час, беручи
за одиницю міри годину, платять більше (extra рау), хоч часто
в пропорції до смішного малій.35 Нормальний робочий день
існує тут як дріб дійсного робочого дня, і цей останній часто протягом
цілого року є довший від нормального.\footnote{
«Норма плати за наднормовий час (у виробництві мережива) така
мала, півпенні тощо за годину, що вона стоїть у гострому контрасті до
того масового лиха, якого зазнає від неї здоров’я та життєва сила робітників\dots{}
Крім того, здобутий таким чином невеликий надлишок до заробітної
плати робітникові доводиться часто витрачати на різні засоби,
що підживляють його сили». («Children’s Employment Commission. 2nd
Report», p. XVI, n. 117).
} Зростання ціни
праці із здовженням робочого дня поза певну нормальну межу
в різних галузях брітанської промисловости призводить до того
результату, що низька ціна праці в так званий нормальний час змушує
робітника працювати наднормовий час, за який краще платять,
якщо він взагалі хоче одержати достатню заробітну плату.\footnote{
Наприклад, y шотландських білильнях. «У деяких частинах
Шотландії цю промисловість провадилося (перед заведенням фабричного
закону 1862 р.) за системою наднормового часу, тобто 10 годин вважали
за нормальний робочий день. Заце робітник діставав Ішилінґ 2 пенси.
Але до цього треба ще додати щодня наднормовий час у 3--4 години,
за що платили по 3 пенси за годину. Наслідки цієї системи: робітник,
що працював тільки нормальний час, міг заробити лише 8 шилінґів на
тиждень. Без наднормового часу цієї заробітної плати не вистачало».
(«Reports of Insp. of Fact. for 30 th April 1863», p. 10). «Підвищена плата
}

36    Наприклад, y виробництві шпалер перед недавнім заведенням
фабричного закону. «Ми працюємо без перерв на їжу, так що 10 У2-годинний
робочий день кінчається близько пів на п’яту по півдні, а все
дальше є наднормовий час, що рідко закінчується перед восьмою годиною
вечора, і ми в дійсності протягом цілого року працюємо наднормовий
час». (Мг. Smith’s Evidence у «Children’s Employment Commission.
1 st Report», p. 125).
\index{i}{0462}  %% посилання на сторінку оригінального видання
Законодавче обмеження робочого дня покладає кінець цій втісі.\footnote{
Див. «Reports of Insp. of Fact, for 30 th April 1863», p. 5. Цілком
справедливо критикуючи цей стан речей, лондонські будівельні робітники
заявили підчас великого страйку та льокавту 1860 р., що вони згодяться
на погодинну заробітну плату лише на двох умовах: 1) щоб разом
із ціною робочої години було установлено нормальний робочий день на
9 або 10 годин та щоб ціна за годину десятигодинного робочого дня була
більша, ніж ціна за годину дев’ятигодинного; 2) щоб за кояшу годину
понад нормальний день платилося, як за наднормовий час, відповідно вище.
}

Загальновідомий факт, що чим довший у якійсь галузі промисловосте
робочий день, тим нижча заробітна плата.\footnote{
«Це дуже дивна річ, що там, де довгий робочий день є загальне
правило, загальне правило є й низька заробітна плата» («It is a very
notable thing, too, that where long hours are the rule, small wages are also
so»). («Reports of Insp. of Fact, for 31 st October 1863», p. 9). «Праця, що
добуває мізерну кількість засобів існування, звичайно є надмірно подовжена
» («The work which obtains the scanty pittance of food is for the most
part excessively prolonged»). («Public Health, Sixth Report 1864», p. 15).
} Фабричний
інспектор А. Редґрев ілюструє це порівняльним оглядом
двадцятирічного періоду від 1839 до 1859 р., з якого видно, що
на фабриках, підлеглих законові про десятигодинний робочий
день, заробітна плата підвищилася, тимчасом як на фабриках,
де працювали від 14 до 15 годин на день, вона зменшилась.\footnote{
«Reports of Insp. of Fact. for 30 th April 1860», p. 31, 32.
}

Зі закону, що «при даній ціні праці поденна або потижнева
плата залежить від кількосте постачуваної праці», випливає
насамперед, що чим нижча ціна праці, тим більша мусить бути
кількість праці, бо тим довший мусить бути робочий день, щоб
робітник забезпечив собі хоча б мізерну пересічну заробітну плату.
Низька ціна праці діє тут як спонука здовжувати робочий час.\footnote{
Так, наприклад, в Англії ручні робітники-цвяхарі в наслідок
низької ціни праці мусили працювати 15 годинна день, щоб одержати якнаймізернішу
тижневу плату. «Багато, багато годин на день та в усякий.
}

Але і, навпаки, здовження робочого часу, з свого боку, викликає
зниження ціни праці, а разом з ним і зниження поденної
або потижневої плати.

Визначення ціни праці дробом

денна вартість робочої сили/робочий час даного числа годин

показує, що просте здовження робочого дня знижує ціну праці,
якщо при цьому немає ніякої компенсації. Але ті самі обставини,

за наднормовий час — це спокуса, якій робітники не можуть протистояти».
(«Reports of Insp. of Fact, of 30 th. April 1848», p. 5). Палітурня
в лондонському Сіті вживає до роботи дуже багато молодих дівчат
14--15 років, і саме на підставі контракту для учнів, що приписує певні
робочі години. А проте вони працюють останній тиждень кожного місяця
до 10, 11, 12 і навіть до 1 години вночі разом із старшими робітниками
в дуже мішаному товаристві. «Хазяїни спокушають (tempt) їх підвищеною
платою та грішми на добру вечерю», яку вони беруть у сусідніх
шинках. Велика розпуста, поширена таким способом серед цієї
young immortals\footnote*{
— безсмертної молоді. \emph{Ред.}
} («Child. Empl. Comm.» V Rep., p. 44, n. 191), компенсується
тим, що вони оправляють, між іншим, і багато біблій та інших
душоспасенних книг.
\index{i}{0463}  %% посилання на сторінку оригінального видання
що дають змогу капіталістові надовго здовжити робочий день,
дають йому спочатку змогу, а, кінець-кінцем, і примушують
його знизити ціну праці й номінально, поки не знизиться ціла ціна
збільшеного числа годин, отже, поденна або потижнева плата.
Нам досить тут буде зазначити дві обставини. Коли одна людина
робить роботу 1 або 2 людей, то подання праці зростає, якщо
навіть подання робочих сил, що є на ринку, і лишається стале.
Конкуренція, створена таким чином серед робітників, дає змогу
капіталістові знижувати ціну праці, і, навпаки, падіння ціни
праці дає йому змогу ще дужче збільшити робочий час.\footnote{
Коли б якийсь фабричний робітник відмовився, наприклад, працювати
встановлене велике число годин, «його одразу замінили б кимось
іншим, готовим працювати скільки завгодно, і він лишився б без праці»
(«he would very shortly be replaced by somebody who would work any
length of time and thus be tbrown out of employment»). («Reports of Insp.
of Fact. for. 31 st October 1848». Evidence p. 39, n. 58). «Якщо одна людина
виконує працю двох\dots{} норма зиску звичайно підвищується\dots{} бо додаткове
подання праці знижує її ціну» («If one man performs the work of
two\dots{} the rate of profits will generally be raised\dots{} in conséquence of the
additional supply of labour having diminished its price»). («Senior:
«Three Lectures on the Rate of Wages», London 1830. p. 14).
} Однак
незабаром ця змога порядкувати ненормальною, тобто такою,
що перевищує пересічний суспільний рівень, кількістю неоплаченої
праці стає засобом конкуренції між самими капіталістами.
Частина ціни товару складається з ціни праці. Неоплачену частину
ціни праці немає потреби зараховувати в ціну товару.
Її можна подарувати покупцеві товару. Це — той перший крок,
до якого веде конкуренція. Другий крок, до якого вона примушує,
— це те, щоб із продажної ціни товару виключити також,
принаймні, якусь частину ненормальної додаткової вартости, створеної
через здовження робочого дня. Таким способом утворюється
спочатку спорадично, а потім поступінно фіксується ненормально
низка продажна ціна товару, яка відтепер стає постійною
основоью мізерної заробітної плати при надмірному
робочому часі, так само як вона первісно була продуктом цих
обставин. Ми лише коротко відзначаємо цей рух, бо аналіза конкуренції
тут не є наше завдання. Однак даймо на хвилину слова
самому капіталістові. «В Бермінґемі конкуренція між хазяїнами
така велика, що дехто з нас як підприємець примушений робити
таке, що він посоромився б зробити за інших умов; а проте й
таким способом не добудеш більше грошей (and yet no more
monev is made), a тільки сама публіка має з того користь».\footnote{
«Children’s Employment Commission, 3 rd Report». Evidence, p. 66, p. 22.
}
Пригадаймо собі два сорти лондонських пекарів, що з них одні
продають хліб за повну ціну (the «fullpriced» bakers), а другі
продають його нижче за його нормальну ціну («the underpri-

час мусить він тяжко працювати, щоб здобути 11 пенсів, або 1 шилінґ,
та ще з того 2\sfrac{1}{2}—3 пенси відпадає на зужиткування знаряддя, на паливо,
відпадки заліза». («Children’s Employaient Commission. З rd. Report», p.
136, n. 671). За той самий робочий час жінки заробляють щотижня
лише 5 шилінґів (там же, стор. 137, п. 674).
\index{i}{0464}  %% посилання на сторінку оригінального видання
ced», «the undersellers»). «Fullpriced» так виказують на своїх
конкурентів перед парляментською слідчою комісією: «Вони
існують лише тим, що, поперше, дурять публіку, фальсифікуючи
товар, а, подруге, витискують із своїх людей 18 годин праці
за плату дванадцятигодинної праці\dots{} Неоплачена праця (the
unpaid labour) робітника — ось засіб, яким вони борються в
конкуренції\dots{} Конкуренція між хазяїнами-пекарями є причина
труднощів усунути нічну працю. Underseller, той,* що продає
свій хліб нижче ціни витрат, яка змінюється разом із ціною
борошна, поповнює свої втрати тим, що витискує із своїх людей
більше праці. Коли я від своїх людей витискую тільки 12 годин
праці, а мій сусіда, навпаки, 18 або 20, то він мусить мене побити
в продажній ціні. Коли б робітники могли добитися плати за
наднормовий час, то тоді б цьому маневрові скоро прийшов би
кінець\dots{} Значне число тих, що працюють в underseller’ів — то
чужинці, підлітки та інші, що примушені задовольнятися майже
всякою заробітною платою, яку можуть дістати».\footnote{
«Report etc. relative to the Grievances complained of by the
journeymen bakers», London 1862, p. LII і там же, Evidence, n. 479,
359, 27). А проте й fullpriced, як ми згадували вище та як це визнає сам
оборонець їхній, Беннет, примушують своїх людей починати працю об
11 годині вечора або й раніш та часто здовжують її до 7 години наступ
ного вечора». (Там же, стор. 22).
}

Ця єреміяда ще й тим цікава, що вона показує, як у мозку
капіталіста відбивається лише зовнішня видимість продукційних
відносин. Капіталіст не знає того, що й нормальна ціна праці
включає певну кількість неоплаченої праці та що саме ця неоплачена
праця є джерело його зиску. Категорії додаткового робочого
часу для нього взагалі не існує, бо цей час міститься в нормальному
робочому дні, і він вважає, що оплачує цей день у поденній
заробітній платі. Щоправда, він визнає існування наднормового
часу, здовження робочого дня поза межу, що відповідає звичайній
ціні праці. Щодо свого конкурента, underseller’а, який продає
нижче проти нормальної ціни, то він обстоює навіть, щоб той
вище оплачував (extra pay) за цей наднормовий час. Знову ж
таки він не знає, що ця вища плата так само включає неоплачену
працю, як і ціна звичайної робочої години. Наприклад,
ціна однієї години дванадцятигодинного робочого дня є 3 пенси,
тобто вартість, спродукована за половину робочої години, тимчасом
як ціна наднормової робочої години є 4 пенси, тобто вартість,
спродукована за \sfrac{2}{3} робочої години. В першому випадку капіталіст
присвоює собі безплатно половину робочої години, у
другому — третину.

Розділ дев’ятнадцятий
Відштучна плата

Відштучна заробітна плата є не що інше, як перетворена
форма почасової плати, так само як почасова плата є перетворена
форма вартости або ціни робочої сили.
