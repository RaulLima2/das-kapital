охоче платили б за кращі помешкання, коли б їх можна було
дістати. Тимчасом вони із своїми родинами занепадають і хоріють,
а солодкувато-ліберальний Форстер, член парляменту, проливає
сльози захоплення з приводу благодаті вільної торговлі й зисків
видатних голів Bradford’у від вовняних підприємств. У звіті
з 5 вересня 1865 р. д-р Белл, один із бредфордських лікарів для
бідних, пояснює жахливу смертність хорих на тиф у його окрузі
їхніми житловими умовами: «В одному льоху на 1.500 кубічних
футів живе десять осіб... На вулицях. Vincentstrasse, Green
Air Place і the Leys є 223 доми з 1.450 мешканцями, 435 ліжками
і 36 кльозетами... На кожне ліжко — а під ліжком я розумію
всякий жмут брудного ганчір’я або купу стружок — припадає
пересічно 3,3 особи, на декотрі 4 й 6 осіб. Багато спить без ліжка
просто на підлозі, не роздягаючись, молоді чоловіки й жінки,
жонаті й нежонаті — все це як попало, одне побіч одного. Чи
треба ще додати, що ці житла здебільша темні, вогкі, брудні,
смердючі нори, цілком непридатні для людського мешкання?
Це — центри, звідки поширюються недуги й смерть, що виривають
свої жертви навіть з-серед заможних (of good circumstances),
які допустили до того, щоб ця моровиця гноїлася в нашому
середовищі».\footnote{
Там же, стор. 114.
}

Третє після Лондону місце щодо житлових злиднів посідає
Брістол. «Тут, в одному з найбагатших міст Европи, якнайбільший
надмір глибокої бідности («blank poverty») і житлових
злиднів».\footnote{
Там же, стор. 50.
}

с) Бродяча людність

А тепер звернімося до верстви людности, сільської своїм походженням,
але здебільша занятої в промисловості. Вона становить
легку піхоту капіталу, яку відповідно до своїх потреб
він кидає то в один пункт, то в інший. Коли вона не в поході,
то «стоїть табором». Працю бродячих робітників використовують
на різні будівельні й дренажні операції, вироблення цегли,
випалювання вапна, будову залізниць тощо. Вони є рухлива
колона, що переносить у ті місцевості, навколо яких вони отаборюються,
заразливі недуги: віспу, тиф, холеру, скарлятину
й т. ін.\footnote{
«Public Health. Seventh Report», London 1865, p. 18.
} У підприємствах із значною витратою капіталу, як

Edwardstreet. Nr. 4        1    кімната    17 осіб.
Jorkstreet. Nr. 34        1        2 родини
Salt Piestreet      1  26 осіб

Льохи

Regent Square        1    льох    8 осіб
Acrestreet        1        7 осіб
Robert’s Court. Nr. 33    1     7 осіб
Back Prattstreet, помешкання
використовується як мідярня    1         7 осіб
Ebenezerstreet. Nr. 27           1            6 осіб

(«Public Health. Eighth Report», стор. 111).