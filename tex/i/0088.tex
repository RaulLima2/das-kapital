товару міряє ступінь притяжної сили його щодо всіх елементів
речового багатства, отже, міряє суспільне багатство свого посідача.
Для варварсько-примітивного посідача товарів, навіть
для західньоевропейського селянина, вартість є невіддільна від
форми вартости, а тому збільшення золотого й срібного скарбу
є для нього збільшення вартости. Правда, вартість грошей змінюється,
чи то в наслідок зміни їхньої власної вартости, чи то
в наслідок зміни вартости товарів. Але це, з одного боку, не
заважає тому, що 200 унцій золота як і раніш, мають більше
вартости, ніж 100, 300 — більше, ніж 200 і т. д., з другого, —
що натуральна металева форма цієї речі лишається загальною
еквівалентною формою всіх товарів, безпосередньо суспільним
втіленням усякої людської праці. Потяг до скарботворення з
самої природи своєї безмірний. Якісно, або своєю формою, гроші
не мають меж, тобто вони є загальний представник речового
багатства, бо вони безпосередньо можуть перетворюватись на
кожний товар. Але разом з тим кожна реальна грошова сума є
кількісно обмежена, отже, вона є купівельний засіб з обмеженою
силою. Ця суперечність поміж кількісною обмеженістю і якісною
безмежністю грошей завжди спонукує збирача скарбів до сізіфової
праці акумуляції. З ним справа стоїть так само, як з
тим завойовником світу, що з кожною новою країною здобуває
лише якийсь новий кордон.

Щоб затримати в себе золото як гроші, а тому і як елемент
скарботворення, треба перешкодити йому циркулювати, або розпускатись
як засобу купівлі в засобах споживання. Тому збирач
скарбів жертвує золотому фетишеві бажання свого тіла.
Він навсправжки приймає євангелію відречення. Але, з другого
боку, він може витягти з циркуляції в грошах лише те, що він
дав їй у товарах. Що більш він продукує, то більш може він
продати. Тим то працьовитість, ощадність і скупість становлять
його кардинальні чесноти; багато продавати й мало купувати —
в цьому вся його політична економія.\footnote{
«Збільшити якомога число продавців усіх товарів, зменшити якомога
число покупців їх — ось основні пункти, на які сходять усі заходи
політичної економії» («Accrescere quanto più si puó il numero de’ venditori
d’ogni merce, diminuire quanto piú si puó il numero dei compratori, questi
sono і cardini sui quali si raggirano tutte le operazioni di economia politica»).
(Verri: «Meditazioni sulla Economia Politica», Custodi, ParteModerna,
vol. 15, p. 52).
}

Поруч із безпосередньою формою скарбу йде його естетична
форма, володіння золотими й срібними товарами. Воно зростає
разом із багатством буржуазного суспільства. «Soyons riches
ou paraissons riches»\footnote*{
Будьмо багаті або видаваймося багатими. Ред.
} (Дідро). Таким чином утворюється почасти
чимраз поширеніший ринок для золота й срібла, незалежно від
їхніх грошових функцій, почасти — лятентне джерело постачання
грошей, що особливо швидко тече за часів суспільних бур.

Скарботворення виконує різні функції в економії металевої
циркуляції. Найближча його функція виникає з умов обігу золо-