є поки що тільки гусінню капіталіста, мусить купувати товари
за їхньою вартістю, за їхньою вартістю продавати, і, все ж таки,
наприкінці процесу витягати більше вартости, ніж він авансував.
Його перетворення з гусені на метелика, з лише посідача
грошей на дійсного капіталіста, мусить відбуватися в сфері
циркуляції, і в той самий час воно мусить відбуватись не
в сфері циркуляції. Такі умови цієї проблеми. Ніс Rhodus,
hiс salta!\footnote*{
Дослівно: Тут Родос, тут стрибай. — Античне прислів’я з байок
Езопа: відповідь родосців одному хвалькові, що вихвалявся своїми
стрибками. Вживається в значенні: отут покажи свою вмілість. Ред.
}

3. Купівля і продаж робочої сили

Зміна вартости грошей, що повинні перетворитися на капітал,
не може відбутися в самих цих грошах, бо як засіб купівлі і як
засіб виплати вони реалізують ціну товарів, які за них купують
або за які ними платять, а коли вони залишаються в своїй власній
формі, то вони тверднуть у, так би мовити, скам’янілу, незмінну
величину вартости.\footnote{
«У формі грошей... капітал не продукує зиску» («In the form of
money... capital is productive of no profit»). (Ricardo: «Principles of
Political Economy», 3 rd ed. London 1821, p. 267).
} Так само не може виникнути ця зміна з
другого акту циркуляції, з перепродажу товару, бо цей акт лише
перетворює товар із натуральної форми назад у грошову форму.
Отже, зміна мусить статися з товаром, що купується в першому
акті Г — Т, але не з його вартістю, бо обмінюється еквіваленти,
товар оплачується за його вартістю. Отже, зміна може виникнути
лише з його споживної вартости, як такої, тобто з його споживання.
Щоб здобувати вартість з споживання якогось товару, нашому
посідачеві грошей мусило б пощастити відкрити в межах сфери
циркуляції, на ринку, такий товар, що сама його споживна вартість
мала б своєрідну властивість бути за джерело вартости,
отже, такий товар, що його фактичне споживання саме було б
упредметненням праці, а тому й творенням вартости. І посідач
грошей находить на ринку такий специфічний товар: це здатність
до праці, або робоча сила.

Під робочою силою або здатністю до праці ми розуміємо
сукупність фізичних та інтелектуальних здібностей, які існують
в організмі, живій особистості людини, і що їх вона пускає в рух
щоразу, коли продукує якібудь споживні вартості.

касуються й сами собою зводяться на пересічну ціну як свою внутрішню
норму (Regel). Ця остання є провідна зірка, приміром, для купця або
промисловця в кожному підприємстві, яке функціонує довший час. Отже,
купець або промисловець знає, що, коли розглядати довший період у
цілості, товари дійсно продається не нижче і не вище, а за їхніми пересічними
цінами. Отже, коли б безстороннє мислення було взагалі в його інтересах,
то він мусив би поставити перед собою проблему утворення капіталу
ось як: як може постати капітал за реґулювання цін пересічною ціною,
тобто в останній інстанції вартістю товару? Я кажу «в останній інстанції»,
бо пересічні ціни не збігаються безпосередньо з величиною вартости токарів,
як то гадають А. Сміс, Рікардо й інші.