до самої себе. Вона відрізняє себе як первісну вартість від самої
себе як додаткової вартости, так само як бог-отець одрізняє
себе від самого себе як бога-сина, і обидві вони одного віку
і становлять фактично лише одну особу, бо лише через додаткову
вартість у 10 фунтів стерлінґів авансовані 100 фунтів стерлінґів
стають капіталом, і скоро тільки вони ним стали, скоро тільки
народився син, а через сина й отець, знову зникає їхня ріжниця
і обидва вони є єдине: 110 фунтів стерлінґів.

Отже, вартість стає вартістю, що процесує, грішми, що процесують,
і як така — капіталом. Вона виходить із циркуляції,
вступає знову до неї, зберігається й помножується в ній, повертається
назад із неї збільшеною, і завжди знову й знов починає той
самий кругобіг.\footnote{
«Капітал... вартість, що невпинно помножується» (Sismondi: «Nouveaux
Principes d’Economie Politique», vol. I, p. 89).
} Г — Г', гроші, що плодять гроші — money which
begets money — так звучить описання капіталу в устах його
перших тлумачів, меркантилістів.

Купити, щоб продати, або точніше — купити, щоб дорожче
продати, Г — Т — Г', здається, правда, формою, що властива
лише одному родові капіталу, купецькому капіталові. Але й
промисловий капітал є гроші, що перетворюються на товар і
через продаж товару перетворюються знов на більшу кількість
грошей. Акти, що відбуваються між купівлею і продажем поза
сферою циркуляції, нічого не змінюють у цій формі руху. Нарешті,
в капіталі, що приносить проценти, циркуляція Г — Т — Г'
виступає у скороченій формі, в своєму результаті без посередніх
ланок, так би мовити, у ляпідарному стилі, як Г — Г' гроші, що
дорівнюють більшій кількості грошей, вартість, що є більша за
саму себе.

Отже, в дійсності Г — Т — Г' є загальна формула капіталу,
як він безпосередньо з’являється у сфері циркуляції.

2. Суперечності загальної формули

Форма циркуляції, що в ній гроші перетворюються на капітал,
суперечить усім раніш розвинутим законам про природу товару,
вартости, грошей і самої циркуляції. Від простої товарової
циркуляції її відрізняє зворотна послідовність тих самих двох
протилежних процесів, продажу й купівлі. Але яким чином
ця суто формальна ріжниця могла спричинити такі магічні переміни
в природі цього процесу?

Навіть більше: ця зворотна послідовність існує лише для
одного з тих трьох ділових приятелів, що ведуть між собою
справу. Як капіталіст я купую товар в А і перепродую його В,
а як простий посідач товарів я продаю товар В і потім купую
товар в А. Для ділових приятелів А і В цієї ріжниці не існує.
Вони виступають лише як покупці або продавці товарів. Я сам
щоразу виступаю проти них як простий посідач грошей або посі-