суканої шерсти та шовкових фабриках, виявилося, що на певному
ступені розвитку надзвичайне поширення фабричних галузей
може сполучатися не тільки з відносним, але й з абсолютним
зменшенням числа вживаних робітників. 1860 р., коли з наказу
парляменту розпочато спеціяльний перепис усіх фабрик Об’єднаного
Королівства, налічувалося 652 фабрики в тій частині
фабричних округ Ланкашіру, Чешіру та Йоркшіру, яку доручено
фабричному інспекторові Р. Бекерові; з цих фабрик 570 мали:
парових ткацьких варстатів — 85.622, веретен (за винятком веретен
на сукання) — 6.819.146, кінських сил у парових машинах —
27.439, у водяних колесах — 1.390, занятих осіб — 94.119.
Навпаки, 1865 р. на цих самих фабриках було: ткацьких варстатів
— 95.163, веретен — 7.025.031, кінських сил у парових машинах
— 28.925, у водяних колесах — 1.445, занятих осіб —
88.913. Отже, від 1860 р. до 1865 р. зріст цих фабрик становив
у парових ткацьких варстатах 11\%, у веретенах — 3\%, у парових
кінських силах — 5\%, тимчасом як число занятих осіб за той
самий період зменшилося на 5,5\%.\footnote{
«Reports of Insp. of Fact. for 31 st October 1865», p. 58 і далі.
Але одночасно було дано вже й матеріяльну базу для вживання чимраз
більшого числа робітників: було засновано 110 нових фабрик з 11.625 паровими
ткацькими варстатами, 628.756 веретенами, 2.695 паровими й водяними
кінськими силами (Там же).
} Між 1852 і 1862 рр. сталося
значне збільшення англійської вовняної фабрикації, тимчасом як
число вживаних робітників лишилося майже без змін. «Це показує,
в якій великій мірі новозаведені машини витиснули працю

і як мало він розуміє рух продукції, а все ж він принаймні почуває,
що машини дуже фатальна інституція, скоро заведення їх перетворює
занятих робітників на павперів, тимчасом як розвиток їх покликає до
життя більше рабів праці, ніж вони були вбили. Кретинізм його власного
погляду можна висловити лише його власними словами: «Les classes
condamnées à produire et à consommer diminuent, et les classes qui dirigent
le travail, qui soulagent, consolent et éclairent toute la population,
se multiplient... et s’approprient tous les bienfaits qui résultent de la diminution
des frais du travail, de l’abondance des productions et du bon marché
des consommations. Dans cette direction, l’espèce humaine s’élève aux plus
hautes conceptions du génie, pénètre dans les profondeurs mystérieuses de la
religion, établit les principes salutaires de la morale (яка є в тому, щоб «s’approprier
tous les bienfaits etc.»), les lois tutélaires de la liberté (liberté pour
«les classes condamnées à produire»?) et du pouvoir, de l’obéissance et de
la justice, du devoir et de l’humanité». («Кляси, засуджені на продукцію
та споживання, зменшуються, а кляси, що керують працею, дають поміч,
утіху та освіту цілому народові, збільшуються... та присвоюють собі
всі блага, що є результат зменшення витрат праці, рясности продуктів
та подешевшання предметів споживання. В цьому напрямі людський
рід підноситься до найвищих концепцій генія, доходить таємних глибин
релігії, встановлює спасенні принципи моралі (яка є в тому, щоб «присвоювати
собі всі блага й т. ін.»), закони до охорони волі (волі для «кляс,
засуджених на продукцію»?) та влади, покірливости та справедливости,
обов’язку та гуманности»). Ці теревені маємо в «Des Systèmes d’Economie
Politique etc. Par M. Ch. Ganilh». 2 ème ed. Paris 1821, vol. I, p. 224.
Порівн. там же, стор. 212.