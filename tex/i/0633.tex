3. Криваве законодавство проти експропрійованих, починаючи
в кінця XV століття. Закони для зниження заробітної плати

Вигнаних у наслідок розпуску февдальних дружин і ґвалтовної,
переводжуваної поштовхами експропріяції земель, цих
вільних, як птиці, пролетарів, мануфактура, що тоді поставала,
ніяк не могла поглинути так само швидко, як вони з’являлися
на світ. З другого боку, ці люди, раптово вибиті з їхньої звичайної
життєвої колії, не могли так само раптом призвичаїтись до
дисципліни нових обставин. Вони масами перетворювалися на
жебраків, розбійників, волоцюг, почасти з власного нахилу,
але здебільшого під примусом обставин. Звідси криваве законодавство
проти волоцюзтва по всіх країнах Західньої Европи
наприкінці XV і протягом усього XVI століття. Батьків теперішньої
робітничої кляси покарано насамперед за те, що їх перетворено
на волоцюг і павперів. Законодавство розглядало
їх як «добровільних» злочинців і виходило з того припущення,
що від їхньої доброї волі залежить і далі працювати серед старих
обставин, які вже не існували.

В Англії це законодавство почалося за Генріха VII.

Генріх VIII, 1530: старі й нездатні до праці жебраки дістають
дозвіл жебракувати. Навпаки, працездатних волоцюг слід карати
батогами й замикати до в’язниць. Їх слід прив’язувати ззаду до
тачки й катувати, доки потече з їхнього тіла кров, а потім узяти
від них присягу, що вони повернуться туди, де народились,
або туди, де перебували останні три роки, і «візьмуться до роботи»
(to put himself to labour). Яка жорстока іронія! Акт 27
Генріха VIII повторює цей закон та загострює його новими додатками:
якщо когобудь удруге зловлять на волоцюзтві, то його
треба знову покарати батогами та відрізати йому піввуха; а
коли кого утретє зловлять на волоцюзтві, то його, як тяжкого
злочинця й ворога громадянства, треба покарати на смерть.

Едвард VI: статут першого року його королювання, 1547,
приписує віддавати кожного, хто ухиляється від праці, у рабство
тій особі, що донесе на нього як на нероба. Хазяїн повинен годувати
свого раба хлібом і водою, давати йому легкі напої й такі
м’ясні покидьки, які він вважатиме за відповідні. Він має право
батогами й кайданами силувати його до всякої, навіть найогиднішої
праці. Коли раб відлучиться на два тижні, то його слід
засудити на довічне рабство й наложити на його лоб або щоку
тавро «S»; коли ж він утече втретє, то його слід покарати на
смерть як зрадника держави. Хазяїн може його продати, відписати
у спадщину, віддати в найми, як раба, цілком так само,

джерел продукції втрачає країна в наслідок цього насильного спустошення,
видно з того, що лісова площа Ben Aulder могла б прогодувати
15.000 овець, і що площа цього лісу становить лише 1/30 частину всієї
мисливської площі Шотляндії... Вся ця мислівська земля є цілком непродуктивна...
з неї така сама користь, як коли б її затопити у хвилях
Північного моря. Міцна рука законодавства мусіла б покласти край цим
імпровізованим пустелям».
