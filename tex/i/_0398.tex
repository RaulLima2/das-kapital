\parcont{}  %% абзац починається на попередній сторінці
\index{i}{0398}  %% посилання на сторінку оригінального видання
спільно, пічок, будівель тощо, отже, одно слово, більша концентрація
засобів продукції й відповідно до цього більша конґльомерація
робітників. Головне заперечення, палко повторюване
кожною мануфактурою, якій загрожувало заведення фабричного
закону, сходило власне на те, що потрібно буде більших витрат
капіталу, щоб провадити далі підприємство в його старому обсязі...
Щождо проміжних форм між мануфактурою й домашньою працею
і щодо самої домашньої праці, то вони з обмеженням робочого
дня й дитячої праці втрачають під собою ґрунт. Безмежна експлуатація
дешевих робочих сил становить одним-одну основу їхньої
конкурентоспроможности.

Посутньою умовою фабричного виробництва, особливо від того
часу, як воно підпадає реґулюванню робочого дня, є нормальна
забезпеченість результату, тобто продукція певної кількости товару
або досягнення бажаного корисного ефекту протягом даного
часу. Далі, приписані законом перерви реґульованого робочого
дня мають за передумову можливість раптом і періодично припиняти
роботу без шкоди для виробу, що перебуває в процесі продукції.
Цієї забезпеченості результату і можливости припиняти
роботу, певна річ, легше можна досягти в суто механічних виробництвах,
аніж там, де певну ролю відіграють хемічні й фізичні
процеси, як от, приміром, у ганчарстві, білильництві, фарбівництві,
пекарстві та в більшості металевих мануфактур. Там, де
існує рутина необмеженого робочого дня, нічної праці й вільного
марнування людського життя, кожну стихійну перешкоду легко
вважається за вічну «природну межу» продукції. Ніяка отрута
не нищить поганих комах так певно, як фабричний закон нищить
такі «природні межі». Ніхто не кричав про «неможливості» так
голосно, як пани з ганчарства. 1864 р. їм октройовано фабричний
закон, і вже по 16 місяцях усі неможливості зникли. Викликані
фабричним законом «поліпшені методи виготовляти ганчарну
масу (slip) за допомогою тиску замість випарювання, нова конструкція
печей на сушіння невипаленого товару й т. ін. — це
події великої ваги для ганчарства, вони означають такий проґрес
у ганчарній справі, якого не можна знайти в останньому столітті...
Температуру печей значно знижено при значному ж зменшенні
споживання вугілля й швидкому діянні на ганчарну масу».\footnote{
«Keports of Insp. of Fact, for 31 st October 1865», p. 96, 127.
}
Всупереч усім пророкуванням зросла не ціна витрат на глиняні
вироби, а маса продукту, так що вивіз за 12 місяців, від грудня
1864 р. до грудня 1865 р., дав надлишок вартости в 138.628 фунтів
стерлінґів проти пересічного вивозу за три попередні роки. У фабрикації
сірників вважали за природний закон, що підлітки,
навіть тоді, коли нашвидку ковтали свій обід, мусилй вмочувати
патички сірників у теплу фосфорову композицію, отруйна пара
якої била їм у лице. Разом із доконечністю економізувати час
фабричний акт (1864 р.) примусив завести «dipping machine»
(машину для вмочування сірників у фосфор), з якої пара не може
\parbreak{}  %% абзац продовжується на наступній сторінці
