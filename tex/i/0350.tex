нервову систему, машинова праця пригнічує багатобічну гру
мускулів і відбирає всяку змогу вільної фізичної й інтелектуальної
діяльности.187 Навіть полегшення праці робиться засобом тортур,
бо машина не визволяє робітника від праці, а відбирає його
праці зміст. Всякій капіталістичній продукції, оскільки вона є
не тільки процес праці, але й процес зростання вартости капіталу,
є спільне те, що не робітник вживає умов праці, а, навпаки,
умови праці вживають робітника, але тільки при машиновій
системі це перекручення набуває технічно-очевидної реальности.
В наслідок свого перетворення на автомат засіб праці підчас
самого процесу праці протистоїть робітникові як капітал,
як мертва праця, що опановує й висисає живу робочу силу. Відокремлення
інтелектуальних сил процесу продукції від ручної
праці та перетворення їх у владу капіталу над працею завершується,
як ми вже раніш казали, у великій промисловості, побудованій
на основі машин. Частинна вправність індивідуального,
спустошеного машинового робітника зникає як незначна другорядна
річ перед наукою, перед велетенськими силами природи
і перед суспільною масовою працею, що втілені в системі машин
та разом з нею становлять владу «хазяїна» (master). Тому цей хазяїн,
у мозку якого машини нероздільно зрослися з його монополією
на них, у випадках колізії вигукує зневажливо до «рук»:
«Хай фабричні робітники в своїх власних інтересах запам’ятають,
що їхня праця в дійсності є дуже низький сорт навченої
праці; що жодної іншої праці не можна легше вивчити та що,
зважаючи на її якість, жодної праці не оплачується ліпше; що
жодної іншої праці не можна придбати за такий короткий час та
в такому великому розмірі, сяк-так привчивши найменш досвідчених
осіб. Машини хазяїна відіграють у дійсності далеко важливішу
ролю в справі продукції, ніж праця і вправність робітника,
яких можна навчитися за шість місяців і яких може навчитися
кожен сільський наймит». 188

даючи щодня 15 годин за одноманітним рухом машини, виснажується
швидше, ніж коли вона протягом того самого часу працює фізично. Ця праця
догляду, яка, може, могла б бути за корисну гімнастику для розуму,
коли б не тривала надто довго, своєю надмірністю руйнує кінець-кінцем
і розум і саме тіло». («Un homme s’use plus vite en surveillant quinze
heures par jour l’évolution uniforme d’un mécanisme, qu’en exerçant dans
le même espace de temps, sa force physique. Ce travail de surveillance,
qui servirait peut - être d’utile gymnastique à l’intelligence, s’il n’etait pas
trop prolongé, détruit à la longue, par son excès, et l’intelligence et le corps
même»). (G. de Molinari: «Etudes Economiques», Paris 1846, p. 49).

187    F. Engels, там же, стор. 216. (Партвидав «Пролетар», 1932 р.,
стор. 186).

188 «The factory operatives should keep in wholesome remembrance
the fact that theirs is really a low species of skilled labour; and that there
is none which is more easily acquired or of its quality more amply remunerated,
or which, by a short training of the least expert can be more quickly
as well as abundantly acquired... The master’s machinery really plays
a far more important part in the business of production than the labour
and the skill of the operative, which six months education can teach, and
