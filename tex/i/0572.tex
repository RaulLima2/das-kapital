от будування залізниць тощо, підприємець сам здебільшого постачає
своїй армії дерев’яні курені й т. ін., імпровізовані селища
що не мають ніяких гігієнічних засобів, не підлягають ніякому
контролеві місцевої влади, але дуже вигідні для пана підприємця,
що подвійно визискує робітників — як промислових солдатів
і як квартирантів. Залежно від того, скільки дір має курінь —
одну, дві чи три, мешканцеві, тобто копальникові й т. ін., доводиться
платити на тиждень 1, 3, 4, шилінґи.130 Досить буде одного
прикладу. У вересні 1864 р., — повідомляє д-р Сімон, — міністер
внутрішніх справ сер Джордж Ґрей одержав такого звіта
від голови Nuisance Removal Committee\footnote*{
Комітет у справі боротьби в антисанітарними умовами. Ред.
} в парафії Sevenoaks:
«Іще 12 місяців тому віспа в цій парафії була цілком невідома.
Незадовго перед цим почалися роботи коло будови залізниці
від Lewisham до Tunbridge. Опріч того, що головні роботи провадились
у безпосередньому сусідстві з цим містом, тут ще й
улаштовано головне депо цілого підприємства. Тому тут працювало
багато робітників. Через те, що неможливо було помістити
їх усіх у котеджах, то підприємець, пан Джей, побудував уздовж
залізничної колії на різних пунктах курені для житла робітників.
Ці курені не мали жодної вентиляції, ані зливів на нечисть;
крім того, вони з доконечности були надмірно переповнені,
бо кожний наймач мусив приймати інших мешканців, хоч би
яка численна була його власна родина, і хоч у кожному курені
було лише дві кімнати. За лікарським звітом, що ми його одержали,
наслідок був той, що ці бідолахи мусили ночами зносити
всі муки задухи, щоб захистити себе від заразних випарів з брудних
калюж і кльозетів, що були зараз же під вікнами. Нарешті,
один лікар, що мав нагоду відвідати ці курені, передав нашому
комітетові скаргу. В якнайгіркіших висловах говорив він про
стан цих так званих помешкань і побоювався дуже серйозних
наслідків, якщо не вживеться деяких санітарних заходів. Приблизно
рік перед тим згаданий Джей зобов’язався збудувати дім,
куди негайно мали ізолювати занятих у нього робітників, скоро
вони захоріють на заразливі недуги. Наприкінці останнього
липня він повторив цю обіцянку, але не зробив найменшого
кроку, щоб виконати її, дарма що від того часу трапилось кілька
випадків віспи і двоє чоловіка від неї померло. 9 вересня лікар
Келсон повідомив мене про нові випадки віспи в цих куренях,
змальовуючи їхній стан як жахний. Для вашої (міністра) інформації
мушу я додати, що в нашій парафії є ізольований дім, так
званий пошесний дім, де ходять за парафіянами, недужими на
заразливі хороби. Цей дім ось уже кілька місяців постійно переповнений
пацієнтами. В одній родині померло п’ятеро дітей від
віспи і пропасниці. Від 1 квітня до 1 вересня цього року трапилось
не менш як 10 смертних випадків од віспи, 4 з них у згаданих
куренях, у цих джерелах пошестей. Подати число занеду-

150 Там же, стор. 165.