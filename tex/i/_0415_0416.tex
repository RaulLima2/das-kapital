\parcont{}  %% абзац починається на попередній сторінці
\index{i}{0415}  %% посилання на сторінку оригінального видання
інспекторам, і одним махом збільшив таким чином сферу контролю
цих інспекторів більш ніж на 100.000 майстерень, — на 300 самих
тільки цегелень, — персонал інспекторів з винятковою дбайливістю
збільшили всього на вісім помічників, дарма що й перед
тим він був надто малий.\footnote{
Персонал фабричної інспекції складався з 2 інспекторів, 2 помічників
і 41 субінспекторів. Нових 8 субінспекторів призначено 1871 р.
Загальна сума видатків на проведення фабричних законів в Англії,
Шотландії й Ірландії становила 1871--1872 рр. лише 25.347 фунтів стерлінґів,
включаючи й судові витрати на процеси проти порушень закону.
}

Отже, в цьому англійському законодавстві 1867 р. вражає,
з одного боку, накинута парляментові панівних кляс доконечність
у принципі погодитись на такі надзвичайні й широкі заходи
проти ексцесів капіталістичної експлуатації, а з другого боку,
половинчастість, неохота і mala fides,\footnote*{
— нечесність. \emph{Ред.}
} з якими парлямент
потім дійсно проводив у життя ці заходи.

Слідча комісія 1862 р. також запропонувала нову реґляментацію
гірничої промисловости, промисловости, яка від усіх інших
відрізняється тим, що в ній інтереси землевласників і промислових
капіталістів ідуть пліч-о-пліч. Протилежність цих двох
груп інтересів сприяла фабричному законодавству; відсутности
цієї протилежности досить для того, щоб пояснити проволікання
та викрути щодо гірничого законодавства.

Слідча комісія 1840 р. зробила такі жахливі і обурливі викриття
і викликала такий скандал на всю Европу, що парлямент
мусив рятувати своє сумління Mining Act’oм 1842 р., в якому
він обмежився забороною праці під землею для жінок і дітей,
молодших від 10 років.

Потім 1860 р. видано Mines’ Inspection Act,\footnote*{
— закон про інспекцію над копальнями. \emph{Ред.}
} що згідно з ним
спеціяльно призначені державні урядовці мають наглядати за гірничими
підприємствами, і дітей між 10 і 12 роками не можна в
тих підприємствах вживати до праці, якщо вони не мають шкільного
посвідчення або не відвідують школи протягом певного
числа годин. Цей закон лишився цілком мертвою буквою через
на сміх мале число призначених інспекторів, мізерність їхніх
уповноважень та інші причини, які докладніше з’ясується в
дальшому викладі.

Одна з найновіших Синіх Книг про гірничі підприємства —
це «Report from the Select Committee on Mines, together with\dots{}
Evidence, 23 July 1866». Це — праця комітету, складеного з
членів нижньої палати й уповноваженого закликати й вислухувати
свідків; товстий том in folio, де сам «Report» має всього
лише п’ять рядків такого змісту: комітет нічого не може сказати,
треба переслухати ще більше свідків!

Спосіб допиту свідків нагадує cross examinations\footnote*{
— перехресний допит. \emph{Ред.}
} по англійських
судах, де адвокат безсоромними і заплутаними перехресними
\index{i}{0416}  %% посилання на сторінку оригінального видання
запитами намагається спантеличити свідка й перекрутити
його слова. Тут у ролі адвокатів виступають сами члени парламентської
слідчої комісії, між ними власники й експлуататори копалень;
свідки тут робітники копалень, здебільша кам’яновугільних.
Ця ціла фарса надто характеристична для духу капіталу, так що
не можна не подати тут декілька витягів. Для легшого огляду
я подаю результати слідства й т. ін. за рубриками. Нагадую, що
питання і обов’язкові відповіді в англійських Синіх Книгах
нумеровані, і що свідки, чиї свідчення тут цитується, є робітники
кам’яновугільних копалень.

1. Праця дітей від 10 років по копальнях. Праця разом з
неминучим ходінням від і до копалень триває звичайно 14 —
15 годин, винятково довше, від 3, 4, 5 години ранку до 4--5 години
вечора (№№ 6, 452, 83). Дорослі робітники працюють двома
змінами, або по 8 годин, але для підлітків, щоб заощадити на
видатках, такої зміни нема (№№ 80, 203, 204). Малих дітей уживають
головно щоб відчиняти й зачиняти двері в різних відділах
копальні, а старших дітей — до тяжкої роботи: перевозити вугілля
й т. ін. (№№ 122, 739, 1747). Довгий робочий день під землею
триває до 18 або 22 року життя, коли відбувається перехід до
власне копальневої праці (№ 161). Дітей і підлітків тепер тяжче
мордують працею, ніж колибудь у попередні часи (№№ 1663 —
67). Копальневі робітники майже одноголосно вимагають парляментського
закону про заборону копальневої праці для дітей,
молодших за 14 років. Але ось Гессей Вівіян (сам експлуататор
копальні) питає: «Чи не залежить ця вимога від більших або
менших злиднів батьків?» — А містер Брюс: «Чи це не жорстоко,
якщо батько помер або покалічений тощо, відбирати в родини
цей ресурс? Адже ця заборона мусить мати силу, як загальне
правило. Чи хочете ви підземну працю дітей до 14 років заборонити
в усіх випадках?» Відповідь: «В усіх випадках» (№№ 107
до 110). Вівіян: «А якщо працю дітей до 14 років по копальнях
заборонять, то чи не посилатимуть батьки дітей на фабрики
тощо? — Як правило, ні» (№ 174). Робітник: «Відчиняти й зачиняти
двері, здається, легко. Але це виснажна праця. Не кажучи
вже про постійний протяг, дитина сидить там немов у в’язниці,
цілком так, наче в темній тюремній камері». Буржуа Вівіян:
«А не може дитина, вартуючи при дверях, читати, якщо вона
матиме світло? — ІІоперше, вона мусила б купити собі свічку.
Але, крім того, їй цього і не дозволили б. Її поставили, щоб пильнувала
справи, вона має виконувати певний обов’язок. Я ніколи
не бачив, щоб якабудь дитина читала в копальні» (№№ 141--160).

2. Виховання. Копальневі робітники вимагають закона про
обов’язкове навчання дітей, як на фабриках. Вони заявляють,
що той пункт закону 1860 р., який вимагає шкільної посвідки для
того, щоб вживати до праці дітей 10--12 років, є чисто ілюзоричний.
«Педантична» процедура допитування капіталістичними
слідчими стає тут справді забавною. (№ 115). «Чи цей закон більше
потрібний проти підприємців, чи проти батьків? — Проти тих
\parbreak{}  %% абзац продовжується на наступній сторінці
