що дають змогу капіталістові надовго здовжити робочий день,
дають йому спочатку змогу, а, кінець-кінцем, і примушують
його знизити ціну праці й номінально, поки не знизиться ціла ціна
збільшеного числа годин, отже, поденна або потижнева плата.
Нам досить тут буде зазначити дві обставини. Коли одна людина
робить роботу 1 або 2 людей, то подання праці зростає, якщо
навіть подання робочих сил, що є на ринку, і лишається стале.
Конкуренція, створена таким чином серед робітників, дає змогу
капіталістові знижувати ціну праці, і, навпаки, падіння ціни
праці дає йому змогу ще дужче збільшити робочий час.42 Однак
незабаром ця змога порядкувати ненормальною, тобто такою,
що перевищує пересічний суспільний рівень, кількістю неоплаченої
праці стає засобом конкуренції між самими капіталістами.
Частина ціни товару складається з ціни праці. Неоплачену частину
ціни праці немає потреби зараховувати в ціну товару.
Її можна подарувати покупцеві товару. Це — той перший крок,
до якого веде конкуренція. Другий крок, до якого вона примушує,
— це те, щоб із продажної ціни товару виключити також,
принаймні, якусь частину ненормальної додаткової вартости, створеної
через здовження робочого дня. Таким способом утворюється
спочатку спорадично, а потім поступінно фіксується ненормально
низка продажна ціна товару, яка відтепер стає постійною
основоью мізерної заробітної плати при надмірному
робочому часі, так само як вона первісно була продуктом цих
обставин. Ми лише коротко відзначаємо цей рух, бо аналіза конкуренції
тут не є наше завдання. Однак даймо на хвилину слова
самому капіталістові. «В Бермінґемі конкуренція між хазяїнами
така велика, що дехто з нас як підприємець примушений робити
таке, що він посоромився б зробити за інших умов; а проте й
таким способом не добудеш більше грошей (and yet no more
monev is made), a тільки сама публіка має з того користь».43
Пригадаймо собі два сорти лондонських пекарів, що з них одні
продають хліб за повну ціну (the «fullpriced» bakers), а другі
продають його нижче за його нормальну ціну («the underpri-

час мусить він тяжко працювати, щоб здобути 11 пенсів, або 1 шилінґ,
та ще з того 2\sfrac{1}{2}—3 пенси відпадає на зужиткування знаряддя, на паливо,
відпадки заліза». («Children’s Employaient Commission. З rd. Report», p.
136, n. 671). За той самий робочий час жінки заробляють щотижня
лише 5 шилінґів (там же, стор. 137, п. 674).

42    Коли б якийсь фабричний робітник відмовився, наприклад, працювати
встановлене велике число годин, «його одразу замінили б кимось
іншим, готовим працювати скільки завгодно, і він лишився б без праці»
(«he would very shortly be replaced by somebody who would work any
length of time and thus be tbrown out of employment»). («Reports of Insp.
of Fact. for. 31 st October 1848». Evidence p. 39, n. 58). «Якщо одна людина
виконує працю двох... норма зиску звичайно підвищується... бо додаткове
подання праці знижує її ціну» («If one man performs the work of
two... the rate of profits will generally be raised... in conséquence of the
additional supply of labour having diminished its price»). («Senior:
«Three Lectures on the Rate of Wages», London 1830. p. 14).

43 «Children’s Employment Commission, 3 rd Report». Evidence, p. 66, p. 22.
