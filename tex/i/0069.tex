товару іншим разом з тим призводить до того, що в руках третьої
особи лишається товар - гроші.72 Циркуляція завжди стікає грошовим
потом.

Нема нічого безглуздішого, як та догма, ніби циркуляція
товарів доконечно зумовлює рівновагу продажів і купівель, з
тієї причини, що кожний продаж є купівля й vice versa. * Коли
цим хочуть сказати, що число дійсно проведених продажів дорівнює
такому самому числу купівель, то це пласка тавтологія.
Але цією догмою хочуть довести, що продавець веде за собою
на ринок свого покупця. Продаж і купівля є тотожний акт як
взаємне відношення між двома полярно протилежними особами:
посідачем товарів і посідачем грошей. Вони становлять два полярно
протилежні акти як учинки тієї самої особи. Тим то тотожність
продажу й купівлі містить у собі й те, що товар стає некорисним,
коли він, кинутий в альхемічну реторту циркуляції, не
виходить із неї у формі грошей, коли посідач товарів його не продасть,
отже, коли посідач грошей його не купить. Ця тотожність
містить у собі далі те, що цей процес, якщо він удасться, становить
певну павзу, певний період в житті товару, який може тривати
довше або коротше. Через те, що перша метаморфоза товару є
одночасно продаж і купівля, цей частинний процес є разом з
тим самостійний процес. Покупець має товар, продавець має
гроші, тобто товар, який зберігає форму, що робить його здатним
до циркуляції, незалежно від того, чи він раніше або пізніше
знову з’явиться на ринку. Ніхто не може продати без того, щоб
хтось інший не купив. Але ніхто не потребує негайно купувати
через те тільки, що він сам щось продав. Циркуляція товарів
ламає часові, місцеві й індивідуальні межі обміну продуктів
саме тим, що наявну тут безпосередню тотожність між відчуженням
через обмін власного продукту праці й привласненням
чужого вона розколює на два протилежні акти — продаж і купівлю.
А що ці самостійні один проти одного процеси становлять
унутрішню єдність, — то це так само говорить і про те, що їхня
внутрішня єдність рухається в зовнішніх протилежностях. **
Коли зовнішнє усамостійнення внутрішньо несамостійних актів, —
бо ж вони доповнюють один одного, — доходить до якогось певного
пункту, то єдність їхня проявляється ґвалтовно — через
кризу. Іманентна товарові протилежність споживної вартости й
вартости, приватної праці, що разом з тим мусить з’являтися
як безпосередньо суспільна праця, осібної конкретної праці,
що разом з тим має значення тільки абстрактної загальної праці,
між персоніфікацією речей і зречевленням осіб, — ця іманентна

72    Примітка до другого видання. Хоч і як впадає на очі не явище,
однак політико-економи здебільша не помічають його, особливо ж вульгарні
прихильники вільної торговлі.

* — навпаки. Ред.

** У французькому виданні це речення подано так: «Правда, що купівля
є доконечне доповнення продажу, але не менш справедливо, що
їхня єдність є єдність протилежностей». («Le Capital etc.», v. I, eh. Ill, p. 47).
Ред.
