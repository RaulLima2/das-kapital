бітника не може бути такого вільного часу, без такого надлишкового
часу не може бути додаткової праці, а тому й капіталістів,
але також не може бути й рабовласників, февдальних баронів,
одне слово, жодної кляси великих власників. 1

Таким чином можна говорити про природну базу додаткової
вартости, але лише в тому цілком загальному розумінні, що в
природі немає жодної абсолютної перешкоди, яка б не дозволяла
одній людині звалювати з себе на іншу людину працю, потрібну
для її власного існування, наприклад, так само, як у природі
не існує жодних абсолютних перешкод для того, щоб одна людина
вживала для харчування м’яса іншої.1а Ні в якому разі не
слід, як це іноді робилося, сполучати містичні уявлення з
цим стихійним розвитком продуктивности праці. Тільки тоді,
коли люди тяжкою працею вибилися з свого первісного тваринного
стану, отже, коли сама їхня праця до деякої міри вже є
усуспільнена, — лише тоді постають відносини, за яких додаткова
праця однієї людини стає умовою існування іншої. На початках
культури здобуті продуктивні сили праці незначні, але так само
незначні й потреби, що розвиваються разом з розвитком засобів
для задоволення тих потреб та залежно від цього розвитку. Далі,
на тих початках культури частина суспільства, що живе з чужої
праці, є величина зникомо мала супроти маси безпосередніх продуцентів.
З проґресом суспільної продуктивної сили праці ця
частина зростає абсолютно й відносно.2 Зрештою, капіталістичне
відношення постає на економічному ґрунті, який є продукт
довгого процесу розвитку. Наявна продуктивність праці,
з якої воно виходить як з основи, не є дар природи, а дар історії,
яка охоплює тисячі століть.

Якщо абстрагуватися від більш або менш розвиненої форми
суспільної продукції, то залишається, що продуктивність праці
зв’язана з природними умовами. Всі ці умови можна звести до
природи самої людини, як от раса й т. ін., та до природи, що оточує
людину. Зовнішні природні умови розпадаються з економічного
погляду на дві великі кляси: природне багатство на засоби

1 «Саме існування капіталістичних підприємців як осібної кляси
залежить від продуктивности праці» («The very existence of the master-capitalists
as a distinct class is dependent on the productiveness
of industry») - (Ramsay: «An Essay on the Distribution of Wealth»,
Edinburgh 1836, p. 206). «Коли б праці кожної людини вистачало лише
для продукції її власних засобів існування, то не могло б бути й власности»
(«If each man’s labour were but enough to produce his own food,
there could be no property»). (Ravenstone: «Thoughts on the Funding
System», London 1824, p. 14, 15).

1а На основі нещодавно зробленого обчислення лише в досліджених
уже частинах землі живе ще щонайменше чотири мільйони канібалів.

* У диких індіян Америки мало не все належить робітникові, 99\%
продуктів припадає робітникові; в Англії на робітника не припадає й
двох третин» («Among the wild Indians in America, almost every thing
is the labourer’s, 99 parts of an hundred are to be put upon the account of
Labour; In England, perhaps the labourer has not \sfrac{2}{3}»). («The Advantages
of the East-India Trade etc.», London 1720, p. 73).
