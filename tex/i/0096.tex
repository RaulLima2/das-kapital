біги репродукції, ґрунтуються на природних умовах продукції,
пов’язаних із зміною пір року. Ці загальні терміни регулюють
так само і ті платежі, що не випливають безпосередньо з товарової
циркуляції, як ось податки, ренти й т. ін. Маса грошей, потрібна
у певні дні року для цих розпорошених по цілій поверхні суспільства
платежів, спричинює періодичні, але цілком поверхові пертурбації
в економії засобів платежу.106 Із законів швидкости
обігу засобів платежу випливає, що маса засобів платежу, доконечна
для всіх періодичних платежів, хоч які були б їх джерела,
стоїть у прямому відношенні ** до протягу періодів платежу. 107

106 «У клечальний понеділок 1824 р., — оповідає пан Креґ парламентській
слідчій комісії з 1826 р., — в Едінбурзі був такий величезний попит
на банкноти, що на 11 годину ми не мали вже й жодної банкноти в нашому
розпорядженні. Ми звертались по черзі до різних банків, щоб позичити
їх, але не мали змоги одержати ні одної, і багато оборудок довелося перевести
лише за допомогою slips of paper.* Однак уже о 3 годині опівдні
всі банкноти повернулись до тих банків, звідки вони вийшли. Вони змінили
лише декілька рук». Хоч пересічне число банкнот, які дійсно циркулюють
у Шотляндії, становить менш, ніж 3 мільйони фунтів стерлінґів,
все ж у певні дні року, коли настає термін різних платежів, до операцій
притягаються всі банкноти, що їх мають банкіри, всього на 7 мільйонів
фунтів стерлінґів. За цієї нагоди банкноти мають виконати однимодну
й специфічну функцію, і, виконавши її, вони повертаються назад
до тих банків, звідки вийшли. (John Fullarton: «Regulation of Currencies»,
2 nd ed. London 1845, p. 86, примітка). На пояснення додамо, що в Шотляндії
в час виходу твору Фуляртона вклади видавано не чеками, а лише
банкнотами.

107 На питання: «Коли б довелося протягом року виплатити 40 мільйонів,
то чи вистачило б цих 6 мільйонів (золотом) для циркуляції та оборотів,
яких вимагала б у даному разі торговля?» — Петті відповідає з
властивою йому майстерністю: «Я відповідаю: так; коли б обороти відбувались
у короткі переміжки часу, приміром, щотижня, як це буває
серед бідних ремісників і робітників, що одержують плату щосуботи, то
для переведення виплат на 40 мільйонів було б досить 40/52 мільйона.
А коли б обороти відбувалися щокварталу, як то звичайно в нас буває
за виплати ренти й податків, то було б потрібно 10 мільйонів. Отже, коли
ми припустимо, що загалом платежі робиться через різні переміжки часу,
між одним тижнем і 13, то тоді треба скласти 10 мільйонів і 40/52 мільйони
і взяти половину, яка дорівнює 5 1/2 мільйонам, так що, коли б ми мали
5 1/2 мільйонів, то нам уже вистачило б грошей». (Auf die Frage «If there were
occasion to raise 40 millions p. a., whether the same 6 millions (Gold) would
suffice for such revolutions and circulations there of as trade requires?», antwortet
Petty mit seiner gewohnten Meisterschaft: «I answer yes: for the expense
being 40 millions, if the revolutions were in such short circles, viz, weekly,
as happens among poor artizans and labourers, who receive and pay every
Saturday, then 40/52 parts of 1 million of money would answer these ends;
but if the circles be quarterly, according to our custom of paying rent, and
gahering taxes, then 10 millions were requisite. Wherefore supposing payments
in general to be of a mixed circle between one week and 13, then add
10 millions to 40/52, the half of the which will be 5 1/2, so as if we have
5 1/2 mill., we have enough»). (William Petty: «Political Anatomy of Ireland.
1672», ed. London 1691, p. 13, 14).

* — шматків паперу. Ред.

** У німецькому тексті тут замість «у прямому відношенні» сказано
«у зворотному відношенні» («in umgekehrtem Verhältniss»), що, очевидно,
є рукописна помилка. Ред.
