хронічної депресії. Сподіваний період розквіту не хоче надходити;
і кожного разу, коли нам здається, що ми бачимо провісні
симптоми, вони одразу ж розвіюються як дим. Тим часом  кожна
наступна зима знову ставить питання: «Що робити з безробітними?»
Але тим часом, як число безробітних рік-у-рік більшає,
немає нікого, хто відповів би на це питання; і ми могли б майже
обчислити час, коли цим безробітним урветься терпець і вони
візьмуть свою долю у свої власні руки. В такий момент, звичайно,
треба, щоб почувся голос людини, що її вся теорія є наслідок
ціложиттьового вивчання економічної історії та стану Англії,
людини, яку це вивчання довело до висновку, що, принаймні в
Европі, Англія є єдина країна, де неминуча соціальна революція
може бути переведена цілком мирними і леґальними засобами.
Звичайно, ця людина ніколи не забувала додати, що вона навряд
сподівається, щоб англійська панівна кляса підкорилась цій
мирній і леґальній революції без «proslavery rebellion» (повстання
на оборону рабства).

5 листопада 1886 р.

Фрідріх Енґельс
