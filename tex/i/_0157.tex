\parcont{}  %% абзац починається на попередній сторінці
\index{i}{0157}  %% посилання на сторінку оригінального видання
вартість елементів, спожитих на створення продукту, тобто понад вартість засобів продукції і робочої
сили.

Змалювавши ті різні ролі, які різні фактори процесу праці відіграють у творенні вартости продукту,
ми фактично схарактеризували функції різних складових частин капіталу в процесі його власного
зростання. Надлишок цілої вартости продукту понад суму вартости елементів, що брали участь у його
творенні, є надлишок вирослого в своїй вартості капіталу понад первісну авансовану капітальну
вартість. Засоби продукції на одному боці, робоча сила на другому — це лише різні форми існування,
що їх набрала первісна капітальна вартість, скидаючи з себе свою грошову форму і перетворюючись на
фактори процесу праці.

Отже, та частина капіталу, яка перетворюється на засоби продукції,
тобто на сировинний матеріял, допоміжний матеріял і засоби праці, не змінює в процесі продукції
величини своєї вартости. Тому я називаю її сталою частиною капіталу, або, коротше, сталим капіталом.

Навпаки, частина капіталу, перетворена на робочу силу, змінює свою вартість у процесі продукції.
Вона репродукує свій власний еквівалент і, крім того, надлишок, додаткову вартість, що сама може
змінюватись, бути більша або менша. Із сталої величини ця частина капіталу безупинно перетворюється
на величину
змінну. Тому я називаю її змінною частиною капіталу, або, коротше, змінним капіталом. Ті самі
складові частини капіталу, які з погляду процесу праці відрізняються як об’єктивні й суб’єктивні
фактори, як засоби продукції й робоча сила, з погляду процесу зростання вартости відрізняються як
сталий капітал і
змінний капітал.

Поняття сталого капіталу аж ніяк не виключає революцій у вартості його складових частин. Припустімо,
що фунт бавовни коштує сьогодні 6 пенсів, і що завтра через недостатній урожай
бавовни ціна його підноситься до 1 шилінґа. Стару бавовну, яку й далі обробляють, куплено за
вартість у 6 пенсів, але вона додає тепер до продукту частину вартости в 1 шилінґ. А випрядена вже
бавовна, що, може, циркулює вже на ринку як пряжа, так само прилучає до продукту величину вдвоє
більшу, ніж її первісна
вартість. Однак ми бачимо, що ці зміни вартости не залежать від зростання вартости бавовни в самому
процесі прядіння. Коли б стара бавовна зовсім не була ще увійшла в процес праці, то її
можна б було тепер знову продати по 1 шилінґу замість 6 пенсів. Навпаки, що менше процесів праці
вона перейшла, то певніший цей результат. Звідси закон спекуляції: за таких революцій у
вартості спекулювати на сировинному матеріялі в його найменш обробленій формі, отже, радше на пряжі,
ніж на тканині, і радше на самій бавовні, ніж на пряжі. Зміна вартости постає тут у тому процесі, що
продукує бавовну, а не в тому процесі, де вона функціонує як засіб продукції і, отже, як сталий
капітал. Вартість товару, щоправда, визначається кількістю праці, що міститься в ньому, алеж сама ця
кількість є суспільно визначена. Коли
\parbreak{}  %% абзац продовжується на наступній сторінці
