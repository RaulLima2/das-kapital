відомостей. Борошно перечислено на квартери збіжжя (Див.
таблицю на стор. 378).

Величезна, стрибкувата розширність фабричної системи та
її залежність від світового ринку неминуче породжують гарячкову
продукцію і наступне переповнення ринків, із звуженням
яких настає параліч. Життя промисловости перетворюється на
послідовний ряд періодів середнього пожвавлення, розцвіту,
перепродукції, кризи й застою. Непевність і непостійність, що
їх зазнає праця і разом з нею й доля робітника через машинове
виробництво, стають нормальними з цією зміною періодів промислового
циклу. За винятком часів розцвіту, між капіталістами лютує
якнайзавзятіша боротьба за їхнє індивідуальне місце на ринку.
Це їхнє місце на ринку стоїть у прямому відношенні до дешевини
продукту. Крім створеного цим суперництва щодо вживання
поліпшених машин, які замінюють робочу силу, та нових метод
продукції, кожного разу настає такий момент, коли капіталісти
намагаються здешевити товари, силоміць понижуючи заробітну
плату нижче вартости робочої сили.235

нулись у другу промислову країну світу, не втративши при цьому цілком;
свого колоніального характеру. — Ф. Е.].

Вивіз бавовни із Сполучених штатів до Великобританії (в фунтах):

1846 р........... 401.949.393    1852 р............ 765.630.544

1859 р........... 961.707.264    1860 р............ 1.115.890.608

Вивіз збіжжя тощо із Сполучених штатів до Великобританії
(1850—1862 рр.) (в центнерах):

                                                            1850 р.                 1862 р.
Пшениця.......................        16.202.312           41.033.503
Ячмінь..........................           3.669.653           6.624.800
Овес..............................         3.174.801             4.426.994
Жито ............................            388.749                7.108
Пшеничне борошно......            3.819.440           7.207.113
Гречка..........................              1.054                  19.571
Кукурудза.....................          5.473.161         11.694.818
Веrе або Bigg (особливий
рід ячменю).................             2.039                    7.675
Горох............................          811.620              1.024.722
Квасоля........................        1.822 972              2.037.37
Увесь довіз...................          34.365.801            74.083.351

235 У відозві робітників, викинутих на брук льокавтом фабрикантів
чобіт із Лестеру, до «Trade-Societies of England», липень 1866 p., між
іншим, сказано: «Ось уже років із 20 тому, як у чоботарстві Лестеру
відбувся переворот: замість зшивати почали скріпляти гвіздками. Тоді
можна було мати добру заробітну плату. Незабаром ця нова галузь промисловости
дуже поширилась. Велика конкуренція почалася між різними
фірмами, кожна з них намагалась подати найелегантніший товар. Алеж
незабаром виникла гірша конкуренція, а саме — фірми намагались побороти
одна одну на ринку нижчою ціною (undersell). Шкідливі наслідки виявилися
незабаром у пониженні заробітної плати, і ціна на працю спадала
