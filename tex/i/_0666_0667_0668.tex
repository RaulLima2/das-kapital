\parcont{}  %% абзац починається на попередній сторінці
\index{i}{0666}  %% посилання на сторінку оригінального видання
мірі перевищує найдоконечніші засоби існування робітника?.. Без сумніву, робочі коні в сільському
господарстві мають в Англії далеко кращий корм, аніж англійські рільничі робітники, бож коні є цінне
майно».\footnote{
Там же, т. І, стор. 47, 246.
} Але never mind,* адже національне багатство з природи тотожнє з народніми злиднями.

Але як же вилікувати колонії від цієї антикапіталістичної болячки? Коли б хто хотів за одним махом
перетворити всю землю з народньої власности на приватну власність, то цим би він, правда, знищив
корінь зла, але разом з тим — і колонії. Майстерність у тім, щоб одним пострілом убити двох зайців.
Треба, щоб уряд надав незайманій землі штучну ціну, незалежну від закону попиту й подання, ціну, що
примусить еміґранта
працювати довший час найманим робітником, доки він заробить досить грошей, щоб купити собі землю\footnote{
«Ви кажете, що завдяки присвоєнню землі й капіталів, людина, яка не має нічого, крім своїх рук,
находить собі роботу та створює собі дохід\dots{} навпаки, лише завдяки індивідуальному присвоєнню
землі, стається те, що є люди, які не мають нічого, крім своїх рук. Ставлячи людину в безповітряний
простір, ви захоплюєте собі атмосферу. Те саме ви робите, захоплюючи собі землю\dots{} Це все одно, що
кинути людину в простір, де немає багатств, щоб зробити її життя залежним від вашої волі». («C’est,
ajoutez-vous, grâce à l’appropriation du sol et des capitaux que l’homme, qui n’a que ses bras,
trouve de l’occupation, et se fait un revenu\dots{} c’est au contraire, grâce à l’appropriation
individuelle du
sol qu’il se trouve des hommes n’ayant que leurs bras\dots{} Quand vous mettez un homme dans le vide,
vous vous emparez de l’atmosphère. Ainsi faites-vous, quand vous vous emparez du sol. C’est le
mettre dans le vide de richesse, pour ne le laisser vivre qu’à votre volonté»). (Colins: «L’Economie
Politique, Source des Révolutions et des Utopies prétendues Sосіаlistes», Paris 1857, vol. III, p.
267--271 passim.).
* — що з того. \emph{Ред.}
** Все буде якнайкраще в цьому найкращому із світів. \emph{Ред.}
}
й перетворитись на незалежного селянина. З другого боку, фонду, створеного через продаж земель по
ціні, порівняно неприступній для найманого робітника, отже, цього грошового фонду, видушеного із
заробітної плати через порушення святого закону попиту й подання, уряд повинен уживати в міру його
зростання на те, щоб імпортувати з Европи в колонії голоту і підтримувати таким
чином для пана капіталіста ринок найманої праці повним. За таких обставин tout sera pour le mieux
dans le meilleur des mondes possibles.** Оце — велика таємниця «систематичної колонізації». «За цим
пляном, — вигукує тріюмфуючи Векфілд, —
подання праці мусить бути стале й реґулярне; бо, поперше, через те, що жоден робітник не має змоги
купити собі землі доти, доки він не попрацює певний час за гроші, всі еміґранти-робітники, працюючи
комбінованими групами як наймані робітники, продукували б своєму підприємцеві капітал для вживання
ще більшої кількости праці; подруге, кожний, що кинув би найману працю і став би земельним
власником, саме через купівлю землі забезпечував би певний фонд, щоб приставляти
\index{i}{0667}  %% посилання на сторінку оригінального видання
нових робітників у колонії».\footnote{
Wakefield. Там же, т. II, стор. 192.
} Октройована державою ціна землі мусить, звичайно, бути
«достатня» (sufficient price), тобто така висока, «щоб перешкоджати робітикам ставати незалежними
селянами доти, доки не з’являться інші, щоб заступити їхнє місце на ринку найманої праці».\footnote{
Там же, стор. 45.
} Ця
«достатня ціна землі» є не що інше, як пом’якшене означення викупних грошей, які робітник платить
капіталістові за дозвіл покинути ринок найманої праці й заходитися коло обробітку землі. Спочатку
робітник мусить створити панові капіталістові «капітал», щоб він міг експлуатувати більше число
робітників, а потім він мусить приставити на ринок праці «заступника», якого його коштом уряд
транспортує із-за моря для його колишнього пана капіталіста.

Надзвичайно характеристично, що англійський уряд протягом багатьох років запроваджував цю методу
«первісної акумуляції капіталу», рекомендовану паном Векфілдом для вжитку спеціяльно по колоніяльних
країнах. Фіяско було, звичайно, таке саме ганебне, як фіяско з банковим актом Піла. Потік еміґрації
лише повернувся від англійських колоній до Сполучених штатів. Тимчасом проґрес капіталістичної
продукції в Европі, супроводжуваний дедалі більшим утиском з боку уряду, зробив рецепт Векфілда
зайвим. З одного боку, величезний і невпинний потік людей, що рік-у-рік тече до Америки, залишає на
сході Сполучених штатів застійні осади, бо хвиля еміґрації з Европи кидає туди людей на робітничий
ринок швидше, ніж друга хвиля еміґрації встигає занести їх на захід. З другого боку, американська
громадянська війна потягла за собою колосальний національний борг, а разом з ним податковий тиск,
народження найпідлішої фінансової аристократії, роздаровування величезної частини громадських земель
товариствам спекулянтів для експлуатації залізниць, копалень і т. ін., — одно слово, вона потягла за
собою якнайшвидшу централізацію капіталу. Отже, велика республіка перестала бути обітованою землею
для робітників-еміґрантів. Капіталістична продукція йде там велетенськими кроками вперед, хоч спад
заробітної плати й залежність найманого робітника далеко ще не зведені до европейського нормального
рівня. Безсоромне марнотратне роздаровування англійським урядом необроблених колоніяльних земель
аристократам і капіталістам, яке сам Векфілд голосно засуджує, разом із потоком людей, що їх
притягають копальні золота, і з конкуренцією, яку імпорт англійських товарів створює навіть
найдрібнішому ремісникові, породили, особливо в Австралії,\footnote{
Скоро Австралія стала своєю власною законодавицею, вона звичайно, видала закони, сприятливі для
переселенців, але марнотратство земель, що його перевели вже англійці, стоїть на перешкоді. «Перша й
найважливіша мета, яку ставить новий земельний закон з року
} достатнє «відносне перелюднення
\index{i}{0668}  %% посилання на сторінку оригінального видання
робітників», так що майже кожний поштовий корабель приносить із собою лихі звістки про
переповнення австралійського ринку праці — «glut of the Australian labour-market», — а проституція
процвітає там подекуди так само пишно, як і на Haymarket у Лондоні.

Однак нас цікавить тут не стан колоній. Нас цікавить лише таємниця, відкрита в Новому Світі
політичною економією Старого Світу і гучно проголошена нею: капіталістичний спосіб продукції й
акумуляції, отже, і капіталістична приватна власність зумовлюють знищення приватної власности,
основаної на власній праці, тобто зумовлюють експропріяцію робітника.

1862, є в тому, щоб полегшити народові змогу розселюватися» («The first and main object at which the
new Land Act of 1862 aims, is to give increased facilities for the settlement of the people»). («The
Land Law of Victoria by the Hon. G. Duffy, Minister of Public Lands», London 1862, p. 3).
\parbreak{}  %% абзац продовжується на наступній сторінці
