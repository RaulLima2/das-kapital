7. Боротьба за нормальний робочий день. Вплив англійського
фабричного законодавства на інші країни

Читач пригадує собі, що продукція додаткової вартости, або
витягання додаткової праці становить специфічний зміст і мету
капіталістичної продукції, незалежно від тих будь-яких змін
у самому способі продукції, які випливають із підпорядкування
праці капіталові. Він пригадує собі, що з того погляду, який
ми досі розвивали, лише самостійний, а тому і юридично повнолітній
робітник складає як продавець товару умову з капіталістом.
Отже, коли в нашому історичному нарисі головну ролю
відіграє, з одного боку, сучасна промисловість, а з другого —
праця фізично і юридично неповнолітніх, то перша мала для
нас значення лише як осібна сфера висисання праці, а друга —
як особливо яскравий приклад висисання праці. Однак, не забігаючи
наперед до того, що ми розвинемо пізніш, із самого лише
загального зв’язку історичних фактів випливає ось що:

По-перше, в галузях промисловости, насамперед революціонізованих
водою, парою й машинами, в цих перших витворах сучасного
способу продукції, в прядільнях і ткальнях бавовни,
льону і шовку капітал насамперед задовольняє своє прагнення
до безмірного й нещадного подовження робочого дня. Змінений
матеріяльний спосіб продукції і відповідно до нього змінені
соціяльні відносини продуцентів 186 створюють спочатку безмірні
порушення меж робочого дня, а після того вже викликають,
як реакцію, суспільний контроль, що законодатним шляхом
обмежує робочий день з його перервами, реґулює його і робить
його одностайним. Тому протягом першої половини XIX століття
цей контроль з’являється лише як виняткове законодавство.187
Скоро тільки останнє завоювало первісне поле нового способу
продукції, виявилося, що за той час не лише багато інших галузей
продукції опинилося під фабричним режимом у власному

мисловости, що їм загрожувало підведення під фабричне законодавство,
використали ввесь свій вплив на парлямент, щоб зберегти недоторканим
своє «право громадянина» на необмежену експлуатацію робочої сили.
Природно, в ліберальному міністерстві Ґледстона вони знайшли покірних
слуг].*

186 «Поведінка кожної з цих кляс (капіталістів і робітників) — це
результат тих відносин, в яких кожна з них перебуває» («The conduct
of each of these classes (capitalists and workmen) has been the result of
relative situation in which they have been placed»). («Reports etc. for 31st
October 1848», p. 113).

187 «Підприємства, що підлягали обмеженням, були пов’язані з
продукцією тканин за допомогою сили пари або води. Треба було двох
умов для того, щоб фабрика мусила підлягати доглядові: застосовувати
силу пари або води й обробляти деякі спеціяльні волокна». («The employments
placed under restriction were connected with the manufacture of
textile fabrics by the aid of steam or water power. There were two conditions
to which an employment must be subject to cause it to be inspected,
viz. the use of steam or water power, and the manufacture of certain specified
fibres»). («Reports etc. for 31 st October 1864», p. 8).

* Заведене у прямі дужки ми беремо з французького видання. Ред.
