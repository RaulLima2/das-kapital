«Ми бачимо, що мало робітничих родин може харчуватись
принаймні так, як арештанти, не кажучи вже про те, щоб харчуватись
так, як матроси або солдати. Кожний бельгійський арештант
коштував у 1847—1849 рр. пересічно 63 сантими на день,
що проти щоденних витрат на утримання робітника дає ріжницю
в 13 сантимів. Витрати на адміністрацію й догляд вирівнюються
тим, що арештант не платить за квартиру... Але як пояснити, що
велике число, ми могли б сказати величезна більшість, робітників
живе у ще скупіших умовах? Лише тим, що вони вживають
таких заходів, що їхня таємниця відома тільки їм самим; вони
зменшують свою щоденну порцію, їдять житній хліб замість пшеничного;
їдять менше м’яса або й зовсім його не їдять; те саме
з маслом і приправами; ціла родина тулиться в одній або двох
комірках, де дівчата й хлопці сплять разом, часто на тому самому
солом’янику; вони заощаджують на одягу, білизні, засобах підтримувати
чистоту, відмовляють собі приємностей неділями,
коротко — вони засуджують себе на якнайприкріші нестатки.
Скоро робітник дійде до цієї останньої межі, то незначне підвищення
ціни засобів існування, якась затримка в роботі, якась
недуга збільшують його злидні й цілковито руйнують його. Борги
наростають, кредит вичерпується, одяг і найдоконечніші меблі
мандрують у льомбард, і, нарешті, родина просить вписати її
у реєстр бідних».137 Дійсно, в цьому «раю капіталістів» щонайменша
зміна ціни найдоконечніших засобів існування тягне
за собою зміну числа смертних випадків і злочинів! (Див. маніфест
Maatschappij: «De Vlamingen VooruitI» Brussel 1860,
p. 15,16). Ціла Бельґія налічує 930.000 родин, з них, за офіціальною
статистикою, 90.000 багатих (виборців) = 450.000 осіб; 190.000
родин дрібної середньої кляси, міської і сільської, що значна
частина її завжди попадає до лав пролетаріяту = 1.950.000 осіб.
Нарешті, 450.000 робітничих родин = 2.250.000 осіб, що з них
зразкові родини зазнають щастя, яке змалював Ducpétiaux. Із 450.000 робітничих родин понад 200.000
фігурують у списку
бідних!

е) Британський рільничий пролетаріят

Антагоністичний характер капіталістичної продукції й акумуляції
ніде не виявляться брутальніш, ніж у проґресі англійського
сільського господарства (включаючи і скотарство) і в
реґресі англійського сільського робітника. Раніш ніж перейдемо
до його сучасного становища, киньмо оком назад. Сучасне
рільництво в Англії веде свій початок від середини XVIII століття,
хоча переворот у відносинах земельної власности, що з
нього як бази походить змінений спосіб продукції, датується куди
ранішим часом.

137 Ducpétiaux: «Budgets économiques des classes ouvrières en Belgique»,
Bruxelles 1855, p. 151, 154, 155.
