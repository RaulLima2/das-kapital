5. Боротьба між робітником і машиною

Боротьба між капіталістом і найманим робітником починається
разом із виникненням самого капіталістичного відношення. Вона
лютує протягом цілого мануфактурного періоду.\footnote{
Див. між іншим: John Houghton: «Husbandry and Trade improved»,
London 1727. «The Advantages of the East India Trade», 1720.
John Belters: «Proposals for raising a Colledge of Industry», London
1696. «Хазяїни та робітники, на жаль, перебувають у постійній війні
між собою. Незмінна мета хазяїнів — діставати працю для себе якомога
дешевше, і вони не вагаються пускатись на всякі хитрощі, щоб досягти
цієї мети, тимчасом як робітники з такою самою впертістю використовують
усяку нагоду, щоб примусити своїх хазяїнів виконати їхні підвищені
вимоги». («The masters and their workmen are unhappily in a perpetual
state of war with each other. The invariable object of the former is to
get their work done as cheap as possibly: and they do not fail to employ
every artifice to this purpose, whilst the latter are equally attentive to
every occasion of distressing their masters into a compliance with higher
demands»). «An Inquiry into the causes of the Present High Prices of
Provisions», 1767, p.61,62. (Автор — панотець Натаніел Ферстер, що цілком
стоїть на боці робітників).
} Але лише
із заведенням машин робітник починає боротися проти самого
засобу праці, цієї матеріяльної форми існування капіталу. Він
повстає проти цієї певної форми засобу продукції як матеріяльної
основи капіталістичного способу продукції.

Мало не ціла Европа пережила у XVII віці повстання робітників
проти так званої Bandmühle (або інакше Schnurmühle
або Mühlenstuhl»),\footnote*{
— стьожковий млин. Ред.
} проти машини для ткання стьожок та брузументу.\footnote{
Bandmühle винайдено в Німеччині. Італійський абат Лянчелотті
у своїй праці, що появилась у Венеції 1636 р., оповідає: «Антон Міллер
з Данціґу якихось 50 років тому (Лянчелотті писав 1579 р.) бачив у Данціґу
дуже мудру машину, яка виготовляла одночасно 4—6 тканин; міська
рада, турбуючися про те, що цей винахід може поробити масу робітників
старцями, затаїла цей винахід, а винахідника наказала потайки задушити
або втопити». В Ляйдені таку саму машину вперше вжито 1629 р. Зако-
} Наприкінці першої третини XVII віку чернь підчас
заколотів знищила вітряну лісопильню, поставлену якимось голляндцем
коло Лондону. Ще на початку XVII віку тартаки,

це, мовляв, дрібничка. Життя й надії робітника так дуже залежать від
його пальців, що така втрата є для нього надзвичайно серйозна подія.
Слухаючи таке безглузде базікання, я питав їх: «Припустіть, що вам потрібен
один додатковий робітник і до вас з’явилося двоє, обидва з усякого
іншого погляду однаково дужі, але один не має великого або вказівного
пальця, то котрого з них ви вибрали б?» І хвилини не вагаючись, вони
висловилися за того, що має всі пальці... Ці пани фабриканти мають
фалшиві упередження проти того, що вони називають псевдофілантропічним
законодавством». («Reports of Insp. of Fact. for 31 st October
1855»). Ці пани — меткі людці, і не дурно мріють вони про бунт рабовласників!
192 На фабриках, здавна підлеглих фабричному законові з його примусовим
обмеженням робочого часу та іншими його постановами, деякі
давніші лиха зникли. Самб поліпшення машин вимагає на якомусь певному
пункті «поліпшеної конструкції фабричних будівель», що йде на
користь робітникам. (Порівн. «Reports etc. for 31 st October 1863», p. 109).