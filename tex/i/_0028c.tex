\parcont{}  %% абзац починається на попередній сторінці
\index{i}{0028}  %% посилання на сторінку оригінального видання
конкретний корисний рід праці, що міститься в кожному осібному
товаровому еквіваленті, є лише осібна форма виявлення людської
праці, отже, форма, що не вичерпує всіх форм людської праці.
Правда, ця остання має свою повну або вичерпну форму виявлення
в сукупності тих осібних форм виявлення. Але в такому разі вона
не має жодної однорідної форми виявлення.

Однак розгорнута відносна форма вартости складається лише
з суми простих відносних виразів вартости або з рівнань першої
форми, як от:
\begin{align*}
20\text{ метрів полотна} &= 1\text{ сурдутові} \\
20\text{ метрів полотна} &= 10 \text{ фунтам чаю й т. ін.}
\end{align*}

Але кожне з цих рівнань містить у собі й зворотне тотожне
рівнання:
\begin{align*}
1\text{ сурдут} &= 20\text{ метрам полотна} \\
10\text{ фунтів чаю} &= 10 \text{ метрам полотна}
\end{align*}

Дійсно, коли хтось обмінює своє полотно на багато інших товарів
і таким чином виражає вартість полотна в ряді інших товарів,
то й усі інші посідачі товарів неминуче мусять обмінювати
свої товари на полотно й таким чином виражати вартість своїх
різних товарів у тому самому третьому товарі, в полотні. — Отже,
якщо ми обернемо ряд: 20 метрів полотна = 1 сурдутові, або =
10 фунтам чаю, або = і т. ін., тобто, якщо виразимо те зворотне
відношення, яке по суті вже міститься в цьому ряді, то матимемо:

\subsubsection{Загальна форма вартости}

\begin{equation*}
\left.\begin{aligned}
&1\text{ сурдут} &= \\
&10\text{ фунтів чаю} &= \\
&40\text{ фунтів кави} &= \\
&1\text{ квартер пшениці} &= \\
&2\text{ унції золота} &= \\
&\sfrac{1}{2}\text{ тонни заліза} &= \\
&х\text{ товару }А &= \\
&\text{і т. под. товарів} &= \\
\end{aligned}\right\rbrace
20 \text{ метрам полотна}
\end{equation*}

\paragraph{Змінений характер форми вартости}

Тепер товари виражають свої вартості, поперше, просто, бо
вони їх виражають в яко\-мусь од\-ним-од\-но\-му товарі, а подруге,
однорідно, бо вони їх виражають в тому самому товарі. їхня форма
вартости є проста й спільна їм всім, тим то й загальна.

Форми перша, А, і друга, В, досягали лише того, що виражали
вартість якогось товару як щось відмінне від його власної
споживної вартости або від його товарового тіла.

Перша форма, А, подавала такі рівнання вартости, як ось: 1 сурдут
= 20 метрам полотна, 10 фунтів чаю = \sfrac{1}{2} тонни заліза й т. ін.
Вартість сурдута виражається як щось рівне полотну, вартість
\parbreak{}  %% абзац продовжується на наступній сторінці
