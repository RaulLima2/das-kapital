А що робочий день є стала величина й виражається в сталій
величині вартости, що кожній зміні величини додаткової вартости
відповідає зворотна зміна величини вартости робочої сили, що
вартість робочої сили може змінятися лише разом із зміною продуктивної
сили праці, то, за цих умов, цілком ясно, що кожна
зміна величини додаткової вартости постає в наслідок зворотної
зміни величини вартости робочої сили. Тому, коли ми раніш
бачили, що неможлива жодна зміна абсолютних величин вартости
робочої сили й додаткової вартости без зміни їхніх відносних
величин, то тепер бачимо, що жодна зміна їхніх відносних
величин вартости неможлива без зміни абсолютної величини вартости
робочої сили.

За третім законом зміна величини додаткової вартости має
собі за передумову зміну вартости робочої сили, спричинену
зміною продуктивної сили праці. Межу зміни величини додаткової
вартости дано новою межею вартости робочої сили. Але навіть
у тому випадку, коли обставини дозволяють цьому законові
діяти, все ж можуть відбуватися проміжні зміни. Якщо, наприклад,
у наслідок підвищеної продуктивної сили праці вартість
робочої сили падає з 4 шилінґів до 3, або доконечний робочий час
падає з 8 до 6 годин, то ціна робочої сили могла б спасти лише до
З шилінґів 8 пенсів, 3 шилінґів 6 пенсів, 3 шилінґів 2 пенсів
і т. д., а тому додаткова вартість могла б підвищитися лише до
З шилінґів 4 пенсів, 3 шилінґів 6 пенсів, 3 шилінґів 10 пенсів,
і т. д. Ступінь спаду, що його мінімальна межа є 3 шилінґи,
залежить од відносної ваги, яку кидають на терези натиск капіталу,
з одного боку, опір робітників — з другого.

Вартість робочої сили визначається вартістю певної кількости
засобів існування. Із зміною продуктивної сили праці змінюється
вартість цих засобів існування, а не їхня маса. Сама ця
маса, при зростанні продуктивної сили праці, може зростати для
робітника і для капіталіста одночасно і в однаковій пропорції
без якоїбудь зміни між величинами ціни робочої сили й додаткової
вартости. Коли первісна вартість робочої сили дорівнює 3 шилінґам,
а доконечний робочий час становить 6 годин, коли додаткова
вартість теж дорівнює 3 шилінґам, або додаткова праця
становить також 6 годин, то подвоєння продуктивної сили праці,
при незмінному поділі робочого дня, лишило б незмінними ціну
робочої сили й додаткову вартість. Тільки кожна з них виражалася
б у подвійній кількості, але відповідно до цього й здешевілих
споживних вартостей. Хоча ціна робочої сили й лишалася б
незмінна, все ж вона підвищилася б понад її вартість. Коли б
ціна робочої сили впала, але не до мінімальної межі в 1\sfrac{1}{2} шилінґа,
визначеної її новою вартістю, а до 2 шилінґів 10 пенсів, 2 шилінґів
6 пенсів і т. д., то й це падіння ціни все ще репрезентувало б
зростання маси засобів існування. Таким чином при зростанні

загального правила — лихо, яке з ним трапляється у вульґаризуванні
Рікарда так само часто, як і з Ж. Б. Сеєм у його вульґаризуванні А. Сміса.
