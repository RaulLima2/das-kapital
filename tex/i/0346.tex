4. Фабрика

На початку цього розділу ми розглядали тіло фабрики, тобто
організовану систему машин. Далі ми бачили, як машини, присвоюючи
собі жіночу й дитячу працю, збільшують людський
матеріял, що його експлуатує капітал, як вони, безмірно здовжуючи
робочий день, забирають увесь час життя робітника та,
нарешті, як їхній поступ, що дозволяє продукувати велетенські,
чимраз більші маси продукту в щораз коротший час, служить
систематичним засобом, щоб кожного моменту пускати в рух
більше праці, тобто раз-у-раз інтенсивніше визискувати робочу
силу. Тепер ми перейдемо до фабрики як до цілости, і саме в її
найрозвиненішій формі.

Д-р Юр, Піндар автоматичної фабрики описує її, з одного
боку, як «кооперацію різних кляс робітників, дорослих і недорослих,
які з вправністю і пильністю доглядають систему про-

                                                                            Вартість експорту (фун.
стерл.)
                                                       1848 р.                    1851 р.           
   1860 р.               1865 р.
Бавовняні фабрики
Бавовняна пряжа                  5.927.831                6.634.026             9.870.875        
10.351.049
Бавовняні тканини                 16.753.369           23.454.810        42.141.505         
46.903.796
Льнопрядні та коноплепрядні фабрики
Пряжа.....................                  493.449                 951.426             1.801.272   
       2.505.497
Тканини..................                 2.802.789             4.107.396             4.804.803     
    9.155.358
Шовкові фабрики
Пряжа й нитки.........                   77.789                 196.380              826.107        
     768.064
Тканини...................                 510.328               1.130.398          1.587.303       
    1.409.221
Вовняні фабрики
Пряжа.......................               776.975              1.484.544            3.843.450      
   5.424.047
Тканини....................               5.733.829             8.377.183           12.156.998      
 20.102.259

(Див. Сині Книги: «Statistical Abstract for the United Kingdom»,
№ 8 і № 13, London 1861 і 1866).

В Ланкашірі число фабрик збільшилося між 1839 і 1850 рр. лише на
4\%, між 1850 і 1856 — на 19\%, між 1856 і 1862 — на 33\%, тимчасом
як число занятих осіб за обидва одинадцятилітні періоди абсолютно підвищилось,
відносно ж зменшилось. Див. «Reports of Insp. of Fact. for
31 st October 1862», p. 63. В Ланкашірі мають перевагу бавовняні фабрики.
А яке відносно велике місце мають вони взагалі у фабрикації пряжі й
тканини, це видно з того, що з загального числа всіх фабрик такого роду
в Англії, Велзі, Шотляндії та Ірляндії на них самих припадає 45,2\%, а із
загального числа веретен — 83,3\%, із загального числа парових ткацьких
варстатів — 81,4\%, із загального числа парових кінських сил, що дають
рух тим фабрикам, — 72,6\%, а із загального числа занятих осіб — 58,2\%.
(Там же, стор. 62, 63).
