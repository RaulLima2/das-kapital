до того, що, за інших аналогічних обставин, доконечний робочий
час є різний. На додаткову працю вони впливають лише як
природна межа, тобто визначають той пункт, що від нього може
початися праця на інших. Ця природна межа відсувається назад
у тій самій мірі, в який промисловість проґресує. Серед західньоевропейського
суспільства, де робітник лише додатковою працею
купує дозвіл працювати для свого власного існування, легко
постає ілюзія, що давати додатковий продукт є природжена властивість
людської праці.8 Але візьмімо, наприклад, жителів
східніх островів азійського архіпелагу, де саґо дико росте в лісі.
«Коли місцеві жителі, просвердливши діру в дереві, переконуються,
що стрижень уже достиг, вони зрубують дерево, ділять
його на декілька кусків, видирають стрижень, змішують його
з водою і, відцідивши воду, дістають цілком придатне до вжитку
саґове борошно. Одно дерево дає звичайно 300 фунтів, а може
дати 500—600 фунтів. Отже, там ідуть у ліс і рубають собі хліб,
як у нас рубають дерево на паливо.9 Припустімо, що такому
східньоазійському рільникові потрібно 12 робочих годин на
тиждень для задоволення всіх його потреб. Сприятлива природа
безпосередньо дає йому багато вільного часу. Для того, щоб він
цей час продуктивно зуживав на самого себе, потрібен цілий ряд
історичних умов, а для того, щоб він витрачав його як додаткову
працю на чужих осіб, потрібен зовнішній примус. Коли б там
було заведено капіталістичну продукцію, наш молодець мусив би
працювати, може, 6 днів на тиждень, щоб присвоїти собі самому
продукт одного робочого дня. Сприятливість природи не пояснює,

праці. Людські потреби зростають або зменшуються залежно від суворости
або м’якости підсоння, що в ньому люди живуть; отже, розміри, що в
них мешканці різних країн мусять продукувати, не можуть бути однакові,
і не можна визначити ступінь цієї неоднаковости інакше, як тільки у
зв’язку з ступенем теплоти або холоду; звідси можна зробити й той загальний
висновок, що кількість праці, потрібної для певного числа людности,
найбільша в холодному підсонні, найменша в теплому. Бо в першому
не тільки люди більше потребують одягу, але й земля потребує
більше праці на оброблення, аніж у другому». («There are no two countries
which furnish an equal number of the necessaries of life in equal plenty,
and with the same quantity of labour. Men’s wants increase or diminish
with the severity or teinperateness of the climate they live in; consequently
the proportion of trade which the inhabitants of different countries are
obliged to carry on through necessity, cannot be the same, nor is it practicable
to ascertain the degree of variation farther than by the Degrees of
Heat and Cold; from whence one may make this general conclusion, that
the quantity of labour required for a certain number of people is greatest
in cold climates, and least in hot ones; for in the former men not only want
more clothes, but the earth more cultivating than in the latter»). («An
Essay on the Governing Causes of the Natural Rate of Interest», London
1750, p. 60). Автор цього епохального анонімного твору J. Massey.
Юм запозичив із нього свою теорію процента.

8 «Всяка праця мусить» (це, здається, також належить до прав і
обов’язків громадянина) «давати надлишок» («Chaque travail doit
laisser un excédant»). (Proudhon).

9 F. Shouw: «Die Erde, die Pflanze und der Mensch». 2. Auflage. Leipzig
1854, S. 148.
