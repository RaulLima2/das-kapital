ніж колись експлуатовано копальні Потозі. Антагоністичний
характер капіталістичної акумуляції, а тому й капіталістичних
відносин власности взагалі115 стає тут такий наочний, що навіть
офіціальні англійські звіти про цей предмет переповнені єретичними
нападами на «власність і її права». Це лихо так поширилося
з розвитком промисловости, акумуляцією капіталу, зростом
та «прикрашуванням» міст, що лише острах перед пошесними
недугами, які не щадять навіть «шановних», викликав від 1847
до 1864 р. не менше як десять санітарно-поліційних парляментських
актів, а перелякана буржуазія деяких міст, як от Ліверпуль,
Ґлезґо і т. д., почала втручатися в цю справу через свої
муніципалітети. Проте, д-р Сімон у своєму звіті з 1865 р. вигукує:
«Взагалі кажучи, цей лихий стан в Англії лишається без
контролю». З наказу Privy Council в 1864 р. досліджено житлові
умови сільських робітників, в 1865 р. — бідніших кляс по містах.
Майстерні праці д-ра Джульяна Гентера надруковано в
сьомому й восьмому звітах «Public Health». Про сільських робітників
я говоритиму пізніше. Щождо міських житлових умов,
то я насамперед наведу загальну увагу д-ра Сімона: «Хоч
мій офіціяльний погляд, — каже він, — виключно медичний, проте
звичайна гуманність не дозволяє мені не звертати уваги на другий
бік цього лиха. Дійшовши вищого ступеня, це лихо майже
неминуче зумовлює таке заперечення всякої звичайности, таке
брудне змішування тіл і фізичних функцій, таку одверту наготу
статей, що все це має звірячий, а не людський характер. Зазнавати
таких впливів — це зневага, яка стає то глибша, що довше
вона триває. Для дітей, що народилися під цим прокляттям,
воно є хрещення на ганьбу (baptism into infamy). І безнадійним
понад усяку міру є бажання, щоб люди, поставлені в такі умови,
в інших відношеннях прагнули тієї атмосфери цивілізації, що її
суть у фізичній і моральній чистоті».116

Перше місце щодо переповнення помешкань або й щодо абсолютної
непридатности їх для людського житла посідає Лондон.
«Дві обставини, — каже д-р Гентер, — певні: по-перше, в Лондоні
є щось з 20 великих колоній, кожна з яких має приблизно
10.000  осіб, що їхнє злиденне становище переважає все, що будь-коли
бачили деінде в Англії, і це становище є майже цілком результат
поганого стану їхнього житла; по-друге, переповненість і зруйнованість
домів по цих колоніях тепер значно гірші, ніж двадцять
років тому». 117 «Не буде перебільшенням сказати, що

115 «Ніде права особи не жертвовано так одверто й безсоромно на
користь праву власности, як у житлових умовах робітничої кляси. Кожне
велике місто являє собою місце людських жертов, вівтар, на якому рік-у-рік
убивають тисячі людей для Молоха ненажерливости.» (S. Laing: «National Distress», 1844, р. 150).

116 «Public Health, Eighth Report», London 1866, p. 14, примітка.

117 Там же, стор. 89. Щодо дітей із цих колоній д-р Гентер каже:
«Ми не знаємо, як виховували дітей перед цією епохою тісного скупчення
бідних, і сміливим пророком був би той, хто хотів би наперед сказати,
чого можна сподіватися від дітей, які серед умов, у цій країні безприклад-
