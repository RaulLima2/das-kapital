Але первородний гріх діє повсюди. З розвитком капіталістичного
способу продукції, акумуляції й багатства капіталіст
перестає бути простим утіленням капіталу. Він починає відчувати
«людське почуття» до свого власного Адама; до того ж він
стає настільки освіченим, що починає глузувати з фанатичного
аскетизму, як із забобону старомодного збирача скарбів. Тимчасом
як клясичний капіталіст плямує індивідуальне споживання
як прогріх проти своєї функції й як «поздержливість» від
акумуляції, модернізований капіталіст у силі зрозуміти акумуляцію
як «відречення» від особистої насолоди. «Ах, дві душі
живуть у його грудях, одна хоче розлучитися з другою!» («Zwei
Seelen wohnen, ach! in seiner Brust, die eine will sich von der
andern trennen!»).

На історичних початках капіталістичного способу продукції —
а кожний капіталістичний вискочень індивідуально пророблює
цю історичну стадію — жага збагачення й скупість панують як
абсолютні пристрасті. Але проґрес капіталістичної продукції
створює не тільки світ насолод. Разом із спекуляцією і кредитовою
справою він відкриває тисячі джерел раптового збагачення. На
певному щаблі розвитку деякий умовний ступінь марнотратства,
що є разом з тим виставою на показ багатства, а тому й кредитоспроможності!,
стає навіть діловою доконечністю для «нещасного»
капіталіста. Розкоші входять у видатки капіталу на представництво.
До того ж капіталіст багатіє не пропорційно до своєї особистої
праці і свого особистого не-споживання, як, приміром,
збирач скарбів: він багатіє в міру того, як висисає чужу робочу
силу і примушує робітника зрікатися всіх життєвих насолод.
Тому, хоч марнотратство капіталіста ніколи не має щирого
характеру марнотратства февдального пана-гуляки, навпаки,
в основі його завжди криється якнайогидливіше скнарство й
найдріб’язковіша ощадність, проте його марнотратство зростає
із зростом його акумуляції, при чому одне одному не перешкоджає.
Разом із тим у благородних грудях капіталіста розвивається
фавстівський конфлікт між жагою акумуляції і жагою насолод.

й персні, пестить свою пику, видає себе за добру побожну людину
й пишається цим... Лихвар же — страшелезна потвора, як той вовкулак,
що все плюндрує, гірший, ніж Какус, Геріон або Антус. Але він
прибирається й удає із себе побожного, щоб ніхто не бачив, де подіваються
ті воли, що їх він утягує задом у свій барліг. Але Геркулес повинен
чути, як ревуть воли й кричать полонені, повинен шукати Какуса навіть
у скелях і ярах, повинен визволити волів від лиходія. Бо Какус є лиходій,
і той лиходій — побожний лихвар, що все краде, грабує та пожирає.
І однак удає, ніби він нічого лихого не заподіяв, і ніхто не може викрити
його лиходійства, бо волів оін утягнув у свій барліг задом, і вони лишають
такі сліди, ніби їх випустили з барлогу. Так і лихвар хоче обдурити
світ, наче він дає користь і дає світові волів, тимчасом як він
захоплює їх собі й пожирає... І коли грабіжників, розбійників і напасників
колесують і стинав ть їм голови, то в скільки разів більше слід
би колесувати всіх лихварів, вимотувати з них жили... проганяти їх,
проклинати їх та стинати їм голови». (Martin Luther: «An die Pfarrherrn,
wider den Wucher zu predigen», Wittenberg 1540).
