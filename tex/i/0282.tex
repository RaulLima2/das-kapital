різних мануфактур. Великі англійські гути, приміром, сами
фабрикують свої глиняні топильні тиглі, бо від їхньої якости
посутньо залежить, вдасться чи не вдасться виготовлення продукту.
Мануфактура засобу продукції сполучається тут із мануфактурою
продукту. Навпаки, мануфактура продукту може бути
сполучена з мануфактурами, що для них цей продукт служить
знову за сировинний матеріял або що з їхніми продуктами він
пізніше сполучається. Так, наприклад, мануфактура кремінного
скла комбінується із шліфуванням скла та з мосяжоливарством;
останнє служить для виготовлення металевих оправ різноманітних
скляних товарів. Тоді різні комбіновані мануфактури
становлять більш або менш просторово відокремлені відділи цілої
мануфактури, а разом з тим вони є незалежні один від одного
процеси продукції, кожний із своїм власним поділом праці. Хоч
комбінована мануфактура дає деякі вигоди, проте, доки вона
ґрунтується на своїй власній основі, вона не набуває дійсної
технічної єдности. Ця єдність постає лише тоді, коли мануфактура
перетворюється на машинове виробництво.

Мануфактурний період, що незабаром проголошує зменшення
робочого часу, доконечного для продукції товарів, за свій свідомий
принцип,41 спорадично розвиває й уживання машин, особливо
для деяких простих початкових процесів, які можна провадити
лише у великому маштабі та при значній витраті сили. Приміром,
у паперовій мануфактурі здавна почали перемелювати ганчір’я
на паперових млинах, а в металюрґії розробляти руду на так
званих млинах-дробарках.42 Найелементарнішу форму всякої
машини залишила у спадщину Римська імперія у формі водяного
млина.43 Ремісничий період передав у спадщину великі винаходи,
як от компас, стрільний порох, друкарство та автоматичний
годинник. Однак взагалі і в цілому машини відіграють ту другорядну
ролю, яку приписує їм Адам Сміс, поряд поділу праці.44

41    Між іншим, це можна побачити з праць W. Petty, John Bellers,
Andrew Yarranton: «The Advantages of the East-India Trade» та J. Vanderlint.
42    Ще наприкінці XVI віку Франція користується ступою та решетом,
щоб роздробляти та перемивати руди.

43    Усю історію розвитку машин можна простежити на історії млинів
на збіжжя. В Англії ще й досі фабрику називають mill (млин). У
німецьких технологічних творах перших десятиліть XIX віку ми теж
ще подибуємо вираз Mühle (млин) не тільки на означення тих машин,
що їх рухають за допомогою сил природи, але й для всяких мануфактур,
що вживають механічних апаратів.

44    Як побачимо докладніше з четвертої книги цього твору, А. Сміс
не виставив жодної нової тези щодо поділу праці. Але що характеризує
його як політико-економа, який резюмує мануфактурний період, так
це той наголос, що його він робить на поділі праці. Та підрядна роля,
що її А. Сміс приписує машинам, викликала на початку великої промисловости
заперечення Лодерделя, а пізніше, за розвиненішої епохи —
заперечення Юра. А. Сміс переплутує також диференціяцію інструментів,
у якій велику ролю відігравали сами частинні робітники мануфактури,
— з винаходом машин. Але тут не мануфактурні робітники
відіграють ролю, а вчені, ремісники, а то й селяни (Brindley) і т. ін.
