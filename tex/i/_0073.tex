\parcont{}  %% абзац починається на попередній сторінці
\index{i}{0073}  %% посилання на сторінку оригінального видання
випадку мусило б циркулювати більше срібла, ніж раніше золота,
а в другому — менше золота, ніж раніше срібла. В обох випадках
змінилася б вартість грошового матеріялу, тобто товару,
що функціонує як міра вартостей, а тому змінився б і вираз
вартостей товарів у цінах, а в наслідок цього й маса грошей
що циркулюють, грошей, що придаються для реалізації цих
цін. Ми бачили вже, що сфера циркуляції товарів має прогалину,
через яку туди входить золото (срібло та взагалі грошовий
матеріял) як товар певної вартости. Цю вартість наперед припускається
при функціонуванні грошей як міри вартости, отже,
при визначенні цін. Коли падає, наприклад, вартість самої міри
вартости, то це виявляється насамперед у зміні ціни товарів,
які безпосередньо біля джерел продукції благородних металів
обмінюється на них як на товари. Чимала частина інших товарів,
особливо за менш розвинених становищ буржуазного суспільства,
буде ще довгий час цінуватися застарілою, що стала вже тепер
ілюзорною, вартістю міри вартости. Тимчасом один товар заражує
інший своїм відношенням вартости до нього; золоті або
срібні ціни товарів поволі вирівнюються відповідно до пропорцій,
визначених самою їхньою вартістю, поки, кінець - кінцем, усі
товарові вартості почнуть цінуватися відповідно до нової вартости
грошового металю. Цей процес вирівнювання супроводиться
безперервним зростом кількости благородних металів, що припливають
безпосередньо на зміну обмінених на них товарів.
Тому, в тій самій мірі, в якій узагальнюється виправлене визначення
цін товарів, або в якій вартості їхні цінується відповідно
до нової вартости металю, що вже впала й далі падає до певного
пункту, — в тій самій мірі вже є також наявна і додаткова
маса металю, доконечна для реалізації цін. Однобічне спостереження
фактів, що наступили по відкритті нових золотих і срібних
джерел, призвело в XVII і особливо у XVIII віці до неправдивого
висновку, що товарові ціни підвищились, бо почало функціонувати
більше золота й срібла як засобу циркуляції. В дальшому
викладі вартість золота припускається за дану, якою вона
дійсно і є в момент установлення цін.

Отже, за такого припущення маса засобів циркуляції визначається
сумою товарових цін, які мають бути зреалізовані. Якщо
ми припустимо тепер далі, що ціну кожного роду товару дано,
то сума цін товарів залежатиме, очевидно, від маси товарів,
які перебувають у циркуляції. Не треба багато морочити собі
голову, щоб зрозуміти, що коли один квартер пшениці коштує
2\pound{ фунти стерлінґів}, то 100 квартерів коштують 200\pound{ фунтів стерлінґів},
200 квартерів — 400\pound{ фунтів стерлінґів} і~\abbr{т. д.}, отже, разом
із зростом маси пшениці мусить зростати й маса грошей, яка
при продажу пшениці міняється з нею місцями.

Якщо припустити масу товарів за дану, то маса грошей, що
циркулюють, більшатиме й меншатиме разом із коливанням
товарових цін. Вона підноситься й падає, бо в наслідок зміни
цін збільшується або зменшується сума цін товарів. Для цього
\parbreak{}  %% абзац продовжується на наступній сторінці
