так званих освічених клясах Німеччини і, навпаки, відживає
знову в її робітничій клясі].*

В Німеччині політична економія лишається й по цей час чужоземною
наукою. Ґустав фон-Ґіліх у «Geschichtliche Darstellung
des Handels, der Gewerbe usw.», а саме, в перших двох томах
цього твору, виданих 1830 р., з’ясував уже значну частину тих
історичних обставин, які гальмували в нас розвиток капіталістичного
способу продукції, а тому й розвиток сучасного буржуазного
суспільства. Отже, бракувало життьового ґрунту для політичної
економії. Її імпортували з Англії і Франції як готовий товар;
німецькі професори її лишались учнями. Теоретичний вираз чужої
дійсности перетворився в їхніх руках на збірку догм, що їх
вони тлумачили в дусі дрібнобуржуазного світу, який їх оточував,
отже, тлумачили хибно. Не мавши сили цілком заховати почуття
наукового безсилля й неприємну свідомість, що доводиться
відігравати ролю вчителів у царині дійсно їм чужій, вони намагалися
прикритись блискучістю літературно-історичної вчености
або домішкою чужого матеріялу, запозиченого в так званих камеральних
наук, — мішанини з відомостей, що їхній чистилищний
вогонь повинен пройти кожний повний надій кандидат на
німецького бюрократа.

Від 1848 р. капіталістична продукція швидко розвинулась у
Німеччині й нині перебуває вже в періоді свого спекулятивного
розцвіту. Але до наших учених фахівців доля лишилася однаково
немилосердою. Доки вони могли безсторонньо працювати коло
політичної економії, в німецькій дійсності бракувало сучасних економічних
відносин. А коли ці відносини увійшли в життя, то це сталося
за таких обставин, що не дозволяли вже більше безсторонньо
досліджувати в межах буржуазного кругогляду. Оскільки політична
економія є буржуазна, тобто оскільки вона розглядає капіталістичний
лад не як історично минущий ступінь розвитку,
а, навпаки, як абсолютну й останню форму (Gestalt) суспільної
продукції, вона може лишатись науковою тільки доти, доки клясова
боротьба лишається лятентною або виявляється лише в
поодиноких явищах.

Візьмімо Англію. Її клясична політична економія припадає на
період нерозвинутої клясової боротьби. Останній великий представник
клясичної політичної економії, Рікардо, є перший економіст,
який свідомо робить вихідним пунктом своїх досліджень
протилежність клясових інтересів, заробітної плати й зиску, зиску
й земельної ренти, наївно вважаючи цю протилежність за суспільний
природний закон. Але цим самим буржуазна економічна
наука дійшла своєї останньої межі, яку їй несила було переступити.
Ще за життя Рікарда і всупереч йому виступила проти
неї критика в особі Сісмонді. 1

1 Див. мою працю «Zur Kritik der Politischen Oekonomie». 1. Auflage.
Berlin. 1859, S. 39.

* Заведене у прямі дужки беремо з другого німецького видання. Ред.
