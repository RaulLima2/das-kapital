\index{i}{0006}  %% посилання на сторінку оригінального видання
Тому величина вартости якогось товару лишалася б сталою,
коли б потрібний для його продукції робочий час лишався сталим.
Але останній змінюється з кожною зміною продуктивної
сили праці. Продуктивна сила праці визначається різноманітними
обставинами, між іншим, пересічним ступенем вправности робітників,
ступенем розвитку науки та її технологічного застосування,
суспільною комбінацією процесу продукції, обсягом і дієздатністю
засобів продукції та природними умовами. Приміром, та сама
кількість праці в сприятливий рік виражається у 8 четвериках
пшениці, а в несприятливий — тільки в 4. Та сама кількість праці
дає більше металю в багатих, ніж у бідних копальнях і~\abbr{т. ін.}
Діяманти трапляються в земній корі рідко, і тим то знаходження
їх коштує пересічно багато робочого часу. Значить, у невеликому
об’ємі вони репрезентують багато праці. Jacob сумнівається, щоб
золото будь-коли оплачувалось повною його вартістю. Ще
більше це має силу щодо діямантів. За Ешвеґе в 1823~\abbr{р.} ввесь
продукт вісімдесятрічної експлуатації бразілійських діямантових
копалень ще не досяг ціни пересічного півторарічного продукту
бразілійських плантацій цукру або кави, дарма що він репрезентував
куди більше праці, отже, більше вартости. На багатших
копальнях та сама кількість праці реалізувалася б у більшому
числі діямантів, і це зменшило б їхню вартість. Коли б пощастило
з невеликою затратою праці перетворювати вугілля на діямант,
то вартість цього останнього могла б впасти нижче від вартости
цегли. Загалом: що більша продуктивна сила праці, то менший
потрібний для виготовлення якоїсь речі робочий час, то менша
скристалізована в ній маса праці, то менша її вартість. Навпаки,
що менша продуктивна сила праці, то більший робочий час, доконечний
для виготовлення якоїсь речі, то більша її вартість.
Отже, величина вартости товару змінюється просто пропорційно
до кількости й зворотно пропорційно до продуктивної сили праці,
що в ньому реалізується.

[Ми знаємо тепер субстанцію вартости: це — праця. Ми знаємо
міру величини вартости: це — робочий час. Лишається ще проаналізувати
форму вартости, яка саме й надає вартості характеру
мінової вартости. Але спочатку треба дещо докладніше розвинути
вже найдені визначення].\footnote*{
Заведене у прямі дужки ми беремо з першого німецького видання. \emph{Ред.}
}

Річ може бути споживною вартістю, не будучи вартістю. Це
маємо тоді, коли користь від неї людям не є наслідок праці. Наприклад,
повітря, незайманий ґрунт, природні луки, дикорослі
ліси і~\abbr{т. ін.} Річ може бути корисна й бути продуктом людської
праці, не будучи товаром. Хто своїм продуктом задовольняє свою
власну потребу, той, правда, створює споживну вартість, але ж
не товар. Щоб випродукувати товар, він мусить продукувати не
тільки споживну вартість, але споживну вартість для інших,
суспільну споживну вартість. [І не тільки для інших взагалі.
\parbreak{}  %% абзац продовжується на наступній сторінці
