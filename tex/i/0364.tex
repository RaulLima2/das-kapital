що він звільняє 50 робітників, а решту — 50 робітників — уживає
коло машин, які коштують йому 1.500 фунтів стерлінґів. Щоб
справу спростити, ми залишаємо осторонь будівлі, вугілля тощо.
Припустімо, далі, що споживаний кожного року сировинний
матеріял коштує, як і раніш, 3.000 фунтів стерлінґів.214 Чи
«звільнився» через цю метаморфозу якийсь капітал? За старого
способу продукції загальна витрачена сума становила 6.000 фунтів
стерлінґів і складалася наполовину із сталого, наполовину із
змінного капіталу. Тепер вона складається з 4.500 фунтів стерлінґів
(3.000 фунтів стерлінґів на сировинний матеріял та 1.500 фунтів
стерлінґів на машини) сталого та 1.500 фунтів стерлінґів змінного
капіталу. Замість половини, змінна або перетворена на живу робочу
силу частина капіталу становить лише1/4 цілого капіталу. Замість
звільнення відбувається тут зв’язування капіталу в такій формі,
що в ній він перестає обмінюватися на робочу силу, тобто відбувається
перетворення змінного капіталу на сталий. Капітал у 6.000
фунтів стерлінґів, за інших незмінних умов, може тепер давати заняття
не більш, як 50 робітникам. З кожним поліпшенням машин він
дає заняття дедалі меншому числу робітників. Коли б новозаведені
машини коштували менше за суму, що її коштували витиснута
ними робоча сила й знаряддя праці, тобто, наприклад, замість
1.500 фунтів стерлінґів лише 1.000 фунтів стерлінґів, то змінний
капітал у 1.000 фунтів стерлінґів перетворився б на сталий капітал,
тобто був би зв’язаний, а капітал у 500 фунтів стерлінґів
звільнився б. Останній, якщо припустити ту саму річну плату,
становить фонд заняття приблизно для 16 робітників, а звільнено
їх 50 — навіть багато менше, ніж для 16 робітників, бо для того,
щоб ці 500 фунтів стерлінґів перетворити на капітал, треба частину
з них знов перетворити на сталий капітал, і отже, лише частину
з них можна перетворити на робочу силу.

Але припустімо навіть, що виготовлювання нових машин дає
заняття більшому числу механіків. Чи буде це компенсацією
для викинутих на брук шпалерників? У найліпшому випадку
виготовлювання нових машин дасть заняття меншому числу
робітників, ніж витискує вживання машин. Сума в 1.500 фунтів
стерлінґів, яка репрезентує лише заробітну плату звільнених
шпалерників, репрезентує тепер у формі машин: 1) вартість засобів
продукції, потрібних на виготовлення машин; 2) заробітну
плату механікам, що їх виготовляють; 3) додаткову вартість, що
припадає їхньому «хазяїнові». Далі: машина, бувши виготовлена,
аж до самої своєї смерти не потребує, щоб її відновлювали. Отже,
щоб постійно давати заняття додатковому числу механіків, фабриканти
шпалер один по одному мусять витискувати робітників
машинами.

В дійсності ці апологети мають на думці не цей рід звільнення
капіталу. Вони мають на думці засоби існування звільнених робіт-

214 Nota bene. Я подаю ілюстрацію цілком на манір вищеназваних
економістів.
