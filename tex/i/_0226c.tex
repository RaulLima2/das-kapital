\parcont{}  %% абзац починається на попередній сторінці
\index{i}{0226}  %% посилання на сторінку оригінального видання
що міг примушувати восьмилітніх робітників-дітей не лише
надмірно працювати без перерви від другої до пів на дев’яту
годину вечора, але й голодувати!

\settowidth{\versewidth}{Як вексель каже!»}
\begin{verse}[\versewidth]
«Так, серце його, \\
Як вексель каже!»\footnote{
Природа капіталу лишається та сама так у його нерозвинених, як
і в розвинених його формах. У збірці законів, подиктованих території
Нової Мехіки впливом рабовласників незадовго перед початком американської
громадянської війни, сказано: Робітник, оскільки капіталіст
купив його робочу силу, «є його (капіталіста) гроші» («The labourer
is his (the capitalist's) money»). Такий самий погляд був поширений і серед
римських патриціїв. Гроші, що визичали вони плебеєві-винуватцеві,
перетворювалися за допомогою його засобів існування на м'ясо й кров
винуватця. Тим то це «м’ясо й кров» було «їхніми грішми». Звідси шейлоківський
закон 10 таблиць! Гіпотеза Ленґе, нібито кредитори-патриції
влаштовували час-від-часу по той бік Тібру бенкети, де подавано варене
м’ясо винуватців, лишається так само невирішеною, як і гіпотеза
Давмера про християнське причастя.
}
\end{verse}

\noindent{}Те, що фабриканти по-шейлоківському вхопилися за букву
закону 1844~\abbr{р.}, оскільки він реґулює працю дітей, повинно було,
однак, тільки підготувати явний бунт проти цього самого закону,
оскільки він реґулює працю «підлітків і жінок». Пригадаймо, що
скасування «фалшивої Relaissystem» становить головну мету
й головний зміст цього закону. Фабриканти розпочали свій бунт
простою заявою, що пункти закону 1844~\abbr{р.}, які забороняють довільно
вживати праці підлітків і жінок у довільно короткі
періоди п’ятнадцятигодинного фабричного дня, «були порівняно
нешкідливі (comparatively harmless) доти, доки робочий час
обмежувано 12 годинами\footnote*{
У французькому виданні цю фразу зредаґовано так: «Фабриканти
розпочали свій бунт простою заявою, що пункти закону 1844~\abbr{р.}, які забороняють
досхочу вживати праці підлітків і жінок, примушуючи їх о якій
завгодно порі дня переривати і знову братися до праці, були порівняно
дрібницею доти, доки мав силу 12-годинний робочий час». («Le Capital,
etc.», v. I, ch. X, p. 124). \emph{Ред.}
}. Але за десятигодинного закону вони
є нестерпна кривда» (hardship)\footnote{
«Reports etc. for 30 th April 1848». p. 28.
}. Тому вони байдужісінько
заявили інспекторам, що не зважатимуть на букву закону і
самовладно знову запровадять стару систему\footnote{
Так, між іншим, заявив філантроп Ешворд у своєму огидному
квакерівському листі до Леонарда Горнера («Reports etc., April 1849»,
р. 4).
}. Це, мовляв,
буде в інтересах самих же робітників, спантеличених дурними
порадами, бо «дасть змогу платити їм вищу заробітну плату».
«Це однісінький можливий плян, щоб за десятигодинного закону
зберегти промислову перевагу Великобрітанії»\footnote{
Там же, стор. 134.
}. «Можливо,
що за системи змін (Relaissystem) трохи важко викривати порушення
закону, але що з того? (what of that?) Хіба ж можна
великі промислові інтереси цієї країни розглядати як другорядну
річ заради того, щоб фабричним інспекторам і підінспекторам
заощадити трохи більше клопоту (some little trouble)»\footnote{
Там же, стор. 140.
}.
