\parcont{}  %% абзац починається на попередній сторінці
\index{i}{0116}  %% посилання на сторінку оригінального видання
ближче до цього. Перед обміном було вина на 40 фунтів стерлінґів
у руках А і збіжжя на 50 фунтів стерлінґів у руках В, разом
вартости на 90 фунтів стерлінґів. Після обміну маємо ту саму
загальну вартість в 90 фунтів стерлінґів. Вартість, що циркулює,
не збільшилась ні одним атомом, а тільки змінився розподіл її
між А й В. Те, що на одному боці з’являється як додаткова вартість,
є зменшення вартости на другому боці, плюс на одному
боці є мінус на другому. Та сама зміна відбулася б, коли б А, не
прикриваючись формою обміну, безпосередньо украв у В 10 фунтів
стерлінґів. Очевидно, що суму вартости, яка циркулює, не
можна збільшити жодною зміною в її розподілі, так само як єврей
не збільшує маси благородного металю в країні тим, що продає
за одну ґінею один фартинґ із часів королеви Ганни. Ціла
кляса капіталістів країни не може вигравати коштом самої себе.\footnote{
Детю де Трасі, хоч він є член Інституту, — а, може, саме тому,
що він є член Інституту, — був протилежної думки. «Промислові капіталісти,
— каже він, — мають свої зиски з того, що вони всі товари продають
дорожче, ніж коштувала їх продукція. Кому ж продають вони їх?
Поперше, один одному». («Traité de la Volonté et de ses effets», Paris
1826, p. 239).
}

Хоч верть, хоч круть, а факт лишається той самий. Коли обмінюється
еквіваленти, то не постає жодної додаткової вартости,
і коли обмінюється не-еквіваленти, то теж не постає жодної додаткової
вартости.\footnote{
«Обмін двох рівних вартостей не збільшує й не зменшує маси вартостей,
що є в суспільстві. Обмін двох нерівних вартостей\dots{} також нічого
не змінює в сумі суспільних вартостей, хоч і додає до майна одного те,
що він бере від майна іншого». («L’échange qui se fait de deux valeurs
égales n’augmente ni ne diminue la masse des valeurs exisstantes dans la
société. L’échange de deux valeurs inégales\dots{} ne change rien non plus à ia
somme des valeurs sociales, bien qu’il ajoute à la fortune de l’un ce qu’il
ôte de la fortune de l’autre»). (J. B. Say: «Traité d’Economie Politique»,
Paris 1817, vol. II, p. 443, 444). Сей, звичайно, байдужий щодо висновків
з цієї тези, запозичує її майже дослівно в фізіократів. У який спосіб використовував
він для збільшення своєї власної «вартости» забуті за його
часів твори фізіократів, показує такий приклад: «Найславнішу» тезу
Monsi ur Say’a: «On n’achète des produits qu’avec des produits» («Продукти
купується лише за продукти» — там же, т. II, стор. 441) в ориґіналі
у фізіократів подано так: «Les productions ne se paient qu’avec des
productions» («Продукти оплачується лише продуктами»). (Le Trosne:
«De l’Intérêt Social», Physiocrates, éd. Daire, Paris. 1846, p. 899).
} Циркуляція або обмін товарів не створює
жодної вартости.\footnote{
«Обмін не додає жодної вартости до продуктів» («Exchange confers
no value at all upon products»). (F. Wayland: «The Elements of Political
Economy», Boston 1853, p. 168).
}

Звідси зрозуміло, чому в нашій аналізі основної форми капіталу,
форми, що в ній він визначає економічну організацію сучасного
суспільства, ми покищо цілком залишаємо осторонь популярні
і, так би мовити, допотопні форми капіталу — торговельний
капітал та лихварський капітал.

У русі власне торговельного капіталу форма $Г — Т — Г'$,
купити, щоб дорожче продати, з’являється в найчистішому
\parbreak{}  %% абзац продовжується на наступній сторінці
