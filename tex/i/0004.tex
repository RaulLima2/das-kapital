вартості товари мають насамперед різну якість, як мінові вартості
вони можуть різнитися лише щодо кількости, отже, не
містять у собі жодного атома споживної вартости.

Коли залишити осторонь споживну вартість товарових тіл,
то в них зостається ще тільки одна властивість, а саме та, що
вони є продукти праці. Однак і самий продукт праці вже зазнав
у нас перетворення. Абстрагуючись від його споживної вартости,
ми тим самим абстрагуємось і від його тілесних складових частин
і форм, які роблять його споживною вартістю. Це вже не стіл або
дім, або пряжа, абож інша корисна річ. Всі його почуттєво сприймані
властивості зникли. Це вже й не продукт теслярської, будівельної
або ткальської праці, або взагалі якоїбудь іншої певної
продуктивної праці. Разом з корисним характером продуктів
праці зникає й корисний характер втіленої в них праці, отже,
зникають і різні конкретні форми цих праць; вони вже не відрізняються
одна від однієї, а всі вони зведені на однакову людську
працю, абстрактну людську працю.

Розгляньмо тепер Residuum,* що лишається від продуктів
праці після цього зведення. Від них не залишилося нічого іншого,
крім однакової для них усіх примарної предметности (Gegenständlichkeit**),
простого згустка безріжницевої (unterschiedsloser)
людської праці, тобто затрати людської робочої сили незалежно
від форми її затрати. Ці речі виявляють тепер тільки те,
що на їхню продукцію затрачено людську робочу силу, що в них
нагромаджено людську працю. Як кристалі цієї спільної їм суспільної
субстанції, вони є вартості — товарові вартості.

У самому міновому відношенні товарів їхня мінова вартість
з’явилася перед нами як щось цілком незалежне від їхньої споживної
вартости. Якщо ж ми дійсно абстрагуємося від споживної
вартости продуктів праці, то матимемо їх вартість, як її щойно
визначено. Отже, те спільне, що виявляється в міновому відношенні
або міновій вартості товару, є його вартість. Дальший дослід
знов приведе нас до мінової вартости як доконечного способу
виразу, або доконечної форми виявлення товарової вартости; цю
останню, однак, спочатку треба розглянути незалежно від цієї форми.

Отже, споживна вартість, або добро, має вартість лише тому,
що в ній упредметнено або зматеріялізовано абстрактну людську
працю. Як же виміряти величину вартости споживної вартости?
Кількістю «вартостетворчої субстанції», що міститься в ній,
кількістю праці. Кількість самої праці вимірюється часом тривання
праці, а робочий час має знов таки свій маштаб у певних
частинах часу, як от година, день і т. ін.

Могло б здаватися, що коли вартість товару визначається
кількістю праці, витраченої підчас його продукції, то що леда-

* — остачу. Ред.

** Німецьке «Gegenständlichkeit» ми перекладаємо тут, так само як
і всюди далі словом «предметність» у розумінні чогось, що об’єктивно
існує. У французькому виданні це слово перекладено словом «réalité»,
що означає — дійсне існування, дійсний предмет або реальність. Ред.
