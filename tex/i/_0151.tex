\parcont{}  %% абзац починається на попередній сторінці
\index{i}{0151}  %% посилання на сторінку оригінального видання
зужитковує удвоє більше матеріялу, який має удвоє більшу
вартість, і зужитковує удвоє більше машин, що мають удвоє
більшу вартість, отже, зберігає в продукті двох тижнів удвоє
більше вартости, ніж у продукті одного тижня. За даних незмінних
умов продукції робітник зберігає то більше вартости, що
більше вартости він додає; але він зберігає більше вартости не
тому, що додає більше вартости, а тому, що додає її за незмінних
і незалежних від його власної праці умов.

Звичайно, в деякому відносному розумінні можна сказати,
що робітник завжди зберігає старі вартості в тій самій пропорції, в
якій він додає нову вартість. Чи піднесеться вартість бавовни з
1\shil{ шилінґа} на 2\shil{ шилінґи}, чи спаде на 6\pens{ пенсів}, робітник завжди
зберігає в продукті однієї години лише удвоє меншу вартість
бавовни, ніж у продукті двох годин, хоч би й як змінялася ця
вартість. Далі, коли змінюється продуктивність його власної праці,
коли вона підноситься або падає, то він, приміром, за одну робочу
годину випряде більше або менше бавовни, ніж раніше, і відповідно
до цього збереже більшу або меншу вартість бавовни у
продукті однієї робочої години. Але за всім тим він за дві робочі
години збереже удвоє більше вартости, ніж за одну робочу годину.

Вартість, залишаючи осторонь її суто символічний вираз у
знаках вартости, існує лише в якійсь споживній вартості, в якійсь
речі. (Сама людина, розглядувана просто лише як буття робочої
сили, є предмет природи, річ, хоч і жива, самосвідома річ, а сама
праця є речове виявлення цієї сили). Тому, коли гине споживна
вартість, то гине й вартість. Засоби ж продукції не втрачають
своєї вартости одночасно із своєю споживною вартістю, бо в наслідок
процесу праці вони втрачають первісну форму своєї споживної
вартости в дійсності лише на те, щоб у продукті набрати форми
іншої споживної вартости. Але, хоч і як важливо для вартости
існувати в якійсь споживній вартості, для неї, як це доводить
метаморфоза товарів, байдуже, в якій споживній вартості вона
існує. Звідси випливає, що в процесі праці вартість переходить
із засобу продукції на продукт лише тією мірою, якою засіб продукції
разом із своєю самостійною споживною вартістю втрачає
й свою мінову вартість. Він віддає продуктові лише ту вартість,
яку він втрачає як засіб продукції. Але з цього погляду зрізними
речовими факторами процесу праці справа стоїть неоднаково.

Вугілля, що ним опалюють машину, зникає безслідно, так
само мастиво, що ним мастять вісь колеса, і~\abbr{т. ін.} Фарби й інші
допоміжні матеріяли зникають, але виявляються у властивостях
продукту. Сировинний матеріял становить субстанцію продукту,
але змінює свою форму. Отже, сировинний матеріял і допоміжні
матеріяли втрачають самостійну форму, в якій вони увійшли до
процесу праці як споживні вартості. Інша справа з власне засобами
праці. Інструмент, машина, фабричний будинок, бочка й~\abbr{т. ін.} служать у процесі праці лише доти, доки зберігають вони
свою первісну форму, доки й завтра можуть входити в процес
праці в тій самій формі, що й учора. Як за свого життя, тобто
\parbreak{}  %% абзац продовжується на наступній сторінці
