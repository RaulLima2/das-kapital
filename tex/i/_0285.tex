\parcont{}  %% абзац починається на попередній сторінці
\index{i}{0285}  %% посилання на сторінку оригінального видання
між мануфактурним поділом праці і суспільним поділом праці,
який становить загальну основу всякої товарової продукції.

Коли мати на оці лише саму працю, то поділ суспільної продукції
на її великі галузі, як ось рільництво, промисловість
і~\abbr{т. д.}, можна назвати поділом праці взагалі (im Allgemeinen),
поділ цих галузей продукції на роди й підроди — поділом праці
спеціальним (im Besondern), а поділ праці всередині майстерні
можна назвати поділом праці детальним (im Einzelnen)\footnote{
«Поділ праці починається з поділу якнайрізнорідніших професій
і розвивається аж до того поділу, коли значне число робітників поділяють
між собою виготовлення одного й того самого продукту, як, приміром,
у мануфактурі». (\emph{Slorch}: «Cours d’Economie Politique», паризьке
видання, т. І, стор. 173). «У народів, що досягли певного ступеня цивілізації,
ми знаходимо три роди поділу продукції: перший, який ми назвемо
загальним, призводить до поділу продуцентів на рільників, промисловців
та купців: він стосується до трьох головних галузей національної
продукції; другий, який можна було б назвати спеціяльним, є поділ
кожного роду продукції на підроди\dots{} насамкінець, третій рід поділу
продукції, який слід було б назвати поділом професій або праці у власному
значенні слова, є той поділ, що встановлюється всередині окремих реместв
і професій\dots{} що встановлюється всередині більшости мануфактур
та майстерень» («Nous rencontrons chez les peuples parvenus à un certain
degré de civilisation trois genres de divisions d’industrie: la première,
que nous nommerons générale, amène la distinction des producteurs en
agriculteurs, manufacturiers et commerçants, elle se rapporte aux trois principales
branches d’industrie nationale; la seconde, que l’on pourrait appeler
spéciale, est la division de chaque genre d’industrie en espèces\dots{} la troisième
division d’industrie, celle enfin que l’on devrait qualifier\dots{} de division
de besogne ou de travail proprement dit, est celle qui s’établit dans les arts
et métiers séparés\dots{} qui s’établit dans la plupart des manufactures et des
ateliers»). (\emph{Skarbeck}: «Théorie des richesses sociales», 2 ème éd. Paris
1840, vol. I, p. 84. 85).
}.

Поділ праці всередині суспільства та відповідне обмеження
індивідів сферами окремих професій розвивається, як і поділ
праці всередині мануфактури, з протилежних вихідних пунктів.
Всередині родини\footnoteA{
Примітка до третього видання. — Пізніші ґрунтовні досліди
над первісним станом людей довели автора до висновку, що первісно не
родина розвивається в рід, а навпаки, рід був первісною природно
вирослою формою людського суспільства, побудованого на спорідненні
крови, так що багатовідмінні форми родини розвинулися лише пізніше,
коли почали розпадатися родові зв'язки. (\emph{Ф. Е.}).
}, а з дальшим розвитком і всередині роду,
виникає природно вирослий поділ праці з ріжниць статі й віку,
отже, на суто фізіологічній основі, яка, з поширенням громади,
із зростом людности й особливо з виникненням конфліктів поміж
різними родами та з поневоленням одного роду іншим, — поширює
свій матеріял. А з другого боку, як я вже раніш зазначив,
обмін продуктами виникає на тих пунктах, де різні родини,
роди, громади входять у контакт, бо на початках культури не
окремі індивіди, а родини, роди й~\abbr{т. д.} самостійно протистоять одне
одному. Різні громади находять різні засоби продукції й різні
засоби існування в тій природі, що їх оточує. Тим-то спосіб їхньої
продукції, спосіб їхнього життя та продукти дуже відрізняються
\parbreak{}  %% абзац продовжується на наступній сторінці
