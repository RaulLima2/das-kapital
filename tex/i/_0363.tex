\parcont{}  %% абзац починається на попередній сторінці
\index{i}{0363}  %% посилання на сторінку оригінального видання
силує бунтівничну руку праці до покірливости», він обурюється
з того, «що з певного боку обвинувачують механічно-фізичну
науку в тому, що вона віддалася на волю деспотизмові багатих
капіталістів та погодилася бути засобом утиску бідних кляс».
Після довгого та голосного проповідування корисности для робітників
швидкого розвитку машин, він застерігає їх, що вони своїм
опором, страйками й т. ін. прискорюють розвиток машин. «Такі
ґвалтовні заколоти, — каже він, — виявляють короткозорість людини
в її найогидливішій формі, короткозорість людини, що сама
себе робить своїм катом». Декілька сторінок перед тим читаємо
протилежне: «Без тих гострих колізій та перерв, спричинюваних
помилковими поглядами робітників, фабрична система була б
розвинулася далеко швидше та з далеко більшою корисністю
для всіх заінтересованих сторін». А далі знову вигукує: «На
щастя для людности фабричних округ Великобританії поліпшення
в механіці відбуваються лише поступінно». «Несправедливо, —
каже він, — обвинувачують машини в тому, що вони зменшують
заробітну плату дорослих, витискуючи певну частину з них, через
що число їх перевищує потребу в праці. Але ж вони збільшують
попит на працю дітей та підносять таким чином рівень заробітної
плати дорослих». З другого боку, цей самий утішник боронить
низьку заробітну плату дітей тим, що «вона спиняє батьків посилати
надто рано дітей до фабрик». Ціла його книжка — то апологія
необмеженого робочого дня, і коли законодавство забороняє
мордувати 12-літніх дітей більше, ніж 12 годин на добу, то це
нагадує його ліберальній душі найтемніші часи середньовіччя.
Це не заважає йому закликати фабричних робітників до молитви
з подякою провидінню за те, що воно за допомогою машин
«дало їм вільний час розмірковувати над своїми безсмертними
інтересами».\footnote{
Ure: «Philosophy of Manufacture», стор. 368, 7, 370, 280, 321, 281, 475.
}

6. Теорія компенсації відносно робітників, витискуваних
машинами

Цілий ряд буржуазних економістів, як от Джеме Мілл, Мак
Куллох, Торенс, Сеніор, Джон Стюарт Мілл і інші, твердять,
що всі машини, які витискують робітників, завжди й неминуче
звільняють у той самий час відповідний капітал, щоб дати заняття
тим самим робітникам.\footnote{
Рікардо спочатку поділяв цей погляд, але пізніше з характеристичною
для нього науковою безсторонністю та любов’ю до правди виразно
відмовився від нього. Див. David Ricardo: «Principles of Political Economy»,
розд. 31. «On Machinery».
}

Припустімо, що якийсь капіталіст уживає 100 робітників,
приміром, у шпалерній мануфактурі, при річній заробітній платі
в 30 фунтів стерлінґів на кожного. Отже, витрачуваний ним річно
змінний капітал становить 3.000 фунтів стерлінґів. Припустімо,
\parbreak{}  %% абзац продовжується на наступній сторінці
