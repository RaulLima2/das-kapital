ком на кшталт старої діви, вкладає в уста свого «прегарного
ідеалу», капіталіста, такі слова, звернені до «зайвих» робітників,
викинутих на брук додатковим капіталом, що вони сами його
створили: «Ми, фабриканти, збільшуючи капітал, з якого ви
мусите жити, робимо для вас усе, що можемо; а ви мусите зробити
решту, пристосовуючи свою чисельність до засобів існування».\footnote{
Harriet Martineau: «The Manchester Strike», 1842, p. 101.
}

Для капіталістичної продукції ні в якому разі недосить тієї
кількости вільної робочої сили, що її дає природний приріст
людности. Для свого вільного розвитку вона потребує промислової
резервної армії, незалежної від цієї природної межі.

Досі ми припускали, що збільшення або зменшення змінного
капіталу точно відповідає збільшенню або зменшенню числа
занятих робітників.

Однак і за незмінного або навіть зменшеного числа робітників
змінний капітал, що панує над ними, зростає, якщо індивідуальний
робітник дає більше праці і в наслідок цього зростає
його заробітна плата, хоч ціна праці лишається незмінна, а то
навіть падає, тільки повільніше, ніж зростає маса праці. Тоді
приріст змінного капіталу стає показником більшої кількости
праці, але не більшої кількости занятих робітників. Кожний
капіталіст має абсолютний інтерес у тому, щоб видушити певну
кількість праці з меншого, а не з більшого числа робітників,
хоча б останнє коштувало так само дешево, а то й дешевше.
В останньому випадку видатки на сталий капітал зростають
пропорційно до маси праці, пущеної в рух, у першому випадку
вони зростають далеко повільніше. Що більший маштаб продукції,
то вирішальніший є цей мотив. Його вага зростає з акумуляцією
капіталу.

Ми бачили, що розвиток капіталістичного способу продукції
і продуктивної сили праці — одночасно причина й наслідок
акумуляції — дають капіталістові спроможність, за однакової витрати
змінного капіталу, через екстенсивнішу або інтенсивнішу
експлуатацію індивідуальних робочих сил пускати в рух більше
праці. Далі ми бачили, що він за ту саму капітальну вартість
купує більше робочої сили, щораз більше витискуючи навчених
робітників менш навченими, дозрілих робітників — недозрілими,
чоловіків — жінками, дорослих — підлітками й дітьми.

Отже, з проґресом акумуляції більший змінний капітал, з
одного боку, пускає в рух більше праці, не наймаючи більшого
числа робітників, з другого боку, змінний капітал тієї самої
величини пускає в рух більше праці за тієї самої маси робочої
сили і, нарешті, витискуючи робочі сили вищої якости, пускає
в рух більше робочих сил нижчої якости.

Тому продукція відносного перелюднення або звільнення
робітників іде ще швидше, ніж технічний переворот процесу
продукції, і без того прискорюваний проґресом акумуляції, і