\index{i}{0465}  %% посилання на сторінку оригінального видання
За відштучної плати на перший погляд здається, наче та
споживна вартість, яку продає робітник, є не функція його
робочої сили, не жива праця, а праця, упредметнена вже в
продукті, і наче ціна цієї праці визначається не дробом
денна вартість робочої сили/робочий день даного числа годин,
як за почасової плати, а дієздатністю продуцента.\footnote{
«Система відштучної праці становить певну епоху в історії робітника;
це є щось середнє між становищем простого поденника, який залежить
від волі капіталіста, та становищем кооперативного робітника, що
в недалекій будучині обіцяє сполучити у своїй власній особі робітника
й капіталіста. Відштучні робітники фактично сами є для себе хазяїни,
навіть коли працюють за допомогою капіталу свого підприємця». («The
system of piece-work illustrates an epoch in the history of the working
man: it is halfway between the position of the mere daylabourer, depending
upon the will of the capitalist, and the cooperative artizan, who in
the not distant future promises to combine the artizan and the capitalist
in his own person. Piece-workers are in fact their own masters, even whilst
working upon the capital of the employer»). (John Watts: «Trade-Societies
and Strikes, Machinery and Cooperative Societies», Manchester 1865,
p. 52, 53). Я цитую це писаннячко, бо це є справжня кльоака для всіх
давно вже зогнилих апологетичних банальностей. Той самий пан Вотс
працював раніш в оуенівському русі та 1842 р. опублікував інше писаннячко
«Facts and Fictions of Political Economy», де він, між іншим, власність
(property) називає грабіжництвом (robbery). Це вже давно минулося.
}

Певність правдивости цього, побудованого на позірності, погляду
мусила б насамперед дуже захитатися вже від того факту,
що обидві форми заробітної плати одночасно існують одна побіч
однієї в тих самих галузях промисловости. Наприклад, «лондонські
складачі працюють звичайно за відштучну плату, почасова
плата є в них виняток. Навпаки, у складачів на провінції
почасова плата є правило, авідштучна плата виняток. Корабельні
теслярі по гаванях Лондону одержують відштучну плату, по
всіх інших англійських гаванях — почасову».\footnote{
Т. J. Dunning: «Trades-Unions and Strikes», London 1860,
p. 22.
} В Лондоні в
тих самих лимарнях часто за ту саму працю французи дістають
відштучну, англійці — почасову плату. На власне фабриках,
де взагалі переважає відштучна плата, окремі функції праці з
технічних причин виключається з цього способу виміру, і тому
їх оплачують почасово.\footnote{
Ось як це одночасне існування одної поряд одної цих двох форм
заробітної плати сприяє шахрайствам фабрикантів: «Фабрика вживає
400 робітників, що з них половина працює відштучно та безпосередньо
зацікавлена у збільшенні числа робочих годин. Решті 200 робітникам
платять поденно, вони працюють так само довго, алеж нічого не дістають
за наднормову працю\dots{} Праця цих 200 людей протягом півгодини
на день дорівнює праці однієї особи протягом 50 годин, або \sfrac{5}{6} тижневої
праці однієї особи, та є позитивний виграш для підприємця». («А factory
employs 400 people, the half of which work by the piece, and have a direct
interest in working longer hours. The other 200 are paid by the day,
work equally long with the others, and get no more money for their overtime\dots{}
The work of these 200 people for half an hour a day is equal to one
person’s work for 50 hours, or 6/e of one person’s labour in a week, and is
} Однак, само собою ясно, що ріжниця
\index{i}{0466}  %% посилання на сторінку оригінального видання
форм у виплаті заробітної плати нічого не зміняє в її суті, дарма
що одна форма може бути сприятливіша для розвитку капіталістичної
продукції, ніж друга.

Хай звичайний робочий день становить 12 годин, із них 6
оплачених, 6 неоплачених. Припустімо, що спродукована протягом
нього вартість дорівнює 6 шилінґам, отже, вартість, спродукована
протягом однієї робочої години — 6 пенсам. Хай,
далі, досвід показує, що робітник, який працює з середнім
ступенем інтенсивности та вправности, отже, який у дійсності
витрачає лише робочий час, суспільно-доконечний на продукцію
якогось товару, дає протягом 12 годин 24 штуки, хоч поодиноких
окремих продуктів, хоч вимірних частин одного цілого продукту.
Таким чином вартість цих 24 штук, відлічивши частину
сталого капіталу, що міститься в них, буде 6 шилінґів, а вартість
окремої штуки — 3 пенси. Робітник дістає 1\% пенси за
штуку й таким чином заробляє за 12 годин 3 шилінґи. Як за
почасової плати байдуже, чи ми припустимо, що робітник працює
6 годин на себе, а 6 годин на капіталіста, або що з кожної години
він працює половину на себе, а другу — на капіталіста, так само
й тут байдуже, чи ми скажемо, що кожна окрема штука напівоплачена
й напівнеоплачена, або що ціна 12 штук покриває лише
вартість робочої сили, тимчасом як в 12 інших штуках утілюється
додаткова вартість.

Форма відштучної плати так само іраціональна, як і форма
почасової плати. Тимчасом як, наприклад, дві штуки товару,
за вирахуванням вартости зужиткованих на них засобів продукції,
варті, як продукт однієї робочої години, 6 пенсів, робітник дістає
за них ціну в 3 пенси. В дійсності відштучна плата безпосередньо
не виражає ніякого відношення вартости. Тут справа не в тому,
щоб виміряти вартість штуки товару втіленим у ній робочим
часом, а в тому, щоб, навпаки, витрачену робітником працю виміряти
числом спродукованих ним штук товару. За почасової
плати працю вимірюється безпосередньо часом її тривання, за
відштучної плати — кількістю тих продуктів, що в них праця
згусла протягом певного часу.\footnote{
«Заробітну плату можна міряти двома способами: або триванням
праці, або її продуктом» («Le salaire peut se mesurer de deux manières:
ou sur la durée du travail, ou sur son produit»). («Abrège élémentaire des
principes de l’Economie Politique», Paris 1796, p. 32). Автор цієї анонімної
праці — G. Garnier.
} Ціна самого робочого часу кі-

а positive gain to the employer»). («Reports of Insp. of Fact, for 31 st
October 1860», p. 9). «Наднормова праця все ще панує в широких розмірах,
і здебільшого сам закон Гарантує фабрикантам можливість уникнути
викриття та кари. У багатьох моїх попередніх звітах я вже зазначав\dots{}
яку кривду роблять усім робітникам, що дістають не відштучну, а потижневу
заробітну плату» (Overworking, to a very considerable extent,
still prevails; and, in most instances, with that security against detection
and punishment which the law itself affords. I have in many former reports
shown\dots{} the injury to all the workpeople who are not employed on piecework,
but receive weekly wages»). (Leonhard Horner y «Reports of Insp.
of Fact, for 30 th April 1859», p. 8, 9).
\index{i}{0467}  %% посилання на сторінку оригінального видання
нець-кінцем визначається рівнанням: вартість денної праці =
денній вартості робочої сили. Отже, відштучна плата — це лише
змодифікована форма почасової плати.

Розгляньмо тепер трохи ближче характеристичні особливості
відштучної плати.

Якість праці контролюється тут через самий продукт, який
мусить бути пересічної добротности, щоб одержати повну відштучну
ціну. З цього боку відштучна плата стає найстрашнішим
джерелом відраховань із заробітної плати й капіталістичного
шахрайства.

Вона дає капіталістові цілком визначену міру інтенсивности
праці. Лише той робочий час, що втілюється у заздалегідь визначеній
і досвідом встановленій кількості товарів, уважається за
суспільно-доконечний робочий час та оплачується як такий.
Тому по великих кравецьких майстернях Лондону певна штука
продукту праці, наприклад, жилет тощо, зветься годиною,
півгодиною й т. ін., а годину рахується по 6 пенсів. З практики
відомо, скільки пересічно продукту виробляється за годину.
При зміні моди або полагодженнях тощо постає суперечка між
підприємцем та робітником, чи певна штука дорівнює одній годині
і т. д., аж поки й тут досвід вирішить справу. Те саме бачимо й
по лондонських майстернях меблів і т. ін. Коли робітник не має
пересічної дієздатности й тому не може дати певний мінімум денного
продукту, то його звільняють.\footnote{
«Йому (прядунові) видають певну кількість фунтів бавовни, і за
певний час він мусить дати натомість стільки й стільки фунтів ниток
або пряжі певної тонини, і він одержує стільки й стільки плати за кожний
даний ним фунт продукту. Коли продукт щодо якости має хиби, то
на нього накладають пеню; коли ж кількість менша за мінімум, визначений
для даного часу, то його звільняють та замінюють умілішим робітником».
(«So much weight of cotton is delivered to him (the spinner),
and he has to return by a certain time, in lieu of it, a given weight of
twist or yarn, of a certain degree of fineness, and he is paid so much per
pound for all that he so returns. If his work is defective in quality, the
penalty falls on him; if less in quantity than the minimum fixed for a
given time, he is dismissed and an abler operative procured»). (C7re: «Philosophy
of Manufacture», p. 316).
}

А що якість та інтенсивність праці тут контролюються через
саму форму заробітної плати, то догляд за працею у значній мірі
стає зайвий. Тому ця форма становить основу так змальованої
вже раніш сучасної домашньої праці, як і ієрархічно організованої
системи експлуатації та пригнічення. Ця остання система
має дві основні форми. Відштучна плата, з одного боку, полегшує
паразитам всуватися поміж капіталіста й найманого робітника,
перепродавати працю (subletting of labour). Бариш посередників
випливає виключно з ріжниці між ціною праці, яку платить
капіталіст, і тією частиною цієї ціни, яку посередники дійсно
платять робітникові.80 Ця система має в Англії характеристичну

50 «Коли праця переходить через багато рук і кожна особа хоче
мати свою частину зиску, тимчасом як лише остання з них дійсно працює,
тоді плата, яку дістає робітниця, є надто, мізерна» («It is when work
\index{i}{0468}  %% посилання на сторінку оригінального видання
назву «sweating-system» (потогінна система). З другого боку,
відштучна плата дозволяє капіталістові складати з головним робітником
— у мануфактурі з старшим групи, по копальнях із
вибійником і т. ін., на фабриці з власне машиновим робітником —
умову на певну ціну за штуку, при чому головний робітник сам
бере на себе обов’язок за цю ціну вербувати та оплачувати своїх
робітників-помічників. Експлуатація робітників капіталом здійснюється
тут через експлуатацію робітників робітником».\footnote{
Навіть апологет Вотс зауважує: «Було б велике поліпшення
системи відштучної плати, коли б усі робітники, заняті тією самою працею,
були учасниками умови, кожний відповідно до своєї здібности, замість
того, шоб один з них був зацікавлений у надмірній праці своїх
товаришів собі особисто на користь» («It would be a great improvement
to the system of piece-work, if all the men employed on a job were partners
in the contract, each according to his abilities, instead of one man
being interested in overworking his fellows for his own benefit» (Там же,
crop. 53). Про підлоти цієї системи див. «Children’s Employment Commission.
З rd Report», p. 66, n. 22, p. 11, n 124, p. XI, n. 13, 53, 59 і т. д.
}

Коли відштучна плата вже існує, то, ясна річ, особистий інтерес
робітника спонукає його якомога інтенсивніше напружувати
свою робочу силу, а це полегшує капіталістові підвищувати нормальний
ступінь інтенсивности. 51а Так само особистий інтерес
робітника спонукає його здовжувати робочий день, бо таким
чином зростає його поденна або потижнева заробітна плата.\footnote{
«Всі ті, що дістають відштучну плату\dots{} мають користь працювати
поза визначені законом меніі робочого дня. Згоду працювати понаднормовий
час можна спостерігати особливо часто серед жінок-ткачих та мотальниць».
(«All those who are paid by piece-work\dots{} profit by the transgression
of the legal limits of work. This observation as to the willingness
to work overtime, is especially applicable to the women employed as weavers
and reelers»). («Reports of Insp. of Fact, for 30 th April 1858», p. 9).
«Ця система відштучної плати, така корисна для капіталіста, скерована
прямо на те, щоб спонукати молодого ганчаря до наднормової праці
протягом 4--5 років, при чому він дістає відштучну, але дуже низьку плату.
Це одна з головних причин, що зумовлюють фізичну дегенерацію ганчарів».
(«Children’s Employment Commission. 1 st Report», P-XIII).
}

passes through several hands, each of which is to take a share of profits,
while only the last does the work, that the pay which reaches the workwoman
is miserably disproportioned»). («Children’s Employment Commission.
2 nd Report», p. LXX, n. 424).

51a Цьому природному результатові часто допомагають штучно.
Приміром, у машинобудівництві (Engineering Trade) Лондону за звичайнісінький
вважається такий маневр: «капіталіст добирає на старшину
певного числа робітників людину надзвичайної фізичної сили та вправности.
Що чверть року або в інші реченці він виплачує їй додаткову плату
з умовою, щоб вона зробила все можливе для того, щоб спонукати своїх
товаришів у праці, які дістають лише звичайну плату, до якнайкрайнішого
змагання\dots{} Це без дальших коментарів пояснює скарги капіталістів на
те, що тред-юньйони «паралізують енерґію, видатну вправність та
робочу силу» («stinting the action, superior skill and working power»).
IDunning: «Trades-Unions and Strikes», London 1860, p. 23, 23). Через
те, що автор сам є робітник та секретар тред-юньйону, то ці його слова
можуть видатися за перебільшення. Але погляньмо тоді, наприклад,
до «високореспектабельної» («highly respectable») агрономічної енциклопедії
Дж. Ч. Мортона, до статті «Labourer», де цю методу рекомендується
фармерам як випробовану.
\parbreak{}  %% абзац продовжується на наступній сторінці
