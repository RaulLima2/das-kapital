\parcont{}  %% абзац починається на попередній сторінці
\index{i}{0108}  %% посилання на сторінку оригінального видання
набирає у простій циркуляції, упосереднюють лише обмін товарів
і зникають у кінцевому результаті руху. Навпаки, в циркуляції
$Г — Т — Г$ функціонують обидва, товар і гроші, лише як різні
способи існування самої вартости: гроші як її загальна, товар
як її осібна, так би мовити, лише замаскована форма існування.\footnote{
«Не речовина становить капітал, а вартість цієї речовини» («Се
n’est pas la matière, qui fait le capital, mais la valeur de cette matière»).
(\emph{J. B. Say}: «Traité d’Economie Politique», 3-ème éd. Paris 1817, vol. II,
p. 429).
}
Вартість постійно переходить з однієї форми в другу, ніколи не
зникаючи в цьому русі, і таким чином перетворюється на автоматичний
суб’єкт. Коли фіксувати осібні форми виявлення, що
їх у кругобігу свого життя навпереміну набирає вартість, що
самозростає, то виходять такі визначення: капітал є гроші, капітал
є товар.\footnote{
«Обігові гроші (!), що вжиті з продуктивною метою, є капітал»
(«Currency (!) employed to productive purposes is capital»). (\emph{Mac Leod}:
«The Theory and Practice of Banking», London 1855, vol. I, ch. 1, p. 55).
«Капітал — це товари» («Capital is commodities»). (James Mill: «Elements
of Political Economy», London 1821, p. 74).
} Але в дійсності вартість стає тут суб’єктом процесу,
в якому вона, постійно змінюючи свою грошову форму на товарову
й товарову на грошову, сама змінює свою величину, відштовхує
себе як додаткову вартість від самої себе як первісної
вартости, самозростає. Бо рух, що в ньому вона прилучає до себе
додаткову вартість, є її власний рух, отже, її зростання є самозростання.
Вона набула магічної якости плодити вартість через
те, що вона є вартість. Вона виплоджує живі діти або, принаймні,
несе золоті яйця.

Вартість як активний суб’єкт такого процесу, що в ньому
вона, то набираючи, то скидаючи грошову форму й товарову
форму, в цій зміні зберігає себе й зростає, потребує насамперед
самостійної форми, за допомогою якої констатується її тотожність
із нею самою. І цю форму вона має лише в грошах. Тому гроші
становлять вихідний і кінцевий пункт кожного процесу зростання
вартости. Вона дорівнювала 100 фунтам стерлінґів, вона тепер
є 110 фунтів стерлінґів і т. д. Але сами гроші мають тут силу
лише як одна з форм вартости, бо у неї їх дві. Не набравши товарової
форми, гроші не стають капіталом. Отже, гроші тут не виступають
проти товарів вороже, як за скарботворення. Капіталіст
знає, що всі товари, хоч якими обідраними вони виглядали б,
або хоч як погано вони пахли б, по вірі й правді є гроші, так би
мовити, євреї істинного обрізання, а до того ще й чудотворний
засіб із грошей робити більше грошей.

Коли в простій циркуляції вартість товарів супроти їхньої
споживної вартости набирає щонайбільше самостійної форми
грошей, то тут вона раптом виступає як субстанція, що процесує,
сама собою рухається, субстанція, що для неї товар і гроші обоє
є лише форми; навіть більше: замість репрезентувати відношення
товарів вона вступає тепер, так би мовити, у приватне відношення
\parbreak{}  %% абзац продовжується на наступній сторінці
