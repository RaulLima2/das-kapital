\parcont{}  %% абзац починається на попередній сторінці
\index{i}{0383}  %% посилання на сторінку оригінального видання
Але робітникам доводилося страждати не тільки від експериментів
фабрикантів на фабриках та муніципалітетів поза фабриками,
не тільки від знижування заробітної плати та від безробіття, від
нужди та милостині, від хвальних промов лордів та членів Палати
громад. «Нещасні жінки, що через бавовняний голод лишалися
без праці, стали покидьками суспільства й лишилися ними\dots{}
Число молодих проституток збільшилося дужче ніж за останні
25 років».\footnote{
3 листа шефа поліції Гаррі з Болтону в «Reports of Insp. of Fact,
for 3t st October 1865», p. 61, 62.
}

Отже, за перші 45 років брітанської бавовняної промисловості,
від 1770 до 1815 р., ми знаходимо лише п’ять років кризи та застою,
але це був період її світової монополії. Другий 48-річний період,
від 1815 до 1863 р., налічує лише 20 років пожвавлення
й розцвіту на 28 років пригнічення й застою. Від 1815 до 1830 р.
починається конкуренція з континентальною Европою та із
Сполученими штатами. Від 1833 р. силоміць поширюється азійські
ринки коштом «руйнування людської раси». Від часу скасування
хлібних законів, від 1846 до 1863 р., на вісім років середнього
пожвавлення та розцвіту було дев’ять років пригнічення
та застою. Про становище дорослих бавовняних робітників-чоловіків,
навіть за часів розцвіту, можна судити за даними нижченаведеної
примітки.\footnote{
В одній, оголошеній весною 1863 р., відозві бавовняних робітників,
що закликала створити еміграційне товариство, сказано, між іншим:
«Те, що в наші часи велика еміграція фабричних робітників є абсолютно
доконечна, мало хто заперечуватиме. Але дальші факти доводять, що
повсякчас потрібен постійний потік еміґрації, та що без нього нам серед
звичайних обставин не сила відстояти наші позиції: 1814 р. офіціяльна
вартість (лише показник кількости) вивезених бавовняних товарів становила
17.665.378\pound{ фунтів стерлінґів}, їхня ж дійсна ринкова вартість —
20.070.824\pound{ фунти стерлінґів}, 1858 р. офіціяльна вартість вивезених бавовняних
товарів становила 182.221.681\pound{ фунт стерлінґів}, їхня ж дійсна
ринкова вартість лише 43.001.322\pound{ фунти стерлінґів}, так що за вдесятеро
збільшену кількість одержано лише трохи більше, ніж подвоєний еквівалент.
Цей результат, такий болючий для країни взагалі, а для фабричних
робітників зокрема, зумовили різні причини, що діяли разом. Одна
з найголовніших причин — це постійний надмір праці, доконечний для
цієї галузі промисловости, яка під погрозою знищення потребує постійного
поширення ринку. Наші бавовняні фабрики можуть спинитися через
періодичні застої в торговлі, які за сучасного ладу так само неминучі,
як і сама смерть. Але це не спиняє винахідницького духу людини. Хоч
за останні 25 років цю країну покинуло, скромно обчислюючи, 6 мільйонів
чоловік, однак, у наслідок постійного витискування робітників задля
здешевлення продукту, великому процентові самих дорослих чоловіків,
навіть у часи найвищого розцвіту, не сила знайти на фабриках якебудь
заняття на будь-яких умовах». («Reports of Insp. of Fact, for 30 th April
1863», p. 51, 52). В одному з дальших розділів ми побачимо, як пани
фабриканти підчас бавовняної катастрофи всякими способами, навіть за
допомогою державної влади, намагалися перешкодити еміґрації фабричних
робітників.
}
\parbreak{}  %% абзац продовжується на наступній сторінці
