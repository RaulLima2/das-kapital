вимирають від голоду; 1836 р. — великий розцвіт; 1837—1838 рр. —
пригнічений стан і криза; 1839 р. — знову пожвавлення; 1840 р. —
велика депресія, повстання, втручання війська; 1841—1842 рр. —
страшне страждання фабричних робітників; 1842 р. — фабриканти
викидають робітників із фабрик, щоб вимусити скасування збіжжевих
законів. Робітники багатьма тисячами напливають до
Йоркширу, звідти військо проганяє їх назад, їхніх проводирів
віддають під суд у Ланкастері. 1843 р. — великі злидні: 1844 р. —
знову пожвавлення; 1845 р. — великий розцвіт. 1846 р. — спершу
триває піднесення, потім симптоми реакції; скасування збіжжевих
законів; 1847 р. — криза; загальне пониження заробітної плати
на 10 і більше процентів на славу «big loaf» (коровай величезного
розміру, що його пани фритредери обіцяли робітникам підчас
агітації проти збіжжевих законів);* 1848 р. — пригнічення триває;
Менчестер під військовою охороною; 1849 р. — знову пожвавлення.
1850 р. — розцвіт. 1851 р. — спад товарових цін, низька
заробітна плата, часті страйки; 1852 р. — починається поліпшення,
страйки тривають далі, фабриканти загрожують довезти чужоземних
робітників. 1853 р. — зріст вивозу; восьмимісячний страйк
і великі злидні в Престоні. 1854 р. — розцвіт, переповнення ринків.
1855 р. — із Сполучених штатів, Канади, із східньоазійських
ринків надходять звістки про банкрутства; 1856 р. — великий
розцвіт; 1857 р. — криза; 1858 р. — поліпшення; 1859 р. — великий
розцвіт, зріст числа фабрик; 1860 р. — зеніт англійської
бавовняної промисловости; індійські, австралійські й інші ринки
так переповнено, що ще 1863 р. вони ледве поглинули всю заваль;
торговельний договір з Францією; величезний зріст числа фабрик
та машин; 1861 р. — піднесення триває якийсь час далі, реакція,
американська громадянська війна, недостача бавовни; 1862 до
1863 р. — повний крах.

Історія бавовняного голоду надто характеристична, щоб не
спинитись на ній хоч на одну хвилину. З наведених даних про
становище світового ринку за 1860—1861 рр. ми бачимо, що
бавовняний голод прийшовся фабрикантам до речі та почасти був
для них корисний — факт, визнаний у звітах менчестерської
торговельної палати, проголошений у парламенті Палмерстоном
і Дербі, потверджений подіями.236 Певна річ, 1861 р.
поміж 2.887 бавовняними фабриками Об’єднаного Королівства
багато було дрібних фабрик. За звітами фабричного інспектора
А. Редґрева, що його округа включає з тих 2.887 фабрик 2.109 фабрик,
392 фабрики з цих останніх, або 19\%, вживали кожна
менш від 10 парових кінських сил, 345 фабрик, або 16\%, вживали
від 10 до 20 сил, а 1.372 фабрики — 20 і більше кінських сил.237

236 Порівн. «Reports of Insp. of Fact, for 31 st October 1862», p. 30.

237 Там же, стор. 19.

* Цього пояснення вислову «big loaf» немає в німецькому тексті.
Ми беремо його з французького видання, де його подано в дужках у самому
тексті. Ред.
