шенням його власного розміру й числа його підданців. Більша
частина їхнього власного додаткового продукту, який чимраз
більше зростає і перетворюється в дедалі більших розмірах на
додатковий капітал, припливає до них назад у формі засобів
платежу, так що вони можуть поширювати межі свого споживання,
краще влаштовувати свій споживний фонд одягу, меблів
і т. д-і скласти невеликий грошовий резервний фонд. Але як
кращий одяг, ліпший харч, ліпше поводження і більший пекуліюм
не нищать відношення залежности й експлуатації раба,
так само це не нищить відношення залежности й експлуатації
найманого робітника. Підвищення ціни праці в наслідок акумуляції
капіталу свідчить справді лише про те, що розміри й вага
золотого ланцюга, що його сам найманий робітник уже викував
для себе, дозволяють ослабити напругу цього ланцюга. В суперечках
навколо цього предмету здебільшого не добачали головного,
а саме differentia specifica* капіталістичної продукції.
Робочу силу тут купують не для того, щоб через її послуги або
її продукт задовольняти особисті потреби її покупця. Мета покупця
— збільшити вартість свого капіталу, продукувати товари,
що містять у собі більше праці, аніж він оплачує, отже,
що містять у собі таку частину вартости, яка нічого не коштує
йому і яку він проте реалізує через продаж товарів. Продукція
додаткової вартости, або нажива — це абсолютний закон капіталістичного
способу продукції. Робоча сила може знаходити
собі покупців лише остільки, оскільки вона зберігає засоби продукції
як капітал, репродукує свою власну вартість як капітал
і в неоплаченій праці дає джерело додаткового капіталу.76 Отже,
умови продажу робочої сили, незалежно від того, чи вони більш
чи менш сприятливі для робітників, містять у собі доконечність
постійного повторювання її продажу і репродукцію багатства
як капіталу в щораз ширшому розмірі. Заробітна плата, як ми
бачили, з самої природи своєї постійно зумовлює постачання
робітником певної кількости неоплаченої праці. Залишаючи
цілком осторонь випадки зростання заробітної плати при зниженні
ціни праці тощо, збільшення її означає в найкращому
випадку лише кількісне зменшення неоплаченої праці, що її
мусить давати робітник. Це зменшення ніколи не може дійти до
такого пункту, де воно загрожувало б самій капіталістичній
системі. Залишаючи осторонь насильні конфлікти щодо рівня
заробітної плати, — а вже Адам Сміс показав, що взагалі і в ці-

76 Примітка до 2 видання. «Однак межа зайняття робітників так
промислових, як і сільських однакова: а саме можливість для підприємця
добувати зиск із продукту їхньої праці... Якщо норма заробітної
плати зростає так високо, що зиск хазяїна падає нижче пересічного
зиску, то хазяїн перестає вживати робітників або вживає їх лише за
тієї умови, щоб вони згодились на зниження заробітної плати». (John
Wade- «History of the Middle and Working Classes», 3rd. ed., London
1835, p. 241).

* — відмінної ознаки. Ред.
