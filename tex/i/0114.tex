утратити 10.24 В дійсності остаточний результат той, що всі посідачі
товарів продають один одному свої товари на 10\% дорожче
понад їхню вартість, а це цілком те саме, як коли б вони продавали
товари за їхніми вартостями. Такий загальний номінальний
додаток до ціни товарів викликає такий самий ефект, як коли б,
наприклад, товарові вартості цінувалося на срібло замість золота.
Грошові назви, тобто ціни товарів, зростали б, але їхні
вартостеві відношення лишалися б незмінними.

Припустімо, навпаки, що покупець має привілей купувати
товари за ціни, нижчі від їхньої вартости. Тут нема вже навіть
потреби нагадувати, що покупець знову стає продавцем. Він
був продавцем раніш, ніж став покупцем. Він уже втратив як
продавець 10\% раніш, ніж виграв 10\% як покупець.25  Все лишається
знову по-старому.

Отже, утворення додаткової вартости, а тому й перетворення
грошей на капітал, не можна пояснити ані тим, що продавці
продають товари понад їхню вартість, ані тим, що покупці купують
їх нижче від їхньої вартости.26

Проблема аж ніяк не стає простішою, коли крадькома занести
до неї чужі їй відношення, отже, коли, наприклад, разом із полковником
Торренсом скажемо: «Дійсний попит полягає у спроможності
та нахилі (!) споживачів, чи то шляхом безпосереднього
чи то посереднього обміну, давати за товари якусь більшу
кількість усіх складових частин капіталу, ніж коштує їхня продукція».27
В циркуляції продуценти і споживачі протистоять
один одному лише як продавці й покупці. Твердити, що додаткова
вартість постає для продуцента з того, що споживачі пла-

24 «При збільшенні номінальної вартости продукту... продавці не
збагачуються... бо рівно стільки, скільки вони виграють як продавці,
вони втрачають як покупці» («By the augmentation of the nominal value
of the produce... sellers not enriched... since what they gain as sellers, they
prеcisеly expend in the quality of buyers»). («The Essential Principles of
the Wealth of Nations etc.», London 1797, p. 66).

25 «Коли хтось мусить віддати за 18 ліврів таку кількість продукту,
яка варта 24 ліври, то, вживши виручені гроші на купівлю, він так само
одержить за 18 ліврів те, за що раніш плати и 24 ліври» («Si L’on est
forcé de donner pour 18 livres une quantité de telle production qui en valait
24, lorsqu’on employera ce même argent à acheter, on aura également pour
18 ce que l’on payait 24»). (Le Trosne «De l’Intérêt Social»,
Physiocrates, éd. Daire, Paris 1846, p. 897).

26 «Кожен продавець не може завжди підвищувати ціни на свої товари
інакше, як згодившись платити завжди дорожче за товари інших продавців;
і з тієї самої причини кожний споживач, звичайно, не може завжди
платити дешевше за те, що купує, інакше, як згодившись на таке саме зменшення
цін на ті речі, які продає» («Chaque vendeur ne peut donc parvenir
à renchérir habituellement ses marchandises, qu’en se soumettant aussi à
payer habituellement plus cher les marchandises des autres vendeurs; et
par la même raison, chaque consommateur ne peut payer habituellement
moins cher ce qu’il achète, qu’en se soumettant aussi à une diminution
semblable sur le prix des choses qu’il vend»). (Mercier de la Rivière:
«L’Ordre naturel et essentiel», Physiocrates, éd. Daire, II. Partie, p. 555).

27  R. Torrerts: «An Essay on the Production of Wealth», London
1821, p. 349.
