логічному застосуванні й порівняно мало залежний від локальних
умов щодо свого місця застосування. Великий геній Ватта
виявляється у специфікації патенту, який він узяв у квітні 1784 р.,
специфікації, що в ній його парову машину описано не як якийсь
винахід для окремих завдань, але як універсальний чинник
великої промисловости. Він натякає тут на такі способи її застосовування,
що з них деякі, як от, приміром, паровий молот,
заведено в життя лише більше ніж півстоліття пізніше. Однак 
він сумнівався в тому, чи можна буде застосувати парову машину
до мореплавства. Його наступники, Болтон та Ватт, виставили
1851 р. на лондонській промисловій виставці колосальнішу парову
машину для океанських пароплавів.

Лише після того, як знаряддя перетворились із знарядь людського
організму на знаряддя механічного апарату, виконавчої
машини, тільки тоді й рухова машина набула самостійної форми,
цілком емансипованої від меж людської сили. Разом з цим та
поодинока виконавча машина, яку ми досі розглядали, зводиться
на простий елемент машинової продукції. Тепер одна рухова
машина може одночасно рухати багато робочих машин. Зі збільшенням
числа одночасно пущених у рух робочих машин зростає
й рухова машина, а передатний механізм розвивається в широченний
і складний апарат.

Тут треба розрізняти дві форми: кооперацію багатьох однорідних
машин і систему машин.

В одному випадку цілий продукт виробляє та сама робоча
машина. Вона виконує всі ті різні операції, що їх виконував своїм
знаряддям ремісник, наприклад, ткач своїм ткацьким варстатом,
або ті, що їх послідовно виконували ремісники за допомогою
різних знарядь, однаково, чи були вони самостійні ремісники,
чи члени якоїсь мануфактури.100 Приміром, у сучасній мануфактурі
поштових конвертів один робітник за допомогою фальцу
фальцював папір, другий накладав клей, третій одгинав кляпку,
на якій друкується девізу, четвертий витискував девізу й т. ін.,
і при кожній з цих частинних операцій кожний окремий конверт
мусив переходити з рук до рук. Одним-одна машина виготовляти
конверти виконує всі ці операції відразу й виготовляє 3.000 й
більше поштових конвертів за одну годину. Одна американська
машина виготовляти паперові мішечки, виставлена па лондонській
промисловій виставці 1862 р., ріже папір, намазує клей,
фальцює й виготовляє 300 штук за хвилину. Цілий процес, що в
мануфактурі є поділений і виконується послідовно, тут виконує

100 З погляду мануфактурного поділу праці ткацтво було зовсім не
проста, а скорше складна реміснича праця, і тому механічний ткацький
варстат є машина, що виконує дуже різноманітні операції. Взагалі неправильно
думати, ніби сучасні машини первісно опанували такі операції,
які мануфактурний поділ праці вже спростив. Прядіння й ткання за мануфактурного
періоду відокремлено одне від одного на нові роди, їхнє знаряддя
поліпшено та урізноріднено, але самого процесу праці ані скільки
не поділено, він лишався ремісничий. За вихідну точку для машини є не
праця, а засіб праці.
