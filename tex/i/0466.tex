форм у виплаті заробітної плати нічого не зміняє в її суті, дарма
що одна форма може бути сприятливіша для розвитку капіталістичної
продукції, ніж друга.

Хай звичайний робочий день становить 12 годин, із них 6
оплачених, 6 неоплачених. Припустімо, що спродукована протягом
нього вартість дорівнює 6 шилінґам, отже, вартість, спродукована
протягом однієї робочої години — 6 пенсам. Хай,
далі, досвід показує, що робітник, який працює з середнім
ступенем інтенсивности та вправности, отже, який у дійсності
витрачає лише робочий час, суспільно-доконечний на продукцію
якогось товару, дає протягом 12 годин 24 штуки, хоч поодиноких
окремих продуктів, хоч вимірних частин одного цілого продукту.
Таким чином вартість цих 24 штук, відлічивши частину
сталого капіталу, що міститься в них, буде 6 шилінґів, а вартість
окремої штуки — 3 пенси. Робітник дістає 1% пенси за
штуку й таким чином заробляє за 12 годин 3 шилінґи. Як за
почасової плати байдуже, чи ми припустимо, що робітник працює
6 годин на себе, а 6 годин на капіталіста, або що з кожної години
він працює половину на себе, а другу — на капіталіста, так само
й тут байдуже, чи ми скажемо, що кожна окрема штука напівоплачена
й напівнеоплачена, або що ціна 12 штук покриває лише
вартість робочої сили, тимчасом як в 12 інших штуках утілюється
додаткова вартість.

Форма відштучної плати так само іраціональна, як і форма
почасової плати. Тимчасом як, наприклад, дві штуки товару,
за вирахуванням вартости зужиткованих на них засобів продукції,
варті, як продукт однієї робочої години, 6 пенсів, робітник дістає
за них ціну в 3 пенси. В дійсності відштучна плата безпосередньо
не виражає ніякого відношення вартости. Тут справа не в тому,
щоб виміряти вартість штуки товару втіленим у ній робочим
часом, а в тому, щоб, навпаки, витрачену робітником працю виміряти
числом спродукованих ним штук товару. За почасової
плати працю вимірюється безпосередньо часом її тривання, за
відштучної плати — кількістю тих продуктів, що в них праця
згусла протягом певного часу.48 Ціна самого робочого часу кі-

а positive gain to the employer»). («Reports of Insp. of Fact, for 31 st
October 1860», p. 9). «Наднормова праця все ще панує в широких розмірах,
і здебільшого сам закон Гарантує фабрикантам можливість уникнути
викриття та кари. У багатьох моїх попередніх звітах я вже зазначав...
яку кривду роблять усім робітникам, що дістають не відштучну, а потижневу
заробітну плату» (Overworking, to a very considerable extent,
still prevails; and, in most instances, with that security against detection
and punishment which the law itself affords. I have in many former reports
shown... the injury to all the workpeople who are not employed on piecework,
but receive weekly wages»). (Leonhard Horner y «Reports of Insp.
of Fact, for 30 th April 1859», p. 8, 9).

48 «Заробітну плату можна міряти двома способами: або триванням
праці, або її продуктом» («Le salaire peut se mesurer de deux manières:
ou sur la durée du travail, ou sur son produit»). («Abrège élémentaire des
principes de l’Economie Politique», Paris 1796, p. 32). Автор цієї анонімної
праці — G. Garnier.
