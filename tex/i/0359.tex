Засіб праці вбиває робітника. Певна річ, ця безпосередня
протилежність найнаочніше виявляється тоді, коли новозаведена
машина конкурує з традиційним ремісничим або мануфактурним
виробництвом. Але й у межах самої великої промисловости постійне
поліпшування машин і розвиток автоматичної системи діють аналогічно.
«Постійна мета поліпшення машин є в тому, щоб зменшити
ручну працю або вдосконалити ланку в продукційному
ланцюзі фабрики, замінивши люський апарат залізним».\footnote{
«Reports of Insp. of Fact, for 31 st October 1858», p. 43.
}
«Застосування сили пари й води до машин, що їх досі рухалось
рукою, трапляється щодня... Незначні поліпшення в машинах,
що мають на меті заощадити на рушійній сипі, поліпшити продукт,
збільшити продукцію протягом того самого часу, витиснути дитину,
жінку або чоловіка, — такі поліпшення робиться постійно
і, хоч на око вага цих поліпшень невелика, все ж вони дають
важливі результати».\footnote{
«Reports |of Insp. of Fact, for 31 st October 1856», p. 15.
} «Повсюди, де якась операція потребує
чималої вправности та певної руки, її якомога швидше забирають
із рук надто навченого робітника, що має часто нахил до нереґулярности
всякого роду, щоб доручити її осібному механізмові,
який так добре вреґульований, що за ним може наглядати й
мала дитина».\footnote{
Ure: «Philosophy of Manufacture», p. 19. «Велика перевага
машин, що їх уживають на цегельнях, є в тому, що вони роблять хазяїна
незалежним від навчених робітників». («Children’s Employment Commission.
5 th Report», London 1866, p. 180, n. 46).

Додаток до другого видання. Пан А. Стеррок, головний управитель
машинового відділу «Great Northern Railway», висловлюється так про
будування машин (льокомотивів і т. д.): «Дорогих (expensive) англійських
робітників із дня на день потребують щораз менше. Продукція збільшується
через уживання поліпшених інструментів, а ці інструменти із свого
боку обслуговує нижчий рід праці (a low class of labour)... Раніш усі частини
парової машини продукувала неодмінно кваліфікована праця.
Ті самі частини тепер продукує менш кваліфікована праця, але з добрими
інструментами... Під інструментами я розумію машини, що їх уживають
в машинобудуванні». («Royal Commission on Railways. Minutes of Evidence»,
n. 17 862 and 17 863. London 1867).
} «За автоматичної системи талант робітника проґресивно
витискується».\footnote{
Ure: «Philosophy of Manufacture», , p. 20.
} «Поліпшення машин не тільки вимагає
зменшити число дорослих робітників, уживаних, щоб досягти
певного результату, але воно ще й заміняє одну клясу індивідів
на другу клясу, більш навчених на менш навчених, дорослих
на дітей, чоловіків на жінок. Всі ці переміни призводять до постійних
коливань у нормі заробітної плати».\footnote{
Там же, стор. 321.
} «Машини безупинно
викидають дорослих із фабрики».206 Надзвичайну еластичність
машинової системи як наслідок нагромадженого практичного
досвіду, як наслідок наявного вже розміру механічних
засобів та постійного проґресу техніки, виявив нам бурхливий

дорослих робітників-чоловіків працею жінок та дітей або працю навчених
робітників працею чорноробів». (Ure: «Philosophy of Manufacture», p.23).

205 Там же, стор. 23.
