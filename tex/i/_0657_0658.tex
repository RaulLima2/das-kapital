\index{i}{0657}  %% посилання на сторінку оригінального видання
Приватна власність, як протилежність до суспільної, колективної
власности, існує лише там, де засоби праці й зовнішні
умови праці належать приватним особам. Але залежно від того,
чи є ці приватні особи робітники або неробітники, змінюється
й характер самої приватної власности. Безмежна різноманітність
відтінків, які вона являє на перший погляд, відбивають лише
проміжні стани, що лежать між обома цими крайностями.

Приватна власність робітника на його засоби продукції є
основа дрібного виробництва, а дрібне виробництво є доконечна
умова розвитку суспільної продукції й вільної індивідуальности
самого робітника. Правда, цей спосіб продукції існує також у
рамках рабства, кріпацтва й інших відносин залежности. Але
він процвітає, виявляє всю свою енергію, здобуває клясичну
адекватну форму тільки там, де робітник є вільний приватний
власник своїх, ним самим уживаних умов праці, селянин — ріллі,
яку він обробляє, ремісник — інструменту, що на ньому він грає,
як віртоуз.

Цей спосіб продукції має за передумову роздрібнення землі
й усіх інших засобів продукції. Він виключає так концентрацію
засобів продукції, як і кооперацію, поділ праці всередині того
самого продукційного процесу, суспільне панування над природою
й реґулювання її, а також вільний розвиток суспільних продуктивних
сил. Він можливий лише за вузьких примітивних
меж продукції й суспільства. Захотіти його увіковічнити, це
значило б — як справедливо каже Пекер — «декретувати
загальну помірність». Але на якомусь певному ступені розвитку
він сам породжує матеріяльні засоби свого власного знищення.
З цієї хвилини в надрах суспільства починають ворушитися
сили та пристрасті, що почувають себе скутими цим способом
продукції. Він мусить бути знищений, і його знищується. Знищення
його, перетворення індивідуальних і роздрібнених засобів
продукції на суспільно-сконцентровані, отже, перетворення
карликової власности багатьох на колосальну власність
небагатьох, отже, експропріяція в широких народніх мас землі,
засобів існування і знарядь праці — оця жахлива й тяжка
експропріяція народньої маси становить передісторію капіталу.
Вона охоплює цілу низку насильних метод, що з них ми коротко
розглянули лише ті, які становили епоху як методи первісної
акумуляції капіталу. Експропріяцію безпосередніх продуцентів
проводиться з найнещаднішим вандалізмом і під тиском
якнайпідліших, якнайбрудніших, найдріб’язковіших і найшаленіших
пристрастей. Приватну власність, здобуту працею власника,
основану, так би мовити, на зрощенні поодинокого незалежного
робітника з його умовами праці, витісняє капіталістична
приватна власність, основана на експлуатації чужої, але формально
вільної праці\footnote{
«Ми перебуваємо в цілком нових суспільних умовах\dots{} ми намагаємось
відокремити кожний рід власности від кожного роду праці»
(«Nous sommes dans une condition tout-à-fait nouvelle de la société\dots{}
nous tendons à séparer toute espèce de propriété d’avec toute espèce de
travail»). («\emph{Sismondi}: «Nouveaux Principes de l’Economie Politique»,
vol. II, p. 434).
}.

\index{i}{0658}  %% посилання на сторінку оригінального видання
Скоро тільки цей процес перетворення в достатній мірі розклав
старе суспільство углиб і вшир, скоро тільки робітників
перетворено на пролетарів, а їхні умови праці на капітал, скоро
тільки капіталістичний спосіб продукції став на власні ноги,
дальше усуспільнення праці і дальше перетворення землі та
інших засобів продукції на суспільно-експлуатовані, отже, на
спільні засоби продукції, і тим то й дальша експропріяція приватних
власників набуває нової форми. Тепер експропріяції
підлягає вже не робітник, що сам веде самостійне господарство,
а капіталіст, що експлуатує багатьох робітників.

Ця експропріяція здійснюється наслідком гри іманентних законів
самої капіталістичної продукції, через централізацію капіталів.
Один капіталіст побиває багатьох. Пліч-о-пліч із цією
централізацією або експропріяцією багатьох капіталістів небагатьма
розвивається кооперативна форма процесу праці в
щораз ширших, більших розмірах, розвивається свідоме технічне
застосування науки, пляномірна експлуатація землі, перетворення
засобів праці на такі засоби праці, що їх можна
вживати тільки колективно, економізування всіх засобів продукції
через вживання їх як засобів продукції комбінованої,
суспільної праці, вплетіння всіх народів у сіть світового ринку,
а разом з тим інтернаціональний характер капіталістичного режиму.
Разом з постійним меншанням числа маґнатів капіталу,
що узурпують і монополізують усі вигоди цього процесу перетворення,
зростає маса злиднів, пригноблення, рабства, виродження,
експлуатації, але разом з тим і обурення робітничої
кляси, щораз більшої й більшої числом, що її навчає, об’єднує
й організує механізм самого процесу капіталістичної продукції.
Монополія капіталу стає путами того способу продукції, що
зріс за неї і під нею. Централізація засобів продукції та усуспільнення
праці досягають такого пункту, коли вони стають
несполучні з їхньою капіталістичною оболонкою. Її розривається.
Б’є остання година капіталістичної приватної власности.
Експропріяторів експропріюють.

Капіталістичний спосіб присвоєння, що випливає з капіталістичного
способу продукції, а тому й капіталістична власність
є перше заперечення індивідуальної приватної власности, основаної
на власній праці. Але капіталістична продукція з доконечністю
природного процесу породжує заперечення самої себе.
Це є заперечення заперечення. Воно відбудовує не приватну
власність, а індивідуальну власність на основі завоювань капіталістичної
ери, на основі кооперації і спільного володіння землею
й засобами продукції, спродукованих самою ж працею.

Перетворення роздрібненої приватної власности, основаної
на власній праці індивідуумів, на капіталістичну власність, є,
\parbreak{}  %% абзац продовжується на наступній сторінці
