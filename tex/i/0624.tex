нувано до приватних маєтків простою узурпацією.201 Все це
робилося без найменшого додержання етикету законности. Присвоєне
таким шахрайським способом державне майно разом з
понаграбовуваним церковним майном, оскільки останнє не втрачено
підчас республіканської революції, становить основу сучасних
княжих доменів англійської олігархії.202 Буржуазні
капіталісти сприяли цій операції, між іншим, для того, щоб
перетворити землю на предмет вільної торговлі, поширити сферу
великої рільничої продукції, збільшити приплив із села вільних,
як птиці, пролетарів і т. д. До того ж нова земельна аристократія
була природною союзницею нової банкократії, цієї фінансової
шляхти, що тільки но вилупилася з яйця, і великих мануфактуристів,
що тоді спирались на охоронні мита. Англійська буржуазія
так само правильно чинила в своїх інтересах, як шведські
міщани, що, навпаки, спільно з своєю економічною твердинею
— селянством, підтримували королів, що силоміць відбирали
від олігархії коронні землі (починаючи від 1604 р. й
пізніше, за Карла X й Карла XI).

Громадська власність — цілком відмінна від щойно розглянутої
державної власности — була старогерманською інституцією,
що й далі існувала під покровом февдалізму. Ми бачили
вже, як насильна узурпація цієї громадської власности, що її
здебільшого супроводило перетворення орної землі на пасовиська,
почалася наприкінці XV століття й тривала далі в XVI столітті.
Але тоді цей процес відбувався лише як індивідуальний
насильний акт, проти якого законодавство даремно боролося протягом
150 років. Проґрес XVIII століття виявляється в тому,
що тепер сам закон стає знаряддям грабування народньої землі,
хоч поруч із цим великі фармери вшивають і своїх дрібних незалежних
приватних метод. 203 Парляментська форма цього грабування
є «Bills for Inclosures of Commons» (закони про обгороджування
громадських земель), інакше кажучи, декрети, за допомогою
яких землевласники сами собі дарують народню землю
у приватну власність, — декрети експропріяції народу. Сер Ф. М.
Ідн у своїй хитромудрій оборонній адвокатській промові

201 «Незаконне відчуження коронних земель, почасти через продаж,
почасти через дарування, становить скандальний розділ в англійській
історії... величезне ошукання нації (gigantic fraud on the nation)».
(F. W. Newman: «Lectures on Political Economy», London 1851, p. 129,
130). — [Подробиці про те, як сучасні англійські великі землевласники
придбали свої маєтки, див. в «Our old Nobility. By Noblesse Oblige», London
1879. — Ф. E.].

202    Див., наприклад, памфлет E. Burke про герцоґську родину
Бедфордів, нащадком якої є Джон Рессел, «the tomtit of liberalism».

203 «Фармери забороняють cottager’aм (бурлакам) тримати якубудь
живу істоту, крім самих себе, під тією причіпкою, що коли вони триматимуть
худобу або дробину, то крастимуть корм з клунь. Вони кажуть
також: тримайте cottager’iв у біді, якщо ви хочете, щоб вони були працьовитими.
Але справжній факт той, що фармери таким способом узурпують
усі права на громадські землі». («A Political Enquiry into the Consequences
of enclosing Waste Lands», London 1785, p. 75).
