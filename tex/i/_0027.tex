\parcont{}  %% абзац починається на попередній сторінці
\index{i}{0027}  %% посилання на сторінку оригінального видання
визначено як працю, рівнозначну кожній іншій людській праці,
хоч яку натуральну форму має ця остання, отже, однаково, чи
упредметнюється вона в сурдуті, чи в пшениці, в залізі, чи в
золоті й~\abbr{т. ін.} Тому полотно через свою форму вартости перебуває
тепер у суспільному відношенні вже не лише до якогось одного
тільки іншого роду товару, а до цілого товарового світу. Як
товар воно є громадянин цього світу. Разом з цим безмежний ряд
його виразів показує, що для вартости товарів байдуже, яка є
осібна форма споживної вартости, що в ній вона виявляється.

У першій формі: 20 метрів полотна = 1 сурдутові може бути
випадковим той факт, що ці два товари є вимінні в певному кількісному
відношенні. Навпаки, в другій формі одразу ж проступає
захована в ній основа, яка посутньо різниться від її випадкового
виявлення та визначає це виявлення. Вартість полотна лишається
однакова щодо величини, чи виражено її в сурдуті, каві, чи
залізі й~\abbr{т. ін.}, — в безлічі різних товарів, що належать найрізнішим
посідачам. Випадкове відношення двох індивідуальних посідачів
товарів одпадає. Стає ясно, що не обмін реґулює величину
вартости товару, а, навпаки, величина вартости товару реґулює
його мінові відношення.

\paragraph{Осібна еквівалентна форма}

Кожний товар — сурдут, чай, пшениця, залізо й~\abbr{т. ін.} — у
виразі вартости полотна служить за еквівалент, а тому й за тіло
вартости. Певна натуральна форма кожного з цих товарів є тепер
осібна еквівалентна форма поруч багатьох інших. Так само й
різноманітні певні, конкретні корисні роди праці, що містяться
в різних товарових тілах, репрезентують тепер стільки ж осібних
форм здійснення або виявлення людської праці взагалі.

\paragraph{Вади повної або розгорнутої форми вартости}

Поперше, відносний вираз вартости товару є невивершений,
бо ряд його виразів вартости ніколи не закінчується. Ланцюг, у
якім одне вартостеве рівнання прилучається до другого, можна
досхочу продовжувати кожним новим родом товару, який дає
матеріял для нового виразу вартости. Подруге, цей ланцюг становить
строкату мозаїку осібних (auseinanderfallender) і різнорідних
виразів вартости. Нарешті, коли, як це й мусить статися,
відносну вартість кожного товару виражається в цій розгорнутій
формі, то відносна форма вартости кожного товару є безмежний
ряд виразів вартости, відмінний від відносної форми вартости
всякого іншого товару. — Вади розгорнутої відносної форми вартости
відбиваються на відповідній її еквівалентній формі. А що
натуральна форма кожного окремого товарового роду становить
тут осібну еквівалентну форму поряд незчисленних інших осібних
еквівалентних форм, то існують взагалі лише обмежені еквівалентні
форми, з яких кожна виключає іншу. Так само й певний,
\parbreak{}  %% абзац продовжується на наступній сторінці
