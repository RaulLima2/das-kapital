промислові бюлетені про вбитих та покалічених.190а Заощадження
суспільних засобів продукції, що вперше визріває у фабричній
системі, немов у теплиці, перетворюється в руках капіталу разом
з тим на систематичне грабування життєвих умов робітника
підчас його праці, на грабування простору, повітря, світла
та засобів охорони робітника від небезпечних для життя або антигігієнічних
умов продукційного процесу; про влаштування якихось
вигід для робітника нічого й казати.191 Чи не правду казав
Фур’є, називавши фабрики «пом’якшеною каторгою».192

«Я сам, — каже фабричний інспектор Р. Бекер, — мусив нещодавно розпочати
судовий процес проти одного бавовняного фабриканта, бо він у цей тяжкий
та лютий час стягав з кількох «молодих» (понад тринадцять років)
робітників, що в нього працювали, по 10 пенсів за лікарську посвідку
про вік, яка коштує йому тільки 6 пенсів і за яку закон дозволяє стягати
лише 3 пенси, а звичай — нічого... Другий фабрикант, щоб досягти
тієї самої мети без конфлікту з законом, накладає на кожну бідну дитину,
що на нього працює, данину в 1 шилінґ за вивчення вмілости та
таємниці прядіння, скоро тільки лікарська посвідка визнає її за дозрілу
виконувати цю працю. Отже, десь на споді існують течії, що їх треба знати,
щоб зрозуміти такі надзвичайні явища, як страйки за таких часів, як
наші» (мовиться про страйк механічних ткачів на фабриці в Darwen у
червні 1863 р.) «Reports of Insp. of Fact, for 30 th April 1863 p.», p. 50,
51. (Фабричні звіти завжди йдуть далі, ніж їхні офіціяльні дати).

190а Закони для охорони від небезпечних машин дали добрі наслідки.
«Але... тепер існують нові джерела нещасливих випадків, які ще не існували
перед 20 роками, а саме збільшена швидкість машин. Колеса, вали,
веретена і ткацькі варстати женуть тепер із збільшеною силою, яка щораз
зростає: пучки мусять швидше та певніш хапати увірвану нитку, бо
досить вагання або необережности і будуть жертви... Багато нещасливих
випадків зумовлюються старанням робітників швидше скінчити свою
працю. Треба собі пригадати, що для фабрикантів дуже важливо тримати
свої машини в безнастанному русі, тобто безупинно продукувати пряжу й
тканини. Кожна зупинка на одну хвилину — це втрата не тільки на рушійній
силі, але й на продукції. Тому наглядачі за працею, заінтересовані
в кількості продукції, підганяють робітників тримати машини в русі,
а це не менш важливо і для робітників, яким платять від ваги або від
штуки. Тому, хоч у більшості фабрик формально й заборонено чистити
машини підчас їхнього руху, на практиці це загальне явище. Сама ця
обставина спричинила за останні шість місяців 906 нещасливих випадків...
Хоч чищення відбувається день-у-день, але все ж ґрунтовне чищення
машин призначають здебільша на суботу, і воно відбувається здебільше
підчас руху машин... За цю операцію не платять, і через те робітники
силкуються якомога швидше її скінчити. Тим то число нещасливих
випадків у п’ятницю й особливо в суботу далеко більше, ніж іншими
днями тижня. У п’ятницю число нещасливих випадків перевищує пересічне
число за перші чотири дні тижня приблизно на 12%, у суботу ж
число нещасливих випадків перевищує пересічне число за попередні
п’ять день на 25%; але коли взяти на увагу, що фабричний день у суботу
налічує лише 7 1/2 годин, а інші дні тижня 10 1/2, то це перевищення становитиме
більше ніж 65%». («Reports of Insp. of Fact, for 31 st October 1866».
London 1867, p. 9, 15, 16, 17).

191 У першому відділі третьої книги я розповім про похід англійських
фабрикантів, що стосується до недавнього часу, проти тих статтей фабричного
закону, які мають захищати члени «рук» від небезпечних для життя
машин. Тут. досить однієї цитати з офіційного звіту фабричного інспектора
Леонарда Горнера: «Я чув, як фабриканти з непробачливою легкістю
говорили про деякі нещасливі випадки, наприклад, втрата одного пальця —
