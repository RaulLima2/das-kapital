\index{i}{0168}  %% посилання на сторінку оригінального видання
З цих 4 фунтів пряжі, в яких таким чином існує вся новоспродукована вартість денного процесу
прядіння, одна половина репрезентує лише еквівалент вартости спожитої робочої сили, тобто змінний капітал у 3\shil{ шилінґи}, другі
2 фунти пряжі — лише додаткову вартість у 3\shil{ шилінґи.}

А що 12 робочих годин прядуна упредметнюються в б шилінґах вартости, то в 30\shil{ шилінґах} вартости пряжі
упредметнюються 60 робочих годин. Вони існують у 20 фунтах пряжі, з яких \sfrac{8}{10},
або 16 фунтів, є матеріялізація тих 48 робочих годин, що минули перед процесом прядіння, а саме,
тієї праці, що упредметнена в засобах продукції пряжі; навпаки, \sfrac{2}{10}, або 4 фунти пряжі, є
матеріялізація 12 робочих годин, витрачених у самому процесі прядіння.

Раніш ми бачили, що вартість пряжі дорівнює сумі нової вартости, створеної в процесі продукції
пряжі, плюс вартості,які вже перед тим існували в засобах її продукції. Тепер виявилось, яким чином функціонально або в
понятті різні складові
частини вартости продукту можуть бути виражені у відносних частинах самого продукту.

Цей розклад продукту — результату процесу продукції — на кількість продукту, що репрезентує лише
працю, вміщену в
засобах продукції, або сталу частину капіталу, далі на іншу кількість, що репрезентує лише додану в
процесі продукції доконечну працю, або змінну частину капіталу, і, нарешті, на ту
кількість, що репрезентує лише додаткову працю, додану в самому процесі, або додаткову вартість, —
цей розклад є так само простий, як і важливий, як це покаже пізніше застосовування його в заплутаних
і не розв’язаних іще проблемах.

Ми щойно розглядали ввесь продукт як готовий результат дванадцятигодинного робочого дня. Але ми
можемо простежити
його і в процесі його постання, і все ж таки частинні продукти виразити як функціонально відмінні
частини продукту.

Прядун продукує за 12 годин 20 фунтів пряжі, значить, за одну годину 1\sfrac{2}{3} фунта і за 8 годин 13\sfrac{1}{3}
фунтів, отже, частинний
продукт, вартість якого дорівнює цілій вартості бавовни, що її випрядається протягом цілого робочого
дня. Так само частинний продукт дальших 1 години 36 хвилин дорівнює 2\sfrac{2}{3} фунтів пряжі,
і тому репрезентує вартість засобів праці, зужиткованих протягом 12 робочих годин. Так само й за
дальші 1 годину 12 хвилин
прядун продукує 2 фунти пряжі = 3\shil{ шилінґам} — вартість продукту, що дорівнює тій цілій
новоспродукованій вартості, що її він створює за 6 годин доконечної праці. Нарешті, за останні \sfrac{6}{5}
годин він продукує так само 2 фунти пряжі, що їхня вартість дорівнює додатковій вартості, створеній
його полуденною додатковою працею. Цей спосіб обчислення служить англійському фабрикантові для домашнього вжитку, і він скаже,
приміром, що за перші 8 годин, або \sfrac{2}{3} робочого дня, він одержує назад вартість своєї бавовни
й т. ін. Ми бачимо, що ця формула правильна, що фактично це лише перша формула, перенесена з
простору, де готові частини
\parbreak{}  %% абзац продовжується на наступній сторінці
