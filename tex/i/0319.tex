з якою обертається веретено, або від числа вдарів, що їх молот
робить за одну хвилину. Деякі з тих колосальних молотів дають
70 ударів, Ryder’ова патентована ковальська машина, що вживає
парового молота невеликих розмірів на кування веретен, дає
700 ударів на одну хвилину.

Якщо дано ту пропорцію, в якій машина переносить вартість
на продукт, то величина цієї частини вартости залежить від величини
вартости самої машини.110 Що менше праці вона сама містить
у собі, то менше вартости додає вона до продукту. Що менше
вартости віддає вона, то продуктивніша вона й то більш наближається
її служба до служби сил природи. А продукція машин за
допомогою машин зменшує їхню вартість порівняно з їх розмірами
і їхньою дією.

Порівняльна аналіза цін на товари, продуковані ремісничим
або мануфактурним способом, та цін на ті самі товари як продукти
машин, дає взагалі такий результат, що в машиновому продукті
складова частина вартости, яку до нього додає засіб праці, відносно
зростає, але абсолютно меншає. Це значить, що її абсолютна
величина меншає, але її величина супроти загальної вартости
продукту, наприклад, одного фунта пряжі, більшає.111

110 Читач, полонений капіталістичними уявленнями, певна річ, дивується,
що тут немає мови про «процент», що його машина, пропорційно
до своєї капітальної вартости, додає до продукту. Однак, легко зрозуміти,
що машина, — тому що вона, як і будь-яка інша складова частина
сталого капіталу, не створює нової вартости, — не може додавати такої
вартости і під назвою «процент». Далі, ясно, що тут, де йдеться про продукування
додаткової вартости, не можна жодної частини її припустити
a priori під назвою «процент». Капіталістичний спосіб обчислення,
який prima facie* видається безглуздим та суперечним законам утворення
вартости, ми пояснимо в третій книзі цього твору.

111 Ця додавана машиною складова частина вартости падає абсолютно
й відносно там, де машина витискує коні, взагалі робочу худобу,
уживану виключно як рухову силу, а не як машини для обміну речовин.
До речі зауважимо, що Декарт, визначаючи тварини як прості машини,
дивиться очима мануфактурного періоду, відмінно від середньовіччя,
яке вважало тварину за помічника людини так само, як пізніше й пан
фон Галлер в його «Restauration der Staatswissenschaften». Що Декарт
так само, як і Бекон, розглядав зміну способу продукції та практичне
опанування природи людиною як результат зміненої методи думання, —
це показує його «Discours de la Méthode», де, між іншим, сказано: «Можна
(за допомогою методи, яку він увів у філософію) дійти знань, дуже корисних
у житті, і замість тієї спекулятивної філософії, якої навчають по школах,
знайти практичну філософію, за допомогою якої, знаючи силу та дію
огню, води, повітря, зірок і всіх інших навкольних тіл так само достеменно,
як ми знаємо різні ремества наших ремісників, ми могли б тим
самим способом вживати їх для всього того, на що вони придатні, і таким
чином зробитися хазяїнами й владарями природи» та тим самим «пособляти
поліпшенню людського життя» («Il est possible de parvenir à des connaissances
fort utiles à la vie, et qu'au lieu de cette philosophie spéculative
qu’on enseigne dans les écoles, on en peut trouver une pratique, par laquelle,
connaissant la force et les actions du feu, de l’eau, d’air, des astres, et de
tous les autres corps qui nous environnent, aussi distinctement que nous
connaissons les divers métiers de nos artisans, nous les pourrions employer
* — на перший погляд. Ред.
