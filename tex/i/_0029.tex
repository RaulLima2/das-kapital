\parcont{}  %% абзац починається на попередній сторінці 
\index{i}{0029}  %% посилання на сторінку оригінального видання 
чаю — як щось рівне залізу й т. ін., але ці щось рівні полотну
й залізу, ці вирази вартости сурдута й чаю є так само відмінні
один від одного, як і полотно та залізо. На практиці ця форма
з’являється, очевидно, лише за примітивних часів, коли продукти
праці перетворюються на товари лише у випадкових і ізольованих
актах обміну.

Друга форма, В, повніш, ніж перша відрізняє вартість товару від
його власної споживної вартости, бо вартість, приміром, сурдута
в усіх можливих формах протистоїть тепер своїй натуральній
формі, як щось рівне полотну, рівне залізу, рівне чаєві, рівне
будь-чому, а тільки не сурдутові. З другого боку, тут усякий
спільний вираз вартости товарів просто виключено, бо у виразі
вартости кожного окремого товару всі інші товари фігурують
тепер лише в формі його еквівалентів. Розгорнута форма вартости
вперше з’являється фактично тоді, коли продукт праці, приміром,
худобу, вже не винятково, а звичайно обмінюється на різні інші
товари.

Новоздобута форма С виражає вартості товарового світу в
одному й тому самому від нього відокремленому товаровому роді,
приміром, у полотні, і виражає таким чином вартості всіх товарів
через урівняння їх з полотном. Як щось рівне полотну вартість
кожного товару відрізняється тепер не лише від його власної
споживної вартости, але й від усіх споживних вартостей, і саме
через це вона виражена як щось спільне даному товарові з усіма
іншими товарами. Тому лише ця форма справді ставить товари у
взаємні відношення як вартості, або призводить до того, що вони
виступають один для одного як мінові вартості.

Обидві попередні форми виражають вартість кожного окремого
товару або в одним-одному відмінному від нього товарі, або в
цілому ряді багатьох відмінних од нього товарів. В обох випадках
надати собі форми вартости — це, так би мовити, приватна справа
окремого товару, і він робить це без допомоги інших товарів. Ці
останні відіграють проти нього лише пасивну ролю еквіваленту.
Навпаки, загальна форма вартости постає лише як спільна справа
товарового світу. Один якийсь товар набуває загального виразу
вартости лише тому, що одночасно з цим усі інші товари виражають
свою вартість у тому самому еквіваленті, і кожний новопосталий
рід товару мусить за ними те саме робити. Тим самим виявляється,
що предметність вартости (Wertgegenständlichkeit)
товарів, тому що вона є лише «суспільне буття» цих речей, може
бути виражена також лише через їхнє всебічне суспільне відношення,
що, отже, формою вартости товарів мусить бути суспільно
визнана форма.

У формі своєї рівности полотну всі товари виступають тепер
не лише як якісно рівні, як вартості взагалі, але й одночасно як
кількісно порівнянні величини вартости. Через те, що вони відбивають
величини своїх вартостей у одному й тому самому матеріялі,
у полотні, ці величини вартостей відбиваються і навзаєм
одна в одній. Приміром, 10 фунтів чаю = 20 метрам полотна і
\parbreak{}  %% абзац продовжується на наступній сторінці
