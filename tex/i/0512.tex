ніш в Англії, бо французька біднота тяжко працює та дуже
ощадна щодо харчу й одягу; вони споживають головно хліб,
овочі, городину, корінці та сушену рибу; вони дуже рідко їдять
м’ясо, і коли пшениця дорога, то й дуже мало їдять хліба».52
«До того ще, — каже далі наш есеїст, — треба додати, що вони
п’ють лише воду або подібні неміцні напитки, так що вони дійсно
надзвичайно мало витрачають грошей... Подібного стану речей,
безперечно, дуже тяжко добитися, але його можна досягти, як
це виразно доводить наявність його так у Франції, як і в Голландії».
53 Два десятиріччя пізніш один американський шахрай,
янкі, що дістав титул барона, Бенжамен Томсон (інакше граф
Румфорд), з великим успіхом розвивав перед богом і людьми ті
самі філантропічні ідеї. Його «Essays» — це куховарська книга
з рецептами всякого роду, як дорогі нормальні страви робітників
заміняти на суроґати. Ось особливо вдатний рецепт цього
дивовижного «філософа»: «П’ять фунтів ячменю, п’ять фунтів
кукурудзи, на 3 пенси оселедців, на 1 пенс соли, на 1 пенс оцту,
на 2 пенси перцю й городини — разом на суму в 20\sfrac{3}{4} пенсів
маємо юшку для 64 осіб; за пересічних цін на хліб ці витрати
можна навіть знизити до \sfrac{1}{4} пенса на людину (менше ніж 3 пфеніґи)».54
З проґресом капіталістичної продукції фальсифікація
товарів зробила зайвими ідеали Томсона.55

52 Фабрикант із Нортгемптоншіру чинить тут ріа fraus,* який йому
можна вибачити, бо це є порив серця. Він порівнює нібито життя англійських
і французьких мануфактурних робітників, алеж у вищецитованих
словах, як він і сам пізніше признається в замішанні, змальовує він
життя французьких рільничих робітників!

53 Там же, стор. 70, 71. Примітка до третього видання. Нині, завдяки
конкуренції на світовому ринку, що склалася з того часу, ми значно
посунулися наперед. «Якщо Китай, — заявляє своїм виборцям член парляменту
Степлтон, — стане великою промисловою країною, то я не бачу,
як робітнича людність Европи могла б витримати боротьбу, не знижуючись
до рівня своїх конкурентів» («Times», 3 вересня 1873 р.). Отже,
не континентальні, а китайські заробітні плати є вже тепер бажана
мета англійського капіталу.

54 Benjamin Thomson: «Essays, political, economical and philosopical
etc.», 3 volumes, London 1796—1802, vol. I, p. 288. У своєму «The
State of the Poor, or an History of the Labouring Classes in England etc.»
cep Ф. M. Еден дуже рекомендує злиденну румфордову юшку начальникам
робітних домів і докірливо нагадує англійським робітникам,
що, мовляв, «у Шотляндії є багато родин, які замість пшениці, жита й
м’яса цілі місяці живуть вівсяними крупами та ячним бороціном, перемішаним
лише з водою й сіллю, і все таки живуть дуже комфортабельно»
(«and that very comfortably too»). (Там же, книга II, розд. 2, стор. 503).
Подібні «вказівки» ми мали і в XIX віці. «Англійські рільничі робітники,
— читаємо, наприклад, — не хочуть їсти мішанини з гірших сортів
жита. В Шотляндії, де виховання краще, цей забобон, мабуть, невідомий».
(Charles Н. Parry, M. D.: «The Question of the Necessity of the
existing Cornlaws considered», London 1816, p. 69). Однак той самий
Пері нарікає, що англійський робітник тепер (1815 р.) дуже підупав
порівняно з часами Едена (1797 р.).

55 Зі звітів останньої парляментської слідчої комісії у справі фальсифікації
засобів існування бачимо, що навіть фальсифікація ліків в

* — благочестивий обман. Ред.
