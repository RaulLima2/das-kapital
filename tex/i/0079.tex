перших представників її в тій недоречній гіпотезі, що в процес
циркуляції товари входять без ціни, а гроші — без вартости, а
потім, у процесі циркуляції, певну частину товарової купи обмінюється
на певну частину гори металю.\footnote{
Само собою зрозуміло, що кожний окремий рід товару своєю ціною
становить елемент суми цін усіх товарів, які циркулюють. Алеж зовсім
незрозуміло, як неспільномірні одна з одною споживні вартості мають
en masse * обмінюватися на масу золота й срібла, що є в країні. Коли
звести товаровий світ в одним-один сукупний товар, що з нього кожний
товар становить лише певну частину, тоді вийде гарнесенький рахуночок:
сукупний товар = х центнерам золота, товар А = відповідній частині
сукупного товару = тій самій відповідній частині х центнерів золота. Це
відверто висловлює Монтеск’є: «Якщо порівнюють масу золота і срібла,
що є у світі, з масою товарів, яку має світ, то цілком ясно, що кожний
окремий предмет споживання або товар можна порівняти з певного частиною
іншого. Припустімо, що на світі існує лише один предмет споживання,
або товар, або що тільки один предмет споживання, або товар, купується
на ринку й поділяється на частини так само, як гроші; тоді ця частина
цього товару відповідала б певній частині маси грошей; половина цілої
кількости першої — половині цілої кількости другої... Установлення цін
на продукти завжди суттю своєю залежить од відношення між сукупністю
продуктів і сукупністю знаків». («Si l’on compare la masse de l’or et
de l’argent qui est dans le monde, avec la somme des marchandises qui y
sont, il est certain que chaque denrée ou marchandise, en particulier, pourra
être comparée à une certaine portion de l’autre. Supposons qu’il n’y en ait
qu’une seule denrée ou marchandise dans le monde, ou qu’il n’y ait qu’une
seule qui s’achète, et qu’elle se divise comme l’argent: cette partie de cette marchandise
répondra à une partie de la masse de l’argent; la moitié du total
de l’une à la moitié du total de l’autre... l’etablissement du prix des choses
dépend toujours fondamentalement de la raison du total des choses au
total des signes».). (Montesquieu: «Esprit des Lois», Oeuvres, London
1767 vol. 3, p. 12, 13). Про дальший розвиток цієї теорії у Рікарда
і його учнів Джемса Мілла, лорда Оверстона та інших порівн. «Zur
Kritik», стор. 140—146 і стор. 150 і далі («До критики», ДВУ 1926 р.,
стор. 179—190 і далі). Пан Дж. Ст. Мілл із звичною для нього еклектичною
логікою знаходить спосіб бути однакових поглядів зі своїм
батьком, Джемсом Міллом, і одночасно протилежних. Коли порівняємо
текст його компендія «Principles of Political Economy» з передмовою
(перше видання), в якій він сам проголошує себе сучасним Ад. Смісом,
то не знатимемо, з чого більше дивуватися, з наївности цієї людини, чи
з наївности публіки, яка на віру прийняла його за Ад. Сміса, на якого
він схожий приблизно так само, як генерал Вільямс Карс на герцоґа
Велінґтона. Оригінальні досліди пана Дж. Ст. Мілла в царині політичної
* — всією масою. \emph{Ред.}
}

с) Монета. Знак вартости

З функції грошей як засобу циркуляції випливає їхня монетна
форма. Вагова частина золота, уявлена в ціні або в грошовій
назві товарів, мусить протистати їм у циркуляції як однойменний
кусник золота або монета. Так само як і встановлення маштабу
цін, справа карбування монети припадає державі. У різних національних
мундирах, що їх носять золото й срібло як монети,
знову скидаючи їх на світовому ринку, виявляється поділ між
унутрішньою, або національною сферою товарової циркуляції
та її загальною сферою світового ринку.