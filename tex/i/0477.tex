ється довести, що різні національні заробітні плати просто пропорційні
ступеням продуктивности національних робочих днів,
щоб із цього інтернаціонального відношення зробити такий висновок,
що заробітна плата взагалі зростає й падає пропорційно
до продуктивности праці. Ціла наша аналіза продукції додаткової
вартости доводить безглуздість цього висновку, навіть тоді,
коли б Кері довів свій засновок, замість своїм звичаєм скидати
як попало до однієї купи некритично й поверхово назбираний
статистичний матеріял. Найкраще з усього є його твердження,
що справа в дійсності стоїть не так, як це повинно б бути на
основі теорії. Держава саме своїм втручанням перекрутила це природне
економічне відношення. Тому національні заробітні плати
треба обчисляти так, наче та частина їх, що припадає державі
у формі податків, припадала б самому робітникові. Чи не повинен
був би п. Кері далі подумати над тим, чи ці «державні витрати»
не є також «природні плоди» капіталістичного розвитку? Це міркування
цілком гідне людини, яка спершу проголосила капіталістичні
продукційні відносини за вічні закони природи й розуму,
що їхню вільну гармонійну гру порушує лише втручання держави,
а потім зробила відкриття, що диявольський вплив Англії
на світовому ринку, — вплив, що, як здається, не випливає з
природних законів капіталістичної продукції, — робить доконечним
втручання держави, тобто державний захист тих ваконів
природи й розуму, інакше кажучи, робить доконечним запровадження
протекційної системи. Далі він відкрив, що теореми
Рікарда й інших, у яких зформульовані існуючі суспільні протилежності
й суперечності, не є ідеальний продукт дійсного економічного
руху, а що, навпаки, дійсні протилежності капіталістичної
продукції в Англії та інших країнах є результат теорії
Рікарда й інших! Нарешті, він відкрив, що, кінець-кінцем,
торговля нищить природну красу й гармонію капіталістичної
продукції. Ще один крок далі, і він, можливо, зробить відкриття,
що єдине лихо капіталістичної продукдії — це сам капітал.
Тільки людина з такою жахливою некритичністю й такою фалшивою
(de faux aloi) вченістю могла заслужити собі, не вважаючи
на свою протекціоністичну єресь, того, щоб стати таємним джерелом
гармонійної премудрости Бастія й усіх інших сучасних
фритредерів-оптимістів. *

* У другому німецькому виданні тут є така примітка: «У четвертій
книзі я докладніше доведу поверховність його вчености». \emph{Ред.}
