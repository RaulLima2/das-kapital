\index{i}{0082}  %% посилання на сторінку оригінального видання
Тут ідеться тільки про державні паперові гроші з примусовим
курсом. Вони виростають безпосередньо з металевої циркуляції.
Навпаки, кредитові гроші мають за передумову такі відносини,
які, з погляду простої товарової циркуляції, нам зовсім ще невідомі.
Однак мимохідь зауважимо, що так само, як власне паперові
гроші випливають із функції грошей як засобу циркуляції,
так кредитові гроші мають свій природний корінь у функції
грошей як засобу платежу\footnote{
Мандарин фінансів Ван-Мао-Ін дозволив собі подати синові
неба проєкт, що в замаскованій формі мав на меті перетворити китайські
державні асиґнати на розмінні банкноти. У звіті асиґнаційного комітету
з квітня 1854~\abbr{р.} він дістав доброго прочухрана. Чи дістав він також відповідне
число бамбукових ударів, про це не повідомляється. «Комітет, —
сказано наприкінці звіту, — уважно зважив його проєкта й найшов, що
все в ньому виходить на користь купецтву, а для корони немає в ньому
нічого корисного». («Arbeiten der Kaiserlich Russischen Gesandtschaft zu
Peking über China. Aus dem Russischen von Dr.~K.~Abel und F.~A.~Mecklenburg»,
Berlin 1858, Bd.~I, S. 47 і далі). Про постійну втрату металю в
золотих монетах через їхній обіг один «Governor» (управитель) Англійського
банку заявив як свідок перед «Комісією Палати Лордів» (про
«банковий акт»): «Кожного року нова кляса соверенів (не політичних
суверенів, a «sovereign» — назва фунта стерлінґів) стає залегка. Та кляса
монет, що циркулювала протягом року як кляса з повного вагою, стираючись,
утрачає саме стільки, щоб найближчого року повернути шалі
проти себе». (House of Lords’ Committee, 1848, n. 429).
}.

Папірці, що на них надруковано їхні грошові назви, як от
1\pound{ фунт стерлінґів}, 5\pound{ фунтів стерлінґів} і~\abbr{т. ін.}, держава ззовні
вкидає в процес циркуляції. Оскільки вони дійсно циркулюють
замість однойменної суми золота, в їхньому русі відбиваються
лише закони самого грошового обігу. Специфічний закон циркуляції
паперових грошей може постати лише з їхнього відношення,
як представників золота, до золота. І цей закон є просто в тому,
що випуск паперових грошей треба обмежувати такою кількістю,
що в ній дійсно мусило б циркулювати символічно репрезентоване
ними золото (або срібло). Щоправда, кількість золота, яку може
поглинути сфера циркуляції, постійно коливається, стаючи вища
або нижча за певний пересічний рівень. Однак маса засобів циркуляції
в даній країні ніколи не падає нижче певного мінімуму,
що його встановлює досвід. Те, що ця мінімальна маса невпинно
змінює свої складові частини, тобто складається завжди з інших
золотих монет, це, певна річ, нічого не змінює в її обсягу та постійності
її обігу в сфері циркуляції. Тому її можна замінити паперовими
символами. Навпаки, коли ми сьогодні наповнимо паперовими
грішми всі канали циркуляції до повного ступеня їхньої
спроможности поглинати гроші, то завтра в наслідок коливання
товарової циркуляції вони можуть стати переповнені. Губиться
всяку міру. Але якщо папірці перейдуть свою міру, тобто ту
кількість однойменних золотих монет, що могла б циркулювати,
то вони все ж репрезентуватимуть у межах товарового світу,
залишаючи осторонь небезпеку загальної дискредитації, тільки
ту кількість золота, яку визначають іманентні закони товарового
\parbreak{}  %% абзац продовжується на наступній сторінці
