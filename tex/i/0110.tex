дач товарів, покупець або продавець, а саме, в обох рядах оборудок
виступаю проти одного контраґента лише як покупець, а
проти другого лише як продавець, проти одного — лише як
гроші, проти другого — лише як товар; ані проти одного, ані
проти другого я не виступаю як капітал або як капіталіст або
як представник чогось такого, що було б більше, ніж гроші або
товар, або могло б заподіяти інший вплив, крім того, що його
можуть справляти гроші або товар. Для мене купівля в А і продаж
В становлять послідовний ряд. Але зв’язок поміж цими
обома актами існує лише для мене. А немає жодного діла до моєї
оборудки з В, а В — до моєї оборудки з А. Коли б я захотів пояснити
їм особливу заслугу, яку я маю перед ними, обертаючи
послідовність ряду, то вони довели б мені, що я помиляюсь щодо
самого порядку послідовности, і що вся операція почалася не
від купівлі й кінчається не продажем, а, навпаки, почалася від
продажу й завершується купівлею. Справді, мій перший акт,
купівля, з погляду А є продаж, а мій другий акт, продаж, з
погляду В — купівля. Не задовольнившися цим, А й В заявляють,
що цілий цей порядок послідовности був зайвий фокус-покус.
А продасть товар безпосередньо В, а В купить його безпосередньо
в А. Разом з тим вся операція стискується в однобічний
акт звичайної товарової циркуляції, — просто продаж з погляду
А і просто купівлю з погляду В. Отже, обернувши порядок послідовности,
ми не вийшли поза сферу простої товарової циркуляції,
а тому ми мусимо розглянути, чи допускає вона з своєї природи
зростання вартостей, що входять у неї, тобто чи допускає вона
творення додаткової вартости.

Візьмімо процес циркуляції у формі, в якій він виявляється
як простий обмін товарів. Це завжди буває тоді, коли обидва
посідачі товарів купують один в одного товари і в термін платежу
вирівнюють балянс своїх взаємних грошових зобов’язань. Гроші
служать тут за рахункові гроші, щоб виразити вартості товарів
у їхніх цінах, але вони не виступають проти самих товарів речово.
Ясна річ, що, оскільки йдеться про споживну вартість,
виграти можуть обидва обмінювані. Обидва відчужують товари,
які є некорисні для них як споживні вартості, і одержують товари,
що їх вони потребують для споживання. І користь од цього може
бути не лише ця одна. А, що продає вино й купує збіжжя, продукує,
може, більше вина, ніж його зміг би випродукувати за той
самий робочий час рільник В, а рільник В за той самий робочий
час продукує більше збіжжя, ніж його зміг би випродукувати
винар А. Отже, А дістає за таку саму мінову вартість більше
вбіжжя, а В — більше вина, ніж дістав би відповідно кожний
із них без обміну, коли б вони мусили продукувати сами для себе
вино і збіжжя. Таким чином щодо споживної вартости можна
сказати, що «обмін є оборудка, в якій виграють обидві сторони».14

14 «Обмін є дивна оборудка, в якій виграють обидва контрагенти —
завжди (!)» («L’échange est une transaction admirable, dans la quelle les
