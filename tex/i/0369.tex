цю галузь продукції.218 Далі щодо сировинного матеріялу,219 то
не підлягає ніякому сумнівові, наприклад, що бурхливий розвиток
бавовняного прядільництва з оранжерійною швидкістю прискорив
культуру бавовнику в Сполучених штатах, а разом з нею
він не тільки прискорив африканську торговлю рабами, але
одночасно зробив вирощування негрів головним промислом так
званих прикордонних рабовласницьких штатів. Коли 1790 р.
зроблено в Сполучених штатах перший перепис рабів, число їх
становило 697.000, а 1861 року — приблизно чотири мільйони.
З другого боку, не менш певне й те, що розцвіт механічної вовняної
фабрики разом з чимраз більшим перетворенням орного поля на
пасовиська для овець викликав масове вигнання сільських робітників
та перетворення їх на «зайвих». В Ірландії ще й тепер
відбувається цей процес, який її людність, що від 1845 р. встигла
вже зменшитися майже наполовину, ще далі зменшує до тієї
міри, яка точно відповідає потребам її лендлордів та англійських
панів фабрикантів вовни.

Якщо машини захоплюють попередні або проміжні ступені,
які предмет праці має перебігти раніш, ніж він набирає своєї
остаточної форми, то разом із матеріялом праці більшає й попит
на працю в проваджуваних ремісничим або мануфактурним способом
галузях промисловости, які обробляють машиновий
фабрикат. Наприклад, машинове прядіння постачало пряжу так
дешево й так багато, що ручні ткачі спочатку могли, не збільшуючи
видатків, працювати повний час. Таким чином їхній дохід
збільшився.220 Звідси наплив людей до пряділень бавовни,
доки, нарешті, 800.000 бавовняних ткачів, що їх в Англії покликали
були до життя машини jenny, throstle та mule, були
вбиті паровими ткацькими варстатами. Так само разом з надміром
матерії на одяг, продукованої машиновим способом, зростало
число кравців, кравчих, швачок і т. д., поки не з’явилася
швацька машина.

тих на копальнях заліза, міді, олива, цини та всіх інших металів —
319.222.

218    1861 р. в Англії та Велзі працювало коло продукції машин 60.807
осіб, залічуючи сюди і фабрикантів з їхніми прикажчиками й т. д. і всіх
аґентів та купців цієї галузі, але виключаючи продуцентів невеличких
машин, як от швацьких машин і т. ін., так само й продуцентів знарядь
до робочих машин, як от веретен і т. ін. Число всіх цивільних інженерів
становило 3.329,

219 Що залізо — найважливіший сировинний матеріял, то треба тут
зауважити, що 1861 р. в Англії та Велзі був 125.771 робітник, які
працювали по залізоливарнях, з того 123.430 чоловіків, 2.341 жінка.
З-поміж перших 30.810 молодші і 92.620 понад 20 років.

220 «Родина з чотирьох дорослих осіб (бавовняних ткачів) з двома
дітьми, що працювали як winders, одержувала наприкінці останнього та
на початку цього століття 4 фунти стерлінґів на тиждень за десятигодинного
робочою дня: якщо робота мала нагальний характер, вони могли
заробити й більше... Раніш вони завжди страждали від недостатнього
подання пряжі». (Gaskell: «The Manufacturing Population of England»,
London 1833, p. 25—27).
