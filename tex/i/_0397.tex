\parcont{}  %% абзац починається на попередній сторінці
\index{i}{0397}  %% посилання на сторінку оригінального видання
вреґулювання їхньої швидкості, швидке псування легких машин
тощо — виключно такі перешкоди, що досвід навчає їх швидко
переборювати.\footnote{
Наприклад, у депі військового одягу в Pimlico, в Лондоні, на
фабриці сорочок Тіллей і Гендерсон у Лондондері, на фабриці одягу фірми
Тсейт у Лімеріку, що вживає близько 1.200 «рук».
} Коли, з одного боку, концентрація багатьох
робочих машин по великих мануфактурах спонукає вживати сили
пари, то, з другого боку, конкуренція пари з мускулами людини
прискорює концентрацію робітників і робочих машин по великих
фабриках. Так, Англія у велетенській ділянці продукції «wearing
apparel», як і в більшості інших виробництв, переживає тепер
революцію — перетворення мануфактури, ремества й домашньої
праці на фабричне виробництво, — після того як усі ці форми,
що під впливом великої промисловости цілком змінилися,
розпалися і покалічились, давно вже репродукували і навіть
перевищили все страхіття фабричної системи, не давши позитивних
моментів розвитку цієї системи.\footnote{
«Тенденція до фабричної системи» («Tendency to factory system»).
(Там же, стор. LXVII). «Вся галузь продукції перебуває тепер у переходовій
стадії й зазнає таких самих змін, яких зазнали виробництво мережива,
ткацтво тощо» (The whole employement is at this time in a state
of transition, and is undergoing the same change as that effected in the
lace trade, weaving etc.»). (Там же, № 405). «Цілковита революція» («А
complete Revolution»). (Там же, стор. XLVI, № 318). За часів «Children’s
Employment Commission» 1840 р. плетіння панчіх було ще ручною
працею. Починаючи від 1846 р., заведено різнорідні машини, що їх тепер
пускають у рух парою. Загальне число осіб обох статей і всякого
віку, починаючи від трьох років, занятих в англійському виробництві
панчіх, становило 1862 р. приблизно 129.000. З цього числа, за Parliamentary
Return з 11 лютого 1862 р. фабричному законові підлягало
лише 4.063 особи.
}

Цю промислову революцію, що розвивається стихійно, штучно
прискорюється поширенням фабричних законів на всі галузі
промисловости, де працюють жінки, підлітки й діти. Примусове
реґулювання робочого дня щодо його довжини, павз, його початку
й кінця, система змін для дітей, виключення всіх дітей нижче
певного віку й т. ін. примушують, з одного боку, збільшувати
число машин\footnote{
Так, приміром, про ганчарство фірма Cochrane «Britain Pottery,
Glasgow» повідомляє: «Щоб зберігати попередні розміри нашої продукції,
ми вживаємо тепер багато машин, що їх обслуговують некваліфіковані робітники,
і щодня ми переконуємося, що таким способом ми можемо спродукувати
більшу кількість продукту, ніж за старої методи» («То keep up
our quantity, we have gone extensively into machines wrought by unskilled
labour, and every day convinces us that we can produce a greater quantity
than by the old method»). («Reports of Insp. of Fact, for 31st October
1865», p. 13). «Вплив фабричного закону в тому, що він спонукає до даль
шого заведення машин». (Там же, стор. 13, 14).
} і замінювати мускули як рухову силу парою.\footnote{
Наприклад, після поширення фабричного закону на ганчарську
галузь продукції ми бачимо велике збільшення механічних ганчарних
варстатів (power jiggers) замість ручних (handmoved jiggers).
}
З другого боку, щоб виграти на просторі те, що пропадає на часі,
відбувається поширення засобів продукції, використовуваних
\parbreak{}  %% абзац продовжується на наступній сторінці
