\parcont{}  %% абзац починається на попередній сторінці
\index{i}{0042}  %% посилання на сторінку оригінального видання
вартости продукту праці?\footnote{
Одна з основних хиб класичної політичної економії та, що їй ніколи
не вдавалося з аналізи товару і, зокрема, товарової вартости вивести
форму вартости, яка саме й робить вартість міновою вартістю. Саме в особі
своїх найкращих представників, А. Сміса й Рікарда, вона розглядає форму
вартости як щось цілком байдуже або зовнішнє супроти самої природи
товару. Причина цього не лише в тому, що аналіза величини вартости
забирає всю її увагу. Ця причина лежить глибше. Форма вартости продукту
праці є найабстрактніша, але й найзагальніша форма буржуазного
способу продукції, який саме нею характеризується як осібний рід суспільної
продукції, отже, разом з тим — і історично. Коли ж розглядають
буржуазний спосіб продукції як вічну природну форму суспільної продукції,
то неминуче не помічають специфічности форми вартости, отже
специфічности товарової форми, а за дальшого розвитку і грошової форми,
форми капіталу й~\abbr{т. ін.} Тим то ми знаходимо в економістів, які цілком
погоджуються щодо вимірювання величини вартости робочим часом, найстрокатіші
й найсуперечніші уявлення про гроші, тобто про закінчену
форму загального еквіваленту. Особливо яскраво це виступає, наприклад,
при розгляді банкової справи, де поверхово пласких визначень грошей
вже недосить. Тому всупереч цьому постала відбудована меркантильна
система (Ganilh і інші), яка у вартості бачить лише суспільну форму
або, радше, її відблиск, позбавлений усякої субстанції. — Зауважу раз
назавжди, що під клясичною політичною економією я розумію всі економічні
теорії, які, починаючи від W. Petty, досліджують унутрішні зв’язки
буржузних продукційних відносин, усупереч вульґарній економії, яка
обертається лише в межах позірного зв’язку і, щоб дати правдоподібне
пояснення, так би мовити, найгрубіших явищ, пристосувати їх до хатнього
вжитку буржуа, вона знову й знову пережовує матеріял, давно вже поданий
науковою економією, а в усьому іншому обмежується на тім, що
педантично систематизує банальні й самозадоволені уявлення буржуазних
діячів продукції про їхній власний світ — найкращий світ — і оголошує
їх за вічну істину.
} Формули, на чолі яких понаписувано,
що вони належать до такої суспільної формації, де процес продукції
панує над людьми, а не людина над процесом продукції, —
буржуазна свідомість цієї формації вважає за таку саму самозрозумілу
природну доконечність, як і саму продуктивну працю.
Тому вона з передбуржуазними формами суспільно-продукційного
організму поводиться приблизно так само, як отці церкви з
передхристиянськими релігіями.\footnote{
«Економісти мають своєрідну методу. Для них існує лише два
роди інституцій — штучні й природні. Інституції февдальної епохи є
штучні, інституції буржуазні — природні інституції. В цьому вони
подібні до теологів, які теж встановлюють два роди релігій. Кожна релігія,
крім їхньої власної, є людський винахід, тимчасом як їхня власна
релігія є божественне відкриття. Отже, була колись історія, але тепер
її вже немає». («Les économistes ont une singulière manière de procéder.
Il n’y a pour eux que deux sortes d’institutions, celles de l’art et celles de
la nature. Les institutions de la féodalité sont des institutions artificielles,
celles de la bourgeoisie sont des institutions naturelles. Ils ressemblent
en ceci aux théologiens, qui eux aussi établissent deux sortes de religions
Toute religion qui n’est pas la leur est une invention des hommes, tandis
que leur propre religion est une émanation de dieu. — Ainsi il y a eu de
l’histoire, mais il n’y en a plus»), (\emph{K. Marx}: «Misère de la Philosophie.
Réponse à la Philosophie de la Misère par M. Proudhon», 1847, p. 113. —
\emph{K. Маркс}: «Злиденність філософії. Відповідь на філософію злиднів
Прудона». Партвидав України, 1932~\abbr{р.}, стор. 108). Справді, смішненький
пан Бастія, що собі уявляє, начебто стародавні греки й римляни жили
лише з грабіжництва. Аджеж коли люди багато століть живуть із грабіжництва,
то мусить щось бути для грабіжництва, або об’єкт грабіжництва
мусить постійно репродукуватися. Отже, здається, що і у греків і
римлян був теж якийсь процес продукції, отже, якась економія, яка
становила матеріяльну основу їхнього світу цілком так само, як і буржуазна
економія — основу нинішнього світу. Чи, може, Бастія думає,
що спосіб продукції, який базується на рабській праці, базується на системі
грабіжництва? Якщо так, то він стає на небезпечний ґрунт. Коли
такий велетень думки, як Арістотель, помилявся у своїм оцінуванні рабської
праці, то чому не помилятись у своїм оцінуванні найманої праці
такому карликові-економістові, як Бастіа? — Я використовую цю нагоду,
щоб коротко відповісти на закид, зроблений мені в одній німецькоамериканській
газеті з приводу моєї праці «Zur Kritik der Politischen
Oekonomie», 1859. На думку газети, мій погляд, що певний спосіб продукції
і щораз відповідні йому продукційні відносини, коротко кажучи,
«економічна структура суспільства є реальна база, на якій здіймаються
юридична й політична надбудова і якій відповідають певні форми
суспільної свідомости», що «спосіб продукції матеріяльного життя зумовлює
взагалі соціяльний, політичний і інтелектуальний процес життя» —
все це, мовляв, правильно для нинішнього світу, де панують матеріяльні
інтереси, алеж не для середньовіччя, де панувало католицтво, і не для
Атен і Риму, де панувала політика. Передусім дивно, що хтось може
припустити, що ці загальновідомі фрази про середньовіччя й античний
світ лишилися комубудь невідомі. Адже занадто ясно, що середньовіччя
не могло жити з католицтва, а античний світ з політики. Навпаки, той
спосіб, яким вони здобували засоби життя, пояснює, чому головну ролю
відігравала там політика, тут — католицтво. Зрештою, не треба багато
знайомства, приміром, з історією Римської республіки, щоб знати, що
таємницю цієї історії становить історія земельної власности. З другого
боку, вже Дон-Кіхот відпокутував ту свою помилку, що він уявив собі,
наче мандрівне лицарство однаково можна погодити з усіма економічними
формами суспільства.
}

\index{i}{0043}  %% посилання на сторінку оригінального видання
Як дуже спантеличує деяких економістів фетишизм, що пристає
до товарового світу, або предметна зовнішність суспільних
визначень праці, показує, між іншим, нудна й беззмістовна суперечка
про ролю природи в утворенні мінової вартости. А що
мінова вартість є певний суспільний спосіб виражати працю,
зужиту на якусь річ, то й не може вона мати більше природної
речовини, ніж, наприклад, векселевий курс.

Що товарова форма є найзагальніша й найнерозвиненіша
форма буржуазної продукції, — в наслідок чого вона й виникає
рано, хоч і не в такій домінантній, отже, і характеристичній
формі, як нині, — то фетишистичний її характер, здається, можна
побачити ще порівняно легко. У конкретніших формах зникає
навіть ця ілюзія простоти. Звідки постають ілюзії монетарної
системи? З того, що вона вбачала в золоті й сріблі як грошах не
вираз суспільного продукційного відношення, але форму природних
речей з дивними суспільними властивостями. А сучасна політична
економія, що так гордовито глузує з монетарної системи, —
хіба ж її фетишизму не видно як на долоні, скоро вона починає
розглядати капітал? Відколи ж то зникла ілюзія фізіократів,
що земельна рента виростає з землі, а не з суспільства?

Однак, щоб не забігати наперед, тут досить було б іще одного
прикладу щодо самої товарової форми. Коли б товари вміли говорити,
то вони б сказали: наша споживна вартість, можливо,
\parbreak{}  %% абзац продовжується на наступній сторінці
