Доки Елія Вайтней не вигадав 1793 р. cottongin’y, відділення одного
фунта бавовни від насіння коштувало пересічно один робочий
день. У наслідок його винаходу одна негритянка могла добувати
на день 100 фунтів бавовни, а від того часу продуктивність cottongin’y
ще значно збільшилася. Один фунт бавовняних волокон,
продукованих раніше за 50 центів, продавався пізніше по
10 центів з більшим зиском, тобто із включенням більшої кількости
неоплаченої праці. В Індії для відділення волокон від насіння
вживають інструмента, щось наче напівмашини, що зветься
churka; ним один чоловік та одна жінка чистять 28 фунтів на день.
Інструментом churka, що його винайшов перед кількома роками д-р Форбс, один чоловік та один підліток
чистять 250 фунтів на
день; там, де волів, пари або води вживають як рушійної сили,
потрібно лише декількох підлітків та дівчаток у ролі feeders
(подавальників матеріялу до машини). Шістнадцять таких машин,
що їх женуть волами, виконують щодня ту саму працю, яку раніш
виконували пересічно 750 чоловіка.115

Як уже згадано, парова машина при паровому плузі виконує
протягом однієї години, за 3 пенси, або за 1/4 шилінґа, стільки ж
праці, скільки 66 осіб по 15 шилінґів виконують протягом однієї
години. Я повертаюсь до цього прикладу, щоб запобігти помилковому
уявленню. А саме, ці 15 шилінґів зовсім не є вираз праці, яку
додають протягом однієї години 66 робітників. Якщо відношення
додаткової праці до доконечної праці становило 100\%, то ці
66 робітників продукували за годину вартість у 30 шилінґів,
хоч в еквіваленті, який вони сами діставали, тобто в заробітній
платі в 15 шилінґів, втілено лише 33 години. Отже, коли ми припустимо,
що машина коштує стільки ж само, скільки й річна
плата витиснених нею 150 робітників, приміром, 3.000 фунтів
стерлінґів, то ці 3.000 фунтів стерлінґів зовсім не є грошовий
вираз праці, постаченої й доданої до предмету праці цими 150 робітниками,
а є вони грошовий вираз лише тієї частини їхньої
річної праці, яка для них самих виражається в заробітній платі.
Навпаки, грошова вартість машини в 3.000 фунтів стерлінґів
виражає всю ту працю, яку витрачено на її продукцію, хоч би
в якій пропорції ця праця становила заробітну плату для
робітника й додаткову вартість для капіталіста. Отже, якщо
машина коштує стільки ж само, скільки коштує замінювана нею
робоча сила, то упредметнена в ній самій праця завжди куди
менша, ніж жива праця, яку вона заміняє.116

Якщо розглядати машини виключно як засіб удешевлення

115    Порівн. «Paper read by Dr. Watson, Reporter on Products to the
Government of India, before the Society of Arts», 17 April 1860.

116 «Ці німі діячі (машини) завжди є продукт значно меншої праці,
ніж та, яку вони заміняють, навіть і тоді, коли їхня грошова вартість
однакова» («These mute agents are always the produce of much less labour
than that which they displace, even when they are of the same money
value»). (Ricardo: «Principles of Political Economy», 3 rd ed. London
1821, p. 40).
