\parcont{}  %% абзац починається на попередній сторінці
\index{i}{0149}  %% посилання на сторінку оригінального видання
долучення нової вартости до предмету праці й збереження старих
вартостей у продукті є два цілком різні результати, які робітник
здобуває в той самий час, хоч він у той самий час працює
тільки один раз, то цю двоїстість результату, очевидно, можна
пояснити лише двоїстим характером самої його праці. За той
самий час однією своєю властивістю праця мусить творити вартість,
а другою властивістю зберігати або переносити вартість.

Яким чином кожний робітник додає робочий час, а тому й
вартість? Завжди лише у формі своєї своєрідної продуктивної
праці. Прядун додає робочий час лише тим, що пряде, ткач тим,
що тче, коваль тим, що кує. Але в наслідок доцільно-визначеної
форми, в якій вони додають працю взагалі, а тому й нову вартість,
у наслідок прядіння, ткання, кування, засоби продукції — бавовна
й веретена, пряжа й ткацький варстат, залізо й ковадло — стають
елементами утворення продукту, нової споживної вартости\footnote{
«Праця дає новий твір замість зниклого». («Labour gives a new
creation for one extinguished»). («An Essay on the Political Economy of
Nations», London 1821, p. 13).
}.
Стара форма їхньої споживної вартости зникає, але на те тільки,
щоб виринути в новій формі споживної вартости. Але при розгляді
процесу творення вартости виявилось, що, оскільки споживну
вартість уживається доцільно для продукції нової споживної
вартости, робочий час, доконечний для утворення зужиткованої
споживної вартости, становить частину робочого часу, доконечного
для утворення нової споживної вартости, отже, є робочий
час, перенесений із зужиткованих засобів продукції на новий
продукт. Отже, робітник зберігає вартості зужиткованих засобів
продукції або переносить їх як складові частини вартости на продукт
не через долучення своєї праці взагалі, а в наслідок особливого
корисного характеру, в наслідок специфічної продуктивної
форми цієї долучуваної праці. Як така доцільна продуктивна
діяльність, прядіння, ткання, кування, праця вже тільки своїм
дотиком робить мертві засоби продукції живими, надає їм духа
факторів процесу праці і сполучається з ними в продукти.

Коли б специфічна продуктивна праця робітника не була прядінням,
то він не перетворив би бавовну на пряжу, отже, і не
переніс би вартостей бавовни й веретен на пряжу. Навпаки, коли
той самий робітник змінить ремество і стане столяром, то він,
як і раніш, одним робочим днем додаватиме вартість до свого
матеріялу. Отже, він додає вартість своєю працею, розглядуваною
не як праця прядіння або праця столяра, але як абстрактна,
суспільна праця взагалі, і він додає певну величину вартости не
тому, що його праця має якийсь осібний корисний зміст, а тому,
що вона триває якийсь певний час. Отже, у своїй абстрактній
загальній властивості, як витрата людської робочої сили, праця
прядуна додає до вартости бавовни й веретен нову вартість, а
у своїй конкретній, осібній, корисній властивості, як процес
прядіння, вона переносить вартість цих засобів продукції на
\parbreak{}  %% абзац продовжується на наступній сторінці
