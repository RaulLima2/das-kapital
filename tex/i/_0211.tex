\parcont{}  %% абзац починається на попередній сторінці
\index{i}{0211}  %% посилання на сторінку оригінального видання
коли він ще тільки поставав, отже, коли він забезпечував своє
право вбирати в себе достатню кількість додаткової праці не
лише за допомогою сили економічних відносин, але й за допомогою
державної влади, — ці домагання виступають перед нами
цілком скромними, коли їх порівняти з тими поступками, які
він, гарчачи й опираючись, мусить робити у своєму зрілому віці.
Треба було віків, щоб «вільний» робітник у наслідок розвиненого
капіталістичного способу продукції добровільно згодився, тобто
суспільними відносинами був примушений, за ціну своїх звичних
засобів існування продавати ввесь активний час свого
життя, навіть більше — саму свою працездатність, своє первородство
за миску сочевиці. Тому природно, що те здовження
робочого дня, яке капітал силкується за допомогою державної
влади накинути повнолітнім робітникам від половини XIV до
кінця XVII віків, збігається приблизно з тією межею робочою
часу, яку в другій половині XIX віку подекуди ставить держава
перетворенню дитячої крови на капітал. Те, приміром, що нині
в штаті Масачуетс, у цьому донедавна найвільнішому штаті
Північноамериканської республіки, оголошено законною межею
праці для дітей молодших за 12 років, — це в Англії ще в половині
XVII віку було нормальним робочим днем повнокровних ремісників,
кремезних рільничих наймитів і велетнів-ковалів.\footnote{
«Жодна дитина молодша за 12 років не повинна працювати ні в
одній мануфактурі поверх 10 годин на добу» («No child under the age
of 12 years shall be employed in any manufacturing establishment more
than 10 hours in one day»). («General Statutes of Massachusetts». 63, ch. 12).
(Ці постанови видано в 1836--1858 рр.). «Працю протягом 10 годин на
добу в усіх фабриках бавовни, вовни, шовку, паперу, скла, так само як
і на фабриках залізних і мідяних виробів, слід розглядати як законний
робочий день. Приписується також, щоб від цього часу жодного малолітнього,
який працює на будь-якій фабриці, не затримувати за працею й
не примушувати працювати поверх 10 годин денно або 60 годин на тиждень,
і щоб від цього часу жодного малолітнього, що не має 10 років, не приймати
робітником ні на які фабрики в межах цього штату». («Labeur performed
during a period of 10 hours on any day in all cotton, woollen, silk,
paper, glass, and flax factories, or in manufactories of iron and brass,
shall be considered a legal day’s labour. And be it enacted, that hereaften
no minor engaged in any factory shall be holden or required to work more
than 10 hours in any day, or 60 hours in any week; and that herafter no
minor schall be admitted as a worker under the age of 10 years in any factory
within this state»). («State of New Jersey. An act to limit the hours of labour
etc.», 61 and 2). (Закон з 11 березня 1855 p.). «Забороняється в якому-будь
промисловому підприємстві вживати праці малолітніх, що мають
12 років, але не дійшли ще 15 років, більш як 11 годин на добу і ніяк не
раніш 5 годин зранку і не пізніш 7\sfrac{1}{2} години увечері» (No minor who has
attained the age of 12 years, and is under the age of 15 years, shall be employed
in any manufacturing establishment more than 11 hours in any one day,
nor before 5 o’clock in the morning, nor after 7\sfrac{1}{2} in the evening»). («Revised
Statutes of the State of Rhode Island etc.», ch. 39, § 23, 1 st July 1857).
}

Для видання першого «Statute of Labourers» (23, Eduard
III, 1349) безпосереднім приводом (не його причиною, бо такі
законодавства існують століття, коли привід вже не існує) була
величезна чума, яка так зменшила людність, що, як каже один
\parbreak{}  %% абзац продовжується на наступній сторінці
