життя в багатьох частинах Лондону й Ньюкестлю в пекло».\footnote{
Там же, стор. 62.
}

Але й тій частині робітничої кляси, що живе в кращих умовах,
а також дрібним крамарям та іншим елементам дрібної
середньої кляси, в Лондоні чимраз більше дається взнаки прокляття
цих мізерних житлових умов у міру того, як проґресують
«поліпшення», а з ними й зламання старих будинків і кварталів,
у міру того, як зростає число фабрик і наплив людей до головного
міста, нарешті, в міру того, як разом із міською земельною
рентою зростає плата за квартиру. «Плата за квартиру стала
така непомірна, що небагато робітників може оплатити більш
ніж одну кімнату».\footnote{
«Report of the Officer of Health of Si. Martin’s in the Fields», 1865.
} У Лондоні немає майже жодної домовласности,
що не була б обтяжена безліччю «middlemen’ів».\footnote*{
— посередників. Ред.
} Ціна
землі в Лондоні завжди дуже висока порівняно з річними доходами
з неї, бо кожний покупець спекулює на те, щоб раніш або
пізніше знову продати її за Jury Price (за ціну, визначену присяжними
при експропріяціях), або щоб вишахрувати надзвичайне
підвищення її вартости через близькість її до якогось великого
підприємства. Наслідок цього є реґулярна торговля контрактами
найму, що їх скуповують, коли їхній реченець наближається
до кінця. «Від джентльменів у цій справі можна сподіватися, що
вони робитимуть так, як роблять, а саме видушуватимуть з
квартирантів якомога більше, а самий дім передаватимуть своїм
наступникам у якомога злиденнішому стані».\footnote{
«Public Health. Eighth Report», London 1866. p. 91.
} Плата за квартиру
— тижнева, і ці панки нічим не ризикують. У наслідок
того, що залізниці будується в межах міста, «недавно одного
суботнього вечора у східній частині Лондону можна було бачити,
як сила родин, вигнаних із своїх старих помешкань, тинялися
з своїми злиденними пожитками на плечах, ніде не находячи
собі притулку, крім лише в робітному домі.\footnote{
Там же, стор. 88.
} Робітні
доми вже переповнені, а ухвалені парляментом «поліпшенням
почали ще тільки проводити в життя. Коли робітників проганяють
в наслідок руйнування їхніх старих домів, то вони не покидають
своєї парафії або принаймні оселяються на її межі, в найближчій
парафії. «Ясна річ, вони силкуються оселитися якомога
ближче до місця своєї праці. Наслідок той, що замість двох кімнат
родина мусить оселитися в одній кімнаті. Навіть при підвищеній
платі нове помешкання є гірше, ніж те погане, з якого їх вигнано.
Вже половині робітників на Strand’і доводиться ходити дві милі
до місця праці». Цей Strand, що його головна вулиця справляє
на чужинця імпозантне вражіння багатством Лондону, може
бути за приклад того, як напаковують людей у Лондоні. В одній
парафії цього Strand’у санітарний урядовець налічив 581 особу

них, виховуються тепер на небезпечні у майбутній своїй практиці кляси,
проводячи час до півночі з особами всякого віку, п'яними, розпусними та
лайливими». (Там же, стор. 56).