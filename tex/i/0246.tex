ремісникові-майстрові перетворюватись на капіталіста, обмежуючи
дуже незначним максимумом число робітників, яких дозволяли
вживати поодинокому майстрові. Посідач грошей або товарів
тільки тоді дійсно перетворюється на капіталіста, коли авансована
на продукцію мінімальна сума далеко перевищує середньовічний
максимум. Тут, як і в природознавстві, потверджується
правдивість закону, відкритого Геґелем в його «Логіці»,
що прості кількісні зміни на певному пункті перетворюються
на якісні ріжниці.205а

Та мінімальна сума вартости, якою мусить порядкувати поодинокий
посідач грошей або товарів, щоб перетворитися на капіталіста,
змінюється на різних ступенях розвитку капіталістичної
продукції і, за даного ступеня розвитку, вона в різних сферах
продукції різна залежно від їхніх осібних технічних умов. Певні
галузі промисловости вже на початках капіталістичної продукції
потребують такого мінімуму капіталу, якого ще немає в руках
поодиноких індивідів. Це призводить почасти до державних субсидій
для приватних осіб, як от у Франції за часів Кольбера, а в
деяких німецьких державах геть аж і донині, а почасти до утворення
товариств із законною монополією на ведення деяких
галузей промисловости й торговлі 206 — предтеч сучасних акційних
товариств.

нуче, якщо він буде прикутий до якогось одного місця». («The farmer
cannot rely on his own labour; and if he does, I will maintain that he is
a loser by it. His employement should be, a general attention to the whole:
his thrasher must be watched, or he will soon lose his wages in corn not
thrashed out; his mowers, reapers etc. must be looked after; he must constantly
go round his fences: he must see there is no neglect; which would
be the case if he was confined to any one spont»). («An Enquiry into Connection
between the Price o Provisions, and the Size of Farms etc. By a
Farmer», London 1773, p. 12). Цей твір дуже цікавий. На ньому можна
вивчати генезу «фармера-капіталіста» або «merchant farmer»,* як виразно
його тут названо, і почути, як той фармер пишається перед «small
farmer»,** для якого справа по суті сходить на засоби існування. «Кляса
капіталістів, спочатку частинно і кінець-кінцем цілком визволяється
від доконечности ручної праці». (Richard Jones: «Textbook of Lectures on
the Political Economy of Nation», Hertford 1852, Lecture III, p. 39).

205a Молекулярна теорія, застосована в сучасній хемії, вперше науково
розвинута в Льорана й Жерара, спирається саме на цей закон. [Додаток
до 3 видання]. — Для пояснення цієї примітки, досить темної для осіб,
незнайомих із хемією, ми зауважимо, що автор говорить тут про вуглеводневі
сполуки, названі Жераром в 1843 р. спочатку «гомологічними
рядами», що з них кожний має власну альґебричну формулу сполукиНаприклад,
ряд парафіни: Сn, Н2n + 2; ряд нормальних алькоголів:
Сn, Н2n + 2, O; ряд нормальних жирних кислот; Сn, Н2n, O2 і багато інших.
У вищенаведених прикладах за допомогою простого кількісного додавання
С Н2 до молекулярної формули кожного разу твориться якісно відмінне
тіло. Щождо Марксової переоцінки участи Льорана й Жерара в установленні
цього важного факту порівн. Kopp: «Entwicklung der Chemie»,
München 1873, S. 709 і 716, та Schorlemmer: «Rise and Progress of
Organic Chemistry», London 1879, p. 54. — Ф. E.

206 Такі установи Мартин Лютер називає «Die Gesellschaft Monopolia»,
тобто монопольними товариствами.

* — фармера-купця. Ред.

** — дрібним фармером. Ред.
