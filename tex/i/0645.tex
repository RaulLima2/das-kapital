фактурні вироби, для яких саме рільничі округи становлять
ринок збуту. Численні тут і там порозкидані споживачі, досі
обслуговувані багатьма дрібними продуцентами, що працювали
на власну руку, сконцентровуються тепер в один великий ринок,
обслуговуваний промисловим капіталом.234 Так пліч-о-пліч
з експропріяцією селян, що раніш господарювали самостійно, і
з відокремленням їх від засобів продукції відбувається нищення
сільської підсобної промисловости, процес відокремлення мануфактури
від рільництва. І лише знищення сільської домашньої
продукції може надати внутрішньому ринкові країни тих розмірів
і тієї сталости, що їх потребує капіталістичний спосіб продукції.

Однак період мануфактури у власному значенні слова ще не
приводить до радикального перевороту. Пригадаймо, що мануфактура
опановує національну продукцію лише частково, спорадично,
і завжди спирається на міське ремество й домашню
сільську підсобну промисловість як на свою широку базу.
Якщо мануфактура знищує домашню сільську підсобну промисловість
в одній формі, в осібних галузях продукції, у певних
пунктах, то вона створює їх знову в інших пунктах, бо вона до
певної міри потребує її для оброблювання сировинного матеріялу.
Тим то вона створює нову клясу дрібних рільників, що для них
оброблювання землі є лише підсобна галузь, а головне їхнє заняття
є промислова праця, що її продукт вони — безпосередньо
або посередньо через купця — продають мануфактурі. Це —
причина, хоч і не головна, того явища, яке насамперед спантеличує
дослідника англійської історії. Починаючи від останньої
третини XV століття, дослідник натрапляє там на постійні,
лише іноді на короткий час притихлі, нарікання на зріст капіталістичного
господарства на селі й на прогресивне нищення
селянства. З другого боку, він завжди знову знаходить там це
селянство, хоч і в зменшеній кількості і в дедалі гірших умовах.235
Головна причина цього ось у чому: в Англії навпереміну пере-

234 «Коли двадцять фунтів вовни непомітно перетворюються на потрібний
протягом року для родини робітника одяг через власну працю
цієї родини, в перервах поміж іншими її роботами, то це не впадає в очі,
але винесіть цю вовну на ринок, відішліть її на фабрику, звідти до маклера,
потім до торгівця — і ви матимете великі комерційні операції і
номінальний капітал у двадцять разів більший за вартість продукту...
Таким чином робітничу клясу визискується на те, щоб підтримувати
нужденну фабричну людність, паразитарну клясу крамарів і фіктивну
комерційну, грошову й фінансову систему» («Twenty pounds of wool
converted unobtrusively into the yearly clothing of a labourer’s family
by its own industry in the intervals of other work — this makes no show;
but bring it to market, send it to the factory, thence to the broker, thence
to the dealer, and you will have great commercial operations, and nominal
capital engaged to the amount of twenty times its value... The working
class is thus emerced to support a wretched factory population, a parasitical
shopkeeping class, and a fictitious commercial, monetary and financial
system»). (David Urquhart: «Familiar Words», London 1855, p. 120).

235 Виняток являють собою часи Кромвела. Поки існувала республіка,
всі верстви англійської народньої маси піднеслися з того занепаду,
в якому вони були за Тюдорів.
