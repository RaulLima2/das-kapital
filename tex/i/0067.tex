Г — Т, купівля є разом з тим продаж, Т — Г; отже, остання
метаморфоза товару є разом з тим перша метаморфоза якогось
іншого товару. Для нашого ткача життєвий шлях його товару
закінчується біблією, на яку він зворотно перетворив 2 фунти
стерлінґів. Але продавець біблії перетворює вторговані від ткача
2 фунти стерлінґів на горілку-житнівку. Г — Т, кінцева фаза
Т — Г — Т (полотно — гроші — біблія), є разом з тим Т — Г,
перша фаза Т — Г — Т (біблія — гроші — горілка-житнівка).
Через те, що продуцент товару постачає на ринок лише однорідний
продукт, він часто продає його більшими масами, тимчасом
як його різноманітні потреби приневолюють його завжди
роздрібнювати зреалізовану ціну або вторговану суму грошей
на численні купівлі. Тому один продаж є вихідний пункт багатьох
купівель різних товарів. Таким чином кінцева метаморфоза
даного товару становить суму перших метаморфоз інших товарів.

Розглядаючи тепер повну метаморфозу якогось товару, наприклад,
полотна, ми бачимо насамперед, що ця метаморфоза
складається з двох рухів, які один одному протилежні й один
одного доповнюють: Т — Г і Г — Т. Ці два протилежні перетворення
товару відбуваються в двох протилежних суспільних
актах посідача товарів та відбиваються в двох протилежних
економічних функціях останнього. Як аґент продажу, він є продавець,
як аґент купівлі, він є покупець. Але так само, як у
кожному перетворенні товару існують одночасно його обидві
форми, товарова форма й грошова, тільки на протилежних полюсах,
так тому самому посідачеві товарів як продавцеві протистоїть
інший покупець, а як покупцеві йому протистоїть інший продавець.
Так само, як той самий товар послідовно пророблює ці два протилежні
перетворення, з товару стає грішми, а з грошей товаром, так
той самий посідач товарів переміняє ролю продавця на ролю покупця.
Отже, купівля й продаж — це не фіксовані функції, а функції,
що в межах товарової циркуляції постійно переміняють індивідів.

Повна метаморфоза товару в її найпростішій формі має собі
за передумову чотири полюси і три Personae dramatis.* Спочатку
гроші виступають проти товару як образ його вартости, яка
«по тому боці», в чужій кишені, має речову, грубу реальність.
Таким чином проти посідача товарів виступає посідач грошей.
Скоро тільки товар перетворивсь у гроші, останні стають його
минущою еквівалентною формою, що її споживна вартість або
зміст існує «по цей бік», в інших товарових тілах. Як кінцевий
пункт першого перетворення товару гроші є разом з тим вихідний
пункт його другого перетворення. Таким чином продавець
із першого акту стає покупцем у другім акті, де проти нього
виступає третій посідач товарів як продавець.71

71 «Отже, тут є чотири члени й три контраґенти, з яких один виступає
двічі» («II у a done quatre termes et trois contractants, dont l’un intervient
deux fois»). (Le Trosne: «De l’Intérêt Social». Physiocrates, éd. Daire,
Paris 1846, p. 908).

* — дієві особи. Ред.
