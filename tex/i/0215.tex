«Коли святкування сьомого дня в тижні вважається за божу
установу, то це містить у собі те, що інші дні в тижні належать
праці (він має на думці — капіталові, як ми це зараз побачимо),
і примус, щоб цей божий наказ було виконано, неможна ганьбити
як жорстокість... Що людство взагалі з природи має нахил до
комфорту і лінощів, про те переконує нас фатальний досвід із
поведінкою нашої мануфактурної черні, яка працює пересічно
не більш як 4 дні на тиждень, за винятком випадків подорожчання
засобів існування... Припустімо, що бушель пшениці репрезентує
всі засоби існування робітника й коштує 5 шилінґів, а робітник
заробляє своєю працею денно 1 шилінґ. Тоді йому треба працювати
лише 5 днів на тиждень і лише 4 дні, коли бушель коштує
4 шилінґи... Але що в цьому королівстві заробітна плата багато
вища порівняно з ціною засобів існування, то мануфактурний
робітник, який працює 4 дні, має надлишок грошей, і за нього
він решту тижня живе бездільно... Сподіваюся, сказаного досить,
щоб з’ясувати, що помірна праця протягом шістьох днів на тиждень
не є рабство. Наші рільничі робітники працюють саме так,
і з усього видно, що вони найщасливіші серед робітників (labouring
poor),\footnote{
«An Essay etc.». Він сам оповідає на стор. 96, у чому було вже
1770 р. «щастя» англійських рільничих робітників. «Їхня робоча сила
(«their working powers») завжди напружена до краю («on the stretch»);
вони не можуть ні жити гірше, ніж живуть («they cannot live cheaper
that they do»), ні важче працювати» («nor work harder»).
} а голляндці працюють стільки ж у мануфактурах і мають
вигляд дуже щасливого народу. Французи, за винятком безлічі
свят, які переривають робочий час, працюють стільки ж...\footnote{
Протестантизм відіграє значну ролю в генезі капіталу вже хоч
би тим, що він перетворив майже всі традиційні свята на робочі дні.
}
Але наша чернь вбила собі в голову idée fixe,\footnote*{
— невідступну ідею. Ред.
} що їй, як англійцям,
належить за правом народження привілей користуватися більшою
волею й незалежністю, ніж [робітничій людності], у будь-якій
іншій європейській країні. Оскільки ця ідея впливає на
хоробрість наших солдатів, вона може мати деяку користь; але
що менш її мають мануфактурні робітники, то краще для них
самих і для держави. Робітники ніколи не сміють вважати себе
незалежними від своїх начальників («independent of their superiors»)...
Надзвичайно небезпечно потурати наволочі в комерційній
державі — такій, як ось наша, де може 7/8 цілої людности
має лише невеличку власність або зовсім її немає...\footnote{
«An Essay etc.», р. 15, 41, 96, 97, 55, 57.
} Повного
видужання не може бути доти, доки наша промислова біднота не
згодиться працювати 6 днів за ту саму суму, яку вона тепер заробляє
за 4 дні».\footnote{
Там же, стор. 69. Ще 1734 р. Джекоб Вандерлінт пояснив, що таємниця
всіх нарікань капіталістів на лінощі робітничої людности є просто
в тому, що капіталісти хотіли дістати за ту саму заробітну плату шість
робочих днів замість чотирьох.
} З цією метою, так само як і для того, щоб
«викоренити ледарство, розпусту та романтичні химери про волю»,
тобто, щоб «зменшити видатки на бідних, підтримати дух під-