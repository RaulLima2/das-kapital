мет праці, що його вже профільтровано через процес праці й що
сам вже є продукт праці, приміром, насіння в рільництві. Тварини
й рослини, що їх звикли вважати за продукти природи, є не
лише продукти праці, може, попереднього року, але в їхніх теперішніх
формах вони є продукти змін, що склалися протягом
багатьох поколінь під контролем людини, за допомогою людської
праці. Щождо засобів праці зокрема, то величезна більшість їх
навіть при найповерховішому розгляді показує сліди минулої
праці.

Сировинний матеріял може становити головну субстанцію
продукту або ввіходити в процес його утворення лише як допоміжний
матеріял. Допоміжний матеріял споживають засоби праці,
як от вугілля споживає парова машина, олію — колесо, сіно —
робочий кінь, або його додають до сировинного матеріялу, щоб
спричинити в ньому речову зміну, як от хлор — до небіленого
полотна, вугілля — до заліза, фарбу — до вовни, або він допомагає
виконувати саму працю, як от, приміром, матеріяли, що
ними користуються для освітлення й опалення робітних майстерень.
У власне хемічній фабрикації ріжниця між головним матеріялом
і допоміжним матеріялом зникає, бо жоден з уживаних
сировинних матеріялів не появляється знов як субстанція продукту.8
А що всяка річ має багато різних властивостей і тому придатна
для різних способів використовування, то той самий продукт
може бути сировинним матеріялом для дуже різних процесів
праці. Наприклад, зерно є сировинний матеріял для мірошника,
для фабриканта крохмалю, гуральника, скотаря й т. ін. Як
насіння воно стає сировинним матеріялом для своєї власної продукції.
Так само й вугілля як продукт виходить із гірничої промисловости,
а як засіб продукції — увіходить до неї.

Той самий продукт у тому самому процесі праці може служити
за засіб праці й за сировинний матеріял. Приміром, худоба,
що її годують, є оброблюваний сировинний матеріял і разом
з тим засіб для виготовлення гною.

Продукт, що існує в готовій для споживання формі, може знов
стати сировинним матеріялом для іншого продукту, як ось виноград
— сировинним матеріялом для вина. Або праця випускає
свій продукт у формах, у яких він знов може бути вжитий лише
як сировинний матеріял. Сировинний матеріял у такому стані
зветься півфабрикатом або краще звати його ступневим фабрикатом
(Stufenfabrikat), як от бавовна, нитки, пряжа тощо.
Первісний сировинний матеріял, хоч і сам він вже є продукт,
повинен, однак, перейти ще цілий ряд різних процесів, де він
у завжди змінливій формі знову й знов функціонує як сировинний
матеріял аж до останнього процесу праці, який відштовхує
його від себе як готовий засіб існування або як готовий засіб праці.

8  Шторх одрізняє власне сировинний матеріял як «matière» від
допоміжних матеріялів як «matériaux»; Шербулье називає допоміжні
матеріяли «matières instrumentales».
