одна робоча машина, яка працює за допомогою комбінації різних
знарядь. Чи така робоча машина є лише механічне відродження
складнішого ремісничого знаряддя, чи вона є комбінація різнорідних
простих інструментів, спеціялізованих мануфактурою, на
фабриці, тобто в основаній на машиновому виробництві майстерні,
кожного разу знов з’являється проста кооперація, і насамперед
(робітника ми лишаємо тут осторонь) саме як зосередження у просторі
однорідних робочих машин, що одночасно разом функціонують.
Так, ткацька фабрика утворюється через сполучення
багатьох механічних ткацьких варстатів, а швацька фабрика —
через сполучення багатьох швацьких машин у тому самому робітному
приміщенні. Але тут існує технічна єдність, тому що це
велике число однорідних робочих машин одночасно та рівномірно
дістає свій рух від руху спільного першого мотора, руху, що
переноситься на них за допомогою передатного механізму, теж
почасти спільного всім їм, бо від нього розходяться лише осібні
відгалуження для кожної окремої виконавчої машини. Цілком
так само як численні знаряддя становлять лише органи однієї
робочої машини, так само й численні робочі машини становлять
тепер лише однорідні органи того самого рухового механізму.
Але система машин у власному значенні слова заступає
поодиноку самостійну машину лише там, де предмет праці послідовно
перебігає ряд зв’язаних між собою різних частинних
процесів, що їх поступінно виконує низка різнорідних виконавчих
машин, які однак одна одну доповнюють. Тут знову з’являється
характеристична для мануфактури кооперація, що ґрунтується
на поділі праці, але тепер уже як комбінація частинних
робочих машин. Специфічні знаряддя різних частинних робітників,
приміром, у вовняній мануфактурі знаряддя шаповалів,
чухральників вовни, стригунів вовни, прядунів вовни й т. ін.,
перетворюються тепер на знаряддя специфікованих робочих машин,
що з них кожна становить осібний орган для осібної функції
в системі комбінованого робочого механізму. Сама мануфактура
дає машиновій системі по тих галузях, де її заводиться вперше,
взагалі та в цілому природну основу поділу, а тому й організацію
процесу продукції.\footnote{
Перед епохою великої промисловости вовняна мануфактура була
домінантною мануфактурою Англії. Тим то за першу половину XVIII століття
саме в ній пороблено більшість експериментів. Досліди, пороблені
на овечій вовні, стали корисними і для бавовни, що її механічне
оброблення потребує менш тяжких підготовчих праць, так само як пізніше,
навпаки, механічна вовняна промисловість розвинулась на основі
механічного прядіння й ткання бавовни. Поодинокі елементи вовняної
мануфактури, як, наприклад, чесання вовни, заведено у фабричну систему
лише в останні десятиліття. «Вживання механічної сили до чесання вовни...
Дуже поширене від часів заведення «чесальної машини», особливої машини
Лістера... мало безперечно своїм наслідком те, що дуже велике число
людей лишилося без праці. Перше вовну розчісували руками здебільшого
вдома в чесальника. Тепер її звичайно розчісують на фабриці, і ручну
} Однак відразу виступає посутня ріжниця.