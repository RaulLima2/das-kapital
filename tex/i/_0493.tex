\index{i}{0493}  %% посилання на сторінку оригінального видання
Насамперед річна продукція мусить постачити всі ті предмети
(споживні вартості), що з них можна замістити спожиті
протягом року речові складові частини капіталу. Коли відняти
ці предмети, то лишається чистий, або додатковий продукт, у
якому міститься додаткова вартість. А з чого складається цей
додатковий продукт? Може, з речей, призначених на задоволення
потреб і примх кляси капіталістів, отже, з речей, що входять
у їхній споживний фонд? Коли б це було так, то додаткову
вартість прогулялося б усю без остачі, і відбувалася б лише
проста репродукція.

Щоб акумулювати, треба частину додаткової вартости перетворювати
на капітал. Але, не творячи чуда, можна перетворювати
на капітал лише такі речі, які можна вживати в процесі
праці, тобто засоби продукції, і далі такі речі, з яких робітник
може себе утримувати, тобто засоби існування. Отже, частина
річної додаткової праці мусить бути вжита на виготовлення додаткових
засобів продукції і засобів існування понад ту їхню
кількість, яка була потрібна, щоб замістити авансований капітал.
Одно слово, додаткову вартість лише тому можна перетворювати
на капітал, що додатковий продукт, вартістю якого вона
є, вже містить у собі речові складові частини нового капіталу.\footnoteA{
Ми абстрагуємось тут від міжнародньої торговлі, за допомогою
якої нація може перетворювати предмети розкошів на засоби продукції
або на засоби існування, і навпаки. Щоб розглянути предмет досліду в
його чистоті, вільним від перешкодних побічних обставин, ми мусимо тут
увесь торговельний світ розглядати як одну націю і припустити, що капіталістична
продукція всюди вкоренилася й опанувала всі галузі промисловости.
}

Щоб заставити ці складові частини фактично функціонувати
як капітал, кляса капіталістів потребує якогось додатку праці.
Якщо експлуатація занятих уже робітників не зростає ті екстенсивно,
ані інтенсивно, то треба вжити додаткових робочих сил.
Про це також подбав уже механізм капіталістичної продукції,
репродукуючи робітничу клясу як клясу, залежну від заробітної
плати, клясу, що її звичайної заробітної плати вистачає на
те, щоб не тільки забезпечити своє утримання, але й розмноження.
Капіталові треба ці додаткові робочі сили різного віку, постачувані
йому щороку робітничою клясою, долучити тільки до
додаткових засобів продукції, що вже містяться в річній продукції,
— і перетворення додаткової вартости на капітал є готове.
Розглядувана конкретно акумуляція сходить на репродукцію
капіталу в чимраз ширшому маштабі. Кругобіг простої репродукції
змінюється й перетворюється, за висловом Сісмонді, на
спіралю.\footnoteA{
Сісмондівська аналіза акумуляції має ту велику хибу, що Сісмонді
надто вже задовольняється фразою: «Перетворення доходу на капітал»,
не досліджуючи матеріяльних умов цієї операції.
}

Вернімось тепер назад до нашого прикладу. Це стара історія:
Аврам породив Ісаака, Ісаак породив Якова й т. д. Первісний
капітал у 10.000 фунтів стерлінґів дає додаткову вартість
\parbreak{}  %% абзац продовжується на наступній сторінці
