\parcont{}  %% абзац починається на попередній сторінці
\index{i}{0301}  %% посилання на сторінку оригінального видання
а цілий механізм, що функціонує в мануфактурі, не має об’єктивного
кістяка, незалежного від самих робітників, капіталові
постійно доводиться боротися з непокірністю робітників. «Слабощі
людської натури, — вигукує наш приятель Юр, — є такі великі,
що чим робітник вправніший, тим він свавільніший, і тим тяжче
з ним поводитися, і, отже, тим більшої шкоди завдає він своїми
божевільними примхами цілому механізмові».\footnote{
Ure: «Philosophy of Manufacture», p. 20.
} Тому протягом
цілого мануфактурного періоду лунають скарги на брак дисципліни
в робітників.\footnote{
Сказане в тексті далеко більше стосується до Англії, ніж до Франції,
а до Франції більше, ніж до Голляндії.
} І якби в нас і не було свідчень тогочасних
письменників, то вже ті прості факти, що, починаючи від XVI віку,
аж до епохи великої промисловости, капіталові не вдається опанувати
цілий робочий час, який мав у своєму розпорядженні
мануфактурний робітник, що мануфактури недовговічні та що
вони разом з еміґрацією або іміграцією робітників полишають
свою осаду в одній країні і виринають в іншій, — вже ці факти
говорять нам більше, ніж цілі бібліотеки. «Порядок мусить бути
тим або іншим способом установлений» — вигукує 1770 р. не раз
уже цитований автор «Essay on Trade and Commerce». — «Порядку»,
лунає 66 років пізніше з уст д-ра Ендр’ю Юра, «порядку»
бракувало в мануфактурі, що ґрунтувалася «на схоластичній
догмі поділу праці», і «Аркрайт створив порядок».

Разом з тим мануфактура не спромоглася ні охопити суспільну
продукцію в повному її обсягу, ні революціонізувати її в
її корені. Як економічний витвір мистецтва високо підіймалася
вона вгору на широкій основі міського ремества та сільської домашньої
промисловости. Її власна вузька технічна база на певному
ступені розвитку стала в суперечність до утворених нею самою
потреб продукції.

Одним із найдосконаліших витворів мануфактури була майстерня
для продукції самих знарядь праці та особливо складних
механічних апаратів, які вже були тоді в ужитку. «Така майстерня,
— каже Юр, — ставила перед очі поділ праці з його різноманітними
ступенями. Свердло, різак, токарський варстат —
кожне мало свого власного робітника, ієрархічно зв’язаного з
іншими залежно від ступеня його вправности». Цей продукт
мануфактурного поділу праці породив, з свого боку, машини.
Ці останні знищують ремісничу діяльність як регулятивний
принцип суспільної продукції. Таким чином, з одного боку,
усувається технічна доконечність прив’язувати робітника на
цілий його вік до якоїсь частинної функції. З другого боку,
падають ті межі, що їх цей принцип ще ставив пануванню
капіталу.
