\index{i}{0342}  %% посилання на сторінку оригінального видання
«Через те, що кількість продуктів реґулюється переважно
швидкістю машин, то інтерес фабриканта мусить бути в тому,
щоб гнати їх із якнайбільшою швидкістю, яку тільки можна
сполучити з такими умовами: зберігання машин, щоб вони не
надто скоро псувалися, зберігання якости фабрикованих продуктів
та здатність робітника встигати за рухом без більшого напруження,
ніж яке він може безупинно розвивати. Часто буває так,
що фабрикант, дуже поспішаючися, занадто прискорює рух.
Тоді полами та поганий виріб більш ніж урівноважують швидкість,
і він примушений зробити рух машин повільнішим. А що
активний та розсудливий фабрикант знаходить осяжний максимум,
то я дійшов висновку, що неможливо за 11 годин випродукувати
стільки, як за 12 годин. Крім того, я припускав, що робітник,
якому платять відштучно, напружує свої сили до якнайбільшої
міри, скільки він може безупинно витримувати такий
ступінь інтенсивности праці».\footnote{
«Reports of Insp. of Fact. to 30 th April 1845», p. 20.
} Тим то Горнер, усупереч експериментам
Ґарднера й ін., дійшов такого висновку, що дальше скорочення
робочого дня нижче від 12 годин мусило б зменшити
кількість продукту.\footnote{
Там же, стор. 22.
} Він сам через 10 років цитує свої сумніви
з року 1845 на доказ того, як мало він ще тоді тямив в елястичності
машин та людської робочої сили, що обидві в рівній мірі
напружуються до якнайвищого ступеня через примусове скорочення
робочого дня.

А тепер перейдімо до періоду після 1847 р., від часу запровадження
закону про десятигодинний робочий день в англійських
бавовняних, вовняних, шовкових та лляних фабриках.

«Швидкість веретен на throstles зросла на 500 обертів, на
mules на 1.000 обертів на одну хвилину, тобто швидкість веретена
throstles, яка 1839 р. мала 4.500 обертів на хвилину, становить
тепер (1862 р.) 5.000 обертів, становить тепер 6.000 обертів на хвилину; отже,
швидкість збільшилася в першому випадку на \sfrac{1}{10}, а в другому —
на \sfrac{1}{5}».\footnote{
«Reports of Insp. of Fact. for 31 st October 1862», p. 62.
} Джеме Несміс, славетний цивільний інженер із
Patricroft’a коло Менчестеру, в одному листі до Леонарда Горнера
1852 р. пояснив у подробицях про поліпшення, пороблені в паровій
машині між 1848 і 1852 рр. Зауваживши, що парова кінська
сила, яка в офіціяльній фабричній статистиці все ще визначається
за ефектом її в 1828 р.,\footnote{
Це змінилося від часу «Parliamentary Return»\footnote*{
— парляментського звіту. \emph{Ред.}
} 1862 p. Тут виступає справжня парова кінська сила сучасних парових машин
та водяних коліс на місце номінальної (див. примітку 109а, стор. 318). І веретен
на сукання (Dublierspindeln) уже не переплутують із прядільними веретенами
у власному значенні (як у «Returns» 1839, 1850 та 1856); далі
для вовняних фабрик подано число «gigs»,\footnote*{
— ворсувальних машин. \emph{Ред.}
} заведено ріжницю між джу
товими та конопляними фабриками, з одного боку, і лляними — з другого;
нарешті, вперше заведено до звіту панчішне виробництво.
} є лише номінальна й може бути
\index{i}{0343}  %% посилання на сторінку оригінального видання
показником (Index) дійсної сили, він, між іншим, каже:
«Немає ніякого сумніву, що парові машини тієї самої ваги, часто
ті самі ідентичні машини, де пороблено лише сучасні поліпшення,
виконують пересічно на 50\% більше праці, ніж раніш, та що в
багатьох випадках ті самі ідентичні парові машини, які за часів
обмеженої швидкости в 220 футів на хвилину давали 50 кінських
сил, тепер, при зменшеному споживанні вугілля, дають понад 100
кінських сил\dots{} Сучасна парова машина з тією самою кількістю
номінальних кінських сил, у наслідок поліпшення в її конструкції,
зменшення розміру та поліпшення конструкції парового
казана тощо, діє з більшою силою, ніж раніш\dots{} Тому, хоч супроти
номінальної кінської сили вживається те саме число рук, що й
раніш, однак супроти робочих машин вживається менше число
рук».\footnote{
«Reports of Insp. of Fact, for 31 st October 1856», p. 11.
} 1850 р. фабрики Об’єднаного Королівства вживали 134.217
номінальних кінських сил, щоб рухати 25.638.716 веретен та
301.495 ткацьких варстатів. 1856 р. число веретен і ткацьких
варстатів становило відповідно 33.503.580 і 369.205. Коли б
потрібна кінська сила лишилася та сама як 1850 р., то 1856 р.
потрібно було б 175.000 кінських сил. Але за офіціяльними документами
число їх становило лише 161.435, отже, понад 10.000
кінських сил менше, ніж їх потрібно було б на основі розрахунку
з 1850 р.\footnote{
Там же, стор. 14, 15.
} «Останній «Return» з 1850 р. (офіціяльна статистика)
установлює той факт, що фабрична система поширюється з чимраз
більшою швидкістю, число рук у відношенні до машин зменшилося,
парова машина в наслідок економії на силі та інших
метод рухає машини більшої ваги, і що більшої кількости продукту
досягається в наслідок поліпшення робочих машин, змінених
метод фабрикації, збільшеної швидкости машин та багатьох інших
причин».\footnote{
Там же, стор. 20.
} «Великі поліпшення, які пороблено в машинах
усякого роду, дуже підвищили їхню продуктивну силу. Немає
ніякого сумніву, що скорочення робочого дня дало\dots{} стимул
до цих поліпшень. Ці поліпшення й інтенсивніше напруження
робітника призвели до того, що протягом скороченого (на 2 години,
або на одну шосту) робочого дня продукується, щонайменше,
стільки ж продукту, як раніш протягом довшого дня».\footnote{
«Reports etc. for 31 st October 1858», p. 9, 10. Порівн. «Reports
etc. for 30 th April 1860», p. 30 і далі.
}

Як зросло збагачення фабрикантів у наслідок інтенсивнішого
визиску робочої сили, доводить уже одна та обставина, що пересічне
пропорційне зростання англійських бавовняних і т. ін.
фабрик становило 1838 – 1850 рр. 32\%, а 1850 – 1856 рр. – 86\%.

Хоч який великий був проґрес англійської промисловости
за вісім років – від 1848 до 1856 – за панування десятигодинного
\index{i}{0344}  %% посилання на сторінку оригінального видання
робочого дня, все ж його знову значно перевищив дальший
шестирічний період від 1856 до 1862 р. Наприклад, 1856 р. по
шовкових фабриках було 1.093.799 веретен, а 1862 р. — 1.388.544;
ткацьких варстатів 1856 р. було 9.260, а 1862 р. — 10.709. Навпаки,
число робітників 1856 р. становило 56.131, а 1862 р. — 52.429.
Це дає збільшення числа веретен на 26,9\% і ткацьких варстатів
на 15,6\% при одночасному зменшенні робітників на 7\%. 1850 р. на
фабриках суканої вовни було в ужитку 875.830 веретен, 1856 р. —
1.324.549 (приріст на 51,2\%), а 1862 р. — 1.289.172 (зменшення
на 2,7\%). А коли відлічити веретена на сукання (Dublierspindeln),
які фігурують у переліку за 1856 р., але не фігурують у переліку
за 1862 р., то число веретен від 1856 р. майже не змінилося. Навпаки,
швидкість веретен і ткацьких варстатів від 1850 р. в
багатьох випадках подвоїлася. Число парових ткацьких варстатів
по фабриках суканої вовни 1850 р. становило 32.617,
1856 р. — 38.956, а 1862 р. — 43.048. Коло них працювало 1850 р.
79.737 осіб, 1856 р. — 87.794, а 1862 р. — 86.063, та з них було
дітей, молодших за 14 років: 1850 р. — 9.956, 1856 р. — 11.228,
а 1862 р. — 13.178. Отже, не зважаючи на значне збільшення
числа ткацьких варстатів у 1862 р. порівняно з роком 1856, загальне
число вживаних робітників зменшилось, а число експлуатованих
дітей зросло.\footnote{
«Reports of Insp. of Fact, for 31 st October 1862», p. 100 і 130.
}

27 квітня 1863 р. член парляменту Ферранд ось що заявив у
Палаті громад: «Делеґати робітників із 16 округ Ланкашіру й
Чешіру, з доручення яких я говорю, повідомили мене, що в наслідок
поліпшення машин праця на фабриках постійно зростає.
Раніш одна особа з помічниками обслуговувала два ткацькі верстати,
тепер одна особа без помічників обслуговує три варстати,
а часто-густо й чотири варстати і т. д. Як це видно з поданих
фактів, дванадцять годин праці стиснуто тепер менше, ніж у
10 робочих годин. Тому само собою зрозуміло, до якого величезного
розміру зросла тяжкість праці фабричних робітників останніми
роками».\footnote{
За допомогою сучасного парового варстату один ткач на двох
варстатах фабрикує тепер за 60 годин на тиждень 26 сувоїв певного сорту
тканини певної довжини та ширини, а раніш на старому паровому ткацькому
варстаті він міг продукувати лише 4 такі сувої. Витрати на ткання
одного такого сувою вже на початку 1850-х років спали з 2 шилінґів
9 пенсів до 5\sfrac{1}{8} пенсів.

Додаток до другого видання: «Перед З0 роками (1841) від одного прядуна
бавовняної пряжі з трьома помічниками вимагали доглядати лише
за однією парою мюлів із 300 – 324 веретенами. Тепер (кінець 1871 р.)
він з п’ятьма помічниками має доглядати мюлів, що їхнє число веретен
становить 2.200, та продукує, щонайменше, всемеро більше пряжі, ніж
1841 р.». (\emph{Alexander Redgrave}, фабричний інспектор, у «Journal of
the Society of Arts», 5 Januar 1872).
}

Тому, хоч фабричні інспектори невтомно та з повним правом
вихваляють сприятливі результати законів 1844 та 1850 рр.,
все ж вони визнають, що скорочення робочого дня викликало вже
\parbreak{}  %% абзац продовжується на наступній сторінці
