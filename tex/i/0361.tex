Загальний результат механічних поліпшень, заведених в англійській
бавовняній промисловості під впливом американської
громадянської війни, показує оця таблиця:

Число фабрик

                                                                      1858 р.                       
1861 р.                1868 р.

Англія та Велз.......................                   2.046                          2.715        
          2.405
Шотляндія.............................                   152                               163      
                131
Ірляндія................................                     12                                    9
                        13
Об’єднане Королівство.........                 2.210                            2.887               
   2.549

Число парових ткацьких  варстатів

Англія та Велз......................              275.590                        368.125            
  344.719
Шотляндія............................               21.624                          30.110          
       31.864
Ірляндія...............................                  1.633                            1.757     
              2.746
Об’єднане Королівство.........             298.847                        399.992              
379.329

Число веретен

Англія та Велз.....................       25.818.576                    28.352.152          
30.478.228
Шотляндія..........................          2.041.129                      1.915.398             
1.397.546
Ірляндія.............................              150.512                         119.944          
      124.240
Об’єднане Королівство.......        28.010.217                    30.387.494            32.000.014

Чиcло вживаних робітників

Англія та Велз...................               341.170                          407.598            
    357.052
Шотляндія........................                  34.698                            41.237         
        39.809
Ірляндія...........................                     3.345                              2.734    
               4.203
Об’єднане Королівство.....               379.213                           451.569               
401.064

Отже, від 1861 до 1868 р. зникло 338 бавовняних фабрик, тобто
продуктивніший та більший машиновий механізм сконцентрувався
в руках меншого числа капіталістів. Число парових ткацьких
варстатів зменшилося на 20.663; але продукт їхній одночасно
збільшився, так що поліпшений ткацький варстат давав
тепер більше продукту, ніж старий. Нарешті, число веретен
зросло на 1.612.541, тимчасом як число вживаних робітників
зменшилося на 50.505. Отже, ті «тимчасові» злидні, що ними бавовняна
криза душила робітників, збільшив і зміцнив хуткий
та невпинний проґрес машинової системи.

Однак машина діє не тільки як непереможний конкурент,
який завжди напоготові зробити найманого робітника «зайвим».
Капітал голосно й тенденційно проголошує її силою, ворожою
робітникові, та саме як таку вживає її. Вона стає наймогутнішим
бойовим знаряддям придушувати періодичні робітничі повстання,
страйки і т. ін. проти автократії капіталу.208 За Ґаске-

208 «Відносини між хазяїнами й руками по фабриках флінтґлясу та пляшкового
скла — це хронічний страйк». Звідси швидкий розвиток мануфактури
пресованого скла, де головні операції виконуються за допомогою машин.
Одна фірма в Ньюкестлі, яка раніш продукувала 350.000 фунтів дутого
кремінного скла річно, тепер замість цієї кільцости продукує 3.000.500
фунтів пресованого скла». («Children’s Employment Commission. 4 th
Report 1865», p. 262, 263).
