тів стерлінґів будуть витрачені на заробітну плату, або поки
цілий продукт, що його репрезентують 2.000 фунтів стерлінґів,
буде спожитий продуктивними робітниками. Ми бачимо: вся
сила цього арґументу лежить у словах «і т. д.», що посилають
нас від Понтія до Пілата. Дійсно, А. Сміс уриває свій дослід
саме там, де починаються його труднощі.31

Поки ми беремо на увагу лише фонд річної продукції в цілому,
річний процес репродукції легко зрозуміти. Але всі складові
частини річної продукції треба винести на товаровий ринок, і
тут саме починаються труднощі. Рухи поодиноких капіталів і
особистих доходів перехрещуються між собою, переплутуються,
губляться в загальній зміні місць — у циркуляції суспільного
багатства — у тій зміні місць, що спантеличує спостерігача та
ставить дослідові дуже заплутані завдання. У третьому відділі
другої книги я подам аналізу дійсного зв’язку всіх тих явищ.
[Там виявиться, що догма А. Сміта, успадкована всіма його послідовникам,
перешкоджала політичній економії зрозуміти навіть
елементарний механізм суспільного процесу репродукції].* Велика
заслуга фізіократів у тому, що вони в своєму «tableau économique»
вперше зробили спробу дати картину річної продукції
в тому вигляді, в якому вона виходить із циркуляції.32

А втім, само собою зрозуміло, що політична економія не проминула
використати в інтересах кляси капіталістів тезу А. Сміта,
ніби всю перетворену на капітал частину чистого продукту споживає
робітнича кляса.

31    Не вважаючи на свою «Логіку», Дж. Ст. Мілл ніде навіть і не
помічає цієї хибної аналізи своїх попередників, яка навіть у межах буржуазного
горизонту, просто з погляду фахівця, потребує поправок. Він
скрізь реєструє з догматизмом школяра плутанину думок своїх учителів.
Так само й тут: «Сам капітал згодом цілком сходить на заробітну плату,
і навіть коли він через продаж продукту відновлюється, він потім знову
перетворюється на заробітну плату» («The capital itself in the long run
becomes entirely wages, and when replaced by the sale of produce becomes
wages again»).

32    А. Сміс у своєму викладі процесу репродукції, отже, і процес v
акумуляції, в деякому відношенні не тільки не зробив жодного поступу,
але зробив рішучий крок назад порівняно з своїми попередниками, особливо
порівняно з фізіократами. З його ілюзією, згаданою в тексті, пов’язана
справді казкова догма, також перейнята від нього політичною економією,
що ціна товарів складається із заробітної плати, зиску (процента)
і земельної ренти, отже, лише із заробітної плати й додаткової вартосте.
Виходячи з цієї бази, Шторх принаймні наївно признається: «Неможливо
розкласти доконечну ціну на її найпростіші елементи» («H est
impossible de résoudre le prix nécessaire dans ses éléments les plus simples»).
(Storch: «Cours d’Economie Politique», ed. Petersbourg 1815,
vol. II, p. 140, примітка). Гарна економічна наука, що проголошує за
неможливе розкласти ціну товарів на її найпростіші елементні Докладніше
про це питання ми скажемо в третьому відділі другої і в сьомому
відділі третьої книги.

* Заведене у прямі дужки ми беремо з другого німецького видання.
Ред.
