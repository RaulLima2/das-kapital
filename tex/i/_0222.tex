\index{i}{0222}  %% посилання на сторінку оригінального видання
Однак фабриканти дозволили такий «проґрес» не без компенсації
його «реґресом». За їхньою спонукою Палата громад скоротила
мінімальний вік дітей, яких можна було експлуатувати,
з 9 до 8 років, щоб забезпечити для капіталу «додаткове постачання
фабричних дітей»\footnote{
«Що скорочення годин їхньої праці спричиниться до збільшення
числа потрібних для праці (дітей), то вирішено, що додаткове постачання
дітей од 8 до 9 років завстаршки може покрити збільшений
попит» («As a reduction in their hours of work would cause a large number
(of children) to be employed, it was thought that the additional supply of
children from eight to nine years of age, would meet the increased demand»)
(1. c., p. 13).
} — річ, належна йому на основі всіх
божих і людських прав.

Роки 1846 і 47 становлять епоху в економічній історії Англії.
Скасовано хлібні закони, скасовано мито на довіз бавовни й
інших сировинних матеріялів, проголошено волю торговлі за
провідну зірку законодавства! Словом, наставало тисячолітнє
царство. З другого боку, чартистський рух і агітація за десятигодинний
робочий день дійшли цими роками свого найвищого
пункту. Вони знайшли собі спільників у торі, що палали помстою.
Не зважаючи на фанатичний опір зрадливої армії вільної торговлі
з Брайтом і Кобденом на чолі, біла про десятигодинний робочий
день, об’єкт такої довгочасної боротьби, парлямент ухвалив.

Новий фабричний закон з 8 червня 1847 р. ухвалив, що з
1 липня 1847 р. набирає сили попереднє скорочення робочого дня
до 11 годин для «підлітків» (від 13 до 18 років) і для всіх робітниць,
а 1 травня 1848 р. — остаточне обмеження робочого
дня 10 годинами для тих самих категорій. Щодо решти, то цей
закон був лише виправленим додатком до законів 1833 р. і 1844 р.

Капітал розпочав попередній похід з тим, щоб не допустити
до повного проведення в життя закону з 1 травня 1848 р. І власне
сами робітники, нібито навчені досвідом, повинні були допомогти
знову зруйнувати свою власну справу. Момент було обрано влучний.
«Треба собі пригадати, що в наслідок страшної кризи 1846 —
1847 рр. серед фабричних робітників панували великі злидні,
бо багато фабрик працювало лише неповний час, інші зовсім не
працювали. Тому значне число робітників перебувало в найскрутнішому
стані, багато було в боргах. Тим то з досить великою
певністю можна було припустити, що вони ладні будуть згодитися
на довший робочий час, щоб поповнити колишні втрати, сплатити,
може, борги, або викупити з заставничих домів свої меблі, або
змінити на нове продане майно, або придбати нову одежу собі й
своїй родині».\footnote{
«Reports of Insp. of Fact, for 31 st October 1848», p. 16.
} Пани фабриканти старалися збільшити природний
вплив цих обставин загальним зниженням заробітної плати
на 10\%. Це мало бути, так би мовити, посвятинами нової доби
вільної торговлі. Потім, скоро тільки робочий день скорочено
до 11 годин, наступило дальше пониження заробітної плати на
8\sfrac{1}{3}\% і нове подвійне зниження, скоро тільки робочий день остаточно
скорочено було до 10 годин. Тому повсюди, де це дозволяли
\parbreak{}  %% абзац продовжується на наступній сторінці
