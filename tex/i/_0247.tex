\index{i}{0247}  %% посилання на сторінку оригінального видання
Ми не спиняємось на подробицях тих змін, яких зазнало відношення
між капіталістом і найманим робітником у перебігу
процесу продукції, отже, не спиняємось і на дальших визначеннях
самого капіталу. Нам треба тут підкреслити лише небагато головних
пунктів.

У межах процесу продукції капітал розвинувся в командування
над працею, тобто над діющою робочою силою, або самим робітником.
Персоніфікований капітал, капіталіст, стежить за тим, щоб
робітник виконував свою працю дбайливо і з належною мірою
інтенсивности.

Далі капітал розвинувся на примусове відношення, що приневолює
робітничу клясу виконувати більше праці, ніж вимагало
вузьке коло його власних життєвих потреб. І як продуцент чужої
працьовитости, як висмоктувач додаткової праці й експлуататор
робочої сили, капітал щодо енерґії, безмірности й активности
перевищує всі попередні системи продукції, побудовані на безпосередній
примусовій праці.

Капітал спочатку підпорядковує собі працю за тих технічних
умов, що їх він знаходить як дані історичним розвитком. Тому він
не змінює безпосередньо способу продукції. Тим-то продукція
додаткової вартости в тій формі, яку ми розглядали досі, тобто
через просте здовження робочого дня, з’явилася незалежною від
якоїбудь зміни в самому способі продукції. Вона була в старомодній
пекарні не менш активна, ніж у сучасній бавовняній прядільні.

Коли ми розглядаємо процес продукції з погляду процесу
праці, то робітник протистоїть засобам продукції не як капіталові,
а як простим засобам і матеріялам своєї доцільної продуктивної
діяльности. У гарбарні, приміром, він обробляє шкуру як простий
предмет своєї праці. Той, що для нього він гарбує шкуру, не є
капіталіст. Але інакше стояла справа, коли ми розглядали процес
продукції з погляду процесу зростання вартости. Засоби продукції
зразу ж перетворювались на засоби вбирання чужої праці. Це
вже не робітник уживає засобів продукції, а засоби продукції
вживають робітника. Замість споживатися ним як речові елементи
його продуктивної діяльности, вони споживають його як
фермент їхнього власного життєвого процесу, а життєвий процес
капіталу є лише його рух як вартости, що сама з себе зростає.
Перетопні печі й робітні приміщення, що вночі відпочивають
і не вбирають у себе жодної живої праці, є «чиста втрата» («mere
loss») для капіталіста. Тим то перетопні печі й робітні приміщення
створюють «претенсію на нічну працю» робочих сил. Просте
перетворення грошей на предметні фактори процесу продукції,
на засоби продукції, перетворює останні в правний титул і примусовий
титул на чужу працю й на додаткову працю. Ще один
приклад покаже нам, як у свідомості капіталістичних голів одбивається
це властиве капіталістичній продукції й характеристичне
для неї переінакшення, навіть перекручення відношення між
мертвою й живою працею, між вартістю й вартостетворчою
силою. Під час бунту англійських фабрикантів 1848--1850 рр.
\parbreak{}  %% абзац продовжується на наступній сторінці
