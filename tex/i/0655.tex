З розвитком капіталістичної продукції протягом мануфактурного
періоду громадська думка Європи позбулася останніх решток
сорому й сумління. Нації цинічно пишались кожною підлотою,
що була засобом для акумуляції капіталу. Прочитайте, приміром,
наївні торговельні анали, складені щирим А. Андерсоном.

Тут, як тріумф англійської державної мудрости, розхвалюється
той факт, що Англія за Утрехтським миром на основі
угоди асієнто\footnote*{
— угода щодо торговлі рабами. Ред.
} вимусила від Еспанії привілей, що давав їй
право на торговлю неграми, яку вона досі вела лише між Африкою
й англійською Західньою Індією, вести й між Африкою та
еспанською Америкою. Англія здобула право аж до 1743 р. постачати
еспанській Америці щорічно 4.800 негрів. Це забезпечувало
їй разом з тим офіціяльне прикриття для контрабанди.
Ліверпул виріс як велике місто на ґрунті торговлі рабами.
Вона становить його методу первісної акумуляції. І ще й по сей
день «поважні» ліверпулські громадяни лишилися Піндарами
работорговлі, яка — порівн. вже цитований вище твір д-ра Ейкіна
з 1795 р. — «підносить дух комерційної підприємливости аж до
пристрасти, створює славних моряків і приносить колосальні
гроші». В 1730 р. в Ліверпулі коло торговлі рабами працювало
15 кораблів, в 1751р. — 53 кораблі, в 1760 р. — 74, в 1770 р. —
96 і в 1792 р. — 132 кораблі.

Бавовняна промисловість, завівши в Англії рабство дітей,
дала разом з тим поштовх до перетворення рабовласницького
господарства Сполучених штатів, доти більш або менш патріярхального,
на комерційну систему експлуатації. Взагалі приховане
рабство найманих робітників в Европі потребувало, як основи,
рабства sans phrase\footnote*{
— попросту. Ред.
} у Новому Світі.\footnote{
«В 1790 р. в англійській Західній Індії 10 рабів припадало на
одну вільну людину, у французькій — 14 на одну, в голляндській — 23
на одну». (Henry Barougham: «An Inquiry into the Colonial Policy of
the European Powers», Edinburgh 1803, vol. II, p. 74).
}

Tantae molis erat\footnote*{
Стільки праці коштувало. Ред.
} розв’язати «вічні природні закони»
капіталістичного способу продукції, вивершити процес відо-

lion Komitees\footnote*{
— комітет у справах зливків. Ред.
} та інтимний приятель Рікарда) заявив у палаті громад:
«Загальновідомий факт, що разом з цінними речами одного банкрута
призначено на продаж і продано з авкціона, як частину його майна, банду
фабричних дітей, коли можна вжити такого слова. Два роки тому (року
1813) перед King’s Bench\footnote*{
— найвищим судом. Ред.
} розглядувано огидний випадок. Ішлося про
групу хлопчиків. Одна лондонська парафія віддала їх якомусь фабрикантові,
що від себе знову передав їх якомусь іншому. Нарешті, декілька філантропів
знайшло їх у стані повного виснаження від голоду («absolute
famine»). З другим випадком, ще огиднішим, його познайомили, як члена
парламентської слідчої комісії. Декілька років тому одна лондонська
парафія склала контракт з одним ланкашірським фабрикантом, за яким
він зобов’язувався на двадцять здорових купованих ним дітей приймати
одного ідіота».