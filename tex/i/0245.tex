побачимо, що цей закон має силу лише щодо розглянутої досі
форми додаткової вартости.

З попереднього розгляду продукції додаткової вартости випливає,
що не кожну довільну суму грошей або вартости можна
перетворити на капітал; навпаки, передумовою такого перетворення
є певний мінімум грошей, або мінової вартости, в руках
поодинокого посідача грошей або товарів. Мінімум змінного капіталу
є ціна витрат (Kostenpreis)\footnote*{
У французькому виданні тут замість слова «Kostenpreis» маємо
«prix moyen», що значить: «пересічна ціна». \emph{Ред.}
} на одну робочу силу, яку протягом
цілого року день-у-день споживають, щоб здобути додаткову
вартість. Коли б у цього робітника були свої власні засоби
продукції та коли б він задовольнявся життям робітника,
то йому досить було б робочого часу, доконечного для репродукції
його засобів існування, приміром, 8 годин денно. Отже, і
засобів продукції він потребував би лише на 8 робочих годин.
Навпаки, капіталіст, який, крім цих 8 годин, примушує його
працювати ще, приміром, 4 години додаткової праці, потребує
додаткової грошової суми на придбання додаткових засобів продукції.
Однак за нашого припущення він мусів би вже вживати
двох робітників для того, щоб на додаткову вартість, яку він
щодня присвоює, жити як робітник, тобто мати змогу задовольняти
свої доконечні потреби. В цьому випадку метою його продукції
було б тільки підтримання життя, а не збільшення багатства,
а саме останнє й має на меті капіталістична продукція.
Щоб жити лише удвоє краще від звичайного робітника й половину
спродукованої додаткової вартости знову перетворювати
на капітал, він мусів би разом з числом робітників увосьмеро
збільшити мінімум авансованого капіталу. Певна річ, він сам,
подібно до того як його робітник, може безпосередньо прикладати
свої руки до процесу продукції, але тоді він є лише щось середнє
між капіталістом і робітником — «дрібний майстер». На певному
рівні розвитку капіталістичної продукції потрібно, щоб капіталіст
цілий час, протягом якого він функціонує як капіталіст,
тобто як персоніфікований капітал, міг уживати на присвоєння,
а тому й на контроль чужої праці, і на продаж продуктів цієї
праці.\footnote{
«Фармер не може покладатись на свою власну працю, а коли
він це робить, то, на мою думку, він од того втрачає. Його функція — це
доглядати за всім: він мусить стежити за молотником, а то плата за останнім
пропаде, а хліб не вимолотиться; він мусить доглядати за своїми
косарями, женцями й т. ін.: мусить завжди обходити своє господарство;
мусить додивлятися, щоб не було жодного недбальства, алеж воно неми-
} Середньовічні цехи намагалися силоміць перешкодити

людности, або 10 мільйонів годин... Капітал має свої межі зростання.
Кожного даного періоду цих меж можна досягти, застосовуючи ввесь
час, який є в розпорядженні для господарювання». («The labour, that
is the economic time of society, is a given portion, say ten hours a day of
a million of people or ten millions hours... Capital has its boundary of increase.
The boundary may, at any given period, be attained in the actual
extent of economic time employed»). («An Essay on the Political Economy
of Nations», London 1821, p. 47, 49).