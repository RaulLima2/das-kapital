стичного процесу продукції, або проста репродукція, зумовлює
ще інші своєрідні зміни, що стосуються не тільки до змінної
частини капіталу, а й до всього капіталу в цілому.

Якщо додаткова вартість, створювана періодично, наприклад,
щороку, капіталом у 1.000 фунтів стерлінґів, становить 200 фунтів
стерлінґів, і якщо цю додаткову вартість щороку споживається,
то ясно, що після п’ятирічного повторювання того самого
процесу сума спожитої додаткової вартости дорівнює 200 × 5, або
дорівнює первісно авансованій капітальній вартості в 1.000 фунтів
стерлінґів. Коли б річну додаткову вартість споживано лише
частинно, наприклад, лише наполовину, то той самий результат
ми мали б після десятирічного повторювання продукційного
процесу, бо 100 × 10 = 1.000. Взагалі кажучи, авансована
капітальна вартість, поділена на щорічно споживану додаткову
вартість, дає число років або число періодів репродукції, після
скінчення яких первісно авансований капітал споживається
капіталістом і тому зникає. Уявлення капіталіста, що він споживає
продукт чужої неоплаченої праці, додаткову вартість,
та зберігає первісну капітальну вартість, абсолютно нічого не
може змінити в цьому факті. Коли мине якесь певне число років,
присвоєна ним капітальна вартість дорівнює сумі додаткової
вартости, присвоєної ним без еквіваленту протягом того самого
числа років, а спожита ним сума вартости дорівнює первісній
капітальній вартості. Правда, він зберігає у своїх руках капітал,
що його величина не змінилася — капітал, що з нього частина,
будівлі, машини й т. ін., існувала вже тоді, коли він пустив у рух
своє підприємство. Але тут ідеться про вартість капіталу, а не
про його матеріяльні складові частини. Коли хтось споживе
все своє майно, поробивши таку кількість боргів, що дорівнюють
вартості цього майна, то якраз усе це майно й репрезентує лише
загальну суму його боргів. І так само, коли капіталіст спожив
еквівалент свого авансованого капіталу, то вартість цього капіталу
репрезентує лише загальну суму присвоєної ним задурно
додаткової вартости. Жодного атома вартости його старого капіталу
вже далі не існує.

Отже, цілком незалежно від усякої акумуляції, проста безперервність
процесу продукції, або проста репродукція, після
коротшого або довшого періоду неминуче перетворює кожний
капітал у нагромаджений капітал, або в капіталізовану додаткову
вартість. Навіть якщо при своєму вступі в продукційний процес
капітал був власністю підприємця, особисто ним заробленою,
все одно, раніш, або пізніш, він стає присвоєною без еквіваленту
вартістю, або матеріялізацією, в грошовій чи іншій формі, неоплаченої
чужої праці.

Як ми бачили в четвертому розділі, для того, щоб перетворити
гроші на капітал, недосить наявности продукції вартости й товарової
циркуляції. Для цього мусили насамперед протистояти
один одному як покупець і продавець на одному боці посідач
вартости або грошей, на другому — посідач вартостетворчої
