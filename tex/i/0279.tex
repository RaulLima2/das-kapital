продукції, аж поки набере своєї остаточної форми. Навпаки, коли
розглядатимемо майстерню як сукупний механізм, то виявиться,
що сировинний матеріял перебуває одночасно в усіх своїх фазах
продукції. Однією частиною своїх багатьох озброєних знаряддям
рук колективний робітник, скомбінований із частинних робітників,
тягне дріт, тоді як іншими руками та знаряддям він одночасно
вирівнює його, іншими ріже його, загострює й т. д. Послідовність
різних стадій процесу в часі перетворюється на одночасність
існування цих стадій у просторі. Звідси виготовлення більшої
кількости товару протягом того самого часу.36 Ця одночасність
виникає, щоправда, з загальної кооперативної форми цілого
процесу, але мануфактура не тільки находить уже готові умови
кооперації, вона почасти лише сама створює їх, розчленовуючи
ремісничу працю. З другого боку, вона досягає цієї суспільної
організації процесу праці, лише міцно приковуючи того самого
робітника до того самого деталю.

Що частинний продукт кожного частинного робітника разом
з тим є лише осібний ступінь у розвитку того самого продукту,
то один робітник постачає іншому, або одна група робітників
іншій сировинний матеріял. Результат праці одного становить
вихідний пункт для праці іншого. Отже, один робітник тут безпосередньо
дає працю іншим. Робочий час, доконечний для
досягнення у кожному частинному процесі корисного ефекту,
що його має на меті цей процес, установлюється з досвіду, і цілий
механізм мануфактури ґрунтується на тій передумові, що протягом
даного робочого часу досягається якогось даного результату.
Лише за цієї передумови різні процеси праці, процеси, що
один одного доповнюють, можуть відбуватися безперервно,
одночасно та просторово один поруч одного. Ясно, що ця безпосередня
взаємна залежність праць, а тому й робітників примушує
кожного окремого робітника витрачати на свою функцію лише
доконечний робочий час, і таким чином досягається цілком іншої
безперервности, одноманітности, правильности, порядку, 37 а осо-

36 «Він (поділ праці) зумовлює економію часу, розчленовуючи
працю на різні операції, які можна виконувати всі одночасно... В наслідок
одночасного виконування всіх тих різних процесів праці, які одна
людина мусить провадити послідовно, один по одному, утворюється,
наприклад, можливість виробити велике число цілком закінчених
шпильок протягом того самого часу, якого треба на те, щоб обрізати й
загострити одну шпильку». («It (the division of labour) produces also
an economy of time, by separating the work into its different branches, all
of which may be carried on into execution at the same moment... By carrying
on all the different processes at once, which an individual must have
executed separately, it becomes possible to produce a multitude of pins
for instance completely finished in the same time as a single pin might
have been either cut or pointed»). (Dugald Stewart: Works, edited
by Sir W. Hamilton, Edinburgh 1855, vol. Ill, «Lectures on Political
Economy», p. 319).

37 «Що більше різноманітности серед робітників мануфактури... то
більші порядок і реґулярність у кожній роботі, то менше мусить витрачатись
на неї часу, то менше мусить витрачатись праці» («The more variety
