Ми відзначаємо тут лише матеріяльні умови, за яких виконується
фабрична праця. Всі чуттьові органи однаково страждають
від штучно підвищеної температури, від повітря, заповненого
відпадками сировинного матеріялу, від оглушливого шуму
й т. д., не кажучи вже про небезпеку для життя серед густо
поставлених машин, які з регулярністю пір року подають свої

робітники засуджені від дев’ятого року життя аж до самої смерти жити
під моральною та фізичною палицею». (F. Engels: «Die Lage der
arbeitenden Klasse in England», Leipzig 1845. S. 217 ff. — Ф. Енгельс:
«Становище робітничої кляси в Англії», Партвидав, 1932 р., стор. 187 і
далі). Що «кажуть суди», це я поясню на двох прикладах. Один випадок
трапився в Шеффілді наприкінці 1866 р. Там один робітник найнявся
на два роки на металеву фабрику. Посварившися з фабрикантом,
він залишив фабрику й заявив, що ні в якому разі не працюватиме
більше на нього. Його оскаржили за зламання контракту й засудили
на два місяці в’язниці. (Коли фабрикант ламає контракт, то його можна
оскаржити лише перед цивільним судом і ризикує він лише грошовою
карою). Після того, як він одсидів ці два місяці, той самий фабрикант
закликає його на основі старого контракту повернутися до фабрики.
Робітник відмовляється. Він, мовляв, одбув уже кару за зламання контракту.
Фабрикант позиває його знову, суд знову засуджує його, дарма
що один із суддів, містер Ші, прилюдно назвав правничою потворністю,
що людину ціле життя можна періодично знову й знову карати за ту саму
провину або злочин. А цей присуд виніс не «Great Unpaid» (сільські
мирові судді), провінціяльні Dogberries, а один із найвищих судів у
Лондоні. [До четвертого видання. Тепер це скасовано. Тепер в Англії,
за винятком деяких випадків, наприклад, на громадських газівнях, робітник
за зламання контракту відповідає нарівні з підприємцем, і його
можна оскаржити лише перед цивільним судом. — Ф. Е.]. — Другий
випадок був у Wiltshire наприкінці листопада 1863 р. Якихось З0 робітниць
при парових ткацьких варстатах, працюючи в якогось Геррупа,
фабриканта сукна в Leoner’s Mill, Westbury, Leigh, улаштували страйк,

бо цей самий Герруп мав приємну звичку стягати з заробітної плати за
спізнення вранці, а саме 6 пенсів за 2 хвилини, 1 шилінґ за 3 хвилини й
1 шилінґ 6 пенсів за 10 хвилин. При 9 шилінґах за годину це становить
4 фунти стерлінґів 10 шилінґів на день, тимчасом як їхня пересічна річна
плата ніколи не перевищує 10—12 шилінґів на тиждень. Герруп доручив
також одному підліткові повідомляти трубою про фабричні години,
а цей іноді робив це перед шостою годиною вранці; якщо ж руки не з’являлися
саме тоді, коли він кінчав, то брама замикалась, а на тих, що залишалися
за брамою, накладали грошову кару; а що на фабриці не було
годинника, то нещасні руки попадали під владу молодого вартового,
інспірованого від Геррупа. Руки, що почали «страйк», матері родин
та дівчата, заявили, що знову стануть до праці, якщо вартового замінять
годинником та заведуть раціональніший карний тариф. Герруп
оскаржив 19 жінок та дівчат перед судом за зламання контракту. Серед
голосного обурення авдиторії кожну з них засудили до 6 пенсів грошової
кари та до 2 шилінґів 6 пенсів судових витрат. Народна маса провела
Геррупа з суду з шиканням. Одна з улюблених операцій фабрикантів
— це карати робітників відрахуванням із заробітної плати за
кепську якість постачуваного їм матеріялу. Ця метода викликала в 1866 р.
загальний страйк в англійських ганчарняних округах. Звіти «Children’s
Employment Commission» (1863—1866) наводять випадки, коли робітник,
замість одержувати плату, ставав через свою працю та за допомогою
карного регламенту ше й винуватцем своїх ясновельможних «хазяїнів».
Повчальні риси бистродумности фабричних автократів у справі відрахувань
з заробітної плати дала також найновіша бавовняна криза.
