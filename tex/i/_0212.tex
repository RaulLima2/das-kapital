\parcont{}  %% абзац починається на попередній сторінці
\index{i}{0212}  %% посилання на сторінку оригінального видання
письменник-торі, «труднощі найти робітника за справедливі
ціни (тобто ціни, які лишали їхнім хазяїнам справедливу кількість
додаткової праці) справді стали нестерпні»\footnote{
«Sophisms of Free Trade. 7 th edit. London. 1850», p. 205. Той
самий торі зрештою додає: «Парляментські акти, які реґулювали заробітну
плату на шкоду робітникам і на користь тим, що вживали праці,
зберігались протягом довгого періоду в 464 роки. Людність зростала.
Ці закони стали тепер зайвими та обтяжливими». (Там же, стор. 206).
}. Тому закон і
почав диктувати справедливу заробітну плату й межі робочого
дня. Останній пункт, єдиний, що лише нас тут і цікавить, повторено
в статуті 1496~\abbr{р.} (за Генріха VII). Робочий день для всіх
ремісників (artificers) і рільничих робітників од березня до вересня
повинен був тоді тривати, чого однак ніколи не здійснено,
від п’ятої години ранку до сьомої-восьмої години вечора,
але час, призначений на їжу, становив: одну годину на сніданок,
1\sfrac{1}{2} години на обід і півгодини на підвечірок, отже, саме
удвоє більше того, що призначає теперішній фабричний закон\footnote{
Дж. Вед слушно зауважує з приводу цього статуту: «Із статуту
1496~\abbr{р.} виходить, що харч вважали за еквівалент \sfrac{1}{3} доходів ремісників
і \sfrac{2}{3} доходів рільничого робітника, а це свідчить про вищий ступінь незалежности
серед робітників, ніж панує тепер, коли харч рільничих і мануфактурних
робітників становить куди більшу частину їхньої заробітної
плати». (\emph{J. Wade}: «History of the Middle and Working Classes», 3 rd
ed». London 1835, p. 24, 25 і 577). Щодо погляду, нібито ця ріжниця
походить із ріжниці між ціною харчів і одягу тепер і тоді, то, щоб збити
його, досить лише кинуте оком на «Chronicon Pretiosum etc.», By Bishop
Fleetwood. 1 st. ed. London 1707. 2 nd ed. London. 1745.
}. Взимку слід було працювати з тими самими перервами від п’ятої
години ранку до присмерку. Статут Єлизавети з року 1562 для
всіх робітників, що їх «наймано за поденну або тижневу плату»,
не змінює довжини робочого дня, але намагається обмежити час
перерви 2\sfrac{1}{2} годинами влітку й 2 годинами взимку. Обід повинен
тривати лише одну годину, а «півгодинний післяобідній сон» слід
дозволяти лише на час від половини травня до половини серпня.
За кожну годину відсутности на роботі слід утримувати із заробітної
плати 1 пенні (близько 4 копійки). Однак у практиці обставини
були далеко сприятливіші для робітників, ніж у статутах.
Вільям Петті, батько політичної економії й до певної міри винахідник
статистики, в одному творі, опублікованому в останню
третину XVII віку, каже: «Робітники (labouring men, тоді власне
рільничі робітники) працюють по 10 годин на добу і їдять двадцять
разів на тиждень, а саме тричі на день у будні й двічі в неділю;
звідси ясно видно, що коли б вони захотіли попостити в
п’ятницю ввечері і тратити на обід півтори години, тоді як тепер
вони тратять на нього дві години, від 11 до 1 години вранці, отже
коли б вони працювали на \sfrac{1}{20} часу більше й на \sfrac{1}{20} менше споживали,
то це покрило б \sfrac{1}{10} частину згаданого вище податку»\footnote{
\emph{W. Petty}: «Political Anatomy of Ireland, Verbum Sapienti, 1672»,
ed. 1691, p. 10.
}.
Хіба д-р Ендр’ю Юре не мав рації плямувати дванадцятигодинний
біл 1833~\abbr{р.} як вороття до минулих часів темряви? Щоправда,
\parbreak{}  %% абзац продовжується на наступній сторінці
