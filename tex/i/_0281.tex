\parcont{}  %% абзац починається на попередній сторінці
\index{i}{0281}  %% посилання на сторінку оригінального видання
до іншої і~\abbr{т. д.} Отже, усамостійнення цих функцій, або доручення
цих функцій окремим робітникам, стає вигідним лише із збільшенням
числа занятих робітників, але це збільшення мусить
одразу охопити всі групи в однаковій пропорції.

Окрема група, певне число робітників, що виконують ту саму
частинну функцію, складається з однорідних елементів та становить
осібний орган цілого механізму. Однак у різних мануфактурах
сама група є розчленоване робоче тіло, тимчасом як цілий
механізм утворюється через повторювання або помножування
цих продуктивних елементарних організмів. Візьмімо, приміром,
мануфактуру пляшок. Вона розпадається на три посутньо відмінні
фази. Перша, підготовча фаза: виготовлення скляної маси,
тобто суміші піску, вапна й~\abbr{т. ін.}, та перетоплення цієї суміші
на плинну скляну масу\footnote{
В Англії піч для перетоплювання відокремлена від печі для виробу
скляних продуктів, але в Бельгії, наприклад, та сама піч служить
для обох процесів.
}. В цій першій фазі працюють різні
частинні робітники, так само як і в кінцевій фазі: вийманні
пляшок із сушарні, сортуванні та пакуванні їх і~\abbr{т. ін.} Посередині
між цими двома фазами маємо власне виробництво пляшок, або
перероблення плинної скляної маси. Коло того самого отвору
печі для виробу скляних продуктів працює група, яка в Англії
зветься «hole» (діра) і складається з одного bottle maker’a\footnote*{
— пляшкаря. \emph{Ред.}
},
або finisher’a\footnote*{
— робітника, що остаточно закінчує продукт. \emph{Ред.}
}, одного blower’a\footnote*{
— надимача. \emph{Ред.}
}, одного gatherer’a\footnote*{
— збирача. \emph{Ред.}
},одного
putter up\footnote*{
— укладача. \emph{Ред.}
} або whetter off\footnote*{
— шліфувальника. \emph{Ред.}
} і одного taker’а\footnote*{
— приймача. \emph{Ред.}
}.
Ці п’ятеро частинних робітників становлять стільки ж окремих
органів одним-одного робочого тіла, яке може функціонувати лише
як єдність, отже, лише через безпосередню кооперацію п’ятьох.
Якщо бракує однієї з п’ятьох складових частин цього тіла, то
воно паралізується. Але та сама піч для перетоплення скла має декілька
отворів, — в Англії, наприклад, 4--6, — і в кожному з них
міститься глиняний топильний тигель із плинним склом; коло кожного
з них працює своя власна робоча група, складена з таких самих
п’ятьох робітників. Організація кожної поодинокої групи базується
тут безпосередньо на поділі праці, тимчасом як зв’язок поміж
різними однорідними групами полягає в простій кооперації,
що дає можливість через спільне вживання ощадніше використовувати
один із засобів продукції, в даному випадку, піч для перетоплення
скла. Кожна така піч з її 4--6 групами становить гуту,
а скляна мануфактура охоплює кілька таких гут разом із приладдям
і робітниками для початкової й кінцевої фази продукції.

Нарешті, мануфактура, так само, як вона виникає почасти
з комбінації різних реместв, може розвинутися на комбінацію
\parbreak{}  %% абзац продовжується на наступній сторінці
