Він мусить зробити переворот у технічних і суспільних умовах
процесу праці, отже, в самому способі продукції, щоб підвищити
продуктивну силу праці, через підвищення продуктивної сили
праці знизити вартість робочої сили й таким чином скоротити
частину робочого дня, доконечну для репродукції цієї вартости.

Додаткову вартість, продуковану через здовження робочого
дня, я називаю абсолютною додатковою вартістю; навпаки, ту
додаткову вартість, що виникає із скорочення доконечного робочого
часу й відповідної зміни відносних величин обох складових
частин робочого дня — відносною додатковою вартістю.

Для того, щоб знизити вартість робочої сили, підвищення
продуктивної сили праці мусить охопити ті галузі промисловости,
продукти яких визначають вартість робочої сили, отже, або
належать до кола звичайних засобів існування, або можуть їх
заміняти. Але вартість товару визначається не тільки кількістю
праці, яка надає йому викінченої форми, а так само й кількістю
праці, що міститься в засобах його продукції. Приміром, вартість
чобота визначається не тільки працею шевця, а ще й вартістю
шкури, смоли, дратви й т. ін. Отже, підвищення продуктивної
сили праці й відповідне подешевшання товарів у тих галузях
промисловости, які постачають речові елементи сталого капіталу, —
засоби праці й матеріял праці — для виготовлення доконечних
засобів існування, також знижує вартість робочої сили. Навпаки,
в тих галузях продукції, що не постачають ні доконечних засобів
існування, ані засобів продукції до їх виготовлення, підвищення
продуктивної сили праці лишає вартість робочої сили незмінною.

Подешевшання товару знижує, природно, вартість робочої
сили лише pro tanto, тобто лише в тій пропорції, в якій цей товар
увіходить у репродукцію робочої сили. Сорочки, приміром, є
доконечний засіб існування, але тільки один із багатьох. Їхнє
подешевшання зменшує видатки робітника лише на сорочки.
Однак загальна сума доконечних засобів існування складається
лише з різних товарів, із самих продуктів окремих галузей
промисловости, а вартість кожного такого товару становить завжди
якусь відповідну частину вартости робочої сили. Ця вартість
зменшується разом із доконечним для її репродукції робочим
часом, що його загальне скорочення дорівнює сумі його скорочень
по всіх тих окремих галузях продукції. Цей загальний результат
ми розглядаємо тут так, наче б він був безпосереднім результатом
і безпосередньою метою в кожному поодинокому випадку. Коли
поодинокий капіталіст через підвищення продуктивної сили праці
здешевлює, приміром, сорочки, то він цим ніяк не має неодмінно за
мету знизити pro tanto вартість робочої сили, а тому й доконечний
робочий час, а лише оскільки він, кінець-кінцем, допомагає
цьому результатові, він допомагає піднесенню загальної норми
додаткової вартости.\footnote{
«Коли фабрикант, поліпшуючи машини, подвоює кількість своїх
продуктів... він виграє (кінець-кінцем) лише остільки, оскільки він у
} Загальні й доконечні тенденції капіталу
треба відрізняти від форм їхнього виявлення.