40 фунтів кави = 20 метрам полотна; отже, 10 фунтів чаю =
40 фунтам кави, або в 1 фунті кави міститься лише чверть тієї
кількости субстанції вартости, праці, що міститься в 1 фунті чаю.

Загальна відносна форма вартости товарового світу надає
(drückt... auf) виключеному в цього світу еквівалентному товарові,
полотну, характер загального еквіваленту. Власна натуральна
форма полотна є тепер спільна форма вартости товарового
світу, а тому воно тепер є безпосередньо вимінне на всі інші товари.
Його тілесна форма фігурує як видиме втілення, як загальна
суспільна лялечка (Verpuppung) всякої людської праці. Ткацтво,
приватна праця, що продукує полотно, перебуває разом з тим у
загальній суспільній формі, у формі рівности з усіма іншими
працями. Незчисленні рівнання, що з них складається загальна
форма вартости, по черзі прирівнюють працю, здійснену в полотні,
до кожної праці, що міститься в інших товарах, і через те роблять
ткацтво загальною формою виявлення людської праці взагалі.
Таким чином праця, упредметнена в товаровій вартості, виявляється
не лише неґативно, як праця, абстрагована від усіх конкретних
форм і корисних властивостей дійсних праць, але й її
власна позитивна природа виразно виступає наперед. Вона —
зведення всіх дійсних праць до спільного їм характеру людської
праці, до витрати людської робочої сили.

Загальна форма вартости, яка виражає продукти праці просто
як згустки безріжницевої людської праці, свідчить своєю власною
будовою, що вона є суспільний вираз товарового світу. Таким
чином вона виявляє, що в межах цього світу загальнолюдський
характер праці становить її специфічний суспільний характер.

2. Відношення між розвитком відносної форми вартостн й розвитком
еквівалентної форми

Ступеневі розвитку відносної форми вартости відповідає ступінь
розвитку еквівалентної форми. Однак — і це слід добре запам’ятати
— розвиток еквівалентної форми є лише вираз і результат
розвитку відносної форми вартости.

Проста або одинична відносна форма вартости якогось товару
робить інший товар одиничним еквівалентом. Розгорнута форма
відносної вартости, цей вираз вартости якогось товару в усіх
інших товарах, надає їм (prägt ihnen... auf) форми різнорідних
осібних еквівалентів. Нарешті, даний осібний рід товару набирає
загальної еквівалентної форми тому, що всі інші товари роблять
його матеріялом своєї однорідної загальної форми вартости.

Але такою самою мірою, якою розвивається форма вартости
взагалі, розвивається і протилежність між її обома полюсами —
відносною формою вартости й еквівалентною формою.

Уже перша форма — 20 метрів полотна = 1 сурдутові —
містить у собі цю протилежність, але не фіксує її. Відповідно до
того, як читати це рівнання — з лівого боку на правий чи навпаки,
кожен з обох товарових полюсів, і полотно і сурдут, рів-
