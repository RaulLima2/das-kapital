Взагалі, досвід показує капіталістові, що існує постійне
перелюднення, тобто перелюднення проти наявної в даний момент
потреби капіталу самозростати своєю вартістю, хоч це перелюднення
і утворюється з людських зниділих поколінь, — поколінь,
що швидко вимирають і витискують одне одного, поколінь, так
би мовити, позриваних у недостиглому стані.111 Щоправда,
тямущому спостерігачеві досвід показує також, як швидко
й глибоко капіталістична продукція, що, висловлюючись історично,
виникла лише вчорашнього дня, підкопала вже до
самого коріння життя народню силу, як виродження промислової
людности уповільнюється тільки тим, що постійно вбирається
природно-вирослі життєві елементи села, і як навіть сільські
робітники починають уже вимирати, не вважаючи на свіже повітря
й таке могутнє панування серед них principle of natural selection,*
який дозволяє виростати лише найсильнішим індивідам.112
Капітал, що має такі «достатні причини» заперечувати страждання
покоління робітників, яке оточує його, у своєму практичному
русі так само мало керується перспективою майбутнього загнивання
людства, отже, кінець-кінцем, неминучого знелюднення, —
як тією перспективою, що земля може впасти на сонце. За кожної
спекуляції акціями всякий знає, що колись та мусить ударити
грім, але всякий сподівається, що він спаде на голову його ближ-

бавовняний робітник стоїть з кожного погляду вище, ніж його товаришу
недолі на континенті. «Пруський фабричний робітник працює щонайменше
10 годин на тиждень більше, ніж його англійський суперник, а коли він
працює в себе вдома на власному ткацькому варстаті, то відпадає навіть
і ця межа його додаткових робочих годин». («Reports of Insp. of Fact.
31 st October 1855», p. 103). Згаданий вище фабричний інспектор Редґрев
після промислової виставки 1851 р. поїхав на континент, спеціально до
Франції й Прусії, щоб вивчити порядки на фабриках у цих країнах.
Ось що каже він про пруського фабричного робітника: «Він дістає заробітну
плату, якої вистачає лише на ті прості харчі й маленький комфорт, до
якого він звик і яким задовольняється... Він живе гірше й працює важче,
ніж його англійський суперник». («Reports of Insp. of Fact. 31st October
1853», p. 85).

111 «Люди, що надмірно працюють, вмирають навдивовижу швидко,
але місця тих, що гинуть, зараз же поповнюються знов, і часта зміна
осіб не зумовлює жодної зміни на сцені». «England and America»,
London 1833, vol. I, p. 55. (Автор: E. G. Wakefield).

112    Див. «Public Health. Sixth Report of the Medical officer of the
Privy Council. 1863». Опубліковано в Лондоні 1864 p. В цьому звіті
мова йде саме про рільничих робітників. «Графство Sutherland змальовували
таким, наче б у ньому пороблено значні поліпшення, але недавні
розсліди показали, що в округах нього графства, яке колись так славилося
красою чоловіків і сміливістю солдатів, людність виродилася в
хиряву й зниділу расу. В найздоровіших місцевостях, на приморських
узбіччях гір обличчя дітей такі худі й бліді, як вони можуть бути
тільки в гнилій атмосфері якогось лондонського заулка». (Thornton:
«Overpopulation and its Remedy», London 1846, p. 74, 75). Вони дійсно
скидаються на тих 30.000 «gallant Highlanders», ** що їх Ґлезґо збирає
до купи з проститутками й злодіями по своїх wynds і closes.***

* — принципу природного добору. Ред.

** — бравих горян. Ред.

*** — заулках і вертепах. Ред.
