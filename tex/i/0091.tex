ляції перетворився на скарб, бо процес циркуляції перервався на першій фазі, тобто тому, що
перетворену форму товару витягнуто з циркуляції. Засіб платежу вступає в циркуляцію, але лише після
того, як товар уже вийшов із неї. Гроші вже не упосереднюють процес. Вони самостійно замикають його,
як абсолютне буття мінової вартости, або загальний товар. Продавець перетворив товар на гроші, щоб
грішми задовольнити якусь потребу, збирач скарбів — щоб зберегти товар у грошовій
формі, винуватець-покупець — щоб мати змогу заплатити. Коли він не заплатить, то відбудеться
примусовий продаж його майна. Отже, в наслідок суспільної конечности, що випливає з відносин самого
процесу циркуляції, форма вартости товару, гроші, стають тепер самостійною метою продажу.

Покупець зворотно перетворює гроші на товар раніш, ніж він перетворив товар на гроші, або провадить
другу метаморфозу товару поперед першої. Товар продавця циркулює, але реалізує свою ціну лише в
приватноправному титулі на одержання
грошей. Він перетворюється на споживну вартість раніш, ніж перетворився на гроші. Здійснення його
першої метаморфози настає тільки пізніш. *98

Протягом кожного певного відтинку часу процесу циркуляції платіжні зобов’язання, що їм настав термін
платежу, репрезентують суму цін тих товарів, що їх продаж викликав ці зобов’язання. Маса грошей,
потрібна для реалізації цієї суми цін,
залежить насамперед від швидкости обігу засобів платежу.

98 Примітка до другого видання. З дальшої цитати, запозиченої з моєї праці, що з’явилася 1859 р.,
видно, чому я в тексті не звертаю жодної уваги на протилежну форму: «Навпаки, у процесі Г — Т гроші
можуть бути відчужені як дійсний засіб купівлі, і таким чином ціну товару реалізується раніш, ніж
реалізується споживну вартість грошей або відчужується товар. Це відбувається, приміром, у звичайній
формі передплати. Або у формі, якої вживає англійський уряд, купуючи опій в індійських райотів...
Однак у цих випадках гроші функціонують лише в знайомій
уже нам формі засобу купівлі. Капітал звичайно теж авансується у формі грошей... Але цей погляд
постає лише за горизонтом простої циркуляції» («Zur Kritik der Politischen Oekonomie», Berlin 1859,
, S. 119, 120. — «До критики політичної економії», ДВУ, 1926 р., стор. 151).

*  У французькому виданні цей абзац зредаговано так: «Припустімо, що селянин купує у ткача 20 метрів
полотна ціною в 2 фунти стерлінґів, які становлять також ціну одного квартера пшениці, і сплачує ці
2 фунти стерлінґів лише через місяць. Селянин перетворює свою пшеницю на полотно раніш, ніж він
перетворив її на гроші. Отже, він провадить останню метаморфозу свого товару раніше за першу. Потім
він продає
пшеницю за 2 фунти стерлінґів, які він передає ткачеві в умовлений термін. і еальні гроші для нього
тут уже не служать більше за посередника для заміщення пшениці полотном. Це вже зроблено. Навпаки,
для нього гроші є останній акт оборудки, оскільки гроші є абсолютна форма вартости, яку він мусить
віддати, загальний товар. ІЦождо ткача, то його товар циркулював і зреалізував свою ціну, але лише
за посередництвом титулу, що ґрунтується на приватному праві. Його товар увійшов у сферу споживання
інших осіб раніш, ніж він перетворився на гроші. Отже, першу метаморфозу його полотна відкладено,
вона здійснюється лише пізніш, коли настає платіжний термін для селянина». («Le Capital etc.», vol.
I, ch. Ill, p. 56). Ред.
