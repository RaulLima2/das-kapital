\index{i}{0261}  %% посилання на сторінку оригінального видання
Економію на засобах продукції треба взагалі розглядати з
подвійного погляду. По-перше, оскільки вона здешевлює товари
й тим знижує вартість робочої сили. По-друге, оскільки вона змінює
відношення додаткової вартості до цілого авансованого капіталу,
тобто до суми вартостей його сталих та змінних складових
частин. Цей останній пункт ми розглядаємо лише у першому
відділі третьої книги цього твору, куди задля зв’язку ми відносимо
й дещо інше, що можна б уже тут розглянути. Хід аналізу
вимагає саме так розбити тему, і це, зрештою, відповідає духові
капіталістичної продукції. Саме тому, що тут умови праці протистоять
робітникові як самостійні, то й економія на них виступає
як осібна операція, яка аніскільки не обходить робітника і
тому відокремлена від методів, що підвищують його особисту продуктивність.

Та форма праці, коли багато осіб планомірно й спільно, один
поруч одного, працюють у тому самому процесі продукції або в різних,
але зв’язаних між собою процесах продукції, називається
кооперацією\footnote{
«Concours de forces» («сполучення сил»). (\emph{Destutt de Tracy}: «Traité
de la Volonté et de ses effets», Paris 1826, p. 78).
}.

Подібно до того, як сила нападу ескадрону кавалерії або сила
опору полку піхоти посутньо відмінні від суми сил нападу й
опору кожного поодинокого кавалериста й піхотинця, так і механічна
сума сил поодиноких робітників відмінна від тієї суспільної
сили, яка розвивається, коли багато рук одночасно спільно працює
над тією самою неподільною операцією, приміром, коли треба
підняти тягар, покрутити корбою, забрати із шляху якусь перешкоду\footnote{
«Є безліч таких простих операцій, що їх не можна поділити на
частки, і все ж не можна виконати їх без кооперації багатьох рук. Так,
наприклад, підняти велику колоду на віз\dots{} коротко, всяка праця, що
її не можна виконати без співробітництва багатьох рук, які одночасно
помагають одна одній у тому самому неподільному процесі праці». («There
are numerous operations of so simple a kind as not to admit a division into
parts, which cannot be performed without the cooperation of many pairs
of hands. For instance the lifting og a large tree on a wain\dots{} every thing
in short, which cannot be done unless a great many pairs ef hands help each
other in the same undivided employment, and at the same time»). (\emph{E.~G.~Wakefield}: «А View of the Art of Colonization», London 1849, p. 168).
}.
За таких обставин цього результату комбінованої
праці або зовсім не можна було б досягти поодинокими силами,
або, якщо й можна було б, то тільки протягом довшого часу або
лише в карликовому маштабі. Тут справа не тільки в підвищенні
індивідуальної продуктивної сили через кооперацію, але й у
створенні продуктивної сили, яка сама по собі мусить бути масовою
силою\footnote*{У французькому виданні це речення подано так: «Справа не тільки
в підвищенні індивідуальних продуктивних сил, але й у створенні за допомогою
кооперації нової сили, яка функціонує лише як колективна
сила» \emph{Ред.}}\footnoteA{
«Якщо одна людина зовсім не може, а десятеро людей можуть
тільки з найбільшою напругою всіх своїх сил підняти тягар вагою в тонну,
то сто людей посягнуть цього, працюючи кожен лише одним пальцем» («As
one man cannot, and 10 men must strain, to lift a tun of weight, yet
one hundred men can do it only by the strength of a finger of each of them»).
(\emph{John Bellers}: «Proposals for raising a colledge of industry», London
1696, p. 21).
}.
\index{i}{0262}  %% посилання на сторінку оригінального видання

Крім цієї нової сили, яка постає із злиття багатьох сил в одну
колективну силу, вже самий суспільний контакт при більшості
продуктивних праць викликає змагання та своєрідне зворушення
життьового духа (animal spirits), яке збільшує індивідуальну
дієздатність поодиноких осіб, так що дванадцятеро осіб
разом протягом того самого робочого дня в 144 години дадуть
далеко більший сукупний продукт, ніж дванадцять поодиноких
робітників, що з них кожен працюватиме 12 годин, або ніж один
робітник, який працюватиме день за днем 12 днів\footnote{
«Отже, в цьому (коли один фармер уживає на 300 акрах те саме
число робітників, яке 10 дрібних фармерів уживають кожен на 30 акрах),
тобто в такій пропорції робітників, є й така користь, яку не легко зрозуміти
людям, незнайомим із справою на практиці: справді, хто буде
заперечувати, що 1 відноситься до 4, як 3 відноситься до 12; однак на
практиці це не так: підчас жнив і інших спішних робіт справа йде ліпше
й успішніше, коли сполучити значне число рук разом; так, наприклад,
2 возії, 2 навантажники, 2 подавальники, 2 загрібальники й декілька
людей на скиртах або на току зроблять удвоє більше, ніж те саме число
робочих рук, поділених на різні групи по поодиноких фармах». («There
is also an advantage in the proportion of servants, which will not easily be
understood but by practical men; for it is natural to say, as 1 is to 4, so
are 3: 12; but this will not hold good in practice; for in harvest-time
and many other operations which require that kind of despatch, by throwing
many hands together, the work is better, and more expeditiously done:
for exemple, in harvest, 2 drivers, 2 loaders, 2 pitchers, 2 rakers, and the
rest at the rick, or in the barn, will despatch double the work, that the same
number of hands would do, if divided into different gangs, on different
farms»): («An Enquiry into the Connection between the present price of
provisions and the size of farms. By a Farmer», London 1773, p. 7, 8).
}. Це випливає
з того, що людина з природи є якщо й не політична\footnote{
Арістотелеве визначення, власне кажучи, говорить, що людина з природи є міський громадянин.
Для клясичної старовини це так само
характеристичне, як і визначення Франкліна, що людина з природи є
творець знарядь, характеристичне для доби янкі.
}, як думає
Арістотель, то в усякому разі громадська тварина.

Хоч багато осіб одночасно й спільно виконують таку саму або
однорідну працю, все ж індивідуальна праця кожної особи, як
частина колективної праці, може репрезентувати різні фази
самого процесу праці, що через них предмет праці, в наслідок
кооперації, перебігає швидше. Приміром, коли мулярі складають
ряд рук, щоб подавати цеглу від основи риштовання до його
верху, то кожен з них робить те саме, а все ж поодинокі операції
становлять безперервні частини однієї спільної операції, окремі
фази, які кожна цеглина мусить перебігти в процесі праці, і
завдяки чому 24 руки колективного робітника подають цеглу
швидше, ніж дві руки поодинокого робітника, що сходить на
риштовання та спускається з нього\footnote{
«Треба ще зауважити, що такий частинний поділ праці може бути
навіть тоді, коли робітники працюють коло тієї самої справи. Наприклад,
мулярі, які подають із рук до рук цеглу на високе риштовання, виконують
усі ту саму роботу, а все ж між ними є якийсь рід поділу праці, який полягає
в тому, що кожний з них переносить цеглу на певну віддаль, і що всі
вони приставляють її на місце призначення далеко швидше, ніж це було б,
якби кожний з них сам носив свою цеглу на те високе риштовання»
(«On doit encore remarquer que cette division partielle de travail peut se
faire quand même les ouvriers sont occupés d’une même besogne. Des maçons,
par exemple, occupés de faire passer de mains en mains des briques à un
échafaudage supérieur, font tous la même besogne, et pourtant il existe
parmi eux une espèce de division de travail, qui consiste en ce que chacun
d’eux fait passer la brique par un espace donné, et que tous ensemble la
font parvenir beaucoup plus promptement à l’endroit marqué qu’ils ne le
feraient si chacun d’eux portait sa brique séparement jusqu’à l’échafaudage
supérieur»). (\emph{F.~Skarbek}: «Théorie des richesses sociales», 2 éme éd.
Paris 1840, vol. I., p. 97, 98).
}. Предмет праці перебігає
\index{i}{0263}  %% посилання на сторінку оригінального видання
ту саму просторінь за коротший час. З другого боку, комбінацію
праці маємо й тоді, коли, приміром, будівлю розпочинають одночасно
з різних боків, хоч би кооперовані робітники робили те
саме або однорідне. Комбінований робочий день в 144 години,
який охоплює предмет праці з багатьох боків у простороні, бо
комбінований робітник або робітник колективний має очі й руки
спереду й ззаду і є до певної міри всюдисущий, — той робочий
день посуває наперед виготовлення цілого продукту швидше,
ніж 12 дванадцятигодинних робочих днів більш або менш відокремлених
робітників, які мусять братися до своєї праці однобічніше.
Просторово різні частини продукту таким чином вистигають
у той самий час.

Ми підкреслювали, що багато робітників, які один одного
доповнюють, роблять те саме або однорідне, бо ця найпростіша
форма спільної праці відіграє чималу ролю і в найрозвиненішій
формі кооперації. Якщо процес праці складний, то вже сама
маса тих, що спільно працюють, дозволяє розподіляти різні
операції поміж різних робітників, отже, і виконувати їх одночасно
та через це скорочувати робочий час, потрібний, щоб виготовити
цілий продукт\footnote{
«Коли треба виконати складну працю, різні справи треба виконувати
одночасно. Один робить одне, тимчасом як другий робить друге,
і всі разом допомагають досягти результату, якого зовсім не могла б здійснити
одна людина. Один гребе, тимчасом як другий кермує стерном, а
третій закидає невід або б'є рибу бодцем, — і влови риби дають такий
результат, що був би неможливий без такого співробітництва». («Est-il
question d’exécuter un travail compliqué, plusieurs choses doivent être
faites simultanément. L’un en fait une pendant que l’autre en fait une
autre, et tous contribuent à l’effet qu’un seul homme n'aurait pu produire.
L’un rame pendant que l’autre tient le gouvernail, et qu’un troisième jette
le filet, ou harponne le poisson, et la pêche a un succès impossible sans
ce concours»). (\emph{Destutt de Tracy}: «Traité de la Volonté et de ses effets»,
Paris 1826, p. 78).
}.

У багатьох галузях продукції бувають критичні моменти,
тобто визначувані самою природою робочого процесу періоди
часу, протягом яких мусять бути досягнені певні результати
праці. Коли, приміром, треба постригти ватагу овець або зжати
та звезти хліб із певного числа морґів, то кількість і якість продукту
залежить від того, щоб операція почалася в певний час
\parbreak{}  %% абзац продовжується на наступній сторінці
