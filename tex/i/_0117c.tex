\parcont{}  %% абзац починається на попередній сторінці
\index{i}{0117}  %% посилання на сторінку оригінального видання
вигляді. З другого боку, цілий його рух відбувається в межах
сфери циркуляції. А що з самої циркуляції неможливо пояснити
перетворення грошей на капітал, творення додаткової вартости,
то й торговельний капітал видається неможливим, якщо тільки
обмінюється еквіваленти;\footnote{
«Коли б реґулярно обмінювалось незмінні еквіваленти, то торговля
була б неможлива» («Under the rule of invariable equivalents commerce
would be impossible»). (\emph{G. Opdyke}\emph{}: «А Treatise on Political Economy»,
New-York 1851, p. 69). «В основі ріжниці між реальною вартістю й
міновою вартістю лежить певний факт, а саме: вартість речі відмінна
від так званого еквіваленту, що його дано за неї в торговлі, тобто, що
цей еквівалент не є еквівалент». (\emph{F.~Engels}: «Umrisse zu einer Kritik
der Nationalökonomie», in Deutsch-Französische Jahrbücher, Paris 1844,
S. 96).
} тому видається, що існування його
можна вивести лише з подвійного ошукування товаропродуцентів —
як покупців і продавців — ошукування від купця, який паразитично
всувається між них. У цьому розумінні Франклін каже:
«Війна є грабіжництво, торговля — ошукування».\footnote{
\emph{Benjamin Franklin}: Works, vol. 2, ed. Sparks b: «Positions to be
examined concerning National Wealth».
} Щоб пояснити
зростання вартости торговельного капіталу інакше, а не
простим тільки ошукуванням товаропродуцентів, для цього потрібен
довгий ряд посередніх ланок, яких зовсім ще бракує тут,
де товарова циркуляція та її прості моменти становлять однісіньку
нашу передумову.

Що сказано про торговельний капітал, те ще в більшій мірі
має силу щодо лихварського капіталу. В торговельному капіталі
обидва полюси, тобто гроші, що їх кинуто на ринок, і збільшені
гроші, що їх витягнуто з ринку, мають, принаймні, за посередні
ланки купівлю й продаж, рух циркуляції. В лихварському капіталі
форма $Г — Т — Г'$ скорочена до нічим не упосереднених полюсів:
$Г — Г'$, гроші, що обмінюються на більшу кількість грошей, —
форма, яка суперечить природі грошей і тому нез’ясовна з погляду
товарового обміну. Тим то Арістотель і каже: «Через те,
що хрематистика є двоїста — одна належить до торговлі, друга до
економіки, остання потрібна й гідна похвали, перша побудована
на циркуляції і справедливо зазнає огуди (бо вона спирається не
на природу, а на обопільне ошуканство), — то лихварство з повнісіньким
правом ненавидять, бо тут сами гроші є джерело наживи
і їх уживається не на те, на що їх винайдено. Їх призначення —
сприяти товаровому обмінові, а процент робить із грошей більшу
кількість грошей. Звідси й назва його (\textgreek{τόχος} — процент і породжене).
Бо ті, що породжені, подібні до тих, що породили. А процент
— то гроші від грошей, так що з усіх галузей наживи ця
найбільше суперечить природі».\footnote{
\emph{Aristoteles}: «De Republica», lib. 1, c. 10.
}

В дальшому ході нашого досліду ми знайдемо, що так торговельний
капітал, як і капітал, що приносить проценти, є вивідні
форми; разом з тим побачимо, чому історично вони з’явилися
раніш від сучасної основної форми капіталу.
