\parcont{}  %% абзац починається на попередній сторінці
\index{i}{0213}  %% посилання на сторінку оригінального видання
постанови із статутів, про які згадує Петті, мають вагу і для
«apprentices» (учнів). Але як саме стояла справа з дитячою працею
ще наприкінці XVII віку, видно з такого нарікання: «Наші
підлітки тут, в Англії, нічого не роблять аж до самого того часу,
коли вони стають учнями, а тоді, звичайно, потребують вони довгого
часу — сім років, — щоб стати досконалими ремісниками».
Навпаки, Німеччину хвалять, бо там дітей од колиски принаймні
«привчають хоч до якої-будь роботи».\footnote{
«A Discourse on the Necessity of Encouraging Mechanic Industry»,
London 1699» p. 13. Маколей, що зфалшував англійську історію в інтересах
віґів і буржуазії, деклямує: «Практика садовити передчасно
дітей за працю\dots{} панувала в XVII віці в майже неймовірних для тодішнього
стану промисловости розмірах. В Norwich’y, головному центрі
вовняної промисловости, шестилітню дитину вважали за працездатну.
Різні письменники тих часів і між ними деякі такі, що їх вважали за людей
з надзвичайно добрими намірами, згадують з «exultation» (із захопленням)
той факт, що в цьому місті праця самих хлопчаків і дівчат створює
багатство, яке понад їхнє власне утримання становило 12.000 фунтів
стерлінґів річно. Що докладніше ми досліджуємо історію минулого, то
більш находимо підстав, щоб відкинути погляди тих, хто вважає наш
вік багатим на нові соціяльні лиха\dots{} Що є нового, так це інтелігенція,
яка викриває це лихо, та гуманність, що гоїть це лихо». («History of
England», vol. I, p. 419). Маколей міг би далі розказати про те, що «amis
du commerce» XVII віку з «надзвичайно добрими намірами», з «exultation»
оповідають, як в одному домі для бідних у Голляндії примушували
працювати чотирилітню дитину, і що цей приклад «vertu mise en pratique»\footnote*{
— практичної чесноти. \emph{Ред.}
}
проходить y всіх творах гуманістів à la Маколей аж до часів
А. Сміса. Правда, разом з виникненням мануфактури, відмінно від ремества,
помічається сліди експлуатації дітей, яка до певної міри здавна
вже існувала в селян і була то розвиненіша, що важче було ярмо,
яке тяжіло над селянином. Тенденція капіталу ясна, але сами факти
мають ще такий поодинокий характер, як і поява на світ двоголових
дітей. Тим то повні передчуття «amis du commerce» з «exultation» змалювали
ці факти для сучасників і нащадків як щось варте уваги й подиву,
і рекомендували їх для наслідування. Той самий шотляндський сикофант
і красномовець-балакун Маколей каже: «Нині ми чуємо лише про реґрес,
а бачимо лише проґрес». Що за очі, а особливо що за вуха!
}

Ще протягом найбільшої частини XVIII віку, аж до епохи
великої промисловости капіталові в Англії не вдалося виплатою
тижневої вартости робочої сили захопити цілий тиждень робітника;
однак рільничі робітники становлять виняток. Та обставина,
що робітники могли цілий тиждень жити на чотириденну
заробітну плату, не видавалась їм за достатню підставу для того,
щоб працювати на капіталіста й останні два дні. Англійські економісти
одного напряму, ті, що були на службі капіталу, якнайлютіше
нападали на робітників за таку впертість, а економісти
другого напряму боронили робітників. Послухаймо, приміром,
полеміку між Постлетвайтом, що його торговельний словник
мав тоді таку саму славу, яку нині мають аналогічні твори МакКуллоха
й Мак-Ґреґора, і цитованим вище автором «Essay
on Trade and Commerce».\footnote{
Найлютіший з усіх обвинувачів робітників є згаданий у тексті
анонімний автор «An Essay on Trade and Commerce, containing Observations
on Taxation etc.», London 1770. Вже давніш він виступив проти
них у своєму творі «Considerations on Taxes», London 1765. Сюди в
першу чергу слід зарахувати й Полонія Артура Юнґа, невимовного базікала
у статистиці. Серед оборонців робітників визначаються: Jacob Vanderlint
в «Money answers all things», London 1734, Reverend Nathaniel
Forster, D. D. в «An Enquiry into the Causes of the Present Price of Provisions»,
London 1767, Dr. Price і, особливо Postlethwayt, так в одному
додатку до його «Universal Dictionary of Trade and Commerce», як і в
«Great Britain’s Commercial Interest explained and improved», 2nd ed.
London 1775. Сами факти находимо сконстатованими в багатьох інших
письменників того часу, між іншим, у Джосії Текера.
}
\index{i}{0214}  %% посилання на сторінку оригінального видання

Постлетвайт каже, між іншим: «Я не можу закінчити цих
коротких заміток, не звернувши уваги на дуже поширений тривіальний
спосіб вислову. Коли робітник (industrious poor), чуємо
ми з уст багатьох, за п’ять днів може одержати досить для того,
щоб жити, то він не захоче працювати повних шість днів. Тому
вони доходять висновку, що треба податками або якимись іншими
способами вдорожчити навіть доконечні засоби існування, щоб
примусити ремісника й мануфактурного робітника до безперервної
шестиденної праці на тиждень. Я мушу попросити дозволу мати
собі інший погляд, ніж мають ці великі політики, що ламають
списи за постійне рабство робітничої людности цього королівства
(«the perpetual slavery of the working people»); вони забувають
приказку: «all work and no play» (сама тільки праця без
забави робить дурним). Хіба ж не пишаються англійці геніальністю
і вмілістю своїх ремісників і мануфактурних робітників,
що досі здобували для брітанських товарів загальну довіру і
славу? Якій обставині завдячується це? Мабуть, нічому іншому,
а тільки тому способові, яким наша робітнича людність ориґінально-весела,
уміє розважатися. Коли б вони мусили працювати
цілий рік усі шість днів на тиждень, повторюючи з дня на день
ту саму працю, невже це не притупило б їхньої геніяльности й не
зробило б їх, жвавих і вправних, тупо-байдужими, і чи не втратили
б наші робітники через таке вічне рабство своєї слави замість
зберегти її?.. Якої майстерної вправности можна б сподіватися
від таких жорстоко мордованих тварин (hard driven animals)?..
Багато з них виконує за 4 дні таку кількість праці, яку француз
виконує за 5 або 6 днів. Але коли англійці мають бути робітниками,
що вічно обтяжені працею, то можна побоюватись, що вони
ще більш виродяться (degenerate), ніж французи. Коли наш нарід
славиться своєю хоробрістю на війні, то хіба не кажемо ми, що
цим він завдячує, з одного боку, доброму англійському ростбіфові
й пудинґові в його шлунку, а з другого — не в меншій мірі
нашому конституційному духові волі. І чому б більшу геніальність,
енергію і вправність наших ремісників і мануфактурних робітників
не завдячувалось тій волі, з якою вони на свій лад розважаються?
Я сподіваюся, що вони ніколи не втратять ні цих
привілеїв, ні доброго життя, звідки однаково випливають їхня
вмілість у праці і їхня сміливість!»\footnote{
\emph{Postlethwayt}. Там же: «Firts Preliminary Discourse», р. 14.
}

На це автор: «Essay on Trade and Commerce» відповідає ось як:
\parbreak{}
