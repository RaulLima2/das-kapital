\parcont{}  %% абзац починається на попередній сторінці
\index{i}{0588}  %% посилання на сторінку оригінального видання
або спромоги платити відповідну квартирну плату, а від того
вжитку, який будуть ласкаві зробити інші «із свого права порядкувати
своєю власністю так, як їм забажається». Хоч би яка
велика була фарма, немає такого закону, що вимагав би збудувати
на ній певну кількість помешкань для робітників, не кажучи
вже зовсім про пристойність цих помешкань; так само
закон не дає робітникові найменшого права на ту землю, для якої
його праця так само доконечна, як дощ і сонячне світло\dots{} Ще
одна загальновідома обставина кидає на терези тяжку вагу проти
нього\dots{} Це — вплив закону про бідних з його постановами про
оселення й податок на користь бідним.\footnote{
1865 р. цей закон дещо поліпшено. Досвід незабаром нам покаже,
що така латанина ані трішечки не допомагає.
} Під його впливом кожна
парафія має грошовий інтерес у тому, щоб обмежити на мінімумі
число сільських робітників, які живуть у ній, бо, на нещастя,
рільнича праця, замість ґарантувати сільському робітникові,
що тяжко працює, та його родині певну й постійну незалежність,
веде його здебільша, довшим або коротшим обхідним
шляхом, до павперизму, — павперизму, до якого протягом цілого
того шляху робітник так стоїть близько, що всяка хороба
або якийсь тимчасовий брак праці примушують його одразу звертатися
по допомогу до парафії; і тим то кожне оселення рільничої
людности в якійсь парафії, очевидно, є для неї збільшення
податку на користь бідним. Великим землевласникам 168 досить
лише вирішити, що в їхніх маєтках не повинно бути жител для
робітників, — і вони одразу звільняються від половини своєї
відповідальности за бідних\dots{} У якій мірі англійська конституція
й закони мали за мету встановити такого роду безумовну
земельну власність, що дає лендлордові силу «робити з своєю
власністю що йому забажається», поводитися з рільниками
як із чужоземцями і проганяти їх із своєї території, — це питання,
що обговорення його не входить у рамки моїх завдань\dots{}
Це право виганяти — не просто теорія. Воно реалізується
на практиці в якнайбільшому маштабі. Це одна з обставин, що
мають вирішальний вплив на житлові умови сільського робітника\dots{}
Про розміри лиха можна судити на основі останнього
перепису, який показав, що протягом останніх 10 років руйнування
домів, не зважаючи на дедалі більший місцевий попит на
них, проґресувалй у 821 різних округах Англії; таким чином в
1861 р. людність, яка проти 1851 р. зросла на 5\sfrac{1}{3}\%, при чому
ми зовсім не беремо на увагу осіб, що примушені жити не в тих
парафіях, де вони працюють, — позбивано в помешкання, площа
яких зменшилась на 4\sfrac{1}{2}\%\dots{} Скоро тільки процес вилюднення
завершується, — каже д-р Гентер, — у результаті його постає
показне село (showvillage), де число котеджів зведено до незнач-

163 Щоб зрозуміти дальшу цитату, зазначимо, що close villages (закритими
селами) називають такі села, що їхні власники один
або два великі лендлорди, a open villages (відкритими селами) — такі,
що їхні землі належать багатьом дрібним власникам. Саме в цих останніх
будівельні спекулянти і можуть будувати котеджі й хати.
\index{i}{0589}  %% посилання на сторінку оригінального видання
ної кількости й де ніхто не сміє жити, крім чабанів, садівників
і сторожів дичини, — цих постійних слуг, з якими шановне панство
поводиться з ласкою, звичайною для цієї кляси слуг.\footnote{
Таке показне село має дуже принадний вигляд, але воно таке
саме нереальне, як і села, що їх бачила Катерина II підчас своєї подорожі
до Криму. Останніми часами навіть і чабана часто виганяють з цих showvillages.
Напр., біля Market Harborough є пасовисько для овець, що
займає майже 500 акрів, де потрібна праця лише однієї людини. Для скорочення
далеких переходів цими просторими рівнинами, гарними пасовиськами
Leicester’a й Northampton’a, чабанові звичайно давали котедж
на фармі. Тепер йому дають тринадцятий шилінґ на помешкання, що його
він мусить собі шукати далеко у відкритому селі.
}
дле земля потребує оброблення, і ми бачимо, що робітники,
заняті на ній, живуть не в земельного власника, а приходять з
відкритого села, що лежить, може, на віддалі трьох миль, де
їх пустили до себе численні дрібні домовласники після того, як
по закритих селах зруйнували котеджі робітників. Там, де справи
наближаються до цього результату, котеджі своїм злиденним
виглядом звичайно свідчать про долю, що на неї вони засуджені.
Вони перебувають на різних щаблях природної руїни. Поки
дах тримається, робітникові дозволяють платити за помешкання
ренту, і він часто дуже радий, що йому це дозволяють, хоча б
йому доводилося платити таку ціну, як і за добре помешкання.
Але жодного ремонту, жодних полагоджень, крім таких, що їх
зможе зробити сам пребідний мешканець. Коли ж котедж стає,
нарешті, цілком непридатний для житла, то це означає лише,
що число зруйнованих котеджів збільшилось на один і остільки
менше доведеться надалі платити податку на бідних. Тимчасом
як великі землевласники таким способом звалюють із себе податок
на користь бідним через вилюднення належної їм землі, найближче
містечко або відкрите село приймає до себе викинутих
робітників; я кажу найближче, але оце «найближче» може лежати
три-чотири милі від фарми, де робітник має день-удень
тяжко працювати. Таким чином, до його денної праці, так, наче
це була б дурничка, долучається потреба день-у-день проходити
7--8 миль, щоб заробити собі на щоденний хліб. Всі сільські
роботи, що їх виконують його дружина й діти, відбуваються
тепер серед таких самих важких обставин. Але й це ще не все
лихо, що його спричиняє йому віддаленість від місця роботи.
В одкритому селі будівельні спекулянти скуповують шматки землі
і якомога густіше забудовують їх найдешевшими халупами
всякого роду. І в таких злиденних житлах, які, навіть тоді,
коли вони виходять на чисте поле, мають найжахливіші характеристичні
риси найпоганіших міських жител, туляться рільничі
робітники Англії\dots{}\footnote{
«Доми робітників (у відкритих селах, які, звісно, завжди переповнені)
звичайно збудовані рядами, задніми стінами до крайньої лінії
того шматка землі, що його будівельний спекулянт називає своїм. Через
те світло й повітря може проходити до них лише з фасаду». (Звіт д-ра
Гентера в «Public Health. Seventh Report 1864», London 1865, p. 135).
Дуже часто власник пивниці або сільський крамар здає також у найми
} З другого боку, не треба собі уявляти,
\index{i}{0590}  %% посилання на сторінку оригінального видання
що робітник, який живе навіть на оброблюваній ним землі, знаходить
собі таке помешкання, що на нього заслуговує його продуктивне
життя. Навіть у князівських маєтках робітничі котеджі
часто мають якнайзлиденніший характер, Є лендлорди, що вважають
і стайню за досить добре житло для своїх робітників
і їхніх родин, а проте не соромляться видушувати якнайбільше
грошей з винаймання таких помешкань.\footnote{
Винаймач помешкання (фармер або лендлорд) безпосередньо або
посередньо збагачується з праці людини, що їй він платить 10 шилінґів
на тиждень, а потім знову відбирає в цього бідолахи 4 або 5 фунтів
стерлінґів річної квартирної плати за хати, що на вільному ринку не
варті й 20 фунтів стерлінґів, але зберігають свою штучну ціну, бо власник
має силу сказати: «Бери мою хату, або йди геть звідси і шукай собі,
не мавши від мене атестації, якесь інше пристановище»\dots{} Коли людина
хоче поліпшити своє становище й піде на залізницю укладати шини або
на каменярню, то та сама сила знову каже йому: «Працюй в мене за що
низьку плату, або йди геть за тиждень після попередження; забирай свою
свиню, коли вона в тебе є, і поміркуй про те, що ти дістанеш за картоплю,
яка росте на твоєму городі». Коли ж проганяти робітника не в інтересах
власника (або фармера), то він у таких випадках іноді вважає за краще
підвищити квартирну плату, щоб покарати робітника, за те, що він
покинув у нього служити» (Д-р Гентер, там же, стор. 132).
} Хай це буде лише
напівзавалена халупа з однією кімнатою для спання, без печі,
без кльозета, з вікнами, що не відчиняються, без водопостачання,
крім якогось рівчака, без садка, — робітник безпорадний проти
такої несправедливости. А наші санітарно-поліційні закони
(The Nuisances Removal Acts) — це мертва буква. Бо ж проводити
їх доручено тим саме власникам, що здають у найми
такі діри\dots{} Виняткові веселіші картини не повинні засліплювати
нас та закривати перед нами величезну силу фактів, що є ганьба
англійської цивілізації. Дійсно, жахний мусить бути стан речей,
коли, не зважаючи на очевидну потворність теперішніх
помешкань, компетентні спостерігачі одноголосно доходять такого
висновку, що навіть ці повсюдно нікчемні помешкання є ще
безмірно менше лихо, ніж просто кількісний брак помешкань.
Вже віддавна переповнення помешкань сільських робітників

і помешкання. Тоді поряд із фармером він є другий пан для сільського
робітника. Останній мусить бути одночасно і його покупцем. «З 10 шилінґами
на тиждень мінус 4 фунти квартирної плати на рік він зобов’язаний
купувати чай, цукор, борошно, мило, свічки й пиво по цінах,
які сподобається визначити крамареві» (Там же, стор. 134). Ці відкриті
села — це дійсно «карні колонії» англійського рільничого пролетаріату.
Багато з цих котеджів — це чисті постоялі двори, через які проходить
уся бродяча наволоч з околиці. Селянин і його родина, що серед,
найбрудніших обставин часто справді навдивовижу зберегли путящість
і чистоту характеру, тут цілком гинуть. Серед знатних Шейлоків це,
звичайно, мода по-фарисейському знизувати плечима на адресу будівельних
спекулянтів, дрібних власників і відкритих сел. Вони дуже
добре знають, що їхні «закриті села й показні села» є місце народження
«відкритих сел» і не могли б існувати без цих останніх. «Без дрібних
власників одкритих сел найбільша частина сільських робітників мусіла б
спати під деревами тих маєтків, де вони працюють» (Там же, стор. 135).
Система «відкритих» і «закритих» сел панує по всій середній і всій
східній Англії.
\parbreak{}  %% абзац продовжується на наступній сторінці
