рів сумління. Йому ні в якому разі не слід вертатись до ролі
збирача скарбу, який показав нам, що виходить з аскетизму.
А, крім того, де нічого немає, там і імператор втратив своє право.
Хоч і яка була б заслуга його поздержливости, але ж нема нічого,
чим можна було б зокрема оплатити її, бо вартість продукту,
який виходить із процесу, дорівнює лише сумі товарових вартостей,
що їх кинуто до процесу. Отже, нехай же він заспокоїться
на тому, що чеснота є нагорода за чесноту. Та замість того капіталіст
робиться настирливим. Пряжа йому непотрібна. Він продукував
її на продаж. Так нехай продасть її або, ще простіш, на
майбутнє нехай продукує лише речі для своєї власної потреби, —
рецепт, який йому приписав був уже його домашній лікар Мак
Кулох як випробуваний лік проти епідемії перепродукції. Капіталіст
уперто стає диба. Хіба ж робітник міг би своїми лише
десятьма пальцями творити у блакитному повітрі, продукувати
товари з нічого? Хіба ж не він, не капіталіст, дав йому матеріял,
через який і в якому робітник лише й може втілити свою працю?
А що більша частина суспільства складається з таких голяків,
то хіба він своїми засобами продукції, своєю бавовною і своїми
веретенами, не зробив незмірної послуги суспільству, не зробив
послуги самому робітникові, якого він, окрім того, ще й забезпечив
засобами існування? І чи не слід йому зарахувати це як заслугу?
А робітник — хіба він йому не відплативсь послугою, перетворивши
бавовну й веретена на пряжу? Та, крім того, тут річне в
послугах.15 Послуга — це не що інше, як корисний ефект споживної
вартости, чи то товару, чи то праці.\footnote{
З приводу цього я, між іншим, зауважую в «Zur Kritik der Politischen
Oekonomie», Berlin 1859, S. 14. («До критики...», ДВУ, 1926 р.,
crop. 55): «Зрозуміло, яку «послугуо мусить робити категорія «послуг»
(service) економістам такого сорту, як Ж. Б. Сей і Ф. Бастіа».
} Алеж тут ідеться
про мінову вартість. Капіталіст виплатив робітникові вартість
у 3 шилінґи. Робітник повернув йому точний еквівалент, додавши
до бавовни вартість у 3 шилінґи, дав йому вартість за вартість.
Наш приятель, який тільки-по так пишався своїм капіталом,
раптом набирає безпретенсійної постаті свого власного робітника.
Хіба ж він сам не працював? Не виконував праці догляду, нагляду
над прядуном? Хіба ж ця його праця не утворює вартости?
Але тут його власний головний директор і його управитель зни-

І5 «Хвались, прибирайся й чепурись... Але хто бере більше або краще
(ніж він дає), той лихвар, і це значить, що не послугу, а шкоду заподіяв
він своєму ближньому, як це буває підчас злодійства й грабунку. Не все
те послуга й добродійство для ближнього, що зветься послугою й добродійством.
Бо перелюбниця і перелюбник роблять один одному велику
послугу і втіху. Райтер\footnote*{
-військовий, кінник. Ред.
} робить убивцеві-палієві велику райтерську
послугу, допомагаючи йому грабувати по шляхах, нападати на маєтки і
людей. Папісти роблять нашим велику послугу, бо вони не всіх топлять,
палять, забивають, примушують гнити по тюрмах, але декого лишають
живим та виганяють їх або забирають у них усе, що вони мають. Сам чорт
робить своїм прислужникам велику незмірну послугу... Словом, світ
повен великих, удалих щоденних послуг і добродійств. (Martin Luther:
«An die Pfarherrn, wider den Wucher zu predigen usw.», Wittenberg 1540).