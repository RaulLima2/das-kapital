\index{i}{0606}  %% посилання на сторінку оригінального видання 
Англія країна розвиненої капіталістичної продукції й переважно
промислова країна, стекла б кров’ю від такого кровопуску,
якого зазнала Ірляндія. Але тепер Ірляндія є лише рільнича
округа Англії, відділена від неї широким каналом, округа, що
постачає їй збіжжя, вовну, худобу, промислових і військових
рекрутів.

Збезлюднення призвело до того, що багато землі лишилось
необробленою, кількість рільничого продукту дуже зменшилась;\footnote{
Якщо продукт зменшується відносно також і на акр, то не треба
забувати, що Англія протягом півтора віків посередньо експортувала,
ірляндський ґрунт, не залишаючи рільникам навіть засобів відновлювати
складові частини ґрунту.
}
не зважаючи на поширення площі, призначеної для
скотарства, в деяких його галузях сталося абсолютне зменшення,
в інших — ледве помітний проґрес, що раз-у-раз переривався
зворотним рухом. А все ж разом із зменшенням людности
безупинно зростали земельні ренти й фармерські зиски,
хоч останні не так стало, як перші. Причину цього зрозуміти
легко. З одного боку, разом з централізацією фарм і перетворенням
орної землі на пасовиська щораз більша частина сукупного
продукту перетворювалась на додатковий продукт. Додатковий
продукт зростав, хоч сукупний продукт, якого він становить
частину, меншав. З другого боку, грошова вартість цього
додаткового продукту зростала ще швидше, ніж його маса, бо
англійські ринкові ціни на м’ясо, вовну й т. ін. протягом останніх
двадцятьох років і особливо протягом останніх десятьох років
раз-у-раз зростали.

Роздрібнені засоби продукції, що для самого продуцента служать
за засоби праці й існування і що не зростають своєю вартістю
за допомогою прилучення до себе чужої праці, так само
не є капітал, як не є товар продукт, споживаний його власним
продуцентом. Хоч маса засобів продукції, застосованих у рільництві,
і зменшилась разом із зменшенням людности, проте
маса капіталу, застосованого в рільництві, збільшилась, бо
частина раніш роздрібнених засобів продукції перетворилась на
капітал.

Цілий капітал Ірляндії, вкладений поза рільництвом — у
промисловість і торговлю, нагромаджувався протягом останніх
двох десятиліть поволі і з постійними великими коливаннями.
Зате тим швидше розвивалась концентрація його індивідуальних
складових частин. Нарешті, хоч і яке невеличке було його абсолютне
зростання, але відносно, порівняно із зменшенням кількости
людности, він значно збільшився.

Отже, перед нашими очима тут у великому маштабі розгортається
процес, що кращого за нього ортодоксальна економія й
бажати не може для ствердження своєї догми, що злидні виникають
з абсолютного перелюднення і що рівновага відновлюється
через зменшення людности. Це експеримент куди важливіший,
\parbreak{}  %% абзац продовжується на наступній сторінці
