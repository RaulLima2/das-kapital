\parcont{}  %% абзац починається на попередній сторінці
\index{i}{0044}  %% посилання на сторінку оригінального видання
цікавить людей. Нас же вона, як речей, не стосується. Що нас,
власне, стосується як речей, так це наша вартість. Це показують
наші власні взаємовідносини як речей-товарів. Ми відносимось
один до одного лише як мінові вартості. А послухаймо
тепер, як економіст висловлює думки товарової душі: «Вартість
(мінова вартість) є властивість речей, багатство (споживна вартість)
є властивість людей. У цьому розумінні вартість неодмінно
включає обмін, багатство ж — ні».\footnote{
«Value is a property of things, riches of man. Value, in this sense,
necessarily implies exchanges, riches do not». («Observations on some verbal
disputes in Pol. Econ., particularly relating to value and to supply and
demand», London 1821, p. 16).
} «Багатство (споживна вартість) —
це атрибут людини, вартість — атрибут товарів. Людина
або суспільство багаті; перла або діямант мають вартість\dots{}
Перла або діямант мають вартість як перла або діямант».\footnote{
«Riches are the attribute of man, value is the attribute of commodities.
A man or community is rich, a pearl or a diamond is valuable\dots{}
A pearl or a diamond is valuable as a pearl or diamond». (\emph{S. Bailey}:
«А critical Dissertation on the Nature etc. of value», p. 165).
}
Досі ще жоден хемік не винайшов у перлі або в діяманті мінової
вартости. Але економісти-винахідники цієї хемічної субстанції,
які претендують на особливу глибину критичної думки, вважають,
що споживна вартість речей незалежна від їхніх речових властивостей,
а вартість, навпаки, властива їм як речам. Їх запевняє
в цьому та дивна обставина, що споживна вартість речей реалізується
для людини без обміну, отже, в безпосередньому відношенні
між річчю й людиною, тимчасом як їхня вартість, навпаки,
реалізується лише в обміні, тобто в певному суспільному процесі.
Хто не пригадає собі тут сердечного Доґбері, який навчає нічного
сторожа Сіколя: «Бути приємною людиною, це — дар обставин,
а вміння читати й писати дає природа».\footnote{
Автор «Observations» і L. Bailey обвинувачують Рікарда в тому,
що він перетворив мінову вартість з чогось-то лише відносного на щось
абсолютне. Навпаки. Позірну відносність, яку мають ці речі, приміром,
діямант і перла, як мінові вартості, він звів до захованого за цією позірністю
дійсного відношення — до їхньої відносности як простих виразів
людської праці. Коли рікардіянці відповіли Ваіlеу’єві грубо, але невлучно,
то лише через те, що в самого Рікарда вони не знайшли жодного
пояснення щодо внутрішнього зв’язку між вартістю й формою вартости,
або міновою вартістю.
}

\section{Процес обміну}

Товари сами не можуть ходити на ринок і сами не можуть
обмінюватися. Отже, ми мусимо звернутися до їхніх хоронителів —
до посідачів товарів. Товари є речі й тим то не можуть вони чинити
опору людям. Як вони не йдуть з доброї волі, то людина може
вжити сили, тобто взяти їх.\footnote{
В XII віці, так уславленому побожністю, серед цих товарів часто
траплялися дуже делікатні речі. Так, один французький поет того часу
з-поміж товарів, що були на ринку в Landit, налічує поруч з матеріями
для одягу, черевиками, шкурами, сільськогосподарським знаряддя і~\abbr{т. ін.}
також і «femmes folles de leur corps».\footnote*{
— повій. Peд.
}} Щоб ці речі ставити одну до однієї
\index{i}{0045}  %% посилання на сторінку оригінального видання
у відношення як товари, хоронителі товарів мусять ставитись
одні до одних як особи, воля яких панує в тих речах, так що один
лише з волі другого, отже, кожний лише за допомогою акту волі,
спільного одному й другому, присвоює собі чужий товар, відчужуючи
свій власний. Тому вони мусять визнавати себе взаємно
за приватних власників. Це правне відношення, що його формою
є договір, законно чи незаконно складений, є вольове відношення,
що в ньому відбивається економічне відношення. Зміст цього
правного або вольового відношення дається самим економічним
відношенням.\footnote{
Прудон черпає свій ідеал справедливосте, justice éternelle (вічної
справедливости) з правних відносин, відповідних товаровій продукції,
даючи також тим, до речі сказавши, дуже втішний для всіх філістерів
доказ, що форма товарової продукції така сама вічна, як і справедливість.
Потім, навпаки, він хоче реформувати за цим ідеалом дійсну товарову
продукцію й відповідне їй дійсне право. Що подумали б про хеміка, який,
замість студіювати дійсні закони обміну речовин і розв’язувати на їхній
основі певні завдання, захотів би реформувати обмін речовин за «вічними
ідеями», «naturalité» й «affinité» (властивости й споріднености)? Коли
нам кажуть, що «лихвар» суперечить «justice éternelle», «équité éternelle
» (вічній правді), «mutualité éternelle» (вічній взаємності) та іншим
«vérités éternelles» (вічним правдам), то невже ж ми дізнаємося про лихваря
щось більше, ніж знали про нього отці церкви, коли вони казали
що він суперечить «grâce éternelle» (вічному милосердю), «foi éternelle»
(вічній вірі), «volonté éternelle de dieu» (вічній волі божій)?
} Особи існують тут одна для однієї лише як представники
товарів, і тим то як посідачі товарів. У дальшому ході
викладу ми взагалі побачимо, що характеристичні економічні
маски осіб є лише персоніфікації економічних відносин, носіями
яких ці особи протистоять одна одній.

Що саме відрізняє посідача товарів від товару, так це та обставина,
що для товару кожне інше товарове тіло є лише форма виявлення
його власної вартости. Левелер з роду та цинік, товар
завжди готовий обміняти не лише душу, але й тіло на всякий
інший товар, хоч би останній був огидливіший, ніж Маріторна.\footnote*{
Дієва особа з роману Сервантеса «Дон-Кіхот». \emph{Ред.}
} Цю відсутню у товару здібність схоплювати конкретні
властивості товарових тіл посідач товарів доповнює своїми п’ятьма
й більше чуттями. Його товар не має для нього жодної безпосередньої
споживної вартости. А то він не виніс би його на ринок.
Він має споживну вартість для інших. Для посідача товарів
товар безпосередньо має лише споживну вартість бути носієм
мінової вартости й таким чином засобом обміну.\footnote{
«Бо вжиток кожного добра є подвійний. — Один вжиток, властивий
речі як такій, другий — ні; так, сандали можуть бути, щоб взувати
на ноги, і для обміну. Обидва є споживні вартості сандалів, бо й той, хто
обмінює сандали на щось, чого йому бракує, наприклад, на харчі, користується
сандалами як сандалами. Але це не є їхній природний спосіб
ужитку, бо вони існують не для обміну». (\emph{Aristoteles}: «De Republica»,
lib. I, c. 9).
} Через те
\parbreak{}  %% абзац продовжується на наступній сторінці
