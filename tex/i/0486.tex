том продукційного процесу від того, що худоба сама споживає
те, що їсть. Постійне зберігання і репродукція робітничої кляси
лишається постійною умовою репродукції капіталу. Виконання
цієї умови капіталіст може спокійно полишити інстинктові
робітників до самозбереження й розмножування. Капіталіст
дбає лише про те, щоб якомога обмежити їхнє особисте споживання
на найдоконечнішому, і, як небо від землі, він далекий від
тієї південно-американської грубости, з якою робітників примушують
їсти поживніший харч замість менш поживного.8

Тому капіталіст і його ідеолог, політико-економ, розглядають
як продуктивне споживання лише ту частину особистого споживання
робітника, що потрібна для увіковічнення робітничої
кляси, отже, що дійсно мусить бути спожита для того, щоб капітал
міг споживати робочу силу; а те, що робітник споживає поверх
того для своєї насолоди, є непродуктивне споживання.9
Коли б акумуляція капіталу спричинила підвищення заробітної
плати, а тому й збільшення засобів споживання робітника без
збільшеного споживання робочої сили капіталом, то додатковий
капітал був би спожитий непродуктивно.10 Справді, особисте
споживання робітника є для нього самого непродуктивне, бо воно
репродукує лише індивіда, що має потреби; воно є продуктивне
для капіталіста і для держави, бо воно є продукування сили,
що продукує чуже багатство.11

Отже, з суспільного погляду робітнича кляса, навіть поза
безпосереднім процесом праці, є так само приналежність капіталу,
як і мертве знаряддя праці. Навіть її особисте споживання
є в певних межах лише момент процесу репродукції капіталу.
Але йей процес, постійно віддаляючи продукт праці робітничої
кляси від її полюса до протилежного полюса капіталу, дбає
про те, щоб ці самосвідомі знаряддя продукції не втекли. Особисте
споживання робітників дбає, з одного боку, про їхнє власне збе-

8 «Робітники в копальнях Південної Америки, що їхня щоденна
праця (найтяжча, мабуть, у світі) є в тому, щоб витягати на своїх плечах
на поверхню землі' вантаж руди в 180—200 фунтів з глибини 450 футів,
харчуються лише хлібом та бобами; вони воліли б харчуватися
самим хлібом, але їхні пани, відкривши, що на самому хлібі вони не
можуть працювати так дуже, поводяться з ними, як із кіньми, і примушують
їх їсти боби; а боби далеко багатші на кісткову золу, ніж хліб».
(Liebig: «Die Chemie in ihrer Anwendung auf Agrikultur und Physiologie»,
7 Auflage, 1862, част. 1, стор. 194, примітка).

9    James Mill: «Eléments d’Economie Politique», Paris 1823, стор. 23S
і далі.

10 «Коли б ціна на пращо піднеслася так високо, що, не зважаючи
на приріст капіталу, не можна було б уживати більше праці, то я сказав
би, що такий приріст капіталу споживається непродуктивно» (Ricardo:
«Principles of Political Economy», 3 rd ed., London 1821, p. 163).

11 «Єдине продуктивне споживання у власному значенні слова є
споживання або руйнування багатства (він має на думці споживання
засобів продукції) капіталістом з метою репродукції... Робітник... є
продуктивний споживач для особи, що вживає його, і для держави, але,
точно кажучи, не для себе самого». (Malthus: «Definitions in Political
Economy», London 1853, p. ЗО).
