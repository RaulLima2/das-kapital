квартер збіжжя за 3 фунти стерлінґів і за ці 3 фунти стерлінґів
купую одяг, то для мене ці 3 фунти стерлінґів остаточно витрачені.
Я вже не маю ніякого відношення до них. Вони вже належать
торговцеві одягом. А коли я продам другий квартер збіжжя,
то гроші знов повернуться до мене, але не в наслідок першої
оборудки, а лише в наслідок її повторення. Вони знову віддаляться
від мене, скоро тільки я другу оборудку доведу до кінця
і знов куплю. Отже, в циркуляції Т — Г — Т витрата грошей
не має ніякого відношення до їхнього зворотного припливу. Навпаки,
в Г — Т — Г зворотний приплив грошей зумовлюється
характером самої їхньої витрати. Без цього зворотного припливу
операція не вдається; процес переривається і не є ще закінчений,
бо бракує другої його фази, продажу, що доповнює й завершує
купівлю.

Кругобіг Т — Г — Т має за вихідний пункт якийсь товар і
за кінцевий пункт якийсь інший товар, що виходить із циркуляції
і ввіходить у споживання. Тому споживання, задоволення
потреб, одне слово, споживна вартість є його кінцева мета.
Навпаки, кругобіг Г — Т — Г має за вихідний пункт грошовий
полюс, і кінець-кінцем він повертається назад до того самого
полюса. Тому його движний мотив і мета, що його визначає, є
сама мінова вартість.

У простій товаровій циркуляції обидва крайні пункти мають
ту саму економічну форму. Обидва вони є товари. Вони є також
товари однакової величини вартости. Але вони є якісно різні
споживні вартості, приміром, збіжжя й одяг. Обмін продуктів,
обмін різних речовин, що в них виражається суспільна праця,
становить тут зміст руху. Інша справа в циркуляції Г — Т — Г.
На перший погляд вона здається беззмістовною через свою тавтологічність.
Обидва полюси мають таку саму економічну форму.
Вони обидва є гроші, отже, вони не є якісно відмінні споживні
вартості, бо гроші — це саме й є така перетворена форма товарів,
у якій згасають усі їхні особливі споживні вартості. Спочатку
обміняти 100 фунтів стерлінґів на бавовну, а потім знову цю саму
бавовну обміняти на 100 фунтів стерлінґів, тобто манівцями
гроші на гроші, те саме на те саме, — це видається так само безцільною,
як і безглуздою операцією.\footnote{
«Гроші не обмінюють на гроші» — вигукує Мерсьє де ля Рів’єр
на адресу меркантилістів. («L’Ordre naturel et essentiel des sociétés politiques»,
Physiocrates, éd. Daire, p. 486). В одному творі, де мова йде
ex professo про «торговлю» й «спекуляцію», ми читаємо: «Всяка торговля
сходить на обмін різнорідних речей; і користь [для купця?] випливає
саме з цієї різнорідности. Обмін одного фунта хліба на один фунт хліба...
був би без жодної користи... звідси корисний контраст між торговлею
і грою, яка є лише обмін грошей на гроші». (Th. Corbet: «An Inquiry
into the Causes and Modes of the Wealth of Individuals; or the Principles
of Trade and Speculation explained», London 1841, p. 5). Хоч Корбет і не
бачить, що Г — Г, обмін грошей на гроші є форма циркуляції, характеристична
не лише для торговельного капіталу, але й для кожного капіталу,
однак він принаймні признає, що ця форма особливого роду торговлі,
спекуляції, є форма гри; але з’являється Мак Куллох і відкриває, що
} Певна сума грошей може