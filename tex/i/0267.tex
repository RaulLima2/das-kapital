гальні функції, що виникають із руху цілого продуктивного тіла
відмінно від руху його самостійних органів. Окремий скрипаль
дириґує собі сам, оркестра потребує дириґента. Ця функція
керування, догляду та упосереднення стає функцією капіталу,
скоро тільки підпорядкована йому праця стає кооперативною.
Як специфічна функція капіталу функція керування набирає
специфічних характеристичних ознак.

Насамперед рушійним мотивом і визначальною метою капіталістичного
процесу продукції є якомога більше самозростання
капіталу,\footnote{
«Зиск... однісінька мета продукції» («Profits.... is the sole end
of trade»). (J. Vanderlint: «Money answers all Things», London 1734,
p. 11)
} тобто якомога більша продукція додаткової вартости,
отже, якомога більший визиск робочої сили капіталом. Із зростом
маси одночасно експлуатованих робітників зростає і їхній
опір, а тому неминуче зростає і гніт капіталу, щоб перебороти
той опір. Керування капіталіста є не тільки осібна функція, що
виникає з природи суспільного процесу праці й належить до нього,
воно є одночасно й функція визиску суспільного процесу праці
і тому зумовлюється неминучим антагонізмом між визискувачем
і сировинним матеріялом його визиску. Так само із зростом розміру
засобів продукції, що протистоять найманому робітникові
як чужа власність, зростає й доконечність контролю над доцільним
уживанням цих засобів.\footnote{
Часопис англійських філістерів «Spectator» сповіщає в числі
з 3 червня 1866 р., що після заведення чогось на зразок товариського підприємства
між капіталістом та робітником у «Wirework company of Manchester»
«першим результатом було те, що раптом зменшилося псування
матеріялу, бо робітники зрозуміли, що їм, як і всім іншим власникам,
нема нащо псувати своє власне майно, а псування знаряддя та матеріялу
є, може, найбільше, після легкодушних боргів, джерело втрат у промисловості»
(«the first result was a sudden decrease in waste, the men not seeing
why they should waste their own property any more than any other master’s,
and waste is perhaps, next to bad debts, the greatest source of manufacturing
loss»). Той самий часопис викриває ось яку основну хибу в рочдельських кооперативних
спробах: «Вони показали, що робітничі асоціяції можуть
успішно порядкувати крамницями, фабриками та майже всіма формами
промисловости, і що вони надзвичайно поліпшили становище самих робітників,
але! але в такому випадку вони зовсім не залишали якогось виразного
місця для капіталіста» («They showed that associations of workmen
could manage shops, mills, and almost all forms of industry with success,
and they immensely improved the condition of the men, but then they did
not leave a clear place for masters»). Quelle horreur!\footnote*{
Який жах! Ред.
}
} Далі, кооперація найманих робітників
є лише результат діяння капіталу, який їх одночасно вживає.
Зв’язок їхніх функцій та їхня єдність як продуктивного цілого
тіла лежать поза ними, в капіталі, що їх згуртовує та тримає
вкупі. Тим-то зв’язок їхніх праць протистоїть їм ідеально як плян,
практично — як авторитет капіталіста, як сила чужої волі, що
підпорядковує їхню діяльність своїй меті.

Тому, якщо капіталістичне керування своїм змістом є двоїсте
внаслідок двоїстости самого продукційного процесу, що ним