\parcont{}  %% абзац починається на попередній сторінці
\index{i}{0103}  %% посилання на сторінку оригінального видання
що з них один лише продає, другий лише купує, а третій навпереміну
купує й продає.

Але що вже насамперед відрізняє обидва кругобіги $Т — Г — Т$
і $Г — Т — Г$, так це зворотна послідовність тих самих протилежних
фаз циркуляції. Проста товарова циркуляція починається
продажем і закінчується купівлею: циркуляція грошей як капіталу
починається купівлею й закінчується продажем. Там
товар становить вихідний і кінцевий пункт руху, тут — гроші.
В першій формі гроші упосереднюють цілий процес, у другій,
навпаки, товар.

У циркуляції $Т — Г — Т$, кінець-кінцем, гроші перетворюються
на товар, що служить за споживну вартість. Отже, гроші
тут витрачається остаточно. Навпаки, в протилежній формі
$Г — Т — Г$ покупець витрачає гроші, щоб одержати їх як продавець.
Купуючи товар, він подає гроші в циркуляцію, щоб
знов їх із неї вилучити через продаж того самого товару. Він
випускає гроші лише з лукавим наміром знов дістати їх до своїх
рук. Отже, гроші тут лише авансується.\footnote{
«Коли купують якусь річ з тим, щоб знов її продати, то вжиту
на це суму називають авансованими грішми: коли ж річ купують не
на продаж, гроші можна називати витраченими» («When a thing is bought,
in order to be sold again, the sum employed is called money advanced;
when it is bought not to be sold, it may be said to be expended»). (\emph{James
Steuart}: «Works etc. edited by General Sir James Steuart, his son», London
1801, vol. 1, p. 274).
}

У формі $Т — Г — Т т$а сама монета двічі змінює своє місце.
Продавець одержує її від покупця й виплачує її іншому продавцеві.
Цілий процес, що починається одержанням грошей за товар,
кінчається віддачею грошей за товар. Протилежне бачимо у
формі $Г — Т — Г$. Не та сама монета, а той самий товар двічі
змінює тут місце. Покупець одержує його з рук продавця і знову
передає його до рук іншого покупця. Як у простій товаровій
циркуляції дворазова зміна місця тієї самої монети спричинює
її остаточний перехід з одних рук до інших, так само тут дворазова
зміна місця того самого товару спричинює зворотний приплив
грошей до їхнього першого вихідного пункту.

Зворотний приплив грошей до їхнього вихідного пункту не
залежить від того, чи товар продається дорожче, ніж його куплено.
Ця обставина впливає лише на величину грошової суми, що зворотно
припливає. Саме явище зворотного припливу відбувається,
скоро тільки куплений товар знову продається, тобто скоро
тільки кругобіг $Г — Т — Г$ цілком вивершується. Отже, ми тут
маємо почуттєво сприйману ріжницю поміж циркуляцією грошей
як капіталу і циркуляцією їх як просто грошей.

Кругобіг $Т — Г — Т$ є цілковито закінчений, скоро тільки
продаж товару приносить гроші, які знову віддаляються через
купівлю іншого товару. А коли все ж таки відбувається зворотний
приплив грошей до їхнього вихідного пункту, то лише через
поновлення або повторення цілого процесу. Коли я продаю
\parbreak{}  %% абзац продовжується на наступній сторінці
