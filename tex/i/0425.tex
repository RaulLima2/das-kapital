Відділ п'ятий

Продукція абсолютної і відносної
Додаткової вартости

Розділ чотирнадцятий

Абсолютна й відносна додаткова вартість

Спочатку ми розглядали процес праці абстрактно (див. п’ятий
розділ), незалежно від його історичних форм, як процес між
людиною і природою. Там ми казали: «Коли розглядати цілий
процес праці з погляду його результату, [продукту],\footnote*{
Заведене у прямі дужки взято з французького видання. \emph{Ред.}
} то і засоби
праці, і предмет праці, одне й друге, з’являються як засоби
продукції, а сама праця — як продуктивна праця». У примітці
сьомій був додаток: «Цього визначення продуктивної праці,
що випливає з погляду простого процесу праці, зовсім недосить
для капіталістичного процесу продукції». Це нам треба тут
розвинути далі.

Оскільки процес праці є суто індивідуальний, той самий
робітник сполучає всі ті функції, що пізніше розділяються. В індивідуальному
присвоюванні предметів природи для своїх життєвих
цілей він контролює сам себе. Пізніше його контролюють.
Поодинока людина не може впливати на природу, не пускаючи
в рух своїх мускулів під контролем свого власного мозку. Як у
системі природи голова й руки належать одне до одного, так само
і процес праці сполучає працю голови й працю рук. Пізніше ті
праці розділяються аж до ворожої протилежности. Продукт пе-

в Англії), — це універсальний закон рільничої промисловости». Це, — каже
далі Лібіґ, — річ досить дивна, бо Міллові була невідома основа цього закону»
(Liebig, там же, книга 1, стор. 143 і примітка). Не кажучи вже про
помилкове тлумачення слова «праця», під яким Лібіґ розуміє щось інше,
ніж політична економія, в усякому разі «річ досить дивна», що він із Дж.
Ст. Мілла робить першого оповісника теорії, яку Джемс Андерсон опублікував
уперше за часів А. Сміса й повторював її в різних творах аж до
початку XIX віку, теорії, яку 1815 р. присвоїв собі Малтуз, взагалі
майстер у пляґіятах (ціла його теорія залюднення є безсоромний пляґіят),
яку Вест розвинув одночасно з Андерсоном і незалежно від нього,
яку Рікардо 1817 р. зв’язав із загальною теорією вартости й яка від
того часу під ім’ям Рікарда обійшла ввесь світ, яку 1820 р. звульґаризував
Джемс Мілл (батько Дж. Ст. Мілла) і яку, нарешті, повторює, між
іншим, і пан Дж. Ст. Мілл як шкільну догму, що встигла вже зробитися
банальною фразою. Безперечно, Дж. Ст. Мілл завдячує свій, в усякому
разі, «дивний» авторитет майже виключно подібним qui pro quo-