Ясна річ, що коли продукція якоїсь машини коштує стільки ж
праці, скільки заощаджується при вживанні її, то відбувається
просте переміщення праці, тобто загальна сума праці, потрібна
на продукцію товару, не меншає, або продуктивна сила праці не
більшає. Однак ріжниця між працею, якої коштує машина, і
тією працею, яку вона заощаджує, або ступінь її продуктивности,
не залежить, очевидно, від ріжниці між її власного вартістю й
вартістю того знаряддя, яке вона заміняє. Ця ріжниця триває
так довго, поки трудові витрати на машину, а тому й та частина
вартости, яку вона додає до продукту, лишаються меншими від
тієї вартости, яку робітник із своїм знаряддям додав би до предмету
праці. Тим то продуктивність машини вимірюється тим
ступенем, у якому вона заміняє людську робочу силу. За Бейнсом,
на 450 веретен-мюлів із підготовчими машинами, що їх рухає
одна парова кінська сила, припадає 2 1/2 робітника,\footnote{
За річним звітом торговельної палати в Ессені (жовтень 1863 р.)
сталеливарня Круппа за допомогою 161 перетопних, гартівних та цементових
печей, 32 парових машин (1800 р. це було приблизно загальне число
парових машин, що вживалися в Менчестері) та 14 парових молотів, —
які разом репрезентували 1.236 кінських сил, — 49 ковальських горен,
203 виконавчих машин та приблизно 2.400 робітників — випродукувала
1862 р. 13 мільйонів фунтів литої сталі. Тут на одну кінську силу немає
й двох робітників.
} і кожне
автоматичне веретено mule випрядає за десятигодинного робочого
дня 13 унцій пряжі (пересічного нумера), отже, 2 1/2 робітника
випрядають 365 5/8 фунтів пряжі на тиждень. Отже, при своєму
перетворенні на пряжу приблизно 366 фунтів бавовни (для спрощення
ми залишаємо осторонь відпадки) вбирають лише 150 робочих
годин, або 15 десятигодинних робочих днів, тимчасом як із
самопрядом, якщо ручний прядун дає за 60 годин 13 унцій пряжі,
та сама кількість бавовни забрала б 2.700 десятигодинних робочих
днів, або 27.000 робочих годин.\footnote{
Беббедж обчислює, що на Яві самою лише працею прядіння до
вартости бавовни додається майже 117\%. У той самий час (1832) в Англії
загальна вартість, яку при тонкопрядінні додавали до бавовни машини
і праця, становила приблизно 33\% вартости сировинного матеріялу,
(«On the Economy of Machinery», London 1832, p. 214).
} Там, де стару методу
blockprinting, або ручного вибивання перкалю, витиснуло машинове
вибивання, одним-одна машина за допомогою одного чоловіка
або підлітка вибиває за одну годину стільки саме чотирибарвного
перкалю, скільки раніше вибивало 200 чоловіка.\footnote{
Окрім того, при машиновому вибиванні заощаджується фарбу.
}

en même façon à tous les usages auxquels ils sont propres, et ainsi nous
rendre comme maîtres et possesseurs de la nature... contribuer au perfectionnement
de la vie humaine»). У передмові до «Discourses upon Trade»
(1691 p.) сера Дудлея Норта сказано, що метода Декарта, застосована
до політичної економії, почала визволяти її від стародавніх казок і забобонних
уявлень про гроші, торговлю й т. д. Загалом, однак, англійські
економісти давніх часів приєднуються до філософії Бекона й Гоббса, тимчасом
як пізніш «філософом» χατ’ ε’ξοχη'ν\footnote*{
— переважно. Ред.
} політичної економії для Англії,
Франції та Італії став Льокк.