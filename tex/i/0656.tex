кремлення робітників від умов праці, перетворити на одному
полюсі суспільні засоби продукції та засоби існування на капітал,
а на протилежному — народню масу на найманих робітників,
на вільних «працюючих бідняків», — цей витвір мистецтва
сучасної історії.\footnote{
Вислів «labouring poor»\footnote*{
— працюючі бідняки. Peд.
} подибуємо в англійських законах від
часу, коли кляса найманих робітників стає помітна. Labouring poor протистоять,
з одного боку, «idle poor»,\footnote*{
— біднякам-неробам. Peд.
} жебракам і т. ін., з другого боку,
тим робітникам, що не є ще цілком обскубані кури, а є ще власники
своїх засобів праці. Із законів вислів «labouring poor» перейшов
до політичної економії, починаючи від Колпепера, Дж. Чайлда й ін.,
аж до А. Сміса й Ідна. По цьому можна судити, яка bonne foi\footnote*{
— сумлінність. Peд.
} в Едмунда
Берка, цього «execrable political cantononger»\footnote*{
— огидливого політичного крамаря. Peд.
}, коли він вислів
«labouring poor» зве «execrable political cant».\footnote*{
— огидливим політичним перекрученням. Ред.
} Цей сикофант, що,
бувши на утриманні англійської олігархії, відігравав ролю романтика проти
французької революції, так само, як на початку заворушень в Америці
він, бувши на утриманні північно-американських колоній, відігравав
ролю ліберала проти англійської олігархії, в дійсності був наскрізь ординарним
буржуа: «Закони торговлі є закони природи, отже, і закони самого
бога» (Е. Burke: «Thoughts and Details on Scarcity», ed. London
1800, p. 31,32). He диво, що він, вірний законам бога й природи, завжди
продавав себе самого на найкращому ринку! У творах панотця Тукера —
Тукер був піп і торі, але зрештою цілком пристойна людина й путящий
політико-економ — можна знайти дуже гарну характеристику цього
Едмунда Берка за його ліберальних часів. При тій огидливій безхарактерності,
яка панує тепер і побожно вірить у «закони торговлі», треба
знову й знов таврувати таких Берків, що від своїх наступників відрізняються
лише одним — талантом!
} Коли гроші, як каже Ож’є, «родяться на
світ із природними кривавими плямами лише на одній щоці»,\footnote{
Marie Augier: «Du Crédit Public», Paris 1842, p. 265.
}
то капітал, що родиться на світ, прискає кров’ю й брудом від
голови до ніг із усіх своїх пор.\footnote{
«Капітал, — каже «Quarterly Reviewer», — уникає заколотів
і сварок і з природи своєї боязкий. Це цілковита правда, алеж не вся
правда. Капітал боїться відсутности зиску або дуже малого зиску, як
природа боїться порожнечі. При відповідному зиску капітал стає відважним.
При певних 10 процентах його можна вживати повсюди; при 20 процентах
він стає жвавим; при 50 процентах він абсолютно готовий ризикувати;
за 100 процентів він топче ногами всі людські закони; 300 процентів
— і немає такого злочину, що на нього він не ризикнув би, навіть
під загрозою шибениці. Коли заколоти й сварки дають зиск, він заохочує
і до заколотів і до сварок. Докази — контрабанда і торговля рабами».
(Т. J. Dunning: «Trades-Unions and Strikes», London 1860, p. 36).
}

7. Історична тенденція капіталістичної акумуляції

На що ж сходить первісна акумуляція капіталу, тобто його
історична генеза? Оскільки вона не є безпосереднє перетворення
рабів і кріпаків на найманих робітників, отже, не є проста
зміна форми, вона означає лише експропріяцію безпосередніх
продуцентів, тобто розклад приватної власности, основаної на
власній праці.