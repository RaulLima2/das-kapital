3. Еквівалентна форма

Ми бачили, що коли якийсь товар А (полотно) виражає свою вартість у споживній вартості відмінного
від нього товару В (сурдута), то він надає (drückt... auf) цьому останньому специфічної форми
вартости, форми еквіваленту. Товар «полотно» виявляє своє власне вартостеве буття тим, що сурдут, не
набираючи будь-якої іншої форми вартости, відмінної від його тілесної форми, є рівнозначний полотну.
Отже, полотно фактично виражає своє власне вартостеве буття тим, що сурдут є безпосередньо вимінний
на нього. Отже, еквівалентна форма якогось товару є форма його безпосередньої вимінности на інший
товар.

Коли якийсь рід товару, як от сурдут, служить за еквівалент якомусь іншому родові товару, приміром,
полотну, і тому сурдути набирають характеристичної властивости перебувати у формі, безпосередньо
вимінній на полотно, то цим ще аж ніяк не дано ту пропорцію, що в ній сурдути й полотно можуть
обмінюватися між собою. А що величину вартости полотна дано, то пропорція ця залежить від величини
вартости сурдутів. Чи сурдут виражено як еквівалент, а полотно як відносну вартість, чи, навпаки,
полотно як еквівалент, а сурдут як відносну вартість, величина вартости сурдута, як і раніш,
визначається кількістю робочого часу, доконечного для його продукції, отже, вона  визначається
незалежно від його форми вартости. Але скоро тільки рід товару «сурдут» займе місце еквіваленту у
виразі вартости, то величина його вартости не набуває жодного виразу як величина вартости. У
рівнанні вартостей вона фігуруватиме скорше лише як певна кількість даної речі.

Приміром, 40 метрів полотна «варті» — чого? 2 сурдутів. Що рід товару «сурдут» відіграє тут ролю
еквіваленту, що споживна вартість «сурдут» фігурує супроти полотна як тіло вартости, то досить
певної кількости сурдутів, щоб виразити певну кількість вартости полотна. Тому 2 сурдути можуть
виразити величину вартости 40 метрів полотна, але ж ніколи не можуть вони виразити величини своєї
власної вартости, величини вартости сурдутів. Поверхове розуміння цього факту, а саме того, що в
рівнанні вартостей еквівалент завжди має лише форму простої кількости якоїсь речі, якоїсь споживної
вартости, призвело Ваіlеу’а, як і багатьох його попередників і наступників, до тієї помилки, що

не лише його власну вартість проти В, на яке воно обмінюється, а ще й вартість В відносно вартости
А, хоч не сталося ніякої зміни в кількості праці, потрібної для продукції В, тоді падає не лише
доктрина, яка запевняє, що вартість товару регулюється кількістю витраченої на нього праці, але й та
доктрина, за якою витрати продукції якогось товару регулюють його вартість». (J. Broadhurst:
«Political Economy», London 1842, p. 11, 14).

Пан Бродерст міг би так само влучно сказати: пригляньтесь до числових відношень 10/20, 10/50,10/100
і т. д. Число 10 лишається незмінним, а проте його пропорційна величина, його величина щодо
знаменників 20, 50, 100 постійно вменшується. Отже, падає великий принцип, що величину числа,
наприклад, 10, «реґулюється» кількістю одиниць, що є в ньому.
