рювання тих самих оборудок між тими самими особами умови продажу товарів реґулюється умовами їхньої
продукції. З другого боку, використовування деяких родів товарів, наприклад, будинку, продається на
якийсь певний час. Лише по скінченні терміну покупець фактично дістає споживну вартість товару. Тому
він купує товар раніш, ніж сплатить за нього. Один посідач товарів продає наявний товар, а другий
купує лише як представник грошей або як представник майбутніх грошей. Продавець стає кредитором,
покупець — винуватцем. А що метаморфоза товару або розвиток його форми вартости тут змінюється, то й
гроші набувають іншої функції. Вони стають засобом платежу.\footnote{
Лютер відрізняє гроші як засіб купівлі й гроші як засіб платежу:
«Ти робиш мені подвійну шкоду тим, що в одному місці не можу платити,
а в другому не можу купити» («Machest mir einen Zwilling aus dem Schadewacht,
das ich hier nicht bezahlen und dort nicht kauffen kann»). (Martin Luther:
«An die Pfarrherrn, wider den Wucher zu predigen», Wittenberg 1540).
}

Те, що діяч циркуляції стає кредитором або винуватцем, випливає тут із простої товарової циркуляції.
Зміна її форми накладає ці нові печаті на продавця й покупця. Отже, спочатку це ролі так само минущі
й так само навперемінки виконувані тими самими аґентами циркуляції, як і ролі покупця й продавця.
Однак, тепер ця
протилежність вже від самого початку виглядає менш добродушною, і вона здатна до більшої
кристалізації.\footnote{
Про відносини між кредитором і винуватцем серед англійських
купців на початку XVIII віку: «Серед людей торговлі панує тут, в Англії,
такий дух жорстокости, якого не зустрінеш ні в якому іншому суспільстві
або іншому королівстві світу» («Such a spirit of cruelty reigns here
in England among the men of trade, that is not to be met with, in any other
society of men, nor in any other kingdom of the world»). («An Essay оn Credit and the Bankrupt Act.
London», 1707, p. 2).
} Але ці самі відносини можуть виникати й незалежно від товарової циркуляції.
Приміром, клясова боротьба в античному світі рухається, головне, у формі боротьби між кредитором і
винуватцем і кінчається в
Римі загином винуватця-плебея, місце якого заступає раб. У середні віки боротьба ця кінчається
занепадом винуватця-февдала, що втрачає свою політичну владу разом з її економічною базою. Однак за
цих епох грошова форма — а відношення між
кредитором і винуватцем має форму грошового відношення — лише
відбиває антагонізм глибших економічних умов життя.

Вернімось до сфери товарової циркуляції. Одночасного з’явлення еквівалентів, товару і грошей, на
обох полюсах процесу продажу вже не відбувається. Гроші функціонують тепер, поперше, як міра
вартости при визначенні ціни продаваного товару. Установлена контрактом ціна його міряє зобов’язання
покупця, тобто ту грошову суму, яку він винен на певний термін. Вони функціонують, подруге, як
ідеальний засіб купівлі. Хоч вони існують лише як обіцянка покупця сплатити гроші, все ж вони
зумовлюють перехід товарів із рук до рук. Лише коли настає платіжний термін, засіб платежу дійсно
вступає в циркуляцію, тобто з рук покупця переходить до рук продавця. Засіб цирку-