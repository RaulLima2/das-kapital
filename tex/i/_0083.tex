\parcont{}  %% абзац починається на попередній сторінці
\index{i}{0083}  %% посилання на сторінку оригінального видання
світу отже, ту кількість золота, що тільки й може бути репрезентована.
Коли, приміром, маса папірців репрезентує: кожний —
2 унції золота замість 1 унції, то фактично 1 фунт стерлінґів
стає грошовою назвою, приміром, \sfrac{1}{8} унції замість \sfrac{1}{4} унції.
Наслідок є такий самий, як коли б золото змінилося у своїй
функції міри цін. Тому ті самі вартості, що раніш виражались
у ціні 1 фунта стерлінґів, виражаються тепер у ціні 2 фунтів
стерлінґів.

Паперові гроші є знак золота або знак грошей. Їхнє відношення
до товарових вартостей є лише в тому, що останні ідеально виражені
в тій самій кількості золота, яку символічно-почуттєво
репрезентують папірці. Паперові гроші лише остільки є знак
вартости, оскільки вони репрезентують кількості золота, які,
як і всякі інші товарові кількості, є також кількості вартости.\footnote{
Примітка до другого видання. Як неясно розуміють різні функції
грошей навіть найкращі письменники у грошовій справі, показує, приміром,
ось яке місце з Фуляртона: «Щодо нашої внутрішньої торговлі, то
всі ті грошові функції, які звичайно виконують золоті й срібні монети,
може з однаковим успіхом виконати циркуляція нерозмінних банкнот,
що не мають іншої вартости, крім штучної й умовної вартости, якої їм
надає закон; це факт, якого, я думаю, ніхто не заперечуватиме. Вартість
такого роду цілком відповідала б усім призначенням унутрішньої вартости
й навіть усунула б потребу особливого маштабу вартости, коли б лише
кількість білетів, що їх випускається в циркуляцію, не переступала
певних меж». («That, as far as concerns our domestic exchanges, all the
monetary functions which are usually performed by gold and silver coins,
may be performed as effectually by a circulation of inconvertible notes,
having no value but that factitious and conventional value they derive
from the law, is a fact, which admits; I conceive, of no denial. Value of
this description may be made to answer all the purposes of intrinsic value,
and supersede even the necessity for a standard, provided only the quantity
of issues be kept under due limitation»). (\emph{Fullarton}: «Regulation of
Currencies», 2 ed., London 1845, p. 21). Отже, через те, що грошовий
товар може бути заміщений у циркуляції простим знаком вартости, він
є зайвий як міра вартостей і як маштаб цін!
}

Нарешті, запитаймо, чому золото може бути замінене простими
знаками його самого, що не мають жодної вартости? Та, як ми
бачили, воно може бути замінене лише остільки, оскільки воно
у своїй функції монети або засобу циркуляції ізолюється або
усамостійнюється. Правда, усамостійнення цієї функції не відбувається
для поодиноких золотих монет, хоч воно й виявляється
в дальшій циркуляції стертих золотих монет. Кусники золота
є монети або засоби циркуляції саме лише доти, доки вони дійсно
перебувають в обігу. Але те, що немає сили для поодиноких золотих
монет, має силу для тієї мінімальної маси золота, яку можна
замінити паперовими грішми. Вона постійно живе у сфері циркуляції,
безперестанно функціонує як засіб циркуляції, а тому
існує виключно як носій цієї функції. Отже, її рух являє собою
лише невпинне перетворювання одного в один протилежних процесів
товарової метаморфози $Т. — Г — Т$, що в ній супроти товару
стає форма його вартости на те тільки, щоб негайно знову зникнути.
Самостійний вираз мінової вартости товару є тут лише
\parbreak{}  %% абзац продовжується на наступній сторінці
