ний поділ праці протиставить їм духовні потенції матеріяльного
продукційного процесу як чужу власність та силу, що опановує
їх. Цей процес відокремлення починається у простій кооперації,
де капіталіст репрезентує супроти поодиноких робітників єдність
і волю суспільного робочого тіла. Він розвивається в мануфактурі,
яка калічить робітника, перетворюючи його на частинного
робітника. Він завершується у великій промисловості, яка відокремлює
від праці науку як самостійний фактор продукції та
примушує її служити капіталові.67

В мануфактурі збагачення колективного робітника, а тому
й капіталу, на суспільну продуктивну силу зумовлено збіднінням
робітника на індивідуальні продуктивні сили. «Неуцтво є
мати промисловости, як і забобонів. Розумування й фантазія
підпадають помилкам; але звичка рухати ногою або рукою не
залежить ні від одного, ні від другого. Отже, мануфактури процвітають
найкраще там, де найбільше можна звільнитися від
інтелектуальної роботи, так що майстерня може бути розглядувана
як машина, що її частинами є люди».68 Справді, в середині
XVIII віку деякі мануфактури вживали охоче напівідіотів,
щоб виконувати деякі прості операції, які, однак, становили
фабричні таємниці.69

«Інтелект великої більшости людей, — каже А. Сміс, — неминуче
розвивається з їхніх щоденних заняттів і через ці заняття.
Людина, що витратила ціле своє життя на виконування небагатьох
простих операцій... не має нагоди вправляти свій розум... Вона
взагалі стає такою тупою і темною, якою тільки й може стати
людська істота». Змалювавши туподумство частинного робітника,
Адам Сміс каже далі: «Одноманітність його життя без усяких
змін губить, ясна річ, і жвавість його розуму... Вона руйнує
навіть енерґію його тіла та робить його нездатним уживати напружено
й довгочасно своєї сили, хіба лише в тій частинній
праці, до якої його привчили. Таким чином його вправність у його
осібному реместві є, здається, властивість, що він її набуває коштом
своїх інтелектуальних, соціальних та військових здібностей.
Але саме такий є в кожному промисловому й цивілізованому
суспільстві той стан, що в ньому неминуче мусить опинитись
кожний працюючий бідняк (the labouring poor), тобто велика маса
народу».70 Щоб перешкодити тому повному знидінню народньої

67 «Людина науки й продуктивний робітник дуже віддалені один від
одного, і наука замість збільшувати в руках робітника його власні продуктивні
сили для нього самого, майже скрізь поставила себе проти нього.
Знання стає знаряддям, що його можна відокремити від праці та їй протипоставити».
(W. Thompson: «An Inquiry into the Principles of the
Distribution of Wealth», London 1824, p. 274).

68 A. Ferguson: «History of Civil Society», Edinburgh 1767,
part IV, sect. I, p. 280.

69 J. D. Tuckett: «А History of the Past and Present State of
the Labouring Population», London 1846, vol. I, p. 148.

70    A. Smith: «Wealth of Nations», b. V, ch. I, art. II. p. 140, 141.
Як учень А. Ферґюсона, який розвинув усі шкідливі наслідки поділу
