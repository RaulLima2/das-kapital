\parcont{}  %% абзац починається на попередній сторінці
\index{i}{0141}  %% посилання на сторінку оригінального видання
не створила додаткової вартости, отже, гроші не перетворилися
на капітал. Ціна 10 фунтів пряжі дорівнює 15\shil{ шилінґам}, і 15\shil{ шилінґів}
витрачено на товаровому ринку на елементи утворення
продукту або, що те саме, на фактори процесу праці: 10\shil{ шилінґів}
на бавовну, 2\shil{ шилінґи} на зужитковану кількість веретен і
З шилінґи на робочу силу. Набубнявіння вартости пряжі нічого
не помагає, бо вартість її — то лише сума вартостей, що раніш
розподілялись між бавовною, веретенами й робочою силою, а з
такого простого додавання наявних вартостей ніколи не може
виникнути додаткова вартість\footnote{
Це є основна теза, на якій ґрунтується теорія фізіократів про
непродуктивність усякої нерільничої праці, і вона незаперечна для економіста
— з фаху. «Цей спосіб прираховувати до одної речі вартість багатьох
інших (наприклад, до полотна — витрати на споживання ткача), нашаровувати,
так би мовити, на одну вартість декілька інших вартостей, призводить
до того, що вона відповідно до цього зростає. Термін «додавання» дуже
добре змальовує той спосіб, яким утворюється ціна продуктів праці; ця
ціна є лише сума багатьох вартостей, спожитих і доданих одна до однієї
але додавати — це не множити» («Cette façon d’imputer à une seule chose
la valeur de plusieurs autres (par exemple au lin la consommation du tisserand),
d’appliquer, pour ainsi dire, couche sur couche, plusieurs valeurs
sur une seule, fait que celle-ci grossit d’autant\dots{} Le terme d’addition peint trèsbien
la manière dont se forme le prix des ouvrages de main d’oeuvre; ce prix
n’est qu’un total de plusieurs valeurs consommées et additionnées ensemble;
or, additionner n’est pas multiplier»), (\emph{Mercier de la Rivière}: «L’Ordre naturel
et essentiel etc.», Physiocrates, éd. Daire, II Partie, p. 599).
}. Всі ці вартості сконцентровано
тепер в одній речі, але вони так само були сконцентровані в грошовій
сумі в 15\shil{ шилінґів} раніш, ніж подробилась вона через
купівлю трьох товарів.

Сам по собі цей результат не є щось дивне. Вартість 1 фунта
пряжі є 1\shil{ шилінґ} 6\pens{ пенсів}, і тим то за 10 фунтів пряжі наш капіталіст
мусив би заплатити на товаровому ринку 15\shil{ шилінґів.}
Чи він купить для себе готовий будинок на ринку, чи ставитиме
його сам, — жодна з цих операцій не збільшить грошей витрачених
на придбання будинку.

Капіталіст, що розуміється на вульґарній економії, скаже,
може, що він авансував свої гроші з наміром зробити з них більше
грошей. Але ж і шлях до пекла вимощено добрими намірами, і в
капіталіста так само міг бути намір добувати гроші, не продукуючи\footnote{
Так, приміром, за 1844--47 pp. він вилучив із продуктивних підприємств
частину свого капіталу, щоб проспекулювати його на залізничних
акціях. Так, за часів американської громадянської війни він позамикав
фабрики й викинув на брук фабричних робітників, щоб грати па ліверпулській
бавовняній біржі.
}. Він загрожує. Удруге його вже
не зловлять. На майбутнє
він купуватиме на ринку готові товари замість їх самому
фабрикувати. Коли ж усі його брати-капіталісти зроблять те
саме, то де ж він тоді знайде товари на ринку? А грошей їсти
він не може. Він починає повчати нас катехизису. Слід би мати
на увазі його поздержливість. Адже він міг протринькати своїх
15\shil{ шилінґів.} Замість того він продуктивно спожив їх і зробив
з них пряжу. Але зате ж бо тепер у нього є пряжа замість докорів
\index{i}{0142}  %% посилання на сторінку оригінального видання
сумління. Йому ні в якому разі не слід вертатись до ролі
збирача скарбу, який показав нам, що виходить з аскетизму.
А, крім того, де нічого немає, там і імператор втратив своє право.
Хоч і яка була б заслуга його поздержливости, але ж нема нічого,
чим можна було б зокрема оплатити її, бо вартість продукту,
який виходить із процесу, дорівнює лише сумі товарових вартостей,
що їх кинуто до процесу. Отже, нехай же він заспокоїться
на тому, що чеснота є нагорода за чесноту. Та замість того капіталіст
робиться настирливим. Пряжа йому непотрібна. Він продукував
її на продаж. Так нехай продасть її або, ще простіш, на
майбутнє нехай продукує лише речі для своєї власної потреби, —
рецепт, який йому приписав був уже його домашній лікар Мак
Кулох як випробуваний лік проти епідемії перепродукції. Капіталіст
уперто стає диба. Хіба ж робітник міг би своїми лише
десятьма пальцями творити у блакитному повітрі, продукувати
товари з нічого? Хіба ж не він, не капіталіст, дав йому матеріял,
через який і в якому робітник лише й може втілити свою працю?
А що більша частина суспільства складається з таких голяків,
то хіба він своїми засобами продукції, своєю бавовною і своїми
веретенами, не зробив незмірної послуги суспільству, не зробив
послуги самому робітникові, якого він, окрім того, ще й забезпечив
засобами існування? І чи не слід йому зарахувати це як заслугу?
А робітник — хіба він йому не відплативсь послугою, перетворивши
бавовну й веретена на пряжу? Та, крім того, тут річне в
послугах\footnote{
«Хвались, прибирайся й чепурись\dots{} Але хто бере
більше або краще (ніж він дає), той лихвар, і це значить, що не послугу,
а шкоду заподіяв він своєму ближньому, як це буває підчас злодійства й грабунку.
Не все те послуга й добродійство для ближнього, що зветься послугою й
добродійством. Бо перелюбниця і перелюбник роблять один одному велику
послугу і втіху. Райтер\footnote*{
-військовий, кінник. \emph{Ред.}
}
робить убивцеві-палієві велику райтерську
послугу, допомагаючи йому грабувати по шляхах, нападати на маєтки і
людей. Папісти роблять нашим велику послугу, бо вони не всіх топлять,
палять, забивають, примушують гнити по тюрмах, але декого лишають
живим та виганяють їх або забирають у них усе, що вони мають. Сам чорт
робить своїм прислужникам велику незмірну послугу\dots{} Словом, світ
повен великих, удалих щоденних послуг і добродійств. (\emph{Martin Luther}:
«An die Pfarherrn, wider den Wucher zu predigen usw.», Wittenberg 1540).
}. Послуга — це не що інше, як корисний ефект споживної
вартости, чи то товару, чи то праці\footnote{
З приводу цього я, між іншим, зауважую в «Zur Kritik der Politischen
Oekonomie», Berlin 1859, S. 14. («До критики\dots{}», ДВУ, 1926~\abbr{р.},
crop. 55): «Зрозуміло, яку «послугуо мусить робити категорія «послуг»
(service) економістам такого сорту, як Ж. Б. Сей і Ф. Бастіа».
}. Алеж тут ідеться
про мінову вартість. Капіталіст виплатив робітникові вартість
у 3\shil{ шилінґи.} Робітник повернув йому точний еквівалент, додавши
до бавовни вартість у 3\shil{ шилінґи}, дав йому вартість за вартість.
Наш приятель, який тільки-по так пишався своїм капіталом,
раптом набирає безпретенсійної постаті свого власного робітника.
Хіба ж він сам не працював? Не виконував праці догляду, нагляду
над прядуном? Хіба ж ця його праця не утворює вартости?
Але тут його власний головний директор і його управитель знизують
\index{i}{0143}  %% посилання на сторінку оригінального видання
плечима. А тимчасом він з веселою усмішкою вже знову прийняв
свою попередню фізіономію. Він просто морочив нам голову
всіма цими докучливими жаліннями. Він не дасть за цей півшага.
Він полишає ці й подібні пусті виверти й беззмістовні викрути
професорам політичної економії, яким за це власне й платять.
Сам він практична людина, яка, правда, не завжди думає над тим,
що говорить поза своїми справами, алеж завжди знає, що робить
у діловій сфері.

Пригляньмося ближче до справи. Денна вартість робочої
сили становила 3\shil{ шилінґи}, бо в ній самій упредметнено пів робочого
дня, тобто, тому що засоби існування, щоденно потрібні
для продукції робочої сили, коштують пів робочого дня. Але
минула праця, що міститься в робочій силі, і жива праця, яку
вона може виконати, щоденні кошти її утримання й її щоденне
витрачання, це — дві цілком різні величини. Перша визначає
її мінову вартість, друга становить її споживну вартість. Та
обставина, що для підтримання життя робітника протягом 24 годин
потрібно лише пів робочого дня, аж ніяк не заважає йому
працювати цілий день. Отже, вартість робочої сили та використовування
її в процесі праці є дві різні величини\footnote*{
У французькому виданні це речення зредаговано так: «Отже, вартість,
що її має робоча сила і вартість, яку вона може створити, є дві різні
величини» («La valeur que la force de travail possède et la valeur qu’elle
peut créer, different donc de grandeur»), \emph{Peд}.
}. Саме цю
ріжницю вартостей капіталіст, купуючи робочу силу, і мав на
увазі. Корисна властивість робочої сіяли робити пряжу або чоботи
була лише condicio sine qua non\footnote*{
Доконечна умова, тобто умова, що без неї певне явище не може відбутися.
\emph{Ред.}
}, бо для утворення вартости
праця мусить бути витрачена в корисній формі. Але вирішальне
значення мала специфічна споживна вартість цього товару, його
властивість бути джерелом вартости й вартости більшої, ніж має
він сам. Це і є та специфічна послуга, якої сподівається від нього
капіталіст. І він поводиться тут згідно з вічними законами товарового
обміну. Справді, продавець робочої сили, як і продавець
кожного іншого товару, реалізує її мінову вартість і відчужує її
споживну вартість. Він не може одержати першої, не віддаючи
другої. Споживна вартість робочої сили, сама праця, так само
не належить її продавцеві, як споживна вартість проданої олії —
торговцеві олією. Посідач грошей оплатив денну вартість робочої
сили; тому йому належить споживання її протягом дня, денна
праця. Та обставина, що денне утримання робочої сили коштує
лише пів робочого дня, хоч робоча сила може діяти, працювати
цілий день, тобто та обставина, що вартість, яку створює споживання
робочої сили протягом дня, удвоє більша, ніж її власна
денна вартість, є особливе щастя для покупця, але аж ніяк не
є кривда проти продавця.

Наш капіталіст передбачав цей казус, що саме й викликав
його усмішку. Тому робітник находить у майстерні потрібні
\parbreak{}  %% абзац продовжується на наступній сторінці
