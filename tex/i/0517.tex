жати, коли б не прийшли на допомогу нововинайдені машини, що
й перепряли вовну».\footnote{
F. Engels: «Lage der arbeitenden Klasse in England», Leipzig
1845, S. 20. (Ф. Енгельс: «Становище робітничої кляси в Англії», Партвидав
«Пролетар», 1932 р. стор. 62, 63.).
} Упредметнена у формі машин праця,
певна річ, не створила безпосередньо жодного робітника, але
вона дала змогу невеликому числу робітників через додаток
порівняно невеликої кількости живої праці не тільки продуктивно
спожити вовну й додати до неї нову вартість, а й зберегти її стару
вартість у формі пряжі й т. ін. Тим самим вона разом з тим дала
засіб і імпульс до поширеної репродукції вовни. Це є природна
властивість живої праці — зберігати стару вартість, створюючи
нову вартість. Тому із зростом дієздатности, розміру й вартости
засобів продукції, отже, з акумуляцією, яка супроводить розвиток
продуктивної сили праці, праця зберігає й увіковічнює в
вавжди нових формах чимраз більшу й більшу капітальну вартість.\footnote{
Клясична політична економія через недостатню аналізу процесу
праці й процесу зростання вартости ніколи не розуміла гаразд цього
важливого моменту, репродукції, як це можна, приміром, бачити у Рікарда.
Він каже, наприклад: хоч яка буде зміна продуктивної сили,
«мільйон людей продукує на фабриках завжди ту саму вартість». Це
правда, коли дано екстенсивність і ступінь інтенсивности їхньої праці.
Але це не перешкоджає — і Рікардо недобачає цього в деяких своїх висновках
— тому, що за різної продуктивної сили своєї праці мільйон людей
перетворює на продукт дуже різні маси засобів продукції, а тому й зберігає
у своєму продукті дуже різні маси вартости, отже, вартості виготовлених
ними продуктів є дуже різні. Між іншим, треба сказати, що на цьому
прикладі Рікардо даремно силкувався пояснити Ж. Б. Сеєві ріжницю
між споживною вартістю (яку він називає тут wealth, речовим багатством)
і міновою вартістю. Сей відповідає: «Щодо тих труднощів, які
зазначає Рікардо, кажучи, що мільйон людей, вживаючи вдосконаленіших
способів продукції, може спродукувати вдвоє, утроє більше багатств,
не продукуючи більше вартости, то ці труднощі зникнуть, коли ми, як
це й слід, розглядатимемо продукцію як обмін, в якому віддають продуктивні
послуги своєї праці, своєї землі і своїх капіталів, щоб одержати
за це продукти. За допомогою цих продуктивних послуг ми дістаємо всі
продукти, що є на світі... Отож... ми є тим багатші, наші продуктивні
послуги мають тим більшу вартість, чим більшу кількість корисних
речей ми дістаємо за ці послуги в обміні, називаному продукцією».
«Quant à la difficulté qu’élève Mr. Ricardo en disant que, par des procédés
mieux entendus, un million de personnes peuvent produire deux fois,
trois fois autant de richesses, sans produire plus de valeurs, cette diffuculté
n’en est pas une lorsque l’on considère, ainsi qu’on le doit, la production
comme un échange dans lequel on donne les services productifs de son travail,
de sa terre, et de ses capitaux, pour obtenir des produits. C’est par
le moyen de ces sevrices productifs que nous acquérons tous les produits
qui sont au monde... Or... nous sommes d’autant plus riches, nos services
productifs ont d’autant plus de valeur, q’uils obtiennent dans l’échange
appelé production, une plus grande quantité de choses utiles»). (J. B. Sag:
«Lettres à M. Maithus» Paris 1820, p. 168, 169). «Труднощі», які має
вияснити Сей, — вони існують для нього, а не для Рікарда, — є ось у
чому: чому не збільшується вартість споживних вартостей, коли зростав
їхня кількість у наслідок підвищення продуктивної сили праці? Відповідь:
труднощі розв’язується тим, що споживну вартість будемо ласкаві
називати міновою вартістю. Мінова вартість є річ, що так або інакше
(one way or another) зв’язана з обміном. Отже, назвімо продукцію «обмі-
} Ця природна сила праці здається силою самозбереження