\parcont{}  %% абзац починається на попередній сторінці
\index{i}{0223}  %% посилання на сторінку оригінального видання
обставини, заробітну плату зменшено щонайменше на 25\%.\footnote{
«Я переконався, що в людей, які діставали 10 шилінґів тижнево,
скорочено плату на 1 шилінґ у наслідок загального зменшення заробітної
плати на 10\%, а потім іще на 1 шилінґ 6 пенсів через скорочення часу,
разом на 2 шилінґи 6 пенсів, і, не зважаючи на це, більшість твердо обстоювала
десятигодинний біл». (Там же).
}
За таких, так сприятливо підготованих обставин розпочалась
аґітація серед робітників за скасування закону з 1847 р. Не гребували
жодним з тих засобів, що їх можуть дати омана, спокуса,
погрози, — та все було даремно. Щодо півтузіня петицій, в яких
робітники мусили нарікати на те, «що їх утискує цей закон», то
сами ж прохачі на усному допиті заявили, що ті підписи від них
вимушено. «Їх утискує, але хтось інший, а не фабричний закон».\footnote{
«Коли я підписував прохання, то я одночасно заявив, що роблю
цим щось погане. — Так чого ж ви його тоді підписували? — Тому, що коли б
я спротивився, мене б були викинули на вулицю. — Прохач дійсно почував
«себе утисненим», але зовсім не фабричним законом». (Там же, стор. 102).
}
Але якщо фабрикантам не пощастило примусити робітників
говорити в бажаному для них тоні, то тим голосніш вони сами
кричали від імени робітників у пресі і в парляменті. Вони виставляли
фабричних інспекторів як щось на зразок комісарів конвенту,
які без милосердя приносять нещасних робітників у жертву
своїй химері про поліпшення світу. Але й цей маневр не мав
успіху. Фабричний інспектор Леонард Горнер особисто й через
своїх підінспекторів паназбирав численні свідчення на фабриках
Ланкашіру. Якихось 70\% переслуханих робітників висловилося
за 10-годинний робочий день, куди менший відсоток за 11-годинний
і зовсім незначна меншість за старий 12-годинний день.\footnote{
Там же, стор. 17. Таким чином в окрузі пана Горнера переслухано
10.270 дорослих робітників-чоловіків на 181 фабриці. Їхні свідчення
можна найти в додатку до фабричного звіту за півріччя, що кінчається
жовтнем 1848 р. Ці свідчення дають і з іншого боку цінний матеріял.
}

Інший «полюбовний» маневр був такий: примусити дорослих
чоловіків-робітників працювати 12--15 годин, а потім оголосити
цей факт за найкращий вислів щиро пролетарських бажань.
Але «немилосердий» фабричний інспектор Леонард Горнер знову
вже був на своєму місці. Більшість робітників, що працювали
«надмірний час», висловилася, «що їм куди краще було б працювати
по 10 годин за меншу заробітну плату, але для них не було
жодного вибору: серед них так багато безробітних, так багато
прядунів, примушених працювати як звичайні ріесеrs,\footnote*{
— присукальники. \emph{Ред.}
} що
коли б вони відмовилися від здовження робочого дня, так зараз
на їхні місця прийшли б інші, так що для них, мовляв, справа
стоїть так: або працювати довший час, або опинитись на вулиці».\footnote{
Там же. Див. зібрані від самого Леонарда Горнера свідчення
№№ 69, 70, 71, 72, 92, 93 і ті, що зібрав підінспектор А., №№ 51, 52,
58, 59, 62, 70 в «Додатку». Навіть один фабрикант висловився занадто
ясно. Див. № 14 після № 265, там же.
}

Попередній похід капіталу скінчився невдало, і закон про десятигодинний
робочий день набув чинности 1 травня 1848 р. Тимчасом
поразка чартистської партії, що її проводирів позамикано
\parbreak{}  %% абзац продовжується на наступній сторінці
