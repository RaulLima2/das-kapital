\index{i}{*0083}  %% посилання на сторінку оригінального видання
Дальший час, від 1820 до 1830 рр., відзначається в Англії науковим
оживленням у царині політичної економії. Це був період
так вульґаризації та поширення рікардівської теорії, як і боротьби
її зі старою школою. Відсвятковувано блискучі турніри.
Що тоді було зроблено, мало відомо на європейському континенті,
бо полеміку цю здебільшого порозкидувано по журнальних
статтях, випадкових брошурах та памфлетах. Безсторонній характер
цієї полеміки — хоч теорія Рікарда винятково вже й тоді
служила за зброю нападу на буржуазне господарство — пояснюється
обставинами того часу. З одного боку, велика промисловість
сама тільки но вийшла із свого дитячого віку, як це видно з того,
що вона лише від часу кризи 1825 р. починає періодичний кругобіг
свого сучасного життя. З другого боку, клясову боротьбу між
капіталом і працею було відсунуто на задній плян: політично —
боротьбою урядів і февдалів, які згуртувалися біля Священного
Союзу проти народніх мас, керованих буржуазією, економічно —
ворожнечею між індустріяльним капіталом та аристократичною
землевласністю — ворожнечею, що у Франції ховалася за противенством
між дрібною й великою земельною власністю, а в Англії
від часів хлібних законів вибухла відверто. Англійська література
в політичній економії цього часу нагадує період бурі й натиску
у Франції після смерти доктора Кене, але лише так, як
бабине літо нагадує весну. 1830 р. настала криза, що раз назавжди
вирішила справу.

Буржуазія у Франції і в Англії завоювала політичну владу.
З цього часу клясова боротьба, практично й теоретично, набирає
дедалі виразніших і загрозливіших форм. Вона задзвонила
на смерть науковій буржуазній економії. Ішлося тепер уже
не про те, правдива чи неправдива та або інша теорема, а про те,
корисна чи шкідлива вона капіталові, догідна чи недогідна, принятна
для поліції, чи ні. Місце безкорисливого досліду заступають
сутички платних писак, місце безсторонніх наукових розслідів —
нечисте сумління та лихі наміри апологетики. Тимчасом навіть
настирливі трактатці, що їх видавала Anti-Cornlawleague\footnote*{
— союз для боротьби з хлібними законами. \emph{Ред.}
} з
фабрикантами Кобденом і Брайтом на чолі, все ж таки являли
собою якщо не науковий, так хоч історичний інтерес своєю полемікою
проти землевласницької аристократії. Але й це останнє
жало вульґарної економії вирвало у неї фритредерське законодавство
від часів сера Роберта Піла.

Континентальна революція 1848 р. відбилась також і на Англії.
Люди, які претендували ще на наукове значення й хотіли бути
чимось більшим, ніж тільки софістами й сикофантами панівних
кляс, намагались погодити політичну економію капіталу з вимогами
пролетаріяту, яких уже не можна було тепер далі іґнорувати.
Звідси нездарний синкретизм, що найкращим представником
його є Джон Стюарт Мілл. Це — проголошення банкрутства
«буржуазної» економії, яке майстерно висвітлив уже великий
\parbreak{}  %% абзац продовжується на наступній сторінці
