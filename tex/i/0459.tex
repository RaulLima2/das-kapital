екстенсивної величини праці зросла її інтенсивна величина.\footnote{
«Заробітні плати залежать від ціни праці та від кількости виконаної
роботи. Збільшення плати за працю не означає ще неодмінно зросту
ціни праці. Із збільшенням робочого часу і при більшому напруженні
плата за працю може значно зрости, тимчасом як ціна праці може лишитися
незмінною». («The wages of labour depend upon the price of labour
and the quantity of labour performed... An increase in the wages of
labour does not necessarily imply an enhancement of the price of labour.
From fuller employment, and greater exertions, the wages of labour may
be considerably increased, while the price of labour may continue the same»).
(West: «Price of Corn and Wages of Labour», London. 1826, p. 67, 68, 112).
Головного питання: як визначається «ціну праці», Вест, між іншим,
збувається банальними фразами.
}
Тому підвищення номінальної поденної або потижневої заробітної
плати може супроводитися ціною праці, що лишається
незмінною або падає. Те саме має силу і щодо доходу робітничої
родини, скоро тільки кількість праці, давана головою сем’ї,
збільшиться працею членів родини. Отже, існують методи понижати
ціну праці, незалежні від зменшення номінальної поденної
або потижневої заробітної плати.\footnote{
Це правильно відчуває фанатичний представник промислової
буржуазії XVIII віку, не раз цитований нами автор «Essay on Trade
and Commerce», хоч він з’ясовує справу плутано: «Кількість праці,
а не ціна її (під останньою він розуміє номінальну поденну або потижневу
заробітну плату) визначається ціною харчових та інших доконечних засобів
існування. Коли ви дуже понизите ціну засобів існування, то ви
тим самим зменшите у відповідній пропорції кількість праці... Фабриканти
знають, що існують різні шляхи підвищувати та знижувати ціну
праці, незалежно від зміни її номінального розміру». («It is the quantity
of labour and not the price of it, that is determined by the price of provisions
and other necessaries: reduce the price of necessaries very low, and
of course you reduce the quantity of labour in proportion... Master-manufacturers
know, that there are various ways of raising and felling the
price of labour, besides that of altering its nominal amount»). (Там же,
стор. 48 і 61). Н. В. Сеніор у своїх «Three Lectures on the Rate of Wages»,
London 1830, де він використовує твір Веста, не покликаючись на
нього, між іншим, каже: «Робітник головним чином заінтересований
у височині заробітної плати» («The labourer is principally interested
in the amount of wages») (p. 14). Отже, робітник заінтересований головним
чином у тому, що він одержує — в номінальній сумі заробітної плати,
а не в тому, що він дає — не в кількості праці!
}

Але звідси випливає такий загальний закон: якщо дано кількість
денної, тижневої і так далі праці, то денна або тижнева
заробітна плата залежить від ціни праці, яка сама змінюється
або із зміною вартости робочої сили, або з відхиленням ціни робочої
сили від її вартости. Навпаки, якщо дано ціну праці, то денна
або тижнева заробітна плата залежить від кількости денної або
тижневої праці.

Одиниця міри почасової плати, ціна робочої години, є результат
від ділення денної вартости робочої сили на число годин звичайного
робочого дня. Припустімо, що робочий день становить
12 годин, денна вартість робочої сили — 3 шилінґи, тобто вартість,
продуковану за 6 робочих годин. Ціна робочої години за
цих умов є 3 пенси, спродукована протягом неї вартість — 6 пен-