2. Хибне розуміння в політичній економії репродукції
в поширеному маштабі

Перше ніж перейти до деяких докладніших визначень акумуляції
або зворотпого перетворення додаткової вартости на капітал,
треба усунути двозначність, вигадану клясичною економією.

Так само як товари, що їх капіталіст купує для свого власного
споживання за якусь частину додаткової вартости, не служать
йому за засоби продукції і зростання вартости, так само і праця,
яку він купує, щоб задовольняти свої природні й соціяльні
потреби, не є продуктивна праця. Замість, купуючи ті товари
і працю, перетворювати додаткову вартість на капітал, капіталіст,
навпаки, споживає або витрачає її як дохід. Всупереч
старошляхетському принципові, що, як слушно каже Геґель,
«полягає у споживанні того, що є в наявності» і особливо яскраво
виявляється в розкошах особистих послуг, для буржуазної економії
мало вирішальну вагу оповістити акумуляцію капіталу за
перший обов’язок громадянина і невтомно проповідувати таке:
не можна акумулювати, якщо проїдати цілий свій дохід замість
чималу частину його витрачати на вербування додаткових продуктивних
робітників, що дають більше, ніж коштують. З другого
боку, політичній економії треба було боротися з народнім
забобоном, що сплутує капіталістичну продукцію з скарботворенням
28 і тому гадає, ніби нагромаджене багатство є багатство, захищене
від руїни в його наявній натуральній формі, отже, вилучене
із споживання або врятоване і від циркуляції. Замикати гроші
та не пускати їх у циркуляцію — це було б методою, якраз протилежною
перетворенню їх на капітал, а нагромаджувати товари
в розумінні скарботворення — було б чистим безглуздям.28а
Акумуляція товарів у великих масах є результат застою циркуляції
або перепродукції.29 Проте в народній уяві постає, з одного
боку, картина дібр, нагромаджених у споживному фонді багатіїв
та повільно споживаних, а з другого боку, творення запасів —

28 «Жоден політико-економ нашого часу не може під заощадженням
розуміти лише скарботворення; але, поза цією обмеженою й недостатньою
операцією, не можна собі уявити іншого значення даного
вислову щодо народнього багатства, як лише того, що випливає з різного
вжитку заощадженого і є основане на реальній ріжниці між різними
родами праці, утримуваними коштом заощаджень» («No political
economist of the present day can by saving mean mere hoarding: and
beyond this contracted and inefficient proceeding, no use of the term in reference
to the national wealth can well be imagined, but that which must
arise from a different application of what is saved, founded upon a real
distinction between the different kinds of labour maintained by it»).
(Malthus: «Principles of Political Economy», p. 38, 39).

28а Так, у Бальзака, що так ґрунтовно вивчив усі відтінки скупости,
старий лихвар Ґобзек змальовується уже здитинілим, коли він починав
збирати собі скарб із нагромаджених товарів.

29 «Акумуляція капіталу... припинення обміну... перепродукція».
(«Accumulation of stocks... non-exchange... overprodustion»). (Th. Cor.
bet: «An Inquiry into the Causes and Modes of the Wealth of Individuals» London
1841, p. 14).
