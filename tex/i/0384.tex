8. Революціонізування мануфактури, ремества та домашньої
праці великою промисловістю

а) Знищення кооперації, основаної
на реместві й поділі праці

Ми бачили, як машини знищують кооперацію, основану на
реместві, та мануфактуру, основану на поділі ремісничої
праці. Прикладом першого роду є жатка — вона заступає кооперацію
женців. Разючим прикладом другого роду є машина до
фабрикації швацьких голок. За Адамом Смісом 10 чоловіка за
його часів у наслідок поділу праці виготовляли 48.000 голок на
день. А одним-одна машина виготовляє 145.000 голок за 11-годинний
робочий день. Одна жінка або одна дівчина наглядає
пересічно за чотирма такими машинами, і тому продукує тими
машинами до 600.000 голок денно, а за тиждень понад 3.000.000.\footnote{
«Children’s Employment Commission. 4 th Report. 1864», p. Ю8,
n. 447.
}
Оскільки на місце кооперації або мануфактури стає поодинока
робоча машина, вона сама знову може стати основою ремісничого
виробництва. Однак, це відновлення ремісничого виробництва,
яке ґрунтується на машинах, становить лише перехід до фабричного
виробництва, яке звичайно з’являється кожного разу, коли
механічна рушійна сила, пара або вода замінює людські мускули,
що рухали машину. Спорадично і лише переходово дрібне виробництво
може сполучатися з механічною рушійною силою за
допомогою наймання пари, як ми це бачимо по деяких мануфактурах
Бермінґему, або за допомогою вжитку невеличких кальорійних
машин, як от у деяких галузях ткацтва тощо.\footnote{
У Сполучених штатах таке відновлення ремества на машиновій
базі трапляється часто. Саме через це концентрація, за неминучого переходу
до фабричного виробництва, йтиме там семимилевими кроками порівняно
з Европою та навіть з Англією.
} В шовкоткацтві
в Ковентрі стихійно розвинувся експеримент з «cottage-фабриками».\footnote*{
— котеджами-фабриками. \emph{Ред.}
}
В центрі побудованих у формі квадрату рядів котеджів
будується так званий engine house\footnote*{
— машиновий будинок. \emph{Ред.}
} для парової машини,
яку валками сполучають з ткацькими варстатами в котеджах.
У всіх таких випадках наймалось пару, приміром, за 2\sfrac{1}{2} шилінґи
з ткацького варстату. Цю плату за пару треба було сплачувати
щотижня, все одно, чи працювали ткацькі варстати, чи ні. Кожний
котедж мав 2—6 ткацьких варстатів, що належали робітникам,
вони були або куплені на кредит, абож найняті. Боротьба між
котеджем-фабрикою та власне фабрикою тривала понад 12 років.
Вона закінчилася повного руїною цих 300 котеджів-фабрик.\footnote{
Порівн. «Reports of Insp. of Fact, for 31 st October 1865», p. 64.
} Там
де природа процесу від самого початку не вимагала продукції у великому
маштабі, ті галузі промисловосте, які виникли за останні