\parcont{}  %% абзац починається на попередній сторінці 
\index{i}{0153}  %% посилання на сторінку оригінального видання 
самий процес продукції, а як елемент творення вартости — лише
частинами.\footnote{
Тут мовиться не про ремонт засобів праці, машин, будівель тощо.
Машина, що її ремонтують, функціонує не як засіб праці, а як матеріял
праці. Нею не роблять, але її саму обробляють, щоб полагодити її споживну
вартість. Такі роботи полагодження завжди можна, зважаючи
на нашу мету, вважати заліченими до тієї праці, що потрібна на продукцію
засобів праці. В тексті мова йде про таке зужиткування, якого не може
вилікувати жоден лікар і яке поволі доводить до смерти, про «зужиткування
такого роду, яке не можна час від часу полагодити і яке, коли справа
йде, приміром, про ніж, кінець-кінцем доводить його до такого стану,
коли ножівник скаже про нього, що він не вартий нового леза» («that
kind of wear which cannot be repaired from time to time, and which, in the
case of a knife, would ultimately reduce it to a state in which the cutler
would say of it, it is not worth a new blade»). В тексті ми бачили, що якась
машина, приміром, цілком увіходить у кожний окремий процес праці,
але лише частинами у рівночасний процес зростання вартости. По цьому
ми можемо оцінити ось таку плутанину понять: «Пан Рікардо каже про
частину праці інженера, витраченої на продукцію машини для панчіх»
(«Mr. Ricardo speaks of the portion of the labour of the engineer in making
stocking machines»), як про працю, що міститься, приміром, у вартості
пари панчіх. «Проте, вся праця, що продукує кожну пару панчіх... містить
у собі всю працю інженера, а не частину її; бо хоч одна машина виробляє
багато пар, але жодної пари не можна зробити без участи всіх частин
машини». («Yet the total labour that produced each single pair of stockings...
includes the whole labour of the engineer, not a portion; for one
machine makes many pairs, and none of those pairs could have been done
without any part of the machine»). («Observations on certain verbal disputes
in Political Economy, particularly relating to Value, and to Demand and
Supply», London 1821, p. 54). Автор, надзвичайно самозадоволений
«wiseacre»\footnote*{
— удаваний мудрак, самозадоволений дурень. Ред.
}, у своїй плутанині, а тому і в своїй полеміці має рацію лише
остільки, оскільки ні Рікардо ні якийбудь інший економіст ні перед
ним, ні після нього строго не розрізняли обох боків праці, а тому ще
й менше проаналізували різну ролю їх у творенні вартости.
}

З другого боку, засіб продукції, може, навпаки, цілком увіходити
в процес зростання вартости, хоч до процесу праці він
увіходить лише частинами. Нехай при прядінні бавовни з 115 фунтів
день-у-день відпадає 15 фунтів, які дають не пряжу,
а лише бавовняний порох, або, як кажуть англійці, лише devil’s
dust.\footnote*{
— чортячий порох. Ред.
} То все ж, коли ці 15\% відпадків є річ нормальна, неминуча
за пересічного способу оброблення бавовни, то вартість цих
15 фунтів бавовни, які не є елемент пряжі, увіходить у вартість
пряжі цілком так само, як і вартість тих 100 фунтів бавовни,
які становлять її субстанцію. Споживна вартість 15 фунтів бавовни
мусить перетворитися на порох, щоб зробити 100 фунтів
пряжі. Отже, загин цієї бавовни є умова продукції пряжі. Саме
тому вона й віддає свою вартість пряжі. Це має силу для всіх
покидьків процесу праці, принаймні, у тій мірі, у якій ці покидьки
знов таки не являють собою нових засобів продукції, а через
те й нових самостійних споживних вартостей. Так, по великих
фабриках машин у Менчестері можна бачити цілі гори відпадків
заліза, поструганих у формі остружків циклопічними машинами;
надвечір вони на величезних возах мандрують із фабрики на
\parbreak{}  %% абзац продовжується на наступній сторінці
