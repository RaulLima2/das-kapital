\parcont{}  %% абзац починається на попередній сторінці
\index{i}{0234}  %% посилання на сторінку оригінального видання
звіту «комісії про працю дітей» (1863 р.) тієї самої долі зазнали
всі мануфактури глиняних товарів (не лише ганчарні), мануфактура
сірників, капсулів, патронів, фабрики шпалер, фабрики
обрізків плюшу (fustian cutting) і численні процеси, об’єднані
під назвою «finishing» (остаточна апретура). 1863 р. «білильні
на свіжому повітрі»\footnote{
«Білильники на свіжому повітрі» ухилилися від додержування
закону 1860 р. про білильні брехнею, нібито в них жінки вночі не працюють.
Брехню викрили фабричні інспектори, але одночасно й петиції
робітників розбили уявлення парляменту про працю у «білильнях на
вільному повітрі», як про працю на прохолодних запашних луках. В цих
білильнях на повітрі користуються сушарнями з температурою в 90--100°
Фаренгайта, і там працюють переважно дівчата. Є навіть технічний вислів
«cooling» (прохолоджування), що ним зветься принагідне вибігання з
сушарні на свіже повітря. «П’ятнадцять дівчат у сушарні. Температура
від 80 до 90° для полотна, 100° і вище для батисту. Дванадцять дівчат прасують
і складають (батист тощо) у маленькій кімнатці якихось приблизно
10 квадратових футів із щільно закритою піччю посередині. Дівчата стоять
колом круг печі, від якої пашить жахливою жарою і яка швидко висушує
батист для прасувальниць. Число годин для цих рук необмежене. Коли
праці багато, вони працюють багато днів уряд до 9 або до 12 годин вночі».
(«Reports etc. for 31 st Oct. 1862», p. 56). Один лікар заявляє: «Осібних
годин для прохолоджування не дозволяється, але, коли температура стає
надто нестерпна, або коли руки робітниць забруднюються від поту, їм
дозволяється вийти на декілька хвилин\dots{} Мій досвід у лікуванні недуг
цих робітниць примушує мене констатувати, що стан їхнього здоров'я
багато гірший від здоров’я бавовнопрядних робітниць (а капітал у своїх
петиціях до парляменту розмальовував їх у стилі Рубенса, нібито від них
пашить здоров’ям!). Найбільш поширено серед них такі недуги: сухоти,
бронхіт, недуги уразу, гістерія в якнайжахнішій формі та ревматизм.
На мою думку, всі ці недуги походять, посередньо або безпосередньо,
від перегрітого повітря їхніх майстерень і недостачі досить теплого одягу,
який міг би захистити їх при повороті додому від вогкої й холодної атмосфери
зимових місяців». (Там же, стор. 56, 57). Фабричні інспектори зауважують
щодо закону з 1863 р., додатково відвойованого в життєрадісних
власників «білилень на свіжому повітрі»: «Цей закон не лише не забезпечує
охорони робітникам, яку він, як здається, їм забезпечує\dots{} його й
зформульовано так, що ця охорона починається лише тоді, коли дітей
і жінок спіймано за працею після 8 годин вечора, але навіть і тоді приписаний
законом спосіб доказу такий заплутаний, що ледве чи можна
покарати винних». (Там же, стор. 52). «Як закон із гуманними й виховними
цілями, він цілком невдалий. Ледве чи можна назвати гуманним
дозволяти жінкам і дітям або, що сходить на те саме, примушувати їх
працювати по 14 годин денно, а може й більше, з перервами на їжу або
й без них, як доведеться, без обмежень щодо віку, без ріжниці статі
і не звертаючи уваги на суспільні звички родин із тих сусідніх околиць,
де лежать ці білильні»). («Reports etc. for 30 th April 1863», р. 40).
} й пекарні підведено під осібні закони,
з яких перший, між іншим, забороняє працювати дітям, підліткам
та жінкам у нічний час (від 8 години вечора до 6 години
ранку), а другий — вживати праці пекарських підмайстрів, молодших
за 18 років, між 9 годиною вечора й 5 годиною ранку.
До пізніших пропозицій згаданої комісії, які загрожують позбавленням
«волі» всім важливим галузям англійської промисловости,
за винятком рільництва, копалень і транспорту, ми ще
повернемось.\footnoteA{
(Примітка до другого видання). Від року 1866, коли я писав те,
що є в тексті, знову надійшла реакція. [Капіталісти тих галузей про-
}

\index{i}{0235}  %% посилання на сторінку оригінального видання
7. Боротьба за нормальний робочий день. Вплив англійського
фабричного законодавства на інші країни

Читач пригадує собі, що продукція додаткової вартости, або
витягання додаткової праці становить специфічний зміст і мету
капіталістичної продукції, незалежно від тих будь-яких змін
у самому способі продукції, які випливають із підпорядкування
праці капіталові. Він пригадує собі, що з того погляду, який
ми досі розвивали, лише самостійний, а тому і юридично повнолітній
робітник складає як продавець товару умову з капіталістом.
Отже, коли в нашому історичному нарисі головну ролю
відіграє, з одного боку, сучасна промисловість, а з другого —
праця фізично і юридично неповнолітніх, то перша мала для
нас значення лише як осібна сфера висисання праці, а друга —
як особливо яскравий приклад висисання праці. Однак, не забігаючи
наперед до того, що ми розвинемо пізніш, із самого лише
загального зв’язку історичних фактів випливає ось що:

По-перше, в галузях промисловости, насамперед революціонізованих
водою, парою й машинами, в цих перших витворах сучасного
способу продукції, в прядільнях і ткальнях бавовни,
льону і шовку капітал насамперед задовольняє своє прагнення
до безмірного й нещадного подовження робочого дня. Змінений
матеріяльний спосіб продукції і відповідно до нього змінені
соціяльні відносини продуцентів\footnote{
«Поведінка кожної з цих кляс (капіталістів і робітників) — це
результат тих відносин, в яких кожна з них перебуває» («The conduct
of each of these classes (capitalists and workmen) has been the result of
relative situation in which they have been placed»). («Reports etc. for 31st
October 1848», p. 113).
} створюють спочатку безмірні
порушення меж робочого дня, а після того вже викликають,
як реакцію, суспільний контроль, що законодатним шляхом
обмежує робочий день з його перервами, реґулює його і робить
його одностайним. Тому протягом першої половини XIX століття
цей контроль з’являється лише як виняткове законодавство.\footnote{
«Підприємства, що підлягали обмеженням, були пов’язані з
продукцією тканин за допомогою сили пари або води. Треба було двох
умов для того, щоб фабрика мусила підлягати доглядові: застосовувати
силу пари або води й обробляти деякі спеціяльні волокна». («The employments
placed under restriction were connected with the manufacture of
textile fabrics by the aid of steam or water power. There were two conditions
to which an employment must be subject to cause it to be inspected,
viz. the use of steam or water power, and the manufacture of certain specified
fibres»). («Reports etc. for 31 st October 1864», p. 8).
}
Скоро тільки останнє завоювало первісне поле нового способу
продукції, виявилося, що за той час не лише багато інших галузей
продукції опинилося під фабричним режимом у власному

мисловости, що їм загрожувало підведення під фабричне законодавство,
використали ввесь свій вплив на парлямент, щоб зберегти недоторканим
своє «право громадянина» на необмежену експлуатацію робочої сили.
Природно, в ліберальному міністерстві Ґледстона вони знайшли покірних
слуг].\footnote*{
Заведене у прямі дужки ми беремо з французького видання. \emph{Ред.}
}
\parbreak{}  %% абзац продовжується на наступній сторінці
