її нормальних меж, її сферу можна було б поширити лише через
узурпаторське загарбання частини доконечного робочого часу.
Не зважаючи на ту важливу ролю, яку ця метода відіграє в дійсному
русі заробітної плати, тут її виключено в наслідок припущення,
що товари, отже, і робочу силу, купується й продається
за їхньою повною вартістю. Скоро ми це припустили, то робочий
час, доконечний для продукції робочої сили або для репродукції
її вартости, може меншати вже не тому, що плата робітника падає
нижче від вартости його робочої сили, а лише тому, що сама ця
вартість падає. За даної довжини робочого дня здовження додаткової
праці мусить виникати із скорочення доконечного робочого
часу, а не навпаки — скорочення доконечного робочого часу із
здовження додаткової праці. В нашому прикладі для того, щоб
доконечний робочий час зменшився на 1/10, тобто з 10 до 9 годин,
і щоб саме в наслідок цього додаткова праця збільшилася з 2 до
3 годин, вартість робочої сили мусить дійсно знизитися на 1/10.

Але таке зниження вартости робочої сили на 1/10 з свого боку,
має за умову, що ту саму масу засобів існування, яку раніше
продуковано за 10 годин, тепер продукується за 9 годин. Це, однак,
не можливо без підвищення продуктивної сили праці. З даними
засобами якийсь швець має, приміром, зробити пару чобіт
протягом одного дванадцятигодинного робочого дня. Коли б він
мав зробити дві пари чобіт протягом того самого часу, то продуктивна
сила його праці мусила б подвоїтися, а вона не може подвоїтися
без зміни в його засобах праці, або в методі його праці,
абож обох разом. Тому мусить настати революція в продукційних
умовах його праці, тобто у його способі продукції, а тому і в самому
процесі праці. Під підвищенням продуктивної сили праці ми розуміємо
тут взагалі таку зміну в процесі праці, в наслідок якої
скорочується робочий час, суспільно-потрібний на продукцію
якогось товару, так що менша кількість праці набуває сили продукувати
більшу кількість споживної вартости.2 Отже, якщо
досліджуючи продукцію додаткової вартости в тій формі, в якій
ми її досі розглядали, ми припускали спосіб продукції за даний,
то для продукції додаткової вартости через перетворення доконечної
праці на додаткову працю вже ніяк недосить того, що
капітал опановує процес праці в його історично-традиційній,
або історично-наявній формі та лише здовжує його тривання.

2 «Коли вдосконалюються ремества, то це є не що інше, як лише
винахід нових метод, за допомогою яких певний продукт можна виготовити
меншою кількістю людей або (це те саме) за коротший час, ніж
раніш» («Quando si perfezionano le arti, che non è altro che la scoperta
di nuove vie, onde si possa compiere una manufattura con meno gente o
(che è lo stesso) in minor tempo di prima»), (Galiani: «Délia Moneta»,
vol. III збірника Custodi «Scrittori Classici Italiani di Economia Politica».
Parte Moderna. Milano 1801, p. 159). «Економія на витратах про:
дукції не може бути чимось іншим, а тільки економією на кількості
праці, вжитої на продукцію» («L'économie sur les frais de production
ne peut... être autre chose que l’économie sur la quantité de travail employé
pour produire»). (Sismondi: «Etudes etc.», vol. I, p. 22).
