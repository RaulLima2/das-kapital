\parcont{}  %% абзац починається на попередній сторінці
\index{i}{0084}  %% посилання на сторінку оригінального видання
минущий момент. Даний товар негайно заміщується іншим товаром.
Тим то досить лише символічного існування грошей у процесі,
який постійно віддалює їх з одних рук до інших. Їхнє функціональне
буття поглинає, так би мовити, їхнє матеріяльне буття.
Як минуще-зоб’єктивізований рефлекс товарових цін, гроші
функціонують лише як знак себе самих і через те вони також
можуть бути заміщені знаками.\footnote{
3 того, що золото й срібло як монета, або у виключній функції
засобу циркуляції стають знаками самих себе, Нікола Барбон виводить
право урядів «to raise money», тобто надавати, приміром, кількості срібла,
що звалася шелягом назву більшої кількости срібла, як, приміром, таляр,
і таким чином виплачувати кредиторам шеляг замість таляра. «Гроші
стираються й стають легшими, часто переходячи з рук до рук\dots{} При торговельних
операціях вважають на назву грошей і курс їхній, а не на кількість
срібла\dots{} Авторитет суспільної влади робить із кусника металю
гроші». («Money does wear and grow lighter by often telling over\dots{} It is
the denomination and currency of the money that men regard in bargaining,
and not the quantity of silver\dots{} Tis the publick authority upon the metal
that makes it money»). (\emph{N. Barbon}: «A Discourse on coining the new
money lighter, in answer to Mr. Lockes Considerations etc.», London 1696,
p. 29, 30, 45).
} Потрібно лише, щоб грошовий
знак набув об’єктивного суспільного визнання, а такого визнання
паперовий символ набуває шляхом примусового курсу. Цей
державний примус має силу лише в межах даного суспільства,
або у сфері внутрішньої циркуляції, але також тільки в цій
сфері гроші цілком розгортаються у своїй функції засобу циркуляції,
або монети, і тому можуть вони в паперових грошах набрати
зовнішньо відокремленого від своєї металевої субстанції й суто
функціонального способу існування.

\section*{3. Гроші}

Товар, що функціонує як міра вартости, а через це безпосередньо
своїм тілом або через заступника і як засіб циркуляції,
є гроші. Отже, золото (або срібло) є гроші. Як гроші воно функціонує,
з одного боку, там, де воно мусить з’являтись у своїй
золотій (або срібній) тілесності, отже, як грошовий товар, тобто
з’являтись не лише ідеально, як у мірі вартости, і не лише як
щось, що може бути репрезентоване, як у засобі циркуляції;
з другого боку, воно функціонує як гроші там, де його функція —
все одно, чи виконує воно цю функцію власною особою, чи через
заступників, — фіксує його як однісіньку форму вартости або як
однісіньке адекватне буття мінової вартости проти всіх інших
товарів як лише споживних вартостей.\footnote*{
У французькому виданні цей абзац зредаґовано так: «Досі ми розглядали
благородний металь з подвійного аспекту: як міру вартостей і як засіб циркуляції.
Першу функцію він виконує як ідеальні гроші;
в другій функції він може бути заміщений символами. Але є такі функції,
що в них він мусить з’являтись у своїй металевій тілесності, як реальний
еквівалент товарів, або як грошовий товар. Але є ще інша функція, яку він
може виконувати або власною особою або через заступників, але в якій
він завжди протистоїть споживним товарам як однісіньке адекватне
втілення їхньої вартости. В усіх цих випадках ми кажемо, що він функціонує
як гроші у власному значенні в протилежність його функціям міри
вартостей і монети». («Le Capital etc.», v. I, ch. III , p. 53) \emph{Ред.}
}

\subsubsection{Скарботворення}

Безперестанний кругобіг двох протилежних товарових метаморфоз,
або невпинне чергування продажу й купівлі, виявляється
в безупинному обігу грошей або, в функції їх як perpetuum mobile
\index{i}{0085}  %% посилання на сторінку оригінального видання
циркуляції. Гроші імобілізуються, або перетворюються, як
каже Буаґільбер, із meuble на immeuble,\footnote*{
— з рухомих на нерухомі. \emph{Ред.}
} з монети на гроші,
скоро тільки переривається ряд метаморфоз, скоро тільки продаж
не доповнюється наступною купівлею.

Уже з самого початку розвитку товарової циркуляції розвивається
конечність і жагуче бажання затримати в себе продукт
першої метаморфози, перетворену форму товару, або її золоту
лялечку.\footnote{
«Багатство на гроші є не що інше, як\dots{} багатство на продукти,
перетворені на гроші» («Une richesse en argent n’est que\dots{} richesse en
productions, converties en argent»). (Mercier de lа Rivière: «L’Ordre naturel
et essentiel des sociétés politiques». Physiocrates, ed. Daire, I. Partie, Paris.
1848, p. 573). «Вартість продуктів\dots{} зміняє тут лише форму» («Une valeur
en productions\dots{} n’a fait que changer de forme»). (Там же, стор. 486).
} Товар продається не на те, щоб купити інший товар,
але щоб замінити товарову форму на грошову. З простого посереднього
моменту обміну речовин ця зміна форми стає за самоціль.
Преображеній формі (entäusserte Gestalt) товару заважається
функціонувати як його абсолютно відчужуваній формі,
або як його лише минущій грошовій формі. Разом із цим гроші
кам’яніють у скарб, а продавець товарів стає збирачем скарбів.

Саме на початках товарової циркуляції на гроші перетворюється
лише зайвина споживної вартости. Таким чином золото
й срібло сами собою стають суспільними виразами надлишку або
багатства. Ця наївна форма скарботворення увіковічнюється в
тих народів, де традиційному й розрахованому на власне споживання
способові продукції відповідає стало замкнене коло потреб,
приміром, в азійців, особливо в індусів. Вандерлінт, який уявляє
собі, що товарові ціни визначається масою золота й срібла, що
є в країні, запитує себе, чому індійські товари такі дешеві.
Відповідь: тому, що індуси закопують гроші в землю. Від 1602
до 1734 рр., зауважує він, вони закопали 150 мільйонів фунтів
стерлінґів срібла, яке первісно привезено було з Америки до
Европи.\footnote{
«Саме завдяки цьому звичаєві вони тримають свої товари й мануфактурні
вироби на такому низькому рівні цін» («Tis by this practice
they keep all their goods and manufactures at such low rates»). (Vanderlint:
«Money answers all Things», London 1734, p. 95, 96).
} Від 1856 до 1866 рр. отже, за одне десятиліття, Англія
вивезла до Індії й Китаю (експортований до Китаю металь іде в
значній частині знову таки до Індії) 120 мільйонів фунтів стерлінґів
срібла, що раніш було виміняне на австралійське золото.

За розвиненішої товарової продукції кожний товаропродуцент
мусить забезпечити собі nexus rerum, «суспільну рухому
\parbreak{}  %% абзац продовжується на наступній сторінці
