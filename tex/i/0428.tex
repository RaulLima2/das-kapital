форми репродукується подекуди і на базі великої промисловости,
хоч і з цілком зміненою фізіономією.

Коли, з одного боку, для продукції абсолютної додаткової
вартости досить лише формальної підпорядкованости праці капіталові,
досить, наприклад, того, щоб ремісники, які раніш працювали
на самих себе абож як підмайстри цехового майстра,
стали тепер як наймані робітники під безпосередній контроль
капіталіста, то, з другого боку, виявилося, що методи продукції
відносної додаткової вартости є разом з тим методи продукції
абсолютної додаткової вартости. Аджеж безмірне здовження
робочого дня виявилось як найхарактеристичніший продукт великої
промисловости. Взагалі специфічно-капіталістичний спосіб
продукції перестає бути простим засобом продукувати відносну
додаткову вартість, скоро тільки він опановує цілу галузь
продукції, і ще більше — скоро він опановує всі вирішальні
галузі. Він стає тепер загальною, суспільно-панівною формою
подукційного процесу. Як осібна метода продукувати відносну
додаткову вартість, він діє ще лише остільки, оскільки, поперше,
захоплює галузі продукції, досі лише формально підпорядковані
капіталові, отже, поширюючи сферу свого впливу; подруге,
остільки, оскільки галузі промисловости, що підпали вже під
його руку, постійно революціонізуються через зміну метод продукції.

З певного погляду ріжниця між абсолютною і додатковою вартістю
видається взагалі ілюзорною. Відносна додаткова вартість
є абсолютна, бо вона зумовлює абсолютне здовження робочого
дня поза робочий час, доконечний для існування самого робітника.
Абсолютна додаткова вартість є відносна, бо вона зумовлює:
розвиток такої продуктивности праці, що дозволяє обмежити
доконечний робочий час певною частиною робочого дня. Але
коли звернути увагу на рух додаткової вартости, то ця позірна
тотожність зникає. Скоро тільки капіталістичний спосіб продукції
виник і став загальним способом продукції, ріжниця між абсолютною
й відносною додатковою вартістю стає відчутною, коли
йдеться про підвищення норми додаткової вартости взагалі.
Коли припустити, що робочу силу оплачується за її вартістю,
то ми опиняємось перед такою альтернативою: при даній продуктивності
праці й нормальному ступені її інтенсивности норму
додаткової вартости можна підвищити тільки через абсолютне
здовження робочого дня; з другого боку, при даних межах робочого
дня норму додаткової вартости можна підвищити лише через
зміну відносних величин його складових частин, доконечної
праці й додаткової праці, а це, з свого боку, має собі за передумову
зміну продуктивности або інтенсивности праці, якщо
заробітна плата не повинна впасти нижче вартости робочої сили.

Якщо робітник потребує всього свого часу на те, щоб продукувати
засоби існування, потрібні для утримання його самого і
його родини, то йому не лишається вже часу задурно працювати
на третіх осіб. Без певного ступеня продуктивности праці в ро-
