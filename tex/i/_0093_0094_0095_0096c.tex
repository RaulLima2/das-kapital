\parcont{}  %% абзац починається на попередній сторінці
\index{i}{0093}  %% посилання на сторінку оригінального видання
світовому ринку. Як олень кричить за свіжою водою, так душа буржуа кричить за грішми, цим єдиним
багатством.\footnote{
«Цей раптовий перехід від кредитової системи до монетарної системи додає теоретичного страху до
практичної паніки: аґенти циркуляції здригаються перед нерозгадною тайною своїх власних відносин».
(К. Marx: «Zur Kritik der Politischen Oekonomie», Berlin 1859, S. 148. — K. Маркс: «До критики і т.
д.», ДВУ, 1926 р., стор. 157). «Бідний сидить без роботи, бо багатий не має грошей, щоб ужити його
до праці, хоч він має ту саму землю і ті самі робочі руки, щоб виробляти засоби існування й одежу,
як і раніш; але саме це й становить дійсне багатство нації, а зовсім не гроші» («The Poor stand
still, because the Rich have no Money to employ them, though they have the same land and hands to
provide victuals and cloaths, as ever they had; which is the true Riches of a Nation, and not the
Money»). (John Bellers: «Proposals for raising a Colledge of
Industry», London 1696, p. 3).
} Підчас кризи протилежність між товаром і формою його вартости, грішми, підноситься до
абсолютної суперечности. Тим то і форма виявлення грошей не має тут значення. Грошовий голод
залишається той самий, незалежно від того, чи треба платити золотом, чи кредитовими грішми,
приміром, банкнотами.\footnote{
Ось як використовують подібні моменти «amis du commerce»:\footnote*{
— друзі торговлі. Ред.
} «В одному з таких випадків (1839)
старий зажерливий банкір (із Сіті), піднявши верх бюрка, за яким сидів у своєму кабінеті, показав
своєму приятелеві жмути банкнот, заявивши з незвичайною радістю, що там 600.000 фунтів стерлінґів,
які він тримав у себе, щоб загострити потребу в грошах, але сьогодні після трьох годин він пустить
їх в обіг» («On one occasion (1839) an old grasping banker (der City) in his private room raised the
lid of the desk he sat over, and displayed to a friend rolls of banknotes, saying with intense glee
there were 600 000 £ of them, they were held to make money tight, and would all be let out after
three o'clock on the same day»). («The Theory of the Exchanges. The Bank Charter Act of 1814» ,
London 1864 p. 81). Напівофіціяльний орган «The Observer» з 24 квітня 1864 p. зауважує: «Поширюється
ряд дуже курйозних чуток про ті засоби, з яких користувалися, щоб утворити недостачу банкнот. Хоч
дуже сумнівна річ, чи справді вжито подібних трюків, проте зазначені чутки були остільки поширені,
що дійсно заслуговують на згадку». («Some very curious rumours are current of the means which have
been resorted to in order to create a scarcity of Banknotes... Questionable as it would seem, to
suppose that any trick of the kind would be adopted, the report has been so universal that really
deserves mention»).
}

Коли ми тепер розглянемо загальну суму грошей, які циркулюють протягом якогось даного часу, то
побачимо, що вона за даної швидкости обігу засобів циркуляції та засобів платежу дорівнює сумі тих
товарових цін, що їх треба зреалізувати, плюс сума платежів, що їм настав термін платежу, мінус сума
платежів, що урівноважується, мінус, нарешті, число обігів, у яких та сама монета функціонує
навпереміну то як засіб циркуляції, то як засіб платежу. Приміром, селянин продає своє збіжжя за 2
фунти стерлінґів, які таким чином функціонують як засіб циркуляції. Коли настає платіжний термін,
він віддає їх, щоб сплатити борг за полотно, яке постачив йому ткач. Тепер ці самі 2 фунти
стерлінґів функціонують як засіб платежу. Далі ткач купує біблію за готівку ; ці 2 фунти стерлінґів
тепер знов функціонують,
\index{i}{0094}  %% посилання на сторінку оригінального видання
як засіб циркуляції і т. ін. Отже, навіть якщо дано ціни, швидкість грошового обігу й
економію платежів, то маса грошей, що циркулюють протягом якогось періоду, приміром, одного дня, і
маса товарів, що циркулюють, вже не покривають одна одну. Циркулюють гроші, що репрезентують товари,
давно вже витягнуті з циркуляції. Циркулюють товари, що їхній грошовий еквівалент з’явиться лише у
будучині. З другого боку, зобов’язання, які щодня складаються, і зобов’язання, що їм того ж дня
настає термін платежу, є цілком неспільномірні величини.\footnote{
«Сума продажів або договорів, що їх зроблено протягом даного дня, не впливає на кількість
грошей, що циркулюють у цей самий день, але у величезній більшості випадків виражається в цілому
ряді векселів на суму грошей, які можуть увійти в обіг лише в дальші більш-менш віддалені терміни...
Акцептовані векселі або кредити, відкриті сьогодні, зовсім не повинні мати якоїсь подібности щодо
кількости загальної суми або протягу реченців з тими кредитовими операціями, що будуть переведені
завтра або в найближчі дні; багато з переведених сьогодні кредитових операцій і виданих сьогодні
векселів можуть навіть збігатися щодо терміну платежу з багатьма зобов’язаннями, що переведення їх
стосується до ряду різних попередніх цілком невизначених дат; векселі на 12, 6, 3 місяці і на 1
місяць часто збігаються одні з одними, і подається їх до виплати того самого дня... » («The amount
of sales or contracts entered upon during the course of any given day, will not affect the quantity
of money afloat on that particular day, but, in the vast majority of cases, will resolve themselves
into multifarious drafts upon the quantity of money which may be afloat at subsequent dates more or
less distant... The bills granted or credits oponed, to day, need have no resemblance whatever,
either in quantity, amount or duration, to those granted or entered ucon to-morrow or next day; nay,
many of to-day’s bills and credits, when due, fall in with a mass of liabilities whose origins
traverse a range of antecedent dates altogether indefinite, bills at 12, 6, 3 months or 1 often
aggregating together to swell the common liabilities of one particular day...»). («The Currency
Question Reviewed; a letter to the Scotch people. By a Banker in England», Edinburgh 1845, p. 29, 30
passim).
}

Кредитові гроші виникають безпосередньо з функції грошей, як засобу платежу, при чому самі боргові
посвідки за продані товари й собі циркулюють, переносячи боргові вимоги. З другого боку, в міру того
як поширюється кредитова справа, поширюється й функція грошей як засобу платежу. В цій своїй функції
вони набувають власних форм існування, що в них вони перебувають у сфері великих торговельних
оборудок, тимчасом як золоту або срібну монету витискується переважно у сферу дрібної торговлі.\footnote{
Як приклад того, як мало реальних грошей входить у власне торговельні операції, наводимо тут
схему одного з найбільших лондонських торгових домів (Morrison, Dillon et С°) щодо його річних
грошових прибутків і платежів. Його операції 1856 р., що обіймають багато мільйонів фунтів
стерлінґів, тут пропорційно скорочені і зведені до маштабу одного мільйона фунтів стерлінґів:
}

На певній висоті розвитку й ширині обсягу товарової продукції функція грошей як засобу платежу
виходить поза сферу циркуляції товарів. Гроші стають загальним товаром для всіх
\index{i}{0095}  %% посилання на сторінку оригінального видання
договорах.\footnote{
Характер торговлі змінився в такому напрямі, що торговля від
обміну товарів на товари, від постачання й одержання, перейшла до продажу
й платежу; тепер усі операції... відбуваються як чисто грошові
операції». («The Course of Trade being thus turned, from exchanging of
goods for goods, or delivering and taking, to selling and paying, all the
bargains... are now stated upon the foot of a Price in Money»). («An Essay
upon Publick Credit», 3 rd ed. London 1710, p. 8).
} Ренти, податки й т. ін. з постачання в натурі перетворюються
на грошові платежі. Як дуже це перетворення
залежить від загальних умов процесу продукції, показує, приміром,
спроба Римської імперії стягати всі податки грішми —
спроба, яка двічі розбилася. Страшенні злидні французького
селянства за Люї XIV, злидні, що їх так красномовно ганьблять
Буаґільбер, маршал Вобан і інші, були викликані не лише висотою
податків, але й перетворенням натурального податку на
грошовий.\footnote{
«Гроші... стали загальним катом». Фінансова справа є «реторта, в
якій перетворюють на пару страшенну кількість добра й засобів існування,
щоб одержати цей фатальний осад». «Гроші проголошують війну всьому
людському родові» («L’argent... est devenu le bourreau de toutes les choses».
Die Finanzkunst ist das «alambic qui a fait évaporer une quantité
effroyable de biens et de denrées pour faire ce fatal précis». «L’argent declare
la guerre à tout le genre humain»). (Boisguillebert: «Dissertation sur la
nature des richesses, de l’argent et des tributs», ed. Daire, «Economistes
financiers», Pans 1843. vol. I, p. 413, 417, 419).
} Коли, з другого боку, натуральна форма земельної
ренти, яка в Азії є разом з тим головний елемент державних
податків, спирається там на продукційні відносини, що репродукуються
з незмінністю природних відносин, то ця форма платежу
через зворотний вплив підтримує стару форму продукції. Вона
становить одну з таємниць самозбереження турецької держави.
Коли закордонна торговля, накинута Японії Европою, потягне
за собою перетворення натуральної ренти на грошову, то це
станеться коштом загину зразкової рільничої культури Японії.
Вузькі умови економічного існування цієї культури зруйнуються.

У кожній країні встановлюється певні загальні терміни платежів.
Почасти вони, залишаючи осторонь інші циклічні пере-

Прибутки    Ф. ст.

Термінові векселі банкірів
і купців............ 533.596

Чеки на банкірів і інші
виплати пред’явникам. . 357.715

Банкноти провінціяльних
банків................ 9.627

Банкноти Англійського
банку................ 68.554

Золото................. 28.089

Срібло й мідь........... 1.486

Поштові перекази.......... 933

Разом фунтів стерлінґів 1.000.000

Видатки    Ф. ст.

Термінові векселі.... 302.674

Чеки на лондонських банкірів
............... 663.672

Банкноти Англійського
банку............... 22.743

Золото................. 9.427

Срібло й мідь.......... 1.484

Разом фунтів стерлінґів 1.000.000

(«Report from the Select Committee on the Bankacts. July 1858»,
p. LXXI).
\index{i}{0096}  %% посилання на сторінку оригінального видання
біги репродукції, ґрунтуються на природних умовах продукції,
пов’язаних із зміною пір року. Ці загальні терміни регулюють
так само і ті платежі, що не випливають безпосередньо з товарової
циркуляції, як ось податки, ренти й т. ін. Маса грошей, потрібна
у певні дні року для цих розпорошених по цілій поверхні суспільства
платежів, спричинює періодичні, але цілком поверхові пертурбації
в економії засобів платежу.\footnote{
«У клечальний понеділок 1824 р., — оповідає пан Креґ парламентській
слідчій комісії з 1826 р., — в Едінбурзі був такий величезний попит
на банкноти, що на 11 годину ми не мали вже й жодної банкноти в нашому
розпорядженні. Ми звертались по черзі до різних банків, щоб позичити
їх, але не мали змоги одержати ні одної, і багато оборудок довелося перевести
лише за допомогою slips of paper.\footnote*{
— шматків паперу. Ред.
} Однак уже о 3 годині опівдні
всі банкноти повернулись до тих банків, звідки вони вийшли. Вони змінили
лише декілька рук». Хоч пересічне число банкнот, які дійсно циркулюють
у Шотляндії, становить менш, ніж 3 мільйони фунтів стерлінґів,
все ж у певні дні року, коли настає термін різних платежів, до операцій
притягаються всі банкноти, що їх мають банкіри, всього на 7 мільйонів
фунтів стерлінґів. За цієї нагоди банкноти мають виконати однимодну
й специфічну функцію, і, виконавши її, вони повертаються назад
до тих банків, звідки вийшли. (John Fullarton: «Regulation of Currencies»,
2 nd ed. London 1845, p. 86, примітка). На пояснення додамо, що в Шотляндії
в час виходу твору Фуляртона вклади видавано не чеками, а лише
банкнотами.
} Із законів швидкости
обігу засобів платежу випливає, що маса засобів платежу, доконечна
для всіх періодичних платежів, хоч які були б їх джерела,
стоїть у прямому відношенні\footnote*{
У німецькому тексті тут замість «у прямому відношенні» сказано
«у зворотному відношенні» («in umgekehrtem Verhältniss»), що, очевидно,
є рукописна помилка. Ред.
} до протягу періодів платежу.\footnote{
На питання: «Коли б довелося протягом року виплатити 40 мільйонів,
то чи вистачило б цих 6 мільйонів (золотом) для циркуляції та оборотів,
яких вимагала б у даному разі торговля?» — Петті відповідає з
властивою йому майстерністю: «Я відповідаю: так; коли б обороти відбувались
у короткі переміжки часу, приміром, щотижня, як це буває
серед бідних ремісників і робітників, що одержують плату щосуботи, то
для переведення виплат на 40 мільйонів було б досить 40/52 мільйона.
А коли б обороти відбувалися щокварталу, як то звичайно в нас буває
за виплати ренти й податків, то було б потрібно 10 мільйонів. Отже, коли
ми припустимо, що загалом платежі робиться через різні переміжки часу,
між одним тижнем і 13, то тоді треба скласти 10 мільйонів і 40/52 мільйони
і взяти половину, яка дорівнює 5 1/2 мільйонам, так що, коли б ми мали
5 1/2 мільйонів, то нам уже вистачило б грошей». (Auf die Frage «If there were
occasion to raise 40 millions p. a., whether the same 6 millions (Gold) would
suffice for such revolutions and circulations there of as trade requires?», antwortet
Petty mit seiner gewohnten Meisterschaft: «I answer yes: for the expense
being 40 millions, if the revolutions were in such short circles, viz, weekly,
as happens among poor artizans and labourers, who receive and pay every
Saturday, then 40/52 parts of 1 million of money would answer these ends;
but if the circles be quarterly, according to our custom of paying rent, and
gahering taxes, then 10 millions were requisite. Wherefore supposing payments
in general to be of a mixed circle between one week and 13, then add
10 millions to 40/52, the half of the which will be 5 1/2, so as if we have
5 1/2 mill., we have enough»). (William Petty: «Political Anatomy of Ireland.
1672», ed. London 1691, p. 13, 14).
}
