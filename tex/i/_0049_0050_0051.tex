\parcont{}  %% абзац починається на попередній сторінці
\index{i}{0049}  %% посилання на сторінку оригінального видання
її в національному маштабі зроблено вперше століттям пізніше,
за часів буржуазної французької революції.

В міру того, як обмін товарів рве свої тільки місцеві пута,
а товарова вартість через це розвивається на матеріялізацію
людської праці взагалі, грошова форма переходить на товари,
які з природи здатні виконувати суспільну функцію загального
еквіваленту, — на благородні металі.

Що «гроші з природи є золото й срібло, дарма що золото й
срібло з природи не є гроші»,\footnote{
\emph{К. Marx}: «Zur Kritik der Politischen Oekonomie», S. 135 (\emph{К. Маркс}:
«До критики і т. д.», ДВУ, 1926 р., стор. 165.) — «Металі\dots{} з природи
гроші» («І metalli\dots{} naturalmente moneta»). (\emph{Galiani}: «Deila Moneta»
в збірнику Custodi, Parte Moderna, vol. III, p. 72).
} показує нам те, що їхні природні
властивості збігаються з їхніми функціями.\footnote{
Докладніше про це див. у моїй тільки-но цитованій праці, розділ:
«Благородні металі».
} Але досі
ми знаємо лише одну функцію грошей: служити за форму виявлення
товарової вартости, або за матеріял, що в ньому суспільно
виражаються величини товарових вартостей. Адекватною формою
виявлення вартости або матеріялізацією абстрактної і тому рівної
людської праці може бути лише така матерія, що її всі екземпляри
мають однорідну якість. З другого боку, через те, що ріжниця
поміж величинами вартостей є суто кількісна, грошовий товар
мусить бути придатний, щоб визначити суто кількісні ріжниці,
отже, бути придатний до довільного ділення на частини й знову
до складання цілого з його частин. Золото й срібло з природи
мають ці властивості.

Споживна вартість грошового товару подвоюється. Поруч з
осібною його споживною вартістю як товару, — приміром, золото
придається пльомбувати зуби або як сировинний матеріял
для предметів розкошів і т. ін., — він набуває формальної споживної
вартости, яка випливає з його специфічних суспільних
функцій.

Через те, що всі інші товари — це лише осібні еквіваленти
грошей, а гроші — їхній загальний еквівалент, то вони як осібні
товари відносяться до грошей як до загального товару.\footnote{
«Гроші є універсальний товар» («Il danaro è la merce universale»).
(\emph{Verri}: «Meditazioni sulla Economie Politica», p. 16).
}

Ми вже бачили, що грошова форма є лише зрослий з певним
товаром рефлекс відношень до нього всіх інших товарів. Отже,
той факт, що гроші є товар,\footnote{
«Сами срібло й золото, яким ми можемо дати загальну назву грошового
металю, є\dots{} товари\dots{}, вартість яких то підноситься, то падає\dots{}
Вартість грошового металю можна вважати за велику тоді, коли за
незначну вагу його можна купити значну кількість продуктів або фабрикатів
 країни, і т. ін.» («Silver and gold themselves, which we may call
by the general name of Bullion, are\dots{} commodities\dots{} raising and falling
in\dots{} value\dots{} Bullion then may be reckoned to be of higher value, where
the smaller weight will purchase the greater quantity of the product or
manufacture of the country etc.»). («А Discourse of the General Notions
of Money, Trade, and Exchange, as they stand in relations to each other.
By a Merchant», London 1695, p. 7). «Срібло й золото, карбовані й некарбовані,
хоч їх уживається як міру для всіх інших речей, все ж таки
являють собою товари не менш, ніж вино, масло, тютюн, тканини або матерії»
(«Silver and gold, coined or uncoined, tho’they are used fora measure
of all other things, are no less a commodity than wine, oyl, tobacco,
cloth or stuffs»). («А Discourse concerning Trade, and that in particular
of the East-Indies etc.», London 1689, p. 2). «Капітал і багатство королівства
не обмежується лише на самих грошах, так само золото й срібло
не можна вилучити з числа предметів торговлі» («The stock and riches
of the kingdom cannot properly be confined to money, nor ought gold and
silver to de excluded from being merchandize»). («The East-India Trade
a most Profitable Trade», London 1677, p. 4).
} є відкриття лише для того, хто
\index{i}{0050}  %% посилання на сторінку оригінального видання
виходить із готової їхньої форми, щоб потім її аналізувати. Процес
обміну дає товарові, що його він перетворює на гроші, не його
вартість, а його специфічну форму вартости. Переплутування
цих двох визначень приводило до того, що вартість золота й
срібла вважалося за уявлювану.\footnote{
«Золото й срібло мають свою вартість як металі раніш, ніж вони
стають грішми» («L’oro е l’argento hanno valore come metalli anteriore
all’essere moneta»). (\emph{Galiani}: «Della Moneta»). Льокк каже: «Загальна
згода людей надає сріблу задля його властивостей, які роблять
його придатним на ролю грошей, уявлюваної вартости». Навпаки, Ло
каже: «Як різні нації могли б надати одній речі уявлювану вартість\dots{}
або як могла б утриматись ця уявлювана вартість?» Але як мало він сам
розумів справу, видно з цих його слів: «Срібло обмінювалося відповідно
до тієї споживної вартости, яку воно мало, отже, за своєю дійсною вартістю;
в наслідок свого призначення бути\dots{} грішми воно набуло додатково
ще однієї вартости (une valeur additioneile)». (\emph{Jean Law}: «Considérations
sur le numéraire et le commerce» y виданні E. Daire: «Economistes
Financiers du XVIII siècle», p. 469, 470).
} Через те, що гроші в певних
своїх функціях можуть бути заміщені простими знаками грошей,
виникла інша помилка, що гроші є лише знаки. З другого боку,
тут містилося передчуття того, що грошова форма речі супроти
самої речі зовнішня й є лише форма виявлення захованих за нею
людських відносин. У цьому розумінні кожний товар був би
знаком, бо як вартість він є лише речова оболонка витраченої
на нього людської праці.\footnote{
«Гроші є їх (харчових продуктів) знак» («L’argent en (des denrées)
est le signe»). (\emph{V. de Forbonnais}: «Eléments du Commerce»,
Nouv. Edit. Leyde 1766, vol. II, p. 143). «Як знак вони притягаються
харчовими продуктами» («Comme signe il est attiré par les denrées»). (Там
само, p. 155). «Гроші — знак речі, і вони репрезентують річ» («L’argent
est un signe d’une chose et la représente»). (\emph{Montesquieu}: «Esprit des
Lois», Oeuvres, London 1767, vol. III, p. 2). «Гроші не простий знак,
бо вони сами є багатство: вони не репрезентують вартостей, а еквівалентні
їм» («L’argent n’est pas simple signe, car il est lui-même richesse; il
ne représente pas les valeurs, il les équivaut»). (\emph{Le Trosne}: «De l’Intérêt
Social», p. 910). «Коли розглядають поняття вартости, то саму річ
вважають лише за знак, і вона має значення не сама по собі, а лише як
те, чого вона варта». (\emph{Hegel}: «Philosophie des Rechtes», S. 100). Задовго
до економістів уявлення про золото як простий знак та про уявлювану
вартість благородних металів пустили в хід юристи; лакеї й сикофанти
королівської влади, вони протягом цілого середньовіччя обґрунтовували
традиціями Римської імперії й поняттями про гроші з пандектів
право цієї влади фалшувати монету. «Ніхто не може й не сміє сумніватися,
— каже вірний учень юристів, Філіп Валюа, в одному декреті з 1346 р., —
що лише нам і нашій королівській ясновельможності належить\dots{} справа
виготовлення, постачання грошей і всякі розпорядження щодо грошей,
призначати їм такий курс і таку ціну, яку завгодно нам і яку ми визнали
за добре» («Qu’aucum puisse ni doive faire doute, que à nous et à notre
majesté royale n’appartienne seulement\dots{} le mestier, le fait, l’état, la provision
et toute l’ordonnance des monnaies, de donner tel cours, et pour tel
prix comme il nous plait et bon nous semble»). Декретування вартости
грошей імператором було догмою римського права. Було виразно заборонено
трактувати гроші як товар. «Грішми не вільно нікому торгувати,
бо, призначені для загального користування, вони не повинні бути товаром»
(«Pecunias vero nullі emere fas erit, nam in usu publico constitutas,
oportet non esse mercem»). Досконалі пояснення про це див. у \emph{G. F. Pagnini}:
«Saggio sopra il giusto pregio delle cose», 1751 y Custodi, Parte
Moderna, vol. II. Особливо полемізує Паньїні з панами юристами у другій
частині своєї праці.
} Але, проголошуючи ті суспільні
\index{i}{0051}  %% посилання на сторінку оригінального видання
риси, що їх набирають речі, або речові риси, що їх набирають
суспільні визначення праці на основі певного способу продукції,
простими знаками, їх одночасно проголошується за самовільний
рефлективний продукт людини. Це й була улюблена манера
пояснення у XVIII столітті, яку вживано, щоб із загадкових
форм людських відносин, що їхнього процесу постання ще не
могли розшифрувати, принаймні покищо зняти подобу чогось
незрозумілого.\footnote*{
У французькому виданні цю фразу подано так: «Це була улюблена
манера пояснення у XVIII столітті; через те, що тоді ще не могли розшифрувати
ні постання, ані розвитку загадкових форм суспільних відносин,
їх загадковість намагалися відкинути тим, що їх проголошувано
людським винаходом, чимось таким, що не впало з неба». («Le Capital
etc.», v. І, ch. II, p. 39). \emph{Ред.}
}

Раніш ми вже зазначали, що еквівалентна форма товару не
містить у собі кількісного визначення величини його вартости.
Коли відомо, що золото є гроші, а тому й безпосередньо вимінне
на всі інші товари, то з того ще невідомо, чого варті, приміром,
10 фунтів золота. Як і кожний інший товар, золото може виражати
величину своєї власної вартости лише відносно в інших товарах.
Його власна вартість визначається робочим часом, потрібним
на його продукцію, і виражається в такій кількості кожного
іншого товару, в якій скристалізовано стільки ж робочого часу.\footnote{
«Якщо людина може доставити до Лондону унцію срібла з перуанських
копалень за такий самий час, який був би їй потрібний на продукцію
четверика хліба, то перший з цих продуктів є природна ціна другого;
а якщо вона з відкриттям нових і багатших копалень може добути дві
унції срібла так само легко, як раніш одну, то хліб, за інших незмінних
умов, буде такий самий дешевий при ціні в 10 шилінґів за четверик, як
раніш при ціні в 5 шилінґів за четверик» («If a man can bring to London
an ounce of silver out of the earth in Peru, in the same time that he can
produce a bushel of corn, then one the natural price of the other; now if
by reason of new and more easier mines a man can get two ounces of silver
as easily as he formerly did one, the corn will be as cheap at 10 shillings
the bushel, as it was before at 5 shillings, caeteris paribus»). (\emph{William
Petty}: «А Treatise of Taxes and Contributions», London 1667, p. 31).
}
Це фіксування відносної величини вартости золота відбувається
біля джерел його продукції в безпосередній міновій торговлі.
Скоро тільки воно входить як гроші в циркуляцію, то вартість
його вже дано. І коли вже в останні десятиліття XVII віку
добре знали, що гроші є товар, то все ж таки це був лише початок
\parbreak{}  %% абзац продовжується на наступній сторінці
