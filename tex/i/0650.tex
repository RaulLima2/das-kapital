тити такі ціни: за чоловічий скальп 12 років і старше — 100 фунтів стерлінґів у новій валюті, за
спійманого чоловіка — 105 фунтів стерлінґів, за спійману жінку або дитину — 55 фунтів стерлінґів, за
жіночий або дитячий скальп — 50 фунтів стерлінґів!
Кілька десятиліть пізніш колоніяльна система помстилась на нащадках цих побожних отців-піліґримів,
що й собі стали бунтарями. З намови й за гроші англійців їх усіх tomahawked.\footnote*{
— повбивано томагавками. \emph{Ред.}
} Брітанський парлямент
проголосив кровожерство і скальпування
за «засоби, дані йому богом і природою».

Колоніяльна система надзвичайно прискорила розвиток торговлі й мореплавства. «Монопольні товариства»
(Лютер) були могутніми підоймами концентрації капіталу. Колонії забезпечували ринок збуту для
мануфактур, що народжувались, а монополія на ринку забезпечувала їм збільшену акумуляцію капіталу.
Скарби, здобуті поза Европою безпосереднім плюндруванням, поневолюванням, грабіжництвом і
вбивствами, припливали в метрополію і тут перетворювались на капітал. Голляндія, яка перша цілком
розвинула колоніяльну систему, вже 1648 р. дійшла вершини своєї торговельної могутности. В її «майже
виключному посіданні була східньоіндійська торговля й засоби
комунікації поміж европейським південним заходом і північним сходом. Її рибальство, мореплавство й
мануфактури були розвиненіші, ніж у всіх інших країнах. Капітали цієї республіки були, мабуть,
значніші, ніж капітали всіх інших країн Европи разом» (Gülich: «Geschichtliche Darstellung des
Handels etc.» Iena 1830, В. I, p. 371). Ґюліх забуває додати, що народні маси Голляндії вже в 1648
р. більше терпіли від надмірної праці, були більш збіднілі й пригнічені, ніж народні маси всіх інших
країн Европи разом.

За наших часів промислова перевага веде за собою торговельну перевагу. Навпаки, за власне
мануфактурного періоду торговельна перевага забезпечує перевагу промисловости. Відси та вирішальна
роля, яку в ті часи відігравала колоніяльна система. Це був той «чужий бог», що засів на вівтарі
поруч із старими божками Европи й одного чудового дня одним ударом усіх їх поскидав. Колоніяльна
система оголосила здобування зиску за останню й єдину мету людства.

Система публічного кредиту, тобто державних боргів, що її початки ми знаходимо в Ґенуї й Венеції вже
за середньовіччя, захопила цілу Европу підчас мануфактурного періоду. Колоніяльна система з її
морською торговлею і торговельними війнами була за теплицю, що прискорювала її розвиток. Так вона
вкоренилася насамперед у Голляндії. Державний борг, тобто відчуження держави — однаково, чи
деспотичної, чи конституційної, чи республіканської — накладає свою печать на капіталістичну еру.
Однісінька частина так званого національного багатства, що дійсно входить у спільне володіння
сучасних наро-