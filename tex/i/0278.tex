цюють хоч і вдома, але на капіталіста (Fabrikant, établisseur),
геть цілком відмінне від становища самостійного ремісника, що
працює для своїх власних клієнтів.\footnote{
В годинникарстві, в цьому клясичному прикладі гетерогенної
мануфактури, можна дуже докладно вивчити згадані вище диференціяцію
та спеціялізацію робочих інструментів, що випливають із розчленування
ремісничої праці.
}

Другий рід мануфактури, її закінчена форма, продукує
вироби, що перебігають зв’язані між собою фази розвитку, певний
ряд послідовних процесів, як от, приміром, дріт у голчаній
мануфактурі, який проходить через руки 72, а то й 92 специфічних
частинних робітників.

Оскільки така мануфактура комбінує ремества первісно розпорошені,
вона зменшує просторове відокремлення між окремими
фазами продукції виробу. Час на перехід його з однієї
стадії до одної скорочується, і так само зменшується праця, що
упосереднює ці переходи.\footnote{
«За такого тісного співжиття людей на транспортування неодмінно
мусить витрачатись менше часу» («In so close a cohabitation
of the People, the carriage must needs be less»). («The Advantages of the
East-India Trade», London 1720, p. 106).
} Порівняно з ремеством, таким способом
досягається вищої продуктивної сили, і цей виграш виникає
саме з загального кооперативного характеру мануфактури.
З другого боку, властивий мануфактурі принцип поділу праці
зумовлює ізоляцію різних фаз продукції, що усамостійнюються
одна проти одної як відповідна кількість частинних праць ремісничого
характеру. Встановлення і зберігання зв’язку поміж
ізольованими функціями вимагає постійного транспортування
виробу з одних рук до одних та з одного процесу до одного.
З погляду великої промисловости ця обставина виступає як характеристична
та іманентна принципові мануфактури обмеженість,
що удорожчує продукцію.\footnote{
«Ізоляція різних стадій мануфактури, що постає в наслідок вживання
ручної праці, надзвичайно збільшує витрати продукції, при чому
втрата виникає, головне, з самого лише переходу від одного процесу
до одного» («The isolation of the different stages of manufacture consequent
upon the employment of the manual labour adds immensely to
the cost of production, the loss mainly arising from the mere removals from
one process to another»). («The Industry of Nations», London 1855,
Part. II, p. 200).
}

Коли подивимось на певну кількість сировинного матеріялу,
приміром, ганчірок у паперовій мануфактурі або дроту в голчаній
мануфактурі, то побачимо, що цей матеріял перебігає в руках
різних частинних робітників почерговий щодо часу ряд фаз

дуктів, сама по собі дуже утруднює перетворення таких мануфактур на
машинове виробництво великої промисловости, то при продукції годинників
сюди долучаються ще дві інші перешкоди: дрібність і тендітність
їхніх елементів та їхній люксусовий характер, отже і різноманітність їх,
наприклад, ліпші лондонські фірми протягом цілого року ледве чи виробляють
тузінь годинників, які були б подібні один до одного. Фабрика
годинників Vacheron and Constantin, що з успіхом вживає машин,
дає щонайбільше три-чотири відміни годинників, різних щодо величини
й форми.