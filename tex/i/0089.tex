тої або срібної монети. Ми бачили, як з постійними коливаннями
товарової циркуляції щодо розміру, цін і швидкости безперестанно
припливає й відпливає маса грошей, що є в обігу. Отже, вона
мусить бути здатна до звуження та розширення. Часом гроші
мусять притягатися як монета, часом монета відштовхуватися
як гроші. Щоб маса грошей, які дійсно є в обігу, завжди відповідала
ступеневі насичення сфери циркуляції, кількість золота
або срібла, що є у країні, мусить бути більша, ніж та, що виконує
монетну функцію. Ця умова виконується через скарбову
форму грошей. Резервуари скарбів одночасно служать за відпливні
й припливні канали для грошей, що циркулюють, тим то
гроші ніколи не переповнюють каналів свого обігу.96

Ь) Засіб платежу

В розглянутій досі безпосередній формі товарової циркуляції
та сама величина вартости завжди з’являлась двоїсто: як товар —
на одному полюсі, як гроші — на протилежному полюсі. Тому
посідачі товарів увіходили в контакт лише як представники
наявних взаємних еквівалентів. Однак з розвитком товарової
циркуляції розвиваються обставини, в наслідок яких відчуження
товарів відокремлюється у часі від реалізації їхніх цін. Тут
досить буде зазначити найпростіші з цих обставин. Один рід
товару потребує для своєї продукції довшого, інший коротшого
часу. Продукція різних товарів пов’язана з різними сезонами.
Один товар родиться на своєму ринку, інший мусить мандрувати
на далекий ринок. Тим то один посідач товарів може виступити
як продавець раніш, ніж інший як покупець. За постійного повто-

95 «Щоб нація могла вести свою торгівлю, потрібна певна сума
готівки, яка може змінятись, то збільшуючись, то зменшуючись залежно
від обставин... Ці припливи й відпливи грошей урівноважуються
сами собою без жодної допомоги з боку політиків... Відра працюють почережно:
коли мало грошей, то карбують монету, а коли мало грошового
металю — перетоплюють монету знов у зливки». («There is required for
carrying on the trade of the nation, a determinate sum of specifick money,
which varies, and is sometimes more, sometimes less, as the circumstances
we are in require... This ebbing and flowing of money, supplies and accommodates
itself, without any aid of Politicians... The buckets work alternately;
when money is scarce, bullion is coined; when bullion is scarce, money
is metled»). (Sir D. North: «Discourses upon Trade», London 1691, p. 22).
Джон Стюарт Мілл, довгий час урядовець східньо-індійської компанії,
потверджує, що в Індії срібна оздоба все ще функціонує безпосередньо як
скарб: «Срібні оздоби витягуються із сховищ і перекарбовуються в монету,
коли норма проценту висока, і знов повертаються назад, коли норма проценту
падає» («Silver ornaments are brought out and conined when there
is a high rate of interest, and go back agaiu when the rate of interest fales»).
(J. St. Mill’s: «Evidence Reports on Bankacts», 1857, p. 2084). За одним
парламентським документом з 1864 р. про імпорт золота й срібла і експорт
до Індії, імпорт золота й срібла в 1863 р. перевищив експорт на 19.367.764
фунти стерлінґів. За вісім останніх років перед 1864 р. лишок імпорту
благородних металів над експортом їх становив 109.652.917 фунтів стерлінґів.
Протягом цього століття в Індії викарбовано монети далеко понад
200.000.000 фунтів стерлінґів.
