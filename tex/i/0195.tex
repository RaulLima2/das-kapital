на той світ. Причина нещастя — недбальство залізничників. Вони
в один голос заявляють перед присяжними, що років 10—12 тому
їхня праця тривала лише 8 годин на день. За останні 5—6 років
робочий час було доведено до 14, 18 і 20 годин, а за особливо
жвавого припливу пасажирів, як от підчас періоду екскурсій,
робочий час триває часто без перерви 40—50 годин. Вони, мовляв,
звичайні люди, а не циклопи. На якомусь даному пункті
їхня робоча сила відмовляється служити. Їх обіймає одубілість,
мозок перестає думати, очі — бачити. Цілком «respectable Вгіtish,
Juryman»\footnote*{
— поважний британський присяжний засідатель. \emph{Ред.}
} відповідає на це присудом, що відсилає їх до
карного суду за «manslaughter»,\footnote*{
— вбивство. \emph{Ред.}
} і в додатковому пункті в лагідному
тоні висловлює побожне побажання, щоб пани-маґнати
від залізничного капіталу надалі були щедріші, купуючи потрібну
кількість «робочих сил», і щоб виявляли більше «поздержливости»,
або «самовідречення», або «ощадности», висисаючи
оплачену робочу силу.\footnote{
«Reynolds, Newspaper», січень 1866 р. Ця сама тижнева газета щотижня
подає цілу низку нових залізничних катастроф під «sensational headings\footnote*{
— сенсаційними заголовками. \emph{Ред.}
}:
«Fearful and fatal accidents»,\footnote*{
Жахливий і фатальний випадок. \emph{Ред.}
} «Appalling tragedies»\footnote*{
Жахлива трагедія. \emph{Ред.}
} і т. ін.
Це викликало таку відповідь одного робітника з північно-стафордшірської
залізниці: «Кожний знає, до яких наслідків доводить навіть хвилинне
ослаблення уваги в машиніста й кочегара. І як воно може бути інакше
за безмірного здовження праці, в найсуворішу годину, без перерви й відпочинку?
Візьміть як приклад такий випадок із щоденного життя. Минулого
понеділка кочегар почав свою денну працю дуже раннім ранком.
Він скінчив її після 14 годин 50 хвилин. Не встиг він іще напитись чаю,
як його знов покликали до праці. Таким чином, він повинен був працювати
без перерви 29 годин 15 хвилин. Дальші пні тижня він був занятий ось
як: середа — 15 годин, четвер — 15 годин 35 хвилин, п’ятниця — 14\sfrac{1}{2} годин,
субота — 14 годин 10 хвилин, разом 88 годин 40 хвилин за тиждень.
Уявіть же ви собі його здивовання, коли він одержав плату лише за 6 робочих
днів. Цей робітник був новак і спитав, що треба розуміти під робочим
днем. Відповідь: 13 годин, тобто 78 годин на тиждень.. Але як же тоді
з виплатою за додаткові 10 годин 40 хвилин? Після довгих суперечок він
одержав 10 пенсів (не цілих 10 срібних шагів) винагороди за цей час».
(Там же, число з 4 лютого 1866 р.).
}

Із строкатої юрби робітників усіх професій, усякого віку й
статі, що насовуються на нас настирливіше, ніж тіні вбитих на
Одіссея, і обличчя яких на перший же погляд, коли навіть не
зазирати до тих Синіх Книг, що вони тримають їх під пахвами,
вказують на надмірну працю, — з цієї юрби робітників виберімо
собі ще дві постаті: кравчиху й коваля, яскравий контраст між
якими доводить, що перед капіталом усі люди рівні.

В останніх тижнях червня 1863 р. всі лондонські газети принесли
замітку під «сенсаційним» заголовком «Death from simple
Owerwork» («Смерть просто від надмірної праці»). Йшлося про
смерть двадцятилітньої кравчихи Мері Енні Волклей, що працювала
в дуже поважній придворній кравецькій майстерні, яку екс-