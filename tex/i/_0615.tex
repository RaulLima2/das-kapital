\parcont{}  %% абзац починається на попередній сторінці
\index{i}{0615}  %% посилання на сторінку оригінального видання
стичну продукцію, а ця остання — наявність великих мас капіталу
й робочої сили в руках товаропродуцентів. Таким чином
увесь цей рух, здається, обертається у зачарованому колі, з
якого ми не можемо вийти інакше, як припустивши, що капіталістичній
акумуляції передувала «первісна» акумуляція («previous
accumulation» у Адама Сміса), акумуляція, що є не результат
капіталістичного способу продукції, а його вихідний пункт.

Ця первісна акумуляція відіграє в політичній економії приблизно
ту саму ролю, що й первісний гріх у теології. Адам покуштував
яблука, і таким чином гріх увійшов у рід людський.
Походження цієї акумуляції пояснюють, оповідаючи про нього
анекдоту давноминулих часів. За дуже давніх часів були, з
одного боку, працьовиті, розумні й насамперед ощадливі обранці,
а з другого боку — ледарі, голодранці, які прогулювали все,
що мали, і навіть ще більше. Правда, теологічна леґенда про
гріхопадіння розповідає нам, як засуждено людину їсти хліб
у поті чола свого; навпаки, історія економічного гріхопадіння
викриває, як це сталося, що є люди, які цього зовсім не потребують.
Так сталось, що перші нагромадили багатство, а в
других кінець-кінцем нічого було продавати, крім власної
шкури. І з цього гріхопадіння починаються злидні великої маси
людей, що їм все ще, не зважаючи на всю їхню працю, нічого
продати, крім себе самих, і багатство небагатьох, яке невпинно
зростає, хоч вони дуже давно перестали працювати. Такі безглузді
дитячі байки пережовує, наприклад, пан Тьєр, який,
щоб захистити propriété,\footnote*{
— власність. Ред.
} з державно-урочистою серйозністю
підносить ці байки французам, що колись були такі дотепні.
Скоро тільки справа торкається питання про власність, то стає
священним обов’язком кожного дотримуватись поглядів дитячого
букваря як єдино правильних для всякого віку й усіх ступенів
розвитку.\footnote*{
У французькому виданні Маркс додає до цього таку примітку:
«Ґьоте, роздратований цими бреднями, висміює їх у такому діялозі:

Вчитель: Скажи мені, звідки взялося багатство твого батька?

Дитина: Від діда.

Вчитель: А звідки воно взялося в діда?

Дитина: Від прадіда.

Вчитель: А в прадіда?

Дитина: Він загарбав його». Ред.
} Як відомо, в дійсній історії велику ролю відіграють
завоювання, поневолення, розбій, коротко кажучи, насильство.
Але в лагідній політичній економії з давніх часів панувала
ідилія. Право й «праця» з давніх часів були єдиним засобом
збагачення, звичайно, щоразу за винятком «поточного року».
А в дійсності методи первісної акумуляції — це все, що хочете,
тільки не ідилія.

Гроші й товари, так само, як засоби продукції й засоби існування,
сами собою не є капітал. їх треба перетворити на капітал.
Але саме це перетворення може відбуватися лише за певних
обставин, які сходять ось на що: два дуже різні сорти
\parbreak{}  %% абзац продовжується на наступній сторінці
