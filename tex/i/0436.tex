дукції, більшу або меншу кількість продукту. Очевидно, можливі
дуже різні комбінації, відповідно до того, чи один із трьох факторів
сталий, а два змінюються, чи два фактори сталі, а один
змінюється, чи, нарешті, всі три змінюються одночасно. Число
цих комбінацій збільшується ще й через те, що за одночасної
зміни різних факторів величина й напрям змін можуть бути
різні. Далі ми розглядаємо лише головні комбінації.

І. Величина робочого дня й інтенсивність праці сталі (дані),
продуктивна сила праці змінюється

При цьому припущенні вартість робочої сили й додаткової
вартости визначається трьома законами:

По-перше, робочий день даної величини завжди виражається
в тій самій новоспродукованій вартості, хоч би й як змінювалася
продуктивність праці і разом з нею маса продуктів, а тому й
ціна поодинокого товару.

Новоспродукована вартість дванадцятигодинного робочого
дня, є, наприклад, 6 шилінґів, хоч маса спродукованих споживних
вартостей змінюється з продуктивною силою праці, і вартість
6 шилінґів розподіляється, отже, на більшу або меншу кількість
товарів.\footnote*{
У французькому виданні останні два абзаци подано так: «По-перше,
робочий день даної величини продукує завжди ту саму вартість,
хоч би й як змінювалася продуктивність праці.

Якщо одна година праці нормальної інтенсивности продукує вартість
у \sfrac{1}{2} шилінґа, то дванадцятигодинний робочий день може спродукувати
лише вартість у 6 шилінґів. (Ми припускаємо завжди, що вартість грошей
лишається незмінна). Якщо продуктивність праці підвищується або зменшується,
то той самий робочий день дасть більше або менше продуктів,
і вартість у 6 шилінґів розподілиться таким чином на більшу або меншу
кількість товарів». («Le Capital etc.», v. I, ch. XVII, p. 224). \emph{Ред.}
}

По-друге, вартість робочої сили й додаткова вартість змінюються
в протилежному напрямі. Зміна продуктивної сили праці,
її зростання або зменшення, впливає на вартість робочої сили у
зворотному напрямі, на додаткову вартість — у простому.

Новоспродукована вартість дванадцятигодинного робочого
дня є стала величина, наприклад, 6 шилінґів. Ця стала величина
дорівнює сумі додаткової вартости плюс вартість робочої сили,
яку робітник заміщує еквівалентом. Само собою зрозуміло, що
з двох частин сталої величини жодна не може збільшитися без
того, щоб друга не зменшилася. Вартість робочої сили не може
підвищитися з 3 шилінґів до 4 без того, щоб додаткова вартість
не знизилася з 3 шилінґів до 2, а додаткова вартість не може
підвищитися з 3 шилінґів до 4 без того, щоб вартість робочої
сили не знизилася з 3 шилінґів до 2. Отже, за цих обставин неможлива
ніяка зміна абсолютної величини ані вартости робочої
сили, ані додаткової вартости без одночасної зміни їхніх відносних
або пропорціональних величин. Неможливо, щоб вони одночасно
падали або підносилися.