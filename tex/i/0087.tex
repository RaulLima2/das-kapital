Але гроші сами є товар, зовнішня річ, яка може стати приватною
власністю кожного. Таким чином суспільна сила стає приватною
силою приватної особи. Тим то античне суспільство лихословить
гроші як монету, на яку розмінявся їхній економічний та моральний
лад.*92 Сучасне суспільство, яке вже в своєму дитячому
віці витягає\footnote{
«Жадобою сподіваються самого Плутоса витягти з середини землі»
(«Ἐλπιζούσης  πλεονεξίας ἀνάξειν ἐκ τῶν μυκῶν τῆς γῆς αὐτὸν τὸν Πλούτωνα). (Atheneus:
«Deipnosophistai»).
} Плутоса за чуба з надр землі, вітає в золотому
Ґраалі\footnote*{
Ґрааль — y середньовічних леґендах чаша, зроблена з самоцвіту,
в яку нібито зібрано кров з ран розп’ятого Ісуса. Ред.
} блискуче втілення свого власного життєвого принципу.

Товар як споживна вартість задовольняє окрему потребу й
становить окремий елемент речового багатства. Але вартість

And give them title, knee and approbation
With senators of the bench; this is it,
That makes the wappen’d widow wed again
... Come damned earth,
Thou commun whore of mankind.

[О золото, блискуче, жовте, золоте,
Із черні біле робиш, з гнилого — гарне,
Із кривди — правду, з підлого — високе,
З старого — молоде, героя — з боягуза.

Чому це, боги? Як воно

Ваших жерців і слуг одштовхує від вас
І спати не дає здоровим людям?
Цей жовтий раб святе
Пов’язує і рве; благословляє те, що
проклинають,
Обожнює гидку проказу, злодія виносить
В сенат, і ранґ, і честь, і славу
Йому дає; і удову плачущу
За молодого заміж видає.
Метале клятий,
Повія ти для всього людства спільна!]

(Шекспір: «Тімон із Атен»).

92 „Οὐδέν γὰρ ἀνθρώποισιν οἷον ἀργυρὸς
Κακὸν νόμισμα ἔβλαστε, τοῦτο καὶ πόλεις
Πορθεῖ, τόδ’ ἄνδρας ἐξανίστησιν δόμων.
Τόδ’ ἐκδιδάσκει καὶ παραλλάσσει φρένας
Χρηστὰς πρὸς αἰσχρὰ ἀνθρώποις ἔχειν,
Καὶ παντὸς ἔργου δυσσέβειαν εἰδέναι.“

[Немає лиха гіршого за гроші,
За срібло те. Міста воно руйнує,
Людей з домів навіки виганяє,
Невинні душі научає зла,
Скеровує людину до розпусти,
В усякій справі путь показує безбожну].

(Софокл: «Антігона»).

* У французькому виданні це речення подано так: «Тим-то
античне
суспільство лихословить гроші як найактивнішого підривника й руїнника
його економічної організації і народніх звичаїв». («Le Capital etc.» v. I,
ch. III, p. 54). Ред.