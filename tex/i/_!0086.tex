\parcont{}  %% абзац починається на попередній сторінці
\index{i}{*0086}  %% посилання на сторінку оригінального видання
відкрив цей закон, він досліджує в подробицях наслідки, що в
них цей закон виявляється в суспільному житті\dots{} Відповідно до
того Маркс піклується лише про одно: шляхом точного наукового
досліду довести неминучість певного ладу суспільних відносин
і якомога бездоганніше констатувати факти, що служать йому
за вихідні пункти та опору. Для цього цілком досить, якщо він,
довівши неминучість сучасного ладу, довів разом із тим і неминучість
іншого ладу, в який неодмінно мусить перейти перший, все
одно, чи думають про це люди, чи не думають, чи свідомі вони
цього, чи несвідомі. Маркс розглядає суспільний рух як природно-історичний
процес, що ним керують закони, які не тільки незалежні
від волі, свідомости й намірів людей, а, навпаки, сами
визначають волю, свідомість і наміри людей\dots{} Якщо свідомий
елемент відіграє в історії культури таку підпорядковану ролю,
то само собою зрозуміло, що критика, об’єктом якої є сама культура,
менш, ніж щобудь інше, може мати собі за основу якусь форму
або якийсь результат свідомости. Отже, не ідея, а тільки зовнішнє
явище може служити їй за вихідний пункт. Критика сходитиме
на порівняння і зіставлення якогось факту не з ідеєю, але з іншим
фактом. Для неї важливо лише, щоб обидва факти були якомога
точніше досліджені й щоб вони дійсно являли собою один проти
одного різні моменти розвитку, та передусім важливо, щоб не
менш точно було досліджено порядок, послідовність і зв’язок
що в них виявляються ці ступені розвитку. Але, скажуть нам,
загальні закони економічного життя ті самі, однаково, чи
прикладається їх до сучасного життя, чи до минулого. Оце саме
й заперечує Маркс. За Марксом, таких абстрактних законів не
існує\dots{} На його думку, навпаки, кожний історичний період має
свої власні закони\dots{} Скоро життя перебуде даний період розвитку,
перейде з даної стадії розвитку в іншу, то починають ним керувати
вже інші закони. Одно слово, економічне життя подає нам
явище, аналогічне до історії розвитку на інших ділянках біології\dots{}

Давні економісти помилялися щодо природи економічних законів,
прирівнюючи їх до законів фізики й хемії\dots{} Глибша аналіза
явищ довела, що соціяльні організми так само ґрунтовно відрізняються
один від одного, як і організми рослин і тварин\dots{} Навіть
більше, те саме явище підлягає цілком різним законам у наслідок
відмінности в будові цих організмів, різнорідности їхніх окремих
органів, відмінности умов, серед яких вони функціонують, і т. ін.
Маркс заперечує, приміром, що закон залюднення для всіх часів
і для всіх місць однаковий. Навпаки, він запевняє, що кожний
ступінь розвитку має свій власний закон залюднення\dots{} Разом із
зміною ступеня розвитку продуктивних сил змінюються і відносини
і закони, що ними керують. Ставлячи собі за мету з цього
погляду дослідити й пояснити капіталістичний господарський лад,
Маркс лише строго науково формулює мету, яку мусить мати
кожний точний дослід економічного життя\dots{} Наукова цінність
такого досліду — вияснити особливі закони, що керують виникненням,
існуванням, розвитком, смертю даного суспільного організму
\parbreak{}  %% абзац продовжується на наступній сторінці
