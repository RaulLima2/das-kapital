докупи осіб обох статей різного віку й різних нахилів, що заразливість прикладу мусить привести до
зіпсованости й розпусти, — чи може така мануфактура збільшити суму національного й індивідуального
щастя?».245 «В Дербішірі, Нотінґемшірі й особливо
Ланкашірі, — каже Фелден, — ужито недавно винайдені машини на великих фабриках, побудованих над
річками, що могли пускати в рух водяне колесо. Одразу постала потреба в тисячах робочих рук у цих
місцевостях, віддалених від міст; і особливо
Ланкашір, до того часу порівняно мало залюднений і неродючий, потребував тепер перш за все людности.
Потрібні були насамперед маленькі й меткі дитячі руки. Одразу ж повівся звичай набирати учнів (!) із
лондонських, бірмінґемських та інших парафіяльних робітних домів. Таким чином багато-багато тисяч
цих маленьких безпорадних істот від 7 до 13 або 14 років життя вивезено на північ. У хазяїна (тобто
крадія дітей) повівся звичай одягати, годувати й приміщувати своїх учнів у «будинку для учнів»
близько фабрик. Він наймав доглядачів, що мали стежити за їхньою працею. В інтересах цих доглядачів
за рабами було примушувати дітей працювати понад усяку міру, бо їхня плата залежала від кількости
продукту, що її можна було видушити
з дітей. Природний наслідок цього була жорстокість... По багатьох фабричних округах, особливо ж у
Ланкашірі, цих невинних і беззахисних істот, відданих на волю фабрикантів, катували з надзвичайною
жорстокістю. Їх замордовували до смерти надмірною працею... їх били батогами, заковували в кайдани й
катували з найвишуканішою витонченістю й жорстокістю; у багатьох випадках їх виснажували голодом до
шкури-кости, і все ж батогом примушували до праці... В деяких випадках їх доводили навіть до
самогубства!.. Чудові й романтичні долини Дербішіру, Нотінґемшіру та Ланкашіру, заховані від
громадського ока, стали жахливим місцем катувань і — часто вбивства!... Зиски фабрикантів були
величезні. Це лише розпалювало їхню вовчу ненажерливість. Вони почали заводити нічну працю, тобто
тримали напоготові для нічної праці групу робітників, що заступала другу групу робітників,
знесилених денною працею; денна йшла до ліжок, які тільки но покинула нічна група, і навпаки. У
Ланкашірі є народній переказ, що ці ліжка ніколи не простигали».246

245    Eden: «The State of the Poor», v. II, ch. I, p. 421.

246 John Fielden: «The Curse of the Factory System», London 1836, p. 5, 6. Про всі ті гидоти, які
творилися від початку фабричної системи порівн. Dr. Aikin: «Description of the Country from thirty
to forty miles round Manchester», London 1795, p. 129 та Gisborne: «Enquiry into the duties of men»,
1795, vol II. — Через те, що парова машина перенесла
фабрики від сільських водоспадів до центру міст, то «прихильний до поздержливости» капіталіст
(Plusmacher) находив дитячий матеріял під рукою, так що не треба було насильно транспортувати рабів
із робітних домів. — Коли сер Роберт Піл (батько «міністра уважливости») запропонував у 1815 р. біл
для охорони дітей, Ф. Горнер (світило «Bul-
