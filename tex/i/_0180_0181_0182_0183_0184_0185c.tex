\index{i}{0180}  %% посилання на сторінку оригінального видання
Порівняння ненажерливої жадоби до додаткової праці в дунайських
князівствах з такою самою жадобою на англійських
фабриках має особливий інтерес, бо додаткова праця панщини
має самостійну, почуттєво-сприйману форму.

Припустімо, що робочий день становить 6 годин доконечної
й 6 годин додаткової праці. Таким чином вільний робітник
постачає капіталістові щотижня 6 × 6, або 36 годин додаткової
праці. Це те саме, як коли б він працював 3 дні на тиждень для
себе й 3 дні на тиждень задурно для капіталіста. Але це не впадає
в очі. Додаткова праця й доконечна праця розпливаються одна
в одній. Тому я міг би висловити те саме відношення, приміром,
і таким чином, що робітник протягом кожної хвилини працює
30 секунд для себе й 30 секунд для капіталіста й~\abbr{т. ін.} Інакше
стоїть справа з панщиною. Доконечна праця, яку виконує, приміром,
волоський селянин, щоб утримати себе самого, просторово
відокремлена від його додаткової праці на боярина. Одну працю
він виконує на своєму власному полі, другу — на панському
маєтку. Отже, обидві частини робочого часу існують самостійно
одна побіч однієї. У формі панщини додаткова праця точно відокремлена
від доконечної праці. Ці різні форми виявлення, очевидно,
нічого не змінюють у кількісному відношенні між додатковою
працею і доконечною працею. Три дні додаткової праці на
тиждень завжди лишаються трьома днями такої праці, яка не
створює жодного еквіваленту для самого робітника, хоч вона
називатиметься кріпацькою, хоч найманою працею. Однак у
капіталіста ненажерлива спрага до додаткової праці виявляється
у прагненні до безмірного подовження робочого дня, у боярина
простіш — у безпосередній гонитві за панщинними днями.\footnote{
Дальші рядки стосуються становища румунських провінцій, як
воно було перед переворотом від часів кримської війни.
}

В дунайських князівствах панщинна праця була сполучена
з натуральними рентами та всякими іншими атрибутами кріпацтва;
але вона становила значну данину, яку платилося панівній
клясі. За подібних умов панщинна праця рідко випливає з
кріпацтва, навпаки, кріпацтво здебільшого випливає з панщинної
праці.\footnoteA{
[Примітка до третього видання. Це має силу і щодо Німеччини,
особливо щодо Прусії на сході від Ельби. В XV віці німецький селянин
хоч і підлягав майже всюди деяким повинностям у продуктах і праці, але
зрештою, принаймні фактично був вільною людиною. Німецьких колоністів
у Бранденбурзі, Померанії, Шлезьку та східній Прусії навіть юридично
визнавано вільними. Перемога шляхти в селянській війні поклала
цьому кінець. Не лише переможені селяни південної Німеччини стали
знов кріпаками, але вже від половини XVI століття вільних селян східньої
Прусії, Бранденбурґу, Померанії й Шлезьку, а незабаром і Шлезвіґ-Гольштайну
понижено до стану кріпаків. (\emph{Maurer}: «Geschichte
der Fronhöfe, der Bauernhöfe und der Hofverfassung in Deutschland», Erlangen
1862/63, Bd. IV. — \emph{Meitzen}: «Der Boden und die landwirtschaftlichen
Verhältnisse des preußischen Staates nach dem Gebietsumfange von
1866», Berlin 1873. — \emph{Hansen} «Leibeigenschaft in Schleswig-Holstein».) —
F. E.].
} Так було в румунських провінціях. Їхній первісний
\index{i}{0181}  %% посилання на сторінку оригінального видання
спосіб продукції ґрунтувався на громадській власності, але
не на громадській власності в слов’янській або навіть індійській
формі. На одній частині землі самостійно господарювали члени
громад, як на вільній приватній власності, другу частину — ager
publicus\footnote*{
— громадську землю. \emph{Ред.}
} — обробляли вони спільно. Продукти цієї спільної праці
почасти служили за резервний фонд на випадок неврожаю й інших
несподіванок, а почасти були державним фондом для покриття
коштів війни, релігійних і інших громадських видатків. З часом
військові й церковні достойники захопили разом з громадською
власністю і зв’язані з нею повинності. Праця вільних селян на
їхній громадській землі перетворилась на панщинну працю для
розкрадачів громадської землі. Відси одночасно з цим розвинулися
кріпацькі відносини, однак тільки фактично, а не юридично,
аж поки всесвітня визвольниця Росія під приводом скасування
кріпацтва піднесла їх до рівня закону. Кодекс панщинної праці,
оголошений 1831~\abbr{р.} від російського генерала Кісельова, звичайно,
подиктували сами бояри. Таким чином Росія одним ударом завоювала
собі маґнатів дунайських князівств і похвальні оплески
ліберальних кретинів цілої Европи.

За «Réglement organique» — так називається той кодекс панщинної
праці — кожний волоський селянин, окрім безлічі докладно
перелічених натуральних повинностей, має ще супроти так
званого земельного власника такі обов’язки: 1) дванадцять робочих
днів взагалі, 2) один день працювати на полі та 3) один день
возити дрова, — отже, разом 14 днів на рік. Однак з глибоким розумінням
політичної економії робочий день узято не в його звичайному
значенні, а як робочий день, доконечний на виготовлення
пересічного денного продукту, а пересічний денний продукт так
хитро визначено, що й жоден циклоп не впорався б з ним за
24 години. Тому сам реґлямент сухими словами з справжньою
руською іронією пояснює, що під 12 робочими днями треба розуміти
продукт 36 днів ручної праці, під одним днем роботи на полі —
три дні, під одним днем возіння дров — теж три дні, разом це
42 панщинні дні. Та сюди ще додано так звану «Jobagie» — службові
повинності, які треба виконувати на користь землевласника
в надзвичайних випадках, що їх висувають потреби продукції.
Кожне село мусить щорічно постачати для «Jobagie» певний
континґент робітників відповідно до кількости своєї людности.
Ця додаткова панщинна праця для кожного волоського селянина
становить 14 днів. Таким чином, обов’язкова панщинна праця
становить річно 56 днів. Але рільничий рік у Волощині через
поганий клімат має лише 210 днів, з яких припадає 40 днів на
неділі й свята, 30 днів пересічно — на негоду, а разом відпадає
70 днів. Лишається 140 робочих днів. Відношення панщинної
праці до доконечної праці, або \frac{56}{84}, або 66\sfrac{2}{3}\%, виражає норму додаткової
вартости, куди меншу за ту, що реґулює працю англійського
\index{i}{0182}  %% посилання на сторінку оригінального видання
рільничого або фабричного робітника. Однак це лише
законом приписана панщинна праця. A «Réglement organique»
ще в «ліберальнішому дусі», ніж англійське фабричне законодавство,
зумів улегшити собі способи нехтувати себе самого.
Зробивши з 12 днів 54, він знову визначає номінальну денну працю
кожного з тих 54 панщинних днів так, що на дальші дні мусить
припадати якийсь додаток праці. Приміром, за один день слід
виполоти якийсь шмат поля — операція, яка, особливо на кукурудзяному
полі, потребує удвоє більше часу. Установлену законом
денну працю для деяких рільничих робіт можна тлумачити
так, що вона повинна починатися у травні, а кінчатися у жовтні.
Для Молдавії постанови ще суворіші. «Дванадцять панщинних
днів, що їх приписує «Réglement organique» — гукнув один
сп’янілий від перемоги боярин. — складають 365 днів на рік!».\footnote{
Дальші подробиці можна найти у \emph{Е. Regnault}: «Histoire politique
et sociale des Principautés Danubiennes», Paris 1855.
}

Коли «Réglement organique» дунайських князівств був позитивним
висловом ненажерливої жадоби до додаткової праці, яку
леґалізує кожний його параграф, то англійські Factory-Acts
є неґативні вислови тієї самої ненажерливої жадоби. Ці закони
загнуздують прагнення капіталу до безмірного висмоктування
робочої сили, встановлюючи примусове обмеження робочого дня
державою, і до того ж державою, в якій панує капіталіст і лендлорд.
Не кажучи вже про робітничий рух, що з дня на день загрозливіше
зростав, обмеження фабричної праці було подиктоване
тією самою доконечністю, яка примусила виливати ґуано
на англійські поля. Те саме сліпе хижацтво, що в одному випадку
виснажило землю, в другому випадку підривало життєві сили
нації в самому корінні. Періодичні епідемії тут свідчать про це
так само виразно, як і зменшення міри зросту в солдатів у Німеччині
і у Франції.\footnote{
«Загалом, зріст органічних істот поза пересічну міру свого роду
свідчить, у певних межах, про їхній розцвіт. Розмір тіла в людини меншає,
коли її добробутові шкодять фізичні або соціяльні обставини. У всіх європейських
країнах, де існує конскрипція, від часу, як її заведено, середній
зріст дорослої людини й загалом її здатність до військової служби зменшились.
Перед революцією (1789~\abbr{р.}) мінімум для піхотинця у Франції
становив 165 сантиметрів, у 1818~\abbr{р.} (закон з 10 березня) — 157 сантиметрів,
за законом з 21 березня 1852~\abbr{р.} — 156 сантиметрів; пересічно у Франції
більш ніж половину рекрутів визнають за нездатних через недостатній
зріст та фізичні хиби. В Саксонії військова міра була 1780~\abbr{р.} 178 сантиметрів,
тепер — 155 сантиметрів. У Прусії тепер — 157 сантиметрів. За даними
д-ра Маєра в «Bayerischen Zeitung» від 9 травня 1862~\abbr{р.} виходить,
що в Прусії за дев’ятилітній період пересічно із \num{1.000} рекрутів 716 визнають
за нездатних до військової служби: 317 — через недостатній зріст і
399 — через фізичні хиби\dots{} В 1858~\abbr{р.} Берлін не міг виставити відповідного
контингенту рекрутів — бракувало 156 осіб». (\emph{J. v. Liebig}: «Die
Chemie in ihrer Anwendung auf Agrikultur und Physiologie», 7. Auflage.
1862. Bd. I, S. 117, 118).
}

Factory Act 1850~\abbr{р.}, що має тепер (1867) силу, дозволяє пересічний
тижневий день у 10 годин, а саме для перших п’ятьох
днів тижня по 12 годин, від шостої години ранку до шостої години
\index{i}{0183}  %% посилання на сторінку оригінального видання
вечора, але з них закон приділяє півгодини на сніданок і
одну годину на обід, отже, лишається 10\% робочих годин, і для
суботи 8 годин, від шостої години ранку до другої години по
півдні, з чого півгодини відпадає на сніданок. Лишається 60 робочих
годин, по 10\sfrac{1}{2} для перших п’яти днів тижня, 7\sfrac{1}{2} для
останнього дня тижня.\footnote{
Історію фабричного закону з 1850~\abbr{р.} викладено на протязі цього
розділу.
} Для слідкування за виконанням цього
закону призначено окремих наглядачів, фабричних інспекторів
безпосередньо підлеглих міністерству внутрішніх справ, фабричних
інспекторів, що їхні звіти опубліковує щопівроку парлямент.
Отже, вони дають постійну й офіційну статистику про капіталістичну
ненажерливу жадобу до додаткової праці.

Послухаймо хвилинку фабричних інспекторів.\footnote{
Періоду від початку великої промисловости в Англії й до 1845~\abbr{р.}
я торкаюсь лише в деяких місцях і відсилаю читача до твору «Die Lage
der arbeitenden Klasse in England», Von Friedrich Engels. Leipzig 1845
(\emph{Ф. Енґельс}: «Становище робітничої кляси в Англії»). Як глибоко збагнув
Енгельс дух капіталістичного способу продукції, свідчать Factory Reports,
Reports on Mines\footnote*{
— звіти фабричних інспекторів, звіти гірничих інспекторів. \emph{Ред.}
} і~\abbr{т. ін.}, що появилися після 1845 p.; a як напрочуд
гарно він змалював у подробицях становище робітничої кляси,
показує якнайповерховіше порівняння його твору з офіційними звітами
«Children’s Employment Commission»\footnote*{
— комісії для вивчення праці дітей. \emph{Ред.}
} (за 1863--67~\abbr{рр.}), що появилися
18--20 років пізніше. У них йде мова саме про ті галузі промисловости,
де фабричного законодавства ще не було заведено до 1862~\abbr{р.}, а почасти не
заведено ще й досі. Отже, становище, змальоване тут Енгельсом, не зазнало
більш-менш значних змін під впливом зовнішніх сил. Свої приклади
я беру переважно з періоду вільної торговлі після 1848~\abbr{р.}, того райського
часу, про який так казково багато понаговорили німцям стільки ж пащекуваті,
як і з наукового боку вбогі комівояжери вільної торговлі. Зрештою, Англія фігурує тут на першому
місці лише тому, що вона є клясична
представниця капіталістичної продукції і що лише вона одна має
постійну офіціяльну статистику про ті явища, про які тут мовиться.
}

«Фабрикант-ошуканець починає працю чверть години, іноді
раніше, іноді пізніше, перед шостою годиною ранку і кінчає її
чверть години, іноді раніше, іноді пізніше, після шостої годнии
вечора. Він одбирає по 5 хвилин від початку й кінця тієї півгодини,
що номінально призначена на сніданок, і по 10 хвилин від
початку й кінця тієї години, що призначена на обід. В суботу в
нього працюють чверть години, іноді більше, іноді менше, після
другої години по півдні. Таким чином його виграш становить:

\noindent\begin{tabularx}{\textwidth}{X@{}r@{~}l|c}
Перед шостою годиною вранці\dotfill{} & 15 & хвилин & \multirowcell{5}{Разом за 5 днів:\\300 хвилин} \\
Після шостої години вечора\dotfill{} & 15 & \dittomark  & \\
На час сніданку\dotfill{} & 10 & \dittomark  & \\
На час обіду\dotfill{} & 20 & \dittomark & \\
~ & 60 & хвилин & \\
\addlinespace
  & \multicolumn{2}{r}{Суботами:} & \\
Перед шостою годиною вранці\dotfill{} & 15 & хвилин & \multirowcell{3}{Цілий тижневий \\виграш: \\ 340 хвилин} \\
На час сніданку\dotfill{} & 10 & \dittomark & \\
Після другої години по півдні\dotfill{} & 15 & \dittomark & \\
\end{tabularx}


\index{i}{0184}  %% посилання на сторінку оригінального видання
Або 5 годин 40 хвилин на тиждень, а це, помножене на 50
робочих тижнів, за вирахуванням 2 тижнів на свята або випадкові
перерви праці, дає 27 робочих днів».\footnote{
«Suggestions etc. by Mr. L. Horner, Inspector of Factories», у
«Factories Regulation Act, Ordered by the House of Commons to be printed,
9. August 1859», p. 4, 5.
}

«Коли робочий день щодня подовжують на 5 хвилин поза
межі його нормального тривання, то це дає 2\sfrac{1}{2} робочих днів на
рік».\footnote{
«Reports of the Insp. of Fact, for the half year ending October 1856»,
p. 35.
} «Одна додаткова година на день, яку здобувається таким
чином, що шматок часу відривається то тут, то там, робить із
дванадцятьох місяців року тринадцять.\footnote{
«Reports etc. 30 th April 1858», p. 9.
}

Кризи, підчас яких продукція притіняється і працюють лише
«коротший час», лише по декілька днів на тиждень, звичайно,
нічого не змінюють у прагненні до подовження робочого дня.
Що менше робиться оборудок, то більший мусить бути виграш
від зробленої оборудки. Що менше часу можна працювати, то
більше додаткового робочого часу треба працювати. Ось які
звіти подають фабричні інспектори про період кризи від 1857 до
1858~\abbr{рр.}:

«Можна вбачати непослідовність у тому, що можлива будь-яка
надмірна праця в такі часи, коли торговля йде так погано,
але саме поганий стан торговлі й підштовхує безсумлінних людей
до надмірностей; вони забезпечують собі таким чином екстразиск»\dots{}
«В той самий час, — каже Леонард Горнер, — коли
122 фабрики в моїй окрузі цілком залишені, 143 припинили свою
працю, а всі інші працюють короткий час, надмірна праця поза
межі визначеного законом часу триває й далі».\footnote{
«Reports etc.», 1. c., p. 43.
} «Хоч, — каже
пан Говел, — більшість фабрик через поганий стан справ працює
лише половину часу, я дістаю таке число скарг, як і раніш, на
те, що вривають (snatched) щоденно в робітників пів або три
четвертини години, порушуючи час, призначений законом на
їжу й відпочинок».\footnote{
«Reports etc.», 1. c., p. 25.
}

Те саме явище повторюється в менших розмірах за часів страшної
бавовняної кризи від 1861 до 1865~\abbr{рр.}\footnote{
«Reports etc. for the half year ending 30 th April 1861». Див. додаток
№ 2: «Reports etc. 31 st October 1862», p. 7, 52, 53. Надмірності знову
стають численнішими за останнє півріччя 1863~\abbr{р.} Порівн. «Reports etc.
ending 31 st October 1863», p. 7.
}

Коли ми спіймаємо робітників за працею в обідній або в який
інший незаконний час, то нам іноді, виправдуючись, кажуть,
що вони ніяк не хочуть залишити фабрику, і що треба вживати
примусу, щоб заставити їх перервати їхню працю (чищення машин
і~\abbr{т. ін.}), особливо в суботу по півдні. Але коли «руки» лишаються
на фабриці після припинення машин, так це стається лише
через те, що між шостою годиною ранку й шостою годиною вечора,
\index{i}{0185}  %% посилання на сторінку оригінального видання
у визначені законом робочі години, їм не дається часу виковувати
такі роботи».\footnote{
«Reports etc. 31 st October 1860», p. 23. З яким фанатизмом, — за
свідченням фабрикантів на суді, — їхні фабричні руки опираються кожній
перерві фабричної праці, показує такий курйоз. На початку липня 1836~\abbr{р.}
суддю в Люсбері (Yorkshire) повідомили про те, що власники вісьмох великих
фабрик біля Batley порушили фабричний закон. Частину цих панів
обвинувачувано в тому, що вони примушували п’ятеро хлоп’ят 11--15 років
працювати від 6 годин ранку в п’ятницю до 4 години після полудня
в суботу, не дозволяючи їм жодного відпочинку, крім часу на їжу й однієї
години на спання опівночі. І ці діти мали виконувати без відпочинку тридцятигодинну
працю в «shoddy-hole», як воно зветься, те пекло, де розсмикується
вовняне ганчір’я й де повітря так насичене пилюгою, відпадками,
що навіть дорослі робітники примушені постійно зав’язувати собі рота
хусточками, щоб захистити свої легені! Пани обвинувачені давали запевнення
замість присяги, — як квакери, вони були занадто скрупульозно-релігійні
люди для того, щоб давати присягу, — що в своєму великому
милосерді вони були б дозволили безталанним дітям спати по 4 години,
але ці вперті діти зовсім не хотіли іти до ліжка! Панів квакерів
позасуджувано на 20\pound{ фунтів стерлінґів} грошової кари. Драйден передбачав
таких квакерів:
\settowidth{\versewidth}{That feared an oath, but like the devil would lie,} 
\begin{verse}[\versewidth]
Fox full fraught in seeming sanctity,\\
That feared an oath, but like the devil would lie,\\
That look’d like Lent, and had the holy lear,\\
And durst not sinl before he said his prayer! \\
\vinphantom{And }[Лисиця повна святощів фалшивих,\\
\vinphantom{And }Клятьби боїться, але бреше як диявол.\\
\vinphantom{And }Як піст свята з-під ока визирає\\
\vinphantom{And }І зроду не грішить, не помолившись].\\
\end{verse}
}

«Екстра-зиск, що його дає надмірна праця понад визначений
законом час, є, здається, для багатьох фабрикантів надто
велика спокуса, щоб вони могли встояти проти неї. Вони сподіваються
на те, що їх не спіймають, і що, навіть якщо це й
викриється, то незначні грошові кари й невеличкі судові витрати
їм усе ще таки забезпечують бариш».\footnote{
«Reports etc. for 31 st October 1856», p. 34.
} «Там, де додатковий
час добувається помноженням дрібних крадіжок («a multiplication
of small thefts») протягом дня, інспектори надибують на
майже непереможні труднощі, коли хочуть подати докази порушення
закону».\footnote{
Там же, стор. 35.
} Ці «дрібні крадіжки», що їх робить капітал
коштом обіднього часу й відпочинку робітників, фабричні інспектори
називають іще «petty pilferings of minutes», крадіж хвилин,\footnote{
Там же, стор. 48.
}
«snatching a few minutes», вривання хвилин\footnote{
Там же.
} або, вживаючи технічної
мови робітників, «nibbling and cribbling at meal times».\footnote{
Там же.
}\footnote*{
— обгризання і обкрадання часу, призначеного для їжі. \emph{Ред.}
}

Ми бачимо, що в цій атмосфері творення додаткової вартости
додатковою працею не є таємниця. «Коли б ви мені дозволили, —
казав мені якось один дуже поважний фабрикант, — заставляти
працювати щоденно лише на 10 хвилин більше понад визначений
\parbreak{}  %% абзац продовжується на наступній сторінці
