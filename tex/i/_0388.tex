\parcont{}  %% абзац починається на попередній сторінці
\index{i}{0388}  %% посилання на сторінку оригінального видання
у пізніші часи їхнього життя безпутніх, зацураних, розпусних
людей\dots{} Страшним джерелом деморалізації є їхнє житло. Кожен
moulder (формівник, власне навчений робітник та голова групи
робітників) дає своїй групі з семи осіб мешкання та харч у своїй
хаті або котеджі. Чи належать вони до його родини, чи ні, — все
одно, чоловіки, підлітки й дівчата сплять усі в цій хаті. Ця хата
має звичайно дві і лише винятково три кімнати, всі на рівні землі,
з недостатньою вентиляцією. Тіла настільки виснажені великим
потінням за день, що ніхто не додержує ані будь-яких правил
гігієни, ані чистоти або пристойности. Багато з цих хат — це
справжні зразки безладдя, бруду й пилюги\dots{} Найбільше лихо
системи, що вживає молодих дівчат до праці цього роду, в тому,
що вона приковує їх, звичайно починаючи від дитячого віку, на
ціле їхнє пізніше життя до найзіпсутішої наволочі. Раніш, ніж
природа навчить їх, що вони жінки, вони стають грубими, пащекуватими
хлопцюгами («rough, foul-mouthed boys»). Зодягнуті
в мізерне брудне лахміття, ноги голі далеко вище колін, волосся
та обличчя вимазане брудом, вони навчаються зневажати всяке
почуття скромности та соромливости. В обідній час лежать вони,
простягнувшися на полях, або підглядають, як хлопці купаються
в сусідньому каналі. Закінчивши, нарешті, свою тяжку денну
працю, вони одягають кращу одежу та йдуть із чоловіками до
шинку». Що серед цілої цієї кляси панує від самих дитячих літ
величезне пияцтво, це цілком природна річ. «Найгірше — це те,
що цегельники доходять одчаю щодо самих себе. Ви, мій пане, —
сказав один із кращих між ними капелянові з Southallfield’a, —
могли б з таким самим успіхом спробувати піднести та виправити
чорта, як і цегельника» («You might as well try to raise and
improve the devil as a brickie, Sir!»).\footnote{
«Children’s Employment Commission. 5 th Report 1866», P; XVІ,
n. 96, 97, p. 130, n. 39--61. Порівн. також там же, 3rd Report 1864,
p. 48, 56.
}

Про капіталістичне економізування умов праці в сучасній
мануфактурі (під якою тут треба розуміти всі великі майстерні,
окрім власне фабрик) ми находимо офіціяльний та якнайбагатший
матеріял у томах IV (1861 р.) та VI (1864 р.) «Public Health Веport».
Опис workshops (майстерень), особливо майстерень лондонських
друкарів і кравців, далеко перевищує все те, що могла вигадати
найогидливішого фантазія наших романістів. Вплив на
стан здоров’я робітників зрозумілий сам собою. Д-р Сімон, найстарший
лікарський урядовець Privy Council та офіціяльний
видавець «Public Health Reports», каже, між іншим: «У моєму
четвертому звіті (1863 р.) я показав, як на практиці робітникам
не сила обстоювати своє найперше право, право на здоров’я,
а саме добитися того, щоб, хоч до якої праці їх збирав їхній хазяїн,
ця праця, оскільки це від нього залежить, булa вільна від
усіх тих антигігієнічних умов, що їх можна оминути. Я показав,
що в той час, як на практиці самим робітникам не сила добитися
\parbreak{}  %% абзац продовжується на наступній сторінці
