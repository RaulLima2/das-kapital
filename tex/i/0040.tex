ються й тут, та тільки ж суспільно, а не індивідуально. Всі продукти Робінзона були виключно його
особистим продуктом, отже, і безпосередньо предметами споживання для нього самого. Сукупний продукт
товариства є суспільний продукт. Певна частина
цього продукту служить знову за засіб продукції. Вона лишається суспільною. Але другу частину члени
товариства споживають як засоби існування. Тим то вона мусить бути розподілена між ними. Спосіб
цього розподілу змінюватиметься разом із осібним характером самого суспільно-продукційного організму
й відповідного рівня історичного розвитку продуцентів. Лише на те, щоб провести паралелю з товаровою
продукцією, ми припускаємо, що пайку кожного продуцента в засобах існування визначає його робочий
час. Отже, робочий час відігравав би двоїсту ролю. Суспільний пляномірний розподіл робочого часу
реґулює правильне відношення різних функцій праці до різних потреб. А, з другого боку, робочий час
разом з тим служить за міру індивідуальної участи продуцента в спільній праці, а тому і в
індивідуально споживаній частині спільного продукту. Суспільні відношення людей до їхніх праць і
продуктів їхньої праці лишаються тут прозоро прості так в продукції, як і в розподілі.

[Релігійний світ є лише рефлекс реального світу].* Для суспільства товаропродуцентів, що його
загальносуспільні продукційні відносини є в тому, щоб до своїх продуктів ставитись як до товарів,
отже, як до вартостей, і в цій речовій формі відносити одну до однієї свої приватні праці як
однакову людську працю, — для такого суспільства за найвідповіднішу форму релігії є християнство з
його культом абстрактної людини, особливо християнство в його буржуазних формах розвитку,
протестантстві, деїзмі і т. ін. За староазійського, античного й таких інших способів продукції
перетворення продукту на товар, а тому й існування людей як продуцентів товару, відіграє
підпорядковану ролю, яка проте стає то значніша, що сильніша стадія занепаду громадського ладу.
Власне торговельні народи існують тільки в межисвітових просторах старого світу, як боги Епікура,
або як євреї в порах польського суспільства. Ці старі суспільно-продукційні організми куди простіші
й прозористіші, ніж буржуазні, але вони спираються або на незрілість індивідуальної людини, що не
відірвалася ще від пуповиння природного родового зв’язку з іншими людьми, або на безпосередні
відносини панування й рабства. Їхнє існування обумовлено низьким ступенем розвитку продуктивних сил
праці й відповідно обмеженими відносинами людей в процесі творення їхнього матеріяльного життя, а
тому й відповідно обмеженими відносинами людей між собою й до природи. Ця реальна обмеженість
відбивається ідеально в старовинних природних і народніх релігіях. Релігійний рефлекс дійсного світу
може взагалі зникнути лише тоді, коли відносини

* Заведене у прямі дужки ми беремо з французького видання: «Le monde religieux n’est que le reflet
du mond réel». Ред.
