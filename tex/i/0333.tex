останнє зужиткування до певної міри в зворотному відношенні
до її вжитку. 145

Але, крім матеріяльного зужиткування, машина зазнає, так би
мовити, і морального зужиткування. Вона втрачає мінову вартість
або в тій мірі, в якій машини тієї самої конструкції можуть бути
репродуковані дешевше, або в тій, в якій з нею починають конкурувати
кращі машини.146 В обох випадках, хоч і яка молода
та повна життєвої сили була б машина, її вартість визначається вже
не тим робочим часом, який фактично упредметнений у ній, а тим,
що тепер є доконечний на її репродукцію або на репродукцію
кращих машин. Тим то вона є більше або менше зневартнена.
Що коротший той період, протягом якого репродукується її цілу
вартість, то менша небезпека морального зужиткування, а що
довший робочий день, то коротший цей період. Коли машини вперше
вводяться в якусь галузь продукції, то раз-у-раз постають
нові методи дешевшої репродукції їх 147 та поліпшення, які охоплюють
не тільки поодинокі частини або апарати, але й цілу їхню
конструкцію. Тому в перший період життя машин цей осібний
мотив до подовження робочого дня діє якнайгостріше.148

За інших незмінних обставин та за даного робочого дня експлуатація
подвійного числа робітників потребує подвоєння так
тієї частини сталого капіталу, що її витрачається на машини й
будівлі, як і тієї, що її витрачається на сировинний матеріял,
допоміжні матеріяли й т. д. Із здовженням робочого дня маштаб
продукції ширшає, тимчасом як частина капіталу, витрачена
на машини та будівлі, лишається незмінна.149 Тому не тільки

145 «Учинення... шкоди делікатним рухомим частинам металевого
механізму тим, що він не працює» («Occasion... injury to the delicate
moving parts of metallic mechanism by inaction»). (Ure: «Philosophy
of Manufacture», London 1835, p. 281).

146 Згаданий уже раніш «Manchester Spinner» («Times», 26. Nov. 1862),
перелічуючи витрати на машини, каже: «це (а саме відрахування для покриття
зужиткування машини) призначено на повернення втрат, які постійно
постають через заміну машин, раніше ніж їх зужитковано, на інші,
нові й ліпшої конструкції» («It (allowance or deterioration of machinery)
is also intended to cover the loss which is constantly arising from the superseding
of machines before they are worn out by others of a new and better
«construction»).

147 «Загалом вважають, що сконструювати одним-одну машину за
новим моделем коштує уп’ятеро дорожче, ніж реконструкція тієї самої
машини за тим самим моделем». (Babbage: «On the Economy of Machinery
and Manufactures», London 1832, p. 211).

144 «Від декількох років у фабрикації тюлю пороблено такі значні та
численні поліпшення, що машину, яка добре збереглася й коштувала
первісно 1.200 фунтів стерлінґів, через декілька років продавали за
60 фунтів стерлінґів.... Поліпшення поставали одне по одному з такою
швидкістю, що машини лишалися в руках машинобудівельників незакінченими,
бо через щасливіші винаходи вони були вже застарілі». Тим то
в цей період «бурі й натиску» фабриканти тюлю незабаром збільшили первісний
восьмигодинний робочий день до 24 годин з подвійною зміною
робітників. (Там же, стор. 281).

149 «Само собою очевидно, що з припливами та відпливами на ринку
та за навперемінного поширення та скорочення попиту постійно трапля-
