її в національному маштабі зроблено вперше століттям пізніше,
за часів буржуазної французької революції.

В міру того, як обмін товарів рве свої тільки місцеві пута,
а товарова вартість через це розвивається на матеріялізацію
людської праці взагалі, грошова форма переходить на товари,
які з природи здатні виконувати суспільну функцію загального
еквіваленту, — на благородні металі.

Що «гроші з природи є золото й срібло, дарма що золото й
срібло з природи не є гроші»,42 показує нам те, що їхні природні
властивості збігаються з їхніми функціями.43 Але досі
ми знаємо лише одну функцію грошей: служити за форму виявлення
товарової вартости, або за матеріял, що в ньому суспільно
виражаються величини товарових вартостей. Адекватною формою
виявлення вартости або матеріялізацією абстрактної і тому рівної
людської праці може бути лише така матерія, що її всі екземпляри
мають однорідну якість. З другого боку, через те, що ріжниця
поміж величинами вартостей є суто кількісна, грошовий товар
мусить бути придатний, щоб визначити суто кількісні ріжниці,
отже, бути придатний до довільного ділення на частини й знову
до складання цілого з його частин. Золото й срібло з природи
мають ці властивості.

Споживна вартість грошового товару подвоюється. Поруч з
осібною його споживною вартістю як товару, — приміром, золото
придається пльомбувати зуби або як сировинний матеріял
для предметів розкошів і т. ін., — він набуває формальної споживної
вартости, яка випливає з його специфічних суспільних
функцій.

Через те, що всі інші товари — це лише осібні еквіваленти
грошей, а гроші — їхній загальний еквівалент, то вони як осібні
товари відносяться до грошей як до загального товару.44

Ми вже бачили, що грошова форма є лише зрослий з певним
товаром рефлекс відношень до нього всіх інших товарів. Отже,
той факт, що гроші є товар,45 є відкриття лише для того, хто

42 К. Marx: «Zur Kritik der Politischen Oekonomie», S. 135 (К. Маркс:
«До критики і т. д.», ДВУ, 1926 р., стор. 165.) — «Металі.... з природи
гроші» («І metalli... naturalmente moneta»). (Galiani: «Deila Moneta»
в збірнику Custodi, Parte Moderna, vol. III, p. 72).

43 Докладніше про це див. у моїй тільки-но цитованій праці, розділ:
«Благородні металі».

44 «Гроші є універсальний товар» («Il danaro è la merce universale»).
(Verri: «Meditazioni sulla Economie Politica», p. 16).

45 «Сами срібло й золото, яким ми можемо дати загальну назву грошового
металю, є... товари..., вартість яких то підноситься, то падає...
Вартість грошового металю можна вважати за велику тоді, коли за
незначну вагу його можна купити значну кількість продуктів або фабрикатів
 країни, і т. ін.» («Silver and gold themselves, which we may call
by the general name of Bullion, are... commodities... raising and falling
in... value... Bullion then may be reckoned to be of higher value, where
the smaller weight will purchase the greater quantity of the product or
manufacture of the country etc.»). («А Discourse of the General Notions
of Money, Trade, and Exchange, as they stand in relations to each other.
By a Merchant», London 1695, p. 7). «Срібло й золото, карбовані й не-
