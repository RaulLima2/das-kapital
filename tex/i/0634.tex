як усяке інше рухоме майно або худобу. Коли раби задумають
що проти панства, то їх так само слід покарати на смерть. Мирові
судді повинні на заяву панів розшукувати рабів-утікачів.
Коли виявиться, що волоцюга три дні тинявся без праці, то його
слід відіслати до місця його народження, випекти на його грудях
розпеченим залізом тавро «V» і, закувавши в кайдани, вживати
його там на дорожні та всякі інші подібні роботи. Коли волоцюга
неправильно показує своє місце народження, то він на кару
за це мусить стати довічним рабом цього місця, його мешканців
або корпорації, і на нього слід накласти тавро «S». Кожний
має право відібрати у волоцюг їхніх дітей і тримати їх при собі
як учнів, хлопців до 24 років, дівчат до 20 років. Коли вони втечуть,
то до вищезазначеного віку вони мусять бути рабами їхніх
хазяїнів, які мають право на своє бажання заковувати їх у
кайдани, бити батогами й т. ін. Кожний хазяїн може накидати
залізний ланцюжок на шию, ноги або руки свого раба, щоб краще
його пізнавати й бути певнішим, що він не втече.\footnote{
Автор «Essay on Trade etc.», 1770, зауважує: «За королювання
Едварда VI англійці, здається, цілком серйозно почали підохочувати
мануфактури й давати бідним заняття. Це видно з одного вартого уваги
статуту, де сказано, що на всіх волоцюг треба накладати тавра, і т. ін.
(Там же, стор. 8).
} Остання
частина цього статуту передбачає, що деякі бідні повинні працювати
на ту округу або тих осіб, що дають їм їсти й пити та знаходять
для них працю. Такий рід парафіяльних рабів зберігся
в Англії аж до XIX віку під назвою roundsmen (Umgeher).

Єлисавета, 1572: жебраків понад 14 років, що не мають дозволу
жебракувати, слід люто бити батогами та випікати їм тавра
на лівому вусі, коли ніхто не згоджується взяти їх на службу
на два роки; коли це повториться, то жебраків понад 18 років
слід покарати на смерть, якщо ніхто не згоджується взяти їх
на службу на два роки; коли їх спіймають на цьому втретє, то
їх слід нещадно покарати на смерть як державних зрадників.
Аналогічні статути: 18 Єлисавети с. 13 і 1597.\footnoteA{
Томас Мор каже у своїй «Утопії»: «Так то й трапляється, що
жаденний і ненаситний обжера, оця справжня чума своєї батьківщини,
захоплює тисячі акрів землі, обгороджує їх тином або живоплотом, або
силою і кривдами може так зацькувати їхніх власників, що вони примушені
продати все своє майно. Тим або іншим способом, не києм, то палицею,
їх примушують виселятися — цих бідних, простих, бідолашних людей!
Мужчини, женщини, чоловіки, жінки, сироти, вдовиці, охоплені
горем матері з немовлятками, всі члени родини, бідні на засоби існування,
але багаті числом, бо рільництво потребувало багато робочих рук.
Вони бредуть геть, кажу я, з своїх рідних місць, до яких вони звикли,
і ніде не находять собі місця відпочинку; продаж усього їхнього домашнього
скарбу, хоч і невеликої вартости, міг би за інших обставин принести
деяку виручку, але, раптом опинившися на вулиці, вони мусять продати
його за безцінь. І коли вони проблукають таким чином аж поки проїдять
останню шажину, то що ж їм лишається, як не красти? Але тоді
їх вішають, додержуючи всіх форм закону. Або жебракувати? Але тоді
їх кидають у в’язницю, як волоцюг, за те, що вони тиняються й не працюють,
хоч їм ніхто не хоче дати роботу, як би жагуче вони її добивалися».
З-поміж цих бідних утікачів, що їх, як каже Томас Мор, просто
}