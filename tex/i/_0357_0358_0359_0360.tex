\parcont{}  %% абзац починається на попередній сторінці
\index{i}{0357}  %% посилання на сторінку оригінального видання
праці — вівці, коні і т. д., — безпосередні акти насильства становлять
тут першу передумову промислової революції. Спочатку
проганяють робітників із землі, а потім з’являються вівці. І тільки
розкрадання землі у великому маштабі, як ось в Англії, створює
для великого рільництва поле його діяльности.\footnoteA{
[До четвертого видання. — Це стосується й до Німеччини. Там,
де в нас існує велике рільництво, отже, саме на Сході, воно стало можливим
лише через застосування системи «Bauernlegen»,\footnote*{
Так називався в Німеччині процес експропріяції земель у селян;
в Англії цей процес звався «Clearing of Estates» («очищення маєтків» —
у дійсності очищення їх від людей). Див. про це далі розділ 24, §2. \emph{Ред.}
}
яке почалося в XVI віці, особливо від 1648 р. — \emph{Ф. Е.}].
} Тому на своїх початках цей переворот у рільництві на позір має скорше вигляд
політичної революції.

Засіб праці у формі машини відразу стає конкурентом самого
робітника.\footnote{
«Машини й праця перебувають у постійній конкуренції» («Machinery
and labour are in constant competition»). (\emph{Ricardo}: «Principles of
Political Economy». 3 rd ed., London 1821, p. 479).
} Самозростання вартости капіталу за допомогою
машини стоїть у прямому відношенні до числа робітників, умови
існування яких вона нищить. Ціла система капіталістичної продукції
ґрунтується на тому, що робітник продає свою робочу силу
як товар. Поділ праці уоднобічнює робочу силу на цілком частинну
вмілість — керувати частинним знаряддям. Скоро тільки
керування знаряддям переходить до машини, то разом із споживною
вартістю зникає й мінова вартість робочої сили. Робітник не
находить собі покупців, як паперові гроші, що виключені з обігу.
Та частина робітничої кляси, що її машини таким способом перетворюють
у надмірну людність, тобто в таку, яка безпосередньо
вже не потрібна для самозростання капіталу, з одного боку, гине
в нерівній боротьбі старого ремісничого й мануфактурного виробництва
з машиновим виробництвом, з другого боку, переповнює
всі приступніші галузі промисловости, переповнює ринок праці,
а тому знижує ціну робочої сили нижче за її вартість. Великою
втіхою для павперизованих робітників має бути те, що їхні страждання,
мовляв, почасти лише «тимчасові» («а temporary inconvenience»),
а почасти те, що машини, мовляв, лише поступінно
опановують ціле поле продукції, а через те зменшується розмір
та інтенсивність їхнього руйнаційного діяння. Одна втіха побиває
другу. Там, де машина захоплює якесь поле продукції поступінно,
вона породжує хронічні злидні серед робітничих верств, які з
нею конкурують. Там, де перехід відбувається швидко, там її
 вплив є масовий і гострий. Немає в світовій історії жахливішого
видовища, як поступінне вимирання англійських ручних бавовняних
ткачів, що тривало цілі десятиліття і, нарешті, завершилося
1838 р. Багато з них померло з голоду, багато животіло довгий
час із своїм родинами, мавши 2\sfrac{1}{2} пенси на день.\footnote{
Конкуренція між ручним та машиновим тканням перед заведенням
закону з 1833 р. про бідних затягувалася в Англії тим, що заробітну
плату, яка зменшилась далеко нижче за мінімум, поповнювано
допомогами парафій. «1827 р. високопреподобний Тернер був парохом у
Wilmslow’i, у Чешірі, в мануфактурній окрузі. Питання комітету в справі
еміграції й відповіді пана Тернера показують, як підтримували конкуренцію
людської праці проти машин. Питання: «Чи не усунено вживанням
механічних варстатів вживання ручних варстатів?» Відповідь: «Безперечно,
воно усунуло б його ще в більшій мірі, ніж це є в дійсності, коли б
ручні ткачі не мали змоги згоджуватися на зниження заробітної плати».
Питання: «Але, згоджуючися на це, чи не наймаються вони за плату,
якої не вистачає їм для їхнього існування, та чи не сподіваються вони
допомоги з парафії, щоб покрити недостачу?» Відповідь: «Так, і конкуренцію
між ручним варстатом і механічним варстатом фактично підтримує
лише податок для бідних». Отже, ганебний павперизм або еміграція — ось
вигоди, що їх мають робітники у наслідок заведення машин. Із почесних
та до певної міри незалежних ремісників їх зводять на становище плазівної
голоти, що живе з принизливого хліба добродійности. Ось що
вони називають тимчасовими труднощами». («The Rev. Mr. Turner was
in 1827 rector of Wilmslow, in Cheshire, a manufacturing district. The questions
of the Committee on Emigration, and Mr. Turner’s answers show
how the competition of human labour is maintained against machinery.
Question: «Has not the use of the power-loom superseded the use of the
hand-loom?» Answer: «Undoubtedly; it would have superseded them much
more than it has done, if the handloom weavers were not enabled to submit
to ä reduction of wages». Question: «But in submitting he has accepted wages
which are insufficient to support him, and looks to parochial contribution
as the remainder of his support?» Answer: «yes, and in fact the competition
between the hand-loom arid the power-loom is maintained out the poorrates».
Thus degrading pauperism or expatriation, is the benefit which the
industrious receive from the introduction of machinery, to be reduced from
the respectable and in some degree independent mechanic, to the cringing
wretch who lives on the debasing bread of charity. This they call a temporary
inconvenience»). («A Prize Essay on the comparative merits of Competition
and Cooperation», London 1834, p. 29).
} Навпаки, гостро подіяло заведення англійських бавовняних машин у
\index{i}{0358}  %% посилання на сторінку оригінального видання
Східній Індії, генерал-губернатор якої 1834--35 рр. констатував:
«Ледве чи знайдеться аналогія до цих злиднів в історії
торговлі. Рівнини Індії біліють від кісток бавовняних ткачів».
Щоправда, оскільки ці ткачі переставились, покинули це тимчасове
шиття, остільки і машини заподіяли їм лише «тимчасових
страждань». Зрештою, це «тимчасове» діяння машин є перманентне,
бо вони постійно захоплюють нові сфери продукції. Отже,
характер усамостійнення та відчужености, що його капіталістичний
спосіб продукції надає взагалі умовам праці та продуктові
праці супроти робітника, розвивається з виникненням
машин до повного антагонізму.\footnote{
«Та сама причина, яка може збільшити дохід країни (тобто, як
тут же пояснює Рікардо, доходи лендлордів та капіталістів, багатство
(wealth) яких з економічного погляду взагалі дорівнює багатству нації
(wealth of the nation), може одночасно утворити надмір людности та
погіршити становище робітника» («The same cause which may increase
the revenue of the country may at the same time render the population
redundant and deteriorate the condition of the labourer»). (\emph{Ricardo}:
«Principles of Political Economy», 3 rd ed. London 1821, p. 469). «Постійна
мета й тенденція кожного вдосконалення механізму фактично є в тому,
щоб цілком збутися праці людини або зменшити її ціну, замінюючи працю
дорослих робітників-чоловіків працею жінок та дітей або працю навчених
робітників працею чорноробів». (\emph{Ure}: «Philosophy of Manufacture», p.23).
} Саме через це з виникненням машин уперше вибухають жорстокі повстання
робітників проти засобу праці.

\index{i}{0359}  %% посилання на сторінку оригінального видання
Засіб праці вбиває робітника. Певна річ, ця безпосередня
протилежність найнаочніше виявляється тоді, коли новозаведена
машина конкурує з традиційним ремісничим або мануфактурним
виробництвом. Але й у межах самої великої промисловости постійне
поліпшування машин і розвиток автоматичної системи діють аналогічно.
«Постійна мета поліпшення машин є в тому, щоб зменшити
ручну працю або вдосконалити ланку в продукційному
ланцюзі фабрики, замінивши люський апарат залізним».\footnote{
«Reports of Insp. of Fact, for 31 st October 1858», p. 43.
}
«Застосування сили пари й води до машин, що їх досі рухалось
рукою, трапляється щодня\dots{} Незначні поліпшення в машинах,
що мають на меті заощадити на рушійній сипі, поліпшити продукт,
збільшити продукцію протягом того самого часу, витиснути дитину,
жінку або чоловіка, — такі поліпшення робиться постійно
і, хоч на око вага цих поліпшень невелика, все ж вони дають
важливі результати».\footnote{
«Reports |of Insp. of Fact, for 31 st October 1856», p. 15.
} «Повсюди, де якась операція потребує
чималої вправности та певної руки, її якомога швидше забирають
із рук надто навченого робітника, що має часто нахил до нереґулярности
всякого роду, щоб доручити її осібному механізмові,
який так добре вреґульований, що за ним може наглядати й
мала дитина».\footnote{
\emph{Ure}: «Philosophy of Manufacture», p. 19. «Велика перевага
машин, що їх уживають на цегельнях, є в тому, що вони роблять хазяїна
незалежним від навчених робітників». («Children’s Employment Commission.
5 th Report», London 1866, p. 180, n. 46).

Додаток до другого видання. Пан А. Стеррок, головний управитель
машинового відділу «Great Northern Railway», висловлюється так про
будування машин (льокомотивів і т. д.): «Дорогих (expensive) англійських
робітників із дня на день потребують щораз менше. Продукція збільшується
через уживання поліпшених інструментів, а ці інструменти із свого
боку обслуговує нижчий рід праці (a low class of labour)\dots{} Раніш усі частини
парової машини продукувала неодмінно кваліфікована праця.
Ті самі частини тепер продукує менш кваліфікована праця, але з добрими
інструментами\dots{} Під інструментами я розумію машини, що їх уживають
в машинобудуванні». («Royal Commission on Railways. Minutes of Evidence»,
n. 17 862 and 17 863. London 1867).
} «За автоматичної системи талант робітника проґресивно
витискується».\footnote{
\emph{Ure}: «Philosophy of Manufacture», p. 20.
} «Поліпшення машин не тільки вимагає
зменшити число дорослих робітників, уживаних, щоб досягти
певного результату, але воно ще й заміняє одну клясу індивідів
на другу клясу, більш навчених на менш навчених, дорослих
на дітей, чоловіків на жінок. Всі ці переміни призводять до постійних
коливань у нормі заробітної плати».\footnote{
Там же, стор. 321.
} «Машини безупинно
викидають дорослих із фабрики».\footnote{
Там же, стор. 23.
} Надзвичайну еластичність машинової системи як наслідок нагромадженого
практичного досвіду, як наслідок наявного вже розміру механічних
засобів та постійного проґресу техніки, виявив нам бурхливий
\index{i}{0360}  %% посилання на сторінку оригінального видання
розвиток цієї системи, що відбувався під тиском скорочення
робочого дня. Але хто міг би 1860 р., року зенітного розвитку
англійської бавовняної промисловости, передбачати ті чимраз
швидші поліпшення машин і відповідне витискування ручної
праці, що їх викликали три наступні роки під тиском американської
громадянської війни? Щодо цього пункту тут досить кількох
прикладів з офіціяльних даних англійських фабричних інспекторів.
Один менчестерський фабрикант заявляє: «Замість 75 чухральних
машин ми потребуємо тепер лише 12, і вони дають нам
таку саму кількість продуктів такої самої, якщо не ліпшої,
якости\dots{} Заощадження на заробітній платі становить 10 фунтів
стерлінґів на тиждень, заощадження на відпадках бавовни — 10\%».
В одній менчестерській тонкопрядільні «через прискорення руху
й заведення різних автоматичних (self-acting) процесів усунено
в одному відділі \sfrac{1}{4}, у другому більш ніж \sfrac{1}{2} робітничого
персоналу, тимчасом як чесальна машина, що заступила другу чухральну
машину, дуже зменшила число робітників, занятих раніш у чухральному
відділі». Інша прядільна фабрика оцінює свої загальні заощадження
на «руках» у 10\%. Панове Джілмер, фабриканти-прядільники
в Менчестері, заявляють: «Заощадження в нашому відділі
blowing (чищення бавовни) на руках та заробітній платі, досягнуті
в наслідок заведення нових машин, ми оцінюємо у цілу третину\dots{}
у відділах jack frame і drawing frame room витрати на руки та інші
видатки зменшились приблизно на \sfrac{1}{3}, у прядільному відділі видатки
зменшились приблизно на \sfrac{1}{3}. Але це не все: якщо наша пряжа
йде тепер до ткача, то в наслідок застосування нових машин її
так дуже поліпшено, що ткачі продукують більше та ліпші тканини,
аніж із колишньої машинової пряжі».\footnote{
«Reports of Insp. of Fact, for 31 st October 1863», p. 108 і далі.
} Фабричний інспектор
А. Редґрев додає до цього: «Зменшення числа робітників
при збільшенні продукції швидко проґресує; по вовняних фабриках
недавно знову почалося зменшення рук, і це зменшення триває
далі; перед кількома днями один учитель, що мешкає коло Рочделя,
сказав мені, що величезне зменшення школярок по дівочих школах
зумовлене не тільки натиском кризи, а ще й тими змінами
в машинах вовняної фабрики, наслідком яких там сталося зменшення
рук пересічно на 70 робітників половинного часу».\footnote{
Там же, стор.109. Швидке поліпшення машин підчас бавовняної
кризи дозволило англійським фабрикантам зараз же по скінченні американської
громадянської війни знову миттю переповнити світовий ринок.
Уже в останні шість місяців 1866 р. тканин майже не можна було продати.
Тоді почався вивіз товарів у Китай та Індію на комісію, що, природно,
зробило «glut»\footnote*{
— пересичення ринку. \emph{Ред.}
} ще інтенсивнішим. На початку 1867 р. фабриканти вдалися
до свого звичайного зарадчого способу, до зниження заробітної
плати на 5\%. Робітники опирались та заявили, теоретично цілком правильно,
що єдине, чим тут можна зарадити, — це працювати скорочений
час, чотири дні на тиждень. Після довгих вагань капітани промисловости,
як вони сами називали себе, — змушені були згодитися на це, подекуди
із зниженням заробітної плати на 5\%, подекуди без зниження її.
}
