\parcont{}  %% абзац починається на попередній сторінці
\index{i}{0327}  %% посилання на сторінку оригінального видання
В Англії є 16 реєстраційних округ, де на \num{100.000} дітей, молодших
за один рік, припадає пересічно лише \num{9.000} смертних випадків
на рік (в одній окрузі тільки \num{7.047}), у 24 округах — понад \num{10.000},
але менш як \num{11.000}, у 39 округах понад \num{11.000}, але менш як \num{12.000},
у 48 округах понад \num{12.000}, але менш як \num{13.000}, у 22 округах
понад \num{20.000}, у 25 округах понад \num{21.000}, у 17 — понад \num{22.000},
в 11 — понад \num{23.000}, в Ноо, Wolverhampton, Asthon-under-Lyne
і Preston — понад \num{24.000}, в Nottingham, Stockport і Bradford —
понад \num{25.000}, y Wisbeach — \num{26.000} та в Менчестері — \num{26.125}\footnote{
«Sixth Report on Public Health», London 1864, p. 34.
}.
Як показав один офіціяльний лікарський дослід 1861~\abbr{р.}, причина
таких високих норм смертности лежить, залишаючи осторонь
місцеві умови, головне у праці матерів поза власною хатою і в
тих умовах, що випливають із цього, а саме в занедбуванні та
мордуванні дітей, між іншим, у невідповідному харчуванні, в
недостачі харчу, в годуванні дітей препаратами опію і~\abbr{т. д.};
до цього долучається протиприродне відчуження матерів від
своїх дітей і, як наслідок цього, навмисне виголодовування й
отруювання\footnote{
«Він (дослід 1861 p.)\dots{} показав, крім того, що серед зазначених
умов, з одного боку, діти вмирають через те, що матері, працюючи на фабриках,
занедбують своїх дітей та зле поводяться з ними, і що, з другого
боку, матері до такої міри втрачають природні почуття до власних
дітей, що смерть їхня не завдає їм жалю, а іноді навіть\dots{} вони просто вживають
усяких заходів, щоб заподіяти їм смерть» (Там же).
}. Навпаки, в таких рільничих округах, «де жіночої
праці знаходимо найменше, там і норма смертности найнижча»\footnote{
Там же, стор. 454.
}.
Слідча комісія з 1861~\abbr{р.} дала, однак, несподіваний результат,
а саме, що в деяких суто рільничих округах уздовж Німецького
моря норма смертности дітей, молодших за один рік, майже
досягає норми смертности найбільше вславлених із цього погляду
фабричних округ. Тим-то докторові Джульєну Гентерові доручено
було розслідити це явище на місці. Його звіта долучено до
«VI Report on Public Health»\footnote{
Там же, стор. 454--463. «Reports by' Dr.~Henry Julian Hunter
on the excessive mortality of infants in some rural districts of England».
}. Досі гадали, що дітей гублять малярія
та інші недуги, властиві низьким та болотяним місцевостям.
Розслід дав якраз протилежний висновок, а саме, «що та сама
причина, яка знищила малярію, тобто перетворення на родючу
землю того ґрунту, який зимою був болотом, а влітку — злиденним
пасовиськом, — що ця сама причина зумовила надзвичайно
високу норму смертности немовлят»\footnote{
Там же, стор. 35, 455, 456.
}. Сімдесят лікарів, які
в цих округах практикували і яких прослухав д-р Гентер,
були «на диво однієї думки» щодо цього пункту. А саме, одночасно
з революцією в рільничій культурі заведено тут і промислову
систему. «Заміжніх жінок, що працюють разом із дівчатами й
хлопцями, чоловік віддає за певну суму в розпорядження орендареві,
який називається «Gangmeister» та наймає ввесь гурт.
\parbreak{}  %% абзац продовжується на наступній сторінці
