минущий момент. Даний товар негайно заміщується іншим товаром.
Тим то досить лише символічного існування грошей у процесі,
який постійно віддалює їх з одних рук до інших. Їхнє функціональне
буття поглинає, так би мовити, їхнє матеріяльне буття.
Як минуще-зоб’єктивізований рефлекс товарових цін, гроші
функціонують лише як знак себе самих і через те вони також
можуть бути заміщені знаками.85 Потрібно лише, щоб грошовий
знак набув об’єктивного суспільного визнання, а такого визнання
паперовий символ набуває шляхом примусового курсу. Цей
державний примус має силу лише в межах даного суспільства,
або у сфері внутрішньої циркуляції, але також тільки в цій
сфері гроші цілком розгортаються у своїй функції засобу циркуляції,
або монети, і тому можуть вони в паперових грошах набрати
зовнішньо відокремленого від своєї металевої субстанції й суто
функціонального способу існування.

3. Гроші

Товар, що функціонує як міра вартости, а через це безпосередньо
своїм тілом або через заступника і як засіб циркуляції,
є гроші. Отже, золото (або срібло) є гроші. Як гроші воно функціонує,
з одного боку, там, де воно мусить з’являтись у своїй
золотій (або срібній) тілесності, отже, як грошовий товар, тобто
з’являтись не лише ідеально, як у мірі вартости, і не лише як
щось, що може бути репрезентоване, як у засобі циркуляції;
з другого боку, воно функціонує як гроші там, де його функція —
все одно, чи виконує воно цю функцію власною особою, чи через
заступників, — фіксує його як однісіньку форму вартости або як
однісіньке адекватне буття мінової вартости проти всіх інших
товарів як лише споживних вартостей.*

а) Скарботворення

Безперестанний кругобіг двох протилежних товарових метаморфоз,
або невпинне чергування продажу й купівлі, виявляється
в безупинному обігу грошей або, в функції їх як perpetuum mo-

85  3 того, що золото й срібло як монета, або у виключній функції
засобу циркуляції стають знаками самих себе, Нікола Барбон виводить
право урядів «to raise money», тобто надавати, приміром, кількості срібла,
що звалася шелягом назву більшої кількости срібла, як, приміром, таляр,
і таким чином виплачувати кредиторам шеляг замість таляра. «Гроші
стираються й стають легшими, часто переходячи з рук до рук... При торговельних
операціях вважають на назву грошей і курс їхній, а не на кількість
срібла... Авторитет суспільної влади робить із кусника металю
гроші». («Money does wear and grow lighter by often telling over... It is
the denomination and currency of the money that men regard in bargaining,
and not the quantity of silver... Tis the publick authority upon the metal
that makes it money»). (N. Barbon: «A Discourse on coining the new
money lighter, in answer to Mr. Lockes Considerations etc.», London 1696,
p. 29, 30, 45).

*    У французькому виданні цей абзац зредаґовано так: «Досі ми розглядали
благородний металь з подвійного аспекту: як міру вартостей
