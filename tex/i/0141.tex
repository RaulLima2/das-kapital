не створила додаткової вартости, отже, гроші не перетворилися
на капітал. Ціна 10 фунтів пряжі дорівнює 15 шилінґам, і 15 шилінґів
витрачено на товаровому ринку на елементи утворення
продукту або, що те саме, на фактори процесу праці: 10 шилінґів
на бавовну, 2 шилінґи на зужитковану кількість веретен і
З шилінґи на робочу силу. Набубнявіння вартости пряжі нічого
не помагає, бо вартість її — то лише сума вартостей, що раніш
розподілялись між бавовною, веретенами й робочою силою, а з
такого простого додавання наявних вартостей ніколи не може
виникнути додаткова вартість.13 Всі ці вартості сконцентровано
тепер в одній речі, але вони так само були сконцентровані в грошовій
сумі в 15 шилінґів раніш, ніж подробилась вона через
купівлю трьох товарів.

Сам по собі цей результат не є щось дивне. Вартість 1 фунта
пряжі є 1 шилінґ 6 пенсів, і тим то за 10 фунтів пряжі наш капіталіст
мусив би заплатити на товаровому ринку 15 шилінґів.
Чи він купить для себе готовий будинок на ринку, чи ставитиме
його сам, — жодна з цих операцій не збільшить грошей витрачених
на придбання будинку.

Капіталіст, що розуміється на вульґарній економії, скаже,
може, що він авансував свої гроші з наміром зробити з них більше
грошей. Але ж і шлях до пекла вимощено добрими намірами, і в
капіталіста так само міг бути намір добувати гроші, не продукуючи.14 Він загрожує. Удруге його вже
не зловлять. На майбутнє
він купуватиме на ринку готові товари замість їх самому
фабрикувати. Коли ж усі його брати-капіталісти зроблять те
саме, то де ж він тоді знайде товари на ринку? А грошей їсти
він не може. Він починає повчати нас катехизису. Слід би мати
на увазі його поздержливість. Адже він міг протринькати своїх
15 шилінґів. Замість того він продуктивно спожив їх і зробив
з них пряжу. Але зате ж бо тепер у нього є пряжа замість доко-

13    Це є основна теза, на якій ґрунтується теорія фізіократів про
непродуктивність усякої нерільничої праці, і вона незаперечна для економіста
— з фаху. «Цей спосіб прираховувати до одної речі вартість багатьох
інших (наприклад, до полотна — витрати на споживання ткача), нашаровувати,
так би мовити, на одну вартість декілька інших вартостей, призводить
до того, що вона відповідно до цього зростає. Термін «додавання» дуже
добре змальовує той спосіб, яким утворюється ціна продуктів праці; ця
ціна є лише сума багатьох вартостей, спожитих і доданих одна до однієї
але додавати — це не множити» («Cette façon d’imputer à une seule chose
la valeur de plusieurs autres (par exemple au lin la consommation du tisserand),
d’appliquer, pour ainsi dire, couche sur couche, plusieurs valeurs
sur une seule, fait que celle-ci grossit d’autant... Le terme d’addition peint trèsbien
la manière dont se forme le prix des ouvrages de main d’oeuvre; ce prix
n’est qu’un total de plusieurs valeurs consommées et additionnées ensemble;
or, additionner n’est pas multiplier»), (Mercier de la Rivière: «L’Ordre naturel
et essentiel etc.», Physiocrates, éd. Daire, II Partie, p. 599).

14    Так, приміром, за 1844—47 pp. він вилучив із продуктивних підприємств
частину свого капіталу, щоб проспекулювати його на залізничних
акціях. Так, за часів американської громадянської війни він позамикав
фабрики й викинув на брук фабричних робітників, щоб грати па ліверпулській
бавовняній біржі.
