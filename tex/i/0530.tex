лому в таких конфліктах хазяїн завжди лишається хазяїном, —
підвищення ціни праці, що випливає з акумуляції капіталу,
припускає таку альтернативу:

Або ціна праці й далі зростає, бо зріст її не заважає проґресові
акумуляції; в цьому немає нічого дивного, бо каже А. Сміс,
«навіть за зниженого зиску капітали все ж зростають; вони навіть
зростають швидше, ніж раніш... Великий капітал навіть
за меншого зиску взагалі зростає швидше, ніж малий капітал за
великого зиску» («Wealth of Nation», р. 189). У цьому випадку
очевидно, що зменшення неоплаченої праці аж ніяк не заважає
капіталові поширювати своє панування. — Або, — і це є другий
бік альтернативи, — акумуляція в наслідок підвищення ціни
праці слабшає, бо притупляється спонукливий стимул баришу.
Акумуляція меншає. Але з її зменшенням зникає причина її
зменшення, а саме зникає диспропорція між капіталом і робочою
силою, приступною для експлуатації. Отже, механізм капіталістичного
процесу продукції сам усуває ті тимчасові перешкоди,
які він утворює. Ціна праці знову спадає до рівня, що
відповідає потребам зростання капіталу, все одно, чи цей рівень
нижчий, вищий або рівний тому, що його вважалося за нормальний
перед початком зростання заробітної плати. Ми бачимо:
в першому випадку не зменшення абсолютного або відносного
зростання робочої сили або робітничої людности робить капітал
надмірним, а, навпаки, збільшення капіталу робить недостатньою
приступну для експлуатації робочу силу. У другому випадку
не збільшення абсолютного або відносного зростання робочої
сили або робітничої людности робить капітал недостатнім, а,
навпаки, зменшення капіталу робить надмірною приступну для
експлуатації робочу силу, або, точніше, її ціну. Оці абсолютні
рухи акумуляції капіталу відбиваються як відносні рухи в масі
приступної для експлуатації робочої сили, і тому здається, нібито
їх спричиняє власний рух останньої. Вживаючи математичного
вислову: величина акумуляції є незалежна змінна, величина
заробітної плати — залежна, а не навпаки. Так, у промисловому
циклі підчас фази кризи загальний спад товарових цін виражається
як підвищення відносної вартости грошей, а підчас фази
розцвіту — загальне підвищення товарових цін виражається як
спад відносної вартости грошей. Так звана Currency-школа
робить із цього висновок, що за високих цін циркулює замало
грошей, а за низьких — забагато. Її неуцтво й повне нерозуміння
фактів\footnote{
Порівн. К. Marx: «Zur Kritik der Politischen Oekonomie»,
S. 166 і далі. (К. Маркс: «До критики політичної економії», ДВУ 1926,
стор. 171 і далі).
} находять собі гідну паралелю в тих економістів,
які ті явища акумуляції пояснюють тим, що в одному випадку
існує замало, а в другому забагато найманих робітників.

Закон капіталістичної продукції, що лежить в основі нібито
«природного закону залюднення», сходить просто ось на що:
відношення між капіталом, акумуляцією й нормою заробітної