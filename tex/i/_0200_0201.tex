\parcont{}  %% абзац починається на попередній сторінці
\index{i}{0200}  %% посилання на сторінку оригінального видання
не вагаючись, визнав цей факт».\footnote{
Там же.
} «На одній вальцювальні, де
номінальний робочий день тривав од 6 години ранку до 5\sfrac{1}{2}  годин
вечора, один хлопчик працював щотижня чотири ночі щонайменше
до 8\sfrac{1}{2} годин вечора другого дня\dots{} і так протягом
6 місяців». «Другий, дев’яти років, працював іноді три дванадцятигодинні
зміни одну по одній, а мавши десять років, — два дні й
дві ночі вряд». «Третій хлопчик, тепер йому 10 років, працював
від 6 години ранку до 12 години вночі три ночі одну по одній і
до 9 години вечора протягом інших ночей». «Четвертий, тепер
йому 13 років, працював цілий тиждень від 6 години вечора до
12 години другого дня, а іноді три зміни одну по одній, приміром,
від понеділка зранку до вівтірка вночі». «П’ятий, що має тепер
12 років, працював в одній залізоливарні в Stavely від 6 години
ранку до 12 години вночі протягом двох тижнів і вже нездатний,
так працювати далі». Джордж Еллінзворт, дев’яти років: «Я прийшов
сюди минулої п’ятниці. Ми мали почати працю на другий
день о 3 годині вранці. Тому я лишився тут на цілу ніч. Живу я
5 миль відси. Спав у сінях, підстеливши собі шкіряний хвартух
і вкрившись маленькою куциною. Другі два дні я приходив сюди
о 6 годині вранці. Еге, це таки пекуче місце! Поки я сюди прийшов,
я працював так само цілий рік коло домни. Це був величезний
завод на селі. Я починав працю теж у суботу ранком о 3 годині, але я міг
принаймні ходити додому спати, бо це було
недалечко. Іншими днями я починав працю ранком о 6 годині, а
кінчав увечері о 6 або о 7 годині» і т. ін.\footnote{
Там же, стор. XIII. Рівень освіти цих «робочих сил» мусить, природно,
бути такий, яким він виявляється в дальших діялогах з одним із
членів слідчої комісії! Джірімія Гейнс, 12 років: «\dots{} Чотири рази чотири
є вісім, але чотири четвірки (4 fours) є шістнадцять»\dots{} Король для
нього є той, хто має всі гроші й усе золото. «Ми маємо короля; кажуть,
що він є королева, її називають принцеса Олександра. Кажуть, що вона
одружилася з сином королеви. Принцеса — це чоловік». В. Тернер,
дванадцяти років: «Я живу не в Англії. Я гадаю, що є така країна, але
раніш я нічого не знав про це». Джон Морріс, чотирнадцяти років: «Я чув,
що бог створив світ, і що ввесь нарід потопився, опріч однієї людини; я
чув, що то був маленький пташок». Вільям Сміс, п’ятнадцяти років:
«Бог створив чоловіка, чоловік створив жінку». Едвард Тейлор, п’ятнадцяти
років: «Нічого не знаю про Лондон». Генрі Матьюмен, сімнадцяти
років: «Я ходжу іноді до церкви\dots{} Одне ім’я, що вони про нього проповідують,
то був якийсь Ісус Христос, але іншого імени я назвати не можу
та й про нього нічого не можу сказати. Його не забили, а він помер, як
і інші люди. Він де в чому не був такий, як інші люди, бо був, сказати б,
релігійний, а інші ні». («Не was not the same as other people in some ways,
because he was religious in some ways, and others is n’t»). («Там же, 74,
p. XV). «Чорт — добра особа, я не знаю, де він живе». «Христос був погана
людина». («The devil is a good person. I don’t know where he lives». «Christ
was a wicked man»), «Ця дівчинка (10 років), замість God (бог) складає
Dog (собака) і не знає імени королеви». («Children’s Employment Commission.
5 th Report, 1866», p. 55, n. 278). Ta сама система, що й на згаданих
металюрґійних мануфактурах, панує на гутах і на фабриках паперу.
На фабриках паперу, де папір виробляють машинами, нічна праця
є загальне правило для всіх процесів, крім сортування лахміття. В деяких
випадках нічна праця за допомогою змін триває безперестанку цілий
тиждень, звичайно від неділі вночі до 12 години ночі наступної суботи.
Робочий персонал, що працює в денній зміні, робить п’ять днів на тиждень
по 12 годин і один день 18 годин, а ті, що працюють у нічній зміні, — п’ять
ночей по 12 годин і одну ніч 6 годин. В інших випадках кожна зміна працює
по 24 години одна за однією навпереміну. Одна зміна працює 6 годин у
понеділок і 18 годин у суботу, щоб було повних 24 години. В інших випадках
заведено проміжну систему, за якої всі робітники, що працюють
коло машин для вироблення паперу, працюють протягом цілого тижня
щодня 15--16 годин. Ця система, — каже член слідчої комісії Лорд, —
сполучає, здається, в собі всі злигодні 12-годинної і 24-годинної системи
змін. Діти менші за 13 років, підлітки молодші за 18 і жінки працюють
за такої нічної системи. Інколи за дванадцятигодинної системи вони
мусили працювати подвійну зміну — 24 години, — щоб заступити відсутніх.
Свідчення свідків доводять, що хлопчаки і дівчата дуже часто працюють
наднормовий час, що частенько триває без перестанку 24 і навіть 36 годин.
В «безперервному й незмінному процесі» ґлянсування можна побачити
дванадцятирічних дівчаток, які працюють цілий місяць по 14 годин на
день «без жодного регулярного відпочинку або перерви, крім двох або
щонайбільше трьох півгодинних павз для їжі». По деяких фабриках, де
зовсім немає регулярної нічної праці, працюють страшенно багато в наднормовий
час, і «це часто у найбрудніших, наймонотонніших процесах
серед неймовірної спеки». («Children’s Employment Commission. 4 th
Report, 1865», p. XXXVIII and XXXIX).
}

\index{i}{0201}  %% посилання на сторінку оригінального видання
Послухаймо тепер, як сам капітал малює цю 24-годинну
систему. Він, звичайно, поминає мовчанням ексцеси цієї системи,
зловживання нею задля «жорстокого й неймовірного» здовжування
робочого дня. Він говорить лише про систему в її «нормальній»
формі.

«Панове Нейлор і Вікерс, фабриканти сталі, які вживають
600 – 700 робітників, поміж ними лише 10\% молодші за 18 років
життя, а з числа останніх лише 20 хлопчаків працює вночі,
висловлюються ось як: «Хлопці зовсім не страждають від спеки.
Температура щось до 86 – 90°\dots{} У кузнях і вальцювальнях
руки працюють вдень і вночі навпереміну, навпаки, вся інша
праця виконується вдень, від 6 години ранку до 6 години вечора.
В кузні працюють від 12 години до 12. Деякі руки постійно працюють
вночі, без зміни денної і нічної праці\dots{} Ми не вважаємо,
щоб денна або нічна праця різно впливали на здоров’я (добродіїв
Нейлора й Вікерса?), певно, люди сплять краще, коли користуються
з відпочинку в той самий час, ніж як його зміняти\dots{}
Щось із двадцять хлопців, молодших за вісімнадцять років,
працюють разом з нічною зміною. Ми не можемо обійтись (not
well do) без нічної праці підлітків, молодших за 18 років. Наше
заперечення — збільшення витрат продукції. Умілі руки й
керівників відділів знайти не легко, хлопців же можна мати
скільки схочете\dots{} Певна річ, беручи до уваги незначний процент
підлітків, що їх ми вживаємо, обмеження нічної праці мало б
для нас невелику вагу або інтерес».\footnote{
Fourth Report etc., 1865, 79, p. XVI.
}

Пан Дж. Елліс з фірми панів Дж. Бравн і К°, з фабрики
сталі й заліза, які вживають 3.000 чоловіків і підлітків,
і саме для частини важких сталевих і залізних робіт, «змінами,
вдень і вночі», заявляє, що у важких сталевих відділах на двох
\parbreak{}  %% абзац продовжується на наступній сторінці
