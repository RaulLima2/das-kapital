\parcont{}  %% абзац починається на попередній сторінці
\index{i}{0536}  %% посилання на сторінку оригінального видання
додаткового продукту, який, з свого боку, є творчий елемент
акумуляції. Отже, вони є разом з тим методи продукції капіталу
капіталом, або методи прискореної акумуляції. Безперервне
перетворювання додаткової вартости знову на капітал виражається
в зростанні величини капіталу, що входить у процес продукції.
З свого боку це стає основою поширеного маштабу продукції,
основою тих метод підвищення продуктивної сили праці,
які супроводять це поширення, і основою прискореної продукції
додаткової вартости. Отже, якщо певний ступінь акумуляції
капіталу являє собою умову специфічно капіталістичного способу
продукції, то цей останній, з свого боку, спричинює прискорену
акумуляцію капіталу. Тому з акумуляцією капіталу розвивається
специфічно капіталістичний спосіб продукції, а із специфічно
капіталістичним способом продукції — акумуляція капіталу.
Ці обидва економічні фактори силою того складного взаємовідношення,
через яке вони один одному дають поштовх,
зумовлюють ту зміну в технічному складі капіталу, наслідком
якої змінна складова частина стає щораз меншою й меншою порівняно
із сталою складовою частиною.

Кожний індивідуальний капітал є більша або менша концентрація
засобів продукції з відповідним пануванням над більшою
або меншою армією робітників. Кожна акумуляція стає засобом
нової акумуляції. Вона поширює із збільшенням маси багатства,
що функціонує як капітал, його концентрацію в руках індивідуальних
капіталістів, а тому поширює й основу продукції
у великому маштабі і основу специфічно капіталістичних метод
продукції. Зростання суспільного капіталу відбувається через
зростання багатьох індивідуальних капіталів. Якщо припустити
всі інші умови за незмінні, то індивідуальні капітали, а разом
з ними й концентрація засобів продукції зростають у тій пропорції,
в якій вони становлять певні частини цілого суспільного
капіталу. Разом з тим від первісного капіталу відриваються
паростки й функціонують як нові самостійні капітали. При цьому
велику ролю відіграє, між іншим, поділ майна в родинах капіталістів.
Тому з акумуляцією капіталу більш або менше зростає
й число капіталістів. Дві обставини характеризують цей рід
концентрації, що безпосередньо спирається на акумуляцію,
або, краще сказати, є з нею ідентична. Поперше: зростання концентрації
суспільних засобів продукції в руках індивідуальних
капіталістів, за інших незмінних умов, є обмежене ступенем
зростання суспільного багатства. Подруге: кожна частина суспільного
капіталу, вкладена в кожну осібну сферу продукції,
є поділена між багатьма капіталістами, які протистоять один
одному як незалежні товаропродуценти, що один з одним конкурують.
Отже, акумуляція й концентрація, що її супроводить,
не тільки роздрібнюються по багатьох пунктах, але й зростання
капіталів, що функціонують, перехрещується з утворенням нових
капіталів і роздрібненням старих. Тому, якщо акумуляція виявляється,
з одного боку, як щораз більша концентрація засобів
\parbreak{}  %% абзац продовжується на наступній сторінці
