Наприкінці XVIII століття і протягом перших десятиліть
XIX століття англійські фармери й лендлорди примусом добилися
абсолютно мінімальної заробітної плати, виплачуючи рільничим
поденникам у формі заробітної плати менше ніш мінімум,
і додаючи їм решту у формі допомоги від парафій. Ось приклад
тих фарсів, що їх витворяли англійські dogberries при «легальному»
встановленні тарифу заробітної плати: «Коли сквайри
1795 р. встановлювали заробітну плату для Speenhamland’y,
вони саме тоді обідали, але, очевидно, гадали, що робітники
чогось такого не потребують... Вони вирішили, що тижнева
плата має бути 3 шилінґи на людину поки буханець хліба
у 8 фунтів 11 унцій коштує 1 шилінґ і має рівномірно зростати
доти, доки буханець коштуватиме 1 шилінґ 5 пенсів.
Коли ціна хліба піднесеться ще вище, то заробітна плата
має пропорційно меншати, поки ціна буханця дійде 2 шилінґів,
і тоді харчі робітника будуть на \sfrac{1}{5} менші, ніш раніш».56
1814 року в слідчому комітеті палати лордів запитали якогось
А. Беннета, великого фармера, суддю, адміністратора дому для
бідних і таксатора заробітної плати: «Чи додержується якоїсь
пропорції між вартістю денної праці й допомогою робітникам
від парафій?» Відповідь: «Так. Тижневий дохід кожної родини
доповнюється поверх її номінального заробітку настільки, щоб
можна було купити буханець вагою в 1 ґальон (8 фунтів 11 унцій)
і мати ще 3 пенси на людину... Ми припускаємо, що буханця
вагою в 1 ґальон досить на утримання кожної особи в родині
протягом тижня; а 3 пенси — то на одяг; коли парафія захоче
сама постачати одяг, то ці три пенси вона відраховує. Ця практика
панує не тільки всюди на захід від Вілтшіру, але, на мою
думку, і в цілій країні».57 «Таким чином, — вигукує один буржуазний
письменник того часу, — фармери протягом багатьох років
спричинювали деґрадацію поважної кляси своїх земляків, примушуючи
їх шукати собі притулку в робітних домах... Фармер
збільшив свій власний дохід тим, що перешкоджав акумуляції
фонду найпотрібнішого споживання на боці робітників». 58 Яку
ролю в утворенні додаткової вартости, а тому і в утворенні
фонду акумуляції капіталу відіграє за наших днів безпосереднє
грабування із доконечного фонду споживання робітника, показала

Англії є не виняток, а правило. Наприклад, аналіза 34 проб опію, купленого
в 34 різних лондонських аптеках, виявила, що 31 проба були фальсифіковані
домішкою макових головок, пшеничного борошна, ґуми,
глини, піску й т. ін. Багато з них не мали й атома морфіну.

56 G. В. Newnham (barrister at law): «А Review of the Evidence
before the Committees of the two Houses of Parliament on the Cornlaws»,
London 1815, p. 28 n.

57 Там же, стор. 19, 20.

58 Ch. H. Parry: «The Question of the Necessity of the existing Cornlaws
considered», London 1816, p. 77, 69. Панове лендлорди з свого боку
не тільки «винагородили» себе за антиякобінську війну, яку вони вели
від імени Англії, а ще й надзвичайно збагатіли. «За вісімнадцять років
їхні ренти збільшились удвоє, утроє, вчетверо, а у виняткових випадках
навіть ушестеро». (Там же, стор. 100, 101).
