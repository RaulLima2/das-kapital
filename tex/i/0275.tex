щодо його тонкости, ситців і інших матерій з Короманделя щодо
пишноти та тривалости фарб ще ніколи не перевищено. А проте
їх продукують без капіталу, без машин, без поділу праці або
якогось із тих інших засобів, що дають так багато переваг європейській фабрикації. Ткач — то
ізольований індивід, що на замовлення споживача виготовлює тканину, а до того — ще й на ткацькому
варстаті якнайпростішої конструкції, варстаті, що
часом складається лише з дерев’яних грубо позбиваних брусів.
У нього немає навіть апарату для натягування основи, і тому
ткацький варстат мусить лишатися розтягнутим на цілу свою
довжину, та такий він незграбний і широкий, що не може вміститися в хаті продуцента, і тому цей
останній мусить виконувати свою працю на вільному повітрі, перериваючи її повсякчас у негоду».\footnote{
«Historical and descriptive Account of British India etc. by
Hugh Murray, James Wilson etc.», Edinburgh 1832, vol. II, p. 449, 450.
Індійський ткацький варстат дуже високий, бо основу натягується вертикально.
}
Лише ця особлива вмілість, що нагромаджувалася
від покоління до покоління та спадково переходила від батька
до сина, дає індусові, як і павукові, цю віртуозність. А проте
порівняно з більшістю мануфактурних робітників такий індійський ткач виконує дуже складну працю.

Ремісник, що виконує один по одному різні частинні процеси
в продукції якогось виробу, мусить змінювати то місце, то інструменти. Перехід від однієї операції
до іншої перериває хід його праці і становить, так би мовити, пори в його робочому дні. Ці пори
звужуються, якщо він протягом цілого дня безупинно
виконує ту саму операцію, або вони зникають у міру того, як
меншає змінливість його операцій. Збільшена продуктивність
постає тут або із збільшення витрати робочої сили протягом
даного часу, отже, із зросту інтенсивности праці, або із зменшення
непродуктивного споживання робочої сили. А саме: зайва витрата
сили, що її вимагає кожний перехід од спокою до руху, компенсується при довшому триванні осягнутої
вже нормальної швидкости
праці. З другого боку, безперервність одноманітної праці ослабляє
напруження уваги та розмах життєвого духа, який саме в зміні
діяльности знаходить свій відпочинок та принаду.

Продуктивність праці залежить не тільки від віртуозности
робітника, але й від досконалосте його знарядь. Знарядь того
самого роду, як, приміром, різальних, свердлильних, поштовхових та ударних і т. ін., вживається в
різних процесах праці, і той самий інструмент у тому самому процесі праці придається до різних
операцій. Однак, скоро тільки різні операції якогось
процесу праці відокремляться одна від одної, і кожна частинна
операція набуде в руках частинного робітника якнайвідповіднішої, а через це й виключної форми, то
постає доконечність змін

Таким чином ніщо не може їм заважати пильно працювати в своїй професії... А до того, діставши багато
правил від своїх прадідів, вони ревно
дбають про те, шоб винайти нові удосконалення». (Diodorus Siculus:
«Historische Bibliothek», Bd. I, Kap. 74, S. 117, 118).