\parcont{}  %% абзац починається на попередній сторінці
\index{i}{0279}  %% посилання на сторінку оригінального видання
продукції, аж поки набере своєї остаточної форми. Навпаки, коли
розглядатимемо майстерню як сукупний механізм, то виявиться,
що сировинний матеріял перебуває одночасно в усіх своїх фазах
продукції. Однією частиною своїх багатьох озброєних знаряддям
рук колективний робітник, скомбінований із частинних робітників,
тягне дріт, тоді як іншими руками та знаряддям він одночасно
вирівнює його, іншими ріже його, загострює й т. д. Послідовність
різних стадій процесу в часі перетворюється на одночасність
існування цих стадій у просторі. Звідси виготовлення більшої
кількости товару протягом того самого часу.\footnote{
«Він (поділ праці) зумовлює економію часу, розчленовуючи
працю на різні операції, які можна виконувати всі одночасно\dots{} В наслідок
одночасного виконування всіх тих різних процесів праці, які одна
людина мусить провадити послідовно, один по одному, утворюється,
наприклад, можливість виробити велике число цілком закінчених
шпильок протягом того самого часу, якого треба на те, щоб обрізати й
загострити одну шпильку». («It (the division of labour) produces also
an economy of time, by separating the work into its different branches, all
of which may be carried on into execution at the same moment\dots{} By carrying
on all the different processes at once, which an individual must have
executed separately, it becomes possible to produce a multitude of pins
for instance completely finished in the same time as a single pin might
have been either cut or pointed»). (\emph{Dugald Stewart}: Works, edited
by Sir W. Hamilton, Edinburgh 1855, vol. Ill, «Lectures on Political
Economy», p. 319).
} Ця одночасність
виникає, щоправда, з загальної кооперативної форми цілого
процесу, але мануфактура не тільки находить уже готові умови
кооперації, вона почасти лише сама створює їх, розчленовуючи
ремісничу працю. З другого боку, вона досягає цієї суспільної
організації процесу праці, лише міцно приковуючи того самого
робітника до того самого деталю.

Що частинний продукт кожного частинного робітника разом
з тим є лише осібний ступінь у розвитку того самого продукту,
то один робітник постачає іншому, або одна група робітників
іншій сировинний матеріял. Результат праці одного становить
вихідний пункт для праці іншого. Отже, один робітник тут безпосередньо
дає працю іншим. Робочий час, доконечний для
досягнення у кожному частинному процесі корисного ефекту,
що його має на меті цей процес, установлюється з досвіду, і цілий
механізм мануфактури ґрунтується на тій передумові, що протягом
даного робочого часу досягається якогось даного результату.
Лише за цієї передумови різні процеси праці, процеси, що
один одного доповнюють, можуть відбуватися безперервно,
одночасно та просторово один поруч одного. Ясно, що ця безпосередня
взаємна залежність праць, а тому й робітників примушує
кожного окремого робітника витрачати на свою функцію лише
доконечний робочий час, і таким чином досягається цілком іншої
безперервности, одноманітности, правильности, порядку,\footnote{
«Що більше різноманітности серед робітників мануфактури\dots{} то
більші порядок і реґулярність у кожній роботі, то менше мусить витрачатись
на неї часу, то менше мусить витрачатись праці» («The more variety
of artists to every manufacture\dots{} the greater the order and regularity of
every work, the same must needs be done in less time, the labour must
be less»). («The Advantages of the East-India Trade», London 1720, p. 68).
} а особливо
\index{i}{0280}  %% посилання на сторінку оригінального видання
й інтенсивности праці, аніж у незалежному реместві або
навіть у простій кооперації. Та обставина, що на продукцію
якогось товару витрачається лише суспільно-доконечний робочий
час, за товарової продукції з’являється взагалі як зовнішній
примус конкуренції, бо, висловлюючись поверхово, кожний
поодинокий продуцент мусить продавати товар за його ринковою
ціною. Навпаки, у мануфактурі виготовлення даної кількости
продукту протягом даного часу стає технічним законом самого
процесу продукції.\footnote{
Проте цього результату мануфактурне підприємство в багатьох
галузях продукції доходить лише недосконало, бо мануфактура не вміє
з певністю контролювати загальні хемічні й фізичні умови процесу продукції.
}

Однак різні операції потребують неоднакового часу й тому
дають протягом однакового часу неоднакові кількості частинних
продуктів. Отже, коли той самий робітник день-у-день повинен
виконувати завжди лише ту саму операцію, то для різних
операцій мусить бути вжите відносно різне число робітників,
приміром, у черенковій мануфактурі на одного ґлянсувальника
четверо ливарників та двоє відламувачів, бо один ливарник виливає
за годину 2.000 черенків, один відламувач одламує 4.000, а
ґлянсувальник ґлянсує начисто 8.000. Тут принцип кооперації
повертається назад до своєї найпростішої форми, до одночасного
вживання багатьох робітників, що роблять однорідну роботу,
але цей принцип стає тепер виразом певного органічного відношення.
Отже, мануфактурний поділ праці не тільки спрощує
і урізноманітнює якісно відмінні органи суспільного колективного
робітника, а й утворює тривале математичне відношення
для кількісного обсягу цих органів, тобто для відносного числа
робітників або для відносної величини робітничих груп у кожній
окремій функції. Разом з якісним розчленуванням він розвиває
й кількісну норму (Regel) та пропорційність суспільного процесу
праці.

Якщо для певного маштабу продукції на основі досвіду встановлено
якнайвідповіднішу пропорційність між різними групами
частинних робітників, то поширити цей маштаб можна лише тоді,
коли вжити кратну кількість робітників кожної з цих окремих
груп.\footnote{
«Якщо досвід залежно від осібної природи продукту кожної ма
нуфактури показав так найвигідніший спосіб поділу фабрикації на
частинні операції, як і потрібне для цього число робітників, то всі підприємства,
що не вживатимуть точної кратної кількости цього числа
робітників, будуть продукувати з більшими витратами\dots{} Це одна з
причин колосального поширення промислових підприємств». (\emph{Ch. Babbage}:
«On the Economy of Machinery», London 1832. ch. XXI, p. 172,173).
} До цього долучається ще й те, що той самий індивід може
виконувати деякі роботи однаково добре, все одно, чи провадяться
вони у великому чи в малому розмірі, приміром, роботи догляду,
транспортування частинних продуктів з однієї продукційної фази
\parbreak{}  %% абзац продовжується на наступній сторінці
