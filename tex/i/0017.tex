властивість не прозирає крізь його тканину, хоч і як вона буде
прозориста. І у вартостевому відношенні до полотна він має значення
лише з цього боку, отже, як утілена вартість, як тіло вартости.
Не вважаючи на те, що сурдут з’являється застебнений на
всі ґудзики, полотно пізнає в ньому споріднену собі прекрасну
душу вартости. Однак, сурдут не може супроти полотна репрезентувати
вартість без того, щоб вартість одночасно не набрала
для полотна форми сурдута. Так особа А не може поставитися до
особи B як до якоїсь ясновельможности без того, щоб ця ясновельможність
одночасно не набрала для А тілесного вигляду В;
тому то риси обличчя, волосся й ще дещо інше міняється кожного
разу разом зі зміною володаря країни.

Отже, в тому вартостевому відношенні, що в ньому сурдут
становить еквівалент полотна, форма сурдута фігурує як форма
вартости. Тому вартість товару «полотно» виражається в тілі
товару «сурдут», вартість одного товару — у споживній вартості
іншого. Як споживна вартість, «полотно» є річ, почуттєво відмінна
від сурдута, як вартість — воно є «сурдуторівне» й тому
виглядає як сурдут. Таким чином воно набирає форми вартости,
відмінної від його натуральної форми. Його вартостеве буття
виявляється в його рівності сурдутові так само, як овеча натура
християнина виявляється в тому, що він подібний агнцеві божому.

Ми бачимо, що все те, що нам сказала раніш аналіза товарової
вартости, каже caмé полотно, скоро воно вступає в стосунки з
іншим товаром, з сурдутом. Воно лише висловлює свої думки єдино
приступною йому мовою — мовою товарів. Щоб сказати, що
праця в її абстрактній властивості людської праці утворює його
власну вартість, полотно каже, що сурдут, оскільки він йому
рівнозначний, тобто оскільки він є вартість, складається з тієї
самої праці, як і воно, полотно. Щоб висловити, що його виспренна
предметність вартости відмінна від його жорсткого полотняного
тіла, воно каже, що вартість має вигляд сурдута, і тому
воно саме як предмет вартости схоже на сурдут, як дві краплі
води. Зауважмо до речі, що товарова мова, опріч єврейської, має;
багато інших більш або менш точних говірок. Німецька «Wertsein»,
наприклад, менше влучно, ніж романське дієслово valere,
valer, valoir, висловлює, що порівняння товару В з товаром А є
вираз власної вартости товару A. Paris vaut bien une messe.\footnote*{
Париж таки вартий служби божої. Ред.
}

Отже, за допомогою вартостевого відношення натуральна форма
товару В стає формою вартости товару А, або тіло товару В
стає дзеркалом вартости товару А.\footnote{
У деякому відношенні з людиною справа стоїть так, як із товаром.
Через те, що вона родиться на світ ані з дзеркалом, ані як фіхтівський
філософ: «Я є я», то людина спершу видивляється в іншу людину, як у
дзеркало. Лише через відношення до людини Павла як до подібного до
себе, людина Петро відноситься й до себе самої як до людини. Але тим
самим і Павло з шкурою й волоссям, у його Павловій тілесності, стає для
нього за форму виявлення роду «людина».
} Товар А, відносячись до товару
В як до тіла вартости, як до матеріялізації людської праці, ро-