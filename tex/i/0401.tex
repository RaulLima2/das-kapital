На фабриках і мануфактурах, не підданих ще фабричному
законові, панує якнайстрашніша надмірна праця: періодично,
підчас так званого сезону, і спорадично — в наслідок раптових замовлень. У зовнішніх відділах
фабрики, мануфактури й
крамниць, тобто у сфері домашньої праці, і без того цілком нерегулярної, цілком залежної щодо
сировинного матеріялу й замовлень
від примх капіталіста, який тут не є обмежений міркуваннями
про експлуатацію будівель, машин тощо і не ризикує нічим, хіба
тільки шкурою самих робітників, — у цій сфері таким чином систематично вирощують промислову резервну
армію, якою завжди
можна порядкувати і яку протягом однієї частини року винищують якнайжорстокішим примусом до праці, а
протягом другої
частини зводять до стану голоти через брак праці. «Підприємці, — каже «Children’s Employment
Commission», — визискують
звичну нереґулярність домашньої праці, щоб у ті часи, коли
потрібна нагальна праця, здовжувати її до 11, 12, 2 години
вночі, а в дійсності, як каже поширена там фраза, здовжувати
її до «всякої години» та ще до того в помешканнях, «де самого
лише смороду досить, щоб вас збити з ніг (the stench is enough
to knock you down). Ви дійдете, може, до дверей і відчините їх,
але вас пройме жах, і ви не підете далі».\footnote{
«Children’s Employment Commission, 4th Report», p. XXXV,
n. 235 і 237.
} «Чудаки оті наші
підприємці, — каже один із вислуханих свідків, швець, — вони
гадають, нібито дитині не шкодить, коли протягом однієї половини року її на смерть виснажують
працею, а протягом другої
половини майже примушують тинятися без роботи».\footnote{
Там же, стор. 127, n. 56.
}

Як про технічні перешкоди, так само й про ці так звані «промислові звички» («usages which have grown
with the growth of
trade»\footnote*{
— звички, що розвивалися разом з розвитком промислів. Ред.
}) заінтересовані капіталісти твердили й твердять, ніби
вони є «природні межі» продукції — улюблений лемент бавовняних лордів тих часів, коли вперше почав
їм загрожувати фабричний закон. Хоч їхня промисловість більше ніж усяка інша
спирається на світовий ринок, а тому й на судноходство, однак
досвід спростував їхню брехню. Від того часу англійські фабричні
інспектори дивляться на кожну таку «промислову перешкоду»
як на просту викрутку.\footnote{
«Щодо втрат, що їх зазнає торговля в наслідок невчасного виконання замовлень на товари, які
доводиться перевозити морем, то я пригадую
собі, що це був улюблений арґумент панів фабрикантів у 1832 та 1833 рр.
Все, що можна тепер сказати з цього приводу, не має такої ваги, як тоді,
коли пара не скоротила ще наполовину всіх дистанцій і не утворила нових
умов для перевозу. І в той час цей арґумент не витримував критики, а
тепер він зовсім не витримує її». («With respect to the loss of trade by
the non-completion of shipping orders in time, I remember that this was
the pet argument of the factory masters in 1832 and 1833. Nothing that
can be advanced now on this subject could have the force that it had then,
before steam had halved all distances and established new regulations
} Ґрунтовні й сумлінні досліди «Child-