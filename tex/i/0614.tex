geant,* то великі землевласники незабаром відкриють, що Ір-
ляндія із 3\sfrac{1}{2} мільйонами людности все ще бідна країна, а бідна,
тому що перелюднена, отже, збезлюднення її мусить піти ще
значно далі, щоб вона могла виконати своє справжнє призначення
бути за пасовисько для овець і рогатої худоби Англії.188b

Ця корисна метода, які і все гарне на цьому світі, має свій
поганий бік. Рівнобіжно з акумуляцією земельної ренти в Ірляндії
ірляндці акумулюються в Америці. Ірляндець, що його виганяють
вівці та бики, з’являється по той бік океану, як феній.
І проти старої владарки морів повстає чимраз грізніш велетенська
молода республіка.

                                      Acerba fata Romanos agunt
                                       Scelusque fraternae necis.**

Розділ двадцять четвертий

Так звана первісна акумуляція

1. Таємниця первісної акумуляції

Ми бачили, як гроші перетворюються на капітал, як за допомогою
капіталу утворюється додаткова вартість, а з додаткової
вартости — додатковий капітал. Але акумуляція капіталу має за
передумову додаткову вартість, додаткова вартість — капіталі-

188b Як окремі земельні власники й англійське законодавство пляномірно
використовували голод і викликані ним обставини, щоб силоміць
провести революцію в рільництві і звести людність Ірляндії до кількости,
вигідної для лендлордів, це я покажу докладніше у третій книзі
цього твору, у відділі про земельну власність. Там я повернуся й до становища
дрібних фермерів і сільських робітників. Тут я подам лише одну
цитату. Нассау В. Сеніор у своєму посмертному творі «Journals, Conversations
and Essays relating to Ireland». 2 volumes. London 1868,
vol. II, p. 282 каже, між іншим, ось що: «Влучно зауважив д-р Ґ., що в
нас є закон про бідних, і що він є могутнє знаряддя, щоб забезпечити
перемогу лендлордам; друге знаряддя — еміґрація. Жоден друг Ірляндії
не побажає, щоб війна (між лендлордами й дрібними кельтськими
фармерами) тривала далі, — ще менш, щоб вона скінчилась перемогою
фармерів... Що швидше вона (ця війна) скінчиться, що швидше Ірляндія
перетвориться на пасовиська (grazing country) з порівняно нечисленною
людністю, якої треба для пасовиськ, то краще для всіх кляс». — Англійські
хлібні закони 1815 р. забезпечували Ірляндії монополію вільно довозити
хліб у Великобрітанію. Таким чином вони штучно сприяли рільництву.
У 1846 р. разом із скасуванням хлібних законів одразу знищено
і цю монополію. Не кажучи вже про всі інші обставини, лише цієї
події було досить, щоб надати потужного поштовху перетворенню ірляндської
орної землі на пасовиська, концентрації фарм і вигнанню дрібних
селян. Після того, як протягом 1815—1846 рр. уславляли родючість
ірляндського ґрунту і вселюдно оголосили, що з самої природи
його призначено виключно на культивування пшениці, тепер англійські
аґрономи, економісти, політики раптом зробили відкриття, що він придатний
лише для культивування кормових трав! Пан Леонс де Лявернь
поспішив повторити це по той бік каналу. Треба бути такою «серйозною»
людиною, як пан Лявернь, щоб йняти віри таким наївним теревеням.

* — апетит приходить під час їди. Ред.

** Жене римлян сувора доля і злочин братовбивства. Ред.
