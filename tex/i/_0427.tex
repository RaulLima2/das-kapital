\parcont{}  %% абзац починається на попередній сторінці
\index{i}{0427}  %% посилання на сторінку оригінального видання
вартости становить загальну основу капіталістичної системи
й вихідний пункт продукції відносної додаткової вартости. При
продукції відносної додаткової вартости робочий день із самого
початку поділено на дві частини: на доконечну працю й додаткову
працю. Щоб здовжити додаткову працю, скорочують доконечну
працю методами, за допомогою яких еквівалент заробітної
плати продукується за коротший час. Продукція абсолютної
додаткової вартости стосується лише до довжини робочого дня;
продукція відносної додаткової вартости геть цілком революціонізує
технічні процеси праці й суспільні угруповання.\footnote*{
У другому німецькому виданні цей абзац викладено так: «Продукція
абсолютної додаткової вартости і продукція відносної додаткової
вартости з’являються перед нами насамперед як два різні способи продукції,
належні різним епохам розвитку капіталу. Продукція абсолютної
додаткової вартости має за передумову, що речові умови праці є перетворені
на капітал, а робітники — на найманих робітників, що продукти
продукується як товари, тобто на продаж, що процес продукції є разом
з тим процес споживання робочої сили капіталом і тому підпорядкований
безпосередньому контролеві капіталіста, нарешті, що процес праці, отже,
робочий день здовжується поза той час, протягом якого робітник продукує
лише еквівалент вартости своєї робочої сили. Припускаючи загальні
умови товарової продукції за дані, продукція абсолютної додаткової вартости
є просто здовження робочого дня поза межі робочого часу, доконечного
для життя самого робітника, і присвоєння капіталом додаткової
праці. Цей процес може відбуватися і відбувається на основі способів
продукції, що історично передані без усякої допомоги з боку капіталу.
Тоді відбувається лише формальна метаморфоза, або капіталістичний
спосіб експлуатації відрізняється від попередніх, як, наприклад, від
системи рабства, лише тим, що за цієї останньої додаткову працю видушується
безпосереднім примусом, тимчасом як за капіталістичного способу
експлуатації, видушення додаткової праці упосереднюється «вільним»
продажем робочої сили. Отже, продукція абсолютної додаткової
вартости припускає лише формальну підпорядкованість праці капіталові». \emph{Ред.}
}

Отже, продукція відносної додаткової вартости має за передумову
специфічно капіталістичний спосіб продукції, який із
своїми методами, засобами й умовами сам стихійно постає й розвивається
лише на основі формальної підпорядкованости праці
капіталові. На місце формальної підпорядкованости стає реальна
підпорядкованість праці капіталові.

Досить буде тільки зазначити покручні форми, коли додаткову
працю не висисають із продуцента безпосереднім примусом,
а формальна підпорядкованість продуцента капіталові ще не настала.
Тут капітал не опанував ще безпосередньо процесу праці.
Поряд самостійних продуцентів, що працюють коло ремества
або рільництва традиційним, прадідівським способом, виступає
лихвар або купець, лихварський або торговельний капітал, що
паразитично висисає їх. Панування цієї форми експлуатації в
суспільстві виключає капіталістичний спосіб продукції, хоча,
з другого боку, вона й може становити переходовий ступінь до
останнього, як це було за пізнього середньовіччя. Нарешті, як
показує приклад сучасної домашньої праці, певні покручні
\parbreak{}  %% абзац продовжується на наступній сторінці
