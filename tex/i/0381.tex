Більшість дрібних фабрик були ткальні, засновані під час розцвіту
після 1858 р. здебільша спекулянтами, з яких один постачав
пряжу, другий — машини, третій — будівлі; заправляли колишні
overlookers\footnote*{
— фабричні наглядачі. Ред.
} та інші немаєтні люди. Ці дрібні фабриканти здебільша
позагибали. Таку саму долю була б заготовила їм торговельна
криза, до якої не припустив голод на бавовну. Хоч вони
й становили 1/3 числа фабрикантів, однак їхні фабрики увібрали
незрівнянно меншу частину капіталу, вкладеного в бавовняну промисловість.
Щодо розмірів кризи, то, на основі автентичного
цінування, в жовтні 1862 р. стояло без ніякої роботи 60,3\%
веретен та 58\% ткацьких варстатів. Ці числа стосуються до всієї
цієї галузі промисловости і, певна річ, дуже варіюють в окремих
округах. Лише дуже небагато фабрик працювало повний чаc (60 годин на тиждень), решта працювала з
перервами. Навіть
для тих небагатьох робітників, які працювали повний час
та за звичну відштучну плату, тижневий заробіток неминуче зменшувався
через заміну ліпшого сорту бавовни на гірший, бавовни
Sea Island на єгипетську (в тонкопрядільнях), американської та
єгипетської — на суратську (східньоіндійську), чистої бавовни —
на суміш відпадків бавовни й сурату. Коротші волокна бавовнисурату,
її забрудненість, більша ламкість ниток, заміна борошна
при шліхтуванні основи пряжі на всякого роду важкі інґредієнти
тощо — все це зменшувало швидкість машин або число ткацьких
варстатів, за якими міг наглядати один ткач, збільшувало працю
в наслідок хибної роботи машин та зменшувало разом із масою
продукту відштучну заробітну плату. При вживанні сурату та
при повночасній праці втрата робітника доходила до 20, 30 і
більше процентів. Але більшість фабрикантів і норму відштучної
плати знизили на 5, 7 1/2 і 10 процентів. Тому можна зрозуміти
становище тих, що працювали лише 3, 3 1/2 і 4 дні на тиждень,
або тільки по 6 годин денно. 1863 р., після того, як настало вже
відносне поліпшення, тижнева заробітна плата ткачів, прядунів
тощо становила 3 шилінґи 4 пенси, 3 шилінґи 10 пенсів, 4 шилінґи
6 пенсів, 5 шилінґів 1 пенс і т. ін.\footnote{
«Reports of Insp. of Fact, for 31 st October 1865», p. 41—45.
} Навіть при такому злиденному
стані винахідницький дух фабрикантів щодо відраховань
від заробітної плати не завмирав. Почасти це були кари за вади
в продукті в наслідок поганої бавовни, поганих машин і т. ін.
А там, де фабрикант був власником котеджів робітників, він сам
собі платив квартирну плату, відлічуючи її від номінальної
заробітної плати. Фабричний інспектор А. РедГрев оповідає
про selfacting minders (вони наглядають за двома автоматичними
мюлями), які «за чотирнадцятиденну повну працю заробляли
8 шилінґів 11 пенсів; із тієї суми в них відраховували плату за
помешкання, при чому, однак, фабрикант половину цієї суми
повертав їм як подарунок, так що minders’и приносили додому