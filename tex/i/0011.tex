вартости сурдутів підвищується разом з їхньою кількістю. Коли
1 сурдут репрезентує х, то 2 сурдути репрезентують 2х; робочих
днів і т. д. Але припустімо, що праця, доконечна для продукції
одного сурдута, збільшується удвоє або зменшується наполовину.
В першому випадку один сурдут має стільки вартости, скільки
раніше мали два сурдути, в останньому випадку два сурдути
мають лише стільки вартости, скільки раніше мав один сурдут,
хоч в обох випадках один сурдут виконує ту саму службу, що й
раніш, і вміщена в ньому корисна праця має ту саму якість, що
й раніш. Але кількість праці, витрачена на його продукцію,
змінилась.

Більша кількість споживної вартости становить сама по собі
більше речове багатство: два сурдути більше, ніж один; двома
сурдутами можна одягти двох людей, одним сурдутом тільки одну
людину й т. д. Проте, збільшенню маси речового багатства може
відповідати одночасне зменшення величини його вартости. Цей
протилежний рух походить з двоїстого характеру праці. Продуктивна
сила є, звичайно, завжди продуктивна сила корисної, конкретної
праці, і в дійсності вона визначає тільки ступінь діяльносте
доцільної продуктивної праці в даний період часу. Тому
корисна праця стає багатшим або біднішим джерелом продуктів
просто пропорційно до підвищення або падіння її продуктивної
сили. Навпаки, зміна продуктивної сили сама по собі зовсім не
зачіпає праці, репрезентованої вартістю. А що продуктивна сила належить
до конкретної корисної форми праці, то вона, природно, не
може вже більше торкатися праці, скоро тільки ми абстрагувались
від її конкретної корисної форми. Тому та сама праця дає в однакові
періоди часу завжди однакову величину вартости, хоч і як
змінюватиметься продуктивна сила. Але вона дає в однаковий
період часу різні кількості споживних вартостей: більше — коли
продуктивна сила підвищується, і менше коли вона падає. Отже,
та сама зміна продуктивної сили, що збільшує видатність праці,
а тим самим і масу даваних нею споживних вартостей, зменшує
величину вартости цієї збільшеної загальної маси, якщо вона
скорочує суму робочого часу, доконечного для її продукції. І так
само навпаки.

Всяка праця є, з одного боку, затрата людської робочої сили
У фізіологічному розумінні, і в цій властивості однакової людської
або абстрактної людської праці вона творить (bildet) товарову
вартість. Всяка праця є, з другого боку, затрата людської робочої
сили в осібній доцільно-визначеній формі, і в цій властивості
конкретної корисної праці вона продукує споживні вартості.16

16    Примітка до другого видання. Щоб довести, «що сама тільки праця є
остаточна й реальна міра, якою можна цінувати й порівнювати вартість
усіх товарів у всі часи», А. Сміс каже: «Рівні кількості праці мусять в
усі часи й по всіх місцях мати для самого робітника однакову вартість.
При нормальному стані свого здоров'я, сили й діяльности і пересічному
ступені вправности, яку він може мати, він мусить завжди віддавати однакову
частину свого спокою, своєї волі й свого щастя». («Wealth of Nations»,
b. I, ch. V, p. 104, 105). З одного боку, А. Сміс плутає тут (не скрізь)
