\parcont{}  %% абзац починається на попередній сторінці
\index{i}{0169}  %% посилання на сторінку оригінального видання
продукту лежать одна поруч однієї, у час, де вони йдуть одна по одній. Алеж цю формулу можуть
супроводити і дуже варварські ідеї, особливо в головах, які практично так само заінтересовані
в процесі зростання вартости, як і в тому, щоб теоретично зрозуміти його хибно. Так, можна собі
уявити, що, приміром, наш прядун за перші 8 годин свого робочого дня продукує або покриває вартість
бавовни, за дальші 1 годину й 36 хвилин — вартість зужиткованих засобів праці, за дальші 1 годину 12
хвилин — вартість заробітної плати й лише преславетну «останню
годину» присвячує фабрикантові, продукції додаткової вартости. Таким чином прядунові накидають
подвійне диво, а саме, що він нібито продукує бавовну, веретена, парову машину, вугілля,
олію і т. д. в той самий момент, коли він ними пряде, і з одного робочого дня даного ступеня
інтенсивности робить п’ять
таких днів. Саме в нашому випадку продукція сировинного матеріялу й засобів праці потребує 24/6 = 4
дванадцятигодинних робочих днів, а їхнє перетворення на пряжу — ще одного дванадцятигодинного
робочого дня. Що хижацтво вірить у такі дива й що йому ніколи не важко знайти доктринера-сикофанта,
який їх довів би, про це свідчить один славнозвісний в історії приклад.

3. «Остання година» Сеніора

Одного чудового ранку 1836 р. Нассав В. Сеніор, відомий своїми економічними знаннями і своїм чудовим
стилем, цей, сказати б, Кльорен серед англійських економістів, був запрошений
з Оксфорду до Менчестеру, щоб учитися тут політичної економії замість навчати її в Оксфорді.
Фабриканти обрали його на борця проти недавно виданого Factory Act\footnote*{
— фабричного закону. Ред.
} і проти аґітації за
десятигодинний
робочий день, яка тривала ще далі. Із звичною практичною дотепністю вони розпізнали, що пан професор
«wanted а
good deal offinishing».\footnote*{
— потребує ще порядного остаточного оброблення. Ред.
} Тим то вони й виписали його до Менчестеру. Пан професор з свого боку
устилізував лекцію, дану йому в Менчестері фабрикантами, у памфлеті «Letters on the Factory Act, as
it affects the cotton manufacture. London. 1837». Тут можна вичитати, між іншим, такі повчальні
місця:

«За теперішнього закону жодна фабрика, що на ній роблять особи, молодші за 18 років, не може
працювати більш як 11 1/2 годин на день, тобто по 12 годин перших п’ять днів тижня й 9 годин
суботами. Дальша аналіза (!) доводить нам, що на такій фабриці ввесь чистий прибуток походить від
останньої години. Фабрикант
витрачає 100.000 фунтів стерлінґів: 80.000 фунтів стерлінґів на фабричні будівлі й машини, 20.000
фунтів стерлінґів — на сировинний матеріял і заробітну плату. Припускаючи, що капітал обертається
один раз на рік і що гуртовий прибуток становить
\parbreak{}  %% абзац продовжується на наступній сторінці
