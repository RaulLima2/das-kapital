Тому, за винятком відділу про форму вартости, нарікати на цю
книжку, ніби її важко зрозуміти, не можна. Я, звичайно, маю
на увазі таких читачів, що хочуть навчитись чогось нового, отже,
що і сами хочуть думати.

Фізик спостерігає процеси природи або там, де вони виявляються
в щонайвиразнішій формі і де їх якнайменше затемнюють
перешкодні впливи, або по змозі сам робить експерименти серед
умов, що забезпечують чистий перебіг процесу. Те, що я маю
дослідити в цій праці, є капіталістичний спосіб продукції і відповідні
йому відносини продукції та обміну (Verkehr). Англія
є донині клясичною країною цього способу продукції. Це —
причина, чому, головним чином, вона служить за ілюстрацію
до мого теоретичного викладу. Однак, якщо німецький читач
дозволив би собі по-фарисейському знизувати плечима з приводу
становища англійських промислових та рільничих робітників або
оптимістично заспокоювати себе, що в Німеччині справи ще далеко
не такі погані, то я муситиму голосно сказати йому: De
te fabula narratur! *

По суті справа не у вищому або нижчому ступені розвитку
суспільних противенств, що виникають із природних законів капіталістичної
продукції. Справа в самих цих законах, у самих цих
тенденціях, що діють і пробивають собі шлях із залізною доконечністю.
Промислово розвиненіша країна показує менш розвиненій
країні лише картину її власної будучини.

Але залишмо осторонь ці міркування. Там, де капіталістична
продукція в нас цілком вкоренилася, приміром, на фабриках
у власному значенні, — там обставини значно гірші, ніж в Англії,
бо немає противаги фабричних законів. У всіх інших сферах ми,
як і всі інші країни західньоевропейського континенту, терпимо
не лише від розвитку капіталістичної продукції, але ще й від
недостатности цього розвитку. Поруч із сучасною скрутою на
нас тисне цілий ряд скрутних обставин, які дісталися нам у
спадщину, і які походять з того, що й далі животіють стародавні,
пережиті вже, способи продукції, в супроводі суперечних
сучасності суспільних і політичних відносин. Ми терпимо
не лише від живих, але й від мертвих. Le mort saisit le vif! **

Супроти англійської соціяльна статистика Німеччини та інших
країн західньоевропейського континенту мізерна. Однак, вона
досить відкриває вуаль, щоб можна було здогадатися про голову
Медузи, що таїться за ним. Ми ужахнулися б наших власних
обставин, коли б наші уряди й парляменти, як ось в Англії, призначали
періодичні комісії для дослідження економічних обставин,
коли б цим комісіям давали такі самі уповноваження досліджувати
правду, як в Англії, коли б пощастило знайти для цього
таких компетентних, безсторонніх і рішучих людей, як англійські

* — про тебе історію оповідається! Ред.

** Мертвий хапає живого. Ред.
