\parcont{}  %% абзац починається на попередній сторінці
\index{i}{0577}  %% посилання на сторінку оригінального видання
очі в швидко гаснущий огонь. Таке спустіння, така безнадійність
повисла над цими людьми й над їхньою малюсінькою кімнаткою,
що я не хотів би побачити подібну сцену ще раз. «Вони,
пане, нічого не заробили, — сказала жінка, показуючи на своїх
дітей, — нічого за 26 тижнів, і всі наші гроші витрачені, всі гроші,
які я й батько відкладали були за кращих часів, сподіваючися
забезпечити собі резерв на лихий час. Ось дивіться, — майже
диким голосом закричала вона, показуючи нам банкову книжечку
з усіма її акуратними записами вкладених і вибраних грошей,
так що ми могли бачити, як невеличка маєтність почалася
від першого вкладу в 5\shil{ шилінґів}, як вона помалу зросла до
20\pound{ фунтів стерлінґів}, а потім почала знову спадати, з фунтів до
шилінґів, аж доки останній запис зробив книжку такою безвартісною,
як якийсь шматок паперу. Ця родина діставала з робітного
дому раз на день один злиденний обід\dots{} Дальша наша візита
була до дружини одного ірляндця, що працював на корабельні.
Ми найшли її хорою від недостачі харчів, вона лежала в одежі,
простягнувшись на матраці, ледве прикрита шматком килиму,
бо всі постільні речі були в заставі. Нещасні діти доглядали її,
але видно було, що вони сами потребують материної опіки.
Дев’ятнадцять тижнів примусового безділля довели її до цього
стану; оповідаючи нам історію свого гіркого минулого, вона
стогнала так, немов утратила всяку надію на кращу будучину\dots{}
Коли ми вийшли з дому, до нас підбіг якийсь молодий чоловік
і попросив нас зайти до його дому та поглянути, чи не можна
чого зробити для нього. Молода жінка, двоє вродливих діток,
купа заставничих квитків, цілком гола кімната — оце було все,
що він мав нам показати».

Наведемо ще витяг з однієї торійської газети про лихі наслідки
кризи 1866~\abbr{р.} Не треба забувати, що східня частина Лондону,
про яку тут мовиться, є місце пробування не лише будівельників
залізних кораблів, згаданих у тексті цього розділу, але й
робітників так званої «домашньої праці», що її оплата завжди
стоїть нижче мінімуму\dots{} «Жахна драма розгорнулася вчора в
одній частині столиці. Хоч тисячі безробітних Істенду з чорними
жалібними прапорами й не влаштували масової демонстрації,
а проте натовп був досить імпозантний. Пригайдамо собі, як
страждає ця людність. Вона вмирає з голоду. Це — простий і
страшний факт. Їх \num{40.000}. На наших очах, в одному з кварталів
цієї дивної столиці, поруч величезного нагромадження багатства,
яке колибудь тільки бачив світ, поруч цього \num{40.000} людей без
жодної допомоги вмирають голодною смертю! Тепер ці тисячі
вдираються до інших кварталів; завжди напівголодні, вони кричать
нам у вуха про свої страждання, голосять про них до
неба, оповідають нам про свої злиденні халупи, про те, що вони
не можуть найти роботу, і що марна річ була б їм просити
милостині. Місцеві платники податку на користь бідних самі
стоять на межі павперизму в наслідок вимог з боку парафій».
(«Standard», 5 April 1867).
