Інша справа з міновою вартістю. «Людина, що має багато вина,
а не має збіжжя, веде торг з людиною, що в неї багато збіжжя,
але немає вина, і вони обмінюють пшеницю вартістю в 50 на
вино вартістю в 50. Цей обмін не є збільшення мінової вартости
ні для одного, ані для другого, бо вже перед обміном кожний
з них мав вартість, рівну тій, що її він здобув собі за допомогою
цієї операції».\footnote{
Mercier de la Rivière: «L’Ordre naturel et essentiel», Physiocrates, éd.
Daire, IІ. Partie, p. 544.
} Справа зовсім не змінюється, коли між товарами
виступають гроші як засіб циркуляції і акт купівлі почуттєво
відокремлюється від акту продажу.\footnote{
«Само по собі цілком байдуже, чи є одна з цих двох вартостей
гроші, чи обидві вони є звичайні товари» («Que l’une de ces deux valeurs
soit argent, ou qu’elles soient toutes deux marchandises usuelles, rien de
plus indifférent en soi»). (Mercier de la Rivière: «L’Ordre naturel et essentiel».
Physiocrates, éd. Daire, II. Partie, p. 543).
} Вартість товарів є виражена
в їхніх цінах раніш, ніж вони вступають до циркуляції, отже,
вона є передумова циркуляції, а не її результат.\footnote{
«Не контраґенти визначають вартість, її визначено ще до оборудки»
(«Ce ne sont pas les contractants, qui prononcent sur la valeur; elle est
décidée avant la convention»). (Le Trosne: «De l’Intérêt Social», Physiocrates,
éd. Daire, Paris 1846, p. 906).
}

Розглядаючи справу абстрактно, тобто залишаючи осторонь
обставини, які не випливають з іманентних законів простої товарової
циркуляції, ми побачимо, що, крім заміни однієї споживної
вартости на іншу, в ній відбувається лише метаморфоза, проста
зміна форми товару. Та сама вартість, тобто та сама кількість
упредметненої суспільної праці, лишається в руках того самого
посідача товарів спочатку в формі товару, потім у формі грошей,
на які товар перетворився, нарешті, у формі товару, на який
знову перетворилися ці гроші. Ця зміна форми не містить у собі
жодної зміни величини вартости. Зміна, якої зазнає в цьому процесі
сама вартість товару, обмежується на зміні її грошової
форми. Спочатку вона існує як ціна подаваного на продаж товару,
потім як грошова сума, що була вже однак виражена в ціні, і,
нарешті, як ціна еквівалентного товару. Ця зміна форм сама по
собі так само мало містить у собі зміну величини вартогти, як
ось розмін п’ятифунтової банкноти на соверени, півсоверени й
шилінґи. Отже, оскільки циркуляція товару зумовлює лише
зміну форми його вартости, вона зумовлює, — якщо явище відбувається
в чистій формі, — обмін еквівалентів. Тому навіть вульгарна
політична економія, хоч і як мало вона тямить, що таке
вартість, кожного разу, коли вона на свій штиб хоче розглянути
явище в його чистій формі, припускає, що попит і подання урівноважуються,
тобто, що вплив їхній взагалі припиняється. Отже,
коли щодо споживної вартости обидва контраґенти можуть виграти,
то на міновій вартості вони не можуть обидва виграти. Навпаки,

deux contractants gagnent — toujours (!)»). (Destutt de Tracy: «Traité de
la Volonté et de ses effets», Paris 1826, p. 68). Ta сама книга появилася пізніше
під назвою «Traité d’Economie Politique».