ваних у товарах, на зреалізовану у благородних металях працю
країн, що продукують золото й срібло. З другого боку, золото
й срібло безупинно пливуть туди й сюди поміж різними національними
сферами циркуляції, — рух, що йде слідом за безнастанними
коливаннями векселевого курсу.\footnote{
«Векселеві курси підносяться й падають щотижня, в певні моменти
року вони особливо сприятливі для однієї нації, в інші моменти
остільки ж сприятливі для іншої нації» («Exchanges rise and fall every
week, and at some particular times in the year run high against a nation,
and at other times run as high on the contrary»). (N. Barbon:» A Discourseconcerning
coining the new money lighter», London 1696, p. 39).
}

Країни з розвиненою буржуазною продукцією обмежують
скарби, що масами сконцентровані по банкових резервуарах,
потрібним для їхніх специфічних функцій мінімумом.\footnote{
Ці різні функції можуть дійти небезпечного конфлікту, скоро
тільки до них долучиться функція розмінного фонду для банкнот.
} За
деяким винятком, надзвичайне переповнення скарбових резервуарів
понад їхній пересічний рівень свідчить про застій товарової
циркуляції або про перерив течії товарової метаморфози.\footnote{
«Кількість грошей, що перевищує абсолютно потрібну кількість
для внутрішньої торговлі, є мертвий капітал і не приносить країні, що
ними володіє, жодного зиску, хіба тільки тоді, коли їх ввозиться
або вивозиться через торговельні операції» («What money is more than
of absolute necessity for a Home Trade, is dead stock, and brings no profit
to that country it’s kept in, but as it is transported in Trade, as well as
imported»). (John Bellers: «Essays about the Poor», London 1669, p. 12).
«Що маємо робити, коли в нас є забагато монет? Ми можемо тоді найваговитіші
з них знов перетопити й перетворити в люксусові полумиски,
посуд і начиння з золота й срібла; або надіслати їх як товари туди, де їх
потребують і є на них попит; або позичити їх на проценти туди, де рівень
проценту високий» («What if we have too much coin? We may melt down
the heaviest and turn it into the splendour of plate, vessels or ustensils of
gold and silver; or send it out, as a commodity, where the same is wanted
or desired; or let it out at interest, where interest is high»). (W. Petty:
«Quantulumcunque concerning Money», 1682, p. 7). «Гроші — це не
більш, як жир політичного тіла; тому надлишок їх часто заважає його
рухливості, а недостача наводить на нього недугу; як жир мастить мускули
і полегшує їх рух, відживляє за недостачі їжі, заповнює порожнечу й
прикрашає тіло, так само і гроші прискорюють діяльність державного
тіла, відживляють чужоземним продуктом підчас неврожаю в себе дома,
вирівнюють рахунки... і оздоблюють ціле. А проте, — іронічно закінчує
автор, — останнє стосується переважно до тих осіб, що мають забагато
грошей». («Money is but the fat of the Body Politick, whereof too much
does as often hinder its agility, as too little makes it sick... as fat lubricates
the motion of the muscles, feeds in want of victuals, fills up uneven cavities,
and beautifies the body; so doth money in the state quicken its actions,
feeds from abroad in time of dearth at home; evens accounts... and beautifies
the whole; although», ironisch abschlissend, «more especially the particular
persons that have it in plenty»). (W. Petty: «Political Anatomy
of Ireland, 1672». Ed. London, p. 14).
}

постачають золото й срібло, дають їх досить для того, щоб дати таку
доконечну кількість кожній нації» («The mines which are continually giving
gold and silver, do give sufficient to supply such a needful balance to every
nation». (I. Vanderlint: «Money answers all Things», London 1734, p. 40).