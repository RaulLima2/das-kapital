Передмова до першого англійського

видання

Опублікування англійського перекладу «Капіталу» не потребує
жодного обґрунтування. Навпаки, можна було б чекати деяких
пояснень, чому це англійське видання запізнилось до цього часу,
коли відомо, що про теорії, викладені в цій книзі, уже протягом
кількох років завжди згадується в періодичній пресі і поточній
літературі так в Англії, як і в Америці, проти них повстають, їх
боронять, їх пояснюють і хибно пояснюють.

Коли скоро по смерті автора в 1883 році з’ясувалося, що англійське
видання його праці дійсно конче потрібне, то містер Самюел
Мур, старий приятель Маркса та автора цих рядків і може
як ніхто інший обізнаний з самою книгою, заявив про свою згоду
взяти на себе переклад, що його дати публіці жадали виконавці
заповіту Марксового. Умовились, що я маю порівняти рукопис з
ориґіналом і запропоную ті зміни, які визнаю за доцільні. Коли
дедалі більше виявлялось, що професійні обов’язки заважали
містерові Мурові закінчити переклад так швидко, як ми того всі
бажали, ми радо згодились на пропозицію д-ра Евелінґа взяти
на себе частину роботи; одночасно містрес Евелінґ, наймолодша
дочка Марксова, запропонувала взяти на себе перевірку цитат і
відновити ориґінальний текст численних місць, взятих з англійських
авторів та з Синіх Книг і перекладених Марксом на німецьку
мову. Це так усюди й зроблено, крім деяких неминучих винятків.

Ось які частини книги переклав д-р Евелінґ:

1) Розділ X (Робочий день) і XI (Норма і маса додаткової
вартости); 2) шостий відділ (Заробітна плата, що охоплює розділи
XIX—XXII); 3) з розділу ХХІV, частину 4 («Обставини, що»
і т. д.) і до кінця книги, що охоплює останню частину розділу
XXIV, розділ XXV і ввесь сьомий відділ (розділи від XXVI
доХХХІІІ); 4) дві передмови автора. Всю решту книги дав містер
Мур. Тимчасом як кожен з перекладачів відповідає за свою
частину, я відповідаю за всю книгу.

Третє німецьке видання, виключно покладене в основу нашої
роботи, я підготував 1883 року за допомогою зауважень, що їх
залишив автор, подаючи ті місця другого видання, що мали бути
замінені відзначеними місцями французького тексту, опублікованого
в 1873 р.* Зміни, що таким чином постали в тексті другого

* «Le Capital», par Karl Marx. Traduction de M. I. Roy, entièrement
revisée par l’auteur. Цей переклад має, особливо в останній частині книги,
чимало змін у тексті і доповнень до тексту другого німецького видання.
