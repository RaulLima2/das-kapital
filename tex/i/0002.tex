мір для кількісного виміру корисних речей. Ріжниця в товарових
мірах походить почасти з неоднакової природи вимірюваних
предметів, почасти — з угоди.

Корисність якоїсь речі робить її споживною вартістю.\footnote{
«Природна вартість будь-якої речі є в її здатності задовольняти
доконечні потреби або служити вигодам людського життя» («The natural
worth of anything consists in its fitness to supply the necessities, or serve
the conveniences of human life»). (John Locke: «Some Considerations
on the Consequences of the Lowering of Interest», 1691; в «Works» edit.
London 1777, vol. II, p. 28). У XVII віці ми ще часто подибуємо в англійських
письменників «worth» на означення споживної вартости й
«value» на означення мінової вартости; це цілком у дусі мови, яка воліє
безпосередні речі означати словами германського, а рефлективні — словами
романського походження.
}  Але
ця корисність не висить у повітрі. Зумовлена властивостями товарового
тіла, вона не існує без цього останнього. Тому саме товарове
тіло, як от залізо, пшениця, діямант тощо, є споживна
вартість, або добро. Цей його характер не залежить од того, чи
присвоєння його споживних властивостей коштує людині багато,
чи мало праці. При розгляді споживних вартостей певну кількість
їх завжди приймається за наперед визначену, приміром,
тузінь годинників, метр полотна, тонна заліза тощо. Споживні
вартості товарів дають матеріял для осібної науки — товарознавства.\footnote{
У буржуазнім суспільстві панує юридична фікція, ніби кожна людина,
як покупець товарів, має енциклопедичні знання з товарознавства.
}
Споживна вартість реалізується лише в ужитку або споживанні.
Споживні вартості становлять речовий зміст багатства, хоч
яка була б його суспільна форма. В тій суспільній формі, яку ми
маємо розглянути, вони є одночасно речові носії мінової вартости.

Мінова вартість з’являється насамперед як кількісне відношення,
пропорція, що в ній споживні вартості одного роду обмінюються
на споживні вартості іншого роду,\footnote{
«Вартість — це мінове відношення, що є між однією річчю та іншою,
між певною кількістю одного продукту й певною кількістю іншого»
(«La valeur consiste dans le rapport d’échange qui se trouve entre telle
chose et telle autre, entre telle mesure d’une production et telle mesure
d’une autre»). (Le Trosne: «De l’Intérêt Social». Physiocrates, éd. Daire.
Paris. 1846, p. 889).
} відношення, яке
завжди зміняється залежно від часу й місця. Тому мінова вартість
здається чимось випадковим і суто відносним; отже, внутрішня,
іманентна товарові мінова вартість (valeur intrinsèque) здається
якоюсь contradictio in adjecto.*7 Розгляньмо справу ближче.

Якийсь товар, приміром, квартер пшениці, обмінюється на x
вакси до чобіт, або на y шовку, або на z золота і т. ін., коротше,
обмінюється на інші товари в найрізніших пропорціях. Отже,

7 «Ніщо не може мати внутрішньої вартости» («Nothing can have
anin trinsick value»). (N. Barbon: «A Discourse concerning coining the new
money lighter», p. 6), або, як каже Butler:

«The value of a thing

Is just as much as it will bring».

(Річ варта саме стільки, скільки вона приносить).

* — абсурдною суперечністю. Ред.
