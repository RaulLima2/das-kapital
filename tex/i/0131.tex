ника його діяльности на цей предмет. Він користується механічними,
фізичними, хемічними властивостями речей для того, щоб
примусити їх діяти як сили на інші речі, відповідно до своєї мети.2
Предмет, що його робітник опановує безпосередньо, — залишаючи
осторонь захоплювання готових засобів існування, приміром,
овочів, коли лише його власні органи служать за засоби праці, —
є не предмет праці, а засіб праці. Таким чином, даний самою
природою предмет стає органом його діяльности, органом, що
його він долучає до органів свого власного тіла, здовжуючи,
всупереч біблії, природний розмір своєї статури. Земля, являючи
собою його первісну харчову комору, є так само й первісний арсенал
його засобів праці. Вона постачає йому, приміром, камінь,
яким він кидає, тре, тисне, ріже й т. ін. Сама земля є засіб праці,
але функціонування її як засобу праці в рільництві знов же
має за передумову цілий ряд інших засобів праці й порівняно
високий уже розвиток робочої сили.3 Скоро тільки взагалі процес
праці сяк-так розвинеться, то він потребує оброблених уже засобів
праці. В печерах найдавнішої людини ми находимо кам’яне
знаряддя й кам’яну зброю. Поруч із обробленим каменем, деревом,
кістьми й мушлями головну ролю як засіб праці на початках
людської історії відіграє приручена, отже, змінена вже працею,
виплекана тварина.4 Вжиток і створення засобів праці,
хоч у зародковій формі вони вже властиві і деяким породам тварин,
характеризують специфічно людський процес праці. Тим
то Франклін визначає людину, як «a toolmaking animal», як
тварину, що продукує знаряддя праці. Останки засобів праці
мають для вивчання загинулих економічних суспільних формацій
таку саму вагу, яку будова останків костей для пізнавання організації
загинулих тваринних порід. Економічні епохи відрізняються
не тим, що продукують, а тим, як продукують, якими
засобами праці.5 Засоби праці є не лише мірило розвитку людської
робочої сили, але й покажчик суспільних відносин, за яких
люди працюють. Серед самих засобів праці механічні засоби праці,
що їх сукупність можна назвати кістковою і мускульною системою
продукції, подають значно відмінніші характеристичні ознаки
певної епохи суспільної продукції, ніж такі засоби праці, що

2 «Розум є так само хитрий, як і могутній. Хитрість полягає взагалі
в упосереднювальній діяльності, яка, примушуючи об’єкти, відповідно
до їхньої власної природи, діяти один на одного та впливати один на
одного, не втручається безпосередньо до цього пронесу і все ж досягає
здійснення лише своєї мети». (Hegel: «Enzyklopädie. Erster Teil. Die
Logik», Berlin 1840», S. 382).

3 У своїй, зрештою нікчемній, праці «Théorie de l’Economie Politique»,
Paris 1815, Ґаніль влучно перелічує всупереч до фізіократів величезний
ряд процесів праці, які становлять передумову рільництва у власному
значенні слова.

4 У «Réflexions sur la Formation et la Distribution des Richesses»
(1766), Oeuvres, éd. Daire, vol. І, Тюрґо добре розвинув вагу прирученої
тварини для початків культури.

5 З усіх товарів власне люксусові товари мають найменше значення
для технологічного порівняння різних епох продукції.
