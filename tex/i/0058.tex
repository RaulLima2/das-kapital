В наслідок цих історичних процесів, відокремлення назви
грошей від їхніх звичаєвих вагових назов стає народньою звичкою.
А що грошовий маштаб, з одного боку, є цілком умовний,
а з другого боку, потребує загального визнання, то його, кінець-кінцем,
реґулює закон. Певну вагову частину благородного
металю, наприклад, одну унцію золота, офіціально поділяють
на аліквотні частини, що в законному хрищенні дістають
такі назви як фунт, таляр і т. д. Кожну таку аліквотну частину,
що її тоді вважається за справжню одиницю грошової міри, підподіляється
на інші аліквотні частини, що в законному хрищенні
дістають такі назви як шилінґ, пенні і т. д.59 Певні кількості ваги
металю і далі, як і раніш, лишаються маштабом металевих грошей.
Що змінилося, так це тільки поділ на частини й надання назов.

Отже, ціни або кількості золота, на які ідеально перетворено
вартості товарів, виражаються тепер у грошових назвах, або в
законно визнаних рахункових назвах маштабу золота. Таким
чином замість казати: квартер пшениці дорівнює одній унції
золота, в Англії сказали б: він дорівнює 3 фунтам стерлінґів
17 шилінґам 10 1/2 пенсам. Таким чином товари у своїх грошових
назвах виражають, чого вони варті, а гроші функціонують як
рахункові гроші кожного разу, як тільки треба фіксувати якусь
річ як вартість, а тому і в грошовій формі.60

Назва якоїсь речі є щось цілком зовнішнє супроти її природи.
Я не знаю нічого про людину, знаючи лише, що вона зветься
Яків. Так само в грошових назвах: фунт, таляр, франк, дукат
і т. ін. зникає всякий слід вартостевого відношення. Плутанина
щодо таємного значення цих кабалістичних знаків є то значніша,
що грошові назви виражають одночасно і вартість товарів,
і аліквотні частини ваги металю, грошового маштабу.61 З другого

monete, le quali oggi sono ideali, sono le ріù antiche d’ogni nazione, e tutte
furono un tempo reali, e perchè eiano reali con esse si contava»). (Galiani:
«Della Moneta», p. 153).

59 Примітка до другого видання. Пан Давід Уркварт у своїй «Familiar
Words» зауважує про той страхітний (!) факт, що нині фунт, одиниця
англійського маштабу грошей, дорівнює приблизно 1/4 унції золота, таке:
«Це фальсифікація міри, а не встановлення маштабу» («This is falsifying
a measure, not establishing a standard»). В цьому «фалшивому
найменуванні» ваги золота він находить, як і скрізь, лише фальсифікаторську
руку цивілізації.

60    Примітка до другого видання. «Коли запитали Анахарсіса, на що
еллінам гроші, він відповів: на те, щоб рахувати». (Athenaeus: «Deipnosophistai»,
1. IV, 49 v. 2 ed. Schweighäuser, 1802).

61    Примітка до другого видання. «Через те, що гроші, як маштаб цін,
мають ту саму рахункову назву, що й ціни товарів, — отже, приміром,
одна унція золота так само, як і вартість тонни заліза, виражається в
З фунтах стерлінґів 17 шилінґах 10 1/2 пенсах, — то цим їхнім рахунковим
назвам дали назву ціни монети. Звідси постало дивовижне уявлення, що
начебто золото (або срібло) цінується у його власному матеріялі і, відмінно
від усіх інших товарів, дістає від держави фіксовану ціну. Фіксацію
рахункових назов певних вагових кількостей золота сплутано з
фіксацією вартостей цих кількостей ваги». (К. Marx: «Zur Kritik der
Politischen Oekonomie», S. 52. — K. Маркс: «До критики політичної
економії», ДВУ, 1926 р., стор. 89).
