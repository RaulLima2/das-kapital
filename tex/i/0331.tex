Додаючи переважну кількість дітей та жінок до комбінованого
робочого персоналу, машина, кінець-кінцем, ламає опір, що
його чоловік-робітник у мануфактурі ставив ще деспотизмові
капіталу.\footnote{
«Пан E., фабрикант, повідомив мене, що коло своїх механічних
ткацьких варстатів він вживає виключно жіночої праці; він дає перевагу
заміжнім жінкам, особливо жінкам, що мають дома родину, утримання
якої залежить від них; вони куди уважніші та слухняніші, ніж незаміжні,
та мусять до крайности напружувати свої сили, щоб добувати собі доконечні
засоби існування. Таким чином чесноти, властиві жіночому характерові,
повертаються їм на шкоду, — таким чином усе моральне й ніжне
їхньої природи стає засобом їхнього поневолення та джерелом їхнього
страждання». («Ten Hours Factory Bill. The Speech of Lord Ashley, 15 th
March», London 1844, p. 20).
}

b) Здовження робочого дня

Якщо машина є якнаймогутніший засіб збільшувати продуктивну
силу праці, тобто скорочувати робочий час, потрібний для
продукції товару, то, як носій капіталу, стає вона насамперед
у безпосередньо охоплених нею галузях промисловости якнаймогутнішим
засобом здовжувати робочий день поза всяку природну
межу. Вона створює, з одного боку, нові умови, що дозволяють
капіталові давати повну волю цій своїй постійній тенденції,
з другого боку, вона створює нові мотиви до загострення його
ненажерливої жадоби чужої праці.

Насамперед рух та функціонування засобу праці в машині
усамостійнюється проти робітника. Засіб праці стає сам по собі
промисловим perpetuum mobile, яке продукувало б безнастанно,
коли б воно не натрапляло на певні природні межі у своїх помічниках
— людях: на слабощі їхнього тіла й на їхню сваволю.
Тому, як капітал, — а, як такий, автомат має в особі капіталіста
свою свідомість і волю, — засіб праці є надхнений прагненням
звести людські природні межі, що ставлять йому опір, але є елястичні,
до мінімуму опору.\footnote{
«З того часу, як повсюди заведено коштовні машини, людину
примусили працювати далеко більше, ніж їй пересічно під силу» («Sіnce
the general introduction of expensive machinery, human nature has been
forced far beyond its average strength»). (Robert Owen: «Observations
on the effects of the manufacturing system», 2 nd ed. London 1817).
} Але й без того цей опір зменшується
через позірну легкість праці коло машини та більшу покірливість
і гнучкість жіночого й дитячого елементу».\footnote{
Англійці, які охоче розглядають першу емпіричну форму виявлення
речі, як її причину, часто вважають за причину довгого робочого
}

шого часу панує фабричний закон у власному значенні (не Print Work's
Act, що його ми щойно навели в тексті), перешкоди проти пунктів про
виховання за останні роки до певної міри переборено. А в тих галузях
промисловости, які не підведені під фабричний закон, ще й досі цілком
панують погляди фабриканта скла, Дж. Ґедса, який так навчав члена слідчої
комісії Вайта: «Оскільки я розумію, більша освіта, яку дістала останніми
роками певна частина робітничої кляси, є лихо. Вона небезпечна,
бо робить робітників надто незалежними». («Children’s Employment Commission.
4 th Report», London 1865, p. 253).