Крім цієї нової сили, яка постає із злиття багатьох сил в одну
колективну силу, вже самий суспільний контакт при більшості
продуктивних праць викликає змагання та своєрідне зворушення
життьового духа (animal spirits), яке збільшує індивідуальну
дієздатність поодиноких осіб, так що дванадцятеро осіб
разом протягом того самого робочого дня в 144 години дадуть
далеко більший сукупний продукт, ніж дванадцять поодиноких
робітників, що з них кожен працюватиме 12 годин, або ніж один
робітник, який працюватиме день за днем 12 днів.12 Це випливає
з того, що людина з природи є якщо й не політична,\footnote{
Арістотелеве визначення, власне кажучи, говорить, що людина з природи є міський громадянин.
Для клясичної старовини це так само
характеристичне, як і визначення Франкліна, що людина з природи є
творець знарядь, характеристичне для доби янкі.
} як думає
Арістотель, то в усякому разі громадська тварина.

Хоч багато осіб одночасно й спільно виконують таку саму або
однорідну працю, все ж індивідуальна праця кожної особи, як
частина колективної праці, може репрезентувати різні фази
самого процесу праці, що через них предмет праці, в наслідок
кооперації, перебігає швидше. Приміром, коли мулярі складають
ряд рук, щоб подавати цеглу від основи риштовання до його
верху, то кожен з них робить те саме, а все ж поодинокі операції
становлять безперервні частини однієї спільної операції, окремі
фази, які кожна цеглина мусить перебігти в процесі праці, і
завдяки чому 24 руки колективного робітника подають цеглу
швидше, ніж дві руки поодинокого робітника, що сходить на
риштовання та спускається з нього.\footnote{
«Треба ще зауважити, що такий частинний поділ праці може бути
навіть тоді, коли робітники працюють коло тієї самої справи. Наприклад,
мулярі, які подають із рук до рук цеглу на високе риштовання, виконують
} Предмет праці перебігає

one man cannot, and 10 men must strain, to lift a tun of weight, yet
one hundred men can do it only by the strength of a finger of each of them»).
(John Bellers: «Proposals for raising a colledge of industry», London
1696, p. 21).

12 «Отже, в цьому (коли один фармер уживає на 300 акрах те саме
число робітників, яке 10 дрібних фармерів уживають кожен на 30 акрах),
тобто в такій пропорції робітників, є й така користь, яку не легко зрозуміти
людям, незнайомим із справою на практиці: справді, хто буде
заперечувати, що 1 відноситься до 4, як 3 відноситься до 12; однак на
практиці це не так: підчас жнив і інших спішних робіт справа йде ліпше
й успішніше, коли сполучити значне число рук разом; так, наприклад,
2 возії, 2 навантажники, 2 подавальники, 2 загрібальники й декілька
людей на скиртах або на току зроблять удвоє більше, ніж те саме число
робочих рук, поділених на різні групи по поодиноких фармах». («There
is also an advantage in the proportion of servants, which will not easily be
understood but by practical men; for it is natural to say, as 1 is to 4, so
are 3: 12; but this will not hold good in practice; for in harvest-time
and many other operations which require that kind of despatch, by throwing
many hands together, the work is better, and more expeditiously done:
for exemple, in harvest, 2 drivers, 2 loaders, 2 pitchers, 2 rakers, and the
rest at the rick, or in the barn, will despatch double the work, that the same
number of hands would do, if divided into different gangs, on different
farms»): («An Enquiry into the Connection between the present price of
provisions and the size of farms. By a Farmer», London 1773, p. 7, 8).