\parcont{}  %% абзац починається на попередній сторінці
\index{i}{0417}  %% посилання на сторінку оригінального видання
і других» (№ 116). «Більше проти тих, ніж проти других? — Як
мені відповісти на це питання?» (№ 137). «Чи виявляють підприємці
будь-яку охоту пристосувати години праці до навчання
в школі? — Ніколи» (№ 211). «Чи поповнюють копальневі
робітники своє виховання опісля? — Загалом вони стають гіршими,
набувають лихих звичок; вони вдаються в пияцтво і гру
й т. п. та цілком занепадають морально» (№ 109). «Чому не посилають
дітей до вечірніх шкіл? — Здебільша в копальневих округах
немає ніяких шкіл. Але головне в тому, що діти такі виснажені
надмірною довгою працею, що в них заплющуються очі від утоми».
«Отже, — робить висновок буржуа, — ви проти освіти? — Ні в
якому разі, але й т. ін.» (№443). «Чи не примушені власники
копалень і т. ін. законом 1860 р. вимагати шкільних посвідок,
якщо вони вживають до роботи дітей між 10 і 12 роками життя? —
За законом — так, але підприємці цього не роблять» (№ 444).
«На вашу думку, цей пункт закону не завжди виконують? —
Його зовсім не виконують» (№ 717). «Чи дуже цікавляться копальневі
робітники питанням освіти? — Велика більшість» (№ 718).
«Чи турбуються вони про виконання цього закону? — Велика
більшість» (№ 720). «Чому ж вони не примушують виконувати
його? — Той або інший робітник і хотів би не допускати до роботи
підлітків без шкільних посвідок, але тоді його беруть на замітку
(a marked man)» (№ 721). «Хто? — Його підприємець» (№ 722).
«Однак, ви ж не думаєте, що підприємці будуть переслідувати
людину за те, що вона слухає закону? — Я думаю, вони будуть
це робити» (№ 723). «Чому робітники не відмовляються вживати
до праці таких підлітків? — Це не залишено їм на вибір» (№ 1634).
«Ви вимагаєте втручання парляменту? — Якщо має щось реального
статися для виховання дітей копальневих робітників, то
це треба зробити примусово, за допомогою парляментського закону»
(№ 1636). «Чи повинно це мати силу для дітей усіх робітників
Великобрітанії, чи тільки для дітей копальневих робітників
? — Я тут, щоб говорити від імени копальневих робітників»
(№ 1638). " «Чому відрізняти дітей копальневих робітників од
інших? — Бо вони є виняток із правила» (№ 1639). «З якого
погляду? —\footnote{
. Жіноча праця. Хоч робітниць, починаючи від 1824 р. вже
й не вживають до праці під землею, а все ж вони працюють над
} фізичного» (№ 1640). «Чому б виховання для них
мало більшу вартість, аніж для хлопців інших кляс? — Я не
кажу, що воно має для них більшу вартість, але через надмірну
працю в копальнях вони мають менше шансів діставати виховання
в денних і недільних школах» (№ 1644). «А чи не правда,
що такі питання не можна розглядати абсолютно?» (№ 1646).
«Чи досить шкіл в округах? — Ні» (№ 1647). «Коли б держава
вимагала, щоб усіх дітей посилати до школи, то де взяти шкіл
для всіх цих дітей? — Я гадаю, що коли обставини вимагатимуть
цього, то школи постануть сами собою. Велика більшість
не тільки дітей, але й дорослих копальневих робітників не вміє
ні писати, ні читати» (№№ 705, 726).
\parbreak{}  %% абзац продовжується на наступній сторінці
