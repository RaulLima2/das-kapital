Це відношення анітрохи не змінюється від того, що капіталіст
замість однієї купує 100 робочих сил або складає контракт замість
з одним із 100 незалежними один від одного робітниками. Він може
вживати 100 робітників, не кооперуючи їх. Тому капіталіст
оплачує вартість 100 самостійних робочих сил, але не оплачує
комбінованої робочої сили 100 робітників. Як незалежні особи,
робітники є окремі індивіди, що стають у відношення до того
самого капіталу, а не один до одного. Кооперація між ними починається
лише в процесі праці, алеж у процесі праці вони перестають
уже належати самим собі. Вступивши до цього процесу,
вони стають складовою частиною капіталу. Як кооперовані
особи, як члени одного активного організму, вони сами є лише
осібні способи існування капіталу. Тим-то продуктивна сила,
яку робітник розвиває як суспільний робітник, є продуктивна
сила капіталу. Суспільна продуктивна сила праці розвивається
безплатно, скоро тільки робітників поставлено в певні умови,
а капітал ставить їх у ці умови. А що суспільна продуктивна
сила праці нічого не коштує капіталові і що, з другого боку, робітник
не розвиває її раніше, ніж його праця належить капіталові,
то й видається вона продуктивною силою, яку капітал має з
природи, його іманентною продуктивною силою.

Колосальним виявляється ефект простої кооперації у велетенських
спорудах давніх азійців, єгиптян, етрусків і т. ін.
«За минулих часів траплялося, що ці азійські держави після
покриття своїх цивільних та військових видатків мали ще якийсь
лишок засобів існування, що його вони могли витратити на розкішні
й корисні твори. Їхнє панування над робочими силами майже
всієї рільничої людности та виключне право монарха і духівництва
порядкувати тим лишком давали їм засоби будувати ті
могутні монументи, якими вони заповнили країну... При переміщенні
тих велетенських статуй і тих величезних мас, що їх
транспорт викликає подив, марнотратно вживали майже виключно
людської праці. Для цього досить було певного числа робітників
та концентрації їхніх зусиль. Так ми бачимо, як із глибин океану,
з коралевих скель виростають острови, творячи суходіл, дарма
що кожний індивід, що бере участь у творенні їх (depositary),
є дрібний, слабий та нікчемний. Нерільничі робітники азійської
монархії мало що повинні були додати до тієї справи, крім
своїх індивідуальних фізичних зусиль; та їхнє число — це їхня
сила, а влада керування цими масами дала початок тим велетенським
спорудам. Саме концентрація в руках однієї або небагатьох
осіб тих доходів, що з них жили робітники, робила можливими
такі підприємства».\footnote{
R. Jones: «Textbook of Lectures etc.», Hertford 1852, p. 77, 78.
Давні асирійські, єгипетські та інші колекції в Лондоні та по інших
европейських столицях роблять нас свідками тих кооперативних процесів
праці.
} Ця влада азійських та єгипетських царів
або етруських теократів тощо в сучасному суспільстві перейшла
до капіталіста, однаково, чи виступає він як поодинокий капіта-