\parcont{}  %% абзац починається на попередній сторінці
\index{i}{0408}  %% посилання на сторінку оригінального видання
посвячений. Велика промисловість розірвала вуаль, який ховав
од людей їхній власний суспільний процес продукції й робив
різні стихійно повідокремлювані галузі продукції загадками
одну для однієї і навіть для посвячених у кожну з них. Принцип
цієї промисловости — розкладати всякий процес продукції, взятий
сам по собі і насамперед незалежно від виконування його
рукою людини, на його складові елементи — цей принцип створив
цілком новітню науку технології. Строкаті, на позір позбавлені
зв’язку й закостенілі форми суспільного процесу продукції,
розпалися на свідомо пляномірні й, відповідно до бажаного корисного
ефекту, систематично повідокремлювані застосування
природознавства. Технологія відкрила так само ті нечисленні
великі основні форми руху, в яких неодмінно, не вважаючи на
всю різноманітність уживаних інструментів, відбувається вся
продуктивна діяльність людського організму, цілком так само,
як у механіці найскладніший характер машин не ховає того,
що машини є постійне повторювання простих механічних знарядь.
Сучасна промисловість ніколи не розглядає і не трактує наявну
форму якогось процесу продукції як остаточну. Тому її технічна
база є революційна, тимчасом як технічна база всіх попередніх
способів продукції суттю своєю була консервативна\footnote{
«Буржуазія не може існувати, не революціонізуючи постійно
знарядь продукції, отже, і продукційних відносин, отже, і всіх суспільних
відносин. Навпаки, незмінне збереження старого способу продукції було
першою умовою існування всіх попередніх промислових кляс. Постійний
переворот у продукції, невпинне стрясіння всіх суспільних станів, вічна
непевність і рух відзначають буржуазну епоху від усіх інших. Всі тривалі,
заіржавілі відносини з їхнім кортежем традиційно-поважаних ідей і
поглядів гинуть, усі новоутворені старіють раніше, ніж вони встигають
скостеніти. Все стале й непорушне випаровує, все святе втрачає святість,
і люди, нарешті, мусять тверезими очима подивитись на своє життєве
становище, на свої взаємні відносини». (\emph{F.~Engels і К.~Marx}: «Manifest
der Kommunistischen Partei», London 1848, S. 5. — \emph{K.~Маркс
і Ф.~Енґельс}: «Маніфест Комуністичної Партії», Партвидав 1932, стор. 30).
}. Машинами, хемічними процесами й іншими методами вона постійно
робить перевороти в технічній основі продукції і разом з тим
у функціях робітників і в суспільних комбінаціях робочого процесу.
Цим вона так само постійно революціонізує поділ праці
всередині суспільства й безупинно кидає маси капіталу й маси
робітників з однієї галузі продукції до іншої. Через це природа
великої промисловости зумовлює переміну праці, рух функцій,
всебічну рухливість робітника. З другого боку, вона репродукує
у своїй капіталістичній формі старий поділ праці з його закостенілими
спеціяльностями. Ми бачили, як ця абсолютна суперечність
[між технічними потребами великої промисловости і соціяльним
характером, що його вона набирає за капіталістичного ладу]\footnote*{
Заведене у прямі дужки ми беремо з французького видання. \emph{Ред.}
}
знищує увесь спокій, сталість, певність життєвого становища
робітника, постійно загрожуючи йому вибити з його рук разом
\parbreak{}  %% абзац продовжується на наступній сторінці
