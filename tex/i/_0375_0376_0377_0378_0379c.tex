\parcont{}  %% абзац починається на попередній сторінці
\index{i}{0375}  %% посилання на сторінку оригінального видання
бо капітал у 2.000 фунтів стерлінґів за старого способу виробництва
вживав би не 400, а 1.200 робітників. Отже, відносне зменшення
числа вживаних робітників узгоджується з його абсолютним
збільшенням. Вище ми припускали, що із зростом цілого
капіталу склад його лишається незмінний, бо й умови продукції
не змінюються. Але ми вже знаємо, що з кожним кроком розвитку
машинової системи стала частина капіталу, яка складається з
машин, сировинного матеріялу тощо, зростає, тим часом як
змінна частина капіталу, витрачена на робочу силу, падає; разом
з тим знаємо, що ні за якого іншого способу продукції не буває
такого постійного поліпшення машин, а тому й таких змін у
складі цілого капіталу. Але цю постійну зміну так само постійно
переривають періоди спокою та просте кількісне поширення на
даній технічній основі. Тому число вживаних робітників зростає.
Так, число робітників на бавовняних, вовняних, суканої вовни,
лляних та шовкових фабриках Об’єднаного Королівства 1835 р.
становило лише 354.684, тимчасом як 1861 р. число самих ткачів
(обох статей та найрізнішого віку, від 8 років починаючи) при
парових варстатах становило 230.654. Певна річ, цей зріст буде
менш значним, коли взяти на увагу, що ще 1838 р. брітанських
ручних ткачів бавовни разом з їхніми родинами, яким вони сами
давали заняття, налічувалося 800.000 чоловіка,\footnote{
«Страждання ручних ткачів (бавовни і тканини з домішкою
бавовни) були предметом дослідження королівської комісії, але хоч їхні
злидні були визнані й оплакані, все ж поліпшення (!) їхнього становища
віддали на волю випадкові та часові, і можна сподіватися, що ці страждання
тепер (через 20 років!) майже (nearly) зникли, чому, певно, допомогло
сучасне велике поширення парових ткацьких варстатів». («Reports
of lnsp. of Fact, for 31 st October 1856», p. 15).
} не кажучи вже
зовсім про тих ручних ткачів, що їх витиснуто в Азії та на континенті
Европи.

В тих небагатьох увагах, що треба ще зробити про цей пункт,
ми почасти торкаємося суто фактичних відносин, до яких самий
наш теоретичний виклад ще не довів нас.

Поки машинове виробництво поширюється в якійсь галузі промисловосте
коштом традиційного ремества або мануфактури, його
успіхи настільки ж певні, як, наприклад, успіх армії, озброєної
ґвинтівками, проти армії, озброєної луками. Цей перший період,
коли машина ще тільки завойовує собі сферу діяння, має вирішальне
значення через ті надзвичайно високі зиски, що їх вона
допомагає продукувати. Ці зиски не тільки сами по собі становлять
джерело прискореної акумуляції, але вони ще й притягають у
цю сприятливу сферу продукції велику частину суспільного додаткового
капіталу, що постійно знов утворюється й шукає нового
вміщення. Особливі користі того першого періоду бурі й натиску
постійно повторюються по тих галузях продукції, де
машини заводиться вперше. Але, скоро тільки фабрична система
досягає певної ширини існування та певного ступеня зрілости,
скоро тільки, особливо, її власна технічна основа, машини, продукуються
\index{i}{0376}  %% посилання на сторінку оригінального видання
знову ж таки за допомогою машин; скоро тільки відбувається
революція в добуванні вугілля й заліза, а також в обробленні
металів і в транспортовій справі; одне слово, скоро тільки
будуть створені загальні умови продукції, відповідні великій
індустрії, — з цього моменту цей спосіб виробництва набуває елястичности,
здатности до раптового стрибкуватого поширювання, що
не має інших меж, як тільки в сировинному матеріялі та в ринку
для збуту. Машини, з одного боку, безпосередньо сприяють збільшенню
сировинного матеріялу, як, наприклад, cotton gin збільшила
продукцію бавовни.\footnote{
Про інші методи, якими машини впливають на продукцію сировинного
матеріялу, буде згадано в третій книзі.
} З другого боку, дешевина машинового
продукту та переворот у засобах транспорту й комунікації є
знаряддя завоювати чужі ринки. Руйнуючи на цих ринках ремісничу
продукцію, машинове виробництво силоміць перетворює
їх на поля продукції свого сировинного матеріялу. Так, Східня
Індія була примушена продукувати для Великобрітанії бавовну,
вовну, коноплю, джут, індиґо тощо.\footnote{
\noindent{}Вивіз бавовни із Східньої Індії до Великобрітанії:

\noindent{}1846 р. — 34.540.143 фунти, 1860 р. — 204.141.168 фунтів, 1865 р. —
445.947.600 фунтів.

\noindent{}Вивіз вовни із Східньої Індії до Великобрітанії:

\noindent{}1846 р. — 4.570.581 фунт, 1860 р. — 20.214.173 фунти, 1865 р. —
20.679.111 фунтів.
} Постійне «перетворювання»
робітників у «зайвих» по країнах великої промисловости примушує
до штучної еміґрації й колонізації чужих країн, які перетворюються
на місця продукції сировинного матеріялу для метрополії,
як, наприклад, Австралія перетворилася в місце продукції
вовни.\footnote{
\noindent{}Вивіз вовни з рогу Доброї Надії до Великобрітанії:

\noindent{}1846 р. — 2.958.457 фунтів, 1860 р. — 16.574.345 фунтів, 1865 р. —
29.220.623 фунти.

\noindent{}Вивіз вовни з Австралії до Великобрітанії:

\noindent{}1846 р. — 21.789.346 фунтів, 1860 р. — 59.166.616 фунтів, 1865 р. —
109.734.261 фунт.
} Створюється новий, відповідний до розташування головних
центрів машинового виробництва, міжнародній поділ праці,
який перетворює одну частину земної кулі переважно на поле
рільничої продукції [для другої частини земної кулі, яка стає
переважно полем промислової продукції].\footnote*{
Заведене у прямі дужки ми беремо з другого німецького видання.
\emph{Ред.}
} Ця революція стоїть
у зв’язку з переворотами в рільництві, що їх ми тут ще не розглядаємо
докладніше.\footnote{
Самий економічний розвиток Сполучених штатів є продукт европейської,
особливо англійської, великої промисловости. Сполучені штати
в їхньому теперішньому вигляді (1866 р.) все ще треба розглядати як
колонію Европи. [До четвертого видання. Від того часу вони
розвинулись у другу промислову країну світу, не втративши при цьому цілком;
свого колоніального характеру. — \emph{Ф.~Е.}].

\begin{center}
    \captionnew{Вивіз бавовни із Сполучених штатів до Великобрітанії (в фунтах):}

    \begin{tabular}{lrlr}
    1846 р. \dotfill{} & 401.949.393 & 1852 р.\dotfill{} &  765.630.544 \\
    1859 р. \dotfill{} & 961.707.264 & 1860 р.\dotfill{} & 1.115.890.608 \\
    \end{tabular}
\end{center}


\begin{center}
    \captionnew{Вивіз збіжжя тощо із Сполучених штатів до Великобрітанії \\ (1850--1862 рр.) (в центнерах):}

    \begin{tabular}{lrr}
     & \makecell{1850 р.} & \makecell{1862 р.} \\
     \addlinespace
     Пшениця\dotfill  & 16.202.312 & 41.033.503 \\
    Ячмінь\dotfill & 3.669.653 &   6.624.800 \\
    Овес\dotfill & 3.174.801  &  4.426.994\\
    Жито \dotfill & 388.749 & 7.108\\
    Пшеничне борошно\dotfill & 3.819.440 & 7.207.113\\
    Гречка\dotfill & 1.054 & 19.571\\
    Кукурудза\dotfill & 5.473.161 & 11.694.818\\
    Веrе або Bigg (особливий рід ячменю)\dotfill & 2.039 & 7.675\\
    Горох\dotfill & 811.620 & 1.024.722\\
    Квасоля\dotfill & 1.822 972 & 2.037.37\\
    \cmidrule{2-3}
    Увесь довіз\dotfill & 34.365.801 & 74.083.351\\
    \end{tabular}
\end{center}
}

З ініціятиви пана Ґледстона Палата громад 17 лютого 1867 р.
наказала зібрати статистичні відомості про вивіз та довіз в
Об’єднане Королівство всякого роду збіжжя й борошна за час
від 1831 до 1866 р. Нижче я подаю зведення цих статистичних
\index{i}{0377}  %% посилання на сторінку оригінального видання
відомостей. Борошно перечислено на квартери збіжжя (Див.
таблицю на стор. \pageref{original-378}).

\begin{sidewaystable}
  \index{i}{0378}  %% посилання на сторінку оригінального видання
  \label{original-378}
  \centering
  \small
  \caption*{П'ятирічні періоди й 1866 рік}
  \begin{tabularx}{\textheight}{Xrrrrrrrr}
    \toprule
    
     & \makecell{1831\textendash{}1835} & \makecell{1836\textendash{}1840} & \makecell{1841\textendash{}1845}
     & \makecell{1846\textendash{}1850} & \makecell{1851\textendash{}1855} & \makecell{1856\textendash{}1860} 
     & \makecell{1861\textendash{}1865} & \makecell{1866} \\
    
    \midrule

    \addlinespace
    \makecell{Пересічно за рік} \\
    Імпорт (квартери)\dotfill{} &  1.096.373 & 2.389.729 & 2.843.865 & 8.776.552  & 8.345.237 & 10.913.612 & 15.009.871 & 16.457.340 \\
    
    \addlinespace
    \makecell{Пересічно за рік} \\
    
    Експорт (квартери)\dotfill{}  &   225.363  &   251.770  &    139.056 &    155.461  &    307.491 &     341.150  &    302.754  &  216.218 \\
    
    \makehangcell{Перевага імпорту над експортом пересічно за рік\dotfill{}}
        & 871.010 & 2.137.959 & 2.704.809 & 8.621.091  & 8.037.746  & 10.572.462  & 14.707.117  & 16.241.122 \\
    
    \addlinespace
    \makecell{Людність:} \\

    \makehangcell{Пересічне число на рік у кожному періоді\dotfill{}}
        & 24.621.107 & 25.929.507 &
        27.262.569 & 27.797.598 & 27.572.923 & 28.391.544 & 29.381.460 & 29.935.404 \\

    \makehangcell{Пересічна кількість збіжжя тощо в квартерах, що її 
        споживає за рік один індивід, при рівному розподілі між людністю,
        із надлишку проти тубільної продукції\dotfill{}}
        & 0,036 & 0,082 & 0,099 & 0,310  & 0,291  & 0,372  & 0,543  & 0,543 \\
  \end{tabularx}
\end{sidewaystable}

Величезна, стрибкувата розширність фабричної системи та
її залежність від світового ринку неминуче породжують гарячкову
продукцію і наступне переповнення ринків, із звуженням
яких настає параліч. Життя промисловости перетворюється на
послідовний ряд періодів середнього пожвавлення, розцвіту,
перепродукції, кризи й застою. Непевність і непостійність, що
їх зазнає праця і разом з нею й доля робітника через машинове
виробництво, стають нормальними з цією зміною періодів промислового
циклу. За винятком часів розцвіту, між капіталістами лютує
якнайзавзятіша боротьба за їхнє індивідуальне місце на ринку.
Це їхнє місце на ринку стоїть у прямому відношенні до дешевини
продукту. Крім створеного цим суперництва щодо вживання
поліпшених машин, які замінюють робочу силу, та нових метод
продукції, кожного разу настає такий момент, коли капіталісти
намагаються здешевити товари, силоміць понижуючи заробітну
плату нижче вартости робочої сили.\footnote{
У відозві робітників, викинутих на брук льокавтом фабрикантів
чобіт із Лестеру, до «Trade-Societies of England», липень 1866 p., між
іншим, сказано: «Ось уже років із 20 тому, як у чоботарстві Лестеру
відбувся переворот: замість зшивати почали скріпляти гвіздками. Тоді
можна було мати добру заробітну плату. Незабаром ця нова галузь промисловости
дуже поширилась. Велика конкуренція почалася між різними
фірмами, кожна з них намагалась подати найелегантніший товар. Алеж
незабаром виникла гірша конкуренція, а саме — фірми намагались побороти
одна одну на ринку нижчою ціною (undersell). Шкідливі наслідки виявилися
незабаром у пониженні заробітної плати, і ціна на працю спадала
так дуже швидко, що багато фірм платить тепер лише половину первісної
заробітної плати. А все ж, хоч заробітна плата падає нижче й нижче,
зиски з кожною зміною тарифу праці, здається, зростають». — Навіть
несприятливі періоди промисловости фабриканти використовують на те,
щоб через надмірне пониження заробітної плати, тобто безпосередньою
крадіжкою найдоконечніших засобів існування робітника, здобувати
надзвичайні зиски. Ось приклад. Мова йде про кризу шовкоткацтва в
Coventry. «Із свідчень, які я дістав так від фабрикантів, як і від робітників,
безперечно виходить, що заробітна плата понижена в більшому розмірі,
ніж цього вимагала конкуренція чужоземних продуцентів або інші обставини.
Більшість ткачів працює за заробітну плату, знижену на 30--40\%.
Моток стьожки, за який ткач перед п’ятьма роками діставав 6 або 7 шилінґів,
дає йому тепер лише 4 шилінґи 3 пенси або 3 шилінґи 6 пенсів;
за іншу працю, за яку раніш платили 4 шилінґи або 4 шилінґи 3 пенси,
він дістає тепер тільки 2 шилінґи або 2 шилінґи 3 пенси. Заробітну плату
понижено тепер більш, ніж це потрібно було для активізації попиту.
Справді, для багатьох сортів стьожок пониження заробітної плати не
супроводилося навіть якимось пониженням ціни товару». (Звіт комісара
F. D. Longe в «Children’s Employment Commission. 5 th Report 1866»,
p. 114, n. 1).
}


\index{i}{0379}  %% посилання на сторінку оригінального видання
Отже, зростання числа фабричних робітників зумовлено пропорційно
куди швидшим зростанням цілого капіталу, вкладеного
у фабрику. Але цей процес відбувається лише в межах періодів
припливу та відпливу промислового циклу. До того його завжди
перериває технічний прогрес, який то потенціяльно заступає робітника,
то витискує його фактично. Ця якісна зміна в машиновому
виробництві постійно викидає робітників із фабрики або замикав
фабричну браму перед новим потоком рекрутів, тимчасом як
просте кількісне поширення фабрик поглинає разом із викинутими
й свіжі контингенти. Таким чином робітників постійно відштовхують
або притягають, кидають ними туди й сюди, і це
супроводиться постійними змінами щодо статі, віку та вправности
завербованих.

Доля фабричного робітника унаочнюється найкраще, коли
подати короткий огляд долі англійської бавовняної промисловости.

Від 1770 до 1815 рр. бавовняна промисловість п’ять років перегнивала
період пригнічення або застою. Протягом цього першого
45-річного періоду англійські фабриканти мали монополію на
машини та світовий ринок. 1815--1821 рр. — час пригнічення;
1822--1823 рр. — розцвіт; 1824 р. — скасування закону про коаліції,
загальне велике поширення фабрик; 1825 р. — криза;
1826 р. — великі злидні та повстання серед бавовняних робітників;
1827 р. — легке поліпшення; 1828 р. — велике зростання
числа парових ткацьких варстатів і вивозу; 1829 р. — вивіз,
особливо до Індії, перевищує всі попередні роки; 1830 р. —
переповнені ринки, великі злидні; 1831--1833 рр. — тривале
пригнічення; у східньоіндійської компанії відібрано монополію
торговлі із Східньою Азією (Індією та Китаєм); 1834 р. — великий
зріст фабрик та машин, недостача рук; новий закон про бідних
активізує еміграцію сільських робітників до фабричних округ;
очищення сільських графств от дітей, торговля білими рабами.
1835 р. — великий розцвіт, але одночасно ручні бавовняні ткачі
\parbreak{}  %% абзац продовжується на наступній сторінці
