тять за товари понад їхню вартість, значить лише замасковувати
просту тезу: посідач товарів як продавець має привілей продавати
товари дорожче, ніж вони варті. Продавець сам випродукував
товар або заступає його продуцента, але й покупець не меншою
мірою сам випродукував товар, виражений у його грошах,
або заступає його продуцента. Отже, продуцент протистоїть
продуцентові. Вони відрізняються тим, що один купує, а другий
продає. Ми не посунемось ані на крок далі, якщо посідач товарів
під фірмою продуцента продає товар понад його вартість, а під
фірмою споживача платить за товари дорожче, ніж вони варті.28

Тим то послідовні оборонці ілюзії, що додаткова вартість
постає з номінального додатку до ціни або з привілею продавця
продавати товари занадто дорого, мусять припустити існування
кляси, яка лише купує, не продаючи, отже, і лише споживає,
не продукуючи. Існування такої кляси з того нашого погляду,
якого ми досі дійшли, з погляду простої циркуляції, ще не може
бути з’ясоване. Але забіжімо наперед. Гроші, за які постійно
купує така кляса, мусять постійно припливати до неї від самих
посідачів товарів, без обміну, задурно, на підставі хоч якогось
права або насильства. Продавати цій клясі товари понад вартість
— значить лише обманою повертати собі частину задурно
відданих грошей.29 Так малоазійські міста виплачували стародавньому
Римові щорічну грошову данину. За ці гроші Рим купував
у них товари й купував їх занадто дорого. Малоазійці ошукували
римлян, виманюючи у своїх завойовників через торговлю
частину данини. А проте ошуканими лишались малоазійці. За
їхні товари платили їм, як і раніш, їхніми ж власними грішми.
Це не є метода для збагачення або творення додаткової вартости.

Отже, тримаймося в межах товарового обміну, де продавці
є покупці, а покупці — продавці. Наші труднощі походять,
може, з того, що ми розглядали контраґентів лише як персоніфіковані
категорії, а не індивідуально.

Посідач товарів А може бути такий мудрагель, що завжди
зможе обдурити своїх колеґ В й С, тимчасом як ці при найкращому
бажанні не спроможуться на реванш. А продає В вино вартістю
в 40 фунтів стерлінґів і дістає на обмін збіжжя вартістю в
50 фунтів стерлінґів. А перетворив свої 40 фунтів стерлінґів
на 50 фунтів стерлінґів, з меншої кількости грошей зробив більшу
їх кількість і перетворив свій товар на капітал. Придивімось

28 «Думка, що зиски виплачують споживачі, є, безумовно, цілком
безглузда. Хто такі ці споживачі?» («The idea of profits being paid by
the consumers, is, assuredly, very absurd. Who are the consumers?»).
(G. Ramsay: «An Essay on the Distribution of Wealth», Edinburgh 1836, p.183).

29 «Коли комусь бракує попиту, чи порадить йому пан Малтуз позичити
якійсь особі гроші, щоб ця остання купила в нього товари?» — питає
один обурений рікардіянець у Малтуза, що, як і його учень, піп Чолмерс,
вихваляє з економічного погляду клясу виключно покупців або виключно
споживачів. Див.: «An Inquiry into those principles respecting the Nature
of Demand and the Necessity of Consumption, lately advocated by Mr. Malthus
etc.», London 1821, p. 55.
