виконувати свою ролю. Вони перестали б бути капіталом. Вилучені
з циркуляції, вони застигають у скарб, і жодний шаг не наростає
на них, хоч би вони й лежали аж до судного дня. Отже,
скоро йдеться про зростання вартости, то для 110 фунтів стерлінґів
є така сама потреба в зростанні вартости, як і для 100 фунтів
стерлінґів, бож обоє вони є обмежені вирази мінової вартости,
отже, обоє мають те саме призначення наближатися до багатства
взагалі через збільшення своєї величини. Правда, первісно авансована
вартість у 100 фунтів стерлінґів відрізняється на одну хвилину
від додаткової вартости в 10 фунтів стерлінґів, яка наростає
на ній у циркуляції, але ця ріжниця зразу ж геть розпливається
знову. Наприкінці процесу з’являється не первісна вартість
у 100 фунтів стерлінґів на одному боці і додаткова вартість у
10 фунтів стерлінґів на другому, а з’являється єдина вартість у
110 фунтів стерлінґів, яка має форму, так само придатну для того,
щоб почати процес зростання, як і первісні 100 фунтів стерлінґів.
Гроші з’являються наприкінці руху, щоб тільки знову почати
його.5 Тому кінець кожного окремого кругобігу, що в ньому
купівля відбувається задля продажу, вже сам собою становить
початок нового кругобігу. Проста товарова циркуляція — продаж
задля купівлі — служить за засіб досягти кінцевої мети,
що лежить поза межами циркуляції, а саме, присвоїти споживні
вартості, задовольнити потреби. Навпаки, циркуляція грошей
як капіталу є самоціль, бо зростання вартости існує лише в межах
цього раз-у-раз поновлюваного руху. Тому рух капіталу є
безмірний.6

5 «Капітал поділяється... на первісний капітал і бариш, приріст
капіталу... хоч у практиці цей самий бариш зразу ж додається знов до
капіталу й разом із ним пускається в рух». (F. Engels: «Umrisse zu
einer Kritik der Nationalökonomie» in «Deutsch-Französische Jahrbücher,
herausgegeben von Arnold Rüge und K. Marx». Paris 1884, S. 99).

6 Арістотель протиставить економіку хрематистиці. Він виходить
з економіки. Оскільки вона є вмілість надбання, вона обмежується на здобуванні
благ, доконечних для життя і потрібних для дому або для держави.
«Справжнє багатство (δ αληζινος πλουιος) складається з таких споживних
вартостей, бо кількість власности цього роду, достатньої для
доброго життя, не є безмежна. Але існує вмілість надбання іншого роду,
яка переважно і з повним правом називається хрематистикою, вмілість,
у наслідок якої, здається, не існує жодних меж багатства і власности.
Товарова торговля (η χαπηλιχη    дослівно значить торговля на роздріб,
і Арістотель бере цю форму, бо в ній переважає споживна вартість) з природи
не належить до хрематистики, бо тут обмін стосується лише до предметів,
що їм самим (покупцям і продавцям) потрібні». Тому, висновує
він далі, первісною формою товарової торговлі була мінова торговля,
але з її поширенням неминуче постали гроші. З винаходом грошей мінова
торговля неминуче мусила розвинутися в χαπηλιχη, в товарову торговлю,
а ця, всупереч до її первісної тенденції, перетворилась на хрематистику,
на вмілість робити гроші. Хрематистика ж відрізняється від економіки
тим, що «для неї циркуляція є джерело багатства (ποιητιχη χεηματωυ... δια χρηηατωυ σιαβολης). І
вона, здається, ґрунтується на грошах, бо гроші є початок і кінець цього роду обміну (το γχρ νομισμα
στοιχειον χαι περας ιης αλλαγης εστιν). Тому те багатство, до якого прагне хрематистика, є без-
