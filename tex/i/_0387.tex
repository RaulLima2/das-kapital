\parcont{}  %% абзац починається на попередній сторінці
\index{i}{0387}  %% посилання на сторінку оригінального видання
недорослих забезпечила різним лондонським друкарням газет
та книжок славетну назву «бойні».\footnoteA{
«Children’s Employment Commission. 5 th Report 1866», p. 3,
n. 24, p. 6, n. 55, 56, p. 7, n. 59, 60.
} Таку ж саму надмірну
працю, жертвами якої є головним чином жінки, дівчата й діти,
ми бачимо і в палітурнях. Тяжка також праця недорослих у
виробництві линв, нічна праця в солоницях, по свічкових та інших
хемічних мануфактурах; у шовкоткальнях, де не користуються
механічною силою, ми бачимо вбивчу працю дітей, уживаних на
те, щоб рухати ткацькі варстати.\footnote{
Там же, стор. 114, 115, n 6, 7. Комісар правильно зауважує,
що коли машина взагалі заступає людину, так тут підліток буквально
заступає машину.
} Одна з найогидніших, найбрудніших
та найгірш оплачуваних праць, до яких переважно
вживають молодих дівчат та жінок, — це сортування лахміття.
Відомо, що Великобританія, не згадуючи вже про величезну масу
її власного лахміття, є склад для торговлі лахміттям цілого світу.
Це лахміття привозять сюди з Японії, якнайвіддаленіших штатів
Південної Америки та з Канарських островів. Але головні джерела
цього довозу є Німеччина, Франція, Росія, Італія, Єгипет,
Туреччина, Бельгія та Голландія. Лахміття служить для
удобрення, фабрикації клоччя (на матраци до ліжок), shoddy
(штучної вовни) та як сировинний матеріял для паперу. Жінки-сортувальниці
лахміття є передатниці віспи та інших заразливих
недуг, що їх першими жертвами стають вони сами.\footnote{
Див. звіт про торговлю лахміттям і численні ілюстрації в «Public
Health. 8th Report», London 1866, Appendix, p. 196-208.
} За класичний
приклад надмірної праці, тяжкої та непідхожої праці, а
тому й сполученої з нею брутальности до робітників, що їх споживають
від наймолодшого віку, можуть бути, поряд рудень та
копалень, цегельні або майстерні для виробу черепиці, де недавно
винайдену машину застосовують в Англії ще лише спорадично
(1866 р.). Між травнем і вереснем праця триває там від 5 години
ранку до 8 години вечора, а там, де сушіння відбувається на
вільному повітрі, — часто від 4 години ранку до 9 години вечора.
Робочий день, що триває від 5 години ранку до 7 години вечора,
вважається за «скорочений», «помірний». Дітей обох статей
уживають, починаючи від 6, навіть від 4 років життя. Вони працюють
стільки ж годин, а часто й більше, ніж дорослі. Праця
тут тяжка, а літня спека ще збільшує виснаження. В одній цегельні
в Mosley, наприклад, одна дівчина 24 років виробляла
2.000 цеглин на день, їй помагало двоє малих дівчаток, які зносили
глину та складали цеглу до купи. Ці дівчатка витягали
щодня по слизьких краях цегельної ями з глибини 30 футів 10 тонн
глини і переносили її на віддаль 210 футів. «Дитина не може
пройти чистилища цегельні, не зазнавши великої моральної
деґрадації\dots{} Безсоромна мова, яку їм доводиться чути від найніжнішого
віку, неподобні, непристойні й безсоромні звички, серед
яких вони виростають в неуцтві та здичавінні, роблять із них
\parbreak{}  %% абзац продовжується на наступній сторінці
