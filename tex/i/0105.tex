взагалі відрізнятися від іншої суми грошей лише своєю величиною.
Отже, процес Г — Т — Г завдячує свій зміст не якісній
ріжниці своїх полюсів, бо обоє вони є гроші, а лише їхній кількісній
ріжниці. Кінець-кінцем із циркуляції витягається більше
грошей, ніж спочатку туди їх подано. Бавовну, куплену за 100
фунтів стерлінґів, знову продається, приміром, за 100 + 10 фунтів
стерлінґів, або за 110 фунтів стерлінґів. Отже, повна форма
цього процесу є Г — Т — Г', де Г' = Г + ΔГ, тобто дорівнює
первісно авансованій грошовій сумі плюс приріст. Цей приріст,
або надлишок понад первісну вартість, я називаю додатковою
вартістю (surplus value). Отже, первісно авансована вартість не
лише зберігається в циркуляції, але в ній вона ще змінює величину
своєї вартости, долучає до себе якусь додаткову вартість,
або зростає у своїй вартості. І цей рух перетворює її на
капітал.

Правда, можливо також, що у формі Т — Г — Т обидва полюси
T, Т, приміром, збіжжя й одяг, є кількісно різні величини
вартости. Селянин може продати своє збіжжя понад його вартість
або купити одяг нижче від його вартости. З другого боку, його
може обдурити продавець одягу. Однак, така ріжниця вартости
лишається чисто випадковою для цієї форми циркуляції. Ця форма,
цілком протилежно до процесу Г — Т — Г, зовсім не втрачає
значення й рації, коли її обидва полюси, приміром, збіжжя й
одяг, є еквіваленти. Тут рівність їхньої вартости є скорше умова
нормального перебігу процесу.

Повторення або поновлення продажу задля купівлі находить,
як і сам цей процес, міру й доцільність у кінцевій меті, що лежить
поза межами цього процесу, — у споживанні, задоволенні певних
потреб. Навпаки, в купівлі задля продажу початок і кінець
є те саме, гроші, мінова вартість, і вже через це рух цей є безкраїй.
Правда, з Г стало Г + ΔГ, з 100 фунтів стерлінґів — 100 + 10 фунтів
стерлінґів. Але розглянуті лише щодо якости ці 110 фунтів
стерлінґів є те саме, що й 100 фунтів стерлінґів, а саме гроші.
Розглянуті щодо кількости 110 фунтів стерлінґів є обмежена сума
вартости, так само як і 100 фунтів стерлінґів. Коли б ці 110 фунтів
стерлінґів було витрачено як гроші, то вони перестали б

купівля задля продажу є спекуляція, і таким чином відпадає ріжниця
між спекуляцією й торговлею. «Всяка операція, коли одна особа купує
продукт із наміром продати його, є в дійсності спекуляція» («Every
transaction in which an individual buys produce in order to sell it again,
is, in fact, a speculation»). (Mac Culloch: «А Dictionary practical etc.
of Commerce», London 1847, p. 1009). Куди наївнішим є Пінто, цей Піндар
амстердамської біржі: «Торгівля — це гра (це речення він запозичає
у Льокка) і, звичайно, граючи з тим, хто нічого не має, не можна
виграти. Коли б хто протягом довгого часу постійно в усіх вигравав,
то він мусив би добровільно повернути більшу частину свого зиску,
щоб знов почати гру». («Le commerce est un jeu, et ce n’est pas avec des
gueux qu’on peut gagner. Si l’on gagnait longtemps en tout avec tous, il
taudrait rendre de bon accord les plus grandes parties du profit, pour
recommencer le jeu»). (Pinto: «Traité de la Circulation et du Crédit».
Amsterdam 1771, p. 231).
