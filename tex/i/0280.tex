бливо й інтенсивности праці, аніж у незалежному реместві або
навіть у простій кооперації. Та обставина, що на продукцію
якогось товару витрачається лише суспільно-доконечний робочий
час, за товарової продукції з’являється взагалі як зовнішній
примус конкуренції, бо, висловлюючись поверхово, кожний
поодинокий продуцент мусить продавати товар за його ринковою
ціною. Навпаки, у мануфактурі виготовлення даної кількости
продукту протягом даного часу стає технічним законом самого
процесу продукції.\footnote{
Проте цього результату мануфактурне підприємство в багатьох
галузях продукції доходить лише недосконало, бо мануфактура не вміє
з певністю контролювати загальні хемічні й фізичні умови процесу продукції.
39 «Якщо досвід залежно від осібної природи продукту кожної ма
нуфактури показав так найвигідніший спосіб поділу фабрикації на
частинні операції, як і потрібне для цього число робітників, то всі підприємства,
що не вживатимуть точної кратної кількости цього числа
робітників, будуть продукувати з більшими витратами... Це одна з
причин колосального поширення промислових підприємств». (Ch. Babbage:
«On the Economy of Machinery», London 1832. ch. XXI, p. 172,173).
}

 Однак різні операції потребують неоднакового часу й тому
дають протягом однакового часу неоднакові кількості частинних
продуктів. Отже, коли той самий робітник день-у-день повинен
виконувати завжди лише ту саму операцію, то для різних
операцій мусить бути вжите відносно різне число робітників,
приміром, у черенковій мануфактурі на одного ґлянсувальника
четверо ливарників та двоє відламувачів, бо один ливарник виливає
за годину 2.000 черенків, один відламувач одламує 4.000, а
ґлянсувальник ґлянсує начисто 8.000. Тут принцип кооперації
повертається назад до своєї найпростішої форми, до одночасного
вживання багатьох робітників, що роблять однорідну роботу,
але цей принцип стає тепер виразом певного органічного відношення.
Отже, мануфактурний поділ праці не тільки спрощує
і урізноманітнює якісно відмінні органи суспільного колективного
робітника, а й утворює тривале математичне відношення
для кількісного обсягу цих органів, тобто для відносного числа
робітників або для відносної величини робітничих груп у кожній
окремій функції. Разом з якісним розчленуванням він розвиває
й кількісну норму (Regel) та пропорційність суспільного процесу
праці.

Якщо для певного маштабу продукції на основі досвіду встановлено
якнайвідповіднішу пропорційність між різними групами
частинних робітників, то поширити цей маштаб можна лише тоді,
коли вжити кратну кількість робітників кожної з цих окремих
груп.39 До цього долучається ще й те, що той самий індивід може
виконувати деякі роботи однаково добре, все одно, чи провадяться
вони у великому чи в малому розмірі, приміром, роботи догляду,
транспортування частинних продуктів з однієї продукційної фази

of artists to every manufacture... the greater the order and regularity of
every work, the same must needs be done in less time, the labour must
be less»). («The Advantages of the East-India Trade», London 1720, p. 68).