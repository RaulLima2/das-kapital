\parcont{}  %% абзац починається на попередній сторінці
\index{i}{0152}  %% посилання на сторінку оригінального видання
протягом процесу праці, вони зберігають супроти продукту свою
самостійну форму, так само зберігають вони її й після своєї смерти.
Трупи машин, знарядь, робітних приміщень тощо існують завжди
окремо від продуктів, творенню яких вони допомагали. Якщо
тепер розглянути той цілий період, протягом якого служить такий
засіб праці, від дня його вступу до майстерні й до того дня, коли
його викинуть у комірку на мотлох, то ми побачимо, що його
споживну вартість протягом цього періоду геть чисто спожито
працею, і тому його мінова вартість геть цілком перейшла на
продукт. Коли прядільна машина за 10 років оджила свій вік,
то протягом десятилітнього процесу праці ціла її вартість перейшла
на десятирічний продукт. Отже, період життя засобу праці
обіймає більше або менше число процесів праці, що знову й знов
повторюються за його допомогою. З засобом праці справа стоїть
так само, як і з людиною. Кожна людина відмирає день-у-день
на 24 години. Але ні по одній людині точно не пізнати, скільки
днів її життя вже минулося. Однак це не перешкоджає товариствам
страхування життя робити з пересічного часу тривання людського
життя дуже певні і, що ще важливіше, дуже зисковні висновки.
Так само стоїть справа і з засобом праці. З досвіду відомо, як
пересічно довго може протриматись якийсь засіб праці, приміром,
машина певного роду. Припустімо, що її споживна вартість у
процесі праці зберігається лише 6 днів. Таким чином пересічно
вона втрачає кожного робочого дня \sfrac{1}{6} своєї споживної вартости
й тому віддає денному продуктові \sfrac{1}{6} своєї вартости. Таким способом
обчислюється зужитковання всіх засобів праці, отже,
приміром, щоденну втрату їхньої споживної вартости й відповідне
до цього щоденне перенесення їхньої вартости на продукт.

Таким чином ясно виявляється, що засіб продукції ніколи не
віддає продуктові більше вартости, ніж сам тратить у процесі
праці через знищення своєї власної споживної вартости. Коли б
засіб продукції не мав жодної вартости, не мав би чого втрачати,
тобто коли б він сам не був продуктом людської праці, то він і
не віддавав би продуктові жодної вартости. Він служив би як
утворювач споживної вартости, не служачи для утворення мінової
вартости. Так стоїть справа з усіма засобами продукції, що
існують від природи, без допомоги людини: із землею, вітром,
водою, залізом у жилах руди, деревом у пралісі й т. ін.

Тут перед нами виступає друге інтересне явище. Нехай вартість
машини дорівнює, приміром, 10.000 фунтів стерлінґів
і нехай її спрацьовується за 1.000 днів. У цьому випадку \sfrac{1}{1000} вартости
машини переходить щодня з неї самої на її денний продукт.
Одночасно ціла машина функціонує й далі в процесі праці, хоч
і зі щодалі меншою життєвою силою. Отже, виявляється, що
один фактор процесу праці, один засіб продукції цілком увіходить
у процес праці, але лише частинно в процес зростання вартости.
Ріжниця між процесом праці і процесом зростання вартости
відбивається тут на їхніх речових факторах, бо той самий
засіб продукції як елемент процесу праці цілком входить у той
\parbreak{}  %% абзац продовжується на наступній сторінці
