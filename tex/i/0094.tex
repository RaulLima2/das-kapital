нують, як засіб циркуляції і т. ін. Отже, навіть якщо дано ціни, швидкість грошового обігу й
економію платежів, то маса грошей, що циркулюють протягом якогось періоду, приміром, одного дня, і
маса товарів, що циркулюють, вже не покривають одна одну. Циркулюють гроші, що репрезентують товари,
давно вже витягнуті з циркуляції. Циркулюють товари, що їхній грошовий еквівалент з’явиться лише у
будучині. З другого боку, зобов’язання, які щодня складаються, і зобов’язання, що їм того ж дня
настає термін платежу, є цілком неспільномірні величини. 102

Кредитові гроші виникають безпосередньо з функції грошей, як засобу платежу, при чому самі боргові
посвідки за продані товари й собі циркулюють, переносячи боргові вимоги. З другого боку, в міру того
як поширюється кредитова справа, поширюється й функція грошей як засобу платежу. В цій своїй функції
вони набувають власних форм існування, що в них вони перебувають у сфері великих торговельних
оборудок, тимчасом як золоту або срібну монету витискується переважно у сферу дрібної торговлі. 103

На певній висоті розвитку й ширині обсягу товарової продукції функція грошей як засобу платежу
виходить поза сферу циркуляції товарів. Гроші стають загальним товаром для всіх

102 «Сума продажів або договорів, що їх зроблено протягом даного дня, не впливає на кількість
грошей, що циркулюють у цей самий день, але у величезній більшості випадків виражається в цілому
ряді векселів на суму грошей, які можуть увійти в обіг лише в дальші більш-менш віддалені терміни...
Акцептовані векселі або кредити, відкриті сьогодні, зовсім не повинні мати якоїсь подібности щодо
кількости загальної суми або протягу реченців з тими кредитовими операціями, що будуть переведені
завтра або в найближчі дні; багато з переведених сьогодні кредитових операцій і виданих сьогодні
векселів можуть навіть збігатися щодо терміну платежу з багатьма зобов’язаннями, що переведення їх
стосується до ряду різних попередніх цілком невизначених дат; векселі на 12, 6, 3 місяці і на 1
місяць часто збігаються одні з одними, і подається їх до виплати того самого дня... » («The amount
of sales or contracts entered upon during the course of any given day, will not affect the quantity
of money afloat on that particular day, but, in the vast majority of cases, will resolve themselves
into multifarious drafts upon the quantity of money which may be afloat at subsequent dates more or
less distant... The bills granted or credits oponed, to day, need have no resemblance whatever,
either in quantity, amount or duration, to those granted or entered ucon to-morrow or next day; nay,
many of to-day’s bills and credits, when due, fall in with a mass of liabilities whose origins
traverse a range of antecedent dates altogether indefinite, bills at 12, 6, 3 months or 1 often
aggregating together to swell the common liabilities of one particular day...»). («The Currency
Question Reviewed; a letter to the Scotch people. By a Banker in England», Edinburgh 1845, p. 29, 30
passim).

103 Як приклад того, як мало реальних грошей входить у власне торговельні операції, наводимо тут
схему одного з найбільших лондонських торгових домів (Morrison, Dillon et С°) щодо його річних
грошових прибутків і платежів. Його операції 1856 р., що обіймають багато мільйонів фунтів
стерлінґів, тут пропорційно скорочені і зведені до маштабу одного мільйона фунтів стерлінґів:
