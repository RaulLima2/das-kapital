лань неможливо, бо родини, де вони трапляються, ховають їх
у якнайбільшій тайні».131

Робітники на вугільних і інших шахтах належать до найкраще
оплачуваних категорій британського пролетаріату. Якою ціною
вони купують свою заробітну плату, це показано вже раніш.132
Я кину тут оком на їхні житлові умови. Експлуататор шахт,
хоч буде це власник їх, хоч наймач їх, звичайно будує певне
число котеджів для своїх рук. Вони дістають котеджі й вугілля
на опал «даремно», тобто це становить частину заробітної плати,
що її видається in natura. Хто не дістає такого помешкання,
одержує замість цього 4 фунти стерлінґів річно. Гірничі округи
швидко приманюють до себе дуже численну людність, що складається
з самих гірників, а також ремісників, крамарів тощо,
які групуються навколо гірників. Як і скрізь, де густа залюдненість,
земельна рента й тут висока. Тому гірнопромисловець
намагається на якнайтіснішому будівельному терені при вході
в шахту побудувати стільки котеджів, скільки треба, щоб понапихати
туди всі свої робочі руки разом з їхніми родинами.
Коли недалеко відкривають нові шахти або знов починають
працювати в старих, тіснота збільшується. При будуванні котеджів
переважає лише один погляд — «поздержливість» капіталіста
від усяких не абсолютно неминучих витрат готівкою.
«Помешкання шахтарів та інших робітників, що зв’язані з копальнями
Northumberland і Durham, — каже д-р Джуліян Гентер,
— у пересічному є, мабуть, чи не найгірше й найдорожче
з усього того, що в цьому напрямі дає у великому маштабі
Англія, за винятком хіба подібних округ у Monmouthshire. Найбільше
лихо у тому, що кожна кімната надто переповнена, площа
густо забудована великим числом будинків, бракує води й
немає кльозетів, часто вживають методи ставити будинки один
над одним або розділяти їх in flats (так що різні котеджі становлять
поверхи, які вертикально лежать один над одним)... Підприємець
поводиться з усією колонією так, наче вона лише стоїть
табором, а не живе постійно».133 «Виконуючи дані мені інструкції,
— каже д-р Стивенс, — я відвідав більшу частину великих
гірничих селищ Durham Union’у... За дуже небагатьма винятками,
треба сказати, що ніде не звертають найменшої уваги на
заходи охорони здоров’я мешканців... Всі гірники прикріплені

131    Там же, стор. 18, примітка. Опікун бідних у Chapei-en-le-Frith-Union
повідомляє генерального реєстратора: «В Doveholes у великому
горбі вапняного попелу пороблено багато печер. Ці печери служать
за житла для копальників та інших робітників, занятих коло будування
залізниць. Печери тісні, вогкі, без зливів на нечисть І без кльовеїів.
У них немає жодного вентиляційного приладу, за винятком відтулини у
склепінні, і ця відтулина одночасно служить і за димар. Віспа лютує і
вже спричинила декілька смертних випадків (серед троглодитів). (Там же,
примітка 2).

132    Подробиці, наведені на стор. 415 і дальших, стосуються саме до
робітників у кам'яновугільних копальнях. Про ще гірший стан у руднях
порівн. сумлінний звіт Royal Commission з року 1864.

133 Там же, стоp. 180, 182.
