\parcont{}  %% абзац починається на попередній сторінці
\index{i}{0414}  %% посилання на сторінку оригінального видання
часу зробив з вимогами 1842 р. Тому вже 1864 р., коли комісія
опублікувала лише частину своїх звітів, мануфактуру глиняних
виробів (включно й ганчарство), фабрикацію шпалер, сірників,
набоїв і пістонів, як і стриження оксамиту, підпорядковано
тим законам, що мали силу для текстильної промисловости.
У троновій промові з 5 лютого 1867 р. тодішній кабінет торі
оголосив дальші біли, основані на прикінцевих пропозиціях
комісії, яка тимчасом 1866 р. закінчила свою працю.

15 серпня 1867 р. Factory Acts Extension Act, a 21 серпня і
Workshops Regulation Act дістали королівське затвердження;
перший закон реґулює великі, а останній — дрібні галузі промисловости.

Factory Acts Extension Act реґулює домни, залізообробні
заводи, міделиварні заводи, сталеливарні, машинобудівельні заводи,
металеві майстерні, фабрики ґутаперчі, фабрики паперу,
скла, тютюну, далі, друкарні, палітурні і взагалі всі такі промислові
майстерні, де одночасно працює п’ятдесят або більше осіб
щонайменше протягом ста день на рік.

Щоб дати деяке уявлення про розміри охопленої цим законом
сфери, подамо тут деякі з установлених у ньому визначень:

«\emph{Ремество} має (в цьому законі) означати: всяку ручну
працю, виконувану з професії або для заробітку, для або з
нагоди виготовлення, зміни, оздоблення, лагодження або остаточного
оброблення на продаж будь-якого предмету або його
частини.

«\emph{Майстерня} має означати: всяку кімнату або місце під дахом
або під голим небом, де дитина, підліток або жінка працює в
якомусь «реместві» і куди той, хто застосовує до праці таких
дітей, підлітків або жінок, має право входу й контролю.

«\emph{Занятий} має означати: працювати в якомусь «реместві»,
за плату або без плати, в майстра або в одного з батьків, як це
докладніше визначено нижче,

«\emph{Батьки} має означати: батько, мати, опікун або інша особа,
що має опіку або контроль над будь-якою\dots{} дитиною або підлітком».

Пункт сьомий, що визначає кару за вживання до праці дітей,
підлітків і жінок усупереч постановам цього закону, накладає
грошову кару не тільки на власника майстерні, усе одно, чи буде
ним хто з батьків, чи ні, але й на «батьків або інших осіб, які
опікають дитину, підлітка або жінку, або мають безпосередню
користь від їхньої праці».

Factory Acts Extension Act, що стосується до великих підприємств,
далеко поступається проти фабричного закону в наслідок
безлічі мізерних виняткових постанов і боязких компромісів
із капіталістами.

Workshops’ Regulation Act, мізерний у всіх своїх подробицях,
лишився мертвою буквою в руках міської й місцевої влади, якій
доручено його проведення в життя. Коли парлямент 1871 р.
відібрав від неї це уповноваження, щоб передати його фабричним
\parbreak{}  %% абзац продовжується на наступній сторінці
