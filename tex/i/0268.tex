треба керувати, процесу, який, з одного боку, є суспільний процес
праці для виготовлення продукту, з другого — процес зростання
капіталу, то своєю формою це керування є деспотичне. З розвитком
кооперації у великому маштабі деспотизм цей розвиває властиві
йому своєрідні форми. Подібно до того, як капіталіст спочатку
звільняється від ручної праці, скоро тільки його капітал
досягає тієї мінімальної величини, за якої тільки й починається
капіталістична продукція у власному значенні, так само й тепер
він знову таки віддає саму цю функцію безпосереднього й невпинного
нагляду над поодинокими робітниками та групами робітників
осібному ґатункові найманих робітників. Як армія потребує
військових обер-офіцерів та унтер-офіцерів, так і маса робітників,
що працюють разом під командою того самого капіталу,
потребує промислових обер-офіцерів (директорів, managers)
та унтер-офіцерів (доглядачів за працею, foremen, overlookers,
contremaîtres), які підчас процесу праці командують іменем
капіталу. Праця нагляду стає їхньою постійною виключною
функцією. При порівнянні способу продукції незалежних селян
або самостійних ремісників із плянтаторським господарством,
що ґрунтується на рабстві, політико-економ залічує цю працю
нагляду до faux frais продукції.\footnoteA{
Змалювавши «superintendence of labour»\footnote*{
— нагляд над працею. Ред.
} як головну характеристичну
рису рабської продукції в південних штатах Північної Америки,
професор Кернс каже далі: «Селянин-власник (на півночі), що привласнює
собі ввесь продукт своєї землі, не потребує іншої спонуки до праці.
Нагляд тут цілком зайвий». («The peasant proprietor appropriating the
whole produce of his soil, needs no other stimulus to exertion. Superintendence
is here completely dispensed with»). (Cairnes: «The Slave Power»,
London 1862, p. 48, 49).
} Навпаки, розглядаючи капіталістичний
спосіб продукції, він ідентифікує функцію керування,
оскільки вона виникає з природи спільного процесу праці, з
тією самою функцією, оскільки її зумовлює капіталістичний і
тому антагоністичний характер цього процесу.\footnote{
Cep Джемс Стюарт, що взагалі дуже добре розбирається в характеристичних
суспільних одмінах різних способів продукції, зауважує:
«Чи не тому великі підприємства у промисловості руйнують приватні
підприємства, що вони більше наближаються до простоти рабського режиму?»
(«Why do large undertakings in the manufacturing way ruin private
industry, but by coming nearer to the simplicity of slaves?»). («Principles
of Political Economy», London 1767, vol. I, p. 167, 168).
} Капіталіст не
тому капіталіст, що він промисловий керівник, навпаки, він стає
промисловим командиром через те, що він є капіталіст. Вища
команда у промисловості стає атрибутом капіталу так само, як
за февдальних часів вища команда на війні і в суді була атрибутом
земельної власности.22а

Робітник є власник своєї робочої сили лише доти, доки він
як продавець її торгується з капіталістом, а продавати він може
лише те, що він має, свою індивідуальну, поодиноку робочу силу.

22a Тим то Оґюст Конт та його школа могли б доводити вічну доконечність
феодальних панів таким самим способом, як це вони робили
щодо панів капіталу.