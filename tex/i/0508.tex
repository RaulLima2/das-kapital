пустив червоного півня. По цей бік каналу швидко зростав оуенізм,
по той бік — сен-сімонізм і фур'єризм. Тоді настав час
для вульґарної політичної економії. Саме за рік перед тим, як
Нассав В. Сеніор із Менчестеру відкрив, що зиск (включаючи
і процент) із капіталу є продукт неоплаченої «останньої дванадцятої
години праці», він сповістив світові про своє друге відкриття.
«Я, — врочисто сказав він тоді, — заміняю слово капітал,
розглядуваний як знаряддя продукції, на слово поздержливість
(Abstinenz)»\footnote{
Senior: «Principes fondamentaux de l’Economie Politique». Trad.
Arrivabene, Paris 1836, p. 308. Але для прихильників старої класичної
школи це було вже трохи занадто безглуздо. «Пан Сеніор замінює вислів
«праця й капітал» на вислів «праця й поздержливість»... Поздержливість
— це просте заперечення. Не поздержливість, а споживання продуктивно
вживаного капіталу становить джерело зиску». (John Cazenove
у примітці до його видання праці Малтуза «Definitions in Political
Economy», London 1853, стор. 130, примітка). Навпаки, Джон. Ст.
Мілл на одній сторінці списує Рікардову теорію зиску, а на другій приймає
Сеніорову теорію «нагороди за поздержливість» («remuneration
of abstinence»). Банальні суперечності так само рідні для нього, як чужа
для нього геґелівська «суперечність», це джерело всякої діялектики.

Додаток до другого видання. Вульґарному економістові ніколи не
впадала в голову та проста думка, що всяку людську дію можна розглядати
як «поздержливість» від протилежної дії. їсти — значить поздержуватися
від посту, ходити — поздержуватися від стоянки, працювати —
поздержуватися від ледарства, ледарювати — поздержуватися від праці
й т. д. Ці пани добре зробили б, коли б подумали над словами Спінози:
«Determinatio est negatio» («Визначення — це заперечення»).
}. Це — незрівнянний зразок «відкрить» вульґарної
економії 1 Економічну категорію вона заміняє на сикофантську
фразу. Voila tout.\footnote*{
Ось і все. Ред.
} «Коли дикун, — навчає Сеніор, — робить
лук, то він займається промисловістю, але не практикує
поздержливости». Це пояснює нам, як і чому за попередніх суспільних
становищ засоби праці фабрикувалося «без поздержливости»
капіталіста. «Що більше суспільство проґресує, то більше
вимагає воно поздержливости»,\footnote{
Senior, там же, стор. 342.
} саме від тих, хто займається
працею присвоювання собі чужої праці та її продукту. Всі умови
процесу праці перетворюються відтепер на відповідну кількість
актів поздержливости капіталіста. Що збіжжя не тільки їдять,
а й сіють, то це — через поздержливість капіталіста! Що вино
витримують певний час, то це теж через поздержливість капіталіста!\footnote{
«Ніхто... не сіятиме, наприклад, своєї пшениці, лишаючи її
12 місяців у землі, або не триматиме свого вина цілі роки в льоху замість
Одразу спожити ці речі або їхній еквівалент, коли він не сподіватиметься
одержати таким способом збільшену вартість і т. ін.» (No one... will sow
his wheat, f. i., and allow it to remain a twelve-month in the ground,
or leave his wine in a cellar for years, instead of consuming these things
or their equivalent at once — unless he expects to acquire additional value
etc.». (Scrope: «Political Economy». Ed. A. Potter, New Уогк 1841,
p. 133, 134).
} Капіталіст грабує свою власну плоть, коли «позичає(!)
робітникові знаряддя продукції», іншими словами, коли,
сполучивши їх з робочою силою, він вживає їх як капітал, за-