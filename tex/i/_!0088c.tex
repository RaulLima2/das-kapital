\parcont{}  %% абзац починається на попередній сторінці 
\index{i}{*0088}  %% посилання на сторінку оригінального видання 
зі Спінозою, а саме як з «мертвим собакою». Тому-то я відверто
визнав себе за учня того великого мислителя і в розділі про теорію
вартости подекуди навіть кокетував його своєрідною термінологією.
Містифікація, що її зазнає діялектика в руках Геґеля, ні в
якому разі не пошкодила йому першому вичерпно й свідомо викласти
її загальні форми руху. Діялектика в нього стоїть на голові.
Треба її повернути й поставити на ноги, щоб вилущити раціональне
ядро з містичної шкаралупи.

У своїй змістифікованій формі діялектика зробилася модною
в Німеччині, бо вона, здавалося, прославляла те, що існує.
У своїй раціональній формі (Gestalt) вона є прикра і страшна для
буржуазії та її доктринерів-проводирів, бо в позитивне розуміння
того, що існує, вона включає одночасно і розуміння його
заперечення, його неминучої загибелі; бо кожну здійснену форму
вона розглядає в течії руху, отже, і з її минущого боку; бо вона
ні перед чим не вклоняється і самою суттю своєю є критична
й революційна.

Повний суперечностей рух капіталістичного суспільства найгостріше
відчувається практичним буржуа в коливаннях того періодичного
циклу, що його перебігає сучасна промисловість, і що
його найвищий пункт є загальна криза. А криза знову насувається,
хоч вона ще перебуває в підготовчих стадіях, і повсюдністю
своєї арени діяння та інтенсивністю свого впливу вона
втовкмачить діялектику навіть вискочням нової, священної прусько-німецької
імперії.

Карл Маркс

Лондон, 24 січня 1873 р.
