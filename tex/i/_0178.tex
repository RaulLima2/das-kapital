\parcont{}  %% абзац починається на попередній сторінці
\index{i}{0178}  %% посилання на сторінку оригінального видання
поповнити за три дні. Те, що ти таким чином виграєш на праці,
я трачу на субстанції праці. Користування моєю робочою силою
і грабування її є цілком різні речі. Коли пересічний період,
який пересічний робітник може прожити за розумної міри праці,
становить 30 років, то вартість моєї робочої сили, яку ти мені
оплачуєш із дня на день, дорівнює $\frac{1}{365×30} \deq{} \sfrac{1}{10950}$ цілої її
вартости. Коли ж ти споживаєш її за 10 років, то платиш мені
щоденно лише \sfrac{1}{10950} замість \sfrac{1}{3650} усієї її вартости, отже, лише
третину її денної вартости, і тому обкрадаєш мене щодня на дві
третини вартости мого товару. Ти платиш мені за одноденну
робочу силу, тоді як споживаєш триденну. Це перечить нашій
умові й законові товарового обміну. Отож я вимагаю робочого
дня нормальної довжини, і я вимагаю його, не апелюючи до твого
серця, бо в грошових справах кінець усякій добродушності.
Ти можеш собі бути зразковим громадянином, навіть членом
товариства охорони тварин, крім того, ще мати славу святої
людини, але в тій речі, яку ти репрезентуєш супроти мене, не
б’ється серце в грудях. Коли й здається, що там щось б’ється,
так це стукання мого власного серця. Я вимагаю нормального
робочого дня, бо я вимагаю вартости свого товару, як і кожний
інший продавець\footnote{
Підчас великого страйку лондонських builders (будівельних робітників),
1860--61~\abbr{рр.}, що мав на меті скорочення робочого дня до 9 годин,
їхній комітет оголосив заяву, майже тотожну промові нашого робітника.
Заява не без іронії зазначає, що найненажерливіший з «building
masters»\footnote*{— будівельних підприємців. \emph{Ред.}
}, відомий сер М.~Пето, має «славу побожної людини». (Цей
самий Пето 1867~\abbr{р.} цілком збанкрутував разом із — Струсберґом!).
}.

Ми бачимо, що, коли залишити осторонь цілком елястичні
межі, з самої природи товарового обміну не випливає жодних
меж для робочого дня, отже, і жодних меж для додаткової праці.
Капіталіст обстоює своє право як покупець, силкуючись зробити
робочий день якнайдовшим і по змозі зробити два робочі дні з
одного. З другого боку, специфічна природа проданого товару
вимагає певної межі споживання його покупцем, і робітник обстоює
своє право як продавець, силкуючись обмежити робочий день
певного нормальною величиною. Отже, тут є антиномія, право
протиставиться праву, обидва однаково спираються на закон
товарового обміну. В сутичці поміж рівними правами вирішує
сила. Таким чином в історії капіталістичної продукції нормування
робочого дня виступає як боротьба за межі робочого дня —
боротьба між збірним капіталістом, тобто клясою капіталістів,
і збірним робітником, тобто клясою робітників.

\subsection{Ненажерлива жадоба до додаткової праці. Фабрикант і боярин}

Капітал не винайшов додаткової праці. Скрізь, де якась частина
суспільства має монополію на засоби продукції, там робітник,
\parbreak{}  %% абзац продовжується на наступній сторінці
