пускається лише одно, — що за даного часу продукція якоїсь певної
кількости золота коштує даної кількости праці. Щодо руху
товарових цін взагалі, то для нього мають силу всі розвинені
вище закони простого відносного виразу вартости.

За незмінної вартости грошей загальне підвищення товарових
цін може статись лише тоді, коли підносяться вартості товарів;
за незмінних вартостей товарів — коли вартість грошей
падає. Навпаки: за незмінної вартости грошей загальне зниження
товарових цін може статись лише тоді, коли падають
товарові вартості; за незмінних товарових вартостей — коли
підноситься вартість грошей. Звідси аж ніяк не випливає, що
підвищення вартости грошей завжди спричинюється до пропорційного
зниження товарових цін, а зниження вартости грошей —
до пропорційного зросту товарових цін. Це має силу лише для
товарів з незмінною вартістю. Такі товари, вартість яких, приміром,
зростає рівномірно й одночасно з вартістю грошей, зберігають
ті самі ціни. Коли вартість їхня підноситься повільніш
або швидше, ніж вартість грошей, то зниження або підвищення
їхніх цін визначається ріжницею між рухом їхньої вартости й
рухом вартости грошей і т. д.

А тепер повернімось до розгляду форми ціни.

[Ми бачили, що звичаєві назви й підподіли вагового маштабу
металів служать також за маштаб цін].* Але поволі грошові
назви вагових кількостей металю відокремлюються від первісних
назов його ваги з різних причин; з них історично вирішальне
значення мають ось які: 1) Заведення чужоземних грошей у
менш розвинених народів, як от, приміром, у старому Римі
срібні й золоті монети спочатку циркулювали як чужоземні
товари; назви цих чужоземних грошей відрізнялись від тубільних
назов ваги. 2) З розвитком багатства менш благородний
металь витискується більш благородним з функції мірила вартости;
мідь витискується сріблом, срібло — золотом, хоч як це
чергування може перечити всякій поетичній хронології.56 Фунт
стерлінґів був, наприклад, грошовою назвою дійсного фунта
срібла. Скоро тільки золото витиснуло срібло як міру вартости,
цю саму назву почали прикладати, може, лише до 1/15 і т. ін. фунта
золота, залежно від відношення між вартістю золота й
срібла. Фунт як грошова назва і звичаєва вагова назва золота
тепер відокремились.57 3) Протягом століттів різні князі систематично
фалшували монету, в наслідок чого від первісної ваги
грошових монет лишилась фактично тільки назва.58

56    А проте це чергування не має загального історичного значення.

57    Примітка до другого видання. Так, англійський фунт означає
менш, ніж одну третину його первісної ваги, шотляндський фунт перед
унією — лише 1/36, французький лівр — 1/74, еспанський мараведі — менш
як 1/1000, португальський рей — ще куди меншу частку.

58 Примітка до другого видання. «Монети, що їх назви тепер уже
лише ідеальні, є найдавніші монети кожної нації, і всі вони колись
були реальні, а що були реальні, то служили за рахункові гроші» («Le

* Заведене у прямі дужки ми беремо з французького видання. Ред.
