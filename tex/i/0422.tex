явний до цього часу забезпечний клапан цілого суспільного
механізму. Разом із матеріяльними умовами й суспільною комбінацією
процесу продукції воно призводить до зрілости суперечності
й антагонізми капіталістичної форми цього процесу,
а тому одночасно і елементи творення нового й моменти перевороту
старого суспільства.\footnote{
Роберт Оуен, батько кооперативних фабрик і кооперативних
крамниць, який, однак, як зауважено вище, зовсім не поділяв ілюзій
своїх наслідувачів щодо значення цих ізольованих елементів перевороту,
в своїх спробах не тільки фактично виходив із фабричної системи,
але й теоретично оголосив її за вихідний пункт соціяльної революції.
Пан Віссерінґ, професор політичної економії в університеті в Ляйдені,
здається, передбачає щось таке подібне, коли він у своєму «Handboek
van Praktische Staatshuishoudkunde. 1860—62», де всі тривіяльності
вульґарної економії подано в щонайвідповіднішій формі, завзято обстоює
ремісниче виробництво проти великої промисловости. — [До четвертого видання.
— «Нова юридична плутанина», що її породило англійське законодавство
за допомогою одне одному суперечних Factory Acts, Factory
Extension Act і Workshops’ Act, стала кінець-кінцем нестерпною, і
тим то у Factory and Workshop Act 1878 p. зроблено кодифікацію цілого
цього законодавства. Певна річ, докладної критики цього промислового
кодексу Англії, який тепер має силу, тут не можна дати. Тому, може,
досить буде таких уваг. Закон охоплює: 1) текстильні фабрики; тут
майже все лишається по-старому: дозволений робочий час для дітей
понад 10 років становить 5\sfrac{1}{2} годин на день, абож 6 годин, але тоді субота
вільна; для підлітків і жінок — 10 годин на день протягом п’ятьох
днів, у суботу щонайбільше 6\sfrac{1}{2} годин. — 2) Нетекстильні фабрики. Тут
постанови більше, ніж раніш, наближаються до постанов, зазначених у
пункті 1, але все ще існують деякі вигідні для капіталістів винятки, які
в деяких випадках можна ще більше поширити за спеціяльним дозволом
міністра внутрішніх справ. — 3) workshops (майстерні), визначувані
більш-менш так само, як і в попередньому законі; оскільки в них вживають
дітей, підлітків або жінок, workshops поставлено майже нарівні
з нетекстильними фабриками, однак знову таки з полегшенням у деталях.
— 4) Workshops, де не вживають дітей або підлітків, а лише осіб
обох статей понад 18 років; для цих майстерень маємо ще більші полегшення.
— 5) Domestic workshops (хатні майстерні), де працюють тільки
члени родини в помешканні самої родини; ще елястичніші постанови,
і одночасно таке обмеження, що інспектор без окремого дозволу міністра
або судді може відвідувати лише такі помешкання, яких воднораз не
використовується як житло; нарешті, безумовна воля для плетіння з
соломи, плетіння мережива і виробництва рукавиць у межах родини.
Попри всі свої хиби цей закон, поруч із швайцарським федеральним
фабричним законом з 23 березня 1877 р., все ж є найліпший закон у цій
справі. Порівняння його із згаданим швайцарським федеральним законом
особливо цікаве тим, що воно дуже унаочнює переваги й хиби двох законодавчих
метод — англійської, «історичної», яка втручається в справи
від нагоди до нагоди, і континентальної, побудованої на традиціях французької
революції, методи більш узагальнювальної. На жаль, англійський
кодекс у своєму застосуванні до workshops здебільша все ще мертва
буква — в наслідок недостатнього інспекторського персоналу. — Ф. Е.].
}

10. Велика промисловість і рільництво

Революцію, яку велика промисловість викликає у рільництві
і в соціяльних відносинах аґентів рільничої продукції, можна
буде з’ясувати тільки пізніше. Тут досить буде подати коротко
деякі результати тієї революції, антиципуючи їх. Якщо вжиток