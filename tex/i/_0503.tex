\index{i}{0503}  %% посилання на сторінку оригінального видання 
3. Поділ додаткової вартости на капітал і дохід.
Теорія поздержливости

У попередньому розділі ми розглядали додаткову вартість,
зглядно додатковий продукт, лише як індивідуальний споживний
фонд капіталіста, в цьому розділі ми розглядали її досі лише як
фонд акумуляції. Але вона не є ні тільки одне, ні тільки друге,
а одне і друге одночасно. Одну частину додаткової вартости капіталіст
споживає як дохід,\footnote{
Читач зауважить, що слова дохід (revenue) уживається в подвійному
значенні: поперше, щоб означити додаткову вартість, як плід, що
періодично виникає з капіталу; подруге, щоб означити ту частину цього
плоду, яку капіталіст періодично споживає або додає до свого споживного
фонду. Я зберігаю це подвійне значення, бо воно гармоніює із звичайною
термінологією англійських і французьких економістів.
} другу частину її він вживає як капітал,
або акумулює.

За даної маси додаткової вартости одна з цих частин буде то
більша, що менша друга. Якщо припустити всі інші обставини
за незмінні, то те відношення, в якому відбувається цей поділ,
визначає величину акумуляції. Але той, хто робить цей поділ,
є власник додаткової вартости, капіталіст. Отже, цей поділ є
акт його волі. Про частину зібраної ним данини, яку він акумулює,
кажуть, що він її заощаджує, бо не проїдає її, тобто виконує
свою функцію як капіталіст, а саме функцію самозбагачування.\footnote*{
У другому німецькому виданні цей абзац подано повніше: «За
даної маси додаткової вартости величина акумуляції, очевидно, залежить
від поділу додаткової вартости на фонд акумуляції і фонд споживання,
на капітал і дохід. Що більша одна частина, то менша друга. Тому маса
додаткової вартости або додаткового продукту, отже, те багатство, що
ним порядкує країна і що може бути перетворене на капітал, завжди
більше за частину додаткової вартости, дійсно перетвореної на капітал.
Що розвиненіша капіталістична продукція певної країни, що швидша
й масовіша акумуляція, що багатша країна, що колосальніші, отже,
розкіш і марнотратство, то більша ця ріжниця. Якщо залишити осторонь
щорічний приріст багатства, то й те багатство, що є в споживному фонді
капіталіста і що нищиться лише поступінно, має почасти таку натуральну
форму, що в ній воно могло б безпосередньо функціонувати як капітал.
До наявних елементів багатства, які могли б функціонувати в процесі
продукції, належать і ті робочі сили, що їх зовсім не вживається або вживається
на суто умовні, особисті, часто ганебні, послуги. Пропорція, що
в ній додаткова вартість поділяється на капітал і дохід, безперестанно
змінюється і залежить від обставин, що їх тут не доводиться докладно
розглядати. Тому капітал, застосовуваний у певній країні, є нестала
величина, а змінна, завжди мінлива і елястична частина наявного багатства,
що може функціонувати як капітал.

Через те, що постійне присвоювання продукованої робітником додаткової
вартости або додаткового продукту для капіталіста з’являється
як плід, який періодично дає його капітал; через те, що продукт чужої
праці, який він узурпує без жодного еквіваленту, становить періодичний
приріст його приватної власності!, то й поділ цієї додаткової вартости або
додаткового продукту на додатковий капітал і фонд споживання, природно,
упосереднюється актом його власної волі». Ред.
}

Лише остільки, оскільки капіталіст є персоніфікований капітал,
він має історичну цінність і те історичне право на існування,
що, як каже дотепний Ліхновський, «keinen Datum nicht hat»
\parbreak{}  %% абзац продовжується на наступній сторінці
