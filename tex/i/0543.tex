що зростають еластичність капіталу, який функціонує, і те абсолютне
багатство, що з нього капітал становить лише деяку
еластичну частину, не тільки через те, що кредит при кожній:
особливій принаді, одразу ж віддає надзвичайну частину цього
багатства як додатковий капітал до розпорядження продукції,
— технічні умови самого, процесу продукції, машини, засоби
транспорту й т. ін., в якнайбільшому маштабі уможливлюють
якнайшвидше перетворення додаткового продукту на
додаткові засоби продукції. Маса суспільного багатства, що зростає
з прогресом акумуляції, й що її можна перетворити на додатковий
капітал, несамовито рине в старі галузі продукції, що
їхній ринок раптом поширюється, або в нововідкривані галузі»
як ось залізниці й т. ін., що потреба на них випливає з розвитку
старих галузей продукції. В усіх таких випадках треба, щоб була
можливість раптом і без скорочення маштабу продукції в інших
сферах кидати великі маси людей на вирішальні пункти. Ці
маси постачає перелюднення. Характеристичний життьовий шлях
сучасної промисловости, ця форма перериваного невеликими
коливаннями десятирічного циклу періодів середнього оживлення»
продукції під високим тисненням, кризи й застою, ґрунтується
на постійному творенні, більшому або меншому поглиненні й
новоутворенні промислової резервної армії, або перелюднення.
З свого боку, мінливість фаз промислового циклу збільшує
людський матеріял для перелюднення і стає одним з найенерґійніших
факторів репродукції перелюднення.

thousand times less... the whole of the annual savings, added to the fixed
capital, would have no effect in increasing the demand for labour»). (John
Barton: «Observations of the circumstances which influence the Condition
of the Labouring Classes of Society», London 1817, p. 16, 17). «Та
сама причина, в наслідок якої зростає чистий дохід країни, може одночасно
на другому боці зробити людність надмірною і погіршити становище
робітників» («The same cause which may increase the net revenue oft he country
may at the same time render the population redundant, and deteriorate
the condition of the labourer»). (Ricardo: «Principles of Political Economy»,
3 rd cd. London 1821, p. 469). Із збільшенням капіталу «попит (на
працю) відносно дедалі зменшується» («the demand (for labour) will be ina diminishing
ratio»). (Там же, стор. 480, примітка). «Сума капіталу, призначена
на утримання праці, може варіювати незалежно від якихбудь змін
у загальній сумі капіталу... Великі коливання в наявній кількості роботи
й великі страждання можуть ставати частішими в міру того, як сам капітал
зростає». (The amount of capital devoted to the maintenance of labour
may vary, independently of any changes in the whole amount of capital...
Great fluctuations in the amount of employment, and great suffering may
become more frequent as capital itself becomes more plentiful»). (Richard
Jones: «An Introductory Lecture on Political Economy», London 1833, p. 13).
«Попит (на працю) зростатиме... не пропорційно до акумуляції загального
капіталу... Тому всяке збільшення національного капіталу, призначеного
на репродукцію, з прогресом суспільства справлятиме щораз
менший вплив на становище робітника» («Demand (for labour) will rise...
not in proportion to the accumulation of the general capital... Every
augmentation, therefore to the national stock destined for reproduction,
comes, in the progress of society, to have a less and less influence upon the
condition of the labourer»). (G. Ramsay: «An Essay on the Distribution,
of Wealth», Edinburgh 1836, p. 90, 91).
