часто повторюють сучасні економісти, а саме тоді, коли вони
намагаються довести, що розвинена форма обміну товарів, торговля,
є джерело додаткової вартости. «Торговля, — кажуть вони,
наприклад, — додає продуктам вартости, бо ті самі продукти
мають більше вартости в руках споживачів, ніж у руках продуцентів;
отже, її треба розглядати, точно кажучи (strictly), як
акт продукції».\footnote{
S. P. Newman: «Elements of Political Economy», Andover and
New-York. 1835, p. 175.
} Але за товари не платять двічі, раз за їхню
споживну вартість, другий раз за їхню вартість. І коли споживна
вартість товару корисніша покупцеві, ніж продавцеві, то його
грошова форма корисніша продавцеві, ніж покупцеві. А то ж
хіба б він продавав товар? І таким чином можна б так само сказати,
що покупець, точно кажучи (strictly), виконує «акт продукції»,
перетворюючи, наприклад, панчохи купця на гроші.

Коли обмінюється товари або товари й гроші рівної мінової
вартости, отже, еквіваленти, то ясна річ, що ніхто не витягає
з циркуляції більше вартости, ніж подає до неї. Тоді не відбувається
жодного творення додаткової вартости. У своїй чистій
формі процес циркуляції товарів зумовлює обмін еквівалентів.
Однак у дійсності процеси не відбуваються в чистій формі. Припустімо,
отже, що обмінюється не-еквіваленти.

У всякому разі на товаровому ринку протистоять один одному
лише посідачі товарів, і влада, яку вони використовують один
проти одного, — це лише влада їхніх товарів. Речова відмінність
товарів є речовий мотив обміну. Ця відмінність робить посідачів
товарів одного від одного залежними, бо жоден із них не має в
своїх руках предмету своєї власної потреби, а кожний із них
має в себе предмет потреби іншого. Поза цією речовою відмінністю
споживних вартостей товарів існує ще лише одна ріжниця
між ними, ріжниця між їхньою натуральною формою і їхньою
перетвореною формою, між товаром і грішми. І таким чином
посідачі товарів відрізняються між собою лише як продавці —
посідачі товару, і як покупці — посідачі грошей.

Припустімо тепер, що продавець через якийсь нез’ясовний
привілей може продавати товари понад їхню вартість, за 110.
коли вони варті 100, отже, з номінальним додатком 10\% до ціни.
Отже, продавець одержує додаткову вартість у 10. Але після
того, як він був продавцем, він стає покупцем. Третій посідач
товарів зустрічається тепер з ним як продавець і собі користується
привілеєм продавати товар на 10\% дорожче. Таким чином
наш посідач товарів як продавець виграв 10, щоб як покупець

(«Dans la société formée il n’y a pas de surabondant en aucun genre»).
Одночасно він дратує його, зауважуючи, що «коли обидва обмінювачі
одержують однаково більше за однаково менше, кожний з них дістає
рівно стільки, скільки і другий». Через те, що Кондільяк ще нічогісінько
не тямить у справі природи мінової вартости, то він є дуже добрий порадник
для пана проф. Вільгельма Рошера в його власних дитячих поглядах.
Див. його: «Die Grundlagen der Nationalökonomie», 3 Auflage,
1858.