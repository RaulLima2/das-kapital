Капітал С розпадається на дві частини: грошову суму с, витрачену на засоби продукції, і другу
грошову суму v, витрачену на робочу силу; с репрезентує частину вартости, перетворену на сталий
капітал, v — частину вартости, перетворену на змінний капітал. Отже, первісно C — c + v, приміром,
авансований капітал у 500 фунтів стерлінґів = 410 фунтів стерлінґів + 90 фунтів стерлінґів.
Наприкінці процесу праці з’являється товар, що його вартість дорівнює с + v + m, де m є додаткова
вартість, наприклад, 410 фунтів стерлінґів + 90 фунтів стерлінґів + 90 фунтів стерлінґів. Первісний
капітал С перетворився на С',
з 500 фунтів стерлінґів на 590 фунтів стерлінґів. Ріжниця між обома дорівнює m, додатковій вартості
в 90. А що вартість елементів продукції дорівнює вартості авансованого капіталу, то сказати, що
надлишок вартости продукту понад вартість елементів його продукції дорівнює приростові авансованого
капіталу, або дорівнює випродукованій додатковій вартості, — є в дійсності тавтологія.

Однак ця тавтологія потребує детальнішого визначення. З вартістю продукту порівнюється вартість
елементів продукції, зужиткованих при його творенні. Але ми вже бачили, що частина застосованого
сталого капіталу, яка складається із засобів праці, віддає продуктові лише одну частину своєї
вартости, тимчасом як друга частина й далі існує в старій своїй формі. А що остання частина не
відіграє жодної ролі у творенні вартости, то тут від неї треба абстрагуватись. Якщо й занести її в
обчислення, то нічого не зміниться. Припустімо, що с = 410 фунтів стерлінґів і
складається з сировинного матеріялу на 312 фунтів стерлінґів, допоміжних матеріялів на 44 фунти
стерлінґів і зужиткованих
у процесі машин на 54 фунти стерлінґів, а вартість дійсно застосованих машин нехай становить 1.054
фунти стерлінґів. За авансовану на створення вартости продукту ми рахуємо лише вартість у 54 фунти
стерлінґів, яку машини втрачають через своє функціонування і в наслідок цього віддають продуктові.
Коли б ми врахували
й ті 1.000 фунтів стерлінґів, що й далі існують у своїй старій формі, як парова машина й т. ін., то
ми мусили б їх урахувати
на обох боках, на боці авансованої вартости та на боці вартости продукту,26a і таким чином ми
одержали б на одному боці
1.500 фунтів стерлінґів і 1.590 фунтів стерлінґів на другому. Ріжниця, або додаткова вартість, була
б, як і раніше, 90 фунтів стерлінґів. Тому там, де із загального зв’язку викладу не виявля-

26a «Коли ми беремо в обрахунок вартість застосованого основного капіталу, як частину авансованого
капіталу, то ми мусимо наприкінці року взяти в обрахунок лишок вартости цього капіталу, як частину
річного доходу» («If we reckon the value of the fixed capital employed as a part of the advances, we
must reckon the remaining value of such capital at the end of the year as a part of the annual
returns»). (Malthus: «Principles
of Political Economy», 2 nd ed. London 1836, p. 269).
