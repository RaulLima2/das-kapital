\parcont{}  %% абзац починається на попередній сторінці
\index{i}{0206}  %% посилання на сторінку оригінального видання
збереження робочої сили визначає тут межі робочого дня, а,
навпаки, якнайбільша щоденна витрата робочої сили, — хоч і
якою болісно-насильною й тяжкою вона була б, — визначає
межі для часу відпочинку робітника. Капітал зовсім не турбується
про довжину життя робочої сили. Його цікавить виключно тільки
максимум робочої сили, що його можна пустити в рух протягом
одного робочого дня. Він досягає цієї мети через скорочення життя
робочої сили, як от ненажерливий господар на селі досягає піднесення
продуктивности землі через виснаження родючости ґрунту.

Отже, капіталістична продукція, яка в суті є продукція додаткової
вартости, вбирання додаткової праці, продукує через
здовжування робочого дня не лише занепадання (Verkümmerung)
людської робочої сили, від якої відбирає нормальні моральні й
фізичні умови розвитку й діяльности. Вона продукує передчасне
виснаження та смерть самої робочої сили\footnote{
«У наших попередніх звітах ми навели думки різних досвідчених
фабрикантів про те, що надмірна праця\dots{} безумовно веде до передчасного
виснаження людської робочої сили» («We have given in our previous
reports the statements of several experienced manufacturers to the effect
that over-hours\dots{} certainly tend prematurely to exhaust the working
power of the men»). (Там же, 64, p. XIII).
}. Вона здовжує
час продукції для робітника протягом якогось даного періоду,
скорочуючи вік його життя.

Але вартість робочої сили містить у собі вартість товарів,
потрібних для репродукції робітника або для розплодження
робітничої кляси. Отже, коли протиприродне здовження робочого
дня, що його неодмінно домагається капітал у своєму безмірному
прагненні до самозростання, скорочує період життя поодиноких
робітників і разом з тим час тривання їхньої робочої сили, то
стає доконечною швидша заміна зужиткованих робочих сил, отже,
стає доконечним збільшення витрат, що входять у репродукцію
робочої сили, цілком так само, як частини вартости машини, що
їх щоденно треба репродукувати, то більші, що швидше зужитковується
машину. Таким чином власний інтерес капіталу, здається,
вимагає від нього нормального робочого дня.

Рабовласник купує собі робітника так само, як купує собі
коня. Втрачаючи раба, він втрачає капітал, що мусить бути поповнений
новою витратою на невільницькому ринку. Але «хоч
як фатально руйнаційно впливають на людський організм поля
рижу в Ґеорґії й болота Міссісіпі, все ж таки це руйнування людського
життя не таке велике, щоб його не можна було відшкодувати
з переповнених обор негрів у Вірґінії й Кентукі. Економічні
міркування, які могли б давати деяку ґарантію для людяного
поводження з рабом, оскільки вони ідентифікують інтерес господаря
із збереженням раба, ці міркування, після заведення работорговлі,
перетворюються, навпаки, на причину найкрайнішого
виснажування раба, бо скоро тільки його місце можна поповнити
новим рабом, привезеним із чужих обор негрів, то час тривання
його життя стає менш важливим, ніж його продуктивність за
\parbreak{}  %% абзац продовжується на наступній сторінці
