інші, дати додаткові історичні або статистичні матеріяли, додати
критичні зауваження і т. ін. Хоч які є літературні недосконалості
цього французького видання, все ж воно має наукову цінність,
незалежну від оригіналу і до нього навіть повинні вдаватись
читачі, обізнані з німецькою мовою.

Я подаю далі ті місця післямови другого німецького видання,
що стосуються до розвитку політичної економії в Німеччині і до
методи, вжитої в цій праці.\footnote*{
Див. післямову до другого німецького видання, стор. 81*—88* Ред.
}

Карл Маркс

Лондон, 28 квітня 1875 р.