зують плечима. А тимчасом він з веселою усмішкою вже знову прийняв
свою попередню фізіономію. Він просто морочив нам голову
всіма цими докучливими жаліннями. Він не дасть за цей півшага.
Він полишає ці й подібні пусті виверти й беззмістовні викрути
професорам політичної економії, яким за це власне й платять.
Сам він практична людина, яка, правда, не завжди думає над тим,
що говорить поза своїми справами, алеж завжди знає, що робить
у діловій сфері.

Пригляньмося ближче до справи. Денна вартість робочої
сили становила 3 шилінґи, бо в ній самій упредметнено пів робочого
дня, тобто, тому що засоби існування, щоденно потрібні
для продукції робочої сили, коштують пів робочого дня. Але
минула праця, що міститься в робочій силі, і жива праця, яку
вона може виконати, щоденні кошти її утримання й її щоденне
витрачання, це — дві цілком різні величини. Перша визначає
її мінову вартість, друга становить її споживну вартість. Та
обставина, що для підтримання життя робітника протягом 24 годин
потрібно лише пів робочого дня, аж ніяк не заважає йому
працювати цілий день. Отже, вартість робочої сили та використовування
її в процесі праці є дві різні величини.\footnote*{
У французькому виданні це речення зредаговано так: «Отже, вартість,
що її має робоча сила і вартість, яку вона може створити, є дві різні
величини» («La valeur que la force de travail possède et la valeur qu’elle
peut créer, different donc de grandeur»), Peд.
} Саме цю
ріжницю вартостей капіталіст, купуючи робочу силу, і мав на
увазі. Корисна властивість робочої сіяли робити пряжу або чоботи
була лише condicio sine qua non,\footnote*{
Доконечна умова, тобто умова, що без неї певне явище не може відбутися.
Ред.
} бо для утворення вартости
праця мусить бути витрачена в корисній формі. Але вирішальне
значення мала специфічна споживна вартість цього товару, його
властивість бути джерелом вартости й вартости більшої, ніж має
він сам. Це і є та специфічна послуга, якої сподівається від нього
капіталіст. І він поводиться тут згідно з вічними законами товарового
обміну. Справді, продавець робочої сили, як і продавець
кожного іншого товару, реалізує її мінову вартість і відчужує її
споживну вартість. Він не може одержати першої, не віддаючи
другої. Споживна вартість робочої сили, сама праця, так само
не належить її продавцеві, як споживна вартість проданої олії —
торговцеві олією. Посідач грошей оплатив денну вартість робочої
сили; тому йому належить споживання її протягом дня, денна
праця. Та обставина, що денне утримання робочої сили коштує
лише пів робочого дня, хоч робоча сила може діяти, працювати
цілий день, тобто та обставина, що вартість, яку створює споживання
робочої сили протягом дня, удвоє більша, ніж її власна
денна вартість, є особливе щастя для покупця, але аж ніяк не
є кривда проти продавця.

Наш капіталіст передбачав цей казус, що саме й викликав
його усмішку. Тому робітник находить у майстерні потрібні