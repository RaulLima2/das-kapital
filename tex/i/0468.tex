назву «sweating-system» (потогінна система). З другого боку,
відштучна плата дозволяє капіталістові складати з головним робітником
— у мануфактурі з старшим групи, по копальнях із
вибійником і т. ін., на фабриці з власне машиновим робітником —
умову на певну ціну за штуку, при чому головний робітник сам
бере на себе обов’язок за цю ціну вербувати та оплачувати своїх
робітників-помічників. Експлуатація робітників капіталом здійснюється
тут через експлуатацію робітників робітником». 51

Коли відштучна плата вже існує, то, ясна річ, особистий інтерес
робітника спонукає його якомога інтенсивніше напружувати
свою робочу силу, а це полегшує капіталістові підвищувати нормальний
ступінь інтенсивности. 51а Так само особистий інтерес
робітника спонукає його здовжувати робочий день, бо таким
чином зростає його поденна або потижнева заробітна плата.52

passes through several hands, each of which is to take a share of profits,
while only the last does the work, that the pay which reaches the workwoman
is miserably disproportioned»). («Children’s Employment Commission.
2 nd Report», p. LXX, n. 424).

51    Навіть апологет Вотс зауважує: «Було б велике поліпшення
системи відштучної плати, коли б усі робітники, заняті тією самою працею,
були учасниками умови, кожний відповідно до своєї здібности, замість
того, шоб один з них був зацікавлений у надмірній праці своїх
товаришів собі особисто на користь» («It would be a great improvement
to the system of piece-work, if all the men employed on a job were partners
in the contract, each according to his abilities, instead of one man
being interested in overworking his fellows for his own benefit» (Там же,
crop. 53). Про підлоти цієї системи див. «Children’s Employment Commission.
З rd Report», p. 66, n. 22, p. 11, n 124, p. XI, n. 13, 53, 59 і т. д.

51a Цьому природному результатові часто допомагають штучно.
Приміром, у машинобудівництві (Engineering Trade) Лондону за звичайнісінький
вважається такий маневр: «капіталіст добирає на старшину
певного числа робітників людину надзвичайної фізичної сили та вправности.
Що чверть року або в інші реченці він виплачує їй додаткову плату
з умовою, щоб вона зробила все можливе для того, щоб спонукати своїх
товаришів у праці, які дістають лише звичайну плату, до якнайкрайнішого
змагання... Це без дальших коментарів пояснює скарги капіталістів на
те, що тред-юньйони «паралізують енерґію, видатну вправність та
робочу силу» («stinting the action, superior skill and working power»).
IDunning: «Trades-Unions and Strikes», London 1860, p. 23, 23). Через
те, що автор сам є робітник та секретар тред-юньйону, то ці його слова
можуть видатися за перебільшення. Але погляньмо тоді, наприклад,
до «високореспектабельної» («highly respectable») агрономічної енциклопедії
Дж. Ч. Мортона, до статті «Labourer», де цю методу рекомендується
фармерам як випробовану.

52 «Всі ті, що дістають відштучну плату... мають користь працювати
поза визначені законом меніі робочого дня. Згоду працювати понаднормовий
час можна спостерігати особливо часто серед жінок-ткачих та мотальниць».
(«All those who are paid by piece-work... profit by the transgression
of the legal limits of work. This observation as to the willingness
to work overtime, is especially applicable to the women employed as weavers
and reelers»). («Reports of Insp. of Fact, for 30 th April 1858», p. 9).
«Ця система відштучної плати, така корисна для капіталіста, скерована
прямо на те, щоб спонукати молодого ганчаря до наднормової праці
протягом 4—5 років, при чому він дістає відштучну, але дуже низьку плату.
Це одна з головних причин, що зумовлюють фізичну дегенерацію ганчарів».
(«Children’s Employment Commission. 1 st Report», P-XIII).
