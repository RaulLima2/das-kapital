\parcont{}  %% абзац починається на попередній сторінці
\index{i}{0419}  %% посилання на сторінку оригінального видання
що я ніде не знайшов нічого, хоч трохи подібного до жіночої
праці в копальнях. Це — чоловіча праця і праця для дужих
чоловіків. Кращі елементи поміж копальневими робітниками,
які силкуються піднестися, стати людьми, замість знаходити підтримку
в своїх дружин, через них занепадають». Після цілого
ряду перехресних запитань цих буржуа виявляється, нарешті,
таємниця їхнього «милосердя» до вдовиць, бідних родин і т. ін.:
«Власник копальні призначає певних джентельменів для головного
догляду; останні, щоб заробити панської похвали, додержують
політики зробити все якомога найекономніше, і дівчата-робітниці
одержують від 1 шилінґа до 1 шилінґа 6 пенсів денно там,
де чоловік мусив би одержувати 2 шилінґи 6 пенсів» (№ 1816).

4. Жюрі для огляду мерців (№ 360). «Щодо слідства coroner’ів\footnote*{
В Англії — слідчий для огляду мерців у випадках наглої смерти. \emph{Ред.}
}
у ваших округах, чи задоволені робітники судовими процесами,
коли трапляються нещасливі випадки? — Ні, незадоволені»
(№ 816). «Чому ні? — Особливо тому, що на членів жюрі призначають
людей, які абсолютно нічого не тямлять у копальнях. Робітників
ніколи не закликають, хібащо лише як свідків. Загалом
до жюрі закликають сусідніх крамарів, які є під впливом власників
копалень, їхніх покупців, і не розуміють навіть технічних
висловів свідків. Ми бажаємо, щоб копальневі робітники становили
частину членів жюрі. Звичайно присуд суперечить виказам
свідків» (№ 378). «Чи не повинні жюрі бути безсторонніми? —
Так» (№ 379). «А чи будуть робітники безсторонніми? — Я не
бачу ніяких мотивів, чому б їм не бути безсторонніми. Вони
добре розуміють справу» (№ 380). «А чи не виявлятимуть вони
нахилу до того, щоб виносити несправедливо суворі присуди в
інтересах робітників? — Ні, я не думаю».

5. Фалшива міра та вага й т. ін. Робітники вимагають тижневої
виплати замість двотижневої, міряння цебер вагою, а не
на об’єм, захисту проти вживання фалшивої ваги і т. ін.
(№ 1071). «Коли цебра по-шахрайському збільшують, то робітник
може ж, повідомивши наперед за 14 день, покинути працю
в копальні? — Алеж, прийшовши на іншу копальню, він і там
знайде те саме» (№ 1072). «Але він все ж може покинути те місце,
де йому чинять кривду? — Алеж повсюди панує те саме» (№ 1073).
«Але робітник завжди може покинути своє місце, попередивши
про це за 14 днів? — Так». Цього досить!

6. Копальнева інспекція. Робітники страждають не тільки
від нещасливих випадків через вибух газів (№ 234 і далі). «Нам
доводиться так само нарікати на погану вентиляцію в кам’яновугільних
копальнях, бо люди в них ледве можуть дихати;
через це вони стають непридатними до будь-якої праці. Так,
наприклад, саме тепер у тій частині копалень, де я працюю,
отруйне повітря загнало багатьох людей на цілі тижні до ліжка.
У головних ходах здебільша повітря є досить, але мало його
\parbreak{}  %% абзац продовжується на наступній сторінці
