а) Дітей  b) Членів родин    с) Тижнева зароб. плата чоловіків    d) Тижнева заробітна плата дітей  
 е) Тижневий дохід цілої родини    f) Тижнева квартирна плата    Я) Загаль-ний тижневий заробіток з
відрахуванням квартирної плати    h) Тижневий заробіток на людину

Друге село

6    8    7 ш.    1—1ш. 6п.    10 ш. —  1 ш. 6       п.    8 ш. 6        п.  1 ш. 3/4   п.
6    9    7  »     1—1 «  6 п.     7»      —  1 »    3 1/2 »    5»     8 1/2 »    0 »    8 1/2»
8    10  7  »        —                  7»      —  1 »    3 1/2 »    5»     8 1/2 »    0 »    7     
 »
4    6    7  »        —                  7»      —  1 »     6 1/2 »   5»     5 1/2 »    0 »  11     
 »
3    5    7  »        —                  7»      —  1 »     6 1/2 »   5»     5 1/2 »    1 »    1    
  »

Третє село

4    6    7 ш.         —         7 ш.   —     1 ш.    —       6  шил. —     1 ш.       —
3    5    7  »  1—2 ш. —   11»     6п.   0 »    10 п.   10  »       8 п.  2 »     1 1/3п.
0    2    5  »  1—2 » 16 »  5  »     —     1 »     —         4   »    —        2 146

Скасування збіжжевих законів дало надзвичайний поштовх
англійському рільництву. Дренажні роботи якнайбільшого маштабу,147
нова система годівлі худоби в стайнях і засіву штучних
кормових трав, заведення механічних апаратів до угноювання,
нові способи обробляти глинкуватий ґрунт, збільшене вживання
мінерального добрива, застосування парової машини й
усякого роду нових робочих машин і т. д., взагалі інтенсивніша
культура — ось що характеризує цю епоху. Пан Пезі, президент
королівського рільничого товариства, твердить, що (відносні)
господарські витрати через заведення нових машин зменшились
майже удвоє. З другого боку, швидко збільшився позитивний
дохід від землі. Основною умовою нових метод були збільшені
капіталовкладення на акр, отже, і прискорена концентрація
фарм.148 Одночасно й засівна площа збільшилась від 1846 р.
до 1856 р. на 464.119 акрів, не кажучи вже про величезні площі
у східніх графствах, які з кролячих загородок і злиденних пасовиськ
немов чарами перетворено в буйні збіжжеві лани. Ми вже
знаємо, що одночасно з цим зменшилось загальне число осіб,
занятих у рільництві. Щождо власне рільників обох статей і

146 «London Economist», 29 березня 1845 р., р. 290.

147 Земельна аристократія сама авансувала собі для цієї мети фонди
з державної скарбниці, звичайно через парлямент, за дуже низький
процент, що його фармери мали виплачувати їй удвоє.

148    Зменшення числа середніх фармерів ясно видно з рубрик перепису:
«Син фармера, онук, брат, племінник, дочка, онучка, сестра, племінниця»,
одно слово, члени родини самого фармера, що працюють у
нього. Ці рубрики налічували 1851 р. 216.851 особу, 1861 р. лише 176.151
особу. Від 1851 до 1871 р. число фарм, менших від 20 акрів, зменшилось
більш, ніж на 900, число фарм від 50 до 77 акрів спало з 8.253 до 6.370;
те саме й з усіма іншими фермами, меншими від 100 акрів. Навпаки,
протягом тих самих двадцятьох років число великих фарм збільшилось;
число фарм від 300 до 500 акрів збільшилось з 7.771 до 8.410, число
фарм, більших за 500 акрів, — з 2.755 до 3.914, число фарм, більших
за 1000 акрів, — з 492 до 582.
