є головна підпора змальованої раніше погодинної системи.\footnote{
«Скільки разів доводилося нам бачити в деяких майстернях куди
більше найнятих робітників, аніж цього потребувала праця. Часто наймають
робітників, лише передбачаючи якусь працю, при чому ще невідомо
напевне, чи буде вона, а іноді ця праця просто існує лише в уяві;
а що робітникам платять відштучно, то хазяїн нічим не ризикує, бо при
цьому всі втрати, які постають від гаяння часу, падають виключно на
робітників, що лишаються без праці» («Combien de fois n’avons-nous
pas vu, dans certains ateliers, embaucher beaucoup plus d’ouvriers que
ne le demandait le travail à mettre en main? Souvent, dans la prévision
d’un travail aléatoire, quelquefois même imaginaire, on admet des ouvriers:
comme on les paie auxpièces. on se dit qu’on ne court aucun risque, parce
que toutes les pertes de temps seront à la charge des inoccupés»). (H. Grégoire
«Les Typographes devant le Tribunal Correctionnel de Bruxelles»,
Bruxelles 1865, p. 9).
}

З попереднього викладу випливає, що відштучна плата є найвідповідніша
капіталістичному способові продукції форма заробітної
плати. Хоч вона й не є щось нове — відштучна плата офіціяльно
фігурує, між іншим, побіч почасової плати у французьких
та англійських робітничих статутах XIV віку, — проте, ширше
поле вона добуває собі лише протягом власне мануфактурного
періоду. В епоху бурі й натиску великої промисловости, а саме
від 1797 до 1815 р., вона служить за підойму до здовжування
робочого часу та знижування заробітної плати. Душе важливий
матеріял для руху заробітної плати за тих часів находимо в
Синіх Книгах: «Report and Evidence from the Select Committee
on Petitions respecting the Corn Laws» (парламентська сесія
1813—1814 pp.) і «Reports from the Lords’ Committee, on the
state аої the Growth, Commerce and Consumption of Grain, and
all Laws relating thereto» (сесія 1814—1815 pp.). Тут находимо
документальні докази невпинного знижування ціни праці від
початку антиякобінської війни. Наприклад, у ткацтві відштучна
плата так спала, що, не вважаючи на дуже здовжений робочий
день, поденна плата була тепер нижча, ніж раніш. «Реальний дохід
ткача куди менший, ніж раніш: його перевага над звичайним
робітником, колись дуже велика, майже цілком зникла. Справді,
ріжниця в заробітних платах вправної та звичайної праці тепер

Перше видання появилося 1755 р.). Отже, вже Кантільйон, що в нього
чимало чого запозичили Кене, сер Джеме Стюарт та А. Сміс, з'ясовує
тут відштучну плату просто як модифіковану форму почасової плати.
Французьке видання Кантільйона на титульному аркуші позначене як
переклад з англійської мови, але англійське видання: «The Analysis
of Trade, Commerce etc. by Philip Cantillon, late of the City of London
Merchant» не тільки датоване пізніше (1759), а й своїм змістом показує,
що це пізніша переробка. Так, наприклад, у французькому виданні ще
не згадується про Юма, і, навпаки, в англійському Петті майже вже не
фігурує. Англійське видання має менше теоретичне значення, але воно
містить всякий специфічний матеріял про англійську торговлю, торговлю
зливками й т. ін., а цього у французькому тексті немає. Тому, здається,
фраза в заголовку англійського видання, що твір цей «запозичено головно
з рукопису одного високоталановитого джентлмена, нині, вже небіжчика,
і пристосовано і т. ін.» («taken chiefly from the Manüscript of
a very ingenious Gentleman deceased, and adapted etc.»), є щось більше
sa просту вигадку, звичайну для тих часів.