\parcont{}  %% абзац починається на попередній сторінці
\index{i}{0317}  %% посилання на сторінку оригінального видання
ріжниця між вартістю машини й тією частиною вартости, що
періодично переноситься з неї на продукт. Є велика ріжниця між
машиною як вартостетворчим елементом і машиною як продуктотворчим
елементом. Що більший період, протягом якого та сама
машина знову й знов функціонує в тому самому процесі праці,
то ріжниця та більша. Правда, ми бачили, що кожний засіб праці
у власному значенні слова або знаряддя продукції завжди цілком
увіходить у процес праці і завжди лише частинами, пропорційно
до його пересічного щоденного зужиткування, в процес утворення
вартости. Однак ця ріжниця між уживанням та зужиткуванням
куди більша в машини, ніж у знаряддя, бо машина, збудована
з тривалішого матеріялу, живе довше, бо вживання її, реґульоване
строго науковими законами, уможливлює більшу економію
у витрачанні її складових частин та засобів, що їх вона споживає,
бо, нарешті, поле продукції в неї куди більше, ніж у знаряддя.
Якщо залишити осторонь щоденні пересічні витрати машин та
знаряддя, тобто ту складову частину вартости, яку вони додають
до продукту в наслідок пересічного щоденного зужиткування
та споживання допоміжного матеріялу, як от мастива, вугілля
тощо, то вони функціонують задурно, цілком так, як функціонують
сили природи, що існують без допомоги людської праці.
Що більший обсяг продуктивної діяльности машини супроти
знаряддя, то більший обсяг її дармової служби порівняно із
службою знаряддя. Тільки у великій промисловості людина
навчається примушувати продукт своєї минулої, упредметненої
вже праці працювати задурно у великому маштабі, подібно до
якоїсь сили природи.\footnote{
Рікардо звертає свою увагу на цей ефект машин, — зрештою,
так само мало з’ясований в нього, як і загальна ріжниця між процесом
праці й процесом творення додаткової вартости, — іноді з такою винятковістю,
що принагідно забуває про ту складову частину вартости, яку
машини віддають продуктові, та цілком переплутує їх з силами природи.
Так, наприклад: «Адам Сміс ніде не недооцінює тих послуг, що їх роблять
нам сили природи й машини, але дуже достеменно розрізняв природу
вартости, яку вони додають до продуктів\dots{} а що вони виконують свою
працю задурно,\dots{} то й допомога, яку вони нам роблять, нічого не додає
до мінової вартости» («Adam Smith nowhere undervalues the services
which the natural agents and machinery perform for us, but he very justly
distinguishes the nature of the value which they add to commodities\dots{} as
they perform their work gratuitusly\dots{} the assistance which they afford
us, adds nothing to value in exchange»). (\emph{Rikardo}: «Principles of
Political Economy», 3 rd ed. London 1821, p. 336, 337). Розуміється,
замітка Рікарда справедлива щодо Ж. Б. Сея, який плеще, начебто машини
роблять ту «послугу», що вони творять вартість, яка становить
частину «зиску».
}

При розгляді кооперації й мануфактури виявилося, що деякі
загальні умови продукції, як от будівлі й т. д., порівняно з роз’єднаними
умовами продукції ізольованих робітників, економізуються
через спільне споживання, а тому менш удорожчують продукт.
За машинового виробництва не тільки тіло робочої машини
споживається її численними знаряддями, але й ту саму рухову
\parbreak{}  %% абзац продовжується на наступній сторінці
