\index{i}{0293}  %% посилання на сторінку оригінального видання 
5. Капіталістичний характер мануфактури

Скупчення значного числа робітників під командою того
самого капіталу становить природний вихідний пункт так кооперації
взагалі, як і мануфактури. Навпаки, мануфактурний поділ
праці розвиває зріст числа вживаних робітників у технічну
доконечність. Тепер мінімум робітників, яких мусить уживати
поодинокий капіталіст, приписується йому наявним поділом
праці. З другого боку, користі з дальшого поділу праці зумовлені
дальшим збільшенням числа робітників, яке можна перевести
лише за кратного збільшення робітників. Але разом із
змінною складовою частиною капіталу мусить зростати й стала
його частина, поряд розміру спільних умов продукції, як от
будівлі, печі й т. д., особливо мусить зростати й кількість сировинного
матеріялу, та ще й далеко швидше, ніж число робітників.
Маса сировинного матеріялу, споживана протягом даного
часу даною кількістю праці, більшає в тій самій пропорції, в
якій більшає продуктивна сила праці в наслідок поділу праці.
Отже, зріст мінімального розміру капіталу в руках поодинокого
капіталіста або щораз більше перетворювання суспільних засобів
існування та засобів продукції на капітал — це закон, що
виникає з технічного характеру мануфактури.

В мануфактурі, як і в простій кооперації, робоче тіло, що функціонує,
є форма існування капіталу. Суспільний продукційний
механізм, складений із багатьох індивідуальних частинних робітників,
належить капіталістові. Тому продуктивна сила, що виникає
з комбінації праць, видається продуктивною силою капіталу.
Мануфактура у власному значенні не тільки підбиває під команду
й дисципліну капіталу робітника, який раніше був самостійний,
а ще й створює, крім цього, ієрархічну ґрадацію серед самих робітників.
Тим часом як проста кооперація лишає взагалі і в цілому
\parbreak{}  %% абзац продовжується на наступній сторінці
