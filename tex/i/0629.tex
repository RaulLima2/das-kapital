літті прогнаним із сел ґаелам одночасно заборонено еміґрувати,
шоб силоміць позаганяти їх у Ґлезґо та в інші фабричні міста.216
Як приклад тієї методи, що панувала в XIX столітті 217, тут досить
навести «очищення», пороблені герцоґинею Сотерлендською.
Ця економічно освічена особа, скоро взяла в свої руки управління,
зараз же вирішила розпочати радикальне економічне лікування
краю й перетворити на пасовиська для овець ціле графство,
що його людність попередніми подібними процесами була
вже зменшена до 15.000 душ. Від 1814 до 1820 р. цих 15.000 жителів,
приблизно 3.000 родин, систематично проганяли й винищували.
Всі їхні села позруйновано й попалено, всі їхні поля
поперетворювано на пасовиська. Між британськими солдатами,
присланими для екзекуції, та місцевою людністю доходило до
боїв. Одна стара жінка згоріла в полум’ї своєї хати, не схотівши
залишити її. Таким чином ця мадам присвоїла собі 794.000 акрів
землі, що від незапам’ятних часів належала кланові. Для прогнаних
тубільців вона відвела на узмор’ї приблизно 6.000 акрів
землі, по 2 акри на родину. Ці 6.000 акрів до того часу лежали
пустирем, не приносячи їхній власниці ніякого доходу. У своїх
благородних почуттях герцоґиня пішла так далеко, що здала
цю землю в оренду пересічно по 2 шилінґи 6 пенсів ренти за акр
членам клану, які протягом століть проливали за її рід свою кров.
Всю землю, загарбану в клану, вона поділила на 29 великих
овечих фарм, посадивши на кожній одним-одну родину, здебільшого
англійських фармерських наймитів. У 1825 р. замість
15.000 ґаелів там було вже 131.000 овець. Викинута на узмор’я
частина тубільців намагалася прожити з рибальства. Вони по-

216 В 1860 р. насильно експропрійованих вивезено до Канади, при
чому їм понадавано багато брехливих обіцянок. Декотрі з них повтікали
в гори й на сусідні острови. За ними погнались поліціянти, втікачі вступили
з ними в бійку і таки повтікали.

217 «У гірських місцевостях, — писав у 1814 р. Б’юкенен, коментатор
А. Сміса, — старі відносини власности день-у-день зазнають насильного
перевороту... Лендлорд, не звертаючи уваги на спадкових орендарів
(цю категорію тут теж ужито помилково), віддає землю тому, хто дає найбільшу
ціну, і коли цей останній є меліоратор (improver), то він одразу заводить
нову систему культури. Земля, раніше рясно вкрита дрібними рільниками,
була заселена відповідно до того, скільки вона давала продукту;
за нової системи поліпшеної культури й збільшених рент треба продукувати
якнайбільше продукту з якомога меншими витратами, і задля цієї
мети усувають усі ті руки, що поробилися некорисними... Викинуті з
рідних сел шукають собі засобів існування по фабричних містах тощо».
(David Buchanan: «Observations on etc. A. Smith’s Wealth of Nations»,
Edinburgh 1814, vol. IV, p. 144). «Шотландські вельможі експропріювали
цілі родини так, наче виполювали бур’ян; вони поводилися з селами
та їхньою людністю так, як індійці у своїй помсті з лігвами хижих звірів.
Людину продають за смушок, за баранячу ногу, навіть за щось
менше... Підчас нападу на північні провінції Китаю на нараді монголів
було запропоновано винищити людність і її землю перетворити на па «овиська.
Пю пропозицію виконало багато лендлордів горішньої Щотляндії
у своїй власній країні проти своїх власних земляків». (George
Ensor: «An Inquiry concerning the Population of Nations», London 1818,
p. 215, 216).
