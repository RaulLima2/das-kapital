дедалі менше й менше робітників. З другого боку, старий капітал,
що періодично репродукується в новому складі, відштовхує
дедалі більше й більше робітників, що їх він раніше вживав.

3. Прогресивне утворення відносного перелюднення,
або промислової резервної армії\footnote*{
У додатках подаємо уривок з цього параграфа за французьким
виданням. Див. стор. 736. Ред.
}

Акумуляція капіталу, що первісно виступала лише як кількісне
поширення, відбувається, як ми бачили, при безперервній
якісній зміні його складу, при постійному збільшенні його сталої
складової частини коштом змінної.\footnoteA{
Примітка до третього видання. — У власному примірнику Маркса
тут на берегах книги є така увага: «Тут для пізнішого треба зауважити:
коли поширення є лише кількісне, то за більшого або меншого капіталу
в тій самій галузі промисловости зиски відносяться один до одного як
величини авансованих капіталів. Якщо кількісне поширення діє і якісно,
то разом з тим підноситься норма зиску для більшого капіталу». [Ф. Е.].
}

Специфічно капіталістичний спосіб продукції, відповідний
до нього розвиток продуктивної сили праці, спричинена ним
зміна в органічному складі капіталу не тільки йдуть пліч-опліч
з розвитком акумуляції або з зростанням суспільного багатства.
Вони йдуть уперед куди швидше, бо проста акумуляція,
або абсолютне поширення цілого капіталу супроводиться централізацією
його індивідуальних елементів, а технічний переворот
у додатковому капіталі супроводиться технічним переворотом
у первісному капіталі. Отже, з проґресом акумуляції
відношення сталої частини капіталу до змінної, коли воно було
первісно 1: 1, змінюється в 2: 1, 3: 1, 4: 1, 5: 1, 7: 1 і т. д.,
так що із зростанням капіталу на робочу силу замість 1/2 його
загальної вартости перетворюється проґресивно лише 1/3, 1/4,
1/5 1/6 1/8 і т. д., а на засоби продукції, навпаки, — 2/3, 3/4, 4/5,
5/6, 7/8 і т. д. Отже, через те, що попит на працю визначається не
розміром цілого капіталу, а розміром його змінної складової
частини, то із зростанням цілого капіталу попит на працю проґресивно
падає, замість, як це раніше припускалося, більшати
пропорційно до цього зростання. Він падає відносно проти величини
цілого капіталу і в щораз швидшій проґресії із зростанням
цієї величини. Щоправда, із зростанням цілого капіталу зростає
і його змінна складова частина, або додавана до нього робоча
сила, але зростає вона в щораз меншій пропорції. Павзи, що протягом
їх акумуляція діє як просте поширення продукції на
даній технічній основі, скорочуються. Але мало того, що прискорена
в чимраз більшій проґресії акумуляція цілого капіталу
потрібна, щоб поглинути певне додаткове число робітників або
навіть щоб дати заняття тим робітникам, які вже функціонують
та в наслідок постійної метаморфози старого капіталу втрачають
роботу. Це щораз більше зростання акумуляції й централізації
ще й собі перетворюється на джерело нових змін у складі капі-