Отже, загальна кількість грошей, що функціонують протягом
даного періоду часу як засоби циркуляції, визначається, з одного
боку, сумою цін усіх товарів, що циркулюють, а з другого боку —
повільнішим або швидшим потоком їхніх протилежних процесів
циркуляції, від якого залежить, яку частину з тієї суми цін можна
зреалізувати за допомогою тих самих монет. Але сума цін товарів
залежить так від маси, як і від ціни кожного роду товару. Та ці
три фактори: рух цін, маса товарів, що циркулюють, і, нарешті,
швидкість обігу грошей можуть змінятися в різних напрямах
і в різних пропорціях; отже, сума цін, що має бути зреалізована,
а тому й зумовлювана нею маса засобів циркуляції, може таким
чином пророблювати численні комбінації. Ми зазначимо тут лише
ті, що найважливіші в історії товарових цін.

що немає грошей?... Фармер скаржиться... він думає, що коли б у країні
було більше грошей, він дістав би добру ціну за свої товари... Отже, він,
здається, потребує не грошей, а доброї ціни за своє збіжжя й за свою худобу,
що їх він хоче продати, але не може... Чому він не може одержати
доброї ціни?.. 1) Або тому, що в країні є забагато збіжжя або худоби,
так що більшість людей, що приходять на ринок, мають потребу продавати,
так само як він, і лише меншість має потребу купувати, 2) або
тому, що зменшився звичайний вивіз за кордон... 3) або тому, що падає
споживання, коли люди, приміром, через зубожіння, не можуть витрачати
на предмети споживання стільки, скільки витрачали раніш. Отже,
не збільшення кількости грошей допоможе фармерові продати свої продукти,
а усунення однієї з цих трьох причин, які дійсно натискають на
ринок... Так само потребують грошей купець і крамар, тобто вони не можуть
збути своїх товарів через застій на ринку... нація досягає найбільшого
розвитку тоді, коли багатства швидко переходять із рук до рук».
(«Money being... the common measure of buying and selling, every body
who has anything to sell, and cannot procure chapmen for it, is presently
apt to think, that want of money in the kingdom, or country, is the cause
why his goods do not do off; and so, want of money is the common cry;
which is a great mistake... What do these people want, who cry out for
money?.. The Farmer complains... he thinks that were more money in the
country, he should have a price for his goods... Then it seems money is not
his want, but a Price for his corn and cattle, which he would sell, but cannot...
why cannot he get a price?.. 1) Either there is too much corn and cattle
in the country, so that most who come to market have need of selling,
as he has, and few of buying: or, 2) There wants the usual vent abroad by
Transportation... Or, 3) The consumption fails, as when men, by reason of
poverty, do not spend so much in their houses as formerly they did, wherefore
it is not the increase of specifick money, which would at all advance
the farmer’s goods, but the removal of any of these three causes, which
do truly keep, down the market... The merchant and shopkeeper want money
in the same manner, that is, they want a vent for the goods they deal in,
by reason that the markets fail... a nation never thrives better, than when
riches are tost from hand to hand»). (Sir Dudley North: «Discourses upon
Trade», London 1691, p. 11—15 passim). Всі шахрайства Гереншванда
сходять на те, що суперечності, які виникають із природи товару й тому
виявляються в циркуляції товарів, можна усунути через збільшення
засобів циркуляції. З популярної ілюзії, яка застої в процесі продукції
і процесі циркуляції приписує бракові засобів циркуляції, зрештою,
ніяк не випливає зворотне, а саме, що дійсний брак засобів циркуляції
в наслідок, приміром, офіціяльних махінацій з «regulation of
currency» * не може із свого боку викликати застоїв.

* — реґулювання засобів обігу. Ред.
