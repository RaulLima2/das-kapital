\index{i}{0603}  %% посилання на сторінку оригінального видання
З попередньої таблиці маємо такий результат:
\begin{center}
  \newcolumntype{Y}{>{\centering\arraybackslash}X}
  \noindent\begin{tabularx}{\textwidth}{Y Y Y Y}
      Коні & Рогата худоба & Вівці & Свині \\
      Абсолютне зменшення & Абсолютне зменшення & Абсолютне збільшення & Абсолютне збільшення \\
      \num{72.358} & \num{116.626} & \num{146.608} & \num{28.819}\hang{l}{\footnote{Результат був би ще несприятливіший, коли б ми пішли ще далі
      назад. Так, овець 1865~\abbr{р.} було \num{3.688.742}, а року 1856 — \num{3.694.294}; свиней
      року 1865 було \num{1.299.893}, а року 1858 — \num{1.409.833}.}}\\

  \end{tabularx}
\end{center}

Звернімось тепер до рільництва, що постачає засоби існування
для худоби й людей. У дальшій таблиці обчислено збільшення
або зменшення для кожного окремого року порівняно з безпосередньо
попереднім роком. Збіжжя обіймає пшеницю, овес,
ячмінь, жито, квасолю й горох, зеленина — картоплю, турнепс,
білі й червоні буряки, капусту, моркву, пастернак, вику й~\abbr{т. ін.}

\setlength{\tabcolsep}{2pt}

\begin{table}[H]\small
  \settowidth\rotheadsize{шення}

  \begin{flushright}
    \emph{Таблиця В}
  \end{flushright}
  \caption*{Збільшення або зменшення засівної площі й лук (зглядно толок) в акрах}
  \noindent\begin{tabular}{cccccccccc}
  \toprule
    \multirowcell{2}{\makecell{Роки}} &
    Збіжжя &
    \multicolumn{2}{c}{Зеленина} &
    \multicolumn{2}{c}{\makecell{Луки й ко-\\нюшина}} &
    \multicolumn{2}{c}{Льон} &
    \multicolumn{2}{c}{\makecell{Загальна кількість \\ землі для рільни-\\цтва і скотарства}} \\

    \cmidrule(l){2-2}
    \cmidrule(l){3-4}
    \cmidrule(l){5-6}
    \cmidrule(l){7-8}
    \cmidrule(l){9-10}
   &
  \makecell{Змен-\\шення} &
  \makecell{Змен-\\шення} &
  \makecell{Збіль-\\шення} &
  \makecell{Змен-\\шення} &
  \makecell{Збіль-\\шення} &
  \makecell{Змен-\\шення} &
  \makecell{Збіль-\\шення} &
  \makecell{Змен-\\шення} &
  \makecell{Збіль-\\шення} \\
  \midrule
    1861 & \phantom{0}\num{15.701} & \phantom{0}\num{36.974} & \textemdash{} & \num{47.969} & \textemdash{} & \textemdash{} &  \phantom{0}\num{19.271} & \phantom{0}\num{81.873} & \textemdash{} \\
    
    1862 & \phantom{0}\num{72.734} & \phantom{0}\num{74.785} & \textemdash{} & \textemdash{} &  \phantom{0}\num{6.623} & \textemdash{} & \phantom{00}\num{2.055} & \num{138.841} & \textemdash{} \\
    
    1863 & \num{144.719} & \phantom{0}\num{19.358} & \textemdash{} & \textemdash{} &  \phantom{0}\num{7.724} & \textemdash{} & \phantom{0}\num{63.922} & \phantom{0}\num{92.431} & \textemdash{} \\
    
    1864 & \num{122.437} & \phantom{00}\num{2.317} & \textemdash{} & \textemdash{} & \num{47.486} & \textemdash{} & \phantom{0}\num{87.761} & \textemdash{} & \num{10.493} \\
    
    1865 & \phantom{0}\num{72.450} & \textemdash{} & \num{25.241} & \textemdash{} & \num{68.970} & \num{50.159} & \textemdash{} & \phantom{0}\num{28.218} & \textemdash{} \\
    
    1861\textemdash{}1865 & \num{428.041} & \num{107.984} & \textemdash{} & \textemdash{} & \num{82.834} & \textemdash{} & \num{122.850} & \num{330.860} & \textemdash{} \\
  \end{tabular}
\end{table}

\setlength{\tabcolsep}{\tabcolsepdef}

В 1865 році в рубриці «луки» сталося збільшення на \num{127.470} акрів,
головно через те, що площа в рубриці «необроблена пуста
земля й торфовища» зменшилась на \num{101.543} акри. Коли порівняти
1865 рік з 1864, то зменшення збіжжя становитиме \num{246.667}
квартерів, із них пшениці — \num{48.999} квартерів, вівса — \num{166.605} квартерів,
ячменю — \num{29.982} квартери й~\abbr{т. ін.}; зменшення кількости
картоплі, хоч оброблювана під нею площа в році 1865 і збільшилась,
становило \num{446.398} тонн і~\abbr{т. ін.} (див. таблицю \emph{С}).

Від руху людности й рільничої продукції Ірляндії перейдімо
до руху в гаманці її лендлордів, великих фармерів і промислових
капіталістів. Він відбивається у зменшенні і збільшенні прибуткового
податку. Щоб зрозуміти дальшу таблицю \emph{D}, треба зауважити,
що рубрика D (зиски, за винятком зисків фармерів) обіймає
і так звані «професійні» зиски, тобто доходи адвокатів,
лікарів і~\abbr{т. ін.}, а рубрики С й Е, які тут не перелічені окремо,
обіймають і доходи урядовців, офіцерів, державних синекуристів,
держців державних цінних паперів і~\abbr{т. д.}
