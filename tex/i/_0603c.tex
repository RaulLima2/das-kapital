\index{i}{0603}  %% посилання на сторінку оригінального видання
З попередньої таблиці маємо такий результат:

Коні                         Рогата худоба              Вівці                       Свині
Абсолютне             Абсолютне             Абсолютне            Абсолютне
зменшення            зменшення           збільшення          збільшення
72.358                         116.626                   146.608                     28.819 \footnote{
Результат був би ще несприятливіший, коли б ми пішли ще далі
назад. Так, овець 1865 р. було 3.688.742, а року 1856 — 3.694.294; свиней
року 1865 було 1.299.893, а року 1858 — 1.409.833.
}

Звернімось тепер до рільництва, що постачає засоби існування
для худоби й людей. У дальшій таблиці обчислено збільшення
або зменшення для кожного окремого року порівняно з безпосередньо
попереднім роком. Збіжжя обіймає пшеницю, овес,
ячмінь, жито, квасолю й горох, зеленина — картоплю, турнепс,
білі й червоні буряки, капусту, моркву, пастернак, вику й т. ін.

Таблиця В

Збільшення або зменшення засівної площі й лук (зглядно толок) в акрах

Роки    Збіжжя    Зеленина        Луки й конюшина        Льон
Загальна кількість землі для рільництва і скотарства
    Зменшення    Зменшення    Збільшення    Зменшення    Збільшення    Зменшення     Збільшення
Зменшення    Збільшення
1861                15.701      36.974       —        47 969      —            —        19.271
81.873       —
1862                72.734      74.785       —           —        6.623         —          2.055
  138.841     —
1863              144.719      19.358       —          —         7.724         —        63.922
 92.431       —
1864              122.437        2.317       —          —        47.486        —         87.761
    —        10.493
1865                72.450         —         25.241     —       68.970    50.159       —
28.218       —
1861-1865    428.041    107.984      —           —       82.834       —        122.850    330.860
   —

В 1865 році в рубриці «луки» сталося збільшення на 127.470 акрів,
головно через те, що площа в рубриці «необроблена пуста
земля й торфовища» зменшилась на 101.543 акри. Коли порівняти
1865 рік з 1864, то зменшення збіжжя становитиме 246.667
квартерів, із них пшениці — 48.999 квартерів, вівса — 166.605 квартерів,
ячменю — 29.982 квартери й т. ін.; зменшення кількости
картоплі, хоч оброблювана під нею площа в році 1865 і збільшилась,
становило 446.398 тонн і т. ін. (див. таблицю С).

Від руху людности й рільничої продукції Ірляндії перейдімо
до руху в гаманці її лендлордів, великих фармерів і промислових
капіталістів. Він відбивається у зменшенні і збільшенні прибуткового
податку. Щоб зрозуміти дальшу таблицю D, треба зауважити,
що рубрика D (зиски, за винятком зисків фармерів) обіймає
і так звані «професійні» зиски, тобто доходи адвокатів,
лікарів і т. ін., а рубрики С й Е, які тут не перелічені окремо,
обіймають і доходи урядовців, офіцерів, державних синекуристів,
держців державних цінних паперів і т. д.
