\index{i}{0603}  %% посилання на сторінку оригінального видання
З попередньої таблиці маємо такий результат:
\begin{center}
  \newcolumntype{Y}{>{\centering\arraybackslash}X}
  \noindent\begin{tabularx}{\textwidth}{Y Y Y Y}
      Коні & Рогата худоба & Вівці & Свині \\
      Абсолютне зменшення & Абсолютне зменшення & Абсолютне збільшення & Абсолютне збільшення \\
      72.358 & 116.626 & 146.608 & 28.819\hang{l}{\footnote{Результат був би ще несприятливіший, коли б ми пішли ще далі
      назад. Так, овець 1865 р. було 3.688.742, а року 1856 — 3.694.294; свиней
      року 1865 було 1.299.893, а року 1858 — 1.409.833.}}\\

  \end{tabularx}
\end {center}

Звернімось тепер до рільництва, що постачає засоби існування
для худоби й людей. У дальшій таблиці обчислено збільшення
або зменшення для кожного окремого року порівняно з безпосередньо
попереднім роком. Збіжжя обіймає пшеницю, овес,
ячмінь, жито, квасолю й горох, зеленина — картоплю, турнепс,
білі й червоні буряки, капусту, моркву, пастернак, вику й т. ін.

\setlength{\tabcolsep}{2pt}

\begin{table}[h]\small
  \settowidth\rotheadsize{шення}

  \begin{flushright}
    \emph{Таблиця В}
  \end{flushright}
  \caption*{Збільшення або зменшення засівної площі й лук (зглядно толок) в акрах}
  \noindent\begin{tabular}{cccccccccc}
  \toprule
    \multirowcell{2}{\makecell{Роки}} &
    Збіжжя &
    \multicolumn{2}{c}{Зеленина} &
    \multicolumn{2}{c}{\makecell{Луки й ко-\\нюшина}} &
    \multicolumn{2}{c}{Льон} &
    \multicolumn{2}{c}{\makecell{Загальна кількість \\ землі для рільни-\\цтва і скотарства}} \\

    \cmidrule(l){2-2}
    \cmidrule(l){3-4}
    \cmidrule(l){5-6}
    \cmidrule(l){7-8}
    \cmidrule(l){9-10}
   &
  \makecell{Змен-\\шення} &
  \makecell{Змен-\\шення} &
  \makecell{Збіль-\\шення} &
  \makecell{Змен-\\шення} &
  \makecell{Збіль-\\шення} &
  \makecell{Змен-\\шення} &
  \makecell{Збіль-\\шення} &
  \makecell{Змен-\\шення} &
  \makecell{Збіль-\\шення} \\
  \midrule
    1861 & \phantom{0}15.701 & \phantom{0}36.974 & \textemdash{} & 47.969 & \textemdash{} & \textemdash{} &  \phantom{0}19.271 & \phantom{0}81.873 & \textemdash{} \\
    
    1862 & \phantom{0}72.734 & \phantom{0}74.785 & \textemdash{} & \textemdash{} &  \phantom{0}6.623 & \textemdash{} & \phantom{00}2.055 & 138.841 & \textemdash{} \\
    
    1863 & 144.719 & \phantom{0}19.358 & \textemdash{} & \textemdash{} &  \phantom{0}7.724 & \textemdash{} & \phantom{0}63.922 & \phantom{0}92.431 & \textemdash{} \\
    
    1864 & 122.437 & \phantom{00}2.317 & \textemdash{} & \textemdash{} & 47.486 & \textemdash{} & \phantom{0}87.761 & \textemdash{} & 10.493 \\
    
    1865 & \phantom{0}72.450 & \textemdash{} & 25.241 & \textemdash{} & 68.970 & 50.159 & \textemdash{} & \phantom{0}28.218 & \textemdash{} \\
    
    1861\textemdash{}1865 & 428.041 & 107.984 & \textemdash{} & \textemdash{} & 82.834 & \textemdash{} & 122.850 & 330.860 & \textemdash{} \\
  \end{tabular}
\end{table}

\setlength{\tabcolsep}{\tabcolsepdef}

В 1865 році в рубриці «луки» сталося збільшення на 127.470 акрів,
головно через те, що площа в рубриці «необроблена пуста
земля й торфовища» зменшилась на 101.543 акри. Коли порівняти
1865 рік з 1864, то зменшення збіжжя становитиме 246.667
квартерів, із них пшениці — 48.999 квартерів, вівса — 166.605 квартерів,
ячменю — 29.982 квартери й т. ін.; зменшення кількости
картоплі, хоч оброблювана під нею площа в році 1865 і збільшилась,
становило 446.398 тонн і т. ін. (див. таблицю \emph{С}).

Від руху людности й рільничої продукції Ірляндії перейдімо
до руху в гаманці її лендлордів, великих фармерів і промислових
капіталістів. Він відбивається у зменшенні і збільшенні прибуткового
податку. Щоб зрозуміти дальшу таблицю \emph{D}, треба зауважити,
що рубрика D (зиски, за винятком зисків фармерів) обіймає
і так звані «професійні» зиски, тобто доходи адвокатів,
лікарів і т. ін., а рубрики С й Е, які тут не перелічені окремо,
обіймають і доходи урядовців, офіцерів, державних синекуристів,
держців державних цінних паперів і т. д.
