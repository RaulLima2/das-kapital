приємливости і знизити ціну праці в мануфактурах», наш вірний
Екарт капіталу пропонує випробований засіб, а саме: таких
робітників, які вдаються до публічної добродійности, словом,
павперів, замикати до «ідеального робітного дому (an idéal Workhouse).
«Такий дім треба зробити домом жаху» (House of Тerror).127
В цьому «домі жаху», цьому «ідеалі робітного дому», праця повинна
тривати 14 годин на день, залічуючи сюди, однак, відповідний час
на їжу, так, щоб на працю лишалося повних 12 годин».128

Дванадцять годин праці щодня в «Ideal-Workhouse», в домі
жаху року 1770! Шістдесят три роки пізніш, 1833 р., коли англійський
парлямент у чотирьох галузях промисловости зменшив
робочий день дітей від 13 до 18-літнього віку до 12 повних робочих
годин, здавалось, англійській промисловості прийшов останній
час! 1852 р., коли Л. Бонапарте, щоб забезпечити собі підтримку
буржуазії, спробував був зробити замах на встановлений
законом робочий день, французька робітнича людність одноголосно
кричала: «Закон, що обмежує робочий день на 12 годин, —
це єдине благо, яке лишилося нам із законодавства республіки!» 129
В Цюріху працю дітей старших за 10 років обмежено 12 годинами;
в Аарґаві 1862 р. працю дітей від 13 до 16 років скорочено з 12\sfrac{1}{2}
до 12 годин; в Австрії 1860 р. для дітей від 14 до 16 років також
скорочено її до 12 годин.130 Який проґрес від 1770 р. з «exultation»,
вигукнув би Маколей!

«Дім жаху» для павперів, про який капіталістична душа
мріяла ще 1770 р., постав декілька років пізніш у формі велетенського
«робітного дому» для самих мануфактурних робітників.
Він звався фабрикою. І цього разу ідеал зблід перед дійсністю,

127 Там же, р. 242: «Такий ідеальний робітний дім треба зробити «домом
жаху», а не притулком для бідних, де вони дістають добру їжу, теплий
і порядний одяг і дуже мало працюють» («Such ideal workhouse must be
made a «House of Terror», and not an asylum for the poor, where they are
to be plentifully fed, warmly and decently clothed, and where they do but
а little work»).

128 «In htis ideal workhouse the poor shall work 14 hours in a day
allowing proper time for meals, in such manner that there shall remain
12 hours of neat labour». (Там же). «Французи, — каже він, — сміються
з наших ентузіястичних ідей про волю». (Там же, стор. 78).

129 «Вони особливо виступали проти праці, що триває більш, ніж
12 годин на день, тому що закон, який установив такий робочий день, є.
єдине благо, яке лишилося їм від законодавства республіки» («They especially
objected to work beyond the 12 hours per day. because the law which
fixed those hours is the only good which remains to them of the legislation
of the Republic»). («Reports of Insp. of Fact. 31 st October 1856»,
p. 80). Французький закон з 5 вересня 1850 р. про дванадцятигодинний
робочий день — це змінене на користь буржуазії видання декрету тимчасового
уряду з 2 березня 1848 р.; цей закон поширюється однаково на
всі майстерні. Перед цим законом робочий день у Франції був необмежений.
Він тривав на фабриках 14, 15 і більше годин. Див.; «Des classes
ouvrières en France, pendant l'année 1848. Par M. Blanqui». Панові Блянкі,
економістові, a не революціонерові, уряд доручив зробити дослідженнястановища
робітників.

130 І в справі урегулювання робочого дня Бельґія виявляє себе зразковою
буржуазною державою. Лорд Говард де Велден, англійський посол
