5. Боротьба за нормальний робочий день. Примусові закони про
здовження робочого дня від середини XIV до кінця XVII віку

«Що таке робочий день?» Який великий той час, що протягом
його капітал може споживати робочу силу, денну вартість якої
він оплачує? Як далеко можна здовжувати робочий день поза
межі робочого часу, доконечного для репродукції самої робочої
сили? На ці запити, як ми бачили, капітал відповідає: робочий
день налічує повних 24 години на добу, за винятком небагатьох
годин відпочинку, без якого робоча сила абсолютно не в стані
відновити свою службу. Насамперед само собою зрозуміло, що
робітник ціле своє життя є не що інше, як тільки робоча сила,
і що тому цілий вільний його час з природи й на основі права є
робочий час, отже, належить процесові самозростання капіталу.
Час, потрібний для освіти людини, для її інтелектуального розвитку,
для виконання соціальних функцій, для стосунків з приятелями,
для вільної гри фізичних і інтелектуальних життєвих
сил, навіть для святкування неділі, хоча б і в країні святкувальників
суботи 104 — все це є чиста нісенітниця! Але в своєму безмірному,
сліпому прагненні, у своїй вовчій ненажерливій жадобі
до додаткової праці, капітал переступає не лише моральні,
але й суто фізичні максимальні межі робочого дня. Він узурпує
час, потрібний для зросту, розвитку й здорового збереження
тіла. Він грабує час, потрібний на споживання свіжого повітря
й сонячного світла. Він зменшує час на їжу і по змозі приєднує
його до самого процесу продукції, так що харч додається робітникові
як простому засобові продукції, як паровому казанові —
вугілля і машинам — мастиво або олію. Здоровий сон, потрібний,
щоб акумулювати, поновити й відсвіжити життєву силу, капітал
зводить на стільки годин заціпеніння, скільки неодмінно потрібно,
щоб оживити абсолютно виснажений організм. Не нормальне

labour»), просто неймовірна». (Там же, стор. XLIII і XLIV). Тимчасом
«повний самовідречення» капітал скляної промисловості, повертаючись
пізно вночі з клюбу додому й похитуючись від портвайну, по-ідіотичному
мугикає собі під носом: «Britons never, never shall be slaves!» *

104 Приміром, в Англії ще й тепер подекуди засуджують на ув’язнення
робітника за те, що він, працюючи в садку перед своїм домом, порушує
святість суботи. Цей самий робітник дістає кару за зламання контракту,
коли не піде в неділю, хоча б і з релігійних мотивів, на фабрику металю,
паперу або скла. Ортодоксальний парлямент глухий на зневажання святости
суботи, коли воно трапляється у «процесі зростання вартости»
капіталу. В одному меморіялі (серпень 1863 р.), де лондонські поденники
з крамниць, що торгували рибою і птицею, вимагають скасувати недільну
працю, сказано, що їхня праця триває протягом перших 6 днів тижня
пересічно по 15 годин щоденно, а в неділю — 8—10 годин. Із того ж меморіялу
видно, що до цієї «недільної праці» заохочує саме вибагливо
витончена обжерливість аристократичних лицемірів із Exeter Hall.
Ці «святуни», такі ревні «in cute curanda»,** виявляють свою християнську
душу в тій покорі, з якою зносять надмірну працю, злидні й голод
третіх осіб. Obsequium ventris istis (робітникам) perniciosus est.***

* Ніколи, ніколи британці не будуть рабами. Ред.

** — в піклуванні про себе. Ред.

*** Догоджати череву є для них (робітників) згубна річ. Ред.
