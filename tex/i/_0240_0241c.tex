\index{i}{0240}  %% посилання на сторінку оригінального видання
Розділ дев'ятий

Норма й маса додаткової вартости

В цьому розділі, як і раніш, ми припускаємо, що вартість
робочої сили, отже й частина робочого дня, доконечна для репродукції
або збереження робочої сили, є дана, стала величина.

Отже, за такого припущення разом з нормою додаткової вартости
є дана і та її маса, яку поодинокий робітник дає капіталістові
за певний період часу. Коли, приміром, доконечна праця
становить 6 годин на день, що виражаються в кількості золота
в 3 шилінґи = 1 талярові, то 1 таляр є денна вартість однієї
робочої сили, або капітальна вартість, авансована на купівлю
однієї робочої сили. Коли, далі, норма додаткової вартости становить
100\%, то цей змінний капітал у 1 таляр продукує масу
додаткової вартости в 1 таляр, або робітник дає денно масу додаткової
праці в 6 годин.\footnote*{
У французькому виданні цей абзац зредаґовано так: «Припустімо,
що денна пересічна вартість робочої сили становить 3 шилінґи, або 1 таляр,
і що потрібно 6 годин на день, щоб її репродукувати. Отже, шоб
купити таку робочу силу, капіталіст мусить авансувати 1 таляр. Скільки
додаткової вартости дасть йому цей таляр? Це залежить від норми додаткової
вартости. Коли вона становить 50\%, то додаткова вартість становитиме
1/2 таляра, репрезентуючи 3 години додаткової праці: коли ж вона
становить 100\%, то додаткова вартість підвищиться до 1 таляра, репрезентуючи
6 годин додаткової праці. Отже, за даної вартости робочої сили
норма додаткової вартости визначає ту суму додаткової вартости, що її
продукує поодинокий робітник». («Le Capital etc.», v. I, ch. XI, p. 131). Peд.
}

Але змінний капітал є грошовий вираз загальної вартости всіх
робочих сил, що їх капіталіст одночасно вживає. Отже, вартість
змінного капіталу дорівнює пересічній вартості однієї робочої
сили, помноженій на число вживаних робочих сил. Отже, за даної
вартости робочої сили величина змінного капіталу є просто пропорційна
до числа одночасно вживаних робітників. Таким чином,
коли денна вартість однієї робочої сили дорівнює 1 талярові, то
треба авансувати капітал у 100 талярів, щоб експлуатувати денно
100 робочих сил, і капітал у n талярів, щоб експлуатувати n робочих
сил.

Так само, коли змінний капітал в 1 таляр, денна вартість однієї
робочої сили, продукує денну додаткову вартість в 1 таляр, то
змінний капітал у 100 талярів продукує денну додаткову вартість
у 100, а капітал в n талярів — денну додаткову вартість в 1 таляр × n.
Отже, маса випродукованої додаткової вартости дорівнює додатковій
вартості, яку дає робочий день поодинокого робітника,
помноженій на число вживаних робітників. А що далі масу додаткової
вартости, яку продукує поодинокий робітник, за даної вартости
робочої сили, визначає норма додаткової вартости, то звідси
випливає такий перший закон: маса випродукованої додаткової
вартости дорівнює величині авансованого змінного капіталу, помноженій
на норму додаткової вартости, або вона визначається
складним відношенням між числом робочих сил, одночасно експлуатованих
\index{i}{0241}  %% посилання на сторінку оригінального видання
тим самим капіталістом, і ступенем експлуатації
поодинокої робочої сили.\footnote*{
У французькому виданні цей закон подано так: «Сума додаткової
вартости, випродукована змінним капіталом, дорівнює вартості цього
авансованого капіталу, помноженій на норму додаткової вартости, абож
вона дорівнює вартості робочої сили, помноженій на ступінь її експлуатації,
помноженій на число одночасно вживаних робочих сил». («Le Capital
etc.», v. I, ch. XI, p. 131). Peд.
}

Отже, коли ми масу додаткової вартости назвемо М, додаткову
вартість, що її пересічно дає поодинокий робітник за день, — m,
змінний капітал, щоденно авансовуваний на купівлю однієї робочої
сили, — v, цілу суму змінного капіталу, — V, вартість пересічної
робочої сили — k, ступінь її експлуатації — a'/a (додаткова праця/докончена праця),
а число вживаних робітників — n, то матимемо:

M = m/v × V k × a'/a × n

Ми завжди припускаємо, що не лише вартість пересічної робочої
сили є стала, але що і вживані капіталістом робітники зведені
на пересічного робітника. Бувають виняткові випадки, коли
випродукована додаткова вартість зростає не пропорційно до числа
експлуатованих робітників, але тоді й вартість робочої сили не
лишається сталою.

[Здобуток не змінює своєї числової величини, коли множники
його одночасно змінюються у зворотному напрямі»].\footnote*{
Заведене у прямі дужки ми беремо з французького видання. Ред.
}

Тим то в продукції певної маси додаткової вартости зменшення
одного фактора може бути компенсоване збільшенням другого.
Коли змінний капітал меншає, а одночасно в тій самій пропорції
більшає норма додаткової вартости, то маса продукованої додаткової
вартости лишається незмінна. Коли, в умовах попередніх
припущень, капіталіст мусить авансувати 100 талярів, щоб денно
експлуатувати 100 робітників, і коли норма додаткової вартости
становить 50\%, то цей змінний капітал у 100 талярів дає додаткову
вартість у 50 талярів, або ж у 100 х 3 робочі години. Коли
норма додаткової вартости подвоюється, або робочий день здовжується
не від 6 до 9, а від 6 до 12 годин, то зменшений наполовину
змінний капітал у 50 талярів дає так само додаткову
вартість у 50 талярів, або у 50 х 6 робочих годин. Отже,
зменшення змінного капіталу може вирівнюватися пропорційним
підвищенням ступеня експлуатації робочої сили, або зменшення
числа вживаних робітників вирівнюється пропорційним здовженням
робочого дня. Отже, подання праці, яке може вимушувати
\parbreak{}  %% абзац продовжується на наступній сторінці
