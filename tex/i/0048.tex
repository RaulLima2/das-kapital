рами і в унутрішньому житті громади. Їхні кількісні мінові відношення
спочатку цілком випадкові. Вимінними вони є в наслідок
вольового акту їхніх посідачів відчужити їх один одному. Тим часом
потреба в чужих предметах споживання помалу зміцнюється.
Постійне повторювання обміну робить його реґулярним
суспільним процесом. Тим то з часом принаймні частина продукту
праці мусить навмисно продукуватись на обмін. Від цього моменту
фіксується, з одного боку, відокремлення корисности речей
для безпосередньої потреби від корисности їх для обміну. Їхня
споживна вартість відокремлюється від їхньої мінової вартости.
З другого боку, кількісне відношення, в якому вони обмінюються,
стає залежне від самої їхньої продукції. Звичай фіксує їх як
величини вартости.

В безпосередньому обміні продуктів кожний товар є безпосередній
засіб обміну для свого посідача, еквівалент для свого
непосідача, однак лише остільки, оскільки він для останнього
є споживна вартість. Отже, предмет обміну не набуває ще жодної
форми вартости, незалежної від його власної споживної вартости
або від індивідуальних потреб обмінювачів. Доконечність цієї
форми розвивається разом із зростом числа й різноманітности
товарів, що вступають у процес обміну. Завдання виникає одночасно
із засобами його розв’язання. Стосунки, за яких посідачі
товарів обмінюють свої власні продукти на різні інші продукти
й порівнюють їх між собою, не можуть ніколи відбуватися без
того, щоб різні товари різних товаропосідачів у межах їхніх
стосунків не обмінювались на той самий третій рід товару й не
порівнювались з ними як вартості. Такий третій товар, стаючи
еквівалентом для різних інших товарів, безпосередньо набирає,
хоч і у вузьких межах, загальної або суспільної еквівалентної
форми. Ця загальна еквівалентна форма постає й зникає разом
із тим минущим суспільним контактом, який її покликав до життя.
Навпереміну й не на довгий час припадає вона цьому або тому
товарові. Але з розвитком обміну товарів вона міцно зростається
виключно з особливими родами товарів або кристалізується в
грошову форму. З яким саме родом товару вона зростається, це
спочатку залежить од випадку. Однак, взагалі і в цілому тут
вирішують справу дві обставини. Грошова форма зростається
або з тими найважливішими предметами довозу з чужих країн,
які дійсно є первісна форма виявлення мінової вартости тубільних
продуктів, або з предметом споживання, що становить головний
елемент тубільного відчужуваного майна, як, приміром, худоба.
Кочові народи перші розвивають грошову форму, бо все їхнє
майно перебуває в рухомій, отже, безпосередньо відчужуваній
формі, і ще й тому, що ввесь лад їхнього життя завжди ставить
їх у контакт з чужими громадами, а тому й заохочує до обміну
продуктів. Люди часто робили саму людину у формі раба первісним
грошовим матеріялом, але землю — ніколи. Така ідея могла
виникнути лише в уже розвинутому буржуазному суспільстві.
Вона з’явилася в останній третині ХVІІ віку, а спробу здійснити
