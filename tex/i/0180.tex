Порівняння ненажерливої жадоби до додаткової праці в дунайських
князівствах з такою самою жадобою на англійських
фабриках має особливий інтерес, бо додаткова праця панщини
має самостійну, почуттєво-сприйману форму.

Припустімо, що робочий день становить 6 годин доконечної
й 6 годин додаткової праці. Таким чином вільний робітник
постачає капіталістові щотижня 6 × 6, або 36 годин додаткової
праці. Це те саме, як коли б він працював 3 дні на тиждень для
себе й 3 дні на тиждень задурно для капіталіста. Але це не впадає
в очі. Додаткова праця й доконечна праця розпливаються одна
в одній. Тому я міг би висловити те саме відношення, приміром,
і таким чином, що робітник протягом кожної хвилини працює
30 секунд для себе й 30 секунд для капіталіста й т. ін. Інакше
стоїть справа з панщиною. Доконечна праця, яку виконує, приміром,
волоський селянин, щоб утримати себе самого, просторово
відокремлена від його додаткової праці на боярина. Одну працю
він виконує на своєму власному полі, другу — на панському
маєтку. Отже, обидві частини робочого часу існують самостійно
одна побіч однієї. У формі панщини додаткова праця точно відокремлена
від доконечної праці. Ці різні форми виявлення, очевидно,
нічого не змінюють у кількісному відношенні між додатковою
працею і доконечною працею. Три дні додаткової праці на
тиждень завжди лишаються трьома днями такої праці, яка не
створює жодного еквіваленту для самого робітника, хоч вона
називатиметься кріпацькою, хоч найманою працею. Однак у
капіталіста ненажерлива спрага до додаткової праці виявляється
у прагненні до безмірного подовження робочого дня, у боярина
простіш — у безпосередній гонитві за панщинними днями.44

В дунайських князівствах панщинна праця була сполучена
з натуральними рентами та всякими іншими атрибутами кріпацтва;
але вона становила значну данину, яку платилося панівній
клясі. За подібних умов панщинна праця рідко випливає з
кріпацтва, навпаки, кріпацтво здебільшого випливає з панщинної
праці.\footnoteA{
[Примітка до третього видання. Це має силу і щодо Німеччини,
особливо щодо Прусії на сході від Ельби. В XV віці німецький селянин
хоч і підлягав майже всюди деяким повинностям у продуктах і праці, але
зрештою, принаймні фактично був вільною людиною. Німецьких колоністів
у Бранденбурзі, Померанії, Шлезьку та східній Прусії навіть юридично
визнавано вільними. Перемога шляхти в селянській війні поклала
цьому кінець. Не лише переможені селяни південної Німеччини стали
знов кріпаками, але вже від половини XVI століття вільних селян східньої
Прусії, Бранденбурґу, Померанії й Шлезьку, а незабаром і Шлезвіґ-Гольштайну
понижено до стану кріпаків. (Maurer: «Geschichte
der Fronhöfe, der Bauernhöfe und der Hofverfassung in Deutschland», Erlangen
\sfrac{1862}{63}, Bd. IV. — Meitzen: «Der Boden und die landwirtschaftlichen
Verhältnisse des preußischen Staates nach dem Gebietsumfange von
1866», Berlin 1873. — Hansen «Leibeigenschaft in Schleswig-Holstein».) —
F. E.].
} Так було в румунських провінціях. Їхній первіс-

44    Дальші рядки стосуються становища румунських провінцій, як
воно було перед переворотом від часів кримської війни.