\parcont{}  %% абзац починається на попередній сторінці
\index{i}{0472}  %% посилання на сторінку оригінального видання
понижують у тому самому відношенні, в якому зростає число
штук, випродукованих протягом того самого часу,\footnote{
«Продуктивну силу його прядільної машини точно вимірюють,
і норму плати за працю, виконану за допомогою машини, понижують
із зростом її продкутивної сили, хоч не в такій самій пропорції» («The
productive power of his spinning machine is accurately measured, and the
rate of pay for work done with it decreases with, though not as the increase
of its productive power»). (\emph{Ure}: «Philosophy of Manufacture», p. 317).
Останній апологетичний зворот Юр сам знову касує. Він визнає, що при
здовженні мюлі jenny, наприклад, деяке збільшення Праці випливає
з того здовження. Отже, праця меншає не в тій самій мірі, в якій зростає
її продуктивність. Далі: «Через таке збільшення продуктивна сила машини
зростає на одну п’яту. В цьому випадку прядун уже не одержує
за свою працю тієї плати, яку він одержував раніш; але через те, що його
плата зменшується не в тій самій пропорції, тобто не на п’ятину, то це
вдосконалення машини збільшує його дохід за те саме число годин його
роботи» — але, але — «вищесказане потребує деякого виправлення\dots{}
прядун із своїх додаткових шістьох пенсів має зробити додаткові витрати
на збільшення числа малолітніх помічників, яке супроводиться витисненням
частини дорослих робітників» («By this increase the productive power
of the machine will be augmented one-fifth. When this event happens,
the spinner will not be paid at the same rate for work done as he was before»
but as that rate will not be diminished in the ratio of one-fifth, the improvement
will augment his money-earnings for any given membero! hours’work
— the foregoing statement requires a certain modification\dots{} the spinner
has to pay something additional for juvenile aid out of his additional
sixpence, accompanied by displacing a portion of adults») (там же, стор. 320,
321), а це ніяким чином не має тенденції підвищувати заробітну плату.
} отже, в тому
самому відношенні, в якому меншає робочий час, витрачуваний
на ту саму штуку. Ця зміна відштучної плати, хоч вона є суто
номінальна, викликає постійні бої між капіталістом і робітником:
або тому, що капіталіст використовує це як привід»,
щоб дійсно знизити ціну праці, або тому, що підвищення продуктивної
сили праці супроводиться підвищенням її інтенсивности,
абож тому, що зовнішню видимість відштучної плати
робітник приймає серйозно, вважаючи, що йому платять за його
продукт, а не за його робочу силу, і тому опирається такому
зниженню заробітної плати, якому не відповідає зниження продажної
ціни товару. «Робітники пильно стежать за ціною сировинного
матеріялу й ціною фабрикованих продуктів і таким чином
можуть точно визначити зиск своїх хазяїнів».\footnote{
\emph{Н. Fawcell}: «The Economic Position of the British Labourer».
Cambridge and London 1865, p. 178.
} Такі претенсії
капітал по праву відкидає як грубу помилку щодо природи
заробітної плати.\footnote{
У лондонському «Standard’i» з 26 жовтня 1861 p. находимо звіт
про процес фірми Джон Брайт і К° перед рочдельським магістратом
«проти1 представників тред-юньйону килимарів за залякуіання»
(«to prosecute for intimidation the agents of the Carpet Weavers Trades
Union»). «Фірма Брайт завела нові машини, що мали виготовляти 240 ярдів
килимів за той самий час і з тією самою кількістю праці (І), яких
раніш потрібно було на виготовлення 160 ярдів. Робітники не мали жодного
права вимагати будь-яку частину того зиску, який створювався в наслідок
витрати капіталу їхніх підприємців на механічні поліпшення. Тому
пани Брайт запропонували понизити заробітну плату з 1 пенса на
1 пенс за ярд, при чому доходи робітників за виконану працю вони лишали
цілком такими, як і раніш. Але це було номінальне зниження, про яке
робітників — як запевняють — не було заздалегідь попереджено». («Bright’s
partners had introduced new machinery which would turn out 240 yards
of carpet in the time and with the labour (!) previously required to produce
160 yards. The workmen had no claim whatewer to share in the profits
made by the investment of their employer’s capital in mechanical improvements.
Accordingly, Messrs. Bright proposed to lower the rate of pay from
1\sfrac{1}{2} d. per yard to 1 d., leaving the earnings of the men exactly the. same
as before for the same labour. But there was a nominal reduction, of which
the operatives, it is asserted, had not fair warning before hand»).
} Він обурюється проти цього домагання робітників
\index{i}{0473}  %% посилання на сторінку оригінального видання
оподатковувати проґрес промисловости і просто заявляє,
що продуктивність праці взагалі зовсім не обходить робітника.\footnote{
Тред-юньойни, силкуючись утримати заробітну плату на певному
рівні, домагаються участи в зиску від поліпшених машин! (Який
жах!)\dots{} Вони вимагають вищої заробітної плати на тій підставі, що праця
скорочена\dots{} інакше кажучи, вони прагнуть накласти податки на промислові
поліпшення». («On Combination of Trades». New Edit, London
1834, p. 42).
}

\section{Національні відмінності в заробітній платі}

У п’ятнадцятому розділі ми розглядали різноманітні комбінації,
що їх може спричинити зміна абсолютної або відносної
(тобто порівняно з додатковою вартістю) величини вартости робочої
сили, тим часом як, з другого боку, кількість засобів існування,
в яких реалізується ціна робочої сили, і собі може пророблювати
рухи, незалежні\footnote{
«Не точно було б сказати, що заробітна плата (мова тут іде про
її ціну) зросла, якщо вона дає змогу купити більшу кількість якогось
дешевшого продукту» («Ц is not accurate to say that wages are increased,
because they purchase more of a cheaper article»). (\emph{David Buchanan}
у його виданні «Wealth of Nations» А. Сміса, 1814, т. I, стop. 417,
примітка).
} або відмінні від зміни цієї ціни. Як уже зауважено,
через просте перетворення вартости, зглядно ціни робочої
сили на екзотеричну форму заробітної плати, всі подані там
закони перетворюються на закони руху заробітної плати. Те,
що в межах цього руху являє собою змінні комбінац ї, може для
різних країн являти собою одночасні відмінності в національних
заробітних платах. Отже, при порівнянні національних заробітних
плат треба брати на увагу всі моменти, що визначають зміну
у величині вартости робочої сили: ціну і обсяг природних та
історично розвинутих доконечних життєвих потреб, витрати
на виховання робітника, ролю жіночої й дитячої праці, продуктивність
праці, її екстенсивну й інтенсивну величину. Навіть
найповерховіше порівняння потребує насамперед зведення пересічної
денної плати для однакових виробництв у різних країнах
до однаково великих робочих днів. Після такого вирівнепня денних
заробітних плат почасову плату знову треба перевести на
відштучну, бо тільки ця остання є мірило так продуктивности,
як і інтенсивної величини праці.
