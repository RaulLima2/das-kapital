звичайно, більше сировинного матеріялу, веретен і т. ін., ніж
для того, щоб уживати 100 прядунів. Але вартість цих додаткових
засобів продукції може збільшуватися, падати, лишатися незмінною,
бути великою або малою, — це не має значення; вона однаково
лишається без якогобудь впливу на процес зростання вартости,
зумовлюваний тими робочими силами, що пускають ці
засоби в рух. Отже, закон, констатований вище, набирає такої
форми: продуковані різними капіталами маси вартости й додаткової
вартости за даної вартости й однакового ступеня експлуатації
робочої сили є просто пропорційні до величин змінних складових
частин цих капіталів, тобто їхніх складових частин, перетворених
на живу робочу силу.

Цей закон явно суперечить усякому досвідові, побудованому
на видимості. Кожний знає, що прядун бавовни, який, коли
обчислити складові частини всього застосованого капіталу в процентах,
вживає відносно багато сталого й мало змінного капіталу,
з цієї причини добуває собі від нього не менший зиск або не меншу
додаткову вартість, як пекар, що пускає в рух відносно багато
змінного й мало сталого капіталу. Щоб розв’язати цю позірну
суперечність, треба ще багатьох посередніх ланок, як от в елементарній
альґебрі треба багатьох посередніх ланок, щоб зрозуміти,
що \sfrac{0}{0} може репрезентувати дійсну величину. Хоч клясична політична
економія ніколи не формулювала цього закону, але вона
інстинктово цупко тримається його, бо він є доконечний наслідок
закону вартости взагалі. Насильною абстракцією вона силкується
врятувати його від суперечностей виявлення. Пізніш\footnote{
Докладніше про це буде в «Четвертій книзі».\footnote*{
Мова йде про «Теорії додаткової вартости», що їх Маркс мав на
думці видати як «Четверту книгу Капіталу». Ред.
}
}
ми побачимо, як школа Рікарда спіткнулася на цій перешкоді.
Вульґарна економія, яка «справді таки нічого не навчилась»
тут, як і всюди, чваниться позірністю явища, заперечуючи закон
явища. Вона думає, всупереч Спінозі, що «неуцтво є достатня
підстава».

Працю, яку день-у-день пускає в рух сукупний капітал
якогось суспільства, можна розглядати як одним-один робочий
день. Коли, приміром, число робітників — 1 мільйон, а пересічний
робочий день одного робітника становить 10 годин, то суспільний
робочий день складається з 10 мільйонів годин. За даної
довжини цього робочого дня, — хоч межі його визначаються фізичними,
хоч соціяльними умовами, — маса додаткової вартости може
бути збільшена лише через збільшення числа робітників, тобто
робітничої людности. Зріст людности становить тут математичну
межу для продукції додаткової вартости сукупним суспільним
капіталом. Навпаки, за даної величини людности цю межу становить
можливе здовження робочого дня.\footnote{
«Праця, що є вжитий на господарювання час суспільства, являє
собою величину дану, приміром, 10 годин денно на кожного з мільйона
} В найближчому розділі