\parcont{}  %% абзац починається на попередній сторінці
\index{i}{0585}  %% посилання на сторінку оригінального видання
він нічого не має, окрім того, що абсолютно доконечне для його
існування. Він дійшов точки замерзання, що з неї, як із бази,
виходять усі обрахунки фармера. Хай буде що буде, хоч щастя,
хоч нещастя, його це не торкається».\footnote{
Там же, стор. 134.
}

В 1863 р. проведено офіціяльне дослідження умов харчування
і праці злочинців, засуджених на заслання й на громадські примусові
роботи. Результати його подано у двох товстих синіх
книгах. «Старанне порівняння, — читаємо там, між іншим, —
харчу злочинців по в’язницях Англії із харчами павперів по
робітних домах і вільних сільських робітників цієї ж країни
доводить безперечно, що перші дістають далеко кращі харчі,
ніж будь-яка з тих двох інших кляс»,\footnote{
«Report of the Commissioners\dots{} relating to Transpotration and
Penal Servitude», London 1863, p. 42, 50.
} тимчасом як «маса
праці, якої вимагають від засуджених на громадські примусові
роботи, становить приблизно половину того, що виконує звичайний
сільський робітник».\footnote{
Там же, стор. 77. «Memorandum by the Lord Chief Justice».
} Ось деякі характеристичні свідчення
свідків. Джон Сміс, директор едінбурзької в’язниці, свідчить.
№ 5056: «Харчі по англійських в’язницях далеко кращі, ніж
харчі звичайних сільських робітників». № 5075: «Це факт, що
звичайні рільничі робітники Шотландії дуже рідко дістають яке
будь м’ясо». № 3047: «Чи не знаєте ви якихбудь підстав для
того, що злочинців треба годувати далеко краще (much better),
ніж звичайних сільських робітників? — Звичайно, не знаю».
№ 3048: «Чи не вважаєте ви за доцільне робити дальші експерименти,
щоб харчі арештантів, засуджених на громадські примусові
роботи, наблизити до харчів вільних сільських робітників»?\footnote{
Там же, том II, «Evidence».
} «Сільський робітник, — читаємо там, — міг би сказати:
«Я тяжко працюю й не дістаю досить їсти. Коли я був у
в’язниці, я працював не так тяжко, а харчів мав удостачу, і тому
мені краще бути у в’язниці, ніж на волі».\footnote{
Там же, том I, Appendix, стор. 280.
} Із таблиць, доданих
до першого тому звіту, складено таке порівняльне зведення.

Тижнева кількість харчів\footnoteA{
Там же, стор. 274, 275.
}

Азотові  складові  частини (унцій)
Безазотові  складові  частини (унцій)
Мінеральні  складові  частини (унцій)
Загальна  сума (унцій)

Злочинець у портлендській
в’язниці\dotfill                28,95                150,06
4,68                  183,69
Матрос королівської
фльоти\dotfill                29,63                152,91
4,52                   187,06
Солдат\dotfill                25,55                114,49
3,94                   143,98
Каретник (робітник)\dotfill                24,53                162,06                4,23
             190,82
Складач\dotfill                21,24                100,83
3,12                   125,19
Сільський робітник\dotfill                 17,73                 118,06                3,29
              139,08
