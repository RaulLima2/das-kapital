\parcont{}  %% абзац починається на попередній сторінці
\index{i}{0318}  %% посилання на сторінку оригінального видання
машину разом із частиною передатного механізму спільно споживають
численні робочі машини.

Якщо дано ріжницю між вартістю машини і тією частиною
вартости, яку вона переносить на свій денний продукт, то ступінь,
у якому ця частина вартости удорожчує продукт, залежить насамперед
від розміру продукту, так би мовити, від його поверхні.
Пан Бейнс із Блекберна в одній лекції, опублікованій 1858 р.,
обчислює, що «кожна реальна механічна кінська сила\footnoteA{
Примітка до третього видання. — Одна «кінська сила» дорівнює
силі 33.000 футо-фунтів на 1 хвилину, тобто силі, яка за 1 хвилину
підносить 33.000 фунтів на 1 фут (англійський) або 1 фунт на 33.000 футів\footnote*{
В перечисленні на метричні міри одна кінська сила дорівнює силі
75 кілограмо-метрів за 1 секунду, тобто силі, яка за 1 секунду підносить
75 кг на висоту 1 метра. \emph{Ред.}
}. Це й є та кінська сила, про яку вище сказано. Але у звичайній діловій
мові, а також подекуди в цитатах цієї книги розрізняються «номінальні»
й «комерційні», або «індикаторні» кінські сили тієї самої машини. Давня,
або номінальна кінська сила обчислюється виключно з руху толока та
з діяметра циліндра, при чому цілком не звертається уваги на тиск пари
та швидкість толока. Фактично це означає ось що: парова машина має,
наприклад, 50 кінських сил, якщо її пускається в рух з таким самим малим
тиском пари та з такою самою малою швидкістю толока, як це було
за часів Болтона та Ватта. Але два останні фактори від тих часів надзвичайно
зросли. Для того, щоб виміряти ту механічну силу, яку в наші часи
дійсно дає якась машина, винайдено індикатора, що показує тиск
пари. Швидкість толока визначити легко. Таким чином міра «індикаторної»
або «комерційної» кінської сили якоїсь машини — це математична
формула, в якій одночасно береться на увагу діяметр циліндра,
височина піднесення толока, швидкість і тиск пари і в якій цим
показується, скільки разів по 33.000 футо-фунтів машина справді дає
за хвилину. Тим то номінальна кінська сила може в дійсності давати три,
чотири й навіть п’ять індикаторних або дійсних кінських сил. Це — щоб
пояснити різні дальші цитати. — \emph{Ф. Е.}].
} гонить
450 автоматичних веретен mule з пристроями, або 200 веретен
throstle, або 15 ткацьких варстатів для 40-цалевої тканини
разом із пристроями підтягувати ланцюга, шліхтувати й т. д.».
Денні витрати однієї парової кінської сили та зужиткування машин,
що їх вона пускає в рух, у першому випадку розподіляються
на денний продукт 450 веретен mule, у другому — 200 веретен
throstle, у третьому — 15 механічних ткацьких варстатів, так
що через це на одну унцію пряжі або на один метр тканини переноситься
лише якусь незначну частину вартости. Те саме й у
вищенаведеному прикладі з паровим молотом. Через те, що
його щоденне зужиткування, споживання вугілля й т. д. розподіляються
на величезні маси заліза, які він щодня виковує, то
до кожного центнера заліза долучається лише незначна частина
вартости, яка була б дуже велика, якби цим циклопічним інструментом
вбивали дрібні цвяшки.

За даної сфери діяльности робочої машини, отже, за даної
кількости її знарядь або, якщо мова йде про силу, за даного
розміру її знарядь маса продуктів залежить від швидкости, з
якою ця машина оперує, отже, наприклад, від тієї швидкости,
\parbreak{}  %% абзац продовжується на наступній сторінці
