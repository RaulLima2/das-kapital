\index{i}{0198}  %% посилання на сторінку оригінального видання

\subsection{Денна й нічна праця. Система змін}

З погляду процесу зростання вартости сталий капітал, засоби
продукції існують лише на те, щоб вбирати в себе працю і з кожною
краплею праці — відповідну кількість додаткової праці.
Якщо вони цього не роблять, то вже саме їхнє існування становить
для капіталіста неґативну втрату, бо ж протягом того часу,
коли вони лежать без діла, вони репрезентують марно авансований
капітал: утрата ця стає позитивною, скоро тільки на поновлення
перерваної продукції треба додаткових витрат. Здовження
робочого дня поза межі природного дня геть аж у ніч діє
лише як паліятив, лише приблизно заспокоює вампірову спрагу
живої крови праці. Тому присвоєння праці протягом усіх 24 годин
доби є іманентне прагнення капіталістичної продукції. А що
фізично неможливо день і ніч безупинно висисати ті самі робочі
сили, то, щоб перемогти фізичні перешкоди, потрібно зміни робочих
сил, споживаних вдень і вночі, зміни, яка допускає різні
методи, наприклад, її можна зорганізувати так, що частина робочого
персоналу один тиждень працює вдень, а другий тиждень
вночі й~\abbr{т. ін.} Як відомо, така система змін, таке перемінне господарство
панувало за часів юнацького розцвіту англійської бавовняної
промисловости і~\abbr{т. ін.}, і процвітає в наші часи, між
іншим, на бавовняних прядільнях Московської губерні. Як система
цей 24-годинний процес продукції існує ще й нині в багатьох
досі «вільних» галузях промисловости Великобрітанії, між іншим
у домнах, кузнях, вальцювальних та інших металевих мануфактурах
Англії, Велзу й Шотляндії. Робочий процес охоплює тут,
окрім 24 годин шістьох робочих днів, здебільшого ще й 24 години
неділі. Робітники складаються з чоловіків та жінок, дорослих
і дітей обох статей. У віці дітей і молоді є всі переходові ступені
від 8 (в деяких випадках від 6) до 18 років\footnote{
«Children’s Employment Commission». Third Report. London
1864, p. IV, V, VI.
}. У деяких галузях
дівчата й жінки працюють уночі разом із чоловічим персоналом\footnote{
«Так у Стафордшірі, як і в південному Велзі, молоді дівчата й жінки
працюють у кам’яновугляних копальнях та коксувальнях не лише вдень,
а й уночі. У звітах, подаваних до парляменту, не раз зазначувано це
явище як причину великого й загальновідомого лиха. Ці жінки, що працюють
разом із чоловіками й ледве відрізняються від них своїм одягом,
покриті брудом і сажею, наражаються на небезпеку згубити свій моральний
характер через утрату самоповаги, а це є неминучий наслідок їхньої
нежіночої праці». («Both in Staffordshire and in South Wales young girl
and women are employed on the pit banks and on the coke heaps, not only
by day, but also night. This practice has been often noticed in Reports presented
to Parliament, as being attended with great and notorious evils.
These females, employed with the men, hardly distinguished from them
in their dress, and begrimed with dirt and smoke, are exposed to the deterioration
of character arising from the loss of self-respect which can hardly
fail to follow from their unfeminine occupation»). (Там же, 194, p. XXVI.~Порівн. Fourth Report (1865), 61, p. XIII). Те саме й на гутах.
}.
