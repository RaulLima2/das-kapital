\parcont{}  %% абзац починається на попередній сторінці
\index{i}{0329}  %% посилання на сторінку оригінального видання
не містить у собі жодної постанови, яка може забезпечити досягнення
цієї мети. Він не постановляє нічого, хіба тільки те, що
діти на певне число годин (3 години) на день повинні бути замкнені
серед чотирьох стін якогось приміщення, званого школою, та
що хазяїн дітей мусить діставати про те щотижня якусь посвідку
від особи, яка підписується своїм ім’ям як учитель або вчителька
школи»\footnote{
Leonhard Horner у «Reports of Insp. of Fact., for 30 th June 1857»,
p. 17.
}. Перед виданням у 1844~\abbr{р.} виправленого фабричного
закону частенько траплялися такі посвідки про відвідування
школи, які вчитель або вчителька підписували хрестом, бо вони
сами не вміли писати. «Відвідавши одного разу школу, яка видавала
такі посвідки, я так здивувався неуцтву вчителя, що сказав
йому: «Скажіть, будь ласка, мій пане, чи вмієте ви читати?»
Його відповідь була така: «Іh jeh, Еbbеs (summat)»\footnote*{
Так, поганенько (поверхово). \emph{Ред.}
}. Задля свого
виправдання він додав: «Та все ж я доглядаю своїх учнів». Підчас
розроблення закону 1844~\abbr{р.} фабричні інспектори скаржились
на жахливий стан приміщень, званих школами, посвідки яких
вони мусили, згідно з законом, визнавати за цілком правні. Але
добились вони тільки того, що, починаючи від 1844~\abbr{р.}, «вчитель
власноручно мусив вписувати числа у шкільній посвідці та власною
рукою підписувати своє ймення і прізвище»\footnote{
Leonhard Horner у «Reports of Insp. of Fact., fo 31 st October
1855», p. 18, 19.
l38 Sir John Kincaid у «Reports of Insp. of Fact., for 31 st October
1858», p. 31, 32.
}. Сер Джон
Кінкед, фабричний інспектор у Шотляндії, оповідає про подібні
факти із свого службового досвіду. «Першу школу, яку ми відвідали,
утримувала якась містрес Анн Кіллін. Коли я попрохав її
написати своє прізвище, вона відразу зробила одну помилку,
почавши його буквою С, але зараз же, поправляючись, сказала,
що її прізвище починається на К. Однак, оглядаючи її підпис у
книзі шкільних посвідок, я помітив, що вона підписувалася різно,
а її письмо не залишало найменшого сумніву про те, що вона
нездатна вчителювати. Та й сама вона призналась, що не може
складати реєстра\dots{} В другій школі я побачив шкільну кімнату
15 футів завдовжки та 10 футів завширшки та налічив у цім приміщенні
75 дітей, які верещали щось незрозуміле»\footnote{Sir John Kincaid у
«Reports of Insp. of Fact., for 31 st October
1858», p. 31, 32.
}. «Однак,
не тільки по таких жахливих норах діти дістають шкільні посвідки,
а не навчання; по багатьох школах, де вчитель компетентна
людина, всі його зусилля майже цілком розбиваються об
перемішану аж до запаморочливости юрбу дітей усякого віку,
починаючи від трилітніх дітей. Заробіток учителя, в найліпшому
випадку мізерний, цілком залежить від числа пенсів, одержуваних
від якнайбільшого числа дітей, яке тільки можливо напхати в
одну кімнату. До цього треба ще додати злиденні шкільні меблі,
брак книжок та іншого навчального приладдя й пригнітний
\parbreak{}  %% абзац продовжується на наступній сторінці
