\index{i}{0258}  %% посилання на сторінку оригінального видання
\section{Кооперація}
Капіталістична продукція, як ми бачили, починається фактично
лише там, де той самий індивідуальний капітал одночасно експлуатує
значне число робітників, отже, лише там, де процес праці
поширює свій обсяг і постачає продукти у великому маштабі.
Праця значного числа робітників у той самий час, у тому самому
приміщенні (або, як хто хоче, на тому самому полі праці), для продукції
того самого ґатунку товарів, під командою того самого
капіталіста, становить історично й логічно вихідний пункт капіталістичної
продукції. Щодо самого способу продукції мануфактура,
приміром, на початках свого розвитку ледве чи відрізняється
від цехової ремісничої промисловости чимось іншим, як
хібащо більшим числом робітників, яких одночасно експлуатує
той самий капітал. Майстерню цехового майстра тільки розширено.

Отже, спочатку ріжниця є лише кількісна. Ми бачили, що
маса додаткової вартости, яку продукує даний капітал, дорівнює
додатковій вартості, яку дає окремий робітник, помноженій на
число одночасно експлуатованих робітників. Це число само по
собі нічого не змінює в нормі додаткової вартости або в ступені
експлуатації робочої сили; щодо продукції товарової вартости
взагалі, то для неї всякі якісні зміни робочого процесу, очевидно,
не мають значення. Це випливає з природи вартости. Якщо один
дванадцятигодинний робочий день упредметнюється в 6 шилінґах,
то 1.200 таких робочих днів — у 6 шилінґах × 1.200. В одному
випадку в продукті втілилось 12 × 1.200 робочих годин, у другому
тільки 12 робочих годин. У продукції вартости велике число
має значення завжди тільки як багато окремих одиниць. Отже, для
продукції вартости немає ніякої ріжниці, чи 1.200 робітників продукують
поодиноко, чи спільно під командою того самого капіталу.

А проте, в певних межах тут відбувається модифікація. Праця,
упредметнена у вартості, є праця пересічної суспільної якости,
тобто виявлення якоїсь пересічної робочої сили. Але пересічна
величина існує завжди лише як пересічна багатьох різних індивідуальних
величин того самого роду. В кожній галузі промисловости
індивідуальний робітник, Петро чи Павло, більш чи менш
відхиляється від пересічного робітника. Ці індивідуальні відхилення,
які в математиці називаються «помилками», компенсуються
і зникають, скоро тільки взяти разом велике число робітників.
Славетний софіст і сикофант Едмунд Берк навіть запевняє
на підставі свого практичного досвіду як фармер, що всяка індивідуальна
ріжниця праці зникає вже «для такої незначної групи»,
як 5 рільничих наймитів, отже, п’ять перших-ліпших дорослих
англійських рільничих наймитів протягом того самого часу виконають
разом стільки ж праці, як і яких-будь інших п’ять англійських
рільничих наймитів.\footnote{
«Безперечно, щодо сили, вправности та сумлінної пильности, існує
велика ріжниця між вартістю праці однієї людини і вартістю праці іншої
людини. Але я цілком певен на підставі моїх докладних спостережень, що
п’ятеро перших-ліпших людей разом постачають кількість праці, рівну
кількості праці всяких інших п'ятьох людей зазначеного мною віку; це
значить, що з цих п’ятьох робітників один має всі властивості доброго
робітника, другий є недобрий, а троє інших будуть із цього погляду середні,
наближаючись то до першого, то до другого. Отже, вже в такій
невеликій групі, як п’ятеро людей, ви знайдете повнотою все те, що вза-
галі можуть дати п'ятеро людей». («Unquestionably, there is a great deal
of difference between the value of one man’s labour and that of another,
from strength, dexterity and honest application. But I am quite sure, from
my best observation, that any given five men will, in their total, afford a
proportion of labour equal to any other five within the periods of life I have
stated; that is, that among such five men there will be one possessing all
the qualifications of a good workman, one bad, and the other three middling,
and approximating to the first and the last. So that in so small a platoon
as that of even five, you will find the full complement of all that five men
can earn»). (\emph{E. Burke}: «Thoughts and Details on Searcity», London
1800, p. 16). Порівн., опріч того, Кетле про середнього індивіда.
} Хоч би й що, а ясно, що сукупний
\index{i}{0259}  %% посилання на сторінку оригінального видання
робочий день значного числа одночасно експлуатованих робітників,
поділений на число робітників, сам по собі є день пересічної
суспільної праці. Припустімо, що робочий день поодинокого
робітника є, приміром, 12 годин. Тоді робочий день дванадцятьох
робітників, експлуатованих одночасно, становить сукупний
робочий день у 144 години, і хоч праця кожного поодинокого
з тих дванадцятьох робітників може більш або менш відхилятися
від пересічної суспільної праці; отже, поодинокий робітник
може потребувати на ту саму роботу трохи більш або трохи менш
часу, все ж робочий день кожного поодинокого робітника, як
одна дванадцята частина сукупного робочого дня в 144 години,
має пересічну суспільну якість. Але для капіталіста, що експлуатує
дванадцятьох робітників, робочий день існує як сукупний
робочий день дванадцятьох. Робочий день кожного поодинокого
робітника існує лише як відповідна частина сукупного робочого
дня цілком незалежно від того, чи працюють ці дванадцятеро
разом, допомагаючи один одному, чи ввесь зв’язок між їхніми
працями є лише в тому, що вони працюють для того самого капіталіста.
Навпаки, коли що два з цих 12 робітників працюватимуть
у дрібного майстра, то лише випадково кожний поодинокий майстер
випродукує, можливо, ту саму масу вартости і зреалізує,
отже, загальну норму додаткової вартости. Тут завжди бувають
індивідуальні відхилення. Коли б робітник зуживав на продукцію
якогось товару значно більше часу, ніж це суспільно-потрібно,
коли б індивідуально-доконечний для нього робочий
час значно відхилявся від суспільно-доконечного або пересічного
робочого часу, то його праця не мала б значення пересічної праці,
а його робоча сила — значення пересічної робочої сили. Цієї робочої
сили або зовсім не можна було б продати, або її можна було б
продати лише за ціну, нижчу від пересічної вартости робочої сили.
Отже, завжди припускається певний мінімум працездатности, і ми
побачимо пізніш, що капіталістична продукція находить засоби,
щоб виміряти цей мінімум. А проте цей мінімум відхиляється
від пересічної величини, хоч і доводиться сплачувати пересічну
\parbreak{}  %% абзац продовжується на наступній сторінці
