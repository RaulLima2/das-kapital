запрошує промислову колонію працювати у його маєтках, а
потім він як власник поверхні землі не дає змоги зібраним ним
робітникам найти відповідне помешкання, доконечне для їхнього
влиття. Орендар копалень [капіталістичний експлуататор] не
має жодного грошового інтересу опиратися цьому двозначному
торгові, бо він добре знає, що коли вимоги землевласника непомірні,
то наслідки цього спадуть не на нього, що робітники, на
яйих вони спадуть, надто несвідомі, щоб знати свої права на здоров’я,
і що ані якнайнепристойніші житла, ані якнайгниліша
вода ніколи не будуть приводом до страйку».135

d) Вплив криз на найкраще оплачувану
частину робітничої кляси

Раніш ніж перейти до власне рільничих робітників, я хочу
ще показати на одному прикладі, як впливають кризи навіть на
найкраще оплачувану частину робітничої кляси, на її аристократію.
Пригадаймо собі ось що: 1857 рік приніс одну з тих
великих криз, що ними кожного разу завершується промисловий
цикл. Найближча криза припала на 1866 р. Ця криза, що у власне
фабричних округах була антиципована в наслідок бавовняного
голоду, який загнав чимало капіталу із звичайної сфери його
вміщення у великі центри грошового ринку, цим разом набрала
переважно фінансового характеру. Сиґналом її вибуху в травні
1866 р. було банкрутство одного з велетенських лондонських
банків, слідом за яким наступив крах безлічі спекуляційних
фінансових товариств. Однією з великих лондонських галузей
промисловости, що їх вразила катастрофа, була будова кораблів
із заліза. Маґнати цієї галузі промисловости за часів спекуляції
не лише перепродукували понад усяку міру, але, крім
того, ще й перейняли на себе контракти на величезні замовлення,
сподіваючись, що джерело кредиту й далі залишатиметься невичерпним.
Тепер же постала страшенна реакція, яка і в інших
галузях лондонської промисловости136 триває до цього часу,
кінець березня 1867 р. Для характеристики становища робітників
наведімо таке місце з докладного звіту одного кореспон-

136 Там же, стор. 16.

137 «Масове голодування лондонських бідняків!(«Wholesale starvation
of the London Poor!»)... Останніми днями мури лондонських будинків
позаклеювано величезними плякатами з такою дивовижною об’явою:
«Ситі бики, зголоднілі люди! Ситі бики покинули свої кришталеві
палаци, щоб відгодувати багатіїв у їхніх розкішних світлицях, тимчасом
як зголоднілі люди гинуть і вмирають у своїх злиденних норах». Плякати
з цим зловішим написом постійно поновлюються. Ледве позривають і
позаліплюють одну партію, як одразу на тому самому або іншому прилюдному
місці знов появляється нова... Це нагадує ті передвіщання, що підготовляли
французький народ до подій 1789 р. Нині, коли англійські
робітники з своїми жінками й дітьми вмирають від холоду і з голоду,
мільйони англійських грошей, продукт англійської праці, вкладають
у російські, еспанські. італійські й інші закордонні позики». («Reynolds'
Newspaper, 20 Januar 1867»).
