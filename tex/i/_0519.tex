\parcont{}  %% абзац починається на попередній сторінці
\index{i}{0519}  %% посилання на сторінку оригінального видання
вони, як ми вже згадували раніш, роблять ту саму дарову службу,
що й сили природи, — вода, пара, повітря, електрика й~\abbr{т. ін.}
Ця дарова служба минулої праці, охоплена й одушевлена живою
працею, акумулюється із зростом маштабу акумуляції.

А що минула праця завжди фігурує в одягу капіталу, тобто
пасив праці робітників $А, В, С$ і~\abbr{т. ін.} фігурує як актив неробітника
$X$, то буржуа й політико-економи не знаходять слів
для вихвалення заслуг минулої праці, яка, на думку шотляндського
генія Мак Куллоха, мусить навіть діставати якусь окрему
винагороду (процент, зиск і~\abbr{т. ін.}).\footnote{
Мак Куллох вибрав патент на «wages of past laboure»\footnote*{
— винагороду за минулу працю. \emph{Ред.}
} далеко раніш, ніж Сеніор вибрав патент на «wages of abstinence».\footnote*{
— винагороду за поздержливість. \emph{Ред.}
}
} Отже, чимраз більше й
більше значення минулої праці, яка у формі засобів продукції
бере участь у живому процесі праці, приписують її відчуженій
від самого робітника формі, а саме формі праці, що являє собою
його минулу й неоплачену працю, — тобто приписують її
формі капіталу. Практичні аґенти капіталістичної продукції
та її ідеологічні базікала так само нездібні уявити собі засіб
продукції окремо від антагоністичної суспільної характеристичної
маски, властивої йому за наших часів, як рабовласник
нездібний уявити собі самого робітника окремо від його характеристичних
рис як раба.

[Нарешті, за інших незмінних обставин, величина випродукованої
додаткової вартости, а тому й акумуляція, визначається
величиною авансованого капіталу].\footnote*{— Заведене у прямі дужки ми беремо з другого німецького видання.
\emph{Ред.}}

За даного ступеня експлуатації робочої сили маса додаткової
вартости визначається числом одночасно визискуваних робітників,
а це останнє відповідає, хоч і в змінній пропорції, величині
капіталу. [Із загальною величиною капіталу зростає і його
змінна складова частина, хоч і не в такій самій пропорції. Що
більший той маштаб, у якому продукує індивідуальний капіталіст,
то більше число робітників, що їх він одночасно експлуатує,
або маса неоплаченої праці, яку він собі присвоює\footnoteA{
[У третій книзі ми побачимо, що на пересічну норму зиску різних
сфер продукції не впливає властивий кожній із них поділ капіталу на
сталий і змінний елемент, і так само побачимо, що це явище лишена
иозір суперечить викладеним нами законам про природу й продукцію
додаткової вартости].\footnote*{— Заведене у прямі дужки ми беремо з другого німецького видання.
\emph{Ред.}}
}]. Отже,
що більше зростає капітал через послідовні акумуляції, то більше
зростає й сума вартости, яка розпадається на фонд споживання
та фонд акумуляції. Тим-то капіталіст може жити розкішніш
і разом з тим більше «поздержуватися». І кінець-кінцем усі
пружини продукції діють то енерґійніше, що більше розширюється
разом із масою авансованого капіталу маштаб продукції.
