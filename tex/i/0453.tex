Клясична політична економія запозичила без дальшої критики
з щоденного життя категорію «ціна праці», щоб потім поставити
собі питання, як визначається цю ціну? Вона пізнала
незабаром, що зміна відношення між попитом і поданням не
пояснює нічого в ціні праці, як і в ціні всякого іншого товару,
крім її зміни, тобто коливання ринкових цін понад або нижче
деякої величини. Якщо попит і подання взаємно покриваються,
то, за інших незмінних умов, коливання в цінах припиняється.
Але тоді попит і подання перестають щобудь пояснювати. Ціна
праці, коли попит і подання покриваються, є її природна ціна,
визначена незалежно від відношення між попитом і поданням,
природна ціна, що її таким чином і знайдено як справжній предмет
аналізи. Або брали довший період коливань ринкової ціни,
наприклад, один рік, і тоді відкривали, що її зростання або
спадання вирівнюється в якусь середню пересічну величину,
сталу величину. Само собою зрозуміло, цю пересічну величину
треба визначати інакше, а не тими відхиленнями від неї самої,
що взаємно покриваються. Ця ціна праці, що панує над випадковими
ринковими цінами праці та їх реґулює, — ця «доконечна
піна» (фізіократи), або «природна ціна» праці (Адам Сміт) може
бути, як і при інших товарах, лише її вартістю, вираженою в
грошах. Цим способом політична економія гадала від випадкових
цін праці пробитися до її вартости. Як і для інших товарів,
цю вартість визначали потім витратами продукції. Але що таке
витрати продукції, витрати продукції робітника, тобто витрати,
потрібні на те, щоб спродукувати або репродукувати самого
робітника? Цим питанням політична економія несвідомо підмінила
первісне питання, бо, досліджуючи витрати продукції
праці як такої, вона крутилася мов у колі, і ніяк не могла рз’їнити
з місця. Отже, те, що вона називає вартістю праці (value of labour),
є в дійсності вартість робочої сили, яка існує в особі
робітника і так само відмінна від своєї функції, від праці, як
відмінна машина від її операцій. Захоплена ріжницею між рин-

effrayante, il ne voit qu’une ellipse gramaticale. Donc toute la société actuelle,
fondée sur le travail marchandise, est désormais fondée sur une licence
poétique, sur une expression figurée. La société veut-elle «éliminer tous
les inconvénients», qui la travaillent, eh bienl qu’elle élimine les termes
malsonnants, qu’elle change de langage, et pour cela elle, n’a qu’a s’adresser
à l’Académie pour lui demander une nouvelle édition de son dictionnaire»).
(K. Marx: «Misère de la Philosophie», p. 34, 35. — K. Маркс: «Злиденність
філософії», Партвидав 1932 р., стор.55). Природно, ще зручніше під «вартістю»
зовсім нічого не розуміти. Тоді можна залежно від обставин підводити
під цю категорію все. Так робить, наприклад, Ж. Б. Сей. Що
таке «вартість» («valeur»)? Відповідь: «Те, чого варта дана річ» («C’est
ce qu’une chose vaut»). А що таке ціна (prix)? Відповідь: «Вартість даної
речі, виражена в грошах» («La valeur d’une chose exprimée en monnaie».
A чому «праця землі має... вартість»? («le travail de la terre... une
valeur»?) «Тому, що за неї дають певну ціну» («parce qu’on у met un
prix»). Отже вартість, є те, чого варта річ, а земля має «вартість» тому,
що вартість її «виражають у грошах». У всякому разі це дуже проста
метода поясняти «чому і як» щодо речей.
