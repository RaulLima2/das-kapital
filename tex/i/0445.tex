2) Збільшення інтенсивности і продуктивної сили праці при
одночасному скороченні робочого дня.

Підвищення продуктивної сили праці та зростання її інтенсивносте
в одному напрямі діють однаково. Одне і друге збільшує
масу продуктів, продукованих протягом певного часу. Отже,
одне і друге скорочує ту частину робочого дня, що її робітник
потребує на продукцію своїх засобів існування — або їх еквіваленту.
Абсолютну мінімальну межу робочого дня визначає
взагалі ця його доконечна складова частина, яку однак можна
скорочувати. Коли б цілий робочий день скоротився до цієї останньої,
то зникла б додаткова праця — річ за капіталістичного
режиму неможлива. Усунення капіталістичної форми продукції
дозволяє обмежити робочий день доконечною працею. Однак, за
інших незмінних умов, остання поширила б свої рамки. З одного
боку, тому що життєві умови робітника покращали б і його життєві
потреби збільшилися б. З другого боку, довелося б до доконечної
праці залічити частину теперішньої додаткової праці,
саме працю, потрібну на те, щоб утворити суспільний резервний
фонд і фонд акумуляції.

Що більше зростає продуктивна сила праці, то більше можна
скорочувати робочий день, а що більше скорочується робочий
день, то більше може зростати інтенсивність праці. З суспільного
погляду продуктивність праці зростає також з її економією. Ця
остання включає не тільки економію на засобах продукції, але й
уникання всякої некорисної праці. Тимчасом як у кожному індивідуальному
підприємстві капіталістичний спосіб продукції примушує
до економії, його анархічна система конкуренції породжує
якнайбезмірніше марнотратство суспільних засобів продукції
та робочих сил поряд безлічі функцій, тепер неминучих, але по
суті зайвих.

За даної інтенсивносте й продуктивности праці частина суспільного
робочого дня, доконечна для матеріяльної продукції,
є то коротша, отже, частина часу, завойована для вільної, розумової
й суспільної діяльносте індивідів, є то більша, що рівномірніше
поділено працю поміж усіма дієздатними членами суспільства,
що менше одна суспільна верства може звалити природну
доконечність праці з себе на інші верстви. З цього погляду,
абсолютна межа для скорочення робочого дня є вселюдність праці.
У капіталістичному суспільстві вільний час однієї, кляси створюється
перетворенням усього життя мас на робочий час.

ще раніш стали робітниками, мусили з тієї самої причини присвятити
більшу частину свого часу збільшенню продукції». («A principal cause
of the increase of capital, during the war, proceeded from the greater exertions,
and perhaps the greater privations of the labouring classes, the most
numerous in every society. More women and children were compelled, by
necessitous circumstances, to enter upon laborious occupations; and former
workmen were, from the same cause, obliged to devote a greater portion
of their time to increase production»). («Essays on Political Economy in
which are illustrated the Principal Causes of the Present National Distress»,
London 1830, p. 248).
