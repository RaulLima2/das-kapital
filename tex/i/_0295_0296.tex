\parcont{}  %% абзац починається на попередній сторінці
\index{i}{0295}  %% посилання на сторінку оригінального видання
ний поділ праці протиставить їм духовні потенції матеріяльного
продукційного процесу як чужу власність та силу, що опановує
їх. Цей процес відокремлення починається у простій кооперації,
де капіталіст репрезентує супроти поодиноких робітників єдність
і волю суспільного робочого тіла. Він розвивається в мануфактурі,
яка калічить робітника, перетворюючи його на частинного
робітника. Він завершується у великій промисловості, яка відокремлює
від праці науку як самостійний фактор продукції та
примушує її служити капіталові.\footnote{
«Людина науки й продуктивний робітник дуже віддалені один від
одного, і наука замість збільшувати в руках робітника його власні продуктивні
сили для нього самого, майже скрізь поставила себе проти нього.
Знання стає знаряддям, що його можна відокремити від праці та їй протипоставити».
(W. Thompson: «An Inquiry into the Principles of the
Distribution of Wealth», London 1824, p. 274).
}

В мануфактурі збагачення колективного робітника, а тому
й капіталу, на суспільну продуктивну силу зумовлено збіднінням
робітника на індивідуальні продуктивні сили. «Неуцтво є
мати промисловости, як і забобонів. Розумування й фантазія
підпадають помилкам; але звичка рухати ногою або рукою не
залежить ні від одного, ні від другого. Отже, мануфактури процвітають
найкраще там, де найбільше можна звільнитися від
інтелектуальної роботи, так що майстерня може бути розглядувана
як машина, що її частинами є люди».\footnote{
A. Ferguson: «History of Civil Society», Edinburgh 1767,
part IV, sect. I, p. 280.
} Справді, в середині
XVIII віку деякі мануфактури вживали охоче напівідіотів,
щоб виконувати деякі прості операції, які, однак, становили
фабричні таємниці.\footnote{
J. D. Tuckett: «А History of the Past and Present State of
the Labouring Population», London 1846, vol. I, p. 148.
}

«Інтелект великої більшости людей, — каже А. Сміс, — неминуче
розвивається з їхніх щоденних заняттів і через ці заняття.
Людина, що витратила ціле своє життя на виконування небагатьох
простих операцій\dots{} не має нагоди вправляти свій розум\dots{} Вона
взагалі стає такою тупою і темною, якою тільки й може стати
людська істота». Змалювавши туподумство частинного робітника,
Адам Сміс каже далі: «Одноманітність його життя без усяких
змін губить, ясна річ, і жвавість його розуму\dots{} Вона руйнує
навіть енерґію його тіла та робить його нездатним уживати напружено
й довгочасно своєї сили, хіба лише в тій частинній
праці, до якої його привчили. Таким чином його вправність у його
осібному реместві є, здається, властивість, що він її набуває коштом
своїх інтелектуальних, соціальних та військових здібностей.
Але саме такий є в кожному промисловому й цивілізованому
суспільстві той стан, що в ньому неминуче мусить опинитись
кожний працюючий бідняк (the labouring poor), тобто велика маса
народу».\footnote{
A. Smith: «Wealth of Nations», b. V, ch. I, art. II. p. 140, 141.
Як учень А. Ферґюсона, який розвинув усі шкідливі наслідки поділу
} Щоб перешкодити тому повному знидінню народньої
\index{i}{0296}  %% посилання на сторінку оригінального видання
маси, яке виникає з поділу праці, А. Сміс радить державну
організацію народньої освіти, хоч і в обережно гомеопатичних
дозах. Послідовно полемізує проти цього його французький
перекладач та коментатор Ґ. Ґарньє, що за часів першої французької
імперії, природна річ, перетворився на сенатора. На його
думку, народня освіта суперечить основним законам поділу праці;
завівши її, «ми засудили б цілу нашу суспільну систему». «Як
і всякі інші поділи праці, — каже він, — поділ між працею фізичною
та розумовою\footnote{
Вже Ферґюсон каже («History of Civil Society», part IV, sect. I.
p. 281): «І саме думання в цьому віці поділу праці стає осібною професією»
(«and thinking itself, in this age of separations, may become a peculiar
craft»).
} стає дедалі виразнішим та рішучішим у
міру того, як багатіє суспільство (він слушно прикладає цей
вислів до капіталу, земельної власности та їхньої держави).
Подібно до всіх інших і цей поділ праці є наслідок минулого та
причина майбутнього проґресу\dots{} Невже ж уряд має право протидіяти
цьому поділові праці та спиняти його в природному його
розвитку? Невже він має право частину державних прибутків
витрачати на спробу перемішати та сплутати дві кляси праці,
що прагнуть свого розділу й відокремлення?».\footnote{
G. Garnier, т. V його перекладу, стор. 2--5.
}

Деяке розумове й фізичне скалічення є невіддільне навіть від
поділу праці всередині суспільства як цілости. Але що мануфактурний
період проводить це суспільне розчленування галузей
праці значно далі і що, з другого боку, цей період з властивим
йому поділом праці вперше захоплює життя індивіда в самому
його корені, то він також уперше дає матеріял та поштовх до
промислової патології.\footnote{
Рамацціні, професор практичної медицини в Падуї, опублікував
1713 р. свою працю «De morbis artificum», перекладену 1781 p.
французькою мовою і знову передруковану 1841 р. в «Encyclopédie des
Sciences Médicales, 7 ème Discours: Auteurs Classiques». Період великої
промисловости, природно, дуже збільшив його каталог робітничих хороб.
Дивись, між іншим, «Hygiène physique et morale de l’ouvrier dans les
grandes villes en général, et dans la ville de Lyon en particulier. Par le
Dr. A. L. Fonterel», Paris 1858 і «Die Krankheiten, welche verschiedenen
Ständen, Altern und Geschlechtern eigentümlich sind». 6 Bände, Ulm I860.
1854 p. Society of Arts призначило слідчу комісію щодо промислової
патології. Реєстр зібраних цією комісією документів можна найти в каталозі
«Twickenham Economic Museum». Дуже важливі офіціальні «Reports
on Public Health». Див. також Eduard Reich, M. D.: «Ueber
die Entartung des Menschen», Erlangen 1868.
}

«Ділити людину — це значить карати її на смерть, якщо вона

праці, А. Сміс цілком ясно розумів цей пункт. На початку свого твору,
де він ex professo прославляє поділ праці, він лише мимохідь указує на
нього як на джерело суспільних нерівностей. Лише у п’ятій книзі про
державні доходи він репродукує Ферґюсона. В «Misère de la Philosophie»
я вже з’ясував усе потрібне про історичне відношення між Фергюсоном,
А. Смісом, Лемонтеєм та Сеєм у їхній критиці поділу праці; там
я також уперше подав мануфактурний поділ праці як специфічну форму
капіталістичного способу продукції (стор. 122 і далі) («Злиденність філософії»
, Партвидав 1932 р., стор. 113 і далі).
\parbreak{}  %% абзац продовжується на наступній сторінці
