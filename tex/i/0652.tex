сучасників раптова поява цієї зграї банкократів, фінансистів, рантьє, маклерів, stockjobber’iв\footnote*{
— біржовиків, що спекулюють на цінних державних паперах, акціях тощо. Ред.
} і
біржових вовків, це можна бачити з тогочасних творів, наприклад, творів Белінґброка.\footnoteA{
«Коли б татари захопили тепер Европу, то дуже важко було б з’ясувати їм, хто такий є в нас
фінансист» («Si les Tartares inondaient aujourd’hui l’Europe, il faudrait bien des affaires pour
leur faire entendre ce que c’est qu’un financier parmi nous»). (Montesquieu: «Esprit des lois», éd.
Londres 1769, vol. IV, p. 33).
}

Разом з державними боргами виникла система міжнароднього кредиту, за якою часто криється одне з
джерел первісної акумуляції капіталу в того або іншого народу. Так підлоти венецької хижацької
системи становлять таку скриту основу капіталістичного
багатства Голляндії, якій занепадуща Венеція позичала великі грошові суми. Такі самі відносини були
між Голляндією та Англією. Вже на початку ХVІІІ століття голляндські мануфактури були значно
перевищені, і Голляндія перестала бути панівною торговельною і промисловою нацією. Тому 1700 — 1776
рр. за одне з головних занять Голляндії стає позичання величезних капіталів, особливо своєму
могутньому конкурентові — Англії. Подібні відносини маємо нині між Англією та Сполученими
штатами. Багато капіталів, що сьогодні з’явилися без метричного свідоцтва в Сполучених штатах, є
лише капіталізована вчора в Англії кров дітей.

А що державні борги спираються на державні доходи, які мусять покривати річні проценти й інші
подібні платежі, то сучасна податкова система стала доконечним доповненням системи національних
позик. Позики дають урядові змогу покривати
надзвичайні видатки таким чином, що платник податків не відчуває цього відразу, але згодом ці позики
все ж вимагають підвищення податків. З другого боку, підвищення податків, спричинене нагромадженням
один по одному роблених боргів, примушує
уряд при нових надзвичайних видатках щоразу брати нові позики. Тому сучасна фіскальна система, що її
вісь становлять податки на найдоконечніші засоби існування (отже, їх подорожчання), має в собі самій
зародок автоматичного підвищування податків. Надмірне оподаткування — це не якийсь винятковий
випадок: навпаки, це — принцип. Тому в Голляндії, де вперше заведено цю систему, великий патріот Де
Вітт вихвалював її у своїх «Maximes» як найкращу систему зробити найманого робітника покірливим,
скромним, працьовитим та... переобтяженим надмірною працею. Однак той руйнаційний вплив, що його
сучасна фіскальна система справляє на становище найманих робітників, нас тут цікавить менш, ніж
зумовлена нею насильна експропріяція селянина, ремісника, коротко — всіх складових частин нижчої
верстви середньої кляси. Про це немає двох думок навіть серед буржуазних економістів.
Експропріяторське діяння