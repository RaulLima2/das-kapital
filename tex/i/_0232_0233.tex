\index{i}{0232}  %% посилання на сторінку оригінального видання
Закон з 1850 р. перетворив лише для «підлітків і жінок»
п’ятнадцятигодинний період від пів на шосту ранку до пів на
дев’яту вечора на дванадцятигодинний період від шостої години
ранку й до шостої години вечора. Отже, не для дітей, яких усе
ще можна було експлуатувати 1/2 години перед початком і 2 1/2
години по скінченні цього періоду, хоч загальний час тривання
їхньої праці не повинен був перевищувати 6 1/2 годин. Під час обговорення
закону фабричні інспектори подали до парляменту статистичні
дані про ганебні зловживання, до яких приводила ця
аномалія. Але все це даремно. Потайний намір був у тому, щоб
за допомогою дітей у роки розцвіту знов догнати робочий день
дорослих до 15 годин. Досвід дальших трьох років довів, що
така спроба мусіла б розбитись об опір дорослих робітниківчоловіків.\footnote{
«Reports etc. for 30 th April 1853», p. 31.
}
Тим то закон з 1850 р. й доповнено, нарешті, 1853 р.
забороною «вживати праці дітей ранком перед початком і ввечері
по скінченні праці підлітків і жінок». Починаючи з цього часу,
фабричний закон 1850 р. реґулював, за деякими винятками, в
підпорядкованих йому галузях промисловости робочий день усіх
робітників.\footnote{
За часів найвищого розквіту англійської бавовняної промисловости,
в роках 1859—1860, деякі фабриканти приманою високої заробітної
плати за наднормовий час пробували підохотити дорослих прядунів до
здовження робочого дня. Прядуни на ручних варстатах і на сельфакторах
поклали кінець цій спробі, подавши меморіял своїм підприємцям, де, між
іншим, зазначено: «Сказати по правді, наше життя є тягар для нас, і
поки ми прикуті до фабрики майже на 2 дні (20 годин) у тижні
більше, ніж інші робітники, то ми почуваємо себе в країні гелотами й
сами собі докоряємо за те, що увіковічнюємо таку систему, яка фізично
й морально шкодить нам самим і нашим нащадкам... Тим то з повною
пошаною доводимо до вашого відома, що від першого дня нового року
не працюватимемо й хвилини довше понад 60 годин тижнево, від шостої
години до шостої години, відлічуючи законом призначені перерви на
1 1/2 години». («Reports etc. for 30 th April 1860», p. 30).
} Від часу оголошення першого фабричного закону
проминуло тепер півстоліття.\footnote{
Про засоби порушувати цей закон, що їх дає редакція цього закону,
див. Parliamentary Return: «Factory Regulations Acts» (6 серпня 1859 р.) і
там само Leonhard Horner: «Suggestions for Amending the Factory Acts to
enable the Inspectors to prevent illegal working, now become very prevalent».
}

Поза свою первісну сферу законодавство вийшло вперше через
«Printworks’ Act» (закон про перкалеві фабрики тощо), виданий
1845 р. Нехіть, з якою капітал допустив цю нову «екстраваґантність»,
промовляє з кожного рядка закону! Він обмежує
робочий день дітей 8—13 років і жінок 16 годинами — від шостої
години ранку до десятої години вечора, не призначаючи жодної
взаконеної перерви на їжу. Він дозволяє примушувати до праці
робітників-чоловіків старших від 13 років довільно цілий
день і цілу ніч.\footnote{
«За останнє півріччя (1857) у моїй окрузі дітей 8 років і старших
справді катують від 6 години ранку й аж до 9 години вечора». («Reports
etc. for 31 st October 1857», p. 39).
} Це — парляментський викидень.\footnote{
«Закон про перкалеві фабрики вважається за невдалий так шодо його
постанов про навчання, як і щодо його постанов про охорону праці» («The-
}

\index{i}{0233}  %% посилання на сторінку оригінального видання
А все ж принцип,\footnote*{
Мається на увазі принцип законодавчого втручання в промислові
справи. \emph{Ред.}
} перемігши у великих галузях промисловости,
які є найспецифічніший витвір сучасного способу продукції,
переміг остаточно. Дивовижний розвиток цих галузей промисловости
на протязі часу від 1853 р. до 1860 р., який відбувався
поруч фізичного й морального відродження фабричних робітників,
розкрив очі найдурнішим. Сами фабриканти, що в них у
півстолітній громадянській війні крок за кроком відвойовано
законодавчі обмеження й реґулювання робочого дня, хвальковито
вказували на контраст між цими галузями промисловости
й тими сферами експлуатації, що лишилися ще «вільними».\footnote{
Так, приміром, висловлюється E. Поттер у листі до «Times’y»
з 24 березня 1863 р. «Times» нагадав йому про бунт фабрикантів проти
десятигодинного закону.
}
Фарисеї «політичної економії» проголосили тепер переконання
про доконечність законодавчим шляхом реґулювати робочий день
новим характеристичним здобутком їхньої «науки».\footnote{
Так, між іншим, висловлюється пан В. Ньюмарч, співробітник
і видавець «History of Prices» Тука. Невже ж це науковий проґрес —
робити боягузливі поступки громадській думці?
} Легко
зрозуміти, що після того, як фабричні маґнати скорилися перед
неминучим і примирилися з ним, сила опору капіталу помалу
слабшала, тоді як у той самий час сила наступу робітничої кляси
зростала разом із зростом числа її спільників серед суспільних
верств, безпосередньо не заінтересованих. Цим то й пояснюється
порівняно швидкий проґрес від 1860 р.

Фарбарні й білильні\footnote{
Виданий 1860 р. закон про білильні та фарбарні установляє,
що робочий день від 1 серпня 1861 р. тимчасово скорочується до 12, а
від 1 серпня 1862 р. остаточно до 10 годин, тобто до 10 1/2 годин у робочі
дні та 7 1/2 годин суботами. Але ось настав лихий 1862 р., і повторився
старий фарс. Пани фабриканти звернулись до парляменту з петицією
стерпіти ще один-однісінький рік дванадцятигодинну працю підлітків
і жінок... «За сучасного стану справ (підчас бавовняного голоду) було б
дуже корисно для робітників, коли б їм дозволили працювати по 12 годин
щодня й діставати по змозі якнайбільшу заробітну плату... Вже було
пощастило внести до парляменту біл у цьому дусі, але він провалився
через аґітацію робітників у білильнях Шотляндії». («Reports etc. for
31 st October 1862», p. 14, 15). Капітал, побитий таким чином тими самими
робітниками, що іменем їхнім він претендував говорити, відкрив тепер
за допомогою юридичних окулярів, що закон з 1860 р., подібно до всіх
парляментських законів про «охорону праці», складений поплутаними,
покрученими словами, дає привід не поширювати його на категорії робітників
«calenderers»\footnote*{
— пресувальники сукна. \emph{Ред.}
} і «finishers».\footnote*{
— апретери. \emph{Ред.}
} Англійська юрисдикція, завжди
вірний наймит капіталу, санкціонувала це закарлюцтво постановою так
званого «Common Pleas».\footnote*{
— цивільний суд. \emph{Ред.}
} «Це викликало велике незадоволення серед
робітників, і дуже шкода, що ясні наміри законодавства нищиться під
приводом хибного окреслення слів». (Там же, стор. 18).
} підведено під фабричний закон 1850 р.
ще 1860 р., а мереживні й панчішні — 1861 р. Внаслідок першого

Printworks Act is admitted to be a failure, both with reference to its educational
and protective provisions»). («Reports etc. for 31 st Oct. 1862», p. 52).
\parbreak{}  %% абзац продовжується на наступній сторінці
