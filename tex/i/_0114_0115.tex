\parcont{}  %% абзац починається на попередній сторінці
\index{i}{0114}  %% посилання на сторінку оригінального видання
утратити 10.\footnote{
«При збільшенні номінальної вартости продукту\dots{} продавці не
збагачуються\dots{} бо рівно стільки, скільки вони виграють як продавці,
вони втрачають як покупці» («By the augmentation of the nominal value
of the produce\dots{} sellers not enriched\dots{} since what they gain as sellers, they
prеcisеly expend in the quality of buyers»). («The Essential Principles of
the Wealth of Nations etc.», London 1797, p. 66).
} В дійсності остаточний результат той, що всі посідачі
товарів продають один одному свої товари на 10\% дорожче
понад їхню вартість, а це цілком те саме, як коли б вони продавали
товари за їхніми вартостями. Такий загальний номінальний
додаток до ціни товарів викликає такий самий ефект, як коли б,
наприклад, товарові вартості цінувалося на срібло замість золота.
Грошові назви, тобто ціни товарів, зростали б, але їхні
вартостеві відношення лишалися б незмінними.

Припустімо, навпаки, що покупець має привілей купувати
товари за ціни, нижчі від їхньої вартости. Тут нема вже навіть
потреби нагадувати, що покупець знову стає продавцем. Він
був продавцем раніш, ніж став покупцем. Він уже втратив як
продавець 10\% раніш, ніж виграв 10\% як покупець.\footnote{
«Коли хтось мусить віддати за 18 ліврів таку кількість продукту,
яка варта 24 ліври, то, вживши виручені гроші на купівлю, він так само
одержить за 18 ліврів те, за що раніш плати и 24 ліври» («Si L’on est
forcé de donner pour 18 livres une quantité de telle production qui en valait
24, lorsqu’on employera ce même argent à acheter, on aura également pour
18 ce que l’on payait 24»). (\emph{Le Trosne} «De l’Intérêt Social»,
Physiocrates, éd. Daire, Paris 1846, p. 897).
} Все лишається знову по-старому.

Отже, утворення додаткової вартости, а тому й перетворення
грошей на капітал, не можна пояснити ані тим, що продавці
продають товари понад їхню вартість, ані тим, що покупці купують
їх нижче від їхньої вартости.\footnote{
«Кожен продавець не може завжди підвищувати ціни на свої товари
інакше, як згодившись платити завжди дорожче за товари інших продавців;
і з тієї самої причини кожний споживач, звичайно, не може завжди
платити дешевше за те, що купує, інакше, як згодившись на таке саме зменшення
цін на ті речі, які продає» («Chaque vendeur ne peut donc parvenir
à renchérir habituellement ses marchandises, qu’en se soumettant aussi à
payer habituellement plus cher les marchandises des autres vendeurs; et
par la même raison, chaque consommateur ne peut payer habituellement
moins cher ce qu’il achète, qu’en se soumettant aussi à une diminution
semblable sur le prix des choses qu’il vend»). (\emph{Mercier de la Rivière}:
«L’Ordre naturel et essentiel», Physiocrates, éd. Daire, II. Partie, p. 555).
}

Проблема аж ніяк не стає простішою, коли крадькома занести
до неї чужі їй відношення, отже, коли, наприклад, разом із полковником
Торренсом скажемо: «Дійсний попит полягає у спроможності
та нахилі (!) споживачів, чи то шляхом безпосереднього
чи то посереднього обміну, давати за товари якусь більшу
кількість усіх складових частин капіталу, ніж коштує їхня продукція».\footnote{
\emph{R. Torrerts}: «An Essay on the Production of Wealth», London
1821, p. 349.
}
В циркуляції продуценти і споживачі протистоять
один одному лише як продавці й покупці. Твердити, що додаткова
вартість постає для продуцента з того, що споживачі платять
\index{i}{0115}  %% посилання на сторінку оригінального видання
за товари понад їхню вартість, значить лише замасковувати
просту тезу: посідач товарів як продавець має привілей продавати
товари дорожче, ніж вони варті. Продавець сам випродукував
товар або заступає його продуцента, але й покупець не меншою
мірою сам випродукував товар, виражений у його грошах,
або заступає його продуцента. Отже, продуцент протистоїть
продуцентові. Вони відрізняються тим, що один купує, а другий
продає. Ми не посунемось ані на крок далі, якщо посідач товарів
під фірмою продуцента продає товар понад його вартість, а під
фірмою споживача платить за товари дорожче, ніж вони варті.\footnote{
«Думка, що зиски виплачують споживачі, є, безумовно, цілком
безглузда. Хто такі ці споживачі?» («The idea of profits being paid by
the consumers, is, assuredly, very absurd. Who are the consumers?»).
(\emph{G. Ramsay}: «An Essay on the Distribution of Wealth», Edinburgh 1836, p.~183).
}

Тим то послідовні оборонці ілюзії, що додаткова вартість
постає з номінального додатку до ціни або з привілею продавця
продавати товари занадто дорого, мусять припустити існування
кляси, яка лише купує, не продаючи, отже, і лише споживає,
не продукуючи. Існування такої кляси з того нашого погляду,
якого ми досі дійшли, з погляду простої циркуляції, ще не може
бути з’ясоване. Але забіжімо наперед. Гроші, за які постійно
купує така кляса, мусять постійно припливати до неї від самих
посідачів товарів, без обміну, задурно, на підставі хоч якогось
права або насильства. Продавати цій клясі товари понад вартість
— значить лише обманою повертати собі частину задурно
відданих грошей.\footnote{
«Коли комусь бракує попиту, чи порадить йому пан Малтуз позичити
якійсь особі гроші, щоб ця остання купила в нього товари?» — питає
один обурений рікардіянець у Малтуза, що, як і його учень, піп Чолмерс,
вихваляє з економічного погляду клясу виключно покупців або виключно
споживачів. Див.: «An Inquiry into those principles respecting the Nature
of Demand and the Necessity of Consumption, lately advocated by Mr. Malthus
etc.», London 1821, p. 55.
} Так малоазійські міста виплачували стародавньому
Римові щорічну грошову данину. За ці гроші Рим купував
у них товари й купував їх занадто дорого. Малоазійці ошукували
римлян, виманюючи у своїх завойовників через торговлю
частину данини. А проте ошуканими лишались малоазійці. За
їхні товари платили їм, як і раніш, їхніми ж власними грішми.
Це не є метода для збагачення або творення додаткової вартости.

Отже, тримаймося в межах товарового обміну, де продавці
є покупці, а покупці — продавці. Наші труднощі походять,
може, з того, що ми розглядали контраґентів лише як персоніфіковані
категорії, а не індивідуально.

Посідач товарів $А$ може бути такий мудрагель, що завжди
зможе обдурити своїх колеґ $В$ й $С$, тимчасом як ці при найкращому
бажанні не спроможуться на реванш. $А$ продає $В$ вино вартістю
в 40 фунтів стерлінґів і дістає на обмін збіжжя вартістю в
50 фунтів стерлінґів. $А$ перетворив свої 40 фунтів стерлінґів
на 50 фунтів стерлінґів, з меншої кількости грошей зробив більшу
їх кількість і перетворив свій товар на капітал. Придивімось
\parbreak{}  %% абзац продовжується на наступній сторінці
