\parcont{}  %% абзац починається на попередній сторінці
\index{i}{0272}  %% посилання на сторінку оригінального видання
разом із звичкою і здібність провадити своє давнє ремество в
його повному розмірі. З другого боку, його зоднобічнена робота
набирає тепер у межах звуженої сфери діяльности якнайдоцільнішої
форми. Первісно каретна мануфактура з’явилась як комбінація
самостійних реместв. Поволі вона стає поділом каретної
продукції на її різні осібні операції, з яких кожна кристалізується
у виключну функцію одного робітника, а сукупність їх
виконується спілкою таких частинних робітників. Так само
постала сукняна мануфактура й цілий ряд інших мануфактур
із комбінації різних реместв під командою того самого капіталу.\footnote{
Щоб дати сучасніший приклад цього способу утворення мануфактури,
наводимо таку цитату. Шовкопрядіння й шовкоткацтво в
Ліоні та Німі «мають цілком патріярхальний характер; ці галузі промисловости
застосовують багато жінок та дітей, але не надривають їхніх
сил і не деморалізують їх; вони лишають їх у їхніх чудових долинах Дроми,
Вара, Ізери й Воклюзи для того, щоб вони розводили там черву й розмотували
її опряди; ніколи ця продукція не набирає характеру справжньої
фабрики. Щоб бути так добре додержаним\dots{} цей принцип поділу праці
набирає спеціяльного характеру. Існують свої спеціялісти — мотальниці,
сукальники, фарбівники, шліфувальники і, нарешті, ткачі, але вони не
зосереджені в тому самому будинку й не залежать від того самого хазяїна;
вони всі незалежні» («\dots{} est toute patriarcale; elle emploie beaucoup de
femmes et d’enfants, mais sans les épuiser ni les corrompre; elle les laisse
dans leurs belles vallées de la Drôme, du Var, de l’Isère, de Vaucluse, pour
y élever des vers et dévider leurs cocons: jamais elle n’entre dans une véritable
fabrique. Pout être aussi bien observé\dots{} le principe de la division
du travail, sy revêt d’un caractère spécial. Il y a bien des dévideuses, des
moulineurs, des teinturiers, des encolleurs, puis des tisserands; mais ils
ne sont pas réunis dans une même établissement, ne dépendent pas d’un
même maître: tous ils sont indépendants»), (A. Blanqui: «Cours l'Economie
industrielle. Recueilli par A. Blaise», Paris 1838--39, p. 79).
Від того часу як Блянкі це писав, цих різних незалежних робітників
почасти зосереджено на фабриках. [До 4 видання. — А від того часу
як Маркс це писав, на цих фабриках укоренився механічний ткацький
варстат і швидко витискує ручний. Крефельдська шовкова промисловість
може теж заспівати такої самої пісні. — Ф. Е.).
}

Але мануфактура виникає і протилежним шляхом. Той самий
капітал експлуатує одночасно і в тій самій майстерні багатьох
ремісників, які роблять ту саму або подібну роботу, приміром,
виробляють папір, або черенки, або голки. Це є кооперація в
найпростішій формі. Кожен з цих ремісників (може з одним
або двома підмайстрами) виготовляє цілий товар, отже, і виконує
послідовно різні операції, потрібні для його виготовлення.
Він працює й далі своїм старим ремісницьким способом.
Алеж зовнішні обставини незабаром примушують інакше використовувати
концентрацію робітників в тому самому приміщенні
й одночасність їхніх праць. Приміром, протягом якогось певного
часу треба постачити більшу кількість готового товару. Тому
працю розподіляють. Замість доручати тому самому ремісникові
виконувати різні операції послідовно в часі, ці операції відокремлюють
одну від одної, ізолюють, ставлять просторово одну поряд
одної, і кожну з них доручають окремому ремісникові, так що
всі ці операції разом одночасно виконують кооперовані робітники.
\parbreak{}  %% абзац продовжується на наступній сторінці
