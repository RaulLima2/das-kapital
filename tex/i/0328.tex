Ці гурти мандрують часто милями від своїх сел; вранці й увечері
можна їх зустріти на сільських шляхах; жінки одягнуті в короткі
спідниці та у відповідні кофти й чоботи, іноді у штани; жінки ці
дуже сильні та здорові з вигляду, але попсовані розпусними
звичками та безоглядні щодо тих згубних наслідків, які через
їхню пристрасть до такого активного й незалежного способу
життя спадають на їхніх дітей, що нидіють вдома».\footnote{
Там же, стор. 456.
} Всі явища
фабричних округ, між іншим, замасковане вбивання дітей та
годування дітей різними препаратами опію, репродукуються тут
у ще більшій мірі.\footnote{
Споживання опію дорослими робітниками та робітницями дедалі
більше поширюється в рільничих округах Англії так само, як у фабричних
округах. «Розвинути торговлю препаратами опію... є головна мета
деяких підприємливих гуртових торговців. Дроґісти визнають опій за
головний товар» (там же, стор. 459). Немовлята, яким дають опій, «скулюються
в маленькі діди або зморщуються в маленькі малпи» (там же,
cтop. 460). Ми бачимо, як Індія й Китай мстяться на Англії.
} «Моє обізнання з тим лихом, що його породжувала
всяка довга промислова праця дорослих жінок, мусить
виправдати мою глибоку огиду до неї», — каже д-р Сімон,
лікар-урядовець англійського Privy Council\footnote*{
— Таємної Ради. Ред.
} і головний редактор
звітів «Public Health».\footnote{
Там же, стор. 37.
} «Справді, — вигукує фабричний
інспектор Р. Бекер в одному офіціяльному звіті, — це буде
щастя для мануфактурних округ Англії, коли кожній заміжній
жінці, що має родину, заборонять працювати на будь-якій
фабриці».\footnote{
«Reports of Insp. of Fact, for 31 st October 1862», p. 59. Цей фабричний
інспектор раніше був лікарем.
}

Моральне покалічення, що випливає з капіталістичної експлуатації
жіночої та дитячої праці, так вичерпно змалював Ф. Енґельс
у своєму «Становищі робітничої кляси в Англії» та інші письменники,
що я тут лише згадую про це. Але інтелектуальне здичавіння,
штучно утворюване перетворенням недозрілих людей на
прості машини для фабрикації додаткової вартости, яке треба
глибоко відрізняти від того природного неуцтва, що лишає розум
облогом, не псуючи його здатностей розвиватися, самої його природної
родючости, — це здичавіння примусило, нарешті, навіть
англійський парламент зробити початкове навчання законною
умовою для «продуктивного» споживання дітей, молодших за
14 років, по всіх галузях промисловости, підлеглих фабричному
законові. Дух капіталістичної продукції ясно висвічує в недбайливій
редакції так званих пунктів про виховання у фабричних
законах, в браку того адміністративного механізму, без якого це
примусове навчання здебільша стає ілюзорним, в опозиції фабрикантів
навіть проти такого закону про навчання та в їхніх вивертах
і крутійствах, щоб оминути його на практиці. «Ганьбити
слід би саме тільки законодавство, бо воно видало оманливий
закон (delusive law), що, дбаючи на позір про виховання дітей