garding Industrial Questions and Trade’s Unions». Закордонні
представники англійської корони зазначають тут без околичностей,
що в Німеччині, Франції, одно слово, по всіх культурних
державах європейського континенту, переворот у наявних відносинах
між капіталом і працею так само помітний і так само неминучий,
як і в Англії. Одночасно, по той бік Атлантійського океану
пан Вейд, віце-президент Сполучених штатів Північної Америки,
заявив на публічних мітинґах: «По скасуванні рабства
на порядок дня стає переворот у відносинах капіталістичної й
земельної власності!» Це — ознаки часу, яких не сховаєш уже
ні під пурпуровою мантією, ні під чорною рясою. Це не значить,
що завтра станеться чудо. Вони показують, що навіть серед панівних
кляс прокидається прочуття, що теперішнє суспільство —
це не твердий кристал, а організм, який здатний до перетворень
і який постійно перебуває в процесі перетворення.

Другий том цієї праці розглядатиме процес циркуляції капіталу
(книга II) і форми (Gestaltungen) капіталістичного процесу
в цілому (книга III), останній, третій том (книга IV) — історію
теорій.

Кожний присуд наукової критики я радо вітатиму. Щождо
забобонів так званої громадської думки, перед якою я ніколи не
поступався, то за своє гасло вважатиму я, як і раніш, слова великого
фльорентійця: Segui il tuo corso, е lascia dir le genti!*

Карл Маркс

Лондон, 25 липня 1867 р.

* Прямуй своїм шляхом, а люди кажуть хай, що хочуть! Ред.
