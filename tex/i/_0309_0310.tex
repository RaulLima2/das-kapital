\parcont{}  %% абзац починається на попередній сторінці
\index{i}{0309}  %% посилання на сторінку оригінального видання
одна робоча машина, яка працює за допомогою комбінації різних
знарядь. Чи така робоча машина є лише механічне відродження
складнішого ремісничого знаряддя, чи вона є комбінація різнорідних
простих інструментів, спеціялізованих мануфактурою, на
фабриці, тобто в основаній на машиновому виробництві майстерні,
кожного разу знов з’являється проста кооперація, і насамперед
(робітника ми лишаємо тут осторонь) саме як зосередження у просторі
однорідних робочих машин, що одночасно разом функціонують.
Так, ткацька фабрика утворюється через сполучення
багатьох механічних ткацьких варстатів, а швацька фабрика —
через сполучення багатьох швацьких машин у тому самому робітному
приміщенні. Але тут існує технічна єдність, тому що це
велике число однорідних робочих машин одночасно та рівномірно
дістає свій рух від руху спільного першого мотора, руху, що
переноситься на них за допомогою передатного механізму, теж
почасти спільного всім їм, бо від нього розходяться лише осібні
відгалуження для кожної окремої виконавчої машини. Цілком
так само як численні знаряддя становлять лише органи однієї
робочої машини, так само й численні робочі машини становлять
тепер лише однорідні органи того самого рухового механізму.
Але система машин у власному значенні слова заступає
поодиноку самостійну машину лише там, де предмет праці послідовно
перебігає ряд зв’язаних між собою різних частинних
процесів, що їх поступінно виконує низка різнорідних виконавчих
машин, які однак одна одну доповнюють. Тут знову з’являється
характеристична для мануфактури кооперація, що ґрунтується
на поділі праці, але тепер уже як комбінація частинних
робочих машин. Специфічні знаряддя різних частинних робітників,
приміром, у вовняній мануфактурі знаряддя шаповалів,
чухральників вовни, стригунів вовни, прядунів вовни й т. ін.,
перетворюються тепер на знаряддя специфікованих робочих машин,
що з них кожна становить осібний орган для осібної функції
в системі комбінованого робочого механізму. Сама мануфактура
дає машиновій системі по тих галузях, де її заводиться вперше,
взагалі та в цілому природну основу поділу, а тому й організацію
процесу продукції.\footnote{
Перед епохою великої промисловости вовняна мануфактура була
домінантною мануфактурою Англії. Тим то за першу половину XVIII століття
саме в ній пороблено більшість експериментів. Досліди, пороблені
на овечій вовні, стали корисними і для бавовни, що її механічне
оброблення потребує менш тяжких підготовчих праць, так само як пізніше,
навпаки, механічна вовняна промисловість розвинулась на основі
механічного прядіння й ткання бавовни. Поодинокі елементи вовняної
мануфактури, як, наприклад, чесання вовни, заведено у фабричну систему
лише в останні десятиліття. «Вживання механічної сили до чесання вовни\dots{}
Дуже поширене від часів заведення «чесальної машини», особливої машини
Лістера\dots{} мало безперечно своїм наслідком те, що дуже велике число
людей лишилося без праці. Перше вовну розчісували руками здебільшого
вдома в чесальника. Тепер її звичайно розчісують на фабриці, і ручну
працю усунено, за винятком деяких осібних родів праці, в яких усе ще віддають
перевагу вовні, розчісаній руками. Багато з ручних чесальників
знайшли працю на фабриках, але продукт ручного чесальника такий незначний
супроти продукту машини, що дуже велике число чесальників
лишилося без праці». («The application of power to the process of combing
wool\dots{} extensively in operation since the introduction of the «combing
machine», especially Lister’s\dots{} undoubtedly had the effect of throwing
a very large number of men out of work. Wool was formerly combed by
hand, most frequently in the cottage of the comber. It is how very generally
combed in the factory, and hand labour is superseded, except in some particular
kinds of work, in which hand-combed wool is still preferred. Many
of the handcombers found employment in the factories, but the produce of
the handcomber bears so small a proportion to that of the machine, that
the employment of a very large number of combers has passed away»). («Reports
of Insp. of Fact, for 31st October 1856», p. 16).
} Однак відразу виступає посутня ріжниця.
\index{i}{0310}  %% посилання на сторінку оригінального видання
В мануфактурі робітники мусять окремо або групами виконувати
кожний окремий частинний процес своїм ручним знаряддям.
Якщо робітник і пристосовується тут до процесу, то й процес
теж уже перед тим пристосовано до робітника. За машинової
продукції цей суб’єктивний принцип поділу відпадає. Цілий
процес розкладається тут об’єктивно, розглядуваний сам по собі,
на свої складові фази, і проблема виконання кожного частинного
процесу та сполучення різних частинних процесів розв’язується
за допомогою технічного застосування механіки, хемії й т. д.,\footnote{
«Отже, принцип фабричної системи є в заміні\dots{} поділу або розчленування
праці між робітників розкладом процесу на його посутні
складові елементи» («The principle of the factory system, then, is to substitute\dots{}
the partition of a process into its essential constituents, lor the
division or gradation of labour among artizans»). (Ure: «Philosophy of
Manufacture», p. 20).
}
при чому, природно, теоретична концепція, як і раніш, мусить
бути вдосконалена через нагромаджений широкий практичний
досвід. Кожна частинна машина постачає для тієї що безпосередньо
йде за нею, її сировинний матеріял; а що всі вони функціонують
одночасно, то продукт так само безупинно перебуває на
різних ступенях процесу свого творення, як і в процесі переходу
з однієї фази продукції в іншу. Як у мануфактурі безпосередня
кооперація частинних робітників створює певні кількісні відношення
між окремими групами робітників, так і в розчленованій
системі машин те, що частинні машини одна одній постійно дають
роботу, створює певне відношення між їх числом, їх розміром
та їхньою швидкістю. Комбінована робоча машина, тепер розчленована
система різнорідних окремих робочих машин та груп
робочих машин, є то досконаліша, що безупинніший є цілий її
рух, тобто, що з меншими перервами переходить сировинний матеріял
від своєї першої фази до останньої фази, отже, що більше
сировинний матеріял пересувається з однієї фази продукції в
іншу фазу самим механізмом замість людської руки. Якщо в
мануфактурі ізоляція окремих процесів є принцип, даний самим
поділом праці, то в розвинутій фабриці панує, навпаки, безперервність
окремих процесів.

Система машин, хоч базується вона на простій кооперації
\parbreak{}  %% абзац продовжується на наступній сторінці
