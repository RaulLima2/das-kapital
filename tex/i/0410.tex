з такими ферментами перевороту та їхньою метою, знищенням
старого поділу праці. Однак розвиток суперечностей певної історичної
форми продукції — це єдиний історичний шлях її розкладу
й утворення нової. «Ne sutor ultra crepidam!»\footnote*{
— Шевче, тримайся колодок своїх. Ред.
} — це nec plus
ultra\footnote*{
— найвищий ступінь, апогей. Ред.
} ремісничої премудрости, стало страшенною дурістю від
того моменту, коли годинникар Ватт вигадав парову машину,
голяр Аркрайт — прядільну машину, ювелірний робітник Фултон
— пароплав.809

Оскільки фабричне законодавство реґулює працю по фабриках,
мануфактурах тощо, воно спочатку здається тільки втручанням
у експлуататорські права капіталу. Навпаки, всяке регулювання
так званої домашньої праці\footnote{
Вона, зрештою, здебільша має характер домашньої праці і в
дрібних майстернях, як ми це бачили в мануфактурі мережива і в плетінні
з соломи і як це можна було б докладніше показати особливо на металевих
мануфактурах у Шеффілді, Бермінґемі й т. ін.
} виявляється відразу ж
як прямий замах на patria potestas, тобто, висловлюючись сучасною
мовою, на батьківський авторитет, крок, що перед ним делікатний
і чутливий англійський парлямент з удаваним жахом
подавався назад. Однак сила фактів примусила, нарешті, визнати,
що велика промисловість руйнує разом з економічною основою
старої родини й відповідної їй родинної праці й самі старі родинні
відносини. Неминуче треба було оголосити право дітей. «На лихо,
— читаємо в кінцевому звіті «Children’s Employment Commission»
з 1866 р., — з усіх виказів свідків ясно, що ні від кого
не треба так дуже боронити дітей обох статей, як від їхніх власних
батьків». Система безмірної експлуатації дитячої праці взагалі
й домашньої праці зокрема тим «підтримується, що батьки нестримно
й безконтрольно використовують самовільну й нещадну
владу над своїми молодими й тендітними нащадками... Не можна
давати батькам абсолютної влади робити з своїх дітей просто
машини, щоб добувати з них стільки та стільки тижневого заро-

309    Джон Беллерс, справжній феномен в історії політичної економії,
ще наприкінці XVII віку з повного ясністю розумів доконечність знищити
теперішнє виховання й поділ праці, що породжують гіпертрофію й атрофію
на обох полюсах суспільства, хоч і в протилежному напрямі. Між
іншим, він чудово каже: «Вчитись у лінощах — це лише трохи щось ліпше,
ніж учитись лінощів... Фізична праця — це первісна божа установа...
Праця так само потрібна для здоров’я тіла, як харч для його життя;
бо ті неприємності, що їх людина уникає через лінощі, впадуть на неї
через недугу... Праця додає олії до лямпи шиття, думання запалює її...
Дурненька дитяча праця (пророчий закид проти Базедових і сучасних
тупих наслідувачів їх) лишає дитячий розум дурненьким». («An idle
learning being little better than the Learning of Idlenes... Bodily Labour,
it’s a primitive institution of God... Labour being as proper for the
bodies health, as eating is for its living; for what pains a man saves by
Ease, he will find in Disease... Labour adds oyl to the lamp of life when
thinking inflames it... A childish silly employ, leaves the children’s minds
silly»). («Proposals for raising a Colledge of Industry of all useful Trades
and Husbandry», London 1696, p. 12, 14, 18).