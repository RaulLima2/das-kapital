незгладиму ганьбу англійської робітничої кляси те, що на своєму
прапорі проти капіталу, що мужньо боровся, за «повну волю
праці», вона написала «рабство фабричних законів».\footnote{
Ure, французький переклад «Philosophie des Manufactures», Paris
1836, vol. II, p. 39, 40, 67, 77 etc.
}

Франція поволі плентається за Англією. Треба було Лютневої
революції, щоб з’явився на світ дванадцятигодинний закон,\footnote{
B Compte Rendu\footnote*{
— звіті. Ред.
} «Інтернаціонального Статистичного Конґресу
в Парижі, 1855 р.» сказано, між іншим: «Французький закон,
що обмежує тривання денної праці по фабриках і майстернях 12 годинами,
не встановлює для цієї праці певних сталих годин (періодів часу); лише
для дитячої праці приписано період між 5 годиною ранку й 9 годиною вечора.
Тим то частина фабрикантів користується з права, яке дає їм
це фатальне замовчування, щоб примушувати працювати день-у-день
без перерви, хіба що за винятком неділь. Вони вживають для цього двох
різних змін робітників, що з них жодна не перебуває в майстерні більш,
ніж 12 годин, хоч продукція в підприємстві триває і вдень і вночі. Вимоги
закону задоволено, але чи задоволено також вимоги гуманности?» Окрім
«руйнаційного впливу нічної праці на людський організм», підкреслюється
також «фатальний вплив спільного перебування вночі обох статей
у тих самих скупо освітлених майстернях».
}
що мав далеко більше хиб, ніж його англійський ориґінал. А, проте,
французька революційна метода виявляє теж свої своєрідні
переваги. Одним замахом вона диктує всім майстерням і фабрикам
однаково одну й ту саму межу робочого дня, тимчасом як
англійське законодавство проти волі то на цьому, то на тому
пункті робить під натиском обставин поступки, вишукуючи найпевніший
шлях, щоб вигадати якусь нову юридичну хитромудрість.\footnote{
«Наприклад, у моїй окрузі в тому самому фабричному будинку
той самий фабрикант як білильник і фарбар підлягає «законові про білильні
й фарбарні», як перкалевиробник — «Printwork’s Act’ові» і як finisher —
«фабричному актові»... («Report of Mr. Baker» у «Reports etc. for 31 st
Oct. 1861», p. 20). Перелічивши різні постанови цих законів і ускладнення,
що з того випливають, пан Бекер каже: «Ми бачимо, як важко забезпечити
виконання цих трьох парляментських законів, коли власник фабрики
захоче обійти закон». Але панам юристам це забезпечує процеси.
}
З другого боку, французький закон проголошує як
принцип те, що в Англії виборюють лише в ім’я дітей, неповнолітніх
і жінок, і чого тільки останніми часами домагаються як
загального права.\footnote{
Так, фабричні інспектори зважуються, нарешті, сказати: «Ці
заперечення (капіталу проти законного обмеження робочого часу) повинні
впасти перед широким принципом прав праці... є пункт, після якого право
хазяїна порядкувати працею свого робітника припиняється, і сам робітник
може порядкувати своїм часом навіть тоді, коли ще немає мови про повне
виснаження його» («These objections must succumb before the broad
principle of the rights of labour... there is a time when the master’s right
in his workmann’s labour ceases and his time becomes his own, even if
there was no exhaustion in the question»). («Reports etc. for 31 st Oct.
1862», p. 54).
}

У Сполучених штатах Північної Америки всякий самостійний
робітничий рух лишався паралізованим так довго, доки
одну частину республіки згиджувало рабство. Праця білошку-