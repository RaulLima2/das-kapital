\parcont{}  %% абзац починається на попередній сторінці
\index{i}{0480}  %% посилання на сторінку оригінального видання
в таких натуральних формах, що вже сами собою виключають
особисте споживання.

Якщо продукція має капіталістичну форму, то й репродукція
має таку саму форму. Так само як за капіталістичного способу
продукції процес праці є лише засіб для процесу зростання
вартости, так само й репродукція є лише засіб репродукувати
авансовану вартість як капітал, тобто як вартість, що самозростає.
Характеристична економічна маска капіталіста тримається
на якійсь людині лише тому, що її гроші безупинно функціонують
як капітал. Якщо, наприклад, авансована сума грошей
у 100 фунтів стерлінґів перетворилася цього року на капітал
і випродукувала додаткову вартість у 20 фунтів стерлінґів,
то найближчого й дальших років вона мусить повторити ту саму
операцію. Як періодичний приріст капітальної вартости, або
періодичний плід капіталу, що процесує, додаткова вартість набирає
форми доходу, який виникає з капіталу.\footnote{
«Багаті, що споживають продукти чужої праці, дістають їх лише
через акти обміну (купівлю товарів). Тому здається, що їхні резервпі
фонди повинні незабаром вичерпатися... Але в суспільному устрої багатство
набуло сили репродукувати себе за допомогою чужої праці... Багатство,
як і праця, та за допомогою праці, дає щорічно плід, який можна
щороку знищувати без того, щоб багатий зробився біднішим. Цей плід є
Дохід, що виникає з капіталу». (Sismondi: «Nouveaux Principes a Economie
Politique», vol. I, p. 81, 82).
}

Якщо цей дохід служить капіталістові лише за фонд споживання,
або якщо його споживають так само періодично, як і
добувають, то, за інших незмінних обставин, відбувається проста
репродукція. Хоч ця остання є просте повторення процесу продукції
в тому самому маштабі, однак це просте повторювання або
безперервність процесу продукції надає йому певних нових рис,
або, точніше кажучи, усуває ті позірні риси, що видаються властивими
йому, коли його розглядати як ізольований процес.

[Розгляньмо спочатку частину капіталу, авансовану на заробітну
плату, тобто змінний капітал].\footnote*{
Заведене y прямі дужки беремо з французького видання. Ред.
}

За вступний акт до процесу продукції є купівля робочої сили
на певний час, і цей вступний акт постійно поновлюється, скоро
тільки минає той термін, на який продано працю, отже і певний
період продукції, тиждень, місяць і т. д. Але робітникові платять
лише після того, як його робоча сила вже функціонувала і зреалізувала
в товарах так свою власну вартість, як і додаткову вартість.
Отже, робітник випродукував так додаткову вартість, яку
ми покищо розглядаємо лише як споживний фонд капіталіста,
як і фонд для своєї власної оплати, змінний капітал, — випродукував
раніше, ніж цей капітал приплив до нього назад у формі
заробітної плати, і робітника вживають до праці лише доти,
доки він постійно репродукує цей фонд своєї оплати. Звідси
згадана нами в шістнадцятому розділі під числом II формула
економістів, яка змальовує заробітну плату як пайку в самому
\parbreak{}  %% абзац продовжується на наступній сторінці
