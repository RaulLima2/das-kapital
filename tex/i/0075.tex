(зглядно купівель) або частинних метаморфоз, що в них ті самі
монети лише один раз змінюють місце, або пророблюють лише
один обіг, а з другого боку — багато почасти паралельних,
почасти посплітуваних між собою більш-менш багаторозгалужених
рядів метаморфоз, що в них ті самі монети пророблюють
більш або менш значну кількість обігів. Однак загальне число
обігів усіх однойменних монет, що перебувають у циркуляції,
дає пересічне число обігів окремих монет, або пересічну швидкість
грошового обігу. Маса грошей, що їх на початку, приміром,
денного процесу циркуляції кидають у нього, визначається,
певна річ, сумою цін товарів, що циркулюють одночасно й просторово
один побіч одного. Але в межах процесу одна монета
стає, так би мовити, відповідальною за інші. Коли одна прискорює
швидкість свого обігу, то цим затримується швидкість обігу
іншої або остання й зовсім вилітає із сфери циркуляції, бо ця
сфера може поглинути лише таку масу золота, яка, помножена
на пересічне число обігів поодиноких її елементів, дорівнює сумі
цін, що мають бути зреалізовані. Тому, коли зростає число обігів
монет, то маса їх, що перебуває в циркуляції, меншає. Коли
число обігів монет меншає, то маса їх зростає. Через те, що за
даної пересічної швидкости обігу маса грошей, яка може функціонувати
як засіб циркуляції, є дана, то досить лише кинути
в циркуляцію, приміром, певну кількість однофунтових банкнот,
щоб витягти з неї рівно стільки саме золотих соверенів, — трюк
добре відомий усім банкам.

Як в обігу грошей взагалі виявляється лише процес циркуляції
товарів, тобто їхній кругобіг через протилежні метаморфози,
так у швидкості грошового обігу виявляється швидкість зміни
товарових форм, безупинне встрявання одного ряду метаморфоз
в інший, сквапність обміну речовин, швидке зникання товарів
зі сфери циркуляції й так само швидка заміна їх новими товарами.
Отже, у швидкості обігу грошей виявляється поточна єдність
протилежних фаз, що одна одну доповнюють, перетворення
споживної форми на форму вартости і зворотне перетворення
форми вартости на споживну форму, або єдність обох процесів,
продажу й купівлі. Навпаки, в загаянні грошового обігу виявляється
відокремлення й усамостійнення цих процесів як протилежностей,
застій переміни форм, а тому і обміну речовин. Звідки
постає цей застій, цього, певна річ, з самої циркуляції пізнати
не можна. Вона показує лише саме явище. Вульґарний погляд,
помічаючи, що з загаянням грошового обігу гроші не так часто
з’являються і зникають на всіх пунктах периферії циркуляції,
шукає пояснення цього явища в недостатній кількості засобів
циркуляції.77

77 «Через те, що гроші становлять... загальну міру купівель і продажів,
кожний, хто має щось на продаж, але не находить покупця, схиляється
до думки, що брак грошей у королівстві або країні є причина,
через яку він не може збути свої товари, і таким чином усі скаржаться на
«брак грошей»; але це велика помилка... Чого хочуть ті, які кричать,
