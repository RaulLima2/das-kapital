б’є безперервно. Але постійний приплив до міст має своєю передумовою постійне лятентне перелюднення
в самих селах, що його розміри стають помітні лише тоді, коли вивідні канали відкриваються винятково
широко. Тим то плату сільського робітника
знижують до мінімуму і він завжди стоїть однією ногою в болоті павперизму.

Третя категорія відносного перелюднення, застійна, становить частину активної робітничої армії, але
разом із тим надзвичайна нереґулярність її занять дає капіталові невичерпний резервуар вільної
робочої сили. Її життєве становище падає нижче за
пересічний нормальний рівень робітничої кляси, і саме це робить її широкою основою осібних галузей
капіталістичної експлуатації. Її характеризують максимум робочого часу й мінімум заробітної плати.
Ми вже вивчили під рубрикою домашньої праці її головну форму. Вона рекрутується постійно із зайвих
робітників великої промисловости й рільництва, і особливо також з робітників галузей промисловости,
що гинуть, тих галузей, де ремісничу продукцію перемагає мануфактурна, мануфактурну — машинова. Її
розміри більшають у міру того, як з розмірами та енерґією акумуляції проґресує «творення зайвих»
робітників. Але разом з тим вона становить той елемент робітничої кляси, який сам себе репродукує й
увіковічнює і який бере порівняно більшу участь у загальному зростанні робітничої кляси, ніж решта
її елементів. Справді, не тільки число народжень і випадків смерти, але й абсолютна величина родин є
зворотно пропорційна до височини заробітної плати, отже, і до маси засобів існування, що ними
порядкують різні категорії робітників. Цей закон капіталістичного суспільства звучав би якимсь
безглуздям серед дикунів, а то й навіть серед цивілізованих колоністів. Він нагадує нам про масову
репродукцію індивідуально малосильних і жорстоко переслідуваних видів тварин.\footnote{
«Злидні... здається... сприяють розмножуванню» («Poverty... seems... favourable to generation»).
(A. Smith: «Wealth of Nations», b. I, ch. 8, p. 195). На думку ґалянтного й дотепного абата Ґаліяні
це навіть надзвичайно мудра установа божа: «Господь зробив так, що людей, які виконують дуже корисну
роботу, родиться найбільше» («Iddio fa che
gli uomini che esercitano mestieri di prima utilità nasconoabbondantemente»). (Galiani: «Della
Moneta», vol. III збірки Custodi «Scrittori Clas sici Italiani di Economia Politica». Parte Moderna.
Milano 1801, p. 78). «Злидні аж до крайніх меж голоду й епідемій не гальмують зросту людности, а
мають тенденцію абільшувати її» («Misery, up to the extreme point of famine and pestilence, instead
of checking, tends to increase population»).
(S. Laing: «National Distress», 1844, p. 69). Зілюструвавши це статистичними даними, Лен каже далі:
«Коли б усі жили в достатках, то земля швидко лишилася б без людей» («If the people were all in easy
circumstances, the world would soon be depopulated»).
}

сільських округ. Але тим часом, як у сільських округах людність протягом наступних десятьох років
зросла лише на півмільйона, в 580 містах вона зросла на 1.554.067. Приріст людности по селах
становить 6,5\%, по містах — 17,3\%. Ріжниця в нормі приросту є наслідок переселення з сел до міст.
Три чверті загального приросту людности припадає на міста». («Census etc.», vol. III, p. 11, 12).