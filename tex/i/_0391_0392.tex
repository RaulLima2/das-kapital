\parcont{}  %% абзац починається на попередній сторінці
\index{i}{0391}  %% посилання на сторінку оригінального видання
14--20 дітей і примушують їх 15 годин на добу займатися такою
роботою, яка сама по собі виснажує людину своєю нудністю та
монотонністю та до того ще й виконується в умовах, що якнайдужче
руйнують здоров’я. Навіть наймолодші діти працюють з
напруженою увагою та дивовижною швидкістю, майже ніколи
не дозволяючи своїм пальцям відпочити або рухатися повільніше.
Коли до них звертаються з запитаннями, то вони не підводять очей
від роботи, боячися втратити хоча б один момент». «Довга палиця»
служить для «mistresses» за засіб спонукати дітей до праці то
більше, що більше здовжується робочий час. «Діти поступінно
втомлюються та стають неспокійні, як птиці, під кінець того довгого
часу, протягом якого їх прив’язано до роботи, монотонної, шкідливої
для очей, виснажливої через одноманітність позиції тіла.
Це справжня рабська праця» («Their work is like slavery»).\footnote{
«Children’s Employment Commission. 2 nd. Report 1864» p. XIX.
XX XXI.
}
Там, де жінки працюють разом із своїми власними дітьми вдома
в сучасному значенні цього слова, тобто в найманій кімнаті,
часто в якійсь халупці на горищі, це становище ще гірше, якщо
це тільки можливо. Таку роботу роздають на 80 миль навколо
Нотінґему. Коли дитина, що працює в крамниці, виходить із неї
о дев’ятій або десятій годині вечора, то часто їй дають на дорогу
ще цілий клунок для того, щоб вона закінчила роботу вдома.
Капіталістичний фарисей, що його репрезентує один з його наймитів,
робить це, звичайно, зворушливо приказуючи: «це для
матері», алеж сам дуже добре знає, що нещасній дитині доведеться
самій присісти та допомагати матері.\footnote{
Там же, стор. XXI, XXVI.
}

Промисловість плетіння мережива поширена головним чином
у двох рільничих округах Англії: в мереживній окрузі Honiton,
20--30 миль поздовж південного берега Девонширу, включаючи
й небагато місць Північного Девону, та в другій окрузі, що
охоплює більшу частину графств Букінґему, Бедфорду, Нортґемптону
та сусідні частини Оксфордширу та Гетінґдонширу.
Котеджі рільничих поденників звичайно являють собою й приміщення
для праці. Деякі мануфактуристи вживають понад
3.000 таких домашніх робітників, головне, дітей та підлітків,
виключно жіночої статі. Тут повторюються ті умови, що ми їх
описали при розгляді lace finishing. Тільки замість «mistresses
houses» тут виступають так звані «lace schools» (школи мережива),
що їх у своїх халупках тримають бідні жінки. Починаючи від
п’ятого року, а іноді й раніше, і до 12 або 15 року, працюють
діти по цих школах; протягом першого року наймолодші працюють
від 4 до 8 години, а потім від 6 години ранку до 8 та 10 години
вечора. «Загалом кажучи, кімнати — це звичайні комірки невеличких
котеджів, камін у них забито, щоб не було протягу, мешканці
іноді й зимою огріваються лише теплом свого власного
тіла. В інших випадках ці так звані шкільні кімнати — це помешкання,
\index{i}{0392}  %% посилання на сторінку оригінального видання
подібні до маленьких комірок без опалення\dots{} Переповнення
цих закутків та зумовлена цим переповненням зіпсованість
повітря доходять часто крайнього ступеня. До цього ще
долучається шкідливий вплив стоків, клозетів, гнилі й іншого
бруду, що є звичайна річ при вході до невеликих котеджів». Щодо
помешкань, то «в одній школі мережива 18 дівчат і хазяйка,
35 кубічних футів на кожну особу; в іншій, де нестерпний сморід,
18 осіб, 24\sfrac{1}{2} кубічних фута на людину. Трапляється й таке, що
в цій промисловості вживають до праці дітей 2--2\sfrac{1}{2} років».261

Там, де в сільських графствах Букінґему та Бедфорду припиняється
плетіння мережива, починається плетіння в соломи. Воно
поширене по значній частині Гертфордширу та по західніх і північних
частинах Есексу. 1861 р. коло плетіння з соломи та коло
виготовлення солом’яних брилів працювало 40.043 особи, з них
3.815 чоловічої статі всякого віку, решта — жіночої статі, при
чому 14.913 молодші від 20 років, з них 7.000 дітей. Замість шкіл
мережива з’являються тут «straw plait schools» (школи плетіння
з соломи). Тут дітей починають учити плести з соломи звичайно
від 4 року, іноді між 3 та 4 роком життя. Виховання вони, звичайно,
не дістають ніякого. Сами діти називають початкові школи
«natural schools» (натуральними школами) відмінно від тих кровососних
установ, де їх тримають за працею просто для того, щоб
вони виготовили роботу, наказану їм від їхніх напівзголоднілих
матерів — здебільша 30 ярдів на день. Ці матері потім часто
примушують їх працювати ще вдома до 10, 11, 12 години вночі.
Солома ріже їм пучки й рот, яким вони її постійно змочують.
Згідно з загальним поглядом медичних урядовців Лондону, що
його зрезюмував д-р Беллярд, 300 кубічних футів на кожну
особу становлять мінімум об’єму для спальні або робітної
кімнати. Але в школах плетіння з соломи помешкання ще менші,
ніж у школах мережива, а саме на кожну особу в них припадає
12\sfrac{2}{3}, 17, 18\sfrac{1}{2} і менше ніж 22 кубічні фути. «Менші з цих чисел, —
каже комісар Байт, — дають менший об’єм, ніж половина того,
що його заняла б дитина, коли б її запакувати в коробку з вимірами
по 3 фути кожний». Отакі радощі життя дітей до 12 або
14 років. Бідні, занепалі батьки думають тільки про те, щоб
якомога більше видушити з дітей. Діти, вирісши, звичайно,
зовсім не дбають про своїх батьків та кидають їх. «Немає нічого
дивного в тому, що неуцтво й розпуста панують серед людности,
так вихованої\dots{} Її моральність є на щонайнижчому щаблі\dots{} Багато
жінок має нешлюбних дітей, і деякі з них у такому недозрілому
віці, що сами знавці кримінальної статистики з дива німіють
перед цим фактом».262 І батьківщина цих зразкових родин є
Англія — зразкова християнська країна Европи, як каже граф
Монталямбер, — людина, певна річ, найкомпетентніша в справах
християнства!
