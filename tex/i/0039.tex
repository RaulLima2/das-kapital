новлять основу даного суспільства, праці та її продуктам не потрібно
набирати відмінної від їхньої реальности фантастичної
форми. Вони увіходять у суспільний рух як натуральні служби
і натуральні повинності. За безпосередню суспільну форму праці
є тут її натуральна форма, її осібність, а не її загальність, як це
є на основі товарової продукції. Панщанну працю так само добре
вимірюється часом, як і працю, що продукує товари, але кожний
кріпак знає, що це певна кількість його особистої робочої сили,
яку він витрачає на службі своєму панові. Десятина, яку він має
віддавати попові, є для нього ясніша, ніж благословення попове.
Тому, хоч би й що думати про характеристичні маски, що в них
тут люди протистоять одні одним, в усякому разі суспільні відносини
осіб у їхній праці виявляються тут як їхні власні особисті
відносини, і не є вони переодягнуті в суспільні відносини речей,
продуктів праці.

Щоб розглянути спілкову, тобто безпосередньо усуспільнену
працю, нам не треба звертатись до її природно вирослої форми,
яку ми подибуємо на порозі історії всіх культурних народів.\footnote{
Примітка до другого видання. «Останніми часами поширився
смішний забобон, нібито форма первісної громадської власности є специфічна
слов’янська, навіть виключно російська форма. Це є та праформа,
яку ми можемо довести в римлян, германців, кельтів, а в індійців ще й нині
ми знаходимо цілу низку різноманітних зразків цієї форми, хоч уже
почасти й зруйнованих. Глибше студіювання азійських, а особливо індійських
форм громадської власности показало б нам, як із різних форм
первісної громадської власности постають різні форми її розпаду. Так,
наприклад, різні оригінальні типи римської й германської приватної
власности можна вивести з різних форм індійської громадської власности».
(К. Marx: «Zur Kritik der Politischen Oekonomie», S. 10. — K. Маркс.
«До критики політичної економії», ДВУ, 1926 р., стор. 51).
}
Ближчий приклад дає нам сільська патріярхальна індустрія селянської
родини, що для власних потреб продукує хліб, худобу,
пряжу, полотно, одяг і т. ін. Ці різні речі протистоять цій родині
як різні продукти її родинної праці, але вони не протистоять
одна одній як товари. Різні праці, що витворюють ці продукти, —
рільництво, скотарство, прядільництво, ткацтво, кравецтво і т. ін. —
є суспільні функції у своїй натуральній формі, бо це функції
родини, яка має свій власний природно вирослий поділ праці
так само, як і товарова продукція. Ріжниця статі й віку, як і
зміни природних умов праці, зумовлені зміною пір року, реґулюють
розподіл праці між членів родини й робочий час поодиноких
членів родини. Але витрата індивідуальних робочих сил,
вимірювана часом її тривання, вже від самого початку з’являється
тут як суспільне визначення самих праць, бо індивідуальні робочі
сили від самого початку функціонують тут лише як органи спільної
робочої сили родини.

Нарешті, уявімо собі, для різноманітности, товариство вільних
людей, що працюють спільними засобами продукції і свідомо
витрачають свої численні індивідуальні робочі сили як одну суспільну
робочу силу. Всі визначення робінзонової праці повторю-