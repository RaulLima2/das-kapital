характеристичної форми якоїсь осібної епохи розвитку капіталістичного
способу продукції. Щонайбільше, вона є приблизно
така на початках мануфактури,\footnote{
«Хіба поєднання вправности, працьовитости та змагання багатьох,
що виконують ту саму працю, не є спосіб посувати наперед цю працю?
І хіба Англія могла б якимось іншим способом довести свою вовняну мануфактуру
до такої досконалости?» («Whether the united skill, industry
and emulation of many together on the same work be not the way to advance
it? And whether it had been otherwise possible for England, to have
carried on her Woollen Manufacture to so great perfection?»). (Berkeley: «The
Querist», London 1750, p. 56, § 521).
} зорганізованої ще на ремісничий
штиб, та в тих великих рільничих господарствах, які відповідають
епосі мануфактури й посутньо відрізняються від селянського
господарства тільки масою одночасно вживаних робітників
та розміром сконцентрованих засобів продукції. Проста кооперація
все ще є переважна форма по таких галузях продукції, де
капітал оперує у великому маштабі, а поділ праці й машини не
відіграють значної ролі.

Кооперація лишається основною формою капіталістичного способу
продукції, хоч проста її форма сама з’являється як осібна
форма поряд інших розвиненіших її форм.

Розділ дванадцятий
Поділ праці та мануфактура
1. Двояке походження мануфактури

Кооперація, що ґрунтується на поділі праці, утворює собі
свою клясичну форму в мануфактурі. Як характеристична форма
капіталістичного процесу продукції вона домінує протягом мануфактурного
періоду у власному значенні, який триває приблизно
від середини ХVІ віку до останньої третини XVIII.

Мануфактура виникає двояким способом.

Або робітників різнорідних самостійних реместв, що через
їхні руки мусить переходити продукт аж до останньої стадії його
виготовлення, згуртовують в одній майстерні під командою того
самого капіталіста. Приміром, карета була продуктом спільної
праці великого числа незалежних ремісників, як от стельмаха,
римаря, кравця, слюсаря, мідяра, токаря, позументаря, скляра,
маляра, лаківника, позолотника і т. ін. Каретна мануфактура
сполучає всіх цих різних ремісників в одній майстерні, де вони
одночасно спільно працюють. Правда, карету не можна золотити
раніш, ніж її зроблено. Але якщо одночасно роблять багато карет,
то одну якусь частину можна завжди золотити, тимчасом як
інша частина пробігає ранішу фазу продукційного процесу. До
цього часу ми все ще стоїмо на ґрунті простої кооперації, яка
находить готовим свій людський та речовий матеріял. Алеж
дуже скоро настає ґрунтовна зміна. Кравець, слюсар, мідяр і
т. ін., що працює тільки коло карет, утрачає крок за кроком