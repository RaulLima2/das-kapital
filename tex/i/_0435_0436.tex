\parcont{}  %% абзац починається на попередній сторінці
\index{i}{0435}  %% посилання на сторінку оригінального видання
її ринкової ціни (!), він начебто авансує ріжницю своєму підприємцеві
(?) і т. ін.».9а В дійсності робітник авансує даром
капіталістові свою працю протягом тижня і т. д., з тим, щоб
наприкінці тижня і т. ін. одержати її ринкову ціну; за Міллом,
це перетворює робітника на капіталіста! На пласкій рівнині
й грудка землі видається горбом; пласкість нашої сучасної буржуазії
можна зміряти калібром її «великих мислителів».

Розділ п’ятнадцятий

Зміна величини ціни робочої сили та додаткової
вартости

[У третьому відділі, сьомий розділ, ми аналізували норму
додаткової вартости, але лише з погляду продукції абсолютної
додаткової вартости. У четвертому відділі ми знайшли додаткові
визначення. Тут нам треба зрезюмувати все посутнє про це].\footnote*{
Заведене у прямі дужки ми беремо з другого німецького видання.
\emph{Ред.}
}

Вартість робочої сили визначається вартістю звичних доконечних
засобів існування пересічного робітника. Маса цих засобів
існування, хоч форма їхня і може змінятись, для даної епохи
й даного суспільства є дана, а тому її треба розглядати як сталу
величину. Змінюється лише вартість цієї маси. Ще два інші
фактори, входять у визначення вартости робочої сили. З одного
боку, витрати на її розвиток, які змінюються із зміною способу
продукції, з другого боку, природні ріжниці робочої сили, тобто,
чи є вона чоловіча або жіноча, дорослих робітників або підлітків.
Споживання цих різних робочих сил, знову ж таки зумовлюване
способом продукції, створює велику ріжницю у витратах репродукції
робітничої родини та у вартості дорослого робітникачоловіка.
Однак у дальшому досліді обидва ці фактори не береться
на увагу.\footnoteA{
Розглянутий на стор. 253--254 випадок, тут, природно, також
виключено. (Примітка до третього видання — Ф. Е.).
}

Ми припускаємо: 1) що товари продається за їхньою вартістю;
2) що ціна робочої сили може іноді піднестися понад свою вартість,
але ніколи не падає нижче від неї.

Зробивши такі припущення, ми виявили, що відносні величини
ціни робочої сили й додаткової вартости визначаються
трьома обставинами: 1) довжиною робочого дня, або екстенсивною
величиною праці; 2) нормальною інтенсивністю, або її інтенсивною
величиною, тобто тією обставиною, що певну кількість
праці витрачається за певний час; 3) нарешті, продуктивною
силою праці, тобто тією обставиною, що та сама кількість праці
за той самий час дає, залежно від ступеня розвитку умов про-

9a J. St. Mill: «Principles of Political Economy», London 1868,
p. 252, 253 passim. (Вищенаведені місця перекладено з французького
видання «Капіталу». — Ф. Е.).
\index{i}{0436}  %% посилання на сторінку оригінального видання
дукції, більшу або меншу кількість продукту. Очевидно, можливі
дуже різні комбінації, відповідно до того, чи один із трьох факторів
сталий, а два змінюються, чи два фактори сталі, а один
змінюється, чи, нарешті, всі три змінюються одночасно. Число
цих комбінацій збільшується ще й через те, що за одночасної
зміни різних факторів величина й напрям змін можуть бути
різні. Далі ми розглядаємо лише головні комбінації.

І. Величина робочого дня й інтенсивність праці сталі (дані),
продуктивна сила праці змінюється

При цьому припущенні вартість робочої сили й додаткової
вартости визначається трьома законами:

По-перше, робочий день даної величини завжди виражається
в тій самій новоспродукованій вартості, хоч би й як змінювалася
продуктивність праці і разом з нею маса продуктів, а тому й
ціна поодинокого товару.

Новоспродукована вартість дванадцятигодинного робочого
дня, є, наприклад, 6 шилінґів, хоч маса спродукованих споживних
вартостей змінюється з продуктивною силою праці, і вартість
6 шилінґів розподіляється, отже, на більшу або меншу кількість
товарів.\footnote*{
У французькому виданні останні два абзаци подано так: «По-перше,
робочий день даної величини продукує завжди ту саму вартість,
хоч би й як змінювалася продуктивність праці.

Якщо одна година праці нормальної інтенсивности продукує вартість
у \sfrac{1}{2} шилінґа, то дванадцятигодинний робочий день може спродукувати
лише вартість у 6 шилінґів. (Ми припускаємо завжди, що вартість грошей
лишається незмінна). Якщо продуктивність праці підвищується або зменшується,
то той самий робочий день дасть більше або менше продуктів,
і вартість у 6 шилінґів розподілиться таким чином на більшу або меншу
кількість товарів». («Le Capital etc.», v. I, ch. XVII, p. 224). \emph{Ред.}
}

По-друге, вартість робочої сили й додаткова вартість змінюються
в протилежному напрямі. Зміна продуктивної сили праці,
її зростання або зменшення, впливає на вартість робочої сили у
зворотному напрямі, на додаткову вартість — у простому.

Новоспродукована вартість дванадцятигодинного робочого
дня є стала величина, наприклад, 6 шилінґів. Ця стала величина
дорівнює сумі додаткової вартости плюс вартість робочої сили,
яку робітник заміщує еквівалентом. Само собою зрозуміло, що
з двох частин сталої величини жодна не може збільшитися без
того, щоб друга не зменшилася. Вартість робочої сили не може
підвищитися з 3 шилінґів до 4 без того, щоб додаткова вартість
не знизилася з 3 шилінґів до 2, а додаткова вартість не може
підвищитися з 3 шилінґів до 4 без того, щоб вартість робочої
сили не знизилася з 3 шилінґів до 2. Отже, за цих обставин неможлива
ніяка зміна абсолютної величини ані вартости робочої
сили, ані додаткової вартости без одночасної зміни їхніх відносних
або пропорціональних величин. Неможливо, щоб вони одночасно
падали або підносилися.
