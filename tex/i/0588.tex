або спромоги платити відповідну квартирну плату, а від того
вжитку, який будуть ласкаві зробити інші «із свого права порядкувати
своєю власністю так, як їм забажається». Хоч би яка
велика була фарма, немає такого закону, що вимагав би збудувати
на ній певну кількість помешкань для робітників, не кажучи
вже зовсім про пристойність цих помешкань; так само
закон не дає робітникові найменшого права на ту землю, для якої
його праця так само доконечна, як дощ і сонячне світло... Ще
одна загальновідома обставина кидає на терези тяжку вагу проти
нього... Це — вплив закону про бідних з його постановами про
оселення й податок на користь бідним.162 Під його впливом кожна
парафія має грошовий інтерес у тому, щоб обмежити на мінімумі
число сільських робітників, які живуть у ній, бо, на нещастя,
рільнича праця, замість ґарантувати сільському робітникові,
що тяжко працює, та його родині певну й постійну незалежність,
веде його здебільша, довшим або коротшим обхідним
шляхом, до павперизму, — павперизму, до якого протягом цілого
того шляху робітник так стоїть близько, що всяка хороба
або якийсь тимчасовий брак праці примушують його одразу звертатися
по допомогу до парафії; і тим то кожне оселення рільничої
людности в якійсь парафії, очевидно, є для неї збільшення
податку на користь бідним. Великим землевласникам 168 досить
лише вирішити, що в їхніх маєтках не повинно бути жител для
робітників, — і вони одразу звільняються від половини своєї
відповідальности за бідних... У якій мірі англійська конституція
й закони мали за мету встановити такого роду безумовну
земельну власність, що дає лендлордові силу «робити з своєю
власністю що йому забажається», поводитися з рільниками
як із чужоземцями і проганяти їх із своєї території, — це питання,
що обговорення його не входить у рамки моїх завдань...
Це право виганяти — не просто теорія. Воно реалізується
на практиці в якнайбільшому маштабі. Це одна з обставин, що
мають вирішальний вплив на житлові умови сільського робітника...
Про розміри лиха можна судити на основі останнього
перепису, який показав, що протягом останніх 10 років руйнування
домів, не зважаючи на дедалі більший місцевий попит на
них, проґресувалй у 821 різних округах Англії; таким чином в
1861 р. людність, яка проти 1851 р. зросла на 5\sfrac{1}{3}\%, при чому
ми зовсім не беремо на увагу осіб, що примушені жити не в тих
парафіях, де вони працюють, — позбивано в помешкання, площа
яких зменшилась на 4\sfrac{1}{2}\%... Скоро тільки процес вилюднення
завершується, — каже д-р Гентер, — у результаті його постає
показне село (showvillage), де число котеджів зведено до незнач-

162 1865 р. цей закон дещо поліпшено. Досвід незабаром нам покаже,
що така латанина ані трішечки не допомагає.

163 Щоб зрозуміти дальшу цитату, зазначимо, що close villages (закритими
селами) називають такі села, що їхні власники один
або два великі лендлорди, a open villages (відкритими селами) — такі,
що їхні землі належать багатьом дрібним власникам. Саме в цих останніх
будівельні спекулянти і можуть будувати котеджі й хати.
