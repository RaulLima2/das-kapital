плуатованих тим самим капіталістом, і ступенем експлуатації
поодинокої робочої сили.\footnote*{
У французькому виданні цей закон подано так: «Сума додаткової
вартости, випродукована змінним капіталом, дорівнює вартості цього
авансованого капіталу, помноженій на норму додаткової вартости, абож
вона дорівнює вартості робочої сили, помноженій на ступінь її експлуатації,
помноженій на число одночасно вживаних робочих сил». («Le Capital
etc.», v. I, ch. XI, p. 131). Peд.
}

Отже, коли ми масу додаткової вартости назвемо М, додаткову
вартість, що її пересічно дає поодинокий робітник за день, — m,
змінний капітал, щоденно авансовуваний на купівлю однієї робочої
сили, — v, цілу суму змінного капіталу, — V, вартість пересічної
робочої сили — k, ступінь її експлуатації — a'/a (додаткова праця/докончена праця),
а число вживаних робітників — n, то матимемо:

M = m/v × V k × a'/a × n

Ми завжди припускаємо, що не лише вартість пересічної робочої
сили є стала, але що і вживані капіталістом робітники зведені
на пересічного робітника. Бувають виняткові випадки, коли
випродукована додаткова вартість зростає не пропорційно до числа
експлуатованих робітників, але тоді й вартість робочої сили не
лишається сталою.

[Здобуток не змінює своєї числової величини, коли множники
його одночасно змінюються у зворотному напрямі»].\footnote*{
Заведене у прямі дужки ми беремо з французького видання. Ред.
}

Тим то в продукції певної маси додаткової вартости зменшення
одного фактора може бути компенсоване збільшенням другого.
Коли змінний капітал меншає, а одночасно в тій самій пропорції
більшає норма додаткової вартости, то маса продукованої додаткової
вартости лишається незмінна. Коли, в умовах попередніх
припущень, капіталіст мусить авансувати 100 талярів, щоб денно
експлуатувати 100 робітників, і коли норма додаткової вартости
становить 50\%, то цей змінний капітал у 100 талярів дає додаткову
вартість у 50 талярів, або ж у 100 х 3 робочі години. Коли
норма додаткової вартости подвоюється, або робочий день здовжується
не від 6 до 9, а від 6 до 12 годин, то зменшений наполовину
змінний капітал у 50 талярів дає так само додаткову
вартість у 50 талярів, або у 50 х 6 робочих годин. Отже,
зменшення змінного капіталу може вирівнюватися пропорційним
підвищенням ступеня експлуатації робочої сили, або зменшення
числа вживаних робітників вирівнюється пропорційним здовженням
робочого дня. Отже, подання праці, яке може вимушувати