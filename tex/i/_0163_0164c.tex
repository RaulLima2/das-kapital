\parcont{}  %% абзац починається на попередній сторінці
\index{i}{0163}  %% посилання на сторінку оригінального видання
то він мусив би, за інших незмінних обставин, працювати, як і раніш, пересічно ту саму відповідну
частину дня, щоб спродукувати вартість своєї робочої сили й таким чином придбати засоби існування,
потрібні для його власного існування й постійної репродукції. Але через те, що протягом тієї частини
робочого дня,
коли він продукує денну вартість робочої сили, приміром, 3 шилінґи, він продукує лише еквівалент її
вартости, яку вже виплатив йому капіталіст,\footnoteA{
[Примітка до третього видання. Автор уживає тут загальнопоширеної економічної мови. Нагадаймо,
що на сторінці 126 показано, як у дійсності не капіталіст «авансує» робітника, а робітник
капіталіста. — Ф. Е.].
} отже, лише компенсує новоствореною вартістю
авансовану змінну капітальну вартість, то ця продукція вартости з’являється як проста репродукція.
Отже, ту частину
робочого дня, протягом якої відбувається ця репродукція, я називаю доконечним робочим часом, а
працю, витрачену
за цей час, — доконечною працею.\footnote{
Досі ми вживали в цьому творі слова «доконечний робочий час» на означення того робочого часу, що
взагалі є суспільно-доконечний для продукції якогось товару. Відтепер ми вживатимемо його й щодо
того
робочого часу, який є доконечний для продукції специфічного товару — робочої сили. Вживати тих самих
termini technici\footnote*{
— технічних термінів. \emph{Ред.}
} в різних значеннях незручно, але цілком уникнути цього не можна в жодній науці.
Порівн. приміром, вищу й нижчу математику.
} Доконечною для робітника, бо ця праця не залежить від суспільної
форми його праці.
Доконечною для капіталу й капіталістичного світу, бо база цього
світу є постійне існування робітника.

Другий період процесу праці, протягом якого робітник працює поза межами доконечної праці, щоправда,
коштує йому
праці, витрати робочої сили, але не утворює для нього жодної вартости. Цей період утворює додаткову
вартість, що всміхається до капіталіста принадою створення з нічого. Цю частину робочого
дня я називаю додатковим робочим часом, а витрачену протягом його працю — додатковою працею (surplus
labour). Так само, як для пізнання вартости взагалі має вирішальне значення розглядати її просто як
згусток робочого часу, як лише упредметнену працю, так для пізнання додаткової вартости має
вирішальне
значення розглядати її просто як згусток додаткового робочого часу, як лише упредметнену додаткову
працю. Лише та форма, що в ній цю додаткову працю витискується з безпосереднього продуцента,
робітника, відрізняє економічні суспільні формації, приміром, суспільство рабства від суспільства
найманої праці.\footnote{
Пан Вільгельм Тукідід Рошер із справді ґотшедівською геніяльністю відкриває, що коли утворення
додаткової вартости або додаткового продукту і сполучену з цим акумуляцію ми завдячуємо нині
«ощадності» капіталіста, який «вимагає за це, приміром, процента», то, навпаки, «на
найнижчих ступенях культури\dots{} сильніші приневолюють слабших до ощадности». («Die Grundlagen der
Nationalökonomie». 3 Auflage. 1858, S. 78). До заощаджешшя праці? чи до заощадження надлишкових
продуктів, яких немає? Поруч із справжнім неуцтвом апологетичний страх перед сумлінною аналізою
вартости й додаткової вартости, а також і перед можливістю небезпечного й непринятного для поліції
результату, — ось
}

\index{i}{0164}  %% посилання на сторінку оригінального видання
Через те, що вартість змінного капіталу дорівнює вартості купленої ним робочої сили, через те, що
вартість цієї робочої
сили визначає доконечну частину робочого дня, а додаткову вартість із свого боку визначає надлишкова
частина робочого дня, то звідси випливає: додаткова вартість відноситься до змінного капіталу, як
додаткова праця до доконечної, абож норма додаткової вартости m/v = додаткова праця/доконечна праця.
Обидві пропорції виражають те саме відношення в різній формі: раз у формі упредметнецої
праці, другий — у формі текучої праці.

Тому норма додаткової вартости є точний вираз ступеня експлуатації робочої сили капіталом, або
робітника капіталістом.\footnoteA{
Примітка до другого видання. Хоч норма додаткової вартости є точний вираз ступеня експлуатації
робочої сили, вона однак не є вираз абсолютної величини експлуатації. Приміром, коли доконечна праця
= 5 годинам і додаткова праця = 5 годинам, то ступінь експлуатації дорівнює
100\%. Величину експлуатації вимірюється тут 5 годинами. Коли ж доконечна праця дорівнює 6 годинам і
додаткова праця 6 годинам, то ступінь експлуатації в 100\% лишається незмінний, тимчасом як величина
експлуатації зростає на 20\%, з 5 на 6 годин.
}

За нашим припущенням, вартість продукту дорівнювала 410 фунтам стерлінґів + 90 фунтів стерлінґів +
90 фунтів стерлінґів,
авансований капітал дорівнював 500 фунтам стерлінґів. Через те, що додаткова вартість дорівнює 90 і
авансований капітал 500, то на основі звичайного способу обчислення вийшло б, що норма
додаткової вартости (яку плутають із нормою зиску) дорівнює 18\%, — процент, низький рівень якого міг
би зворушити пана Кері й інших гармоністів. В дійсності ж норма додаткової вартости є не m/С, або
m/с + v, отже не \sfrac{90}{500}, а \sfrac{90}{90} = 100\%, більш ніж уп’ятеро проти того, що на позір становить
ступінь експлуатації. Хоч у даному разі ми не знаємо ні абсолютної величини робочого дня, ні періоду
процесу праці (день, тиждень і т. д.), ні, нарешті, числа робітників, що їх одночасно пускає в рух
змінний капітал у 90 фунтів стерлінґів, все ж норма додаткової вартости m/v тим, що вона може бути
перетворена на формулу
додаткова праця / доконечна праця, докладно показує нам взаємне відношення обох складових частин
робочого дня. Воно дорівнює 100\%. Отже, робітник одну половину дня працює на себе, другу на
капіталіста. Отже, метода обчислення норми додаткової вартости, коротко
кажучи, така: ми беремо цілу вартість продукту й припускаємо, що стала капітальна вартість, яка лише
знов у з’являється у вартості продукту, дорівнює нулеві. Сума вартости, яка після цього

що приневолює Рошера й К° більш або менш імовірні мотиви, що ними капіталіст виправдує присвоєння
наявної додаткової вартости, перекручувати на причини її постання.
\parbreak{}  %% абзац продовжується на наступній сторінці
