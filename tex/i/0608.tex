частина пустирів і торфовищ, що їх раніш не використовували,
служить тепер для поширення скотарства. Дрібні й середні
фармери — я залічую сюди всіх тих, що обробляють не більше
як 100 акрів землі — все ще становлять приблизно \sfrac{8}{10} із загального
числа. 186c  Конкуренція капіталістичної рільничої продукції
щораз більше й більше душить їх, і тим то вони постійно постачають
клясі найманих робітників нових рекрутів. Однісінька
велика промисловість Ірляндії, фабрикація полотна, потребує
порівняно мало дорослих робітників-чоловіків і, не зважаючи
на її поширення після подорожчання бавовни в 1861—1866 рр.,
вона взагалі вживає лише порівняно незначну частину людности.
Як усяка інша велика промисловість, вона постійними коливаннями
у своїй власній сфері постійно продукує відносне перелюднення,
навіть і за абсолютного зростання маси робітників, яку
вона поглинає. Злидні сільської людности є за п’єдестал для велетенських
фабрик сорочок тощо, робітнича армія яких розпорошена
здебільшого по селах. Тут ми знову таки бачимо змальовану
раніш систему домашньої праці — систему, де недостатня
плата за роботу і надмірна праця служать за методичні засоби
продукувати «зайвих» робітників. Нарешті, хоч зменшення людности
не має тут таких руйнаційних наслідків, як у країні з розвиненою
капіталістичною продукцією, проте й тут воно відбувається
не без постійного зворотного впливу на внутрішній ринок.
[Еміґрація лишає по собі не лише порожні будинки, але й зруйнованих
квартироздавців].* * Та прогалина, що її створює тут
еміґрація, не лише зменшує місцевий попит на працю, але також
і доходи дрібних крамарів, ремісників, взагалі дрібних промисловців.
[Кожне нове виселення перетворює частину дрібної середньої
кляси на пролетарів].* Звідси зменшення доходів між
60 і 10 фунтами стерлінґів у таблиці Е.

Ясну картину становища сільських поденників в Ірляндії
ми маємо у звітах інспекторів ірляндської адміністрації в справах
про бідних (1870).\footnoteA{
«Reports from the Poor Law Inspectors on the wages of Agricultural
Labourers in Dublin, 1870». Порівн. також «Agricultural Labourers
(Ireland) Return etc. dated 8th March 1861», London, 1862.
} Урядовці такого уряду, що тримається
лише за допомогою баґнетів і стану облоги, то явного, то прихованого,
мусять бути обережними у висловах, чим їхні колеґи
в Англії нехтують; а проте вони не дозволяють своєму урядові
уколисувати себе ілюзіями. За їхніми відомостями, рівень
заробітної плати на селі, і досі все ще дуже низький, все ж за
останні двадцять років підвищився на 50—60\% і становить
тепер пересічно 6—9 шилінґів на тиждень. Але за цим позір-

186с Примітка до другого видання. Згідно з однією таблицею в Murphy:
«Ireland, Industrial, Political and Social», 1870, 94,6\% усіх земель
є фарми, менші від 100 акрів кожна і 5,4\% — фарми понад 100 акрів.

* Заведене у прямі дужки ми беремо з другого німецького видання.
\emph{Ред.}
