таку інтенсивність праці, що руйнує здоров’я робітників, отже,
руйнує саму робочу силу. «В більшості бавовняних фабрик, фабрик
суканої вовни та шовкових фабрик той запал, що виснажує
робітника і є потрібний для праці коло машин, рух яких останніми
роками так надзвичайно прискорено, є, здається, однією
з причин тієї надмірної смертности від недуг на легені, яку виявив
д-р Ґрінхов у своєму найновішому вартому уваги звіті». 176
Немає найменшого сумніву, що тенденція капіталу, скоро тільки
закон раз назавжди покладе край здовжуванню робочого дня,
а саме тенденція відшкодовувати себе систематичним підвищенням
ступеня інтенсивности праці та перетворенням усякого поліпшення
машин на засіб до більшого висисання робочої сили, мусить
незабаром знову привести до того поворотного пункту, де знов
стає неминучим скорочення робочого дня. 177 З другого боку,
швидкий поступ англійської промисловости за час від 1848 р. до
наших часів, тобто за періоду десятигодинного робочого дня, ще
дужче перевищує розвиток її за час від 1838 до 1847 р., тобто за
періоду дванадцятигодинного робочого дня, ніж цей останній
перевищує розвиток промисловости протягом півстоліття від часу
заведення фабричної системи, тобто за періоду необмеженого
робочого дня.178

176 «Reports of Insp. of Fact, for 31 st October 1861», p. 25, 26.

177    Тепер (1867 p.) у Ланкашірі почалася серед фабричних робітників
аґітація за восьмигодинний робочий день.

178    Декілька нижченаведених чисел показують поступ власне «фабрик»
в Об’єднаному Королівстві, починаючи від 1848 р.:

                                                                                        Розмір
експорту
                                                             1848 р.               1851 р.          
        1860 р.                 1865 р.
Бавовняні фабрики

Бавовняна пряжа....135.831.162 фун.   143.966.106 фун.  197.343.655 фун. 103.751.455 фун.
Нитки до шиття......                      —                 4.392.176 фун.    6.287.554 фун.   
4.648.611 фун.
Бавовняні тканини....1.091.373.930 ярд.  1.543.161.789 ярд.    2.776.218.427 ярд.    2.015.237.851
ярд.

Льнопрядні та коноплепрядні фабрики

Пряжа...................  11.722.182 фун.    18.841.326 фун.    31.210.612 фун.    36.777.334 фун.
Тканини...............  88.901.519 ярд.    129.106.753 ярд.    143.996.773 ярд.    247.012.329 ярд.

Шовкові фабрики

Пряжа й нитки................  194.815 фун.    462.513 фун.      897.402 фун.         812.589 фун.
Тканини.........................              —             1.181.455 ярд.    1.307.293 ярд.   
2.869.837 ярд.

Вовняні фабрики

Пряжа............................               —       14.670.880 фун.    27.533.968 фун.   
31.669.267 фун.
Тканини.........................              —     151.231.153 ярд.    190.371.537 ярд.   
278.837.418 ярд.
