\parcont{}  %% абзац починається на попередній сторінці
\index{i}{0634}  %% посилання на сторінку оригінального видання
як усяке інше рухоме майно або худобу. Коли раби задумають
що проти панства, то їх так само слід покарати на смерть. Мирові
судді повинні на заяву панів розшукувати рабів-утікачів.
Коли виявиться, що волоцюга три дні тинявся без праці, то його
слід відіслати до місця його народження, випекти на його грудях
розпеченим залізом тавро «V» і, закувавши в кайдани, вживати
його там на дорожні та всякі інші подібні роботи. Коли волоцюга
неправильно показує своє місце народження, то він на кару
за це мусить стати довічним рабом цього місця, його мешканців
або корпорації, і на нього слід накласти тавро «S». Кожний
має право відібрати у волоцюг їхніх дітей і тримати їх при собі
як учнів, хлопців до 24 років, дівчат до 20 років. Коли вони втечуть,
то до вищезазначеного віку вони мусять бути рабами їхніх
хазяїнів, які мають право на своє бажання заковувати їх у
кайдани, бити батогами й т. ін. Кожний хазяїн може накидати
залізний ланцюжок на шию, ноги або руки свого раба, щоб краще
його пізнавати й бути певнішим, що він не втече.\footnote{
Автор «Essay on Trade etc.», 1770, зауважує: «За королювання
Едварда VI англійці, здається, цілком серйозно почали підохочувати
мануфактури й давати бідним заняття. Це видно з одного вартого уваги
статуту, де сказано, що на всіх волоцюг треба накладати тавра, і т. ін.
(Там же, стор. 8).
} Остання
частина цього статуту передбачає, що деякі бідні повинні працювати
на ту округу або тих осіб, що дають їм їсти й пити та знаходять
для них працю. Такий рід парафіяльних рабів зберігся
в Англії аж до XIX віку під назвою roundsmen (Umgeher).

Єлисавета, 1572: жебраків понад 14 років, що не мають дозволу
жебракувати, слід люто бити батогами та випікати їм тавра
на лівому вусі, коли ніхто не згоджується взяти їх на службу
на два роки; коли це повториться, то жебраків понад 18 років
слід покарати на смерть, якщо ніхто не згоджується взяти їх
на службу на два роки; коли їх спіймають на цьому втретє, то
їх слід нещадно покарати на смерть як державних зрадників.
Аналогічні статути: 18 Єлисавети с. 13 і 1597.\footnoteA{
Томас Мор каже у своїй «Утопії»: «Так то й трапляється, що
жаденний і ненаситний обжера, оця справжня чума своєї батьківщини,
захоплює тисячі акрів землі, обгороджує їх тином або живоплотом, або
силою і кривдами може так зацькувати їхніх власників, що вони примушені
продати все своє майно. Тим або іншим способом, не києм, то палицею,
їх примушують виселятися — цих бідних, простих, бідолашних людей!
Мужчини, женщини, чоловіки, жінки, сироти, вдовиці, охоплені
горем матері з немовлятками, всі члени родини, бідні на засоби існування,
але багаті числом, бо рільництво потребувало багато робочих рук.
Вони бредуть геть, кажу я, з своїх рідних місць, до яких вони звикли,
і ніде не находять собі місця відпочинку; продаж усього їхнього домашнього
скарбу, хоч і невеликої вартости, міг би за інших обставин принести
деяку виручку, але, раптом опинившися на вулиці, вони мусять продати
його за безцінь. І коли вони проблукають таким чином аж поки проїдять
останню шажину, то що ж їм лишається, як не красти? Але тоді
їх вішають, додержуючи всіх форм закону. Або жебракувати? Але тоді
їх кидають у в’язницю, як волоцюг, за те, що вони тиняються й не працюють,
хоч їм ніхто не хоче дати роботу, як би жагуче вони її добивалися».
З-поміж цих бідних утікачів, що їх, як каже Томас Мор, просто
таки примушують красти, «за королювання Генріха VIII покарано на
смерть \num{72.000} великих і малих злодіїв». (Hollinshed: «Description of
England», vol. I, p. 186). За часів Єлисавети волоцюг вішали цілими лавами;
звичайно не проходило року, щоб там або деінде не повісили 300
або 400 осіб». (Strype: «Annals of the Reformation and Establishment
of Religion, and other Various Occurrences in the Church of England
during Queen Elisabeth’s Happy Reign», 2 nd ed. 1725, vol. II). За тим самим
Стріпом в Сомерсетшірі за один лише рік покарано на смерть 40 осіб,
потавровано 35, покарано батогами 37, а 183 «очайдушних злочинців»
випущено на волю. Однак, каже, він, «у це велике число обвинувачених,
у наслідок недбальства мирових суддів і безглуздого співчуття
з боку народу, не входить і п’ятина гідних кари злочинців». Він додає:
«Інші графства Англії були не у кращому становищі, ніж Сомерсетшір,
а багато навіть у гіршому».
}

\index{i}{0635}  %% посилання на сторінку оригінального видання
Яків І: особу, що тиняється й жебрачить вважається за волоцюгу.
Мирові судді в Petty Sessions\footnote*{
— малих сесіях. \emph{Ред.}
} уповноважені віддавати
таких осіб на прилюдну кару батогами й замикати їх у в’язниці
на шість місяців, спіймавши їх перший раз, і на два роки, спіймавши
їх удруге. Підчас ув’язнення їх слід карати батогами так часто
й так багато, як це вважають за відповідне мирові судді\dots{} Непоправних
і небезпечних волоцюг слід таврувати, випікаючи
їм на лівому плечі літеру «R», і вживати до примусових праць,
а коли їх іще раз спіймають на жебрацтві — нещадно карати
на смерть. Ці постанови мали силу аж до початку XVIII віку,
скасовано їх тільки актом 12 Анни с. 23.

Подібні закони були й у Франції, де в середині XVII століття
утворилось у Парижі так зване «королівство волоцюг»
(royaume des truands). Ще на початку королювання ЛюдовікаХІV
(ордонанс від 13 липня 1777 р.) кожну здорову людину між 16 і
60 роками засилали на ґалери, коли вона не мала засобів існування
й певної професії. Подібні постанови є в статуті Карла V
для Нідерляндів від жовтня 1537 р., перший едикт штатів і міст
Голляндії від 19 березня 1614 р., плякат Сполучених Провінцій
від 25 червня 1649 р. і т. д.

Таким чином, сільську людність, силоміць позбавлену землі,
вигнану й перетворену на волоцюг, за допомогою жахливо терористичних
законів, батогами, тавруванням і катуванням привчили
до дисципліни, доконечної для системи найманої праці.

Мало того, що умови праці виступають на одному полюсі як
капітал, а на другому полюсі як люди, що не мають на продаж
нічого, крім своєї власної робочої сили. Недосить також і того,
що їх примушують добровільно продавати себе. З розвитком капіталістичної
продукції розвивається робітнича кляса, що в наслідок
свого виховання, традиції, звичок визнає вимоги цього способу
продукції за само собою зрозумілі закони природи. Організація
розвиненого капіталістичного процесу продукції ламає
всякий опір; постійне створювання відносного перелюднення
тримає закон попиту й подання, а тим то й заробітну плату
в межах, відповідних до потреби самозростання капіталу; німий
гніт економічних відносин закріпляє панування капіталіста над
\parbreak{}  %% абзац продовжується на наступній сторінці
