
\index{i}{0159}  %% посилання на сторінку оригінального видання
Капітал С розпадається на дві частини: грошову суму с, витрачену на засоби продукції, і другу
грошову суму v, витрачену на робочу силу; с репрезентує частину вартости, перетворену на сталий
капітал, v — частину вартости, перетворену на змінний капітал. Отже, первісно C — c \dplus{} v, приміром,
авансований капітал у 500\pound{ фунтів стерлінґів} \deq{} 410\pound{ фунтів стерлінґів} \dplus{} 90\pound{ фунтів стерлінґів}.
Наприкінці процесу праці з’являється товар, що його вартість дорівнює с \dplus{} v \dplus{} m, де m є додаткова
вартість, наприклад, 410\pound{ фунтів стерлінґів} \dplus{} 90\pound{ фунтів стерлінґів} \dplus{} 90\pound{ фунтів стерлінґів}. Первісний
капітал С перетворився на С',
з 500\pound{ фунтів стерлінґів} на 590\pound{ фунтів стерлінґів}. Ріжниця між обома дорівнює m, додатковій вартості
в 90. А що вартість елементів продукції дорівнює вартості авансованого капіталу, то сказати, що
надлишок вартости продукту понад вартість елементів його продукції дорівнює приростові авансованого
капіталу, або дорівнює випродукованій додатковій вартості, — є в дійсності тавтологія.

Однак ця тавтологія потребує детальнішого визначення. З вартістю продукту порівнюється вартість
елементів продукції, зужиткованих при його творенні. Але ми вже бачили, що частина застосованого
сталого капіталу, яка складається із засобів праці, віддає продуктові лише одну частину своєї
вартости, тимчасом як друга частина й далі існує в старій своїй формі. А що остання частина не
відіграє жодної ролі у творенні вартости, то тут від неї треба абстрагуватись. Якщо й занести її в
обчислення, то нічого не зміниться. Припустімо, що с \deq{} 410\pound{ фунтів стерлінґів} і
складається з сировинного матеріялу на 312\pound{ фунтів стерлінґів}, допоміжних матеріялів на 44\pound{ фунти
стерлінґів} і зужиткованих
у процесі машин на 54\pound{ фунти стерлінґів}, а вартість дійсно застосованих машин нехай становить \num{1.054}\pound{ фунти стерлінґів}. За авансовану на створення вартости продукту ми рахуємо лише вартість у 54\pound{ фунти
стерлінґів}, яку машини втрачають через своє функціонування і в наслідок цього віддають продуктові.
Коли б ми врахували
й ті \num{1.000}\pound{ фунтів стерлінґів}, що й далі існують у своїй старій формі, як парова машина й~\abbr{т. ін.}, то
ми мусили б їх урахувати
на обох боках, на боці авансованої вартости та на боці вартости продукту\footnoteA{
«Коли ми беремо в обрахунок вартість застосованого основного капіталу, як частину авансованого
капіталу, то ми мусимо наприкінці року взяти в обрахунок лишок вартости цього капіталу, як частину
річного доходу» («If we reckon the value of the fixed capital employed as a part of the advances, we
must reckon the remaining value of such capital at the end of the year as a part of the annual
returns»). (Malthus: «Principles
of Political Economy», 2 nd ed. London 1836, p. 269).
}, і таким чином ми
одержали б на одному боці
\num{1.500}\pound{ фунтів стерлінґів} і \num{1.590}\pound{ фунтів стерлінґів} на другому. Ріжниця, або додаткова вартість, була
б, як і раніше, 90\pound{ фунтів стерлінґів}. Тому там, де із загального зв’язку викладу не виявляється
\index{i}{0160}  %% посилання на сторінку оригінального видання
протилежне, під сталим капіталом, авансованим на продукцію
вартости, ми завжди розуміємо лише вартість зужиткованих
у продукції засобів продукції.

Припустивши таку передумову, повернімось до формули

С \deq{} с \dplus{} v, яка перетворюється на С' \deq{} с \dplus{} v \dplus{} m, а саме через це
перетворення й С перетворюється на С'. Відомо, що вартість сталого
капіталу лише знов з’являється в продукті. Отже, вартість,
дійсно новоспродукована в процесі, є відмінна від цілої вартости
продукту, добутої в процесі, тим-то вона становить не
c \dplus{} v \dplus{} m, або 410\pound{ фунтів стерлінґів} \dplus{} 90\pound{ фунтів стерлінґiв} \dplus{} 90\pound{ фунтів
стерлінґів}, як то здається на перший погляд, а v \dplus{} m, або
90\pound{ фунтів стерлінґів} \dplus{} 90\pound{ фунтів стерлінґів}, тобто не 590\pound{ фунтів
стерлінґів}, а 180\pound{ фунтів стерлінґів}. Коли б с, сталий капітал,
дорівнював 0, іншими словами, коли б існували такі галузі промисловости,
де капіталіст не мав би застосовувати жодних спродукованих
засобів продукції — ані сировинного матеріялу, ані
допоміжних матеріялів, ні знаряддя праці, а мав би застосовувати
тільки такі матеріяли, які існують з природи, і робочу
силу, тоді на продукт не переносилося б жодної частини сталої
вартости. Тоді цей елемент вартости продукту, в нашому прикладі
410\pound{ фунтів стерлінґів}, відпав би, але новоспродукована вартість
у 180 фунтів, що містить у собі 90\pound{ фунтів стерлінґів} додаткової
вартости, лишалася б цілком такого самого розміру, як коли б
с являло собою найбільшу суму вартости. Ми мали б С \deq{} 0 \dplus{} v \deq{} v,
і С', вирослий у своїй вартості капітал, дорівнював би v \dplus{} m, а
С' мінус С, як і раніш, дорівнювало б m. Навпаки, коли б m дорівнювало
0, іншими словами, коли б робоча сила, вартість якої
авансується у змінному капіталі, продукувала лише еквівалент,
тоді б С \deq{} с \dplus{} v і С' (вартість продукту) \deq{} с \dplus{} v \dplus{} 0, а тому С \deq{} С'.
Авансований капітал не зріс би своєю вартістю.

В дійсності ми вже знаємо, що додаткова вартість є лише наслідок
тієї зміни вартости, яка відбувається з v, з частиною капіталу,
перетвореною на робочу силу, що, отже, v \dplus{} m \deq{} Δv (v плюс
приріст v). Але дійсна зміна вартости й відношення, в якому
змінюється вартість, затемнюються тим, що в наслідок зростання
складової змінної частини капіталу зростає також і ввесь авансований
капітал. Він був 500, а стає 590. Отже, аналіза процесу в його
чистій формі вимагає, щоб ми цілком абстрагувалися від тієї
частини вартости продукту, в якій лише знов з’являється стала
капітальна вартість, тобто припустили, що сталий капітал с
дорівнює нулеві, і таким чином застосували той закон математики,
що ним вона оперує змінними й сталими величинами, коли
стала величина тільки через додавання або віднімання зв’язана
зі змінною.
