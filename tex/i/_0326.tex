\parcont{}  %% абзац починається на попередній сторінці
\index{i}{0326}  %% посилання на сторінку оригінального видання
фабрикант домагається таких хлопчаків, які мають вигляд, наче
їм уже минуло 13 років. Раптове — іноді — зменшення числа
дітей, молодших за 13 років, експлуатованих фабрикантами, яке
так дивує в англійській статистиці за останні 20 років, було, за
словами самих фабричних інспекторів, здебільша справою тих
certifying surgeons, що перекручували вік дітей відповідно до
експлуататорської жадоби капіталістів та баришницьких потреб
батьків. В Bethnal Green, цій ославленій окрузі Лондону, щопонеділка
й щовівтірка влаштовують одкритий базар, на якому
діти обох статей, починаючи від 9-літнього віку, сами себе наймають
на лондонські шовкові мануфактури. «Звичайні умови —
1 шилінґ 8 пенсів на тиждень (це належить батькам) і 2 пенси —
для мене самого, та чай». Контракти мають силу тільки на тиждень.
Сцени й мова підчас цього базару справді обурливі.\footnote{
«Children’s Employment Commission. 5 th Report», London 1866,
p. 81. n.31. [До 4 видання. Шовкову промисловість у Bethnal Green тепер
майже знищено. — \emph{Ф. Е.}].
} Ще й досі
трапляється в Англії, що жінки беруть «хлопчаків із робітного
дому й наймають їх якомубудь покупцеві за 2 шилінґи
6 пенсів на тиждень».\footnote{
«Children’s Employment Commission. 3 rd Report», London 1864,
p. 53, n. 15.
} Всупереч законодавству ще й досі щонайменше
2.000 хлопчаків продається у Великобританії їхніми
власними батьками як живі сажотрусні машини (дарма що існують
машини для заміни їх).\footnote{
Там же, 5 th Report, p. XXIII, n. 137.
} Зумовлена машинами революція у
правних відносинах між покупцем та продавцем робочої сили,
позбавивши всю цю угоду навіть подоби контракту між вільними
особами, дала пізніш англійському парляментові юридичну
підставу виправдуватися за втручання держави у фабричний
режим. Щоразу, коли фабричний закон обмежує дитячу працю
в нереґляментованих досі галузях промисловости 6 годинами,
знову й знову лунає голосіння фабрикантів: частина батьків,
мовляв, забирають своїх дітей із реґляментованих тепер галузей
промисловости, щоб запродати їх на такі, де ще панує «воля
праці», тобто, де діти, молодші за 13 років, примушені працювати
як дорослі, отже, на такі галузі, де за них можна дорожче
взяти. А що капітал із своєї природи левелер, тобто вимагає,
як свого природженого права, рівности в умовах експлуатації
праці в усіх сферах продукції, то й законодавче обмеження дитячої
праці в одній галузі промисловости стає причиною обмеження
її в іншій галузі.

Ми вже раніш відзначали фізичний занепад дітей, підлітків
і жінок робітників, що їх машини кидають на експлуатацію капіталу
спочатку безпосередньо по фабриках, що виростають на
основі машин, а потім посередньо, в усіх інших галузях промисловости.
Тому ми тут спинимося лише на одному пункті, на жахливій
смертності робітничих дітей у перші роки їхнього життя.
\parbreak{}  %% абзац продовжується на наступній сторінці
