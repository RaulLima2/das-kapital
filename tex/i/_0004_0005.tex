\parcont{}  %% абзац починається на попередній сторінці
\index{i}{0004}  %% посилання на сторінку оригінального видання
вартості товари мають насамперед різну якість, як мінові вартості
вони можуть різнитися лише щодо кількости, отже, не
містять у собі жодного атома споживної вартости.

Коли залишити осторонь споживну вартість товарових тіл,
то в них зостається ще тільки одна властивість, а саме та, що
вони є продукти праці. Однак і самий продукт праці вже зазнав
у нас перетворення. Абстрагуючись від його споживної вартости,
ми тим самим абстрагуємось і від його тілесних складових частин
і форм, які роблять його споживною вартістю. Це вже не стіл або
дім, або пряжа, абож інша корисна річ. Всі його почуттєво сприймані
властивості зникли. Це вже й не продукт теслярської, будівельної
або ткальської праці, або взагалі якоїбудь іншої певної
продуктивної праці. Разом з корисним характером продуктів
праці зникає й корисний характер втіленої в них праці, отже,
зникають і різні конкретні форми цих праць; вони вже не відрізняються
одна від однієї, а всі вони зведені на однакову людську
працю, абстрактну людську працю.

Розгляньмо тепер Residuum\footnote*{
-- остачу. \emph{Ред.}
}, що лишається від продуктів
праці після цього зведення. Від них не залишилося нічого іншого,
крім однакової для них усіх примарної предметности (Gegenständlichkeit\footnote*{
Німецьке «Gegenständlichkeit» ми перекладаємо тут, так само як
і всюди далі словом «предметність» у розумінні чогось, що об’єктивно
існує. У французькому виданні це слово перекладено словом «réalité»,
що означає — дійсне існування, дійсний предмет або реальність. \emph{Ред.}
}),
простого згустка безріжницевої (unterschiedsloser)
людської праці, тобто затрати людської робочої сили незалежно
від форми її затрати. Ці речі виявляють тепер тільки те,
що на їхню продукцію затрачено людську робочу силу, що в них
нагромаджено людську працю. Як кристалі цієї спільної їм суспільної
субстанції, вони є вартості -- товарові вартості.

У самому міновому відношенні товарів їхня мінова вартість
з’явилася перед нами як щось цілком незалежне від їхньої споживної
вартости. Якщо ж ми дійсно абстрагуємося від споживної
вартости продуктів праці, то матимемо їх вартість, як її щойно
визначено. Отже, те спільне, що виявляється в міновому відношенні
або міновій вартості товару, є його вартість. Дальший дослід
знов приведе нас до мінової вартости як доконечного способу
виразу, або доконечної форми виявлення товарової вартости; цю
останню, однак, спочатку треба розглянути незалежно від цієї форми.

Отже, споживна вартість, або добро, має вартість лише тому,
що в ній упредметнено або зматеріялізовано абстрактну людську
працю. Як же виміряти величину вартости споживної вартости?
Кількістю «вартостетворчої субстанції», що міститься в ній,
кількістю праці. Кількість самої праці вимірюється часом тривання
праці, а робочий час має знов таки свій маштаб у певних
частинах часу, як от година, день і~\abbr{т. ін.}

Могло б здаватися, що коли вартість товару визначається
кількістю праці, витраченої підчас його продукції, то що ледачіша
\index{i}{0005}  %% посилання на сторінку оригінального видання
або невміліша людина, то вищу вартість матиме її товар,
бо то більше часу потребує вона для виготовлення товару. Але
праця, що становить субстанцію вартостей, є однакова людська
праця, затрата тієї самої людської робочої сили. Сукупна робоча
сила суспільства, що виражається у вартостях товарового світу,
має тут значення однієї і тієї самої людської робочої сили, дарма
що вона складається з безлічі індивідуальних робочих сил. Кожна
з цих індивідуальних робочих сил є така сама людська робоча
сила, як і всяка інша, оскільки вона має характер суспільної
пересічної робочої сили і функціонує як така суспільна пересічна
робоча сила, отже, оскільки вона на продукцію якогось товару
потребує лише пересічно доконечного або суспільно-доконечного
робочого часу. Суспільно-доконечний робочий час є робочий
час, потрібний на те, щоб виготовити якусь споживну вартість
при даних суспільно-нормальних умовах продукції і суспільному
пересічному ступені вмілости та інтенсивности праці. Приміром,
в Англії після заведення парового ткацького варстату,
щоб перетворити певну кількість пряжі в тканину, досить було,
може, вдвоє менше праці, ніж раніш. Англійський ручний ткач
на ділі потребував на це перетворення того самого робочого часу,
що й раніш, але продукт його індивідуальної робочої години
репрезентував тоді лише половину суспільної робочої години, і
через це його вартість зменшилась удвоє проти колишньої.

Отже, величина вартости якоїсь споживної вартости визначається
лише кількістю суспільно-доконечної праці або кількістю
робочого часу, суспільно-доконечного, щоб виготовити
її\footnote{
Примітка до другого видання: «Вартість засобів споживання, коли
їх обмінюється один на один, визначається кількістю праці, доконечно-потрібної
і звичайно витрачуваної на продукцію їх» («The value of them
(the necessaries of life) when they are exchanged the one for another, is
regulated by the quantity of labour necessarily required, and commonly
taken in producing them»). («Some Thoughts on the Interest of Money in
general, and particulary in the Public Funds etc.», London, p. 36). Цей
вартий уваги анонімний твір минулого століття не датований. Однак
з його змісту видно, що він вийшов у світ за часів Ґеорґа II, приблизно
року 1739 або 1740.
}. Кожен поодинокий товар вважається тут взагалі за пересічний
екземпляр свого роду\footnote{
«Всі продукти того самого роду становлять, власне кажучи, одну
масу, що її ціну визначається загально і не зважаючи на поодинокі обставини»
(«Toutes les productions d’un même genre ne forment proprement
qu’une masse, dont le prix se détermine en général et sans égard aux circonstances
particulières»). (\emph{Le Trosne}: «De l’Intérêt Social», p. 893).
}. Тому всі товари, що в них міститься
однакова кількість праці, або що їх можна виготовити протягом
однакового робочого часу, мають однакову величину вартости.
Вартість одного товару відноситься до вартости всякого іншого
товару, як робочий час, доконечний на те, щоб випродукувати
один товар, відноситься до робочого часу, доконечного на те,
щоб випродукувати всякий інший товар. «Як вартості, всі товари
є тільки певна маса застиглого робочого часу»\footnote{
\emph{K.~Marx}: «Zur Kritik der Politischen Ökonomie», S. 6. (\emph{K.~Маркс}.
«До критики політичної економії». ДВУ, 1926~\abbr{р.}, стор. 48).
}.
