\index{i}{0657}  %% посилання на сторінку оригінального видання
Приватна власність, як протилежність до суспільної, колективної
власности, існує лише там, де засоби праці й зовнішні
умови праці належать приватним особам. Але залежно від того,
чи є ці приватні особи робітники або неробітники, змінюється
й характер самої приватної власности. Безмежна різноманітність
відтінків, які вона являє на перший погляд, відбивають лише
проміжні стани, що лежать між обома цими крайностями.

Приватна власність робітника на його засоби продукції є
основа дрібного виробництва, а дрібне виробництво є доконечна
умова розвитку суспільної продукції й вільної індивідуальности
самого робітника. Правда, цей спосіб продукції існує також у
рамках рабства, кріпацтва й інших відносин залежности. Але
він процвітає, виявляє всю свою енергію, здобуває клясичну
адекватну форму тільки там, де робітник є вільний приватний
власник своїх, ним самим уживаних умов праці, селянин — ріллі,
яку він обробляє, ремісник — інструменту, що на ньому він грає,
як віртоуз.

Цей спосіб продукції має за передумову роздрібнення землі
й усіх інших засобів продукції. Він виключає так концентрацію
засобів продукції, як і кооперацію, поділ праці всередині того
самого продукційного процесу, суспільне панування над природою
й реґулювання її, а також вільний розвиток суспільних продуктивних
сил. Він можливий лише за вузьких примітивних
меж продукції й суспільства. Захотіти його увіковічнити, це
значило б — як справедливо каже Пекер — «декретувати
загальну помірність». Але на якомусь певному ступені розвитку
він сам породжує матеріяльні засоби свого власного знищення.
З цієї хвилини в надрах суспільства починають ворушитися
сили та пристрасті, що почувають себе скутими цим способом
продукції. Він мусить бути знищений, і його знищується. Знищення
його, перетворення індивідуальних і роздрібнених засобів
продукції на суспільно-сконцентровані, отже, перетворення
карликової власности багатьох на колосальну власність
небагатьох, отже, експропріяція в широких народніх мас землі,
засобів існування і знарядь праці — оця жахлива й тяжка
експропріяція народньої маси становить передісторію капіталу.
Вона охоплює цілу низку насильних метод, що з них ми коротко
розглянули лише ті, які становили епоху як методи первісної
акумуляції капіталу. Експропріяцію безпосередніх продуцентів
проводиться з найнещаднішим вандалізмом і під тиском
якнайпідліших, якнайбрудніших, найдріб’язковіших і найшаленіших
пристрастей. Приватну власність, здобуту працею власника,
основану, так би мовити, на зрощенні поодинокого незалежного
робітника з його умовами праці, витісняє капіталістична
приватна власність, основана на експлуатації чужої, але формально
вільної праці.\footnote{
«Ми перебуваємо в цілком нових суспільних умовах... ми намагаємось
відокремити кожний рід власности від кожного роду праці»
(«Nous sommes dans une condition tout-à-fait nouvelle de la société...
}
