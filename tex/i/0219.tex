нехтували всіма виданими за останні 22 роки законами про дитячу
працю, їм ще й підсолоджено пілюлю. Парламент ухвалив, що
від 1 березня 1834 р. жодна дитина молодша від 11 років, від
1 березня 1835 р. жодна дитина молодша від 12 років і від 1 березня
1836 р. жодна дитина молодша від 13 років не повинні
працювати на фабриці більш, ніж 8 годин! Цей такий надто обережний
проти «капіталу» «лібералізм» був то більше вартий
подяки, що Д-р Фар, сер. А. Карлейл, сер Б. Бреді, сер. К. Белл,
м-р. Ґетрі та інші, коротко — найвидатніші physicians і surgeons*
Лондону в своїх свідченнях перед Палатою громад заявили, що
periculum in mora!** Д-р Фар висловився з цього приводу навіть
ще гостріше: «Законодавство однаково доконечне, щоб запобігти
передчасній смерті, хоч у яких формах ту смерть спричиняється,
а на цей спосіб (фабричний спосіб) безперечно треба дивитися
як на одну з найжорстокіших метод спричинення смерти».135
Той самий «реформований» парлямент, який із делікатного почуття
до панів фабрикантів ще на цілі роки загнав дітей, молодших
від 13 років, у пекло фабричної праці протягом 72 годин на
тиждень, навпаки, емансипаційним актом, що теж давав волю
краплями, з самого ж початку заборонив плянтаторам примушувати
негрів-рабів працювати більш ніж 45 годин на тиждень!

Але капітал, аж ніяк нечутливий до всіх цих поступок, розпочав
тепер на цілі роки шумну аґітацію. Вона оберталась, головне,
навколо віку категорій, які під назвою дітей мали працювати
не більш ніж 8 годин і які до певної міри підлягали примусовому
навчанню. За капіталістичною антропологією дитячий
вік кінчався десятим роком або, щонайвище, одинадцятим. Що
ближче надходив термін повного здійснення фабричного закону,
фатальний 1836 р., то скаженіше лютувала фабрикантська зграя.
І їй дійсно пощастило так дуже залякати уряд, що останній
1835 р. запроєктував знизити межу дитячого віку з 13 до 12 років.
Тимчасом грізно наростав pressure from without.* ** І не стало відваги
в Палати громад. Вона відмовилася від того, щоб кидати
тринадцятилітніх дітей під колісницю Джаґернавта**** капіталу
більш, ніж на 8 годин удень, і акт 1833 р. набув повної сили.
Він лишався без зміни до червня 1844 р.

Протягом десятиріччя, коли він реґулював фабричну працю
спочатку частково, а потім цілком, офіціяльні звіти фабричних
інспекторів ущерть повні нарікань на неможливість провести його
в життя. А що закон 1833 р. залишив на волю панів-капіталістів
призначати в межах п’ятнадцятигодинного періоду, від пів на шосту
ранку до пів на дев’яту вечора, ту годину, коли кожний «під-

135 «Legislation is equally necessary for the prevention of death, in
any form in which it can be prematurely inflicted, and certainly this must
be viewed as most cruel mode of inflicting it».

* — лікарі й хірурги. Ред.

** — у загаянні небезпека. Ред.

*** — натиск зовні. Ред.

**** Один із видів індійського бога Вішну; під колісницю, на якій возять
його ідола, фанатичні індуси кидаються, щоб дати себе роздавити. Ред.
