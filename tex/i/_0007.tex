\parcont{}  %% абзац починається на попередній сторінці
\index{i}{0007}  %% посилання на сторінку оригінального видання
Середньовічний селянин продукував чиншевий хліб для февдальних
панів, а десятину для попів. Але ні чиншевий хліб, ні десятина
не ставали товаром через те, що їх продуковано для інших.
Щоб стати товаром, продукт мусить через обмін перейти до когось
іншого, що для нього він служить за споживну вартість].\footnoteA{
Прим, до 4 вид. Я поставив уміщені в дужки слова, бо без них
дуже часто виникало непорозуміння, ніби в Маркса всякий продукт,
споживаний будь-ким іншим, а не його продуцентом, є товар. — \emph{Ф. Е.}
} Нарешті,
жодна річ не може бути вартістю, не будучи предметом
споживання. Коли вона некорисна, то й працю, що в ній міститься,
не вважається за працю, і тому вона не творить вартости.

\section*{2. Двоїстий характер праці, репрезентованої товарами
}

Первісно товар з’явився перед нами як щось двоїсте: споживна
вартість і мінова вартість. Пізніше виявилось, що й праця, оскільки
вона виражена у вартості, не має вже більше тих ознак,
які їй належать як творцеві споживних вартостей. Я перший
критично довів цю двоїсту природу праці, що міститься в товарах.\footnote{
\emph{К. Marx:} «Zur Kritik der Politischen Oekonomie», Berlin 1859,
S. 12, 13 und passim. (\emph{K. Маркс.} «До критики політичної економії», ДВУ
1926 р., cтop. 46--47 і далі).
} А що цей пункт є основний пункт, що на ньому ґрунтується
розуміння політичної економії, то його треба тут детальніше
висвітлити.

Візьмімо два товари, приміром, один сурдут і 10 метрів полотна.
Хай перший має подвійну вартість останніх, так що, коли
10 метрів полотна = \emph{w}, то сурдут = \emph{2w}.

Сурдут є споживна вартість, що задовольняє якусь окрему
потребу. Щоб його виготовити, потрібен якийсь певний рід продуктивної
діяльности. Вона визначається своєю метою, характером
операції, предметом, засобами й результатом. Працю, що її
корисність таким способом виражається в споживній вартості її
продукту або в тім, що її продукт є споживна вартість, ми коротко
називаємо корисною працею. З цього погляду її треба завжди
розглядати стосовно до її корисного ефекту.

Як сурдут і полотно є якісно різні споживні вартості, так само
кравецька праця, що виготовляє сурдут, відрізняється якісно
від праці ткача, що продукує полотно. Коли б ці речі не були
якісно різні споживні вартості, а тому й продукти якісно різних
корисних праць, то тоді взагалі вони не могли б протистояти одна
одній як товари. Сурдут не обмінюється на сурдут, певна споживна
вартість не обмінюється на таку саму споживну вартість.

У сукупності різнорідних споживних вартостей або товарових
тіл виявляється сукупність так само різноманітних щодо ґатунку,
роду, родини, підроду й відміни різних корисних праць, — виявляється
суспільний поділ праці. Він є умова існування товарової
продукції, хоч товарова продукція, навпаки, не є умова існування
суспільного поділу праці. У староіндійській громаді праця є
\parbreak{}  %% абзац продовжується на наступній сторінці
