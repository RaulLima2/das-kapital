\parcont{}  %% абзац починається на попередній сторінці
\index{i}{0290}  %% посилання на сторінку оригінального видання
при поділі праці всередині майстерні, при поділі праці всередині
суспільства діє лише а posteriori як унутрішня, німа, сприймана
лише в барометричних змінах ринкових цін природна доконечність,
що переборює безладну сваволю товаропродуцентів. Мануфактурний
поділ праці припускає безумовний авторитет капіталіста
супроти людей, які є лише прості члени належного йому цілого
механізму; суспільний поділ праці протиставить одного одному
незалежних товаропродуцентів, які не визнають жодного іншого
авторитету, крім конкуренції, крім того примусу, що його справляє
на них тиск їхніх взаємних інтересів, так само як у царстві
тварин bellum omnium contra omnes\footnote*{
— боротьба всіх проти всіх. \emph{Ред.}
} більш або менш підтримує
умови існування всіх видів. Тому та сама буржуазна свідомість,
яка прославляє мануфактурний поділ праці, довічне приковання
робітника до якоїсь детальної операції та безумовну підпорядкованість
частинного робітника капіталові, як організацію
праці, що підвищує її продуктивну силу, — ця сама буржуазна
свідомість так само голосно ганьбить усякий свідомий суспільний
контроль та реґулювання суспільного процесу продукції як замах
на незаймані права власности, на волю та «геніальність» індивідуального
капіталіста, геніяльність що сама себе визначає.
Це дуже характеристично, що ентузіясти-апологети фабричної
системи не вміють нічого сильнішого сказати проти всякої загальної
організації суспільної праці, як тільки те, що така організація
перетворила б ціле суспільство на одну фабрику.

Якщо анархія суспільного поділу праці й деспотія мануфактурного
поділу праці зумовлюють одна одну в суспільстві капіталістичного
способу продукції, то, навпаки, попередні форми
суспільства, в яких відокремлення промислів розвивалося стихійно,
потім кристалізувалось і, нарешті, закріпилося законом,
подають, з одного боку, образ пляномірної та авторитарної
організації суспільної праці, тимчасом як, з другого боку, вони
цілком виключають поділ праці всередині майстерні або розвивають
його в карликовому маштабі, або лише спорадично й випадково.\footnote{
«Можна\dots{} встановити, як загальне правило, що чим менше поділ
праці всередині суспільства підлягає авторитетові, тим дужче поділ
праці розвивається всередині майстерні й тим більше він тут підлягає
авторитетові однієї особи. Таким чином щодо поділу праці авторитет
у майстерні та авторитет у суспільстві зворотно пропорційні один
одному» («On peut\dots{} établir en règlegénérale, que moins l’autorité préside
à la division du travail dans l’intérieur de la société, plus la division du
travail se développe dans l’intérieur de l’atelier, et plus elle y est soumise
à l’autorité d’un seul. Ainsi l’autorité dans l’atelier et celle dans la société
par rapport à division du travail, sont en raison inverse l’une de l’autre»),
(\emph{K. Marx}: «Misère de la Philosophie», Paris 1847. p. 130, 131 — \emph{К. Маркс}:
«Злиденність філософії», Партвидав 1932 р., стор. 118--119).
}

Приміром, ті старовинні дрібні індійські громади, які почасти
ще й досі існують, ґрунтуються на спільному посіданні
землі, на безпосередній сполуці рільництва й ремества та на
\parbreak{}  %% абзац продовжується на наступній сторінці
