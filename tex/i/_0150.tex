\parcont{}  %% абзац починається на попередній сторінці
\index{i}{0150}  %% посилання на сторінку оригінального видання
продукт і таким чином зберігає їхню вартість у продукті. Звідси
двоїстість результату праці за той самий відтинок часу.

Через просте кількісне долучений праці долучається нову
вартість, а в наслідок якости долученої праці старі вартості засобів
продукції зберігаються в продукті. Цей двоїстий ефект тієї
самої праці як наслідок двоїстого її характеру виразно виявляється
в різних явищах.

Припустімо, що якийсь винахід дає прядунові змогу випрясти
за 6 годин стільки бавовни, скільки раніш випрядалося за 36 годин.
Праця його, як доцільна корисна продуктивна діяльність,
ушестеро збільшила свою силу. Продукт її є вшестеро більший —
36 фунтів пряжі замість 6 фунтів. Але ці 36 фунтів бавовни вбирають
у себе тепер лише стільки робочого часу, скільки раніше
6 фунтів. До них додається нової праці вшестеро менше, ніж за
старої методи, отже, додається лише \sfrac{1}{6} тієї вартости, що її долучалося
до них раніш. З другого боку, в продукті, в 36 фунтах
пряжі міститься тепер ушестеро більша вартість бавовни. Протягом
цих 6 годин прядіння зберігається й переноситься на продукт
вшестеро більшу вартість сировинного матеріялу, хоч до
того самого сировинного матеріялу додається вшестеро меншу
нову вартість. Це показує, як та властивість праці, в наслідок
якої праця протягом того самого неподільного процесу зберігає
вартості, посутньо відмінна від тієї її властивости, в наслідок
якої вона творить вартість. Що більше доконечного робочого
часу входить у ту саму кількість бавовни протягом операції
прядіння, то більша нова вартість, що її додається до бавовни,
але що більше фунтів бавовни випрядається за той самий робочий
час, то більша стара вартість, що зберігається в продукті.

Припустімо, навпаки, що продуктивність праці прядіння лишається
незмінна, отже, прядун потребує стільки ж часу, як і раніш,
на те, щоб один фунт бавовни перетворити на пряжу, але що
мінова вартість самої бавовни змінюється, а саме, що ціна фунта
бавовни зростає або падає в шість разів. В обох випадках прядун
і далі додає до тієї самої кількости бавовни той самий робочий
час, отже, ту саму вартість; і в обох випадках за однаковий час
він продукує однакову кількість пряжі. Однак вартість, яку він
з бавовни переносить на пряжу, на продукт, в одному випадку
вшестеро менша, у другому — вшестеро більша, ніж раніш. Так
само стоїть справа й тоді, коли засоби праці дорожчають або
дешевшають, але незмінно роблять ту саму послугу в процесі
праці.

Коли технічні умови процесу прядіння лишаються незмінні
й так само не відбувається жодних змін вартости з його засобами
продукції, то прядун, як і раніш, споживає за однаковий робочий
час однакові кількості сировинного матеріялу й машин, вартість
яких лишається незмінна. Вартість, яку він зберігає в продукті,
в цьому випадку прямо пропорційна до тієї нової вартости, яку
він прилучає. За два тижні він додає удвоє більше праці, ніж
за один тиждень, отже, удвоє більше вартости: одночасно він
\parbreak{}  %% абзац продовжується на наступній сторінці
