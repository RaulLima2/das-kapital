ності вампір, який висисає його, не випускає його «доти, доки для
експлуатації лишається ще хоч однісінький мускул, однісінька
жила, однісінька крапля крови».199 Щоб «боронити» себе проти
змія, який мучить їх, робітники мусять об’єднатися і, як кляса,
добитися державного закону, могутньої суспільної перешкоди,
яка самим їм заважала б шляхом добровільного договору з капіталом
продавати на смерть і рабство себе й своїх нащадків.200
Замість бучного каталогу «невідійманнх прав людини» з’являється
скромна Magna Charta* обмеженого законом робочого
дня, яка, «нарешті, з’ясовує, коли кінчається час, що його робітник
продає, і коли починається той час, що належить йому самому».201
Quantum muta tus ab illo! **

their hands, and the sweat of their brows»). («Reports etc. for 30 th April
1850», p. 45). «Вільна праця, коли її можна так назвати, навіть у вільній
країні потребує для своєї охорони міцної руки закону» («Free Labour,
if so it may be termed, even in a free country requires the strong arm of
the law to protect it»). («Reports etc. for 31 st Oct. 1864», p. 34). «Дозволити,
а це те саме, що примусити... працювати по 14 годин денно з перервами
на їжу або без них і т. ін.» («То permit, which is tantamount to
compelling... to work 14 hours a day with or without meals etc.»). («Reports
etc. for 30 th April 1863», p. 40).

199 Friedrich Engels: «Lage der arbeitenden Klasse in England»,
стор. 5. (Ф. Енгельс: «Становище робітничої кляси в Англії», Партвидав
«Пролетар», 1932 р.).

200 Десятигодинний біл у підпорядкованих йому галузях промисловости
«врятував робітників од цілковитого виродження і захистив їхнє
фізичне здоров’я». («Reports etc. for 31 st Oct. 1859», p. 47). «Капітал
(на фабриках) ніколи не може тримати машини в русі понад обмежений
період часу без того, щоб не шкодити здоров’ю й моральності робітників,
що в нього працюють; а робітники не в стані самих себе захистити».
(Там же, стор. 8).

201 «Ще більша користь у тому, що, нарешті, виясняється ріжниця
між власним часом, що належить робітникові, та часом, що належить
його хазяїнові. Робітник знає тепер, коли кінчається той час, що його
він продає, і коли починається той час, що належить йому; знаючи це,
він має змогу наперед розподілити свої хвилини часу, відповідно до своїх
потреб». («А still greater boon is, the distinction at least made clear between
the worker’s own time and his master’s. The worker knows now when
that which he sells is ended, and when his own begins, and by possessing a sure
foreknowledge of this, is enabled to pre-arrange his own minutes for his
own purpose’s»). (Там же, стор. 52). «Зробивши робітників хазяїнами
їхнього власного часу, вони (фабричні закони) дали їм моральну силу,
яка приведе їх однієї днини до захоплення політичної влади» («Ву
making them masters of their own time, they have given them a moral energy
which is directing them to the eventual possession of political power»).
(Там же, стор. 47). Із стриманою іронією і в дуже обережних висловах
натякають фабричні інспектори на те, що теперішній десятигодинний
закон до певної міри звільнив також і капіталіста від його природної
брутальности, властивої йому просто лише як персоніфікації капіталу, та
дав йому час для деякої «освіти». Раніш «хазяїн не мав часу на щось
інше, окрім набування грошей, а робітник не мав часу на щось інше,
крім праці» («the master had no time for anything but money; the servant
had no time for anything but labour»). (Там же, стор. 48).

* — основні права. Ред.

** Яка велика ріжниця проти попереднього! Ред.
