\parcont{}  %% абзац починається на попередній сторінці
\index{i}{0264}  %% посилання на сторінку оригінального видання
і скінчилася в певний час. Період, що протягом його треба виконати
процес праці, є тут так само приписаний, як от за вловів
оселедців. Поодинока людина може з одного дня викраяти лише
один робочий день, приміром, у 12 годин, але кооперація, приміром,
із 100 людей збільшує дванадцятигодинний день до робочого
дня в \num{1.200} годин. Короткість строку праці компенсується
величиною маси праці, що її кидається у вирішальний момент
на поле продукції. Вчасний результат залежить тут від одночасного
вживання багатьох комбінованих робочих днів, а обсяг
корисного ефекту — від числа робітників, яке однак завжди
лишається меншим від числа тих робітників, що змогли б протягом
того самого часу виконати ту саму працю, працюючи кожен
окремо.\footnote{
«Виконання її (рільничої праці) у критичний момент має величезну
вагу» («The doing of it at the critical juncture, is of so much the
greater consequence»). («An Inquiry into the Connection between the present
price etc.», p. 7). «У рільництві немає важливішого фактора, ніж час»
(\emph{Liebig}: «Ueber Theorie und Praxis in der Landwirtschaft», 1856, S. 23).
} Це через брак такої кооперації на заході Сполучених
штатів гине рік-у-рік сила хліба, а в тих частинах Східньої
Індії, де англійське панування знищило давню громаду, — сила
бавовни.\footnote{
«Дальше лихо, що його ледве чи хто міг сподіватись у країні,
яка вивозить праці більше, ніж усяка інша країна, за винятком хіба
Китаю та Англії, — це неможливість знайти достатню кількість робочих
рук для збирання бавовни. В наслідок цього значна частина врожаю лишається
незібрана, а другу частину його збирають із землі після того, як
бавовна вже висипалась і через це втратила належний колір і почасти
згнила; таким чином через те, що у відповідний час бракує робочих рук,
плянтатор фактично примушений відмовитися від великої частини того
врожаю, що його з такою тривогою сподівається Англія». («The next
evil is one which one would scarcely expect to find in a country which exports
more labour than any other in the world, with the exception perhaps of
China an England — the impossibility of procuring a sufficient number of
hands to clean the cotton. The consequence of this is that large quantities of
the crop are left unpicked, while another portion is gathered from the ground,
when it has fallen, and is of course discoloured and partially rotted, so
that for want of labour at the proper season the cultivator is actually forced
to submit to the loss of a large part of that crop for which England is so
anxiously looking»). (\emph{Bengal Hurkaru}: «Bi-Monthly Overland Summary
of News. 22 nd July 1861»).
}

З одного боку, кооперація дозволяє поширити просторову
сферу праці, а тому для певних процесів праці, як от, приміром,
за осушування ґрунту, будування гребель, іриґації, будування
каналів, шляхів, залізниць тощо, вона потрібна вже в наслідок
просторової зв’язаности предмету праці. З другого боку, кооперація
уможливлює просторово звужувати, порівняно з маштабом
продукції, поле продукції. Це обмеження просторової сфери праці
за одночасного поширення сфери її діяння, через що заощаджується
багато непродуктивних витрат (faux frais), постає із зосередження
робітників, зближення різних процесів праці та концентрації
засобів продукції.\footnote{
«З проґресом рільництва всю ту, а, може, і ще значнішу кількість
капіталу й праці, яку колись уживали для поверхового оброблення 500 акрів,
концентрується тепер для досконалішого оброблення 100 акрів».
Хоч «проти кількости вживаного капіталу й праці просторінь і скоротилася,
проте сфера продукції поширилася супроти тієї сфери, що її
раніш мав або експлуатував поодинокий незалежний аґент продукції»).
(«In the progress of culture all, and perhaps more than all the capital an
labour which once loosely occupied 500 acres, are now concentrated for the
more complete tillage of 100. Relatively to the amount of capital and labour
employed, space is concentrated, it is an enlarged sphere of production,
as compared to the sphere of production formely occupied or worked upon
by one single, independent agent of production»). (\emph{R. Jones}: «An Essay,
on the Distribution of wealth. Part I. On Rent», London 1831, p. 191,
199).
}

\index{i}{0265}  %% посилання на сторінку оригінального видання
Проти рівновеликої суми відокремлених індивідуальних робочих
днів комбінований робочий день продукує більші маси споживних
вартостей і тому зменшує робочий час, потрібний, щоб
досягти певного корисного ефекту. Чи комбінований робочий
день у даному випадку дістає цю збільшену продуктивну силу
тому, що він підносить механічну силу праці, чи тому, що поширює
її просторову сферу діяння; чи тому, що він супроти маштабу
продукції просторово звужує продукційне поле; чи тому, що в
критичний момент він пускає в рух багато праці за короткий час;
чи тому, що заохочує поодиноких осіб до змагання та напружує
їхній життєвий дух; чи тому, що він накладає печать безперервности
та багатобічности на однорідні операції багатьох осіб; чи
тому, що виконує одночасно різні операції, чи тому, що економізує
засоби продукції через спільний ужиток їх; чи тому, що надає
індивідуальній праці характеру пересічної суспільної праці, —
за всяких обставин специфічна продуктивна сила комбінованого
робочого дня є суспільна продуктивна сила праці, або продуктивна
сила суспільної праці. Вона випливає із самої кооперації. У пляномірному
співробітництві з іншими робітник стирає свої індивідуальні
межі й розвиває свою родову спроможність.\footnote{
«Сила кожної людини мінімальна, але сполука мінімальних сил
утворює спільну силу, більшу за суму цих сил, так що сили через саме
своє об’єднання можуть зменшити час та збільшити сферу своєї акції»
(«La forza di ciascuno uomo è minima, ma la riunione delle minime forze
forma una forza totale maggiore anche della somma delle forze medesime
fino a che le forze per essere riunite possono diminuere il tempo ed accrescere
lo spazio della loro azione»). (\emph{G. R. Carli} примітка до \emph{P. Verri}: «Meditazioni
sulla Economia Politica». vol. XV, p. 196). [«Колективна
праця дає такі результати, яких ніколи не могла б дати індивідуальна
праця. Отже, у міру того як зростатиме кількість людности, продукти
об’єднаної промисловости значно переважатимуть суму, що її ми мали
в наслідок простого складання, обчисленого на основі цього зросту\dots{}
У сфері механічних робіт так само, як і в сфері наукових робіт, людина
може протягом одного дня фактично зробити більше, ніж ізольований
індивід протягом усього свого життя. Аксіома математиків, що ціле
дорівнює сумі частин, прикладена до нашого предмету, вже не є правильна.
Щодо праці, цієї великої основи існування людства, то можна
сказати, що продукт об’єднаних зусиль значно переважає все те, що
могли б колибудь спродукувати зусилля поодиноких і розрізнених індивідів».
— \emph{Th. Sadler}: «The Law of Population», London 1850].\footnote*{
Наведене тут у прямих дужках ми беремо з французького видання.
(«Le Capital etc.», v. I, ch. XIII, p. 143). \emph{Ред.}
}
}
