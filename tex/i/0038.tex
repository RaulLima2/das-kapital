Саме такого роду форми і становлять категорії буржуазної
політичної економії. Це — суспільно-визнані дійсними, отже,
об’єктивні форми мислення для продукційних відносин цього
історично визначеного суспільного способу продукції — товарової
продукції. Тим то ввесь містицизм товарового світу, всі чари й
примари, що обгортають туманом продукти праці, які продукується
на основі товарової продукції, все це одразу зникає, скоро
тільки ми звернемось до інших форм продукції.

А що політична економія закохана в робінзонадах,29 то нехай
і у нас передусім з’явиться Робінзон на його острові. Хоч і який
він з роду скромний, та все ж він має задовольняти різноманітні
потреби, і тому мусить він виконувати корисні праці різного
роду: виробляти знаряддя, фабрикувати меблі, приручати ляму,
рибалчити, полювати й т. ін. Про молитву тощо ми тут не говоримо,
бо наш Робінзон находить у тому приємність і вважає подібну
діяльність за відпочинок. Не зважаючи на те, що його продуктивні
функції такі різні, він знає, що вони є тільки різні форми
діяльности того самого Робінзона, отже, лише різні роди людської
праці. Сама нужда примушує його точно розподілювати свій час
на різні функції. Чи та чи інша функція забере йому місця більше
чи менше в його цілій діяльності, це залежить від більших або
менших труднощів, які треба йому перебороти, щоб досягти наміченого
корисного ефекту. Досвід навчає його цього, і наш Робінзон,
що врятував з розбитого корабля годинник, головну книгу,
чорнило й перо, починає зразу ж, як чистокровний англієць, вести
книги про себе самого. Його інвентар складається з реестра предметів
споживання, які він посідає, різних операцій, потрібних для
продукції предметів споживання, нарешті, робочого часу, якого
йому коштує пересічно певна кількість цих різних продуктів. Всі
відношення між Робінзоном і речами, які становлять його власноручно
створене багатство, тут такі прості й прозорі, що навіть
пан М. Вірт зрозумів би їх, ані трохи не напружуючи розуму.
А проте в них містяться всі істотні визначення вартости.

Перенесімось тепер з ясного острова Робінзона в темне середньовіччя
Европи. Замість незалежної людини ми бачимо тут усіх
залежними — кріпаків і сеньйорів, васалів і сюзеренів, парафіян
і попів. Особиста залежність характеризує тут суспільні відносини
матеріяльної продукції так само, як і інші збудовані на ній сфери
життя. Але саме тому, що відносини особистої залежности ста-

29 Примітка до другого видання. І Рікардо не міг обійтися без робінзонади.
«Прарибалка і прамисливець у нього зараз же виступають
як посідачі товарів, що обмінюють рибу на дичину в пропорції, відповідній
до робочого часу, упредметненого в цих мінових вартостях. За цієї
нагоди він вдається в той анахронізм, що прарибалка й прамисливець
у своїх обрахунках знарядь праці користуються рахівничими таблицями,
що їх уживалось на лондонській біржі в 1817 р. «Паралелограми пана
Овена» є, здається, єдина суспільна форма, що її він знав крім буржуазної».
(Karl Marks: «Zur Kritik der Politischen Oekonomie», S. 38, 39. —
K. Маркс. «До критики політичної економії», ДВУ 1926 року,
стор. 76, 77).
