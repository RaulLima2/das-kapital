дії та з англійських рільничих округ», яка вона величезна розміром,
як вона 1860 р. дала 5/13 усієї англійської експортної
торговлі, як вона через декілька років знову зросте через поширення
ринку, особливо індійського, і через примусовий достатній
«довіз бавовни по 6 пенсів за фунт». Він каже далі: «Час — один,
два, може, три роки — випродукує потрібну кількість... І я хотів би
тоді поставити питання, чи варта ця промисловість того, щоб її
підтримувати, чи варто тримати машини (а саме живі робочі
машини) в порядку і чи не найбільша дурість думати про те, щоб
відмовитися від них? Я думаю, що так. Я визнаю, що робітники
не є власність («І allow that the workers are not a property»),
не власність Ланкашіру й хазяїнів; але вони — сила обох; вони
є духовна й навчена сила, що її не можна замістити протягом
життя однієї ґенерації; навпаки, інші машини, коло яких вони
працюють («the mere machinery which they work»), можна здебільшого
з користю замінити й поліпшити за дванадцять місяців.14
Заохочуйте до еміґрації робочої сили, або дозволяйте (І)
її, але ж, що тоді станеться з капіталістом? («Encourage
or allow the working power to emigrate, and what of the capitalist?»
Цей крик серця нагадує придворного маршала Кальба)...
Зберіть вершки робітників — і основний капітал буде в значній
мірі зневартнений, а оборотний капітал не зможе боротися при
недостатньому поданні праці нижчого сорту... Нам кажуть,
що робітники сами бажають еміґрації. Це дуже природна річ,
що вони це роблять... Зменшуйте, придушуйте бавовняне виробництво,
відбираючи в нього робочі сили (by taking away its working
power), зменшуючи, приміром, на третину або на 5 мільйонів
видатки в заробітній платі, але що тоді станеться з найближчою
над робітниками вищою клясою, — з дрібними крамарями?..
Що станеться з земельними рентами, з платою за наймання котеджів?..
З дрібним фармером, кращим домовласником і землевласником?
А тепер скажіть, чи може бути якийсь плян самозгубніший
для всіх кляс країни, ніж оцей плян ослабити націю
експортом її найкращих фабричних робітників і зневартненням
частини її найпродуктивнішого капіталу й багатства?» «Я раджу
позику в 5—6 мільйонів, розподілену на 2 або 3 роки, що нею
порядкуватимуть приставлені до адміністрації опіки над бідними

14    Пригадаймо собі, що той самий капітал співає іншої пісеньки за
звичайних обставин, коли йдеться про зниження заробітної плати. Тоді
«хазяїни», як один, заявляють (див. четвертий відділ, примітку 188):
«Хай фабричні робітники в своїх власних інтересах запам’ятають, що їхня
праця в дійсності є дуже низький сорт навченої праці; що жодної іншої
праці не можна легше вивчити та що, зважаючи на її якість, жодної
праці не оплачується ліпше; що жодної іншої праці не можна придбати
за такий короткий час та в такому великому розмірі, сяк-так привчивши
найменш досвідчених осіб. Машини хазяїна (що їх, як ми тепер
чуємо, можна з користю замінити й поліпшити за дванадцять місяців)
відіграють у дійсності далеко важливішу ролю в справі продукції, ніж
праця і вправність робітника (яких тепер не можна замінити за тридцять
років), яких можна навчитися за шість місяців і яких може навчитися
кожен сільський наймит».
