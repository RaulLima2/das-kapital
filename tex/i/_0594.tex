\parcont{}  %% абзац починається на попередній сторінці
\index{i}{0594}  %% посилання на сторінку оригінального видання
5. Essex.

По багатьох парафіях цього графства людність меншає рівнобіжно
із зменшенням числа котеджів. Однак не менш ніж у
22 парафіях зруйнування будинків не спинило зростання людности
або не призвело до її виселення, яке відбувається всюди під
назвою «переселення до міст». У Fingringhoe, парафії, що займає
3.443 акри, в 1851 р. було 145 хат, у 1861 р. — вже лише 110, але
людність не хотіла йти геть і умудрилась зростати навіть за
таких умов. У Ramsden Crags 1851 р. 252 особи жили в 61 хаті
а в 1861 р. 262 особи тулилися вже в 49 хатах. У Basilden в 1851 р
на 1.827 акрах жило 157 осіб у 35 хатах, наприкінці цього десятиліття
— 180 осіб у 27 хатах. У парафіях Fingringhoe, South
Farnbridge, Widford, Basilden і Ramsden Crags у 1851 p. на
8.449 акрах жило 1.392 особи в 316 хатах, в 1861 р. на тій самій
площі — 1.473 особи у 249 хатах.

6. Herefordshire.

Це маленьке графство більше, ніж яке інше в Англії, потерпіло
від «духу виселення». В Nadby переповнені котеджі,
звичайно з двома спальнями, належать здебільша фармерам.
Вони легко здають їх у найми за 3 або 4\pound{ фунти стерлінґів} на рік
і платять заробітну плату в 9 шилінґів на тиждень!

7. Huntingdonshire.

У Hartford’i в 1851 р. було 37 хат, незабаром після цього
в цій маленькій парафії на 1.720 акрів зруйновано 19 котеджів;
число мешканців становило в 1831 р. 452 особи, в 1852 р. — 832,
а в 1861р. — 341. Досліджено 14 cots з однією спальнею в кожному.
В одному з них живе одне подружжя, 3 дорослі сини, одна
доросла дівчина, 4 дітей, разом 10; в іншому — 3 дорослих,
6 дітей. Одна з цих кімнат, де спало 8 осіб, мала 12 футів ІОцалів
завдовжки, 12 футів 2 цалі завширшки, 6 футів 9 цалів заввишки;
пересічно, не відраховуючи площі виступів, на людину припадало
130 кубічних футів. У 14 спальнях — 34 дорослих і 33 дітей. Ці
котеджі рідко мають садочки, але багато мешканців могло орендувати
маленькі шматки землі по 10 або 12 шилінґів за rood (\sfrac{1}{4} акра).
Ці allotments\footnote*{
— парцелі. \emph{Ред.}
} лежать далеко від хат, сами хати не мають кльозетів.
Члени родини мусять ходити або на свої парцелі, щоб там
полишати свої екскременти, або, як це, вибачайте за слово, робиться
тут, наповняти ними шухляду шафи. Як вона заповниться,
її висувають і випорожнюють там, де її вміст потрібний. В Японії
кругобіг умов життя відбувається охайніше.

8. Lincolnshire.

Langtoft: Один чоловік живе тут у хаті Wright’a з своєю дружиною,
її матір’ю й 5 дітьми; в хаті є кухня, комірка-полоскальний
над кухнею — спальня; кухня й спальня мають 12 футів

\parbreak{}  %% абзац продовжується на наступній сторінці
