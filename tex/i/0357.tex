праці — вівці, коні і т. д., — безпосередні акти насильства становлять
тут першу передумову промислової революції. Спочатку
проганяють робітників із землі, а потім з’являються вівці. І тільки
розкрадання землі у великому маштабі, як ось в Англії, створює
для великого рільництва поле його діяльности.196а Тому на своїх
початках цей переворот у рільництві на позір має скорше вигляд
політичної революції.

Засіб праці у формі машини відразу стає конкурентом самого
робітника. 197 Самозростання вартости капіталу за допомогою
машини стоїть у прямому відношенні до числа робітників, умови
існування яких вона нищить. Ціла система капіталістичної продукції
ґрунтується на тому, що робітник продає свою робочу силу
як товар. Поділ праці уоднобічнює робочу силу на цілком частинну
вмілість — керувати частинним знаряддям. Скоро тільки
керування знаряддям переходить до машини, то разом із споживною
вартістю зникає й мінова вартість робочої сили. Робітник не
находить собі покупців, як паперові гроші, що виключені з обігу.
Та частина робітничої кляси, що її машини таким способом перетворюють
у надмірну людність, тобто в таку, яка безпосередньо
вже не потрібна для самозростання капіталу, з одного боку, гине
в нерівній боротьбі старого ремісничого й мануфактурного виробництва
з машиновим виробництвом, з другого боку, переповнює
всі приступніші галузі промисловости, переповнює ринок праці,
а тому знижує ціну робочої сили нижче за її вартість. Великою
втіхою для павперизованих робітників має бути те, що їхні страждання,
мовляв, почасти лише «тимчасові» («а temporary inconvenience»),
а почасти те, що машини, мовляв, лише поступінно
опановують ціле поле продукції, а через те зменшується розмір
та інтенсивність їхнього руйнаційного діяння. Одна втіха побиває
другу. Там, де машина захоплює якесь поле продукції поступінно,
вона породжує хронічні злидні серед робітничих верств, які з
нею конкурують. Там, де перехід відбувається швидко, там її
 вплив є масовий і гострий. Немає в світовій історії жахливішого
видовища, як поступінне вимирання англійських ручних бавовняних
ткачів, що тривало цілі десятиліття і, нарешті, завершилося
1838 р. Багато з них померло з голоду, багато животіло довгий
час із своїм родинами, мавши 2 1/2 пенси на день. 198 Навпаки,
гостро подіяло заведення англійських бавовняних машин у

196а [До четвертого видання. — Це стосується й до Німеччини. Там,
де в нас існує велике рільництво, отже, саме на Сході, воно стало можливим
лише через застосування системи «Bauernlegen»,* яке почалося
в XVI віці, особливо від 1648 р. — Ф. Е.].

197 «Машини й праця перебувають у постійній конкуренції» («Machinery
and labour are in constant competition»). (Ricardo: «Principles of
Political Economy». 3 rd ed., London 1821, p. 479).

198    Конкуренція між ручним та машиновим тканням перед заведенням
закону з 1833 р. про бідних затягувалася в Англії тим, що заро-

* Так називався в Німеччині процес експропріяції земель у селян;
в Англії цей процес звався «Clearing of Estates» («очищення маєтків» —
у дійсності очищення їх від людей). Див. про це далі розділ 24, §2. Ред.
