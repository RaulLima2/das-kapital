\parcont{}  %% абзац починається на попередній сторінці
\index{i}{0444}  %% посилання на сторінку оригінального видання
незмінна, тим часом як її абсолютна величина зростає, і, залежно
від ступеня здовження робочого дня, можуть зростати обидві.

В період від 1799 р. до 1815 р. зростання цін на засоби існування
в Англії призвело до номінального підвищення заробітної
плати, хоч дійсна, виражена в засобах існування плата і впала.
Звідси Вест і Рікардо зробили такий висновок, що зменшення
продуктивности рільничої праці спричинилося до зменшення
норми додаткової вартости, і це припущення факту, що існував
тільки в їхній фантазії, вони зробили вихідним пунктом важливої
аналізи кількісного співвідношення між величинами заробітної
плати, зиску та земельної ренти. Але в наслідок підвищеної
інтенсивности праці й вимушеного здовження робочого часу
додаткова вартість зросла тоді і абсолютно і відносно. Це був
той період, коли безмірне здовження робочого дня здобуло собі
прав громадянства,\footnote{
«Хліб і праця рідко коли йдуть цілком пліч-о-пліч; але існує
очевидна межа, що поза нею їх ніяк не можна розлучити. Щодо незвичайних
зусиль, роблених робітничими клясами за часів дорожнечі, зусиль,
що призвели до спаду заробітної плати, констатованого у свідченнях
(а саме перед парляментськими комітетами 1814--15 рр.), то вони
становлять більше заслугу поодиноких осіб і звичайно сприяють зростові
капіталу. Але жодна гуманна людина не забажає, щоб вони лишалися
постійними й неослаблими. Як тимчасова полегкість, вони варті
подиву, але коли б вони зробилися постійним явищем, то результат був
би такий самий, як коли б людність країни збільшилася до крайніх меж,
визначених засобами її існування». («Corn and Labour rarely march
quite abreast; but there is an obvious limit, beyond which they cannot
be separated. With regard to the unusual exertions made by the labouring
classes in periods of dearness, which produce the fall of wages noticed in
the evidence (Parliamentary Committees of Inquiry 1814--15), they are
most meritorius in the individuals, an certainly favour the growth of capital.
But no man of humanity could wish to see them constant and unremitted.
They are most admirable as a temporary relif; but if they were
constantly in action, effects of a similar kind would result from them, as
from the population of a country being pushed to the every extreme limits
of its food»). (\emph{Malthus}: «Inquiry into the Nature and Progress of Rent»,
London 1815, p. 48n). Робить честь Малтузові те, що він тут робить
наголос на здовженні робочого дня, про що він безпосередньо зазначає
ще в іншому місці свого памфлету, тимчасом як Рікардо й інші, не вважаючи
на наявність кричущих фактів, кладуть в основу всіх своїх дослідів
сталу величину робочого дня. Але консервативні інтереси, що їхнім
рабом був Малтуз, заважали йому бачити, що безмірне здовження робочого
дня разом з надзвичайним розвитком машин та експлуатацією жіночої
і дитячої праці мусило зробити «зайвою» велику частину робітничої
кляси, особливо після того, як припинилися зумовлений війною попит
та англійська монополія на світовому ринку. Природно, куди зручніше
було й куди більше відповідало інтересам панівних кляс, перед якими
Малтуз вклонявся чисто по-попівському, поясняти це «перелюднення»
з вічних законів природи, ніж з лише історичних природних законів
капіталістичної продукції.
} період, що його в Англії виразно характеризує
прискорене зростання капіталу на одному боці й павперизму
на другому.\footnote{
«Головна причина зростання капіталу за часів війни випливала
з великих зусиль і, може бути, з ще більших нестатків робітничих кляс,
найчисленніших у кожному суспільстві. В наслідок нужденних обставин
більше число жінок і дітей примушені були взятися до праці; а ті, які
ще раніш стали робітниками, мусили з тієї самої причини присвятити
більшу частину свого часу збільшенню продукції». («A principal cause
of the increase of capital, during the war, proceeded from the greater exertions,
and perhaps the greater privations of the labouring classes, the most
numerous in every society. More women and children were compelled, by
necessitous circumstances, to enter upon laborious occupations; and former
workmen were, from the same cause, obliged to devote a greater portion
of their time to increase production»). («Essays on Political Economy in
which are illustrated the Principal Causes of the Present National Distress»,
London 1830, p. 248).
}
\index{i}{0445}  %% посилання на сторінку оригінального видання

2) Збільшення інтенсивности і продуктивної сили праці при
одночасному скороченні робочого дня.

Підвищення продуктивної сили праці та зростання її інтенсивности
в одному напрямі діють однаково. Одне і друге збільшує
масу продуктів, продукованих протягом певного часу. Отже,
одне і друге скорочує ту частину робочого дня, що її робітник
потребує на продукцію своїх засобів існування — або їх еквіваленту.
Абсолютну мінімальну межу робочого дня визначає
взагалі ця його доконечна складова частина, яку однак можна
скорочувати. Коли б цілий робочий день скоротився до цієї останньої,
то зникла б додаткова праця — річ за капіталістичного
режиму неможлива. Усунення капіталістичної форми продукції
дозволяє обмежити робочий день доконечною працею. Однак, за
інших незмінних умов, остання поширила б свої рамки. З одного
боку, тому що життєві умови робітника покращали б і його життєві
потреби збільшилися б. З другого боку, довелося б до доконечної
праці залічити частину теперішньої додаткової праці,
саме працю, потрібну на те, щоб утворити суспільний резервний
фонд і фонд акумуляції.

Що більше зростає продуктивна сила праці, то більше можна
скорочувати робочий день, а що більше скорочується робочий
день, то більше може зростати інтенсивність праці. З суспільного
погляду продуктивність праці зростає також з її економією. Ця
остання включає не тільки економію на засобах продукції, але й
уникання всякої некорисної праці. Тимчасом як у кожному індивідуальному
підприємстві капіталістичний спосіб продукції примушує
до економії, його анархічна система конкуренції породжує
якнайбезмірніше марнотратство суспільних засобів продукції
та робочих сил поряд безлічі функцій, тепер неминучих, але по
суті зайвих.

За даної інтенсивности й продуктивности праці частина суспільного
робочого дня, доконечна для матеріяльної продукції,
є то коротша, отже, частина часу, завойована для вільної, розумової
й суспільної діяльносте індивідів, є то більша, що рівномірніше
поділено працю поміж усіма дієздатними членами суспільства,
що менше одна суспільна верства може звалити природну
доконечність праці з себе на інші верстви. З цього погляду,
абсолютна межа для скорочення робочого дня є вселюдність праці.
У капіталістичному суспільстві вільний час однієї, кляси створюється
перетворенням усього життя мас на робочий час.
