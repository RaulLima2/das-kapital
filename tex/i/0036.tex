тичних зносинах, у обміні продуктів; отже, суспільно-корисний
характер їхніх приватних праць мозок їхній відбиває в тій формі,
що продукт праці мусить бути корисний, і то саме для інших
людей, а суспільний характер рівности різнорідних праць він
відбиває у формі спільного характеру вартости цих матеріяльно
відмінних речей, продуктів праці.

Отже, люди ставлять у взаємні відношення продукти своєї
праці як вартості не тому, що ці речі мають для них значення лише
речових оболонок однорідної людської праці. Навпаки. Прирівнюючи
в процесі обміну один до одного свої різнорідні продукти
як вартості, вони тим самим прирівнюють одну до однієї свої
різні праці як людську працю. Вони несвідомі цього, але вони
це роблять.\footnote{
Примітка до другого видання. Тим то, коли Galiani каже: Вартість
є відношення між двома персонами — «La Richezza è una ragione
tra due persone», — то він мусив би був додати: відношення, заховане під
речовою оболонкою. (Galiani: «Della Moneta», р. 220, vol. III збірника
Custodi: «Scrittori Classici Italiani di Economia Politica». Parte
Moderna. Milano 1801).
} Отже, у вартости на чолі не написано, що вона є.
Скорше вартість перетворює кожний продукт праці на суспільний
гієрогліф. Пізніше люди силкуються розшифрувати значення того
гієрогліфу, збагнути таємницю свого власного суспільного продукту,
бож визначення предметів споживання як вартостей є
так само їхній суспільний продукт, як і мова. Пізніше наукове
відкриття, що продукти праці, оскільки вони є вартості, є лише
речові вирази людської праці, витраченої на їхню продукцію,
становить епоху в історії розвитку людства, але ні в якому разі
не усуває предметної зовнішности суспільного характеру праці.
Те, що має силу лише для цієї осібної форми продукції, для товарової
продукції, а саме, що специфічний суспільний характер
незалежних одна від однієї праць полягає в їхній рівності як
людської праці взагалі і набирає форми характеру вартости продуктів
праці, — це для людей, захоплених відношенням товарової
продукції, видається, як перед цим відкриттям, так і після нього,
так само незмінним і так само природним, як і те, що форма повітря
як фізична тілесна форма й далі існує, хоч наукова аналіза
й розклала повітря на його складові елементи.

Осіб, що обмінюють продукти, передусім практично цікавить
питання, скільки чужих продуктів вони дістануть за свій власний
продукт, тобто в яких пропорціях обмінюються продукти. Скоро
ці пропорції набирають певної звичної сталости, тоді здається,
нібито вони випливають із самої природи продуктів праці; приміром,
здається, що одна тонна заліза й дві унції золота мають однакову
вартість, подібно до того, як 1 фунт золота і 1 фунт заліза, не
зважаючи на те, що їхні фізичні та хемічні властивості є неоднакові,
мають однакову вагу. В дійсності характер вартости продуктів
праці закріпляється лише в наслідок функціонування їх як
вартостей певної величини. Величини вартостей змінюються постійно
незалежно від волі, передбачення й діяльности осіб, що