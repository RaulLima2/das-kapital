сів. Якщо ж робітника вживатимуть до праці менше ніж 12 годин
на день (або менше ніж 6 днів на тиждень), наприклад, лише
6 або 8 годин, тоді він одержить за цієї ціни праці лише 2 або
1\sfrac{1}{2} шилінґи денної плати.\footnote{
Вплив такої ненормальної недостачі роботи цілком відрізняється
від впливу загального примусового законодавчого скорочення робочого
дня. Перша не має нічого спільного з абсолютною довжиною робочого
дня й може постати так само при п’ятнадцятигодинному, як і при шестигодинному
робочому дні. Нормальну ціну праці в першому випадку обчислюється
з того, що робітник працює пересічно 15 годин, у другому — з
того, що він працює пересічно 6 годин денно. Тому вплив лишається той
самий, якщо робітник працює в одному випадку лише 7\sfrac{1}{2}, а в другому
лише 3 години.
} А що, за нашим припущенням, він
мусить працювати пересічно 6 годин на день, щоб випродукувати
лише денну плату, відповідну вартості його робочої сили, що,
за тим самим припущенням, він працює лише половину з кожної
години на себе самого, а половину на капіталіста, то ясно, що
він не може одержати для себе вартости, випродукованої протягом
6 годин, коли працюватиме менше ніж 12 годин. Якщо раніше
ми бачили руйнаційні наслідки надмірної праці, то тут ми відкриваємо
джерела тих страждань, які постають для робітника від
недостачі роботи.

Коли погодинну заробітну плату встановлюється так, що
капіталіст не зобов’язується платити якусь поденну або потижневу
заробітну плату, а зобов’язується платити тільки за ті
робочі години, протягом яких йому забажається вживати робітника,
то він може вживати його на час, менший від того, що його
первісно покладено в основу визначення погодинної заробітної
плати або одиниці міри для ціни праці. А що ця одиниця міри
визначається відношенням

денна вартість робочої сили/робочий день даного числа годин,

то вона, природно, втрачає всякий сенс, скоро тільки робочий
день перестає мати в собі певне число годин. Знищується зв’язок
між оплаченою й неоплаченою працею. Тепер капіталіст може
видушити з робітника певну кількість додаткової праці, не даючи
йому для роботи робочого часу, доконечного для його власного
утримання. Він може знищити всяку реґулярність праці й цілком
залежно від своєї вигоди, сваволі та інтересів моменту, навпереміну
то примушувати робітника до неймовірної надмірної
праці, то залишати його почасти або й зовсім без роботи. Він
може з того приводу, що він нібито платить «нормальну ціну
праці», ненормально здовжувати робочий день без будь-якої відповідної
компенсації для робітника. Звідси цілком раціональне
повстання лондонських будівельних робітників (1860 р.) проти
спроби капіталістів накинути їм таку погодинну заробітну плату.
Законодавче обмеження робочого дня покладає кінець цій неподобності,
хоч, природно, не знищує недостачі роботи, що випли-