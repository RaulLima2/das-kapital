\parcont{}  %% абзац починається на попередній сторінці
\index{i}{0511}  %% посилання на сторінку оригінального видання
робітника\footnote{
«Рікардо каже: «На різних стадіях розвитку суспільства акумуляція
капіталу або засобів уживати працю (тобто експлуатувати її)
є більш або менш швидка і в усіх випадках мусить залежати від продуктивних
сил праці. Продуктивні сили праці взагалі найбільші там, де
існує надмір родючої землі». Якщо в цьому реченні продуктивні сили
праці означають мізерність тієї частини кожного продукту, яка припадає
тим, що продукують його своєю ручною працею, то ця теза є тавтологія,
бо частина, яка залишилась, є той фонд, що з нього, коли цього захочеться
власникові його («if the owner pleases»), можна акумулювати капітал.
Але здебільшого цього не буває там, де країна є найродючіша».
(«Observations on certain verbal disputes etc.», p. 74, 75).
}. У відділах про продукцію додаткової вартости ми
постійно припускали, що заробітна плата принаймні дорівнює
вартості робочої сили. Однак на практиці примусове зниження
заробітної плати нижче від цієї вартости відіграє надто важливу
ролю, і тому ми мусимо хоч на хвилину спинитися на ньому
В певних межах воно фактично перетворює доконечний фонд
споживання робітника на акумуляційний фонд капіталу.

«Заробітні плати, — каже Дж. Мілл, — не мають продуктивної
сили; вони — ціна продуктивної сили; заробітні плати не
беруть участи, поряд самої праці, в продукції товарів, так само
як і ціна самих машин. Коли б працю можна було мати, не купуючи
її, заробітні плати були б зайві»\footnote{
«\emph{J. St. Mill}: Essays on some unsetled Questions of Political Economy»,
London 1844, p. 90.
}. Але коли б робітники
могли жити з повітря, то їх і не можна було б купити ні за яку ціну.
Отже, зниження заробітної плати до нуля є межа в математичному
розумінні, що її ніколи не можна досягти, хоч до неї можна
завжди наближатися. Постійна тенденція капіталу — це знизити
заробітну плату до цього нігілістичного рівня. Часто цитований
мною письменник XVIII століття, автор «Essay он Trade and
Commerce», виказує лише якнайінтимнішу таємницю душі англійського
капіталу, заявляючи, що історичне життєве завдання
Англії — це знизити англійську заробітну плату до французького
й голляндського рівня\footnote{
«An Essay on Trade and Commerce», London 1770, p. 44. У грудні
I860 p. і в січні 1867~\abbr{р.} «Times» умістив подібні серцевиливи англійських
посідачів копалень, де описувалось щасливий стан бельгійських
копальневих робітників, які не вимагали і не діставали нічого більше,
а тільки те, що було доконечно потрібне, щоб жити для своїх «хазяїнів».
Бельгійські робітники терплять багато, і все це для того, щоб фігурувати
в «Times» як зразкові робітники! Страйк бельгійських копальневих
робітників (коло Marchienne) на початку лютого 1867~\abbr{р.}, придушений
порохом і оливом, був відповіддю на цю атестацію.
}. Він, між іншим, наївно каже: «Коли
наша біднота (символічна назва робітників) хоче жити в розкошах\dots{}
то, певна річ, її праця мусить бути дорога\dots{} Погляньте
тільки на силу-силенну зайвих речей («heap of superfluities), —
аж волосся стає диба, — що їх споживають наші мануфактурні
робітники, як ось горілка, джин, чай, цукор, чужоземні
овочі, міцне пиво, ситець, табака, тютюн і~\abbr{т. ін.}»\footnote{
Там же, стор. 44, 46.
}. Він цитує
твір одного фабриканта з Нортгемптоншіру, що, звівши очі до
неба, лементує: «Праця на цілу третину дешевша у Франції,
\parbreak{}  %% абзац продовжується на наступній сторінці
