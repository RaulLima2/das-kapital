тає чудовий плід цього cercle vicieux\footnote*{
— зачарованого кола. Ред.
} — так звана Gangsystem
система артілей або ватаг (Gang-oder Bandensystem), про яку
я скажу тут декілька слів.\footnote{
Шостий і останній «Report of Children’s Employment Commission»,
опублікований наприкінці березня 1867 р., каже лише про систему
рільничих артілей.
}

Система артілей процвітає майже виключно в Lincolnshire
Huntingdonshire, Cambridgeshire, Norfolk, Suffolk і Nott nghamshire,
спорадично — в сусідніх графствах Nothampton, Bedford
і Rutland. Як приклад візьмімо тут Lincolnshire. Значна
частина цього графства — це нова земля, колишнє болото, абож
земля, як і в інших названих східніх графствах, відвойована від
моря. Парова машина наробила чудес при осушуванні. Колишня
драговина й пісковий ґрунт красіють тепер у наряді буйного
збіжжя і дають якнайвищу ренту. Те саме стосується й до штучно
здобутого наносного ґрунту, як от на острові Axholme та інших
парафіях на побережжі Trent’y. В міру того як виникали нові
фарми, не лише не будували нових котеджів, але руйнували і
старі, а робітників постачали з відкритих сел, віддалених за
кілька миль, порозкидуваних здовж сільських доріг, що в’ються
схилами горбів. Лише там людність раніш знаходила собі захист
від тривалих зимових поводей. Робітники, що живуть на фармах
розміром від 400 до 1.000 акрів (їх тут звуть «confined labourers»\footnote*{
— «прикріплені робітники». Ред.
}),
служать виключно для постійних важких польових робіт, виконуваних
кіньми. На кожні 100 акрів (1 акр — 40,49 ара, або
1,584 пруських морґів) пересічно припадає ледве один котедж.
Один фармер, що орендував колишню драговину, свідчить перед
слідчою комісією: «Моя фарма має більш ніж 320 акрів, все це
саме орне поле. Котеджів на ній немає. Тепер у мене живе один
робітник. Чотири робітники, що доглядають моїх коней, живуть

між іншим, раніш цитовану працю Colins’a: «L’Economie Politique», і
Karl Marx: «Der Achtzehnte Brumaire des Louis Bonaparte», 2 Aufl.
Hamburg 1869 p., стор. 91 і далі. (К. Маркс: «Вісімнадцяте Брюмера Люї
Бонапарта», Партвидав «Пролетар» 1932, стор. 100 і далі). В 1846 р.
міська людність Франції становила 24,42\%, сільська — 75,58\%, в 1861р.
міська — 28,86\% сільська — 71,14\%. Протягом останніх п’яти років
зменшення проценту сільської людности ще більше. Уже в 1846 р. П’ер
Дюпон співає в своєму «Ouvriers»:

«Mal vêtus, logés dans des trous,
Sous les combles, dans les décombres,
Nous vivons avec les hiboux
Et les larrons, amis des ombres».

(«Усі в дранті, тут на горищах,
Серед руїн, в льохах живемо ми,
Де сови лиш та злодії нічні
Ховаються, охочі до пітьми»).