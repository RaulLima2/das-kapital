не вагаючись, визнав цей факт».\footnote{
Там же.
} «На одній вальцювальні, де
номінальний робочий день тривав од 6 години ранку до 5 1/2  годин
вечора, один хлопчик працював щотижня чотири ночі щонайменше
до 8 1/2 годин вечора другого дня... і так протягом
6 місяців». «Другий, дев’яти років, працював іноді три дванадцятигодинні
зміни одну по одній, а мавши десять років, — два дні й
дві ночі вряд». «Третій хлопчик, тепер йому 10 років, працював
від 6 години ранку до 12 години вночі три ночі одну по одній і
до 9 години вечора протягом інших ночей». «Четвертий, тепер
йому 13 років, працював цілий тиждень від 6 години вечора до
12 години другого дня, а іноді три зміни одну по одній, приміром,
від понеділка зранку до вівтірка вночі». «П’ятий, що має тепер
12 років, працював в одній залізоливарні в Stavely від 6 години
ранку до 12 години вночі протягом двох тижнів і вже нездатний,
так працювати далі». Джордж Еллінзворт, дев’яти років: «Я прийшов
сюди минулої п’ятниці. Ми мали почати працю на другий
день о 3 годині вранці. Тому я лишився тут на цілу ніч. Живу я
5 миль відси. Спав у сінях, підстеливши собі шкіряний хвартух
і вкрившись маленькою куциною. Другі два дні я приходив сюди
о 6 годині вранці. Еге, це таки пекуче місце! Поки я сюди прийшов,
я працював так само цілий рік коло домни. Це був величезний
завод на селі. Я починав працю теж у суботу ранком о 3 годині, але я міг принаймні ходити додому
спати, бо це було
недалечко. Іншими днями я починав працю ранком о 6 годині, а
кінчав увечері о 6 або о 7 годині» і т. ін.\footnote{
Там же, стор. XIII. Рівень освіти цих «робочих сил» мусить, природно,
бути такий, яким він виявляється в дальших діялогах з одним із
членів слідчої комісії! Джірімія Гейнс, 12 років: «... Чотири рази чотири
є вісім, але чотири четвірки (4 fours) є шістнадцять»... Король для
нього є той, хто має всі гроші й усе золото. «Ми маємо короля; кажуть,
що він є королева, її називають принцеса Олександра. Кажуть, що вона
одружилася з сином королеви. Принцеса — це чоловік». В. Тернер,
дванадцяти років: «Я живу не в Англії. Я гадаю, що є така країна, але
раніш я нічого не знав про це». Джон Морріс, чотирнадцяти років: «Я чув,
що бог створив світ, і що ввесь нарід потопився, опріч однієї людини; я
чув, що то був маленький пташок». Вільям Сміс, п’ятнадцяти років:
«Бог створив чоловіка, чоловік створив жінку». Едвард Тейлор, п’ятнадцяти
років: «Нічого не знаю про Лондон». Генрі Матьюмен, сімнадцяти
років: «Я ходжу іноді до церкви... Одне ім’я, що вони про нього проповідують,
то був якийсь Ісус Христос, але іншого імени я назвати не можу
та й про нього нічого не можу сказати. Його не забили, а він помер, як
і інші люди. Він де в чому не був такий, як інші люди, бо був, сказати б,
релігійний, а інші ні». («Не was not the same as other people in some ways,
because he was religious in some ways, and others is n’t»). («Там же, 74,
p. XV). «Чорт — добра особа, я не знаю, де він живе». «Христос був погана
людина». («The devil is a good person. I don’t know where he lives». «Christ
was a wicked man»), «Ця дівчинка (10 років), замість God (бог) складає
Dog (собака) і не знає імени королеви». («Children’s Employment Commission.
5 th Report, 1866», p. 55, n. 278). Ta сама система, що й на згаданих
металюрґійних мануфактурах, панує на гутах і на фабриках паперу.
На фабриках паперу, де папір виробляють машинами, нічна праця
є загальне правило для всіх процесів, крім сортування лахміття. В деяких
випадках нічна праця за допомогою змін триває безперестанку цілий
}