\parcont{}  %% абзац починається на попередній сторінці 
\index{i}{*0094}  %% посилання на сторінку оригінального видання 
видання, в цілому відповідають змінам, що їх Маркс у ряді рукописних
вказівок наказав зробити в англійському перекладі, наміченому
до видання в Америці вже з десяток років тому, але відкладеному
головне за браком тямущого й пристойного перекладача.
Цей рукопис дав у наше розпорядження наш старий приятель,
містер Ф. А. Зорґе в Гебекені, штат Нью-Джерсей. В ньому
відзначені ще деякі дальші вставки з французького видання;
але що цей манускрипт на кілька років старший, ніж останні вказівки
до третього видання, то я не вважав за право моє користатися
з нього інакше, як у виняткових випадках і особливо тоді,
де це допомагало нам розв’язати труднощі. Так само брали ми
французький текст для більшости важких місць, як вказівку на те,
чим сам автор готовий був пожертувати, коли доводилось жертвувати
чимось з загального сенсу ориґіналу, перекладаючи його.

Ще є труднощі, що від них ми не могли звільнити читача: вживання
деяких виразів у розумінні, відмінному не тільки від щоденного
слововживання, але й від слововживання звичайного в
політичній економії. Та це було неминуче. Кожне нове розуміння
науки несе з собою революцію у фаховій термінології цієї науки.
Це доводить найкраще хемія, де вся термінологія ґрунтовно змінюється
приблизно щодвадцять років і де майже не можна знайти
такої органічної сполуки, що не перейшла б цілий ряд назов.
Політична економія взагалі задовольнилась з того, що брала
вирази торгового й промислового життя точно такими, як вони
були, і оперувала ними, зовсім при цьому випускаючи з уваги, що
в такий спосіб вона обмежувалась вузьким колом ідей, виражених
цими термінами. Так, навіть клясична політична економія, хоч
і була цілком свідома того, що так зиск, як і рента є лише підрозділи,
частки тієї неоплаченої частини продукту, що її мусить
давати робітник своєму підприємцеві (першому присвоювачеві
цієї частини продукту, хоч і не останньому виключному власникові
її), все ж вона ніколи не підносилась понад звичайне розуміння
зиску і ренти, ніколи не досліджувала цієї неоплаченої
частини продукту (що її Маркс називає додатковим продуктом)
в її сукупності як ціле і тому ніколи не могла дійти ясного розуміння
ані її походження й її природи, ані тих законів, що реґулюють
дальший розподіл її вартости. Так само всю промисловість,
крім сільського господарства і ремества, однаково охоплювалось
терміном мануфактури і тим самим стиралося ріжницю між двома
великими і посутньо різними періодами економічної історії: періодом
власне мануфактури, основаної на поділі ручної праці,
і періодом сучасної промисловости, основаної на машиновому
виробництві. Однак, само собою зрозуміло, що теорія, яка розглядає
сучасну капіталістичну продукцію як просту переходову
стадію в економічній історії людства, мусить вживати інших
термінів, ніж звичайні в тих письменників, що розглядають цю
форму продукції як вічну й остаточну.

Не зайво буде сказати кілька слів про методу автора наводити
цитати. В більшості випадків цитати, як звичайно, служать за
\parbreak{}  %% абзац продовжується на наступній сторінці
