Проте одинична форма вартости сама собою переходить у досконалішу
форму. Правда, і за допомогою цієї останньої форми вартість
якогось товару виражається лише в одному товарі іншого
роду. Але це цілком байдуже, якого роду є цей другий товар: чи
це сурдут, залізо, чи пшениця й т. ін. Отже, залежно від того,
як даний товар стає у вартостеве відношення до того або іншого
роду товару, виникають різні прості вирази вартости одного й
того самого товару.22a Число можливих виразів його вартости
обмежено лише числом відмінних од нього родів товару. Тому
одиничний вираз вартости товару перетворюється на ряд його
різних простих виразів вартости, що його можна продовжувати
досхочу.

В. повна або розгорнута форма вартости

z товару А = u товару В, або = υ товару С, або = w товару D,
або = х товару Е, або = й т. ін.

(20 метрів полотна = 1 сурдутові, або = 10 фунтам чаю, або = 40 фунтам
кави, або = 1 квартерові пшениці, або = 2 унціям золота, або = \sfrac{1}{2} тонни
заліза, або = й т. ін.).

1. Розгорнута відносна форма вартости

Вартість якогось товару, приміром, полотна, виражено тепер
у безлічі інших елементів товарового світу. Кожне інше товарове
тіло стає дзеркалом вартости полотна.23 Таким чином сама ця
вартість лише тепер справді з’являється як згусток безріжницевої
людської праці. Бо й працю, що творить її, тепер виразно

22а Примітка до другого видання. Наприклад, у Гомера вартість
однієї речі виражається в ряді різних речей.

23 Через те говорять про сурдутову вартість полотна, коли вартість
його виражають у сурдутах, про його збіжжеву вартість, коли виражають
її в збіжжі, і т. ін. Кожний такий вираз показує, що те, що виявляється
у споживних вартостях «сурдут», «збіжжя» й т. ін., є вартість полотна.
«Вартість кожного товару позначає його відношення в обміні…,
так що ми можемо говорити про неї як про збіжжеву вартість, сукняну
вартість, залежно від того, з яким товаром його порівнюється; отже, є
тисячі різних родів вартости, стільки, скільки є товарів, і всі вони однаково
реальні й однаково номінальні» («The value of any commodity denoting
its relation in exchange…, we may speak of it as… corn-value,
cloth-value according to the commodity with which it is compared; and
hence there are a thousand different kinds of value, as many kinds
of value as there are commodities in existence, and all are equally real
and equally nominal»). («A Critical Dissertation on the Nature, Measure
and Causes of value: chiefly in reference to the writings of Mr. Ricardo
and his followers». By the Author of «Essays on the Formation etc. of Opinions»,
London 1825, p. 39). S. Bailey, автор цієї анонімної праці, що
свого часу наробила чимало шуму в Англії, уявляє собі, що, вказуючи
на ці строкаті відносні вирази тієї самої товарової вартости, він знищив
усі визначення поняття вартости. Роздратовання, з яким напала на нього
школа Рікарда, наприклад, у «Westminster Review», доводить нам, що
Bailey, не зважаючи на свою обмеженість, прозондував усе ж таки
хибні місця теорії Рікарда.
