та постійно дбає про ремонт їх, як от інженери, механіки, столярі
й т. д. Це вища кляса робітників, почасти з науковою освітою,
почасти реміснича, що стоїть поза колом фабричних робітників
та лише прилучена до них.\footnote{
Це характеристично для навмисного статистичного шахрайства,
яке можна було б викрити зрештою аж до дрібниць, що англійське фабричне
законодавство рішуче виключає із сфери свого впливу згаданих щойно
в тексті робітників, як нефабричних робітників, а з другого боку —
«Returns», опубліковані парляментом, так само рішуче залічують до
категорії фабричних робітників не тільки інженерів, механіків і т. д.,
але й директорів фабрик, прикажчиків, розсильних, доглядачів складів,
упаковників і т. д., словом, усіх людей, за винятком самого власника
фабрики.
} Цей поділ праці є чисто технічний.

Всяка праця коло машини вимагає привчання робітника
замолоду, щоб він учився пристосовувати свій власний рух до
одностайного безупинного руху автомату. Оскільки сукупність
фабричних машин сама становить систему різноманітних комбінованих
машин, що функціонують одночасно, остільки й з ґрунтована
на ній кооперація вимагає розподілу різнорідних груп
робітників поміж різнорідними машинами. Але машинове виробництво
усуває (hebt auf) доконечність закріпляти, як у мануфактурі,
цей розподіл постійним прикріпленням тих самих робітників
до тієї самої функції.\footnote{
Юр визнає це. Він каже, що робітників «у крайньому разі» можна
перекидати з волі директора від однієї машини до іншої, та, тріюмфуючи,
вигукує: «Подібна зміна явно суперечить старій рутині, яка поділяє
працю і одному робітникові дає завдання виготувати головку до шпильки,
а другому шліхтувати її вістря» («Philosophy of Manufacture», р. 22).
Він скорше повинен був би себе запитати, чому «стару рутину» кидають
на автоматичній фабриці лише «у крайньому разі».
} А що цілий рух фабрики виходить не від
робітника, а від машини, то може відбуватися постійна зміна
осіб без перерви в процесі праці. Найразючіший доказ на це дає
система змін (Relaissystem), запроваджена в життя підчас бунту
англійських фабрикантів 1848—1850 рр. Нарешті, та швидкість,
з якою навчаються в юнацькому віці працювати коло машини,
так само усуває доконечність виховувати осібну клясу робітників
на виключно машинових робітників.\footnote{
Коли є недостача людей, як це було, приміром, за часів американської
громадянської війни, то буржуа вживає фабричних робітників
тільки до найважчих робіт, як будування вулиць і т. ін. Англійські
«ateliers nationaux» * 1862 й наступних років для безробітних бавовняних
робітників тим відрізняються від французьких року 1848, що в останніх
робітник мав на кошти держави виконувати непродуктивну прашо, в
перших же — продуктивну міську працю на користь буржуа, і до того ж
дешевше, ніж регулярні робітники, в конкуренцію з дкими таким чином
кинуто безробітного. «Фізичний вигляд бавовняного робітника безперечно
покращав. Це я приписую... оскільки мова йде про чоловіків, праці
коло громадських робіт на свіжому повітрі» («The physical appearance
of the cotton opératives is unquestionably improved. This I attribute...
as to the men, to outdoor labour on public Works»). (Тут говориться
про престонських фабричних робітників, які працювали коло висушування
«Престонського болота»). («Reports of Insp. of Fact. for October
1865», p. 59).
* — національні майстерні. \emph{Ред.}
} Щождо праці простих