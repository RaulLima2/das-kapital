\parcont{}  %% абзац починається на попередній сторінці
\index{i}{0433}  %% посилання на сторінку оригінального видання
чому він тепер працює 6 днів на тиждень або чому він дає 5 днів
додаткової праці. Вона пояснює лише те, чому його доконечний
робочий час обмежений одним днем на тиждень. Але його додатковий
продукт ні в якому разі не виникає з якоїсь природженої
таємничої властивости людської праці.

Продуктивні сили праці, зумовлені природою, так само як
і історично розвинені, суспільні продуктивні сили, видаються
продуктивними силами капіталу, що до нього долучено працю.

Рікардо ніколи не турбується про походження додаткової
вартости. Він розглядає її як річ, іманентну капіталістичному
способові продукції, який в його очах є природна форма суспільної
продукції. Там, де він говорить про продуктивність праці,
він шукає в ній не причину буття додаткової вартости, а лише
причину, що визначає її величину. Навпаки, його школа голосно
оповістила, що продуктивність праці є причина виникнення
зиску (читай: додаткової вартости). В усякому разі це — проґрес
проти меркантилістів, які, з свого боку, надлишок ціни
продуктів понад їхні витрати продукції виводять з обміну, з
продажу їх понад їхню вартість. Однак і школа Рікарда обійшла
лише цю проблему, а не розв’язала її. Справді, ці буржуазні
економісти інстинктивно відчували, що дуже небезпечна річ надто
глибоко досліджувати пекуче питання про походження додаткової
вартости. Але що сказати, коли через півстоліття після Рікарда
пан Джон Стюарт Мілл поважно констатує свою перевагу над
меркантилістами, поганенько повторюючи нікчемні викрути перших
вульґаризаторів Рікарда?

Мілл каже: «Причина зиску в тому, що праця продукує
більше, аніж потрібно для її утримання». Покищо це лише стара
пісня, але Мілл хоче додати й дещо своє: «Або, змінюючи форму
цієї тези: причина, чому капітал дає зиск, у тому, що харчі,
одяг, сировинний матеріял і засоби праці тривають довший час,
аніж це потрібно для їхньої продукції». Мілл сплутує тут тривання
робочого часу з часом тривання його продуктів. З цього
погляду пекар, продукти якого тривають лише один день, ніколи
не міг би витягти із своїх найманих робітників такий самий зиск,
як машинобудівник, продукти якого тривають двадцять і більше
років. Звичайно, коли б гнізда птахів трималися тільки стільки
часу, скільки потрібно, щоб їх збудувати, то довелося б птахам
давати собі раду без гнізд.

Установивши цю основну істину, Мілл констатує потім свою
перевагу над меркантилістами: «Отже, ми бачимо, що зиск походить
не з випадкового факту обміну, а з продуктивности праці;
ввесь зиск даної країни завжди визначається продуктивною силою
праці, однаково, чи відбувається будь-який обмін, чи ні. Коли б
не було поділу професій, то не було б ані купівлі, ані продажу,
але все ж був би зиск». Отже, тут обмін, купівля і продаж, оці
загальні умови капіталістичної продукції, є лише простий випадок,
а зиск існує навіть і тоді, коли немає купівлі й продажу
робочої сили!
\parbreak{}  %% абзац продовжується на наступній сторінці
