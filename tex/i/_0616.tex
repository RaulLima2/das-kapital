\parcont{}  %% абзац починається на попередній сторінці
\index{i}{0616}  %% посилання на сторінку оригінального видання
товаровласників мусять протистати один одному і увійти в контакт:
з одного боку, власники грошей, засобів продукції та засобів
існування, які жадають через купівлю чужої робочої сили
збільшувати присвоєну ними суму вартости; з другого боку,
вільні робітники, продавці своєї власної робочої сили, отже, продавці
праці. Вільні робітники у двоякому значенні: ані вони
сами не належать безпосередньо до засобів продукції, як раби
кріпаки тощо, ані засоби продукції не належать їм, як це є у
селянина тощо, який самостійно господарює, — вони вільні від
них, цілком позбавлені їх. Разом з цією поляризацією товарового
ринку дано основні умови капіталістичної продукції. Капіталістичні
відносини мають за передумову відокремлення робітників
від власности на засоби здійснення праці. Скоро тільки
капіталістична продукція стає на власні ноги, вона не лише
підтримує це відокремлення, але й репродукує його невпинно
в щораз більшому маштабі. Отже, процес, який утворює капіталістичні
відносини, не може бути чимось іншим, як процесом
відокремлення робітника від власности на умови його праці,
процесом, що, з одного боку, перетворює суспільні засоби існування
й продукції на капітал, а з другого — безпосередніх продуцентів
на найманих робітників. Отже, так звана первісна
акумуляція є не що інше, як історичний процес відокремлення
продуцента від засобів продукції. Він з’являється як «первісний»
тому, що становить передісторію капіталу й відповідного
йому способу продукції.

Економічна структура капіталістичного суспільства виникла
з економічної структури февдального суспільства. Розклад
останнього визволив елементи першого.

Безпосередній продуцент, робітник, лише відтоді міг порядкувати
своєю особою, коли він перестав бути прикріпленим до
землі або кріпаком, залежним від іншої особи. Далі, щоб зробитись
вільним продавцем робочої сили, який подає свій товар
усюди, де є на нього попит, робітник мусив визволитись від влади
цехів, від цехових статутів про учнів і підмайстрів та інших
обмежувальних щодо праці приписів. Таким чином, історичний
рух, що перетворює продуцентів на найманих робітників, з’являється,
з одного боку, як визволення продуцентів від февдального
та цехового примусу; і лише цей бік існує для наших буржуазних
істориків. Але, з другого боку, і нововизволені лише
відтоді стають продавцями самих себе, коли в них пограбовано
всі їхні засоби продукції та всі ґарантії існування, забезпечувані
їм старовинними февдальними інституціями. І історію цієї
експропріяції їх вписано в літописи людства мовою меча й вогню.

Промислові капіталісти, ці нові владарі, мусіли з свого боку
витиснути не лише цехових майстрів, але й февдалів, що володіли
джерелами багатства. З цього боку їхнє піднесення є наслідок
переможної боротьби проти февдальної влади та її обурливих
привілеїв, а також проти цехів і тих пут, що їх ці останні накладають
на вільний розвиток продукції та вільну експлуатацію
\parbreak{}  %% абзац продовжується на наступній сторінці
