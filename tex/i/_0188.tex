\parcont{}  %% абзац починається на попередній сторінці 
\index{i}{0188}  %% посилання на сторінку оригінального видання 
окрузі Stoke працює по ганчарнях лише 30,6\%, а у Wolstanton
лише 30,4\% чоловічої людности у віці понад 20 років, однак
з-поміж чоловіків цієї категорії в першій окрузі більше, ніж
половина, а у другій окрузі приблизно дві п’ятих із загального
числа смертних випадків з грудних недуг припадає на ганчарів.
Д-р Бутройд, лікар, що практикує в Hanley, свідчить: «Кожне
наступне покоління ганчарів карликуватіше й слабіше, ніж попереднє».
Так само інший лікар, пан Мак-Бін, каже: «Від того
часу, як я перед 25 роками почав свою практику серед ганчарів,
наочне виродження цієї кляси виявилось у дедалі швидшому
зменшенні зросту й ваги». Ці свідчення взято із звіту д-ра
Ґрінхова з 1860 р.\footnote{
«Public Health. Зrd Report etc.», p. 102, 104, 105.
}

Із звіту комісарів 1863 р. ми наводимо ось що: Д-р Дж.
Т. Ерледж, головний лікар шпиталю в північному Стафордшірі,
каже: «Як кляса, ганчарі, чоловіки й жінки... являють собою
фізично й морально вироджену людність. Вони звичайно низькі
на зріст, поганої статури й часто нездужають на викрив грудини.
Вони передчасно старіються й живуть недовго; флегматичні й
малокровні, вони виявляють слабість своєї статури нападами
впертої диспепсії, розладів діяльности печінки й нирок та ревматизму.
Але передусім піддаються вони грудним недугам: запаленню
легенів, сухотам, бронхітові й астмі. Одна форма цієї
останньої спеціяльно властива їм і відома під назвою ганчарської
астми, або ганчарських сухот. Золотуха, що захоплює залози,
кості й інші частини тіла, є недуга, на яку слабує більше ніж
дві третини ганчарів. Те, що виродження (degenerescence) людности
цієї округи не є ще значно більше, завдячує виключно
припливові нових елементів із суміжних селянських округ і
шлюбам із здоровшими расами». Пан Чарлз Пірсон, що недавно
ще перед тим був лікарем у тому самому шпиталі, пише
в одному листі до члена комісії Льонге, між іншим, таке: «Я можу
говорити лише на основі своїх особистих спостережень, а не з
статистичних даних, але я не вагаюся запевняти вас, що в мене
раз-у-раз скипало обурення, коли я дивився на цих бідних
дітей, що їхнє здоров’я кидано на поталу ненажерливости їхніх
батьків і працедавців». Він перелічує причини недуг серед ганчарів
і закінчує найголовнішою з них — «long hours» («довгий
робочий день»). Звіт комісії висловлює надію, що «мануфактура,
яка має таке видне становище в очах цілого світу, не плямуватиме
себе більш тим, що її величезні успіхи супроводяться психічним
виродженням, численними тілесними хоробами і передчасною
смертю робітників, що через працю і вправність їхню
досягнуто таких великих результатів».\footnote{
«Children’s Employment Commission, 1863», p. 24, 22, XI.
} Сказане тут про ганчарні
в Англії стосується також і до ганчарень у Шотляндії.\footnote{
Там же, стор. XLVII.
}

Мануфактура сірників починається з 1833 р., коли винайдено
спосіб прикріпляти фосфор до самого сірникового патичка. Від
\parbreak{}  %% абзац продовжується на наступній сторінці
