у зворотному відношенні до часу її діяння. Тому в певних межах
виграється на інтенсивності праці те, що втрачається на часі її
тривання. Але, щоб робітник дійсно таки витрачав більше робочої
сили, про це капітал дбає за допомогою методи оплати.158 У мануфактурах,
наприклад, у ганчарнях, де машини не відіграють
жодної ролі або відіграють незначну ролю, запровадження фабричного
закону дало разючий доказ того, що саме лише скорочення
робочого дня навдивовижу підносить реґулярність, одноманітність,
лад, безупинність та енерґію праці.159 Однак цей результат
у власне фабриці видавався сумнівним, бо залежність робітника
від безупинного та одноманітного руху машини давно створила
вже тут якнайсуворішу дисципліну. Тому, коли 1844 р.
почали обговорювати питання про скорочення робочого дня
нижче від 12 годин, то фабриканти майже одноголосно заявили,
що «їхні наглядачі по різних робітних приміщеннях пильнували,
щоб руки не гаяли часу», що «ступінь недріманости та уважности
робітників ледве чи можна підвищити» («the extent of vigilance
and attention on the part of the workmen») та, припускаючи, що всі
інші обставини, як, приміром, швидкість руху машин і т. ін.,
не змінюються, «було б безглуздям сподіватися в добре впоряджених
фабриках якогось значного результату від збільшення
уважности робітників і т. ін.»).160 Це твердження збито експериментами.
Пан Р. Ґарднер запровадив 20 квітня 1844 р. на двох
своїх великих фабриках у Престоні замість 12-годинного лише
11-годинний робочий день. Приблизно через рік виявився
той результат, що «за однакових витрат одержано однакову
кількість продукту, та що всі робітники заробили за 11 годин
стільки саме заробітної плати, скільки раніш за 12 годин». 161
Я тут лишаю осторонь експерименти в прядільних та чухральних
відділах, бо вони були зв’язані із збільшенням швидкости машин
(на 2\%). Навпаки, у ткальному відділі, де, крім того, ткано
дуже різні сорти легеньких фантастичних квітчастих виробів,
не сталося жодних змін в об’єктивних умовах продукції. Результат
був такий: «Від 6 січня до 20 квітня 1844 р. за дванадцятигодинного
робочого дня пересічна тижнева заробітна плата кожного
робітника становила 10 шилінґів 1\sfrac{1}{2} пенса, від 20 квітня до
29 червня 1844 р. за одинадцятигодинного робочого дня пересічна
тижнева заробітна плата — 10 шилінґів 3\sfrac{1}{2} пенса».162 Тут за
11 годин продукували більше, ніж раніш за 12 годин, виключно
в наслідок більшої й рівномірнішої витривалости робітників та

158 Особливо за допомогою відштучної плати, форми, що її розглядається
в шостому відділі книги.

159 Див. «Reports of Insp. of Fact. for 31st October 1865».

160 «Reports of Insp. of Fact. for 1844 and the quarter ending 30th
April 1845», p. 20, 21.

161 Там же, стор. 19. Через те, що відштучна плата лишилась та
сама, висота тижневої плати залежала від кількости продукту.

162 Там же, стор. 22.
