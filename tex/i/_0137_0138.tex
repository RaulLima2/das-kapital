\parcont{}  %% абзац починається на попередній сторінці
\index{i}{0137}  %% посилання на сторінку оригінального видання
є рішучий проґресист, проте він фабрикує чоботи не задля них
самих. Споживна вартість за товарової продукції взагалі не є
та річ, «qu’on aime pour lui-même»\footnote*{
— «яку люблять задля неї самої». \emph{Ред.}
}. Споживні вартості взагалі
продукується тут лише тому й остільки, що й оскільки вони є
матеріяльний субстрат, носії мінової вартости. І наш капіталіст
дбає про дві речі. Поперше, він хоче продукувати споживну
вартість, що має мінову вартість, предмет, призначений на продаж,
товар. А, подруге, він хоче продукувати товар, вартість
якого вища від суми вартости товарів, потрібних для його продукції,
тобто засобів продукції й робочої сили, на які він авансував
на товаровому ринку свої власні гроші. Він хоче продукувати
не тільки споживну вартість, але й товар, не тільки споживну
вартість, але й вартість, і не тільки вартість, а ще й додаткову
вартість.

А що тут мова йде про товарову продукцію, то ми на ділі,
очевидно, розглядали досі лише один бік процесу. Як сам товар
є єдність споживної вартости й вартости, так і процес продукції
товару мусить бути єдністю процесу праці й процесу творення
вартости.

Розгляньмо тепер процес продукції ще й як процес творення
вартости.

Ми знаємо, що вартість кожного товару визначається кількістю
праці, зматеріялізованої в його споживній вартості, робочим
часом, суспільно"=доконечним для його продукції. Це має силу
і для продукту, що його наш капіталіст дістав як результат процесу
праці. Отже, слід насамперед обчислити працю, упредметнену
в цьому продукті.

Нехай це буде, приміром, пряжа.

Для виготовлення пряжі потрібно було насамперед сировинного
матеріялу, приміром, 10 фунтів бавовни. Яка вартість бавовни,
цього тепер не доводиться досліджувати, бо капіталіст
купив її на ринку за її вартістю, приміром, за 10\shil{ шилінґів.} У ціні
бавовни потрібну для її продукції працю вже виражено як загальносуспільну
працю. Припустімо далі, що зужиткована на
перероблення бавовни кількість веретен, які для нас репрезентують
усі інші вжиті засоби праці, — має вартість у 2\shil{ шилінґи.}
Коли маса золота в 12\shil{ шилінґів} є продукт 24 робочих годин, або
2 робочих днів, то з цього насамперед випливає, що у пряжі упредметнено
2 робочі дні.

Та обставина, що бавовна змінила свою форму, а зужиткована
кількість веретен зовсім зникла, не повинна спантеличувати
нас. За загальним законом вартости, 10 фунтів пряжі, приміром,
є еквівалент 10 фунтів бавовни й чверти веретена, коли вартість
40 фунтів пряжі дорівнює вартості 40 фунтів бавовни плюс вартість
одного цілого веретена, тобто, коли потрібно однакового
робочого часу, щоб випродукувати одну й другу сторони цього
рівнання. У цьому випадку той самий робочий час репрезентується
\index{i}{0138}  %% посилання на сторінку оригінального видання
раз у споживній вартості пряжі, другий — у споживних
вартостях бавовни й веретена. Отже, для вартости однаковісінько,
чи виявляється вона у пряжі, веретені чи бавовні. Те, що
веретено й бавовна, замість спокійно лежати одне побіля одного,
у процесі прядіння увіходять у сполуку, яка змінює їхні споживні
форми, перетворює їх на пряжу, — це так само не впливає на
їхню вартість, як коли б їх були замінили через звичайний обмін
на еквівалентну кількість пряжі.

Потрібний для продукції бавовни робочий час є частина робочого
часу, потрібного на продукцію пряжі, що для неї бавовна
становить сировинний матеріял, а тому цей робочий час, потрібний
для продукції бавовни, міститься у пряжі. Так само стоїть
справа й щодо робочого часу, потрібного для продукції тієї кількости
веретен, без ужиткування або споживання якої не можна
бавовни перетворити на пряжу\footnote{
«Не лише та праця, що її прикладається безпосередньо до товарів,
впливає на вартість товарів, а й та, що міститься у знарядді, інструментах,
будинках, які допомагають цій праці» («Not only the labour applied
immediately to commodities affects their value, but the labour also which
is bestowed on the implements, tools, and buildings with which such
labour is assisted»). (\emph{Ricardo}: «The Principles of Political Economy», 3rd,
ed. London. 1821, p. 16).
}.

Отже, оскільки йдеться про вартість пряжі, тобто про потрібний
для її виготовлення робочий час, ми можемо розглядати різні
особливі, відокремлені один від одного в часі й просторі процеси
праці, які мусять відбутися для того, щоб випродукувати саму
бавовну й зужитковані веретена, а потім для того, щоб з бавовни
й веретен зробити пряжу, — як різні послідовні фази одного й
того самого процесу праці. Вся праця, що міститься у пряжі,
є минула праця. Та обставина, що робочий час, потрібний для
продукції елементів творення пряжі, минув давніше, тобто є
давноминулий час, а праця, безпосередньо витрачена на кінцевий
процес, на прядіння, навпаки, ближча до теперішнього часу,
тобто є просто минулий робочий час, не має ніякого значення.
Коли на будову будинку потрібна якась певна маса праці, приміром,
30 робочих днів, то загальна кількість робочого часу,
втіленого в будинку, ніяк не змінюється від того, що тридцятий
робочий день увійшов до продукції 29 днями пізніше, ніж перший
робочий день. І, таким чином, робочий час, що міститься в матеріялі
й засобах праці, ми можемо розглядати зовсім так, наче
його витрачено лише в давнішній стадії процесу прядіння, перед
тією працею, що її долучено наприкінці у формі прядіння.

Отже, вартості засобів продукції, бавовни й веретен, виражені
в ціні 12\shil{ шилінґів}, становлять складові частини вартости пряжі,
або вартости продукту.

Але тут треба виконати дві умови. Поперше, бавовна й веретена
мусили дійсно служити для продукції споживної вартости.
В нашому випадку з них мусила постати пряжа. Для вартости
байдуже, яка споживна вартість є її носій, але якась споживна
\parbreak{}  %% абзац продовжується на наступній сторінці
