\parcont{}  %% абзац починається на попередній сторінці
\index{i}{0621}  %% посилання на сторінку оригінального видання
закріпити за кожним котеджем сільського найманого робітника
також по 4 акри землі й забороняло йому приймати до свого
котеджу пожильців. Ще в 1627 р., за Якова І, Роджера Крокера
з Фронт-Міллу засуджено за те, що він у маєтку Фронт-Мілл
збудував котедж і не прирізав до нього 4 акрів землі; ще в 1638 р.,
за Карла І, призначено королівську комісію, яка мала примусити
до виконування старих законів, особливо закону про 4 акри
землі; ще Кромвел заборонив в околицях Лондону 4 милі довкола
міста будувати доми, не забезпечивши їх 4 акрами землі. Ще в
першій половині XVIIІ століття скаржилися, коли до котеджу
сільського робітника не додавали 1 або 2 акрів землі. А тепер
сільський робітник щасливий, коли біля його хати є садок або
коли недалеко від неї він може найняти клаптик землі. «Земельні
власники й фармери, — каже д-р Гентер, — у цій справі всі заодно.
Кілька акрів землі коло котеджа зробили б робітника
занадто вже незалежним».\footnote{
Dr. Hunter у «Public Health. Seventh Report 1864», London 1865,
p. 134. — «Кількість землі, установлену (старими законами), вважали б
сьогодні за надто велику для робітників і радше за придатну для того;
щоб перетворити їх на дрібних фармерів» («The quantity оf land assigned
would now be judged too great for labourers, and rather as likely to
convert them into small farmers»). (George Roberts: «The Social History
of the People of the Southern Counties of England in past centuries», London
1856, p. 184, 185).
}

Реформація і колосальний крадіж церковних маєтків, що
супроводив її, дали в XVI столітті новий страшенної сили поштовх
процесові насильницької експропріяції народньої маси.
Католицька церква була на початок реформації февдальною
власницею великої частини англійської землі. Знищення манастирів
тощо, перетворило їхніх мешканців на пролетаріят.
Самі церковні маєтки здебільшого пороздаровувано хижим королівським
фаворитам або порозпродувано за безцінь спекулянтамфармерам
та міщанам, які виганяли давніх спадкових орендарів
цілими масами і з’єднували докупи їхні господарства. Гарантоване
законом право власности збіднілих рільників на частину
церковної десятини тишком у них одібрано.\footnote{
«Право бідних на частину десятини встановлено старовинними
статутами» («The right of the poor to share in the tithe, is established by
the tenure of ancient statutes»). (J. D. Tucket: «А History of the Past
and Present State of the Labouring Population», London 1846, vol. II,
p. 804, 805).
} «Pauper ubique
jacet»\footnote*{
Бідні повсюди. \emph{Ред.}
} — вигукнула королева Єлисавета після однієї подорожі
своєї по Англії. На 43 році її королювання уряд був, нарешті,
змушений офіціально визнати павперизм, заводячи податок на
користь бідним. «Автори цього закону соромилися пояснити
його мотиви, і тим то вони, всупереч усім звичаям, випустили
його в світ без будь-якого preamble (вступного пояснення).\footnote{
William Cobbett: «А History of the Protestant Reformation», § 471.
}
Акт 16, Карла І, 4, надав цьому законові постійної сили, і тільки
\parbreak{}  %% абзац продовжується на наступній сторінці
