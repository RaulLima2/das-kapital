Тим то за всякої металевої циркуляції наявні вже назви вагового
маштабу становлять і первісні назви грошового маштабу,
або маштабу цін.

Як міра вартостей і маштаб цін гроші виконують дві цілком
різні функції. Мірою вартостей вони є як суспільне втілення
людської праці, маштабом цін — як фіксована вага металю.
Як міра вартости вони служать на те, щоб перетворювати вартості
дуже різноманітних товарів на ціни, на уявлювані кількості
золота; як маштаб цін вони вимірюють ці кількості золота. Мірою
вартостей товари вимірюються як вартості, навпаки, маштаб
цін вимірює різні кількості золота якоюсь його даною кількістю,
а не вартість якоїсь кількости золота вагою іншої кількости.
Для маштабу цін якась певна вага золота мусить бути фіксована
як одиниця міри. Тут, як і за всіх інших визначень міри однойменних
величин, вирішує справу стійкість одиниці міри. Таким
чином маштаб цін виконує свою функцію то краще, що незмінніше
та сама кількість золота служить за одиницю міри. За міру
вартости золото може служити лише тому, що воно самé є продукт
праці, отже, в можливості змінна вартість.\footnote{
Примітка до другого видання. У творах англійських авторів
панує неймовірна плутанина понять міри вартостей (measure of value)
і маштабу цін (standard of value). Вони постійно плутають ці функції,
а через це і їхні назви.
}

Насамперед ясно, що зміна вартости золота ніяким чином не
заважає його функції як маштабу цін. Хоч як змінюватиметься
вартість золота, різні кількості його лишаються завжди в тому
самому вартостевому відношенні поміж собою. Коли б вартість
золота зменшилась на 1000\%, то 12 унцій золота, як і раніш, мали б
у 12 разів більше вартости, ніж одна унція, а в цінах ідеться
лише про відношення різних кількостей золота поміж собою.
А що, з другого боку, вага однієї унції золота з пониженням
або підвищенням її вартости лишається цілком незмінна, то так
само мало змінюється й вага аліквотних частин її; таким чином
золото як сталий маштаб цін завжди робить ті самі послуги, хоч
як змінюється його вартість.

Зміна вартости золота не заважає і його функції як міри вартости.
Ця зміна стосується всіх товарів одночасно, отже, за інших
незмінних умов, лишає їхні взаємні відносні вартості незмінними,
хоч ці останні визначатимуться тепер всі у вищих або
нижчих золотих цінах, аніж раніш.\footnote*{
До цього Маркс дає у французькому виданні «Капіталу» таку примітку:
«Вартість грошей може безупинно змінюватись і все ж таки вони
можуть служити за міру вартости так само добре, як тоді, коли б їхня
вартість лишалась цілком сталою». (Bailey: «Money and its Vicissitudes»,
London 1837, p. 11). Ред.
}

Так за виразу вартости якогось товару в споживній вартості
якогось іншого товару, як і за цінування товарів у золоті при-

was introduced at a later period into a coinage adapted only to silver, an
ounce of gold cannot be coined into an adequate number of pieces»). (Maclaren:
«History of the Currency», London 1858, p. 16).