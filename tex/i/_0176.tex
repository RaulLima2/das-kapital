\parcont{}  %% абзац починається на попередній сторінці
\index{i}{0176}  %% посилання на сторінку оригінального видання
Тому робочий день можна визначити, але сам по собі він є величина
невизначена.\footnote{
«Робочий день є величина невизначена, він може бути довгий або
короткий» («А day’s labour is vague, it may be long or short»). («An
Essay on Trade and Commerce, containing Observations on Taxation etc.»,
London 1770, p. 73).
}

Хоч робочий день є не стала, а змінна величина, однак, з
другого боку, він може змінятися лише в певних межах. Але мінімальні
межі його визначити не можна. Певна річ, коли ми припустимо,
що лінія здовження \emph{bc}, або додаткова праця, дорівнює
нулеві, то тоді матимемо мінімальну межу, а саме ту частину дня,
яку робітник доконечно мусить працювати, щоб утримати себе
самого. Але на базі капіталістичного способу продукції доконечна
праця завжди може становити лише частину його робочого дня,
отже, робочий день ніколи не може бути скорочений до цього
мінімуму. Навпаки, робочий день має певну максимальну межу.
Його не можна здовжити поза певну межу. Цю максимальну
межу визначається подвійно. Поперше, фізичними межами робочої
сили. Людина може протягом природного дня в 24 години
витратити тільки певну кількість життєвої сили. Приміром, кінь
може працювати з дня на день лише 8 годин. Протягом однієї
частини дня сила мусить відпочивати, спати, протягом другої
частини людині треба задовольняти інші фізичні потреби: живитися,
підтримувати чистоту, одягатися й т. ін. Крім цих суто
фізичних меж здовження робочого дня натрапляє на моральні
межі. Робітник потребує часу, щоб задовольняти інтелектуальні
й соціяльні потреби, що їхній обсяг і кількість визначаються
загальним станом культури. Отже, зміни, що їх зазнає робочий
день, рухаються в цих фізичних і соціяльних межах. Але обидві
ці межі дуже елястичної природи й дають найбільший простір
для варіяцій. Так ми находимо робочі дні у 8, 10, 12, 14, 16, 18
годин, отже, дні дуже різноманітної довжини.

Капіталіст купив робочу силу за її денною вартістю. Йому
належить її споживна вартість протягом одного робочого дня.
Отже, він здобув собі право примушувати робітника працювати
для нього протягом одного дня. Але що таке робочий день?\footnote{
Питання це безконечно важливіше, ніж знамените питання сера
Роберта Піла, звернене до бермінґемської торговельної палати: «What
is a pound?»\footnote*{
Що таке фунт стерлінґів? \emph{Ред.}
} — питання, яке він міг поставити лише через те, що для
нього була так само неясна природа грошей, як і для «little shilling men»\footnote*{
— людців від шилінґів. \emph{Ред.}
}
із Бермінґему.
}
У всякому разі щось менше за природний день життя. На
скільки? У капіталіста є свій власний погляд на цю ultima Thule,\footnote*{
— останню межу. \emph{Ред.}
}
доконечну межу робочого дня. Як капіталіст він є лише персоніфікований
капітал. Його душа — душа капіталу. Але капітал
має одним-однісіньке життєве прагнення — прагнення самозростати,
утворювати додаткову вартість, вбирати своєю сталою
\parbreak{}  %% абзац продовжується на наступній сторінці
