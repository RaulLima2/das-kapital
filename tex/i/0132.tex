служать лише як сховища предметів праці й що їхню сукупність
взагалі можна назвати судинною системою продукції, як от
труби, бочки, коші, жбани тощо. Лише в хемічній фабрикації
вони відіграють значну ролю.\footnoteA{
Примітка до другого видання. Хоч і як мало дотеперішня історична
наука знає розвиток матеріяльної продукції, основу всього суспільного
життя, отже, і всієї дійсної історії, проте принаймні передісторичні часи
поділено — на основі природничо-наукових, а не так званих історичних,
дослідів — за матеріялом знарядь праці і зброєю на кам’яний вік,
бронзовий вік і залізний вік.
}

Крім тих речей, що упосереднюють діяння праці на предмет
праці і тому так або інакше служать за провідників діяльности,
процес праці залічує до своїх засобів у ширшому розумінні всі
речові умови, які взагалі потрібні, щоб відбувся процес. Вони
не входять у нього безпосередньо, але без них він або зовсім
не може відбуватися або відбувається лише в недосконалій формі.
За загальний засіб праці цього роду є знов таки сама земля, бо
вона дає робітникові locus standi,\footnote*{
— місце, на якому він стоїть. Ред.
} а його процесові — поле
діяльности (field of employment). За засоби праці цього роду,
але такі, що вже оброблялись працею, приміром, є робочі будинки,
канали, шляхи й т. ін.

Отже, у процесі праці діяльність людини спричинює за допомогою
засобів праці зміну в предметі праці, яку вона собі заздалегідь
поставила за мету. Процес згасає в продукті. Продукт його є
споживна вартість, речовина природи, через зміну форми пристосована
до людських потреб. Праця сполучилась із своїм предметом
праці. Вона упредметнена, а предмет праці оброблено. Те, що
на боці робітника з’являлось у формі неспокою (Unruhe), з’являється
тепер на боці продукту як властивість спокою (ruhende
Eigenschaft), у формі буття. Робітник пряв, і продукт є прядиво.

Коли розглядати цілий процес з погляду його результату,
продукту, то засоби праці й предмет праці, обидва, з’являються
як засоби продукції,\footnote{
Видається парадоксом, коли, приміром, рибу, якої ще не впіймано,
називати засобом продукції для рибальства. Але досі ще не винайдено
вмілости ловити рибу у водах, де її немає.
} а сама праця — як продуктивна праця.\footnote{
Цього визначення продуктивної праці, що випливає з погляду простого
процесу праці, зовсім недосить для капіталістичного процесу продукції.
}

Коли якась споживна вартість виходить із процесу праці як
продукт, то інші споживні вартості, продукти попередніх процесів
праці, входять у нього як засоби продукції. Та сама споживна
вартість, що є продукт одного процесу праці, становить засіб
продукції для іншого процесу праці. Тому продукти є не лише
результат, але разом з тим і умова процесу праці.

За винятком добувальної промисловости, що знаходить свій
предмет праці в самій природі, як от гірництво, полювання,
рибальство й т. ін. (рільництво лише остільки, оскільки воно в
першу чергу обробляє саму цілину), всі інші галузі промисловости
обробляють предмет, що є сировинний матеріял, тобто пред-