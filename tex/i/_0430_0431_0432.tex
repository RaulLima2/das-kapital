\parcont{}  %% абзац починається на попередній сторінці
\index{i}{0430}  %% посилання на сторінку оригінального видання
існування, отже, родючість ґрунту, багаті рибою води й т. ін.,
і природне багатство на засоби праці, як от водоспади, судноплавні
річки, дерево, металі, вугілля й т. ін. На початках культури
має переважне значення перша форма природного багатства,
на вищих ступенях розвитку — друга форма. Порівняйте,
наприклад, Англію з Індією, або — в античному світі — Атени
й Корінт з країнами на узбережжі Чорного моря.

Що менше число природних потреб, які абсолютно треба
задовольняти, і що більша природна родючість ґрунту та сприятливість
підсоння, то менший робочий час, доконечний для утримання
й репродукції продуцента. Отже, то більший може бути й
надлишок його праці на інших супроти його праці на самого себе.
Так, уже Діодор зауважує про давніх єгиптян: «Просто неймовірно,
як мало праці й витрат коштує їм виховання їхніх дітей.
Вони варять для них першу-ліпшу просту страву; дають їм
їсти й долішню частину папіруса, яку можна присмажити на
вогні, та коріння й стебла болотяних рослин, почасти сирі, почасти
варені й печені. Діти здебільша ходять без взуття й одягу,
бо повітря там дуже м’яке. Тому дитина коштує своїм батькам,
доки виросте, в цілому не більше, як двадцять драхм. Цим головне
й можна пояснити, що в Єгипті така численна людність,
у наслідок чого й можна було збудувати там такі великі споруди».\footnote{
\emph{Diodorus Siculus}: «Bibliotheca historica», lib. I, c. 80.
}
Однак великі споруди давнього Єгипту своє існування
завдячують менше чисельності його людности, ніж тій обставині,
що відносно великою частиною людности можна було порядкувати
для цієї справи. Як індивідуальний робітник може давати
тим більше додаткової праці, чим менший його доконечний робочий
час, цілком так само чим менша частина робітничої людности,
потрібна для продукції доконечних засобів існування,
тим більша та її частина, якою можна порядкувати для іншої
справи.

Скоро капіталістичну продукцію дано як передумову, то, за
інших незмінних обставин і за даної довжини робочого дня, величина
додаткової праці буде змінюватися залежно від природних
умов праці, особливо ж залежно від родючости ґрунту. Але звідси
ні в якому разі не випливає протилежне, а саме те, що найродючіший
ґрунт є найвідповідніший для зростання капіталістичного
способу продукції. Цей спосіб припускає панування людини над
природою. Занадто марнотратна природа «водить людину, як
дитину, на мотузочку». Вона не робить власний розвиток людини
природною доконечністю.\footnote{
«Перше (природні багатства), будучи найсприятливішим і найкориснішим,
робить народ безтурботним, чванливим та схильним до всяких
надмірностей, тимчасом як друге приневолює до дбайливости, науки,
мистецтва та розумної політики» («The first (natural wealth), as it
is most noble and advantageous, so doth it make the people careless, proud,
and given to all excesses; whereas the second enforceth vigilancy, literature,
arts and policy»). («England’s Treasure by Foreign Trade. Or the Balance
of our Foreign Trade is the Rule of our Treasure. Written by Thomas Mun,
of London, Merchant, and now published for the common good by his son
John Mun», London 1669, p. 181, 182). «Я також не можу уявити собі
більшого прокляття для народу, як бути закинутим на клапоть землі,
де сама природа рясно продукує засоби існування, а підсоння вимагає
або дозволяє лише мало турбуватися про одяг і житло\dots{} Можлива й протилежна
крайність. Ґрунт, що з нього навіть працею не можна вирвати
ніякого продукту, так само недобрий, як і той ґрунт, що рясно родить
без ніякої праці». («Nor can I conceive a greater curse upon a body of people,
than to be thrown upon a spot of land, where the productions for subsistence
and food were, in great measure, spontaneous, and the climate required
or admitted little care for raiment and covering\dots{} there may be an extreme
on the other side. A soil incapable of produce by labour is quite as bad as
a soil that produces plentifully without any labour»). («An Inquiry into
the Present High Price of Provisions», London 1767, p. 10).
} Не тропічне підсоння з його буйною
\index{i}{0431}  %% посилання на сторінку оригінального видання
рослинністю, а помірна смуга є батьківщина капіталу. Не абсолютна
родючість ґрунту, а його диференційованість, різноманітність
його природних продуктів становить природну основу
суспільного поділу праці та через зміну природних умов, серед
яких живе людина, спонукає її до урізноманітнення її власних
потреб, здібностей, засобів праці та способів праці. Доконечність
суспільно контролювати якусь силу природи, економно користуватися
нею, присвоювати її собі або приборкувати у великому
маштабі за допомогою споруд, зроблених людською рукою, —
ось що відіграє найвирішальнішу ролю в історії промисловости.
Наприклад, уреґулювання води в Єгипті,\footnote{
Доконечність обчислювати періоди розливу Ніла створила єгипетську
астрономію, а з нею й панування касти жерців як керівників
рільництва. «Сонцестояння — це той момент року, коли починається
розлив Ніла, і єгиптяни мусили стежити за цим сонцестоянням з особливою
увагою. Для них важливо було встановити цей тропічний рік для
того, щоб реґулювати свої рільничі роботи. Тому вони мусили шукати
на небі виразного знаку його повороту» («Le solstice est le moment de
l’année où commence la crue du Nil, et celui que les Egyptiens ont dû
observer avec le plus d’attention\dots{} C’était cette année tropique qu’il leur
importait de marquer pour se diriger dans leurs opérations agricoles. Ils
durent donc chercher dans le ciel un signe apparent de son retour»). (\emph{Cuvier}:
«Discours sur les révolutions du globe». Ed. Hoefer. Paris 1863, p. 141).
} Льомбардії, Голляндії
й т. ін. Або в Індії, Персії й т. ін., де зрошування штучними
каналами постачає ґрунтові не тільки конче потрібну воду, але
разом з її намулом приставляє з гір мінеральне добриво. Каналізація
— ось у чому була таємниця розцвіту промисловости Еспанії
та Сіцілії під арабським пануванням.\footnote{
Однією з матеріяльних основ державної влади над малими, незв'язаними
між собою продукційними організмами Індії, було реґулювання
водопостачання. Мохаммеданські володарі Індії розуміли це краще,
ніж їхні англійські нащадки. Ми нагадаємо лише про голод 1866 р., що
коштував життя більш ніж мільйонові індусів в окрузі Оріссі Бенґальського
президентства.
}

Сприятливі природні умови завжди дають лише можливість
додаткової праці, алеж ніколи не дійсність додаткової праці,
отже, і додаткової вартости, або додаткового продукту. Різні
природні умови праці призводять до того, що та сама кількість
праці по різних країнах задовольняє різні маси потреб,\footnote{
«Немає двох країн, що давали б однакову кількість доконечних
засобів існування в однаковій достатності та при однакових затратах
праці. Людські потреби зростають або зменшуються залежно від суворости
або м’якости підсоння, що в ньому люди живуть; отже, розміри, що в
них мешканці різних країн мусять продукувати, не можуть бути однакові,
і не можна визначити ступінь цієї неоднаковости інакше, як тільки у
зв’язку з ступенем теплоти або холоду; звідси можна зробити й той загальний
висновок, що кількість праці, потрібної для певного числа людности,
найбільша в холодному підсонні, найменша в теплому. Бо в першому
не тільки люди більше потребують одягу, але й земля потребує
більше праці на оброблення, аніж у другому». («There are no two countries
which furnish an equal number of the necessaries of life in equal plenty,
and with the same quantity of labour. Men’s wants increase or diminish
with the severity or teinperateness of the climate they live in; consequently
the proportion of trade which the inhabitants of different countries are
obliged to carry on through necessity, cannot be the same, nor is it practicable
to ascertain the degree of variation farther than by the Degrees of
Heat and Cold; from whence one may make this general conclusion, that
the quantity of labour required for a certain number of people is greatest
in cold climates, and least in hot ones; for in the former men not only want
more clothes, but the earth more cultivating than in the latter»). («An
Essay on the Governing Causes of the Natural Rate of Interest», London
1750, p. 60). Автор цього епохального анонімного твору J. Massey.
Юм запозичив із нього свою теорію процента.
} отже,
\index{i}{0432}  %% посилання на сторінку оригінального видання
до того, що, за інших аналогічних обставин, доконечний робочий
час є різний. На додаткову працю вони впливають лише як
природна межа, тобто визначають той пункт, що від нього може
початися праця на інших. Ця природна межа відсувається назад
у тій самій мірі, в який промисловість проґресує. Серед західньоевропейського
суспільства, де робітник лише додатковою працею
купує дозвіл працювати для свого власного існування, легко
постає ілюзія, що давати додатковий продукт є природжена властивість
людської праці.\footnote{
«Всяка праця мусить» (це, здається, також належить до прав і
обов’язків громадянина) «давати надлишок» («Chaque travail doit
laisser un excédant»). (\emph{Proudhon}).
} Але візьмімо, наприклад, жителів
східніх островів азійського архіпелагу, де саґо дико росте в лісі.
«Коли місцеві жителі, просвердливши діру в дереві, переконуються,
що стрижень уже достиг, вони зрубують дерево, ділять
його на декілька кусків, видирають стрижень, змішують його
з водою і, відцідивши воду, дістають цілком придатне до вжитку
саґове борошно. Одно дерево дає звичайно 300 фунтів, а може
дати 500--600 фунтів. Отже, там ідуть у ліс і рубають собі хліб,
як у нас рубають дерево на паливо.\footnote{
\emph{F. Shouw}: «Die Erde, die Pflanze und der Mensch». 2. Auflage. Leipzig
1854, S. 148.
} Припустімо, що такому
східньоазійському рільникові потрібно 12 робочих годин на
тиждень для задоволення всіх його потреб. Сприятлива природа
безпосередньо дає йому багато вільного часу. Для того, щоб він
цей час продуктивно зуживав на самого себе, потрібен цілий ряд
історичних умов, а для того, щоб він витрачав його як додаткову
працю на чужих осіб, потрібен зовнішній примус. Коли б там
було заведено капіталістичну продукцію, наш молодець мусив би
працювати, може, 6 днів на тиждень, щоб присвоїти собі самому
продукт одного робочого дня. Сприятливість природи не пояснює,
\parbreak{}  %% абзац продовжується на наступній сторінці
