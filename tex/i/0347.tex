дуктивних машин, що її безнастанно пускає в рух центральна
сила (перший мотор)», а з другого боку, «як велетенський автомат,
складений з безлічі механічних та самосвідомих органів,
які оперують у згоді й без перерви, щоб продукувати той самий
предмет, так що всі ці органи підпорядковані одній рушійній
силі, яка сама собою рухається». Ці обидва визначення зовсім
не є ідентичні. В першому комбінований колективний робітник
або суспільне робоче тіло з’являється як домінантний суб’єкт,
а механічний автомат — як об’єкт; у другому сам автомат є
суб’єкт, а робітники як свідомі органи лише додані до його
несвідомих органів і разом з цими останніми підпорядковані
центральній рушійній силі. Перше визначення можна прикласти
до кожного можливого вживання машин у великому маштабі,
друге характеризує капіталістичне вживання машин, а тому й
сучасну фабричну систему. Тим то Юр любить також змальовувати
центральну машину, що від неї виходить рух, не тільки як
автомата, але і як автократа. «По цих величезних майстернях
благодатна сила пари збирає довкола себе міріяди своїх підданців
».\footnote{
«Ure: «Philosophy of Manufacture», р. 18.
}

Разом із робочим знаряддям і віртуозність керувати ним
переходить від робітника до машини. Видатність знаряддя звільняється
від особистих рамок людської робочої сили. Тим самим
усунуто (aufgehoben) ту технічну основу, на якій ґрунтується
поділ праці в мануфактурі. Тому замість тієї ієрархії спеціялізованих
робітників, що характеризує мануфактуру, на автоматичній
фабриці виступає тенденція зрівняти, або знівелювати ті
праці, які мають виконувати помічники машин,\footnote{
Там же, стор. 20. Порівн. K. Marx: «Misère de la Philosophie»,
Paris 1847. p. 140, 141. (K. Маркс: «Злиденність філософії», Партвидав
«Пролетар», 1932 р., стор. 126, 127).
} замість штучно
витворених ріжниць між частинними робітниками виступають
переважно природні ріжниці віку та статі.

Оскільки поділ праці відроджується на автоматичній фабриці,
він є насамперед розподіл робітників між різними спеціялізованими
машинами та розподіл мас робітників, які однак не являють
собою організованих груп, між різними відділами фабрики, де
вони працюють коло однорідних виконавчих машин, що стоять
рядом одна побіч однієї; отжеж, серед них існує тільки проста
кооперація. Розчленовану групу мануфактури тут замінено
зв’язком головного робітника з небагатьма помічниками. Посутня
ріжниця є поміж робітниками, які справді працюють коло виконавчих
машин (сюди належать декілька робітників, що доглядають
рушійної машини та опалюють її), і простими підручними
(майже виключно діти) цих машинових робітників. До підручних
у більшій або меншій мірі залічується всіх «feeders» (які
лише подають машинам матеріял праці). Побіч цих головних кляс
виступає чисельно незначний персонал, що доглядає всіх машин