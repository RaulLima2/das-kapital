ced», «the undersellers»). «Fullpriced» так виказують на своїх
конкурентів перед парляментською слідчою комісією: «Вони
існують лише тим, що, поперше, дурять публіку, фальсифікуючи
товар, а, подруге, витискують із своїх людей 18 годин праці
за плату дванадцятигодинної праці... Неоплачена праця (the
unpaid labour) робітника — ось засіб, яким вони борються в
конкуренції... Конкуренція між хазяїнами-пекарями є причина
труднощів усунути нічну працю. Underseller, той,* що продає
свій хліб нижче ціни витрат, яка змінюється разом із ціною
борошна, поповнює свої втрати тим, що витискує із своїх людей
більше праці. Коли я від своїх людей витискую тільки 12 годин
праці, а мій сусіда, навпаки, 18 або 20, то він мусить мене побити
в продажній ціні. Коли б робітники могли добитися плати за
наднормовий час, то тоді б цьому маневрові скоро прийшов би
кінець... Значне число тих, що працюють в underseller’ів — то
чужинці, підлітки та інші, що примушені задовольнятися майже
всякою заробітною платою, яку можуть дістати».\footnote{
«Report etc. relative to the Grievances complained of by the
journeymen bakers», London 1862, p. LII і там же, Evidence, n. 479,
359, 27). А проте й fullpriced, як ми згадували вище та як це визнає сам
оборонець їхній, Беннет, примушують своїх людей починати працю об
11 годині вечора або й раніш та часто здовжують її до 7 години наступ
ного вечора». (Там же, стор. 22).
}

Ця єреміяда ще й тим цікава, що вона показує, як у мозку
капіталіста відбивається лише зовнішня видимість продукційних
відносин. Капіталіст не знає того, що й нормальна ціна праці
включає певну кількість неоплаченої праці та що саме ця неоплачена
праця є джерело його зиску. Категорії додаткового робочого
часу для нього взагалі не існує, бо цей час міститься в нормальному
робочому дні, і він вважає, що оплачує цей день у поденній
заробітній платі. Щоправда, він визнає існування наднормового
часу, здовження робочого дня поза межу, що відповідає звичайній
ціні праці. Щодо свого конкурента, underseller’а, який продає
нижче проти нормальної ціни, то він обстоює навіть, щоб той
вище оплачував (extra pay) за цей наднормовий час. Знову ж
таки він не знає, що ця вища плата так само включає неоплачену
працю, як і ціна звичайної робочої години. Наприклад,
ціна однієї години дванадцятигодинного робочого дня є 3 пенси,
тобто вартість, спродукована за половину робочої години, тимчасом
як ціна наднормової робочої години є 4 пенси, тобто вартість,
спродукована за \sfrac{2}{3} робочої години. В першому випадку капіталіст
присвоює собі безплатно половину робочої години, у
другому — третину.

Розділ дев’ятнадцятий
Відштучна плата

Відштучна заробітна плата є не що інше, як перетворена
форма почасової плати, так само як почасова плата є перетворена
форма вартости або ціни робочої сили.