\setlength{\tabcolsep}{4.5pt}
\index{i}{0605}  %% посилання на сторінку оригінального видання
\begin{table}
\begin{flushright}
  \emph{Таблиця D}
\end{flushright}

\caption*{Доходи, що підлягають оподаткуванню, у фунтах стерлінґів}
\footnotesize
  \noindent\begin{tabularx}{\textwidth}{Xcccccc}
  \toprule
    & 1860 & 1861 & 1862 & 1863 & 1864 & 1865 \\
  \cmidrule(rl){2-7}

  \makecell{Рубрика А} \\

  \makehangcell{Земельна рента} & 13.893.829 & 13.003.554 & 13.398.938 & 13.494.091 & 13.470.700 & 13.801.616 \\

  \addlinespace
  \makecell{Рубрика В} \\

  \makehangcell{Зиски фармерів} 
    & \phantom{0}2.765.387 & \phantom{0}2.773.644 & \phantom{0}2.937.899 
    & \phantom{0}2.938.823 & \phantom{0}2.930.874 & \phantom{0}2.946.072 \\

  \addlinespace
  \makecell{Рубрика D} \\

  \makehangcell{Промисловий і ін\-ший зиск\dotfill{}}
    & \phantom{0}4.891.652 & \phantom{0}5.836.203 & \phantom{0}4.858.800 
    & \phantom{0}4.846.497 & \phantom{0}4.546.147 & \phantom{0}4.850.199 \\

  \addlinespace
  \makehangcell{Сума всіх рубрик від А до~Е\dotfill{}} & 22.962.885 & 22.998.394 & 23.597.574 & 23.658.631 & 23.236.298 &    23.930.340\hang{l}{\footnotemark{}}

  \end{tabularx}

\end{table}
\footnotetext{«Tenth Report of the Commissioners of Inland Revenue», London
1866.}
\setlength{\tabcolsep}{\tabcolsepdef}

Під рубрикою D збільшення прибутку від 1853 до 1864 р.
становило пересічно лише 0,93\%, тимчасом як у Великобрітанії
воно за той самий період становило 4,58\%. Дальша таблиця показує
розподіл зисків (за винятком зисків фармерів) у 1864 і
1865 рр.

\begin{table}

  \begin{flushright}
    \emph{Таблиця Е}
  \end{flushright}

  \caption*{Рубрика D. Доходи, що складаються з різних зисків (понад 60 фунтів
  стерлінґів) в Ірландії}
  
  \small

  \settowidth{\myheight}{\small\tablefont 4.368.610, розподілений}
  \noindent\begin{tabularx}{\textwidth}{X>{\hangindentdef}p{\myheight}>{\hangindentdef}p{\myheight}}

  \toprule
    & \makecell{1864 р.} & \makecell{1865 р.} \\
    & \makecell{Фунтів стерлінґів} & \makecell{Фунтів стерлінґів} \\
    \midrule
  
  \makehangcell{Загальний річний дохід\dotfill{}} &
  4.368.610, розподілений між особами 17.467 &
  4.669.979, розподілений між 18.081 особою \\

  \makehangcell{Річний дохід понад 60 ф. ст. і нижче за 100 ф. ст\dotfill{}} &
  \samewidth{0.}{~}238.626, розподілений між 5.015 особами &
  \samewidth{0.}{~}222.575, розподілений між 4.703 особами \\

  \makehangcell{Із загального річного доходу\dotfill{}} &
  1.979.066, розподілений між 11.321 особами &
  2.028.471, розподілений між 12.184 особами \\

  \makehangcell{Решта загального річного доходу\dotfill{}} &
  2.150.818, розподілений між 1.131 особою &
  2.418.933, розподілений між 1.194 особами \\

  \multicolumn{1}{r}{\ldelim\{{10}{*}[З нього:]} &
  1.083.906, розподілений між 910 особами &
  1.097.937, розподілений між 1.044 особами \\

  &
  1.066.912, розподілений між 121 особою &
  1.320.996, розподілений між 186 особами \\

  &
  \samewidth{0.}{~}430.535, розподілений між 105 особами &
  \samewidth{0.}{~}584.458, розподілений між 122 особами \\

  &
  \samewidth{0.}{~}646.377, розподілений між 26 особами &
  \samewidth{0.}{~}736.448, розподілений між 28 особами \\

  &
  \samewidth{0.}{~}262.610, розподілений між 3 особами &
  \samewidth{0.}{~}274.528, розподілений між 3 особами\footnotemark{} \\

  \end{tabularx}
\end{table}
\footnotetext{Загальний річний дохід під рубрикою D відхиляється тут від
  чисел попередньої таблиці, бо закон допускає деякі відрахування.}
