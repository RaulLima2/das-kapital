\index{i}{0605}  %% посилання на сторінку оригінального видання
\begin{table}
\begin{flushright}
  \emph{Таблиця D}
\end{flushright}

\caption*{Доходи, що підлягають оподаткуванню, у фунтах стерлінґів}
\footnotesize
  \noindent\begin{tabularx}{\textwidth}{p{2cm} X X X X X X}

  \toprule
& 1860 & 1861 & 1862 & 1863 & 1864 & 1865 \\
\cmidrule{2-7}
\mbox{Рубрика А}
\mbox{Земельна рента} & 13.893.829 & 13.003.554 & 13.398.938 & 13.494.091 & 13.470.700
& 13.801.616 \\
Рубрика В
\mbox{Зиски фармерів} & 2.765.387 & 2.773.644 & 2.937.899 & 2.938.823 & 2.930.874 & 2.946.072 \\
Рубрика D
\mbox{Промисловий} і інший зиск & 4.891.652 & 5.836.203 & 4.858.800 & 4.846.497 & 4.546.147 & 4.850.199 \\
\mbox{Сума всіх рубрик}
від А до Е & 22.962.885 & 22.998.394 & 23.597.574 & 23.658.631 & 23.236.298 &    23.930.340\footnotemark{} % ця мітка у заголовку
\\
 % текст примітки прямо під заголовком

  \end{tabularx}

\end{table}
\footnotetext{«Tenth Report of the Commissioners of Inland Revenue», London
1866.}

Під рубрикою D збільшення прибутку від 1853 до 1864 р.
становило пересічно лише 0,93\%, тимчасом як у Великобрітанії
воно за той самий період становило 4,58\%. Дальша таблиця показує
розподіл зисків (за винятком зисків фармерів) у 1864 і
1865 рр.

\begin{table}
  \begin{flushright}
    \emph{Таблиця Е}
  \end{flushright}

\caption*{Рубрика D. Доходи, що складаються з різних зисків (понад 60 фунтів
  стерлінґів) в Ірландії}
\small

  \noindent\begin{tabularx}{\textwidth}{X X X}
\toprule
    & 1864 р. & 1865 р. \\
\cmidrule{2-3}
    & Фунтів стерлінґів &  Фунтів стерлінґів \\
  Загальний річний дохід &
  4.368.610, розподілений між особами 17.467 &
  4.669.979, розподілений між 18.081 особою \\

  Річний дохід понад 60 ф. ст. і нижче за 100 ф. ст &
  238.626, розподілений між 5.015 особами &
  222.575, розподілений між 4.703 особами \\

  Із загального річного доходу &
  1.979.066, розподілений між 11.321 особами &
  2.028.471, розподілений між 12.184 особами \\

  Решта загального річного доходу &
  2.150.818, розподілений між 1.131 особою &
  2.418.933, розподілений між 1.194 особами \\

  &
  1.083.906, розподілений між 910 особами &
  1.097.937, розподілений між 1.044 особами \\

  &
  1.066.912, розподілений між 121 особою &
  1.320.996, розподілений між 186 особами \\
  % TODO: потрібно застосувати пакет multirow, і збільшити фігурну дужку
  З нього: \Bigg\{ &
  430.535, розподілений між 105 особами &
  584.458, розподілений між 122 особами \\

  &
  646.377, розподілений між 26 особами &
  736.448, розподілений між 28 особами \\

  &
  262.610, розподілений між 3 особами &
  274.528, розподілений між 3 особами\footnotemark{}
  \\

  \end{tabularx}
\end{table}
\footnotetext{Загальний річний дохід під рубрикою D відхиляється тут від
  чисел попередньої таблиці, бо закон допускає деякі відрахування.}
