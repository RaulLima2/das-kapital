\setlength{\tabcolsep}{4.5pt}
\index{i}{0605}  %% посилання на сторінку оригінального видання
\begin{table}
\begin{flushright}
  \emph{Таблиця D}
\end{flushright}

\caption*{Доходи, що підлягають оподаткуванню, у фунтах стерлінґів}
\footnotesize
  \noindent\begin{tabularx}{\textwidth}{Xcccccc}
  \toprule
    & 1860 & 1861 & 1862 & 1863 & 1864 & 1865 \\
  \cmidrule(rl){2-7}

  \makecell{Рубрика А} \\

  \makehangcell{Земельна рента} & \num{13.893.829} & \num{13.003.554} & \num{13.398.938} & \num{13.494.091} & \num{13.470.700} & \num{13.801.616} \\

  \addlinespace
  \makecell{Рубрика В} \\

  \makehangcell{Зиски фармерів} 
    & \phantom{0}\num{2.765.387} & \phantom{0}\num{2.773.644} & \phantom{0}\num{2.937.899} 
    & \phantom{0}\num{2.938.823} & \phantom{0}\num{2.930.874} & \phantom{0}\num{2.946.072} \\

  \addlinespace
  \makecell{Рубрика D} \\

  \makehangcell{Промисловий і ін\-ший зиск\dotfill{}}
    & \phantom{0}\num{4.891.652} & \phantom{0}\num{5.836.203} & \phantom{0}\num{4.858.800} 
    & \phantom{0}\num{4.846.497} & \phantom{0}\num{4.546.147} & \phantom{0}\num{4.850.199} \\

  \addlinespace
  \makehangcell{Сума всіх рубрик від А до~Е\dotfill{}} & \num{22.962.885} & \num{22.998.394} & \num{23.597.574} & \num{23.658.631} & \num{23.236.298} &    \num{23.930.340}\hang{l}{\footnotemark{}}

  \end{tabularx}

\end{table}
\footnotetext{«Tenth Report of the Commissioners of Inland Revenue», London
1866.}
\setlength{\tabcolsep}{\tabcolsepdef}

Під рубрикою D збільшення прибутку від 1853 до 1864 р.
становило пересічно лише 0,93\%, тимчасом як у Великобрітанії
воно за той самий період становило 4,58\%. Дальша таблиця показує
розподіл зисків (за винятком зисків фармерів) у 1864 і
1865 рр.

\begin{table}

  \begin{flushright}
    \emph{Таблиця Е}
  \end{flushright}

  \caption*{Рубрика D. Доходи, що складаються з різних зисків (понад 60\pound{ фунтів
  стерлінґів}) в Ірландії}
  
  \small

  \settowidth{\myheight}{\small\tablefont \num{4.368.610}, розподілений}
  \noindent\begin{tabularx}{\textwidth}{X>{\hangindentdef}p{\myheight}>{\hangindentdef}p{\myheight}}

  \toprule
    & \makecell{1864 р.} & \makecell{1865 р.} \\
    & \makecell{Фунтів стерлінґів} & \makecell{Фунтів стерлінґів} \\
    \midrule
  
  \makehangcell{Загальний річний дохід\dotfill{}} &
  \num{4.368.610}, розподілений між особами \num{17.467} &
  \num{4.669.979}, розподілений між \num{18.081} особою \\

  \makehangcell{Річний дохід понад 60 ф. ст. і нижче за 100 ф. ст\dotfill{}} &
  \samewidth{0.}{~}\num{238.626}, розподілений між \num{5.015} особами &
  \samewidth{0.}{~}\num{222.575}, розподілений між \num{4.703} особами \\

  \makehangcell{Із загального річного доходу\dotfill{}} &
  \num{1.979.066}, розподілений між \num{11.321} особами &
  \num{2.028.471}, розподілений між \num{12.184} особами \\

  \makehangcell{Решта загального річного доходу\dotfill{}} &
  \num{2.150.818}, розподілений між \num{1.131} особою &
  \num{2.418.933}, розподілений між \num{1.194} особами \\

  \multicolumn{1}{r}{\ldelim\{{10}{*}[З нього:]} &
  \num{1.083.906}, розподілений між 910 особами &
  \num{1.097.937}, розподілений між \num{1.044} особами \\

  &
  \num{1.066.912}, розподілений між 121 особою &
  \num{1.320.996}, розподілений між 186 особами \\

  &
  \samewidth{0.}{~}\num{430.535}, розподілений між 105 особами &
  \samewidth{0.}{~}\num{584.458}, розподілений між 122 особами \\

  &
  \samewidth{0.}{~}\num{646.377}, розподілений між 26 особами &
  \samewidth{0.}{~}\num{736.448}, розподілений між 28 особами \\

  &
  \samewidth{0.}{~}\num{262.610}, розподілений між 3 особами &
  \samewidth{0.}{~}\num{274.528}, розподілений між 3 особами\footnotemark{} \\

  \end{tabularx}
\end{table}
\footnotetext{Загальний річний дохід під рубрикою D відхиляється тут від
  чисел попередньої таблиці, бо закон допускає деякі відрахування.}
