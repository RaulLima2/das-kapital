суспільно поділена, але продукти в наслідок цього не стають товарами.
Або, щоб узяти ближчий приклад, на кожній фабриці
праця є систематично поділена, але цей поділ постає не з того,
що робітники обмінюються своїми індивідуальними продуктами.
Тільки продукти самостійних і незалежних одна від однієї приватних
праць протистоять один одному як товари.

Отже, ми бачили, що в споживній вартості кожного товару
міститься якась певна доцільна продуктивна діяльність, або корисна
праця. Споживні вартості не можуть протистояти одна
одній як товари, коли в них не містяться якісно різні корисні
праці. У такому суспільстві, продукти якого загально набирають
форми товарів, тобто в суспільстві товаропродуцентів, ця якісна
ріжниця корисних праць, які виконується незалежно одну від
однієї як приватні справи самостійних продуцентів, розвивається
в багаточленову систему, в суспільний поділ праці.

Сурдутові зрештою байдуже, чи його носитиме кравець, чи
той, хто його замовив. В обох випадках він функціонує як споживна
вартість. Так само мало змінюється само по собі відношення
між сурдутом і тією працею, яка його витворила, в наслідок
того, що кравецтво стає особливою професією, самостійним
членом суспільного поділу праці. Там, де потреба в одежі примушувала
людину кравцювати, вона кравцювала цілі тисячоліття,
раніш ніж з людини зробився кравець. Але буття сурдута,
полотна, всякого елементу речового багатства, що його природа
не дає в готовому вигляді, завжди мусило бути наслідком спеціяльної,
доцільної продуктивної діяльности, яка пристосовує окремі
природні матеріяли до окремих людських потреб. Тому праця,
як творець споживних вартостей, як корисна праця, є незалежна
від усіх суспільних форм умова існування людини, вічна природна
доконечність, що упосереднює обмін речовин між людиною і природою,
тобто людське життя.

Споживні вартості: сурдут, полотно тощо, коротко — товарові
тіла, є сполуки двох елементів: природної речовини і праці. Коли
відлічити загальну суму всіх різних корисних праць, що містяться
в сурдуті, полотні й ін., то завжди лишатиметься матеріяльний
субстрат, даний природою без участи людини. Людина може оперувати
у своїй продукції лише так, як сама природа, тобто може
лише змінювати форми речовин.\footnote{
«Всі світові явища, хоч походять вони від людини, хоч від самих
загальних фізичних законів, не дають нам поняття про дійсну творчість,
а є лише перетворення речовин. Сполука й розділ — ось ті однісінькі
елементи, що їх людський розум завжди находить, аналізуючи ідею репродукції;
так стоїть справа при репродукції вартости (споживної вартости,
хоч Verri в цій своїй полеміці з фізіократами сам не знає гаразд, про яку
саме вартість він говорить) і багатства, коли земля, повітря й вода на
ниві перетворюються на збіжжя, а перероблені людською рукою клейкі
слизоти деяких комах перетворюються на шовк, або поодинокі кусні
металю сполучаються докупи й стають годинником». («Tutti і fenomeni
dell' universo, sieno essi prodotti dell’uomo, ovvero delle universali leggi
della fisica, non ci danno idea di attuale creazione, ma unicamente di una
modificazione della maieria. Accostare e separare sono gli unici elementi-
} Навіть більше: в самій цій праці