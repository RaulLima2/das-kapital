Які ж то, на думку Векфілда наслідки цього сумного стану в колоніях?\footnote*{
У другому німецькому виданні це речення зформульовано так: «Який же то результат панівної в
колоніях системи приватної власности, основаної на власній праці, а не на експлуатації чужої
праці?». Ред.
} «Варварська система
розпорошености» продуцентів і національного майна.\footnote{
Wakefield: «England and America», London 1833, vol. II, p. 52.
270 Там же, стор. 191, 192.
} Роздрібнення засобів
продукції поміж численних самостійно господарюючих власників нищить з централізацією капіталу всі
основи комбінованої праці. Кожне розраховане на довгий час підприємство, що поширюється на багато
років і потребує витрати основного капіталу, наражається, переводячи свої справи, на перешкоди. В
Европі капітал не гає ані хвилинки, бо робітнича кляса становить там його живу приналежність, її там
з лишком, і він завжди може нею порядкувати. Але в колоніяльних країнах! Векфілд
оповідає надзвичайно сумну анекдоту. Він мав розмову з кількома капіталістами з Канади й штату
Нью-Йорк, де хвилі еміґрації часто спиняються, лишаючи по собі осад «зайвих» робітників. «Наш
капітал, — зідхає один з персонажів мелодрами, — наш капітал був напоготові для багатьох операцій,
що для свого виконання потребують чималого часу; але хіба ми могли починати такі операції з
робітниками, які — ми знали це — незабаром повернули б нам спину? Коли б ми були певні, що зможемо
вдержати в себе працю цих еміґрантів, ми охоче були б їх негайно найняли, та ще й за високу ціну. Ще
більше: навіть упевнені, що втратимо їх, ми все ж були б їх найняли, коли б були певні, що матимемо
нове подання праці, відповідно до наших потреб».270

Після того, як Векфілд так пишно змалював контраст між англійським капіталістичним рільництвом з
його «комбінованою» працею і розпорошеним американським селянським господарством, він мимохіть
пробовкнувся й про зворотний бік
медалі. Він змальовує американську народню масу як заможну, незалежну, підприємливу й порівняно
освічену, тимчасом як «англійський рільничий робітник є жалюгідний голодранець (a miserable wretch),
павпер... У якій іншій країні, крім Північної Америки й деяких нових колоній, заробітна плата за
вільну працю, вживану в рільництві, хоч у якійбудь вартій згадки

ni les autres périssent?» (Molinari: «Etudes Economiques», Paris 1846, p. 51, 52). Пане Молінарі,
пане Молінарі! Що це буде з десятьма заповідями, з Мойсеєм та пророками, із законом попиту й
подання, коли в Европі «entrepreneur»\footnote*{
— підприємець. Ред.
} може скорочувати part légitime\footnote*{
— законну пайку. Ред.
} робітника, а в Західній
Індії робітник part légitime підприємця. І скажіть, будь ласка, що це таке, ота «part légitime», що
її, як ви сами призналися, капіталіст
в Европі щодня не доплачує? Молінарі страшенно хочеться там, у колоніях, де робітники такі «прості»,
що «експлуатують» капіталістів, поліційними заходами надати належної чинности законові попиту й
подання, що в інших випадках діє автоматично.