Послухаймо тепер з приводу цього два глибокі зідхання від щирого серця. Пани В. Кукслей (фабриканти
цвяхів, ланцюгів і т. ін. з Брістола) з доброї волі завели у своєму підприємстві норми фабричного
закону. «Через те, що стара, нереґульована система існує й далі в сусідніх підприємствах, їм
доводиться зазнавати такої кривди, що їхніх робітників-підлітків заманюють (enticed) працювати після
шостої години вечора в іншому місці». «Це, — кажуть вони цілком природно, — для нас і кривда і
втрата, бо це виснажує частину сили підлітків, що повна користь від неї належить нам».315 Пан Дж.
Сімпсон (фабрикант паперових коробок і мішечків із Лондону) заявляє комісарам «Children’s Employment
Commission»: «Він, мовляв, підпише всяку петицію на користь заведенню фабричних законів. Хоч що там,
а тепер, по закритті майстерні, він ніколи не має вночі спокою («he always felt restles at night»),
тільки но згадає, що інші примушують працювати довше й виривають у нього з-під носа замовлення».316
«Це була б кривда, — каже «Children’s Employment Commission», резюмуючи, — проти більших підприємців
підводити їхні фабрики під фабричний закон, тимчасом як дрібні підприємства цієї самої галузі
продукції не підлягають ніякому законодавчому обмеженню робочого часу. До несправедливости
неоднакових умов конкуренції щодо робочого часу, яка випливає з того, що по дрібних майстернях
робочий час не є обмежений, для великих фабрикантів долучається ще й та шкода, що подання праці
жінок і підлітків відхилилося б від них у бік майстерень, непідлеглих фабричному законові. Нарешті,
це дало б поштовх до збільшення дрібніших майстерень, які майже без винятку найменше сприятливі для
здоров’я, комфорту, виховання й загального поліпшення становища народу».317

В своєму кінцевому звіті «Children’s Employment Commission» пропонує підвести під фабричний закон
понад 1.400.000 дітей, підлітків і жінок, що з них приблизно половину експлуатує дрібне виробництво
й домашня праця.318 «Коли б, — каже комісія —

315 «Children's Employment Commission. 5 th Report», p. X, n. 35.

316    Там же, стор. ІХ, n. 28.

317    Там же, стор. XXV, n. 165, 176. Про переваги великого виробництва проти карликового порівн.
«Children’s Employment Cominission. З rd Report», p. 13, n. 144, p. 25, n. 121, p. 26, n. 125, p.
27, n. 140 і т. ін.

318    Галузі промисловости, на які треба поширити фабричне законодавство, такі: мануфактура
мережива, плетіння панчіх, плетіння з соломи, мануфактура одягу та її численні відміни, виробництво
штучних квіток, виробництво рукавиць, чоботарство, капелюшництво, кравецтво, всі металюрґійні
фабрики від домн до фабрик голок і т. ін., фабрики паперу, мануфактури скла, тютюнова мануфактура,
ґумові фабрики, фабрикація берд (для ткацтва), ручне килимництво, мануфактура парасолів, фабрикація
веретен і шпульок, друкарство, палітурництво, торговля папером для писання («Stationery», сюди
належить виготовлення паперових коробок, мап, фарб для паперу тощо), виробництво линв, мануфактура
аґатових оздоб, цегельні, шовкові мануфактури, Coventry ткацтво, солеварні, фабрики свічок і
цементу, цукроварні, виробництво сухарів, різні вироби з дерева й інші мішані роботи.
