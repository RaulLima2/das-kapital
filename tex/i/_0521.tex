\parcont{}  %% абзац починається на попередній сторінці
\index{i}{0521}  %% посилання на сторінку оригінального видання
і навіть акумуляція\footnote{
«Політико-економи надто схильні розглядати\dots{} певну кількість
капіталу й певне число робітників як знаряддя продукції однорідної
сили й певної однорідної інтенсивности функціонування\dots{} Ті\dots{}
хто твердять\dots{} що товари е єдині фактори продукції\dots{} доводять, що
продукцію взагалі не можна поширити, бо для такого поширення треба б спочатку
збільшити засоби існування, сировинні матеріяли і знаряддя, а це фактично
сходить на те, що ніякого зросту продукції не може бути без попереднього
її зросту, або, інакше кажучи, що ніякий зріст продукції неможливий».
(\emph{S.~Bailey}: «Money and its Vicissitudes», p. 58 і 70). Бейлі
критикує цю догму головним чином з погляду циркуляції.
}. Догму цю використовували і сам Бентам,
і Малтуз, і Джеме Мілл, Мак Куллох і~\abbr{т. д.} з апологетичною
метою, а саме з метою представити одну частину капіталу, змінний
капітал, або капітал, що обмінюється на робочу силу, як
сталу величину. Створено байку, що речове існування змінного
капіталу, тобто та маса засобів існування, яку він репрезентує
для робітників, або так званий робочий фонд, є нібито осібна
частина суспільного багатства, визначена природними межами,
що їх ніяк не можна переступити. Щоб пустити в рух ту частину
суспільного багатства, що має фунціонувати як сталий капітал,
або, висловлюючи це речево, як засіб продукції, потрібна
певна маса живої праці. Вона є дана технологічними умовами. Але
не дані ані число робітників, потрібне, щоб пустити в рух цю
масу праці, бо воно змінюється разом із зміною ступеня експлуатації
індивідуальної робочої сили, ані ціна цієї робочої сили;
дана лише мінімальна межа тієї ціни, до того ж дуже еластична.
Факти, що лежать в основі цієї догми, такі. З одного боку, робітник
не має голосу при поділі суспільного багатства на засоби
споживання неробітників і на засоби продукції. З другого
боку, робітник лише у сприятливих виняткових випадках може
поширити так званий «робочий фонд» коштом «доходів» багатих\footnote{
Дж.~Ст.~Мілл каже у своєму творі «Principles of Polit. Economy»,
т. II, стор. 259, 260: «Продукт праці за наших часів поділяється у зворотній
пропорції до праці: найбільша частина припадає тим, що ніколи
не працюють, дальша щодо величини частина — тим, чия праця майже
лише номінальна, і таким чином, за низхідною скалею, нагорода щораз
більше спадає в міру того, як праця стає тяжча й неприємніша, так що
за найвтомнішої і найвиснажливішої фізичної праці людина не може з
певністю розраховувати навіть на задоволення життєвих потреб». Щоб
уникнути непорозуміння, зауважу, що коли таких людей, як Дж.~Ст.~Мілл і~\abbr{т. д.}, і треба ганьбити за суперечності між їхніми старими економічними
догмами і їхніми сучасними тенденціями, то все ж було б цілком
несправедливо змішувати їх до купи із зграєю вульгарно-економічних
апологетів.
}.
До якої недоладної тавтології приводить спроба перебрехати
капіталістичні межі робочого фонду на його суспільні
природні межі, показує, між іншим, професор Фавсет: «Обіговий
капітал\footnote{
\emph{Н.~Fawcett}. Prof, of Political Economy at Cambridge: «The Economic
Position of the British Labourer», London 1865, p. 120.
} якоїсь країни, — каже він, — це є робочий фонд
тієї країни. Тим то, щоб обрахувати пересічну грошову плату,
яку дістає кожний робітник, нам треба просто лише поділити
\parbreak{}  %% абзац продовжується на наступній сторінці
