важає то рільництво, то скотарство, і залежно від цих періодів
змін коливаються й розміри селянської продукції. Лише велика
промисловість з її машинами дає сталу базу для капіталістичного
рільництва, радикально експропріює величезну більшість сільської
людности й вивершує відокремлення рільництва від домашньої
сільської промисловости, вириваючи її коріння: прядіння
і ткацтво.\footnote{
Текет знає, що з мануфактур у власному значенні слова та із
зруйнування сільських або домашніх мануфактур виникає із заведенням
машин велика вовняна промисловість (Tuckett: «A History of the Past
and Present State of the Labouring Population», vol. I, p. 139—143).
«Плуг, ярмо були винаходом богів і заняттям героїв; хіба ж ткацький
варстат, веретено й прядка менш благородні походженням? Ви відокремлюєте
прядку від плуга, веретено від ярма, і маєте фабрики й робітні
доми, кредит і паніку, дві ворожі нації, рільничу й комерційну».
(David Urquhart: «Familiar Words», London 1855, p. 122). Але ось з’являється
Kepi і обвинувачує Англію, звичайно, не без підстав, у тому,
що вона намагається перетворити всі інші країни у виключно рільничі,
щоб стати для них фабрикантом. Він запевняє, що таким чином зруйновано
Туреччину, бо там «ніколи не дозволялось (Англією) землевласникам
і рільникам зміцнити своє становище через природну спілку плуга
з ткацьким варстатом, борони з молотком». («The Slave Trade», p. 125).
На його погляд, сам Уркварт є один із головних винуватців зруйнування
Туреччини, де він в інтересах Англії пропагував вільну торговлю.
Але найкраще те, що Кері, до речі великий холоп Росії, хоче за допомогою
протекційної системи спинити той процес відокремлення, що його вона
в дійсності прискорює.
} Тим то лише вона завойовує для промислового
капіталу ввесь внутрішній ринок.\footnote{
Філантропічні англійські економісти, як от Мілл, Роджерс, Ґолдвін,
Сміс, Фавсет і т. ін., та ліберальні фабриканти, як от Джон Брайт
і компанія, запитують англійських земельних аристократів, як бог
запитував Каїна про брата його Авеля: де поділись тисячі наших freeholder’iв?\footnote*{
— самостійних селян. Ред.
}.
Та ви то сами звідки взялися? Із знищення цих freeholder’iв.
Чому ви не питаєте далі: де поділись наші самостійні ткачі, прядуни,
ремісники?
}

6. Генеза промислового капіталіста

Генеза промислового\footnote{
«Промисловий» уживається тут у протилежність до «рільничого».
В розумінні «категорії» фармер є такий самий промисловий капіталіст,
як і фабрикант.
} капіталіста відбувалася не з такою
поступінністю, як генеза фармера. Без сумніву, деякі дрібні
цехові майстрі, і ще більше самостійні дрібні ремісники і навіть
наймані робітники перетворювались на дрібних капіталістів,
а потім, поволі, за допомогою більше поширюваної експлуатації
найманої праці й відповідної акумуляції, — на капіталістів sans
phrase.\footnote*{
— попросту. Ред.
} У дитячому періоді капіталістичної продукції справа
здебільша стояла так, як у дитячому періоді середньовічного
міського ладу, де питання про те, хто з кріпаків-утікачів повинен
бути майстром, а хто слугою, вирішувано здебільша залежно
від того, хто з них раніш утік. Однак черепаша хода цієї методи
зовсім не відповідала торговельним потребам нового світового