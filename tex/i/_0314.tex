\parcont{}  %% абзац починається на попередній сторінці
\index{i}{0314}  %% посилання на сторінку оригінального видання
про цілковитий переворот у будівництві парусних суден, комунікаційну
й транспортову справу поступінно пристосовано до способу
продукції великої промисловости за допомогою системи річкових
пароплавів, залізниць, океанських пароплавів і телеграфу.
Але величезні маси заліза, що їх доводилося тепер кувати, зварювати,
різати, свердлити й формувати, потребували, з свого боку,
таких циклопічних машин, що їх створити не сила було мануфактурному
будівництву машин.

Отже, велика промисловість мусила опанувати характеристичний
для неї засіб продукції, саму машину, та продукувати машини
за допомогою машин. Тільки тоді вона створила адекватну собі
технічну базу і стала на свої власні ноги. Із зростанням машинового
виробництва першими десятиліттями XIX віку машина дійсно
поступінно опанувала фабрикацію виконавчих машин. Однак,
тільки в другу третину XIX віку величезніше будівництво
залізниць та океанське пароплавство покликали до життя ті
циклопічні машини, що їх уживали до конструкції перших
моторів.

Найпосутнішою продукційною умовою для фабрикації машин
за допомогою машин була така рухова машина, що була б здатна
розвивати силу всякої величини, алеж разом з тим цілком піддаватися
контролеві. Така машина існувала вже у формі парової
машини. Але одночасно треба було машиновим способом продукувати
потрібні для окремих частин машин строго геометричні
форми, такі, як от лінія, площінь, коло, циліндер, конус і куля.
Цю проблему розв’язав Генрі Модслей в першому десятилітті
XIX віку, винайшовши slide-rest, що його незабаром зроблено
автоматичним і в змодифікованій формі перенесено з токарського
варстату, для якого він був спочатку призначений, на інші будівельні
машини. Цей механічний пристрій замінює не якесь окреме
знаряддя, а саму людську руку, яка виготовляє певну форму, наближаючи
та припасовуючи вістря токарського інструменту тощо
до матеріялу праці, або спрямовуючи його на матеріял праці,
наприклад, залізо. Таким чином пощастило геометричні форми
окремих частин машин «продукувати з такою легкістю, точністю
і швидкістю, яких не міг би ніякий нагромаджений досвід дати
руці найвправнішого робітника».\footnote{
«The Industry of Nations», London 1855, part. II, p. 239. Тут
саме сказано: «Хоч і яким простим та незначним здавався б цей додаток
до варстату, ми, здається, не перебільшуючи, можемо сказати, що його
вплив на вдосконалення й поширення машин був так само великий, як і
вплив удосконалень, що їх поробив Ватт у самій паровій машині. Заведення
того додатку призвело відразу до вдосконалення та подешевшання
всяких машин і стимулювало до нових винаходів та вдосконалень».
(«Simple and outwardly unimportant as this appendage to lathes may appear,
it is not, we belive, averring too much to state, that its influence in
improving and extending the use of machinery has been as great as that
produced by Watt’s improvements of the steam-engine itself. Its introduction
went at once to perfect all machinery, to cheapen it, and to stimulate
invention and improvement»).
}
