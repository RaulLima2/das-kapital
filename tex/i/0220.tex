літок» і кожна «дитина» можуть починати свою дванадцятигодинну,
зглядно восьмигодинну працю, переривати й закінчувати
її, і так само залишив на їхню волю призначати різні години
на їжу для різних осіб, то ці пани швидко винайшли нову
«Relaissystem», за якої робочих коней не переміняють на певних
поштових станціях, а раз-у-раз наново запрягають на перемінних
станціях. Ми не зупиняємось довше на принадах цієї
системи, бо пізніш муситимемо повернутись до неї. Але вже і з
першого погляду ясно, що ця система знищила не лише дух, а й
саму букву цілого фабричного закону. Як могли б фабричні
інспектори за такої складної бухгальтерії щодо кожної окремої
дитини й кожного підлітка примусити фабрикантів додержувати
визначеного законом робочого часу й давати на визначений законом
час перерву на їжу? На більшості фабрик незабаром знов
безкарно почали процвітати давніші жорстокі неподобства.
На нараді з міністром унутрішніх справ (1844) фабричні інспектори
довели неможливість якогобудь контролю за нововигаданої
системи змін.136 Але тимчасом обставини дуже змінилися. Фабричні
робітники, особливо від 1838 р., зробили своїм економічним гаслом
десятигодинний біл, подібно до того, як Charter* вони зробили своїм
політичним гаслом. Навіть частина фабрикантів, що вреґулювала
фабричну продукцію згідно з законом 1833 р., закидала парламент
меморіалами про неморальну «конкуренцію» «фальшивих братів»,
що їм їхня більша нахабність або щасливіші місцеві обставини
дозволяють ламати закон. До того, хоч і як дуже хотілось би
окремим фабрикантам дати повну волю своїй давнішній ненажерливості,
оратори й політичні провідники класи фабрикантів радили
змінити поведінку й мову щодо робітників. Вони розпочали
були похід за скасування хлібних законів і для перемоги потребували
допомоги робітників! Тому вони й обіцяли не лише подвоїти
вагу буханця хліба, але й ухвалити закон про десятигодинний
робочий день у тисячолітньому царстві Free Trade.**18
Отже, їм то менше треба було боротися проти тих заходів, що
повинні були лише здійснити закон 1833 р. Нарешті, торі, що їхньому
найсвятішому інтересові, земельній ренті, загрожувала
небезпека, у своїм філантропічнім обуренні загриміли на «ганебну
поведінку» 138 своїх ворогів.

Так з’явився додатковий фабричний закон з 7 червня 1844 р.
Він набув сили 10 вересня 1844 р. Ним ставиться під охорону закону
нову категорію робітників, а саме жінок, старших за 18 років.

136 «Reports of Insp. of Fact, for 31 st October 1849», p. 6.

137 «Reports of Insp. of Fact, for 31 st October 1848», p. 98.

138    Зрештою, Леонард Горнер офіціально вживає вислову «nefarious
practices» (ганебна поведінка). («Reports of Insp. of Fact, for 31 st
October 1859», p. 7).

* Хартія. Мова йде про хартію чартистів. Її пункти: загальне виборче
право, щорічне переобрання парламенту, оплата депутатів парламенту,
таємне голосування, рівні виборчі округи і скасування майнового
цензу. Ред.

** — вільної торговлі. Ред.
