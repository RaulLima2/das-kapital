Отже, зростання числа фабричних робітників зумовлено пропорційно
куди швидшим зростанням цілого капіталу, вкладеного
у фабрику. Але цей процес відбувається лише в межах періодів
припливу та відпливу промислового циклу. До того його завжди
перериває технічний прогрес, який то потенціяльно заступає робітника,
то витискує його фактично. Ця якісна зміна в машиновому
виробництві постійно викидає робітників із фабрики або замикав
фабричну браму перед новим потоком рекрутів, тимчасом як
просте кількісне поширення фабрик поглинає разом із викинутими
й свіжі контингенти. Таким чином робітників постійно відштовхують
або притягають, кидають ними туди й сюди, і це
супроводиться постійними змінами щодо статі, віку та вправности
завербованих.

Доля фабричного робітника унаочнюється найкраще, коли
подати короткий огляд долі англійської бавовняної промисловости.

Від 1770 до 1815 рр. бавовняна промисловість п’ять років перегнивала
період пригнічення або застою. Протягом цього першого
45-річного періоду англійські фабриканти мали монополію на
машини та світовий ринок. 1815—1821 рр. — час пригнічення;
1822—1823 рр. — розцвіт; 1824 р. — скасування закону про коаліції,
загальне велике поширення фабрик; 1825 р. — криза;
1826 р. — великі злидні та повстання серед бавовняних робітників;
1827 р. — легке поліпшення; 1828 р. — велике зростання
числа парових ткацьких варстатів і вивозу; 1829 р. — вивіз,
особливо до Індії, перевищує всі попередні роки; 1830 р. —
переповнені ринки, великі злидні; 1831—1833 рр. — тривале
пригнічення; у східньоіндійської компанії відібрано монополію
торговлі із Східньою Азією (Індією та Китаєм); 1834 р. — великий
зріст фабрик та машин, недостача рук; новий закон про бідних
активізує еміграцію сільських робітників до фабричних округ;
очищення сільських графств от дітей, торговля білими рабами.
1835 р. — великий розцвіт, але одночасно ручні бавовняні ткачі

так дуже швидко, що багато фірм платить тепер лише половину первісної
заробітної плати. А все ж, хоч заробітна плата падає нижче й нижче,
зиски з кожною зміною тарифу праці, здається, зростають». — Навіть
несприятливі періоди промисловости фабриканти використовують на те,
щоб через надмірне пониження заробітної плати, тобто безпосередньою
крадіжкою найдоконечніших засобів існування робітника, здобувати
надзвичайні зиски. Ось приклад. Мова йде про кризу шовкоткацтва в
Coventry. «Із свідчень, які я дістав так від фабрикантів, як і від робітників,
безперечно виходить, що заробітна плата понижена в більшому розмірі,
ніж цього вимагала конкуренція чужоземних продуцентів або інші обставини.
Більшість ткачів працює за заробітну плату, знижену на 30—40\%.
Моток стьожки, за який ткач перед п’ятьма роками діставав 6 або 7 шилінґів,
дає йому тепер лише 4 шилінґи 3 пенси або 3 шилінґи 6 пенсів;
за іншу працю, за яку раніш платили 4 шилінґи або 4 шилінґи 3 пенси,
він дістає тепер тільки 2 шилінґи або 2 шилінґи 3 пенси. Заробітну плату
понижено тепер більш, ніж це потрібно було для активізації попиту.
Справді, для багатьох сортів стьожок пониження заробітної плати не
супроводилося навіть якимось пониженням ціни товару». (Звіт комісара
F. D. Longe в «Children’s Employment Commission. 5 th Report 1866»,
p. 114, n. 1).
