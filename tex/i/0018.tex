бить споживну вартість В матеріялом виразу своєї власної вартости.
Вартість товару А, виражена таким чином у споживній
вартості товару В, має форму відносної вартости.

b) Кількісна визначеність відносної форми вартости

Кожний товар, що його вартість має бути виражена, являє собою
дану кількість якогось предмету споживання — 15 шефлів
пшениці, 100 фунтів кави тощо. Ця дана кількість товару містить
у собі певну кількість людської праці. Отже, форма вартости має
виразити не лише вартість взагалі, але й кількісно визначену
вартість, або величину вартости. Тим то у вартостевому відношенні
товару А до товару В, полотна до сурдута, рід товару «сурдут»
не лише якісно прирівнюється до полотна, як тіло вартости
взагалі, але й до певної кількости полотна, наприклад, до 20 метрів
полотна прирівнюється певну кількість тіла вартости, або
еквіваленту, приміром, 1 сурдут.

Рівнання: «20 метрів полотна = 1 сурдутові, або: 20 метрів
полотна варті 1 сурдута» має за передумову, що в 1 сурдуті міститься
рівно стільки субстанції вартости, як і в 20 метрах полотна,
що, отже, обидві кількості товарів коштують рівну кількість
праці, або рівну кількість робочого часу. Але робочий час,
доконечний для продукції 20 метрів полотна або 1 сурдута, змінюється
з кожною зміною в продуктивній силі ткацтва або кравецтва.
Вплив таких змін на відносний вираз величини вартости
треба тепер розглянути докладніше.

I. Хай вартість полотна змінюється,19 тимчасом як вартість
сурдута лишається стала. Коли робочий час, доконечний для продукції
полотна, подвоюється, приміром, у наслідок того, що зменшується
родючість землі, яка родить льон, то подвоюється і його
вартість. Замість рівнання: 20 метрів полотна = 1 сурдутові ми
мали б: 20 метрів полотна = 2 сурдутам, бо 1 сурдут містить у
собі тепер лише половину того робочого часу, що міститься в
20 метрах полотна. Навпаки, коли доконечний для продукції
полотна робочий час зменшиться наполовину, приміром, у наслідок
поліпшення ткацьких варстатів, то й вартість полотна спаде
наполовину; отже, відповідно до цього ми мали б тепер: 20 метрів
полотна = \sfrac{1}{2} сурдута. Отже, за незмінної вартости товару В
відносна вартість товару А, тобто його вартість, виражена в товарі
В, підвищується й падає просто пропорційно до вартости
товару А.

II. Хай вартість полотна лишається стала, тимчасом як вартість
сурдута змінюється. Коли за цих обставин доконечний для
продукції сурдута робочий час подвоюється, приміром, у наслідок
недостатнього збору вовни, то ми замість: 20 метрів полотна =

19 Вислову «вартість» уживається тут, як і в деяких місцях раніш,
для позначення кількісно визначеної вартости, тобто величини вартости.
