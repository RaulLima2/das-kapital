буття, то він містить у собі лятентно можливість замінити металеві
гроші в їхній монетній функції знаками з іншого матеріялу або
символами. Технічні труднощі карбування монети з найдрібніших
вагових частин золота або срібла і та обставина, що первісно
за міру вартости служили нижчі металі замість благородніших,
срібло замість золота, мідь замість срібла, — і тому вже циркулювали
як гроші в той час, коли благородніший металь скинув
їх із трону, — ці обставини історично пояснюють ролю срібних
і мідяних знаків як заступників золотих монет. Вони заступають
золото в сфері товарової циркуляції там, де монета циркулює
найшвидше і через те найшвидше стирається, тобто там, де купівлі
й продажі без упину відновлюються в найдрібнішому
маштабі. Щоб перешкодити цим супутникам золота вкоренитися
замість самого золота, закон визначає дуже незначні пропорції,
в яких їх мусять при платежах приймати замість золота. Ті осібні
сфери, у яких різні сорти монет циркулюють, звичайно, переплутуються
між собою. Дрібна монета з’являється побіч золота
для виплати дробових частин найменшої золотої монети; золото
постійно увіходить у цю роздрібну циркуляцію, але так само
постійно його викидають звідти через розмін на дрібну монету.\footnote{
«Коли маса срібла ніколи не перевищує тієї кількости, що потрібна
для дрібних платежів, то його не можна зібрати в кількості, потрібній
для більших платежів... Вживання золота для великих платежів неминуче
включає вживання його і в дрібній торговлі. Хто має золоту монету, той
оплачує нею дрібні закупи й дістає разом із купленим товаром решту
сріблом. Таким шляхом той лишок срібла, який інакше зібрався б у дрібного
крамаря, відтягається від нього й розпорошується в загальній циркуляції.
Але коли б срібла було стільки, скільки потрібно, щоб перевести
дрібні платежі не користуючись із золота, то дрібний крамар одержував
би за дрібні закупи срібло й останнє неминуче нагромаджувалося б у
його руках». «If silver never exceed what is wanted for the smaller payments,
it cannot be collected in sufficient quantities for the larger payments...
the use of gold in the main pauyments necessarily implies also
its use in the retail trade: those who have gold coin, ofering them for small
purchases, and receiving with the commodity purchased a balance of silver
in return. Dy which means the surplus of silver that would otherwise encumber
the retail dealer, is drawn of and dispersed into general circulation.
But if there is as much silver as will transact the small payments independent
of gold, the retail dealer must then receive silver for small purchases; and
it must of necessity accumulate in his hands»). (David Buchanan: «Inquiry
into the Taxation and Commercial Policy of Great Britain», Edinburgh
1884, p. 248, 249)
}

Металевий зміст срібних або мідяних знаків закон визначає
самовільно. В обігу вони стираються ще швидше, ніж золота
монета. Тому їхня монетна функція стає фактично цілком незалежною
від їхньої ваги, тобто від будь-якої вартости. Монетне
буття золота цілком відділяється від його субстанції вартости.
Тому відносно безвартісні речі, папірці, можуть функціонувати
як монети замість золота. В металевих грошових знаках їхній
суто символічний характер ще до деякої міри є захований. У паперових
грошах він виступає вже наочно. Ми бачимо: ce n’est que
le premier pas qui coûte.*

* — тяжко зробити лише перший крок. Ред.
