\parcont{}  %% абзац починається на попередній сторінці
\index{i}{0130}  %% посилання на сторінку оригінального видання
як продавець своєї власної робочої сили, той стан, коли людська
праця не визволилась була ще із своєї первинної інстинктивної
форми, відходить у глибини праісторичних часів. Ми припускаємо
працю в такій формі, що в ній вона належить виключно людині.
Павук виконує операції, що подібні до операцій ткача, а бджола
будовою своїх воскових комірочок засоромить не одного архітектора-людину.
Але найгіршого архітектора від найкращої
бджоли з самого ж початку відрізняє те, що він, раніш ніж збудувати
комірочку з воску, вже збудував її у своїй голові. Наприкінці
процесу праці виходить такий результат, який на початку
цього процесу вже існував в уяві робітника, отже, вже
існував ідеально. Він не лише змінює форму того, що дала природа;
в тому, що дала природа, він здійснює одночасно й свою
свідому мету, яка як закон визначає спосіб і характер його діяння
і якій він мусить підпорядкувати свою волю. І це підпорядковання
не є відокремлений акт.\footnote*{
У французькому виданні це речення подано так: «І це підпорядковання
не є короткочасне» («Et cette subordination n’est pas momen
tanée»). \emph{Peд}.
} Окрім напруження органів, що працюють,
потрібна на цілий час праці доцільна воля, яка виявляється
в увазі, і потрібна вона то більше, що менше праця захоплює
робітника своїм змістом, способом та характером її виконання,
тобто, що менше він тішиться нею як грою своїх власних
фізичних і інтелектуальних сил.

Прості моменти процесу праці є: 1) доцільна діяльність, або
сама праця, 2) предмет праці й 3) засоби праці.

Земля (економічно розуміємо під нею й воду), що первісно
постачає \footnote{«Здається, — та воно так і справді є, — що природні продукти
землі, яких є лише обмежена кількість і які існують цілком незалежно
від людини, є дані природою цілком так само, як дають юнакові невеличку
суму грошей, щоб справити його на шлях якоїсь діяльности й дати йому
спроможність надбати собі статки» («The earth’s spontaneous productions
being in small quantity, and quite independent of man, appear,
as it were, to be furnished by nature, in the same way as a small sum is given
to a young man, in order to put him in a way of industry, and of making
his fortune»). (\emph{James Steuart}: «Principles of Political Economy», Ed.
Dublin. 1770» vol. I, p. 116).}
людині харчі, готові засоби існування, існує без будь-якої
допомоги людини як загальний предмет людської праці.
Всі речі, які праця відриває лише від їхнього безпосереднього
зв’язку з усесвітом, є предмети праці, дані природою, приміром,
риба, яку ловлять, відривають од її життєвої стихії, води, дерево,
що рубають у пралісі, руда, яку видобувають із її жил. Навпаки, коли сам предмет праці є вже, так би мовити, профільтрований
попередньою працею, то ми звемо його сировинним матеріялом,
як ось видобута вже руда, що її потім промивають. Усякий сировинний
матеріял є предмет праці, та не всякий предмет праці є
сировинний матеріял. Предмет праці є сировинний матеріял тільки
тоді, коли він уже зазнав зміни, спричиненої працею.

Засіб праці є якась річ або комплекс речей, що їх робітник
ставить поміж собою й предметом праці й що служать за провід-
\parbreak{}
\index{i}{0131}  %% посилання на сторінку оригінального видання
ника його діяльности на цей предмет. Він користується механічними,
фізичними, хемічними властивостями речей для того, щоб
примусити їх діяти як сили на інші речі, відповідно до своєї мети.\footnote{
«Розум є так само хитрий, як і могутній. Хитрість полягає взагалі
в упосереднювальній діяльності, яка, примушуючи об’єкти, відповідно
до їхньої власної природи, діяти один на одного та впливати один на
одного, не втручається безпосередньо до цього пронесу і все ж досягає
здійснення лише своєї мети». (\emph{Hegel}: «Enzyklopädie. Erster Teil. Die
Logik», Berlin 1840», S. 382).
}
Предмет, що його робітник опановує безпосередньо, — залишаючи
осторонь захоплювання готових засобів існування, приміром,
овочів, коли лише його власні органи служать за засоби праці, —
є не предмет праці, а засіб праці. Таким чином, даний самою
природою предмет стає органом його діяльности, органом, що
його він долучає до органів свого власного тіла, здовжуючи,
всупереч біблії, природний розмір своєї статури. Земля, являючи
собою його первісну харчову комору, є так само й первісний арсенал
його засобів праці. Вона постачає йому, приміром, камінь,
яким він кидає, тре, тисне, ріже й т. ін. Сама земля є засіб праці,
але функціонування її як засобу праці в рільництві знов же
має за передумову цілий ряд інших засобів праці й порівняно
високий уже розвиток робочої сили.\footnote{
У своїй, зрештою нікчемній, праці «Théorie de l’Economie Politique»,
Paris 1815, Ґаніль влучно перелічує всупереч до фізіократів величезний
ряд процесів праці, які становлять передумову рільництва у власному
значенні слова.
} Скоро тільки взагалі процес
праці сяк-так розвинеться, то він потребує оброблених уже засобів
праці. В печерах найдавнішої людини ми находимо кам’яне
знаряддя й кам’яну зброю. Поруч із обробленим каменем, деревом,
кістьми й мушлями головну ролю як засіб праці на початках
людської історії відіграє приручена, отже, змінена вже працею,
виплекана тварина.\footnote{
У «Réflexions sur la Formation et la Distribution des Richesses»
(1766), Oeuvres, éd. Daire, vol. І, Тюрґо добре розвинув вагу прирученої
тварини для початків культури.
} Вжиток і створення засобів праці,
хоч у зародковій формі вони вже властиві і деяким породам тварин,
характеризують специфічно людський процес праці. Тим
то Франклін визначає людину, як «a toolmaking animal», як
тварину, що продукує знаряддя праці. Останки засобів праці
мають для вивчання загинулих економічних суспільних формацій
таку саму вагу, яку будова останків костей для пізнавання організації
загинулих тваринних порід. Економічні епохи відрізняються
не тим, що продукують, а тим, як продукують, якими
засобами праці.\footnote{
З усіх товарів власне люксусові товари мають найменше значення
для технологічного порівняння різних епох продукції.
} Засоби праці є не лише мірило розвитку людської
робочої сили, але й покажчик суспільних відносин, за яких
люди працюють. Серед самих засобів праці механічні засоби праці,
що їх сукупність можна назвати кістковою і мускульною системою
продукції, подають значно відмінніші характеристичні ознаки
певної епохи суспільної продукції, ніж такі засоби праці, що
\parbreak{}  %% абзац продовжується на наступній сторінці
