15\%, річний оборот товарів фабрики мусить становити вартість у 115.000 фунтів стерлінґів... Кожна з
23 половин робочої години продукує щодня 5/115, або 1/23 цих 115.000 фунтів стерлінґів. З цих 23/23,
що становлять цілість цих 115.000 фунтів стерлінґів (constituting the whole 115.000 £), 20/23, тобто
100.000 фунтів стерлінґів з цих 115.000 повертають назад лише капітал, 1/23, або 5.000 фунтів
стерлінґів, із гуртового прибутку в 115.000 (!) покривають зужитковання фабрики і машин. Решта ж,
2/23, тобто обидві останні півгодини кожного дня, продукують чистий прибуток
у 10\%. Тому, коли б фабрика за незмінних цін могла працювати 13 годин замість 11 1/2, то за
збільшення обігового
капіталу приблизно на 2.600 фунтів стерлінґів чистий прибуток більше ніж подвоївся б. З другого
боку, коли б робочі години скоротити на 1 годину на день, то чистий прибуток зник би, а коли б
скоротити на 1 1/2 години, то зник би й гуртовий прибуток».32

І це пан професор називає «аналізою»! Коли він повірив лементові фабрикантів, що робітники кращий
час дня витрачають
на продукцію, отже, на репродукцію або на покриття вартости будівель, машин, бавовни, вугілля й т.
ін., то всяка аналіза
була зайва. Він мав просто відповісти: Мої панове! Коли ви примусите працювати 10 годин замість 11
1/2, то за інших незмін-

32    Сеніор, там же, стор. 12, 13. Ми не спиняємось на курйозах, що для нашої мети не мають ваги,
наприклад, на твердженні, що фабриканти зараховують до прибутку, брутто або нетто, брудного або
чистого, покриття зужиткованих машин і т. ін., тобто складову частину капіталу. Ми не
спиняємося й на тому, правильні чи неправильні ці числові дані. ІЦо вони
варті не більш, як ця так звана «аналіза», це довів Леонард Горнер в «А Letter to Mr. Senior etc.
London. 1837». Леонард Горнер, один із Factory Inquiry Commissioners * 1833 p. і фабричний
інспектор, у дійсності
фабричний цензор до 1859 р., здобув собі невмирущі заслуги перед англійською робітничою клясою. Все
своє життя він боровся не лише з розлютованими фабрикантами, але й з міністрами, для яких
незрівнянно важливіше було рахувати «голоси» фабрикантів у палаті громад, ніж
робочі години «рук» на фабриці.

Додаток до 32 примітки. Незалежно від фалшивости змісту сам виклад Сеніора є плутаний. Сказати він,
власне, хотів ось що. Фабрикант щоденно вживає робітників протягом 11 1/2 годин, або 23/2 години.
Цілорічна праця так само, як і поодинокий робочий день, складається з 11 1/2,
або 23/3 години (помножених на число робочих днів протягом року). За такої передумови 23/2 робочої
години продукують річний продукт у 115.000 фунтів стерлінґів; 1/2 робочої години продукує 20/23 X
115.000 фунтів стерлінґів; 20/2 робочої години продукує 20/23 X 115.000 фунтів стерлінґів = 100.000
фунтам стерлінґів, тобто вони лише повертають
авансований капітал. Лишаються 3/2 робочої години, що продукують 3/23 X 115.000 фунтів стерлінґів =
15.000 фунтів стерлінґів, тобто гуртовий прибуток. З цих 3/2 робочої години 1/2 робочої години
продукує 1/23 X 115.000 фунтів стерлінґів = 5.000 фунтів стерлінґів, тобто продукує
лише покриття зужитковання фабрики й машин. Останні дві половини робочої години, тобто остання
робоча година, продукує 2/23 X 115.000 фунтів стерлінґів = 10.000 фунтів стерлінґів, тобто чистий
зиск.
У тексті Сеніор перетворює останні 2/23 продукту на частину самого робочого
дня.

* — членів комісії для розсліду відносин по фабриках. Ред.
