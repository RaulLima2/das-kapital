\index{i}{0378}  %% посилання на сторінку оригінального видання
\begin{table}
  \centering
  \caption*{П'ятирічні періоди й 1866 рік}
  \begin{tabularx}{\textwidth}{Xrrrrrrrr}
    \toprule
    \makecell{1831--1835} & \makecell{1836--1840} & \makecell{1841--1845} & \makecell{1846--1850}
    & \makecell{1851--1855} & \makecell{1856--1860} & \makecell{1861--1865} & \makecell{1866} \\
    \midrule
    \makecell{Пересічно за рік} \\
    Імпорт (квартери)\dotfill{} &  1.096.373 & 2.389.729 & 2.843.865 & 8.776.552  & 8.345.237 & 10.913.612 & 15.009.871 & 16.457.340 \\
    \makecell{Пересічно за рік} \\
    Експорт (квартери)\dotfill{}  &   225.363  &   251.770  &    139.056 &    155.461  &    307.491 &     341.150  &    302.754  &  216.218 \\
    Перевага імпорту над експортом пересічно за рік\dotfill{}  & 871.010 & 2.137.959 & 2.704.809 & 8.621.091  & 8.037.746  & 10.572.462  & 14.707.117  & 16.241.122 \\
    \makecell{Людність:} \\
    Пересічне число на рік у кожному періоді\dotfill{} & 24.621.107 & 25.929.507 &
        27.262.569 & 27.797.598 & 27.572.923 & 28.391.544 & 29.381.460 & 29.935.404 \\
    Пересічна кількість збіжжя тощо в квартерах, що її споживає за рік один індивід, при рівному
    розподілі між людністю, із надлишку проти тубільної продукції\dotfill{} & 0,036  &
    0,082 & 0,099 & 0,310  & 0,291  & 0,372  & 0,543  & 0,543 \\
  \end{tabularx}
\end{table}

