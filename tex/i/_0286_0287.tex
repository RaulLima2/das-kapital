\parcont{}  %% абзац починається на попередній сторінці
\index{i}{0286}  %% посилання на сторінку оригінального видання
між собою. Це та природно виросла відмінність, що при контакті
громад викликає взаємний обмін продуктами, а тому й поступінне
перетворення цих продуктів на товари. Обмін не створює ріжниці
між сферами продукції, а ставить різні сфери продукції у зв’язок
одну з одною, перетворюючи їх таким чином на більш або менш
залежні одну від одної галузі суспільної сукупної продукції.
Тут суспільний поділ праці постає через обмін між первісно різними,
але одна від однієї незалежними сферами продукції. Там,
де фізіологічний поділ праці становить вихідний пункт, окремі
органи безпосередньо зв’язаної цілости роз’єднуються, розкладаються
— при чому головний поштовх цьому процесові розкладу
дає обмін товарів із чужими громадами — і усамостійнюється
до того пункту, де зв’язок різних праць упосереднюється обміном
продуктів як товарів. В одному випадку втрачає свою самостійність
те, що раніш було самостійне, у другому всамостійнюється
те, що було раніш несамостійне.

Основою всякого розвиненого та обміном товарів упосереднюваного
поділу праці є відокремлення міста від села.\footnote{
Сер Джемс Стюарт найкраще висвітлив цей пункт. Як мало відомий
за наших часів його твір, який з’явився десять років перед «Wealth
of Nations» А. Сміса, видно, між іншим, із того, що прихильники
Малтуза навіть не знають, що Малтуз у першому виданні своєї праці
про «Людність», за винятком суто декляматорської частини, майже все
вписав із Стюарта та з творів попів Уоллеса і Тавнсенда.
} Можна
сказати, що ціла економічна історія суспільства резюмується
в русі цієї протилежности, на якій ми тут, однак, не спинятимемося
довше.

Як для поділу праці всередині мануфактури матеріяльну передумову
становить певне число одночасно вживаних робітників,
так само для поділу праці всередині суспільства за матеріяльну
передумову є кількість людности і її густота, яка тут заступає
місце концентрації людей у тій самій майстерні.\footnote{
«Певна густота людности потрібна так для того, щоб могли розвинутися
соціяльні відносини, як і для того, щоб склалася така комбінація
сил, за якої зростає продуктивність праці» («There is a certain
density of population which is convenient, both for social intercourse,
and for that combination of powers by which the produce of labour is increased»).
(\emph{James Mill}: «Elements of Political Economy», London 1821,
p. 50). «Із зростанням числа робітників продуктивна сила суспільства
підвищується у складній пропорції, тобто пропорційно до цього зростання
числа робітників, помноженого на результат поділу праці між ними»
(«As the number of labourers increases, the productive power of society
augments in the compound ratio of that increase, multiplied by the
effects of the division of labour»). (\emph{Th. Hodgskin}: «Popular Political Economy»,
p. 125, 126).
} Але ця густота
людности є щось відносне. Країна, відносно мало залюднена,
з розвиненими засобами комунікації, має густішу людність, ніж
залюдненіша країна з нерозвиненими засобами комунікації, і з
цього погляду, приміром, північні штати Американського союзу
густіше залюднені, ніж Індія.\footnote{
Від 1861 p. в наслідок великого попиту на бавовну по деяких
взагалі дуже залюднених округах Східньої Індії поширилася продукція
бавовни коштом продукції рижу. В наслідок цього у більшій частині
країни настав голод, бо за браком засобів комунікації, тобто за браком
фізичного зв'язку, недостачу рижу в одній окрузі не можна було поповнити
довозом з інших округ.
}

\index{i}{0287}  %% посилання на сторінку оригінального видання
Що продукція та циркуляція товарів є загальна передумова
капіталістичного способу продукції, то мануфактурний поділ
праці вимагає достиглого вже до певного ступеня розвитку поділу
праці всередині суспільства. Навпаки, мануфактурний поділ
праці, з свого боку, впливає на суспільний поділ праці, розвиваючи
та розгалужуючи його. З диференціяцією інструментів
праці чимраз більше диференціюються й ті галузі промисловости,
які ці інструменти продукують.\footnote{
Так, у Голляндії фабрикація ткацьких човників уже в XVII ст.
становила осібну галузь промисловости.
} Якщо мануфактурне виробництво
захопить якийсь промисел, що досі був зв’язаний з іншими
як головний або підсобний промисел і виконувався тим самим
продуцентом, то відразу постає відокремлення та взаємне усамостійнення
цих реместв. Якщо воно захопить якийсь окремий
щабель продукції якогось товару, то різні щаблі його продукції
перетворюються на різні незалежні промисли. Ми вже зазначали,
що там, де продукт є лише механічно сполучена цілість частинних
продуктів, частинні праці сами можуть знову таки усамостійнитися
в окремі ремества. Для того, щоб досконаліше здійснити
поділ праці всередині якоїсь мануфактури, та сама галузь
продукції поділяється на різні, почасти цілком нові мануфактури,
залежно від ріжниці її сировинних матеріялів або тих різних
форм, яких може набирати той самий сировинний матеріял.
Так, у самій лише Франції вже в першій половині XVIII віку
ткали понад 100 різнорідних шовкових матерій, а в Авіньйоні,
приміром, було законом, що «кожний учень повинен посвячувати
себе цілком лише одному родові фабрикації й не смів вчитися
виготовлювати одночасно декілька родів продуктів». Територіальний
поділ праці, який окремі галузі продукції прив’язує
до окремих округ країни, набуває нового імпульсу від мануфактурного
виробництва, що експлуатує всякі особливості.\footnote{
«Хіба англійську вовняну мануфактуру не поділено на різні
частини або галузі, прив’язані до окремих місць, де продукується виключно
або переважно такі продукти: тонкі сукна у Сомерсетшірі, грубі —
в Йоркшірі, подвійної ширини — в Ексетері, шовк — у Сандбері, креп —
у Норвічі, піввовняні полотна — у Кендалі, покривала — в Уітнеї і
т. ін.?» («Whether the Woollen Manufacture of England is not divided
into several parts or branches appropriated to particular places, where
they are only or principally manufactured: fine cloths in Somersetshire,
coarse in Yorkshire, long ells at Exeter, saies and Sandbury, crapes at
Norwich, linseys at Kendal, blankets at Whitney, and so forth?»).
(\emph{Berkсley}: «The Querist», 1750, § 520).
} Для мануфактурного
періоду багатий матеріял для поділу праці всередині
суспільства дають поширення світового ринку й колоніальна
система, що належать до загальних умов існування мануфактурного
періоду. Тут не місце далі доводити, як цей поділ праці
\parbreak{}  %% абзац продовжується на наступній сторінці
