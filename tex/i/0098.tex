Світові гроші функціонують як загальний засіб платежу,
загальний засіб купівлі і абсолютно суспільна матеріялізація
багатства взагалі (universal wealth). Функція засобу платежу
для вирівнювання інтернаціональних балянсів,переважає. Звідси
гасло меркантильної системи — торговельний балянс! 109 Золото
й срібло служать за міжнародній засіб купівлі головним чином
тоді, коли раптом порушиться звичну рівновагу обміну речовин
між різними націями. Нарешті, вони функціонують як абсолютна
суспільна матеріялізація багатства тоді, коли йдеться не
про купівлю і не про платіж, а про перенесення багатства з однієї

дуже рідко 30—60 грамів золота. Срібло рідко трапляється в суцільному
вигляді, але зате в осібних рудах, що порівняно легко відділяються від жилової
породи й мають здебільшого від 40 до 90\% срібла; або воно міститься
в менших кількостях у мідяних, оливових рудах тощо, які вже сами собою
варті розроблення. Вже звідси виходить, що тимчасом як праця,
витрачувана на видобування золота, скорше збільшилась, праця, витралювана
на видобування срібла значно зменшилась, отже, і зниження
вартости срібла пояснюється цілком природно. Це зменшення вартости
виразилося б іще в більшому спаді цін, якби ціни на срібло ще й тепер
не підтримувано на високому рівні штучними засобами. Але срібні поклади
Америки покищо приступні до розроблення ще в малій частині, і таким
чином є всі надії на те, що вартість срібла ще довгий час падатиме. Падінню
вартости срібла ще більше мусить сприяти відносне зменшення попиту
на срібло для предметів споживання й розкошів, заміна срібла товарами
з накладного срібла, алюмінієм тощо. З цього можна оцінити утопізм
біметалевих уявлень, що, мовляв, міжнародній примусовий курс
знов піднесе срібло до старого відношення вартости 1: 15\sfrac{1}{2}. Скорше
срібло дедалі більш і більш утрачатиме свою грошову функцію на світовому
ринку. — Ф. Е.].

109 Противники меркантильної системи, яка за мету світової торговлі
вважала вирівнювання торговельного балянсу золотом'і сріблом, з свого
боку зовсім не розуміли функції світових грошей. Як неправильне розуміння
законів, що регулюють масу засобів циркуляції, лише відбивається
у неправильному розумінні міжнароднього руху благородних металів,
це я докладно показав на прикладі Рікарда («Zur Kritik der Politischen
Oekonomie», S. 150 і далі. — «До критики політичної економії», ДВУ
1926 р., стор. 186 і далі). Його помилкову догму: «Несприятливий торговельний
балянс завжди постає лише в наслідок надлишку засобів циркуляції..
Вивіз монети викликає її дешевина, і він є не наслідок, а причина
несприятливого балянсу» («An unfavourable balance of trade never
arises but from a redundant currency... The exportation of the coin is caused
by its cheapness, and is not the effect, but the cause of an unfavourable
balance»). (Ricardo: «The high Price of Bullion etc.», p. 11, 12, 14)
находимо ми вже в Барбона: «Торговельний балянс, коли такий є, не
є причина вивозу грошей за кордон з якоїсь країни, вивіз грошей відбувається
в наслідок ріжниці вартостей грошового металю в різних країнах»
(«The Balance of Trade, if there be one, is not the cause of sendingaway
the money out of a nation; but that proceeds from the difference of the value
of Bullion in every country»). (N. Bardon: «A Discourse on coining
the new money lighter», London 1696, p. 59). Мак Куллох «The Literature
of Political Economy, a classified catalogue», London 1845, хвалить
Барбона за цю антиципацію теорії Рікарда, алеж розсудливо оминає
навіть згадувати про ті наївні форми, що в них у Барбона ще виявляються
абсурдні припущення «currency principle». Безкритичність і навіть нечесність
цього каталога доходить апогею в розділах про історію теорії
грошей, бо Мак Куллох тут крутить хвостом як сикофант лорда Обер-
