рів як однойменні величини, якісно однакові й кількісно порівнянні.
Таким чином воно функціонує як загальна міра вартостей,
і, насамперед, лише через цю функцію золото, цей специфічний
еквівалентний товар, стає грішми.

Не гроші роблять товари спільномірними. Навпаки, через
те, що всі товари як вартості є упредметнена людська праця, отже,
сами по собі спільномірні, вони можуть міряти свої вартості
тим самим специфічним товаром, перетворюючи його таким чином
на їхню спільну міру вартости, або на гроші. Гроші як міра
вартости є доконечна форма виявлення іманентної товарам міри
вартости — робочого часу.\footnote{
Питання, чому гроші не репрезентують безпосередньо самого робочого
часу, — так, щоб, приміром, якась банкнота репрезентувала х робочих
годин, — сходить просто на питання, чому на основі товарової продукції
продукти праці мусять набирати форм товарів, бо товарова форма
включає роздвоєння товару на товар і грошовий товар; або на інше питання:
чому приватну працю не можна розглядати як безпосередню суспільну
працю, тобто як протилежність приватної. В іншому місці я докладно
розібрав плаский утопізм теорії «робочих грошей» на основі товарової
продукції («Zur Kritik der Politischen Oekonomie», S. 61 ff. — «До критики
політичної економії», ДВУ, 1926 р., стор. 98 і далі). Тут зауважу лише,
що, наприклад, «робочі гроші» Оуена так само мало є гроші, як, приміром,
театральний квиток. Оуен припускає як передумову безпосередньо
усуспільнену працю, форму продукції, діяметрально протилежну товаровій
продукті. Робочий сертифікат у нього лише констатує індивідуальну
участь продуцента у спільній праці та його індивідуальні вимоги щодо споживання
певної частини спільного продукту. Але Оуенові не спадало й на
думку припускати, з одного боку, товарову продукцію і, з другого боку, все ж
намагатися обійти її доконечні умови за допомогою грошових фокусів.
}

Вираз вартости якогось товару в золоті — х товару А = у
грошового товару — є його грошова форма, або його ціна. Тепер
досить одним-одного рівнання, як от: 1 тонна заліза = 2 унціям
золота, щоб подати вартість заліза в суспільно-визнаній формі.
Цьому рівнанню не треба вже більше бути ланкою в ряді рівнань
вартости інших товарів, бо еквівалентний товар, золото, має
вже характер грошей. Тому загальна відносна форма вартости
товарів має тепер знову вигляд своєї первісної, простої або одиночної
відносної форми вартости. З другого боку, розгорнутий
відносний вираз вартости або безкраїй ряд відносних виразів
вартости стає специфічною відносною формою вартости грошового
товару. Але цей ряд тепер уже суспільно даний у цінах
товарів. Читайте в прайскуранті спершу ціни, а потім назви
товарів — і ви знайдете величину вартости грошей, виражену
в усіх можливих товарах. Навпаки, гроші не мають жодної
ціни. Щоб узяти участь у цій однорідній відносній формі вартости
інших товарів, вони мусили б відноситись до самих себе як до
свого власного еквіваленту.

Ціна, або грошова форма товарів, як і взагалі їхня форма вартости,
є відмінна від їхньої видимої, реальної тілесної форми, отже,
вона є лише ідеальна, або уявна форма. Вартість заліза, полотна,
пшениці й т. ін. існує, хоч і невидимо, в самих цих речах; вона
виявляється в рівності їх із золотом, у їхнім відношенні до золота,