6. Боротьба за нормальний робочий день. Примусове законодавче
обмеження робочого часу. Англійське фабричне законодавство
від 1833 до 1864 р.

Після того, як капіталові треба було століть, щоб здовжити
робочий день до його нормальних максимальних меж, а потім і
поза ці межі, до меж природного дванадцятигодинного дня,131
почалося від часів народження великої промисловости в останній
третині XVIII віку лявиноподібне, ґвалтовне й безмірне перекидання
всіх перепон. Всі межі, які ставили звичаї й природа,
вік і стать, день і ніч — всі їх геть поруйновано. Навіть поняття
про день і ніч, по-селянському прості у старих статутах, так
розпливлися, що одному англійському судді ще 1860 р. треба
було вжити дійсно талмудичного мудрування, щоб пояснити
силою «судового присуду», що таке день і що таке ніч. 132 Капітал
святкував свої оргії.

Скоро тільки приголомшена гуркотом продукції робітнича
кляса сяк-так знов отямилась, вона почала ставити опір, спочатку
на батьківщині великої індустрії, в Англії. Однак протягом
трьох десятиліть поступки, що їх вона здобула впертою боротьбою,
лишилися суто номінальні. За час від 1802 й до 1833 р.
парлямент видав п’ять законів про працю, але був настільки
хитрий, що не вотував жодного шеляга на примусове їх запровадження,
на потрібний персонал урядовців і т. ін.133 Закони лиши-

у Брюсселі,12 травня 1862 р. доводить до відома Foreign Office*: «Міністер
Рожіє заявив мені, що дитячої праці ніяк не обмежує ні загальний закон,
ні місцеві постанови; що уряд протягом трьох останніх років кожної сесії
мав думку внести до палати законопроєкта з цього приводу, але він завжди
натрапляв на непереможну перешкоду в ревнивому остраху перед
всяким законодавством, що суперечить принципові повної волі праці»!

131 «Це безперечно жалюгідний факт, що якась кляса людей мусить
мучитися коло праці по 12 годин на день. Коли залічити сюди ще обідній
час і час, потрібний на те, щоб дійти до майстерні й назад, то це в дійсності
становитиме 14 із 24 годин на добу... Залишаючи навіть здоров’я осторонь,
я сподіваюся, що ніхто не зважиться заперечувати, що з морального
погляду таке цілковите пожирання часу трудящих кляс, безперестанно
від раннього віку, від 13 року життя, а по «вільних» галузях промисловости
навіть від значно ранішого віку, є надзвичайно шкідливе й страшне
лихо... В інтересах суспільної моралі, щоб виховати дужу людність
і щоб забезпечити великій масі народу змогу розумно користуватися
життям, треба вперто домагатися того, щоб по всіх галузях промисловости
від кожного робочого дня лишалась якась частина на відпочинок і дозвілля».
(Leonhard Horner у «Reports of Insp. of Fact, for 31 st December
1841»).

132 Див. «Judgment of Mr. J. H. Otway, Belfast, Hilary Sessions,
County Antrim 1860».

133 Дуже характеристичний для режиму Люї-Філіпа, цього короля-буржуа,
є той факт, що однісінький виданий при ньому фабричний
закон з 22 березня 1841 р. ніколи не був запроваджений у життя. Та й цей
закон стосується лише до дитячої праці. Він визначає вісім годин працідля
дітей 8—12 років, дванадцять годин — для дітей 12—16 років і
т. ін., але допускає багато винятків, які дозволяють нічну працю навіть
для восьмилітніх дітей; догляд за виконанням цього закону і примус

* — англійського міністерства закордонних справ. Ред.
