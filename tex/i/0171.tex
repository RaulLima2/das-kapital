них умов щоденне споживання бавовни, машин і т. ін. зменшиться на 1\sfrac{1}{2} години. Отже, ви виграєте
рівно стільки, скільки втрачаєте. На майбутнє ваші робітники будуть витрачати на репродукцію
або покриття авансованої вартости капіталу на 1\sfrac{1}{2} години менше. А коли б він не повірив їм на
слово і, як тямуща у справі людина, визнав би за потрібне аналізу, то мусив би насамперед попрохати
панів фабрикантів, щоб у питанні, яке стосується
виключно відношення чистого прибутку до величини робочого дня, вони не плутали в одну купу машини й
фабричні будівлі,
сировинний матеріял і працю, а були ласкаві поставити сталий капітал, вміщений у фабричних будівлях,
машинах, сировинному матеріялі тощо, по один бік, а капітал, авансований на заробітну
плату, — по другий бік. Коли б тоді виявилося, приміром, що за обчисленням фабрикантів робітник за
\sfrac{2}{2} робочої години, або за 1 годину, репродукує або покриває заробітну плату, то нашому
аналітикові треба було б далі так казати:

За вашими даними, робітник за передостанню годину продукує свою заробітну плату, а за останню — вашу
додаткову вартість або чистий прибуток. А що протягом однакового часу він продукує
однакові вартості, то продукт передостанньої години має таку
саму вартість, що й продукт останньої. Далі, він продукує вартість лише остільки, оскільки він
витрачає працю, і кількість
його праці вимірюється його робочим часом. Останній, за вашими даними, становить 11\sfrac{1}{2} годин на
день. Одну частину з цих 11\sfrac{1}{2} годин
він витрачає на продукцію, або на покриття своєї заробітної плати, другу частину — на продукцію
вашого чистого прибутку. Чогось іншого він не робить протягом дня. А що, за вашими даними, його
заробітна плата й постачена від нього додаткова вартість є рівновеликі вартості, то він, очевидно,
продукує свою заробітну плату за 5\sfrac{3}{4} годин, а ваш чистий прибуток за другі
5\sfrac{3}{4} годин. А що, далі, вартість пряжі, спродукованої за дві години, дорівнює сумі вартости його
заробітної плати плюс ваш
чистий прибуток, то ця вартість пряжі мусить вимірятися 11\sfrac{1}{2} робочими
годинами, продукт передостанньої години мусить вимірятися 5\sfrac{3}{4} робочими годинами, продукт останньої
години — так
само. Ми дійшли тут до дражливого пункту. Отже, увага! Передостання робоча година є така сама
звичайна робоча година, як і
перша. Ni plus, ni moins. * То ж яким чином прядун міг би за одну робочу годину спродукувати
вартість пряжі, що репрезентує 5\sfrac{3}{4} робочих годин? В дійсності він і не робить такого дива. Та
споживна вартість, яку він продукує за одну робочу годину, є певна кількість пряжі. Вартість цієї
пряжі вимірюється 5\sfrac{3}{4} робочими годинами, з яких 4\sfrac{3}{4} вже без його допомоги містяться в спожитих
протягом години засобах продукції, в бавовні, в машинах і т. ін., а \sfrac{4}{4}, або 1 годину, додав він
сам. Отже, що його заробітну плату продукується за 5\sfrac{3}{4} годин, а пряжа, сродукована за одну годину,
так само містить у собі 5\sfrac{3}{4} робочих годин, то в тому зовсім немає

* Не більше й не менше. Ред.
