\parcont{}  %% абзац починається на попередній сторінці
\index{i}{0292}  %% посилання на сторінку оригінального видання
знову відбудовуються на тому самому місці з тією самою назвою,\footnote{
«У цій простій формі\dots{} мешканці країни жили від споконвічних
часів. Межі сел змінялися рідко; і хоч сами села інколи й зазнавали
ушкоджень, а то навіть і цілковитого спустошення в наслідок воєн,
голоду та пошестей, а все ж вони й далі існували цілі віки під тією самою
назвою, в тих самих межах, з тими самими інтересами, і навіть з тими
самими родинами. Загибель або поділ держав мало турбує мешканців;
доки село лишається цілим, їм байдуже, під чиєю владою воно опинилося,
якому суверенові воно підлягає; їхнє внутрішнє економічне життя
лишається незмінне». («Under this simple form\dots{} the inhabitants of the
country have lived since time immemorial. The boundaries of the villages
have been but seldom altered; and though the villages themselves have
been sometimes injured, and even desolated by war, famine, and disease,
the same name, the same limits, the same interests, and even the same
families, have continued for ages. The inhabitants give themselves non
trouble about the breaking up and division of kingdoms; while the village
remains entire, they care not to what power it is transferred ort to what
sovereign it devolves; its internal economy remains unchanged»). (\emph{Th. Stamford
Raffles}, late Lieut. Gov. of Java: «The History of Java», London 1817.
vol. II, p. 285).
} дає нам ключ до зрозуміння таємниці незмінности азійських
суспільств, незмінности, яка так гостро контрастує з постійним
розпадом і новоутворенням азійських держав та безперервною
зміною династій. Бурі, що відбуваються в хмарній царині політики,
не зачіпають структури основних економічних елементів
суспільства.

Як це вже раніше зазначалось, цехові закони пляномірно
перешкоджали перетворенню цехового майстра на капіталіста,
точно обмежуючи число підмайстрів, яких міг тримати поодинокий
майстер. Так само майстер міг вживати підмайстрів виключно
в тому реместві, де він сам був майстром. Цех із запалом
боронився проти усяких замахів купецького капіталу, —
єдиної вільної форми капіталу, що протистояла йому. Купець
міг купити всякі товари, та тільки не працю як товар. Його терпіли
лише як скупника-розповсюдника (Verleger) продуктів
ремества. Якщо зовнішні обставини викликали поступ у розвитку
поділу праці, то наявні цехи розколювались на підроди абож
поряд старих засновувалися нові цехи, однак не об’єднуючи різних
реместв в одній майстерні. Тим то цехова організація, хоч і в
якій великій мірі зумовлювані нею відокремлення, ізоляція й
розвиток реместв належать до матеріяльних умов існування
мануфактурного періоду, все ж виключала мануфактурний поділ
праці. Взагалі і в цілому, робітник і його засоби продукції
лишались зв’язані одне з одним так само, як слимак із своєю
шкаралупою, і таким чином бракувало першої основи мануфактури
— усамостійнення засобів продукції як капіталу супроти
робітника.

Тим часом як поділ праці в цілому суспільстві, хоч упосереднюється
він товаровим обміном, хоч ні, належить до найрізнорідніших
економічних суспільних формацій, мануфактурний
поділ праці є цілком специфічний витвір капіталістичного способу
продукції.
