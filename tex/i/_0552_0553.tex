\parcont{}  %% абзац починається на попередній сторінці
\index{i}{0552}  %% посилання на сторінку оригінального видання
є суперечність самого руху капіталу. Він потребує більших мас робітників молодшого віку, менших мас
— дорослого. Ця суперечність не більш кричуща за ту другу, що нарікають на брак робочих рук у той
самий час, коли багато тисяч викинуто на
брук через те, що поділ праці приковує їх до якоїсь певної галузі продукції.\footnote{
Тимчасом як протягом останнього півріччя 1866 р. в Лондоні лишилося без роботи 80.000--90.000
робітників, у фабричному звіті про це саме півріччя читаємо: «Здається, не зовсім правда, що попит
створює подання саме в ту хвилину, коли це потрібно. Щодо праці справа стояла інакше, бо протягом
останнього року багато машин не працювало через брак рук». («It does not appear absolutely true to
say that demand
will always produce supply just at the moment when it is needed. It has not done so with labour, for
much machinery had been idle last year for want of hands»). («Report of Insp. of Fact. for 31st
October 1866», p. 81).
} До того ж, капітал
споживає робочу силу так швидко, що здебільшого робітник середнього віку є вже більше або менше
виснажений. Він попадає в ряди зайвих або його витискують
із вищого щабля на нижчий щабель. Саме в робітників великої промисловости ми натрапляємо на
найкоротший протяг життя. «Д-р. Лі, санітарний урядовець Менчестеру, сконстатував, що в тому місті
пересічний протяг життя заможної кляси
38 років, а робітничої кляси — лише 17 років. У Ліверпулі він становить для першої кляси 35 років,
для другої — 15. Отже, з цього випливає, що упривілейована кляса має асиґнату на життя (have a lease
of life) понад удвоє більшу, ніж її менш щасливі співгромадяни».\footnoteA{
Промова, що її виголосив на відкритті санітарної конференції в Бермінґемі 15 січня 1875 р. Дж.
Чемберлен, тодішній мер міста, теперішній (1883) міністер торговлі.
} За цих обставин для абсолютного
зростання цієї частини пролетаріяту потрібна така форма, при якій чисельність її зростала б, не
зважаючи на швидке виснажування її елементів. Отже, потрібна швидка зміна поколінь робітників.
(Цей закон не має сили для решти кляс людности).
Цю суспільну потребу задовольняється ранніми шлюбами, — неминучий наслідок відносин, серед яких
живуть робітники великої промисловости, — і тією премією, яку дає експлуатація дітей робітників за
продукцію їх.

Скоро тільки капіталістична продукція опановує рільництво, або в міру того, як вона опановує
рільництво, попит на сільську робітничу людність абсолютно меншає з акумуляцією капіталу, що тут
функціонує, так що відштовхування робітничої людности
тут не доповнюється, як у нерільничій промисловості, більшим притяганням. Тому частина сільської
людности завжди готова перейти в ряди міського або мануфактурного пролетаріяту і лише вичікує
сприятливих умов для цього перетворення. (Слова мануфактура тут уживається в розумінні всякої
нерільничої
промисловости).\footnote{
За переписом 1861 р. в Англії і Велзі налічувалося «781 місто з 10.960.998 жителями, тимчасом як
по селах і сільських парафіях налічувалося лише 9.105.226 жителів\dots{} У перепису 1851 р. фігурувало
580 міст, що їх людність приблизно дорівнювала людності прилеглих до них
сільських округах. Але тим часом, як у сільських округах людність протягом наступних десятьох років
зросла лише на півмільйона, в 580 містах вона зросла на 1.554.067. Приріст людности по селах
становить 6,5\%, по містах — 17,3\%. Ріжниця в нормі приросту є наслідок переселення з сел до міст.
Три чверті загального приросту людности припадає на міста». («Census etc.», vol. III, p. 11, 12).
} Отже, це джерело відносного перелюднення
\index{i}{0553}  %% посилання на сторінку оригінального видання
б’є безперервно. Але постійний приплив до міст має своєю передумовою постійне лятентне перелюднення
в самих селах, що його розміри стають помітні лише тоді, коли вивідні канали відкриваються винятково
широко. Тим то плату сільського робітника
знижують до мінімуму і він завжди стоїть однією ногою в болоті павперизму.

Третя категорія відносного перелюднення, застійна, становить частину активної робітничої армії, але
разом із тим надзвичайна нереґулярність її занять дає капіталові невичерпний резервуар вільної
робочої сили. Її життєве становище падає нижче за
пересічний нормальний рівень робітничої кляси, і саме це робить її широкою основою осібних галузей
капіталістичної експлуатації. Її характеризують максимум робочого часу й мінімум заробітної плати.
Ми вже вивчили під рубрикою домашньої праці її головну форму. Вона рекрутується постійно із зайвих
робітників великої промисловости й рільництва, і особливо також з робітників галузей промисловости,
що гинуть, тих галузей, де ремісничу продукцію перемагає мануфактурна, мануфактурну — машинова. Її
розміри більшають у міру того, як з розмірами та енерґією акумуляції проґресує «творення зайвих»
робітників. Але разом з тим вона становить той елемент робітничої кляси, який сам себе репродукує й
увіковічнює і який бере порівняно більшу участь у загальному зростанні робітничої кляси, ніж решта
її елементів. Справді, не тільки число народжень і випадків смерти, але й абсолютна величина родин є
зворотно пропорційна до височини заробітної плати, отже, і до маси засобів існування, що ними
порядкують різні категорії робітників. Цей закон капіталістичного суспільства звучав би якимсь
безглуздям серед дикунів, а то й навіть серед цивілізованих колоністів. Він нагадує нам про масову
репродукцію індивідуально малосильних і жорстоко переслідуваних видів тварин.\footnote{
«Злидні\dots{} здається\dots{} сприяють розмножуванню» («Poverty\dots{} seems\dots{} favourable to generation»).
(\emph{A. Smith}: «Wealth of Nations», b. I, ch. 8, p. 195). На думку ґалянтного й дотепного абата Ґаліяні
це навіть надзвичайно мудра установа божа: «Господь зробив так, що людей, які виконують дуже корисну
роботу, родиться найбільше» («Iddio fa che
gli uomini che esercitano mestieri di prima utilità nasconoabbondantemente»). (\emph{Galiani}: «Della
Moneta», vol. III збірки Custodi «Scrittori Clas sici Italiani di Economia Politica». Parte Moderna.
Milano 1801, p. 78). «Злидні аж до крайніх меж голоду й епідемій не гальмують зросту людности, а
мають тенденцію абільшувати її» («Misery, up to the extreme point of famine and pestilence, instead
of checking, tends to increase population»).
(\emph{S. Laing}: «National Distress», 1844, p. 69). Зілюструвавши це статистичними даними, Лен каже далі:
«Коли б усі жили в достатках, то земля швидко лишилася б без людей» («If the people were all in easy
circumstances, the world would soon be depopulated»).
}

