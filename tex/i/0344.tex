ного робочого дня, все ж його знову значно перевищив дальший
шестирічний період від 1856 до 1862 р. Наприклад, 1856 р. по
шовкових фабриках було 1.093.799 веретен, а 1862 р. — 1.388.544;
ткацьких варстатів 1856 р. було 9.260, а 1862 р. — 10.709. Навпаки,
число робітників 1856 р. становило 56.131, а 1862 р. — 52.429.
Це дає збільшення числа веретен на 26,9\% і ткацьких варстатів
на 15,6\% при одночасному зменшенні робітників на 7\%. 1850 р. на
фабриках суканої вовни було в ужитку 875.830 веретен, 1856 р. —
1.324.549 (приріст на 51,2\%), а 1862 р. — 1.289.172 (зменшення
на 2,7\%). А коли відлічити веретена на сукання (Dublierspindeln),
які фігурують у переліку за 1856 р., але не фігурують у переліку
за 1862 р., то число веретен від 1856 р. майже не змінилося. Навпаки,
швидкість веретен і ткацьких варстатів від 1850 р. в
багатьох випадках подвоїлася. Число парових ткацьких варстатів
по фабриках суканої вовни 1850 р. становило 32.617,
1856 р. — 38.956, а 1862 р. — 43.048. Коло них працювало 1850 р.
79.737 осіб, 1856 р. — 87.794, а 1862 р. — 86.063, та з них було
дітей, молодших за 14 років: 1850 р. — 9.956, 1856 р. — 11.228,
а 1862 р. — 13.178. Отже, не зважаючи на значне збільшення
числа ткацьких варстатів у 1862 р. порівняно з роком 1856, загальне
число вживаних робітників зменшилось, а число експлуатованих
дітей зросло.174

27 квітня 1863 р. член парляменту Ферранд ось що заявив у
Палаті громад: «Делеґати робітників із 16 округ Ланкашіру й
Чешіру, з доручення яких я говорю, повідомили мене, що в наслідок
поліпшення машин праця на фабриках постійно зростає.
Раніш одна особа з помічниками обслуговувала два ткацькі верстати,
тепер одна особа без помічників обслуговує три варстати,
а часто-густо й чотири варстати і т. д. Як це видно з поданих
фактів, дванадцять годин праці стиснуто тепер менше, ніж у
10 робочих годин. Тому само собою зрозуміло, до якого величезного
розміру зросла тяжкість праці фабричних робітників останніми
роками». 175

Тому, хоч фабричні інспектори невтомно та з повним правом
вихваляють сприятливі результати законів 1844 та 1850 рр.,
все ж вони визнають, що скорочення робочого дня викликало вже

174 «Reports of Insp. of Fact, for 31 st October 1862», p. 100 і 130.

175 За допомогою сучасного парового варстату один ткач на двох
варстатах фабрикує тепер за 60 годин на тиждень 26 сувоїв певного сорту
тканини певної довжини та ширини, а раніш на старому паровому ткацькому
варстаті він міг продукувати лише 4 такі сувої. Витрати на ткання
одного такого сувою вже на початку 1850-х років спали з 2 шилінґів
9 пенсів до 5 1/8 пенсів.

Додаток до другого видання: «Перед З0 роками (1841) від одного прядуна
бавовняної пряжі з трьома помічниками вимагали доглядати лише
за однією парою мюлів із 300—324 веретенами. Тепер (кінець 1871 р.)
він з п’ятьма помічниками має доглядати мюлів, що їхнє число веретен
становить 2.200, та продукує, щонайменше, всемеро більше пряжі, ніж
1841 р.». (Alexander Redgrave, фабричний інспектор, у «Journal of
the Society of Arts», 5 Januar 1872).
