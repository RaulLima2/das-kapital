\parcont{}  %% абзац починається на попередній сторінці
\index{i}{0186}  %% посилання на сторінку оригінального видання
законом час, ви вкладали б мені до кишені річно 1.000\pound{ фунтів
стерлінґів}».\footnote{
Там же, стор. 48.
} «Атоми часу є елементи баришу».\footnote{
«Moments are the element of profit». («Reports of the Insp. etc.
30 th April 1860», p. 56).
}

З цього боку немає нічого характеристичнішого, як назва
«full times»\footnote*{
— повний час. \emph{Ред.}
} для робітників, що працюють повний час, і «half
times»\footnote*{
— половина часу. \emph{Ред.}
} для дітей до тринадцятилітнього віку, яким дозволяється
працювати лише по 6 годин.\footnote{
Цей вислів має офіціальне право громадянства так на фабриці,
як і у фабричних звітах.
} Робітник є тут не що більше,
як персоніфікований робочий час. Усі індивідуальні ріжниці
сходять на ріжницю між «Vollzeitler» і «Halbzeitler».\footnote*{
— робітником повночасним і робітником півчасним. \emph{Ред.}
}

\subsection{
Галузі англійської промисловости без законодавчих меж
експлуатації
}

Досі ми розглядали прагнення здовжувати робочий день,
ненажерливий вовчий голод за додатковою працею, на такому
полі, де безмірні зловживання, не перевищені і навіть — як каже
один буржуазний англійський економіст, — жорстокостями еспанців
проти червоношкурих Америки,\footnote{
«Ненажерливість власників фабрик призводить до того, що в погоні
за баришем вони допускаються таких жорстокостей, яких ледве чи
перевищили жорстокості еспанців підчас завойовування Америки в гонитві
за золотом» («The cupidity of mill-owners, whose cruelties in pursuit
of gain, have hardly been exceeded by those perpetrated by the Spaniards
on the conquest of America in the pursuit of gold»). (John Wade:
«History of the Middle and Working Classes», 3 id ed. London 1835, p-114).
Теоретична частина цієї книги, свого роду нарис політичної економії,
містить у собі дещо оригінальне для свого часу, приміром, про торговельні
кризи. Щождо історичної частини, то вона є безсоромний пляґіят із Sir
М. Eden: «History of the Poor», London 1799.
} спричинилися, нарешті,
до того, що капітал закували у ланцюги законодавчого регулювання.
А тепер киньмо оком на деякі галузі промисловости, де
висисання робочої сили або ще й тепер вільне від тих законодавчих
пут, або було таким ще зовсім недавно.

«Пан Бровтон, суддя графства, як голова мітингу, який
відбувся в нотінгемському міському будинку 14 січня 1860 р.,
заявив, що серед частини міської людности, занятої виробництвом
мережива, панують такі страшні злидні й нужда, що решта цивілізованого
світу ще таких не знає\dots{} О 2, 3, 4 годині ранку 9 —
10-літніх дітей виривають із їхніх брудних ліжок і примушують
тільки за мізерний харч працювати до 10,11,12 години вночі,
в наслідок чого нидіють їхні члени, корчиться тіло, тупіють риси
їх обличчя, і їхнє ціле людське єство дубіє в німій нерухомості, на
яку навіть глянути страшно. Це для нас не диво, що пан Малет
і інші фабриканти виступили з протестом проти всякої дискусії.
Система, як її описав панотець Монтегіо Вальпі, — це система безмежного
\index{i}{0187}  %% посилання на сторінку оригінального видання
рабства, — рабства з кожного погляду, соціяльного, фізичного,
морального й інтелектуального\dots{} Що подумати про місто,
яке скликає прилюдний мітинг на те, щоб просити про обмеження
робочого часу для чоловіків на 18 годин на добу!.. Ми деклямуємо
проти вірджінських і каролінських плянтаторів. Але хіба їхня
торговля неграми з усіма страхіттями батога й баришування людським
м’ясом огидніша, ніж це повільне душогубство, яке відбувається
для того, щоб на користь капіталістам вироблялося
серпанки й комірчини?»\footnote{
«London Daily Telegraph», з 17 січня 1860 р.
}

Ганчарня (Pottery) Стафордшіру була протягом останніх
22 років предметом трьох парляментських слідств. Результати
цих слідств наведено у звіті пана Скрайвена з 1841 р., поданому
членам «Children’s Employment Commission», у звіті д-ра
Ґрінхов з 1860 р., опублікованому за розпорядженням лікарського
урядовця Privy Council («Public Health», 3 rd Report,
I, 112--113), нарешті, y звіті пана Льон Ге з 1863 р., у «First
Report of the Children’s Employment Commission» з 13 червня
1863 p. Для мого завдання досить узяти із звітів з 1860 і 1863 рр.
деякі свідчення дітей, що сами були об’єктом експлуатації. Із
становища дітей можна робити висновки й про становище дорослих,
а особливо дівчат і жінок, і до того ж в такій галузі промисловости,
поруч з якою бавовнопрядіння й т. ін. може видаватися
дуже приємною й здоровою працею.\footnote{
Порівн. Engels: «Lage der arbeitenden Klasse in England»,
Leipzig 1845, S. 249--251. (Енгельс: «Становище робітничої кляси в
Англії». Партвидав «Пролетар», 1932 р., стор. 233--236).
}

Вільгельм Вуд, дев’яти років, «почав працювати, мавши
7 років 10 місяців». Спочатку він був «van moulds» (носив до
сушні виготовлений товар у формах і приносив назад порожні
форми). Цілий тиждень день-у-день приходив о 6 годині вранці
й кінчав роботу так десь коло 9 години вечора. «Я цілий тиждень
працюю щодня до 9 години вечора. Так було, приміром, протягом
останніх 7--8 тижнів». Отже, п’ятнадцять годин праці для семилітньої
дитини! Дж. Меррей, дванадцятилітнє хлоп’я, свідчить:
«І run moulds and turn jigger (я ношу форми та кручу колесо).
Я приходжу о 6, іноді о 4 годині вранці. Я працював цілу останню
ніч до 8 години сьогоднішнього ранку. Я не спав від минулої
ночі. Крім мене працювало ще 8 або 9 хлопчиків цілу останню
ніч без перерви. За винятком одного, всі знов прийшли сьогодні
вранці. Я дістаю 3 шилінґи 6 пенсів (1 таляр 5 шагів) на тиждень.
Я не дістаю більше, коли працюю цілу ніч. Останнього тижня
я працював дві ночі». Фернігав, десятилітнє хлоп’я: «Мені не
завжди лишається ціла година на обід, часто лише півгодини;
це буває щочетверга, п’ятниці й суботи».\footnote{«Children’s Employment Commission. 1 st Report etc. 1863», Appendix,
p. 16, 19, 18.
}

Д-р Ґрінхов заявляє, що вік життя в ганчарняних округах
Stoke-upon-Trent і Wolstanton надзвичайно короткий. Хоч в
\parbreak{}  %% абзац продовжується на наступній сторінці
