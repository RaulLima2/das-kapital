кількість грошей в обігу, або грошового матеріялу, залежить від
його власної вартости. Ілюзія, що товарові ціни, навпаки, визначаються
масою засобів циркуляції, а ця остання, своєю чергою,
масою грошового матеріялу, що є в країні,\footnote{
«Ціни продуктів кожної нації, певна річ, зростатимуть у міру
того, як зростатиме кількість золота й срібла серед народу; отже, коли
кількість золота й срібла, яку має якась нація, меншає, то й ціни на всі
продукти мусять спадати пропорційно до цього зменшення кількости грошей»
(«The prices of things will certainly rise in every nation, as the gold
and silver increase amongst the people: and, consequently, where the gold
and silver decrease in any nation, the prices of all things must fall proportionably
to such decrease of money»). (Jacob Vanderlint: «Money answers
all Things», London 1734, p. 5). Ближче порівнання праці Вандерлінта
й «Essays» Юма не лишає в мені найменшого сумніву, що Юм
знав і використав цей, зрештою, значний твір Вандерлінта. Погляд, що
маса засобів циркуляції визначає ціни, находимо і в Барбона та ще давніших
письменників. «Жодної невигоди, — каже Вандерлінт, — не може
постати з нічим не обмеженої торговлі, а тільки велика користь... бо коли
грошова готівка якоїсь нації зменшуватиметься під впливом вільної
торговлі, — а цьому повинні стати на перешкоді заборони, — то в тих націй,
що до них припливає готівка, неминуче зростатимуть ціни на всі
речі відповідно до збільшення грошової готівки. А... продукти наших мануфактур
і всякі інші продукти дійдуть тут швидко таких низьких цін,
що торговельний балянс повернеться на нашу користь і гроші знову припливатимуть
назад до нас». («No inconvenience can arise by an unrestrained
trade, but very great advantage... since, if the cash of the nation be decreased
by it, which prohibitions are designed to prevent, those nations that
get the cash will certainly find every thing advance in price, as the cash
increases amongst them. And... our manufactures and every thing else, will
soon become so moderate as to turn the balance of trade in our favour, and thereby
fetch the money back again»). («Money answers all Things», p. 44).
* — з обов’язку. Peд.
} має свої коріння у

is to be like wise taken from the frequency of commutations, and from
the bigness of payments»). (William Petty: «A. Treatise on Taxes and
Contributions», London 1667, p. 17). Теорію Юма боронив супроти
Дж. Стюарта й інших А. Юнґ у своїй «Political Arithmetic», London
1774, де їй присвячено окремий розділ: «Prices depend on quantity of
money», p. 112 і далі. У своїй «Zur Kritik der Politischen Oekonomie»
S. 149 (ДВУ, 1926 p., стор. 178) я роблю таку увагу: «Питання про кількість
монет, що циркулюють, він (А. Сміс) мовчки усуває, цілком неправильно
розглядаючи гроші як простий товар». Але це стосується лише
до тих місць, де А. Сміс розглядає гроші ex officio.* Однак, принагідно,
приміром, у критиці попередніх систем політичної економії, він висловлюється
правильно: «Кількість грошей у кожній країні регулюється вартістю
тих товарів, що їх вони пускають у рух... Вартість благ, що їх щорічно
купують і продають у якійсь країні, потребує певної кількости грошей
для того, щоб пустити блага в обіг і розподілити поміж їхніми споживачами;
вжити більшої суми грошей не можна. Канал циркуляції
неминуче притягає до себе суму, якої досить для того, щоб наповнити
його, і ніколи не допускає жодного надміру». («The quantity of coin in
every country is regulated by the value of the commodities which are to
be circulated by it... The value of goods annually bought and sold in any
country requires a certain quantity of money to circulate and distribute
them to their proper consumers, and can give employment to no more. The
channel of circulation necessarily draws to itself a sum sufficient to fill
it, and nevet admits any more»). (Wealth of Nations, 1. 5, ch. 1). Подібно
до цього А. Сміс починає свій твір ex officio апотеозою поділу праці. Наприкінці,
в останній книзі про джерела державних доходів, він принагідно
поновлює напади свого вчителя А. Ферґюсона на поділ праці.