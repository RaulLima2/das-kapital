\parcont{}  %% абзац починається на попередній сторінці
\index{i}{0319}  %% посилання на сторінку оригінального видання
з якою обертається веретено, або від числа вдарів, що їх молот
робить за одну хвилину. Деякі з тих колосальних молотів дають
70 ударів, Ryder’ова патентована ковальська машина, що вживає
парового молота невеликих розмірів на кування веретен, дає
700 ударів на одну хвилину.

Якщо дано ту пропорцію, в якій машина переносить вартість
на продукт, то величина цієї частини вартости залежить від величини
вартости самої машини.\footnote{
Читач, полонений капіталістичними уявленнями, певна річ, дивується,
що тут немає мови про «процент», що його машина, пропорційно
до своєї капітальної вартости, додає до продукту. Однак, легко зрозуміти,
що машина, — тому що вона, як і будь-яка інша складова частина
сталого капіталу, не створює нової вартости, — не може додавати такої
вартости і під назвою «процент». Далі, ясно, що тут, де йдеться про продукування
додаткової вартости, не можна жодної частини її припустити
a priori під назвою «процент». Капіталістичний спосіб обчислення,
який prima facie* видається безглуздим та суперечним законам утворення
вартости, ми пояснимо в третій книзі цього твору.
} Що менше праці вона сама містить
у собі, то менше вартости додає вона до продукту. Що менше
вартости віддає вона, то продуктивніша вона й то більш наближається
її служба до служби сил природи. А продукція машин за
допомогою машин зменшує їхню вартість порівняно з їх розмірами
і їхньою дією.

Порівняльна аналіза цін на товари, продуковані ремісничим
або мануфактурним способом, та цін на ті самі товари як продукти
машин, дає взагалі такий результат, що в машиновому продукті
складова частина вартости, яку до нього додає засіб праці, відносно
зростає, але абсолютно меншає. Це значить, що її абсолютна
величина меншає, але її величина супроти загальної вартости
продукту, наприклад, одного фунта пряжі, більшає.\footnote{
Ця додавана машиною складова частина вартости падає абсолютно
й відносно там, де машина витискує коні, взагалі робочу худобу,
уживану виключно як рухову силу, а не як машини для обміну речовин.
До речі зауважимо, що Декарт, визначаючи тварини як прості машини,
дивиться очима мануфактурного періоду, відмінно від середньовіччя,
яке вважало тварину за помічника людини так само, як пізніше й пан
фон Галлер в його «Restauration der Staatswissenschaften». Що Декарт
так само, як і Бекон, розглядав зміну способу продукції та практичне
опанування природи людиною як результат зміненої методи думання, —
це показує його «Discours de la Méthode», де, між іншим, сказано: «Можна
(за допомогою методи, яку він увів у філософію) дійти знань, дуже корисних
у житті, і замість тієї спекулятивної філософії, якої навчають по школах,
знайти практичну філософію, за допомогою якої, знаючи силу та дію
огню, води, повітря, зірок і всіх інших навкольних тіл так само достеменно,
як ми знаємо різні ремества наших ремісників, ми могли б тим
самим способом вживати їх для всього того, на що вони придатні, і таким
чином зробитися хазяїнами й владарями природи» та тим самим «пособляти
поліпшенню людського життя» («Il est possible de parvenir à des connaissances
fort utiles à la vie, et qu'au lieu de cette philosophie spéculative
qu’on enseigne dans les écoles, on en peut trouver une pratique, par laquelle,
connaissant la force et les actions du feu, de l’eau, d’air, des astres, et de
tous les autres corps qui nous environnent, aussi distinctement que nous
connaissons les divers métiers de nos artisans, nous les pourrions employer
* — на перший погляд. Ред.
}

\index{i}{0320}  %% посилання на сторінку оригінального видання
Ясна річ, що коли продукція якоїсь машини коштує стільки ж
праці, скільки заощаджується при вживанні її, то відбувається
просте переміщення праці, тобто загальна сума праці, потрібна
на продукцію товару, не меншає, або продуктивна сила праці не
більшає. Однак ріжниця між працею, якої коштує машина, і
тією працею, яку вона заощаджує, або ступінь її продуктивности,
не залежить, очевидно, від ріжниці між її власного вартістю й
вартістю того знаряддя, яке вона заміняє. Ця ріжниця триває
так довго, поки трудові витрати на машину, а тому й та частина
вартости, яку вона додає до продукту, лишаються меншими від
тієї вартости, яку робітник із своїм знаряддям додав би до предмету
праці. Тим то продуктивність машини вимірюється тим
ступенем, у якому вона заміняє людську робочу силу. За Бейнсом,
на 450 веретен-мюлів із підготовчими машинами, що їх рухає
одна парова кінська сила, припадає 2 1/2 робітника,\footnote{
За річним звітом торговельної палати в Ессені (жовтень 1863 р.)
сталеливарня Круппа за допомогою 161 перетопних, гартівних та цементових
печей, 32 парових машин (1800 р. це було приблизно загальне число
парових машин, що вживалися в Менчестері) та 14 парових молотів, —
які разом репрезентували 1.236 кінських сил, — 49 ковальських горен,
203 виконавчих машин та приблизно 2.400 робітників — випродукувала
1862 р. 13 мільйонів фунтів литої сталі. Тут на одну кінську силу немає
й двох робітників.
} і кожне
автоматичне веретено mule випрядає за десятигодинного робочого
дня 13 унцій пряжі (пересічного нумера), отже, 2 1/2 робітника
випрядають 365 5/8 фунтів пряжі на тиждень. Отже, при своєму
перетворенні на пряжу приблизно 366 фунтів бавовни (для спрощення
ми залишаємо осторонь відпадки) вбирають лише 150 робочих
годин, або 15 десятигодинних робочих днів, тимчасом як із
самопрядом, якщо ручний прядун дає за 60 годин 13 унцій пряжі,
та сама кількість бавовни забрала б 2.700 десятигодинних робочих
днів, або 27.000 робочих годин.\footnote{
Беббедж обчислює, що на Яві самою лише працею прядіння до
вартости бавовни додається майже 117\%. У той самий час (1832) в Англії
загальна вартість, яку при тонкопрядінні додавали до бавовни машини
і праця, становила приблизно 33\% вартости сировинного матеріялу,
(«On the Economy of Machinery», London 1832, p. 214).
} Там, де стару методу
blockprinting, або ручного вибивання перкалю, витиснуло машинове
вибивання, одним-одна машина за допомогою одного чоловіка
або підлітка вибиває за одну годину стільки саме чотирибарвного
перкалю, скільки раніше вибивало 200 чоловіка.\footnote{
Окрім того, при машиновому вибиванні заощаджується фарбу.
}

en même façon à tous les usages auxquels ils sont propres, et ainsi nous
rendre comme maîtres et possesseurs de la nature... contribuer au perfectionnement
de la vie humaine»). У передмові до «Discourses upon Trade»
(1691 p.) сера Дудлея Норта сказано, що метода Декарта, застосована
до політичної економії, почала визволяти її від стародавніх казок і забобонних
уявлень про гроші, торговлю й т. д. Загалом, однак, англійські
економісти давніх часів приєднуються до філософії Бекона й Гоббса, тимчасом
як пізніш «філософом» χατ’ ε’ξοχη'ν\footnote*{
— переважно. Ред.
} політичної економії для Англії,
Франції та Італії став Льокк.
\parbreak{}  %% абзац продовжується на наступній сторінці
