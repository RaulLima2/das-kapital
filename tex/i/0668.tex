нення робітників», так що майже кожний поштовий корабель приносить із собою лихі звістки про
переповнення австралійського ринку праці — «glut of the Australian labour-market», — а проституція
процвітає там подекуди так само пишно, як і на Haymarket у Лондоні.

Однак нас цікавить тут не стан колоній. Нас цікавить лише таємниця, відкрита в Новому Світі
політичною економією Старого Світу і гучно проголошена нею: капіталістичний спосіб продукції й
акумуляції, отже, і капіталістична приватна власність зумовлюють знищення приватної власности,
основаної на власній праці, тобто зумовлюють експропріяцію робітника.

1862, є в тому, щоб полегшити народові змогу розселюватися» («The first and main object at which the
new Land Act of 1862 aims, is to give increased facilities for the settlement of the people»). («The
Land Law of Victoria by the Hon. G. Duffy, Minister of Public Lands», London 1862, p. 3).
