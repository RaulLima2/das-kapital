природи речей, і нарікати разом з Россі: «Уявляти собі спроможність
працювати (puissance de travail), абстрагуючись від засобів
існування робітників підчас процесу продукції — це значить
уявляти собі якусь химеру (être de raison). Хто каже «праця»,
хто каже «спроможність працювати», той каже одночасно «робітник
і засоби існування», «робітник і заробітна плата».\footnote{
Rossi: «Cours d’Economie Politique», Bruxelles 1842, p. 370.
} Хто
каже «спроможність працювати», той ще не каже «праця», так
само, як хто каже «спроможність перетравлювати», ще не каже
«перетравлювання». Останній процес потребує, як відомо, ще
чогось більшого, ніж добрий шлунок. Хто каже «спроможність
працювати», той не абстрагується від засобів існування, потрібних
для її існування. Навпаки, вартість цих засобів існування
є виражена в її вартості. Коли спроможність працювати не можна
продати, то робітникові з неї користи мало, навпаки, він відчуває
як жорстоку природну доконечність те, що його спроможність
працювати потребувала для своєї продукції певної кількости засобів
існування і знову раз-у-раз потребує їх для своєї репродукції.
Тоді він разом із Сісмонді відкриває, що «спроможність
працювати... є ніщо, коли її не можна продати».\footnote{
Sismondi: «Nouveaux Principes d’Economie Politique», vol. 2, p. 113.
}

Зі своєрідної природи цього специфічного товару, робочої
сили, випливає, що зі складанням контракту між покупцем і
продавцем споживна вартість робочої сили в дійсності ще не
переходить до рук покупця. Її вартість, подібно до вартости
всякого іншого товару, було визначено раніш, ніж вона увійшла
до циркуляції, бо певну кількість суспільної праці було вже
витрачено на продукцію робочої сили, але її споживна вартість
полягає у виявленні сили, що відбувається лише пізніше. Тому
відчуження сили й дійсне виявлення її, тобто її буття як споживної
вартости, не збігаються щодо часу. Але там, де йдеться
про такі товари,\footnote{
«Всяку працю оплачується після її закінчення» («All labour is
paid, after it has ceased»). (An Inquiry into those Principles respecting
the Nature of Demand etc»., London 1821, p. 104). «Комерційний кредит
мусив був початись у той момент, коли робітник, це перше джерело продукції,
через свою ощадність мав змогу чекати на плату за свою працю
до кінця тижневого, двотижневого, місячного, тримісячного й т. ін. реченця
» («Le crédit commercial a dû commencer au moment où l’ouvrier,
premier artisan de la production, a pu, au moyen de ses économies, attendre
le salaire de son travail jusqu’à la fin de la semaine, de la quinzaine, du
mois, du trimestre etc.»). (Ch. Ganilh: «Des Systèmes de l’Economie
Politique», 2-ème éd. Paris 1821, vol. 2, p. 150).
} що в них формальне відчуження споживної
вартости через продаж і її дійсне використання покупцем щодо
часу не збігаються, — там гроші покупця функціонують здебільша
як засіб платежу. По всіх країнах капіталістичного способу
продукції робочу силу оплачується лише після того, як вона
вже функціонувала протягом реченця, установленого контрактом,
приміром, наприкінці кожного тижня. Тому повсюди робітник
авансує капіталістові споживну вартість своєї робочої сили;