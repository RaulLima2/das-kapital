управителя (bailiff, Vogt, який раніше сам був кріпаком. Рільничі
наймані робітники складалися почасти з селян, що використовували
свій вільний час, працюючи у великих землевласників,
почасти із самостійної, відносно й абсолютно малочисельної
кляси найманих робітників у власному значенні слова. Останні
фактично разом з тим теж були селянами, що самостійно господарювали,
бо, крім своєї заробітної плати, вони діставали котедж
і 4 або й більш акрів поля. Окрім того, вони спільно з справжніми
селянами користалися з громадської землі, на якій вони пасли
свою худобу й яка разом з тим давала їм паливо — дрова, торф
і т. ін. 191 По всіх країнах Европи февдальна продукція характеризується
поділом землі поміж якомога більшим числом васалів.
Сила февдального пана, як і всякого суверена, спиралась не на
розміри його ренти, а на число його підданців, а останнє залежало
від числа селян, що господарювали самостійно.192 Тим-то,
хоч англійська земля після норманського завоювання була
поділена на величезні баронства, що з них окремі часто охоплювали
900 давніших англосаксонських лордств, проте вона була
вкрита дрібними селянськими господарствами, що лише дене-де
перемежалися великими панськими маєтками. Та і відносини,
за одночасного розквіту міст, яким відзначається X V століття,
уможливили те народнє багатство, яке так красномовно
змальовує канцлер Фортескю у своїх «Laudibus Legum Angliae»,
але вони виключали капіталістичне багатство.

Пролог до перевороту, що створив основу капіталістичного
способу продукції, відбувся в останній третині XV і в перші
десятиліття XVI століть. Масу вільних, як птиці, пролетарів
було викинуто на робітничий ринок у наслідок розпуску февдаль-

цілковита власність на землю]. Пересічний дохід цих дрібних землевласників...
оцінюється в 60—70 фунтів стерлінґів. Обчислено, що число
тих, хто обробляв власну землю, було більше, ніж число орендарів чужої
землі». (Macaulay: «History of England», 10th ed. London 1854,
vol. I, p. 333—334). — «Ще в останній третині XVII століття 4/5 англійської
людности були рільники» (там же, стор. 413). — Я цитую Маколея,
бо він, як систематичний фальсифікатор історії, по змозі применшує
подібні факти.

191 Ніколи не слід забувати, що навіть кріпак був не тільки власником
— правда, власником, що мусив платити данину — земельних
парцель, що належали до його дому, але й співвласником громадської
землі. «Селянин є тут (у Шльонську) кріпак» («Le paysan у (en Silésie)
est serf»). Проте ці кріпаки (serfs) є посідачі громадської землі. «Досі
ще не вдалося схилити жителів Шльонську до поділу громадських
земель, тимчасом як у Новій Марці немає вже жодного села, де б цього
поділу не проведено з якнайбільшим успіхом» («On n’a pas pu encore
engager les Silésiens au partage des communes, tandis que dans la nouvells
Marche, il n’y a guère de village où ce partage ne soit exécuté avec le plue
grand succès»). (Mirabeau: «De la Monarchie Prussienne», London 1788,
vol. II, p. 125, 126).

192    Японія з її суто февдальною організацією земельної власности
та з її розвиненим дрібноселянським господарством дає куди вірніший
образ європейського середньовіччя, ніж усі наші книги з істори, здебільшого
подиктовані буржуазними забобонами. Занадто воно вже вигідно
бути «ліберальним» коштом середньовіччя.
