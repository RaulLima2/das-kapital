\parcont{}  %% абзац починається на попередній сторінці
\index{i}{0062}  %% посилання на сторінку оригінального видання
обмін товару на золото, то недобачають якраз того, що треба б
бачити, а саме того, що робиться з формою. Недобачають, що золото
як простий товар не є гроші, і що інші товари у своїх цінах
сами відносять себе до золота як до своєї власної грошової
форми.

Спочатку товари входять у процес обміну непозолочені, непоцукровані,
а так, як їх мати породила. Процес обміну створює
подвоєння товару на товар і гроші, — зовнішню протилежність,
у якій вони виявляють свою іманентну протилежність між споживною
вартістю й вартістю. В цій протилежності товари як
споживні вартості виступають проти грошей як мінової вартости.
З другого боку, і один і другий бік цієї протилежности
є товари, отже, єдності споживної вартости й вартости. Але ця
єдність ріжниць на кожному з обох полюсів з’являється у зворотному
вигляді, а через те одночасно виражає їхнє взаємне відношення.
Реально товар є споживна вартість, його вартостеве
буття з’являється лише ідеально в ціні, яка ставить його у відношення
до золота, що протистоїть йому, як до його реальної
форми вартости. Навпаки, речовина золота фігурує лише як матеріялізація
вартости, як гроші. Тому золото реально є мінова
вартість. Його споживна вартість лише ідеально з’являється в
ряді відносних виразів вартости, в якому воно відноситься до
товарів, що протистоять йому, як до кола своїх реальних споживних
форм. Ці протилежні форми товарів є дійсні форми руху
їхнього процесу обміну.

Підімо тепер з якимось посідачем товарів, приміром, з нашим
старим знайомим ткачем полотна, на арену процесу обміну, на
товаровий ринок. Його товар, 20 метрів полотна, має визначену
ціну. Вона дорівнює 2\pound{ фунтам стерлінґів}. Він обмінює полотно
на 2\pound{ фунти стерлінґів} і, як людина старозавітня, знову обмінює
2\pound{ фунти стерлінґів} на фамільну біблію тієї самої ціни. Полотно
для нього лише товар, носій вартости, воно відчужується за
золото, форму вартости полотна, і в цій формі знову відчужується
за інший товар, за біблію, яка, однак, як предмет споживання,
має помандрувати в дім ткача і там задовольняти потребу
в душоспасному читанні. Отже, процес обміну товару здійснюється
в двох метаморфозах, що одна одній протилежна і взаємно
одна одну доповнює, — в перетворенні товару на гроші й у зворотному
перетворенні його з грошей на товар.\footnote{
«Вогонь, — як каже Геракліт, — перетворюється на все, і все
перетворюється на вогонь подібно до того, як товари перетворюються на
золото, а золото на товари» («\textgreek{Ἐκ δὲ τοῦ\dots{} πυρὸς ἀνταμείβεσθαι πάντα φησὶν ὁ Ἡράκλειτος, καὶ πὺρ
ἁπάντων, ὥσπερ χρυσοῦ χρήματα καὶ χρημάτων χρυσός}»), (\emph{F. Lassale}:
«Die Philosophie Herakleitos des Dunkeln», Berlin 1858, Bd. I, p. 222).
У примітці до цього місця (стор. 224, прим. 3) Ляссаль неправильно розглядає
гроші лише як простий знак вартости.
} Моменти товарової
метаморфози є одночасно операції посідача товарів: продаж,
обмін товару на гроші, купівля, обмін грошей на товар,
і єдність обох актів: продаж, щоб купити.
