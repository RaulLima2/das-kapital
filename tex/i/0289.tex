ний продукт частинних робітників перетворюється на товар.58а
Поділ праці всередині суспільства упосереднюється купівлею та
продажем продуктів різних галузей праці, зв’язок між частинними
робітниками в мануфактурі — продажем різних робочих
сил тому самому капіталістові, який вживає їх як одну комбіновану
робочу силу. Мануфактурний поділ праці має собі за передумову
концентрацію засобів продукції в руках одного капіталіста,
суспільний поділ праці — розпорошення засобів продукції
між багатьма один від одного незалежними продуцентами товарів.
Тимчасом як у мануфактурі залізний закон пропорційного числа
або пропорційности реґулює (subsumiert) розподіл певних робочих
мас між певними функціями, випадок і сваволя ведуть свою
вередливу гру в поділі товаропродуцентів і їхніх засобів продукції
поміж різними суспільними галузями праці. Правда, різні сфери
продукції постійно намагаються дійти рівноваги, бо, з одного
боку, кожний товаропродуцент мусить виробляти споживну вартість,
отже, задовольняти осібну суспільну потребу, але обсяг
цих потреб кількісно є різний і внутрішній зв’язок сполучає
різні маси потреб в одну природно вирослу систему: з другого ж
боку, закон вартости товарів визначає, скільки з усього робочого
часу, який суспільство має в своєму розпорядженні, може воно
витратити на продукцію кожного окремого роду товару. Але ця
постійна тенденція різних сфер продукції дійти рівноваги виявляється
лише як реакція проти постійного нищення (Aufhebung)
цієї рівноваги. Норма, що її а priori і пляномірно дотримуються

cтину цілого, а через те, що кожна частина сама по собі не має ніякої
вартости або корисности, то тут немає нічого такого, що робітник міг би
взяти і сказати: це мій продукт, це я залишаю собі» («There is no longer
anything which we can call the natural reward of individual labour.
Each labourer produces only some part of a whole, and each part, having no
value or utility of itself, there is nothing on which the labourer can seize,
and say: it is my product, this I will keep for myself»). («Labour defended
against the claims of Capital», London 1825, p. 25). Автором цієї
прегарної праці є цитований вище Т. Годжскін.

58a Примітка до другого видання. — Янкі практично зілюстрували
цю ріжницю між суспільним і мануфактурним поділом праці. Одним
з нових податків, вигаданих у Вашинґтоні за часів громадянської війни,
був акциз у 6\% на «всі промислові продукти». Питання: що таке промисловий
продукт? Законодавець відповідає: Кожна річ є продукт,
«якщо вона зроблена» (when it is made), а вона є зроблена, якщо готова
для продажу. Ось один із багатьох прикладів. Мануфактури Нью-Йорку
і Філадельфії за старих часів «робили» парасолі з усіма їхніми причандалами.
Але що парасоля є mixtum compositum * цілком різнорідних
складових частин, то ці останні поволі поробилися продуктами незалежних
одна від одної й розкиданих по різних місцях галузей продукції,
їхні частинні продукти входили як самостійні товари в парасольну
мануфактуру, яка лише сполучає їх в одну цілість. Янкі охристили такого
роду продукти «assembled articles» (збірними продуктами) — назва,
яку вони заслужили собі саме як «збирачі» податків. Так, парасоля «збирала»
спочатку 6\% акцизу з ціни кожного із своїх елементів, а потім
знову 6\% з ціни цілого продукту.

* — складне сполучення. Ред.
