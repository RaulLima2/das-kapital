Тут нічого не зарадить і пояснення обміну більшої кількости
праці на меншу ріжницею форми, тим, що в одному випадку праця
є упредметнена, в другому — жива.\footnote{
«Треба було згодитися (ще одна відміна «contrat social»), що кожного
разу, коли обмінюватиметься працю виконану на працю, яка ще
має бути виконана, останній (капіталіст) матиме вартість вищу, ніж перший
(робітник)» («Il a fallu convenir que toutes les fois qu’il échangerait du
travail fait contre du travail à faire, le dernier (le capitaliste) aurait une valeur
supérieure au premier (le travailleur)». Simonde de Sismondi: «De
la Richesse Commerciale», Genève 1803, vol. I, p. 37).
} Це тим недоладніша річ,
що вартість товару визначається не кількістю дійсно упредметненої
в ньому праці, а кількістю живої праці, доконечної для
його продукції. Нехай якийсь товар репрезентує 6 робочих годин.
Якщо пороблять такі винахода, що через них той товар можна
буде випродукувати протягом 3 годин, то й вартість випродукованого
вже товару спаде наполовину. Тепер він репрезентує
замість попередніх 6 годин тільки 3 години доконечної суспільної
праці. Отже, величину вартости товару визначає кількість праці,
потрібна на його продукцію, а не предметна форма цієї праці.

Фактично на товаровому ринку власникові грошей безпосередньо
протистоїть не праця, а робітник. Те, що останній продає,
є його робоча сила. Скоро тільки дійсно починається його
праця, вона вже не належить йому, отже, він її не може вже
більше продати. Праця є субстанція й іманентна міра вартостей,
але сама вона не має вартости.26

У вислові «вартість праці» поняття вартости не тільки цілком
погашено, але ще й перетворено на свою протилежність. Це такий
самий іраціональний вислів, як от, наприклад, вартість землі.
Однак такі іраціональні вислови випливають із самих продукційних
відносин. Це — категорії для форм виявлення посутніх
відносин. Що речі в своєму виявленні часто з’являються
покрученими, це досить відомо в усіх науках, крім політичної
економії.26

25 «Праця, виключна міра вартости... творець усякого багатства,
не є товар» («Labour, the exclusive standard of value... the creator of
ail wealth, no commodity»). (Th. Hodaskin: «Popular Political Economy»,
p. 186).

26 Навпаки, поясняти такі вислови просто як licentia poetica\footnote*{
— поетичні вільності. Ред.
} — це
свідчить лише про безсилля аналізи. Тим то на фразу Прудона: «Про
працю кажуть, що вона має вартість не як про власне товар, а маючи
на оці ті вартості, що їх припускається за потенціяльно вміщені в ньому.
Вартість праці — вираз фігуральний і т. ін.» («Le travail est dit valoir,
non pas en tant que marchandise lui-même, mais en vue de valeurs qu’on
suppose renfermées puissanciellement en lui. La valeur du travail est une
expression figurée etc.»), — я зауважую: «В праці-товарі, який має
жахливу реальність, він убачає тільки граматичну еліпсу. Отже,
виходить, що ціле сучасне суспільство, яке ґрунтується на товарі-праці,
відтепер ґрунтується на поетичній вільності, на фігуральному вислові.
Якщо суспільство захоче «усунути всі недоладності», що його мучать
що ж І — хай воно усуне лише недоброзвучні, вислови, змінить мову, а для
цього досить лише звернутися до академії з вимогою випустити нове
видання її словника» («Dans le travail-marchandise, qui est d une réalité