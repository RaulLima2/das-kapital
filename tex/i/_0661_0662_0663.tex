\parcont{}  %% абзац починається на попередній сторінці
\index{i}{0661}  %% посилання на сторінку оригінального видання
слуги, який міг би постелити йому постіль або набрати води з річки.\footnote{
Е. G. Wakefield: «England and America». London 1833, vol. II, p. 33.
258 Там же, т. І, стор. 17, 18.
259 Там же, стор. 18.
* — суспільного договору. \emph{Ред.}
} Безталанний пан Піл! Він усе
передбачив, та забув лише експортувати англійські продукційні відносини на Лебединий берег.

Щоб зрозуміти дальші відкриття Векфілда, потрібні два попередні зауваження. Ми знаємо, що коли
засоби продукції та засоби існування є власність безпосереднього продуцента, то вони не є капітал.
Вони стають капіталом лише за таких умов, коли вони разом з тим служать за засоби експлуатації та за
засоби упідлеглення робітника. Але ця їхня капіталістична душа в голові політико-економа з’єднана
таким тісним подружнім зв’язком із їхньою речовою субстанцією, що він за всяких обставин називає їх
капіталом, навіть і тоді, коли вони є сáме протилежність капіталу. Так стоїть справа й у Векфілда.
Далі: роздрібнення засобів продукції, як індивідуальної власности багатьох незалежних один від
одного, самостійно господарюючих робітників, він називає рівним поділом капіталу. З
політико-економом трапляється те саме, що і з февдальним юристом. Цей останній і на суто грошові
відносини наклеює свої февдальні правні етикетки.

«Коли б, — каже Векфілд, — капітал був поділений поміж усіма членами суспільства рівними пайками, то
жодна людина не була б заінтересована в тому, щоб акумулювати капіталу більш, ніж вона може
застосувати своїми власними руками. Так до певної міри стоїть справа в нових американських колоніях,
де жадоба до земельної власности перешкоджає існуванню кляси найманих робітників».258 Отже, поки
робітник має змогу акумулювати для себе самого, — а це він може робити, поки він лишається власником
своїх засобів продукції, — доти капіталістична акумуляція й капіталістичний спосіб продукції
неможливі. Бракує доконечної для цього кляси найманих робітників. Але як же тоді в старій Европі
здійснено експропріяцію в робітника його умов праці, яким чином, отже, створено там капітал і
найману працю? За допомогою contrat social* дуже ориґінального характеру. «Людство\dots{} засвоїло собі
просту методу активізувати акумуляцію капіталу», яка, звичайно, від часів Адама здавалась йому
останньою й єдиною метою його буття: «воно поділилось на власників капіталу і власників праці\dots{} цей
поділ був результатом добровільного порозуміння та погодження» (Kombination).259 Одне слово, маса
людства сама себе експропріювала на славу «акумуляції капіталу». А тепер треба б думати, що інстинкт
цього самовідданого фанатизму мусив би вільно виявитися саме в колоніях, де тільки й існують люди й
умови, які могли б перенести contrat
\index{i}{0662}  %% посилання на сторінку оригінального видання
social із царства мрій у царство дійсности. Але навіщо тоді взагалі «систематична колонізація»
протилежно до природної колонізації? Але, але: «сумнівно, чи в північних штатах американського союзу
хоч десята частина людности належить до категорії найманих робітників\dots{} В Англії\dots{} велика маса
народу складається з найманих робітників».\footnote{
Там же, стор. 42, 43, 44.
} В дійсності нахилу до самоекспропріяції на славу
капіталові в трудящого людства так небагато, що рабство, навіть за Векфілдом, є єдина природна
основа колоніяльного багатства. Його систематична колонізація є просто pis aller,\footnote*{
Pis aller — французький вираз: щось, до чого вдаються, коли немає нічого кращого. \emph{Ред.}
} бо ж йому
доводиться мати справу з вільними людьми, а не з рабами. «Перші еспанські поселенці на Сан-Домінґо
не діставали робітників із Еспанії. Але без робітників [тобто без рабства] капітал був би загинув
або принаймні скоротився б до таких дрібних розмірів, що всякий індивід міг би застосувати його
своїми власними руками. Так воно в дійсності й сталося в останній заснованій англійцями колонії, де
великий капітал у насінні, худобі й знарядді загинув через недостачу найманих робітників, і де жоден
поселенець не має капіталу більше, ніж він може застосувати своїми власними руками».\footnote{
Там же, т. II, стор. 5.
}

Ми бачили: експропріяція землі в народніх мас становить основу капіталістичного способу продукції.
Навпаки, суть вільних колоній у тому, що маса землі є ще народня власність, і тому кожний поселенець
може частину її перетворити на свою приватну власність і на свій індивідуальний засіб продукції, не
перешкоджаючи цим пізнішому поселенцеві зробити те саме.\footnote{
«Щоб стати елементом колонізації, земля не лише повинна бути необробленою, але й бути
громадською власністю, яку можна перетворити на приватну власність». (Там же, т. II, стор. 125).
} В цьому таємниця так процвітання колоній
як і їхніх болячок — їхнього опору проти вселення капіталу. «Де земля дуже дешева й усі люди вільні,
де кожний може з свого бажання дістати шматок землі для самого себе, там праця не лише дуже дорога,
беручи до уваги ту пайку, що припадає робітникові з його продукту, але й взагалі важко хоч за
якубудь ціну дістати комбіновану працю».\footnote{
Там же, т. І, стор. 247.
} А що в колоніях немає ще відокремлення робітника від
умов праці й їхньої основи, від землі, або відокремлення таке існує лише спорадично або на занадто
обмеженому просторі, то там ще немає й відокремлення рільництва від промисловости і не знищена ще
сільська домашня промисловість. Але звідки ж тоді там узятися внутрішньому ринкові для капіталу? «За
винятком рабів та їхніх хазяїнів, що скомбіновують капітал і працю для великих підприємств, жодна
частина людности Америки не працює виключно коло рільництва.
\index{i}{0663}  %% посилання на сторінку оригінального видання
Вільні американці, що сами обробляють землю, мають одночасно ще багато інших занять. Частину
вживаних ними меблів і знарядь вони звичайно виготовлюють сами. Вони часто будують свої власні
будинки й постачають продукти своєї власної
промисловости на якнайдальші ринки. Вони одночасно прядуни й ткачі, вони фабрикують мило й свічки,
взуття і одяг для свого власного вжитку. В Америці рільництво часто є побічне заняття коваля,
мірошника або крамаря».\footnote{
Там же, стор. 21, 22.
} Де ж тут лишається серед таких чудаків поле для «поздержливости»
капіталіста?

Велика принадність капіталістичної продукції в тому, що вона не лише постійно репродукує найманого
робітника як найманого робітника, але й пропорційно до акумуляції капіталу завжди продукує відносне
перелюднення найманих робітників.
Таким чином закон попиту й подання праці утримується в належній колії, коливання заробітної плати
вганяється у межі, вигідні для капіталістичної експлуатації, і, нарешті, ґарантується стільки
доконечну соціяльну залежність робітника від
капіталіста, те відношення абсолютної залежности, що його політико-економ може у себе дома, в
метрополії, пишномовно перебріхувати на вільне договірне відношення між покупцем і продавцем, між
двома однаково незалежними посідачами товарів,
посідачем товару капітал і посідачем товару праця. Але в колоніях ця чудова ілюзія зникає. Абсолютна
кількість людности тут зростає куди швидше, ніж у метрополії, бо багато робітників приходить тут на
світ уже дорослими, і все ж ринок праці тут завжди неповний. Закон попиту й подання праці тут цілком
крахує. З одного боку, старий світ постійно вкидає туди капітал, жаждущий експлуатації, охоплений
потребою в
поздержливості; з другого боку, реґулярна репродукція найманих робітників як найманих робітників
наражається на якнайнеприємніші й почасти непереможні перешкоди. Де ж тут думати про продукцію
зайвих найманих робітників пропорційно
до акумуляції капіталу! Сьогоднішній найманий робітник на завтра стає незалежним, самостійно
господарюючим селянином або ремісником. Він зникає з ринку праці, та тільки не в робітний дім. Це
постійне перетворювання найманих робітників на незалежних продуцентів, що працюють не на капітал, а
на самих себе, і збагачують не пана капіталіста, а самих себе, із свого боку надзвичайно шкідливо
впливає на стан
ринку праці. Не тільки ступінь експлуатації найманого робітника лишається до непристойности низький.
Найманий робітник, крім цього, втрачає разом із своєю залежністю й почуття залежности від
поздержливого капіталіста. Відси всі ті прикрості,
що їх так відважно, так пишномовно й так зворушливо змальовує нам Е. Ґ Векфілд.

Подання найманої праці, — скаржиться він, — і непостійне, і нерівномірне, і недостатнє. Воно «не
лише завжди занадто
\parbreak{}  %% абзац продовжується на наступній сторінці
