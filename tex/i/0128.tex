ної гармонії речей, або під проводом всемудрого провидіння,
здійснюють лише справу своєї взаємної вигоди, спільної користи,
спільних інтересів.

У той момент, коли ми розстаємося з цією сферою простої
циркуляції, або товарового обміну, з якої фритредер vulgaris
запозичає свої погляди, поняття й маштаб для своїх думок про
суспільство капіталу й найманої праці, ми помічаємо, що вже де
в чому зміняється, здається, фізіономія наших dramatis personae.
Колишній посідач грошей іде попереду як капіталіст, посідач
робочої сили йде слідком за ним як його робітник; один іде,
многозначно усміхаючися й жадаючи взятися до справи; другий —
боязко, опираючись, як людина, що винесла на ринок свою власну
шкуру й не має вже нічого іншого сподіватись, як тільки того,
що цю шкуру будуть гарбарювати.
