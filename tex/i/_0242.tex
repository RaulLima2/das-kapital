\parcont{}  %% абзац починається на попередній сторінці
\index{i}{0242}  %% посилання на сторінку оригінального видання
капітал, є в певних межах незалежне від подання робітників.\footnote{
Цей елементарний закон, здається, невідомий вульґарним економістам
які, Архімеди навиворіт, гадають, що у визначенні ринкових цін
праці попитом і поданням вони знайшли пункт опори не для того, щоб
перевернути світ, а щоб спинити його рух.
}
Навпаки, зменшення норми додаткової вартости лишає масу
випродукованої додаткової вартости незмінною, коли величина
змінного капіталу або число вживаних робітників пропорційно
зростає. [Змінний капітал у 100 талярів, що експлуатує 100 робітників
при нормі додаткової вартости в 100\%, продукує 100 талярів
додаткової вартости. Норма додаткової вартости може зменшитися
вдвоє, але сума її лишається та сама, коли одночасно
подвоюється змінний капітал].\footnote*{
Заведене у прямі дужки ми беремо з французького видання. Ред.
}

Однак, компенсація числа робітників або величини змінного
капіталу збільшенням норми додаткової вартости або здовженням
робочого дня має межі, що їх не сила переступити. Хоч яка
буде вартість робочої сили, отже, все одно, чи робочий час,
доконечний для утримання робітника, становить 2 чи 10 годин,
загальна вартість, яку робітник може продукувати день-у-день,
є завжди менша від вартости, в якій упредметнюються 24 робочі
години, менша від 12 шилінґів, або 4 талярів, коли вони є грошовий
вираз 24 упредметнених робочих годин.

[Щодо додаткової вартости, то її межі ще вужчі. Коли частина
робочого дня, доконечна для покриття денної заробітної плати,
становить 6 годин, то від природного дня залишається тільки 18 годин,
що з них, за біологічними законами, частина потрібна для
відпочинку робочої сили. Припустімо, що 6 годин є мінімальна
межа цього відпочинку; коли здовжити робочий день до його максимальної
межі, до 18 годин, то додаткова праця становитиме
лише 12 годин і, отже, спродукує вартість лише в 2 таляри].\footnote*{
Цей абзац ми беремо з французького видання. Ред.
}

За нашої попередньої передумови, за якою потрібно на день
6 робочих годин, щоб репродукувати саму робочу силу, або
покрити капітальну вартість, авансовану на її купівлю, змінний
капітал у 500 талярів, який уживає 500 робітників при нормі
додаткової вартости в 100\%, або за 12-годинного робочого дня,
продукує денно додаткову вартість у 500 талярів, або 6 × 500
робочих годин. А капітал у 100 талярів, який денно вживає
100 робітників при нормі додаткової вартости в 200\%, або за 18-годинного
робочого дня, продукує масу додаткової вартости лише
в 200 талярів, або 12 × 100 робочих годин. Ціла нововипродукована
ним вартість, еквівалент авансованого змінного капіталу
плюс додаткова вартість, ніколи не може досягти суми 400 талярів,
або 24 × 100 робочих годин пересічно на день. Абсолютна
межа пересічного робочого дня, який з природи є завжди менший
від 24 годин, становить абсолютну межу для компенсації зменшення
змінного капіталу підвищенням норми додаткової вартости,
або зменшення числа експлуатованих робітників підвищенням
\parbreak{}  %% абзац продовжується на наступній сторінці
