тією самою працею, оскільки вона утворює вартість, виявилась
тепер як ріжниця між різними сторонами процесу продукції.

Як єдність процесу праці й процесу утворення вартости процес
продукції є процес продукції товарів; як єдність процесу
праці й процесу зростання вартости він є капіталістичний процес
продукції, капіталістична форма товарової продукції.

Вже раніш зазначено, що для процесу зростання вартости
цілком байдуже, чи є присвоєна капіталістом праця проста, суспільна
пересічна праця, чи складніша праця, праця вищої специфічної
ваги. Праця, що є виша, складніша супроти суспільної пересічної
праці, є виявлення такої робочої сили, що на її освіту витрачено
більше коштів, продукція якої коштує більше робочого
часу і яка тому має вищу вартість, ніж проста робоча сила.
Коли вартість цієї сили вища, то й виявляється вона у вищій
праці й тим то упредметнюється за той самий час у відповідно
вищих вартостях. Але хоч і яка була б ріжниця щодо ступеня
між працею прядуна й працею ювеліра, та частка праці, якою
ювелірний робітник повертає тільки вартість своєї власної робочої
сили, якісно аж ніяк не відрізняється від тієї додаткової
частки праці, що нею він утворює додаткову вартість. Тут, як і
раніш, додаткову вартість утворюється лише в наслідок кількісного
надлишку праці, в наслідок здовженого тривання того самого
процесу праці: в одному випадку — процесу продукції пряжі,
в другому випадку — процесу ювелірної продукції.\footnote{
Ріжниця між складною і простою працею, між «skilled» і «unskilled
labour», ґрунтується почасти просто на ілюзіях, або, принаймні,
на ріжницях, які давним-давно перестали бути реальними і існують далі
лише як традиційні умовності, а почасти на безпораднішому становищі
певних верств робітничої кляси, через яке їм більш, ніж іншим, не сила
вибороти оплату своєї робочої сили за її вартістю. Випадкові обставини
відіграють при цьому таку велику ролю, що того самого роду праці міняють
своє місце. Там, наприклад, де фізична сила (Substanz) робітничої кляси
ослаблена й порівняно вичерпана, як от по всіх країнах розвиненої капіталістичної
продукції, грубі роботи, що потребують багато мускульної сили,
взагалі набирають вищого характеру порівняно з делікатнішими роботами,
які спадають до ступеня простої праці; приміром, праця bricklayer’à (муляра)
в Англії має значно вищий ступінь, ніж праця ткача дамасту.
З другого боку, праця fustian cutter (робітника, що стриже менчестер),
хоч вона й коштує більшого фізичного напруження і, опріч того, дуже нездорова,
фігурує як «проста» праця. Зрештою, не треба уявляти собі, що
так звана «skilled labour» має кількісно значні розміри в національній
праці. Лен обчислює, що в Англії (і у Велзі) існування більш ніж
11 мільйонів людей спирається на просту працю. Відлічивши один мільйон
аристократів і півтора мільйона павперів, волоцюг, злочинців, тих, що
живуть із проституції і т. ін., з 18 мільйонів загальної кількости людности
за часів писання його твору, лишається 4.650.000 осіб середньої
кляси, залічуючи сюди дрібніших рантьє, урядовців, письменників,
артистів, учителів і т. д. Щоб одержати цих 4\sfrac{2}{3} мільйона, він залічує до
працівної частини середньої кляси, крім банкірів, тощо всіх краще оплачуваних
«фабричних робітників»! Навіть bricklayer’і (мулярі) опинилися
серед «кваліфікованих робітників». Після цього лишається в нього
}

З    другого боку, в кожному процесі утворення вартости вища
праця завжди мусить зводитися на суспільну пересічну працю,