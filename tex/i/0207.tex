той час, поки воно триває. Тому основний принцип рабовласницького
господарства по тих країнах, куди ввозять рабів, такий:
найуспішніша економія є в тому, щоб із людської худоби (human
chattle) видушити якнайбільшу кількість праці протягом якнайменшого
часу. Саме в країнах з тропічною культурою, де річний
зиск часто дорівнює цілому капіталові плянтацій, життя негрів
якнайнещадніше кидається в жертву. Саме рільництво Західньої
Індії, вікова колиска казкового багатства, пожерло мільйони
людей африканської раси. Саме тепер на Кубі, де доходи рахується
на мільйони, де плянтатори є князі, ми бачимо, що кляса
рабів не тільки дістає щонайгірше харчування і зазнає якнайвиснажувальнішої
й безперестанної муки, але й чимала частина
її рік-у-рік просто винищується через повільне катування
надмірною працею й недостачу сну й відпочинку».106

Mutato nomine de te fabula narratur.*  Замість работорговлі
читай — робочий ринок, замість Кентукі й Вірґінія — Ірляндія
й рільничі округи Англії, Шотляндії й Велзу, замість Африка —
Німеччина! Ми чули, як надмірна праця спустошує лави лондонських
пекарів, а, проте, лондонський робочий ринок завжди
переповнений німецькими й іншими кандидатами на смерть по
пекарнях. Ганчарні, як ми бачили, є одна з галузей промисловости,
де життя робітників якнайкоротше. Та чи бракує від того ганчарів?
Джосія Веґвуд, винахідник сучасної ганчарні, сам із роду
звичайний робітник, заявив 1785 р. перед Палатою громад,
що в цілій цій мануфактурі працює від 15 до 20 тисяч осіб.107
Року 1861 людність тільки міських центрів цієї промисловости
Великобрітанії становила 101.302 чол. «Бавовняна промисловість
існує вже 90 років. За час трьох поколінь англійської раси
вона пожерла дев’ять поколінь бавовняних робітників».108
Певна річ, підчас окремих епох гарячкового розцвіту робочий
ринок виявляв чималі прогалини. Так було, приміром, року 1834.
Але тут панове фабриканти запропонували Poor Law Commissioners**
відправляти «надмір людности» рільничих округ на північ,
заявивши, що «фабриканти його поглинули б і спожили б». 109
Це були їхні власні слова. «За згодою Poor Law Commissioners
призначено аґентів до Менчестеру. Виготовлено й вручено цим
аґентам списки рільничих робітників. Фабриканти кинулися до
бюр, і після того, як вони повибирали все, що їм було потрібно,
цілі родини повисилано з півдня Англії. Ці людські вантажі із
значками, як паки товарів, постачувано каналом і вантажевими
возами; декотрі пошкандибали за ними пішки, а багато збилося
з шляху й напівголодні блукали по промислових округах. Це роз-

106  J. С. Cairns: «The Slave Power», London 1862, p. 110, 111.

107 John Ward: «History of the Borough of Stoke-upon-Trent», London
1834, p. 42.

108 Промова Феранда в Палаті громад 27 квітня 1863 р.

109 «That the manufacturers would absorb it and use it up. Those were
the very words used by the cotton manufacturers». (Там же).

* Під іншим іменем тут про тебе мова. Ред.

** — комісії у справах бідних. Ред.
