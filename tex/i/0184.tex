Або 5 годин 40 хвилин на тиждень, а це, помножене на 50
робочих тижнів, за вирахуванням 2 тижнів на свята або випадкові
перерви праці, дає 27 робочих днів».\footnote{
«Suggestions etc. by Mr. L. Horner, Inspector of Factories», у
«Factories Regulation Act, Ordered by the House of Commons to be printed, ,
9. August 1859», p. 4, 5.
}

«Коли робочий день щодня подовжують на 5 хвилин поза
межі його нормального тривання, то це дає 2\sfrac{1}{2} робочих днів на
рік». 50 «Одна додаткова година на день, яку здобувається таким
чином, що шматок часу відривається то тут, то там, робить із
дванадцятьох місяців року тринадцять.\footnote{
«Reports etc. 30 th April 1858», p. 9.
}

Кризи, підчас яких продукція притіняється і працюють лише
«коротший час», лише по декілька днів на тиждень, звичайно,
нічого не змінюють у прагненні до подовження робочого дня.
Що менше робиться оборудок, то більший мусить бути виграш
від зробленої оборудки. Що менше часу можна працювати, то
більше додаткового робочого часу треба працювати. Ось які
звіти подають фабричні інспектори про період кризи від 1857 до
1858 рр.:

«Можна вбачати непослідовність у тому, що можлива будь-яка
надмірна праця в такі часи, коли торговля йде так погано,
але саме поганий стан торговлі й підштовхує безсумлінних людей
до надмірностей; вони забезпечують собі таким чином екстразиск»...
«В той самий час, — каже Леонард Горнер, — коли
122 фабрики в моїй окрузі цілком залишені, 143 припинили свою
працю, а всі інші працюють короткий час, надмірна праця поза
межі визначеного законом часу триває й далі».\footnote{
«Reports etc.», 1. c., p. 43.
} «Хоч, — каже
пан Говел, — більшість фабрик через поганий стан справ працює
лише половину часу, я дістаю таке число скарг, як і раніш, на
те, що вривають (snatched) щоденно в робітників пів або три
четвертини години, порушуючи час, призначений законом на
їжу й відпочинок».\footnote{
«Reports etc.» 1. c., p. 25.
}

Те саме явище повторюється в менших розмірах за часів страшної
бавовняної кризи від 1861 до 1865 рр.\footnote{
«Reports etc. for the half year ending 30 th April 1861». Див. додаток
№ 2: «Reports etc. 31 st October 1862», p. 7, 52, 53. Надмірності знову
стають численнішими за останнє півріччя 1863 р. Порівн. «Reports etc.
ending 31 st October 1863», p. 7.
}

Коли ми спіймаємо робітників за працею в обідній або в який
інший незаконний час, то нам іноді, виправдуючись, кажуть,
що вони ніяк не хочуть залишити фабрику, і що треба вживати
примусу, щоб заставити їх перервати їхню працю (чищення машин
і т. ін.), особливо в суботу по півдні. Але коли «руки» лишаються
на фабриці після припинення машин, так це стається лише
через те, що між шостою годиною ранку й шостою годиною ве-

50 «Reports of the Insp. of Fact, for the half year ending October 1856»,
p. 35.