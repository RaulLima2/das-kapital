нього, і саме після того, як йому самому пощастить зібрати золотий
дощ і заховати його в безпечне місце. Après moi le déluge!* —
оце гасло кожного капіталіста й кожної капіталістичної нації.
Тому капітал не звертає найменшої уваги на здоров’я та життя
робітника там, де суспільство не примушує його звертати на це
увагу.113 На скарги на фізичний і інтелектуальний занепад,
передчасну смерть, катування надмірою працею він відповідає:
невже мають боліти нам ті муки, коли вони збільшують нашу
втіху (зиск)? Але взагалі і в цілому це навіть не залежить від
доброї або злої волі поодинокого капіталіста. Вільна конкуренція
накидає поодинокому капіталістові іманентні закони капіталістичної
продукції як зовнішній примусовий закон. 114

Установлення нормального робочого дня є результат багатовікової
боротьби між капіталістом і робітником. Але історія цієї
боротьби виявляє дві протилежні течії. Порівняймо, приміром,
англійське фабричне законодавство нашого часу з англійськими
робітничими статутами від ХІV аж далеко до половини XVIII
століття.116 Тоді як сучасний фабричний закон насильно скорочує
робочий день, ці статути намагаються насильно його здовжити.
Певна річ, домагання капіталу в ембріональному стані,

113 «Хоч здоров'я людности і є такий важливий елемент національного
капіталу, ми боїмося, що доведеться визнати, що капіталісти зовсім
не мають нахилу берегти й цінити цей скарб... Фабрикантів примушено
звертати увагу на здоров'я робітників». («Times», 5 листопада 1861).
«Чоловіки West Riding’a стали сукноробами для цілого людства… здоров’я
робітничої людности було віддано на жертву і протягом декількох
поколінь раса була б виродилася, але наступила реакція. Обмежено
години дитячої праці й т. ін.» («Report of the Registrar General for October
1861»).

114 Тим-то ми бачимо, наприклад, що на початку 1863 р. 26 фірм,
які посідають величезні ганчарні в Staffordshire, між ними й Дж. Веґвуд
із синами, прохають у поданому меморіялі «насильного втручання держави».
«Конкуренція з іншими капіталістами» не дозволяє, мовляв, їм
зробити жодного «добровільного» обмеження робочого часу дітей і т. ін.
«Тим то, хоч би ми й як нарікали на вищезгадане лихо, його ніяк не можна
було б усунути шляхом якоїсь згоди поміж фабрикантами... Зважаючи на
всі ці пункти, ми прийшли до того переконання, що потрібен примусовий
закон». («Children’s Employment, Commission. 1st Report 1863»,
p. 322).

Додаток до примітки 114. Ще яскравіший приклад дає нам найближче
минуле. Високі ціни на бавовну за часів гарячкового розвитку справ спонукали
власників ткалень бавовни у Blackburn’i за взаємною згодою між
собою на якийсь певний реченець скоротити на своїх фабриках робочий
час. Реченець цей скінчився приблизно з кінцем листопада (1871 р.)
Тимчасом багатші фабриканти, в яких прядіння було сполучене з тканням,
використали скорочення продукції, зумовлене цією згодою, на те, щоб
поширити свої власні підприємства й таким чином коштом дрібних підприємців
здобути великі зиски. Опинившись у скрутному становищі,
останні звернулися до фабричних робітників, закликаючи їх серйозно
заходитися коло аґітації за дев’ятигодинний робочий день, і обіцяли їм
грошову допомогу на цю справу.

115    Ці робітничі статути, які ми знаходимо одночасно і у Франції,
Нідерляндах і т. д., формально скасовано в Англії лише 1813 р., вже після
того, як їх давно були усунули сами продукційні відносини.

* Після мене хоч потоп! Ред.
