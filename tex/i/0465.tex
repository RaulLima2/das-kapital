За відштучної плати на перший погляд здається, наче та
споживна вартість, яку продає робітник, є не функція його
робочої сили, не жива праця, а праця, упредметнена вже в
продукті, і наче ціна цієї праці визначається не дробом
денна вартість робочої сили/робочий день даного числа годин,
як за почасової плати, а дієздатністю продуцента.45

Певність правдивости цього, побудованого на позірності, погляду
мусила б насамперед дуже захитатися вже від того факту,
що обидві форми заробітної плати одночасно існують одна побіч
однієї в тих самих галузях промисловости. Наприклад, «лондонські
складачі працюють звичайно за відштучну плату, почасова
плата є в них виняток. Навпаки, у складачів на провінції
почасова плата є правило, авідштучна плата виняток. Корабельні
теслярі по гаванях Лондону одержують відштучну плату, по
всіх інших англійських гаванях — почасову».46 В Лондоні в
тих самих лимарнях часто за ту саму працю французи дістають
відштучну, англійці — почасову плату. На власне фабриках,
де взагалі переважає відштучна плата, окремі функції праці з
технічних причин виключається з цього способу виміру, і тому
їх оплачують почасово.47 Однак, само собою ясно, що ріжниця

45 «Система відштучної праці становить певну епоху в історії робітника;
це є щось середнє між становищем простого поденника, який залежить
від волі капіталіста, та становищем кооперативного робітника, що
в недалекій будучині обіцяє сполучити у своїй власній особі робітника
й капіталіста. Відштучні робітники фактично сами є для себе хазяїни,
навіть коли працюють за допомогою капіталу свого підприємця». («The
system of piece-work illustrates an epoch in the history of the working
man: it is halfway between the position of the mere daylabourer, depending
upon the will of the capitalist, and the cooperative artizan, who in
the not distant future promises to combine the artizan and the capitalist
in his own person. Piece-workers are in fact their own masters, even whilst
working upon the capital of the employer»). (John Watts: «Trade-Societies
and Strikes, Machinery and Cooperative Societies», Manchester 1865,
p. 52, 53). Я цитую це писаннячко, бо це є справжня кльоака для всіх
давно вже зогнилих апологетичних банальностей. Той самий пан Вотс
працював раніш в оуенівському русі та 1842 р. опублікував інше писаннячко
«Facts and Fictions of Political Economy», де він, між іншим, власність
(property) називає грабіжництвом (robbery). Це вже давно минулося.

46    Т. J. Dunning: «Trades-Unions and Strikes», London 1860,
p. 22.

47    Ось як це одночасне існування одної поряд одної цих двох форм
заробітної плати сприяє шахрайствам фабрикантів: «Фабрика вживає
400 робітників, що з них половина працює відштучно та безпосередньо
зацікавлена у збільшенні числа робочих годин. Решті 200 робітникам
платять поденно, вони працюють так само довго, алеж нічого не дістають
за наднормову працю... Праця цих 200 людей протягом півгодини
на день дорівнює праці однієї особи протягом 50 годин, або 5/6 тижневої
праці однієї особи, та є позитивний виграш для підприємця». («А factory
employs 400 people, the half of which work by the piece, and have a direct
interest in working longer hours. The other 200 are paid by the day,
work equally long with the others, and get no more money for their overtime...
The work of these 200 people for half an hour a day is equal to one
person’s work for 50 hours, or 6/e of one person’s labour in a week, and is
