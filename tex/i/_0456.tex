\parcont{}  %% абзац починається на попередній сторінці
\index{i}{0456}  %% посилання на сторінку оригінального видання
мають однакову ціну, а та сама річ має змінну ціну, дарма що
споживна вартість цієї речі або потреба в ній не змінюються].\footnote*{
Заведене у прямі дужки ми беремо з першого німецького видання. \emph{Ред.}
}

Далі, через те, що мінова вартість і споживна вартість сами
по собі є величини неспільномірні, то вислови «вартість праці»,
«ціна праці здаються не більш іраціональними, аніж, наприклад,
вислів «вартість бавовни», «ціна бавовни». До цього долучається
ще й те, що робітникові платять після того, як він дав
уже свою працю. Але в своїй функції засобу платежу гроші
навпісля реалізують вартість або ціну постаченого продукту,
отже, в даному випадку вартість або ціну постаченої праці.
Нарешті, та «споживна вартість», що її робітник постачає капіталістові,
є фактично не його робоча сила, а її функція, певна
корисна праця, праця кравця, шевця, прядуна й т. ін. Та обставина,
що та сама праця, з іншого боку, є загальний вартостетворчий
елемент, ця властивість, якою вона відрізняється від усіх
інших товарів, лишається поза сферою звичайної свідомости.

Коли станемо на погляд робітника, що за свою дванадцятигодинну
працю дістає, приміром, вартість, спродуковану шестигодинною
працею, скажімо 3 шилінґи, то для нього його дванадцятигодинна
праця є справді засіб купівлі 3 шилінґів. Вартість
його робочої сили може змінятися разом із зміною вартости його
звичайних засобів існування, може підвищитися з 3 до 4 шилінґів,
або спасти з 3 до 2 шилінґів, або за незмінної вартости його робочої
сили її ціна може, в наслідок зміни відношення між попитом
і поданням, підвищитися до 4 шилінґів або зменшитись до 2 шилінґів,
але все одно він завжди дає 12 годин праці. Тим-то кожна
зміна величини еквіваленту, що його він дістає, здається йому
неодмінно зміною вартости або ціни його 12 робочих годин.
Навпаки, ця обставина привела Адама Сміса, що розглядає
робочий день як сталу величину,\footnote{
Лише випадково, говорячи про відштучну плату, А. Сміс натякає
на зміну робочого дня.
} до твердження, що вартість
праці є стала, дарма що вартість засобів існування змінюється
і той самий робочий день через це виражається для робітника
в більшій або меншій кількості грошей.\footnote*{
У французькому виданні Маркс тут наводить відповідну цитату:
«Хоч і як змінюється кількість дібр, що їх робітник дістає за свою
працю, ціна (виражена в праці), яку він платить, завжди лишається
однаковою. Дійсно, за цю ціну можна одного разу купити більшу кількість
цих дібр, а другого — меншу; але змінюється тут вартість цих
дібр, а не вартість праці, яка їх купує\dots{} Однакові кількості праці
завжди мають однакову вартість» (\emph{A. Smith}: «Wealth of Nations», кн. I,
розд. 5). \emph{Ред.}
}

Коли ми, з другого боку, візьмемо капіталіста, то він хоче
дістати якомога більше праці за якомога менше грошей. Тим-то
практично його цікавить лише ріжниця між ціною робочої сили
й тією вартістю, яку створює її функціонування. Але він намагається
купувати всі товари якомога дешевше і завжди пояснює
\parbreak{}  %% абзац продовжується на наступній сторінці
