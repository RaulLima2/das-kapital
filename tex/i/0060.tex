тости на ціну це доконечне відношення з’являється як мінове
відношення даного товару до грошового товару, що існує поза
ним. Але в цьому відношенні так само добре може бути виражена
величина вартости товару, як і той плюс або мінус, з яким його
за даних обставин відчужується. Отже, можливість кількісного
незбігу ціни з величиною вартости, або відхилення ціни від величини
вартости, є в самій формі ціни. Це не вада цієї форми, а
навпаки, це робить її адекватною формою способу продукції,
при якому правило може пробивати собі шлях лише як сліпо
діючий закон пересіччя іреґулярностей (Durchschnittsgesetz der
Regellosigkeit).\footnote*{
У французькому виданні кінець цього речення зформульовано
так: «... способу продукції, при якому правило здійснюється як закон
лише через сліпу гру іреґулярностей, що, у пересічному, компенсують,
паралізують і нищать одна одну». («Le Capital etc», ch. Ill, p. 43). Ред.
}

Однак форма ціни не лише допускає можливість кількісного
незбігу величини вартости з ціною, тобто величини вартости
з її власним грошовим виразом, але може ховати в собі ще й
якісну суперечність, так що ціна взагалі перестає бути виразом
вартости, дарма що гроші є лише форма вартости товарів. Речі,
які сами по собі не є товари, приміром, сумління, честь і т. ін.,
можуть бути продажними для своїх посідачів за гроші і таким
чином через свою ціну набути товарової форми. Отже, річ формально
може мати ціну, не маючи вартости. Вираз ціни стає тут
уявлюваним так само, як і певні величини в математиці. З другого
боку, і уявлювана форма ціни, — як, приміром, ціна некультивованого
ґрунту, що не має вартости, бо в ньому не упредметнено
жодної людської праці, — може ховати в собі дійсне вартостеве
відношення або відношення, вивідне з цього останнього.

Ціна, як і відносна форма вартости взагалі, виражає вартість
якогось товару, приміром, тонни заліза, тим що певна кількість
еквіваленту, наприклад, унція золота, завжди є безпосередньо
вимінна на залізо, тимчасом як залізо, навпаки, аж ніяк не в
безпосередньо вимінне на золото. Отже, щоб практично впливати
як мінова вартість, товар мусить скинути з себе своє природне
тіло, перетворитися з лише уявлюваного золота на дійсне
золото, хоч би таке перевтілення було для нього «прикріше»,
ніж для геґелівського «поняття» перехід від доконечности до
свободи, або для морського рака скинути з себе свою шкаралупу,
абож для св. Єроніма скинути з себе старого Адама.\footnote{
Коли Єронімові замолоду доводилося багато боротися із своєю
матеріяльною плоттю, на що вказує його боротьба в пустелі з образами
гарних жінок, то на старість йому довелось боротися з духовною плоттю.
«Мені здавалось, — каже він, приміром, — що я стою перед суддею світу».
«Хто ти?» — запитав голос. «Я — християнин». — «Брешеш, — гримнув
на мене суддя світу, — ти лише ціцероніянець».
} Поруч
із своєю реальною формою, наприклад, заліза, товар може мати
в ціні ідеальну форму вартости або уявлювану форму золота,
але він не може бути одночасно дійсним залізом і дійсним золотом.
Для того, щоб дати йому ціну, досить прирівняти його до