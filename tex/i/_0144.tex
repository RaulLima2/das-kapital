\parcont{}  %% абзац починається на попередній сторінці
\index{i}{0144}  %% посилання на сторінку оригінального видання
засоби продукції не лише для шестигодинного, але для дванадцятигодинного
пронесу праці. Коли 10 фунтів бавовни вбирали
в себе 6 робочих годин і перетворювались на 10 фунтів пряжі, то
20 фунтів бавовни вберуть у себе 12 робочих годин  перетворяться
на 20 фунтів пряжі. Розгляньмо продукт здовженого процесу
праці. В 20 фунтах пряжі упредметнено тепер 5 робочих днів:
4 — у спожитій кількості бавовни й веретен, а один день бавовна
увібрала в себе протягом процесу прядіння. Але грошовий вираз
5 робочих днів є 30 шилінґів, або 1 фунт стерлінґів і 10 шилінґів.
Отже, це є ціна 20 фунтів пряжі. Фунт пряжі коштує, як і
раніш, 1 шилінґ і 6 пенсів. Але сума вартости товарів, кинутих у
процес, становила 27 шилінґів. Вартість пряжі становить 30 шилінґів.
Вартість продукту зросла на одну дев’яту понад вартість,
авансовану на його продукцію. Таким чином 27 шилінґів перетворились
на 30 шилінґів. Вони породили додаткову вартість
у 3 шилінґи. Нарешті трюк удався. Гроші перетворилися
на капітал.

Всі умови проблеми розв’язано й законів товарового обміну
ані трохи не порушено. Еквівалент обмінено на еквівалент.
Капіталіст як покупець платив за кожний товар — бавовну,
веретена, робочу силу — за його вартістю. А потім він зробив
те, що робить кожний інший покупець товарів: він споживав
їхню споживну вартість. Процес споживання робочої сили, який
разом з тим є процес продукції товару, дав продукт у 20 фунтів
пряжі вартістю в 30 шилінґів. Тепер капіталіст повертається
на ринок і продає товари, тимчасом як раніш він на ньому
купував товари. Він продає 1 фунт пряжі за 1 шилінґ 6 пенсів —
ні на шаг більше ні менше від його вартости. І все ж він вилучає
з циркуляції на 3 шилінґи більше, ніж первісно подав до неї.
Цілий цей процес, перетворення його грошей на капітал, відбувається
у сфері циркуляції й відбувається не в ній. Він відбувається
за посередництвом циркуляції, бо зумовлює його купівля
робочої сили на товаровому ринку; не в циркуляції, бо остання
лише підготовляє процес зростання вартости, який відбувається
у сфері продукції. Таким чином «tout pour le mieux dans le meilleur
des mondes possibles».\footnote*{
Все є якнайкраще в цьому якнайкращому з світів. \emph{Ред.}
}

Перетворюючи гроші на товари, що служать за речові
елементи утворення нового продукту або за фактори процесу
праці, прилучаючи до їхньої мертвої предметности живу робочу
силу, капіталіст перетворює вартість — минулу, упредметнену,
мертву працю — на капітал, на вартість, що сама з себе зростає,
на одушевлену потвору, що починає «працювати» так, наче вона
захоплена любовною жагою.

Коли ми тепер порівняємо процес утворення вартости й процес
зростання вартости, то побачимо, що процес зростання вартости є
не що інше, як процес утворення вартости, продовжений за межі
певного пункту. Коли процес утворення вартости триває лише
\parbreak{}  %% абзац продовжується на наступній сторінці
