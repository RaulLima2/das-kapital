\parcont{}  %% абзац починається на попередній сторінці
\index{i}{0516}  %% посилання на сторінку оригінального видання
техніки, — то продуктивніші і, розглядувані щодо розміру
їхньої дієздатности, дешевші машини, знаряддя, апарати і~\abbr{т. ін.}
стають на місце старих. Старий капітал репродукується в продуктивнішій
формі, не кажучи вже про невпинні зміни деталів
у наявних засобах праці. Друга частина сталого капіталу, сировинний
матеріял і допоміжний матеріял, репродукується невпинно
протягом року, а якщо він походить із рільництва, то здебільшого
щорічно. Отже, кожне заведення ліпшої методи й~\abbr{т. ін.}
діє тут майже одночасно і на додатковий капітал і на той капітал,
що вже функціонує. Кожний проґрес на полі хемії не тільки урізноманітнює
число корисних речовин і застосування вже відомих,
поширюючи тим самим разом із зростанням капіталу сферу його
прикладання. Одночасно він навчає повертати екскременти процесу
продукції і споживання назад у кругобіг процесу репродукції,
отже, він створює нові матеріяли для капіталу без попередньої
витрати капіталу. Подібно до того, як через просте підвищення
напруження робочої сили збільшується експлуатація
природних багатств, так само наука й техніка створює для капіталу,
що функціонує, незалежну від даної його величини силу
поширюватись. Вони одночасно впливають і на ту частину первісного
капіталу, що увійшла в стадію свого відновлення. У своїй
новій формі капітал захоплює для себе задурно той суспільний
проґрес, що відбувся за спиною його старої форми. Правда, цей
розвиток продуктивної сили супроводиться частинним зневартненням
капіталів, що функціонують. Оскільки це зневартнення
дає себе гостро відчувати через конкуренцію, головний тягар
його спадає на робітника: капіталіст намагається поповнити свої
втрати через підвищену експлуатацію робітника.

Праця переносить на продукт вартість спожитих нею засобів
продукції. З другого боку, вартість і маса засобів продукції,
що їх пускає в рух дана кількість праці, зростає пропорційно
до того, як праця стає продуктивнішою. Отже, хоч та сама кількість
праці й додає до своїх продуктів завжди лише ту саму суму
нової вартости, а все ж із зростом продуктивности праці зростає
та стара капітальна вартість, яку вона одночасно переносить
на продукти.

Наприклад, коли англійський і китайський прядун працюватимуть
однакове число годин і з однаковою інтенсивністю, то за
тиждень вони обидва вироблять рівні вартості. Не зважаючи
на цю рівність, існує величезна ріжниця між вартістю тижневого
продукту англійця, що працює за допомогою потужного
автомата, і китайця, що має лише самопряд. За той самий час,
за який китаєць випрядає один фунт бавовни, англієць випрядає
декілька сот фунтів. У кілька сот разів більша сума старих вартостей
збільшує вартість продукту англійця, в якому ті старі
вартості зберігаються в новій корисній формі, і таким чином
можуть знову функціонувати як капітал. «1782~\abbr{р.}, — повідомляє
Ф.~Енґельс, — увесь збір вовни за три попередні роки (в Англії)
лежав ще необроблений через брак робітників і мусив би ще лежати,
\index{i}{0517}  %% посилання на сторінку оригінального видання
коли б не прийшли на допомогу нововинайдені машини, що
й перепряли вовну»\footnote{
\emph{F.~Engels}: «Lage der arbeitenden Klasse in England», Leipzig
1845, S. 20. (\emph{Ф.~Енґельс}: «Становище робітничої кляси в Англії», Партвидав
«Пролетар», 1932~\abbr{р.} стор. 62, 63.).
}. Упредметнена у формі машин праця,
певна річ, не створила безпосередньо жодного робітника, але
вона дала змогу невеликому числу робітників через додаток
порівняно невеликої кількости живої праці не тільки продуктивно
спожити вовну й додати до неї нову вартість, а й зберегти її стару
вартість у формі пряжі й~\abbr{т. ін.} Тим самим вона разом з тим дала
засіб і імпульс до поширеної репродукції вовни. Це є природна
властивість живої праці — зберігати стару вартість, створюючи
нову вартість. Тому із зростом дієздатности, розміру й вартости
засобів продукції, отже, з акумуляцією, яка супроводить розвиток
продуктивної сили праці, праця зберігає й увіковічнює в
вавжди нових формах чимраз більшу й більшу капітальну вартість\footnote{
Клясична політична економія через недостатню аналізу процесу
праці й процесу зростання вартости ніколи не розуміла гаразд цього
важливого моменту, репродукції, як це можна, приміром, бачити у Рікарда.
Він каже, наприклад: хоч яка буде зміна продуктивної сили,
«мільйон людей продукує на фабриках завжди ту саму вартість». Це
правда, коли дано екстенсивність і ступінь інтенсивности їхньої праці.
Але це не перешкоджає — і Рікардо недобачає цього в деяких своїх висновках
— тому, що за різної продуктивної сили своєї праці мільйон людей
перетворює на продукт дуже різні маси засобів продукції, а тому й зберігає
у своєму продукті дуже різні маси вартости, отже, вартості виготовлених
ними продуктів є дуже різні. Між іншим, треба сказати, що на цьому
прикладі Рікардо даремно силкувався пояснити Ж.~Б.~Сеєві ріжницю
між споживною вартістю (яку він називає тут wealth, речовим багатством)
і міновою вартістю. Сей відповідає: «Щодо тих труднощів, які
зазначає Рікардо, кажучи, що мільйон людей, вживаючи вдосконаленіших
способів продукції, може спродукувати вдвоє, утроє більше багатств,
не продукуючи більше вартости, то ці труднощі зникнуть, коли ми, як
це й слід, розглядатимемо продукцію як обмін, в якому віддають продуктивні
послуги своєї праці, своєї землі і своїх капіталів, щоб одержати
за це продукти. За допомогою цих продуктивних послуг ми дістаємо всі
продукти, що є на світі\dots{} Отож\dots{} ми є тим багатші, наші продуктивні
послуги мають тим більшу вартість, чим більшу кількість корисних
речей ми дістаємо за ці послуги в обміні, називаному продукцією».
«Quant à la difficulté qu’élève Mr.~Ricardo en disant que, par des procédés
mieux entendus, un million de personnes peuvent produire deux fois,
trois fois autant de richesses, sans produire plus de valeurs, cette diffuculté
n’en est pas une lorsque l’on considère, ainsi qu’on le doit, la production
comme un échange dans lequel on donne les services productifs de son travail,
de sa terre, et de ses capitaux, pour obtenir des produits. C’est par
le moyen de ces sevrices productifs que nous acquérons tous les produits
qui sont au monde\dots{} Or\dots{} nous sommes d’autant plus riches, nos services
productifs ont d’autant plus de valeur, q’uils obtiennent dans l’échange
appelé production, une plus grande quantité de choses utiles»). (\emph{J.~B.~Say}:
«Lettres à M.~Maithus» Paris 1820, p. 168, 169). «Труднощі», які має
вияснити Сей, — вони існують для нього, а не для Рікарда, — є ось у
чому: чому не збільшується вартість споживних вартостей, коли зростає
їхня кількість у наслідок підвищення продуктивної сили праці? Відповідь:
труднощі розв’язується тим, що споживну вартість будемо ласкаві
називати міновою вартістю. Мінова вартість є річ, що так або інакше
(one way or another) зв’язана з обміном. Отже, назвімо продукцію «обміном»
праці й засобів продукції на продукт — і стане ясно як день, що
ми дістанемо тим більше мінової вартости, чим більше споживних вартостей
дає нам продукція. Іншими словами, що більше споживних вартостей,
наприклад, панчіх, дає один робочий день фабрикантові панчіх,
то багатший він на панчохи. Однак раптом Сеєві спадає на думку, що «зі
збільшенням кількости» панчіх їхня «ціна» (яка, природно, не має нічого
спільного з міновою вартістю) падає, «бо конкуренція примушує їх (продуцентів)
віддавати продукти за стільки, скільки вони їм коштують»
(«parce que la concurrence les (les producteurs) oblige à donner les produits
pour ce qu’ils leur coûtent»). Звідки ж береться зиск, коли капіталіст
продає товари за цінами, що їх вони йому коштують? Але облишмо
це. Сей заявляє, що в наслідок підвищеної продуктивности кожен дістав
тепер в обмін за той самий еквівалент дві пари панчіх замість однієї, як
це було раніш, і~\abbr{т. д.} Результат, до якого він доходить, є саме та теза
Рікарда, яку він хотів був збити. Після такої величезної напруги думки
він, тріюмфуючи, звертається до Малтуза з такими словами: «Така, мій
пане, є добре пов’язана доктрина, що без неї, я це заявляю, неможливо
пояснити якнайбільші труднощі в політичній економії й особливо питання,
яким чином можливо, щоб нація стала багатшою тоді, коли вартість її
продуктів меншає хоч багатство і складається з вартостей» («Telle est,
monsieur, la doctrine bien liée sans laquelle il est impossible, le je déclare,
d’expliquer les plus grandes difficultés de l’économie politique et notamment,
comment il se peut qu’une nation soit plus riche lorsque ses produits
diminuent de valeur, quoique la richesse soit de la valeur»). (Там же, стор.
170). Один англійський економіст зауважує з приводу подібних фокусів
у «Листах» Сея: «Ці афектовані манери патякати («those affected ways
of talking») становлять y цілому те, що пан Сей залюбки називає своєю
доктриною і що він радить Малтузові викладати в Гертфорді, як це вже
робиться «в багатьох місцях Европи». Він каже: «Коли в усіх цих тезах
найдете дещо парадоксальним, то погляньте на ті речі, що їх ці тези виражають,
і я смію сподіватися, що вони здаватимуться вам дуже простими
й дуже розумними» («Si vous trouvez une physionomie de paradoxe â
toutes ces propositions, voyez les choses qu’elles expriment, et j’ose croire
qu’elles vous paraîtront fort simples et fort raisonnables»). Безсумнівно,
але в результаті того самого процесу вони видаватимуться всім, чим хочете,
та тільки не оригінальним або важливим». («An Inquiry mto those
Principles respecting the Nature of Demand etc.», p. 116. 110).
}. Ця природна сила праці здається силою самозбереження
\index{i}{0518}  %% посилання на сторінку оригінального видання
того капіталу, до якого долучено працю, — цілком так само,
як суспільні продуктивні сили праці здаються його властивостями,
і так само, як постійне присвоювання додаткової праці
капіталістом здається постійним самозростанням вартости капіталу.
Всі сили праці здаються силами капіталу, як усі форми
вартости товару — формами грошей.

Із зростанням капіталу зростає ріжниця між застосованим
і спожитим капіталом. Інакше кажучи, зростає вартість і речова
маса засобів праці, як от будівлі, машини, дренажні труби, робоча
худоба, апарати всякого роду, — засобів праці, що протягом
довшого або коротшого періоду функціонують в постійно повторюваних
процесах продукції, або служать, щоб досягти певних
корисних ефектів, у повному своєму обсягу, тимчасом як зужитковуються
вони лише поступінно і тому втрачають свою вартість
лише частинами, отже, і лише частинами переносять її на продукт.
У тій самій мірі, в якій ці засоби праці служать як продуктотворці,
не додаючи до продукту вартости, отже, застосовуються
цілком, а споживаються лише частинно, в цій самій мірі,
\parbreak{}  %% абзац продовжується на наступній сторінці
