\parcont{}  %% абзац починається на попередній сторінці
\index{i}{0607}  %% посилання на сторінку оригінального видання
аніж так дуже уславлена малтузіянцями чума в середині
XIV століття. До речі зауважмо, що коли само но собі по-школярському
наївно було прикладати до умов продукції і відповідних
відносин людности XIX століття маштаб XIV століття,
то ця наївність, крім того ще й недобачила, що коли
по цей бік каналу, в Англії, слідом за чумою і зменшенням
людности, що супроводило її, ішло визволення й збагачення
сільської людности, то по той бік каналу, у Франції, слідом за
чумою йшло ще більше поневолення й ще більше зростання
злиднів.\footnoteA{
Через те, що Ірляндію вважають за обітовану землю «принципу
залюднення», Т. Садлер перед опублікуванням своєї праці про
людність видав свою славетну книгу «Ireland, its Evils and their
Remedies», 2 nd, ed. London 1829, де він, порівнюючи статистичні дані
про окремі провінції і про окремі графства в кожній провінції, показує,
що злидні там не просто пропорційні, як того хоче Малтуз, а зворотно
пропорційні до кількости людности.
}

Голод в Ірляндії в 1846~\abbr{р.} знищив понад мільйон людей, але
лише бідняків. Багатству країни він не заподіяв найменшої
шкоди. Наступне по цьому двадцятирічне і все чимраз масовіше
виселення не знищило, як це було за тридцятирічної війни,
разом із людьми і їхніх засобів продукції. Ірляндський геній
винайшов цілком нову методу, немов чарами, переносити бідний
люд за тисячі миль від арени його злиднів. Еміґранти, які переселилися
до Сполучених штатів, щорічно надсилають додому гроші,
— засоби на дорогу тим, що лишилися ще у країні. Кожна
партія, що еміґрує цього року, тягне за собою наступного року
нову партію. Таким чином еміґрація не тільки нічого не коштує
Ірляндії, а ще й становить одну з найдохідніших галузей її експортових
операцій. Нарешті, вона є систематичний процес, що
не лише спричинює тимчасові прогалини в масі людности, але
й щорічно витягує з неї більше людей, ніж заміщує щорічний
приріст, так що абсолютна кількість людности рік-у-рік
меншає.\footnoteA{
За часів від 1851 до 1874~\abbr{р.} загальне число еміґрантів становило
\num{2.325.922}.
}

Які ж були наслідки для тих робітників, що лишилися в Ірляндії
і звільнилися від перелюднення? Наслідки такі: відносне
перелюднення нині таке саме велике, як і перед 1864~\abbr{р.}, заробітна
плата є така сама низька, муки праці ще більше зросли, а злидні
на селі знову загрожують новою кризою. Причини дуже прості.
Революція в рільництві відбувалась рівнобіжно з еміґрацією.
Творення відносного перелюднення йшло ще швидше, ніж абсолютне
зменшення людности. Досить кинути оком на таблицю
С, щоб побачити, як перетворення орної землі на пасовиська
мусило в Ірляндії діяти ще гостріше, ніж в Англії. В Англії разом
із скотарством зростає культура зеленини, в Ірляндії вона падає.
Тимчасом як велику кількість оброблюваних раніш земель лишають
необробленими або перетворюють на постійні луки, більша
\parbreak{}  %% абзац продовжується на наступній сторінці
