шена витрата зумовлює збільшене відновлення.43 Коли власник
робочої сили працював сьогодні, то він мусить мати змогу й завтра
повторити той самий процес за тих самих умов сили й здоров’я.
Отже, суми засобів існування мусить вистачати, щоб індивіда,
який працює, утримати як такого в його нормальному життєвому
стані. Сами природні потреби, як от їжа, одяг, паливо, помешкання
тощо, є різні залежно від кліматичних і інших природних
особливостей країни. З другого боку, розміри так званих доконечних
потреб, як і спосіб задоволення їх, сами є історичний
продукт і тому здебільша залежать од культурного рівня країни
і, між іншим, значною мірою і від того, серед яких умов, отже,
з якими звичками й життєвими вимогами утворилася кляса вільних
робітників.\footnote{
Порівн. «Overpopulation and its Remedy», London 1846, von W. Th.
Thornton.
} Отже, протилежно до інших товарів визначення
вартости робочої сили містить у собі історичний і моральний
елемент. Однак для певної країни і для певного періоду пересічна
кількість доконечних засобів існування є дана.

Власник робочої сили є смертний. Отже, щоб раз-у-раз
з’являтися на ринку, а це є передумова безупинного перетворювання
грошей на капітал, продавець робочої сили мусить увічнити
себе, «як увічнює себе кожний живий індивід — через
розплодження».\footnote{
Petty.
} Робочі сили, що сходять із ринку через виснаження
і смерть, мусять постійно поповнюватись, принаймні,
таким самим числом нових робочих сил. Отже, сума засобів
існування, доконечних для продукції робочої сили, обіймає й
засоби існування для заступників, тобто дітей робітників, так
що ця раса своєрідних товаропосідачів увічнюється на товаровому
ринку.\footnote{
«Її (праці) натуральна ціна... складається з такої кількости засобів
існування і комфорту, яка, залежно від природи й клімату й відповідно
до звичаїв даної країни, доконечна для утримання самого робітника
й для того, щоб дати йому змогу утримувати родину, яка спроможна
була б забезпечити незменшуване постачання праці на ринку» («Its
g (labour’s) natural price... consists in such a quantity of necessaries, and
comforts of life, as, from the nature of the climate, and the habits of the
country, are necessary to support the labourer, and to enable him to rear
such a family as may preserve, in the market, an undiminished supply of
labour»). (R. Torrens: «An Essay on the external Corn Trade», London
1815, p. 62). Слово «праця» тут неправильно поставлено замість «робоча
сила».
}

Для модифікації загальної людської натури в такий спосіб,
щоб людина набула вправности та досвіду в певній галузі праці,
щоб стала розвинутою й специфічною робочою силою, потрібно
певної освіти або виховання, яке, із свого боку, коштує більшої
або меншої суми товарових еквівалентів. Ця сума витрат виховання
змінюється залежно від більш чи менш складного харак-

41 «Тому староримський villicus, що стояв як управитель на чолі
рільничих рабів, діставав «злиденніше утримання, ніж раби, бо мав
легшу працю». (Th. Mommsen: «Römische Geschichte», 1856, стор. 810).