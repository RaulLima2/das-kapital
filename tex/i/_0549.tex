\parcont{}  %% абзац починається на попередній сторінці
\index{i}{0549}  %% посилання на сторінку оригінального видання
підвищення її. Приміром, коли тижнева заробітна плата становить
20 шилінґів і підвищується до 22 шилінґів, то це є підвищення
на 10\%; навпаки, коли вона становить лише 7 шилінґів
і підвищується до 9, то це є підвищення на 28\sfrac{4}{7}\%, а це вже
звучить дуже значно. В усякому разі фармери зчинили галас,
і навіть «London Economist» з приводу цих голодних заробітків
цілком серйозно почав верзти про «а general an substantial
advance» («загальний і значний проґрес»).84 А що зробили
фармери? Може вони очікували, доки в наслідок цієї блискучої
плати сільські робітники так дуже порозмножаться, щоб їхня
заробітна плата мусила знову впасти, як то уявляє собі справу
догматичний мозок економіста? Вони завели більше машин,
і вмить робітники знову стали «зайвими» в такій мірі, що задовольнила
навіть фермерів. Тепер у рільництво вкладено
«більше капіталу», ніж перед тим, і в продуктивнішій формі.
Разом з тим попит на працю зменшився не лише відносно,
а й абсолютно.

Ця економічна фікція сплутує закони, що реґулюють загальний
рух заробітної плати або відношення між робітничою клясою,
тобто сукупною робочою силою, і сукупним суспільним капіталом,
із тими законами, що розподіляють робітничу людність
між окремими сферами продукції. Коли, наприклад, у наслідок
сприятливої коньюнктури акумуляція в певній галузі продукції
особливо жвава, зиски в ній вищі від пересічних, додатковий
капітал рине туди, то, звичайно, більшає попит на працю й заробітна
плата. Вища заробітна плата притягає більшу частину
робітничої людности до тієї сфери, що є в сприятливому стані,
доки її не насичується робочою силою, а заробітна плата на довгий
час знову не спаде до свого попереднього пересічного рівня,
а то й нижче його, якщо приплив був занадто великий. Тоді
приплив робітників до даної галузі промисловості! не тільки
припиняється, але навіть відступає місце відпливові. Тут політико-економ
уявляє собі, ніби він бачить, «де і яким чином»
із збільшенням заробітної плати відбувається абсолютне збільшення
числа робітників, а з абсолютним збільшенням числа робітників
— зменшення заробітної плати, але в дійсності він бачить
лише місцеве коливання ринку праці в якійсь окремій сфері
продукції, він бачить лише явища розподілу робітничої людности
між різними сферами вкладання капіталу залежно від змінних
потреб капіталу.

Промислова резервна армія підчас застою й середнього процвітання
тисне на активну робітничу армію й загнуздує її домагання
підчас періоду перепродукції й пароксизмів. Отже, відносно
перелюднення — це той стрижень, що навколо нього рухається
закон попиту й подання праці. Воно втискує поле діяльности
цього закону в межі, що абсолютно відповідають жадобі капіталу
до експлуатації й панування. Тут буде до речі вернутися до одного

81 «Economist», Januar, 21, 1860.
\parbreak{}  %% абзац продовжується на наступній сторінці
