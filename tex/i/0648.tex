Різні моменти первісної акумуляції капіталу розподіляються
тепер у більш-менш хронологічному порядку між різними країнами,
головним чином між Еспанією, Портуґалією, Голляндією,
Францією та Англією. В Англії наприкінці XVII століття вони
систематично об’єднуються в колоніяльну систему, систему державних
позик, сучасну податкову систему й систему протекціонізму.
Ці методи ґрунтуються почасти на найбрутальнішому
насильстві, наприклад, на колоніяльній системі. Але всі вони
використовували державну владу, концентроване й організоване
насильство суспільства, щоб надзвичайно прискорити процес
перетворення февдального способу продукції на капіталістичний
спосіб продукції і скоротити його переходові стадії. Насильство —
це є повитуха кожного старого суспільства, вагітного новим.
Саме насильство є економічний фактор (Potenz).

Про християнську колоніяльну систему В. Гавітт, людина,
що робить з християнства свою спеціяльність, каже ось що:
«Варварство й нечестиві жорстокі вчинки так званих християнських
рас, по всіх частинах світу й проти всіх народів, що їх
вони мали змогу підкорити, не находять собі жодної паралелі
ані в одній добі світової історії, ані в якої раси, хоч би яка вона
була дика, темна, немилосерда й безсоромна».241 Історія голляндського
колоніяльного господарства — а Голляндія була
зразковою капіталістичною нацією XVII століття — «розгортає
перед нами незрівнянну картину зради, підкупу, вбивства
й підлоти».242 Немає нічого характеристичнішого для Голляндії,
як її система крадежу людей на Целебесі, щоб добувати рабів
для острова Яви. Для цієї мети муштровано спеціяльних викрадачів
людей. Злодій, перекладач і продавець були головними аґентами
цієї торговлі людьми, а тубільні принци — головними продавцями.
Покрадену молодь переховували по целебеських таємних в’язницях,
тримаючи їх там доти, доки вже дорослими їх можна було
відправляти на кораблях, навантажених рабами. Один офіціяльний
звіт каже: «В місті Макасар, наприклад, є повно таємних в’язниць,
одна жахливіша за одну, і в кожній повнісінько нещасних
жертов ненажерливости й тиранства, позаковуваних у кайдани
й силоміць відірваних від їхніх родин». Щоб захопити Малаку,
голляндці підкупили португальського губернатора. Року 1641
він упустив їх до міста. Вони одразу ж подалися до його будинку
й вбили його потайки, щоб «здержатись» від виплати умовленої
суми підкупу в 21.875 фунтів стерлінґів. Спустошення і збезлюднення
йшло за ними всюди, де ступала їхня нога. Баньюванґі,

241 William Howitt: «Colonization and Christianity. A Popular History
of the Treatment of the Natives by the Europeans in all their Colonies»,
London 1838, p. 9. Про поводження з рабами є добра компіляція в Charles
Comte: «Traité de la Legislation», 3-me éd., Bruxelles 1837. Треба детально
простудіювати цю працю, щоб побачити, що робить буржуа з
самого себе і своїх робітників там, де він може без перешкод перероблювати
світ па свій образ і свою подобу.

242 Thomas Stamford Raffles, late Lieut. Gov. of that island: «Java
and its dependencies», London 1877.
