припускаючи, що вартість грошей не змінюється, — продукує
завжди ту саму нову вартість у 6 шилінґів, хоч би як ця сума
вартости поділялась між еквівалентом вартости робочої сили й
додатковою вартістю. Але якщо внаслідок підвищення продуктивної сили праці вартість денних засобів
існування, а тому й
денна вартість робочої сили падає з 5 шилінґів до 3, то додаткова
вартість зростає з 1 шилінґа до 3 шилінґів. Щоб репродукувати
вартість робочої сили раніше було потрібно 10, а тепер треба
лише 6 робочих годин. Чотири робочі години стали вільні, і їх
можна прилучити до сфери додаткової праці. Звідси іманентне
прагнення й постійна тенденція капіталу підвищувати продуктивну
силу праці, щоб здешевити товари та через здешевлення товарів
здешевити самого робітника.5

Абсолютна вартість товару для капіталіста, що його продукує, сама по собі байдужа. Капіталіста
цікавить лише додаткова
вартість, що міститься в товарі й що її можна реалізувати в продажі. Реалізація додаткової вартости
включає й повернення авансованої вартости. А що відносна додаткова вартість зростає просто
пропорційно до розвитку продуктивної сили праці, тимчасом як
вартість падає зворотно пропорційно до того самого розвитку,
отже, що той самий ідентичний процес здешевлює товари та збільшує додаткову вартість, яка міститься
в них, то й розв’язується
та загадка, що капіталіст, який дбає лише про продукцію мінової
вартости, постійно намагається знизити мінову вартість товарів, —
суперечність, якою один з основників політичної економії, а
саме Кене, мучив своїх супротивників, що так і не дали йому на
неї відповіді. «Ви визнаєте, — каже Кене, — що чим більше
можна без шкоди для продукції заощадити витрат та зменшити
дорогі роботи при фабрикації промислових продуктів, тим корис-

5 «У тій самій пропорції, в якій меншають видатки робітника, буде
зменшено і його заробітну плату, якщо тільки разом з цим промисловість
звільняють від обмежень» («In whatever proportion the expenses of a
labourer are diminished, in the same proportion will his wages be diminished,
if the restraints upon industry are at the same time taken off»).
(«Considerations concerningt taking off the Bounty on Corn exported etc.»,
London 1752, p. 7). «Інтереси промисловости вимагають, щоб хліб і
взагалі всякі харчові речі були якомога дешевші: бо те, що їх робить
дорожчими, робить дорожчою і працю... по всіх країнах, де промисловість вільна від обмежень, ціна на
предмети харчування мусить впливати
на ціну праці. Цю останню завжди понижують, коли дешевшають потрібні
засоби існування». («The interest of trade requires, that corn and all provisions should be as cheap
as possible; for whatver makes them dear, must
make labour dear also... in all countries, where industry is not restrained,
the price of provisions must affect the Price of Labour. This will always
be diminished when necessaries of life grow cheaper»). (Там же, стор. 3).
«Заробітну плату понижують в тій самій пропорції, в якій зростають
продуктивні сили. Правда, машини здешевлюють засоби існування, але
вони також і робітників роблять дешевшими». («Wages are decreased
in the same proportion as the powers of production increase. Machinery,
it is true, cheapens the necessaries of life, but it also cheapens the labourer
too»). («А Prize Essay on the comparative merits of Competition and
Cooperation», London 1834, p. 27).
