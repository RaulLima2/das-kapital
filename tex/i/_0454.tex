\parcont{}  %% абзац починається на попередній сторінці 
\index{i}{0454}  %% посилання на сторінку оригінального видання 
новими цінами праці й її так званою вартістю, відношенням цієї
вартости до норми зиску, до товарових вартостей, продукованих
працею, і т. ін., клясична політична економія ніколи не могла
відкрити, що хід аналізи веде не тільки від ринкових цін праці
до її так званої вартости, але ще й до того, щоб і цю вартість
самої праці звести до вартости робочої сили. Несвідома цього
результату своєї власної аналізи, некритично приймаючи категорії
«вартість праці», «природна ціна праці» й т. ін., як останні
адекватні вирази розглядуваного відношення вартости, клясична
політична економія заплуталася, як побачимо пізніше, в нерозв’язній
плутанині й суперечностях, давши тим самим надійну
операційну базу для банальностей вульґарної політичної економії,
яка принципово визнає лише зовнішню видимість явищ.

Погляньмо тепер насамперед, як вартість і ціна робочої сили
виражаються у своїй перетвореній формі, у формі заробітної
плати.

Ми знаємо, що денну вартість робочої сили обчислюється
відповідно до певного протягу життя робітника, що йому відповідає
певна довжина робочого дня. Припустімо, що звичайний
робочий день становить 12 годин, а денна вартість робочої сили
З шилінґи — грошовий вираз вартости, що репрезентує 6 робочих
годин. Якщо робітник одержує 3 шилінґи, то він одержує вартість
своєї робочої сили, яка функціонує протягом 12 годин.
Якщо тепер цю денну вартість робочої сили виразити як вартість
денної праці, то матимемо таку формулу: дванадцятигодинна
праця має вартість у 3 шилінґи. Таким чином вартість
робочої сили визначає вартість праці, або, виражаючи в грошах,
доконечну ціну праці. Навпаки, якщо ціна робочої сили
відхиляється від її вартости, то так само й ціна праці відхиляється
від її так званої вартости.

А що вартість праці є лише іраціональний вираз вартости
робочої сили, то само собою випливає, що вартість праці завжди
мусить бути менша, ніж спродукована працею вартість, бо капіталіст
завжди примушує робочу силу функціонувати довше,
ніж треба для репродукції її власної вартости. У вищенаведеному
прикладі вартість робочої сили, яка функціонує протягом
12 годин, дорівнює 3 шилінґам — вартості, що на її репродукцію
робоча сила потребує 6 годин. Навпаки, спродукована
нею вартість дорівнює 6 шилінґам, бо вона фактично функціонувала
12 годин, а спродукована нею вартість залежить не від її
власної вартости, а від часу тривання її функції. Таким чином
ми маємо той на перший погляд недоладний результат, що праця,
яка творить вартість у 6 шилінґів, має вартість у 3 шилінґи.\footnote{
Порівн. «Zur Kritik, der Politischen Oekonomie», Berlin 1859,
S. 40. («До критики політичної економії», ДВУ 1926 р., стор. 78), де я
зазначаю, що, при досліджуванні капіталу треба розв’язати таку проблему:
«Яким чином продукція на базі мінової вартости, яка визначається
лише робочим часом, веде до того результату, що мінова вартість
праці менша за мінову вартість її продукту?»
}
