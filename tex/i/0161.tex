Інша трудність випливає з первісної форми змінного капіталу.
Так, у вищенаведеному прикладі С' = 410 фунтів стерлінґів сталого
капіталу + 90 фунтів стерлінґів змінного капіталу + 90 фунтів
стерлінґів додаткової вартости. Але 90 фунтів стерлінґів
є дана, отже, стала величина, і тому здається недоречною річчю
розглядати її як змінну величину. Але 90v фунтів стерлінґів або
90 фунтів стерлінґів змінного капіталу є тут у дійсності лише
символ того процесу, що його перебігає ця вартість. Частина
капіталу, авансована на купівлю робочої сили, є певна кількість
упредметненої праці, отже, є така ж стала величина вартости,
як і вартість купленої робочої сили. Але в самому процесі продукції
на місце авансованих 90 фунтів стерлінґів виступає діюща
робоча сила, на місце мертвої — жива праця, на місце величини,
що перебуває в стані спокою, — рухлива величина, на місце сталої
— змінна. Результат цього процесу є репродукція υ плюс
приріст υ. З погляду капіталістичної продукції цілий цей процес
є самовільний, автоматичний рух первісної сталої вартости,
перетвореної на робочу силу. Цій вартості приписується процес
і його результат. Тим то, коли формула: 90 фунтів стерлінґів
змінного капіталу, або вартість, що самозростає, видається суперечною,
то вона виражає лише суперечність, іманентну капіталістичній
продукції.

На перший погляд видається дивним, що сталий капітал
прирівнюється нулеві. А проте це така операція, що її постійно
робиться в щоденному житті. Приміром, коли хто хоче обчислити
зиск Англії з бавовняної промисловости, то він насамперед віднімає
ціну бавовни, виплачену Сполученим Штатам, Індії, Єгиптові
й т. ін., тобто прирівнює нулеві ту капітальну вартість, що
лише знову з’являється у вартості продукту.

Звичайно, відношення додаткової вартости не лише до тієї
частини капіталу, що з неї вона безпосередньо виникає й що зміну
її вартости вона виражає, а й до цілого авансованого капіталу,
має своє велике економічне значення. Тим-то у третій книзі ми
докладно розглядаємо це відношення. Для того, щоб одна частина
капіталу через перетворення її на робочу силу виросла в своїй
вартості, друга частина капіталу мусить бути перетворена на
засоби продукції. Для того, щоб змінний капітал функціонував,
сталий капітал мусить бути авансований у відповідних пропорціях,
залежно від певного технічного характеру процесу праці.
Однак та обставина, що для певного хемічного процесу треба
реторт і іншого посуду, не заважає в аналізі абстрагуватися від
самої реторти. Якщо творення вартости й зміну вартости розглядати
сами по собі, тобто в чистій формі, то засоби продукції, ці
речові форми сталого капіталу, постачають лише речовину, в
якій повинна застигнути текуча сила, що творить вартість. Тим
то й природа цієї речовини не має значення, тобто не має значення,
чи буде це бавовна чи залізо. І вартість цієї речовини теж не має
значення. Вона лише мусить бути наявна в такій достатній масі,
