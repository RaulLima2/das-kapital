\parcont{}  %% абзац починається на попередній сторінці
\index{i}{0471}  %% посилання на сторінку оригінального видання
куди незначніша, ніж колибудь за попередніх періодів.\footnote{
«Remarks on the Commercial Policy of Great Britain», London
1815, p. 48.
} Як
мало користи дали сільському пролетаріятові підвищена інтенсивність
і здовження праці, посталі разом з відштучною платою,
показує дальше місце, запозичене з одного партійного твору,
написаного в оборону лендлордів та орендарів: «Переважну
частину рільничих операцій виконують люди, яких наймають
поденно або відштучно... їхня тижнева плата становить приблизно
12 шилінґів, і хоч можна припустити, що робітник за відштучної
плати, через більше стимулювання до праці, заробить на 1 або,
може, і на 2 шилінґи більше, ніж за потижневої плати, проте,
цінуючи цілий його дохід, ми бачимо, що його втрати від безробіття
протягом року зрівноважують цей додатковий заробіток...
Далі ми побачимо взагалі, що заробітні плати цих робітників
стоять у певному відношенні до ціни доконечних засобів існування,
так що робітник з двома дітьми може утримати свою родину,
не вдаючись по допомогу до парафії».\footnote{
«А Defence of the Landowners and Farmers of Great Britain», London
1814, p. 4, 5.
} [Отже, коли б цей
робітник мав трьох дітей, то йому довелося б удаватися по допомогу
до суспільної благодійности].\footnote*{
Заведене у прямі дужки ми беремо з французького видання. Ред.
} Малтуз зауважив тоді з
приводу опублікованих парляментом фактів: «Признаюся, я з
незадоволенням дивлюсь на велике поширення практики відштучної
плати. Справді, тяжка праця протягом 12 або 14 годин
денно на якийсь більш-менш довгий період часу — це надто багато
для людської істоти».\footnote{
Malthus: «Inquiry into the Nature etc. of Rent.», London 1815,
p. 49 Note.
}

В майстернях, підпорядкованих фабричному законові, відштучна
плата стає загальним правилом, бо тут капітал може збільшувати
робочий день тільки через підвищення інтенсивности».\footnote{
«Робітники відштучної плати становлять, мабуть, 4/s усіх робітників
на фабриках». («Reports of Insp. of Fact, for 30 th April
1858», p. 9).
}

Із зміною продуктивности праці змінюється й робочий час,
що його репрезентує та сама кількість продукту. Отже, змінюється
й відштучна плата, бо вона є вираз ціни певного робочого
часу. В нашому вищенаведеному прикладі за 12 годин спродуковано
24 штуки, тимчасом як вартість, спродукована за 12 годин,
була 6 шилінґів, денна вартість робочої сили — 3 шилінґи,
ціна робочої години — 3 пенси й заробітна плата за штуку —
1 1/2 пенса. Одна штука вбирала в себе 1/2 робочої години. Якщо
той самий робочий день даватиме, приміром, у наслідок подвоєння
продуктивности праці 48 штук замість 24 і якщо всі інші
обставини лишаться незмін іі, то відштучна плата зменшиться
з 1 1/2 пенса на 3/4 пенса, бо тепер кожна штука репрезентує лише
У± замість Уг робочої години. 1\% пенсах24 = 3 шилінґам, і так
само 3/4 пенса х 48 = 3 шилінґам. Інакше кажучи, відштучну плату
\parbreak{}  %% абзац продовжується на наступній сторінці
