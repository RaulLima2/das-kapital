
\index{i}{0440}  %% посилання на сторінку оригінального видання
\manualpagebreak{}
\subsection{Сталий робочий день, стала продуктивна сила праці,
інтенсивність праці змінюється}

Зростання інтенсивносте праці припускає збільшену витрату
праці протягом того самого часу. Тому інтенсивніший робочий
день утілюється в більшій кількості продуктів, аніж менш
інтенсивний з тим самим числом годин. Правда, і при підвищенні
продуктивної сили праці той самий робочий день дає
більше продуктів. Але в останньому випадку вартість одиниці
продукту спадає, бо він коштує менше праці, ніж раніш; у першому
ж випадку вона лишається незмінна, бо продукт коштує
стільки ж праці, що й раніш. Число продуктів тут зростає, але
ціна їхня не спадає. З їхнім числом зростає сума їхніх цін, тимчасом
як при підвищенні продуктивної сили праці та сама сума
вартосте лише виражається в збільшеній масі продуктів. Отже,
при однаковому числі годин інтенсивніший робочий день утілюється
в більшій новоспродукованій вартості, отже, при незмінній
вартості грошей, — у більшій кількості грошей. Новоспродукована
протягом нього вартість змінюється відповідно до того,
як відхиляється його інтенсивність від нормального суспільного
ступеня. Отже, той самий робочий день виражається не в сталій,
як раніш, а у змінній новоспродукованій вартості, наприклад,
інтенсивніший дванадцятигодинний робочий день у 7\shil{ шилінґах},
8\shil{ шилінґах} і~\abbr{т. д.} замість 6\shil{ шилінґів}, як це було за дванадцятигодинного
робочого дня звичайної інтенсивносте. Ясно, що коли
новоспродукована протягом робочого дня вартість змінюється,
скажімо, з 6\shil{ шилінґів} на 8\shil{ шилінґів}, то й обидві частини цієї
новоспродукованої вартости, ціна робочої сили й додаткова вартість,
можуть одночасно зростати, в однаковій або неоднаковій
мірі. Ціна робочої сили й додаткова вартість можуть одна й друга
зрости одночасно з 3\shil{ шилінґів} до 4, коли новоспродукована вартість
зростає з 6 до 8\shil{ шилінґів.} Тут підвищення ціни робочої
сили не включає неодмінно підвищення її ціни понад її вартість.
Навпаки, воно може супроводитися падінням її вартости. Це
завжди буває тоді коли підвищення ціни робочої сили не компенсує
її прискореного зужитковування.

Ми знаємо, що, за деякими минущими винятками, зміна
продуктивности праці тільки тоді викликає зміну величини вартости
робочої сили а тому й величини додаткової вартости, коли
продукти розглядуваної галузі промисловости стають предметами
звичайного споживання робітника. Тут ця межа відпадає.
Хоч величина праці змінюється екстенсивно або інтенсивно, цій
зміні її величини відповідає зміна величини новоспродукованої
нею вартости, незалежно від природи того предмету, в якому
ця вартість виражається.

Коли б інтенсивність праці одночасно й рівномірно підвищилася
в усіх галузях промисловости, то новий вищий ступінь інтенсивности
праці став би звичайним, нормальним суспільним ступенем,
і тому перестав би вважатись за екстенсивну величину.
\parbreak{}  %% абзац продовжується на наступній сторінці
