талу,\footnote{
Чому ця іманентна суперечність не доходить до свідомости поодинокого
капіталіста, а тому і до свідомости політичної економії, яка поділяє
його погляд, це ми побачимо з перших відділів третьої книги.
} знову таки спонукає його до найбільш насильного здовжування
робочого дня, щоб зменшення відносного числа експлуатованих
робітників компенсувати збільшенням не тільки
відносної, але й абсолютної додаткової праці.

Отже, коли капіталістичне вживання машин, з одного боку,
утворює нові могутні мотиви до безмірного здовжування робочого
дня та робить переворот у самому способі праці й характері суспільного
робочого організму в такий спосіб, що ламає опір супроти
цієї тенденції, то, з другого боку, воно, почасти через підбивання
під владу капіталу неприступних раніш верств робітничої кляси,
почасти через звільнення витиснених машиною робітників, продукує
надмірну робітничу людність,\footnote{
Одна з великих заслуг Рікарда\footnote*{
У французькому виданні тут сказано: «Одна з заслуг Сісмонді
та Рікарда в тому, що вони зрозуміли і т. д.». Ред.
} в тому, що він зрозумів, що
машина є засіб продукції не тільки товарів, але й «redundant
population».\footnote*{
— надмірної людности. Ред.
}
} яка примушена коритися
законові, що його диктує їй капітал. Звідси те варте уваги явище в
історії сучасної промисловости, що машина нищить усі моральні та
природні межі робочого дня. Звідси той економічний парадокс, що
наймогутніший засіб до скорочення робочого часу перетворюється
в найпевніший засіб до того, щоб увесь час життя робітника та
його родини перетворити на робочий час, що ним порядкує капітал
для збільшення своєї вартости. «Коли б, — мріяв Арістотель,
цей найбільший мислитель старовини, — коли б кожне знаряддя
могло з наказу або передчування виконувати свою власну працю
так, як майстерні витвори Дедала рухалися сами собою, або як
триніжки Гефеста з своєї власної охоти заходжувалися коло святої
праці, коли б так сами собою ткали ткацькі човники, то не потрібні
були б ані майстрові помічники, ані панові раби».\footnote{
F. Biese: «Die Philosophie des Aristoteles», Berlin 1842, Bd. II,
S. 408.
} І Антіпарос,
грецький поет за часів Ціцерона, вітав винахід водяного
млина, щоб молоти мливо, цю елементарну форму кожної
продуктивної машини, як визвольника рабинь і відновника золотого
віку!\footnote{
Подаємо тут Штольберґів переклад цього вірша, бо вірш цей,
цілком так само, як і наведені вище цитати про поділ праці, характеризує
протилежність між античними та сучасними поглядами.

«Schonet der mahlenden Hand, о Müllerinnen, und schlafet
Sanft! Es verkünde der Hahn euch den Morgen umsonst!
Däo hat die Arbeit der Mädchen den Nymphen befohlen,
Und itzt hüpfen sie leicht über die Räder dahin,
Dass die erschütterten Achsen mit ihren Speichen sich wälzen,
Und im Kreise die Last drehen des wälzenden Steins.
}

«Поганці, ах, ті поганці!» Вони, як це виявив мудрий Бастія,