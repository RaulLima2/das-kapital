\parcont{}  %% абзац починається на попередній сторінці
\index{i}{0189}  %% посилання на сторінку оригінального видання
1845~\abbr{р.} вона швидко розвинулася в Англії і з густо заселених
частин Лондону поширилась особливо на Менчестер, Бермінґем,
Ліверпул, Брістол, Норвіч, Ньюкестл, Ґлезґо, розносячи з
собою спазматичне стиснення щелеп — недугу, яку один віденський
лікар уже в 1845~\abbr{р.} визнав за своєрідну хоробу сірничкарів.
Половина робітників — це діти, молодші 13 років, і
підлітки, молодші 18 років. Унаслідок шкідливого впливу на
здоров’я та гидких умов праці в цій мануфактурі, про неї пішла
така погана слава, що лише найзанепаліша частина робітничої
кляси, напівголодні вдовиці тощо постачає для неї дітей — «обідраних,
напівголодних, цілком занехаяних і невихованих дітей»\footnote{
Там же, стор. LIV.
}.
З тих свідків, яких переслухував член комісії Байт (1863~\abbr{р.}),
270 не мали 18 років, 50 мало менш як 10 років, 10 були лише
у 8 і 5 лише в 6 році життя. Робочий день, що довжина його коливається
між 12, 14 і 15 годинами, нічна праця, нереґулярний
час для їжі, при чому їсти здебільшого доводиться в самих майстернях,
затруєних фосфором. Данте знайшов би, що ця мануфактура
перевищує всі його найжахніші фантазії про пекло.

На фабриці шпалер грубші сорти їх друкується машинами,
кращі — руками (block printing). Найжвавіший період перебігу
справ припадає на час від початку місяця жовтня до кінця
квітня. Протягом цього періоду праця триває часто від 6 години
ранку до 10 години вечора, а то й до пізньої ночі, і майже без
перерви.

Дж.~Леч свідчить: «Останньої зими (1862~\abbr{р.}) із 19 дівчат 6
не могли працювати далі, захорівши від надмірної праці. Щоб не
дати їм заснути, я мусив на них кричати». В.~Деффі: «Від утоми
діти часто не могли тримати очей розплющеними; в дійсності й
нам самим це часто ледве вдається». Дж.~Ляйтбурн: «Мені
13 років\dots{} минулої зими ми працювали до 9 години вечора, а
зиму перед тим до 10 години. Останньої зими я майже щовечора
кричав від болю на поранених ногах». Ґ.~Епеден: «Цього мого
хлопчика, коли йому було лише 7 років, я день-у-день носив
на плечах туди й назад по снігу, і він звичайно працював по
16 годин\elli{!..} Часто я ставав навколішки, щоб нагодувати його в той
час, як він сам стояв біля машини, бо він не смів ні відійти від
неї, ні припинити її». Сміс, компаньйон-управитель однієї
менчестерської фабрики: «Ми (він має на гадці ті його «руки»,
що працюють на «нас») працюємо без перерви на їжу, так що
десятигодинний робочий день кінчається о 4\sfrac{1}{2}  годині вечора, а
все, що далі, є надробочий час»\footnote{
Цього не можна вважати за додатковий робочий час у нашому розумінні.
Ці пани розглядають 10\sfrac{1}{2}-годинну працю як нормальний
робочий день, що містить у собі отож і нормальну додаткову працю. Опісля
починається «надробочий час», оплачуваний трохи краще. Пізніш при
пагоді ми побачимо, що вживання робочої сили протягом так званого
нормального дня оплачується нижче вартости, так що «надробочий
час» — то лише хитрощі капіталістів, щоб видушити більше «додаткової
праці»; зрештою, це лишається без зміни й тоді, коли робочу силу, вживану
протягом «нормального дня», дійсно оплачується цілком.
}. (Не знати, чи той пан Сміс
дійсно нічого не їсть протягом  10\sfrac{1}{2} годин?) «Ми (той самий
Сміс) рідко коли кінчаємо працю перед 6 годиною вечора (він
\index{i}{0190}  %% посилання на сторінку оригінального видання
має на думці: споживання «наших» живих машин, тобто робочу
силу), так що фактично ми (iterum Crispinus) працюємо надробочий
час протягом цілого року\dots{} Діти й дорослі (152 дітей і
підлітків нижче 18 років і 140 дорослих) однаково працювали
протягом останніх 18 місяців — пересічно щонайменше по 7 днів
і 5 годин тижнево, або 78\sfrac{1}{2} годин на тиждень. Для 6 тижнів,
кінчаючи 2 травня поточного року (1863), пересічний результат
був вищий — 8 днів, або 84 години на тиждень!» Однак  цей самий
пан Сміс, що відчуває таку велику симпатію до pluralis majestatis\footnote*{
Множина величности, тобто вживання займенника «ми» замість
«я», як це робили владарі й царі. \emph{Ред.}
},
додає підсміхаючись: «Машинова праця легка». А фабриканти,
що вживають Block Printing, кажуть: «Ручна праця
здоровіша від машинової». Загалом же пани фабриканти з обуренням
висловлюються проти проекту «припиняти машини бодай
на час їжі». Пан Отлей, управитель фабрики шпалер у Боро
(у Лондоні), каже: «Закон, який дозволяв би нам робочий день
від 6 години вранці до 9 години вечора, був би нам (!) дуже до вподоби,
але робочий день від 6 години ранку до 6 години вечора,
який приписує Factory Act, нам (!) не підходить\dots{} Ми спиняємо
свою машину підчас обіду (що за великодушність!). Це припинення
не спричинює жодної вартої згадки втрати на папері й фарбах».
«Але, — додає він із співчуттям, — я добре розумію, що втрата,
сполучена з цим, не дуже приємна річ». Звіт комісії наївно гадає,
що страх деяких «видатних фірм» утратити час, тобто час, протягом
якого присвоюється чужу працю, і через те «втратити зиск»,
що цей страх не є ще «достатня основа», щоб дітей, молодших
від 13 років, і молодь до 18 років «позбавляти їжі» протягом
12--16 годин, або щоб їм постачали харч так, як засобам праці
постачають допоміжні матеріяли: машині — воду й вугілля,
вовні — мило, колесам — мастиво і~\abbr{т. ін.}, тобто підчас самого
процесу продукції\footnote{
«Children’s Employment Commission, 1863», Evidence, p. 123,
124, 125, 140 і LIV.
}.

В жодній галузі промисловости в Англії — (ми лишаємо осторонь
машинове виробництво хліба, яке тільки-но починає прокладати
собі шлях) — не зберігся донині такий стародавній
і навіть, як це можна побачити в поетів з часів Римської імперії,
дохристиянський спосіб продукції, як пекарство. Але капітал,
як зазначено раніш, є спочатку байдужий щодо технічного характеру
того процесу праці, який він опановує. Він бере його
спочатку таким, яким його находить.

Неймовірну фальсифікацію хліба, особливо в Лондоні, відкрив
уперше комітет Палати громад у справі «про фалшування
харчів» (1855/56~\abbr{р.}) і твір д-ра Гесселя: «Adulterations
\index{i}{0191}  %% посилання на сторінку оригінального видання
detected»\footnote{
Галун, дрібно розтертий, або змішаний із сіллю, є нормальний
предмет торговлі, що має характеристичну назву «baker’s stuff» («порошок
пекарів»).
}. Як наслідок того відкриття оголошено закона з 6 серпня
1860~\abbr{р.}: «for preveting the adulteration of articles of food
and drink»\footnote*{
Про заходи, запобіжні проти фальсифікації харчів і напоїв. \emph{Ред.}
}, закона без ніякої сили, бо ж він, розуміється, з
надзвичайною делікатністю ставиться до кожного фритредера,
що має намір через купівлю та продаж фальшованих товарів «to
turn an honest penny\footnote*{
— заробити копійку чесною працею. \emph{Ред.}
}»\footnote{
Сажа є, як відомо, дуже енерґійна форма вуглеця і служить за
добриво, яке капіталістичні коминярі продають англійським фармерам.
1862~\abbr{р.} в одному судовому процесі брітанському «Juryman» (судді) довелося
вирішувати, чи така сажа, що до неї без відома покупця домішано
90\% пилу й піску, є «правдива» сажа в «комерційному» значенні слова,
чи вона є «фальсифікована» сажа в «законному» значенні. «Amis du
commerce»\footnote*{
Друзі торговлі. \emph{Ред.}
} вирішили, що це «правдива» комерційна сажа, і відкинули
скаргу орендаря, який, крім того, мусив ще заплатити судові витрати.
}. Сам комітет сформулював більш-менш
наївно своє переконання, що свобода торговлі означає по суті
торговлю фальшованим, або, як це дотепно кажуть англійці, «софістикованими
продуктами». Дійсно, такого роду «софістика»
вміє краще за Протагора робити з білого чорне і з чорного біле,
і краще за елеатів демонструвати ad oculos\footnote*{
— наочно. \emph{Ред.}
} тільки подобу всього
реального\footnote{
Французький хемік Шевальє у розвідці про «софістикацію» товарів
налічує для багатьох із 600 продуктів, що їх він розглядає, до 10, 20,
30 різних способів фалшування. Він додає, що не знає всіх способів, і
згадує не про всі, які він знає. Для цукру він наводить 6 способів фальсифікації,
для маслинової олії — 9, для масла — 10, для соли — 12, молока
— 19, хліба — 20, для горілки — 23, для борошна — 24, для шоколяди
— 28, для вина — 30, для кави — 32 і~\abbr{т. ін.} Навіть милосердого господа-бога
не минула ця доля. Див. \emph{Ronard de Card}: «De la falsification
des substances sacramentelles», Paris 1856.
}.

В кожному разі комітет звернув увагу громадянства на його
«хліб насущний», а тим самим і на пекарство. Одночасно на публічних
мітинґах і в петиціях, звернених до парляменту, залунав
крик лондонських пекарських підмайстрів про надмірну працю
й~\abbr{т. ін.} Крик зробився такий настирливий, що пана Г.~С.~Тріменгіра,
члена не раз уже згадуваної комісії 1863~\abbr{р.}\footnote{
«Report etc. relating to the Grievances complained of by the
Journeymen Bakers etc.», London 1862 і «Second Report etc.», London
1863.
}, призначено
було на королівського слідчого комісара. Його звіт разом із виказами
свідків зворушив публіку, не серце її, а її шлунок. Правда,
начитаний у біблії англієць знав, що людину, якщо вона не є з
ласки божої ні капіталіст, ні лендлорд і не має синекури, призначено
на те, щоб у поті чола свого їсти хліб свій, та він не знав
того, що він сам мусить день-у-день з’їдати в своєму хлібі
певну кількість людського поту, змішаного з виділенням гнійних
ґуль, павутинням, трупами тарганів, з гнилими німецькими дріжджами,
\index{i}{0192}  %% посилання на сторінку оригінального видання
не кажучи вже про галун, пісок і інші приємні мінеральні
домішки. Тому, не зважаючи на її святість, «вільна торговля»,
«вільне» перед тим пекарство піддано під догляд державних
інспекторів (наприкінці парляментської сесії 1863~\abbr{р.}), і той самий
парляментський акт заборонив пекарським підмайстрам, молодшим
за 18 років, працювати між дев’ятою годиною вечора й п’ятою
годиною ранку. Останній додатковий пункт свідчить красномовніш,
ніж цілі томи, про надмірну працю в цій галузі промисловости,
що від неї віє такою патріярхальністю.

«Праця лондонського пекарського підмайстра починається
звичайно об 11 годині ночі. В цей час він робить тісто, дуже втомний
процес, що триває від \sfrac{1}{2} до \sfrac{3}{4} години залежно від величини
та якости печива. Потім він лягає на місильну дошку, що разом
з тим служить і за покришку діжі, де виробляється тісто, і засинає
на декілька годин, підклавши один лантух з-під борошна під
голову й накрившися другим. Після того починається швидка й
безупинна чотиригодинна праця: викидають із діжі тісто, важать
його, формують, садовлять до печі, виймають із печі й~\abbr{т. ін.} Температура
пекарні сягає 75 і навіть 90 градусів\footnote*{
За Фаренгайтом; за Цельсієм це становить 24--32 градуси, за
Реомюром — 19--26 градусів. \emph{Ред.}
}, а в невеличких пекарнях
вона скорше буває більша, аніж менша. Коли справу печення
хліба, булок тощо закінчено, починається розподілювання хліба;
значна частина робітників, скінчивши тількищо описану важку
нічну працю, протягом дня розносить хліб у кошах або розвозить
його на візках від одного дому до другого, а в переміжках іноді працює
і в пекарні. Залежно від пори року та розміру підприємства
праця кінчається між першою й шостою годиною по півдні, тоді коли
інша частина підмайстрів працює в пекарні до пізнього вечора»\footnote{
Там же. «First Report etc.», р. VI.
}.
«Підчас лондонського сезону підмайстри в пекарів Вестенда, що
продають хліб за «повну» ціну, реґулярно починають працювати
об 11 годині вночі і працюють коло печення хліба з однією або
двома часто дуже короткими перервами до 8 години найближчого
ранку. Потім до 4, 5, 6, а то навіть і до 7 години вони розносять
хліб або печуть бісквіти в пекарні. Після закінчення праці вони
відпочивають, засипаючи на 6 годин, часто лише на 5 або 4 години.
У п’ятницю праця завжди починається раніш, так щось о 10 годині
вечора, і триває безперестанку, чи то при виготовленні чи розношуванні
хліба, до 8 години вечора наступної суботи, але ж здебільшого
до 4 або 5 години в ніч під неділю. І в першорядних
пекарнях, що продають хліб за «повну ціну», неділями знов таки
доводиться працювати протягом 4--5 годин, щоб підготовити
роботу наступного дня\dots{} Ще довший робочий день пекарських
підмайстрів у «underselling masters» (що продають хліб нижче
від повної ціни), а ці останні становлять, як це вже раніш зазначено,
більш ніж \sfrac{3}{4} лондонських пекарів, але праця їхня
майже цілком обмежена пекарнею, бо їхні майстри-хазяї, за
винятком постачання хліба до дрібних крамниць, продають хліб
\parbreak{}  %% абзац продовжується на наступній сторінці
