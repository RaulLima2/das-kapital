провінція Яви, що 1750 р. налічувала понад 80.000 мешканців,
1811 р. мала їх лише 8.000. Оце вам doux commerce!*

Англійська східньоіндійська компанія, як відомо, крім політичної
влади у Східній Індії, здобула собі виключну монополію
на торговлю чаєм, як і взагалі на торговлю з Китаєм і на транспорт
товарів з Европи та до Европи. Але мореплавство по узбережжі
Індії й між островами, як і торговля всередині Індії,
зробилися монополією вищих службовців цієї компанії. Монополії
на сіль, опій, бетель** та інші товари були невичерпними
джерелами багатства. Службовці компанії сами визначали ціни
на товари й обдирали нещасних індусів, як сами хотіли. Генерал-губернатор
брав участь у цій приватній торговлі. Його фаворити
діставали контракти на умовах, що дозволяли їм краще за альхеміків
виробляти золото з нічого. Великі багатства виростали
як гриби після дощу, первісна акумуляція відбувалась повним
ходом без авансування жодного шилінґа. Судовий процес Воррен
Гастінґза аж кишить такими прикладами. Подаємо один із
них. Один контракт на постачання опію передано якомусь Сюлейвенові
в момент його від’їзду — з офіціяльного доручення —
до частини Індії, дуже віддаленої від районів продукції опію.
Сюлейвен продає свій контракт за 40.000 фунтів стерлінґів якомусь
Біннові; того самого дня Бінн перепродує його за 60.000 фунтів
стерлінґів, а останній покупець і виконавець контракту заявляє,
що й він після всього цього здобув величезний бариш.
За одним документом, внесеним до парляменту, компанія та її
службовці примусили індусів за час від 1757 р. до 1766 р. подарувати
їм 6 мільйонів фунтів стерлінґів! В 1769—1770 рр.
англійці створили голод, закупивши ввесь риж і відмовившись
перепродувати його інакше, як за казкові ціни.243

Поводження з тубільцями було, звичайно, найжорстокіше
на плянтаціях, призначених виключно для експортної торговлі,
як от у Західній Індії, а також у багатих і густо залюднених
країнах, що стали жертвою грабіжництва й убивства, як от Мехіко
та Східня Індія. Однак і в колоніях у власному значенні
слова виявився християнський характер первісної акумуляції
капіталу. Пуритани Нової Англії, ці тверезі віртуози протестантизму,
ухвалили в 1703 р. на своїй Assembly*** видавати премію
в 40 фунтів стерлінґів за кожний скальп індійця і за кожного
спійманого червоношкурого, в 1720 р. — премію в 100 фунтів
стерлінґів за кожний скальп, а в 1744 р., після того, як Massachusetts-Bay
проголосив одно плем’я бунтарським, ухвалено пла-

243 1866 р. в самій тільки провінції Оріса померло з голоду понад
мільйон індусів. Проте намагалися збагатити індійську державну касу
за допомогою цін, за якими голодним продавали засоби існування.

* — лагідна торговля. Ред.

** Бетель — рослина, що належить до перцевих; впливає збудно на
нервову систему. В тропічній Азії дуже поширена звичка жувати цю рослину.
Ред.

*** — законодавчі збори. Ред.
