копальнях капітал вважав за остільки згідне із своїм моральним
кодексом, і особливо із своєю головною книгою, примушувати
голих жінок та дівчат, часто разом із чоловіками, працювати по
вугільних та інших копальнях, що лише після тієї заборони він
удався до машин. Янкі винайшли машини розбивати камінь.
Англійці їх не вживають, бо «нещасний» (wretch — технічний
вислів в англійській політичній економії на означення рільничих
робітників), що виконує цю працю, дістає оплату такої
незначної частини своєї праці, що машини удорожчили б цю
продукцію для капіталіста.\footnote{
«Машини... часто не можуть знайти вжитку доти, доки праця (він
має на оці заробітну плату) не піднесеться» («Machinery... can frequently
not be employed until labour rises»). (Ricardo: «Principles of
Political Economy», 3 rd. ed, London 1821, p. 479).
} В Англії замість коней іноді все
ще вживають жінок, щоб тягати барки каналами тощо,\footnote{
Див. «Report of the Social Science Congress at Edinburgh. October
1863».
} бо
праця, потрібна на продукцію коней та машин, є математично
дана кількість, тимчасом як праця, потрібна на утримання жінок
із надмірної людности, є нижча від усякого обрахунку. Тим то
ніде немає безсоромнішого марнотратства людської сили на всякі
дрібниці, як саме в Англії, в цій країні машин.

3. Безпосередні діяння машинового виробництва на робітників

За вихідний пункт великої промисловості є, як це вже
показано, революція в засобі праці, а зреволюціонізований засіб
праці набирає своєї найрозвиненішої форми в розчленованій
системі машин на фабриці. Перше ніж розглядати, як до цього
об’єктивного організму додається людський матеріял, розгляньмо
деякі загальні діяння тієї революції на самого робітника.

а) Присвоювання капіталом додаткових робочих
сил. Жіноча та дитяча праця

Оскільки машина робить мускульну силу зайвою, вона стає
засобом, щоб уживати робітників без мускульної сили або робітників
з недостатнім фізичним розвитком, але з більшою гнучкістю
членів. Тому жіноча й дитяча праця була першим словом капіталістичного
вживання машин! Таким чином цей могутній засіб

ної машини». («Employers of labour would not unnecessarily retain two sets
fo children under thirteen... In fact one class of manufacturers, the spinners
of woollen yarn, now rarely employ children under thirteen years ages, i. e.
half-times. They have introduced improved and new machinery of various
kinds which altogether supersedes the employment of children; f. і: I will
mention one process as an illustration of this diminution in the number
of children, wherein, by thy addition of an apparatus, called a piecingmachine,
to existing machines, the work of six or four half-times, according
to the peculiarity of each machine, can be performed by one young
person... the half-time system» стимулювала «the invention of the
piecing-machine»). (Reports of Insp of Fact, for 31 st October 1858»).