Звідси постає реакція, змальована вже при розгляді почасової
плати, не кажучи вже про те, що здовження робочого дня, навіть
коли відштучна плата лишається стала, само по собі включає
вже зниження ціни праці.

При почасовій заробітній платі, за деякими винятками, панує
рівна заробітна плата за ті самі функції, тимчасом як за відштучної
плати ціну робочого часу вимірюється, щоправда, певною
кількістю продукту, але денна або тижнева плата змінюється
залежно від індивідуальних ріжниць між робітниками, з яких
один дає за даний час мінімум, другий — пересічну кількість,
а третій — більш за пересічну кількість продукту. Отже, щодо
справжнього доходу робітників тут постають дуже великі ріжниці
залежно від різної вправности, сили, енергії, витривалости
й т. ін., індивідуальних робітників.\footnote{
«Там, де в певному підприємстві за працю платять від штуки...
заробітні плати різних робітників щодо суми можуть дуже значно відрізнятися
одна від однієї... Але за поденної плати існує звичайно однакова
норма... визнана так підприємцем, як і робітником за. норму заробітної
плати для середніх робітників даного підприємства». («Where the
work in any trade is paid for by the piece at so much per job... wages may
very materially differ in amount... But in work by the day there is generally
an uniform rate... recognized by both employer and employed as the
standard of wages for the general run of workmen in the trade»). (Dunning:
«Trades-Unions and Strikes», London 1860, p. 17).
} Це, звичайно, нічого не
змінює в загальному відношенні між капіталом та найманою
працею. Поперше, індивідуальні ріжниці для цілої майстерні
вирівнюються, так що майстерня за певний робочий час дає пересічну
кількість продукту, і вся заплачена заробітна плата буде
пересічною заробітною платою даної галузі промисловости. Подруге,
пропорція між заробітною платою та додатковою вартістю
лишається незмінна, бо індивідуальній платі поодинокого робітника
відповідає індивідуально спродукована ним маса додаткової
вартості!. Але, даючи більший простір для індивідуальносте,
відштучна плата прагне, з одного боку, до того, щоб розвивати в
робітників індивідуальність, а тим самим і почуття волі, самостійність
та самоконтроль, з другого боку, — розвивати поміж
ними конкуренцію. Тому вона має тенденцію разом із підвищенням
індивідуальних заробітних плат понад пересічний рівень
знижувати самий цей рівень. Але там, де певна відштучна
плата віддавна вкоренилась як традиція і де, отже, її зниження
являє собою особливі труднощі, — там хазяїни винятково вдавалися
також до насильного перетворення відштучної плата на почасову.
Проти цього вибухнув, наприклад, 1860 р. великий страйк
серед ткачів стьожок у. Ковентрі.\footnote{
«Працю ремісників-підмайстрів регулюється або поденно або
відштучно (à la journée ou à la pièce)... Хазяїни приблизно знають,
скільки роботи можуть виконати за день робітники в кожному реместві,
і тому вони платять їм часто пропорційно до виконаної ними роботи;
таким чином ці підмайстри в своєму власному інтересі працюють стільки,
скільки лише можуть, без усякого догляду». (Cantillon: «Essai sur la
Nature du Commerce en Général», ed. Amsterdam 1756, p. 185 та 202.
} Нарешті, відштучна плата