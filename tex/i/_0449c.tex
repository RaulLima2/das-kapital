\parcont{}  %% абзац починається на попередній сторінці
\index{i}{0449}  %% посилання на сторінку оригінального видання
$\frac{\text{Неоплачена праця}}{\text{оплачена праця}}$ — це є тільки популярний вираз для
$\frac{\text{додаткова праця}}{\text{доконечна праця}}$. Капіталіст оплачує вартість, зглядно ціну
робочої сили, що відхиляється від вартости, та через обмін дістає в своє порядкування саму живу
робочу силу. Використовування ним цієї робочої сили розпадається на два періоди. Протягом одного
періоду робітник продукує тільки вартість,
яка дорівнює вартості його робочої сили, отже, тільки її еквівалент.
За авансовану ним ціну робочої сили капіталіст таким
чином дістає продукт такої самої ціни. Виходить так, наче він
купив був продукт на ринку готовим. Навпаки, в періоді додаткової
праці використовування робочої сили утворює для капіталіста
вартість, за яку він не дав жодного еквіваленту, яка
йому нічого не коштує.\footnote{
Хоч фізіократи й не дійшли таємниці додаткової вартости, все ж
їм було ясно, що додаткова вартість є «незалежне багатство, яким можна
порядкувати і яке він (власник) продає, не купивши його» («une richesse
indépendante et disponible, qu’il n’a point achetée et qu'il vend»). (\emph{Turgot}:
«Réflexions sur la Formation et la Distribution des Richesses», p. 11).
} Це функціонування (Flüssigmachung)
робочої сили дістається йому даром. В цьому розумінні додаткову
працю можна назвати неоплаченою працею.

Отже, капітал — це не тільки командування над працею,
як каже А. Сміс. Він є суттю своєю командування над неоплаченою
працею. Всяка додаткова вартість, хоч у якій особливій
формі — зиску, процента, ренти і т. ін. — вона пізніш кристалізується,
своєю субстанцією є матеріялізація неоплаченого робочого
часу. Таємниця самозростання капіталу роз’яснюється тим,
що капітал порядкує певною кількістю неоплаченої чужої праці.
