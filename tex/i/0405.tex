Хоч і які мізерні в цілому постанови фабричного закону про
виховання, все ж вони оголосили початкове навчання за обов’язкову
умову праці.297 Їхній успіх вперше довів, що можна сполучати
навчання і гімнастику 298 з ручною працею, отже, і ручну
працю з навчанням і гімнастикою. З свідчень учителів фабричні
інспектори незабаром виявили, що фабричні діти, хоч їх навчають
удвоє менше, ніж звичайних школярів, здобувають стільки ж
знання, а часто й більше, ніж ті останні. «Справа проста. Ті, що
проводять у школі тільки половину дня, завжди мають свіжу
голову й майже завжди здатні й охочі вчитися. Система поперемінного
чергування праці й навчання робить одне заняття відпочинком
від другого, отже, вона значно відповідніша для дитини,
ніж безперервність одного з цих двох заняттів. Хлопчина, що від
самого ранку сидить у школі, особливо ж у спеку, ніяк не може
змагатися з якимось іншим, що жвавий і втямливий приходить до
школи від своєї праці».299 Дальші докази можна знайти в промові
Сеніора на соціологічному конґресі в Едінбурзі 1863 р. Сеніор
зазначає тут, між іншим, і те, що однобічний непродуктивний і
довгий шкільний день дітей вищих і середніх кляс без користи
збільшує працю вчителя, «тимчасом як він не тільки даремно,
але й з абсолютною шкодою для дітей забирає їм час, виснажує
їхнє здоров’я й енергію».300 Із фабричної системи, як можна про-

297 За англійським фабричним законом батьки не можуть посилати
дітей молодших від 14 років до «контрольованих» фабрик, не даючи
їм одночасно початкової освіти. Фабрикант відповідає за недодержання
закону. «Навчання при фабриках обов'язкове та є умова праці» («Factory
education is compulsory, and it is a condition of labour»). («Reports of
Insp. of Fact, for 31 st. October 1863», p. 111).

298    Про найкращі наслідки сполучення гімнастики (а для хлопців
і військових вправ) з обов’язковим навчанням дітей по фабриках і школах
для бідних дивись промову Н. В. Сеніора на сьомому, щорічному конґресі
«National Association for the Promotion of Social Science» в «Report
of Procedings etc.», London 1863, p. 63, 64, а також звіти фабричних інспекторів
з 31 жовтня 1865 р., стор. 118, 119, 120, 126 і далі.

299 «Reports of Insp. of Fact, for 31 st October 1865», p. 118. Один
наївний фабрикант шовку заявив слідчому комісарові «Children’s Employment
Commission» ось що: «Я цілком переконаний, що справжній секрет
продукувати вправних робітників знайдено у сполученні праці з навчанням,
починаючи від дитинства. Звичайно, праця не повинна бути ні
надто напружена, ні осоружна, ані нездорова. Я бажав би, щоб мої
власні діти мали працю й забави як відпочинок від школи». («Children’s
Employment Commission. 5 th Report», p. 82, n. 36).

300 Сеніор, там само, стор. 66. Яким чином велика промисловість на
певному ступені розвитку через переворот у способі матеріяльної продукції
і в суспільних продукційних відносинах робить переворот і в головах,
показує яскраво порівняння промови Н. В. Сеніора з 1863 р. з його філіппікою
проти фабричного закону 1833 р., або порівняння поглядів згаданого
конгресу з тим фактом, що в певних сільських частинах Англії бідним
батькам ще й досі заборонено під загрозою голодної смерти навчати
своїх дітей. Так, наприклад, пан Снелл повідомляє як про звичайну практику
в Сомерсетшірі, що коли бідна людина подається до парафії
по допомогу, то її примушують забрати своїх дітей із школи. Так, пан
Воллестоп, піп з Feltham’y, оповідає про випадки, коли деяким родинам
