\parcont{}  %% абзац починається на попередній сторінці
\index{i}{0340}  %% посилання на сторінку оригінального видання
економії їхньої часу. Тимчасом як вони діставали ту саму заробітну
плату та вигравали одну годину вільного часу, капіталіст
одержував ту саму масу продуктів і заощаджував видатки на
вугілля, газ і т. ін. за одну годину. Подібні експерименти проведено
з однаковим успіхам на фабриках панів Геррекса та
Джексона.\footnote{
Там же, стор. 21. Моральний елемент відіграв значну ролю у
згаданих вище експериментах. «Ми, — заявляли робітники фабричному
інспекторові, — ми працюємо з більшою захопленістю, ми постійно маємо
на оці нагороду — змогу раніше піти на ніч; бадьорий та активний
дух проймає всю фабрику від наймолодшого помічника до найстаршого
робітника, і ми можемо більше допомагати один одному» («We work with
more spirit, we have the reward ever before us of getting away sooner at
night, and one active and cheerful spirit pervades the whole mill, irom the
youngest piecer to the oldest hand, and we can greatly help each other»).
(Там же).
}

Скоро тільки скорочення робочого дня, що насамперед утворює
суб’єктивну умову конденсації праці, а саме дає робітникові
спромогу протягом даного часу витрачати більше сили, скоро
тільки це скорочення робочого дня з наказу закону стає примусовим,
— машина в руках капіталу стає об’єктивним і систематично
вживаним засобом на те, щоб видушувати більше праці
протягом того самого часу. Цього досягають двояким способом:
збільшенням швидкости машин та збільшенням кількости машин,
що їх має доглядати той самий робітник, тобто поширенням поля
його праці. Поліпшення конструкції машин почасти є доконечне,
щоб збільшити натиск на робітника, почасти воно само собою
супроводить інтенсифікацію праці, бо обмеження робочого дня
примушує капіталіста до якнайсуворішої економії у витратах
продукції. Поліпшення парової машини збільшує число ударів
її толока на одну хвилину та одночасно дає змогу, через більше
заощадження сили, гнати тим самим мотором більший механізм
при незмінному або навіть зменшеному споживанні вугілля.
Поліпшення передатного механізму зменшує тертя, а також — і це
дуже виразно відрізняє сучасні машини від старих — зводить
діяметр і вагу великих та малих валків до мінімуму, який разу-раз
меншає. Нарешті, поліпшення робочих машин зменшують,
при збільшеній швидкості та поширеній діяльності, їхні розміри,
як от у сучасному паровому ткацькому варстаті, або збільшують
разом із корпусом розмір та число знарядь, що їх вони рухають,
як от у прядільній машині, або за допомогою непомітних дрібних
змін деталів збільшують рухливість цих знарядь, як от у середині
п’ятдесятих років збільшено швидкість веретен у selfacting mule
на одну п’яту.

Скорочення робочого дня до 12 годин датується в Англії від
1833 р. Вже 1836 р. один англійський фабрикант заявив: «Порівняно
з попереднім часом праця, яку доводиться виконувати на
фабриках, дуже зросла в наслідок збільшення уваги та діяльности,
що їх вимагає від робітника значно збільшена швидкість
\parbreak{}  %% абзац продовжується на наступній сторінці
