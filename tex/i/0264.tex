і скінчилася в певний час. Період, що протягом його треба виконати
процес праці, є тут так само приписаний, як от за вловів
оселедців. Поодинока людина може з одного дня викраяти лише
один робочий день, приміром, у 12 годин, але кооперація, приміром,
із 100 людей збільшує дванадцятигодинний день до робочого
дня в 1.200 годин. Короткість строку праці компенсується
величиною маси праці, що її кидається у вирішальний момент
на поле продукції. Вчасний результат залежить тут від одночасного
вживання багатьох комбінованих робочих днів, а обсяг
корисного ефекту — від числа робітників, яке однак завжди
лишається меншим від числа тих робітників, що змогли б протягом
того самого часу виконати ту саму працю, працюючи кожен
окремо.16 Це через брак такої кооперації на заході Сполучених
штатів гине рік-у-рік сила хліба, а в тих частинах Східньої
Індії, де англійське панування знищило давню громаду, — сила
бавовни.17

З одного боку, кооперація дозволяє поширити просторову
сферу праці, а тому для певних процесів праці, як от, приміром,
за осушування ґрунту, будування гребель, іриґації, будування
каналів, шляхів, залізниць тощо, вона потрібна вже в наслідок
просторової зв’язаности предмету праці. З другого боку, кооперація
уможливлює просторово звужувати, порівняно з маштабом
продукції, поле продукції. Це обмеження просторової сфери праці
за одночасного поширення сфери її діяння, через що заощаджується
багато непродуктивних витрат (faux frais), постає із зосередження
робітників, зближення різних процесів праці та концентрації
засобів продукції.18

16 «Виконання її (рільничої праці) у критичний момент має величезну
вагу» («The doing of it at the critical juncture, is of so much the
greater consequence»). («An Inquiry into the Connection between the present
price etc.», p. 7). «У рільництві немає важливішого фактора, ніж час»
(Liebig: «Ueber Theorie und Praxis in der Landwirtschaft», 1856, S. 23).

17 «Дальше лихо, що його ледве чи хто міг сподіватись у країні,
яка вивозить праці більше, ніж усяка інша країна, за винятком хіба
Китаю та Англії, — це неможливість знайти достатню кількість робочих
рук для збирання бавовни. В наслідок цього значна частина врожаю лишається
незібрана, а другу частину його збирають із землі після того, як
бавовна вже висипалась і через це втратила належний колір і почасти
згнила; таким чином через те, що у відповідний час бракує робочих рук,
плянтатор фактично примушений відмовитися від великої частини того
врожаю, що його з такою тривогою сподівається Англія». («The next
evil is one which one would scarcely expect to find in a country which exports
more labour than any other in the world, with the exception perhaps of
China an England — the impossibility of procuring a sufficient number of
hands to clean the cotton. The consequence of this is that large quantities of
the crop are left unpicked, while another portion is gathered from the ground,
when it has fallen, and is of course discoloured and partially rotted, so
that for want of labour at the proper season the cultivator is actually forced
to submit to the loss of a large part of that crop for which England is so
anxiously looking»). (Bengal Hurkaru: «Bi-Monthly Overland Summary
of News. 22 nd July 1861»).

18 «З прогресом рільництва всю ту, а, може, і ще значнішу кількість
капіталу й праці, яку колись уживали для поверхового оброблення 500 ак-
