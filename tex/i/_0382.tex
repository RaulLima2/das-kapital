\parcont{}  %% абзац починається на попередній сторінці
\index{i}{0382}  %% посилання на сторінку оригінального видання
цілих 6 шилінґів 11 пенсів. Тижнева плата ткачів наприкінці
1862 р. починалась з 2 шилінґів 6 пенсів».\footnote{
«Reports of Insp. of Fact, for 31 st October 1863», p. 41, 42.
} Плату за помешкання
часто відраховували від заробітної плати навіть і тоді, коли
руки працювали лише короткий час.\footnote{
Там же, стор. 51.
} Не диво, що в деяких
частинах Ланкашіру почалось щось ніби голодна чума! Але найхарактеристичніше
було те, що революціонізування процесу продукції
відбувалося коштом робітника. Це були справжні expérimenta
in corpora vili,\footnote*{
— експерименти на нічого не вартих тілах. \emph{Ред.}
} як от експерименти анатомів на жабах.
«Хоч я, — каже фабричний інспектор Редґрев, — подав дійсні
доходи робітників по багатьох фабриках, але з цього не можна
зробити такого висновку, ніби вони тиждень-у-тиждень дістають
ту саму суму. Доходи робітників зазнають якнайбільших
коливань через постійне експериментування («experimentalizing»)
з боку фабрикантів... їхні заробітки зростають або падають залежно
від якости бавовняної сумішки; вони то наближаються
до їхніх попередніх заробітків, відхиляючися від них лише на
15\%, то в найближчий або другий тиждень падають на 50 —
60\%».\footnote{
Там же, стор. 50, 51.
} Ці експерименти роблено не тільки коштом засобів
існування робітників. Робітники мусили поплатитись усіма своїми
п’ятьма почуттями. «Ті робітники, що працювали коло відкривання
паків бавовни, оповідали мені, що нестерпний сморід
доводить їх до непритомности. Тим робітникам, що їх уживають
до праці по відділах мішання та чухрання бавовни, порох та бруд,
що вилітають із бавовни, подразнюють дишні шляхи, викликають
кашель та стиснене дихання... Через те, що волокна короткі,
при шліхтуванні додають до пряжі багато всякого матеріялу, а
саме всяких суроґатів замість борошна, що його вживали раніш.
Звідси млості та диспепсія в ткачів. Через порох панує бронхіт,
а так само запал горла, далі шкіряні недуги через подразнення
шкури брудом, що є в сураті». З другого боку, суроґати борошна
були для панів фабрикантів за джерело наживи, бо збільшували
вагу бавовни. Ці суроґати давали те, що «15 фунтів перепряденого
сировинного матеріялу важили 26 фунтів».\footnote{
Там же, стор. 62, 63.
} У звіті фабричних
інспекторів з 30 квітня 1864 р. ми читаємо: «Промисловість
користується тепер цим допоміжним джерелом у справді таки
непристойній мірі. Від дуже авторитетної особи я знаю, що восьмифунтову
тканину виготовляють із 5 1/4 фунтів бавовни та 2 3/4 фунтів
шліхти. В іншій 5 1/4-фунтовій тканині було 2 фунти шліхти.
Це були звичайні шертинґи для вивозу. До інших сортів іноді
додають 50\% шліхти, так що фабриканти можуть вихвалятись
та дійсно вихваляються, що вони багатіють, «продаючи тканини
дешевше, ніж номінально коштує вміщена в них пряжа».\footnote{
«Reports etc. for ЗО th April 1864», p. 27.
}
\parbreak{}  %% абзац продовжується на наступній сторінці
