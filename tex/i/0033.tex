лент поруч з іншими товаровими еквівалентами. Помалу воно
почало у вужчих або ширших сферах функціонувати як загальний
еквівалент. Скоро тільки воно завоювало собі монополію на це
місце у виразі вартостей товарового світу, воно стало грошовим
товаром, і лише від того моменту, коли воно вже стало грошовим
товаром, форма Б відрізняється від форми С, або загальна
форма вартости перетворюється на грошову форму.

Простий відносний вираз вартости якогось товару, приміром,
полотна, в товарі, що вже функціонує як грошовий товар, приміром,
у золоті, є форма ціни. Отже, «форма ціни» полотна така:

20 метрів полотна = 2 унціям золота,

або, коли 2 фунти стерлінґів є монетна назва 2 унцій золота,

20 метрів полотна = 2 фунтам стерлінґів.

Трудність у понятті грошової форми обмежується зрозумінням
загальної еквівалентної форми, отже, загальної форми вартости
взагалі, форми С. Але форма С зводиться ретроспективно
на форму В, на розгорнуту форму вартости, а конститутивним
елементом цієї останньої є форма А: 20 метрів полотна = 1 сурдутові,
або х товару А = у товару В. Тому проста товарова форма
є зародок грошової форми.

4. Фетишистичний характер товару та його таємниця

На перший погляд товар видається цілком зрозумілою, тривіяльною
річчю. Його аналіза виявляє, що це дуже чудернацька
річ, повна метафізичних тонкощів і теологічних примх. Як споживна
вартість він не має в собі нічого загадкового, хоч розглядати
його з того погляду, що своїми властивостями він задовольняє
людські потреби, хоч з того, що самих цих властивостей він
набуває лише як продукт людської праці. Цілком зрозуміло,
що людина своєю діяльністю змінює форми природних матеріялів
так, що робить їх корисними для себе. Наприклад, змінює форму
дерева, виробляючи з нього стіл. А все ж таки стіл лишається
деревом, ординарною, сприйманою почуттям річчю. Але скоро
тільки він виступає як товар, він перетворюється в почуттєвонадпочуттєву
річ. Він не лише стоїть на землі своїми ніжками,
але проти всіх інших товарів він стає на голову, і ця його дерев’яна
голова плодить химери куди дивовижніші, ніж коли б
він почав самохіттю танцювати.25

Отже, містичний характер товару випливає не з його споживної
вартости. Так само мало випливає він із змісту визначень вартости.
Бо, по-перше, хоч які різні можуть бути корисні праці або продуктивні
діяльності, все ж таки — це фізіологічна істина, що
вони є функції людського організму, і що кожна така функція,

25    Пригадаймо собі, що Китай і столи почали танцювати тоді, коли
ввесь інший світ, здавалося, стояв спокійно — pour encourager les autres.*

* — щоб підбадьорити інших. Ред.
