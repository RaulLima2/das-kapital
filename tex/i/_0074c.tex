\parcont{}  %% абзац починається на попередній сторінці
\index{i}{0074}  %% посилання на сторінку оригінального видання
зовсім не потрібно, щоб одночасно підносились або падали ціни
всіх товарів. Досить підвищення цін певного числа головних
товарів в одному випадку або зниження їхніх цін у другому,
щоб підвищити або понизити належну до реалізації суму цін
усіх товарів, що циркулюють, отже, і щоб пустити в циркуляцію
більше або менше грошей. Чи зміна цін товарів одбиває дійсну
зміну вартостей чи просто коливання ринкових цін, вплив на
масу засобів циркуляції лишається той самий.

Припустімо, що дано якесь число продажів, або частинних
метаморфоз, не зв’язаних між собою, що відбуваються одночасно,
отже, просторово одна побіч однієї, приміром, 1 квартера пшениці,
20 метрів полотна, 1 біблії, 4 ґальонів горілки-житнівки. Коли
ціна кожного товару є 2 фунти стерлінґів, отже, належна до реалізації
сума цін є 8 фунтів стерлінґів, то в циркуляцію мусить
увійти маса грошей в 8 фунтів стерлінґів. Навпаки, коли ці самі
товари є члени відомого нам уже ряду метаморфоз: 1 квартер
пшениці — 2 фунти стерлінґів — 20 метрів полотна — 2 фунти
стерлінґів — 1 біблія — 2 фунти стерлінґів — 4 ґальони горілки —
2 фунти стерлінґів, то 2 фунти стерлінґів спричинюють послідовно
циркуляцію різних товарів, реалізуючи послідовно їхні
ціни, отже, і суму цін в 8 фунтів стерлінґів, щоб спочити, кінецькінцем,
у руках гуральника. Вони роблять чотири обіги. Ця
кількаразова зміна місць тієї самої монети репрезентує подвійну
зміну форми товару, його рух через дві протилежні стадії циркуляції
і сплетіння метаморфоз різних товарів.\footnote{
«Лише продукти пускають їх (гроші) в рух і примушують їх
циркулювати\dots{} Швидкість їхнього (грошей) руху заступає їхню кількість.
Коли виникає потреба в них, вони тільки переходять із рук до рук, не
зупиняючись ані на хвилину». («Ce sont les productions qui (l’argent
mettent en mouvement et le font circuler\dots{} La célérité de son mouvement
(sc. de l’argent) supplée à sa quantité. Lorsqu’il en est besoin, il ne fait
que glisser d’une main dans l’autre sans s’arrêter un instant»). (\emph{Le Trosne}:
«De l’Intérêt Social», Physiocrates, éd. Daire. Paris 1846, p. 915 sq.).
} Протилежні й
що одна одну доповнюють фази, які перебігає цей процес, не можуть
відбуватися одна поруч однієї просторово, а наступають
одна по одній лише в часі. Тому переміжки часу становлять міру
тривання цього процесу, або швидкість обігу грошей вимірюється
числом обігів тієї самої монети за даний час. Нехай процес
циркуляції зазначених чотирьох товарів триває, приміром, один
день. Тоді сума цін, що має бути зреалізована, становитиме
8 фунтів стерлінґів, число обігів тієї самої монети за день — 4
і кількість грошей, що циркулюють — 2 фунти стерлінґів, або
для даного переміжну часу процесу циркуляції:
$\frac{\text{Сума цін товарів}}{\text{Число обігів однойменних монет}} =$
масі грошей, що функціонують
як засіб циркуляції. Цей закон має загальне значення.
Процес циркуляції якоїсь країни за якийсь даний відтинок часу
охоплює, правда, з одного боку, багато ізольованих, що відбуваються
одночасно й просторово один поруч одного, продажів
\parbreak{}  %% абзац продовжується на наступній сторінці
