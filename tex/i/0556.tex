сплутують з ним аналогічні, але посутньо відмінні явища передкапіталістичних
способів продукції.

Венеціанський чернець Ортес, один із найбільших письменників-економістів
XVIII віку, розглядає антагонізм капіталістичної
продукції як загальний природний закон суспільного
багатства. «Економічне добро й економічне зло в якійсь нації
завжди зрівноважуються (il bene ed il male economico in una
nazione sempre all’istessa misura), повнява дібр в одних є завжди
недостача дібр в інших (la copia dei beni in alcuni sempre eguale
alla mancanza di esse is altri). Велике багатство небагатьох завжди
супроводиться абсолютним грабуванням доконечного в далеко
більшого числа інших. Багатство якоїсь нації відповідає її людності,
а злидні її відповідають її багатству. Працьовитість одних
вимушує ледарство інших. Бідні й нероби є неминучий продукт
багатих і працьовитих» і т. д.89 Через якихось десять років
після Ортеса англікансько-протестантський піп Тавнсенд цілком
грубим способом розхвалював злидні як доконечну умову
багатства. «Законодатний примус до праці є пов’язаний із чималими
труднощами, насильством і шумом, тимчасом як голод не
тільки є мирний, мовчазний, безупинний натиск, але, являючи
собою якнайприроднішу спонуку до промисловости і праці, викликає
якнайдужче напруження». Отже, все сходить на те, щоб
для робітничої кляси зробити голод перманентним, і про це, за
Тавнсендом, дбає принцип залюднення, який є особливо активний
серед бідних. «Це є, здається, природний закон, що бідні
до певної міри легкодумні (improvident) (а саме так легкодумні,
що приходять на світ без золотої ложки в роті), так що завжди
знаходяться люди (that there always may be some) для виконання
найнижчих, найбрудніших і найпаскудніших функцій у суспільстві.
Запас людського щастя (the fund of human happiness) через
те дуже збільшується, делікатніші люди (the more delicate) увільнені
від мук праці й можуть без перешкод іти за своїм вищим
покликанням і т. д... Закон про бідних має тенденцію зруйнувати
гармонію і красу, симетрію й порядок цієї системи, що її

пролетаріят, що дедалі більше зростає» («De jour en jour il devient donc
plus clair que les rappotrs de production dans lesquels se meut la bourgeoisie
n’ont pas un caractère un, un caractère simple, mais un caractère de duplicité;
que dans les mêmes rapports dans lesquels se produit la richesse, la
misère se produit aussi: que dans les mêmes rapports dans lesquels il y a
développement des forces productives, il y a une force productive de répression;
que ces rapports ne produisent la richesse bourgeoise, c’est à dire
la richesse de la classe bourgeoise, qu’en anéantissant continuellement la
richesse des membres intégrants de cette classe et en produisant un prolétariat
toujours croissant». (K. Marx: «Misère de la Philosophie», p. 116.
— K. Маркс: «Злиденність філософії», Партвидав 1932, стор. 110).

89 G. Ortes: «Delia Economia Nazionale libri sei», 1777, y Custodi.
Parte Moderna, vol. XXI, p. 6, 9, 22, 25 etc. Ортес каже (там же, стор. 32):
«Замість вигадувати нікчемні системи, як зробити народи щасливими,
я хочу обмежитися на розсліді причин їхнього нещастя» («In luoco di
progettar sistemi inutili per la felicità de popoli, mi limiterô a investigare
la ragione delia loro infelicità»).
