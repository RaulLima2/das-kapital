\index{i}{0249}  %% посилання на сторінку оригінального видання
Відділ четвертий

Продукція відносної додаткової вартости

Розділ десятий

Поняття відносної додаткової вартости

Ту частину робочого дня, яка продукує лише еквівалент
оплаченої капіталістом вартости робочої сили, ми досі розглядали
як величину сталу, чим вона в дійсності і є за даних умов продукції,
на даному ступені економічного розвитку суспільства. Понад
цей доконечний для нього робочий час робітник міг працювати
ще 2, 3, 4, 6 і т. д. годин. Від величини цього здовження залежали
норма додаткової вартости й величина робочого дня. Якщо
доконечний робочий час був сталою величиною, то цілий робочий
день, навпаки, був величиною змінною. Припустімо тепер такий
робочий день, що його величина й поділ на доконечну й додаткову
працю є дані. Нехай лінія ас, а—————b——с, репрезентує,
приміром, дванадцятигодинний робочий день, відтинок
аb — 10 годин доконечної праці, відтинок bс — 2 години додаткової
праці. Яким чином можна збільшити продукцію додаткової
вартости, тобто здовжити додаткову працю, без усякого дальшого
здовження ас, або незалежно від усякого дальшого здовження ас?

Не зважаючи на те, що межі робочого дня ас дано, bс, здається,
можна здовжити, якщо не через здовження його поза кінцевий
пункт с, який разом з тим є кінцевий пункт робочого дня ас, то
через переміщення його початкового пункту b у протилежному напрямі,
в бік а. Припустімо, що в лінії а————b' — b—с
відтинок b'b дорівнює половині bс, тобто дорівнює одній робочій
годині. Коли тепер за дванадцятигодинного робочого дня
ас пункт b посунуто до b', то bс поширюється до b'с, додаткова
праця зростає наполовину, з 2 до 3 годин, хоч робочий день,
як і раніш, має лише 12 годин. Але це поширення додаткової
праці bс до b'с, від 2 до 3 годин, очевидно, неможливе без одночасного
скорочення доконечної праці, аb до аb', з 10 до 9 годин.
Здовженню додаткової праці відповідало б скорочення доконечної
праці, або частина того робочого часу, що його досі
робітник фактично зуживав для себе самого, перетворюється на
робочий час для капіталіста. Що змінилося б тут, так це не довжина
робочого дня, а його поділ на доконечну працю й додаткову
працю.
