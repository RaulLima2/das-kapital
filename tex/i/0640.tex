робітники не сміють порозуміватись між собою щодо своїх інтересів,
не сміють спільно діяти, щоб послабити «свою абсолютну,
майже рабську залежність», бо саме цим «вони порушують свободу
своїх сі-devant maîtres,\footnote*{
— колишніх цехових хазяїнів. \emph{Ред.}
} теперішніх підприємців (свободу
тримати робітників у рабстві!), бо об’єднання проти деспотії
колишніх цехових хазяїнів є — відгадайте! — є відновлення цехів,
скасованих французькою конституцією.\footnote{
Bûchez et Roux: «Histoire Parlementaire», t. X, p. 193 —-195.
}

4. Генеза капіталістичних фермерів

Розглянувши процес ґвалтовного створення вільних, як птиці,
пролетарів, криваву дисциплину, що перетворила їх на найманих
робітників, брудні державні заходи, що, разом із збільшенням
ступеня експлуатації праці, поліційними засобами збільшували
акумуляцію капіталу, ми опинилися перед питанням: звідки взялися
первісно капіталісти? Бо експропріяція сільської людности
створює безпосередньо лише великих землевласників. Що ж до
генези фармерів, то ми можемо її, так би мовити, намацати рукою,
бо це повільний процес, що триває цілі століття. Сами кріпаки,
що поряд з ними існували і вільні дрібні землевласники, перебували
в душе різних майнових відносинах, а тому й емансипація
їх відбулася серед дуже різних економічних умов.

В Англії перша форма фармера є bailiff,\footnote*{
— управитель маєтка. \emph{Ред.}
} що сам є кріпак.
Його становище подібне до становища староримського villicus’a,
але з вужчою сферою діяльности. У другій половині XIV століття
bailiff’а замінює фармер, якому лендлорд постачає насіння,
худобу й рільниче знаряддя. Становище його не дуже відрізняється
від становища селянина. Він лише більше експлуатує
найманої праці. Швидко він стає metayer, фармером, що замість
грошової ренти платить землевласникові частиною продукту.
Він постачає одну частину потрібного для рільництва капіталу,
лендлорд — другу. Цілий продукт обидва ділять між собою у
пропорції, визначеній контрактом. В Англії ця форма швидко
зникає, щоб віддати місце фармерові у власному значенні, який
збільшує вартість свого власного капіталу, вживаючи найманих
робітників, і частину додаткового продукту віддає лендлордові
грішми або in natura як земельну ренту.

Поки, протягом XV віку незалежний селянин і рільничий
наймит, що поруч служби з найму ще й самостійно господарював,
збагачуються своєю працею, становище фармера і розміри його
продукції лишаються однаково помірні. Рільнича революція
останньої третини XV віку, що потім тривала майже цілий XVI вік
(за винятком, однак, останніх десятиліть), збагачує фармера з
такою швидкістю, з якою вона збіднює сільську людність.227

227 «Фармери, — каже Гаррісон y своєму «Description of England», —
що їм раніше важко було платити 4 фунти стерлінґів ренти, платять тепер