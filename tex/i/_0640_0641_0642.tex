\parcont{}  %% абзац починається на попередній сторінці
\index{i}{0640}  %% посилання на сторінку оригінального видання
робітники не сміють порозуміватись між собою щодо своїх інтересів,
не сміють спільно діяти, щоб послабити «свою абсолютну,
майже рабську залежність», бо саме цим «вони порушують свободу
своїх сі-devant maîtres\footnote*{
— колишніх цехових хазяїнів. \emph{Ред.}
}, теперішніх підприємців (свободу
тримати робітників у рабстві!), бо об’єднання проти деспотії
колишніх цехових хазяїнів є — відгадайте! — є відновлення цехів,
скасованих французькою конституцією\footnote{
\emph{Bûchez et Roux: } «Histoire Parlementaire», t. X, p. 193—195.
}.
\subsection{Генеза капіталістичних фермерів}
Розглянувши процес ґвалтовного створення вільних, як птиці,
пролетарів, криваву дисциплину, що перетворила їх на найманих
робітників, брудні державні заходи, що, разом із збільшенням
ступеня експлуатації праці, поліційними засобами збільшували
акумуляцію капіталу, ми опинилися перед питанням: звідки взялися
первісно капіталісти? Бо експропріяція сільської людности
створює безпосередньо лише великих землевласників. Що ж до
генези фармерів, то ми можемо її, так би мовити, намацати рукою,
бо це повільний процес, що триває цілі століття. Сами кріпаки,
що поряд з ними існували і вільні дрібні землевласники, перебували
в душе різних майнових відносинах, а тому й емансипація
їх відбулася серед дуже різних економічних умов.

В Англії перша форма фармера є bailiff\footnote*{– управитель маєтка. \emph{Ред.}
}, що сам є кріпак.
Його становище подібне до становища староримського villicus’a,
але з вужчою сферою діяльности. У другій половині XIV століття
bailiff’а замінює фармер, якому лендлорд постачає насіння,
худобу й рільниче знаряддя. Становище його не дуже відрізняється
від становища селянина. Він лише більше експлуатує
найманої праці. Швидко він стає metayer, фармером, що замість
грошової ренти платить землевласникові частиною продукту.
Він постачає одну частину потрібного для рільництва капіталу,
лендлорд — другу. Цілий продукт обидва ділять між собою у
пропорції, визначеній контрактом. В Англії ця форма швидко
зникає, щоб віддати місце фармерові у власному значенні, який
збільшує вартість свого власного капіталу, вживаючи найманих
робітників, і частину додаткового продукту віддає лендлордові
грішми або in natura як земельну ренту.

Поки, протягом XV віку незалежний селянин і рільничий
наймит, що поруч служби з найму ще й самостійно господарював,
збагачуються своєю працею, становище фармера і розміри його
продукції лишаються однаково помірні. Рільнича революція
останньої третини XV віку, що потім тривала майже цілий XVI вік
(за винятком, однак, останніх десятиліть), збагачує фармера з
такою швидкістю, з якою вона збіднює сільську людність\footnote{
«Фармери, — каже Гаррісон y своєму «Description of England», —
що їм раніше важко було платити 4\pound{ фунти стерлінґів} ренти, платять тепер 40, 50, 100\pound{ фунтів стерлінґів} і вважають, що вони зробили невигідну операцію,
коли по закінченні орендного контракту вони не зможуть відкласти
для себе суми, що дорівнювала б ренті за шість-сім років».
}.
\index{i}{0641}  %% посилання на сторінку оригінального видання
Узурпація громадських випасів і~\abbr{т. ін.} дозволяє фармерові значно
збільшити кількість своєї худоби майже без витрат, а худоба дає
йому багато угноєння для землі.

В ХVІ столітті сюди долучається ще один вирішально-важливий
момент. Того часу орендні контракти були довготермінові,
часто на 99 років. Безперервне падіння вартости благородних
металів, а тому й грошей дало фармерам золоті плоди. Передусім
воно, не кажучи вже про всі інші вищерозглянуті обставини,
понизило заробітну плату. Частину цієї заробітної плати долучувано
до фармерського зиску. Невпинний зріст цін на збіжжя,
вовну, м’ясо, одне слово, на всі рільничі продукти, збільшував
грошовий капітал фермера без жодної його участи, тимчасом
як земельна рента, яку він мусив платити, зменшувалась у наслідок
зневартнення грошей підчас тривання контракту\footnote{
Про вплив зневартнення грошей у XVI столітті на різні кляси
суспільства див. «А Compendious or Briefe Examination of Certayne Ordinary
Complaints of Diverse of our Countrymen in these our Days. By
W. S., Gentleman», London 1581. Діялогічпа форма цього твору сприяла
тому, що його довго приписували Шекспірові, і ше року 1751 цей твір
знову видано під його йменням. Автор цього твору є Вільям Стафорд.
В одному місці лицар (knight) резонує так;

Лицар: «Ви, мій сусіде, рільнику, ви, пане крамарю, і ви, добродій
котлярю, так само і всі інші ремісники, ви можете гаразд зарадити
собі. Бо наскільки всі речі стали тепер дорожчі ніж були раніш, настільки
ви підвищуєте ціни своїх товарів і продукти своєї праці, які продаєте.
А ми не маємо нічого, що могли б продавати, ми не маємо нічого, на що
могли б підвищити ціну, щоб відшкодувати себе за ті речі, які ми мусимо
купувати». В іншому місці лицар так запитує доктора: «Скажіть мені,
будь ласка, яких саме людей ви маєте на думці? І насамперед хто, на
ваш погляд, не терпить при цьому жодних втрат?» Доктор: «Я маю на
увазі всіх тих, що живуть із купівлі й продажу; бо коли вони дорого
купують, то потім так само дорого і продають». Лицар: «А хто ті, що,
як ви кажете, виграють на цьому?» Доктор: «Звичайно, всі ті, що господарюють
на маєтках або фармах, платячи стару ренту; бо, платячи за старою
нормою, вони продають за новою, тобто вони платять за свою землю
дуже дешево, а всі продукти, що виростають на ній, продають дорого\dots{}»
Лицар: «Ну, а що ж то за люди ті, що, як ви кажете, втрачають на цьому
більше, ніж ті виграли». Доктор: «Вся шляхта, джентлмени та всі інші,
що живуть з фіксованої ренти або з фіксованого утримання, або сами
не господарюють на своїй землі, або не займаються купівлею і продажем».
(Knight: «You, my neighbour, the husbandman, you Maister Mercer,
and you Goodman Copper, with other artificers, may save yourselves
metely well. For as much as all things are deerer than they were, so much
do you arise in the pryce of your wares and occupations that yee sell agayne.
But we have nothing to sell where by we might advance ye pryce there of,
to countervaile those things that we must buy agayne». — «I pray you, what
be those sorts that ye meane. And, first, of those that yee thinke should
have no losse hereby?» — Doktor: «I meane all these that live by buying
and selling, for, as they buy deare, they sell thereafter». — Knight: «What
is the next sorte that yee say would win by it?» — Doktor: «Marry, all
such as have takings or fearmes in their owne manurance (тобто cultivation)
at the old rent, for where they pay after the olde rate, they sell after the newe — that is, they pays for their lande good cheape, and sell all things
growing thereof deare\dots{}» Knight: «What sorte is that which, ye sayde
should have greater losse hereby, than these men had profit? — Doktor:
«It is all noblemen, gentlemen, and all other that live either by a stinted
rent or stypend, or do not manure (cultivate) the ground, or doe occupy no
buying and selling»).
}. Таким
чином фармер багатів одночасно коштом своїх найманих
\index{i}{0642}  %% посилання на сторінку оригінального видання
робітників і коштом свого лендлорда. Отже, немає нічого дивного
в тому, що наприкінці XVI віку Англія мала клясу багатих
для тодішніх часів «капіталістичних фармерів»\footnote{
У Франції régisseur, на початку середніх віків управитель і збирач
повинностей на користь февдальним панам, швидко стає homme
d’affaires\footnote*{
— ділком. \emph{Ред.}
}, що за допомогою вимагання, шахрайства й~\abbr{т. ін.}, виростає
на капіталіста. Ці régisseurs сами були іноді вельможними панами. Наприклад:
«Оцей рахунок пан Жак де Торес, лицар-кастелян у
Безансоні, подає своєму панові, що веде рахунки в Діжоні для пана
герцога і графа Бургундського, про ренти, які належались від зазначеного
кастелянства від 25 грудня 1359~\abbr{р.} до 28 грудня 1360 р». («C’est
li compte que messire Jacques de Thoraine, chevalier chastelain sor Besançon
rent es seigneur tenant les comptes à Dijon pour monseigneur le
duc et comte de Bourgoigne, des rentes appartenant à la dite chastellenie,
depuis XXVe jour de décembre MCCCLIX jusq’au XXVIIIe jour de décembre
MCCCLX»). (\emph{Alexis Monteil} : «Traité des Matériaux, manuscrits etc.» v. I,
p. 234 і далі). Вже тут видно, як у всіх сферах суспільного життя левина
пайка попадає до рук посередників. Наприклад, в економічній царині фінансисти,
биржовики, купці, дрібні крамарі збирають вершки з усіх справ;
у царині громадського права адвокат скубе супротивників; у політиці
депутат має більше значення, ніж виборець, міністер — більше ніж суверен
; у релігії бога відсувають на задній плян святі «посередники», а цих
останніх знову таки витискують попи, які й собі є неминучі посередники
між добрим пастирем і його вівцями. У Франції, як і в Англії, великі
февдальні території були поділені на безліч дрібних господарств, але
на умовах куди менш сприятливих для сільської людности. В XIV столітті
виникли оренди, фарми або terriers. Число їх невпинно зростало і
значно перевищило \num{100.000}. Вони платили грішми або in natura земельну
ренту, що її розмір коливався від \sfrac{1}{12} до \sfrac{1}{5} продукту. Terriers називалися
денами, підленами і~\abbr{т. ін.} (fiefs, arrière-fiefs), залежно від вартости й
розміру доменів, що з них деякі мали лише декілька арпенів. Всі ці
terriers мали в тій або іншій мірі судову владу над людністю; такої влади
було чотири ступені. Можна зрозуміти, який гніт відчурала сільська
людність під владою усіх цих дрібних тиранів. Монтейль каже, що у Франції
за тих часів було \num{160.000} судів там, де тепер досить \num{4.000} судових установ
(залічуючи сюди й мирових суддів).
}.

\subsection {Зворотний вплив рільничої революції на промисловість.
Утворення внутрішнього ринку для промислового капіталу}

Експропріяція і зганяння сільської людности, що відбувалися
поштовхами й постійно відновлювалися, постачали, як ми бачили,
для міської промисловости щораз нові маси пролетарів, які
стояли цілком поза цеховими відносинами, — мудра обставина,
яка примушує старого А. Андерсона (не треба сплутувати його
з Джемсом Андерсоном) в його історії торговлі увірувати в безпосереднє
втручання провидіння. Ми мусимо ще хвилину спинитися
на цьому елементі первісної акумуляції. Розрідженню незалежної,
самостійно господарюючої сільської людности відпові-

\parbreak{}  %% абзац продовжується на наступній сторінці
