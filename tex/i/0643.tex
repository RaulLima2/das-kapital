дало не лише згущення промислового пролетаріату, як ось Жофруа
Сент-Ілер пояснює згущення світової матерії в одному
місці розрідженням її в іншому.230 Не зважаючи на зменшення
числа обробників землі, вона давала стільки ж продуктів, скільки
й раніш, а то й більше, бо революція у відносинах земельної
власности відбувалася в супроводі поліпшених метод обробітку
землі, поширеної кооперації, концентрації засобів продукції й
т. ін., бо сільських найманих робітників не тільки примушували
інтенсивніше працювати,231 але й поле продукції, на якому
вони працювали сами на себе, дедалі більше скорочувалось.
Отже, разом із звільненням частини сільської людности звільняються
також і її колишні засоби існування. Вони перетворюються
тепер на речовий елемент змінного капіталу. Селянин,
позбавлений власности, мусить купувати вартість цих засобів
існування у свого нового пана, у промислового капіталіста, в
формі заробітної плати. З тубільним сировинним матеріялом,
постачуваним для промисловосте рільництвом, сталося те саме,
що і з засобами існування. Він перетворився на елемент сталого
капіталу.

Припустімо, наприклад, що частину вестфальських селян,
які за часів Фрідріха II всі займалися прядінням, хоч і не шовку,
то льону, силоміць експропрійовано й зігнано з землі, а другу
частину їх, ту, що залишилась, перетворено на наймитів великих
фармерів. Одночасно виростають великі льонопрядні й ткацькі
підприємства, де ці «звільнені» працюють як наймані робітники.
Льон має цілком такий самий вигляд, як і раніш. Жодне волоконце
в ньому не змінилось, але нова соціяльна душа вселилась
у його тіло. Він становить тепер частину сталого капіталу власника
мануфактури. Поділений раніше поміж безлічі дрібних
продуцентів, що сами обробляли його й разом із своїми родинами
випрядали маленькими порціями, він тепер зосереджений у руках
одного капіталіста, що примушує інших прясти і ткати на
нього. Додаткова праця (Extraarbeit), витрачена в прядінні
льону, реалізувалася раніш у додатковому доході (Extraeinkommen)
безлічі селянських родин або також — за часів Фрідріха
II — у податках pour le roi de Prusse.* Тепер вона реалізується
в зиску небагатьох капіталістів. Веретена і ткацькі варстати,
порозділені раніш по селах, зосереджені тепер по небагатьох
великих робітних казармах, так само як робітники і сировинний
матеріял. І веретена, і ткацькі варстати, і сировинний
матеріял перетворюються відтепер із засобів незалежного існування
прядунів і ткачів на засоби панування232 над ними й висисання
з них неоплаченої праці. По великих мануфактурах і по

230    У своїх «Notions de Philosophie Naturelle», Paris 1838.

231    Цей пукт підкреслює сер Джемс Стюарт.

232 «Я дозволю вам, — каже капіталіст, — мати честь служити мені
з умовою, що ви віддасте мені те небагато, що у вас залишилося, за ту
тяжку працю, яку я виконую, командуючи вами» (Je permettrai que

* — для пруського короля. Ред.
