Як дуже спантеличує деяких економістів фетишизм, що пристає
до товарового світу, або предметна зовнішність суспільних
визначень праці, показує, між іншим, нудна й беззмістовна суперечка
про ролю природи в утворенні мінової вартости. А що
мінова вартість є певний суспільний спосіб виражати працю,
зужиту на якусь річ, то й не може вона мати більше природної
речовини, ніж, наприклад, векселевий курс.

Що товарова форма є найзагальніша й найнерозвиненіша
форма буржуазної продукції, — в наслідок чого вона й виникає
рано, хоч і не в такій домінантній, отже, і характеристичній
формі, як нині, — то фетишистичний її характер, здається, можна
побачити ще порівняно легко. У конкретніших формах зникає
навіть ця ілюзія простоти. Звідки постають ілюзії монетарної
системи? З того, що вона вбачала в золоті й сріблі як грошах не
вираз суспільного продукційного відношення, але форму природних
речей з дивними суспільними властивостями. А сучасна політична
економія, що так гордовито глузує з монетарної системи, —
хіба ж її фетишизму не видно як на долоні, скоро вона починає
розглядати капітал? Відколи ж то зникла ілюзія фізіократів,
що земельна рента виростає з землі, а не з суспільства?

Однак, щоб не забігати наперед, тут досить було б іще одного
прикладу щодо самої товарової форми. Коли б товари вміли говорити,
то вони б сказали: наша споживна вартість, можливо,

біжництва, то мусить щось бути для грабіжництва, або об’єкт грабіжництва
мусить постійно репродукуватися. Отже, здається, що і у греків і
римлян був теж якийсь процес продукції, отже, якась економія, яка
становила матеріяльну основу їхнього світу цілком так само, як і буржуазна
економія — основу нинішнього світу. Чи, може, Бастія думає,
що спосіб продукції, який базується на рабській праці, базується на системі
грабіжництва? Якщо так, то він стає на небезпечний ґрунт. Коли
такий велетень думки, як Арістотель, помилявся у своїм оцінуванні рабської
праці, то чому не помилятись у своїм оцінуванні найманої праці
такому карликові-економістові, як Бастіа? — Я використовую цю нагоду,
щоб коротко відповісти на закид, зроблений мені в одній німецькоамериканській
газеті з приводу моєї праці «Zur Kritik der Politischen
Oekonomie», 1859. На думку газети, мій погляд, що певний спосіб продукції
і щораз відповідні йому продукційні відносини, коротко кажучи,
«економічна структура суспільства є реальна база, на якій здіймаються
юридична й політична надбудова і якій відповідають певні форми
суспільної свідомости», що «спосіб продукції матеріяльного життя зумовлює
взагалі соціяльний, політичний і інтелектуальний процес життя» —
все це, мовляв, правильно для нинішнього світу, де панують матеріяльні
інтереси, алеж не для середньовіччя, де панувало католицтво, і не для
Атен і Риму, де панувала політика. Передусім дивно, що хтось може
припустити, що ці загальновідомі фрази про середньовіччя й античний
світ лишилися комубудь невідомі. Адже занадто ясно, що середньовіччя
не могло жити з католицтва, а античний світ з політики. Навпаки, той
спосіб, яким вони здобували засоби життя, пояснює, чому головну ролю
відігравала там політика, тут — католицтво. Зрештою, не треба багато
знайомства, приміром, з історією Римської республіки, щоб знати, що
таємницю цієї історії становить історія земельної власности. З другого
боку, вже Дон-Кіхот відпокутував ту свою помилку, що він уявив собі,
наче мандрівне лицарство однаково можна погодити з усіма економічними
формами суспільства.
