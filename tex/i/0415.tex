інспекторам, і одним махом збільшив таким чином сферу контролю
цих інспекторів більш ніж на 100.000 майстерень, — на 300 самих
тільки цегелень, — персонал інспекторів з винятковою дбайливістю
збільшили всього на вісім помічників, дарма що й перед
тим він був надто малий.\footnote{
Персонал фабричної інспекції складався з 2 інспекторів, 2 помічників
і 41 субінспекторів. Нових 8 субінспекторів призначено 1871 р.
Загальна сума видатків на проведення фабричних законів в Англії,
Шотландії й Ірландії становила 1871—1872 рр. лише 25.347 фунтів стерлінґів,
включаючи й судові витрати на процеси проти порушень закону.
}

Отже, в цьому англійському законодавстві 1867 р. вражає,
з одного боку, накинута парляментові панівних кляс доконечність
у принципі погодитись на такі надзвичайні й широкі заходи
проти ексцесів капіталістичної експлуатації, а з другого боку,
половинчастість, неохота і mala fides,\footnote*{
— нечесність. \emph{Ред.}
} з якими парлямент
потім дійсно проводив у життя ці заходи.

Слідча комісія 1862 р. також запропонувала нову реґляментацію
гірничої промисловости, промисловости, яка від усіх інших
відрізняється тим, що в ній інтереси землевласників і промислових
капіталістів ідуть пліч-о-пліч. Протилежність цих двох
груп інтересів сприяла фабричному законодавству; відсутности
цієї протилежности досить для того, щоб пояснити проволікання
та викрути щодо гірничого законодавства.

Слідча комісія 1840 р. зробила такі жахливі і обурливі викриття
і викликала такий скандал на всю Европу, що парлямент
мусив рятувати своє сумління Mining Act’oм 1842 р., в якому
він обмежився забороною праці під землею для жінок і дітей,
молодших від 10 років.

Потім 1860 р. видано Mines’ Inspection Act,\footnote*{
— закон про інспекцію над копальнями. \emph{Ред.}
} що згідно з ним
спеціяльно призначені державні урядовці мають наглядати за гірничими
підприємствами, і дітей між 10 і 12 роками не можна в
тих підприємствах вживати до праці, якщо вони не мають шкільного
посвідчення або не відвідують школи протягом певного
числа годин. Цей закон лишився цілком мертвою буквою через
на сміх мале число призначених інспекторів, мізерність їхніх
уповноважень та інші причини, які докладніше з’ясується в
дальшому викладі.

Одна з найновіших Синіх Книг про гірничі підприємства —
це «Report from the Select Committee on Mines, together with...
Evidence, 23 July 1866». Це — праця комітету, складеного з
членів нижньої палати й уповноваженого закликати й вислухувати
свідків; товстий том in folio, де сам «Report» має всього
лише п’ять рядків такого змісту: комітет нічого не може сказати,
треба переслухати ще більше свідків!

Спосіб допиту свідків нагадує cross examinations*** по англійських
судах, де адвокат безсоромними і заплутаними перехрес-

* * * — перехресний допит. \emph{Ред.}
