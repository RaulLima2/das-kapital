\parcont{}  %% абзац починається на попередній сторінці
\index{i}{0636}  %% посилання на сторінку оригінального видання
робітником. Правда, позаекономічне, безпосереднє насильство
все ще практикують, алеж тільки винятково. За звичайного
перебігу справ робітника можна віддати на волю «природним
законам продукції», тобто тій його залежності від капіталу,
що випливає з самих умов продукції, ними ґарантується та увічнюється.
Не те бачимо підчас історичної генези капіталістичної
продукції. Новонароджувана буржуазія потребує державної
влади, і вона використовує її, щоб «реґулювати» заробітну плату,
тобто примусово тримати її в межах, сприятливих для добування
додаткової вартости, щоб здовжувати робочий день і самого робітника
тримати на нормальному щаблі залежности. Оце й є
посутній момент так званої первісної акумуляції.

Кляса найманих робітників, що виникла в останній половині
ХІУ століття, становила тоді й наступного століття лише дуже
незначну складову частину народу; у такому своєму стані вона
мала дуже міцну опору в самостійному селянському господарстві
на селі і в цеховій організації по містах. І на селі, і по містах
хазяїни й робітники соціяльно стояли близько одні до одних.
Підпорядковання праці капіталові було лише формальне, тобто
самий спосіб продукції не мав ще специфічно-капіталістичного
характеру. Змінний елемент капіталу дуже переважав над сталим
елементом. Тому попит на найману працю швидко зростав
з кожною акумуляцією капіталу, тимчасом як подання найманої
праці лише поволі йшло за попитом. Значна частина національного
продукту, перетвореного пізніше на фонд акумуляції капіталу,
тоді ще увіходила в споживний фонд робітника.

Законодавство про найману працю, що від самого початку
мало на меті експлуатацію робітника і в своєму дальшому розвитку
завжди незмінно вороже до нього,\footnote{
«Завжди, коли законодавство намагається реґулювати суперечки
між хазяїнами та їхніми робітниками, його дорадниками є хазяїни», каже
А. Сміс («Whenever the legislature attempts to regulate the differences
between masters and their workmen, its counsellors are always the masters).
«Власність — ось дух законодавства», каже Ленґе («L’esprit des
lois, c’est la propriété»).
} починається в Англії
із Statute of Labourers\footnote*{
— статуту про робітників. \emph{Ред.}
} Едварда III, виданого 1349 р. У Франції
йому відповідає ордонанс 1350 р, виданий від імени короля
Жана. Англійське і французьке законодавство йдуть паралельно
і тотожні щодо змісту. Оскільки робітничі статути намагаються
примусити до подовження робочого дня, я не повертаюся до них,
бо цей пункт розглянуто вже вище (8 розділ, 5).

Statute of Labourers видано в наслідок настирливих скарг
палати громад [тобто покупців праці].\footnote*{
Заведене у прямі дужки беремо з французького видання. \emph{Ред.}
} «Раніш, — наївно каже
один торі, — бідні вимагали такої високої заробітної плати, що
вони тим загрожували промисловості й багатству. Тепер їхня
заробітна плата така низька, що знову таки загрожує промисловості
й багатству, але інакше і, може бути, ще небезпечніш,
\parbreak{}  %% абзац продовжується на наступній сторінці
