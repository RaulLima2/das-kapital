Що продукція та циркуляція товарів є загальна передумова
капіталістичного способу продукції, то мануфактурний поділ
праці вимагає достиглого вже до певного ступеня розвитку поділу
праці всередині суспільства. Навпаки, мануфактурний поділ
праці, з свого боку, впливає на суспільний поділ праці, розвиваючи
та розгалужуючи його. З диференціяцією інструментів
праці чимраз більше диференціюються й ті галузі промисловости,
які ці інструменти продукують.54 Якщо мануфактурне виробництво
захопить якийсь промисел, що досі був зв’язаний з іншими
як головний або підсобний промисел і виконувався тим самим
продуцентом, то відразу постає відокремлення та взаємне усамостійнення
цих реместв. Якщо воно захопить якийсь окремий
щабель продукції якогось товару, то різні щаблі його продукції
перетворюються на різні незалежні промисли. Ми вже зазначали,
що там, де продукт є лише механічно сполучена цілість частинних
продуктів, частинні праці сами можуть знову таки усамостійнитися
в окремі ремества. Для того, щоб досконаліше здійснити
поділ праці всередині якоїсь мануфактури, та сама галузь
продукції поділяється на різні, почасти цілком нові мануфактури,
залежно від ріжниці її сировинних матеріялів або тих різних
форм, яких може набирати той самий сировинний матеріял.
Так, у самій лише Франції вже в першій половині XVIII віку
ткали понад 100 різнорідних шовкових матерій, а в Авіньйоні,
приміром, було законом, що «кожний учень повинен посвячувати
себе цілком лише одному родові фабрикації й не смів вчитися
виготовлювати одночасно декілька родів продуктів». Територіальний
поділ праці, який окремі галузі продукції прив’язує
до окремих округ країни, набуває нового імпульсу від мануфактурного
виробництва, що експлуатує всякі особливості.55 Для мануфактурного
періоду багатий матеріял для поділу праці всередині
суспільства дають поширення світового ринку й колоніальна
система, що належать до загальних умов існування мануфактурного
періоду. Тут не місце далі доводити, як цей поділ праці

бавовни коштом продукції рижу. В наслідок цього у більшій частині
країни настав голод, бо за браком засобів комунікації, тобто за браком
фізичного зв'язку, недостачу рижу в одній окрузі не можна було поповнити
довозом з інших округ.

54 Так, у Голландії фабрикація ткацьких човників уже в XVII ст.
становила осібну галузь промисловости.

55 «Хіба англійську вовняну мануфактуру не поділено на різні
частини або галузі, прив’язані до окремих місць, де продукується виключно
або переважно такі продукти: тонкі сукна у Сомерсетшірі, грубі —
в Йоркшірі, подвійної ширини — в Ексетері, шовк — у Сандбері, креп —
у Норвічі, піввовняні полотна — у Кендалі, покривала — в Уітнеї і
т. ін.?» («Whether the Woollen Manufacture of England is not divided
into several parts or branches appropriated to particular places, where
they are only or principally manufactured: fine cloths in Somersetshire,
coarse in Yorkshire, long ells at Exeter, saies and Sandbury, crapes at
Norwich, linseys at Kendal, blankets at Whitney, and so forth?»).
(Berkсley: «The Querist», 1750, § 520).
