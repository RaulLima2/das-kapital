\index{i}{0455}  %% посилання на сторінку оригінального видання
Ми бачимо далі: вартість у 3 шилінґи, в якій виражається
оплачена частина робочого дня, тобто шестигодинна праця,
з’являється як вартість або ціна цілого дванадцятигодинного
робочого дня, що має в собі 6 неоплачених годин. Отже, форма
заробітної плати затирає всякий слід поділу робочого дня на
доконечну й додаткову працю, на оплачену й неоплачену працю.
Уся праця здається оплаченою працею. При панщанній праці
праця кріпака на себе самого та його примусова праця на пана
відрізняються цілком виразно, відрізняються щодо місця й часу.
При рабській праці навіть та частина робочого дня, що протягом
неї раб покриває тільки вартість своїх власних засобів існування,
отже, коли він фактично працює на себе самого, здається працею
на його пана. Вся його праця здається неоплаченою працею.\footnote{
«Morning Star», до безглуздя наївний орган лондонських фритредерів,
за часів американської громадянської війни не переставав запевняти
з моральним обуренням, на яке тільки спроможна людина, що
негри в «Confederate States»\footnote*{
— південних штатах. \emph{Ред.}
} працюють цілком задурно. Йому треба
було б потрудитись порівняти денні витрати такого негра з денними витратами
вільного робітника, наприклад, у лондонському East End'i.
}

Навпаки, при найманій праці навіть додаткова праця або
неоплачена праця здається оплаченою. Там відношення власности
приховує працю раба на себе самого, тут грошове відношення
приховує дармову працю найманого робітника.

Тому зрозуміле вирішальне значення перетворення вартости
й ціни робочої сили на форму заробітної плати або на вартість
і ціну самої праці. На цій формі виявлення, що робить невидимим
дійсне відношення та показує саме його протилежність,
ґрунтуються всі правні уявлення так робітника, як і капіталіста,
всі містифікації капіталістичного способу продукції, всі
його ілюзії про свободу, всі апологетичні викрути вульґарної
економії.

Якщо всесвітній історії треба було багато часу на те, щоб
викрити таємницю заробітної плати, то, навпаки, нема нічого
легшого, як зрозуміти доконечність, raisons d’être\footnote*{
— причини існування. \emph{Ред.}
} цієї форми виявлення.

Обмін між капіталом та працею при спостереженні насамперед
виявляється у цілком такому самому вигляді, як купівля
і продаж усіх інших товарів. Покупець дає певну суму грошей,
продавець — якусь річ, відмінну від грошей. Правна свідомість
убачає тут щонайбільше речову ріжницю, яка виражається
в правно-еквівалентних формулах: «Do ut des», «do ut facias»,
«facio ut des» і «facio ut facias».\footnote*{
— «Даю, щоб ти дав», «даю, щоб ти зробив», «роблю, щоб ти дав»,
«роблю, щоб ти зробив». \emph{Ред.}
}

[Далі, кожну купівлю і продаж супроводить ілюзія, ніби
те, що оплачують, це — споживна вартість товару, хоч цю ілюзію
збиває вже той простий факт, що найрізноманітніші речі
\parbreak{}  %% абзац продовжується на наступній сторінці
