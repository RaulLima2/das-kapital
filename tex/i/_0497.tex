\parcont{}  %% абзац починається на попередній сторінці
\index{i}{0497}  %% посилання на сторінку оригінального видання
передумову ріжницю їхніх споживних вартостей і не має абсолютно
ніякого чинення до споживання їх, яке починається лише
після того, як торг уже закінчено.

Отже, первісне перетворення грошей на капітал відбувається
в якнайточнішій згоді з економічними законами товарової продукції
і з правом власности, що з них випливає. А проте воно
приводить до таких наслідків:

1) продукт належить капіталістові, а не робітникові;

2) вартість цього продукту, окрім вартости авансованого капіталу,
має в собі додаткову вартість, що коштувала робітникові
праці, а капіталістові нічого не коштувала, і все ж таки стає
правною власністю капіталіста;

3) робітник і надалі зберіг свою робочу силу й може знову її
продати, якщо знайде покупця.

Проста репродукція — це лише періодичне повторювання
цієї першої операції; щоразу гроші знову й знову перетворюються
на капітал. Отже, закон не ламається, навпаки, він
має лише нагоду постійно виявлятися. «Декілька послідовних
актів обміну лише зробили з останнього представника першого».\footnote*{
«Plusieurs échanges successifs n’ont fait du dernier que le représentant
du premier». (\emph{Sismondi}: «Nouveaux Principes d’Economie Politique»,
vol. I, p. 70).
}

А, проте, ми бачили, що досить простої репродукції, щоб цій
першій операції, оскільки ми розглядали її як ізольований акт,
надати цілком зміненого характеру. «Поміж тими, що розподіляють
між себе національний дохід, одні [робітники], щороку
набувають новою працею нового права на нього, другі [капіталісти]
вже раніш набули постійного права на нього за допомогою
первісної праці».\footnote*{
«Parmi ceux qui partagent le revenue national, les uns y acquierènt
chaque année un nouveau droit par un nouveau travail, les autres y ont acquis
antérieurement un droit permanent par un travail primitif». (\emph{Sismondi},
i. c., p. 110).
} Як відомо, царина праці — це не одним-одна
царина, де перворідність творить чудеса.

Справа ані скільки не зміниться й тоді, коли просту репродукцію
замінюється репродукцією в поширеному маштабі, акумуляцією.
За першої капіталіст прогулює цілу додаткову вартість,
за другої він виявляє чесноту громадянина, споживаючи
лише якусь частину додаткової вартости й перетворюючи решту
на гроші.

Додаткова вартість є його власність, вона ніколи не належала
комусь іншому. Якщо він авансує її на продукцію, то це авансування
він робить із свого власного фонду так само, як того
дня, коли він уперше з’явився на ринку. Що цей фонд цим разом
походить із неоплаченої праці його робітників, це абсолютно не
має значення для справи. Коли робітника В вживають до праці,
оплачуючи його додатковою вартістю, випродукованою робітником
А, то, по-перше, А постачив цю додаткову вартість, діставши
\parbreak{}  %% абзац продовжується на наступній сторінці
