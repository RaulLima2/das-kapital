Цей результат стає неминучим, скоро тільки робітник сам
вільно продає свою робочу силу як товар. Але тільки відтепер товарова
продукція стає загальною й робиться типовою формою продукції;
тільки відтепер кожний продукт продукується з самого
початку на продаж і всяке продуковане багатство проходить через
циркуляцію. Тільки тоді, коли наймана праця стає базою товарової
продукції, ця остання накидає себе цілому суспільству;
але тільки тоді розгортає вона також всі заховані в ній потенції.
Сказати, що втручання найманої праці перекручує товарову
продукцію, значить сказати, що товарова продукція, щоб лишитися
неперекрученою, мусить не розвиватися. В тій самій мірі,
в якій товарова продукція розвивається за своїми власними
іманентними законами на капіталістичну продукцію, в тій самій
мірі закони власности товарової продукції перетворюються на
закони капіталістичного присвоювання.24

Ми бачили, що навіть при простій репродукції кожен авансований
капітал, хоч яким чином його первісно придбано, перетворюється
на акумульований капітал, або на капіталізовану
додаткову вартість. Але в потоці продукції ввесь первісно авансований
капітал взагалі стає безконечно малою величиною (magnitudo
evanescens у математичному розумінні) порівняно з безпосередньо
акумульованим капіталом, тобто із зворотно перетвореною
на капітал додатковою вартістю або додатковим продуктом,
однаково, чи функціонує він у руках того, хто його нагромадив,
чи в чужих руках. Тому політична економія визначає капітал
взагалі як «акумульоване багатство» (перетворену додаткову
вартість або дохід), «що його знову вживають до продукції додаткової
вартости»,25 а капіталіста як «власника додаткового
продукту».26 Той самий погляд, хоч і в іншій формі, маємо і
в тому вислові, що ввесь наявний капітал є акумульований або
капіталізований процент, бо процент є лише частина додаткової
варгосіи 27

24    Тим-то можна дивуватися з хитромудрости Прудона, що хоче знищити
капіталістичну власність, протиставлячи їй вічні закони власности
товарової продукціїї

25 «Капітал, тобто акумульоване багатство, що його вживають, щоб
одержати зиск» («Capital, viz: accumulated wealth employed with a
view to profit»). (Malthus: «Principles of Political Economy», p. 262).
«Капітал... складається з багатства, що заощаджується з доходу, і вживається,
щоб одержати зиск» («Capital... consists of wealth saved from
revenue, and used with a view to profit»). (R. Jones: «An Introductory
Lecture on Political Economy», London 1833, p. 262).

26 «Посідачі додаткового продукту або капіталу» («The possessors
of surplusproduce or capital»). («The Source and Remedy of the National
Difficulties. A Letter to Lord John Russel», London 1821).

27 «Капітал із складними процентами на кожну частину заощадженого
капіталу до такої міри виріс, що все багатство світу, яке дає дохід,
віддавна є вже процент від капіталу» («Capital, with compound interest
on every portion of capital saved, is so all engrossing, that all the wealth
in the world from which income is derived, has long ago become the
interest on capital»). («London Economist», 19 July 1859).
