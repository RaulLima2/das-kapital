\parcont{}  %% абзац починається на попередній сторінці
\index{i}{0507}  %% посилання на сторінку оригінального видання
доконечність? Коли для клясичної політичної економії пролетар
є лише машина продукувати додаткову вартість, то
й капіталіста вона розглядала лише як машину перетворювати
цю додаткову вартість на додатковий капітал. Вона трактує
його історичну функцію з надзвичайною серйозністю. Щоб ґарантувати
серце капіталіста від лихого конфлікту між жагою
насолод і жадобою до збагачення, Малтуз на початку двадцятих
років цього століття обстоював такий поділ праці, що призначав
справу акумуляції капіталістові, дійсно занятому в продукції,
а справу марнотратства — іншим учасникам додаткової вартости,
земельній аристократії, людям, що дістають утримання від держави,
церкви й~\abbr{т. ін.} Надзвичайно важливо, каже він, «відокремити
пристрасть до видатків від пристрасти до акумуляці і» («the
passion for expenditure and the passion for accumulation»).\footnote{
\emph{Malthus}: «Principles of Political Economy», p. 319, 320.
}
Пани капіталісти, що давно вже поперетворювалися на розкішників
і світських людей, зчинили галас. Як, — вигукнув один
їхній проводир, рікардіянець, — пан Малтуз проповідує високі
земельні ренти, високі податки й~\abbr{т. ін.} для того, щоб через непродуктивних
споживачів постійно підганяти промисловців! Щоправда,
продукція, продукція в щораз ширшому маштабі, таке
наше гасло. Але «через такий процес продукція куди більше
гальмується, ніж розвивається. Крім того, не зовсім справедливо
(nor is it quite fair) підтримувати таким чином у ледарстві певне
число осіб для того лише, щоб підганяти інших, при чому з характеру
цих осіб можна бачити («who are likely, from their characters»),
що вони успішно функціонуватимуть, коли їх примусити
функціонувати».\footnote{
«An Inquiry into those principles respecting the Nature of Demand
etc., p. 67.
} Але, якщо цей рікардіянець вважає за несправедливе
підганяти промислового капіталіста до акумуляції,
збираючи жир з його юшки, то він, навпаки, вважає за доконечне
звести заробітну плату робітника по змозі на мінімум,
«щоб підтримати його працьовитість». Він ні на хвилину не
затаює й того, що таємниця добування додаткової вартости
(Plusmacherei) — це присвоювання неоплаченої праці. «Збільшений
попит на роботу з боку робітників — це значить не що
інше, як їхній нахил брати менше з свого власного продукту для
самих себе, а більшу частину з нього лишати для своїх підприємців;
і коли кажуть, що це в наслідок зменшення споживання
(з боку робітників) викликає glut (переповнення ринку, перепродукцію),
то я на це можу відповісти лише, що glut — це синонім
високого зиску».\footnote{
Там же, стор. 50.
}

Вчені суперечки про те, як найкорисніше для акумуляції
поділити витягнену з робітників здобич між промисловим капіталістом
і неробою-землевласником тощо, припинилися перед
липневою революцією. Незабаром після того міський пролетаріят
ударив на сполох у Ліоні, а сільський пролетаріят в Англії
\parbreak{}  %% абзац продовжується на наступній сторінці
