Якщо додамо всіх тих, що працюють по текстильних фабриках,
до персоналу копалень та рудень, то матимемо 1.208.442;
якщо ж число перших додамо до персоналу всіх металюрґійних
заводів і мануфактур, то матимемо загальне число 1.039.605 —
в обох випадках менше, ніж число сучасних хатніх рабів. От
який величний результат капіталістичної експлуатації машин!

7. Відштовхування і притягування робітників із розвитком
машинового виробництва. Кризи в бавовняній промисловості

Всі серйозні представники політичної економії визнають,
що заведення в життя машин впливає неначе чума на робітників
у тих традиційних ремествах і мануфактурах, з якими машина
насамперед починає конкурувати. Майже всі вони бідкаються
над рабством фабричного робітника. Але який той великий козир,
що ним усі вони козиряють? Це те, що машини після всіх страхіть
періоду заведення їх у життя та розвитку їх, кінець-кінцем,
не зменшують, а збільшують число рабів праці! Так, політична
економія захоплюється огидною теоремою — огидною для всякого
«філантропа», що вірує у вічну природну доконечність капіталістичного
способу продукції, — теоремою, що навіть фабрика,
яка вже основана на машиновому виробництві, після певного
періоду зросту, після коротшого або довшого «переходового
часу» починає мучити більше число робітників, ніж те, яке вона
первісно викинула на брук! 226

Правда, з деяких прикладів, як от на англійських фабриках

мірною працею молодих служниць, це — ганьба для їхніх господинь.
Випадково я знайомий з багатьма з цих «рабинь», як їх дехто називає,
і співчуваю їм від усього серця. Вони мусять рано вставати та працювати
до самісінької ночі. Вони сплять у підвальних комірках із нечистю або
по горищах із пацюками. Вони харчуються покидьками. Їх лають і шельмують,
їх переслідують брутальні хазяйські сини, їх мучать 4 або 5 дітей;
під дощ їх ганяють по пиво, інколи їх б’ють розгнівані господині. Тижнями
їм не дозволяють піти до церкви. їм платять дуже мало; якщо вони
захоріють, їх відсилають до їхніх родичів, коли в них є родичі, або ж до
шпиталю, або до притулку для бідних. Не диво, що вони мають острах
і огиду до пристойної праці і готові «піти світ за очі, хоч к чорту», і це
вони, ці бідолашні маленькі рабині, залюбки й роблять. Я бачив, як вони
плакали, оповідаючи про свої страждання, побої, голод і холод, про те,
як їх прогнали з їхнього «місця», коли вони захоріли, як жили вони
тоді з продажу свого одягу, і як, нарешті, коли все було продано, вони
утопли в мерзоті, дедалі більше занепадаючи. На жаль, лише дехто їм
співчуває». Ред.

226 Ґаніль, навпаки, вважає за остаточний результат машинового виробництва
абсолютне зменшення числа рабів праці, що їхнім коштом годується
потім збільшене число «gens honnêtes»,* що розвивають свою відому
«perfectibilité perfectible»,** [що її так надхненно висміяв Фур’є].*** Хоч

* — порядних людей. Ред.

** — здібну вдосконалюватися здібність до вдосконалення. Ред.

*** Заведений у прямі дужки кінець речення беремо з французького
видання. Ред.
