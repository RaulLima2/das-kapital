Їх урівняно в усьому з підлітками, їхній робочий час обмежено
12 годинами, нічну працю їм заборонено й т. ін. Отже, законодавство
вперше побачило себе примушеним безпосередньо й офіціяльно
контролювати й працю повнолітніх. У фабричному звіті 1844 —
1845 рр. говориться з іронією: «До нашого відома не дійшов
жоден випадок, щоб дорослі жінки нарікали на це встрявання
в їхні права».\footnote{
«Reports etc. for 30 th Sept. 1844», p. 15.
} Робочий день дітей, молодших від 13 років,
скорочено до 6\sfrac{1}{2} і, серед певних умов, до 7 годин на день.\footnote{
Закон дозволяє вживати праці дітей протягом десятьох годин,
коли вони працюють не щоденно, а через день. Взагалі ж це застереження
лишилося без наслідків.
}

Щоб усунути зловживання фалшивої «Relaissystem», закон
ухвалив, між іншим, такі важливі детальні постанови: «Робочий
день для дітей і підлітків треба рахувати з того моменту, що з
нього якась дитина або підліток починає уранці на фабриці
працювати». Так що, коли, приміром, А починає працю о 8 годині
ранку, а В о 10 годині, то робочий день для В мусить усе-таки кінчитися тією самою годиною, що й для
А. Початок робочого
дня повинен показувати якийсь громадський годинник, наприклад,
найближчий залізничний годинник, і з ним треба погоджувати
фабричний дзвін. Фабрикант має вивісити у фабриці надруковану
великими літерами об’яву з означенням годин початку,
кінця й перерв робочого дня. Дітей, що починають свою працю
перед 12 годиною ранку, забороняється вживати знов до праці
після 1 години по півдні. Отже, післяобідня зміна мусить складатися
з інших дітей, ніж передобідня. 1\sfrac{1}{2} години на їжу мусять
бути призначені для всіх робітників, що стоять під охороною закону,
в одну й ту саму пору дня, при чому одна година, принаймні,
перед третьою годиною по півдні. Дітей або підлітків забороняється
вживати до роботи більш як 5 годин перед першою годиною
по півдні, якщо їм не дається принаймні півгодинної перерви на
їжу. Дітям, підліткам і жінкам не можна в час, призначений на
їжу, лишатися у фабричному приміщенні, де відбувається якийбудь
процес праці, і т. ін.

Ми бачили, що ці дрібничкові постанови, які з такою військовою
одноманітністю за ударом дзвону реґулюють час, межі й
перерви праці, зовсім не були продуктом парляментських вигадок.
Вони поволі розвивалися з обставин, як природні закони
сучасного способу продукції. Формулювання їх, офіціяльне
визнання й проголошення державою були результатом довгочасної
клясової боротьби. Одним із найближчих наслідків їх
було те, що у практиці й робочий день дорослих фабричних робітників-чоловіків
підведено під ті самі обмеження, бо в більшості
процесів продукції не можна обійтися без співробітництва дітей,
підлітків і жінок. Тим-то взагалі і в цілому, протягом періоду
від 1844 до 1847 рр. дванадцятигодинний робочий день набув загальної
й однакової сили по всіх галузях промисловости, що
підлягали фабричному законодавству.