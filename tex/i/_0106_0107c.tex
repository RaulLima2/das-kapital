\parcont{}  %% абзац починається на попередній сторінці
\index{i}{0106}  %% посилання на сторінку оригінального видання
виконувати свою ролю. Вони перестали б бути капіталом. Вилучені
з циркуляції, вони застигають у скарб, і жодний шаг не наростає
на них, хоч би вони й лежали аж до судного дня. Отже,
скоро йдеться про зростання вартости, то для 110\pound{ фунтів стерлінґів}
є така сама потреба в зростанні вартости, як і для 100\pound{ фунтів
стерлінґів}, бож обоє вони є обмежені вирази мінової вартости,
отже, обоє мають те саме призначення наближатися до багатства
взагалі через збільшення своєї величини. Правда, первісно авансована
вартість у 100\pound{ фунтів стерлінґів} відрізняється на одну хвилину
від додаткової вартости в 10\pound{ фунтів стерлінґів}, яка наростає
на ній у циркуляції, але ця ріжниця зразу ж геть розпливається
знову. Наприкінці процесу з’являється не первісна вартість
у 100\pound{ фунтів стерлінґів} на одному боці і додаткова вартість у
10\pound{ фунтів стерлінґів} на другому, а з’являється єдина вартість у
110\pound{ фунтів стерлінґів}, яка має форму, так само придатну для того,
щоб почати процес зростання, як і первісні 100\pound{ фунтів стерлінґів}.
Гроші з’являються наприкінці руху, щоб тільки знову почати
його\footnote{
«Капітал поділяється\dots{} на первісний капітал і бариш, приріст
капіталу\dots{} хоч у практиці цей самий бариш зразу ж додається знов до
капіталу й разом із ним пускається в рух». (\emph{F.~Engels}: «Umrisse zu
einer Kritik der Nationalökonomie» in «Deutsch-Französische Jahrbücher,
herausgegeben von Arnold Rüge und K. Marx». Paris 1884, S. 99).
}. Тому кінець кожного окремого кругобігу, що в ньому
купівля відбувається задля продажу, вже сам собою становить
початок нового кругобігу. Проста товарова циркуляція — продаж
задля купівлі — служить за засіб досягти кінцевої мети,
що лежить поза межами циркуляції, а саме, присвоїти споживні
вартості, задовольнити потреби. Навпаки, циркуляція грошей
як капіталу є самоціль, бо зростання вартости існує лише в межах
цього раз-у-раз поновлюваного руху. Тому рух капіталу є
безмірний\footnote{
Арістотель протиставить економіку хрематистиці. Він виходить
з економіки. Оскільки вона є вмілість надбання, вона обмежується на здобуванні
благ, доконечних для життя і потрібних для дому або для держави.
«Справжнє багатство (\textgreek{δ αληζινος πλουιος}) складається з таких споживних
вартостей, бо кількість власности цього роду, достатньої для
доброго життя, не є безмежна. Але існує вмілість надбання іншого роду,
яка переважно і з повним правом називається хрематистикою, вмілість,
у наслідок якої, здається, не існує жодних меж багатства і власности.
Товарова торговля (\textgreek{η χαπηλιχη} дослівно значить торговля на роздріб,
і Арістотель бере цю форму, бо в ній переважає споживна вартість) з природи
не належить до хрематистики, бо тут обмін стосується лише до предметів,
що їм самим (покупцям і продавцям) потрібні». Тому, висновує
він далі, первісною формою товарової торговлі була мінова торговля,
але з її поширенням неминуче постали гроші. З винаходом грошей мінова
торговля неминуче мусила розвинутися в \textgreek{χαπηλιχη}, в товарову торговлю,
а ця, всупереч до її первісної тенденції, перетворилась на хрематистику,
на вмілість робити гроші. Хрематистика ж відрізняється від економіки
тим, що «для неї циркуляція є джерело багатства (\textgreek{ποιητιχη χεηματωυ\dots{} δια χρηηατωυ σιαβολης}). І
вона, здається, ґрунтується на грошах, бо гроші є початок і кінець цього роду обміну (\textgreek{το γχρ νομισμα
στοιχειον χαι περας ιης αλλαγης εστιν}). Тому те багатство, до якого прагне хрематистика, є
безмежне. Як кожна вмілість, що її мета має для неї значення не засобу,
а останньої кінцевої мети, є безмежна у своєму прагненні, бо старається
дедалі більше наблизитись до неї, тимчасом як умілості, що мають на оці
лише засоби для здійснення мети, не є безмежні, бо сама мета ставить їм
межі, так само для цієї хрематистики немає жодних меж у її меті, бо її
мета — це абсолютне збагачення. Економіка, а не хрематистика має
межу\dots{} перша має на меті щось відмінне від самих грошей, друга — лише
збільшення їх\dots{} Сплутування обох форм, що переходять одна в одну, дає
привід декому дивитись на збереження й безмірне збільшення грошей,
як на останню мету економіки». (\emph{Aristoteles}: «De Republica», ed. Bekker,
lib. I, c. 8 und 9 passim).
}.
\index{i}{0107}  %% посилання на сторінку оригінального видання

Як свідомий носій цього руху посідач грошей стає капіталістом.
Його особа, або, скорше, його кишеня, є пункт, звідки виходять
і куди повертаються гроші. Об’єктивний зміст тієї циркуляції
— зростання вартости — є його суб’єктивна мета, і лише
остільки, оскільки дедалі більше присвоєння абстрактного багатства
є єдиним движним мотивом його операцій, — лише остільки
він функціонує як капіталіст або персоніфікований, наділений
волею й свідомістю капітал. Отже, споживну вартість ніколи не
можна розглядати як безпосередню мету капіталіста\footnote{
«Товари (тут у значенні споживних вартостей) не є кінцева мета
капіталіста-купця\dots{} його кінцева мета є гроші» («Commodities are
not the terminating object of the trading capitalist\dots{} money is his terminating
object»). (\emph{Th. Chalmers}: «On Political Economy etc.» 2 nd ed. London
1832, p. 165, 166).
}. І не окремий
бариш, а лише безупинний рух одержування баришу є його
мета\footnote{
«Хоч купець і не вважає за ніщо вже одержаний бариш, а все ж
завжди має на меті майбутній бариш» («Il mercante non conta quasi per
niente il lucro fatto, ma mira sempre al futuro»). \emph{(A. Genovesi}: «Lezioni
di Economia Civile», 1765. Видання італійських економістів von Custodi,
Parte Moderna, vol. VIII, p. 139).
}. Це абсолютне прагнення до збагачення, ця пристрастна
гонитва за вартістю\footnote{
«Непогасна жага баришу, auri sacra fames, завжди характеризує
капіталіста». (\emph{Мас Culloch}: «The Principles of Political Economy»,
London 1830, p. 179). Цей погляд, звичайно, не заважає тому самому
Мак Куллохові й К° при теоретичних труднощах, приміром, при аналізі
перепродукції, перетворити того самого капіталіста на доброго буржуа,
що для нього справа ходить лише про споживну вартість і що в нього
навіть розвивається дійсно вовчий апетит до чобіт, капелюхів, яєць,
перкалю й інших в якнайвищій мірі звичайних сортів споживної вартости.
} є спільне в капіталіста і збирача скарбу,
але тимчасом як збирач скарбу є лише збожеволілий капіталіст,
капіталіст є раціональний збирач скарбу. Того безперестанного
зростання вартости, якого прагне збирач скарбу, рятуючи\footnote{
«\textgreek{Σώζειν}» (рятувати) — це один з найхарактеристичніших висловів
греків на означення скарботворення. Так само англійське «to save» значить
одночасно і «рятувати» і «зберігати».
} гроші від циркуляції, розумніший капіталіст досягає, віддаючи їх знову
й знов до циркуляції\footnoteA{
«Ту безмежність, якої речі не осягають, рухаючись в однім напрямі,
вони осягають шляхом кругобігу» («Quell’ infinito che le cose
non hanno nella progressione, la hanno nel giro»). (\emph{Galiani}, p. 156).
}.

Ті самостійні форми, грошові форми, які вартість товарів
\parbreak{}  %% абзац продовжується на наступній сторінці
