(«не має жодної дати»). Лише остільки його власна минуща
доконечність криється в минущій доконечності капіталістичного
способу продукції. Але остільки ж рушійним мотивом його
діяльности є не споживна вартість і не споживання, а мінова
вартість та її збільшення. Як фанатик зростання вартости, він
нещадно примушує людство до продукції задля продукції, отже,
до розвитку суспільних продуктивних сил і до створення тих
матеріяльних умов продукції, які тільки й можуть становити
реальну базу вищої суспільної форми, що її основний принцип
є повний і вільний розвиток кожного індивіда. Лише як персоніфікація
капіталу капіталіст є респектабельний. У цій ролі
він так само, як і збирач скарбів, пройнятий жагою абсолютного
збагачування. Але те, що в збирача скарбів становить індивідуальну
манію, у капіталіста є діяння суспільного механізму,
в якому він є лише одне колесо. Крім того, розвиток капіталістичної
продукції робить доконечним невпинне збільшення капіталу,
вкладеного в промислове підприємство, а конкуренція накидає
кожному індивідуальному капіталістові іманентні закони
капіталістичного способу продукції як зовнішні примусові закони.
Конкуренція примушує його невпинно збільшувати свій
капітал, щоб зберегти його, а збільшувати його він може лише
за допомогою проґресивної акумуляції.

Тим то, оскільки вся діяльність капіталіста є лише функція
капіталу, обдарованого в його особі волею і свідомістю, його
власне приватне споживання в його очах має значення грабежу
в акумуляції його капіталу подібно до того, як в італійській
бухгальтерії приватні видатки фігурують на сторінці дебету
капіталіста проти його капіталу. Акумуляція — це завойовання
світу суспільного багатства. Разом з масою експлуатованого людського
матеріялу вона поширює безпосереднє й посереднє панування
капіталіста.34

34    На прикладі старомодної, хоч і постійно відновлюваної форми
капіталіста, — на прикладі лихваря, Лютер дуже добре унаочнює властолюбство
як елемент жадоби до збагачення. «Поганці могли збагнути
своїм розумом, що лихвар тричі злодій і душогуб. Ми ж, християни, так
шануємо їх, що мало не молимося на них задля їхніх грошей... Той, хто
висисає в другого його харч, хто грабує і краде, так само є душогубець
(оскільки це від нього залежить), як і той, що голодом мордує когось
та заганяє на той світ. Але лихвар робить усе це, і все ж він спокійно
сидить у своєму кріслі, хоч і мав би висіти на шибениці, де б його шматувало
стільки ворон, скільки він накрав золотих, якби тільки на ньому
було стільки м’яса, шоб усі ті ворони могли пошматувати те м’ясо та
поділити між собою. Малих злодіїв вішають на шибениці... Малих злодіїв
тримають по в’язницях, а великі ходять собі, пишаючися, в золоті
та шовках... Отже, немає й більшого ворога людини на землі (крім чорта),
як скнара та лихвар, бо він хоче бути богом над усіма людьми. Турки,
вояки, тирани теж лихі люди, однак вони мусять давати людям жити й
визнають, що вони лихі люди й вороги; вони можуть, і-навіть мусять
іноді змилуватися над деким. Але лихвар і скнара хотів би, шоб увесь
світ пропадав з голоду, спраги, суму й нужди; він хотів би все, що навколо
нього є, мати лише собі, щоб усяк діставав усе від нього, наче від бога,
і був навіки його кріпаком. Він носить пишні мантії, золоті ланцюжки
