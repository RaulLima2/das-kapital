\parcont{}  %% абзац починається на попередній сторінці
\index{i}{0469}  %% посилання на сторінку оригінального видання
Звідси постає реакція, змальована вже при розгляді почасової
плати, не кажучи вже про те, що здовження робочого дня, навіть
коли відштучна плата лишається стала, само по собі включає
вже зниження ціни праці.

При почасовій заробітній платі, за деякими винятками, панує
рівна заробітна плата за ті самі функції, тимчасом як за відштучної
плати ціну робочого часу вимірюється, щоправда, певною
кількістю продукту, але денна або тижнева плата змінюється
залежно від індивідуальних ріжниць між робітниками, з яких
один дає за даний час мінімум, другий — пересічну кількість,
а третій — більш за пересічну кількість продукту. Отже, щодо
справжнього доходу робітників тут постають дуже великі ріжниці
залежно від різної вправности, сили, енергії, витривалости
й т. ін., індивідуальних робітників.\footnote{
«Там, де в певному підприємстві за працю платять від штуки\dots{}
заробітні плати різних робітників щодо суми можуть дуже значно відрізнятися
одна від однієї\dots{} Але за поденної плати існує звичайно однакова
норма\dots{} визнана так підприємцем, як і робітником за норму заробітної
плати для середніх робітників даного підприємства». («Where the
work in any trade is paid for by the piece at so much per job\dots{} wages may
very materially differ in amount\dots{} But in work by the day there is generally
an uniform rate\dots{} recognized by both employer and employed as the
standard of wages for the general run of workmen in the trade»). (\emph{Dunning}:
«Trades-Unions and Strikes», London 1860, p. 17).
} Це, звичайно, нічого не
змінює в загальному відношенні між капіталом та найманою
працею. Поперше, індивідуальні ріжниці для цілої майстерні
вирівнюються, так що майстерня за певний робочий час дає пересічну
кількість продукту, і вся заплачена заробітна плата буде
пересічною заробітною платою даної галузі промисловости. Подруге,
пропорція між заробітною платою та додатковою вартістю
лишається незмінна, бо індивідуальній платі поодинокого робітника
відповідає індивідуально спродукована ним маса додаткової
вартості. Але, даючи більший простір для індивідуальности,
відштучна плата прагне, з одного боку, до того, щоб розвивати в
робітників індивідуальність, а тим самим і почуття волі, самостійність
та самоконтроль, з другого боку, — розвивати поміж
ними конкуренцію. Тому вона має тенденцію разом із підвищенням
індивідуальних заробітних плат понад пересічний рівень
знижувати самий цей рівень. Але там, де певна відштучна
плата віддавна вкоренилась як традиція і де, отже, її зниження
являє собою особливі труднощі, — там хазяїни винятково вдавалися
також до насильного перетворення відштучної плата на почасову.
Проти цього вибухнув, наприклад, 1860 р. великий страйк
серед ткачів стьожок у Ковентрі.\footnote{
«Працю ремісників - підмайстрів реґулюється або поденно або
відштучно (à la journée ou à la pièce)\dots{} Хазяїни приблизно знають,
скільки роботи можуть виконати за день робітники в кожному реместві,
і тому вони платять їм часто пропорційно до виконаної ними роботи;
таким чином ці підмайстри в своєму власному інтересі працюють стільки,
скільки лише можуть, без усякого догляду». (\emph{Cantillon}: «Essai sur la
Nature du Commerce en Général», ed. Amsterdam 1756, p. 185 та 202. Перше видання появилося 1755 р.). Отже, вже Кантільйон, що в нього
чимало чого запозичили Кене, сер Джемс Стюарт та А. Сміс, з'ясовує
тут відштучну плату просто як модифіковану форму почасової плати.
Французьке видання Кантільйона на титульному аркуші позначене як
переклад з англійської мови, але англійське видання: «The Analysis
of Trade, Commerce etc. by Philip Cantillon, late of the City of London
Merchant» не тільки датоване пізніше (1759), а й своїм змістом показує,
що це пізніша переробка. Так, наприклад, у французькому виданні ще
не згадується про Юма, і, навпаки, в англійському Петті майже вже не
фігурує. Англійське видання має менше теоретичне значення, але воно
містить всякий специфічний матеріял про англійську торговлю, торговлю
зливками й т. ін., а цього у французькому тексті немає. Тому, здається,
фраза в заголовку англійського видання, що твір цей «запозичено головно
з рукопису одного високоталановитого джентлмена, нині, вже небіжчика,
і пристосовано і т. ін.» («taken chiefly from the Manuscript of
a very ingenious Gentleman deceased, and adapted etc.»), є щось більше
зa просту вигадку, звичайну для тих часів.
} Нарешті, відштучна плата
\index{i}{0470}  %% посилання на сторінку оригінального видання
є головна підпора змальованої раніше погодинної системи.\footnote{
«Скільки разів доводилося нам бачити в деяких майстернях куди
більше найнятих робітників, аніж цього потребувала праця. Часто наймають
робітників, лише передбачаючи якусь працю, при чому ще невідомо
напевне, чи буде вона, а іноді ця праця просто існує лише в уяві;
а що робітникам платять відштучно, то хазяїн нічим не ризикує, бо при
цьому всі втрати, які постають від гаяння часу, падають виключно на
робітників, що лишаються без праці» («Combien de fois n’avons-nous
pas vu, dans certains ateliers, embaucher beaucoup plus d’ouvriers que
ne le demandait le travail à mettre en main? Souvent, dans la prévision
d’un travail aléatoire, quelquefois même imaginaire, on admet des ouvriers:
comme on les paie auxpièces. on se dit qu’on ne court aucun risque, parce
que toutes les pertes de temps seront à la charge des inoccupés»). (\emph{H. Grégoire}
«Les Typographes devant le Tribunal Correctionnel de Bruxelles»,
Bruxelles 1865, p. 9).
}

З попереднього викладу випливає, що відштучна плата є найвідповідніша
капіталістичному способові продукції форма заробітної
плати. Хоч вона й не є щось нове — відштучна плата офіціяльно
фігурує, між іншим, побіч почасової плати у французьких
та англійських робітничих статутах XIV віку, — проте, ширше
поле вона добуває собі лише протягом власне мануфактурного
періоду. В епоху бурі й натиску великої промисловости, а саме
від 1797 до 1815 р., вона служить за підойму до здовжування
робочого часу та знижування заробітної плати. Душе важливий
матеріял для руху заробітної плати за тих часів находимо в
Синіх Книгах: «Report and Evidence from the Select Committee
on Petitions respecting the Corn Laws» (парламентська сесія
1813 -- 1814 pp.) і «Reports from the Lords’ Committee, on the
state of the Growth, Commerce and Consumption of Grain, and
all Laws relating thereto» (сесія 1814 -- 1815 pp.). Тут находимо
документальні докази невпинного знижування ціни праці від
початку антиякобінської війни. Наприклад, у ткацтві відштучна
плата так спала, що, не вважаючи на дуже здовжений робочий
день, поденна плата була тепер нижча, ніж раніш. «Реальний дохід
ткача куди менший, ніж раніш: його перевага над звичайним
робітником, колись дуже велика, майже цілком зникла. Справді,
ріжниця в заробітних платах вправної та звичайної праці тепер

\parbreak{}  %% абзац продовжується на наступній сторінці
