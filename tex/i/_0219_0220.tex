\parcont{}  %% абзац починається на попередній сторінці
\index{i}{0219}  %% посилання на сторінку оригінального видання
нехтували всіма виданими за останні 22 роки законами про дитячу
працю, їм ще й підсолоджено пілюлю. Парламент ухвалив, що
від 1 березня 1834~\abbr{р.} жодна дитина молодша від 11 років, від
1 березня 1835~\abbr{р.} жодна дитина молодша від 12 років і від 1 березня
1836~\abbr{р.} жодна дитина молодша від 13 років не повинні
працювати на фабриці більш, ніж 8 годин! Цей такий надто обережний
проти «капіталу» «лібералізм» був то більше вартий
подяки, що Д-р Фар, сер. А.~Карлейл, сер Б.~Бреді, сер. К.~Белл,
м-р. Ґетрі та інші, коротко — найвидатніші physicians і surgeons\footnote*{
— лікарі й хірурги. \emph{Ред.}
} Лондону в своїх свідченнях перед Палатою громад заявили, що
periculum in mora!\footnote*{
— у загаянні небезпека. \emph{Ред.}
} Д-р Фар висловився з цього приводу навіть
ще гостріше: «Законодавство однаково доконечне, щоб запобігти
передчасній смерті, хоч у яких формах ту смерть спричиняється,
а на цей спосіб (фабричний спосіб) безперечно треба дивитися
як на одну з найжорстокіших метод спричинення смерти»\footnote{
«Legislation is equally necessary for the prevention of death, in
any form in which it can be prematurely inflicted, and certainly this must
be viewed as most cruel mode of inflicting it».
}.
Той самий «реформований» парлямент, який із делікатного почуття
до панів фабрикантів ще на цілі роки загнав дітей, молодших
від 13 років, у пекло фабричної праці протягом 72 годин на
тиждень, навпаки, емансипаційним актом, що теж давав волю
краплями, з самого ж початку заборонив плянтаторам примушувати
негрів-рабів працювати більш ніж 45 годин на тиждень!

Але капітал, аж ніяк нечутливий до всіх цих поступок, розпочав
тепер на цілі роки шумну аґітацію. Вона оберталась, головне,
навколо віку категорій, які під назвою дітей мали працювати
не більш ніж 8 годин і які до певної міри підлягали примусовому
навчанню. За капіталістичною антропологією дитячий
вік кінчався десятим роком або, щонайвище, одинадцятим. Що
ближче надходив термін повного здійснення фабричного закону,
фатальний 1836~\abbr{р.}, то скаженіше лютувала фабрикантська зграя.
І їй дійсно пощастило так дуже залякати уряд, що останній
1835~\abbr{р.} запроєктував знизити межу дитячого віку з 13 до 12 років.
Тимчасом грізно наростав pressure from without\footnote*{
— натиск зовні. \emph{Ред.}
}. І не стало відваги
в Палати громад. Вона відмовилася від того, щоб кидати
тринадцятилітніх дітей під колісницю Джаґернавта\footnote*{
Один із видів індійського бога Вішну; під колісницю, на якій возять
його ідола, фанатичні індуси кидаються, щоб дати себе роздавити. \emph{Ред.}
} капіталу
більш, ніж на 8 годин удень, і акт 1833~\abbr{р.} набув повної сили.
Він лишався без зміни до червня 1844~\abbr{р.}

Протягом десятиріччя, коли він реґулював фабричну працю
спочатку частково, а потім цілком, офіціяльні звіти фабричних
інспекторів ущерть повні нарікань на неможливість провести його
в життя. А що закон 1833~\abbr{р.} залишив на волю панів-капіталістів
призначати в межах п’ятнадцятигодинного періоду, від пів на шосту
ранку до пів на дев’яту вечора, ту годину, коли кожний «підліток»
\index{i}{0220}  %% посилання на сторінку оригінального видання
і кожна «дитина» можуть починати свою дванадцятигодинну,
зглядно восьмигодинну працю, переривати й закінчувати
її, і так само залишив на їхню волю призначати різні години
на їжу для різних осіб, то ці пани швидко винайшли нову
«Relaissystem», за якої робочих коней не переміняють на певних
поштових станціях, а раз-у-раз наново запрягають на перемінних
станціях. Ми не зупиняємось довше на принадах цієї
системи, бо пізніш муситимемо повернутись до неї. Але вже і з
першого погляду ясно, що ця система знищила не лише дух, а й
саму букву цілого фабричного закону. Як могли б фабричні
інспектори за такої складної бухгальтерії щодо кожної окремої
дитини й кожного підлітка примусити фабрикантів додержувати
визначеного законом робочого часу й давати на визначений законом
час перерву на їжу? На більшості фабрик незабаром знов
безкарно почали процвітати давніші жорстокі неподобства.
На нараді з міністром унутрішніх справ (1844) фабричні інспектори
довели неможливість якогобудь контролю за нововигаданої
системи змін\footnote{
«Reports of Insp. of Fact, for 31 st October 1849», p. 6.
}. Але тимчасом обставини дуже змінилися. Фабричні
робітники, особливо від 1838~\abbr{р.}, зробили своїм економічним гаслом
десятигодинний біл, подібно до того, як Charter\footnote*{
Хартія. Мова йде про хартію чартистів. Її пункти: загальне виборче
право, щорічне переобрання парламенту, оплата депутатів парламенту,
таємне голосування, рівні виборчі округи і скасування майнового
цензу. \emph{Ред.}
} вони зробили своїм
політичним гаслом. Навіть частина фабрикантів, що вреґулювала
фабричну продукцію згідно з законом 1833~\abbr{р.}, закидала парламент
меморіалами про неморальну «конкуренцію» «фальшивих братів»,
що їм їхня більша нахабність або щасливіші місцеві обставини
дозволяють ламати закон. До того, хоч і як дуже хотілось би
окремим фабрикантам дати повну волю своїй давнішній ненажерливості,
оратори й політичні провідники класи фабрикантів радили
змінити поведінку й мову щодо робітників. Вони розпочали
були похід за скасування хлібних законів і для перемоги потребували
допомоги робітників! Тому вони й обіцяли не лише подвоїти
вагу буханця хліба, але й ухвалити закон про десятигодинний
робочий день у тисячолітньому царстві Free Trade\footnote*{
— вільної торговлі. \emph{Ред.}
}\footnote{
«Reports of Insp. of Fact, for 31 st October 1848», p. 98.
}. Отже, їм то менше треба було боротися проти тих заходів, що
повинні були лише здійснити закон 1833~\abbr{р.} Нарешті, торі, що їхньому
найсвятішому інтересові, земельній ренті, загрожувала
небезпека, у своїм філантропічнім обуренні загриміли на «ганебну
поведінку»\footnote{
Зрештою, Леонард Горнер офіціально вживає вислову «nefarious
practices» (ганебна поведінка). («Reports of Insp. of Fact, for 31 st
October 1859», p. 7).
} своїх ворогів.

Так з’явився додатковий фабричний закон з 7 червня 1844~\abbr{р.}
Він набув сили 10 вересня 1844~\abbr{р.} Ним ставиться під охорону закону
нову категорію робітників, а саме жінок, старших за 18 років.
\parbreak{}  %% абзац продовжується на наступній сторінці
