ren’s Employment Commission» справді доводять, що в деяких
галузях промисловости регулювання робочого дня тільки рівномірніше розподілило б на цілий рік ту
масу праці, якої вживається вже в цих галузях; 288 що це реґулювання є перше раціональне
обмеження людовбивчих, нісенітних, пустотливих примх моди,289
які сами по собі не відповідають системі великої промисловости;
що розвиток океанського судноходства й засобів комунікації
взагалі усунув власне технічну підставу сезонової праці; 290
що всі інші обставини, яких нібито не можна контролювати,
усувається поширенням будівель, додатковими машинами, збільшенням числа одночасно занятих робітників
291 і зворотним впливом, що його всі ці зміни справляють на систему великої торгівлі.292  Проте
капітал, як він це не раз заявляв устами своїх

for transit. It quite failed at that time of proof when put to the test, and
again it will certainly fail should it have to be tried»). («Reports of Insp.
of Fact, for 31 st October 1862», p. 54, 55).

288 «Children’s Employment Commission, 4 th Report», p. XVIII,
n. 118.

289 ДжоніБеллерс уже 1699 р. зауважує: «Непостійність мод збільшує
число бідних. Вона спричиняє два великі лиха: 1) робітники бідують
узимку від недостачі праці, бо торговці матеріями і ткачі-хазяїни не
наважуються витрачати свої капітали, щоб дати робітникам роботу, поки
настане весна й виявиться, яка буде мода; 2) по весні бракує робітників,
і ткачі-хазяїни мусять брати багато учнів, щоб задовольнити потреби
королівства протягом кварталу або півроку; це відбирає руки від рільництва, позбавляє село робочих
сил і здебільша переповнює міста жебраками; а ті, що соромляться жебракувати, зимою помирають з
голоду».
(«The uncertainty of fashions does increase necessitous Poor. It has two
great mischiefs in it: 1st) The journeymen are miserable in winter for want
of work, the mercers and master-weavers not daring to layout their stocks
to kepp the journeymen imployed before the spring comes and they know
what the fashion will then be; 2dly) In the spring the journeymen are
not sufficient, but the master-weavers must draw in many prentices, that
they may supply the trade of the kingdom in a quarter or half a year, which
robs the plow of hands, drains the country of labourers, and in a great part
stocks the city with beggars, and starves some in winter that are ashamed
to beg»). («Essays about the Poor, Manufactures etc.», p. 9).

290 «Children’s Employment Commission. 5 th Report», p. 171, n. 31.

291    Так, наприклад, у свідченнях бретфордських торговців-експортерів читаємо: «За цих обставин
ясно, що немає потреби примушувати
дітей працювати по крамницях довше, ніж від 8 години ранку до 7—7V2 годин вечора. Це — справа лише
додаткових видатків і додаткових рук.
[Дітям не треба було б працювати до пізньої ночі, коли б деякі підприємці не були такі жадні на
бариші: додаткова машина коштує лише
16 або 18 фунтів стерлінґів]... Всі труднощі випливають із недостатнього
устаткування та недостатнього помешкання». (Там же, стор. 171, n. 31,
36 і 38).

292 Там же. Один лондонський фабрикант, який, зрештою, розглядає
примусове реґулювання робочого дня як засіб захисту робітників проти
фабрикантів і самих фабрикантів проти великої торговлі, свідчить: «Нашу
промисловість притискують торговці-експортери, які, приміром, відсилаючи товари вітрильним кораблем,
хочуть до початку певного сезону
бути вже на місці і разом з тим сховати собі до кишені ріжницю між
фрахтом вітрильного корабля й пароплава; абож із двох пароплавів вони
хочуть вибрати собі той, що відпливає раніш, щоб з’явитися на закордонному ринку попереду своїх
конкурентів».
