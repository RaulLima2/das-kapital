Проти рівновеликої суми відокремлених індивідуальних робочих
днів комбінований робочий день продукує більші маси споживних
вартостей і тому зменшує робочий час, потрібний, щоб
досягти певного корисного ефекту. Чи комбінований робочий
день у даному випадку дістає цю збільшену продуктивну силу
тому, що він підносить механічну силу праці, чи тому, що поширює
її просторову сферу діяння; чи тому, що він супроти маштабу
продукції просторово звужує продукційне поле; чи тому, що в
критичний момент він пускає в рух багато праці за короткий час;
чи тому, що заохочує поодиноких осіб до змагання та напружує
їхній життєвий дух; чи тому, що він накладає печать безперервности
та багатобічности на однорідні операції багатьох осіб; чи
тому, що виконує одночасно різні операції, чи тому, що економізує
засоби продукції через спільний ужиток їх; чи тому, що надає
індивідуальній праці характеру пересічної суспільної праці, —
за всяких обставин специфічна продуктивна сила комбінованого
робочого дня є суспільна продуктивна сила праці, або продуктивна
сила суспільної праці. Вона випливає із самої кооперації. У пляномірному
співробітництві з іншими робітник стирає свої індивідуальні
межі й розвиває свою родову спроможність. 19

рів, концентрується тепер для досконалішого оброблення 100 акрів».
Хоч «проти кількости вживаного капіталу й праці просторінь і скоротилася,
проте сфера продукції поширилася супроти тієї сфери, що її
раніш мав або експлуатував поодинокий незалежний аґент продукції»).
(«In the progress of culture all, and perhaps more than all the capital an
labour which once loosely occupied 500 acres, are now concentrated for the
more complete tillage of 100. Relatively to the amount of capital and labour
employed, space is concentrated, it is an enlarged sphere of production,
as compared to the sphere of production formely occupied or worked upon
by one single, independent agent of production»). (R. Jones: «An Essay,
on the Distribution of wealth. Part I. On Rent», London 1831, p. 191,
199).

19 «Сила кожної людини мінімальна, але сполука мінімальних сил
утворює спільну силу, більшу за суму цих сил, так що сили через саме
своє об’єднання можуть зменшити час та збільшити сферу своєї акції»
(«La forza di ciascuno uomo è minima, ma la riunione delle minime forze
forma una forza totale maggiore anche della somma delle forze medesime
fino a che le forze per essere riunite possono diminuere il tempo ed accrescere
lo spazio della loro azione»). (G. R. Carli примітка до P. Verri: «Meditazioni
sulla Economia Politica». vol. XV, p. 196). [«Колективна
праця дає такі результати, яких ніколи не могла б дати індивідуальна
праця. Отже, у міру того як зростатиме кількість людности, продукти
об’єднаної промисловости значно переважатимуть суму, що її ми мали
в наслідок простого складання, обчисленого на основі цього зросту...
У сфері механічних робіт так само, як і в сфері наукових робіт, людина
може протягом одного дня фактично зробити більше, ніж ізольований
індивід протягом усього свого життя. Аксіома математиків, що ціле
дорівнює сумі частин, прикладена до нашого предмету, вже не є правильна.
Щодо праці, цієї великої основи існування людства, то можна
сказати, що продукт об’єднаних зусиль значно переважає все те, що
могли б колибудь спродукувати зусилля поодиноких і розрізнених індивідів».
— Th. Sadler: «The Law of Population», London 1850].*

* Наведене тут у прямих дужках ми беремо з французького видання.
(«Le Capital etc.», v. I, ch. XIII, p. 143). Ред.
