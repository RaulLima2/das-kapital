\parcont{}  %% абзац починається на попередній сторінці
\index{i}{0501}  %% посилання на сторінку оригінального видання
явище, властиве всім способам продукції й що на ньому ми на
хвилину спішимося в аналізі процесу циркуляції.\footnote*{
У французькому виданні це речення подано так: «Звичайний спосіб
вислову сплутує також капіталістичну акумуляцію, що є процес
продукції, з двома іншими економічними явищами, а саме: з нагромадженням
у споживному фонді багатіїв дібр, які споживаються лише
повільно, та з творенням запасів споживання — явищем, властивим усім
способам продукції». («Le Capital etc.», v. I, ch. XXIV, p. 257). \emph{Ред.}
}

Отже, в цьому розумінні клясична політична економія має
рацію, коли вона підкреслює як характеристичний момент процесу
акумуляції те, що додатковий продукт мусить споживатись
продуктивними робітниками, а не непродуктивними. Але тут починається
й її помилка. А. Сміт завів моду малювати акумуляцію як
просте споживання додаткового продукту продуктивними робітниками,
або малювати капіталізацію додаткової вартости як просте
перетворення її на робочу силу. Послухаймо, наприклад, Рікарда:
«Треба зрозуміти, що всі продукти країни споживаються; але величезна
ріжниця, яку тільки можна собі уявити, є в тому, чи споживаються
вони тими, що репродукують якусь іншу вартість, чи тими,
що її не репродукують. Коли ми кажемо, що дохід заощаджується
й додається до капіталу, то ми розуміємо під цим, що ту частину
доходу, про яку кажуть, що її додається до капіталу, споживається
продуктивними, а не непродуктивними робітниками. Немає
більшої помилки, як припускати, що капітал збільшується через
неспоживання».\footnote{
\emph{Ricardo}: «Principles of Political Economy», 3rd, ed. London
1821, p. 163, примітка.
} Немає більшої помилки, як та, що її за
А. Смітом проказують Рікардо і всі пізніші економісти, а саме,
що «ту частину доходу, про яку кажуть, що її додається до капіталу,
споживається продуктивними робітниками». За цим уяввленням
вся додаткова вартість, що перетворюється на капітал,
ставала б змінним капіталом. Навпаки, вона, як і первісно авансована
вартість, поділяється на сталий капітал і змінний капітал,
на засоби продукції й робочу силу. Робоча сила є та форма, що
в ній змінний капітал існує в процесі продукції. В цьому процесі
її саму споживає капіталіст. Вона ж своєю функцією, працею,
споживає засоби продукції. Одночасно гроші, заплачені при
купівлі робочої сили, перетворюються на засоби існування, що
їх споживає не «продуктивна праця», а «продуктивні робітники».
За допомогою аналізи, цілком хибної в своїй основі, А. Сміс
доходить такого недоладного результату, що хоч кожний індивідуальний
капітал і поділяється на сталу і змінну складову
частину, все ж суспільний капітал сходить лише на змінний
капітал, або його витрачають лише на виплату заробітної плати.
Нехай, наприклад, фабрикант сукна перетворює \num{2.000}\pound{ фунтів стерлінґів}
на капітал. Одну частину цих грошей він витрачає на купівлю
ткачів; другу — на купівлю вовняної пряжі, машин і т. д.
Але люди, що в них він купує пряжу й машини, знову оплачують
частиною з тих грошей працю і т. д., поки всі \num{2.000} фунтів
\index{i}{0502}  %% посилання на сторінку оригінального видання
стерлінґів будуть витрачені на заробітну плату, або поки
цілий продукт, що його репрезентують \num{2.000}\pound{ фунтів стерлінґів},
буде спожитий продуктивними робітниками. Ми бачимо: вся
сила цього арґументу лежить у словах «і т. д.», що посилають
нас від Понтія до Пілата. Дійсно, А. Сміс уриває свій дослід
саме там, де починаються його труднощі.\footnote{
Не вважаючи на свою «Логіку», Дж. Ст. Мілл ніде навіть і не
помічає цієї хибної аналізи своїх попередників, яка навіть у межах буржуазного
горизонту, просто з погляду фахівця, потребує поправок. Він
скрізь реєструє з догматизмом школяра плутанину думок своїх учителів.
Так само й тут: «Сам капітал згодом цілком сходить на заробітну плату,
і навіть коли він через продаж продукту відновлюється, він потім знову
перетворюється на заробітну плату» («The capital itself in the long run
becomes entirely wages, and when replaced by the sale of produce becomes
wages again»).
}

Поки ми беремо на увагу лише фонд річної продукції в цілому,
річний процес репродукції легко зрозуміти. Але всі складові
частини річної продукції треба винести на товаровий ринок, і
тут саме починаються труднощі. Рухи поодиноких капіталів і
особистих доходів перехрещуються між собою, переплутуються,
губляться в загальній зміні місць — у циркуляції суспільного
багатства — у тій зміні місць, що спантеличує спостерігача та
ставить дослідові дуже заплутані завдання. У третьому відділі
другої книги я подам аналізу дійсного зв’язку всіх тих явищ.
[Там виявиться, що догма А. Сміта, успадкована всіма його послідовникам,
перешкоджала політичній економії зрозуміти навіть
елементарний механізм суспільного процесу репродукції].\footnote*{
Заведене у прямі дужки ми беремо з другого німецького видання.
\emph{Ред.}
} Велика
заслуга фізіократів у тому, що вони в своєму «tableau économique»
вперше зробили спробу дати картину річної продукції
в тому вигляді, в якому вона виходить із циркуляції.\footnote{
А. Сміс у своєму викладі процесу репродукції, отже, і процес v
акумуляції, в деякому відношенні не тільки не зробив жодного поступу,
але зробив рішучий крок назад порівняно з своїми попередниками, особливо
порівняно з фізіократами. З його ілюзією, згаданою в тексті, пов’язана
справді казкова догма, також перейнята від нього політичною економією,
що ціна товарів складається із заробітної плати, зиску (процента)
і земельної ренти, отже, лише із заробітної плати й додаткової вартосте.
Виходячи з цієї бази, Шторх принаймні наївно признається: «Неможливо
розкласти доконечну ціну на її найпростіші елементи» («H est
impossible de résoudre le prix nécessaire dans ses éléments les plus simples»).
(\emph{Storch}: «Cours d’Economie Politique», ed. Petersbourg 1815,
vol. II, p. 140, примітка). Гарна економічна наука, що проголошує за
неможливе розкласти ціну товарів на її найпростіші елементні Докладніше
про це питання ми скажемо в третьому відділі другої і в сьомому
відділі третьої книги.
}

А втім, само собою зрозуміло, що політична економія не проминула
використати в інтересах кляси капіталістів тезу А. Сміта,
ніби всю перетворену на капітал частину чистого продукту споживає
робітнича кляса.
