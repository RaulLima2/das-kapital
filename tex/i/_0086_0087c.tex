\parcont{}  %% абзац починається на попередній сторінці
\index{i}{0086}  %% посилання на сторінку оригінального видання
заставу».\footnote{
«Гроші є застава» («Money is a pledge»). (\emph{John Bellers}: «Essays
about the Poor, Manufactures, Trade, Plantations and Immorality», London
1699, p. 13).
} Його потреби безперестанно відновляються й безперестанно
спонукують купувати чужі товари, тимчасом як продукція
й продаж його власного товару коштують часу й залежать
од випадків. Щоб купити, не продаючи, він мусить раніш продати,
не купуючи. Ця операція, проведена в суспільному маштабі,
здається, сама собі суперечить. Однак біля джерел своєї продукції
благородні металі обмінюється безпосередньо на інші товари.
Тут відбувається продаж (з боку посідача товарів) без купівлі
(з боку посідача золота або срібла).\footnote{
Купівля в категоричному розумінні має за свою передумову, що
золото або срібло є власне вже перетворена форма товару, або продукт
продажу.
} І пізніші продажі без наступних
купівель упосереднюють лише дальший розподіл благородних
металів серед усіх посідачів товарів. Таким чином на
всіх пунктах циркуляції постають золоті й срібні скарби найрізнішого
розміру. Разом з можливістю затримати товар як
мінову вартість або мінову вартість як товар прокидається й
жадоба на золото. З поширенням товарової циркуляції зростає й
влада грошей, цієї завжди готової до бою абсолютно суспільної
форми багатства. «Золото — чудова річ! Хто його має, той пан
над усім, чого він бажає. За допомогою золота можна навіть
душам відчинити двері в рай» (Колюмб, у листі з Ямайки, 1503 р.).
А що по грошах не пізнати, щó на них перетворено, то на гроші
перетворюється все, — товари й не-товари. Все стає предметом
продажу й купівлі, все можна купити й продати. Циркуляція
стає величезною суспільною ретортою, в яку все впадає, щоб
знову вийти звідти у формі грошового кристалу. Проти цієї
альхемії не можуть устояти навіть мощі святих, не кажучи вже
про менш грубі res sacrosanctae, extra commercium hominum.\footnote*{
Святі речі, що ними люди не торгують. \emph{Ред.}
}\footnote{
Генріх III, найхристияніший король Франції, грабує в манастирів
і т. ін. їхні реліквії, щоб перетворити їх на гроші. Відомо, яку ролю
в історії Греччини відіграло пограбування скарбів дельфійського храму
фокійцями. Мешканням бога товарів у стародавніх народів були, як відомо,
храми. Вони були «святими банками». У фінікійців, народу торговельного,
par excellence, гроші вважалося за преображену форму (entäusserte
Form) всіх речей. Отже, було цілком природно, коли дівчата, що в
свята богині кохання віддавалися чужинцям, жертвували богині монету,
одержану ними в нагороду.
} Як у грошах стирається всяка якісна ріжниця товарів, так гроші
й собі, як радикальний левелер, затирають усяку ріжницю.\footnote{
\begin{verse}
\vspace{-\dimexpr\baselineskip+\topsep}
Gold! yellow, glittering precious gold! \\
Thus much of this, will make black white; foul, fair, \\
Wrong, right; base, noble; old, young; coward, valiant. \\
\dots{} What this, you gods! Why this \\
Will lug jour priests and servants from your sides; \\
Pluck staut men’s pillows from below their heads. \\
This yellow slave \\
Will knit and break religions; bless the accurs'd; \\
Make the hoar leprosy ador’d; place thieves \\
And give them title, knee and approbation \\
With senators of the bench; this is it, \\
That makes the wappen’d widow wed again \\
\dots{} Come damned earth, \\
Thou commun whore of mankind. \\

\hspace{2em}[О золото, блискуче, жовте, золоте, \\
\hspace{2em}Із черні біле робиш, з гнилого — гарне,\\
\hspace{2em}Із кривди — правду, з підлого — високе,\\
\hspace{2em}З старого — молоде, героя — з боягуза.\\
\hspace{5em}Чому це, боги? Як воно\\
\hspace{2em}Ваших жерців і слуг одштовхує від вас\\
\hspace{2em}І спати не дає здоровим людям?\\
\hspace{2em}Цей жовтий раб святе\\
\hspace{2em}Пов’язує і рве; благословляє те, що\\
\hspace{5em}проклинають,\\
\hspace{2em}Обожнює гидку проказу, злодія виносить\\
\hspace{2em}В сенат, і ранґ, і честь, і славу\\
\hspace{2em}Йому дає; і удову плачущу\\
\hspace{2em}За молодого заміж видає.\\
\hspace{2em}Метале клятий,\\
\hspace{2em}Повія ти для всього людства спільна!]\\
\hspace{4em}(Шекспір: «Тімон із Атен»).
\end{verse}
}
\index{i}{0087}  %% посилання на сторінку оригінального видання
Але гроші сами є товар, зовнішня річ, яка може стати приватною
власністю кожного. Таким чином суспільна сила стає приватною
силою приватної особи. Тим то античне суспільство лихословить
гроші як монету, на яку розмінявся їхній економічний та моральний
лад.\footnote*{
У французькому виданні це речення подано так: «Тим-то
античне суспільство лихословить гроші як найактивнішого підривника й руїнника
його економічної організації і народніх звичаїв». («Le Capital etc.» v. I,
ch. III, p. 54). \emph{Ред.}
}\footnote{
\begin{verse}
\vspace{-\dimexpr\baselineskip+\topsep}
\textgreek{„Οὐδέν γὰρ ἀνθρώποισιν οἷον ἀργυρὸς \\
Κακὸν νόμισμα ἔβλαστε, τοῦτο καὶ πόλεις \\
Πορθεῖ, τόδ’ ἄνδρας ἐξανίστησιν δόμων. \\
Τόδ’ ἐκδιδάσκει καὶ παραλλάσσει φρένας \\
Χρηστὰς πρὸς αἰσχρὰ ἀνθρώποις ἔχειν, \\
Καὶ παντὸς ἔργου δυσσέβειαν εἰδέναι.“} \\
\hspace{2em}[Немає лиха гіршого за гроші, \\
\hspace{2em}За срібло те. Міста воно руйнує, \\
\hspace{2em}Людей з домів навіки виганяє, \\
\hspace{2em}Невинні душі научає зла, \\
\hspace{2em}Скеровує людину до розпусти, \\
\hspace{2em}В усякій справі путь показує безбожну]. \\
\hspace{4em}(Софокл: «Антігона»).
\end{verse}
} Сучасне суспільство, яке вже в своєму дитячому
віці витягає\footnote{
«Жадобою сподіваються самого Плутоса витягти з середини землі»
(«\textgreek{Ἐλπιζούσης  πλεονεξίας ἀνάξειν ἐκ τῶν μυκῶν τῆς γῆς αὐτὸν τὸν Πλούτωνα}). (\emph{Atheneus}:
«Deipnosophistai»).
} Плутоса за чуба з надр землі, вітає в золотому
Ґраалі\footnote*{
Ґрааль — y середньовічних леґендах чаша, зроблена з самоцвіту,
в яку нібито зібрано кров з ран розп’ятого Ісуса. \emph{Ред.}
} блискуче втілення свого власного життєвого принципу.

Товар як споживна вартість задовольняє окрему потребу й
становить окремий елемент речового багатства. Але вартість
\parbreak{}  %% абзац продовжується на наступній сторінці
