\parcont{}  %% абзац починається на попередній сторінці
\index{i}{0508}  %% посилання на сторінку оригінального видання
пустив червоного півня. По цей бік каналу швидко зростав оуенізм,
по той бік — сен-сімонізм і фур'єризм. Тоді настав час
для вульґарної політичної економії. Саме за рік перед тим, як
Нассав В. Сеніор із Менчестеру відкрив, що зиск (включаючи
і процент) із капіталу є продукт неоплаченої «останньої дванадцятої
години праці», він сповістив світові про своє друге відкриття.
«Я, — врочисто сказав він тоді, — заміняю слово капітал,
розглядуваний як знаряддя продукції, на слово поздержливість
(Abstinenz)»\footnote{
Senior: «Principes fondamentaux de l’Economie Politique». Trad.
Arrivabene, Paris 1836, p. 308. Але для прихильників старої класичної
школи це було вже трохи занадто безглуздо. «Пан Сеніор замінює вислів
«праця й капітал» на вислів «праця й поздержливість»... Поздержливість
— це просте заперечення. Не поздержливість, а споживання продуктивно
вживаного капіталу становить джерело зиску». (John Cazenove
у примітці до його видання праці Малтуза «Definitions in Political
Economy», London 1853, стор. 130, примітка). Навпаки, Джон. Ст.
Мілл на одній сторінці списує Рікардову теорію зиску, а на другій приймає
Сеніорову теорію «нагороди за поздержливість» («remuneration
of abstinence»). Банальні суперечності так само рідні для нього, як чужа
для нього геґелівська «суперечність», це джерело всякої діялектики.

Додаток до другого видання. Вульґарному економістові ніколи не
впадала в голову та проста думка, що всяку людську дію можна розглядати
як «поздержливість» від протилежної дії. їсти — значить поздержуватися
від посту, ходити — поздержуватися від стоянки, працювати —
поздержуватися від ледарства, ледарювати — поздержуватися від праці
й т. д. Ці пани добре зробили б, коли б подумали над словами Спінози:
«Determinatio est negatio» («Визначення — це заперечення»).
}. Це — незрівнянний зразок «відкрить» вульґарної
економії 1 Економічну категорію вона заміняє на сикофантську
фразу. Voila tout.\footnote*{
Ось і все. Ред.
} «Коли дикун, — навчає Сеніор, — робить
лук, то він займається промисловістю, але не практикує
поздержливости». Це пояснює нам, як і чому за попередніх суспільних
становищ засоби праці фабрикувалося «без поздержливости»
капіталіста. «Що більше суспільство проґресує, то більше
вимагає воно поздержливости»,\footnote{
Senior, там же, стор. 342.
} саме від тих, хто займається
працею присвоювання собі чужої праці та її продукту. Всі умови
процесу праці перетворюються відтепер на відповідну кількість
актів поздержливости капіталіста. Що збіжжя не тільки їдять,
а й сіють, то це — через поздержливість капіталіста! Що вино
витримують певний час, то це теж через поздержливість капіталіста!\footnote{
«Ніхто... не сіятиме, наприклад, своєї пшениці, лишаючи її
12 місяців у землі, або не триматиме свого вина цілі роки в льоху замість
Одразу спожити ці речі або їхній еквівалент, коли він не сподіватиметься
одержати таким способом збільшену вартість і т. ін.» (No one... will sow
his wheat, f. i., and allow it to remain a twelve-month in the ground,
or leave his wine in a cellar for years, instead of consuming these things
or their equivalent at once — unless he expects to acquire additional value
etc.». (Scrope: «Political Economy». Ed. A. Potter, New Уогк 1841,
p. 133, 134).
} Капіталіст грабує свою власну плоть, коли «позичає(!)
робітникові знаряддя продукції», іншими словами, коли,
сполучивши їх з робочою силою, він вживає їх як капітал, замість
\index{i}{0509}  %% посилання на сторінку оригінального видання
з’їдати парові машини, бавовну, залізниці, добриво, робочих
коней тощо або, як це собі по-дитячому уявляє вульгарний
економіст, протринькати «їхню вартість» на розкоші
й інші засоби споживання.\footnote{
«Нестатки, що їх бере на себе капіталіст, позичаючи (цього
евфемізму вжито на те, щоб за випробованою манерою вульґарних економістів
ідентифікувати найманого робітника, визискуваного промисловим
капіталістом, із самим промисловим капіталістом, що позичає гроші
в капіталістів-кредиторів!) свої знаряддя продукції робітникові, замість
присвятити їхню вартість своєму власному споживанню, перетворивши
їх на предмети споживання або втіх» («La privation que s’impose le
capitaliste, en prêtant ses instruments de production au travailleur au lieu
d’en consacrer la valeur à son propre usage, en la transformant en objets
d'utilité ou d’agrément»). (G. de Molinari: «Etudes Economiques», Paris
1846, p. 36).
} Як саме кляса капіталістів має
це зробити, — це таємниця, що її досі вперто зберігає вульгарна
економія. Досить. Світ живе лише з самокатування капіталіста,
цього сучасного покаянного поклонника Вішну. Не тільки
акумуляція, а й просте «збереження капіталу потребує постійного
напруження, щоб устояти проти спокуси з’їсти його».\footnote{
«La conservation d’un capital exige... un effort... constant pour
résister à la tentation de le consommer». (Cuurcelle Seneuil: «Traité théorique
et pratique des entreprises industrielles», 2 ème éd. Paris 1857, p. 20).
}
Отже, проста гуманність, очевидно, вимагає визволити капіталіста
від цього мучеництва і спокуси, визволити його таким самим
способом, яким недавно через скасування рабства визволено
ґеорґійського рабовласника від тяжкої дилеми — чи прогуляти
геть чисто на шампанське додатковий продукт, видушений із
негрів-рабів, чи знову перетворити його частково на додаткову
кількість негрів і землі.

В найрізніших суспільно-економічних формаціях відбувається
не тільки проста репродукція, але ще й — правда, в різних
розмірах — репродукція в поширеному маштабі. Щораз більше
продукують і більше споживають, отже, і більше продукту перетворюють
на засоби продукції. Але цей процес не є акумуляція
капіталу, а тому й не є він функція капіталіста, доки засоби продукції
робітника, а тому і його продукт і його засоби існування
не протистоять ще йому в формі капіталу.\footnote{
«Осібні кляси доходу, що найбільше сприяють прогресові національного
капіталу, змінюються на різних стадіях розвитку, а тому вони
цілком різні в націй, що стоять на різних ступенях розвитку... Зиск...
на давніших стадіях суспільного розвитку... є незначне джерело акумуляції
по івняно із заробітною платою й рентою... Коли сили національної
праці до певної міри зростають, то відносне значення зиску як
джерела акумуляції зростає». («The particular classes of income which
yield the most abundantly to the progress of national capital, change at
different stages of their progress, and are therefore entirely different in
nations occupying different positions in that progress... Profits... unimportant
source of accumulation, compared with wages and rents, in the
earlier stages of society... When a considerable advance in the powers of
national industry has actually taken place, profits rise into comparative
importance as a source of accumulation»). (Richard Jones: «Textbook
etc.», p. 16, 21).
} Померлий перед
кількома роками Річард Джонс, наступник Малтуза на катедр
\parbreak{}  %% абзац продовжується на наступній сторінці
