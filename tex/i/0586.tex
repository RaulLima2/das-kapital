Загальний результат досліджень лікарської комісії 1863 р.
про стан харчування гірше харчованих кляс народу вже відомий
читачеві. Читач пригадує собі, що харчі більшої частини родин
сільських робітників стоять нижче мінімальної міри, потрібної,
«щоб забезпечити себе від хороб у наслідок голоду». Так стоїть
справа особливо по всіх суто рільничих округах — Cornwall, Devon
Somerset, Witts, Stafford, Oxford, Berks і Herts. «Кількість
харчів, що їх дістає сільський робітник, — каже д-р Сімон, —
більша, ніж показує пересічна кількість, бо сам він дістає куди
більшу частину засобів існування, доконечну для його праці,
аніж інші члени його родини; в найбідніших округах він дістає
майже все м’ясо або сало. Та кількість харчів, що припадає на
жінку, а так само й на дітей у період їхнього швидкого зросту,
в багатьох випадках і майже по всіх графствах недостатня, особливо
щодо азоту».\footnote{
«Public Health. Sixth Report 1863», p. 238, 249, 261,262.
} Наймити й наймички, що живуть у самих
фармерів, харчуються добре. Число їх із 288.277 в 1851 р. спало
до 204.962 в 1861 р. «Праця жінок на полі, — каже д-р Сміс, —
хоч би й якими взагалі шкідливими наслідками вона супроводилася,
за сучасних обставин є дуже корисна для родини, бо дає
їй засоби на взуття, одяг, квартирну плату, і таким чином змогу
краще харчуватись».\footnote{
Там же, стор. 262.
} Одним із найвизначніших результатів
цього дослідження було виявлення того факту, що сільський
робітник в Англії харчується куди гірше, ніж в інших частинах
Сполученого королівства («is considerably the worst fed»), як
це видно з нижченаведеної таблиці.

Тижневе споживання вуглецю й азоту пересічно
на одного сільського робітника
                                                                                Вуглецю             
             Азоту
                                                                                (грани)             
               (грани)
Англія.....................                                           40,673                        
     1,594
Велз........................                                           48,354                       
       2,031
Шотландія..............                                            48,980                           
   2,348
Ірландія.................                                             43,366                        
     2,439 161

161 Там же, стор. 17. Англійський сільський робітник дістає лише
четвертину тієї кількости молока й лише половину тієї кількости хліба,
яку дістає ірляндський сільський робітник. Кращі умови харчування
ірландського сільського робітника відзначив уже А. Юнґ на початку
цього століття в своїй «Tour through Ireland». Причина цього та, що бідний
ірляндський фармер куди гуманніший, ніж багатий англійський.
А щодо Велзу, то наведені в тексті дані не стосуються до його південнозахідньої
частини. «Всі тамошні лікарі згоджуються на тому, що збільшення
проценту смертности від туберкульози, золотухи й т. ін. інтенсивно
вростає з погіршанням фізичного стану людности, і це погіршання всі
приписують злидням. Денне утримання сільського робітника обчислюють
там у 5 пенсів, у багатьох округах фармер (що й сам бідує) платить іще
менше. Шматок засоленого м’яса, сухий, як тверде червоне дерево, і ледве
