на своїх сеньйорів за заробітну плату взагалі нижчу від заробітної
плати звичайних поденників, не кажучи вже про ті невигоди
і втрати, які постають для них у наслідок того, що в критичну
пору сівби або жнив вони мусять занедбувати свої власні поля».187і

Отже, незабезпеченість та іреґулярність заняття, часті й довготривалі
перерви у роботі — всі ці симптоми відносного перелюднення
фігурують у звітах інспекторів адміністрації в справах
про бідних, як так само численні тяготи ірляндського рільничого
пролетаріату. Ми пригадуємо собі, що подібні явища ми бачили
й серед англійського сільського пролетаріату. Але ріжниця
в тому, що в Англії, промисловій країні, промислову резервну
армію рекрутують на селі, тимчасом як в Ірляндії, у рільничій
країні, рільничу резервну армію рекрутують у містах, притулках
вигнаних сільських робітників. В Англії зайві сільські робітники
перетворюються на фабричних, в Ірляндії ж, загнані в міста,
вони, справляючи тиск на заробітну плату по містах, все ж залишаються
сільськими робітниками і примушені завжди вертатися
назад у села, щоб знайти собі роботу.

Автори офіціяльних звітів резюмують свої висновки про
матеріяльний стан рільничих поденників так: «Хоч вони живуть
надзвичайно ощадно, проте їхньої заробітної плати ледве вистачає
на те, щоб здобути харчі для себе й своєї родини і заплатити
за своє житло; на одяг вони потребують додаткових доходів...
Атмосфера їхніх мешкань разом з іншими злиднями робить цю
клясу часто здобиччю тифу і сухот».\footnoteA{
Там же, стор. 21, 13.
} Після цього немає чого
дивуватися, що, за одноголосним свідченням авторів звітів,
хмура незадоволеність охоплює ряди цієї кляси, що вона сумує
за минулим, ненавидить теперішнє, зневірюється в будучині,
«піддається згубним впливам демагогів» і охоплена лише однією
idée fixe — еміґрувати до Америки. Така та блаженна країна,
на яку збезлюднення, велика малтузіянська панацея, перетворило
зелений Ерін!

Щоб побачити, як благоденствують ірляндські мануфактурні
робітники, досить одного прикладу:

«Під час моєї останньої інспекторської подорожі на півночі
Ірляндії, — каже англійський фабричний інспектор Роберт Бекер,
— мене вразило, як один вправний ірляндський робітник
силкувався із своїх злиденних коштів дати освіту своїм дітям.
Я точно передаю його оповідання, як я чув його з власних його
уст. Що він справді вправний фабричний робітник, видно з того,
що його вживали до виробу товарів на менчестерський ринок.
Джонсон: З професії я beefier, і працюю від 6 години ранку до
11 години вночі, від понеділка до п’ятниці; суботами ми кінчаємо
о 6 годині вечора і маємо 3 години на обід і відпочинок. В мене
п’ятеро дітей. За цю працю я дістаю 10 шилинґів 6 пенсів на тиждень;
моя дружина теж працює і заробляє на тиждень 5 шилін-

187i Там же, стор. 30.