вартістю в 1 шилінґ, то він продає його на 3 пенси вище від його
індивідуальної вартости й таким чином реалізує понаддодаткову
вартість (Extramehrwert) у 3 пенси. Але, з другого боку, дванадцятигодинний
робочий день виражається тепер для нього у
24 штуках товару замість 12, як це було раніш. Отже, для
того, щоб продати продукт одного робочого дня, він потребує
подвійного збуту або удвоє більшого ринку. За інших незмінних
обставин його товари завойовують більший ринок лише через
зниження їхніх цін. Тому капіталіст продаватиме їх понад їхню
індивідуальну вартість, але нижче від їхньої суспільної вартости,
приміром, по 10 пенсів за штуку. Таким чином із кожної окремої
штуки він усе ще видобуде понаддодаткову вартість в 1 пенс. Це
підвищення додаткової вартости відбувається для нього однаково,
належить чи не належить його товар до кола доконечних
засобів існування, отже однаково, входить чи не входить він як
визначальний момент у загальну вартість робочої сили. Отже,
незалежно від цієї останньої обставини для кожного поодинокого
капіталіста існує мотив здешевлювати товари через підвищення
продуктивної сили праці.

Однак навіть у цьому випадку підвищена продукція додаткової
вартости постає із скорочення доконечного робочого часу та
з відповідного здовження додаткової праці. За  Доконечний робочий
час становив 10 годин, або денна вартість робочої сили становила
5 шилінґів, додаткова праця становила 2 години, а тому
денно продукована додаткова вартість становила 1 шилінґ. Але
наш капіталіст продукує тепер 24 штуки товару, які він продає
по 10 пенсів за штуку, або всі разом за 20 шилінґів. А що вартість
засобів продукції дорівнює 12 шилінґам, то 14 2/5 штуки товару
покривають лише авансований сталий капітал. Дванадцятигодинний
робочий день виражається в 9 3/5 штуках, що ще залишаються.
А що ціна робочої сили дорівнює 5 шилінґам, то доконечний робочий
час виражається в 6 штуках продукту, а додаткова праця
у 3 3/5 штуках. Відношення доконечної праці до додаткової, яке за
пересічних суспільних умов становило 5: 1, тепер становить
уже лише 5: 3. До того самого результату дійдемо й таким способом.
Вартість продукту дванадцятигодинного робочого дня
дорівнює 20 шилінґам. Із них 12 належать до вартости засобів
продукції — вартости, що лише знову з’являється. Отже, лиша-

3а «Зиск, що дістає якась людина, залежить не від того, що вона
панує над продуктом праці інших людей, а від того, що вона панує над
самою працею. Коли вона може продати свої продукти за вищу ціну, тимчасом
як заробітна плата її робітників лишається незмінна, вона, очевидно,
матиме користь.... Тоді меншої частини того, що вона продукує, вистачить
на те, щоб пустити в рух цю працю, отже, більша частина лишається їй
самій». («А man’s profit does not depend upon his command of the produce
of other men’s labour, but upon his command of labour itself. If he can sell
his goods at a higher price, while his workmen's wages remain unaltered,
he is clearly benefited.... A smaller proportion of what he produces is sufficient
to put that labour into motion, and a larger proportion consequently
remains for himself»). («Outlines of Political Economy», London 1832,
p. 49, 50).
