Східній Індії, генерал-губернатор якої 1834—35 рр. констатував:
«Ледве чи знайдеться аналогія до цих злиднів в історії
торговлі. Рівнини Індії біліють від кісток бавовняних ткачів».
Щоправда, оскільки ці ткачі переставились, покинули це тимчасове
шиття, остільки і машини заподіяли їм лише «тимчасових
страждань». Зрештою, це «тимчасове» діяння машин є перманентне,
бо вони постійно захоплюють нові сфери продукції. Отже,
характер усамостійнення та відчужености, що його капіталістичний
спосіб продукції надає взагалі умовам праці та продуктові
праці супроти робітника, розвивається з виникненням
машин до повного антагонізму.199 Саме через це з виникненням
машин уперше вибухають жорстокі повстання робітників проти
засобу праці.

бітну плату, яка зменшилась далеко нижче за мінімум, поповнювано
допомогами парафій. «1827 р. високопреподобний Тернер був парохом у
Wilmslow’i, у Чешірі, в мануфактурній окрузі. Питання комітету в справі
еміграції й відповіді пана Тернера показують, як підтримували конкуренцію
людської праці проти машин. Питання: «Чи не усунено вживанням
механічних варстатів вживання ручних варстатів?» Відповідь: «Безперечно,
воно усунуло б його ще в більшій мірі, ніж це е в дійсності, коли б
ручні ткачі не мали змоги згоджуватися на зниження заробітної плати».
Питання: «Але, згоджуючися на це, чи не наймаються вони за плату,
якої не вистачає їм для їхнього існування, та чи не сподіваються вони
допомоги з парафії, щоб покрити недостачу?» Відповідь: «Так, і конкуренцію
між ручним варстатом і механічним варстатом фактично підтримує
лише податок для бідних». Отже, ганебний павперизм або еміграція — ось
вигоди, що їх мають робітники у наслідок заведення машин. Із почесних
та до певної міри незалежних ремісників їх зводять на становище плазівної
голоти, що живе з принизливого хліба добродійности. Ось що
вони називають тимчасовими труднощами». («The Rev. Mr. Turner was
in 1827 rector of Wilmslow, in Cheshire, a manufacturing district. The questions
of the Committee on Emigration, and Mr. Turner’s answers show
how the competition of human labour is maintained against machinery.
Question: «Has not the use of the power-loom superseded the use of the
hand-loom?» Answer: «Undoubtedly; it would have superseded them much
more than it has done, if the handloom weavers were not enabled to submit
to ä reduction of wages». Question: «But in submitting he has accepted wages
which are insufficient to support him, and looks to parochial contribution
as the remainder of his support?» Answer: «yes, and in fact the competition
between the hand-loom arid the power-loom is maintained out the poorrates»
. Thus degrading pauperism or expatriation, is the benefit which the
industrious receive from the introduction of machinery, to be reduced from
the respectable and in some degree independent mechanic, to the cringing
wretch who lives on the debasing bread of charity. This they call a temporary
inconvenience»). («A Prize Essay on the comparative merits of Competition
and Cooperation», London 1834, p. 29).

199 «Та сама причина, яка може збільшити дохід країни (тобто, як
тут же пояснює Рікардо, доходи лендлордів та капіталістів, багатство
(wealth) яких з економічного погляду взагалі дорівнює багатству нації
(wealth of the nation), може одночасно утворити надмір людности та
погіршити становище робітника» («The same cause which may increase
the revenue of the country may at the same time render the population
redundant and deteriorate the condition of the labourer»). (Ricardo:
«Principles of Political Economy», 3 rd ed. London 1821, p. 469). «Постійна
мета й тенденція кожного вдосконалення механізму фактично є в тому,
щоб цілком збутися праці людини або зменшити її ціну, замінюючи працю
