\parcont{}  %% абзац починається на попередній сторінці
\index{i}{0066}  %% посилання на сторінку оригінального видання
вартости золото стало тому, що всі товари міряли ним свої вартості
й таким чином зробили його уявною протилежністю своєї
споживної форми, формою своєї вартости. Реальними грішми
воно стає тому, що товари через процес свого всебічного відчужування
роблять його своєю дійсно преображеною (entäusserten)
або перетвореною споживною формою, а тому й дійсною формою
своєї вартости. В своїй вартостевій формі товар стирає з себе
всякий слід своєї природної споживної вартости, всякий слід
осібної корисної праці, яка його створила, щоб перетворитись
на одноманітну суспільну матеріялізацію безріжницевої людської
праці. Отже, по грошах не пізнати, якого роду є той товар,
що на них перетворився. У своїй грошовій формі один товар
виглядає цілком так само, як і інший. Тим то гроші можуть бути
і сміттям, хоч сміття не є гроші. Припустімо, що ті два золоті,
за які наш ткач відчужує свій товар, є перетворена форма квартера
пшениці. Продаж полотна, $Т — Г$, є разом з тим купівля
його, $Г — Т$. Але продажем полотна цей процес починає такий
рух, який кінчається своєю протилежністю, купівлею біблії;
купівлею полотна він кінчає той рух, що почався своєю протилежністю,
продажем пшениці. $Т — Г$ (полотно — гроші), ця перша
фаза процесу $Т — Г — Т$ (полотно — гроші — біблія), є разом
з тим $Г — Т$ (гроші — полотно) — остання фаза іншого процесу
$Т — Г — Т$ (пшениця — гроші — полотно). Перша метаморфоза
якогось товару, його перетворення з товарової форми на гроші,
є завжди разом з тим друга протилежна метаморфоза якогось
іншого товару, його зворотне перетворення з грошової форми на
товар\footnote{
Виняток становить, як раніш зауважено, продуцент золота або
срібла, який обмінює свій продукт без попереднього продажу його.
}.

$Г — Т$. Друга, або кінцева метаморфоза товару — купівля.
А що гроші є преображена форма (entäusserte Gestalt) всіх інших
товарів, або продукт загального їхнього відчуження, то й є вони
абсолютно відчужуваний товар. Вони читають усі ціни ззаду
наперед, і таким чином вони відбиваються в усіх товарових тілах
як у матеріялі, покірному для їх власного перетворення на товар.
Ціни, ці закохані очі, якими товари підморгують грошам, показують
разом з тим межі здатности грошей перетворюватися, а
саме їх власну кількість. А що товар, ставши грішми, зникає,
то на грошах непомітно, як саме вони дістаються до рук їх посідача,
або що саме на них перетворено. Вони не пахнуть, non olet,
хоч яке їхнє походження. Якщо вони, з одного боку, репрезентують
проданий товар, то, з другого — товари, що їх можна
купити\footnote{
«Якщо гроші в наших руках репрезентують речі, які ми можемо
забажати купити, то вони також репрезентують і речі, які ми продали за
ці гроші» («Si l’argent représente, dans nos mains, les choses, que nous
pouvons désirer d’acheter, il y représente aussi les choses que nous avons
vendues pour cet argent»). (\emph{Mercier de la Rivière}: «L’Ordre naturel et
essentiel des sociétés politiques», p. 586).
}.
