\index{i}{0335}  %% посилання на сторінку оригінального видання
Машина продукує відносну додаткову вартість не тільки тим,
що вона безпосередньо зневартнює робочу силу та посередньо
здешевлює її, здешевлюючи товари, потрібні для її репродукції,
але ще й тим, що, при першому спорадичному заведенні її, вона
перетворює вживану посідачем машини працю на працю вищого
ступеня і більшої ефективности (potenzierte Arbeit), суспільну
вартість машинового продукту підвищує понад його індивідуальну
вартість, і таким чином дає змогу капіталістові денну вартість
робочої сили покривати меншою частиною вартости денного продукту.
Тому підчас цього переходового періоду, коли машинове
виробництво лишається своєрідною монополією, бариші є надзвичайно
великі, і капіталіст силкується якнайґрунтовніше визискати
цей «перший час молодого кохання» за допомогою якнайбільшого
здовження робочого дня. Великий бариш загострює ненажерливу
жадобу ще більшого баришу.

У міру того, як машина стає загальним явищем у тій самій
галузі продукції, суспільна вартість машинового продукту знижується
до його індивідуальної вартости та потверджується той
закон, що додаткова вартість постає не з тих робочих сил, що їх
капіталіст замістив машиною, а, навпаки, з тих робочих сил,
яких він коло неї вживає. Додаткова вартість походить тільки
із змінної частини капіталу, і ми вже бачили, що масу додаткової
вартости визначають два фактори — норма додаткової вартости
та число одночасно вживаних робітників. За даної довжини робочого
дня норма додаткової вартости визначається тим відношенням,
що в ньому робочий день розпадається на доконечну працю
й додаткову працю. Число одночасно вживаних робітників визначається,
з свого боку, відношенням змінної частини капіталу до
сталої. Тепер ясно, що хоч як машинове виробництв збільшувало б
через піднесення продуктивної сили праці додаткову працю коштом
доконечної праці, воно доходить цього результату лише тим,
що зменшує число робітників, вживаних якимось даним капіталом.
Воно перетворює на машини, отже, на сталий капітал, що не продукує
жодної додаткової вартости, частину капіталу, який раніш
був змінний, тобто перетворювався на живу робочу силу. З двох
робітників, приміром, неможливо витиснути стільки додаткової
вартости, скільки з 24. Якщо кожний із 24 робітників дає за 12 годин
праці лише одну годину додаткової праці, то разом вони дають
24 години податкової праці, тимчасом як уся праця двох робітників
становить лише 24 години. Отже, вживання машин з метою
продукції додаткової вартости містить у собі іманентну суперечність,
бо з двох факторів додаткової вартости, що її дає капітал
даної величини, машини збільшують один фактор, норму додаткової
вартости, лише тим, що зменшують другий фактор —
число робітників. Ця іманентна суперечність виявляється, скоро
тільки з загальним поширенням машин у якійсь галузі промисловости
вартість продукованого машиновим способом товару стає
реґулятивною суспільною вартістю всіх товарів того самого
роду, і це є та суперечність, яка, не доходячи до свідомости капіталу\footnote{
Чому ця іманентна суперечність не доходить до свідомости поодинокого
капіталіста, а тому і до свідомости політичної економії, яка поділяє
його погляд, це ми побачимо з перших відділів третьої книги.
},
\index{i}{0336}  %% посилання на сторінку оригінального видання
знову таки спонукає його до найбільш насильного здовжування
робочого дня, щоб зменшення відносного числа експлуатованих
робітників компенсувати збільшенням не тільки
відносної, але й абсолютної додаткової праці.

Отже, коли капіталістичне вживання машин, з одного боку,
утворює нові могутні мотиви до безмірного здовжування робочого
дня та робить переворот у самому способі праці й характері суспільного
робочого організму в такий спосіб, що ламає опір супроти
цієї тенденції, то, з другого боку, воно, почасти через підбивання
під владу капіталу неприступних раніш верств робітничої кляси,
почасти через звільнення витиснених машиною робітників, продукує
надмірну робітничу людність\footnote{
Одна з великих заслуг Рікарда\footnote*{
У французькому виданні тут сказано: «Одна з заслуг Сісмонді
та Рікарда в тому, що вони зрозуміли і~\abbr{т. д.}». \emph{Ред.}
} в тому, що він зрозумів, що
машина є засіб продукції не тільки товарів, але й «redundant
population»\footnote*{
— надмірної людности. \emph{Ред.}
}.
}, яка примушена коритися
законові, що його диктує їй капітал. Звідси те варте уваги явище в
історії сучасної промисловости, що машина нищить усі моральні та
природні межі робочого дня. Звідси той економічний парадокс, що
наймогутніший засіб до скорочення робочого часу перетворюється
в найпевніший засіб до того, щоб увесь час життя робітника та
його родини перетворити на робочий час, що ним порядкує капітал
для збільшення своєї вартости. «Коли б, — мріяв Арістотель,
цей найбільший мислитель старовини, — коли б кожне знаряддя
могло з наказу або передчування виконувати свою власну працю
так, як майстерні витвори Дедала рухалися сами собою, або як
триніжки Гефеста з своєї власної охоти заходжувалися коло святої
праці, коли б так сами собою ткали ткацькі човники, то не потрібні
були б ані майстрові помічники, ані панові раби»\footnote{
\emph{F. Biese}: «Die Philosophie des Aristoteles», Berlin 1842, Bd. II,
S. 408.
}. І Антіпарос,
грецький поет за часів Ціцерона, вітав винахід водяного
млина, щоб молоти мливо, цю елементарну форму кожної
продуктивної машини, як визвольника рабинь і відновника золотого
віку!\footnote{
Подаємо тут Штольберґів переклад цього вірша, бо вірш цей,
цілком так само, як і наведені вище цитати про поділ праці, характеризує
протилежність між античними та сучасними поглядами.

\settowidth{\versewidth}{Lasst uns leben das Leben das Väter, und lasst uns der Gaben}
\begin{verse}[\versewidth]
«Schonet der mahlenden Hand, о Müllerinnen, und schlafet\\
Sanft! Es verkünde der Hahn euch den Morgen umsonst!\\
Däo hat die Arbeit der Mädchen den Nymphen befohlen,\\
Und itzt hüpfen sie leicht über die Räder dahin,\\
Dass die erschütterten Achsen mit ihren Speichen sich wälzen,\\
Und im Kreise die Last drehen des wälzenden Steins.\\
Lasst uns leben das Leben das Väter, und lasst uns der Gaben\\
Arbeitslos uns freun, welche die Göttin uns schenkt».

\vinphantom{test}(«Gedichte aus dem Griechischen übersetzt von\\
\vinphantom{test}Christian Graf zu Stolberg. Hamburg. 1782»).

(«Руки свої бережіть, о млинарки, і спіте спокійно, —\\
Півні нехай сповіщають про ранок — для вас то байдуже.\\
Део роботу дівочу на німф відтепер всю поклала,\\
Німфи віднині на колесах легко і жваво танцюють,\\
І обертаються осі і крутяться спині із ними,\\
І перемелюють зерно важкеє на кам'яних жорнах.\\
Отже, живімо як предки жили, без утомної праці\\
І заживаймо дарів, подарованих нам від богині»).\\
\end{verse}

}

«Поганці, ах, ті поганці!» Вони, як це виявив мудрий Бастія,
\index{i}{0337}  %% посилання на сторінку оригінального видання
а перед ним ще мудріший Мак-Куллох, нічого не розуміли з
політичної економії та християнства. Вони, між іншим, не розуміли,
що машина — найвипробуваніший засіб до здовження робочого
дня. Вони виправдували рабство однієї людини лише як
засіб до повного людського розвитку іншої. Але, щоб проповідувати
рабство мас із метою перетворити небагато грубих або
напівосвічених вискочнів у «eminent spinners», «extensive sausage
makers» та «influential shoe black dealers»\footnote*{
— «видатних прядунів», «великих ковбасників» і «впливових
продавців вакси». \emph{Ред.}
}, для цього їм
бракувало специфічно-християнського почуття.

\subsubsection[Інтенсифікація праці]{Інтенсифікація праці \footnotemarkZ{}}
\footnotetextZ{
У французькому виданні Маркс додає до цього таку примітку;
«Словом інтенсифікація праці ми позначаємо методи, що роблять працю
напруженішою» («Par le mot intensification nous designons les procedés
qui rend le travail plus intense»). \emph{Ред.}
}

Безмірне здовження робочого дня, що його продукують машини
в руках капіталу, приводить пізніше, як ми вже бачили, до
реакції з боку суспільства, життю якого воно загрожувало в
самому корені, а тим самим і до законодатно обмеженого нормального
робочого дня. На основі останнього набирає вирішальної
ваги явище, з яким ми вже раніш зустрічались, а саме інтенсифікація
праці. При аналізі абсолютної додаткової вартости йшлося
насамперед про екстенсивну величину праці, а ступінь її інтенсивности
припускалось за даний. Тепер ми маємо розглянути
перетворення екстенсивної величини на інтенсивну величину,
тобто на величину, вимірювану щодо ступеня.

Само собою зрозуміло, що разом із розвитком машин та з нагромадженням
досвіду спеціяльною клясою машинових робітників
природно зростає швидкість, а тим самим і інтенсивність
праці. Так, в Англії протягом цілого півстоліття здовження
робочого дня йде поруч зростання інтенсивности фабричної праці.
Однак зрозуміло, що при такій праці, де йдеться не про минущі
пароксизми, а про реґулярну одноманітність, що з дня на день
повторюється, мусить настати пункт, коли здовження робочого
\parbreak{}  %% абзац продовжується на наступній сторінці
