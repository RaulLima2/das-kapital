лише показником (Index) дійсної сили, він, між іншим, каже:
«Немає ніякого сумніву, що парові машини тієї самої ваги, часто
ті самі ідентичні машини, де пороблено лише сучасні поліпшення,
виконують пересічно на 50% більше праці, ніж раніш, та що в
багатьох випадках ті самі ідентичні парові машини, які за часів
обмеженої швидкости в 220 футів на хвилину давали 50 кінських
сил, тепер, при зменшеному споживанні вугілля, дають понад 100
кінських сил... Сучасна парова машина з тією самою кількістю
номінальних кінських сил, у наслідок поліпшення в її конструкції,
зменшення розміру та поліпшення конструкції парового
казана тощо, діє з більшою силою, ніж раніш... Тому, хоч супроти
номінальної кінської сили вживається те саме число рук, що й
раніш, однак супроти робочих машин вживається менше число
рук».170 1850р. фабрики Об’єднаного Королівства вживали 134.217
номінальних кінських сил, щоб рухати 25.638.716 веретен та
301.495 ткацьких варстатів. 1856 р. число веретен і ткацьких
варстатів становило відповідно 33.503.580 і 369.205. Коли б
потрібна кінська сила лишилася та сама як 1850 р., то 1856 р.
потрібно було б 175.000 кінських сил. Але за офіціяльними документами
число їх становило лише 161.435, отже, понад 10.000
кінських сил менше, ніж їх потрібно було б на основі розрахунку
з 1850 р. 171 «Останній «Return» з 1850 р. (офіціяльна статистика)
установлює той факт, що фабрична система поширюється з чимраз
більшою швидкістю, число рук у відношенні до машин зменшилося,
парова машина в наслідок економії на силі та інших
метод рухає машини більшої ваги, і що більшої кількости продукту
досягається в наслідок поліпшення робочих машин, змінених
метод фабрикації, збільшеної швидкости машин та багатьох інших
причин».112 «Великі поліпшення, які пороблено в машинах
усякого роду, дуже підвищили їхню продуктивну силу. Немає
ніякого сумніву, що скорочення робочого дня дало... стимул
до цих поліпшень. Ці поліпшення й інтенсивніше напруження
робітника призвели до того, що протягом скороченого (на 2 години,
або на одну шосту) робочого дня продукується, щонайменше,
стільки ж продукту, як раніш протягом довшого дня».178

Як зросло збагачення фабрикантів у наслідок інтенсивнішого
визиску робочої сили, доводить уже одна та обставина, що пересічне
пропорційне зростання англійських бавовняних і т. ін.
фабрик становило 1838—1850 рр. 32%, а 1850—1856 рр. — 86%.

Хоч який великий був проґрес англійської промисловости
за вісім років — від 1848 до 1856 — за панування десятигодин-

товими та конопляними фабриками, з одного боку, і лляними — з другого;
нарешті, вперше заведено до звіту панчішне виробництво.

170 «Reports of Insp. of Fact, for 31 st October 1856», p. 11.

171 Там же, стор. 14, 15.

172 Там же, стор. 20.

173 «Reports etc. for 31 st October 1858», p. 9, 10. Порівн. «Reports
etc. for 30 th April 1860», p. 30 і далі.
