\parcont{}  %% абзац починається на попередній сторінці
\index{i}{0252}  %% посилання на сторінку оригінального видання
Він мусить зробити переворот у технічних і суспільних умовах
процесу праці, отже, в самому способі продукції, щоб підвищити
продуктивну силу праці, через підвищення продуктивної сили
праці знизити вартість робочої сили й таким чином скоротити
частину робочого дня, доконечну для репродукції цієї вартости.

Додаткову вартість, продуковану через здовження робочого
дня, я називаю абсолютною додатковою вартістю; навпаки, ту
додаткову вартість, що виникає із скорочення доконечного робочого
часу й відповідної зміни відносних величин обох складових
частин робочого дня — відносною додатковою вартістю.

Для того, щоб знизити вартість робочої сили, підвищення
продуктивної сили праці мусить охопити ті галузі промисловости,
продукти яких визначають вартість робочої сили, отже, або
належать до кола звичайних засобів існування, або можуть їх
заміняти. Але вартість товару визначається не тільки кількістю
праці, яка надає йому викінченої форми, а так само й кількістю
праці, що міститься в засобах його продукції. Приміром, вартість
чобота визначається не тільки працею шевця, а ще й вартістю
шкури, смоли, дратви й т. ін. Отже, підвищення продуктивної
сили праці й відповідне подешевшання товарів у тих галузях
промисловости, які постачають речові елементи сталого капіталу, —
засоби праці й матеріял праці — для виготовлення доконечних
засобів існування, також знижує вартість робочої сили. Навпаки,
в тих галузях продукції, що не постачають ні доконечних засобів
існування, ані засобів продукції до їх виготовлення, підвищення
продуктивної сили праці лишає вартість робочої сили незмінною.

Подешевшання товару знижує, природно, вартість робочої
сили лише pro tanto, тобто лише в тій пропорції, в якій цей товар
увіходить у репродукцію робочої сили. Сорочки, приміром, є
доконечний засіб існування, але тільки один із багатьох. Їхнє
подешевшання зменшує видатки робітника лише на сорочки.
Однак загальна сума доконечних засобів існування складається
лише з різних товарів, із самих продуктів окремих галузей
промисловости, а вартість кожного такого товару становить завжди
якусь відповідну частину вартости робочої сили. Ця вартість
зменшується разом із доконечним для її репродукції робочим
часом, що його загальне скорочення дорівнює сумі його скорочень
по всіх тих окремих галузях продукції. Цей загальний результат
ми розглядаємо тут так, наче б він був безпосереднім результатом
і безпосередньою метою в кожному поодинокому випадку. Коли
поодинокий капіталіст через підвищення продуктивної сили праці
здешевлює, приміром, сорочки, то він цим ніяк не має неодмінно за
мету знизити pro tanto вартість робочої сили, а тому й доконечний
робочий час, а лише оскільки він, кінець-кінцем, допомагає
цьому результатові, він допомагає піднесенню загальної норми
додаткової вартости.\footnote{
«Коли фабрикант, поліпшуючи машини, подвоює кількість своїх
продуктів\dots{} він виграє (кінець-кінцем) лише остільки,
оскільки він у наслідок цього має змогу дешевше зодягати
робітника\dots{} і оскільки таким
чином на робітника припадає менша частина цілого продукту». («Ram-Say:
«An Essay on the Distribution of Wealth», Edinburgh 1836, p. 168,169)
}
Загальні й доконечні тенденції капіталу
треба відрізняти від форм їхнього виявлення.

\index{i}{0253}  %% посилання на сторінку оригінального видання
Тут не час розглядати, яким способом іманентні закони капіталістичної
продукції виявляються у зовнішньому русі капіталів,
набирають сили як примусові закони конкуренції, і тому доходять
до свідомости поодинокого капіталіста як рушійні мотиви
його діяльности; але в усякому разі само собою зрозуміло: наукова
аналіза конкуренції можлива лише після того, як уже пізнано
внутрішню природу капіталу, цілком так само, як видимий рух
небесних тіл зрозумілий лише тому, хто знає їхній дійсний, але
почуттєво несприйманий рух. Однак, щоб зрозуміти продукцію
відносної додаткової вартости лише на основі здобутих уже результатів,
треба зауважити ось що.

Якщо одна робоча година виражається в кількості золота
в 6 пенсів, або \sfrac{1}{2} шилінґа, то протягом дванадцятигодинного
робочого дня продукується вартість у 6 шилінґів. Припустімо, що
за даної продуктивности праці виготовляється 12 штук товару
протягом цих дванадцятьох робочих годин. Вартість засобів продукції,
сировинного матеріялу тощо, зужиткованих на кожну
штуку, хай буде 6 пенсів. За цих обставин кожна штука товару
коштує 1 шилінґ, або 12 пенсів, а саме 6 пенсів є вартість засобів
продукції та 6 пенсів — нова вартість, додана до них працею
підчас їхнього перероблення. Припустімо тепер, що якомусь
капіталістові пощастить подвоїти продуктивну силу праці і тому
замість 12 продукувати 24 штуки цього товару протягом дванадцятигодинного
робочого дня. За незмінної вартости засобів
продукції вартість окремої штуки товару спадає тепер до 9 пенсів,
 а саме, 6 пенсів за вартість засобів продукції, а 3 пенси становлять
нову вартість, додану останньою працею. Хоч продуктивна
сила праці й подвоїлася, робочий день, як і раніш, створює
лише нову вартість у 6 шилінґів, яка, однак, розподіляється
тепер на подвійну кількість продуктів. Отже, на кожний окремий
продукт припадає замість \sfrac{1}{12} лише \sfrac{1}{24} цієї загальної вартости,
замість 6 пенсів 3 пенси, або, що сходить на те саме, до засобів
продукції підчас їхнього перетворення на продукт додається на
кожну штуку товару тепер лише півгодини праці замість цілої,
як це було раніш. Індивідуальна вартість цього товару стоїть
тепер нижче від його суспільної вартости, тобто він коштує менше
робочого часу, аніж велика маса тих самих товарів, які випродуковано
за пересічних суспільних умов. Одна штука цього товару
коштує пересічно 1 шилінґ, або репрезентує 2 години суспільної
праці; із зміною способу продукції вона коштує лише 9 пенсів,
або містить у собі лише 1\sfrac{1}{2} години праці. Але дійсна вартість
якогось товару є не його індивідуальна, а його суспільна вартість,
тобто її вимірюється не тим робочим часом, що його фактично
коштує товар продуцентові в окремому випадку, а робочим часом,
суспільно потрібним на його продукцію. Отже, коли капіталіст,
що вживає нової методи, продає свій товар за його суспільною

\parbreak{}  %% абзац продовжується на наступній сторінці
