заміщувати працю та робітників перетворився відразу на засіб
збільшувати число найманих робітників, підбиваючи під безпосереднє
панування капіталу всіх членів родини без ріжниці статі
й віку. Примусова праця на капіталіста узурпувала не тільки час
дитячих забав, а ще й час вільної праці в колі домашніх, у прийнятих
звичаєм межах, для потреб самої родини.\footnote{
Підчас бавовняної кризи, що супроводила американську громадянську
війну, англійський уряд послав д-ра Едварда Сміса до Ланкашіру,
Чешіру й т. д., щоб дати звіта про стан здоров’я бавовняних робітників.
Е. Сміс, між іншим, повідомляє: «З погляду гігієни криза, крім
того, що вона витиснула робітників із фабричної атмосфери, дала чимало
й інших користей. Жінки робітників мають тепер потрібний вільний
час, щоб нагодувати груддю своїх дітей замість отруювати їх Cordial’ем
Ґодфрея (препаратом з опію). Вони тепер мають час учитися варити страви».
На нещастя, припало це куховарство на той час. коли вони не мали
чого їсти. Але ми бачимо, як капітал для свого самозростання узурпував
працю родини, потрібну для самого споживання родини. Так само кризу
використано на те, щоб по окремих школах учити дочок робітників шити.
Отже, треба було американської революції та світової кризи, щоб дівчата-робітниці,
які прядуть для цілого світу, навчилися шити.
}

Вартість робочої сили було визначено не тільки робочим
часом, потрібним, щоб утримати поодинокого дорослого робітника,
а ще й часом, потрібним, щоб утримати робітничу родину.
Викидаючи всіх членів робітничої родини на ринок праці, машини
розподіляють вартість робочої сили чоловіка на всю його родину.
Тому вони знижують вартість його робочої сили. Може бути купівля
родини, розпарцельованої на чотири робочі сили, коштує й
більше, ніж раніш коштувала купівля робочої сили голови родини,
але зате тепер чотири робочі дні стають на місце одного дня, і їхня
ціна падає пропорційно надлишкові додаткової праці чотирьох
над додатковою працею одного. Тепер для існування однієї родини
четверо мусять постачати капіталові не тільки працю, а ще й
додаткову працю. Таким чином машина з самого початку, разом
із збільшенням людського матеріялу експлуатації, цього справжнього
поля капіталістичного визиску,\footnote{
«Збільшення числа робітників було велике в наслідок дедалі
більшої заміни праці чоловіків працею жінок, а особливо праці дорослих
працею дітей. Троє дівчаток у віці 13 років із заробітною платою від 6
до 8 шилінґів на тиждень замінили дорослого чоловіка, що його плата
коливається між 18 і 45 шилінґами». («The numerical increase of labourers
has been great, through the growing substitution of female for male, and
above all of childish for adult, labour. Three girls of 13, at wages from of
6 sh. to 8 sh. a week, have replaced the one man of mature age, of wages
varying from 18 sh. to 45 sh.»). (Th. de Quincey: «The Logic of Political
Economy», London 1844, p. 147 n.). Через те, що певних функцій родини,
як от, приміром, догляд та годування груддю дітей і т. д., не можна зовсім
усунути, то матері родин, конфісковані капіталом, мусять сяк чи так
} збільшує одночасно і
ступінь експлуатації.

Машини також ґрунтовно революціонізують формальний вираз
капіталістичного відношення, контракт між робітником і капіталістом.
На основі товарового обміну першою передумовою було
те, що капіталіст і робітник протистояли один одному як вільні