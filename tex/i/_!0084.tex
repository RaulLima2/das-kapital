\parcont{}  %% абзац починається на попередній сторінці
\index{i}{*0084}  %% посилання на сторінку оригінального видання
російський учений і критик Н. Чернишевський у своїх «Нарисах
політичної економії за Міллом».

Отже, у Німеччині капіталістичний спосіб продукції дійшов
своєї стиглости вже після того, як його антагоністичний характер
шумно виявився у Франції й Англії в історичних боях, тимчасом
як німецький пролетаріят мав уже куди рішучішу теоретичну
клясову свідомість, аніж німецька буржуазія. Тому, скоро тільки
тут буржуазна наука політичної економії стала, здавалося, можливою,
як уже знову зробилась вона неможливою.

Серед цих обставин її проводирі поділились на два табори.
Одні, розсудливі, ласі до наживи, люди практики, згуртувались
під прапором Бастія, найповерховішого, а тому й найудатнішого
представника вульґарно-економічної апологетики; інші, горді
професорською гідністю своєї науки, пішли за Дж. Ст. Міллом у
його спробі примирити непримиренне. Як за клясичного періоду
буржуазної економії, так само і за часів її занепаду німці лишилися
тільки учнями, прихильниками й наслідувачами, дрібними
крамарями-розносниками продукції великих закордонних фірм.

Отже, своєрідний історичний розвиток німецького суспільства
виключав тут можливість усякого ориґінального дальшого розвитку
«буржуазної» політичної економії, але не виключав її
критики. Оскільки така критика репрезентує взагалі якусь клясу,
вона може репрезентувати лише ту клясу, історичне призначення
якої — зробити переворот у капіталістичному способі продукції
й остаточно знищити кляси, тобто вона може репрезентувати лише
пролетаріят.

Вчені й невчені проводирі німецької буржуазії намагалися
спочатку замовчати «Капітал», як це їм пощастило з моїми давнішими
працями. Але скоро ця тактика перестала відповідати
обставинам часу, вони під тим приводом, що критикують мою
книгу, подали низку вказівок, «щоб заспокоїти буржуазну свідомість»,
але натрапили в робітничій пресі — див., приміром,
статті Йозефа Діцґена в «Volksstaat’i» — на сильніших борців,
яким і досі не спромоглися відповісти.\footnote{
Беззубі базікала німецької вульґарної політичної економії лають
стиль і спосіб викладу моєї праці. Ніхто гостріше не може оцінити літературні
вади «Капіталу», ніж я сам. Та все ж на користь і задоволення
цих панів та їхньої публіки я процитую тут думку англійської та російської
критики. Цілком ворожа моїм поглядам «Saturday Review» у своїй
нотатці з приводу першого німецького видання каже: виклад «дає й
найсухішим економічним питанням своєрідну принаду (charm)». «С.-П.
Ведомости» зазначають у числі з 20 квітня 1872~\abbr{р.}: «виклад, за винятком
небагатьох занадто спеціяльних частин, відзначається приступною зрозумілістю,
ясністю і, не зважаючи на наукову височінь предмету, надзвичайною
жвавістю. З цього боку автор\dots{} зовсім не подібний до більшости
німецьких учених\dots{} які пишуть свої книги такою темною й сухою мовою,
що у звичайного смертного від того голова тріщить». Та в читачів сучасної
німецької національ-ліберальної професорської літератури тріщить
щось зовсім інше, а не голова.
}

Чудовий російський переклад «Капіталу» з’явився весною
1872~\abbr{р.} в Петербурзі. Видання в \num{3.000} примірників тепер уже
\parbreak{}  %% абзац продовжується на наступній сторінці
