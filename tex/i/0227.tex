Звичайно, всі ці викрути нічого не помогли. Фабричні інспектори
вдалися до суду. Але незабаром на міністра внутрішніх
справ сера Джорджа Грея спала така хмара петицій від фабрикантів,
що в обіжнику з 5 серпня 1848 р. він наказав інспекторам
«не позивати взагалі за порушення букви закону, поки не буде
доведене зловживання Relaissystem’ою з метою примусити підлітків
і жінок працювати понад десять годин». Після цього фабричний
інспектор Ф. Стюарт дозволив так звану систему змін протягом
п’ятнадцятигодинного періоду фабричного дня для цілої Шотландії,
де вона незабаром знов розцвіла, як колись. Навпаки,
англійські фабричні інспектори заявили, що міністер не має
жодної диктаторської влади припинити чинність закону, і далі
провадили судові переслідування проти Proslavery rebels.

Алеж нащо було притягати до суду, коли суди, county magistrates,\footnote{
Ці «county magistrates», «great unpaid»,\footnote*{
— величні неоплачувані. Ред.
} як їх називає В. Кобе
— це щось наче безплатні мирові судді, що їх обирають із почесних
осіб графства. В дійсності вони являють собою патримоніяльні суди панівних
кляс.
}
виправдували притягуваних до права? По цих судах
засідали пани фабриканти, щоб самих себе судити. Ось приклад.
Якийсь Іскрідж із бавовнопрядної фірми Кершоу, Лізе і К°
подав був фабричному інспекторові своєї округи схему Relaissystem,
призначену для його фабрики. Одержавши відмову,
він спочатку тримався пасивно. Декілька місяців пізніш якийсь
індивід, на ім’я Робінзон, теж бавовняник, і коли не П’ятниця,
то в усякому разі родич Іскріджа, став перед Borough Justices\footnote*{
— мировими суддями. Ред.
}
у Стокпорті, обвинувачуваний у тому, що завів у себе таку
систему змін, яку вигадав Іскрідж. Засідало четверо суддів,
серед них три бавовняні фабриканти, з тим самим неминучим
Іскріджем на чолі. Іскрідж виправдав Робінзона й заявив: що є
законоправне для Робінзона, те справедливе й для Іскріджа.
Покликаючись на свій власний судовий присуд, що набрав правної
сили, він зараз же завів цю систему й на своїй власній фабриці.\footnote{
«Reports etc. for 30 th April 1849», p. 21, 22. Порівн. подібні
приклади там же, crop. 4, 5.
}
Певна річ, уже самий склад таких суддів був явним порушенням
закону.\footnote{
Законом 1 і 2 Вільяма IV, с. 24, s. 10, відомим під назвою фабричного
закону сера Джона Гобговза, забороняється кожному посідачеві
бавовнопрядної або ткацької фабрики, а також і батькові, синові або
братові такого посідача виконувати обов'язки мирового судді в питаннях,
які стосуються до фабричного закону.
} «Такі судові фарси, — каже інспектор Хоуелл, —
аж волають по ліки... або пристосуйте закон до таких присудів,
або віддайте вирішення справ не такому вже порочному трибуналові,
який свої присуди пристосував би до закону... в усіх таких
випадках. Дуже бажано, щоб посада судді була платна!»\footnote{
«Reports etc. for 30 th April 1849».
}

Коронні юристи проголосили фабрикантську інтерпретацію
закону 1848 р. за недоладну, але рятівники суспільства не дали