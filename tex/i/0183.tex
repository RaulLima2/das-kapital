дини вечора, але з них закон приділяє півгодини на сніданок і
одну годину на обід, отже, лишається 10\% робочих годин, і для
суботи 8 годин, від шостої години ранку до другої години по
півдні, з чого півгодини відпадає на сніданок. Лишається 60 робочих
годин, по 10\sfrac{1}{2} для перших п’яти днів тижня, 7\sfrac{1}{2} для
останнього дня тижня.47 Для слідкування за виконанням цього
закону призначено окремих наглядачів, фабричних інспекторів
безпосередньо підлеглих міністерству внутрішніх справ, фабричних
інспекторів, що їхні звіти опубліковує щопівроку парлямент.
Отже, вони дають постійну й офіційну статистику про капіталістичну
ненажерливу жадобу до додаткової праці.

Послухаймо хвилинку фабричних інспекторів.48
«Фабрикант-ошуканець починає працю чверть години, іноді
раніше, іноді пізніше, перед шостою годиною ранку і кінчає її
чверть години, іноді раніше, іноді пізніше, після шостої годнии
вечора. Він одбирає по 5 хвилин від початку й кінця тієї півгодини,
що номінально призначена на сніданок, і по 10 хвилин від
початку й кінця тієї години, що призначена на обід. В суботу в
нього працюють чверть години, іноді більше, іноді менше, після
другої години по півдні. Таким чином його виграш становить:

Перед шостою годиною вранці. . 15 хвилин
Після шостої години    вечора... 15»
На час сніданку............................10»
На час обіду..................................20»
                                                                   60    хвилин

Разом за 5 днів: 300 хвилин

Суботами:

Перед шостою годиною вранці. . 15 хвилин
На час сніданку........................... 10»
Після другої години по півдні. .   15»

Цілий тижневий
виграш:
340 хвилин

47    Історію фабричного закону з 1850 р. викладено на протязі цього
розділу.

48    Періоду від початку великої промисловости в Англії й до 1845 р.
я торкаюсь лише в деяких місцях і відсилаю читача до твору «Die Lage
der arbeitenden Klasse in England», Von Friedrich Engels. Leipzig 1845
(Ф. Енґельс: «Становище робітничої кляси в Англії»). Як глибоко збагнув
Енгельс дух капіталістичного способу продукції, свідчать Factory Reports,
Reports on Mines* і т. ін., що появилися після 1845 p.; a як напрочуд
гарно він змалював у подробицях становище робітничої кляси,
показує якнайповерховіше порівняння його твору з офіційними звітами
«Children’s Employment Commission» ** (за 1863—67 рр.), що появилися
18—20 років пізніше. У них йде мова саме про ті галузі промисловости,
де фабричного законодавства ще не було заведено до 1862 р., а почасти не
заведено ще й досі. Отже, становище, змальоване тут Енгельсом, не зазнало
більш-менш значних змін під впливом зовнішніх сил. Свої приклади
я беру переважно з періоду вільної торговлі після 1848 р., того райського
часу, про який так казково багато понаговорили німцям стільки ж пащекуваті,
як і з наукового боку вбогі комівояжери вільної торговлі. Зрештою, Англія фігурує тут на першому
місці лише тому, що вона є клясична
представниця капіталістичної продукції і що лише вона одна має
постійну офіціяльну статистику про ті явища, про які тут мовиться.

* — звіти фабричних інспекторів, звіти гірничих інспекторів. Ред.

** — комісії для вивчення праці дітей. Ред.
