пружин, виготівник циферблятів, виготівник спіральних пружин,
робітник, що робить дірки для каменя та вставляє рубін, виготівник
стрілок, виготівник коробки для годинника, виготівник
шрубів, позолотник з багатьма підрозділами, як от, приміром,
колісник (виріб мідяних та сталевих коліщат знову поділено),
Triebmacher, Zeigenverkmacher, acheveur de pignon (закріплює
коліщата в належних місцях, полірує facettes і т. ін.), Zapfenmacher,
planteur de finissage (вставляє в механізм різні коліщата
та пружини), finisseur de barillet (вирізує зубці, поширює дірочки
до належного розміру та закріплює установку), Hemmungmacher
і, як підрозділ цієї галузі, виготівник циліндрів, виготівник
трибків, виготівник маятників, Raquettemacher (тобто виготівник
механізму, що реґулює годинник), planteur d’échappement
(Hemmungmacher у власному значенні); далі: repasseur de barillet
(виготовлює коробку для пружини та закріпляє її установку),
ґлянсувальник сталі, ґлянсувальник коліщат, ґлянсувальник
шруб, маляр цифр, Blattmacher (покриває мідь емалем),
fabricant de pendants (виготовлює лише кільця до годинникової
коробки), finisseur de charnière (вставляє мосяжевий штифт
всередину коробки й т. ін.), faiseur de secret (виготовлює пружину,
що відкриває кришку годинника), ґравер, ciseleur, полірувальник
годинникової коробки і т. д. і т. д., нарешті, repasseur, що
складає окремі частини годинника докупи та пускає годинника
в рух. Лише небагато частин годинника переходить через різні
руки, і всі ці membra disjecta збираються лише в руках того,
хто, кінець-кінцем, сполучає їх в один цілий механізм. Це
зовнішнє відношення готового продукту до його різнорідних
елементів тут, як і в подібних роботах, лишає комбінацію частинних
робітників у тій самій майстерні випадковою. Самі
частинні праці знов таки можуть провадитись як незалежні
одне від одного ремества, як от у кантоні Ваадт та Невшатель,
тимчасом як у Женеві, приміром, існують великі мануфактури
годинників, тобто існує безпосередня кооперація частинних робітників
під командою одного капіталу. І в останньому випадку
цифербляти, пружини й коробки рідко виготовлюють у самій
мануфактурі. Комбіноване мануфактурне виробництво зисковне
тут лише за виняткових умов, бо конкуренція поміж робітниками,
що хочуть працювати вдома, надзвичайно велика, роздрібнення
продукції на масу гетерогенних процесів мало дає змоги застосовувати
спільні засоби праці; крім того, при роздрібненій фабрикації
капіталіст заощаджує собі видатки на робітні приміщення
й т. д.32 Однак становище й цих частинних робітників, які пра-

32 Женева в 1854 р. випродукувала 80.000 годинників, що не складає
навіть і п’ятої частини продукції годинників кантону Невшатель.
Chaux-de-Fonds, що його можна розглядати як єдину мануфактуру годинників,
сам щорічно дає удвоє більше, ніж Женева. Від 1850 і до 1861 р.
Женева постачила 750.000 годинників. Див. «Report from Geneva on
the watch Trade» в «Reports by H. M. ’s Secretaries of Embassy and Legation
on the Manufactures, Commerce etc.». № 6 1863. Якщо відсутність
зв’язку між процесами, на які розпадається продукція складних про-
