продукція разом з тим примушує відновити його систематично
як закон, що реґулює суспільну продукцію, і у формі, адекватній
повному розвиткові людини. У рільництві, як і в мануфактурі,
капіталістичне перетворення продукційного процесу є
разом з тим мартиролог продуцентів, засіб праці є разом з тим
засіб поневолення, засіб експлуатації й засіб павперизації робітника,
суспільна комбінація процесу праці є разом з тим організоване
пригнічення індивідуальної життьової сили робітника,
його волі й самостійности. Розпорошеність сільських робітників
по великих просторах ламає одночасно силу їхнього
опору, тимчасом як концентрація міських робітників підносить
її. У сучасному рільництві, так само і в міській промисловості
підвищення продуктивної сили й більша ефективність праці
купується ціною нищення й виснаження самої робочої сили.
І кожний проґрес капіталістичного рільництва — це не тільки
проґрес у вмілості грабувати робітника, але разом з тим і у вмілості
грабувати ґрунт, кожний проґрес у піднесенні родючости
його на даний час — це разом з тим проґрес у руйнуванні тривалих
джерел цієї родючости. Що більш якась країна, як, приміром,
Сполучені штати Північної Америки, виходить від великої
промисловости як бази свого розвитку, то швидший цей
процес руйнування.\footnote{
Порівн. Liebig: «Die Chemie in ihrer Anwendung auf Agrikultur
und Physiologie». 7. Auflage 1862, особливо також «Einleitung
in die Naturgesetze des Feldbaues» у першому томі. Вияснення неґативного
боку сучасного рільництва з погляду природознавства — це одна
з невмирущих заслуг Лібіґа. Його історичні нариси з історії рільництва,
хоч вони й не без грубих помилок, також висвітлюють деякі питання.
Можна пожалкувати, що він навмання зважується висловлювати
ось які погляди: «Продовжуване далі роздрібнювання й частіше переорювання
підвищує обмін повітря всередині поруватих частин землі, збільшує
й поновлює поверхню цих частин землі, на яку має впливати повітря;
але легко зрозуміти, що додатковий здобуток із поля не може бути,
пропорційний до витраченої на поле праці, а зростає в куди меншій пропорції».
«Цей закон, — додає Лібіґ, — уперше висловив Дж. Ст. Мілл у своїм
«Principles of Political Economy», v. I, p. 17, ось так: «Те, що продукт
землі за інших рівних умов зростає в дедалі меншій пропорції порівняно
до зростання числа вживаних робітників (навіть відомий закон Рікарда
пан Мілл повторює тут у фалшивому формулюванні, бо через те, що «зменшення
числа вживаних робітників» («the decrease of the labourers employed»)
в Англії постійно відбувалося поруч із проґресом рільництва, закона,
вигаданого для Англії і в Англії, не можна було б застосувати, принаймні
} Тому капіталістична продукція розвиває
техніку й комбінування суспільного процесу продукції, але
лише так, що вона разом з цим підриває джерела виникнення
всякого багатства: землю й робітника.

го потворного й неприродного поділу». («You divide the people into two
hostile camps of clownish boors and emasculated dwarfs. Good heavens!
a nation divided into agricultural and commercial interests calling itself
sane, nay styling itself enlightened and civilized, not only in spite of,
but in consequence of this monstrous and unnatural division»). (David
Urquhart: «Familiar Words», London 1855, p. 119). Це місце показує
одночасно силу і слабість такого роду критики, яка вміє обмірковувати
та ганьбити сучасність, але не вміє її зрозуміти.