\parcont{}  %% абзац починається на попередній сторінці
\index{i}{0356}  %% посилання на сторінку оригінального видання
зрозуміла. Якщо сказати, приміром, що в Англії потрібно було б
100 мільйонів чоловіка на те, щоб колишнім самопрядом випрясти
ту бавовну, яку тепер випрядає машиною 500.000 чоловіка, так це,
природно, ще не означає, що машина заступила ті мільйони, які
ніколи не існували. Це означає тільки, що багато мільйонів
людей потрібно було б на те, щоб замінити прядільну машину.
Навпаки, коли сказати, що паровий ткацький варстат викинув
в Англії 800.000 ткачів на брук, так тут говориться не про наявні
машини, що їх треба б замінити якимось певним числом робітників,
а про існування якогось певного числа робітників, що їх
фактично замінили або витиснули машини. Протягом мануфактурного
періоду основою лишалося ремісниче виробництво, дарма
що воно вже розклалось. Нових колоніяльних ринків не могло
задовольнити відносно мале число міських робітників, що лишилися
з середньовіччя, і разом із тим мануфактури у власному
значенні відкрили нові сфери продукції для сільської людности,
прогнаної із розкладом февдалізму з землі. Отже, тоді в поділі
праці та в кооперації по майстернях виразно виявився позитивний
бік, а саме те, що вони підносили продуктивність праці вживаних
робітників.\footnote{
Сер Джеймс Стюарт навіть і вплив машин розглядав ще цілком у
цьому розумінні. «Отже, я розглядаю машини як засіб збільшити (у можливості)
число робітників, яких не треба харчувати... Чим ефект, що його
дає машина, відрізняється від того, що його дають нові мешканці?»
(«Je considère donc les machines comme des moyens d’augmenter (virtuellement)
le nombre des gens industrieux qu’on n’est pasobbligé de nourrir...
En quoi l’effet d’une machine diffère-1-il de celui de nouveaux habitants?»).
(«Principles of Political Economy», французьке видання, том І, книга 1,
розд. XIX). Багато наївніший Петті, який каже, що вона заступає «полігамію».
Цей погляд підходить хібащо до деяких частин Сполучених штатів.
Навпаки: «Рідко коли можна з успіхом уживати машин для того, щоб
зменшити працю окремого робітника; їх будування потребувало б більше
часу, ніж можна було б заощадити, вживаючи їх. Дійсну користь вони
дають тільки тоді, коли ними працюють у великому маштабі, коли одна
машина може допомагати праці тисяч. Відповідно до цього найбільше
вживають їх по найзалюдивніших країнах, де й найбільше є безробітних...
До вживання їх спричиняється не брак людей, а легкість, з якою можна
маси людей притягти до праці». («Machinery can seldom be used with success
to abridge the labours of an individual; more time would be lost in its
construction than could be saved bu its application. It is only really useful
when it acts on great masses, when a single machine can assist the work
of thousands. It is accordingly in the most populous countries, where there
are most idle men, that it is always most abundant... It is not called into
use by a scarcity of men, but by the facility with which they can be
brought to act in masses»). (Piercy Ravenstone: «Thoughts on the Funding
System and its Effects», London 1824, p. 45).
} Правда, кооперація та комбінація засобів
праці в руках небагатьох, застосовані в рільництві, викликали
в багатьох країнах задовго перед періодом великої промисловости
великі, раптові й ґвалтовні революції у способі продукції, а тим
самим і в життєвих умовах та в засобах заняття сільської людности.
Але ця боротьба точиться спочатку більше поміж великими
та дрібними землевласниками, ніж поміж капіталом та найманою
працею; з другого боку, оскільки робітників витискують засоби
\parbreak{}  %% абзац продовжується на наступній сторінці
