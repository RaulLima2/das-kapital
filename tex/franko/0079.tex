Яков І: Кождий, хто ходит від села до села і жебрає,
узнаєсь волоцюгою. Мирові суді мают право засудити го на
прилюдне бичованє і за першим разом на 6 місяців, за
другим на 2 роки тюрми. Підчас сидженя в тюрмі мают
бути так часто і так богато бичовані, як се мировий судя
узнасть за добре.... Непоправні і небеспечні волоцюги мают
бути на лівім плечи напятновані буквою R і заставлені до
робіт примусових, а як їх ще коли придиблют на жебранині,
то мают бути без милосердя і без сповіди повішені. Ті устави
(в рукоп. „уставі“), правосильні аж до перших літ 18. віку,
знесені зістали доперва 12. уст. Анни, розд. 23.

Подібні устави бачимо і в Франції, де в половині 17.
віку завязалось було ціле царство волоцюгів (truands) в Парижи.
Ще в початку панованя Людовіка XVI. (Указ з дня
13. липня 1777) кождий здорово збудований чоловік від 16
до 60 літ віку, скоро був без удержаня і не мав означеного
занятя, мав бути висланий на ґалєри. Подібні також : устава
Карля V. для Нідерляндів з 6. жовтня 1537, перший едікт
держав і міст голяндських з 19. марта 1614., оповіщене Сполучених
провінцій з д. 25. червня 1649 і богато других.
Ось яким способом, — батогами, пятнованєм та тортурами
па підставі нелюдських, кровавих устав увігнано мужиків,

ломайся, вій присилуй їх забиратися, — бідні, прості, нещасливі душі!
Мужчини її женщини, чоловіки й жінки, сироти без батьків, удови, плачучі
матері с пеленковими дітьми, і вся челядь, убога добром, а богата
роботами, бо рільництво вимагає богато рук. І волочутся вони, кажу вам,
з знакомих, рідних місць, не находячи пристанівку. Якби при й не таких
обставинах, то моглиб бодай що то вторгувати за свій, хоть і не дуже
цінний, домашний спряток; але раптово повпкидувані, мусят усе продавати
за песій гріш. А коли перебурлачат посліднпй свій гріш, то щож
тоді мают робити, як не красти, а відтак, боже добрий, по всій формі та
правді згинути на шибеници або пуститися на жебри. А й тоді ще їх
попрут до вязниць як волоцюгів, що-ді плентаются, а нічо не робят.
А що там судови до того, що їх ніхто не хоче взяти на роботу, хоть би
й як радо самі на ню напрошувались!“ і таких бідних утікачів, котрих
но словам Томаса Моруса присилувано до крадіжи, „за панованя Генріха
VIII., повішено 72000 великих та дрібних злодіїв“. (Ноllingshed, Dеscription
of England, т. І, стор. 186). За часів Єлисавети „вішано волоцюгів
цілими рядами; а прецінь не було такого року, в котрім би на
однім або другім пляцу не повішено їх 300—400“ (Strype`s Annals, т. II),
Той сам Страйп свідчит, що в Соммерcетшайрі за оден рік повішено 40
люда, папятновано 35, бито батогами 37, а випущено 183 „непоправних
злочинців“. А такій, каже він, „те велике число оскаржених не становит
ще й пятої части всіх злочинців, дякувати недбальству мирових судів
і глупому милосердю народа“. Він додає: „Прочі англійські ґрафства
зовсім не стояли ліпше від Соммерсетшайра, а богато стояло в тім згляді
ще далеко гірше“.
