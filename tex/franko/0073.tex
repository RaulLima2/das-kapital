цілі німецькі князівства), а також окремою формою ґрунтової власности, котру так насильно перемінюют
в приватну власність. Ті ґрунти, то була власність повіту (clan), — начальник або „великий чоловік“
був тілько титулярним властивцем, як представник повіту, так само, як королева англійська є
титулярною властителькою всего ґрунту Англії. Тот переворот, котрий в Шотляндії почався по посліднім
повстаню претендента, мож слідити в перших єго початках у письмах Джемса Стеєрта і Джемса Андерсона\footnote{
Стеєрт каже: „Рента в тих околицях (він хибно називає рентою тоту оплату, яку обивателі повіту
(taskmen) складали начальникови повіту) зовсім незначна в стосунку до обширности піль, але що до
числа осіб, котрих удержує одна аренда, мож сміло твердити, що оден кусник ґрунту в шотлянських
горах виживлює десять раз більше людей, ніж так само заобширний ґрунт в найбогатших рівнинах“.
}. В 18. віці заборонено притім Ґелям, прогнаним з ґрунтів, виселюватись в чужі краї, щоб їх таким
способом силою попхнути до Ґлязґова і других фабричних міст\footnote{
1860 вивожено тих насильно вивласнених хліборобів до Канади, отуманивши їх фальшивими
обіцянками. Деякі повтікали в гори і на сусідні пусті острови. Поліція пустилася за ними в погоню,
прийшло до бійки і втікачі здужали вирватися та порозбігатись.
}. За примір
методи пануючої в девятнайцятім віці\footnote{
„В шотляндських горах“, каже Бюкенен, коментатор А. Сміта, 1814, „день в день насильно затираєся
давний власностевий порядок\dots Сільский льорд, без огляду на дідичних арендаторів (знов хибно
названі тексмени) винаймає ґрунт тому, хто найбільше платит, а коли той належит до меліораторів
(imprower), то зараз заводит новий спосіб управи поля. Ґрунт, давнійше покритий дрібними
властивцями, був в стосунку до своєї плодовитости досить заселений; при новім сістемі поліпшеної
управи і побільшеної ренти одержуєсь як мож найбільше плодів як мож найменьшим коштом, і для таго
віддалюются робітники, котрі стали тепер непотрібними. Ті вигнанці з рідних хат шукают відтак
утриманя в фабричних містах і т. д. (David Buchanan: „Observations on A. Smith’s Wealth of Nations.
Edinb. 1814“.) „Шотляндські маґнати вивласнили цілі родини, немов хопту випололи: вони так обійшлися
з селами й людністю, як Інди розїдлі пімстою з дикими звірями по норах\dots Чоловіка продают
за овече руно, за волове стегно, ба ні, ще за меньшу дрібницю\dots Підчас нападу на північні
провінції Хіни була на раді Монголів така думка, щоб усіх мешканців витратити, а їх край перемінити
в степ. Тоту раду богато північно-шотляндських маґнатів дословно виповнили в своїм власнім краю і на
своїх власних земляках“. (Джордж Ензер: „An Inquiry concerning the Population of Nations. Lond.
1818“. Стор. 215, 216.
} досить буде ту навести „обчищуваня“ герцоґині Созерлєнд.
Тота в економії вишколена особа постановила зараз в початку свого панованя взятися до радікального
ліку економічного, і ціле ґрафство, в котрім задля давнійших подібних процесів оста-