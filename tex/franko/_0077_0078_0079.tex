\parcont{}
\index{franko}{0077}
а з беззглядною жорстокістю переведена переміна феодальної та окружної (Clan-)
власности в новійшу приватну власність, — ось які іділлічні були способи
первісного нагромадженя капіталу. Вони здобули ґрунт для капіталістичного
рільництва, втягли землю в обсяг капіталу, а міському промислови
достатчили потрібних „рук“, т. є. вольного і голого пролєтаріяту.

III.   Кроваві устави протів пролєтаріїв при кінци XV. віку.

Вольний і голий пролєтаріят, вигнаний с хат і ґрунтів
через скасованє феодальних дворів і через насильне раз-
заразом вивласнюванє, не міг відразу перелятися весь до
новоповстаючих мануфактур так швидко, як швидко сам
повстав. А при тімже се були люде, викинені раптово с привичного
способу житя, — а такі люде не швидко можут
застосоватися до яких небудь нових, непривичних порядків.
На першій порі з них поробилися маси жебраків, розбійників,
волоцюг, — деякі з наклінности, а найбільша часть під гнетом обставин. С
кінцем XV. і підчас цілого XVI. віку бачимо проте в цілій Західній Европі
кроваві устави протів волоцюгів. Батьки нинішної робітницької верстви мусіли
на самім вступі відбути страшну кару, — за що? За то, що їх перемінено в волоцюг
та голоту. Праводавці вважали їх „добровільними переступцями“ і думали, що
тілько від їх доброї волі залежит — працювати далі серед давних обставин, котрі
між тим зо світа щезли.

В Англії почалось те праводавство під Генріхом VII.

Генріх VIII., 1530: Старі і неспосібні до праці жебраки одержуют дозвіл на
жебрацтво. За то здорові й міцні волоцюги карані будут батогами й арештом. Вони
мают бути привязані ззаду до тачок і бичовані доти, доки не поплине кров з їх
тіла, — відтак мусят зложити присягу, вернути на місце уродженя або там, де
пробули послідні 3 роки, і „засісти до праці“ (to put himself to labour). Що за
безсердечна насмішка! В 27 уст. Генріха VIII повторена попередна устава, але
заострена новими додатками. Як кого другий раз зловят на волоцюгованю, то такого
бичувати ще раз і відтяти му пів вуха. За третим разом непоправного волоцюгу,
як тяжкого злочинця і ворога суспільности — вкарати смертю.

Едуард VI.: Устава с першого року єго панованя 1547, наказує, що скоро хто
отягаєся від праці, той має бути присуджений на невольника тій особі, котра
донесла урядови о єго неробстві. Пан має годувати невольника хлібом і водою,
слабими напитками і такими обрізками мяса, які му видадутся відповідними. Він
має право всилувати го батогами \index{franko}{0078}
та зелізними ланцами до всякої, хотьби й як гидкої роботи. Коли невольник на 14
день віддалится, то зістає засуджений на віковічну неволю і має бути на чолі
або на лици напятнований буквою S, а коли до трох раз утече, то має бути
вкараний смертю, як зрадник держави. Пан може го продати, передати в наслідство,
визичити другому в неволю, зовсім так, як усяке друге рухоме добро, як худобу.
Коли невольники в чім небудь станут супротів панів, то мают також бути покарані
смертю. Мирові судьї повинні за отриманим остереженєм слідити за волоцюгами.
Коли покажеся, що такий волоцюга три дни волочився без діла, то такого
відставити на місце, де родився, роспеченим зелізом напятнувати на груди буквою
V і тамій в зелізних ланцюхах уживати до замітаня вулиці або до якої небудь
їншої служби. Коли волоцюга подасть фальшиво місце вродженя, то за кару має
бути віковічним невольником тої громади, тих мешканців або того товариства і
напятнований буквою S. Кождий має право відобрати у волоцюги єго діти і яко
помічників та термінаторів держати хлопців до 24, дівчат до 20 літ. Коли вони
втечут, то мают аж до тих літ бути невольниками майстра, а тому вільно їх
заковувати в ланци, бити і пр., як му сподобаєсь. Кождий пан може заложити
зелізну обручку на шию, руку або ногу свого невольника, щоби міг го ліпше
пізнати і бути певним, що му не втече\footnote{
Автор книжки „Essay on Trade and Commerce“ 1770, каже: „Під панованєм Едварда
VI. взялись були Англічане зовсім, здаєсь, серйозно до піддвигненя мануфактур і
затрудненя бідних. Се бачимо з одної дивовижної устави, в котрій приписуєсь, що
всі волоцюги мают бути пятновані, і т. д. (Essay on Trade and Commerce, стор.
8).
}. Послідна часть тої устави наказує, щоб
деяких бідних брали на себе громади або поєдинчі люде; ті мают їм давати їсти
й пити і старатись для них о роботу. Тот рід громадських невольників удержувався
в Англії гет ще в 19. віці під назвою roundsmen (люде, що ходят від хати до
хати).

Єлисавета, 1572: жебраки без дозволу і віком понад 14 літ мают бути без
милосердя бичовані і напятновані на лівім вусі, хіба що їх хто схоче взяти на
два роки на службу; в разі повтореня, коли мают над 18 літ, мают бути — смертю
карані, скоро їх ніхто не схоче взяти на два роки на службу; за третим разом
мают без милосердя як зрадники державні бути покарані смертю. Подібна також 18.
устава Єлисавети, розділ 13, і устава з р. 1597 \footnote{
Томас Морус каже в своїй „Утопії“: „Так то дієся, що оден захланний і неситий
ненаїсник, правдива чума нашої вітчини, може тисячі екрів ґрунту збити до купи
і обпалькувати, обгородити одним плотом, або силою та кривдою до того довести
єго властивців, що вони будут мусіли все спродувати. Сяким чи таким способом,
чи там гнись чи ломайся, вій присилуй їх забиратися, — бідні, прості, нещасливі
душі! Мужчини її женщини, чоловіки й жінки, сироти без батьків, удови, плачучі
матері с пеленковими дітьми, і вся челядь, убога добром, а богата
роботами, бо рільництво вимагає богато рук. І волочутся вони, кажу вам,
з знакомих, рідних місць, не находячи пристанівку. Якби при й не таких
обставинах, то моглиб бодай що то вторгувати за свій, хоть і не дуже
цінний, домашний спряток; але раптово повикидувані, мусят усе продавати
за песій гріш. А коли перебурлачат послідний свій гріш, то щож
тоді мают робити, як не красти, а відтак, боже добрий, по всій формі та
правді згинути на шибеници або пуститися на жебри. А й тоді ще їх
попрут до вязниць як волоцюгів, що-ді плентаются, а нічо не робят.
А що там судови до того, що їх ніхто не хоче взяти на роботу, хоть би
й як радо самі на ню напрошувались!“ і таких бідних утікачів, котрих
но словам Томаса Моруса присилувано до крадіжи, „за панованя Генріха
VIII., повішено 72000 великих та дрібних злодіїв“. (Ноllingshed, Dеscription
of England, т. І, стор. 186). За часів Єлисавети „вішано волоцюгів
цілими рядами; а прецінь не було такого року, в котрім би на
однім або другім пляцу не повішено їх 300--400“ (Strype`s Annals, т. II),
Той сам Страйп свідчит, що в Соммерcетшайрі за оден рік повішено 40
люда, папятновано 35, бито батогами 37, а випущено 183 „непоправних
злочинців“. А такій, каже він, „те велике число оскаржених не становит
ще й пятої части всіх злочинців, дякувати недбальству мирових судів
і глупому милосердю народа“. Він додає: „Прочі англійські ґрафства
зовсім не стояли ліпше від Соммерсетшайра, а богато стояло в тім згляді
ще далеко гірше“.
}.

\index{franko}{0079}
Яков І: Кождий, хто ходит від села до села і жебрає,
узнаєсь волоцюгою. Мирові суді мают право засудити го на
прилюдне бичованє і за першим разом на 6 місяців, за
другим на 2 роки тюрми. Підчас сидженя в тюрмі мают
бути так часто і так богато бичовані, як се мировий судя
узнасть за добре\dots{} Непоправні і небеспечні волоцюги мают
бути на лівім плечи напятновані буквою R і заставлені до
робіт примусових, а як їх ще коли придиблют на жебранині,
то мают бути без милосердя і без сповіди повішені. Ті устави
(в рукоп. „уставі“), правосильні аж до перших літ 18. віку,
знесені зістали доперва 12. уст. Анни, розд. 23.

Подібні устави бачимо і в Франції, де в половині 17.
віку завязалось було ціле царство волоцюгів (truands) в Парижи.
Ще в початку панованя Людовіка XVI. (Указ з дня
13. липня 1777) кождий здорово збудований чоловік від 16
до 60 літ віку, скоро був без удержаня і не мав означеного
занятя, мав бути висланий на ґалєри. Подібні також: устава
Карля V. для Нідерляндів з 6. жовтня 1537, перший едікт
держав і міст голяндських з 19. марта 1614., оповіщене Сполучених
провінцій з д. 25. червня 1649 і богато других.
Ось яким способом, — батогами, пятнованєм та тортурами
па підставі нелюдських, кровавих устав увігнано мужиків, \parbreak{}
