\parcont{}
\index{franko}{0063}
відповідь на те дуже проста: жерело капіталістичного ладу,
се не що їнше, як той процес відділюваня робітника
від власности, від средств продукційних.
Сей процес з одного боку перемінює суспільну істенину
(средства продукційні і знадоби до житя) в капітал, а з другого
боку перемінює беспосередних витвірців в наемних
робітників. Так назване „первісне нагромадженє капіталу“,
се затим не що їнше, як історичний процес відділюваня
продуцента від средств продукційних. Він і справді „первісний“,
бо становит вступ до історії капіталу і капіталістичної
продукції.

На перший погляд видно, то той процес роскладовий
обнимає собою цілий ряд історичних процесів і то ряд двоякий:
з одного боку нищенє тих відносин, котрі робітника
робили власністю третих осіб, їх средством продукційним,
— з другого боку вивласнюванє беспосередних витвірців,
витискане їх с посіданя средств продукційних. Процес роскладовий,
се затим на ділі ціла історія розвитку новійшої
буржоазної суспільности. Се була б зовсім не трудна історія,
колиб буржоазні історики не були єї нам вказали виключно
в рожевім світлі еманціпації робітників, а були б звернули
увагу й на то, якими способами в тій історії визискуванє
феодальне перемінилося в визискуванє капіталістичне. Початок
розвитку становила неволя робітника. В дальшім тягу
того розвитку неволя осталась, тілько в зміненій формі.
Але ми ту не будем вдаватися в розбір середновікових рухів.
Хоть капіталістична продукція вже в 14. і 15. віці розпочалася
в деяких місцях над Середземним морем, то прецінь
ера капіталістична починався аж від 16. віку. Там,
де вона росцвитає, давно вже знесено панщину і середновікове
міщанство також як раз хилится до впадку.

Епохи в історії того роскладового процесу становят ті
хвилі, коли великі маси людей нараз і силою відривано від
усіх средств до житя та праці і як свобідннх і голих пролетаріїв
перто на робучий торг. Вивласнюванє робітників
з ґрунту і посідлости становит основу цілого процесу. Тож
і ми насамперед мусимо переглянути історію того вивласненя.
В різних краях вона проявляєся в різних окрасках
і переходит різні фази в неоднакім порядку. Тілько в Англії,
котру ми проте беремо за примір, вона має клясичну форму\footnote{
В Італії, де капіталістична продукція розвилась найраньше, найраньше
також увільнено кріпаків. Тількож при тім увільненю вони не
одержали права па ґрунти, хотьби й за сплатою індемнізації. так що
„воля“ перемінила італіянських кріпаків відразу в голих пролєтаріїв, котрі
крім того по містах, стоячих ще переважно від римських часів, найшли
вже готових нових панів.
}.
