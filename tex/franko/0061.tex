Початок і історичний розвиток
капіталістичної продукції в Англії.

1. Первісне нагромадженє капіталу.

Ми бачили, що гроші стают капіталом тоді, коли служат
до купованя робучої сили. Ми бачили, що капітал
раз-ураз намагає творити надзвигаку вартости, а надзвишка —
вбільшує капітал. Між тим, щоб капітал міг нагромаджуватись,
мусит уже вперед витворюватись надзвишка;
щоб могла витворюватись надзвишка, мусит істнувати капіталістична
продукція, а щоб тота істнувала, мусит уже
вперед більша маса капіталу бути нагромаджена в руках
поединчих богатирів. Здаєсь затим, що весь той процес
полягає на якімось „первіснім“ нагромадженю, котре мало
місце перед капіталістичною продукцією, котре, значит, не
було випливом капіталістичної продукції, а є її джерелом.
