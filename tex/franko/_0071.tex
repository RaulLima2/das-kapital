\parcont{}
\index{franko}{0071}
каже Річард Прайс, „всі ґрунти будут в руках кількох великих арендаторів, то з дрібних арендаторів (про них Прайс
казав уперед ось що: „множество дрібних властивців і арендаторів, що вдержуют самі себе й свої родини добутками з ґрунту, котрий оброблюют, доходами з овець, дробу, свиней і т. д.,
котрі випасают на громадських толоках, так, що для вдержаня їм мало що приходится докуповувати“)
пороблятся люде, котрі будут мусіли працею заробляти на прожиток собі і другим, і все, чого їм
треба, будут мусіли купувати на торзі\dots{} Бути може, що праці тоді буде більше, бо більше буде примусу\dots{} Міста й
мануфактури будут змагатися, бо до них напхаєся більше людей шукаючих заняття. Се тота дорога, по
котрій зовсім природно пре концентрація аренд і по котрій вона дійсно довгі вже літа чим раз далі
посуває Англію“. Загальне вліянє „прилучень“ ось як описує Прайс: „Взагалі положінє нижчих верстов
народа майже в кождім згляді погіршилося. Дрібні властивці та арендаторі зруйновані та зведені до
стану наємннків та комірників; а рівночасно й о прожиток в тім стані стало далеко тяжше“\footnote{
В наведеній книжці Р. Прайса, стор. 147, 159. Се нагадує стародавний Рим, котрого порядки ось як
описує Аппіан в „Історії римських війн домашних“, кн. І, 7: „Богачі забрали в свої руки няйбільшу
часть неподілених ґрунтів. Вони задуфали на обставини часу, що їм тих ґрунтів ніхто вже не відбере, і
скуповували проте сусідні частки бідних, по части за їх згодою, а по части відбирали їм силою, — так, що замісць
поєдинчих  піль богачі оброблювали переважно обширні лани. Притім
вони уживали невольників до управи поля і годівлі худоби, бо свобідних
людей позабирано їм від праці до війська. Посіданє невольників приносило їм іще й тоту велику
користь, що невольники — вільні від військової служби — могли без перепони множитися і плодили
богато дітей. Таким способом постягали магнати всі богацтва до себе, і цілі околиці вкриті були
невольниками. А правдивих Італьців ставало між тим усе меньше, — їх руйнували: бідність, податки та
військова служба. А хоть часом і настав супокій, то вони зовсім не могли підпомочися, бо весь ґрунт
був у богацьких руках, котрі замісць свобідних людей воліли мати до праці невольників“. Сесь уступ
описує часи перед правом Ліцінія. Військова служба, котра так прудко прискорила руїну римських
плебеїв, була також головним средством, при помочи котрого Карло Великий перемінив вольних німецьких
селян в кріпаків так швидко, мов петрушку в розсаднику зростив.
}. І справді, наслідки забору громадських ґрунтів і доконаного тим забором перевороту в рільництві далися
так прудко і прикро почути сільским робітникам, що, як сам Еден признає, між 1765 а 1780 плата їх
почала знижуватися до крайної границі і уряд мусів поповнювати єї датками запомоговими. „Плата їх“,
каже Еден, „не вистатчала вже зовсім для потреб житя“.
