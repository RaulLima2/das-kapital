капітал через ужитє наємних робітників і одну часть надвишки витворів, грішми чи натурою, платят дідичови яко ренту ґрунтову. Доки в 15. віці незалежний мужик, а також сільский наймит, що попри наймитство й сам про себе веде ґосподарство, збогачуются самі власною працею, доти й обставини тай обсяг продукційний арендатора остаются дуже скромні. Переворот в рільництві, що почався в послідній
третині 15. віку і трівав через цілий 16 вік крім єго послідних десятиліть, збогатив го майже так само прудко, як прудко зубожив мужиків 25). Загарбанє громадських пасовиск і т. д. дозволяє му богато побільшувати число худоби майже без ніяких видатків, а між тим худоба достатчувала му далеко більше обірнику для поправи ґрунту. В 16. віці причинюєсь ще одна рішучо важна обставина. Тоді арендові контракти були довгі, нераз де з на 99 літ. А ту в 16. віці вартість золота та срібла, а разом з ним і вартість грошей раз-ураз вменьшуєсь, і арендаторам се принесло золоті плоди. Не зважаючи на прочі, вперед згадані обставини, арендаторі першим ділом вменьшили робочу плату. Те, що урвано робітникам на платі, побільшувало
арендовий зиск. А з другого боку ціна збіжя, вовни, мяса, — одим словом, всіх плодів рільничих, раз-ураз вбільшуєсь, через що змагаєся грошевий капітал арендатора
без єго причинку, — а притім ще ренту ґрунтову дідичови платит він давними, стратившими на вартости, грішми. Таким способом він збогачуєсь рівночасно на кошт своїх наймитів і свого дідича. Не диво (в рукоп. „даво“) затим, що вже с кінцем 16. віку витворилась в Англії окрема верства як на тодішні обставини богатих „капіталістичних“ арендаторів 26).

25)  „Арендаторі“, каже Гаррізен в своїй „Description of England“, „котрим давнійше годі було заплатити 4 ф. шт. ренти, платят тепер по 40, 50 та 100 ф. шт. і ще кажут, що їм зле повелося, коли по упливі арендового контракту не зложили бодай тілько готівки, кілько виносит 6—7-милітна рента“.

26) В Франції з „Regisseur-ів“, т. є. панських окономів та тивунів середновікових поробилися швидко т. зв. hommes d'affaires, т. є. люде, що туманництвом та шахрайством подороблялися капіталів. Такі окономи, то були нераз великі пани. Як в Англії, так і в Франції великі феодалні добра поділені були на богато дрібних ґосподарств, але з условинами далеко гіршими для мужиків. В 14. віці повстают і ту аренди, звані ту „fermes“ або „terriers“. Число їх раз-ураз змагалося і дійшло гет понад
100000. Вони платили чи то грішми чи натурою ренту ґрунтову, котра виносила від 12-тої до 5-тої части річного здобутку. Ті terriers були цілими або частковими леннами як до вартости і обєму ґрунтів, котрі нераз виносили заледво кілька прутів. Всі арендаторі мали до певної степені (степенів було штири) власть судову над мужиками, жиючими
