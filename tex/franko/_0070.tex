\parcont{}
\index{franko}{0070}
льокайський і від самоволі лєндльордів залежний збрід, розросталися тимчасом з рабунку державних
дібр, а ще більше з сістематичного загарбуваня громадських ґрунтів ті великі аренди, котрі в 18. в.
звано арендами капіталовими або купецькими. Чим більше вони розросталися, тим більше селян витискано
з їх давних домівок, тим більше пролєтарів перлося до міст, до промислу.

Але 18. вік не понимав ще так досконало, як 19., що „богацтво національне“, а вбожество народне —
одно й то само. Про те горячі спори в тогочасній економічній літературі зза „прилучуваня громадських
ґрунтів“. З великої
маси матеріялу, який маю під руками, подаю отсе кілька виривків, бо в них живо малюєся тодішне
положінє.

„В многих округах в Гертфордшайрі“, пише з обуренєм Томас Урайт, „зіллято 24 аренди, кожда пересічно
в 50--150 екрів, усего в 3 аренди“. „В Нортгемтоншайрі і Лінкольншайрі загалом поприлучувано
громадські ґрунти до приватних дібр, а повсталі відси нові льордства поперевертано в толоки. Через
те в многих льордствах не ореся тепер і 50 екрів, де вперед орано 1500\dots{} Звалища колишних хат,
стоділ, стаєнь і т. д., се єдині сліди по давнійших мешканцях. З соток домів і родин де в яких селах
полишалося по 8--10. В найбільшій части округів, де прилучуванє почалося ледво від 15--20 літ назад,
уже властивців ґрунтових дуже мало супротів того, що було вперед. Се ще звичайна річ, коли 4 або 5
богатих годівників худоби посідают недавно позлучувані льордства, на котрих уперед жило 20--30
арендаторів і богато\footnote*{
В рукописі: богати.
} дрібних властивців та комірників. Всіх їх з родинами й цілим спрятком
повикидано гет, а з ними й богато таких родин, котрі у них зарабляли собі прожиток“. (Се пише ч.
Аддіґтон). І прилучували сусідні лєндльорди на підставі Bills for enclosures не тілько перелоги, але
часто й управні ґрунти, котрі громада або винаймала поєдинчим ґаздам за певною оплатою, або
оброблювала спільно. Говорю ту про прилучуванє царини
і загалом управних ґрунтів. Навіть писателі, котрі боронят „прилучуваня“, признают, що воно в тім
разі вменьшило управу піль, підняло в гору ціни за живність і причинилося до обезлюдненя сіл\dots{} А
навіть прилучуванє пустих ґрунтів, яке тепер відбуваєся, відбирає бідному часть утриманя
і вбільшує аренди, котрі й так уже за великі“\footnote{
Др.~Річард Прайс в своїй кнпжці: „Observations on Reversionary
Payments“, т. II, стор. 155. Прошу читати Форстера, Аддінґтона, Кента,
Прайса і Джемса Андерсона, а порівнати се з нужденною балаканкою
та вонючими похвалами Мак-Кельльока в єго списі: „The Literature of
Political Economy. Lond. 1845“.
}. „Коли“,
\parbreak{}
