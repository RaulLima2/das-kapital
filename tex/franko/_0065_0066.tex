\index{franko}{0065}
Перший крок перевороту, що поклав основу капталістичній продукції, припадає в послідній третині 15 і
в першій чверти 16 віку. Тоді скасовано феодальне дворацтво, котре, як справедливо замічає Джемс
Стюерт, ,,залякало  всі хати і двори безхосенно". Через те викинено масу голих пролєтаріїв на
робучий торг. Хоть королівська власть, що й сама виросла з буржуазного розвитку, намагаючи до
неограниченого панованя, силою скасувала те великопанське дворацтво, то прецінь вона не була єдиною
причиною нового перевороту. Ні, в упертім опорі протів королівства та
парляменту витворили великі пани-феодали далеко більшу масу пролєтаріяту, прогонюючи силою
хліборобів з ґрунту і посідлости, хоть хлібороби мали до тих ґрунтів більше право, ніж вони, і
забираючи для себе громадські ґрунти. Беспосередний товчок до того в Англії дав іменно росцвіт
фляндрійської вовняної мануфактури і звязане з ним підскоченє цін вовни. Стара феодальна шляхта
вигибла в великих феодальних війнах, а нова шляхта — се були діти свого часу, для котрих гроші були
силою понад всі сили. З вірного поля пасовиська для овець! — се став тепер їх загальний оклик.
Гаррізен в своїй „Description of England. Prefixed to Holinshed’s Chronicles“ описує, як
вивласнюванє дрібних ґаздів руйнує край. „Але що нашим великим самозванцям до того?“ Мешканя ґаздів
та коттеджі робітників валят вони силою або прогнавши людей лишают пустками. „Коли перездримо
давнійші інвентарі кождої домінії, то побачимо, що незлічимі хати та дрібні ґаздівства пощезали, що
ґрунт годує далеко меньше люда, що богато міст підупало, хоть деякі нові підносятся… Мож би
чимало наросповідатися про місточка та села, зруйновані для того, щоб було місце на толоки для
овець; тілько самотні панські двори стоят серед тих толок“. Правда, наріканя тих старих літописів
усе пересаджені, але вони досадно малюют те вражінє, яке на самих сучасників робив переворот
обставин продукційних. Порівнанє між письмами канцлєрів Фортеске і Томаса Моруса вказує наглядно
пропасть між 15. а 16. віком. „Із золотого віку — каже справедливо Зорнтон — попали англійські
робітники без ніяких перехідних ступнів прямо в зелізну“.

Праводавство злякалось сего перевороту. Воно не стояло ще на такім високім ступни цівілізації, де
„богацтво народне“, т. є. богацтво капіталістів і безграничне висисанє та зубожінє маси люду
становит верх премудрости
політичної. В своїй історії Генріха VII. каже Бекон: „В тім часі (1489) посипалися скарги на то, що
вірне поле перемінюєсь в пасовиська, котрих лехко може дозирати кілька пастухів. Ґрунти, що вперед
виарендовувались на кілька літ, на доживотну або щорічну умову, тепер зіллято разом
\index{franko}{0066}
с панськими. Се підкопало добробуток люду, а через те й міста, церкви, десятини\dots{} Щоб зарадити
тому лиху, проявили король і парлямент дивну на ті часи мудрість\dots{} Вони видали право протів того
обезлюднюючого край загарбуваня громадських ґрунтів (depopulating inclosures) і невідлучної
від него обезлюднюючої ґосподарки толочної (depopulating pasture[s])“. Оден акт Генріха VII. з р.
1489 заказує руйнувати хліборобські хати, до котрих належит що найменьше 20 екрів ґрунту. Генріх
VIII відновив той самий указ. Говорится там між їншим, що „многі аренди і огромні отари, особливо
овець, нагромаджуются в немногих руках, через що дохід
з ґрунту дуже вбільшився, а рільництво дуже підупало, церкви і хати повалено, дивовижні маси народа
стали неспосібні вдержувати себе і свої родини“. Указ наказує затим відбудовувати повалені хутори,
означує, кілько має бути вірного поля в стосунку до овечих толок і т. д. Їнший акт з р. 1533
жалуєсь, що деякі властивці мают по 24000 овець, і ограничує їх число на 2000 \footnote{
В своїй „Утопії“ говорит Томас Морус про дивовижний край, де
„вівці їдят людей“.
}. Наріканя народа і
праводавство протів вивласнюваня дрібних арендаторів та хліборобів, що почалось від Генріха VII і
трівало зо 150 літ
— не помогли нічо. Чому не помогли, пояснює нам Бекон, сам того не знаючи. „Акт Генріха VII, — каже
він в своїх „Essays, civil and moral“, Sect. 20, — був глибоко і дивно обдуманий. Він утворив
сільскі ґаздівства і хліборобські доми певного нормального розміру, т. є. вдержав для них таку
пропорцію ґрунту, котра давала їм змогу плодити на світ підданих доста заможних і не придавлених
нуждою, так що плуг був в руках властивців, а не наємників\footnote{
Бекон пояснює далі звязок між свобідним, заможним селянством
а доброю інфантерією. „Се була дивно важна річ для сили і мужности
королівства — мати аренди достаточного розміру, щоб дільних мужів
забеспечити від нужди і велику часть ґрунту краєвого запевнити в посіданє джоменам, т. є. людім
середної заможности між шляхтою а халупниками (cottagers) та наймитами. Бо се загальна думка
найліпших знавців воєнного діла\dots{} що головна сила армії, се інфантерія або піхота. Але щоб
витворити добру інфантерію, тре людей вихованих не в притиску ані в нужді, але свобідно і в певній
заможности. Коли затим яка держава вросте переважно в шляхту та делікатне панство, а хлібороби та
ратаї зійдут на простих зарібників та наймитів або халупників, т. є. жебраків з власною хатою, то
така держава може мати добру кінницю, але доброї піхоти не буде мати. Се видно в Італії і Франції і
деяких других заграничних краях, де справді все або шляхта або нужденні зарібники\dots{} Дійшло там до
того, що ті краї мусят уживати наємного зброду Швейцарів та др. для своєї піхоти: відти то й пішло,
що ті держави мают богато людий, а мало вояків“. („The Reign of   Henry VII.“ і т. д.).
}. А між
\parbreak{}
