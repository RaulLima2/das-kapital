\parcont{}
\index{franko}{0069}
ще не було розгарбане за революції, се основа нинішних
князівських посідлостей англійської оліґархії\footnote{
Прошу прочитати н. пр. Е. Борка памфлєт про родину герцоґів
Бедфорд, котрої потомок, льорд Джон Россель — один з головних стовпів
теперішного лібералізму.
}. Капіталісти
з міщан радо дивилися на ті операції, між їншим і для
того, бо ґрунти через те робилися чистим товаром, а сільскі
пролєтарії, обідрані до крихти, чим раз більше тислися
до міст за роботою. Вони поступали зовсім відповідно для
власної користи, так само, як шведські міщане, котрих економічною
опорою було селянство і котрі затим дружно с селянами
помогали королям (від р. 1604, пізнійше під Карлом
X. і Карлом XI.) силою видирати коронні добра з рук
маґнатів.

Власність громадська, се була староґерманська встанова,
котра животіла під покривкою феодальства. Ми бачили,
як тоті громадські ґрунти силою загарбувано, при
чім по більшій части рілю перемінювано в толоки. Се почалося
с кінцем 15. віку і трівало далі в 16. Але тоді було
се все такі особистим насилєм, супротів котрого праводавство
дармо боролося цілих 150 літ. Поступ 18. віку проявляєся
тим, що само право від тепер починає підпирати рабунок
громадських ґрунтів, хоть великі арендаторі побіч
того не закидают і своїх дрібних незалежних способиків на
власну руку\footnote{
„Арендатори заказуют коттеджерам (халупникам) держати будь
яку будь живу тварь крім себе самих, а то тому, бо як будут держати
худобу або дріб, то будут мусіли з їх стоділ красти пашу. У них є приповідка:
„держи халупника в бідности, то вдержиш го в пильности“.
А властиво все діло ту в тім, що арендаторі таким способом привласнили
собі виключне право на громадські ґрунти“, („А Political Enquiry into
the Consequences of enclosing Waste Lands. Lond. 1785“, стор. 75.).
}.  Парляментарною формою, в якій відбувалися
ті рабунки, були „Bills for Inclosures of Commons“ (Закони
про прилученє громадських ґрунтів). Се були декрети, котрими
сільскі льорди роздаровували власність народну самі
собі на власність приватну, — правдиві декрети обдираня
народа. Сер Ф. М. Еден, котрий хитро, як правдивий адвокат,
доказує, що ґрунти громадські, се властиво приватна
власність сільских льордів, що настали намісць феодалів,
— сам же зараз збиває всі свої докази, коли домагався
„загальної постанови парляменту для прилученя громадських
ґрунтів (до дібр приватних)“, — значит, признає, що
для їх переміни в приватну власність конечно треба парламентарного
замаху, — а з другого боку сам домагався
від праводавства „відшкодованя“ для вивласнених бідаків.

Між тим коли замісць незалежних їоменів (ґаздів) настали
„tenants-at-will“, т. є. дрібні арендаторі на оден рік,
\parbreak{}
