\index{franko}{0072}

Послухаймо ще хвильку, що говорит оден защитник „прилучуваня“ а противник Р.~Прайса: „Зовсім
фальшива тота гадка, що край обезлюднів, бо не видно-ді людей, працюючих в чистім поли. Коли їх
тепер убуло по селах, то за то прибуло їх по містах\dots{} Коли дрібні ґазди-рільники перемінились в
наємних робітників, то через те сама кількість добутої праці стає більша, а се прецінь користь
пожадана для суспільности (тілько що, розумієся, самі „перемінені“ не належат до тої
суспільности!)\dots{} Добутку буде більше, коли скомбінована праця тих наємників буде ужита в \so{одній}
аренді; таким способом повстане надвишка витворів, котра піде до мануфактур, а через те й
мануфактур, тих жерел нашого богацтва, стане більше в стосунку до витвореної многоти збіжя“.

Незамутимий супокій, з яким суспільний економіст глядит на найзухвальше топтанє „святого права
власности“, на найгидше знущанє над людьми, коли йно все то робится для того, щоб покласти підвалину
капіталістичній продукції, проявляє між їншими торій і „філянтроп“ сер Ф.~М.~Еден. Цілий ряд
рабунків, головництв і притисків народних, серед яких відбувалося вивласнюванє люду ві(д) послідної
третини 15. до кінця 18. віку, викликає у него тілько сей супокійно-радісний вивід: „Належита
пропорція між вірними полями а толоками мусіла бути встановлена. Ще в цілім 14. і найбільшій части
15. віку на оден екр толоки приходилося 2, 3, а навіть 4 екри вірного поля. В половині 16. віку
перемінилася пропорція: 3 екри толоки приходили на 2 екри рілі, а пізнійше 2 екри толоки на 1 екр
рілі, аж поки вкінци не вийшла належита пропорція: 3 екри толоки на 1 екр рілі“.

В 19. віці защезла, розумієся, й память про звязок між хліборобами а власністю громадською. Не
згадую вже зовсім о найпослідних часах, — але чи одержали селяне хоть оден шелюг відплати за тих
3,511.770 екрів громадського ґрунту, котрі їм зрабовано між роками 1801 а 1831 і котрі с
парляментарними формальностями сіль(с)кі льорди подарували собі самим?

Послідний великий процес вивласнюваня хліборобів, се вкінци т. зв. „\textenglish{Clearing of Estates}“ (обчищуванє
дібр, або радше вимітанє з них людей). Тото „обчищуванє“, се вершок усіх англійських способів, які
ми доси бачили. Там, де вже не осталося незалежних ґаздів-хліборобів, доходит до вимітаня коттеджів,
так, що хліборобські робітники не можут уже найти й кусничка місця для замешканя на тім ґрунті,
котрий оброблюют. Властиве „обчищуванє дібр“ відзначуєся нечуваною сістематичністю і огромним
розміром, в якім тота операція нараз виконуєсь (в Шотляндії н. пр.
вона відбувалася нараз на просторах таких завбільшки, як
\parbreak{}
