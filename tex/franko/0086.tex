VI.    Вліяніє рільничого перевороту па промисл.
Промисловий капітал здобуває собі в краю ринок
відбутовий.

Раптове і частими нападами повторюване вивласнюванє
та прогонюванє мужиків достатчило, як ми бачили,
міському промислови раз-за-разом маси пролєтаріїв, не належачих
зовсім до ніяких цехових звязків, — дуже мудра
подія, про котру старший Андерзен (не треба го мішати
з Джемсом Андерзеном) в своїй історії торговлі каже, що
се прямо боже провидініє так зробило. Ще хвилю мусимо
задержатися над тим складником первісного нагромадженя
капіталів. Не тілько що по селах убуло незалежного, самоґосподаруючого
мужицтва, а по містах прибуло промислового
пролєтаріяту, так, як після Жаффроа Сент-Улєра світової
матерії в одних місцях убуває, між тим коли в других
місцях вона згущаєсь. Помимо меньшого числа оброблюючих
рук ґрунт видавав прото однако або й ще більше
плодів, бо разом с переворотом в ґрунтових відносинах
власностевих настали також ліпші способи управи, більша
кооперація, зосередженє средств продукційних і т. д., а з другого
сільські наємники не тілько силувані були до тяжілої
праці — на се головно напирає сер Джемс Стеарт, —
а й обсяг їх домашної продукції, де вони працювали самі
на себе, чим раз більше вменьшувався. З освободженєм
одної части мужицтва освободжені зістали також єго давні
средства прожитку. Вони стают тепер матеріяльним складником
змінного капіталу*. Бездомний та немаючий мужик
мусит окупувати собі ті средства прожитку від свого
нового пана, промислового капіталіста, в формі робучої
плати. Як зі средствами прожитку, так само сталося й з домашним
рільничим сирим матеріялом, котрого переробкою
займався промисл. Той сирий матеріял став частиною по-

на їх ґрунтах. Лехко понити, якого притиску мусів дізнаватп люд від
усіх тих дрібних тиранів. Монтейль каже, що тоді було в Франції 160000
судів, де тепер вистарчає (враз із мировими судами) 4000 трибуналів.

* Звісно, що Маркс ділит капітал на постійний (constant) і змінний
(variabel), а то після того, чи в довшім протягу продукції вартість
єго зміняєся, чи ні. І так машини, сирий матеріял, будинки фабричні
і т. д., се капітал постійний, бо продукція не змінює в загальній сумі
єго вартости, а то, що убуде вартости на машинах і приладах і пр.,
котрі зуживаются при роботі, прибуває самим витворам, котрі через переробку
зискуют на вартости. Між тим друга часть капіталу, а іменно
тота, котра йде на наймленє і удержано робітника і містится в понятю
робучої плати, се капітал змінний, бо по кождім процесі продукційнім
капіталіст добуває з него більше, ніж видав. Робітник витворює вартість
більшу, ніж тота, яку одержав в формі робучої платп. (Прим. перев.)
