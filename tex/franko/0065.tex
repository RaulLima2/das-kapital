Перший крок перевороту, що поклав основу капталістичній продукції, припадає в послідній третині 15 і
в першій чверти 16 віку. Тоді скасовано феодальне дворацтво, котре, як справедливо замічає Джемс
Стюерт, ,,залякало  всі хати і двори безхосенно". Через те викинено масу голих пролєтаріїв на
робучий торг. Хоть королівська власть, що й сама виросла з буржуазного розвитку, намагаючи до
неограниченого панованя, силою скасувала те великопанське дворацтво, то прецінь вона не була єдиною
причиною нового перевороту. Ні, в упертім опорі протів королівства та
парляменту витворили великі пани-феодали далеко більшу масу пролєтаріяту, прогонюючи силою
хліборобів з ґрунту і посідлости, хоть хлібороби мали до тих ґрунтів більше право, ніж вони, і
забираючи для себе громадські ґрунти. Беспосередний товчок до того в Англії дав іменно росцвіт
фляндрійської вовняної мануфактури і звязане з ним підскоченє цін вовни. Стара феодальна шляхта
вигибла в великих феодальних війнах, а нова шляхта — се були діти свого часу, для котрих гроші були
силою понад всі сили. З вірного поля пасовиська для овець! — се став тепер їх загальний оклик.
Гаррізен в своїй „Description of England. Prefixed to Holinshed’s Chronicles“ описує, як
вивласнюванє дрібних ґаздів руйнує край. „Але що нашим великим самозванцям до того?“ Мешканя ґаздів
та коттеджі робітників валят вони силою або прогнавши людей лишают пустками. „Коли перездримо
давнійші інвентарі кождої домінії, то побачимо, що незлічимі хати та дрібні ґаздівства пощезали, що
ґрунт годує далеко меньше люда, що богато міст підупало, хоть деякі нові підносятся.... Мож би
чимало наросповідатися про місточка та села, зруйновані для того, щоб було місце на толоки для
овець; тілько самотні панські двори стоят серед тих толок“. Правда, наріканя тих старих літописів
усе пересаджені, але вони досадно малюют те вражінє, яке на самих сучасників робив переворот
обставин продукційних. Порівнанє між письмами канцлєрів Фортеске і Томаса Моруса вказує наглядно
пропасть між 15. а 16. віком. „Із золотого віку — каже справедливо Зорнтон — попали англійські
робітники без ніяких перехідних ступнів прямо в зелізну“.

Праводавство злякалось сего перевороту. Воно не стояло ще на такім високім ступни цівілізації, де
„богацтво народне“, т. є. богацтво капіталістів і безграничне висисанє та зубожінє маси люду
становит верх премудрости
політичної. В своїй історії Генріха VII. каже Бекон: „В тім часі (1489) посипалися скарги на то, що
вірне поле перемінюєсь в пасовиська, котрих лехко може дозирати кілька пастухів. Ґрунти, що вперед
виарендовувались на кілька літ, на доживотну або щорічну умову, тепер зіллято разом
