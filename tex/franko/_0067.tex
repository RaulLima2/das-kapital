\parcont{}
\index{franko}{0067}
тим капіталістична продукція намагала як раз до противного: до залежности людової маси, до поверненя
єї в наємників, а єї средств продукційних — в капітал. Тотож давнійше праводавство стараєсь утримати
також 4 екри ґрунту при коттеджи сільского наємного робітника, заказуючи му між тим приймати
комірників до свого коттеджу. Ще 1627, за Якова І. засуджено Роджера Крокера с Фронтміль за то,
що-ді в своїх добрах давав коттеджі своїм наємникам без приписаних 4 екрів ґрунту; ще 1638, за Карла
І. визначена була королівська комісія для примусового поновленя давних прав, а іменно права про 4
екри ґрунту, конечного придатку до наємницького коттеджу; ще Кромвель заказав будувати дім в 4 милях
довкола Льондону без конечного придатку до него в 4 екри ґрунту. Ще в першій половині 18. віку чути
жалі, що коттеджі сільских робітників не мают придатку в 1--2 екри. А нині той робітник щасливий,
коли має малесенький огородець або коли далеко від хати може де винаймити пару прутів поля.
„Властивці (льорди ґрунтові) і арендаторі (фермери) — каже др. Гонтер — ідут ту рука об руку. Кілька
екрів ґрунту при коттеджу зробилиб робітника занадто незалежним“.

Новий, страшний товчок до насильного вивласнюваня маси народа дала в 16 віці реформація і разом з
нею кольоссальна\footnote*{
В рук. кольоссалька.
} крадіж церковних дібр. Католическа церков до реформації була феодальним
властивцем великої части англійських ґрунтів і земель. На тих ґрунтах сиділо богато чиншовників,
котрих по зруйнованю монастирів поперто між пролєтаріят. Самі церковні добра в великій части
пороздаровувано захланним королівським підлизайкам або за псі гроші попродано
спекулянтам-арендаторам та міщанам, котрі давних, дідичних чиншовників масами попроганяли, а їх
ґосподарства вигирили. Правно забеспечену власність збіднілих селян (їм признана була одна частка
десятин церковних) мовчки загарбано. „Pauper ubique jacet“, сказала королева Єлисавета по обїзді
Англії. В 43. році єї панованя прийшлось вкінци й офіціяльно признати пауперізм через заведенє
податку на бідних. „Установці того податку
— каже Уілліям Коббет в своїй „Історії протестанської реформації“ — встидалися висказати єго причину
і пустили го в світ без усякого вступного виводу (preamble) протів усякого праводавчого звичаю“.
Карло І. признав той податок сталим на завсігди і справді аж 1834 єго підвишшено\footnote{
В Шотляндії скасовано кріпацтво кількасот літ пізнійше, ніж в Англії. Ще 1698 сказав Флєтчер іс
Сельгоуну в шотлянськім парляменті: „Число жебраків в Шотляндії виносит не меньше 200.000. Єдиний
}.
