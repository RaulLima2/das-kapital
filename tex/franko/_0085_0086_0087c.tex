\parcont{}
\index{franko}{0085}
капітал через ужитє наємних робітників і одну часть надвишки витворів, грішми чи натурою, платят
дідичови яко ренту ґрунтову. Доки в 15. віці незалежний мужик, а також сільский наймит, що попри
наймитство й сам про себе веде ґосподарство, збогачуются самі власною працею, доти й обставини тай
обсяг продукційний арендатора остаются дуже скромні. Переворот в рільництві, що почався в послідній
третині 15. віку і трівав через цілий 16 вік крім єго послідних десятиліть, збогатив го майже так
само прудко, як прудко зубожив мужиків\footnote{
„Арендаторі“, каже Гаррізен в своїй „Description of England“, „котрим давнійше годі було
заплатити 4 ф. шт. ренти, платят тепер по 40, 50 та 100 ф. шт. і ще кажут, що їм зле повелося, коли
по упливі арендового контракту не зложили бодай тілько готівки, кілько виносит 6--7-милітна рента“.
}. Загарбанє громадських пасовиск і т. д. дозволяє му
богато побільшувати число худоби майже без ніяких видатків, а між тим худоба достатчувала му далеко
більше обірнику для поправи ґрунту. В 16. віці причинюєсь ще одна рішучо важна обставина. Тоді
арендові контракти були довгі, нераз де з на 99 літ. А ту в 16. віці вартість золота та срібла, а
разом з ним і вартість грошей раз-ураз вменьшуєсь, і арендаторам се принесло золоті плоди. Не
зважаючи на прочі, вперед згадані обставини, арендаторі першим ділом вменьшили робочу плату. Те, що
урвано робітникам на платі, побільшувало
арендовий зиск. А з другого боку ціна збіжя, вовни, мяса, — одим словом, всіх плодів рільничих,
раз-ураз вбільшуєсь, через що змагаєся грошевий капітал арендатора
без єго причинку, — а притім ще ренту ґрунтову дідичови платит він давними, стратившими на вартости,
грішми. Таким способом він збогачуєсь рівночасно на кошт своїх наймитів і свого дідича. Не диво (в
рукоп. „даво“) затим, що вже с кінцем 16. віку витворилась в Англії окрема верства як на тодішні
обставини богатих „капіталістичних“ арендаторів\footnote{
В Франції з „Regisseur-ів“, т. є. панських окономів та тивунів середновікових поробилися швидко
т. зв. hommes d'affaires, т. є. люде, що туманництвом та шахрайством подороблялися капіталів. Такі
окономи, то були нераз великі пани. Як в Англії, так і в Франції великі феодалні добра поділені були
на богато дрібних ґосподарств, але з условинами далеко гіршими для мужиків. В 14. віці повстают і ту
аренди, звані ту „fermes“ або „terriers“. Число їх раз-ураз змагалося і дійшло гет понад
100000. Вони платили чи то грішми чи натурою ренту ґрунтову, котра виносила від 12-тої до 5-тої
части річного здобутку. Ті terriers були цілими або частковими леннами як до вартости і обєму
ґрунтів, котрі нераз виносили заледво кілька прутів. Всі арендаторі мали до певної степені (степенів
було штири) власть судову над мужиками, жиючими
на їх ґрунтах. Лехко понити, якого притиску мусів дізнавати люд від
усіх тих дрібних тиранів. Монтейль каже, що тоді було в Франції 160000
судів, де тепер вистарчає (враз із мировими судами) 4000 трибуналів.
}.
\index{franko}{0086}
\subsection{Вліяніє рільничого перевороту па промисл.
Промисловий капітал здобуває собі в краю ринок
відбутовий.}

Раптове і частими нападами повторюване вивласнюванє
та прогонюванє мужиків достатчило, як ми бачили,
міському промислови раз-за-разом маси пролєтаріїв, не належачих
зовсім до ніяких цехових звязків, — дуже мудра
подія, про котру старший Андерзен (не треба го мішати
з Джемсом Андерзеном) в своїй історії торговлі каже, що
се прямо боже провидініє так зробило. Ще хвилю мусимо
задержатися над тим складником первісного нагромадженя
капіталів. Не тілько що по селах убуло незалежного, самоґосподаруючого
мужицтва, а по містах прибуло промислового
пролєтаріяту, так, як після Жаффроа Сент-Улєра світової
матерії в одних місцях убуває, між тим коли в других
місцях вона згущаєсь. Помимо меньшого числа оброблюючих
рук ґрунт видавав прото однако або й ще більше
плодів, бо разом с переворотом в ґрунтових відносинах
власностевих настали також ліпші способи управи, більша
кооперація, зосередженє средств продукційних і т. д., а з другого
сільські наємники не тілько силувані були до тяжілої
праці — на се головно напирає сер Джемс Стеарт, —
а й обсяг їх домашної продукції, де вони працювали самі
на себе, чим раз більше вменьшувався. З освободженєм
одної части мужицтва освободжені зістали також єго давні
средства прожитку. Вони стают тепер матеріяльним складником
змінного капіталу\footnote*{
Звісно, що Маркс ділит капітал на постійний (constant) і змінний
(variabel), а то після того, чи в довшім протягу продукції вартість
єго зміняєся, чи ні. І так машини, сирий матеріял, будинки фабричні
і т. д., се капітал постійний, бо продукція не змінює в загальній сумі
єго вартости, а то, що убуде вартости на машинах і приладах і пр.,
котрі зуживаются при роботі, прибуває самим витворам, котрі через переробку
зискуют на вартости. Між тим друга часть капіталу, а іменно
тота, котра йде на наймленє і удержано робітника і містится в понятю
робучої плати, се капітал змінний, бо по кождім процесі продукційнім
капіталіст добуває з него більше, ніж видав. Робітник витворює вартість
більшу, ніж тота, яку одержав в формі робучої платп. (Прим. перев.)
}. Бездомний та немаючий мужик
мусит окупувати собі ті средства прожитку від свого
нового пана, промислового капіталіста, в формі робучої
плати. Як зі средствами прожитку, так само сталося й з домашним
рільничим сирим матеріялом, котрого переробкою
займався промисл. Той сирий матеріял став частиною постійного
\index{franko}{0087}
капіталу. Се бачимо не тілько в Англії. За часів
Фрідріха II. бачимо н. пр., що часть вестфальських мужиків,
котрі всі прядут лен, — хоть ще не шовк, — насилу
вивласнено і прогнано з хат і ґрунтів, а прочу часть перемінено
в наймитів великих арендаторів. Рівночасно повстают
великі прядильні і ткальні льну, де „освободжені“ наймаются
на роботу. Лен виглядає так само, як виглядав уперед.
Ані одно волоконце в нім не змінилося, але нова соціяльна
душа вступила в єго тіло. Тепер він становит часть постійного
капіталу панів мануфактуристів. Давнійше розділений
між множество дрібних витвірців, котрі го самі управляли
і пряли, він тепер згромадився в руках одного капіталіста,
котрий других заставляв для себе прясти і ткати. Виложена
в прядильни надвишка праці становила давнійше надвишку
доходу незлічених родин мужицьких, або також — за часів
Фрідріха II, йшла на extra-податки pour le roi de Prusse.
Тепер вона становит зиск немногих капіталістів. Веретена
і ткацькі станки, давнійше розсіяні широко по краю, тепер
стовпилися в кількох великих касарнях робучих, так само
й робітники, так само й сирий матеріял. І веретена і ткацькі
станки і сирі матеріяли зі средств незалежного прожитку
для прядильників і ткачів від тепер переміеюются в средства
командованя над ними і висисаня з них бесплатної
праці. По великих мануфактурах не видно того так, як по
\linebreak[4]
\makebox[\linewidth]{\dotfill}
\makebox[\linewidth]{\dotfill}

\begin{center}
\emph{[На цьому уривається збережений рукопис Франка]}
\end{center}
