а з беззглядною жорстокістю переведена переміна феодальної та окружної (Clan-) власности в новійшу приватну власність, — ось які іділлічні були способи первісного нагромадженя капіталу. Вони здобули ґрунт для капіталістичного рільництва, втягли землю в обсяг капіталу, а міському промислови достатчили потрібних „рук“, т. є. вольного і голого пролєтаріяту.

III.   Кроваві устави протів пролєтаріїв при кінци XV. віку.

Вольний і голий пролєтаріят, вигнаний с хат і ґрунтів через скасованє феодальних дворів і через насильне раз-заразом вивласнюванє, не міг відразу перелятися весь до
новоповстаючих мануфактур так швидко, як швидко сам повстав. А при тімже се були люде, викинені раптово с привичного способу житя, — а такі люде не швидко можут
застосоватися до яких небудь нових, непривичних порядків. На першій порі з них поробилися маси жебраків, розбійників, волоцюг, — деякі з наклінности, а найбільша часть під гнетом обставин. С кінцем XV. і підчас цілого XVI. віку бачимо проте в цілій Західній Европі кроваві устави протів волоцюгів. Батьки нинішної робітницької верстви мусіли на самім вступі відбути страшну кару, — за що? За то, що їх перемінено в волоцюг та голоту. Праводавці вважали їх „добровільними переступцями“ і думали, що тілько від їх доброї волі залежит — працювати далі серед давних обставин, котрі між тим зо світа щезли.

В Англії почалось те праводавство під Генріхом VII.

Генріх VIII., 1530: Старі і неспосібні до праці жебраки одержуют дозвіл на жебрацтво. За то здорові й міцні волоцюги карані будут батогами й арештом. Вони мают бути привязані ззаду до тачок і бичовані доти, доки не поплине кров з їх тіла, — відтак мусят зложити присягу, вернути на місце уродженя або там, де пробули послідні 3 роки, і „засісти до праці“ (to put himself to labour). Що за безсердечна насмішка! В 27 уст. Генріха VIII повторена попередна устава, але заострена новими додатками. Як кого
другий раз зловят на волоцюгованю, то такого бичувати ще раз і відтяти му пів вуха. За третим разом непоправного волоцюгу, як тяжкого злочинця і ворога суспільности —
вкарати смертю.

Едуард VI.: Устава с першого року єго панованя 1547, наказує, що скоро хто отягаєся від праці, той має бути присуджений на невольника тій особі, котра донесла урядови
о єго неробстві. Пан має годувати невольника хлібом і водою, слабими напитками і такими обрізками мяса, які му видадутся відповідними. Він має право всилувати го бато-
