\parcont{}
\index{franko}{0083}
Уайтбрід внесок устави, яка може бути найменьша плата
для робітників рільничих\dots{} Хоть Пітт супротивлявся тому
внескови, то прецінь і сам признав, що „положінє вбогих
страшенне (cruel)“. Вкінци 1813 скасовано устави про реґуляцію
плати. Вони стались смішним недоріцтвом, відколи
капіталіст порядив у своїй фабриці після власних приватних
прав, а плата рільничого робітника давно впала понизше
мінімум конечного до прожитку, і мусіла до висоти
того мінімум доповнюватися с „податку на бідних“. Постанови
„Устави робітницької“ що до згоди між майстром
а наємним робітником, що до вимовленя терміну і т. д.,
постанови дозволяючі тілько цівільну скаргу на недодержуючого
умови майстра, а крімінальну скаргу на недодержуючого
умови робітника, — ті постанови стоят ще й доси
в повній силі. Нелюдські ухвали супротів стоваришень
скасовано 1825 з ляку перед грізною поставою пролєтаріяту.
Парлямент зніс їх дуже нерадо\footnote{
Деякі останки устави протів стоваришень знесено аж 1859 р.
(Додаток до 2. вид.) Устава з 29. жовтня 1871. зносит всі устави
протів стоваришень і урядово признає „Робучі Звязки“ (Trades Unions).
Але в однім додатковім акті с того самого дня, п. н. „An Act to amend
the Criminal Law relating to violence, threats and molestation” — устави
протів стоваришень щасливо воскресли в новій формі. Сесь акт піддає
іменно робітників за вживанє деяких средств воєнних протів майстрів
під окремі устави крімінальні, а судят робітників на підставі тих устав
самі ж майстри, яко мирові судьї. Два роки передтим та сама палата
послів і тот сам Ґлядстон. що 1871 винайшли нові проступки на робітників,
вихвалювали при другім єго читанню один внесок до устави, в котрім
чесним способом роблено конець всяким окремим праводавствам
протів робітників. Вихвалювали, вихвалювали, тай хитро-мудро стали на
другім читанню. (Звісно, що в Англійськім парляменті кождий внесок,
заким одержит силу права, мусит бути три рази читаний і більшістю голосів
принятий. Прим, перев.) Цілі два роки відволікано сю справу, аж
поки „велике ліберальне сторонництво“ не звязалось зі своїми противниками
і не почулося задосить сильним, щоб разом стати — протів спільного
ворога — робітників.
},  той сам парлямент, що
сам довгі столітя с цинічним безвстидством виступав як
неустаюче стоваришенє капіталістів супроті робітників.

Сейчас в початках революційної бурі поквапилась французька
буржоазія інощо здобуте право стоваришень знов
видерти робітникам. В декреті с 14. червня 1791 оголосила
вона, що всі робітницькі стоваришеня, се „замах на свободу
і признані права чоловіка“, за котрий накладаєсь кара
500 ліврів і позбавленє на рік актівних прав горожанських.
Се право, котре конкуренційну боротьбу між капіталом
а працею силою поліційно-державною втискає в такі границі,
які вигідні для капіталу, перетрівало революції та зміни
\parbreak{}
