стійного капіталу. Се бачимо не тілько в Англії. За часів
Фрідріха II. бачимо н. пр., що часть вестфальських мужиків,
котрі всі прядут лен, — хоть ще не шовк, — насилу
вивласнено і прогнано з хат і ґрунтів, а прочу часть перемінено
в наймитів великих арендаторів. Рівночасно повстают
великі прядильні і ткальні льну, де „освободжені“ наймаются
на роботу. Лен виглядає так само, як виглядав уперед.
Ані одно волоконце в нім не змінилося, але нова соціяльна
душа вступила в єго тіло. Тепер він становит часть постійного
капіталу панів мануфактуристів. Давнійше розділений
між множество дрібних витвірців, котрі го самі управляли
і пряли, він тепер згромадився в руках одного капіталіста,
котрий других заставляв для себе прясти і ткати. Виложена
в прядильни надвишка праці становила давнійше надвишку
доходу незлічених родин мужицьких, або також — за часів
Фрідріха II, йшла на extra-податки pour le roi de Prusse.
Тепер вона становит зиск немногих капіталістів. Веретена
і ткацькі станки, давнійше розсіяні широко по краю, тепер
стовпилися в кількох великих касарнях робучих, так само
й робітники, так само й сирий матеріял. І веретена і ткацькі
станки і сирі матеріяли зі средств незалежного прожитку
для прядильників і ткачів від тепер переміеюются в средства
командованя над ними і висисаня з них бесплатної
праці. По великих мануфактурах не видно того так, як по

[На цьому уривається збережений рукопис Франка]
