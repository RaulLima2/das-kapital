притім міцно обезпечували самостійні ґаздівства по селах,
а цехові звязки по містах. По селах і містах не було великої
суспільної ріжниці між майстрами а робітниками.
Підчиненє праці під капітал було тілько формальне, т. є.
продукція сама пе мала ще на собі окремої капіталістичної
ціхи. Попит за наємною працею змагався прото дуже швидко
за кождим нагромадженєм капіталу, — між тим рук готових
найматися до праці прибувало дуже поволи. Велика
часть витворів суспільних, що пізнійше стала фондом вбільшуючим
капітал, тоді переходила ще в руки робітника для
єго власного зужитку.

Праводавство про наємну працю, згори вже вицілене
на визискуванє робітника і в своїм розвитку йому завсігди
однаково неприхильне, почалося в Англії від виданя „Устави
робітницької“ (Statute of Labourers) Едвардом III., 1349.
Рівночасно видано в Франції Указ 1350 р. в імени короля
Жана. Англійські і французькі устави виходят рівнобіжно
і зовсім однакі що до змісту.
Устава робітницька зістала видана за про голосні наріканя
послів. „Давнійше“, каже наївно оден Торі, „жадали
бідні такої великої плати за роботу, що промисл і богацтво
були загрожені. Тепер плата така низька, що знов грозит
промисловії й богацтву і то може ще небеспечнійше ніж
тоді“. Установлено правну тарифу платну для міст і сіл,
за роботу (в рукоп. „робуту“) на дни й від штуки. Сільскі робітники
повинні винайматися на рік, міські „с прилюдного
торгу“. Під карою тюрми заборонено платити висшу плату
від означеної в уставі; а хто бере більшу плату, того кара
виносит більше, ніж сама плата. Так само ще в розд. 18
і 19. устави о учениках ремісницьких, виданої за Єлисавети,
грозится карою 10 день тюрми тому, хто платит більше,
а 21 день тюрми тому, хто бере більшу плату від правом
приписаної. Устава з р. 1360. заострила кари і навіть дала
майстрам право силувати робітників мусом до праці за таку
плату, яка означена в тарифі. Всякі звязки, угоди, присяги
і т. д., котрими взаїмно сполучилися теслі з мулярами,
узнані неважними. Стоваришеня робітницькі караются як
тяжка провина від 14. віку до 1825, в котрім скасовано
устави протів стоваришень. Дух „Робітницької устави“ з р.
1349 і її потомків просвічує ясно й с тих устав протів стоваришень.
Се тота сама засада: держава приписує, кілько
мож найбільше платити робітникови, але хрань боже, щоб
хоть натякнула на те, кілько мож йому найменьше платити!

В 16. віці, як звісно, положінє робітників дуже погіршилося.
Правда, грішми плачено більше, тількож що ціна
прошей стала меньша, а ціна товарів без міри більша. На
ділі затим і плана вменышилася. А прецінь устави для єї
зпиженя трівают далі порівно з обрізуванєм вух та пятно-
