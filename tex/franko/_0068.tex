\index{franko}{0068}
Але сесі беспосередні наслідки реформації не були
пайтривкійші. Церковна власність, се була реліґійна підпора
старосвіцьких порядків ґрунтових. Впала вона, то й їм не
довго було вже встоятись.

Ще в послідних десятилітях 17. віку було джоменів
(самостійних ґаздів хліборобів) більше, ніж арендаторів.
Вони творили головну силу Кромвеля і — як свідчит сам
Маколєй — визначувались дуже корисно супротів роспитих
паничів та їх прислужників — сільских попів. Ще навіть
сільскі наємники були співвластивцями громадського ґрунту.
Аж около 1750. щезли джомени зовсім, а в послідних десятиліттях
18. віку щезли послідні сліди громадських ґрунтів
хліборобських. Ми ту не берем на ввагу чисто економічних
двигачів рільничого перевороту, але глядимо тілько па noсторонні,
насильні товчки.

За реставрації Стюартів перевели великі властивці
ґрунтів правним способом такий самий рабунок, який в прочій
Европі робився і без правних оборотів. Вони знесли
феодальні ґрунтові порядки, т. є. скасували всі ті повинности,
які припадали державі з ґрунтів, „відшкодували“ державу
тим, що наложили податки на хліборобів та прочу
масу народа, а самі забрали в тісну приватну власність усі
добра, над котрими вперед мали лиш феодальну зверхність,
і накинули вкінци народові! такі права осідленя (laws of
settlement), котрі, mutatis mutandis, так само повліяли на
англійських хліборобів, як указ татарина Бориса Ґодунова
на россійських хліборобів.

„Преславна революція“ (glorious Revolution) з Вільгельмом
III Оранським дала панованє в руки ґрунтових та капіталістичних
богатирів. Вони почали нову еру тим, що до
роскраданя державних ґрунтів, котре доси велося скромно
і тайком, взялися тепер на кольосальний розмір. Ті ґрунти
роздаровувано, продавано за песі гроші або й прямо без
даня рації прилучувано до приватних дібр\footnote{
„Безправна рострата коронних дібр чи то через продаж, чи через
роздарованє, становит огидну картку англійської історії\dots{} Се величезне
окраденє народа\dots{}“ (F. W. Newmann: „Lectures on Political Economy.
London, 1851“. стор. 129, 130).
}. Все то робилося
без найменьшої вваги на правні формальности. Ті закрадені
добра державні ураз із церковним фурфантєм, яке

спосіб заради, який можу радити я, републіканець з засади, се — вернути
назад давне кріпацтво і всіх тих поробити невольниками, хто сам
собі не може заробити на прожиток“. А економіст Еден пише: „Від свободи
хлопів почавсь пауперізм\dots{} Мануфактури і торговля, се правдиві
родичі наших нищих“. Еден так, як і тамтой шотлянський „републіканець
з засади“, в тім тілько помиливсь, що не знесенє кріпацтва зробило хлібороба
пролєтаріем та нищим, а загарбанє єго власности.
\parbreak{}
