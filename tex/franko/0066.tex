с панськими. Се підкопало добробуток люду, а через те й міста, церкви, десятини.... Щоб зарадити
тому лиху, проявили король і парлямент дивну на ті часи мудрість.... Вони видали право протів того
обезлюднюючого край загарбуваня громадських ґрунтів (depopulating inclosures) і невідлучної
від него обезлюднюючої ґосподарки толочної (depopulating pasture[s])“. Оден акт Генріха VII. з р.
1489 заказує руйнувати хліборобські хати, до котрих належит що найменьше 20 екрів ґрунту. Генріх
VIII відновив той самий указ. Говорится там між їншим, що „многі аренди і огромні отари, особливо
овець, нагромаджуются в немногих руках, через що дохід
з ґрунту дуже вбільшився, а рільництво дуже підупало, церкви і хати повалено, дивовижні маси народа
стали неспосібні вдержувати себе і свої родини“. Указ наказує затим відбудовувати повалені хутори,
означує, кілько має бути вірного поля в стосунку до овечих толок і т. д. Їнший акт з р. 1533
жалуєсь, що деякі властивці мают по 24000 овець, і ограничує їх число на 2000 6. Наріканя народа і
праводавство протів вивласнюваня дрібних арендаторів та хліборобів, що почалось від Генріха VII і
трівало зо 150 літ
— не помогли нічо. Чому не помогли, пояснює нам Бекон, сам того не знаючи. „Акт Генріха VII, — каже
він в своїх „Essays, civil and moral“, Sect. 20, — був глибоко і дивно обдуманий. Він утворив
сільскі ґаздівства і хліборобські доми певного нормального розміру, т. є. вдержав для них таку
пропорцію ґрунту, котра давала їм змогу плодити на світ підданих доста заможних і не придавлених
нуждою, так що плуг був в руках властивців, а не наємників 7. А між

6 В своїй „Утопії“ говорит Томас Морус про дивовижний край, де
„вівці їдят людей“.

7 Бекон пояснює далі звязок між свобідним, заможним селянством
а доброю інфантерією. „Се була дивно важна річ для сили і мужности
королівства — мати аренди достаточного розміру, щоб дільних мужів
забеспечити від нужди і велику часть ґрунту краєвого запевнити в посіданє джоменам, т. є. людім
середної заможности між шляхтою а халупниками (cottagers) та наймитами. Бо се загальна думка
найліпших знавців воєнного діла.... що головна сила армії, се інфантерія або піхота. Але щоб
витворити добру інфантерію, тре людей вихованих не в притиску ані в нужді, але свобідно і в певній
заможности. Коли затим яка держава вросте переважно в шляхту та делікатне панство, а хлібороби та
ратаї зійдут на простих зарібників та наймитів або халупників, т. є. жебраків з власною хатою, то
така держава може мати добру кінницю, але доброї піхоти не буде мати. Се видно в Італії і Франції і
деяких других заграничних краях, де справді все або шляхта або нужденні зарібники.... Дійшло там до
того, що ті краї мусят уживати наємного зброду Швейцарів та др. для своєї піхоти: відти то й пішло,
що ті держави мают богато людий, а мало вояків“. („The Reign of   Henry VII.“ і т. д.).
