гами та зелізними ланцами до всякої, хотьби й як гидкої роботи. Коли невольник на 14 день
віддалится, то зістає засуджений на віковічну неволю і має бути на чолі або на
лици напятнований буквою S, а коли до трох раз утече, то має бути вкараний смертю, як зрадник
держави. Пан може го продати, передати в наслідство, визичити другому в неволю, зовсім так, як усяке
друге рухоме добро, як худобу. Коли невольники в чім небудь станут супротів панів, то мают також
бути покарані смертю. Мирові судьї повинні за отриманим остереженєм слідити за волоцюгами. Коли
покажеся, що такий волоцюга три дни волочився без діла, то такого відставити на місце, де родився,
роспеченим зелізом папятнувати на груди буквою V і тамій в зелізних ланцюхах
уживати до замітаня вулиці або до якої небудь їншої служби. Коли волоцюга подасть фальшиво місце
вродженя, то за кару має бути віковічним невольником тої громади,
тих мешканців або того товариства і напятнований буквою S. Кождий має право відобрати у волоцюги єго
діти і яко помічників та термінаторів держати хлопців до 24, дівчат до 20 літ. Коли вони втечут, то
мают аж до тих літ бути невольниками майстра, а тому вільно їх заковувати в ланци, бити і пр., як му
сподобаєсь. Кождий пан може заложити зелізну обручку на шию, руку або ногу свого невольника, щоби
міг го ліпше пізнати і бути певним, що му не втече 21). Послідна часть тої устави наказує, щоб
деяких бідних брали на себе громади або поєдинчі люде; ті мают їм давати їсти
й пити і старатись для них о роботу. Тот рід громадських невольників удержувався в Англії гет ще в
19. віці під назвою roundsmen (люде, що ходят від хати до хати).

Єлисавета, 1572: жебраки без дозволу і віком понад 14 літ мают бути без милосердя бичовані і
напятновані на лівім вусі, хіба що їх хто схоче взяти на два роки на службу; в разі повтореня, коли
мают над 18 літ, мают бути — смертю карані, скоро їх ніхто не схоче взяти на два роки на службу; за
третим разом мают без милосердя як зрадники державні бути покарані смертю. Подібна також 18. устава
Єлисавети, розділ 13, і устава з р. 1597 22).

21)    Автор книжки „Essay on Trade and Commerce“ 1770, каже: „Під панованєм Едварда VI. взялись
були Англічане зовсім, здаєсь, серйозно до піддвигненя мануфактур і затрудненя бідних. Се бачимо з
одної дивовижної устави, в котрій приписуєсь, що всі волоцюги мают бути пятновані, і т. д. (Essay on
Trade and Commerce, стор. 8).

22)    Томас Морус каже в своїй „Утопії“: „Так то дієся, що оден захланний і неситий ненаїсник,
правдива чума нашої вітчини, може тисячі екрів ґрунту збити до купи і обпалькувати, обгородити одним
плотом, або силою та кривдою до того довести єго властивців, що вони будут мусіли все спродувати.
Сяким чи таким способом, чи там гнись чи
