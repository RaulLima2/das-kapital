\index{i}{0001}  %% посилання на сторінку оригінального видання
\chapter{Товар і гроші}
\section{Товар}
\subsection{Два фактори товару: споживна вартість і вартість (субстанція
вартости, величина вартости)}

Багатство суспільств, що в них панує капіталістичний спосіб
продукції, з’являється як «величезне нагромадження товарів»,\footnote{
К. Marx: «Zur Kritik der Politischen Oekonomie», Berlin 1859,
S. 4. (K. Маркс: «До критики політичної економії», ДВУ, 1926 р.,
стор. 45).
}
поодинокий товар — як його елементарна форма. Тому наш
дослід починається аналізою товару.

Товар є передусім зовнішній предмет, річ, яка своїми властивостями
задовольняє ті або інші людські потреби. Природа цих
потреб, чи походять вони, приміром, від шлунку, чи від фантазії,
ані трохи не змінює справи.\footnote{
«Бажання має собі за передумову потребу; це апетит духа, і він
для нього так само природний, як голод для тіла\dots{} більша частина (речей)
має свою вартість, тому що вони задовольняють потреби духа» («Desire
implies want; it is the appetite of the mind, and as natural as hunger to
the body\dots{} the greatest number (of things) have treir value from supplying
the wants of the mind»). (Nikolas Barbon: «А Discourse concerning coining
the new money lighter, in answer to Mr. Locke’s Considerations
etc.», London 1696, p. 2, 3).
} Тут також не в тому справа, як саме
річ задовольняє людську потребу: чи безпосередньо, як засіб
існування, тобто як предмет споживання, чи обхідним шляхом,
як засіб продукції.

Кожну корисну річ, як от залізо, папір тощо, можна розглядати
з подвійного погляду: з погляду якости і з погляду кількости.
Кожна така річ є сукупність багатьох властивостей, і тому
може бути корисна з різних боків. Відкрити ці різні боки, а значить,
і різноманітні способи вживання речей — це справа історичного
розвитку.\footnote{
«Речі мають унутрішню властивість (vertue — так специфічно
означує Барбон споживну вартість), що всюди є однакова, як от, приміром,
властивість магнету притягати залізо» («Things have an intrinsick vertue,
which in all places have the same vertue; as the loadstone te attract iron»).
(N. Barbon: «А Discourse concerning coining the new money lighter»,
p. 16). Властивість магнету притягати залізо стала корисною лише тоді,
коли за її допомогою відкрито магнетову полярність.
} Такий характер має й винахід суспільних
\parbreak{}  %% абзац продовжується на наступній сторінці
