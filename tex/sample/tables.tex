\documentclass{kapital}
\begin{document}
Офіціяльні лікарські розсліди показали
згодом, що, навпаки, «пересічний відсоток смертности в
округах шовкової промисловости винятково високий, а серед жіночої
частини людности навіть вищий, ніж в округах бавовняної
промисловости Ланкашіру». \footnote{
Там же, стор. 27. Загалом фізичний стан робітничої людности,
що підлягала фабричному законові, дуже поліпшився. Всі лікарські
свідчення погоджуються в цьому, і мої особисті спостереження різного

Розділ \UkrNumToName{59}

\begin{tiny}
\noindent\begin{tabularx}{\textwidth}{Xrrrr}
  \toprule
  & \multicolumn{4}{c}{\so{Розмір експорту}} \\
  \cmidrule{2-5}
  & \multicolumn{1}{c}{1848 р.} & \multicolumn{1}{c}{1851 р.} &
    \multicolumn{1}{c}{1860 р.} & \multicolumn{1}{c}{1865 р.}\\
  \cmidrule{2-5}
  
  \multicolumn{1}{c}{\emph{Бавовняні фабрики}} \\
  Бавовняна пряжа\dotfill{} & 135.831.162 фун. & 143.966.106 фун. & 197.343.655 фун.  & 103.751.455  фун. \\
  Нитки до шиття\dotfill{} & \makecell{---}    & 4.392.176 фун.   & 6.287.554 фун.    & 4.648.611  фун.\\

  Бавовняні тканини\dotfill{} & 1.091.373.930\samewidth{фун.}{ярд.} & 1.543.161.789\samewidth{фун.}{ярд.} &  2.776.218.427\samewidth{фун.}{ярд.} & 2.015.237.851\samewidth{фун.}{ярд.} \\
  
  \makecell{\emph{Льнопрядні та}\\\emph{коноплепрядні фабрики}} \\
  Пряжа\dotfill{} & 11.722.182 фун. & 18.841.326 фун. &  31.210.612 фун.  &  36.777.334 фун. \\
  Тканини\dotfill{} &  88.901.519 ярд. &   129.106.753 ярд. &   143.996.773 ярд.  &  247.012.329 ярд. \\

  \multicolumn{1}{c}{\emph{Шовкові фабрики}} \\
  Пряжа й нитки\dotfill{} &  194.815 фун. &   462.513 фун.  &    897.402 фун.  & 812.589 фун. \\
  Тканини\dotfill{}       & \multicolumn{1}{c}{---} &    1.181.455 ярд. &   1.307.293 ярд.& 2.869.837 ярд. \\
  
  \multicolumn{1}{c}{\emph{Вовняні фабрики}} \\
  Пряжа\dotfill{}   & \multicolumn{1}{c}{---} &     14.670.880 фун.  &  27.533.968 фун. &31.669.267 фун. \\
  Тканини\dotfill{} & \multicolumn{1}{c}{---} &    151.231.153 ярд.  & 190.371.537 ярд. &278.837.418 ярд. \\
\end{tabularx}
\end{tiny} 
часу переконали мене цього. А все ж, навіть залишаючи осторонь неймовірно
високу смертність дітей у перших роках життя, офіціяльні
звіти д-ра Ґрінхова свідчать про несприятливий стан здоров’я у фабричних
округах порівняно з «рільничими округами з нормальним здоров'ям».
На доказ подаю, між іншим, оцю таблицю з його звіту від 1861 р.:

\newlength{\myheight}
\setlength{\myheight}{10em}

\noindent\begin{tabularx}{\textwidth}{Xccccc}
  \toprule 
  \multicolumn{1}{c}{Назва округ} &
  \rotatebox[origin=c]{90}{\parbox[c]{\myheight}{Відсоток дорослих чоловіків, що працюють у мануфактурі}} & 
  \rotatebox[origin=c]{90}{\parbox[c]{\myheight}{Смертність від нездужання на легені на кожні 100.000 чоловік.}} & 
  \rotatebox[origin=c]{90}{\parbox[c]{\myheight}{Відсоток дорослих жінок, що працюють у мануфактурі}} &
  \rotatebox[origin=c]{90}{\parbox[c]{\myheight}{Смертність від нездужання на легені на кожні 100.000 жінок}} & 
  Рід праці жінок \\

  \midrule
  Wigam\dotfill{}            & 14,9 & 598 & 18,0 & 644 & бавовна \\
  Blackburn\dotfill{}        & 42,6 & 708 & 34,9 & 734 & \ditto{бавовна} \\
  Halifax\dotfill{}          & 37,3 & 547 & 20,4 & 564 & вовна \\
  Bradford\dotfill{}         & 41,9 & 611 & 30,0 & 603 & \ditto{вовна} \\
  Macclesfield\dotfill{}     & 31,0 & 691 & 26,0 & 804 & шовк \\
  Leck\dotfill{}             & 14,9 & 588 & 17,9 & 705 & \ditto{шовк} \\
  Stoke-upon-Trent\dotfill{} & 36,6 & 721 & 19,3 & 665 & ганчарство \\
  Woolstanton\dotfill{}      & 30,4 & 726 & 13,9 & 727 & \ditto{ганчарство} \\
  \noindent\parbox[b]{\hsize}{Вісім здорових рільничих округ\dotfill{}}
  &  --- & 305 &  --- & 340 & --- \\
\end{tabularx}

}

\begin{table}
  \centering
  \caption*{Пересічне число веретен на одну фабрику}
  \begin{tabular}{l}
    Англія\makebox[0.3\textwidth]{\dotfill{}}12.600 \\
    Швайцарія\dotfill{}8.000 \\
    Австрія\dotfill{}7.000 \\
    Саксонія\dotfill{}4.500 \\
    Бельґія\dotfill{}14.000 \\
    Франція\dotfill{}1.500 \\
    Прусія\dotfill{}1.500 \\
  \end{tabular}
\end{table}
\end{document}

