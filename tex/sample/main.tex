\documentclass[12pt, a4paper, final]{memoir}

\usepackage{mathtools}
\everymath{\displaystyle}
\usepackage{xfrac}

\usepackage{unicode-math}
\setmathfont{pt-serif-pro}[
	Path = fonts/ ,
    BoldFont = *-bold.otf ,
	ItalicFont = *-italic.otf ,
	BoldItalicFont = *-bolditalic.otf
]

\usepackage{fontspec}  
\setmainfont{pt-serif}[
	Path = fonts/ ,
    BoldFont = *-bold.ttf ,
	ItalicFont = *-italic.ttf ,
	BoldItalicFont = *-bolditalic.ttf
]

\usepackage{polyglossia}
\setdefaultlanguage{ukrainian}
\setotherlanguages{english}

\usepackage{microtype}

\frenchspacing

\DeclareSymbolFont{cyrletters}{\encodingdefault}{\familydefault}{m}{it}
\newcommand{\makecyrmathletter}[1]{%
  \begingroup\lccode`a=#1\lowercase{\endgroup
  \Umathcode`a}="0 \csname symcyrletters\endcsname\space #1
}
\count255="409
\loop\ifnum\count255<"44F
  \advance\count255 by 1
  \makecyrmathletter{\count255}
\repeat

%\gappto\captionsukrainian{\renewcommand{\chaptername}{Відділ}}
%\renewcommand{\thesection}{Розділ \arabic{section}}

\begin{document}

\chapter{Закон тенденції норми зиску до падіння}
\section{Закон як такий}
При даній заробітній платі і при даному робочому дні змінний капітал, наприклад, в 100, представляє певне число приведених у рух робітників; він є показник цього числа. Припустімо, наприклад, що 100 фунтів стерлінгів становлять заробітку плату 100 робітників, скажімо, за 1 тиждень.\footnote{якийсь текст} Якщо ці 100 робітників виконують стільки ж необхідної праці, скільки додаткової праці, якщо вони, отже, щодня працюють стільки\footnote{A sample\footnotemark{} footnote.}\footnotetext{A subfootnote.} ж часу на себе самих, тобто для репродукції своєї заробітної плати, скільки на капіталістів, тобто для виробництва додаткової вартості, то вся вироблена ними вартість буде = 200 фунтам стерлінгів, а вироблена ними додаткова вартість становитиме 100 фунтів стерлінгів. Норма додаткової вартості $\frac{m}{v}$ була б = 100\%. Однак, ця норма додаткової вартості, як ми бачили, виражалася б у дуже різних нормах зиску, залежно від різного розміру сталого капітаталу $c$, а тому й усього капіталу $К$, бо норма зиску $=\frac{m}{K}$. При нормі додаткової вартості в 100\%,
\begin{align*}
\text{якщо } c&=50, v=100, \text{ то } & р'=\frac{100}{150}=66\frac{2}{3}\%;\\
\text{якщо } c&=100, v=100, \text{ то } &  р'=\frac{100}{200}=50\%;\\
\text{якщо } c&=200, v=100, \text{ то } & р'=\frac{100}{300}=33\frac{1}{3}\%;\\
\text{якщо } c&=300, v=100, \text{ то } & р'=\frac{100}{400}=25\%;\\
\text{якщо } c&=400, v=100, \text{ то } & р'=\frac{100}{500}=20\%.
\end{align*}

Таким чином при незмінному ступені експлуатації праці та сама норма додаткової вартості виражалася б у падаючій нормі зиску, бо разом з матеріальним розміром сталого капіталу зростає, хоч і не в тій самій пропорції, і розмір вартості сталого, а разом з ним і всього капіталу.


Якщо ми далі припустимо, що ця ступнева зміна в складі капіталу відбувається не тільки в окремих сферах виробництва, але більш-менш в усіх або, принаймні, у вирішальних сферах виробництва, так що вона таким чином рівнозначна зміні в пересічному органічному складі сукупного капіталу, належного певному суспільству, то таке ступневе наростання сталого капіталу порівняно з змінним неминуче мусить мати своїм результатом \emph{ступневе зниження загальної норми зиску} при незмінній нормі додаткової вартості, або при незмінному ступені експлуатації праці капіталом. Але виявилось, як закон капіталістичного способу виробництва, що з розвитком цього способу виробництва відбувається відносне зменшення змінного капіталу порівняно з сталим капіталом і, отже, порівняно з усім капіталом, який приводиться в рух. Це означає тільки те, що те саме число робітників, та сама кількість робочої сили, якою можна розпоряджатися при змінному капіталі даного розміру вартості, в наслідок особливих методів виробництва, що розвиваються в капіталістичному виробництві, за той самий час приводить в рух, переробляє, продуктивно споживає постійно зростаючу масу засобів праці, машин і всякого роду основного капіталу, сировинних і допоміжних матеріалів, отже і сталий капітал постійно зростаючого розміру вартості. Це прогресуюче відносне зменшення змінного капіталу порівняно з сталим і, отже, з усім капіталом, тотожне з дедалі вищим пересічним органічним складом суспільного капіталу. Це — так само тільки інший вираз прогресуючого розвитку суспільної продуктивної сили праці, який виявляється саме в тому, що за допомогою зростаючого застосування машин і взагалі основного капіталу при тому самому числі робітників за той самий час, тобто з меншою кількістю і праці, перетворюється в продукти більша кількість сировинних і допоміжних матеріалів. Цьому зростаючому розмірові вартості сталого капіталу — хоч він тільки віддалено представляє зростання дійсної маси споживних вартостей, з яких речево скла-і дається сталий капітал —відповідає зростаюче здешевлення продукту. Кожний індивідуальний продукт, розглядуваний сам по собі, містить у собі меншу суму праці, ніж на нижчому ступені виробництва, де відношення капіталу, витраченого на працю, до капіталу, витраченого на засоби виробництва, є незрівняно більша величина. Отже, гіпотетичний ряд, наведений нами на початку цього розділу, виражає дійсну тенденцію капіталістичного виробництва. Це останнє разом з прогресуючим відносним зменшенням змінного капіталу порівняно з сталим створює дедалі вищий органічний склад сукупного капіталу, безпосереднім наслідком чого є те, що норма додаткової вартості при незмінному і навіть при зростаючому ступені експлуатації праці виражається в дедалі нижчій загальній нормі зиску. (Далі буде показано, чому це зниження виступає не в цій абсолютній формі, а більше в тенденції до прогресивного падіння.) Отже, прогресуюча тенденція загальної норми зиску до зниження є тільки \emph{властивий капіталістичному способові виробництва вираз} прогресуючого розвитку суспільної продуктивної сили праці. Цим не сказано, що норма зиску не може тимчасово падати і з інших причин, але цим доведено, як само собою зрозумілу з суті капіталістичного способу виробництва необхідність, що з розвитком цього способу виробництва загальна пересічна норма додаткової вартості мусить виражатись у падаючій загальній „ нормі зиску. Через те що маса вживаної живої праці постійно зменшується порівняно з масою упредметненої праці, яку вона приводить в рух, порівняно з масою продуктивно споживаних засобів виробництва, то й відношення тієї частини цієї живої праці, яка неоплачена і упредметнюється в додатковій вартості, до розміру вартості всього вживаного капіталу мусить постійно зменшуватись. Але це відношення маси додаткової вартості до вартості всього вживаного капіталу становить норму зиску, яка через це мусить постійно падати.

Хоч і яким простим здається цей закон після того, що ми досі розвинули, проте всій дотеперішній політичній економії не вдалося відкрити його, як ми це побачимо в одному з дальших відділів. Вона бачила явище і мучилася в суперечливих спробах пояснити його. Але при тій великій важливості, яку цей закон має для капіталістичного виробництва, можна сказати, що він становить таємницю, над розв’язанням якої б’ється вся політична економія від часів Адама Сміта, і що ріжниця між різними школами від часів А. Сміта полягає в різних спробах розв’язати цю таємницю. З другого ж боку, якщо взяти до уваги, що дотеперішня політична економія хоч напомац і підходила до розрізнення сталого і змінного капіталу, але ніколи не спромоглась ясно сформулювати його; що вона ніколи не представляла додаткову вартість відокремлено від зиску, а зиск взагалі ніколи не представляла у чистому вигляді в відміну від його різних усамостійнених одна проти одної складових частин, — як промисловий зиск, торговельний зиск, процент, земельна рента; що вона ніколи грунтовно не аналізувала ріжниці в органічному складі капіталу, а тому й утворення загальної норми зиску,— то перестає бути загадковим те, що їй ніколи не вдавалося розв’язати цю загадку.

Ми навмисно виклали цей закон раніше, ніж показали розпад зиску на різні усамостійнені одна проти одної категорії. Незалежність цього викладу від розпаду зиску на різні частини, які припадають різним категоріям осіб, прямо доводить незалежність закону в його всезагальності від такого розпаду і від взаємних відношень між категоріями зиску, які виникають з цього розпаду. Зиск, про який ми тут говоримо, є тільки інша назва самої додаткової вартості, яка тільки представлена у відношенні до всього капіталу, а не у відношенні до змінного капіталу, з якого вона виникає. Отже, падіння норми зиску виражає спадаюче відношення самої додаткової вартості до всього авансованого капіталу, і тому воно незалежне від будьякого розподілу цієї додаткової вартості між різними категоріями.

Ми бачили, що на певному ступені капіталістичного розвитку, коли склад капіталу $c:v = 50:100$, норма додаткової вартості в 100\% виражається в нормі зиску в $66\sfrac{2}{3}\%$ і що на вищому ступені розвитку, коли $c:v$ як $400:100$, та сама норма додаткової вартості виражається в нормі зиску тільки в 20\%. Те, що стосується до різних послідовних ступенів розвитку в одній країні, стосується і до різних ступенів розвитку, які існують одночасно один поряд одного в різних країнах. У нерозвиненій країні, де перший склад капіталу є пересічний, загальна норма зиску була б $=66\sfrac{2}{3}\%$ тимчасом як у країні другого складу капіталу, з значна вищим ступенем розвитку, вона була б $=20\%$.

Ріжниця обох національних норм зиску могла б зникнути і навіть стати протилежною в наслідок того, що в менш розвиненій країні праця була б менш продуктивною, тому більша кількість праці виражалася б у меншій кількості того самого товару, більша мінова вартість виражалася б у меншій споживній вартості, отже, робітник мусив би вживати більшу частину свого часу на репродукцію своїх власних засобів існування або їх вартості і меншу частину на створення додаткової вартості, давав би менше додаткової праці, так що норма додаткової вартості була б нижча. Якщо, наприклад, у менш розвиненій країні робітник працював би $\sfrac{2}{3}$ робочого дня на себе самого і $\sfrac{1}{3}$, на капіталіста, то, зберігаючи припущення вищенаведеного прикладу, та сама робоча сила оплачувалася б у розмірі $133\sfrac{1}{3}$ і дала б надлишок тільки в $66\sfrac{2}{3}$. Змінному капіталові в $133\sfrac{1}{3}$ відповідав би сталий капітал в 50. Отже, норма додаткової вартості становила б тут $133\sfrac{1}{3}:66\sfrac{2}{3}=50\%$, а норма зиску $183\sfrac{1}{3}:66\sfrac{2}{3}$, або приблизно $36\sfrac{1}{2}\%$.

Через те що ми досі ще не дослідили різних складових частин, на які розпадається зиск,— отже, вони для нас ще не існують,— то ми тільки для того, щоб уникнути непорозумінь, зауважимо наперед таке. При порівнянні країн різних ступенів розвитку, а саме країн з розвиненим капіталістичним виробництвом і таких, де праця ще формально не підпорядкована капіталові, хоча в дійсності робітник експлуатується капіталістом (наприклад, в Індії, де райот господарює як самостійний селянин, отже, його виробництво, як таке, ще не підпорядковане капіталові, хоч лихвар може видушити з нього в формі процента не Тільки всю його додаткову працю, але навіть — капіталістично висловлюючись —частину його заробітної плати), було б великою помилкою, коли б хтонебудь схотів міряти висоту національної норми зиску висотою національного рівця процента. В такому проценті міститься весь зиск і навіть більше ніж зиск, тимчасом як у країнах розвиненого капіталістичного виробництва процент виражає тільки відповідну частину виробленої додаткової вартості або зиску. З другого боку, тут рівень процента переважно визначається такими відносинами (позики лихварів знаті, власникам земельної ренти), які не мають нічого спільного з зиском, а, навпаки, показують тільки, в якій мірі лихвар привласнює собі земельну ренту.

В країнах різного ступеня розвитку капіталістичного виробництва і тому різного органічного складу капіталу норма додаткової вартості (один з факторів, що визначають норму зиску) може стояти вище в тій країні, де нормальний робочий день коротший, ніж у тій країні, де він довший. \emph{Поперше}, якщо англійський робочий день у 10 годин в наслідок своєї вищої інтенсивності дорівнює австрійському робочому дневі в 14 годин, то при однаковому розподілі робочого дня 5 годин додаткової праці англійця можуть на світовому ринку представляти вищу вартість, ніж 7 годин австрійця. А \emph{подруге}, в Англії додаткову працю може становити більша частина робочого дня, ніж в Австрії.

Закон спадаючої норми зиску, в якій виражається та сама або навіть зростаюча норма додаткової вартості, означає, інакше кажучи, таке: якщо взяти якусь певну кількість пересічного суспільного капіталу, наприклад, капітал в 100, то частина його, представлена в засобах праці, дедалі зростає, а частина, представлена в живій праці, дедалі зменшується. Отже, через те що вся маса живої праці, додаваної до засобів виробництва, зменшується порівняно з вартістю цих засобів виробництва, то порівняно з вартістю всього авансованого капіталу зменшується також і неоплачена праця і та частина вартості, в якій вона виражається. Або: з усього витраченого капіталу все менша й менша частина перетворюється в живу працю, і тому весь цей капітал вбирає порівняно з своєю величиною все менше й менше додаткової праці, хоч одночасно з цим відношення неоплаченої частини вживаної праці до її оплаченої частини може зростати. Відносне зменшення змінного і збільшення сталого капіталу, хоч обидві ці частини абсолютно зростають, є, як ми вже сказали, тільки інший вираз зростаючої продуктивності праці.

Припустім, що капітал в 100 складається з $80c + 20v$, а ці останні $=20$ робітникам. Норма додаткової вартості нехай буде 100\%, тобто робітники працюють півдня на себе, півдня на капіталіста. Нехай у другій, менш розвиненій країні капітал буде $20c+80v$, і ці останні $=80$ робітникам. Але цим робітникам потрібно $\sfrac{2}{3}$ робочого дня для себе й тільки $\sfrac{1}{3}$ вони працюють на капіталіста. При всіх інших однакових умовах, у першому випадку робітники виробляють вартість в 40, у другому — в 120. Перший капітал виробляє $80c + 20v + 20m=120$; норма зиску $=20\%$; другий капітал $20c + 80v + 40m=140$; норма зиску $=40\%$. Отже, в другому випадку вона вдвоє більша, ніжу першому, хоч у першому випадку норма додаткової вартості, $=100\%$, вдвоє більша, ніж у другому випадку, де вона становить тільки 50\%. Але зате однакової величини капітал привласнює собі в першому випадку додаткову працю тільки 20, а в другому 80 робітників.

Закон прогоесуючого падіння норми зиску або відносного зменшення привласнюваної додаткової праці порівняно з масою упредметненої праці, яка приводиться в рух живою працею, аж ніяк не виключає зростання абсолютної маси праці, яка приводиться в рух і експлуатується суспільним капіталом, а тому й зростання абсолютної маси привласнюваної ним додаткової праці; так само цей закон не виключає того, що капітали, які є в розпорядженні окремих капіталістів, командують дедалі більшою масою праці, а тому й додаткової праці, — останнє навіть у тому випадку, коли число робітників, якими вони командують, не зростає.

Якщо взяти робітниче населення даної чисельності, наприклад, два мільйони, якщо взяти, далі, як дані, довжину і інтенсивність пересічного робочого дня, а також заробітну плату, а разом з тим і відношення між необхідною і додатковою працею, то сукупна праця цих двох мільйонів, а також їх додаткова праця, яка виражається в додатковій вартості, завжди виробляє вартість однакової величини. Але з зростанням маси сталого— основного і обігового — капіталу, який приводиться в рух цією працею, падає відношення цієї величини вартості до вартості цього капіталу, яка зростає разом з його масою, хоч і не в тій самій пропорції. Це відношення, а тому й норма зиску, падає, хоч капітал командує такою самою масою живої праці, як і раніше, і вбирає таку саму масу додаткової праці. Відношення змінюється не тому, що зменшується маса живої праці, а тому, що збільшується маса упредметненої вже праці, яку вона приводить в рух. Зменшення тут відносне, не абсолютне, і в дійсності нічим не зв’язане з абсолютною величиною приведеної в рух праці й додаткової праці. Падіння норми зиску виникає не з абсолютного, а тільки з відносного зменшення змінної складової частини всього капіталу, з її зменшення .порівняно з сталою складовою частиною.

Те саме, що має значення для даної маси праці і маси додаткової праці, має значення і для зростаючого числа робітників, а тому, при даних припущеннях, і для зростаючої маси праці, яка взагалі є в розпорядженні, і зокрема для її неоплаченої частини, для додаткової праці. Якщо робітниче населення зростає з двох мільйонів до трьох, якщо змінний капітал, виплачений йому в формі заробітної плати, так само становив раніше два мільйони, а тепер становить три мільйони, а сталий капітал, .Навпаки, підвищується з 4 до 15 мільйонів, то при даних припущеннях (незмінний робочий день і незмінна норма додатковаИ'нартості) маса додаткової праці, додаткової вартості зростаэ наполовину, на 50\%, з двох мільйонів до трьох. Проте, не зважаючи на це зростання абсолютної маси додаткової праці, а тому й додаткової вартості на 50\%, відношення змінного капіталу до сталого впало б з $2:4$ до $3:15$, і відношення додаткової вартості до всього капіталу було б таке (в мільйонах):
\begin{align*}
\text{I. } & 4c+2v+2m; K=6, p'=33\sfrac{1}{3}\%\\
\text{II. } & 15c+3v+3m; K=18, p'=16\sfrac{2}{3}\%
\end{align*}

Тимчасом як маса додаткової вартості підвищилась наполовину, норма зиску впала наполовину порівняно з попередньою. Але зиск є тільки додаткова вартість, обчислена на суспільний капітал, і тому маса зиску, його абсолютна величина, розглядувана з точки зору всього суспільства, дорівнює абсолютній величині додаткової вартості. Отже, абсолютна величина зиску, його сукупна маса, зросла б на 50°/0, не зважаючи на величезне зменшення цієї маси зиску відносно авансованого сукупного капіталу або не зважаючи на величезне зменшення загальної норми зиску. Отже, число вживаних капіталом робітників, тобто абсолютна маса праці, яка ним приводиться в рух, тому й абсолютна маса вбираної ним додаткової праці, тому й маса виробленої ним додаткової вартості, тому й абсолютна маса виробленого ним зиску може зростати і зростати прогресивно, не зважаючи на прогресивне падіння норми зиску. Це не тільки \emph{може} бути. Це — залишаючи осторонь минущі коливання — \emph{мусить} так бути на базі капіталістичного виробництва.

Капіталістичний процес виробництва є разом з тим істотно і процес нагромадження. Ми показали, як з розвитком капіталістичного виробництва маса вартості, яка мусить бути просто репродукована, збережена, збільшується і зростає разом з підвищенням продуктивності праці, навіть якщо вживана робоча сила лишається незмінною. Але з розвитком суспільної продуктивної сили праці ще більше зростає маса вироблюваних споживних вартостей, частину яких становлять засоби виробництва. А добавна праця, через .привласнення якої це додаткове багатство може бути знову перетворене в капітал, залежить не від вартості, а від маси цих засобів виробництва (включаючи й засоби існування), бо в процесі праці робітник має справу не з вартістю, а з споживною вартістю засобів виробництва. Однак, само нагромадження і дана разом з ним концентрація капіталу є матеріальний засіб підвищення продуктивної сили. Але це зростання засобів виробництва передбачає зростання робітничого населення, створення населення робітників, яке відповідає додатковому капіталові і загалом і в цілому навіть завжди перевищує його потреби, отже, створення перенаселення робітників. Тимчасовий надлишок додаткового капіталу порівняно з робітничим населення яке є в його розпорядженні, справляв би двоякий вплив. З темного боку, він ступнево збільшував би робітниче населення шляхом підвищення заробітної плати, отже, пом’якшенням згубних впливів, що скорочують приріст робітників, і полегшенням шлюбів; а з другого боку, шляхом застосування методів, які створюють відносну додаткову вартість (введення й поліпшення машин), він ще далеко швидше створив би штучне відносне перенаселення, яке з свого боку — бо в капіталістичному виробництві злидні породжують населення,—знов таки є теплицею дійсного швидкого збільшення чисельності населення. Тому з природи капіталістичного процесу нагромадження — який е тільки моментом капіталістичного процесу виробництва — само собою випливає, що збільшена маса засобів виробництва, призначених для перетворення в капітал, завжди знаходить під рукою відповідно збільшене і навіть надлишкове робітниче населення, яке можна експлуатувати. Отже, з розвитком процесу виробництва і нагромадження \emph{мусить} зростати маса придатної до привласнення і привласнюваної додаткової праці, а тому й абсолютна маса зиску, привласнюваного суспільним капіталом. Але ті самі закони виробництва і нагромадження разом з масою сталого капіталу підвищують у дедалі більшій прогресії і його вартість,— швидше, ніж вони підвищують вартість змінної частини капіталу, обмінюваної на живу працю. Отже, одні й ті самі закони зумовлюють для суспільного капіталу зростаючу абсолютну масу зиску і падаючу норму зиску.

Ми тут цілком залишаємо осторонь те, що та сама величина вартості з прогресом капіталістичного виробництва і відповідного йому розвитку продуктивної сили суспільної праці та при помноженні галузей виробництва, отже й продуктів, представляє прогресивно зростаючу масу споживних вартостей і насолод.

Хід розвитку капіталістичного виробництва і нагромадження зумовлює процеси праці в дедалі більшому масштабі, отже, в дедалі більших розмірах, і відповідно до цього зумовлює зростаюче авансування капіталу на кожне окреме підприємство. Тому зростаюча концентрація капіталів (супроводжена в той самий час, хоч і в меншій мірі, зростанням числа капіталістів) є так само однією з матеріальних умов капіталістичного виробництва, і нагромадження, як і одним із створюваних ним самим результатів. Рука в руку і у взаємодії із цим відбувається прогресуюча експропріація більш чи менш безпосередніх виробників. Таким чином для одиничних капіталістів стає зрозумілим, що вони мають у своєму розпорядженні дедалі зростаючі робітничі армії (як би сильно не падав їх змінний капітал порівняно з сталим), що маса привласнюваної ними додаткової вартості, а тому й зиску, зростає одночасно з падінням норми зиску і не зважаючи на це падіння. Якраз ті самі причини, які концентрують маси робітничих армій під командою окремих капіталістів, збільшують також масу за--стосовуваного основного капіталу, як і сировинних та допоміжних матеріалів,— збільшують відносно швидше, ніж масу вживаної жилої праці.

Далі, тут слід тільки згадати, що при даному робітничому населенні, якщо норма додаткової вартості зростає — чи то в наслідок здовження або інтенсифікації робочого дня, чи в наслідок зниження вартості заробітної плати в результаті розвитку продуктивної сили праці, — маса додаткової вартості, а тому й абсолютна маса зиску, мусить зрости, не зважаючи на відносне зменшення змінного капіталу порівняно з сталим.

Той самий розвиток продуктивної сили суспільної праці, ті самі закони, які виражаються у відносному зменшенні змінного капіталу порівняно з усім капіталом і в прискореному разом з цим нагромадженні, тоді як, з другого боку, нагромадження, впливаючи в протилежному напрямі, стає вихідним пунктом дальшого розвитку продуктивної сили і дальшого відносного зменшення змінного капіталу, — цей самий розвиток, залишаючи осторонь тимчасові коливання, виражається в дедалі дужчому збільшенні всієї вживаної робочої сили, в дедалі більшому зростанні абсолютної маси додаткової вартості, а тому й зиску.

В якій же формі мусить виражатися цей двоїстий закон породжуваного одними й тими самими причинами зменшення \emph{норми} зиску і одночасного збільшення абсолютної \emph{маси} зиску? Закон, оснований на тому, що при даних умовах привласнювана маса додаткової праці, отже й додаткової вартості, зростає, І що, коли розглядати сукупний капітал або кожний окремий капітал тільки як частину сукупного капіталу, зиск і додаткова вартість є тотожні величини?

Візьмімо певну частину капіталу, на яку ми обчислюємо норму зиску, наприклад, 100. Припустім, що ці 100 представляють пересічний склад сукупного капіталу, скажімо, $80v+20c$. В другому відділі цієї книги ми бачили, яким чином пересічна норма зиску в різних галузях виробництва визначається не особливим складом капіталу кожної з них, а його пересічним суспільним складом. З відносним зменшенням змінної частини порівняно з сталою, і, отже, порівняно з усім капіталом в 100, норма зиску при незмінному і навіть зростаючому ступені експлуатації праці падає, падає відносна величина додаткової вартості, тобто відношення її до вартості всього авансованого капіталу в 100. Але падає не тільки ця відносна величина. Величина додаткової вартості або зиску, що його вбирає весь капітал в 100, падає абсолютно. При нормі додаткової вартості в 100\% капітал в $60c + 40v$ виробляє масу додаткової вартості, а тому й зиску, в 40; капітал в $70c + 30v$ виробляє масу зиску в ЗО; при капіталі в $80c+20v$ зиск падає до 20. Це падіння стосується до маси додаткової вартості, а тому й зиску, і випливає з того, що оскільки весь капітал в 100 приводить в рух менше живої праці взагалі, а при незмінному ступені експлуатації і менше додаткової праці, він виробляє менше додаткової вартості. Коли якусь частину суспільного капіталу, отже, капіталу пересічного суспільного складу, взяти за одиницю міри для виміряння додаткової вартості,—а це робиться при всякому обчисленні зиску, — то взагалі відносне падіння додаткової вартості і її абсолютне падіння є тотожні. Норма зиску в наведених вище випадках знижується з 40\% до 30\% і до 20\%, бо в дійсності маса додаткової вартості, а тому й зиску, вироблена тим самим капіталом, падає абсолютно з 40 до 30 і до 20. Через те що величина вартості капіталу, відносно якої вимірюється додаткова вартість, є дана, $=100$, то зменшення відношення додаткової вартості до цієї незмінної величини може бути тільки іншим виразом зменшення абсолютної величини додаткової вартості й зиску. Справді, це — тавтологія. Але те, що таке зменшення настає, випливає, як уже було показано, з природи розвитку капіталістичного процесу виробництва.

Але, з другого боку, ті самі причини, які викликають абсолютне зменшення додаткової вартості, а тому й зиску на даний капітал, а тому також і обчислюваної в процентах норми зиску, ці самі причини приводять до зростання привласнюваної суспільним капіталом (тобто сукупністю капіталістів) абсолютної маси додаткової вартості, а тому й зиску. Як же це мусить виразитись, як це може виразитись, або які умови передбачаються цією позірною суперечністю?

Якщо кожна відповідна частина, $=100$, суспільного капіталу, отже, кожні 100 капіталу пересічного суспільного складу, є величина дана, і тому для неї зменшення норми зиску збігається І із зменшенням абсолютної величини зиску саме через те, що тут капітал, яким вони вимірюються, є величина стала, то, ‘навпаки, величина сукупного суспільного капіталу, як і капіталу, який знаходиться в руках окремих капіталістів, є змінна величина, яка, щоб відповідати припущеним умовам, мусить змінюватись у зворотному відношенні до зменшення своєї змінної частини.

В попередньому прикладі, при процентному складі капіталу в $60c + 40v$, Додаткова вартість або зиск на капітал був 40, а тому й норма зиску була 40\%. Припустім, що при цій висоті складу сукупний капітал становив один мільйон. В такому разі сукупна додаткова вартість, а тому й сукупний зиск становив 400000. Якщо потім склад буде $=80c + 20v$, то при незмінному І тупені експлуатації праці додаткова вартість, або зиск, на кожні 100 $=20$. Але через те що додаткова вартість, або зиск, як ми їюказали, щодо своєї абсолютної маси зростає, не зважаючи на цю падаючу норму зиску або дедалі менше створення додаткової вартості кожною сотнею капіталу,— наприклад, зростає, скажімо, з 400000 до 440000,— то це можливе тільки тому, що сукупний Капітал, який утворився одночасно з цим новим складом, зріс До 2 200000. Маса приведеного в рух сукупного капіталу зросла До 220\%, тимчасом як норма зиску впала на 50\%. Коли б капітал тільки подвоївся, то при нормі зиску в 20\% він міг би Іиробити тільки таку саму масу додаткової вартості й зиску, як старий капітал в 1000000 при нормі зиску в 40\%. Коли б він зріс менше ніж удвоє, то він виробив би менше додаткової вартості або зиску, ніж раніше виробляв капітал в 1000000, який для того, щоб при своєму попередньому складі підвищити свою додаткову вартість з 400000 до 440000, мав би зрости з 1000000 тільки до 1100000.

Тут виявляється вже раніш викладений закон, що з відносним зменшенням змінного капіталу, отже, з розвитком суспільної продуктивної сили праці, потрібна все більше зростаюча маса всього капіталу, щоб приводити в рух ту саму кількість робочої сили і вбирати ту саму масу додаткової праці. Отже, в тій самій мірі, в якій розвивається капіталістичне виробництво, розвивається можливість появи відносно надлишкового робітничого населення, не тому, що продуктивна сила суспільної праці зменшується, а тому, що вона збільшується, тобто не в наслідок абсолютної невідповідності між працею і засобами існування або засобами виробництва цих засобів існування, а в наслідок невідповідності, яка виникає з капіталістичної експлуатації праці, невідповідності між прогресуючим ростом капіталу і його відносно меншаючою потребою в зростанні населення.

Якщо норма зиску падає на 50\%, то вона падає наполовину. Тому, щоб маса зиску лишилась та сама, капітал мусить подвоїтися. Для того, щоб при падаючій нормі зиску маса зиску лишилась незмінною, множник, який показує зростання всього капіталу, має бути рівний дільникові, який показує падіння норми зиску. Якщо норма зиску падає з 40 до 20, то для того, щоб результат лишався попереднім, весь капітал мусить, навпаки, зрости у відношенні $20:40$. Коли б норма зиску впала з 40 до 8, то капітал мусив би зрости у відношенні $8:40$, тобто вп’ятеро. Капітал в 1000000 при 4О\% виробляє 400000 і капітал в 5000000 при 8\% виробляє так само 400000. Таке зростання потрібне для того, щоб результат лишався попереднім. Навпаки, для того щоб результат збільшився, капітал мусить зростати в більшій пропорції, ніж падає норма зиску. Іншими словами: для того, щоб змінна складова частина всього капіталу не тільки абсолютно лишалась попередньою, але й абсолютно зростала, хоч її процентне відношення до всього капіталу падає, весь капітал мусить зростати в більшій пропорції, ніж падає процентне відношення змінного капіталу до всього капіталу. Він мусить зростати настільки, щоб при його новому складі йому потрібна була для купівлі робочої сили не тільки стара змінна частина капіталу, але ще більша за неї. Якщо змінна частина капіталу, рівного 100, падає з 40 до 20, то весь капітал мусить зрости більше, ніж до 200, для того щоб можна було вжити змінний капітал більший, ніж 40.

Навіть коли б експлуатована маса робітничого населенші лишалась незмінною і збільшилася б тільки довжина і інтенсивність робочого дня, маса вживаного капіталу мусила б зрости, бо вона мусить зрости навіть для того, щоб при зміненому складі капіталу можна було вживати ту саму масу праці при попередніх відношеннях експлуатації.

Отже, той самий розвиток суспільної продуктивної сили праці виражається з прогресом капіталістичного способу виробництва, з одного боку, в тенденції до прогресуючого падіння норми зиску, а з другого боку, в постійному зростанні абсолютної маси привласнюваної додаткової вартості або зиску; так що загалом відносному зменшенню змінного капіталу і зиску відповідає абсолютне \textbf{збільшення} обох. Ця двобічна дія, як ми вже показали, може виразитись тільки в зростанні всього капіталу в швидшій прогресії, ніж та, в якій падає норма зиску. Для того, щоб при вищому складі капіталу або при відносно сильнішому збільшенні сталого капіталу можна було вжити абсолютно зрослий змінний капітал, весь капітал мусить \textit{зрости} не тільки відповідно до вищого складу, але ще швидше. З цього випливає, що чим більше розвивається капіталістичний спосіб 
\[виробництва\]
, тим більша й більша маса капіталу потрібна для того, щоб уживати ту саму робочу силу, і ще більша для того, щоб \(уживати\) вирослу робочу силу. Отже, зростаюча продуктивна сила праці на капіталістичній базі з необхідністю створює постійне позірне перенаселення робітників. Якщо змінний капітал становить тільки $\sfrac{1}{6}$ всього капіталу замість колишньої $\sfrac{1}{2}$, то, щоб можна було вжити ту саму робочу силу, весь капітал мусить потроїтись; а для того, щоб можна було вжити подвійну робочу силу, він мусить пошестеритись.

Дотеперішня політична економія, яка не зуміла була пояснити закон падіння норми зиску, вказувала на підвищення маси зиску, зростання абсолютної величини зиску, чи то для окремих капіталістів, чи для суспільного капіталу, як на свого роду підставу для утішення, але й вона базується на самих тільки загальних місцях і можливостях.

Те, що маса зиску визначається двома факторами, поперше, нормою зиску і, подруге, масою капіталу, вжитого для одержання цієї норми зиску,— це просто тавтологія. Тому та обставина, що зростання маси зиску можливе, не зважаючи на одночасне падіння норми зиску, є тільки вираз цієї тавтології і не допомагає ні на крок посунутися вперед, бо цілком так само можливе й те, що капітал зростатиме без зростання маси зиску і що він може навіть зростати і в тому випадку, коли вона падає. 100 при 25\% дає 25, 400 при 5\% дає тільки 20. 
\footnote{„We should also expect that, however the rate of the profits of stock might diminish in consequence of the accumulation of capital on the land and the rise of wages, yet the aggregate amount of profits would increase. Thus, supposing that, with repeated accumulations of 100 000 the rate of profits should fall from 20 to 19, to 18, to 17 per cent., a constantly diminishing rate; we should expect that the whole amount of profits received by those successive owners of capital would be always progressive; that It would be greater when the capital was 200000 than when 100000 still greater when 300000; and so on, increasing, though at a diminishing rate, with every increase of capital. This progression, howewer, is only true for a certain time; thus, 19 per cent, on 200 000 £ is more than 20 on 10000£, again 18 per cent on 300000 is more than 19 per cent, on 200000£ but after capital has accumulated to a large amount, and profits have fallen, the further accumulation diminishes the aggregate of profits. Thus, suppose the accumulation should be 1 00 0 000 £, and the profits 7 per cent, the whole amount of profits will be 70000; now if an addition of 100000 £ capital bemade tothemilllon, and profits should fall to 6 per cent., 66 000 or a diminution of 4000 £ will be received by the owners of stock, although the whole amount of stock will be increased from 1 000 £ to
1 100000£.“ [„Нам слід, отже, сподіватися, що хоча норма зиску на капітал може зменшитися в наслідок нагромадження капіталу в країні і підвищення заробітної плати, однак загальна сума зиску збільшиться. Так, якщо ми припустимо, що при послідовному нагромадженні 100000 фунтів стерлінгів норма зиску впаде з 20\% До 19\%, до 18\% і до 17\%, тобто постійно зменшуватиметься, то слід сподіватися, що вся сума зиску, одержувана цими послідовними власниками капіталу, постійно зростатиме; що вона буде більша при капіталі в 200 000 фунтів стерлінгів, ніж при капіталі в 100000 фунтів стерлінгів, і ще більша при капіталі в 300 000 фунтів стерлінгів і т. д., зростаючи з кожним збільшенням капіталу, не зважаючи на зменшення норми. Однак, таке зростання має місце тільки на протязі певного часу; так, 19\% від 200000 фунтів стерлінгів є більше, ніж 20\% від 100000 фунтів стерлінгів, 18\% від 300000 фунтів стерлінгів знов таки більше, ніж 19\% від 200000 фунтів стерлінгів; але після того, як капітал уже нагромадився до великої суми, а зиски зменшились, дальше нагромадження зменшує загальну суму зиску. Так, якщо припустимо, що нагромадження становить 1000000 фунтів стерлінгів, а зиск 7\%. то загальна сума зиску становитиме 70000 фунтів стерлінгів; якщо тепер до капіталу в мільйон буде додано 100 000 фунтів стерлінгів і зиск знизиться до 6\%, то власники капіталу одержать 66000 фунтів стерлінгів, або на 4000 фунтів стерлінгів менше, хоч загальна сума капіталу зросла з 1000000 фунтів стерлінгів до 1 100000 фунтів стерлінгів“. Ricardo: .Principles of Political Economy“, розд. VII (.Works“ вид. Мак-Куллоха, 1852, стор. 68 [69]). В дійсності тут припускається, що капітал зростає з 1000 000 до 1 100000, тобто на 10\%, тимчасом як норма зиску падає з 7 до 6, тобто на 14\%, Hlnc illae lacrimae [звідси ці сльози].}
Але якщо ті самі причини, які викликають падіння норми зиску, сприяють нагромадженню, тобто утворенню додаткового капіталу, і якщо кожен додатковий капітал приводить в рух добавну працю і виробляє добавну додаткову вартість; якщо, з другого боку, просте зниження норми зиску включає вже і той факт, що сталий капітал, а тому й весь старий капітал, зріс,— то весь цей процес перестає бути таємничим. Ми далі побачимо, до яких умисних фальшувань в обчисленнях вдаються для того, щоб по-шахрайському відкинути можливість збільшення маси зиску при одночасному зменшенні норми зиску.

Ми показали, як ті самі причини, які викликають тенденцію загальної норми зиску до падіння, зумовлюють прискорене нагромадження капіталу, а тому й зростання абсолютної величини або загальної маси привласнюваної ним додаткової праці (додаткової вартості, зиску). Як усе в конкуренції, а тому й у свідомості агентів конкуренції, так і цей закон — я маю на думці цей внутрішній і необхідний зв’язок між двома явищами, які, як здається, одне одному суперечать—виступає у перекрученому вигляді. Очевидно, що в межах вищенаведених пропорцій капіталіст, який розпоряджається великим капіталом, одержує більшу масу зиску, ніж дрібний капіталіст, який, видимо, одержує високий зиск. Далі, найповерховіше спостереження конкуренції показує, що при певних обставинах, коли більший капіталіст хоче захопити для себе місце на ринку, витиснути дрібніших капіталістів,— як, наприклад, за часів кризи,— він використовує це практично, тобто навмисно знижує свою норму зиску, щоб витиснути з ринку дрібніших капіталістів. Так само й купецький капітал — про який ми пізніше скажемо докладніше — показує явища, завдяки яким зниження зиску здається наслідком розширення підприємства, а разом з тим і капіталу. Власне науковий вираз замість помилкового розуміння ми дамо пізніше. Подібні поверхові погляди є результатом порівнення норм зиску, одержуваних в окремих галузях підприємств залежно від того, чи підпорядковані вони режимові вільної конкуренції чи монополії. Цілком банальне уявлення, яке створюється в головах агентів конкуренції, ми знаходимо в нашого Рошера, а саме, що таке зниження норми зиску є „розумніше й гуманніше“. Зменшення норми зиску представлено тут як наслідок збільшення капіталу і зв’язаного з цим розрахунку капіталістів, що при меншій нормі зиску маса зиску, яку вони кладуть собі в кишеню, буде більша. Все це (за винятком того, що є в А. Сміта, про що пізніше) основане на цілковитому нерозумінні того, щб таке взагалі є загальна норма зиску, і на тому грубому уявленні, що ціни дійсно визначаються шляхом надбавки більш-менш довільної частки зиску до дійсної вартості товарів. Хоч які грубі ці уявлення, все ж вони з необхідністю виникають з того перекрученого способу й вигляду, в якому імманентні закони капіталістичного виробництва виявляються в сфері конкуренції.
\end{document}