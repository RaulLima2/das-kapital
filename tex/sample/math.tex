\documentclass{kapital}
\begin{document}

   Книжку набрано гарнітурами Alegreya ht (2011) \frac{X}{X} та Alegreya Sans ht (2013), які спроектував аргентинський дизайнер XXXXXXXXXXXXXXXXXXX Хуан Пабло дел Перал. Математичні знаки набрані гарнітурою STIX2Math, що розроблена в рамках ініціативи Scientific and Technical Information Exchange (STIX) font creation project. Книжку набрано гарнітурами Alegreya ht (2011)  та Alegreya Sans ht (2013), які спроектував аргентинський дизайнер  Хуан Пабло дел Перал. Математичні знаки набрані гарнітурою STIX2Math, що розроблена в рамках ініціативи Scientific and Technical Information Exchange (STIX) font creation project.

   Формули $1$1 $Т$\emph{Т}та їх вигляд \emph{x}$x = 20 : 40 $ з іншого боку $Т - т + г - Г - Т'$

   Дробь в текстовом 1\sfrac{1}{2} $\frac{A}{A}$\frac{A}{A} режиме $1\sfrac{48}{56}$ и 3\frac{\text{asd}}{400} и продолжение.

   \splitfrac{Р}{Зп} $\frac{\text{Р}}{\text{Зп}}$

   тгп Дробь $—$— в текстовом 1\sfrac{1}{2} $\frac{1}{2}$\frac{1}{2} режиме $1\sfrac{48}{56}$ и 3\frac{\text{asd}}{400} и продолжение.

   \[
   	1 + 1\frac{48}{56}=5
   \]

   Text $Г − Т \splitfrac{Р}{Зп} \dots{} П \dots{} (Т + т) — (Г + г)$

   то матимемо: \[
M = 
\begin{cases}
\frac{m}{v} × V & \\
k × \frac{a'}{a} × 1\frac{48}{56} × n & \\
\end{cases}
\]

   % \index{i}{0446}  %% посилання на сторінку оригінального видання 
Розділ шістнадцятий

Різні формули норми додаткової вартости

Ми бачили, що норма додаткової вартости виражається в таких
формулах:

I. додаткова вартість/змінний капітал (m/v) =
додаткова вартість/вартість робочої сили =
додаткова праця/доконечна праця

Дві перші формули виражають у формі відношення вартостей
те саме, що третя виражає у формі відношення відтинків часу,
що протягом їх ці вартості продукується. Ці формули, що одна
одну доповнюють, є строго логічні. Тим то ми находимо їх у клясичній
політичній економії, правда, щодо суті, але виробленими
несвідомо. Зате ми бачимо там такі вивідні формули:

II. додаткова праця\footnote*{
У французькому виданні Маркс заводить цю формулу в дужки
і дає до цього таку примітку: «Ми заводимо першу формулу в дужки,
бо ясно вираженого поняття додаткової праці ми не знаходимо в буржуазній
політичній економії». Ред.
}/робочий день =
додаткова вартість/вартість продукту =
додатковий продукт/сукупний продукт

Ту саму пропорцію виражено тут навпереміну то у формі
робочих часів, то у формі вартостей, що в них вони втілюються,
то у формі продуктів, що в них існують ці вартості. Звичайно,
припускається, що під вартістю продукту треба розуміти лише
вартість, новоспродуковану протягом робочого дня, а сталу частину
вартости продукту виключено.

У всіх цих формулах дійсний ступінь експлуатації праці, або
норму додаткової вартости, виражено неправильно. Хай робочий
день буде 12 годин. Якщо інші припущення нашого попереднього
прикладу лишаються незмінні, то в цьому випадку дійсний
ступінь експлуатації праці виразиться в таких пропорціях:

6    годин додаткової праці/6 годин доконечної праці =
додаткова вартість у 3 шилінґи/змінний капітал у 3 шилінґи = 100\%.

Навпаки, за формулою II ми маємо:

6 годин додаткової праці/робочий день у 12 годин =
додаткова вартість у 3 шилінґи/новоспродукована вартість у 6 шилінґів =
50\%.

Ці вивідні формули в дійсності виражають ту пропорцію,
що в ній робочий день або новоспродукована протягом нього
\parbreak{}  %% абзац продовжується на наступній сторінці

   \parcont{}  %% абзац починається на попередній сторінці
\index{i}{0447}  %% посилання на сторінку оригінального видання
вартість поділяються між капіталістом і робітником. Тому, якщо
розглядати їх як безпосередні вирази ступеня самозростання
капіталу, то дійшлося б такого неправильного закону: додаткова
праця або додаткова вартість ніколи не може досягти 100\%.\footnote{
Див., наприклад, «Dritter Brief an v. Kirchmann von Rodbertus.
Widerlegung der Ricardoschen Theorie von der Grundrente und Begründung
einer neuen Rententheorie», Berlin 1854. Я пізніше повернуся
до цього твору, який, не вважаючи на його хибну теорію земельної ренти,
доходить суті капіталістичної продукції. — [Додаток до третього видання.
— Ми бачимо тут, як доброзичливо цінував Маркс своїх попередників,
коли находив у них якийсь справжній крок наперед, якусь вірну
нову думку. Тимчасом опубліковані листи Родбертуса до Руд. Маєра
обмежують до певної міри вищенаведене визнання. Там читаємо: «Треба
врятувати капітал не тільки від праці, але й від себе самого, а цього в
дійсності можна найкраще досягти, якщо розглядати діяльність капіталіста-підприємця як народньо- й
державногосподарську функцію,
покладену на нього капіталістичною власністю, а його дохід — як певну
форму утримання, бо ми ще не знаємо ніякої іншої соціяльної організації.
Але утримання повинні бути вреґульовані і знижені, коли вони
забагато відбирають від заробітної плати. Таким способом слід також
відбити напад Маркса на суспільство — так я назвав би його книгу...
Взагалі Марксова книга — це не так дослід про капітал, як полеміка
проти теперішньої форми капіталу, яку він сплутує з самим поняттям
капіталу, з чого саме й постають його помилки». («Briefe usw. von Dr.
Rodbertus-Jagetzow, herausgegeben von Dr. Rud. Meyer», Berlin 1881,
Bd. I, S. 111, 48. Brief von Rodbertus). У таких ідеологічних банальностях
зникають справді сміливі напади Родбертусових «соціяльних листів».
— Ф. Е.].
}
А що додаткова праця завжди може становити лише певну частину
робочого дня, або додаткова вартість — лише певну частину
новоспродукованої вартости, то додаткова праця завжди є неодмінно
менша, ніж робочий день, або додаткова вартість завжди
є менша, ніж новоспродукована вартість. Але для того, щоб
відноситися одна до однієї як 100/100 вони мусили б бути між собою
рівні. Щоб додаткова праця забрала цілий робочий день (тут
мова йде про пересічний день робочого тижня, робочого року
й т. ін.), доконечна праця мусила б упасти до нуля. Але коли
зникає доконечна праця, то зникає й додаткова праця, бо остання
є лише функція першої. Отже, пропорція додаткова праця/робочий день =
додаткова вартість/новоспродукована вартість ніколи не може досягти межі
100/100, а ще менше — підвищитися до (100 + х)/100. Але це цілком можлива річ
для норми додаткової вартости, або дійсного ступеня експлуатації
праці. Візьмімо, приміром, обчислення пана Л. де Ляверня,
що за ним англійський рільничий робітник дістає лише чвертину,
а капіталіст (фармер) — три чверті продукту\footnote{
Ту частину продукту, яка тільки покриває витрачений сталий
капітал, само собою зрозуміло, з цього обчислення виключено. — Пан
Л. де Лявернь, сліпий прихильник Англії, подає скорше надто низьке,
ніж високе відношення.
} або його вартости, не-
\parbreak{}  %% абзац продовжується на наступній сторінці

\end{document}
