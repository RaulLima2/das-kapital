Що в період кризи бракує платіжних засобів, це очевидно само собою.
Перетворність *) векселів у гроші заступила місце метаморфози самих товарів,
і то саме за таких часів то більше, що більше частина торговельних фірм працює
тільки на кредит. Невігласне та недоладне банкове законодавство як от
років 1844—45, може зробити цю грошову кризу тяжчою. Але жодне банкове
законодавство не в стані тієї кризи усунути.

При такій системі продукції, коли всі взаємозв’язки процесу репродукції спирається
на кредиті, то якщо кредит раптом припинено та має силу лише платіж готівкою,
очевидно, мусить наступати криза, надзвичайно велика гонитва за платіжними
засобами. Тому на перший погляд вся криза здається лише кредитовою кризою
та грошовою кризою. І справді, справа — лише в перетворності векселів у гроші.
Але ці векселі представляють, здебільша, дійсні купівлі та продажі, що їхній пошир,
який значно пересягає межі суспільної потреби, кінець-кінцем, лежить в основі
всієї кризи. Однак поряд цього величезна маса цих векселів представляє лише
шахрайські операції, що виходячи тепер на денне світло, врахують; далі — спекуляції,
ведені та ще й нещасливо на чужий капітал; насамкінець — товарові
капітали, що знецінилися або й зовсім не можуть бути продані; або зворотний приплив
капіталів, що фактично вже ніколи не відбудеться. Всю цю штучну систему
ґвалтовного поширу процесу продукції не можна, звичайно, вилікувати тим, що
якийсь банк, напр., Англійський банк, дасть — у своїх паперах — усім спекулянтам
капітал, що його їм бракує, та купить усі знецінені товари за їхні старі
номінальні вартості. Проте все тут виявляється перекручено, бо в цьому паперовому
світі ніде не виступає реальна ціна та її реальні моменти, а тільки
зливки, металеві гроші, банкноти, векселі, цінні папери. Це перекручення
виявляється особливо в центрах, де, як от в Лондоні, зосереджено всі грошові
підприємства країни; весь процес стає незрозумілим; вже менше помічається це
в центрах продукції.

Проте, з приводу того надміру промислового капіталу, що виявляється підчас
криз, треба зауважити ось що: товаровий капітал сам про себе є одночасно
грошовий капітал, тобто певна сума вартости, висловлена в ціні товару. Як споживча
вартість, є він певна кількість певних речей споживання, що їх підчас
кризи є понад міру. Але як грошовий капітал сам про себе, як потенціяльний
грошовий капітал, він зазнає повсякчас поширу та скорочення. Напередодні кризи та
протягом її товаровий капітал скорочується в своїй властивості, як потенціяльний
грошовий капітал. Для своїх державців та їхніх кредиторів (а так само як і забезпечення
для векселів та позик) він становить менше грошового капіталу, ніж тоді,
коли його скуповували та коли робилось засновані на ньому дисконтові та заставні
операції. Коли такий має бути зміст твердження, що грошовий капітал певної
країни підчас скрути меншає, то є це тотожне з тим, що ціни товарів спали.
Однак такий крах цін тільки вирівнює їхнє колишнє набубнявіння (Aufblähung).

Доходи непродуктивних кляс та тих, що живуть з сталих доходів, лишаються,
здебільша, незмінні підчас такого набубнявіння цін (Preisaufblähung), що
розвивається поряд надмірної продукції та надмірної спекуляції. Тому їхня спожиткова
спроможність відносно меншає, а разом з тим меншає й їхня здібність повертати

різноманітніші ділянки, так що капітал розподіляється далеко більше, а місцеву надмірну спекуляцію
легше перемогти. В наслідок цього всього більшість старих огнищ криз і нагод до утворення криз
усунено
або дуже зменшено. Поряд цього конкуренція на внутрішньому ринку відступає перед картелями та
трестами, тим часом коли на зовнішньому ринку її обмежує охоронне мито, що ним оточили себе всі
великі
промислові країни, крім Англії. Але це саме охоронне мито становить не що інше, а тільки озброєння
для остаточної загальної промислової війни, що має вирішити справу панування на світовому ринку.
Оттак кожен з тих елементів, що діє проти повторювання колишніх криз, ховає в собі зародок далеко
більш могутньої майбутньої кризи. — Ф. Е].

*) Перетворність, нім. Konvertibilität, властивість і можливість перетворюватись, переходити
з одного стану в інший. Пр. Ред.
