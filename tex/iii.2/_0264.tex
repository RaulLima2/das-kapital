\parcont{}  %% абзац починається на попередній сторінці
\index{iii2}{0264}  %% посилання на сторінку оригінального видання
Отже, з товарів. Але вартість товарів, за нашим припущенням, визначається
в першу чергу ціною праці, що продукує їх, заробітною платою. Заробітна плата
є тут передумова і розглядається як елемент, що конституює ціну товарів. Таким
чином, ця ціна повинна бути визначена відношенням подання праці до капіталу.
Ціна самого капіталу дорівнює ціні товарів, що з них він складається. Попит
капіталу на працю дорівнює поданню товарів. А подання капіталу дорівнює
поданню суми товарів даної ціни, а цю ціну в першу чергу реґулюється
ціною праці, а ціна праці й собі знову таки дорівнює тій частині товарової
ціни, з якої складається змінний капітал, що віддається робітникові, в
обмін на його працю; а ціна товарів, що з них складається цей змінний капітал
в першу чергу знов таки й собі визначається ціною праці, бо вона визначається
цінами заробітної плати, зиску й ренти. Отже, щоб визначити заробітну плату,
ми не можемо виходити з капіталу як з передумови, бо вартість самого капіталу
співвизначається заробітною платою.

Крім того, притягування до цієї справи конкуренції нам не допомагає.
Конкуренція призводить до того, що ринкові ціни праці підвищуються
або падають. Але припустімо, що попит і подання праці взаємно урівноважуються.
Чим буде визначатись тоді заробітна плата? Конкуренцією. Але ж
ми саме припустили, що конкуренція перестала діяти, як визначальний чинник,
— що, в наслідок рівноваги її обох протилежно спрямованих сил, її вплив
знищено. Адже ми якраз хочемо знайти природну ціну заробітної плати,
тобто не ту ціну праці, яка регулюється конкуренцією, а ту, яка навпаки
реґулює її.

Не лишається нічого іншого, як визначити потрібну ціну праці потрібними
засобами існування робітника. Але ці засоби існування є товари, що мають
певну ціну. Отже, ціна праці визначається ціною потрібних засобів існування,
а ціна засобів існування, як і ціна всіх інших товарів, визначається в першу
чергу ціною праці. Отже, визначена ціною засобів існування ціна праці визначається
кінець-кінцем ціною праці. Ціна праці визначає саму себе. Інакше кажучи,
ми не знаємо, чим визначається ціна праці. Праця має тут взагалі ціну,
бо її розглядається як товар. Отже, щоб говорити про ціну праці, ми мусимо
знати, що таке взагалі ціна. Але що таке ціна взагалі, ми цим шляхом якраз
і не можемо дізнатись.

Але припустімо, що цим, таким втішним, способом ми визначили потрібну
ціну праці. Як же стоїть справа з пересічним зиском, з зиском кожного капіталу
в нормальних умовах, з зиском, який становить другий елемент ціни товару?
Пересічний зиск мусить бути визначений пересічною нормою зиску; а чим визначається
ця остання? Конкуренцією між капіталістами? Але конкуренція вже має
своєю передумовою існування зиску. Вона має своєю передумовою різні
норми зиску, отже і різні зиски в тих самих або різних галузях продукції. Конкуренція
може впливати на норму зиску лише остільки, оскільки вона впливає на
ціни товарів. Конкуренція може призвести лише до того, що продуценти в межах
даної сфери продукції продають свої товари по однакових цінах, а в межах різних
сфер продукції — по цінах, що дають їм однаковий зиск, тобто однакову пропорційну
додачу до ціни товару, вже почасти визначеної заробітною платою.
Конкуренція може тому лише вирівнювати ріжницю в нормі зиску. Для того,
щоб можливо було вирівнювати неоднакові норми зиску, зиск як елемент товарової
ціни мусить уже бути в наявності. Конкуренція його не створює. Вона
підвищує або понижує, але не створює того рівня, який встановлюється, коли
вирівняння дійсно настало. І коли ми говоримо про потрібну норму зиску, ми
хочемо пізнати саме ту норму зиску, яка незалежна від руху конкуренції, яка
сама з свого боку реґулює конкуренцію. Пересічна норма зиску здійснюється
коли настає рівновага, між силами конкурентних капіталістів. Конкуренція може
\parbreak{}  %% абзац продовжується на наступній сторінці
