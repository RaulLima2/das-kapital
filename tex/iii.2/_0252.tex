\parcont{}  %% абзац починається на попередній сторінці
\index{iii2}{0252}  %% посилання на сторінку оригінального видання
які витрачається дохід, тобто правити за засоби споживання, перебігають протягом
року різні ступені, наприклад, вовняна пряжа, сукно. На одному ступені
вони становлять частину сталого капіталу, на другому — їх особисто споживається,
отже, цілком, входять в склад доходу. Можна, отже, уявити собі разом
з А. Смітом, що сталий капітал є лише позірний елемент товарової вартости,
який в загальному зв’язку зникає. Таким самим чином відбувається далі обмін
змінного капіталу на дохід. Робітник купує на свою заробітну плату частину
товарів, що становить його дохід. Одночасно він покриває цим самим для капіталіста
грошову форму змінного капіталу. Нарешті: частина продуктів, що становлять
сталий капітал, покривається або in natura, або за посередництвом обміну
між самими продуцентами сталого капіталу; процес, до якого споживачі
не мають жодного чинення. Коли спустити це з уваги, то може постати зовнішня
видимість, що дохід споживачів покриває ввесь продукт, отже і сталу частину
вартости.

5) Крім плутанини, яку вносить перетворення вартостей на ціни продукції,
виникає ще й інша в наслідок перетворення додаткової вартости на різні окремі
форми доходу, самостійні одна проти однієї і залічені до різних елементів продукції,
на зиск і ренту. При цьому забувається, що вартості товарів є основою, і що
розпадання цієї товарової вартости на окремі складові частини, і дальший розвиток
цих складових частин вартости у форми доходу, їх перетворення на відносини
різних посідачів різних чинників продукції до цих окремих складових
частин вартости, їх розподіл між цими посідачами згідно з певними категоріями
і титулами, нічого не змінює у самому визначенні вартости й законів її.
Так само мало змінюється закон вартости тією обставиною, що вирівнювання
зиску, тобто розподіл сукупної додаткової вартости між різними капіталами, і
перешкоди, що почасти (в абсолютній ренті) ставляться землеволодінням на
шляху цього вирівнювання, призводять до відхилу регуляційних пересічних цін
товарів від їхніх індивідуальних вартостей. Це впливає знов таки тільки на
добавку додаткової вартости до цін різних товарів, але не знищує самої додаткової
вартости і сукупної вартости товарів як джерела цих різних складових
частин ціни.

Тут перед нами quid pro quo, яке ми розглядаємо в дальшім розділі, і
яке неминуче зв’язане з ілюзією, що нібито вартість виникає з її власних
складових частин. А саме: спершу різні складові частини вартости товару набувають
в доходах самостійних форм, і як такі доходи їх залічують до окремих
речових елементів продукції, як до їхніх джерел, замість залічити їх до
вартости товару як до їхнього джерела. Вони дійсно залічуються до зазначених
окремих джерел, але не як складові частини вартости, а як доходи, як складові
частини вартости, що дістаються цим певним категоріям агентів продукції: робітникові,
капіталістові, земельному власникові. А проте, можна уявити собі, що
ці складові частини вартости замість виникати від розкладу товарової вартости,
навпаки, лише створюють її своїм сполученням; тоді саме й виникає чудове
порочне коло: вартість товарів виникає з суми вартости заробітної плати, зиску,
ренти, а вартість заробітної плати, зиску, ренти в свою чергу визначається
вартістю товарів і т. ін.\footnote{
«Оборотний капітал, витрачений на матеріяли, сировий матеріял і викінчені вироби, сам
складається з товарів, що їх потрібна ціна складена з тих самих елементів; так що, розглядаючи
сукупність товарів у певній країні, було б зайво зачислювати цю частину оборотного капіталу до
елементів потрібної ціни». (Storch, Cours d’Ec. Pol., II, p. 140). Під цими елементами оборотного
капіталу Шторх розуміє (основний — це тільки змінена форма оборотного) сталу частину вартости.
«Правда, що заробітна плата робітника так само як і частина зиску підприємця, яка складається з
заробітних плат, коли розглядати їх як частину засобів існування, і собі складається з товарів, що
}.
