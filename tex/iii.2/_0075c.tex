\index{iii2}{0075}  %% посилання на сторінку оригінального видання
Повосьме. Відпливи металу, здебільша, є симптом зміни в стані закордонної
торговлі, а ця зміна, своєю чергою, є ознака того, що знову достигають
умови для кризи\footnote{
За Newmarch’oм відплив золота закордон може поставати з причин троякого роду, а саме:
1) від суто-комерційних причин, тобто тоді, коли довіз був більший, ніж вивіз, як то було між 1836
та 1844 роками та знову в 1847 роді, коли був головне значний довіз збіжжя; 2) від того, що треба
добувати засоби для приміщення англійського капіталу закордоном, як то було в 1857 роді, коли
будували
залізниці в Індії; та 3) від того, що остаточно витрачається кошти закордоном, як от в 1853
та 1854 роках на військові справи на Сході.
}.

Подев’яте. Платіжний баланс може бути сприятливий для Азії й несприятливий
для Европи й Америки\footnote{
1918. Newmarch «Якщо ви візьмете Індію та Китай разом, якщо ви візьмете на увагу
обороти між Індією та Австралією та ще важливіші обороти між Китаєм та Сполученими Штатами — а в
цих випадках торговля є трибічна й вирівнюються рахунки за нашим посередництвом... тоді слушно,
що торговельний балянс був несприятливий не тільки для Англії, але й для Франції та Сполучених
Штатів». (В. А 1857).
}.

Довіз благородного металу відбувається переважно при двох моментах.
З одного боку, за тієї першої фази низького рівня проценту, що настає по
кризі та є вияв обмеження продукції; а потім за другої фази, коли рівень проценту
підноситься, але ще не досягнув своєї середньої висоти. Це — фаза, коли
зворотні припливи капіталів відбуваються легко, комерційний кредит великий,
а тому й попит на позичковий капітал зростає непропорційно поширові продукції.
В обох фазах, коли позичкового капіталу є порівняно багато, надмірний
приплив капіталу, що існує в формі золота та срібла, отже, в такій формі,
в якій він насамперед може функціонувати лише як позичковий капітал, — мусить
значно виливати на рівень проценту, а тому й на загальний розвиток справ.

З другого боку: відплив, невпинний значний вивіз благородного металу
настає тоді, коли вже постають труднощі щодо зворотного припливу капіталів,
коли ринки переповнені, а подоба розцвіту зберігається тільки за допомогою
кредиту; отже, коли вже є дуже збільшений попит на позичковий капітал, а тому
й рівень проценту вже досяг, принаймні, своєї середньої величини. Серед цих
обставин, що відбиваються у відпливі саме благородного металу, значно більшає
вплив невпинного витягування капіталу в такій формі, в якій він існує безпосередньо
як грошовий позичковий капітал. Це мусить безпосередньо впливати
на рівень проценту. Але замість обмежувати кредитові операції, це піднесення
рівня проценту поширює їх та приводить до надмірного напруження всіх їхніх
допоміжних засобів. Тому цей період передує крахові.

Newmarch’а питають (В. А. 1857): «1520. Отже, число векселів у циркуляції
зростає разом з рівнем проценту? — Здається, так. — 1522. За спокійних
звичайних часів головна книга є дійсне знаряддя обміну; але коли постають
труднощі, коли, напр., серед таких обставин, що мною вище були наведені, підвищується
норма банкового дисконту... тоді операції сходять цілком сами собою
до виписування векселів; ці векселі не тільки придатніші до того, щоб бути за
законний доказ зробленої справи, але вони й зручніші для дальших закупів
і насамперед їх можна уживати, як засіб кредиту, щоб визичати капітал», —
До цього долучається те, що, коли банк серед до певної міри загрозливих обставин
підвищує свою норму дисконту, а в наслідок цього одночасно стає ймовірним,
що банк обмежить реченець векселів, котрі йому доводиться дисконтувати,
— постає загальне побоювання, що це буде розвиватися crescendo\footnote{
Crescendo. від лат. cresco, — росту, чим раз більшаючи, зростаючи. Пр. Ред.
}. Отже,
\parbreak{}  %% абзац продовжується на наступній сторінці
