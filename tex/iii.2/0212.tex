ком іншого характеру. Петті, Кантільйон, взагалі письменники, що ближче стоять
до доби февдалізму, беруть земельну ренту як нормальну форму додаткової
вартости взагалі, тимчасом як зиск для них ще не визначився і поєднується
з заробітною платою, або, щонайбільше, виступає як та частина цієї
додаткової вартости, що її капіталіст витискує з земельного власника. Отже, вони
виходять з такого стану, коли, поперше. хліборобська людність становить ще рішучу
переважну частину нації, і коли, подруге, земельний власник ще є тією
особою, яка, користуючись монополією земельної власности, у першу чергу привласнює
надмірну працю беспосередніх продуцентів, коли, отже, земельна власність
все ще є головна умова продукції. Для них ще не могло існувати такої
постави питання, що, навпаки, з погляду капіталістичного способу продукції,
намагається дослідити, яким чином земельна власність досягає того, що
віднімає від капіталу частину спродукованої ним (тобто вичавленої з безпосередніх
продуцентів) і в першу чергу привласненої вже ним додаткової вартости.

У фізіократів труднощі вже іншого характеру. Як дійсно перші систематичні
тлумачі капіталу, вони намагалися аналізувати природу додаткової вартости
взагалі. Для них ця аналіза збігається з аналізою ренти, однісінької
форми, в якій для них існує додаткова вартість. Капітал, що дає ренту, або
хліборобський капітал, є для них однісінький капітал, що продукує додаткову
вартість, і пущена ним в рух хліборобська праця є однісінька, що створює додаткову
вартість, отже, з капіталістичного погляду цілком послідовно однісінька
продуктивна праця. Продукцію додаткової вартости вони цілком слушно вважають
за визначальний момент. Їм, залишаючи осторонь інші заслуги, про які мова
буде в книзі ІV, належить насамперед та велика заслуга, що від торговельного
капіталу, який функціонує тільки в сфері циркуляції, вони звернулись до продуктивного
капіталу, протилежно до меркантильної системи, яка за своїм грубим
реалізмом була справжньою вульґарною економією тієї доби, що її практичними
інтересами було відсунуто цілком на задній плян початки наукової аналізи
у Петті та його послідовників. Між іншим, тут, при критиці меркантильної системи,
мова йде лише про її погляди на капітал та додаткову вартість. Вже
давніш ми відзначали, що продукцію на світовий ринок і перетворення продукту
на товар, а тому і на гроші, монетарна система справедливо проголосила за передумову
і умову капіталістичної продукції. В її продовженні, в меркантильній системі,
переважну ролю відіграє вже не перетворення товарової вартости на гроші, а створення
додаткової вартости, але розглядається воно з іраціонального погляду сфери
циркуляції, до того ж так, що ця додаткова вартість виступає в формі додаткових
грошей, в надмірі торговельного балансу. Разом з тим справді характеристичне
для заінтересованих купців і фабрикантів того часу і адекватне тому періодові
капіталістичного розвитку, який вони репрезентують, є те, що при перетворенні
хліборобських февдальних громад на промислові, і при відповідній промисловій
боротьбі націй на світовому ринку, справа залежить від прискореного розвитку
капіталу, що досягається не так званим природним шляхом, а примусовими заходами.
Величезна ріжниця є в тому, чи перетворюється національний капітал на промисловий
поступово і повільно, чи це перетворення прискорюється в часі, в наслідок податків,
що ними вони в формі охоронних мит оподатковували переважно земельних
власників, середніх і дрібних селян і ремесло, в наслідок прискореної експропріяції
самостійних безпосередній продуцентів, в наслідок насильницької прискореної
акумуляції і концентрації капіталів, коротко, в наслідок прискореного
створення умов капіталістичного способу продукції. Разом з тим це становить
величезну ріжницю в капіталістичній і промисловій експлуатації природної національної
продуктивної сили. Тому національний характер меркантильної системи
в устах її оборонців є не просто фраза. З тієї притоки, що їх ніби цікавить тільки
багатство нації та допоміжні ресурси держави, вони в дійсності проголошують
