Пошосте. За винятком хіба 1837 року, дійсна криза завжди вибухала
тільки по зміні вексельного курсу, тобто скоро довіз благородного металу знову
переважав вивіз.

В 1825 році дійсний крах постав по тому, як відплив золота припинився.
В 1839 році відбувався відплив золота, не викликавши краху В 1847 році
відплив золота припинився в квітні, а крах стався в жовтні. В 1857 році відплив
золота закордон припинився на початку листопада, а крах стався лише
пізніше в листопаді. Особливо виразно виявляється це в кризі 1847 року, коли
відплив золота спинився вже в квітні, викликавши відносно легку передкризу,
а власне криза в справах вибухла потім тільки у жовтні.

Далі наведені свідчення складено перед Secret Committee of the House of
Lords on Commercial Distress 1848; викази свідків (evidence) видруковано тільки
в 1857 році (цитовано їх теж як: С. D. 1848/57).

Свідчення Тука. В квітні 1847 року постала скрута, що, власне кажучи,
дорівнювала паніці, але триваючи порівняно недовго, не супроводилась жодними
більш-менш значними комерційними банкрутствами. В жовтні, коли скрута була
далеко інтенсивніша, ніж за будь-який період у квітні, сталося майже нечуване
число комерційних банкрутств (2190). — В квітні вексельні курси, особливо в операціях
з Америкою, змусили нас експортувати значну кількість золота для оплати
незвичайно великого імпорту; тільки з незвичайно енергійним напруженням банк
припинив відплив золота та підвищив курс. (2197) — В жовтні вексельні курси
були сприятливі для Англії. (2198) — Зміна їх почалася на третім тижні квітня.
(3000). — Вони коливались у липні та серпні; від початку серпня вони були
увесь час сприятливі для Англії (3001). Відплив золота в серпні постав від
попиту для внутрішньої циркуляції.

J. Morris, управитель Англійського банку: Хоч вексельний курс, починаючи
від серпня 1847 р., став сприятливий для Англії і з тієї причини відбувався довіз
золота, проте, металевий запас у банку меншав. «2200.000 ф. ст. золотом пішли
з банку в країну в наслідок внутрішнього попиту». (137). — Це пояснюється,
з одного боку, збільшенням числа робітників, що мали працю коло будування
залізниць, а з другого боку, «бажанням банкірів мати підчас кризи власний
золотий запас» (І47).

Palmer, колишній управитель, а від 1811 року директор Англійського банку:
«684. Протягом цілого періоду від середини квітня 1847 року аж до
дня припинення чинности банкового акту 1844 року вексельні курси були
сприятливі для Англії».

Отже, відплив металу, що в квітні 1847 року викликав лише саму грошову
паніку, є тут, як і скрізь, тільки попередник кризи й перетворився на
приплив металу раніше, ніж та криза вибухла. В 1839 році при значному пригніченні
в справах відбувався дуже значний відплив металу — за збіжжя і т. ін., —
але без кризи та грошової паніки.

Посьоме. Скоро загальні кризи закінчились, золото й срібло — коли не
вважати на приплив нового благородного металу з країн його продукції —
знову розподіляються в тих відношеннях, в яких вони, як осібний скарб
були в різних країнах за часів рівноваги. За всіх інших незмінних обставин
відносна величина скарбу в кожній країні визначається ролею її на світовому
ринку. З країни, що має золота й срібла понад норму, вони відпливають, припливаючи
до іншої; ці рухи припливу та відпливу відновлюють тільки первісний
розподіл їх проміж скарбами різних націй. Проте, цей перерозподіл здійснюється
під впливом різних обставин, що їх згадується при розгляді вексельного курсу.
Скоро відновлено нормальний розподіл, то з цього момену настає спочатку зростання,
а потім знову настає відплив. [Ця остання теза має силу, звичайно,
тільки для Англії, як осередку світового грошового ринку. — Ф. Е.].
