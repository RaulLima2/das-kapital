покоєння, навіть справжня паніка, не було поставлено жодної межі для видання
банкнот, яке тільки й могло полегшити цей скрутний стан».

Так висловлюється людина, що протягом 39 років сиділа в дирекції Англійського
банку. Послухаймо тепер приватного банкіра Twells’a, що, починаючи
від 1801 року, є спільник в фірмі Spooner, Attwoods and С°. З-поміж усіх свідків
перед В. С. 1857 року він єдиний дає змогу виробити погляд на дійсний стан
країни та бачить наближення кризи. Щодо решти його поглядів, то є він певний
ґатунок бірмінґемських Little-Shilling теоретиків, як і його спільники, брати
Attwood’и, що ту школу заснували, (див. Zur Kritik der pol. Ök. стор. 59). Він
каже: «4488. Як впливав, на вашу думку, акт 1844 року? — Коли б мав я
відповідати як банкір, то я сказав би, що він мав цілком прегарний вплив,
давши банкірам та [грошовим] капіталістам всякого роду багатий урожай. Але
він мав дуже лихий вплив для чесного працьовитого ділка, що йому потрібна
була сталість норми дисконту, щоб він міг з певністю робити свої обрахунки...
він, той акт, зробив з визичання грошей операцію, надзвичайно прибуткову, —
4489. Він [банковий акт] дає змогу лондонським акційним банкам виплачувати
акційникам 20—22%? — Один банк нещодавно виплатив 18%; другий, думається
мені, 20%; вони мають всяку підставу дуже рішуче боронити той акт, — 4490. Дрібні ділки та поважні
купці, що не мають великого капіталу... їх він
дуже тисне. Однісінький засіб, що його я маю, щоб дізнатися цього, той, що я
бачу дивовижну масу їхніх неоплачених акцептів. Ці акцепти завжди невеликі,
приблизно з 20—100 ф. ст., багато з них не оплачено та вертаються неоплачені
назад в усі частини країни, а це завжди є ознака пригнічення серед...
дрібних торговців», — 4494. Він заявляє, що тепер справи не прибуткові. Дальші
його уваги є важливі, бо він бачив заховане існування кризи тоді, коли ще
ніхто з решти не прочував її.

«4494. Ціни на Mincing Lane’i тримаються ще досить добре, але нічого
не продається, не можна ні за яку ціну продати; тримаються лише номінальні
ціни». — 4495. Він оповідає про один випадок: якийсь француз послав маклерові
на Mincing Lane’i товарів на 3.000 ф. ст. на продаж за певну ціну. Маклер
не в стані одержати ту ціну, француз не в стані продати нижче від тієї
ціни. Товар лежить, але французові треба грошей. Отже, маклер позичає йому 1.000 ф. ст., так що
француз виставляє на маклера тримісячний вексель на 1.000 ф. ст. під забезпечення товарів. По трьох
місяцях надходить тому векселеві
реченець платежа, але товарів ще й досі не можна продати. Тоді маклерові
доводиться оплатити той вексель, і хоч він має покриття на 3.000 ф. ст.,
проте, він не може його перетворити на готівку та опиняється в скрутному
стані. Оттак один тягне за собою другого до загину. — 4496. Щождо значного
вивозу... коли справи всередині країни в пригніченому стані, то це неминуче
викликає й міцний вивіз. — 4497. Чи, на вашу думку, внутрішнє споживання зменшилось?
— Дуже значно... цілком надзвичайно... дрібні торговці тут
найкращий авторитет. — 4498. А проте, довіз дуже великий; чи не вказує це
на велике споживання? — Так, якщо ви маєте змогу продати; але багато
товарових складів повні девізного товару; в тому прикладі, що про нього я щойно
оповідав, імпортовано було на 3.000 ф. ст. товарів, і їх не можна було продати.

«4514. Якщо гроші дорогі, — то чи не скажете ви, що тоді капітал дешевий?
— Так.» — Отже, ця людина зовсім не поділяє Оверстонової думки, що
високий рівень проценту є те саме, що дорогий капітал.

Як тепер провадять справи: 4516... «Інші дуже хапаються, роблять
величезні експортові та імпортові операції далеко понад ту міру, що її дозволяє
їхній капітал; про це не може бути ані найменшого сумніву. Цим людям
може в тих справах пощастити; вони можуть через якусь щасливу пригоду
надбати велике майно та все поплатити. Це до великої міри така система, що
