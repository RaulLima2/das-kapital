\parcont{}  %% абзац починається на попередній сторінці
\index{iii2}{0118}  %% посилання на сторінку оригінального видання
специфічну особливість ренти (і взагалі хліборобського продукту) те, що на основі
товарової продукції — і точніше капіталістичної продукції, яка в усьому обсязі
своєму є товарова продукція — в дійсності є спільне всім галузям продукції
і всім їхнім продуктам.

З поступом суспільного розвитку величина земельної ренти (і разом з нею
і вартість землі) розвивається як наслідок усієї суспільної праці. Поруч з цим
зростає, з одного боку, ринок і попит на продукти землі, з другого боку — попит
безпосередньо на саму землю, як конкурентну умову продукції для всеможливих.
також і нехліборобських галузей підприємств. Точніше — рента, коли мова
йде лише про власне хліборобську ренту, а разом з нею і вартість землі, розвивається
рівнобіжно з ринком для продукту землі, а тому з ростом нехліборобської
людности; з її потребами і попитом почасти на харчові засоби, почасти
на сирові матеріяли. В природі капіталістичного способу продукції лежить те
явище, що він постійно скорочує хліборобську людність проти нехліборобської,
бо в промисловості (у вузькому розумінні) зростання сталого капіталу проти
змінного зв’язане з абсолютним зростом — хоч відносним зменшенням — змінного
капіталу; тимчасом як у хліборобстві змінний капітал, потрібний для експлуатації
певної дільниці землі, зменшується абсолютно; отже, може зростати лише остільки,
оскільки оброблюється нова земля, а це знову таки має за свою передумову ще
більший зріст нехліборобської людности.

В дійсності тут немає явища специфічно своєрідного для хліборобства та
його продуктів. На основі товарової продукції і її абсолютної форми, капіталістичної
продукції, це явище ще більше має силу щодо всіх інших галузей продукції
та продуктів.

Ці продукти є товари, споживні вартості, що мають мінову вартість, і саме
мінову вартість, яку можна реалізувати, перетворити на гроші, лише в тій мірі,
в якій інші товари становлять еквівалент їх, в якій інші товари протистоять їм, як
товари і як вартості; отже, в тій мірі, в якій вони продукуються не як безпосередні
засоби існування самих їхніх продуцентів, а як товари, як продукти,
що стають споживними вартостями лише через перетворення їх на мінову
вартість (гроші), через їх відчуження. Ринок для цих товарів розвивається
за посередництвом суспільного поділу праці; відокремлення продуктивних робіт
перетворює їх відповідні продукти взаємно на товари, на еквіваленти у відношенні
один до одного, заставляє їх взаємно відігравати ролю ринку один для
одного. Тут немає нічого специфічно своєрідного для хліборобських продуктів.

Рента може розвинутися як грошова рента лише на основі товарової
продукції, точніше — капіталістичної продукції, і вона розвивається в тій самій
мірі, в якій хліборобська продукція стає товаровою продукцією; отже, в тій самій
мірі, в якій нехліборобська продукція розвивається як самостійна щодо неї продукція,
бо в тій самій мірі хліборобський продукт стає товаром, міновою вартістю
і вартістю. В тій самій мірі, як з капіталістичною продукцією розвивається товарова
продукція, а тому продукція вартости, — розвивається продукція додаткової
вартости і додаткового продукту. Але в тій самій мірі, як розвивається
остання, розвивається і здібність земельної власности, користуючись своєю монополією
на землю, захоплювати чим раз більшу частину цієї додаткової вартости,
а тому й підвищувати вартість своєї ренти, і ціну самої землі. В розвитку
цієї додаткової вартости і додаткового продукту капіталіст є ще самодіяльний
аґент. ІЦождо землевласника, то він має лише захоплювати ростучу таким
чином без його участи частину додаткового продукту і додаткової вартости.
Ось у чому є спеціфічна особливість його становища, а не в тому, що вартість
продуктів землі, а тому і землі, дедалі більше зростає в тій мірі, в якій
поширюється ринок для них, зростає попит, а разом з ним і світ товарів, що
протистоїть продуктові землі, отже, іншими словами, в тій мірі, в якій зростає
\parbreak{}  %% абзац продовжується на наступній сторінці
