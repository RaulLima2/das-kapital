Р і к    Банкноти в 5—10 ф. ст. Відсотки    Банкноти в 20—100 ф. ст. Відсотки    Банкноти в 200—1000
ф. ст. Відсотки    Разом  фунтів  ст.
1844         9263    45,7    5735    28,3    5253    26,0    20241
1845         9698    46,9    6082    29,3    4942    28,6    20723
1846         9918    48,9    5771    28,5    4590    22,6    20286
1847         9591    50,1    5498    28,7    4066    21,2    19155
1848         8732    48,3    5046    27,9    4307    23,8    18085
1849         8692    47,2    5234    28,5    4777    24,3    18403
1850         9164    47,2    5587    28,8    4646    24,0    19398
1851        9362    48,8    5554    28,5    4557    23,4    19473
1852        9839    45,0    6161    28,2    5856    26,8    21856
1853         10699    47,3    6393    28,2    5541    24,5    22653
1854         10565    51,0    5910    28,5    4234    20,5       20709
1855         10628    53,6    5706    28,9    3459    17,5    19793
1856         10680    54,4    5645    28,7    3324    16,9    19648
1857         10659    54,7    5567    28,6    3241    16,7    19467

Як значно зменшено ужиток грошей в гуртовій торговлі до невеличкого
мінімуму, про це свідчить таблиця, надрукована в книзі І, розд. III, примітка 103,
що подана банковій комісії фірмою Morrison Dillon and С°, однією з тих найбільших
лондонських фірм, де дрібний торговець може закупити ввесь свій запас
товарів усякого роду.

За свідченням W. Newmarch’a перед банковою комісією 1857 р. № 1741,
заощадженню засобів циркуляції сприяли ще й інші обставини: поштовий тариф
в 1 пенс (das Penny-Briefporto), залізниці, телеграфи, коротко — поліпшені засоби
комунікації; так що Англія тепер має змогу при приблизно тій самій циркуляції
банкнот робити у п’ятеро-шестеро більш операцій. Але до цього значно
спричинилося теж виключення з циркуляції банкнот, більших за 10 ф. ст. 1 це
здається йому природним поясненням того, що в Шотляндії та Ірландії, де в
циркуляції є навіть банкноти в 1 ф. ст., циркуляція банкнот зросла приблизно
на 31% (1747). Вся сума банкнот, що є в циркуляції у Сполученому Королівстві,
включаючи й банкноти в 1 ф. ст., становить, хай 39 мільйонів ф. ст.
(1749). Сума золота в циркуляції-70 мільйонів ф. ст. (1750). В Шотляндії
в 1834 році було в циркуляції банкнот 3.120.000 ф. ст.; в 1844 році —
3.020.000 ф. ст.; в 1854 році — 4.050.000 ф. ст. (1752).

Вже з цього випливає, що збільшення числа банкнот в циркуляції ніяк
не в волі банків, що видають банкноти, поки ці банкноти можна кожного часу
розмінювати на грші (Goeld). [Про паперові гроші, що їх не можна розміняти,
тут взагалі немає мови; нерозмінні банкноти можуть лише там правити за загальний
засіб циркуляції, де їх фактично підпирає державний кредит, як от, напр., тепер
в Росії. Отже вони підлягають уже розвинутим законам (книга І, розд. III, 2, с:
монета, знак вартости) про нерозмінні державні паперові гроші. — Ф. Е.].

Число банкнот в циркуляції реґулюється потребами обороту, й кожна
зайва банкнота повертається негайно назад до того, хто її видав. Оскільки в
Англії взагалі лише банкноти Англійського банку обертаються як законний
платіжний засіб, то й можемо ми тут знехтувати незначною, а до того ще й лише
місцевою циркуляцією банкнот провінціяльних банків.
