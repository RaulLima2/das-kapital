\parcont{}  %% абзац починається на попередній сторінці
\index{iii2}{0082}  %% посилання на сторінку оригінального видання
в № 1804 — що порушилось би «вартість капіталу» — дві дуже відмінні речі.
Рівень проценту може впливати на вексельні курси, а курси можуть впливати
на рівень проценту, але при змінних курсах може рівень проценту, а при змінному
рівні проценту можуть курси бути сталими. Вілсон не хоче зрозуміти того,
що при відправі капіталу закордон сама лише форма, що в ній його відправлено,
породжує таку ріжницю у впливі, тобто, що ріжниця форм капіталу має таку
вагу, й то саме лише його грошова форма, що дуже вже суперечить теорії
освіченої економії. Newmarch дає Вілсонові однобічну відповідь, оскільки зовсім
не звертає його уваги на те, що він так раптом і безпідставно перескочив від вексельного
курсу до рівня проценту Newmarch відповідає на питання № 1804
непевно та вагаючись: «Безперечно, коли треба добути 12 міл., то — оскільки
це стосується рівня проценту — немає істотного значіння, чи відправлять
закордон ці 12 міл. в формі благородного металу, чи в формі матеріялів.
Однак я думаю» [прегарне оце «однак» як перехід до того, щоб сказати прямо
протилежне], «що це не так то вже не має істотного значіння» [це не має істотного
значіння, але все ж це і не немає істотного значіння], «бо в одному
випадку 6 міл. ф. ст. мали б негайно повернутися; в другому ж випадку вони
повернулися б не так хутко. Тому деяку» [яка визначеність!] «ріжницю становитиме,
чи ті 6 міл. витрачено тут в країні, чи їх цілком відправлено закордон».
Що має значити те, що ті 6 міл. мають негайно повернутись? Оскільки
ті 6 міл. ф. ст. витрачено в Англії, то вони існують у шинах, локомотивах і т. ін,
що їх відправиться до Індії, звідки вони не вернуться, й їхня вартість вертатиметься
через амортизацію, отже, дуже повільно, тимчасом коли 6 міл. благородного
металу, можливо, дуже швидко повернуться in natura. Оскільки ті 6 міл.
витрачено на заробітну плату, їх з’їли; але гроші, що в них ті 6 міл. авансовано,
перебувають, як і раніше, в циркуляції країни або являють вони запас.
Те саме має силу й щодо зиску продуцентів шин та щодо тієї частини тих 6 міл.,
що повертає їхній сталий капітал. Отже, двозначну фразу про повертання
капіталу Newmarch ужив тільки на те, щоб не сказати прямо: гроші лишилися
в країні, й оскільки вони функціонують, як позичковий грошовий капітал, ріжннця
для грошового ринку (коли не вважати на те, що, скажемо, циркуляція
могла б поглинути більше металевих грошей) сходить тільки на те, що їх витрачається
за рахунок А, а не В. Приміщення такого роду, де капітал переноситься
до чужих країн в товарах, а не в благородному металі, може впливати на вексельні
курси (і то саме не з тією країною, де капітал приміщено) лише остільки,
оскільки продукція цих експортованих товарів потребує додаткового імпорту
інших чужоземних товарів. В такому разі ця продукція не в стані компенсувати
цей додатковий імпорт. Але те саме відбувається при кожному експорті на кредит,
однаково, чи той експорт призначено для приміщення капіталу, чи для
звичайних торговельних цілей. Опріч того, цей додатковий імпорт, зворотно
впливаючи, може викликати додатковий попит на англійські товари, напр.,
збоку колоній, або Сполучених Штатів.

\pfbreak

Перед тим Newmarch казав, що в наслідок тратт Ост-індської компанії
вивіз з Англії до Індії більший за довіз. Сер Charles Wood бере його щодо
цього пункту у перехресний допит. Цей надмір англійського вивозу до Індії, понад
довіз з Індії фактично здійснюється завдяки довозові з Індії, що за нього Англія
не платить жодного еквіваленту: тратти Ост-індської компанії (тепер ост-індського
уряду) сходять на данину, що її беруть з Індії. Напр., в 1855 році:
довіз з Індії до Англії становив 12.670.000 ф. ст.; англійський вивіз до Індії —
10.350.000 ф. ст. Балянс на користь Індії — 2.250.000 ф. ст. «Коли б справу
тим вичерпувалось, довелося б ці 2.250.000 ф. ст. у якійсь формі відправити
\parbreak{}  %% абзац продовжується на наступній сторінці
