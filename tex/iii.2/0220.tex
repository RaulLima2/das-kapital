земельна рента є нормальною формою додаткової вартости, а тому і додаткової
праці, тобто всієї надмірної праці, яку безпосередній продуцент мусить даром,
отже, на ділі примусово, виконувати на власника найістотнішої умови його
праці, на власника землі, — хоч цей примус уже не протистоїть йому в старій
брутальній формі. Зиск, — коли ми, фалшиво антиципуючи, назвемо так той
дріб надміру його праці над потрібного працею, що він його привласнює самому
собі, — до такої міри не має визначального впливу на ренту продуктами, що
радше можна було б сказати, що він виростає за спиною останньої і має свою
природну межу в розмірі ренти продуктами. Остання може досягати такого
розміру, що є поважною загрозою репродукції умов праці, самих засобів продукції,
більш або менш унеможливлює поширення продукції і знижує задоволення
потреб безпосереднього продуцента до фізичного мінімуму засобів
існування. Так буває саме в тому випадку, коли цю форму знаходить готового
і починає експлуатувати торговельна нація-завойовник, як, наприклад, англійці
в Індії.

IV. Грошова рента.

Під грошовою рентою ми розуміємо тут — на відзнаку від промислової
або комерційної земельної ренти, що ґрунтується на капіталістичному способі
продукції і становить лише надмір над пересічним зиском, — земельну ренту,
що виникає з простого перетворення форми ренти продуктами, так само, як ця
остання сама була лише перетвореною відробітною рентою. Замість продукту
безпосередній продуцент має тут виплачувати власникові землі (чи то буде
держава, чи приватна особа) ціну продукту. Отже, надміру продукту в його
натуральній формі вже не досить, його мусять перетворити з цієї натуральної
форми на грошову форму. Хоч безпосередній продуцент, як і давніш, продовжує
продукувати сам, принаймні, більшу частину своїх засобів існування, проте,
частина його продукту мусить тепер бути перетворена на товар, продукуватися
як товар. Отже, характер всього способу продукції більш або менш змінюється.
Він втрачає свою незалежність, свою відокремленість від зв'язку з суспільством.
Відношення витрат продукції, в які тепер входять в більшій чи меншій мірі і
грошові витрати, стає за вирішальне; в усякому разі стає вирішальним
надмір тієї частини гуртового продукту, що її треба перетворити на гроші
над тією частиною, яка, з одного боку, мусить стати знову засобом репродукції
і, з другого боку, безпосереднім засобом існування. А проте, база цього
роду ренти, хоч і наближається до свого розпаду, все ще лишається та сама,
що і при ренті продуктами, яка становить вихідний пункт. Безпосередній продуцент
є, як і давніш, спадковий або інакше традиційний посідач землі, який
повинен виплачувати земельному власникові, як власникові цієї найістотнішої
умови його продукції, надмірну примусову працю, тобто неоплачену, виконувану
без еквівалента працю, в формі додаткового продукту, перетвореного на
гроші. Власність на умови праці, відмінні від землі, хліборобське знаряддя та
інше рухоме майно спочатку фактично, а потім й юридично, перетворюється на
власність безпосередніх продуцентів вже за попередніх форм, і ще більше доводиться
припускати це для такої форми, як грошова рента. Спочатку спорадичне,
потім відбуваючись більш або менш у національному маштабі, перетворення
ренти продуктами на грошову ренту, має своєю передумовою вже порівняно
значний розвиток торгівлі, міської промисловосте товарової продукції взагалі, а
разом з тим і грошової циркуляції. Далі воно має своєю передумовою ринкову
ціну продуктів, і те, що вони продаються більш або менш близько до
своєї вартости, чого може і не бути за колишніх форм. На Сході Европи ми
можемо почасти ще на власні очі спостерігати процес цього перетворення.
