\parcont{}  %% абзац починається на попередній сторінці
\index{iii2}{0154}  %% посилання на сторінку оригінального видання
капіталу нічого не змінило в ріжницях між В, С і D. Тому частина продукту,
що перетворюється на ренту, зменшується.

Коли б вищенаведений наслідок — задоволення попиту при виключенні
землі А — був спричинений тич, що більше, ніж подвійна кількість капіталу
вкладалася б в землю С або D або в обидві разом, то справа набула б іншого
вигляду. Наприклад, коли б третє вкладення капіталу було зроблено на С:

\begin{table}[h]
  \begin{center}
    \emph{Таблиця ІVa}
    \footnotesize

  \begin{tabular}{c c c c c c c c c c c}
    \toprule
      \multirowcell{2}{\makecell{Рід \\землі}} &
      \multirowcell{2}{\rotatebox[origin=c]{90}{Акри}} &
      \rotatebox[origin=c]{90}{Капітал} &
      \rotatebox[origin=c]{90}{Зиск} &
      \rotatebox[origin=c]{90}{\makecell{Ціна про- \\ дукції}} &
      \multirowcell{2}{\rotatebox[origin=c]{90}{\makecell{Продукт \\ в кварт.}}} &
      \rotatebox[origin=c]{90}{\makecell{Продажна \\ ціна}} &
      \rotatebox[origin=c]{90}{Здобуток} &
      \multicolumn{2}{c}{Рента} &
      \multirowcell{2}{\makecell{Норма \\надзиску}} \\

      \cmidrule(rl){3-3}
      \cmidrule(rl){4-4}
      \cmidrule(rl){5-5}
      \cmidrule(rl){7-7}
      \cmidrule(rl){8-8}
      \cmidrule(rl){9-10}

       &  &  ф. ст. & ф. ст. & ф. ст. & & ф. ст. & ф. ст. & Кварт. & ф. ст. &  \\
      \midrule

      B & 1 &  \phantom{0}5\phantom{\sfrac{1}{2}} & 1\phantom{\sfrac{1}{2}} & \phantom{0}6 & \phantom{0}4 & 1\sfrac{1}{2} & \phantom{0}6\phantom{\sfrac{1}{2}} & 0 & \phantom{0}0\phantom{\sfrac{1}{2}}   & \phantom{00}0\% \\
      C & 1 &  \phantom{0}7\sfrac{1}{2}           & 1\sfrac{1}{2}           & \phantom{0}9 & \phantom{0}9 & 1\sfrac{1}{2} & 13\sfrac{1}{2}                     & 3 & \phantom{0}4\sfrac{1}{2}            & \phantom{0}60\%\\
      D & 1 &  \phantom{0}5\phantom{\sfrac{1}{2}} & 1\phantom{\sfrac{1}{2}} & \phantom{0}6 & \phantom{0}8 & 1\sfrac{1}{2} & 12\phantom{\sfrac{1}{2}}           & 4 & \phantom{0}6\phantom{\sfrac{1}{2}}  & 120\%\\
     \cmidrule(rl){1-1}
     \cmidrule(rl){2-2}
     \cmidrule(rl){3-3}
     \cmidrule(rl){4-4}
     \cmidrule(rl){5-5}
     \cmidrule(rl){6-6}
     \cmidrule(rl){8-8}
     \cmidrule(rl){9-9}
     \cmidrule(rl){10-10}

     Разом & 3 & 17\sfrac{1}{2} & 3\sfrac{1}{2} & 21 & 21 & & 30\sfrac{1}{2} & 7 & 10\sfrac{1}{2} &\\
  \end{tabular}

  \end{center}
\end{table}

Продукт з С збільшився тут проти таблиці ІV з 6 кватерів до 9, надпродукт
— з 2 квартерів до 3, грошова рента зросла з 3 ф. стерл. до 4\sfrac{1}{2} ф.
стерл. Проти таблиці II, де грошова рента була 12 ф. стерл. і таблиці І,
де вона була 6 ф. стерл., вона навпаки зменшилась. Загальна сума ренти визначена
в збіжжі = 7 квартерів, зменшилась проти таблиці II (12 квартерів),
збільшилась проти таблиці І (6 квартерів); визначена в грошах (10\sfrac{1}{2}ф. стерл.)
зменшилася проти обох (18 ф. стерл. і 36 ф. стерл.).

Коли б у землю В було вкладено третій капітал в 2\sfrac{1}{2} ф. стерл., то хоч
це й змінило б масу продукції, але не зачепило б ренти, бо згідно з припущенням
послідовні вкладення капіталу не вносять жодної ріжниці в землю того
самого роду, а земля В ренти не дає.

Навпаки, коли ми припустимо, що третій капітал вкладається в землю D,
замість С, то ми матимемо:

\begin{table}[h]
  \begin{center}
    \emph{Таблиця ІVb}
    \footnotesize

  \begin{tabular}{c c c c c c c c c c c}
    \toprule
      \multirowcell{2}{\makecell{Рід \\землі}} &
      \multirowcell{2}{\rotatebox[origin=c]{90}{Акри}} &
      \rotatebox[origin=c]{90}{Капітал} &
      \rotatebox[origin=c]{90}{Зиск} &
      \rotatebox[origin=c]{90}{\makecell{Ціна про- \\ дукції}} &
      \multirowcell{2}{\rotatebox[origin=c]{90}{\makecell{Продукт \\ в кварт.}}} &
      \rotatebox[origin=c]{90}{\makecell{Продажна \\ ціна}} &
      \rotatebox[origin=c]{90}{Здобуток} &
      \multicolumn{2}{c}{Рента} &
      \multirowcell{2}{\makecell{Норма \\надзиску}} \\

      \cmidrule(rl){3-3}
      \cmidrule(rl){4-4}
      \cmidrule(rl){5-5}
      \cmidrule(rl){7-7}
      \cmidrule(rl){8-8}
      \cmidrule(rl){9-10}

       &  &  ф. ст. & ф. ст. & ф. ст. & & ф. ст. & ф. ст. & Кварт. & ф. ст. &  \\
      \midrule

      B & 1 &  \phantom{0}5\phantom{\sfrac{1}{2}} & 1\phantom{\sfrac{1}{2}} & \phantom{0}6 & \phantom{0}4 & 1\sfrac{1}{2}  & \phantom{0}6 & 0 & \phantom{0}0 & \phantom{00}0\% \\
      C & 1 &  \phantom{0}5\phantom{\sfrac{1}{2}} & 1\phantom{\sfrac{1}{2}} & \phantom{0}6 & \phantom{0}6 & 1\sfrac{1}{2}  & \phantom{0}9 & 2 & \phantom{0}3 & \phantom{0}60\%\\
      D & 1 &  \phantom{0}7\sfrac{1}{2}           & 1\sfrac{1}{2}           & \phantom{0}9 & \phantom{0}12 & 1\sfrac{1}{2} & 18           & 6 & \phantom{0}9 & 120\%\\
     \cmidrule(rl){1-1}
     \cmidrule(rl){2-2}
     \cmidrule(rl){3-3}
     \cmidrule(rl){4-4}
     \cmidrule(rl){5-5}
     \cmidrule(rl){6-6}
     \cmidrule(rl){8-8}
     \cmidrule(rl){9-9}
     \cmidrule(rl){10-10}

     Разом & 3 & 17\sfrac{1}{2} & 3\sfrac{1}{2} & 21 & 22 & & 33 & 8 & 12 &\\
  \end{tabular}

  \end{center}
\end{table}

Тут загальна кількість продукту = 22 кварт., більша ніж удвоє проти
загальної кількости продукту таблиці І, хоч авансований капітал є лише 17\sfrac{1}{2}
ф. стерл. проти 10 ф. стерл., отже, не подвоївся. Далі, загальна кількість продукту
на 2 квартерн більша, ніж загальна кількість продукту у таблиці II, хоч
в останній авансований капітал більший, а саме 20 ф. стерл.
