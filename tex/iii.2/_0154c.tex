\parcont{}  %% абзац починається на попередній сторінці
\index{iii2}{0154}  %% посилання на сторінку оригінального видання
капіталу нічого не змінило в ріжницях між В, С і D. Тому частина продукту,
що перетворюється на ренту, зменшується.

Коли б вищенаведений наслідок — задоволення попиту при виключенні
землі А — був спричинений тич, що більше, ніж подвійна кількість капіталу
вкладалася б в землю С або D або в обидві разом, то справа набула б іншого
вигляду. Наприклад, коли б третє вкладення капіталу було зроблено на С:

Таблиця ІVа.

Рід землі    Акри    Капітал    Зиск    Ціна  продукції    Продукт в квартер    Продажна  ціна   
Здобуток    Рента        Норма  надзику
                                В збі-жжі    В  грошах
        ф. стер. ф. стер. ф. стер. ф. стер. ф. стер. Кварт. ф, стер.
B          1    5          1          6    4    11/2      6    0    0    0
C,          1    71/2,    11/2          9    9    11/2          131/2    3    41/2    60\%
D          1    5          1          6    8    11/2         12    4    6    120\%
Разом  3    171/2,    3 1/2    21    21             311/2    7    101/2

Продукт з С збільшився тут проти таблиці ІV з 6 кватерів до 9, надпродукт
— з 2 квартерів до 3, грошова рента зросла з 3 ф. стерл. до 41/2 ф.
стерл. Проти таблиці II, де грошова рента була 12 ф. стерл. і таблиці І,
де вона була 6 ф. стерл., вона навпаки зменшилась. Загальна сума ренти визначена
в збіжжі = 7 квартерів, зменшилась проти таблиці II (12 квартерів),
збільшилась проти таблиці І (6 квартерів); визначена в грошах (101/2ф-стерл.)
зменшилася проти обох (18 ф. стерл. і 36 ф. стерл.).

Коли б у землю В було вкладено третій капітал в 21/2 ф. стерл., то хоч
це й змінило б масу продукції, але не зачепило б ренти, бо згідно з припущенням
послідовні вкладення капіталу не вносять жодної ріжниці в землю того
самого роду, а земля В ренти не дає.

Навпаки, коли ми припустимо, що третій капітал вкладається в землю D,
замість С, то ми матимемо:

Таблиця ІVв.

Рід землі    Акри    Капітал    Зиск    Ціпа  продукції    Продукт в квартер. Продажна ціна    
Здобуток    Рента        Норма надзиску

ф. стер. ф. стер. ф. стер. ф. стер. ф. стер. кварт. ф. стер.
В            1    5            1            6    4    11/2     6    0    0    0
С            1    5            1            6    6    11/2     9    2    3    60\%
D             1    71/2    11/2    9    12    11/2      18    6    9    120\%
Разом.    3    171/2    3 1/2    21    22                33    8    12

Тут загальна кількість продукту = 22 кварт., більша ніж удвоє проти
загальної кількости продукту таблиці І, хоч авансований капітал є лише 171/2
ф. стерл. проти 10 ф. стерл., отже, не подвоївся. Далі, загальна кількість продукту
на 2 квартерн більша, ніж загальна кількість продукту у таблиці II, хоч
в останній авансований капітал більший, а саме 20 ф. стерл.
