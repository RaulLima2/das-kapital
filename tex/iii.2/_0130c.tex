\index{iii2}{0130}  %% посилання на сторінку оригінального видання
Перше і головне припущення є, що поліпшення в хліборобстві нерівномірно
впливає на землі різних родів, і тут воно більше впливає на кращі землі
С і D, ніж на А і В. Досвід довів, що, звичайно, справа так і стоїть, хоч може
статись і зворотне. Коли б поліпшення більше впливало на гірші землі, ніж на
кращі, то рента з останніх понизилася б замість підвищитись. — Але з абсолютним
зростом родючості! всіх родів землі у таблиці одночасно припускається зріст
вищої відносної родючості кращих родів землі С і D, а тому зріст ріжниці в продукті за однакової
величини застосованого капіталу, а тому і зріст диференційної ренти.
Друге припущення є в тому, що з зростанням всього продукту відповідно зростає і загальна потреба в
ньому. По-перше, не слід уявляти собі це зростання раптовим; воно відбувається поступово, доти, доки
не встановиться ряд III.
По-друге, невірно, нібито споживання потрібних засобів існування не зростає разом з їхнім
здешевленням. Скасування хлібних законів в Англії (дивись Newman) довело зворотне, і протилежне
уявлення постало лише тому, що великі і раптові ріжниці в урожаях, які пояснюються тільки погодою,
спричиняють то неспіврозмірне пониження, то неспіврозмірне підвищення цін збіжжя.
Коли в цьому разі раптове і скороминуще здешевлення не встигає справити повного впливу на поширення
споживання, то зворотне явище спостерігається в тому випадку, коли здешевлення випливає із зменшення
самої регуляційної ціни продукції, отже, коли воно має тривалий характер. По-трете, частина збіжжя
може бути спожита у вигляді горілки або пива. А зростаюуче споживання обох цих продуктів ніяк не
обмежено вузькими межами. По-четверте, тут справа залежить почасти від приросту людности, почасти
від
експорту збіжжя в тих країнах, що вивозять збіжжя — як от Англія, до і
пізніше половини XYII1 століття, і де тому потребу реґулюеться межами не самого
тільки національного споживання. Нарешті, збільшення і здешевлення
продукції пшениці може мати своїм наслідком, що замість жита або вівса за
головний засіб харчування маси народу стане пшениця, так що вже в наслідок
самого цього ринок для неї зросте подібно до того, як при зменшенні кількості
продукту і збільшенні його ціни може постати зворотне явище. — При цих припущеннях, отже, і при
взятих числових відношеннях, ряд III дає той наслідок,
що ціна знижується з 60 до ЗО шил. за квартер, отже на 50°/о; що продукція проти ряду І зростає з 10
до 23 квартерів, отже, на 130°/о; що рента з землі В лишається незмінною, рента з землі С
подвоюється, азі) більше, ніж подвоюється, і що загальна сума ренти підвищується з 18 до 22 ф.
стерл., отже, на 22 1/9 0/0

З порівняння цих трьох таблиць (при чому ряд 1 треба брати подвійно:
у висхідному напрямку від А до D і в низхідному від D до А), що їх можна
розглядати або як дані ступені хліборобства, за даного стану суспільства, наприклад,

Таблиця 111

Рід  землі    Продукт        Витрата  капіталу    Ціна про-дукції квартера    Зиск        Р е    н т
а
    Квар-тер и    Шилінґи            Квар-тери    Шилінґи    Квар-тери    Шилінґи
А    2    00    50    зо    ги    10    0    0
в    4    120    50    15    2і/з    70    2    60
с    7    210    50    8*/f'    Ь1/.    160    5    150
D    10    300    50    6    8] /з    250    8    240
Разом 23                            15    450
\parbreak{}  %% абзац продовжується на наступній сторінці
