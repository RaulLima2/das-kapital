власности, бо покупець платив за неї, як за всякий інший товар, еквівалент,
і більша частина земельної власности перейшла, таким чином, з рук у руки.
Але в такому випадку це міркування виправдувало б і рабство, бо для рабовласника,
що заплатив за раба готівкою, здобуток від рабської праці становить
лише процент на капітал, витрачений на його купівлю. Висновувати з купівлі
і продажу земельної ренти виправдання її існування — це значить взагалі
виправдувати її існування її існуванням.

Хоч як важливо для наукової аналізи земельної ренти, — тобто самостійної,
специфічної економічної форми земельної власности на основі капіталістичного
способу продукції, — хоч як важливо вивчити її в чистому вигляді вільною від
усіх домішок, що фальсифікують і затушовують її, не менш важливо, з другого
боку, для розуміння практичних наслідків земельної власности і навіть
для теоретичного пізнання маси фактів, які суперечать поняттю і природі земельної
ренти і проте є формами існування земельної ренти, — не менш важливо
обізнатись з елементами, що призводять до цих затемнень теорії.

Практично земельною рентою видається все, що орендар платить земельному
власникові в формі орендних грошей за дозвіл обробляти землю.
Хоч би з яких складових частин складалася ця дань, хоч би з яких джерел
вона походила, для неї з власне земельною рентою спільне є те, що монополія
на дільниці землі дає так званому земельному власникові силу брати
дань, накладати контрибуцію. Для неї з власне земельною рентою спільне є те,
що вона визначає земельну ціну, котра, як показано вище, є не що інше, як
капіталізований дохід від здачі землі в оренду.

Ми вже бачили, що процент на долучений до землі капітал може являти
таку чужородну складову частину земельної ренти, складову частину, що з поступом
економічного розвитку завжди мусить становити дедалі більший додаток
до загальної суми рент даної країни. Але навіть лишаючи цей процент осторонь,
можливо, що частина орендних грошей, а в певних випадках орендні
гроші цілком — отже, за цілковитої відсутности власне земельної ренти, а
тому за умов, коли земля дійсно не має вартости, — становлять відрахування
або з пересічного зиску, або з нормальної заробітної плати, або з того
й тієї одночасно. Ця частина чи то зиску, чи то заробітної плати — видається
тут як земельна рента, бо вона не потрапляє, як це буває звичайно, до промислового
капіталіста, або найманого робітника, а виплачується в формі орендних
грошей земельному власникові. В економічному розумінні слова ні та ні друга
частина не являє собою земельної ренти, але практично вона являє дохід
земельного власника, економічне використання його монополії, цілком так само,
як справжня земельна рента, і цілком так само як остання впливає вона на
земельну ціну, визначаючи її.

Ми не говоримо тут про такі відносини, коли земельна рента, що є відповідна
капіталістичному способові продукції форма реалізації земельної власности,
формально існує, але коли немає самого капіталістичного способу продукції,
коли сам орендар не є промисловий капіталіст, або характер його господарювання
не є капіталістичний. Таке становище, наприклад, в Ірляндії. Орендар
тут пересічно дрібний селянин. Те, що він виплачує, як орендну плату, земельному
власникові, часто поглинає не тільки частину його зиску, тобто його
власної додаткової праці, на яку він має право, як власник своїх власних
знарядь праці, але і частину тієї нормальної заробітної плати, яку він одержував
би в інших умовах за таку саму кількість праці. Крім того, земельний
власник, який тут анічогісінько не робить для поліпшення ґрунту, експропріює
у нього його маленький капітал, що його він долучає до землі здебільша
власною працею — експропріює його так само, як це зробив би лихвар в аналогічних
умовах. Ріжниця тільки та, що лихвар ризикує в таких операціях,
