\parcont{}  %% абзац починається на попередній сторінці
\index{iii2}{0078}  %% посилання на сторінку оригінального видання
Англії, то в Лондоні підноситься ціна марки, висловлена у стерлінґах, а в Гамбурзі
й Берліні спадає ціна стерлінґа, висловлена в марках. Якщо ця перевага
платіжних зобов’язань Англії проти Німеччини знову не зрівноважується, напр.,
за допомогою переваги закупів Німеччини в Англії, то ціна у стерлінґах за
вексель у марках на Німеччину мусить піднестися до тієї точки, поки не стане
вигідніше посилати до Німеччини на оплату замість векселів метал — золоті гроші
або зливки. Це — типовий перебіг справи.

Коли цей вивіз благородних металів набере більшого обсягу та триватиме
довший час, то він вплине на запас Англійського банку й англійському грошовому
ринкові, насамперед, Англійському банкові, доведеться ужити захисних заходів.
Ці останні, як ми вже бачили, сходять, головним чином, на піднесення
рівня проценту. При значному відпливі золота, на грошовому ринку постає звичайно
скрута, тобто попит на позичковий капітал у грошовій формі значно
переважає подання, і вищий рівень проценту випливає відси цілком сам собою;
норма дисконту, декретована Англійським банком, відповідає станові справ та
здійснюється на ринку. Але трапляються теж випадки, коли відплив металу
постає з інших причин, а не з звичайних операційних комбінацій (напр., з причин
позик чужим державам, приміщення капіталу закордоном і т. ін.), коли стан
лондонського грошового ринку, як такого, ніяк не виправдує значного підвищення
норми проценту; тоді Англійському банкові доводиться спочатку значними
позиками на «одкритому ринку», «зробити гроші рідкими», як то кажуть, і тим
способом штучно утворити стан, що виправдує або робить потрібним певне підвищення
проценту; маневр, що стає для банку з кожним роком чимраз тяжчим. —
Ф. Е.]. Як впливає це підвищення норми проценту на вексельний курс, показують
такі свідчення про банкове законодавство 1857 р. перед комісією нижчої
палати (цитовано як R А. або В. С., 1857).

Джон Стюарт Міл: «2176. Коли стан справ стає тяжким... постає значний
спад цін на цінні папери... чужинці купують залізничні акції тут в Англії,
або англійські власники закордонних залізничних акцій продають їх закордоном...
і відповідно усувається відправа золота закордон. — 2182. Велика та багата
кляса банкірів та торговців цінними паперами, що за їх допомогою звичайно
вирівнюється розмір проценту і стан комерційного барометра (pressure) між
різними країнами... завжди позирає яких би то купити таких цінних паперів,
що обіцяють підвищення ціни... за найкраще місце для закупу таких
паперів буде їм країна, що відправляє золото закордон. — 2183. Ці приміщення
капіталу відбувалися в 1847 році в значному маштабі, достатньому для того, щоб
зменшити відплив золота».

I. G. Hubbard, колишній управитель, а з 1838 року один з директорів
Англійського банку: «2545. Багато є європейських цінних паперів... що циркулюють
по всіх різних європейських грошових ринках, і скоро ці папери впали
ціною на одному ринку на 1 чи 2\%, їх негайно скуповують для відправи до
тих ринків, де їхня вартість ще тримається. — 2565 Чи чужоземні країни не
заборгувалися дуже купцям в Англії?.. — Дуже значно. — 2566. Отже, самого
одержання цих боргів могло б вистачити, щоб пояснити дуже велике нагромадження
капіталу в Англії? — В 1847 році ми, кінець-кінцем, здобули собі колишні
позиції, викресливши кілька мільйонів, що їх були раніш винні Англії
Америка та Росія». [Англія теж була винна саме цим країнам «кілька мільйонів» за
збіжжя і, здебільша, не занедбала теж «викреслити» ці борги за допомогою
банкрутств англійських довжників. Дивись вище звіт про банковий акт 1857 року,
розд. 30, ст. 31] — «2572. В 1847 році курс між Англією та Петербургом був
дуже високий. Коли було видано урядового листа, що уповноважував банк видавати
банкноти, не додержуючись приписаної межі з 14 міл.» [понад золотий
запас], «то поставили умову, що дисконт треба тримати на рівні 8\%. В той
\parbreak{}  %% абзац продовжується на наступній сторінці
