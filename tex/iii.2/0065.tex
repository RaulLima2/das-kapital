вати ці чотиримісячні векселі». (2097). Отже, і в провінції припинення чинности
банкового акту вплинуло, як визволення. — «2102. Минулого жовтня»
[1847 року] «майже всі американські закупники, що купують тут товари, негайно
обмежили, скільки змоги було, свої замовлення: а як звістка про подорожчання
грошей дійшла до Америки, всі нові замовлення припинилися. — 2134.
Збіжжя та цукор являли спеціяльні випадки. На збіжжевий ринок вплинули
сподіванки на урожай, а на цукор вплинули величезні запаси та довізи — 2163.
Багато з наших платіжних зобов’язань до Америки... зліквідувалось примусовими
продажами товарів, що їх відправлено на комісію, а багато, побоююсь,
анульовано тутешніми банкрутствами. — 2196. Якщо я добре пригадую, на нашій
фондовій біржі в жовтні 1847 платилося до 70%».

[Криза 1837 року з її довготривалими лихими наслідками, що до них
року 1842 долучилося ще справжнє покриззя, та заінтересоване осліплення
промисловців та купців, що ніяк не хотіли бачити надпродукції — аджеж за
вульґарною політичною економією вона є безглуздя та неможливість! — кінець-кінцем
породили ту плутанину в головах, що дозволила школі-currency перетворити
свою догму на практику в національному маштабі. Банкове законодавство
1844—45 років було переведено.

Банковий акт 1844 року поділяє англійський банк на відділ видання банкнот
і на відділ банковий. Перший одержує забезпечень — здебільша з паперів державного
боргу — на 14 мільйонів та ввесь металевий скарб, що має складатися
щонайбільше на 1/4 з срібла, і видає банкноти на суму, рівну загальній сумі
тих забезпечень і того скарбу. Оскільки ці банкноти не є в руках публіки, вони
лежать в банковому відділі, являючи разом з невеликою кількістю монети (щось
з мільйон), потрібної до щоденного вжитку, завжди готовий запас банку. Емісійний
відділ видає публіці золото за банкноти та банкноти за золото; решту
зносин з публікою обслуговує банковий відділ. Приватні банки, що в 1844 році
мали право видавати власні банкноти в Англії та Велсі, зберігають це право,
проте, видання ними банкнот обмежено певним контингентом; коли якийсь з цих
банків перестає видавати власні банкноти, то Англійський банк може збільшити
суму своїх непокритих банкнот на 2/3 невикористуваного тим банком континґенту;
цим способом та сума протягом часу до 1892 року підвищилась від 14 до 16 1/2
мільйонів ф. ст. (точно — 16.450.000 ф. ст.).

Отже, замість кожних п’ятьох фунтів золотом, що відпливають з банкового
скарбу, до емісійного відділу вертається банкнота-п’ятифунтівка й її
нищиться там; замість кожних п’ятьох соверенів, що надходять до скарбу, в
циркуляцію йде п’ятифунтівка-банкнота. Таким способом здійснюється па практиці
Оверстонова ідеальна паперова циркуляція, що регулюється точно за законами
металевої циркуляції, й тим, як стверджують теоретики currency, кризи на віки
вічні унеможливлено.

Але в дійсності розділ банку на два незалежні відділи відібрав у дирекції
змогу вільно порядкувати у рішучі моменти всіма вільними засобами, так що
могли трапитися випадки, коли банковому відділові загрожувало банкрутство, в
той час коли емісійний відділ мав незайманих кілька мільйонів золотом, та, крім
того, ще отих своїх 14 мільйонів забезпечень. І це могло то легше статись, що
майже в кожній кризі буває такий відтинок часу, коли постає великий відплив
золота за кордон, а покривати його доводиться, переважно, металевим скарбом
банку. А за кожні п’ять фунтів, що відпливають тоді за кордон, забирають з
внутрішньої циркуляції країни п’ятифунтівку-банкноту, отже, кількість засобів
циркуляції меншає саме тоді, коли їх найбільше уживається та є в них найбільша
потреба. Отже, банковий акт 1844 року провокує ввесь торговельний
світ безпосередньо до того, щоб на початку кризи своєчасно відкладати собі певний
запасний фонд банкнот, отже, до того, щоб ту кризу прискорювати та
