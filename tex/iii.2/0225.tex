громадської землі перетворюються на власників не тільки узурпованої ними
громадської землі, але й самих селянських дільниць.

Тут нам немає потреби докладніше спинятись на власне рабовласницькому
господарстві (яке теж проходить ряд ступенів від патріархальної системи, розрахованої
переважно на власне споживання, до власне плантаторської системи,
що працює на світовий ринок) і на поміщицькому господарстві, в якому земельний
власник провадить обробіток власним коштом, посідає всі знаряддя
продукції і визискує працю наймитів, невільних чи вільних, оплачуваних натурою
чи грішми. Земельний власник і власник знарядь продукції, а тому й
безпосередній визискувач робітників, що належать до числа цих елементів продукції,
тут збігаються. Так само збігаються рента і зиск, поділу різних форм
додаткової вартости не відбувається. Всю додаткову працю, яка тут втілюється
в додатковому продукті, безпосередньо здобуває з робітників власник усіх знарядь
продукції, до яких належить земля, а при первісній формі рабства і сами
безпосередні продуценти. Там, де панує капіталістичний спосіб уявлення, як
на американських плянтаціях, всю цю додаткову вартість розглядається як зиск;
там, де не існує самого капіталістичного способу продукції і куди ще не перенесено
відповідного йому способу уявлення з капіталістичних країн, вона
виступає як рента. В усякому разі, ця форма не являє жодних труднощів. Дохід
земельного власника, хоч би яку назву давали йому, привласнюваний ним
додатковий продукт, що ним можна порядкувати, є тут тією нормальною і панівною
формою, в якій безпосередньо привласнюється всю неоплачену додаткову
працю, і земельна власність створює базу цього привласнення.

Далі, парцелярна власність. Селянин є тут одночасно і вільний власник
своєї землі, яка є головним знаряддям його продукції, конче потрібним полем
для застосування його праці та його капіталу. При цій формі не виплачується
жодної орендної плати: отже, рента не виступає як відокремлена форма додаткової
вартости, хоч в країнах, де в інших сферах розвинувся капіталістичний
спосіб продукції, вона, порівняно з іншими галузями продукції, виступає як
надзиск, але як такий надзиск, що припадає селянинові, як і взагалі ввесь здобуток
від його праці.

Ця форма земельної власности має своєю передумовою, що, як і при її
давніших старих формах, сільська людність має величезну чисельну перевагу
над міською, що, отже, хоч капіталістичний спосіб продукції взагалі й панує,
але відносно він лише мало розвинений, а тому і в інших галузях продукції
концентрація капіталів рухається в вузьких межах, переважає розпорошення
капіталів. По суті справи переважна частина сільсько-господарського продукту
тут мусить споживатись самими продуцентами його, селянами, як беспосередній
засіб існування, і лише надмір над цим може як товар брати участь у торгівлі
з містами. Хоч би як регулювалась тут пересічна ринкова ціна хліборобського
продукту, диференційна рента, надмірна частина ціни товарів з кращих або
краще розташованих земель очевидно мусить тут існувати так само, як за
капіталістичного способу продукції. Навіть тоді, коли ця форма існує за таких
становищ суспільства, коли ще взагалі не розвинулась загальна ринкова ціна,
існує ця диференційна рента; вона виступає тоді у вигляді надмірного додаткового
продукту. Але потрапляє вона до кишені того селянина, що його праця
реалізується в сприятливіших природних умовах. Якраз за цієї форми, де ціна
землі для селянина входить як елемент в фактичні витрати продукції, чи тому,
що з дальшим розвитком цієї форми земля при поділі спадщини дістається замість
певної грошової вартости, чи тому, що при постійних переходах з рук
у руки всієї власности або її складових елементів, сам обробник купує землю,
причому гроші він здобуває здебільша під гіпотеку; де, отже, ціна землі є не
що інше, як капіталізована рента, наперед узятий елемент, і де тому здається,
