коли б капітал, належний їхнім підприємствам, але звідти забраний, вони примістили
не в залізниці, а, напр., в гірничі підприємства, що їхній продукт,
залізо, вугілля, мідь і т. ін., сам становить знову floating capital. Дійсне зменшення
вільного грошевого капіталу через неврожай, довіз збіжжя та вивіз
золота, звичайно було подією, що не мала нічого спільного з залізничним
шахрайством, — «Майже всі торговельні фірми почали більше або менше виснажати
свої підприємства, щоб примістити гроші в залізниці». — «Дуже великі позики,
що їх дали торговельні фірми залізницям, змусили ті фірми занадто вже дуже
спиратись на банки, дисконтуючи векселі в них і таким способом провадити
далі свої торговельні підприємства». (Той самий Hodgson, 1. с., р. 67). «В Менчестері
стались величезні втрати через спекуляцію на залізничних паперах».
(R. Gardner, не раз цитований в книзі І, розд. XIII, 3, с. та по інших місцях,
свідчення № 4877, 1. с.).

Головною причиною кризи 1847 року було колосальне переповнення ринку
та безмежна спекуляція в Ост-Індській торговлі. Але й інші обставини привели
дуже багаті фірми цієї ділянки до банкрутства: «Вони мали досить засобів, але не
сила було їм ті засоби перетворити на виплатні засоби. Цілий їхній капітал був міцно
приміщений у земельній власності на острові Маврикія або в фабриках індиґо та
цукру. Коли вони потім набралися зобов’язань на 500000—600000 ф. ст., то
в них не було платіжних засобів на те, щоб оплатити свої векселі, і кінець-кінцем
виявилося, що вони мусили для оплати своїх векселів цілком покладатись
на свій кредит». (Ch. Turner, великий ост-індський купець в Ліверпулі, № 730,
1. с.). Далі Ґарднер (№ 4872, І. с.): «Одразу після того, як складено умову
з Китаєм, перед країною розгорнулись такі великі перспективи щодо величезного
поширу нашої торговлі з Китаєм, що багато великих фабрик навмисно збудували
для того, щоб виготовляти бавовняні тканини, які мали найбільше поширення на китайському
ринку, і ці фабрики долучилися до всіх наших уже наявних фабрик. —
4874. Як закінчилися ці операції? — Незвичайно руйнаційно, так що бодай і не казати;
я не думаю, щоб повернулося назад більше, ніж 2/3 суми вартости усіх товарів,
відправлених кораблями до Китаю в 1844 та 1845 роках; тому що чай є головний
предмет зворотного експорту, і тому, що в нас збудили великі надії, то й сподівались ми,
фабриканти, з певністю чималого зниження мита на чай». А ось маємо наївно висловлене,
характеристичне credo англійського фабриканта: «Наша торговля з певним
закордонним ринком не обмежується його спроможністю купувати товари, але вона
є обмежена тут, в нашій країні, нашою спроможністю споживати ті продукти,
що їх ми одержуємо назад за наші промислові вироби». (Порівняно бідні країни,
що з ними Англія торгує, звичайно, мають змогу сплатити та спожити скільки
завгодно англійських фабрикатів, але, на жаль, багата Англія не в стані перетравити
продукти, експортовані з тих країн). «4876. Спочатку я відправив трохи
товару, і його продано було приблизно з 15\% збитку в певній надії, що ціна, за
яку мої аґенти могли купити чай, при перепродажу тут дасть такий великий
зиск, що той збиток покриється; але замість зиску я мав втрати часом 25 і до
50\% — 4877. Чи фабриканти експортували за свій рахунок? — Переважно за свій
рахунок; купці, здається, скоро побачили, що з тієї справи нічого не виходило, та
більше заохочували фабрикантів відправляти товар на комісію, ніж сами брали
участь у цьому». — Навпаки, в 1857 році втрати та банкрутства пали переважно
на купців, бо цього разу фабриканти передали до їхніх рук оте переповнення чужих
ринків «за їхній таки рахунок».

Збільшення грошового капіталу, яке постає від того, що в наслідок поширення
банкової справи (див. нижче приклад Іпсвіча, де протягом небагатьох років
безпосередньо перед 1857 р. вклади фармерів збільшилися вчетверо) те, що раніш було
приватним скарбом або монетним запасом, раз-у-раз перетворюється на певний
