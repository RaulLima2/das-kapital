місці, де його уживають, коли доводиться робити закордонні платежі». Цю систему
порушив акт 1845 року: «В наслідок акту, виданого для Шотландії
року 1845, останнім часом відбувся значний відплив золотих монет з Англійського
банку, щоб попередити той лише можливий попит на них в Шотландії,
що, може бути, й ніколи не постав би.. . Від цього часу одну значну суму
реґулярно закріпляють в Шотландії, а друга теж значна сума раз-у-раз мандрує
туди та сюди між Лондоном та Шотландією. Якщо настає час, коли шотландський
банкір чекає збільшеного попиту на свої банкноти, то відправляється
туди скриньку з золотом з Лондону; а коли цей час минув, то та сама скринька,
здебільша навіть не бувши одкрита, вертається назад до Лондону». (Economist
23 жовтня 1847 р.).

А що каже з приводу всього цього батько банкового акту, банкір Samuel
Jones Loyd, alias лорд Оверстон?

Вже 1848 року він повторив перед С. D. комісією лордів, що «грошову
скруту та високий рівень проценту, викликані недостатньою кількістю капіталу,
не можна полегшити збільшеним виданням банкнот» (1514), дарма що
самого лише дозволу урядового листа з 25 жовтня 1847 року на збільшене
видання банкнот було досить, щоб збити тій кризі вістря.

Він лишається при тій думці, що «висока норма рівня проценту та пригнічений
стан фабричної промисловости були неминучим наслідком зменшення
матеріяльного капіталу, що його можна було ужити для промислових та комерційних
цілей». (1604). А проте, пригнічений стан фабричної промисловости
протягом місяців був у тому, що матеріяльний товаровий капітал понад міру
наповнював комори, але його просто не сила було продати, і саме тому матеріяльний
продуктивний капітал цілком або напів лежав без діла, щоб не продукувати
ще більше того товарового капіталу, що його не сила було продати.

І перед банковою комісією 1857 року він каже: «Через гостре та ретельне
додержування засад акту 1844 року все відбувалося реґулярно та легко, грошова
система певна та непохитна, розцвіт країни безперечний, громадське довір’я
до акту 1844 року щодня зростає. Коли комісія бажає ще дальших
практичних доказів тому, що ті принципи, на які спирається цей акт, здорові,
та доказів тих благодійних наслідків, що їх він забезпечує, то на це ось правдива
та достатня відповідь: подивіться навколо себе; погляньте на сучасний
стан справ в нашій країні, погляньте на задоволення народу; погляньте на
багатства та розцвіт всіх кляс суспільства, і тоді, побачивши все це, комісія
буде в стані вирішити, чи схоче вона повстати проти дальшого існування акту,
що ним досягнуто таких наслідків». (В. С. 1857, № 4189).

На цей дитирамб, що його Оверстон заспівав перед Комісією 14 липня,
відповідь дано було 12 листопада того самого року, тим листом до дирекції
банку, що ним уряд припиняв чинність чудотворного закону 1844 року, щоб
врятувати бодай те, що можна ще було врятувати. — Ф. Е.]

Розділ тридцять п’ятий.

Благородний метал та вексельний курс.

І. Рух золотого скарбу

Щодо нагромадження банкнот підчас скрути треба зауважити, що тут повторюється
те саме явище збирання скарбів у благородному металі, що в початковому
періоді суспільного розвитку завжди бувало за неспокійних часів. Акт
1844 року в своїй чинності являє інтерес тому, що він воліє перетворити ввесь
наявний у країні благородний метал на засоби циркуляції; він намагається вирівняти
відплив золота скороченням, а прилив золота збільшенням засобів циркуляції. Але
