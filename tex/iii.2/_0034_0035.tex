\index{iii2}{0034}  %% посилання на сторінку оригінального видання
Отже, нагромадження позичкового грошового капіталу виявляє почасти
тільки той факт, що всякі гроші, на котрі промисловий капітал перетворюється
в процесі свого кругообороту, набирають форму не тих грошей, що їх
авансують капіталісти репродукції, а тих грошей, що їх вони собі позивають;
так що в дійсності авансування грошей, що мусить відбуватися в процесі
репродукції, видається авансуванням позичених грошей. В дійсності на
основі комерційного кредиту один визичає іншому гроші, що їх той потребує
в процесі репродукції. Але це набирає тепер таку форму, що банкір, якому
частина капіталістів репродукції позичає гроші, визичає їх другій частині
тих капіталістів, при чому цей банкір в такому разі видається благ подателем;
одночасно це набирає таку форму, що порядкування цим капіталом потрапляє
цілком до рук банкіра, як посередника.

Тепер треба навести ще деякі осібні форми нагромадження грошового
капіталу. Капітал стає вільним, напр., з причини спаду цін на елементи продукції,
на сирові матеріяли і т. ін. Якщо промисловець не може безпосередньо поширити
свій процес репродукції, то частину його грошового капіталу виштовхується
як зайву з кругообороту, і перетворюється на позичковий грошовий капітал.
Але, подруге, капітал звільняється у грошовій формі саме в купця, скоро
настають перерви в торговлі. Якщо купець зробив ряд справ і в наслідок таких
перерв може почати новий ряд їх тільки згодом, то реалізовані гроші становлять
для нього тільки скарб, надлишковий капітал. Однак одночасно вони безпосередньо
є нагромадження позичкового грошового капіталу. В першому випадку нагромадження
грошового капіталу означає повторювання процесу репродукції серед
сприятливіших умов, дійсне звільнення частини раніше зв’язаного капіталу,
отже, змогу поширювати процес репродукції тими самими грошовими засобами.
Навпаки, в другому випадку воно означає просту перерву в потоці операцій.
Але в обох випадках звільнений грошовий капітал перетворюється на позичковий
грошовий капітал, представляє нагромадження цього капіталу, впливає рівномірно
на грошовий ринок та на рівень проценту, дарма що в першому випадку
звільнення грошового капіталу означає сприятливі умови процесу дійсного нагромадження,
а в другому — гальмування його. Насамкінець, нагромадженню грошового
капіталу сприяють ті люди, що забезпечують собі певний дохід та усуваються від
репродукції. Що більше добувається зиску протягом промислового циклу, то більше
таких людей. Тут нагромадження позичкового грошового капіталу означає, з одного
боку, дійсне нагромадження (за його відносним обсягом); з другого боку,
лише обсяг перетвору промислових капіталістів на простих грошових капіталістів.

Щождо другої частини зиску, не призначеної для спожитку в формі доходу,
то вона перетворюється на грошовий капітал тільки тоді, коли її не можна
ужити безпосередньо до поширу підприємства в тій сфері продукції, де ту частину
зиску добувається. Це може поставати з двох причин. Або тому, що ця сфера
насичена капіталом. Або тому, що нагромадження для того, щоб мати змогу
функціонувати як капітал, мусить перше досягти певного обсягу, відповідно
до кількісних пропорцій приміщення нового капіталу в цьому певному підприємстві.
Отже, нагромадження передусім перетворюється на позичковий грошовий капітал
і придається до поширу продукції в інших сферах. Коли припустити, що всі
інші обставини однакові, то маса зиску, призначеного до зворотного перетворення
в капітал, залежатиме від маси добутого зиску, отже й від поширу самого процесу
репродукції. Коли ж це нове нагромадження наражається в своєму ужитку
на труднощі, брак сфер приміщення, коли, отже, відбувається переповнення ділянок
продукції позичковим капіталом та надмірне постачання його, то ця багатість
(Plethora) позичкового грошового капіталу доводить тільки обмеженість капіталістичної
продукції. Ажіотаж в кредитовій справі, що по тому приходить, доводить,
що немає жодної позитивної перешкоди для ужитку цього надмірного капіталу.
\index{iii2}{0035}  %% посилання на сторінку оригінального видання
Але, щоправда, є перешкода, що виникає в наслідок законів зросту капітальної
вартости, в наслідок тих меж, що в них вартість капіталу може зростати
як капітал. Багатість грошового капіталу, як такого, не означає неминуче ані
надмірної продукції, ані бодай лише браку сфер для вжитку капітала.

Нагромадження позичкового капіталу полягає просто в тому, що гроші
осідають, як гроші призначені до позичання. Цей процес дуже відмінний від
дійсного перетворення на капітал; це — тільки нагромадження грошей в такій
формі, в якій вони можуть перетворюватись на капітал. Але це нагромадження,
як уже доведено, може виражати моменти, дуже відмінні від дійсного нагромадження.
При постійному поширі дійсного нагромадження це поширене нагромадження
грошового капіталу може почасти бути його результатом, почасти —
результатом моментів, що, відбуваючись одночасно з поширом дійсного нагромадження,
проте цілком відмінні від нього, а почасти, насамкінець, навіть результатом
спину дійсного нагромадження. Вже тому, що нагромадження позичкового
капіталу незвичайно зростає з причини таких моментів, які від дійсного
нагромадження незалежні, а проте відбуваються поряд нього, вже тому мусить
в певні фази циклу раз-у-раз поставати багатість грошового капіталу і ця
багатість мусить розвиватися з розвитком кредиту. Отже, разом з цією багатістю
мусить одночасно розвиватися необхідність поширювати процес продукції поза його
капіталістичні межі: звідси надмірна торговля, надмірна продукція, надмірний
кредит. Одночасно мусить це відбуватися завжди в формах, що викликають
реакцію.

Щодо нагромадження грошового капіталу з земельної ренти, заробітної плати,
і т. ін., то зайво на цьому тут спинятися. Слід підкреслити лише те, що справа
дійсного заощадження та поздержливости (серед тих, що збирають скарби), оскільки
вона дає елементи нагромадження, з поступом капіталістичної продукції, через поділ
праці, припадає тим, що одержують мінімум таких елементів та ще досить часто
гублять своє заощаджене, як от робітники при банкрутствах банків. З одного
боку, промисловий капіталіст не сам «заощаджує» свій капітал, але порядкує чужими
заощадженнями пропорціонально величині власного капіталу; з другого боку,
грошовий капіталіст робить з чужих заощаджень свій капітал, а кредит, що його
дають капіталісти репродукції одні одним, і кредит, що їм дає публіка, повертає
він у джерело свого особистого збагачування. Так знищується до краю остання
ілюзія капіталістичної системи, ніби капітал породжується власною працею та ощадженням.
Не тільки зиск є присвоювання чужої праці, але й капітал, що ним
пускається в рух та визискується чужу працю, складається з чужої власности,
яку грошовий капіталіст дає до розпорядку промисловому капіталістові та за
це своєю чергою визискує його.

Ще треба дещо зауважити про кредитовий капітал.

Як часто та сама грошова монета може фігурувати як позичковий капітал,
це цілком залежить, як уже розвинуто вище, від того:

1) як часто вона реалізує товарові вартості в продажі та платежі, отже,
переносить капітал, і далі від того, як часто вона реалізує дохід. Тому, як
часто вона переходить до інших рук як реалізована вартість — чи то капіталу
або доходу, це залежить, очевидно, від обсягу та маси дійсних оборотів;

2) це залежить від економії на платежах та від розвитку й організації
кредитової справи.

3) Насамкінець, від зв’язку кредиту між собою та швидкости руху кредитів,
так що, коли в одному пункті вона осідає як вклад, в другому негайно
виходить вона знову, як позика.

Навіть, коли припустити за форму, що в ній існує позичковий капітал, лише
форму дійсних грошей, золота або срібла, товарів, що їхня речовина править за
мірило вартостей, то велика частина цього грошового капіталу завжди неминуче
\parbreak{}  %% абзац продовжується на наступній сторінці
