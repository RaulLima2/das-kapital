до життя цю природну умову підвищеної продуктивної сили праці, в такий
спосіб як кожен капітал може воду перетворити в пару. Ця природна умова
трапляється в природі лише місцями, і там, де її немає, її неможливо створити
певного витратою капіталу. Вона зв’язана не з продуктами, створюваними працею,
як машини, вугілля тощо, а з певними природними відносинами певної частини
землі. Та частина фабрикантів, що їм належать водоспади, усувають ту частину
фабрикантів, у яких немає водоспадів, від застосування цієї природної сили,
бо земля — і тим паче земля, що має водну силу, — обмежена. Це не виключає
того, що хоч кількість природних водоспадів у певній країні обмежена, кількість
водяної сили, яку може використовувати промисловість, може бути збільшена.
Водоспад можна штучно відвести, щоб цілком використати його рушійну силу;
коли вже є водоспад, водяне колесо можна удосконалити, щоб більше використати
силу води; там, де для подачі води звичайне колесо непридатне, можна застосувати
турбіни і т. ін. Посідання цією природною силою становить монополію
в руках її посідача, таку умову високої продуктивної сили вкладеного капіталу,
яку не можна створити продукційним процесом самого капіталу\footnote{
Див. про надзиск „Inquiry“ (проти Мальтуса).
}; ця природна
сила, яка може бути так монополізована, завжди зв’язана з землею. Така природна
сила не належить ні до числа загальних умов згаданої сфери продукції, ні до
числа таких її умов, що їх можна створити як загальні умови.

Тепер, коли ми собі уявимо що водоспади разом з прилежною до них землею
перебувають в руках осіб, які вважаються власниками цих частин землі,
землевласниками, то ми побачимо, що вони не дозволяють прикладати капітал
до водоспаду, використовувати його з допомогою капіталу. Вони можуть
дозволити і не дозволити використання водоспаду. Але капітал не може створити
водоспад із себе. Тому надзиск, що постає з цього використання водоспаду, постає
не з капіталу, а з застосування капіталом цієї природної сили, яку монополізувати
можна і яка монополізована. В таких обставинах надзиск перетворюється на земельну
ренту, тобто він дістається власникові водоспаду. Коли фабрикант виплачує
йому за його водоспад 10 ф. ст. на рік, то його зиск становить 15ф. ст.; 15\% на
ті 100 ф. ст., що їх тепер досягають його витрати продукції; і він опиняється
тепер цілком в такому самому становищі, може в кращому, ніж усі інші капіталісти
його сфери продукції, що працюють з допомогою пари. Справа ані трохи
не відмінилась би від того, коли б капіталіст сам був власником водоспаду. Він,
як і раніш, одержував би надзиск в 10 ф. ст. не як капіталіст, а як власник водоспаду,
і саме тому, що цей надмір постає не з його капіталу, як такого, а
з порядкування такою природною силою, що її можна відділити від його капіталу,
що її можна монополізувати, та яка обмежена в своїх розмірах, — саме тому, цей
надмір переворюється на земельну ренту.

Перше: Ясно, що ця рента завжди становить диференційну ренту, бо
вона не ввіходить визначально в загальну ціну продукції товару, а навпаки,
має її за передумову. Вона завжди виникає з ріжниці між індивідуальною
ціною продукції, для окремого капіталу, який порядкує монополізованою природною
силою, і загальною ціною продукції для капіталу, взагалі вкладеного у згадану
сферу продукції.

Друге-. Ця земельна рента постає не з абсолютного підвищення продуктивної
сили застосованого капіталу — зглядно привласненої ним праці, —
що взагалі могло б призвести лише до зменшення вартости товарів; а
з більшої відносної продуктивностн певних окремих капіталів, приміщених
в певну сферу продукції, порівняно з тими приміщенями капіталу, які усунені
від цих виключних, створених природою сприятливих умов підвищення
продуктивної сили. Коли б, наприклад, не зважаючи на те, що вугілля має вар-