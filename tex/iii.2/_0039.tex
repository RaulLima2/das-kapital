\parcont{}  %% абзац починається на попередній сторінці
\index{iii2}{0039}  %% посилання на сторінку оригінального видання
його pro tanto. А проте попит на змінний капітал, отже й попит на грошовий
капітал може разом з цим більшати, і це може підвищувати рівень проценту.
Тоді ринкова ціна робочої сили підвищується понад свою пересічну величину,
має працю число робітників більше за пересічне, й одночасно підноситься
рівень проценту, бо разом з тими обставинами зростає попит на грошовий капітал.
Чимраз більший попит на робочу силу удорожчує цей товар як і кожен інший товар,
підносить його ціну, але не підносить зиску, який головно спирається на відносну
дешевину саме цього товару. Але одночасно — за припущених умов — він підвищує
норму проценту, тому що збільшує попит на грошовий капітал. Коли б
грошовий капіталіст, замість визичати гроші, перетворився на промисловця, то
та обставина, що він має платити за працю дорожче, сама по собі не підвищила
б його зиску, а зменшила б його pro tanto. Коньюнктура обставин може
бути така, що хоч його зиск зростає, але ніколи не від того, що він дорожче
платить за працю. Однак останньої обставини, оскільки вона збільшує попит на
грошовий капітал, є досить, щоб підвищити норму проценту. Коли б заробітна
плата — серед інших несприятливих обставин — зросла з якихось причин, то
піднесення заробітної плати знизило б норму зиску, але піднесло б норму проценту
у тій мірі, в якій воно збільшило б попит на грошовий капітал.

Коли не вважати на працю, то те, що Оверстон зве «попитом на капітал»,
є лише попит на товари. Попит на товари підвищує їхню ціну, чи тому, що
він зростає понад пересічний рівень, чи тому, що подання спадає нижче від
пересічного рівня. Якщо промисловий капіталіст або купець має тепер платити,
напр., 150 ф. ст. за ту саму масу товарів, що за неї він раніш платив 100
ф. ст., то мав би він позичити 150 ф. ст., коли він раніше позичав 100 ф. ст.;
і тому мав би він при 5\% платити 7 1/2 ф. ст. процентів, коли раніше він
платив 5 ф. ст. Маса процентів, що їх йому доводиться платити, зросла б,
бо зросла маса позиченого капіталу.

Спроба пана Оверстона цілком сходить на те, щоб змалювати інтереси позичкового
капіталу та капіталу промислового як тотожні, тимчасом коли його
банковий акт розраховано саме на те, щоб визискувати ріжницю їхніх інтересів
на користь грошового капіталу.

Можливо, що попит на товари, якщо їхнє подання спаде нижче від пересічного
рівня, не поглине більше грошового капіталу, ніж раніше. Доводиться
платити за сукупну вартість товарів ту саму, може бути й меншу, суму, але
за цю суму одержують уже меншу кількість споживчих вартостей. В цьому разі
попит на позичковий грошовий капітал лишиться той самий, отже рівень проценту
не зросте, дарма що попит на товар проти їхнього подання зросте, а тому
і ціна товару піднесеться. Рівень проценту може бути зачеплено лише тоді,
коли зросте сукупний попит на позичковий капітал, а за вище наведеними припущеннями цього не буває.

Але подання певного товару може впасти нижче від пересічного рівня, як
от при неврожаю збіжжя, бавовни і т. ін., а попит на позичковий капітал може
зростати, бо спекулюють на те, щоб іще вище піднести ціни, а найближчий
засіб до піднесення цін той, щоб певну частину того подання забрати на деякий
час з ринку. Але на те, щоб оплатити куплені товари, не продаючи їх,
гроші добувають за допомогою комерційних «вексельних операцій». В цьому
разі попит на позичковий капітал зростає, та рівень проценту може піднестися
в наслідок цієї спроби штучно утруднити довіз товару до ринку. Вищий рівень
проценту тоді свідчить про штучно зменшене подання товарового капіталу.

З другого боку, попит на певний товар може зростати тому, що його подання
зросло й ціна того товару нижча від його пересічної ціни.

В цьому разі попит на позичковий капітал може лишитися той самий або
навіть впасти, бо за ту саму грошову суму можна мати більше товарів. Однак
\parbreak{}  %% абзац продовжується на наступній сторінці
