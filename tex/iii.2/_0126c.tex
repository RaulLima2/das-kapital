\index{iii2}{0126}  %% посилання на сторінку оригінального видання
Після цих попередніх зауважень я хочу коротко подати особливості мого
дослідження в відміну від Рікардо та ін.

\pfbreak

Ми розглянемо спочатку неоднакові наслідки, що їх дають однакові маси
капіталу, застосовані на різних земельних дільницях однакової величини: або,
при земельних дільницях неоднакової величини, наслідки, обчислені щодо однакової
земельної площі.

Дві незалежні від капіталу загальні причини цієї неоднаковости наслідків
є: 1) \emph{Родючість}. (До цього пункту (1) слід вияснити, що взагалі і які
різні моменти розуміються під природною родючістю земель). 2) \emph{Положення}
земельних дільниць. Остання причина є вирішальна для колоній, і взагалі — для
послідовности, в якій можуть іти під обробіток земельні дільниці одна по одній.
Далі ясно, що ці дві різні основи диференційної ренти, родючість і положення,
можуть впливати в протилежному напрямку. Земля може бути добре розташована
і мало родюча, і навпаки.

Ця обставина є важлива, бо вона пояснює нам, чому, обробляючи землі
даної країни, можна поступово переходити від кращої землі до гіршої, так само,
як і навпаки. Нарешті ясно, що прогрес суспільної продукції взагалі, з одного
боку, нівелює вплив положення, як основу диференційної ренти, бо він створює
місцеві ринки і, створюючи засоби сполучення й транспорту, змінює умови
положення; з другого боку, цей проґрес збільшує ріжниці в місцевому положенні
земельних дільниць, відокремлюючи хліборобство від мануфактури і створюючи
великі промислові центри, з одного боку, і відносне відокремлення села, з другого.

Але спочатку ми залишимо цей пункт, положення, не будемо звертати на
нього уваги і розглянемо лише природну родючість. Лишаючи осторонь кліматичні
та інші моменти, ріжниця у природній родючості сходить на ріжницю
хемічного складу верхнього шару ґрунту, тобто на ріжницю в кількості потрібних
для виростання рослин поживних речовин, що містяться в ньому. Проте, коли
припустити дві земельні дільниці з однаковим хемічним складом ґрунту і в
цьому розумінні однакової природної родючости, то дійсна ефективна родючість
буде різна залежно від тієї форми, в якій перебувають ці поживні речовини і в якій
вони більш-менш засвоюються, більш або менш безпосередньо йдуть на живлення
рослин. Отже, почасти від розвитку хліборобської хемії, почасти від розвитку
хліборобської механіки залежить те, якою мірою на земельних дільницях однакової
природної родючости можна дійсно використати цю природну родючість.
Отже, хоч родючість і є об’єктивна властивість ґрунту, проте, економічно вона
постійно має в собі певне відношення, відношення до даного рівня розвитку
хліборобської хемії і механіки, і змінюється разом з цим рівнем розвитку.
Як з допомогою хемічних засобів (наприклад, застосуванням певного текучого
гною на щільнім глинястім ґрунті, абож обпалюванням важкого глинястого
ґрунту), так і з допомогою механічних засобів (наприклад, особливих
плугів для важких ґрунтів) можна усунути перешкоди, що робили такі самі
родючі ґрунти фактично менше родючими (сюди ж належить і дренування ґрунту).
Це може змінити і саму послідовність в обробітку різних родів землі, як це
було, наприклад, щодо легкого піщаного і важкого глинястого ґрунтів за одного
з періодів розвитку англійського хліборобства. Це знов таки показує, яким
чином історично — в послідовному перебізі обробітку — перехід може однаково відбуватися
так від родючіших земель до менш родючих, як і навпаки. Те саме
може статись і в умовах штучно переведених поліпшень у складі ґрунту, або
в умовах простої зміни в методах хліборобства. Нарешті, такий самий результат
може постати з зміни в ієрархії щодо родів ґрунту в наслідок різних умов
підґрунтя, скоро тільки підґрунтя теж починає оброблятися й перетворюється
на зорану землю. Це зумовлюється почасти застосуванням нових хліборобських
\parbreak{}  %% абзац продовжується на наступній сторінці
