\parcont{}  %% абзац починається на попередній сторінці
\index{iii2}{0145}  %% посилання на сторінку оригінального видання
надпродукт з земель А, В або якогось іншого розряду можна було б одержати
лише по вищій ціні, ніж 3 ф. стерл., — тільки в цьому випадку зі зменшенням
продукту від додаткової витрати капіталу на якийсь з розрядів А, В, С, D
було б зв’язане підвищення ціни продукції і реґуляційної ринкової ціни.
Коли б це усталилось на тривалий час і не спричинило б оброблення додаткової
землі А (принаймні, якости А) і взагалі ніякі інші впливи не призвели б
до дешевшого подання, то, за інших незмінних умов, заробітна плата підвищилася б
в наслідок подорожчання хліба і відповідно до цього знизилась би норма зиску.
В цьому випадку було б байдуже, чи задовольнявся б підвищений попит втягненими
до обробітку гіршої, ніж А, землі, чи додатковим приміщенням капіталу
у будь-яку з чотирьох родів землі. Диференційна рента підвищилася б в зв’язку
з пониженням норми зиску.

Цей випадок, коли низхідна продуктивність додаткових капіталів, вкладуваних
в землі, що вже перебувають під культурою, може призвести до підвищення
ціни продукції, пониження норми зиску і створення вищої диференційної
ренти, — бо ця остання за даних умов підвищилася б на всіх родах землі
цілком так само, як коли б гірша, ніж А, земля стала тепер реґулювати ринкову
ціну, — цей випадок Рікардо перетворює в єдиний випадок, в нормальний
випадок, до якого він зводить все створення диференційної ренти II.

Так воно і було б, коли б оброблювалось лише рід землі А і коли б послідовні
вкладення капіталу на ній не були зв’язані з пропорційним приростом
продукту.

Отже, тут у випадку з диференційною рентою II цілком губиться з пам’яти
диференційна рента І.

За винятком цього випадку, коли або подання з оброблюваних земель
недостатнє, і тому ринкова ціна довгочасно перевищує ціну продукції, поки не
почнеться оброблення нових додаткових земель гіршої якости, або поки ввесь продукт
додаткового капіталу, вкладеного в землю різних розрядів, не буде можливости
збувати по вищій ціні продукції, ніж суща до того часу, — за винятком
цього випадку пропорційне зменшення продуктивности додаткових капіталів не
зачіпає регуляційної ціни продукції і норми зиску. А втім, можливі ще три
такі випадки:

а) Коли додатковий капітал, вкладений у землю якогось роду А, В, C, D,
дає лише норму зиску, визначувану ціною продукції на А, то через це не
створюється жодного надзиску, отже, і жодної можливої (евентуальної) ренти;
так само, як коли б почала оброблятися додаткова земля А.

в) Коли додатковий капітал дасть більшу кількість продукту, то, як само
собою зрозуміло, створюється новий надзиск (потенціяльна рента), якщо регуляційна
ціна лишається колишня. Останнє не завжди так буває, саме не буває
тоді, коли ця додаткова продукція виключає землю А з числа оброблюваних, а
разом з тим з числа конкурентних родів землі. В цьому випадку регуляційна ціна
продукції знижується. Норма зиску підвищилась би, коли б з цим було зв’язане
зниження заробітної плати, або коли б дешевший продукт увійшов елементом
в сталий капітал. Коли б додаткові капітали дали вищу продуктивність на
землях кращих родів С і D, то тільки від ступеня підвищення продуктивности
і від маси нововкладених капіталів залежало б, наскільки створення збільшеного
надзиску (отже, і збільшеної ренти) сполучалося б з пониженням ціни і підвищенням
норми зиску. Ця остання може підвищуватися і без пониження заробітної
плати, в наслідок здешевлення елементів сталого капіталу.

с) Якщо додаткове приміщення капіталу відбувається за зменшуваного
надзиску, але все-ж так, що його продукт дає надмір проти продукту такого ж
самого капіталу на землі А, то, якщо тільки збільшене подання не виключить
землі А з числа оброблюваних земель, за всяких умов відбудеться створення
\parbreak{}  %% абзац продовжується на наступній сторінці
