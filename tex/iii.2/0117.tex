потрібні харчові засоби для всього суспільства, тобто і для нехліборобських робітників;
отже, щоб був можливий цей великий поділ праці між хліборобами
і промисловцями, а також між тими з хліборобів, що продукують харч, і тими,
що продукують сирові матеріяли. Хоч праця безпосередніх продуцентів харчу
щодо них самих поділяється на потрібну і додаткову працю, проте, щодо
суспільства вона становить лише потрібну працю, яка потрібна для продукції
харчових засобів. А втім, це саме має силу і щодо всякого поділу праці
всередині всього суспільства, на відміну від поділу праці всередині окремої майстерні.
Це — праця, що потрібна для продукування особливих речей, для задоволення
особливої потреби суспільства в особливих речах. Коли цей поділ праці
пропорційний, то продукти різних груп продаються по їхніх вартостях (при
дальшому розгляді по цінах їхньої продукції) або ж по цінах, які, визначувані
загальними законами, становлять модифікації цих вартостей, — у відповідних
випадках цін продукції. Це справді, є закон вартости, як він виявляється не
у відношенні до окремих товарів, або речей, а у відношенні до кожноразового
сукупного продукту окремих суспільних сфер продукції, які в наслідок поділу
праці стали самостійними; отже, не тільки так, що на кожен окремий товар вжито
лише потрібний робочий час, але й так, що з усього суспільного робочого часу
на різні групи вжито лише конечну пропорційну кількість. Бо умовою лишається
споживна вартість. Але коли споживна вартість окремого товару залежить
від того, що він сам по собі задовольняє будь-яку потребу, то споживна вартість
суспільної маси продуктів залежить від того, що ця маса адекватна кількісна
певній суспільній потребі в продукті кожного осібного роду, а тому і від того,
що працю розподілено пропорційно між різними сферами продукції відповідно
до цих суспільних потреб, що кількісно визначені. (На цей пункт звернути
увагу в зв’язку з розподілом капіталу між різними сферами виробництва). Суспільна
потреба, тобто споживна вартість у суспільній потенції, вступає тут визначально
для кількостей всього суспільного робочого часу, що припадають на
різні окремі сфери продукції. Але це — лише той самий закон, який виявляється
уже у відношенні до окремого товару, а саме, що споживна вартість товару
є передумова його мінової вартости, а тому і його вартості!. Цей пункт має
лише ту дотичність до відношення між потрібного і додатковою працею, що при
порушенні цієї пропорції не може бути реалізована вартість товару, а тому
й додаткова вартість, що міститься в ній. Хай, наприклад, бавовняних тканин
випродуковано пропорційно забагато, хоч у всьому цьому продукті, в цих тканинах
реалізовано лише потрібний для цього в даних умовах робочий час. Але взагалі
на цю осібну галузь витрачено надто багато суспільної праці; тобто частина
продукту некорисна. Тому ввесь продукт продається лише так, як коли б він
був випродукований в потрібній пропорції. Ця кількісна межа тих кількостей
суспільного робочого часу, які можна витратити на різні осібні сфери продукції,
є лише далі розвинений вираз закону вартости взагалі; хоч потрібний робочий
час набуває тут іншого сенсу. Для задоволення певної суспільної потреби треба
стільки от робочого часу. Обмеження тут настає через споживну вартість. Суспільство
в даних умовах продукції на такий от продукт певного роду може
витратити лише стільки й стільки з усього свого робочого часу. Але суб’єктивні
й об'єктивні умови додаткової праці і додаткової вартости взагалі не мають
ніякого чинення до певної форми так зиску, як і ренти. Вони мають силу для
додаткової вартости як такої, хоч би яких особливих форм вона набувала. Тому
вони не пояснюють земельної ренти.

3) Якраз при економічній реалізації земельної власности, в розвитку земельної
ренти, виявляється дуже своєрідним те, що її величина визначається
зовсім не за допомогою її одержувача, а розвитком суспільної праці, що є незалежний
від його допомоги, і в якому він зовсім не бере участи. Тому легко вважати за
