\parcont{}  %% абзац починається на попередній сторінці
\index{iii2}{0133}  %% посилання на сторінку оригінального видання
казати, що вартість продуктів залишилась би та сама в умовах заміни капіталістичної
продукції асоціацією. Тотожність ринкової ціни однорідних товарів є
спосіб, у який, на базі капіталістичного способу продукції і взагалі продукції,
що ґрунтується на обміні товарів між поодинокими продуцентами, пробивається
суспільний характер вартости. Те, що суспільство, розглядуване як споживач,
переплачує за хліборобські продукти, і що становить мінус у реалізації його
робочого часу в хліборобській продукції, це становить тепер плюс для однієї
частини суспільства, для земельних власників.

Друга обставина, важлива для того, про що говориться в дальшім розділі
під рубрикою II, така:

Таблиця I
Рід  землі    Акри    Ціна  продукції    Продукт    Рента в збіжжі    Грошова  рента
А    1    3 ф. ст.    1 кварт.    0    0
В    1    3 > >    2» 1 кварт.    3 ф. ст.
С 1 3» > 3. 2 >\footnote{
кварт. 18 ф. ст.

Справа не тільки в ренті з акра або з гектара, взагалі не тільки в ріжниці
між ціною продукції і ринковою ціною, або між індивідуальною й загальною
ціною продукції з
акра, але також і в тому,
скільки акрів кожного
роду землі обробляється.
Тут важлива безпосередньо
лише величина загальної
суми ренти, тобто сукупної
ренти з усієї оброблюваної
площі; але це дає
нам одночасно можливість
перейти до з’ясовування
того, як підвищується норма
ренти, хоч ціни не
збільшуються, і хоч за
низхідних цін не збільшуються
ріжниці у відносній
родючості різних
родів землі. Вище ми мали:
(див. табл. 1).

Припустімо тепер,
що число оброблюваних
акрів кожного розряду подвоїлося.
В такому разі ми
матимемо: (див. табл. 1а).

Ми припустимо ще
2 випадки; перший, коли
продукція розширюється
на обох гірших родах землі.
Отже, тоді матимемо:
(див. табл. Іb).

І, нарешті, коли маємо
неоднакове поширення
продукції і оброблюваної
площі в чотирьох розрядах:
(див. табл. Iс).

Насамперед, в усіх
цих випадках І, Іа, Іb, Іс
рента з одного акра лишається
та сама; бо в
дійсності продукт однакової
маси капіталу на кожному

Таблиця 1а.

Рід  землі    Акри    Ціна  продукції    Продукт    Рента в збіжжі    Грошова  рента
А    2    6 ф. ст.    2 кварт.    0    0
в    2    6» >    4» 2 кварт.    6 ф. ст.
с    2    6» >    6» 4» 12 >»
п    2    6 > >    8 >    6 >    18» *

Сума 8 акрів
} > >
Б 1 3 > > 4» 3» 9 > >

Сума 4 акри 10 кварт. 18 ф. ст.

20 кварт. 12 кварт. 36 ф. ст.

Таблиця 1b.

Рід  землі    Ціна продукції        Продукт    Рента в збіжжі    Грошова  рента
        На  акр. В сумі
А    4    3 ф. ст.    12 ф. ст.    4 кварт.    0    0
в    4    3» »    12 >» 8» 4 кварт.    12 ф. ст.
с    2    3 > >    6» »    6» 4.    12 > >
І) 2 3.» 6» > 8» 6» 18 > *

Сума 12 акр. 36 ф. ст. 26 кварт. 14 кварт. 42 ф. ст.
\parbreak{}  %% абзац продовжується на наступній сторінці
