чено... на закуп товарів в Англії... залізних шин та дерева й інших матеріялів...
це — витрачання англійського капіталу в самій Англії на певний сорт
товарів, що його відправляється до Індії, й на тому кінець справи. — 1798.
[Weguelin:] Але продукція цих товарів з заліза та дерева, товарів, потрібних для
залізниць, породжує значне споживання закордонних товарів, а це могло б всеж
вилинути на вексельний курс? — Звичайно.

Вілсон гадає, що залізо представляє, здебільша, працю, а заробітна плата,
виплачена за цю працю, представляє здебільша імпортовані товари (1799), й
потім питає далі:

«1801. Але загалом кажучи: якщо товари, випродуковані через споживу
цих імпортованих товарів, вивозиться так, що ми не одержуємо за них назад
жодного еквіваленту, чи то продуктами, чи якось інак, — то чи не впливатиме
це на курс у несприятливому для нас напрямі? — Цей принцип точно висловлює те,
що відбувалося в Англії підчас великого будування залізниць (1845). Протягом
трьох або чотирьох, а то й п’ятьох послідованих років ви витратили на залізниці
30 міл. ф. ст. й майже всю цю суму на заробітну плату. Протягом трьох років,
будуючи залізниці, локомотиви, вагони та станції, ви годували більше число
люду, ніж по всіх фабричних округах разом. Ці люди... витрачали свою заробітну
плату на купівлю чаю, цукру, горілки та інших закордонних товарів; ці
товари доводилось імпортувати; але безперечно, що протягом того часу, коли робилось
ці великі витрати, вексельні курси між Англією та іншими країнами не
дуже були порушені. Не було відпливу благородного металу, навпаки, радше був
приплив його».

1802. Вілсон обстоює ту думку, що при рівновазі торговельного балансу
та паритетному курсі між Англією та Індією надзвичайна відправа заліза та локомотивів
«мусить впливати на вексельний курс з Індією». Newmarch не може
цього зрозуміти, бо поки шини відправляють до Індії, як приміщення капіталу,
Індія не має цього оплачувати в цій або тій формі; він додає до цього: «Я згоден
з тим принципом, що жодна країна не може протягом довшого часу мати
несприятливий вексельний курс з усіма тими країнами, з якими вона торгує;
несприятливий вексельний курс з однією країною неминуче породжує сприятливий
курс з якоюсь іншою». На це Вілсон відповідає йому такою тривіальністю:
«1803. Хіба ж перенесення капіталу не буде однаковим, чи відправлено той
капітал у тій або цій формі? — Так, оскільки мати на увазі боргове зобов’язання. —
1804. Отже, чи відправите ви благородний метал, чи товари, вплив залізничного
будівництва в Індії на ринок капіталу тут у нас був би однаковий і підвищить
вартість капіталу так само, як коли б усе відправлялося у формі благородного
металу?»

Якщо ціни на залізо не піднеслися, то це було в усякому разі доказом того, що
«вартість» уміщеного в шинах «капіталу» не збільшилася. «Вартість», про яку тут
мовиться, є вартість грошового капіталу — рівень проценту. Вілсон хотів би ототожнити
грошовий капітал з капіталом взагалі. Тут передусім той простий факт, що в
Англії було підписано 12 міл. на індійські залізниці. Це — справа, що безпосередньо
не має нічого до діла з вексельними курсами, і призначення тих 12 міл. є теж річ
байдужа для грошового ринку. Якщо грошовий ринок перебуває в сприятливому
стані, то це може взагалі не породжувати жодного впливу на нього, так само
як-от підписки на англійські залізниці в 1844 та 1845 роках не справили
впливу на грошовий ринок. Коли грошовий ринок перебуває вже до певної міри
у скрутному стані, то така підписка, звичайно, могла вилинути на рівень проценту,
але однак лише в напрямі піднесення, а це, за теорією Вілсона, мусило б вплинути на курс сприятливо
для Англії, тобто загальмувати тенденцію до вивозу благородного металу, якщо не до Індії, так бодай
куди інде. Пан Вілсон стрибає від однієї
справи до іншої. В питанні 1802 він каже, що вексельні курси порушилось би;
