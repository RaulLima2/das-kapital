\index{iii2}{0277}  %% посилання на сторінку оригінального видання
Розділ п’ятдесят другий.

Кляси.

Власники самої тільки робочої сили, власники капіталу й земельні власники
що їхніми відповідними джерелами доходів є заробітна плата, зиск і земельна рента,
отже, наймані робітники, капіталісти й земельні власники становлять три великі
кляси сучасного суспільства, яке ґрунтується на капіталістичному способі
продукції.

В Англії сучасне суспільство своєю економічною структурою досягло безперечно
найвищого клясичного розвитку. Проте і тут це клясове розчленування не
виступає ще в цілком чистому вигляді. Навіть і тут середні й переходові ступені
всюди затемнюють межові лінії (правда в селі геть менше, ніж у містах).
А втім це не має значіння для нашого досліду. Ми вже бачили, що постійна
тенденція і закон розвитку капіталістичного способу продукції є в тому, що
засоби продукції дедалі більше відокремлюються від праці, і розпорошені засоби
продукції дедалі більше концентруються в значних масах, що, отже, праця
перетворюється на найману працю, а засоби продукції на капітал. І цій тенденції
відповідає на другому боці самостійне відокремлювання земельної власности
від капіталу й праці\footnote{
F. List слушно зауважує: «Перевага самодостатнього господарства у великих маєтках
свідчить тільки про брак цивілізації, засобів комунікації, тубільних промислів та багатих міст. Тому
ми й знаходимо його всюди в Росії, у Польщі, Угорщині, Мекленбурзі. Давніш воно панувало і в Англії;
з розвитком торговлі й промислу на його місці став поділ на господарства середньої величини та
здавання в оренду» (Die Ackerverfassung, die Zwergwirtschaft und die Auswanderung, 1842, p. 10).
}, або перетворення всякої земельної власности на форму
земельної власности, відповідну капіталістичному способові продукції.

Найближче питання, на яке треба відповісти, таке: що утворює клясу?
причому відповідь ця випливає сама собою з відповіді на інше питання: що
робить з найманих робітників, капіталістів і землевласників утворювачів трьох
великих суспільних кляс?

На перший погляд, це є тотожність доходів і джерел доходу. Перед нами
три великі суспільні групи, що їх члени, індивідууми, які утворюють ці групи,
живуть відповідно з заробітної плати, зиску й земельної ренти, використовуючи
свою робочу силу, свій капітал і свою земельну власність.

Але з цього погляду лікарі і урядовці, наприклад, становили б теж дві
кляси, бо вони належать до двох різних суспільних груп, причому члени кожної
з цих двох груп одержують свої доходи з того самого джерела. Те саме мало б
силу щодо безконечної роздрібнености інтересів і становищ, що до неї призводить
суспільний розподіл праці так серед робітників, як і серед капіталістів та
земельних власників, — напр., розчленовуючи останніх на посідачів виноградників,
орної землі, лісів, копалень, риболовель.

[Тут рукопис уривається].
\parbreak{}  %% абзац продовжується на наступній сторінці
