\parcont{}  %% абзац починається на попередній сторінці
\index{iii2}{0177}  %% посилання на сторінку оригінального видання
дукції з землі В це становить округло 1 1/2 квартера. Надзиск з В визначається,
отже, у відповідній частині продукту з В, в цих 1 1/2 квартерах, які
становлять ренту, визначену в збіжжі, і які продаються по загальній ціні продукції
за 4  1/2 ф. стерл. Але, навпаки, надмірний продукт з акра землі В, надмірний
проти продукту з акра землі А, не можна просто вважати за надзиск,
а тому й за надпродукт. Згідно з припущенням акр землі В продукує 3  1/2 квартери,
акр землі А лише 1 квартер. Надмірний продукт з землі В є, отже,
2 1/2 квартери, але надпродукт є лише 1 1/2 квартери; бо в землю В вкладено
удвоє більший капітал, ніж у землю А, і тому вся ціна продукції тут удвоє
більша. Коли б у землю А також було вкладено 5 ф стерл. і норма продуктивности
лишилася б без зміни, то продукт становив би 2 квартери замість одного,
і таким чином виявилося б, що дійсний надпродукт можна знайти порівнянням
не 3  1/2 і 1, а 3  1/2 і 2; що, отже, він дорівнює не 2  1/2, а лише 1  1/2 квартерам.
Але далі, якби в землю В було вкладено третю порцію капіталу в 2  1/2 ф. стерл.,
що дала б лише 1 квартер, так що він коштував би 3 ф. стерл., як на землі А, то
його продажна ціна в 3 ф. ст. покрила б тільки ціну продукції, дала б лише
пересічний зиск, але не дала б надзиску, а отже і нічого, що могло б перетворитися
на ренту. Продукт з акра будь-якого роду землі, порівняно з продуктом
з акра землі А, не показує ані того, чи є він продукт однакової або більшої
витрати капіталу, ані того, чи надмірний продукт покриває тільки ціну продукції,
чи завдячує він своїм виникненням вищій продуктивності додаткового капіталу.

Друге: З щойно викладеного випливає, що при низхідній нормі продуктивности
додаткових витрат капіталу, за межу котрих, — оскільки мова йде
про створення нового надзиску, — є така витрата капіталу, що покриває лише
ціну продукції, тобто що продукує квартер так само дорого, як рівна витрата
капіталу на землі А, отже, згідно з припущенням, за 3 ф. стерл., — випливає,
що за межу, на якій загальна витрата капіталу на акр землі В перестала б
давати ренту, є та, коли індивідуальна пересічна ціна продукції продукту з
акра землі В підвищилася б до рівня ціни продукції з акра землі А.

Коли на В робляться лише такі додаткові витрати капіталу, що оплачують
ціну продукції і, отже, не створюють надзиску, а тому й нової ренти, то хоч
це й підвищує індивідуальну пересічну ціну продукції квартера, проте, не зачіпає
надзиску, що створився від попередніх витрат капіталу, евентуально ренти. Бо
пересічна ціна продукції завжди лишається нижча від ціни продукції на А, а коли
надмір ціни з квартера і зменшується, то кількість квартерів збільшується
у тому самому відношенні, так що загальний надмір ціни лишається без зміни.

В наведеному випадку дві перші витрати капіталу в 5 ф. стерл. на землі В
продукують 3 1/2 квартери, отже, згідно з припущенням, 1  1/2 квартери ренти =
4  1/2 ф. стерл. Коли сюди прилучиться третя витрата капіталу в 2 1/2 ф. стерл.,
що продукує лише 1 додатковий квартер, то вся ціна продукції (включаючи 20\%
зиску) 4  1/2 квартерів = 9 ф. стерл.; отже, пересічна ціна за квартер = 2 ф. стерл.
Отже, пересічна ціна продукції за квартер на землі В піднеслась з 1 5/7 ф. стерл.
до 2 ф. стерл., надзиск з квартера порівняно з регуляційною ціною А упав
з 1  2/7 ф. стерл. до 1 ф. стерл. Але 1×4  1/2 = 4  1/2 ф. стерл., цілком так само,
як раніш 1  2/7 × 3 1/2 = 4 1/2 ф. стерл.

Коли ми припустимо, що на В було б зроблено ще четверту і п’яту додаткові
витрати капіталу по 2  1/2 ф. стерл., які продукують квартер лише по його
загальній ціні продукції, то весь продукт з акра становив би тепер 6  1/2 квартерів,
а ціна його продукції була б 15 ф. стерл. Пересічна ціна продукції
квартера для В знову підвищилась би з 2*) до 2  4/13 ф. стерл., а надзиск з квартера
\index{iii2}{0178}  %% посилання на сторінку оригінального видання
порівняно з реґуляційною ціною продукції на землі А знову зменшився
б з 1 ф. стерл, до 9/13 ф. стерл. Але ці 9/13 ф. стерл. тут слід
помножити на 6 1/2 квартерів замість колишніх 4 1/2. А 9/13×6 1/2 = 1× 4 1/2 = 4 1/2 ф. стерл.

Звідси насамперед випливає, що за цих обставин не потрібно жодного підвищення
реґуляційної ціни продукції для того, щоб уможливити додаткові витрати
капіталу на рентодайних землях, навіть в такому розмірі, що додатковий
капітал зовсім перестає давати надзиск і дає ще лише пересічний зиск. З
цього випливає далі, що тут сума надзиску на акр лишається без зміни,
хоч би як дуже зменшувався надзиск з квартера; це зменшення завжди урівноважується
відповідним збільшенням квартерів, продукованих на акрі. Для того,
щоб пересічна ціна продукції піднеслась до рівня загальної ціни продукції (отже,
тут досягла б 3 ф. стерл. на землі В), мусять бути зроблені такі додаткові витрати
капіталу, продукт яких мав би вищу ціну продукції, ніж реґуляційна ціна
в 3 ф. стерл. Але ми побачимо, що тільки цього ще не досить, щоб підвищити
пересічну ціну продукції квартера на землі В до рівня загальної ціни продукції
в 3 ф. стерл.

Припустімо, що на землі В було випродуковано:

1) 3 1/2 квартери, що їхня ціна продукції, як і давніш, 6 ф. стерл.; отже,
дві витрати капіталу по  2 1/2 ф. стерл. кожна, при чому обидві дають надзиски,
але низхідної висоти.

2) 1 квартер за 3 ф. стерл.; витрата капіталу, при якій індивідуальна
ціна продукції дорівнювала б реґуляційній ціні продукції.

3) 1 квартер за 4 ф. стерл.; витрата капіталу, при якій індивідуальна
ціна продукції на 25\% вища за реґуляційну ціну.

Ми мали б тоді\footnote{
1/2 квартерів за 13 ф. стерл. дають пересічну ціну продукції в 2 4/11 ф. стерл.
за квартер, отже, за реґуляційної ціни продукції в 3 ф. стерл. надмір
в 7/11 ф. стерл. з квартера, який може перетворитися на ренту. 5 1/2 квартерів,
продані по реґуляційній ціні в 3 ф. стерл. дають 16 1/2 ф. стерл. За вирахуванням
ціни продукції в 13 ф. стерл. залишається 3 1/2 ф. стерл. надзиску, або
ренти, так що ці 3 1/2 ф. стерл., рахуючи по теперішній пересічній ціні продукції
квартера з землі В, тобто по 2 4/11 ф. стерл. за квартер, репрезентують
1 25/52 \footnote*{
В німецькому тексті тут стоїть: «1 5/72». Очевидна помилка. Прим. Ред.
} квартера. Грошова рента понизилася б на 1 ф. стерл., збіжжева
рента приблизно на 1/2 квартера, але, не зважаючи на те, що четверта додаткова
витрата капіталу на В не тільки не створює надзиску, але дає менше, ніж
пересічний зиск, — як і давніш, існує надзиск і рента. Коли ми припустимо, що,
крім витрати капіталу 3), і витрата 2) продукує по ціні, що перебільшує реґуляційну
ціну продукції, то вся продукція становитиме: 3 1/2 квартери за
6 ф. ст. + 2 квартери за 8 ф. ст., разом 5 1/2 квартерів за 14 ф. ст. ціни продукції.
Пересічна ціна продукції квартера була б 2 6/11  ф. ст., що давало б надмір
в 5/11 ф. ст. Ці  5 1/2  квартери, продані по 3 ф. ст., дають 16 1/2 ф. ст.; за вирахуванням
14 ф. ст. ціни продукції, лишається 2 1/2 ф. ст. на ренту. За теперішньої
пересічної ціни продукції на землі В це становило б 55/56 квартера. Отже, рента
все ще одержується, хоч і в меншому розмірі, ніж давніш.

В усякому разі це показує, що на кращих земельних дільницях при додаткових
витратах капіталу, що їхній продукт коштує дорожче, ніж реґуляційна
ціна продукції, рента, принаймні в межах допустимих практикою, мусить не
зникнути, а лише зменшитися, і саме відповідно до того, з одного боку, яку
} 1/2  квартерів з акра за 13 ф. стерл. при витраті капіталу
в 10 ф. стерл.; первісна витрата капіталу зросла б учетверо, але продукт
першої витрати капіталу не збільшився б і втроє.
\parbreak{}  %% абзац продовжується на наступній сторінці
