на А зменшується, бо в такому разі ціна продукції не лишилася б сталою,
а підвищилась би. Але в усіх цих обставинах, тобто, чи буде, надпродукт, що
його дають додаткові капітали, пропорційний їхній величині, чи буде він
вищий чи нижчий від цієї пропорції, — чи лишається, отже, норма надзиску на
капітал, з його зростом, незмінною, чи підвищується вона чи понижується, —
надпродукт і відповідний йому надзиск з акра зростає, отже, зростає і
евентуальна рента, у збіжжі і в грошах. Зростання просто маси надзиску зглядно
ренти, обчислених на акр, тобто збільшення маси, обчисленої на будь-яку постійну
одиницю, отже, в даному разі, на будь-яку певну кількість землі, акр або гектар,
виражається тут, як ростуча пропорція. Тому висота ренти, обчисленої з акра,
вростає за цих умов просто в наслідок збільшення капіталу, вкладеного в землю.
І до того ж це відбувається за незмінних цін продукції, і навпаки при цьому
не має ваги, чи лишається продуктивність додаткового капіталу незмінною, чи
вменшується вона, чи збільшується. Ці останні обставини модифікують розмір,
в якому зростає висота ренти на акр, але нічого не змінюють у факті цього
зросту. Це є явище, що властиве диференційній ренті II і яке відрізняв її від
диференційної ренти І. Коли б додаткові капітали вкладалося не один по
одному послідовно в часі на тій самій землі, а послідовно в просторі один
поряд одного на нових додаткових землях відповідної якости, то збільшилася б
загальна маса ренти і також, як це показано давніш, пересічна рента з усієї
оброблюваної площі, але не висота ренти з акра. За незмінних наслідків щодо
маси вартости усієї продукції і додаткового продукту, концентрація капіталу на
земельній площі меншого розміру підвищує розмір ренти з акра там, де за тих
самих обставин, його розпорошення на більшій земельній площі, за інших незмінних
умов, не справляє такого впливу. Але що більше розвивається капіталістичний
спосіб продукції, то більше концентрується капітал на тій самій земельній
площі, то більше, отже, підвищується рента, обчислена на акр. Тому в
двох країнах, де ціни продукції були б тотожні, ріжниці між різними родами
землі тотожні, і де вкладено було б однакову масу капіталу, але в одній країні
переважно у формі послідовних вкладень на обмеженій земельній площі, в другій
переважно у формі координованих вкладень на ширшій площі, — рента з акра, а
тому і ціна землі, була б вища в першій і нижча в другій країні, хоч маса
ренти в обох країнах була б та сама. Отже, ріжницю в висоті ренти можна
було б пояснити тут не ріжницею природної родючости різних земель і не
кількістю вжитої праці, а виключно різним способом вкладення капіталу.

Кажучи тут про надпродукт, ми завжди маємо на увазі відповідну
частину продукту, в якій репрезентовано надзиск. Взагалі-ж, під додатковим
продуктом або надпродуктом ми розуміємо ту частину продукту, яка становить
всю додаткову вартість, а в поодиноких випадках ту частину продукту, в якій
репрезентовано пересічний зиск. Специфічне значіння, надаване цьому слову,
коли мова йде про капітал, що дає ренту, є за привід до непорозумінь, як це
зазначено давніш.

Розділ сорок другий.

Диференційна рента II. — другий випадок: низхідна
ціна продукції.

Ціна продукції може падати, коли додаткові капіталовкладення відбуваються
за незмінної, низхідної або висхідної норми продуктивности.

І. За незмінної продуктивности додаткових капіталовкладень.

Отже, це означає, що на різних землях відповідно до їхньої відносної
якости, продукт зростає в тій самій мірі, в якій зростає вкладений в них
