\index{iii2}{0169}  %% посилання на сторінку оригінального видання
Варіянт III: Висхідна продуктивність другої витрати (таблиця XXI); це знов
таки зумовлює низхідну продуктивність першої витрати.

Друга видозміна: земля гіршої якости, (позначувана: літерою а)
вступає в конкуренцію; земля А дає ренту.

Варіянт 1: Незмінна продуктивність другої витрати (таблиця XXII).

Варіант 2: Низхідна продуктивність (таблиця XXIII).

Варіант 3: Висхідна продуктивність (таблиця XXIV).

Ці три варіянти відповідають загальним умовам проблеми і не дають
приводу до будь-яких зауважень.

Тепер ми наведемо таблиці:

Таблиця XI.

Рід землі
Ціна продукції. Шил.
Продукт. Бушелі
Продажна ціна. Шил.
Здобуток. Шил.
Рента. Шил.
Підвищення ренти
А 60 10 6 60 0          0
В 60 12 6 72 12       12
C 60 14 6 84 24  2 × 12
D 60 16 6 96 36  3 × 12
E 60 18 6 108 48  4 × 12
120                        10 × 12

За другої витрати капіталу на тій самій землі.

Перший випадок: за незмінної ціни продукції.

Варіянт 1: за незмінної продуктивности другої витрати капіталу.

Таблиця XII.

Рід землі
Ціна продукції. Шил.
Продукт. Бушелі
Продажна ціна. Шил.
Здобуток. Шил.
Рента. Шил.
Підвищення ренти
А 60 + 60 = 120 10 + 10 = 20 6 120 0 0
В 60 + 60 = 120 12 + 12 = 24 6 144 24 24
C 60 + 60 = 120 14 + 14 = 28 6 168 48 2 × 24
D 60 + 60 = 120 16 + 16 = 32 6 192 72 3 × 24
E 60 + 60 = 120 18 + 18 = 36 6 216 96 4 × 24
240 10 × 24

Варіянт 2: за низхідної продуктивности другої витрати капіталу: на землі
А не зроблено другої витрати.

1) Коли земля В стає землею, що не дає ренти.
