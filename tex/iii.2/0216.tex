землевласника. Отже, цей надмір над найпотрібнішими засобами існування, зародок
того, що за капіталістичного способу продукції з’являється як зиск,
тут цілком визначається висотою земельної ренти, яка тут не тільки
є безпосередньо неоплачена додаткова праця, але й виступає як така; неоплачена
додаткова праця для «власника» умов продукції, які тут тотожні з землею, а коли
й відмінні від неї, то вважаються лише її належністю. Що продукту панщанина тут
мусить вистачати для того, щоб крім засобів його існування покрити його умови
праці, це є така обставина, що лишається однаковою для всіх способів продукції,
бо це не є наслідок специфічної форми останніх, а природна умова усякої безперервної
і репродуктивної праці взагалі, всякої тривалої продукції, яка завжди
є одночасно і репродукцією, отже, також репродукцією умов свого власного функціонування.
Далі ясно, що в усіх формах, при яких безпосередній робітник є
«посідач» засобів продукції і умов праці, потрібних для продукування засобів
його власного існування, відношення власности мусить одночасно виступати
як безпосереднє відношення панування і упідлеглення, а безпосередній
продуцент — як невільний; неволя, яка від кріпацтва з панщинною працею
може пом’якшуватись до звичайного оброчного зобов’язання. Згідно з припущенням
безпосередній продуцент посідає тут свої власні засоби продукції,
речові умови праці, потрібні для реалізації його праці і для продукування
засобів його існування: він самостійно провадить своє хліборобство, як і
зв’язану з ним сільсько-домашню промисловість. Ця самостійність не знищується
тим, що, як наприклад в Індії, дрібні селяни з’єднуються поміж себе в
більш або менш природно вирослу продукційну громаду, бо тут мова йде тільки
про самостійність проти номінального землевласника. В таких умовах з них
додаткову працю для номінального земельного власника можна вичавити
тільки поза економічним примусом, хоч би якої форми набував він 44). Дана форма
відрізняється від рабського або плантаторського господарства саме тим, що раб
тут працює з допомогою чужих умов продукції і не самостійно. Отже, потрібні
відносини особистої залежности, особиста неволя, хоч би в якій мірі; і прикріплення
до землі, як придатка до неї, належність в справжньому розумінні.
Коли не приватні земельні власники, а держава безпосередньо протистоїть їм
як земельний власник і разом з цим суверен, як-от в Азії, то рента і податок
збігаються, або радше тоді не існує жодного податку, що був би відмінний від цієї
форми земельної ренти. За таких обставин відносини залежности політично і
економічно не потребують суворішої форми, ніж та, що є спільна усім підданцям
у їхніх відносинах до цієї держави. Держава тут верховний власник землі.
Суверенітет тут — земельна власність, концентрована в національному маштабі.
Але зате тут не існує жодної приватної земельної власности, хоч існує так
приватне, як і громадське посідання й користування землею.

Та специфічна економічна форма, що в ній неоплачену додаткову працю висисається
з безпосередніх продуцентів, визначає відносини панування й упідлеглення,
якими вони виростають безпосередньо з самої продукції, і з свого боку визначально
справляють на неї зворотний вплив. Але на цьому заснована вся структура економічної
громади, що виростає з самих відносин продукції, а разом з тим і її специфічна
політична структура. Безпосереднє відношення власників умов продукції
до беспосередніх продуцентів, — відношення, що всяка його дана форма завжди
природно відповідає певному ступеневі розвитку способу праці, а тому й суспільній
продуктивній силі останньої, — ось в чому ми завжди розкриваємо
внутрішню таємницю, приховану основу всього суспільного ладу, а тому й політичної
форми відносин суверенітету й залежности, коротко, всякої даної

44) По завоюванні країни за найближче завдання для завойовників завжди буди привласнення також і
людей. Пор. Linguet. Див. також Möser.
