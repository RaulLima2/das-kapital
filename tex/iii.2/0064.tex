утворити фонд для закупу шовку; і стан ринків в Індії такий (не вважаючи на
нагромадження там срібла), що для купця вигідніше посилати туди срібло, ніж
тканини або інші британські фабрикати. — 4338. Чи не було великого відпливу
з Франції, що в наслідок його ми одержували срібло? — Так, дуже значний відплив.
— 4344. Замість вивозити шовк з Франції та Італії, ми відправляємо його
туди великими партіями, так бенгальський, як і китайський».

Отож до Азії відправлялося срібло — грошовий металь цієї частини світу —
замість товарів не тому, що ціни на ці товари піднеслися в країні, що їх продукує
(Англія), а тому що впали — впали через надмірний імпорт — в тій країні,
куди їх імпортують; дарма що це срібло Англія одержувала з Франції та почасти
мусила його оплачувати золотом. За currency-теорією при такому імпорті
ціни в Англії мусіли б впасти, а в Індії та Китаю — піднестися.

Другий приклад. Перед комісією лордів (С. D. 1848—1857) Wylie, один
з    перших ліверпульських купців, свідчить так: «1994. Наприкінці 1845 року
не було вигіднішої справи, що давала б такі великі зиски [як бавовнопрядіння].
Запас бавовни був великий, і добру придатну бавовну можна було мати по
4    пенси за фунт, а з такої бавовни можна було випрядати гарний secunda
mule twist № 40, що на нього теж мали витратити коло 4 пенсів, отже разом
витрат щось коло 8 пенсів у прядуна. Цю пряжу великими масами продавалося
в вересні та жовтні 1845 року — і складалися так само великі контракти на
постачання її — по 10 1/2 та 11 1/2 пенсів за фунт, і в деяких випадках прядуни
реалізували зиск, рівний купівельній ціні бавовни. — 1996. Справа була вигідна
до початку 1846 року. — 2000. З березня 1844 року запас бавовни [627.042 паки]
становив подвійну кількість того, що він становить сьогодні [7 березня 1848 року,
коли його було 301.070 паків], а проте ціна була на 1 1/4 пенси за фунт вища».
[6 1/4 пенсів проти 5 пенсів]. Одночасно пряжа — добрий secunda mule twist
№ 40 — впала від 11 1/2—12 пенсів до 9 1/2 пенсів в жовтні та до 7 3/4 пенсів
наприкінці грудня 1847 року; пряжу продавалося за купівельну ціну бавовни,
що з неї було її випрядено (ib, № 2021, 2023). Це виявляє ту заінтересовану
премудрість Оверстона, що гроші мають бути «дорогі», бо капітал «рідкий». 3-го березня
1844 року банковий рівень проценту був 3%; в жовтні та листопаді
дійшов він до 8 та 9% й 7 березня 1848 року становив ще 4%. Ціни на бавовну
— в наслідок цілковитого спину в збуті та в наслідок паніки з відповідним
їй високим рівнем проценту — впали далеко нижче від тієї ціни на неї,
що відповідала станові подання. Наслідок цього було, з одного боку, величезне
зменшення довозу в 1848 році, а з другого боку, зменшення продукції в Америці;
відси новий зріст бавовняних цін в 1849 році.
За Оверстоном, товари були занадто дорогі тому, що занадто багато грошей
було в країні.
«2002. Недавнє погіршення стану бавовняної промисловости завдячує не
бракові сировини, бо ціна впала, дарма що запас бавовни-сировини значно
зменшився». Але в Оверстона маємо приємне переплутування ціни, відповідно
вартости товару, з вартістю грошей, власне з рівнем проценту. Відповідаючи на
питання 2026, Wylie подає свій загальний погляд на currency-теорію, що на
ній Cardwell та сер Charles Wood в травні 1847 року «заснували потребу перевести
банковий акт 1844 року в усій повноті його змісту»: «Ці принципи, на
мою думку, такі, що вони надаватимуть грошам штучну високу вартість, а всім
товарам штучну руйнаційно низьку вартість». — Далі він каже про вплив цього
банкового акту на загальний стан справ: «Що лише з великими втратами можна
було дисконтувати чотиримісячні векселі, — які є звичайні трати фабричних міст
на купців та банкірів за куплені товари, призначені для Сполучених Штатів, —
то і виконання замовлень дуже гальмувалося аж до урядового листа з 25 жовтня»
[припинення чинности банкового акту], «коли знову з’явилася змога дисконту-
