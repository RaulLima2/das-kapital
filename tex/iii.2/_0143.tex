\parcont{}  %% абзац починається на попередній сторінці
\index{iii2}{0143}  %% посилання на сторінку оригінального видання
один по одному на тій самій дільниці землі, чи для капіталів, що застосовуються
до декількох дільниць різнорідної землі, — це нічого не може змінити в ріжниці
родючости або в кількості її продукту, а тому і в створенні диференційної
ренти з частин капіталу, вкладених з більшою родючістю. За рівних капіталовкладень
різну родючість, як і давніш, виявляє земля, але тут та ж сама земля,
за послідовних вкладень різних частин капіталу, дає те, що при диференційній
ренті І дають різні роди землі при вкладенні в них різних частин суспільною
капіталу однакової величини.

Коли той самий капітал в 10 ф. ст., що в таблиці І вкладається різними орендарями
у вигляді самостійних капіталів по 2 1/2 ф ст. в кожен акр чотирьох родів
землі А, В, С і D, замість цього вкладався б послідовно у той самий акр
D, так що перше вкладення дало б 4 квартери, друге — 3, третє — 2 і останнє
— 1 квартер (або ж у зворотній послідовності), то ціна одного квартера, одержуваного
від найменш дохідної частини капіталу, рівна 3 ф. ст., не давала б диференційної
ренти, але визначала б ціну продукції до того часу, поки зберігається
потреба в довозі пшениці, ціна продукції якої є 3 ф. ст. А що згідно з припущенням
продукція провадиться капіталістично і, отже, ціна в 3 ф. ст. має
в собі пересічний зиск, що його взагалі дає капітал в 2 1/2 ф. ст., то три інші
частини по 2 1/2 ф. ст. даватимуть надзиски залежно від ріжниці в кількості
продукту, бо цей продукт продається не по його ціні продукції, а по ціні продукції
найменш дохідного вкладення капіталу в 2 1/2 ф. ст., вкладення, яке не дає ренти
і ціна продукту якого реґулюється за загальними законами цін продукції. Створення
надзиску було б таке саме, як у таблиці І.

Тут знову виявляється, що диференційна рента II має за свою передумову
диференційну ренту І. Мінімум продукту, що дає капітал в 2 1/2 ф. ст., тобто капітал,
вкладений в найгіршу землю, ми взяли тут в 1 квартер. Отже, ми припускаємо,
що орендар землі D, крім тих 2 1/2 ф. ст., які дають йому 4 квартери,
за що він виплачує 3 квартери диференційної ренти, вживає на тій самій землі
2 1/2 ф. ст., що дають йому лише 1 квартер, як це дав би той самий капітал на
найгіршій землі А. В такому випадку це було б таке приміщення капіталу,
що не дає ренти, бо воно дало б орендареві тільки пересічний зиск. Тут не
було б жодного надзиску, щоб перетворитись на ренту. Але, з другого боку, це
зменшення продукту від другого вкладення капіталу в D не зробило б жодного
впливу на норму зиску. Це було б те саме, як коли б 2 1/2 ф. стерл. були знову
вкладені в якийсь новий акр землі сорту А, — обставина, яка жодним способом
не зачіпає надзиску, а, отже, і диференційної ренти з земель А, В, С, D.
Але для орендаря це додаткове вкладення 2 1/2 ф. стерл. у D було б вигідно
в такій самій мірі, як, згідно з нашим припущенням, вкладення первісних
2 1/2 ф. стерл. у акр землі D, хоч вона дає 4 квартери. Коли б далі два наступні
вкладення капіталу по 2 1/2 ф. стерл. кожне дали йому: перше 3, а друге
2 квартери додаткової продукції, то тут знову відбулося б зменшення продукту
порівняно з продуктом від першого вкладення в 2 1/2 ф. стерл. у D, яке
дало 4 квартери, а тому 3 квартери надзиску. Але це було б лише зменшенням
величини надзиску і не зачепило б ні пересічного зиску, ні регуляційної ціни
продукції. Останнє могло б статися лише в тому випадку, коли б додаткова
продукція, яка дає понижений надзиск, зробила б зайвою продукцію з А і таким
чином виключила б А з числа оброблюваних земель. В цьому випадку
з пониженням родючости додаткового вкладення капіталу в акр землі D було б
зв’язане падіння ціни продукції, наприклад, з 3 ф. стерл. до 1 1/2 ф. стерл.,
коли б акр землі В зробився землею, що не дає ренти, отже, реґулює ринкову ціну.

Продукт з D був би тепер = 4 + 1 + 3 + 2 = 10 квартерам, тимчасом
як раніш він дорівнював 4 квартерам. Але ціна квартера, регульована В, впала б
до 1 1/2 ф. стерл. Ріжниця між D і В була б = 10—2 = 8 квартерів, що по
\parbreak{}  %% абзац продовжується на наступній сторінці
