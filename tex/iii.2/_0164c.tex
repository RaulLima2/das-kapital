\parcont{}  %% абзац починається на попередній сторінці
\index{iii2}{0164}  %% посилання на сторінку оригінального видання
і низхідної ціни. Хоч в усіх цих випадках рента може залишитися без зміни
і може понизитись, вона понизилася б значніше, коли б додаткове вживання
капіталу, за інших незмінних обставин не зумовлювало збільшення родючости. Додаткове вкладення
капіталу тоді завжди є за причину відносної висоти ренти,
хоча б абсолютно вона й понизилась.

\section{Дифереційна рента II. — Третій випадок:
висхідна ціна продукції.}

[Підвищення ціни продукції має за свою передумову, що продуктивність землі
найгіршої якости, що не дає ренти, зменшується. Ціна продукції, взята нами за
реґуляційну, може піднестися вище від 3 ф. ст. за кв., лише тоді, коли 2 \sfrac{1}{2} ф. ст.,
витрачені на А, продукуватимуть менш за 1 квартер, або 5 ф. ст. менш за
2 квартери, або коли б довелося обробляти землю ще гіршої якости, ніж А.

За незмінної або навіть висхідної продуктивности другого вкладення капіталу
це було б можливе лише тоді, коли б продуктивність першого вкладення в 2 \sfrac{1}{2} ф. с.
зменшилась. Цей випадок трапляється досить часто. Наприклад, коли виснажений
при поверховій оранці зверхній шар ґрунту дає при старій системі обробітку
дедалі менші врожаї, то витягнений на поверхню з допомогою глибшої
оранки нижній шар за раціонального обробітку починає давати вищі
урожаї, ніж давніш. Але цей сцеціяльний випадок, точно кажучи, сюди не
стосується. Пониження продуктивности першої витрати капіталу в 2\sfrac{1}{2} ф. ст. зумовлює для кращих
земель, навіть коли там припустити аналогічні відношення,
пониження диференційної ренти І; проте тут ми розглядаємо лише диференційну
ренту II. Але тому, що даний спеціяльний випадок не може статися, коли не
припускається існування диференційної ренти II і тому, що він в дійсності
становить відбитий вплив модифікації диференційної ренти І на диференційну
ренту II, то ми наведемо приклад цього випадку.

\begin{table}[h]
  \begin{center}
    \emph{Таблиця VII}
    \footnotesize

  \begin{tabular}{c@{  } c@{  } c@{  } c@{  } c@{  } c@{  } c@{  } c@{  } c@{  } c@{  } с}
    \toprule
      \multirowcell{2}{\makecell{Рід\\ землі}} &
      \multirowcell{2}{Акри} &
      Капітал &
      Зиск &
      \makecell{Ціна\\ продук.} &
      \multirowcell{2}{\makecell{Продукт в\\ квартерах}} &
      \makecell{Продажна \\ ціна} &
      \makecell{Здо-\\буток} &
      \multicolumn{2}{c}{Рента} &
      \multirowcell{2}{\makecell{Норма \\ренти}} \\

      \cmidrule(r){3-3}
      \cmidrule(r){4-4}
      \cmidrule(r){5-5}
      \cmidrule(r){7-7}
      \cmidrule(r){8-8}
      \cmidrule(r){9-9}
      \cmidrule(r){10-10}

       &  & ф. ст. & ф. ст. & ф. ст. & & ф. ст. & ф. ст. & Кварт. & ф. ст. &   \\
      \midrule
      A & 1 & 2\sfrac{1}{2} + 2\sfrac{1}{2} = 5 & 1 & 6 & \phantom{0}\sfrac{1}{2} + 1\sfrac{1}{4} = 1\sfrac{3}{4}                      & 3\sfrac{3}{7} & \phantom{0}6 & 0\phantom{\sfrac{1}{2}} & \phantom{0}0 & \phantom{00}0\% \\
      B & 1 & 2\sfrac{1}{2} + 2\sfrac{1}{2} = 5 & 1 & 6 & 1\phantom{\sfrac{0}{0}} + 2\sfrac{1}{2} = 3\sfrac{1}{2}                     & 3\sfrac{3}{7} & 12           & 1\sfrac{3}{4}           & \phantom{0}6 & 120\% \\
      C & 1 & 2\sfrac{1}{2} + 2\sfrac{1}{2} = 5 & 1 & 6 & 1\sfrac{1}{2} + 3\sfrac{3}{4} = 5\sfrac{1}{4}                               & 3\sfrac{3}{7} & 18           & 3\sfrac{1}{2}           & 12           & 240\%\\
      D & 1 & 2\sfrac{1}{2} + 2\sfrac{1}{2} = 5 & 1 & 6 & 2\phantom{\sfrac{0}{0}} + 5\phantom{\sfrac{0}{0}} = 7\phantom{\sfrac{0}{0}} & 3\sfrac{3}{7} & 24           & 5\sfrac{1}{4}           & 18           & 360\%\\

     \cmidrule(r){3-3}
     \cmidrule(l){6-6}
     \cmidrule(r){8-8}
     \cmidrule(r){9-9}
     \cmidrule(r){10-10}
     \cmidrule(r){11-11}

      Разом & & \phantom{2\sfrac{1}{2} + 2\sfrac{1}{2} =}20 & & & \phantom{2 + 1\sfrac{1}{2} =}17\sfrac{1}{2} & & 60 & 10\sfrac{1}{2} & 36 & 240\%\footnotemarkZ{}\\
  \end{tabular}

  \end{center}
\end{table}
\footnotetextZ{Тут, як і далі в таблицях VIII, IX, і X пересічну норму ренти обчислено не до всього
вкладеного капіталу, а тільки до капіталу, вкладеного в рентодайні дільниці. \emph{Пр. Ред.}} % текст примітки прямо під заголовком

Грошова рента, як і грошовий здобуток, лишаються ті самі, що і в таблиці II.
Підвищена реґуляційна ціна продукції точнісінько покриває те, що втрачено
на кількості продукту; а що ця ціна продукції і кількість продукту змінюються
в зворотному відношенні, то само собою зрозуміло, що здобуток їх лишається
той самий.
