\index{iii2}{0200}  %% посилання на сторінку оригінального видання
Коли б уся земля певної країни, придатна для хліборобства, була вже
здана в оренду, — при чому припускається, як загальне явище, капіталістичний
спосіб продукції і нормальні відносини, — то не було б такої землі, яка не
давала б ренти, але могли б існувати такі приміщення капіталу, окремі частини
капіталу вкладеного в землю, які не давали б ренти; бо, скоро земля здана в
оренду, земельна власність перестає діяти, як абсолютна межа потрібного вкладення
капіталу. Як відносна межа, вона продовжує ще діяти і після цього
в такій мірі, в якій перехід до земельного власника долученного до землі капіталу
ставить тут перед орендарем дуже визначені межі. Тільки в цьому випадку
вся рента перетворилася б на диференційну ренту, на диференційну ренту, яка
визначається не ріжницями в якості землі, а ріжницями між надзисками, що
постають від останніх приміщень капіталу на певній землі, і рентою, яка виплачувалася
б за оренду землі найгіршої кляси. Як межа земельна власність
діє абсолютно лише остільки, оскільки допущення до землі взагалі, як до сфери
приміщення капіталу, зумовлює данину земельному власникові. Коли це допущення
сталося, останній уже не може протиставити ніяких абсолютних меж
кількісному розмірові приміщення капіталу на даній дільниці землі. Будуванню
будинків взагалі кладеться межу земельною власністю третьої особи на ту дільницю
землі, на якій мається збудувати будинок. Але скоро лише ця земля
орендована під будівлю будинків, то вже від орендаря залежить, чи бажає він
збудувати на ній високий чи низький будинок.

Коли б пересічний склад хліборобського капіталу був такий самий або
вищий, ніж пересічний склад суспільного капіталу, то абсолютна рента, знов
таки в щойно дослідженому розумінні, відпала б; тобто відпала б рента, яка
відрізняється так від диференційної ренти, як і від ренти, що ґрунтується на
власне монопольній ціні. Тоді вартість хліборобського продукту не була б
вища від його ціни продукції, і хліборобський капітал пускав би в рух не
більшу кількість праці, отже, реалізував би також не більшу кількість додаткової
праці, ніж нехліборобський капітал. Те саме сталося б тоді, коли б з
проґресом культури склад хліборобського капіталу зрівнявся із пересічним
складом суспільного капіталу.

На перший погляд здається за суперечність припускати, що, з одного
боку, склад хліборобського капіталу підвищується, отже, зростає його стала частина
проти змінної, а з другого боку, що ціна хліборобського продукту має
піднестися остільки високо, щоб нова і гірша, ніж колишня, земля могла виплачувати
ренту, яка в цьому випадку могла б виникнути лише з надміру ринкової
ціни над вартістю і ціною продукції, коротко, лише з монопольної ціни
продукту.

Тут треба відрізняти таке.

Розглядаючи створення норм зиску, ми, насамперед, бачили, що капітали
які, технологічно розглядувані, складені однаково, тобто порівняно з кількістю
машин і сирового матеріялу пускають в рух однакову кількість праці,
можуть, проте, бути різного складу в наслідок того, що сталі частини цих капіталів
мають різну вартість. Сировий матеріял або машини можуть бути в
одному випадку дорожчі, ніж у другому. Щоб пустити в рух таку саму масу
праці (а це, згідно з припущенням, було б потрібне для перероблення такої ж
самої маси сирового матеріялу), в одному випадку довелося б авансувати більший
капітал, ніж у другому, тому що, наприклад, з капіталом 100 я не можу пустити
в рух сднакової кількости праці, коли сировий матеріял, який теж доводиться
оплачувати з цих 100, в одному випадку коштує 40, в другому 20.
Але те, що технологічно ці капітали все ж складені однаково, негайно виявилося
б, скоро ціна дорожчого сирового матеріялу знизилася б до рівня дешевшого.
Відношення вартости змінного і сталого капіталу тоді стали б однакові,
\parbreak{}  %% абзац продовжується на наступній сторінці
