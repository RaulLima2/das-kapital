рента з землі, або як процент на державну позику, обчислюється ним як процент
на гроші, що їх коштувала йому купівля титулу на цей дохід. Як капітал
він може реалізувати їх тільки через перепродаж. Але тоді інший, новий покупець,
потрапляє в таке саме становище, в якому був перший, і жодними переміщеннями
з рук в руки витрачені в такий спосіб гроші не можуть перетворитись
на дійсний капітал для того, хто їх витратив.

При дрібній земельній власності ще дужче зміцнюється та ілюзія, ніби
сама земля має вартість і тому, як капітал, входить в ціну продукції продукту,
цілком так само як машина або сировий матеріял. Але ми бачили, що тільки
в двох випадках рента, а тому і капіталізована рента, ціна землі, може ввійти
визначально в ціну хліборобського продукту. Поперше, коли в наслідок складу
хліборобського капіталу, — капіталу, який не має нічого спільного з капіталом,
витраченим на купівлю землі, — вартість хліборобського продукту стоїть вище за
його ціну продукції, і ринкові відносини дають можливість земельному власникові
використати цю ріжницю. Подруге, коли є монопольна ціна. І те і те
найрідше трапляється при парцелярному господарстві і дрібній земельній
власності, бо саме тут продукція в дуже великій частині своїй задовольняє
власні потреби, і відбувається незалежно від регулювання загальною нормою
зиску. Навіть там, де парцелярне господарство провадиться на орендованій землі,
орендна плата в незрівняно більшій мірі, ніж за будь-яких інших відносин,
має в собі частину зиску і навіть вирахування з заробітної плати; в такому
випадку це є рента лише номінально, а не рента як самостійна категорія
протилежно до заробітної плати і зиску.

Отже, витрата грошового капіталу на купівлю землі не є приміщення хліборобського
капіталу. Вона є pro tanto скорочення того капіталу, що ним могли б
порядкувати дрібні селяни в сфері самої продукції. Вона зменшує pro tanto розмір
їхніх засобів продукції і тому звужує економічну базу репродукції. Вона
упідлеглює дрібного селянина лихвареві, бо в цій сфері власне кредит взагалі
трапляється рідше. Вона є за гальмо для хліборобства, навіть тоді, коли цю
купівлю роблять великі поміщицькі господарства. Вона дійсно суперечить капіталістичному
способові продукції, що йому взагалі байдужа заборгованість
земельного власника, — байдуже те, чи одержав він свій маєток в спадщину, чи
купив. Чи сам він забирає ренту, чи мусить і він її віддати гіпотечному кредиторові,
— це само по собі нічого не змінює в господарюванні на орендованому
маєтку.

Ми бачили, що за даної земельної ренти ціна землі реґулюється розміром проценту.
Коли процент низький, то ціна землі висока, і навпаки. Отже, нормально висока
ціна землі і низький розмір проценту мусили б іти поруч, так, що коли б селянин
в наслідок низького розміру проценту дорого заплатив за землю, то цей самий низький,
розмір проценту мусив би призвести до того, що він на сприятливих умовах здобув би в кредит капітал
для господарювання. В дійсності при пануванні парцелярної
власности справа стоїть інакше. Насамперед, до селян не стосуються загальні закони
кредиту, бо вони мають за свою передумову, що продуцент є капіталіст.
Подруге, там, де панує парцелярна власність, — про колонії тут немає мови —
і парцелярний селянин являє собою головний стовбур нації, створення капіталу,
тобто суспільна репродукція, є порівняно мала, і ще менше створення позикового
грошового капіталу в раніш викладеному розумінні. Воно має своєю передумовою
концентрацію і наявність кляси багатих нероб капіталістів (Massie). Потретє, тут,
де власність на землю становить життєву умову для більшої частини продуцентів
і неодмінну сферу для приміщення їхнього капіталу, ціна землі підвищується,
незалежно від розміру проценту і часто в зворотному відношенні до
нього, в наслідок переваги попиту на земельну власність над поданням. Земля,
продавана парцелями часто дає тут геть вищі ціни, ніж при продажу великими
