\parcont{}  %% абзац починається на попередній сторінці
\index{iii2}{0238}  %% посилання на сторінку оригінального видання
ґрунтується на різних індивідуальних вартостях продукту різних родів землі,
де буде лише щойно згадане визначення; оскільки вона, подруге, ґрунтується на
відмінній від цих індивідуальних вартостей реґуляційній загальній ринковій
вартості, це буде суспільний, здійснюваний за посередництвом конкуренції закон,
який не має ніякого чинення ні до землі, ні до різних ступенів її родючости.

Могло б здаватися, що, принаймні, в формулі: «праця — заробітна плата»
висловлено раціональне відношення. Але цього тут немає так само, як у формулі:
«земля — земельна рента». Оскільки праця створює вартість і втілюється
в вартості товарів, вона не має ніякого чинення до розподілу цієї вартости між
різними категоріями. Оскільки ж вона має спецефично суспільний характер
найманої праці, вона не створює вартости. Взагалі, вже раніш показано, що
заробітна плата або ціна праці є лише іраціональний вираз вартости, або ціни
робочої сили; і певні суспільні умови, в яких продається ця робоча сила, не
мають жодного чинення до праці як загального аґента продукції. Праця зрічевлюється
і в тій складовій частині вартости товару, яка в вигляді заробітної
плати становить ціну робочої сили; вона створює цю частину цілком так само
як інші частини продукту; але вона зрічевлюється в цій частині не більше і
не інакше, ніж у частинах, які становлять ренту або зиск. І взагалі, коли ми
фіксуємо працю в її властивості створювати вартість, то ми розглядаємо її не
в її конкретній формі, яку вона має як умова продукції, а в її суспільній
визначеності, яка відрізняється від визначености її як найманої праці.

Навіть вираз: «капітал — зиск» тут є неправильний. Коли капітал береться
в тому однісінькому відношенні, в якому він продукує додаткову вартість,
саме в його відношенні до робітника, що в ньому він витискує додаткову працю
примусом, який він справляє на робочу силу, тобто на найманого робітника,
то ця додаткова вартість має в собі крім зиску (підприємницкий бариш плюс
процент) і ренту, коротко, всю ще неподілену додаткову вартість. Навпаки тут
капітал як джерело доходу ставиться в відношення тільки до тієї частини, яка
дістається капіталістові. А це ж не додаткова вартість, яку він взагалі здобуває,
а лише та її частина, яку він здобуває для капіталіста. Всякий внутрішній
зв’язок відпадає ще більше, коли ця формула перетворюється на таку: «капітал
— процент».

Коли ми, поперше, розглядали те, що ділить три джерела, то тепер, подруге,
бачимо, що їхні продукти, їхні витвори, доходи, навпаки, належать до тієї
самої сфери, до сфери вартости. Проте, це знищується тим (це є відношення
не тільки між неспівмірними величинами, але й між такими речами, що не
мають жодної загальної міри, жодного відношення одна до однієї, і які є непорівнюванні),
що капітал, подібно до землі й праці, на ділі розглядають просто за
його речевою субстанцією, отже, просто як випродукований засіб продукції,
причому абстрагуються від нього, як від відношення до робітника, і так само
абстрагуються від нього як вартости.

Потрете. Отже, в цьому розумінні формула: капітал — процент (зиск), земля
— рента, праця — заробітна плата являє однакову в усіх частинах і симетричну
невідповідність. Дійсно, якщо наймана праця уявляється не як суспільно
визначена форма праці, але всяка праця з своєї природи уявляється найманою
працею (так уявляється справа заплутаному капіталістичними відносинами
продукції), то й певні специфічні суспільні форми, що їх набувають речові умови
праці — випродуковані засоби продукції і земля — проти найманої праці
(так само як вони навпаки, з свого боку, мають передумовою заробітну
плату) прямо збігаються з речевим буттям цих умов праці, або з тим
виглядом, який вони взагалі мають у дійсному процесі праці, незалежно
від його всякої історично-визначеної суспільної форми, і навіть незалежно від
усякої його суспільної форми. Таким чином, форма умов праці, що має характер
\parbreak{}  %% абзац продовжується на наступній сторінці
