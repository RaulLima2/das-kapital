Для зручности порівняння поновимо насамперед таку таблицю:

Таблиця І.

Земля    Акри    Капітал ф. ст. Зиск ф. ст. Ціна продукції в квартерах    Продукт в квартерах   
Збіжжева рента Кварт. Грошова  рента ф. ст. Норма надзиску

А    1    2 1/2    1/2    3       1     0    0    0
В    1    2 1/2    1/2    1 1/2   2     1    3    120%
С    1    2 1/2    1/2    1       3     2    6    240%
D    1    2 1/2    1/2    3/4       4    3    9      360%
Разом   4    10          10   6      18    180%  пересічно

Коли ми тепер припустимо, що цифра 16 квартерів, що їх даватимуть землі
В, С, D за низхідної норми продуктивности, достатня для того, щоб вилучити
А з числа оброблюваних земель, то таблиця III перетворюється на таку:

Таблиця V.

Рід  землі    Акри    Вкладення  капіталу ф. ст. Зиск ф. ст. Продукт в квартерах    Продажна ціна ф.
ст. Здобуток ф. ст. Збіжж. рента Кварт. Грошова рента ф. ст. Норма надзиску

B     1    2 1/2 + 2 1/2    1    2 + 1 1/2 = 3 1/2    1 5/7    6            0            0          
 0
C    1    2 1/2 + 2 1/2    1    3 + 2 = 5                    1 5/7    8 4/7    1 1/2     2 4/7    51
2/5%
D    1    2 1/2 + 2 1/2    1      4 + 3 1/2 = 7 1/2    1 5/7    12 6/7    4            6 6/7    137
1/5%
Разом.    3    15        16        27 3/7  5 1/2     9 3/7     94 3/10%  пересічно *)

Тут за низхідної норми продуктивности додаткових капіталів і за різного
ступеня цього зменшення на різних землях, реґуляційна ціна продукції знизилася
з 3 ф. стерл. до 1 5/7 ф. стерл. Вкладення капіталу збільшилося наполовину з 10 ф.
стерл. до 15 ф. стерл. Грошова рента зменшилася майже удвоє, з 18 до 9 3/7 ф.
стерл., але збіжжева рента лише на 1/2 **), з 6 квартерів до 5 1/2. Весь продукт
збільшився з 10 до 16, або на 60% ***). Збіжжева рента становить небагато більше
від третини всього продукту. Авансований капітал відноситься до грошової ренти
як 15: 9 3/7, тимчасом як давніш це відношення було 10: 18.

III. За висхідної норми продуктивности додаткових капіталів.

Цей випадок відрізняється від варіянту І, наведеного на початку цього
розділу, де ціна продукції за незмінної норми продуктивности знижується, тільки
тим, що коли потрібна додаткова кількість продукту для того, щоб вилучити
землю А, то це відбувається тут швидше.

Так за низхідної, як і за висхідної продуктивности додаткових вкладень
капіталу може це різно впливати, залежно від того, як ці вкладення розподіляються
між різними родами землі. В міру того, як цей різний вплив

*) Тут пересічну норму надзиску обчислено не до всього вкладеного капіталу, а тільки до капіталу,
вкладеного в рентодайні дільниці С і D. Прим. Ред.

**) В німецькому тексті тут стоїть «1/22». Очевидна помилка. Прим. Ред.

***) В німецькому тексті тут помилково стоїть: «160%». Прим. Ред.
