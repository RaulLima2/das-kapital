\parcont{}  %% абзац починається на попередній сторінці
\index{iii2}{0274}  %% посилання на сторінку оригінального видання
цьому мають на увазі різні титули на ту частину продукту, що припадає
особистому споживанню. Навпаки, ті розподільчі відносини, про які ми щойно
говорили, є базою окремих суспільних функцій, що припадають в межах самого
продукційного відношення певним діячам його, у протилежність безпосереднім
продуцентам. Вони надають самим умовам продукції та їхнім представникам специфічно
суспільної якости. Вони визначають увесь характер і ввесь рух продукції.

Дві характеристичні риси від самого початку відзначають капіталістичний
спосіб продукції:

\emph{Поперше}. Він продукує свої продукти як товари. Не продукування товарів
відрізняє цей спосіб продукції від інших способів продукції, а те, що для
його продуктів переважною і визначальною рисою є їхній товарний характер. Це
включає насамперед і те, що сам робітник виступає як продавець товару, і тому
як вільний найманий робітник, отже, праця виступає взагалі як наймана праця.
Після всього того, що ми виклали до цього часу, зайво було б знову доводити
тут, як відношення між капіталом і найманою працею визначає ввесь характер даного
способу продукції. Головні діячі самого цього способу продукції, капіталіст і
найманий робітник, являють, як такі, лише втілення, персоніфікацію капіталу і
найманої праці; це — певні суспільні характери, що їх надає індивідуумам суспільний
процес продукції; це — продукти даних певних суспільних продукційних
відносин.

Характер 1) продукту як товару, і 2) товару як продукту капіталу вже
включає всю сукупність відносин циркуляції, тобто включає певний суспільний
процес, що його мусять проробити продукти, і в якому вони набирають
певного суспільного характеру; так само він включає певні відносини між
діячами продукції, відносини, що ними визначається вживання їхнього продукту
і його зворотне перетворення чи на засоби існування, чи на засоби
продукції. Але навіть лишаючи це осторонь, з зазначених вище двох характеристичних
особливостей продукту як товару, або товару як капіталістично
випродукованого товару, випливає все визначення вартости і регулювання вартістю
сукупної продукції. В цій цілком специфічній формі вартости праця має
значіння, з одного боку, тільки як суспільна праця; з другого боку, розподіл
цієї суспільної праці і її взаємне довершення, обмін речовин її продуктів, її
упідлеглення перебігові суспільного механізму і включення в цей останній, —
все це підпадає випадковим потягам поодиноких капіталістичних продуцентів,
потягам, що взаємно знищуються. А що ці капіталістичні продуценти протистоять
один одному лише як товаропосідачі, при чому кожен намагається
продати свій товар можливо дорожче (і навіть при урегулюванні продукції
керується нібито тільки своєю сваволею), то внутрішній закон пробивається
лише за посередництвом їхньої конкуренції, їхнього взаємного тиснення один
на одного, що ним взаємно знищуються всі відхили. Лише як внутрішній
закон, що виступає проти окремих діячів продукції як сліпий закон природи,
діє тут закон вартости і проводить суспільну рівновагу продукції серед її випадкових
флюктуацій.

Далі, вже в товарі, і ще більше в товарі як продукті капіталу, включено
зрічевлення суспільно-продукційних визначень і уособлення матеріяльних
основ продукції, що характеризує ввесь капіталістичний спосіб продукції.

\emph{Друга} особливість, що спеціально визначає капіталістичний спосіб продукції,
це є продукція додаткової вартости як безпосередня мета і визначальний
мотив продукції. Капітал продукує переважно капітал, і він досягає цього
лише остільки, оскільки продукує додаткову вартість. При дослідженні відносної
додаткової вартости і, далі, при дослідженні перетворення додаткової вартости
на зиск, ми бачили, як на цьому ґрунтується характеристичний для капіталістичного
періоду спосіб продукції, — особлива форма розвитку суспільних продук-
\parbreak{}  %% абзац продовжується на наступній сторінці
