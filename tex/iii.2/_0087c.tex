\index{iii2}{0087}  %% посилання на сторінку оригінального видання
Що — згідно з тезою 3) при переповненні ринку державець товарів збуватиме
їх дешевше — якщо взагалі зможе продавати їх, — ніж при сподіванці на
швидке вичерпанні наявних запасів, це досить зрозуміла річ. Але менш зрозуміло,
чому з тієї причини має спадати рівень проценту.

Якщо ринок переповнено імпортованими товарами, то рівень проценту може
підноситися в наслідок збільшеного попиту з боку власників, на позичковий
капітал, щоб не довелось їм будь-щобудь викидати товари на ринок. Він може
спадати, бо легкість добування комерційного кредиту тримає попит на банковий
кредит ще на порівняно низькому рівні.

Economist згадує про швидкий вплив на курс 1847 року, викликаний
підвищенням рівня проценту та іншим тисненням на грошовий ринок. Однак не
треба забувати, що не зважаючи на зміну курсів, золото й далі відпливало
до кінця квітня; зміна настала тут тільки на початку травня.

1 січня 1847 року металевий скарб банку становив 15066691 ф. ст.;
рівень проценту 3 1/2\%; тримісячний курс на Париж 25,75; на Гамбурґ 13.10;
на Амстердам 12.3 1/4 5-го березня металевий скарб впав до 11595535 ф. ст.;
дисконт піднісся до 4\%; вексельний курс впав на Париж до 25.67 1/2, на Гамбурґ
до 13.9 1/4, на Амстердам до 12.2 1/2. Відплив золота триває далі; див.
нижченаведену таблицю.

1847 рік
Скарб благородного металу в Англ. банку ф. ст.
Грошовий ринок
Найвищий 3 місячний курс
            Париж    Гамбурґ    Амстерд
Березня 20  11231630   Банк. диск. 4\%   25.67 1/2   13.9 3/4  12.2 1/2
Квітня 3  10246410 „„ 5\%   25.80  13.10     12.3   1/2
„10  9867053  Грошей дуже обмаль 25.90   13.10 1/4   12.4 1/2
„17  9329941  Банк. диск.  5 1/2\%    26.02 1/2    13.10 3/4       12.5 1/2
„24  9213890  Пригнічення       26.05   13.13     12.6
Травня 1 9337716 Висхідне пригнічення 26.15      13.12 3/4    12.6 1/2
„8  9588759  Найбільше « 26.27 1/2      13.15 1/2      12.7 3/4

В 1847 році цілий експорт благородного металу з Англії становив
8602597 ф. ст.

ф. ст.

З того пішло до Сполучених Штатів..........................3226411

»»    »» Франції..........................2479892

 »»    »» Ганзейських міст..........................958781

»»    »» Голяндії..........................247743

Не зважаючи на зміну курсів наприкінці березня відплив золота триває
далі ще протягом цілого місяця: ймовірно, до Сполучених Штатів.

«Ми бачимо тут» [каже Economist 1847, р. 984], як швидко й різко піднесення
рівня проценту й грошова скрута, що постала по тому піднесенні, виправили
несприятливий курс та змінили поток золота, так що золото знову почало
пливти до Англії. Цей вплив склався цілком незалежно від платіжного балянсу.
Вищий рівень проценту породив нижчу ціну на цінні папери, англійські й
чужоземні, спонукавши до великих закупів тих паперів за закордонний рахунок.
\parbreak{}  %% абзац продовжується на наступній сторінці
