доводиться зробити певну витрату, коли доводиться оплатити щось, що давніше
не оплачувалося. Бо під покриттям капіталу, зужиткованого в продукції, слід
розуміти покриття тільки вартостей, що являють собою певні засоби продукції,
Елементи природи, що входять як аґенти, у продукцію, без жодних витрат,
хоч би яку ролю вони завжди відігравали в продукції, входять в неї не як складові
частини капіталу, а як дарова природна сила капіталу, тобто як дарова природна
продуктивна сила праці, яка на базі капіталістичного способу продукції
виступає, подібно до всякої продуктивної сили, як продуктивна сила капіталу.
Отже, коли в продукції бере участь така природна сила, яка первісно
нічого не коштуй, то вона не входить в розрахунок при визначенні ціни, поки
продукт, впготовлюваний з її допомогою, достатній для задоволення потреби.
Але коли в перебізі розвитку потрібно більше продукту, ніж можна виготувати
з допомогою цієї природної сили, тобто коли доведеться випродукувати
цей додатковий продукт без допомоги цієї природної сили, або з допомогою
людини, людської праці, то в капітал ввійде новий додатковий елемент. Отже,
для одержання колишнього продукту потрібно буде відносно більшої витрати
капіталу. За інших рівних умов відбудеться подорожчання продукції.

(З зошиту «Початого в половині лютого 1876 року»),

Диференційна рента і рента як просто процент на капітал,
долучений до землі.

Так звані сталі меліорації, — які змінюють фізичні, почасти й хемічні
властивості ґрунту через операції, що коштують витрат капіталу, і що можуть
розглядатися, як долучення капіталу до землі, — майже всі сходять на те,
щоб певній дільниці землі, ґрунтові в певному обмеженому місці надати таких
властивостей, що їх інший ґрунт в другому місці, часто зовсім близько, має
з природи. Одна земля нівельована з природи, іншу ще доводиться нівелювати:
одна має природні водозбіги, інша потребує штучного дренажу; одна з природи
має глибокий орний шар, на іншій його треба поглибити штучно; один глинястий
ґрунт з природи змішаний з належною кількістю піску, у іншого ще
треба штучно створити це відношення, одна лука зрошується з природи, або
вкривається шаром намулу, на іншій цього доводиться досягати працею, або,
кажучи мовою буржуазної економії, капіталом.

Справді чудна теорія, за якою тут на одній землі, що її відносні вигоди
придбані, рента є процент, а на іншій землі, яка має ці вигоди з природи, не
є процент. (На ділі, при застосуванні цієї теорії, справа перекручується так, що в
наслідок того, що в одному випадку рента дійсно збігається з процентом, її і в інших
випадках, коли фактично цього немає, мусять називати процентом, перебріхують
в процент). Але після того, як зроблено витрату капіталу, земля дає ренту не тому,
що в неї вкладено капітал, а тому, що витрата капіталу зробила землю продуктивнішою
ділянкою приміщення капіталу проти давнішого. Припустімо,
що вся земля певної країни потребує такої витрати капіталу; в такому разі
кожна дільниця землі, на якій вона ще не була зроблена, муситиме пройти цю
стадію і рента (процепт, який в даному випадку дає земля), що її дає земля, на
якій вже було зроблено таку витрату капіталу, так само становить диференційну
ренту, як коли б ця земля з природи мала цю перевагу, а інша земля
мусила б її набувати лише штучним шляхом.

І ця зводжувана до проценту рента стає чистою диференційною рентою,
скоро витрачений капітал буде амортизовано. Інакше той самий капітал як
капітал мусив би існувати подвійно.
