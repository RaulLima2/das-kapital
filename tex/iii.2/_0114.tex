\parcont{}  %% абзац починається на попередній сторінці
\index{iii2}{0114}  %% посилання на сторінку оригінального видання
продукт Англії сходить на 8 міл. ф. ст., а ввесь продукт — на 28 міл. ф. ст.,
як у Франції. Справді це трохи надто сильно, коли п. Лавернь, вводить у свое
обчислення одночасно і кількість продукту і ріжницю цін, так що, коли Англія
продукує певні речі дорожче, ніж Франція, — що визначає тільки більший зиск
для орендарів і земельних власників, — то це виступає як перевага англійського
хліборобства.

Що п. Лавернь не тільки обізнаний з економічними наслідками англійського
сільського господарства, але і поділяє забобони англійських орендарів
і землевласників, це він показує на стор. 48: «Велика шкода звичайно поєднана
з зерновими рослинами... вони виснажують ґрунт, на якому ростуть».
Пан Лавернь не тільки гадає, що інші рослини цього не роблять; він гадає також,
що кормові рослини і корінняки збагачують ґрунт: «Кормові рослини беруть
головні елементи свого росту з атмосфери, тимчасом як ґрунтові вони повертають
більше, ніж беруть з нього. Таким чином, вони подвійним способом — так
безпосередньо як і в наслідок їхнього перетворення на тваринний гній — сприяють
покриттю шкоди, спричиненої зерновими рослинами та іншими виснажними
культурами; тому слід взяти за правило, щоб вони, принаймні, чергувалися з цими
культурами; у цьому і полягає сівозміна Норфолку» (р. 50,51).

Не дивно, що п. Лавернь, котрий вірить цим казкам англійської сільської
благодушности, вірить їй і в тому, що після скасування хлібних мит заробітна плата
англійських поденників ніби втратила свій колишній ненормальний характер.
Дивись, що ми говорили про це давніш, книга І, розд. ХХІ11,5. Проте, послухаймо
ще промови п. Джона Брайта, проголошеної в Бірмінгемі 14 грудня 1865 року.
Поговоривши про 5 мільйонів родин, зовсім не репрезентованих у нарляменті,
він каже далі: «З них 1 мільйон, або, радше, понад 1 мільйон у Сполученому
Королівстві припадає на записаних у зловісні списки павперів. Потім ще один
мільйон, які все ще тримаються вище рівня павперизму, але яким постійно
загрожує небезпека теж зробитися павперами. Їхнє становище й їхні перспективи
не кращі. Подивіться ж, нарешті, на темні, низькі верстви цієї частини суспільства.
Подивіться на їхнє становище покинутих, їхню бідність, їхні муки, їхню цілковиту
безнадійність. Навіть у Сполучених Штатах, навіть у Південних штатах за
панування рабства, кожен негр мав надію, що й йому колись буде воля.
Але для цих людей, для цієї маси нижчих верств в нашій країні не існує —
я можу тут прямо сказати це — ні надії на якесь поліпшення, ні навіть
потягу до цього. Чи читали ви недавно в газетах замітку про Джона Кросса,
хліборобського робітника в Дорсетшайрі? Він працював по 6 день на тиждень,
мав дуже добре посвідчення від свого господаря, на якого він працював 24 роки,
одержуючи тижнево по 8 шилінґів. Джон Кросс на цю заробітну плату мав
утримувати у своїй хатині родину з 7 дітей. Щоб зігріти свою хвору жінку та
її немовлятко, він узяв — висловлюючись язиком закону, можу сказати, украв, —
дерев’яний тин вартістю 6 пенсів. За цей вчинок мирові судді присудили його
до 14 або 20 днів ув’язнення. Я можу вам сказати, що в цілій країні можна
знайти багато тисяч таких випадків, як з Джоном Кроссом, особливо на півдні,
і що становище цих людей таке, що до цього часу найуважніший дослідник не
в силі викрити таємниці, як у них душа в тілі тримається. А тепер подивіться
на всю країну і погляньте на ці 5 мільйонів родин і на одчайдушне становище
цієї верстви. Чи не можна справді сказати, що маса нації, позбавлена
виборчого права, мучиться над роботою, вічно мучиться над роботою і майже
ніколи не знає відпочинку? Порівняйте її з владущою клясою — але, коли я це
зроблю, то мене обвинуватять у комунізмі... але порівняйте цю велику націю,
що побивається над працею і позбавлена виборчого права, з тією частиною,
в якій можна бачити владущі кляси. Подивіться на їхнє багатство, їхню пишність,
їхні розкоші, подивіться на їхню утому — бо й серед них помічається утома,
\parbreak{}  %% абзац продовжується на наступній сторінці
