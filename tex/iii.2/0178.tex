тера порівняно з реґуляційною ціною продукції на землі А знову зменшився
б з 1 ф. стерл, до 9/13 ф. стерл. Але ці 9/13 ф. стерл. тут слід
помножити на 6 1/2 квартерів замість колишніх 4 1/2. А 9/13×6 1/2 = 1× 4 1/2 = 4 1/2 ф. стерл.

Звідси насамперед випливає, що за цих обставин не потрібно жодного підвищення
реґуляційної ціни продукції для того, щоб уможливити додаткові витрати
капіталу на рентодайних землях, навіть в такому розмірі, що додатковий
капітал зовсім перестає давати надзиск і дає ще лише пересічний зиск. З
цього випливає далі, що тут сума надзиску на акр лишається без зміни,
хоч би як дуже зменшувався надзиск з квартера; це зменшення завжди урівноважується
відповідним збільшенням квартерів, продукованих на акрі. Для того,
щоб пересічна ціна продукції піднеслась до рівня загальної ціни продукції (отже,
тут досягла б 3 ф. стерл. на землі В), мусять бути зроблені такі додаткові витрати
капіталу, продукт яких мав би вищу ціну продукції, ніж реґуляційна ціна
в 3 ф. стерл. Але ми побачимо, що тільки цього ще не досить, щоб підвищити
пересічну ціну продукції квартера на землі В до рівня загальної ціни продукції
в 3 ф. стерл.

Припустімо, що на землі В було випродуковано:

1) 3 1/2 квартери, що їхня ціна продукції, як і давніш, 6 ф. стерл.; отже,
дві витрати капіталу по  2 1/2 ф. стерл. кожна, при чому обидві дають надзиски,
але низхідної висоти.

2) 1 квартер за 3 ф. стерл.; витрата капіталу, при якій індивідуальна
ціна продукції дорівнювала б реґуляційній ціні продукції.

3) 1 квартер за 4 ф. стерл.; витрата капіталу, при якій індивідуальна
ціна продукції на 25\% вища за реґуляційну ціну.

Ми мали б тоді 5 1/2  квартерів з акра за 13 ф. стерл. при витраті капіталу
в 10 ф. стерл.; первісна витрата капіталу зросла б учетверо, але продукт
першої витрати капіталу не збільшився б і втроє.

5 1/2 квартерів за 13 ф. стерл. дають пересічну ціну продукції в 2 4/11 ф. стерл.
за квартер, отже, за реґуляційної ціни продукції в 3 ф. стерл. надмір
в 7/11 ф. стерл. з квартера, який може перетворитися на ренту. 5 1/2 квартерів,
продані по реґуляційній ціні в 3 ф. стерл. дають 16 1/2 ф. стерл. За вирахуванням
ціни продукції в 13 ф. стерл. залишається 3 1/2 ф. стерл. надзиску, або
ренти, так що ці 3 1/2 ф. стерл., рахуючи по теперішній пересічній ціні продукції
квартера з землі В, тобто по 2 4/11 ф. стерл. за квартер, репрезентують
1 25/52 *) квартера. Грошова рента понизилася б на 1 ф. стерл., збіжжева
рента приблизно на 1/2 квартера, але, не зважаючи на те, що четверта додаткова
витрата капіталу на В не тільки не створює надзиску, але дає менше, ніж
пересічний зиск, — як і давніш, існує надзиск і рента. Коли ми припустимо, що,
крім витрати капіталу 3), і витрата 2) продукує по ціні, що перебільшує реґуляційну
ціну продукції, то вся продукція становитиме: 3 1/2 квартери за
6 ф. ст. + 2 квартери за 8 ф. ст., разом 5 1/2 квартерів за 14 ф. ст. ціни продукції.
Пересічна ціна продукції квартера була б 2 6/11  ф. ст., що давало б надмір
в 5/11 ф. ст. Ці  5 1/2  квартери, продані по 3 ф. ст., дають 16 1/2 ф. ст.; за вирахуванням
14 ф. ст. ціни продукції, лишається 2 1/2 ф. ст. на ренту. За теперішньої
пересічної ціни продукції на землі В це становило б 55/56 квартера. Отже, рента
все ще одержується, хоч і в меншому розмірі, ніж давніш.

В усякому разі це показує, що на кращих земельних дільницях при додаткових
витратах капіталу, що їхній продукт коштує дорожче, ніж реґуляційна
ціна продукції, рента, принаймні в межах допустимих практикою, мусить не
зникнути, а лише зменшитися, і саме відповідно до того, з одного боку, яку

*) В німецькому тексті тут стоїть: «1 5/72». Очевидна помилка. Прим. Ред.
