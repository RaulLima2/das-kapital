само можуть вони раз за разом знову правити за засіб для повернення тих
позик». (Book II, chap. IV.)

Що та сама монета може здійснювати декілька купівель, відповідно до швидкости
своєї циркуляції, то й може вона так само здійснювати декілька позик, бо купівлі
переносять її з одних рук до інших, а позика становить тільки перенесення з одних
рук до інших, яке відбувається без посередництва купівлі. Для кожного з продавців
гроші є перетворена форма його товару; в наші дні, коли кожна вартість набирає
вигляду капітальної вартости, гроші представляють у різних позиках послідовний
ряд різних капіталів, а це є лише інший вислів тієї попередньої тези, що гроші
можуть послідовно реалізувати різні товарові вартості. Одночасно правлять
вони за засіб циркуляції для того, щоб переносити речові капітали з одних рук до
інших. При позиках гроші переходять з рук до рук не як засіб циркуляції. Поки
вони лишаються в руках позикодавця, вони є в його руках не засіб циркуляції, а
форма існування вартости його капіталу. І в цій формі він передає гроші, позичаючи
їх третій особі. Коли б А позичив гроші В, а В позичив їх С без
посередництва продажів, то ті самі гроші становили б не три капітали, а тільки
один, тільки одну капітальну вартість. Скільки капіталів в дійсності представляють
гроші, це залежить від того, як часто вони функціонують, як форма
вартости різних товарових капіталів.

Те саме, що А. Сміт каже про позики взагалі, має силу й для вкладів,
бо вони є лише особлива назва для тих позик, що їх публіка дає банкірам.
Ті самі гроші можуть бути знаряддям для будь-якого числа вкладів.

«Безперечно правильно, що ті 1000 ф. ст., які хтось сьогодні склав в А,
завтра знову витратиться, й вони являтимуть вклад у В. Другого дня, повернуті
від В, можуть вони являти вклад у С і так далі без кінця. Тому ті самі
1000 ф. ст. у грошах можуть в наслідок ряду передач помножитися до такої
суми вкладів, що її абсолютно не сила визначити. Отже можливо, що 9/10 усіх
вкладів у Сполученому Королівстві існують лише в бухгальтерських записах
по книгах банкірів, що своєю чергою мають провадити між собою розрахунки...
Напр., в Шотляндії, де кількість грошей в циркуляції ніколи не була більшою
за 3 мільйони ф. ст., сума вкладів проте становила 27 мільйонів. І коли б не
було загального штурму на банки по депозити, то ті самі 1000 ф. ст. могли б
перебігаючи свій шлях назад, з тією самою легкістю вирівняти знову таку суму,
що її так само не сила визначити. А що ті самі 1000 ф. ст., що ними сьогодні
хтось оплатив борг крамареві, завтра можуть вирівняти борг останнього купцеві,
день потому — борг купця банкові і т. д. без краю, то і можуть ті сам,
1000 ф. ст. мандрувати з рук до рук і від банку до банку та вирівнювати
всяку суму вкладів, яку тільки можна уявити». (The Currency Question Reviewed,
p. 162, 163).

А що все в цій кредитовій системі подвоюється та потроюється, перетворюючись
у просту химеру, то й має це силу й для «запасних фондів», що в них
нарешті сподіваються знайти дещо солідне.

Послухаймо знову пана Моріса, управителя Англійського банку: «Резерви
приватних банків перебувають в руках Англійського банку, як вклади. Перший
вплив відпливу золота, здається, зачіпає тільки Англійський банк; але той
відплив так само впливатиме й на резерви інших банків, бо він є відплив
частини тих резервів інших банків, що їх вони мають в нашому банку. Так
само впливатиме він і на резерви всіх провінціяльних банків» (Commercial
Distress 1847—48). Отже, кінець-кінцем, запасні фонди у дійсності сходять на
запасний фонд Англійського банку\footnote{
[Як дуже розвинулося таке становище з того часу, показують наведені далі офіційні відомості,
взяті з Daily New s з дня 15 грудня 1892 р., про банкові резерви п’ятнадцятьох найбільших
лондонських
банків у листопаді 1892 р: (див. далі стор. 14).
}. Але й цей запасний фонд має знову двоїсте