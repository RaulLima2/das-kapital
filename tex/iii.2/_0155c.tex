\index{iii2}{0155}  %% посилання на сторінку оригінального видання
На землі Б збіжжева рента проти таблиці I зросла з 3 \footnote*{
В німецькому тексті стоїть: «з 2 квартерів». Явна помилка, як це можна бачити з таблиці І
Прим. Ред.
} квартерів до 6
тимчасом як грошова рента лишилася, як і давніш, 9 ф. стерл.. Проти таблиці II
збіжжева рента з D лишилася колишня, 6 квартерів, але грошова рента знизилась
з 18 ф. стер, до 9 ф. стерл.

Коли розглядати загальні суми ренти, то збіжева рента таблиці IVb = 8
квартерам, більша, ніж рента в таблиці І, що дорівнює 6 квартерам, і більша,
ніж рента в таблиці IVа, що дорівнює 7 квартерам; і навпаки, вона менша, ніж
рента в таблиці II = 12 кварт. Грошова рента в таблиці IVb = 12 ф. стерл.,
більша, ніж грошова рента в таблиці ІVа = 101/2 ф-стерл., і менша від грошової
ренти таблиці І = 18 ф. стерл. і таблиці II = 36 ф. стерл.

Щоб по відпаданні ренти з В в умовах таблиці IVb загальна сума ренти
дорівнювала такій у таблиці I, ми мусимо одержати ще на 6 ф. стерл.
надпродукту, тобто 4 квартери по 11/2 ф. стерл., що є новою ціною продукції.
Тоді ми знову маємо загальну суму ренти в 18 ф. стерл., як у таблиці І. Величина
потрібного на це додаткового капіталу буде різна залежно від того, чи
вкладемо ми його в С або D, чи розподілимо його між обома родами землі.

На С капітал в 5 ф. стерл. дає 2 квартери надпродукту, отже, 10 ф. ст..
додаткового капіталу дадуть 4 квартери додаткового надпродукту. На D було б
досить додаткової витрати в 5 ф. стерл., щоб випродукувати 4 квартери додаткової
збіжжевої ренти при зробленому тут основному припущенні, що продуктивність
додаткових капіталовкладень лишається та сама. Тому здобуваємо
такі наслідки.

Таблиця IVc.

Рід землі    Акри    Капітал    Зиск    Ціна  продукції    Продукт в квартер. Продажна  ціна   
Здобуток    Рента        Норма надзиску
        ф. стер    ф. стер. ф. стер. ф. стер. ф. стер. кварт. ф. стер.
B            1    5            І           6     4    11/2    6    0    0    0
C            1    15            3          18    18    11/2    27    6    9    60\%
D            1    71/2    11/2    9    12    11/2    18    6    9    120\%
Разом.    3    271/2    51/2    33    34                51    12    18

Таблиця VId.

Рід землі    Акри    Капітал    Зиск    Ціна  продукції    Продукт в квартер. Продажна  ціна   
Здобуток    Рента        Норма надзиску
        ф. стер    ф. стер. ф. стер. ф. стер. ф. стер. кварт. ф. стер.

В           1    5            1          6    4    11/2    6     0     0    0
С           1    5            1          6    6    11/2    9     2      3     60\%
D           1    121/2    21/2    15    20    11/2    30    10    15    120\%
Разом.    3    221/2    41/2    27    30                45    12    18
