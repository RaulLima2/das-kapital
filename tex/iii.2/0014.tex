існування. Запасний фонд banking department (банкового відділу) є рівний
надмірові банкнот, що їх банк має право видати понад ті банкноти, що є в
циркуляції. Усталений законом максимум банкнот до емісії є = 14 мільйонам
(для цього не треба жодного металевого резерву; це є приблизна сума боргу
держави банкові) плюс сума запасу благородних металів банку. Отже, коли цей запас
є = 14 мільйонам ф. ст., то банк може емітувати банкнот на 28 мільйонів ф. ст.,
а коли 20 мільйонів з тієї суми є в циркуляції, то запасний фонд банкового відділу
є = 8 мільйонам. Ці 8 мільйонів банкнот становлять тоді законний банкірський
капітал, що ним має порядкувати банк, а одночасно й запасний фонд для його
вкладів. Отже, коли настане відплив золота, що зменшить металевий запас на
6 мільйонів — а в наслідок цього треба знищити стільки ж банкнот, — то запас
банкового відділу спаде від 8 до 2 мільйонів. З одного боку, банк мав би дуже
підвищити свій рівень проценту; з другого боку, ті банки, що складали в нього
свої гроші, та інші вкладники побачили б, що запасний фонд банку, яким забезпечено
їхні вклади в нього, дуже зменшився. В 1857 році чотири найбільші
акційні банки Лондону загрожували, що, коли англійський банк не доб’ється «урядового листа», щоб
припинити чинність банкового акту 1844 року,5 то вони
заберуть свої вклади, від чого банковий відділ став би банкротом. Оттак банковий
відділ може, як от 1847 року, збанкрутувати, тимчасом коли в issue departement
(емісійному відділі) лежить багато мільйонів (напр., в 1847 р. 8 мільйонів),
як ґарантія розмінности банкнот, що перебувають в циркуляції. Але це
останнє є знову ілюзорне.

«Велика частина вкладів, що на них самі банкіри не мають безпосереднього
попиту, йде до рук billbrokers (буквально: векселевих маклерів, а справді —
напівбанкірів), що дають банкірові як забезпечення за свою в нього позику
торговельні векселі, вже дисконтовані ними для різних осіб у Лондоні та в
провінції. Billbroker відповідає перед банкіром за повернення цих money at
call (гроші, що їх мають на вимогу негайно повернути); і ці операції мають
такий величезний обсяг, що пан Neave, сучасний управитель банку [Англійського]
каже у своєму свідченні: «Нам відомо, що один broker мав 5 мільйонів,

  Назва банку  Пасив ф. ст.  Запаси готівкою ф. ст.  У відсотках

City        9317629    746551    8,01
Capital and Counties        11392744    1307483    11,47
Imperial         3987400    447157    11,21
Lloyds         23800937    2966806    12,46
Lon. and Westminster        24671559    3818885    15,50
London and S. Western        5570268    812353    13,58
London joint Stock        12127993    1288977    10,62
London and Midland         8814499    1127280    12,79
London and County         37111035    3600374    9,70
National        11163829    1426225    12,77
National Provincial        41907384    4614780    11,01
Parrs and the Alliance        12794489    1532707    11,93
Prescott and C°          4041058    538517    13,07
Union of Loudon        15502618    2300084    14,84
Williams, Deacon and Manchester and C°. . . .    10452381    1317628    12,60
Разом      232655823    27845807    11,97,

З цих майже 28 мільйонів запасів, принаймні 25 мільйонів покладено до Англійського банку, а
щонайбільше 3 мільйони готівкою є в касах самих 15 банків. Однак запас готівкою в банковому відділі
Англійського банку в тому самому листопаді 1892 р. ніколи не становив повних 16 мільйонів! — Ф. Е.].

5) Припинення чинности банкового акту 1844 року позволяє банкові видавати довільне число банкнот, не
зважаючи на покриття їх тим золотим скарбом, що є в його руках; отже, дозволяє утворювати довільну
кількість паперового фіктивного грошового капіталу й тим способом давати позики банкам та векселевим
маклерам, а через них і торговлі.
