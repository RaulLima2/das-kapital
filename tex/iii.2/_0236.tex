\parcont{}  %% абзац починається на попередній сторінці
\index{iii2}{0236}  %% посилання на сторінку оригінального видання
протягом певного часу, отже, і протягом певного додаткового робочого часу. Отже,
дійсне багатство суспільства і можливість постійно поширювати процес його
репродукції залежить не від протяжности додаткової праці, а від її продуктивности
і від більшого або меншого достатку тих умов продукції, за яких вона відбувається.
Царство волі починається в дійсності лише там, де припиняється праця,
викликана потребою і зовнішньою доцільністю, отже, з природи речей воно
лежить по той бік сфери власне матеріяльної продукції. Як дикун, щоб задовольняти
свої потреби, щоб зберегти і репродукувати своє життя, мусить боротися
з природою, так мусить боротися і цивілізований, і він мусить так боротися
в усіх суспільних формах і при всіх можливих способах продукції. З його
розвитком поширюється це царство природної доконечности, бо його потреби
поширюються; але одночасно поширюються і продуктивні сили, які служать
для їхнього задоволення. Воля в цій царині може бути лише в тому, що соціялізована
людина, асоційовані продуценти раціонально регулюють цей свій обмін
речовин з природою, становлять його під свій громадський контроль, замість того,
щоб він як сліпа сила панував над ними; — лише в тому, що вони провадять його
з найменшою витратою сили і в найгідніших і найадекватніших їхній людській
природі умовах. Але це все ж таки лишається царством доконечности. По той
бік його починається розвиток людської сили, що є самоціль, справжнє царство
волі, яке, проте, може розцвісти лише на цьому царстві доконечности як на
своїй базі. Скорочення робочого дня — основна умова.

Ця додаткова вартість або цей додатковий продукт розподіляється в капіталістичному
суспільстві між капіталістами, — коли ми лишимо осторонь випадкові
коливання розподілу і розглядатимемо його регуляційний закон, його нормівні
межі, — як дивіденд пропорційно тій частині, що належить кожному в суспільному
капіталі. В цьому вигляді додаткова вартість виступає як пересічний
зиск, що дістається капіталові, пересічний зиск, який в свою чергу розпадається
на підприємницький бариш і процент, і який в кожній з цих двох категорій
може дістатися різного роду капіталістам. Це присвоювання і розподіл додаткової
вартости, зглядно додаткового продукту, капіталом, має, проте, свою межу в земельній
власності. Як капіталіст, що функціонує, висмоктує з робітника додаткову
працю, а тим самим висмоктує він у формі зиску додаткову вартість
і додатковий продукт, так земельний власник і собі висмоктує з капіталіста
частину цієї додаткової вартости або додаткового продукту у формі ренти, згідно
з раніш викладеними законами.

Отже, коли ми говоримо тут про зиск як про частину додаткової вартости,
що припадає капіталові, то ми маємо на увазі пересічний зиск (дорівнює підприємницькому
баришеві плюс процент), який уже обмежений вирахуванням
ренти з усього зиску (зиску, що у своїй масі тотожний з усією додатковою
вартістю); тобто припускається вирахування ренти. Отже, зиск з капіталу (підприємницький
бариш плюс процент) і земельна рента є не що інше, як окремі
складові частини додаткової вартости, категорії, що в них остання стає різна,
залежно від того, чи дістається вона капіталові, чи земельній власності, рубрики,
які проте нічого не змінюють в її суті. Складені одна з однією, вони становлять
суму суспільної додаткової вартости. Капітал висмоктує додаткову працю, втілену
в додатковій вартості і додатковому продукті, безпосередньо з робітників. Отже,
в цьому розумінні його можна розглядати як продуцента додаткової вартости.
Земельна властність не має жодного чинення до дійсного процесу продукції.
Її роля обмежується тим, що вона переміщує частину випродукованої додаткової
вартости з кишені капіталіста у свою власну. А проте, земельний власник відіграє
певну ролю в капіталістичному процесі продукції не тільки в наслідок
тиснення, яке він справляє на капітал, і не просто в наслідок того, що велика
земельна власність є передумова і умова капіталістичного способу продукції,
\parbreak{}  %% абзац продовжується на наступній сторінці
