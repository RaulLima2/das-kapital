гляду. Трудність не в тому, щоб взагалі з’ясувати створений хліборобським
капіталом додатковий продукт і відповідну до нього додаткову вартість. Це
питання, радше, є вже розв’язане аналізою додаткової вартости, створюваної
всяким продуктивним капіталом, хоч би в яку сферу він був вкладений. Трудність
є в тому, що треба показати, звідки після того, як додаткова вартість
вирівнялась між різними капіталами на пересічний зиск, на відповідну до
їхніх відносних величин пропорційну частину всієї додаткової вартости, створеної
всім суспільним капіталом у всіх сферах продукції, — звідки після цього вирівняння,
після того як розподіл усієї додаткової вартости, яка взагалі може
бути розподілена, вже очевидно стався — звідки ж тут після цього береться
ще й та надмірна частина цієї додаткової вартости, яку капітал, вкладений
в землю, виплачує в формі земельної ренти земельному власникові.
Цілком лишаючи осторонь практичні мотиви, які спонукали сучасних економістів
як оборонців промислового капіталу проти земельної власности
досліджувати це питання, — мотиви, які ми накреслимо ближче в розділі
про історію земельної ренти, — це питання становило для них, як для теоретиків,
переважний інтерес. Визнати, що появлення ренти на капітал, вкладений
в хліборобство, завдячує особливій дії самої сфери приміщення, властивостям,
належним земній корі, як такій, це значило б відмовитись від самого
поняття вартости, отже, відмовитися від усякої можливости наукового
пізнання в цій галузі. Саме звичайне спостереження, що ренту виплачується
з ціни продукту землі, а це так і є навіть в тому випадку, коли її виплачується
в натуральній формі, скоро тільки орендар здобуває свою ціну продукції,
— показує, оскільки безглуздо надмір цієї ціни над звичайною ціною
продукції, отже, відносну дорожнечу хліборобського продукту, пояснювати надміром
природної продуктивности хліборобської промисловости над продуктивністю
інших галузей промисловости; бо, навпаки, що продуктивніша праця, то
дешевша кожна складова частина її продукту, тому що тим більша маса споживних
вартостей, в якій репрезентована та сама кількість праці, отже, та сама вартість.

Отже, при аналізі ренти вся трудність була в тому, що треба було
пояснити надмір хліборобського зиску над пересічним зиском, з’ясувати не додаткову
вартість, а властиву цій сфері продукції надмірну додаткову вартість,
отже, знов таки не «чистий продукт», а надмір цього чистого продукту над
чистим продуктом інших галузей промисловости. Сам пересічний зиск є продукт,
витвір процесу соціального життя, що відбувається в цілком певних історичних
продукційних відносинах, продукт, що має своєю передумовою, як ми
бачили, дуже широкосяжні посередницькі ланки. Для того, щоб взагалі можна було
говорити про надмір над пересічним зиском, сам цей пересічний зиск мусить
взагалі скластися як маштаб і — як це відбувається за капіталістичного способу
продукції, — як регулятор продукції. Отже, в таких суспільних формах, де ще
немає капіталу, який виконує ту функцію, що вимушує всю додаткову працю
і привласнює в першу чергу собі всю додаткову вартість, отже, де капітал ще
не упідлеглив собі суспільної праці, або упідлеглив її лише місцями, — взагалі
не може бути мови про ренту в сучасному значенні, про ренту як надмір
над пересічним зиском, тобто над пропорційною частиною всякого індивідуального
капіталу в додатковій вартості, спродукованій усім суспільним капіталом. Те
що, наприклад, пан Раssy (дивись далі) говорить вже про ренту в первісному стані
як про надмір над зиском, як про надмір над історично-певного суспільною
формою додаткової вартосте, так що за п. Раssу ця форма могла б, мабуть,
існувати і без суспільства, — свідчить лише про його наївність.

Для колишніх економістів, які взагалі лише починали аналізу капіталістичного
способу продукції, ще нерозвиненого за їхнього часу, аналіза ренти
або взагалі не становила жодних труднощів, або становила лише труднощі ціл-
