\index{iii2}{0063}  %% посилання на сторінку оригінального видання
II. 1844—1853

Час    Металева готівка банку ф. ст. Ринкова норма дисконту в\%    З 15    головних товарів
            Піднеслися  ціною    Впали    Без змін
1844, 1 березня   16.162.000    2 1/4   —    —     —
1845, 1 грудня    13.237.000    4 1/2    11    4 —
1846, 1 вересня  16.366 000    3    7    8 —
1847, 1 »   9.140.000    6    6    6    3
1850, 1 березня   17.126.000    2 1/2    5    9    1
1851, 1 червня   13.705.000    3    2    11    2
1852, 1 вересня  21.853.000    1 3/4    9    5    1
1853, 1 грудня   15.093.000    5        14        1

Hubbard додає до цього таке пояснення: «Як за 10 років, 1834—43, так
і за 1844—53 роки поряд коливання кількости золота в банку в кожному
випадку відбувалося збільшення або зменшення позичкової вартости грошей, що
визичалися під дисконт; а з другого боку, зміни товарових цін в країні виявляють
повну незалежність від розмірів циркуляції, оскільки ці останні виявляються
в коливаннях кількости золота в Англійському банку». (Bank Acts Report, 1857,
II., р. 290 та 291).

Що попит та подання товарів реґулює їхні ринкові ціни, то очевидно, яка
є помилкова Оверстонова ідентифікація попиту на позичковий грошовий капітал
(або, радше, відхилів подання від попиту), оскільки цей попит виявляється
в нормі дисконту, з попитом на дійсний «капітал». Твердження, що товарові
ціни реґулюються коливаннями в сумі currency, ховається тепер за фразою, що
коливання норми дисконту виявляють собою коливання попиту на дійсний речовий
капітал, у відміну від грошового капіталу. Ми бачили, що й Норман, і
Оверстон дійсно стверджували це перед тією самою комісією, та бачили, до яких
нікчемних викрутів при цьому мусіли вони вдаватися, особливо останній, поки
він, кінець-кінцем, таки добре впіймався. (Розділ XXVI). Де дійсно давня вигадка,
ніби зміни в масі наявного золота, збільшуючи або зменшуючи кількість
засобів циркуляції в країні, мусять піднести або знизити товарові ціни в межах
цієї країни. Якщо золото вивозять, то за цією currency-теорією ціни товарів
мусять підноситися в тій країні, куди надходить золото, і разом з цим мусить
підноситися вартість експорту країни, що вивозить золото, на ринку країни, що
довозить золото; навпаки, вартість експорту останньої країни на ринку першої
буде спадати, зростаючи в країні, звідки походить той експорт та куди прямує
золото. Однак, в дійсності зменшення кількости золота підносить лише рівень проценту,
а зменшення тієї кількости знижує його, і коли б цих коливань рівня проценту
не бралось на увагу при визначенні коштів продукції або при визначенні
попиту та подання, то товарових цін вони, ті коливання, цілком не зачіпали б.

У тому самому звіті N. Alexander, шеф однієї значної фірми, що провадить
справи в Індії, так висловлюється про великий відплив срібла до Індії та Китаю
коло середини 50-х років почасти в наслідок китайської громадянської війни, що
припинила збут англійських тканин в Китаю, почасти в наслідок тієї хороби
шовковика в Европі, що дуже обмежила італійське та французьке шовківництво.

«4337. Чи цей відплив іде до Китаю, чи до Індії? — Ви шлете срібло до
Індії та на добру частину його купуєте опіюм, що весь іде до Китаю на те, щоб
\parbreak{}  %% абзац продовжується на наступній сторінці
