є лише фіктивна, тобто лише титул на вартість, цілком так, як от знаки
вартости. Оскільки гроші функціонують в кругообороті капіталу, являють вони,
щоправда, в певний момент грошовий капітал; проте вони не перетворюються на
позичковий грошовий капітал, а їх або вимінюють на елементи продуктивного капіталу,
або ж виплачують як засіб циркуляції, реалізуючи дохід, отже й не можуть
вони перетворитися на позичковий капітал для свого державця. Але оскільки
вони перетворюються на позичковий капітал, і та ж сама сума грошей повторно
представляє позичковий капітал, то очевидно, що вони лише на одному пункті
існують як металеві гроші; на всіх інших пунктах вони існують лише в формі
вимоги на капітал. Нагромадження цих вимог, згідно з припущенням, виникає
з дійсного нагромадження, тобто з перетворення вартости товарового капіталу
і т. ін. на гроші; проте нагромадження цих вимог або титулів, як таке, є відмінне
так від дійсного нагромадження, що з нього воно постав, як і від майбутнього
нагромадження (нового процесу продукції), що обслуговується за посередництвом
визичання грошей.

Prima facie позичковий капітал існує завжди у формі грошей\footnote{
В. А. 1857. Свідчення банкіра Twells’a: «4516. Як банкір, чи ви робите операції із капіталом
чи з грішми? — Ми провадимо операції з грішми. — 4517. В якій формі платять вклади в ваш
банк? — Грішми. — 4518. Як виплачуєте ви її? — Грішми. — Чи можна отже сказати, що вони є дещо
інше, ніж гроші? — Ні».

Оверстон (див. розд. XXVI) раз-у-раз плутається між «capital» та «money». Value of money
означає в нього також і процент, але остільки, оскільки він визначається масою грошей; value of
capital означає процент, оскільки він визначається попитом на продуктивний капітал та зиском, що
його він
дає. Він каже: «4140. Уживати слово капітал дуже небезпечно. — 4148. Вивіз золота з Англії становить
зменшення кількости грошей в країні й він мусить, природна річ, взагалі викликати збільшений
попит на грошовому ринку» [отже, за Оверстоном, не на ринку капіталу]. — «4112. В міру того, як
гроші
відходять з країни, меншає їхня кількість в країні. Це зменшення кількости грошей, що лишаються в
країні, породжує зріст вартости грошей» [первісно за його теорією це означало піднесення вартости
грошей як грошей проти вартостей товарів, піднесення, викликане скороченням циркуляції; при чому,
отже, це піднесення вартости грошей = спадові вартости товарів. А що в проміжний час навіть для
нього безперечно
доведено, що маса грошей в циркуляції не визначає цін, то й має тепер зменшення грошей як
засобів циркуляції підвищувати їхню вартість як капіталу, що дає процент, а разом з тим підвищувати
й рівень проценту]. «І це піднесення вартости решти грошей затримує їхній відплив та триває далі
доти, доки поверне воно назад стільки грошей, скільки треба, щоб знову відновити рівновагу». —
Продовження
про суперечності Оверстона буде далі.
}, пізніше
у формі вимоги на гроші, бо гроші, що в них він спочатку існує, існують тепер
тільки в руках позикоємця у дійсній грошовій формі. Для позикодавця позичковий
капітал перетворився у вимогу на гроші, у титул власности. Тому та сама маса
дійсних грошей може становити дуже різні маси грошового капіталу. Просто
гроші — чи становлять вони реалізований капітал, чи реалізований дохід, —
стають позичковим капіталом за допомогою простого акту визичання, за допомогою
перетворення їх у вклад, якщо розглядати загальну форму за розвинутої
кредитової системи. Вклад є грошовий капітал для вкладника. Однак в руках
банкіра він може бути тільки потенціяльним грошовим капіталом, що лежить
без діла в його касі, замість лежати в касі його власника\footnote{
Тут постає плутанина, бо і те й це є «гроші», і вклад як вимога до банкіра на платіж,
і складені гроші в руках банкіра. Банкір Twells наводить такий приклад перед банковою комісією
1857 року: «Я починаю своє підприємство з 10.000 ф. ст. На 5000 ф. ст. я купую товарів та
беру їх до себе на склад. Другі 5000 ф. ст. я складаю в банкіра, щоб брати по потребі. Однак я
розглядаю цілу суму все ще, як свій капітал, дарма що 5000 ф. ст. з неї перебувають у формі вкладу
або грошей. (4528)». З цього розгортаються тепер такі цікаві дебати: «4531. Отже, ви дали комусь
іншому свої 5000 ф. ст. в банкнотах? — Так. — 4532. Отож ця особа має тепер вклад в 5000 ф. ст. —
Так. — 4533. І ви маєте вклад в 5000 ф. ст.? — Цілком слушно. — 4534. Вона має 5000 ф. ст; грішми,
і ви маєте 5000 ф. ст. грішми? — Так. — 4535. Але ж кінець-кінцем це — не що інше як гроші? —
Ні». Плутанина виникає почасти з такої причини: А, що склав вклад в 5000 ф. ст., може брати
з неї собі частину, порядкує ними так само, як коли б він ще їх мав при собі. Остільки вони
функціонують для нього як потенціяльні гроші. Але в усіх випадках, коли він бере собі якусь частину,
він нищить свій вклад pro tanto. Якщо він бере з банку дійсні гроші, — а його гроші вже визи-
}.