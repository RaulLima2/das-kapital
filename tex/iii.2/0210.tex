вищення продуктивности праці виражається у збільшеній масі продукту. Продукт
становить тепер ту саму вартість, що й давніш, але ціна його складових
частин понизилася, тимчасом, як число цих частин збільшилося. Коли вживається
той самий капітал, це неможливе, бо в цьому випадку та сама вартість
виражається в якій завгодно масі продукту. Але це можливе, коли витрачено
додатковий капітал на гіпс, гуано тощо, коротко кажучи, на такі поліпшення,
що вплив їхній триває багато років. Умова цього є в тому, щоб ціна одного квартера
хоч і знизилась, але не в такому самому відношенні, як зростає число квартерів.

III. Ці різні умови підвищення ренти, а тому і ціни землі взагалі або окремих
родів землі, можуть почасти конкурувати між собою, почасти вони виключають
одна одну і можуть діяти лише навперемінки. Але з вище розвинутого, випливає,
що з підвищення ціни землі не можна без дальших околичностей робити
висновку, що рента підвищилась, і з підвищення ренти, яке завжди спричинює
підвищення ціни землі, не можна без дальших околичностей робити висновку,
що продукт землі збільшився42).

Замість звернутися до дійсних природних причин виснаження ґрунту, які,
проте, в наслідок стану хліборобської хемії в той час були невідомі усім економістам,
що писали про диферецційну ренту, — по допомогу звернулися до того
поверхового погляду, що в просторово обмежений лан не можна вкласти необмежену
масу капіталу; паприклад, Westminster Rewiew заперечує Річардові
Джонсові, що не можливо було б прогодувати цілу Англію обробітком Soho Square.
Хоч це вважається за особливу невигоду хліборобства, але справедливе як раз
зворотне. У хліборобстві можна продуктивно провадити послідовні приміщення
капіталу тому, що сама земля діє як знаряддя продукції, тимчасом як цього зовсім
немає, або є лише в дуже вузьких межах у випадку з фабрикою, де земля
функціонує лише як фундамент, як місце, як просторова операційна база. Правда,
можна — так і робить велика промисловість — саме на відносно невеликім, проти
парцельованого ремесла, просторі концентрувати велику продукційну споруду.
Але за даного ступеня розвитку продуктивної сили завжди потрібен певний простір,
і будування в висоту теж має свої певні практичні межі. Поширення продукції
за ці межі потребує і поширення простору землі. Основний капітал, вкладений
у машини тощо, не поліпшується споживанням, а навпаки, зношується. Внаслідок
нових винаходів і тут можуть статися окремі поліпшення, але, припускаючи
даний ступінь розвитку продуктивної сили, машина при споживанні може
лише погіршуватись. При швидкому розвитку продуктивної сили всю сукупність
старих машин доводиться заміняти вигіднішими, отже, вони гинуть. Навпаки,
земля, коли вона правильно обробляється, дедалі поліпшується. Та перевага
землі, що послідовні приміщення капіталу можуть дати вигоду без втрати колишніх,
одночасно має в собі можливість різної продуктивности цих послідовних
приміщень капіталу.

Розділ сорок сьомий.

Генеза капіталістичної земельної ренти

1. Вступ.

Треба з’ясувати собі, в чому власне є труднощі трактування земельної
ренти з погляду сучасної економії, як теоретичного виразу капіталістичного
способу продукції. Цього ще не розуміє навіть величезне число новітніх письменників,
про що свідчить всяка нова спроба з’ясувати земельну ренту «по
новому». Новіша тут майже завжди є в повороті до давно вже побореного по-

42) Про падіння земельних цін при підвищенні ренти як про факт дивись Passy.
