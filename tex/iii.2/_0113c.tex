\parcont{}  %% абзац починається на попередній сторінці
\index{iii2}{0113}  %% посилання на сторінку оригінального видання
скаржитись, що вони не зможуть платити таких високих рент, як звичайно
платили, бо в наслідок еміграції праця дорожчає». Отже, тут висока земельна
рента прямо ототожнюється з низькою заробітною платою. І оскільки висота
земельної ціни зумовлюється цією обставиною, яка підвищує ренту, остільки підвищення
вартости землі тотожне із знеціненням праці, високий рівень земельної
ціни — з низьким рівнем ціни праці.

Те саме і у Франції. «Орендна плата підвищується, бо на однім боці підвищується
ціна хліба, вина, м’яса, городини і овочів, а на другім боці ціна
праці лишається незмінна. Коли б старі люди порівняли рахунки їхніх батьків, —
що відсунуло б нас назад майже на 100 років, — вони побачили б, що тоді ціна
робочого дня у сільській Франції була достоту така, як і тепер. Ціна м’яса від того
часу збільшилась утроє.. . Хто жертва цього перевороту? Чи багатий, що є власник
здаваної в оренду землі, чи бідняк, що її обробляє?... Зріст орендних
цін є доказ суспільного лиха». (Du Mécanisme de la Société en France et en
Angleterre. Par 1. Rubichon, 2-me édit. Paris 1837, p. 101).

Приклади ренти, як наслідку вирахування, з одного боку, з пересічного
зиску, з другого — з пересічної заробітної плати:

Цитований вище Мортон, сільський аґент і сільсько-господарський інженер,
каже, що в багатьох місцевостях зроблено спостереження, що рента
за великі оренди нижча, ніж за дрібні, бо «конкуренція за останні, звичайно,
більша, ніж за перші, і тому що дрібні орендарі, які рідко мають можливість
узятись до якогось іншого діла, крім хліборобства, вимушені пекучою потребою
знайти підхоже діло, часто погоджуються платити таку ренту, про яку вони
сами знають, що вона надто висока». (John С. Morton, The Resources of Estates.
London 1885. p. 116).

Проте, на його думку, в Англії ця ріжниця поступово згладжується, чому,
як він вважає, дуже сприяє еміґрація кляси саме дрібних орендарів Той самий
Мортон наводить приклад, коли в земельну ренту безперечно входить вирахування
з заробітної плати самого орендаря, а тому ще безперечніше і з заробітної
плати робітників, які в нього працюють. А саме, коли орендні дільниці
менші, ніж 70—80 акрів (30—34 гектари), при яких неможливо держати парокінний
плуг. «Коли орендар не працює своїми власними руками так само
дбало, як будь-який робітник, він не може існувати від своєї оренди. Коли він
виконання роботи покладе на своїх людей, а сам обмежиться виключно наглядом
за ними, то він, найімовірніше, дуже скоро виявить, що не зможе виплачувати
орендної плати» (1. с., р 118). З цього Мортон висновує, що коли орендарі
в краю не дуже бідні, то розміри віддаваних на оренду дільниць не повинні
бути менші за 70 акрів, щоб орендар міг держати двох або трьох коней.

Надзвичайна мудрість пана Léonce de Lavergne, Membre de l’Institut et de
la Société Centrale d’Agriculture. У своїй Economie Rurale de l’Angletterre (цитовано
з англійського перекладу, London 1855), він робить таке порівняння
річних вигід від рогатої худоби, яку у Франції вживається для роботи, а в Англії
не вживається, бо її заміняють коні (р. 42):

Франція: Англія:

молоко...........4 міл. ф. ст. молоко..............16 міл. ф. ст.

м'ясо............16 >» ->    м’ясо............... '20» »»

робота...... 8» »    >    робота............... —» »»

28 міл. ф. ст.    36 міл. ф. ст.

Але вищий продукт для Англії тут виходить лише тому, що згідно з його
власними даними молоко в Англії коштує удвоє дорожче, ніж у Франції, тимчасом
як для м’яса він припускає однакові ціни в обох країнах (р. 35); отже, молочний
\parbreak{}  %% абзац продовжується на наступній сторінці
