\parcont{}  %% абзац починається на попередній сторінці
\index{iii2}{0136}  %% посилання на сторінку оригінального видання
стерл.) становить 240\% замість 180\%. Увесь продукт збільшився з 10 до
36 квартерів.

Порівняно з Іb, де загальне число оброблених акрів, застосований капітал
і ріжниці між обробленими родами землі лишились ті самі, але розподіл
їх інший, продукт становить 36 квартерів замість 26 кварт., пересічна рента
з акра становить 6 ф. стерл. замість 3\sfrac{1}{2} і норма ренти щодо всього авансового
капіталу тієї самої величини — 240\% замість 140\%.

Хоч як би ми стали розглядати різні становища, подані в таблицях Іа,
Іb, Іс, чи як становища, що одночасно існують одно біля одного в різних країнах,
чи як послідовні становища в тій самій країні, — однаково, виявиться таке:
за сталої ціни збіжжя, сталої тому, що продукт з найгіршої землі, яка не дає
ренти, лишається той самий; за незмінної різниці у родючости різних розрядів
оброблюваної землі; при відносно однаковій кількості продукту, а, отже,
при однаковій витраті капіталу на відповідно однакові частини (акри) земельної
площі, оброблюваної в кожному розряді; при сталому в наслідок цього
відношенні між рентами з акра кожного роду землі і при однаковій нормі ренти
на капітал, вкладений у кожну дільницю землі того самого роду — виявиться,
\emph{поперше}, що сума ренти завжди зростає, разом з поширенням оброблюваної
площі, а тому із збільшенням витрати капіталу, за винятком того випадку, коли
весь приріст припав би на землю, що не дає ренти. \emph{Подруге}, так пересічна
рента на акр (загальна сума ренти, поділена на все число оброблюваних
акрів), як і пересічна норма ренти (загальна сума ренти, поділена на ввесь
витрачений капітал) можуть значно варіювати, і хоч обидві в одному напрямку,
але в різних пропорціях у відношенні одна до однієї. Якщо не брати
на увагу того випадку, коли приріст відбувається лише на землі кг
яка не дає ренти, то виявляється, що пересічна рента на акр і пересічна,
норма ренти на капітал, вкладений у хліборобство, залежить від того, які пропорційні
частини всієї оброблюваної землі становлять землі різних розрядів;
або, що схопіть на те саме, від розподілу всього застосованого капіталу між
землями різної родючости. Чи багато, чи мало землі обробляється і тому (за
винятком того випадку, коли приріст припадає лише на А) чи більша, чи
менша є загальна сума ренти, пересічна рента на акр або пересічна рента на
застосований капітал лишається та сама, доки відношення різних родів оброблюваної
землі до всієї її площі лишається те саме. Дарма, що з поширенням
культури і збільшенням застосованого капіталу відбувається підвищення і
навіть значне підвищення загальної суми ренти, пересічна рента на акр і
пересічна норма ренти на капітал понижується, коли поширення земельних
дільниць, що не дають ренти, або дають лише незначну диференційну ренту,
зростає швидше, ніж поширення кращих земельних дільниць, що дають більшу
ренту. Навпаки, пересічна рента на акр і пересічна норма ренти на капітал підвищується в міру того,
як кращі землі починають становити відносно більшу
частину всієї площі, і тому на них припадає відносно більше застосованого
капіталу.

Таким чином, коли розглядати пересічну ренту на акр або гектар усієї
оброблюваної землі, як це звичайно робиться в статистичних працях при порівнянні
різних країн за тієї самої доби, або різних діб у тій самій країні, та
виявляється, що пересічна висота ренти на акр, а тому і загальна сума ренти
в певних пропорціях (хоч зовсім не в тих самих, а в швидше ростучих) відповідає
не відносній, а абсолютній родючості хліборобства в країні, тобто відповідає,
масі продуктів, одержуваній пересічно з однакової земельної площі. Бо що
більшу частину із загальної площі становлять кращі землі, то більша маса
продуктів, одержувана з земельної площі однакової величини за однакової величини
застосованого капіталу, і то більша пересічна рента на акр. Зворотне в зворотному
\index{iii2}{0137}  %% посилання на сторінку оригінального видання
випадку. Тому здається, що рента визначається не відношенням диференційної
родючости, а абсолютною родючістю, і що таким чином закон диференційної
ренти знищується. Тому деякі явища заперечуються, або їх намагаються
пояснити несущими різницями пересічних цін хліба і ріжницями диференційної
родючости оброблюваних дільниць землі, тимчасом, як ці явища ґрунтуються
просто на тому, що відношення загальної суми ренти так до всієї площі оброблюваної
землі, як і до всього капіталу, вкладеного в землю, за однакової родючости
землі, що не дає ренти, а тому і за однакових цін продукції і за
однакової ріжниці між землями різних родів визначаються не тільки рентою
на акр, або нормою ренти на капітал, але в такій же мірі відношенням числа
акрів кожного роду до загального числа оброблюваних акрів; або, що сходить
на те саме, розподілом усього застосованого капіталу між різними родами землі.
До цього часу на цю обставину, дивовижно, зовсім не звертали уваги. В усякому
разі виявляється, і це є важливе для нашого дальшого досліду, що відносна
висота пересічної ренти на акр і пересічна норма ренти, або відношення
загальної суми ренти до всього вкладеного в землю капіталу, може збільшуватися
або зменшуватися просто в наслідок екстенсивного поширення культури,
за незмінних цін, незмінної ріжниці в родючості оброблюваних дільниць землі
і незмінної ренти з акра, або норми ренти на капітал, вкладений в акр
кожного розряду землі, що дійсно дає ренту, або на ввесь капітал, що дійсно
дає ренту.

\pfbreak

Треба зробити ще такі доповнення щодо тієї форми диференційної ренти,
яка досліджена в нас під рубрикою І, і що почасти мають також значіння і
для диференційної ренти II.

\emph{Перше:} Ми бачили, як пересічна рента з акра або пересічна норма
ренти на капітал може підвищитись з поширенням культури, за сталих цін і
незмінної ріжниці в родючості оброблюваних земельних дільниць. Скоро вся
земля в будь-якій країні буде привласнена, вкладення капіталу в землю, культура
і людність досягнуть певної висоти — обставини, наявність яких доводиться
припускати, скоро капіталістичний спосіб продукції став панівним і упідлеглив
собі і хліборобство, — ціна необроблюваної землі різної якости (просто припускаючи
існування диференційної ренти) визначається ціною оброблюваних дільниць
землі однакової якости і рівноцінного положення. Ціна є така сама — за вирахуванням
витрат на обробіток, що приєднується до неї — хоч ця земля і не
дає ренти. Ціна землі, звичайно, є не що інше, як капіталізована рента. Але
і в ціні оброблених земельних дільниць оплачуються лише майбутні ренти, наприклад,
одним заходом виплачується наперед ренти за 20 років, коли міродайний
розмір проценту є 5\%. Коли продається земля, вона продається як така, що дає
ренту, і перспективний характер ренти (яку розглядається тут як витвір землі, чим
вона є тільки з видимости) призводить до того, що необроблювана земля не
відрізняється від оброблюваної. Ціна необроблюваних дільниць землі, як і рента
з них, — а ціна становить лише зосереджену формулу ренти — має суто ілюзорний
характер, поки ці дільниці не будуть дійсно використані. Але вона, таким
чином, визначається a priori і реалізується, скоро знаходяться покупці. Тому,
коли дійсна пересічна рента в певній країні визначається дійсною пересічною
річною сумою ренти і відношенням цієї останньої до всієї оброблюваної площі,
то ціна необроблюваної частини земельної площі визначається ціною оброблюваної
і є тому лише відбиток вкладення капіталу та його наслідків на оброблюваних
земельних дільницях. А що за винятком найгіршої землі, землі усіх родів
дають ренту (а ця рента, як ми побачимо в рубриці II, зростає з масою капіталу
а з відповідною до цієї маси інтенсивністю культури), то і створюється таким чином
\parbreak{}  %% абзац продовжується на наступній сторінці
