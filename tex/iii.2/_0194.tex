\parcont{}  %% абзац починається на попередній сторінці
\index{iii2}{0194}  %% посилання на сторінку оригінального видання
(радше, внаслідок визначення конкуренцією ціни продукції, яка регулює ринкову
ціну) частини ціни товару, яка зводиться до надзиску, — за при чину перенесення
цієї частини ціни від однієї особи до іншої, від капіталіста до
земельного власника. Але земельна власність тут не є причина, яка \emph{створює}
цю складову частину ціни, або те підвищення ціни, яке є передумовою цієї
частини ціни. Навпаки, коли найгірша земля кляси А не може оброблятись, —
хоч оброблення її дало б ціну продукції, — поки вона не дає надміру над цією
ціною продукції, ренти, то земельна власність є творчою основою \emph{цього} підвищення
ціни. \emph{Сама земельна власність створила ренту}. Це анітрохи не
зміниться від того, що, як у другому розгляненому випадку, рента, виплачувана
тепер з землі А, становить диференційну ренту порівняно з тими останніми
додатковими приміщеннями капіталу на старих заорендованих дільницях, які
виплачують лише ціну продукції. Бо та обставина, що оброблення землі А
не може початися, поки регуляційна ринкова ціна не підійметься остільки високо,
що земля А зможе давати ренту, — тільки ця обставина є тут причиною
того, що ринкова ціна підвищується до такого пункту, на якому вона для
останніх приміщень капіталу на старих заорендованих дільницях виплачує,
правда, лише їхню ціну продукції, але таку ціну продукції, яка одночасно
дає ренту для землі А. Та обставина, що остання взагалі мусить виплачувати
ренту, є тут причиною створення диференційної ренти між землею А і останніми
приміщеннями капіталу на старих заорендованих дільницях.

Коли ми взагалі кажемо, що — припускаючи реґулювання збіжжевої ціни
ціною продукції — земля кляси А не виплачує ренти, то ми маємо на увазі
ренту в категоричному значінні слова. Коли орендар виплачує орендну плату,
яка становить вирахування або з нормальної заробітної плати його робітників,
або з його власного нормального пересічного зиску, то він не виплачує жодної
ренти, жодної самостійної складової частини ціни його товару, яка відрізнялася б
від заробітної плати і зиску. Вже давніш ми відзначали, що на практиці це
завжди трапляється. Коли заробітну плату хліборобських робітників у певній
країні взагалі знижують поза нормальний пересічний рівень заробітної плати,
і тому вирахування з заробітної плати, частина заробітної плати, входить, як
загальне правило, до складу ренти, то це не становить жодного винятку для
орендаря найгіршої землі. В тій самій ціні продукції, яка уможливлює оброблення
найгіршої землі, вже ураховується, як складова стаття, ця низька заробітна
плата, і тому продаж продукту по ціні продукції не дає змоги орендареві
цієї землі виплачувати ренту. Земельний власник може також здати свою землю
в оренду робітникові, який буде готовий усе те, або більшу частину того, що
продажна ціна залишає йому поверх заробітної плати, виплатити у формі
ренти другій особі. Проте, в усіх цих випадках зовсім не виплачується дійсної
ренти, хоч виплачується орендна плата. Але там, де існують відносини, відповідні
капіталістичному способові продукції, рента і орендна плата мусили б
збігатися. Отут ми й повинні дослідити якраз це нормальне відношення.

Якщо навіть розглянуті вище випадки, коли за капіталістичного способу
продукції дійсно можуть вкладатися у землю капітали, не даючи при цьому
ренти, — якщо навіть ці випадки нічого не дають для розв’язання нашої проблеми,
то ще значно менше дасть посилання на колоніяльні відносини. Що робить
колонію колонією, — ми говоримо тут лише про власне хліборобські колонії,
— так це не тільки маса родючих земель, що перебувають у природному
стані. Ні, колоніями робить їх радше та обставина, що ці землі не привласнені,
не підлеглі земельній власності. Саме це і зумовлює таку колосальну ріжницю
між старими землями і колоніями, оскільки справа йде про землю:
юридична або фактична відсутність земельної власности, як слушно відзначив
\parbreak{}  %% абзац продовжується на наступній сторінці
