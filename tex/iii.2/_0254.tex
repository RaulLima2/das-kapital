\parcont{}  %% абзац починається на попередній сторінці
\index{iii2}{0254}  %% посилання на сторінку оригінального видання
коли розглядати його тільки з речового боку, цілком відповідає у дикуна зворотному
перетворенню додаткової праці на новий капітал. У процесі акумуляції
безупинно відбувається перетворення такого продукту надмірної праці на капітал;
і та обставина, що всякий новий капітал походить з зиску, ренти або інших
форм доходу, тобто з додаткової праці, призводить до помилкового уявлення,
ніби вся вартість товарів виникає з доходу. Це зворотне перетворення зиску
на капітал при ближчій аналізі показує, навпаки, що додаткова праця, яка завжди
виступає в формі доходу, служить не для збереження, або репродукції старої
капітальної вартости, а, — оскільки її не споживається як дохід — для створення
нового додаткового капіталу.

Вся трудність випливає з того, що вся новодолучена праця, оскільки
створена нею вартість не зводиться до заробітної плати, з’являється в формі
зиску — тут зрозумілого як форма додаткової вартости взагалі, — тобто як
вартість, яка нічого не коштувала капіталістові, що нею, отже, йому не доводиться
покривати нічого авансованого, жодного капіталу. Тому вартість ця існує
в формі вільного, додаткового багатства, або, кажучи коротко, з погляду окремого
капіталіста, у формі його зиску. Але цю новостворену вартість однаково
можна спожити    як продуктивно, так і особисто, однаково і як капітал, і як
дохід. Частина її вже за своєю натуральною формою мусить бути спожита    продуктивно. Отже, ясно, що
щорічно долучувана праця створює так капітал, як і
дохід; як це й виявляється в процесі акумуляції. Але ту частину робочої сили,
яку застосовується на те, щоб наново створити капітал (отже, за аналогією
ту частину робочого дня, яку дикун застосовує не на те, щоб привласнювати
їжу, а на те, щоб виготовляти знаряддя, що ним він привласнює їжу), вже тому
не можна розпізнати, що ввесь продукт додаткової праці, насамперед, виступає
в формі зиску; визначення, яке в дійсності не має нічого до діла з самим
цим додатковим продуктом, а стосується тільки приватного відношення капіталіста
до одержаної ним додаткової вартости. В дійсності додаткова вартість,
створювана робітником, розпадається на дохід і капітал, тобто на засоби споживання
і додаткові засоби продукції. Але старий, залишений від минулого
року, сталий капітал, (залишаючи осторонь ту частину, яка ушкоджена і, отже,
pro tanto, знищена, отже, оскільки його не доводиться репродукувати, — а такі
порушення процесу репродукції стосуються до сфери страхування), розглядуваний
з боку його вартости, не репродукується новодолученою працею.

Ми бачимо далі, що частину новодолученої праці постійно поглинається
в процесі репродукції і покриття зужиткованого сталого капіталу, хоч ця новодолучена
праця розпадається тільки на доходи: заробітну плату, зиск і ренту.
Але при цьому спускається з уваги, 1) що частина вартости продукту цієї
праці являє собою не продукт цієї новодолученої праці, а вже сущий і спожитий
сталий капітал; що тому та частина продуктів, що в ній виявляється
ця частина вартости, теж не перетворюєтьея на дохід, а покриває in natura
засоби продукції цього сталого капіталу; 2) що та частина вартости, що в ній
дійсно втілюється ця новодолучена праця, не споживається in natura як дохід,
а покриває сталий капітал в іншій сфері, де він перетворюється на таку натуральну
форму, в якій він може бути спожитий як дохід, але останній в свою
чергу знов таки не є виключно продукт новодолученої праці.

Оскільки репродукція відбувається в незмінному маштабі, кожен спожитий
елемент сталого капіталу мусить покриватись in natura новим екземпляром відповідного роду, коли не
такої самої кількости й форми, то такої самої дієздатности.
Якщо продуктивна сила праці залишається та сама, то це покриття in
natura включає покриття тієї самої вартости, яку сталий капітал мав у своїй
старій формі. Якщо ж продуктивна сила праці збільшується так, що ті самі
речові елементи можна репродукувати меншою працею, то менша частина вартости
\parbreak{}  %% абзац продовжується на наступній сторінці
