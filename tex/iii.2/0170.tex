Рід землі
Ціна продукції. Шил.
Продукт. Бушелі
Продажна ціна. Шил.
Здобуток. Шил.
Рента. Шил.
Підвищення ренти
А 60 10 6 60 0 0
В 60 + 60 = 120 12 + 8 = 20 6 120 0 0
С    60 + 60 = 120    14 + 9 1/3 = 23 1/3         6    140    20    20
D    60 + 60 = 120    16 + 10 2/3 = 26 2/3    6    160    40    2 × 20
Е 60 + 60 = 120 18 + 12*) = 30 6 180 60 3 × 20
120 6 × 20

2) Кола земля В не стає землею, що зовсім не дає ренти.

Таблиця XIV.

Рід землі
Ціна продукції. Шил.
Продукт. Бушелі
Продажна ціна. Шил.
Здобуток. Шил.
Рента. Шил.
Підвищення ренти
А 60 10 6 60 0 0
В 60 + 60 = 120 12 + 9 = 21 6 126 6 6
С    60 + 60 = 120    14 + 10 1/2 = 24 1/2    6    147     27    6 + 21
D 60 + 60 = 120 16 + 12 = 28 6 168 48 6 + 2 × 21
Е    60 + 60 = 120    18 + 13 1/2 = 31 1/2    6    189     69    6 + 3 × 21
 150  4 × 6 + 6 × 21

Варіянт 3: за висхідної продуктивности другої витрати капіталу; на землі
А тут теж не робиться другої витрати.

Таблиця XV.

Рід землі
Ціна продукції. Шил.
Продукт. Бушелі
Продажна ціна. Шил.
Здобуток. Шил.
Рента. Шил.
Підвищення ренти
А 60 10 6 60 0 0
В 60 + 60 = 120 12 + 15 = 27 6 162 42 42
С    60 + 60 = 120     14 + 17 1/2 = 31 1/2    6    189     69     42 + 27
D 60 + 60 = 120 16 + 20 = 36 6 216 96 42 + 2 × 27
Е    60 + 60 = 120     18 + 22 1/2 = 40 1/2    6    243     123     42 + 3 × 27
330 42 × 4 + 6 × 27

*) В німецькому тексті тут стоїть «20». Очевидна помилка. Прим. Ред.
