становити тільки виняток та раз-у-раз мусить перериватися в багатьох
місцях. Розглядаючи процес репродукції (Книга II, відділ ІІІ), ми бачили,
що продуценти сталого капіталу почасти обмінюють сталий капітал поміж
себе. Тому мають вони змогу більш або менш вирівнювати свої векселі. Те саме
буває й на тій лінії продукції, що являє собою по висхідній лінії послідовні ланки
продукції, а саме коли бавовняний маклер трасує вексель на прядуна, прядун —
на фабриканта ситцю, фабрикант ситцю на експортера, а цей останній на імпортера
(можливо, знову на імпортера бавовни) Але в той же час ці операції не складаються
в коло, а тому й низка вимог не вирівнюється. Вимога, напр., прядуна
до ткача не вирівнюється вимогою постачальника вугілля до фабриканта машин;
прядунові у своєму підприємстві ніколи не створити контр-вимоги до фабриканта
машин, бо його продукт, пряжа, ніколи не ввіходить як елемент у процес репродукції
того фабриканта. Тому такі вимоги мусять вирівнюватися грішми.

Межі цього комерційного кредиту, коли їх розглядати самі про себе,
такі: 1) багатство промисловців та купців, тобто запасний капітал, що вони ним
можуть орудувати, якщо зворотний приплив капіталу забариться, 2) самі зворотні
припливи капіталу. Ці останні можуть щодо часу забаритися, або товарові ціни
можуть протягом того часу впасти, або товарів раптом не можна буде продати
через застій на ринках. Що довший реченець має вексель, то більшим мусить
бути насамперед запасний капітал і то більша є можливість зменшення або
запізнення зворотного припливу капіталу через спад цін або переповнення ринків.
І далі, ці зворотні припливи є то непевніші, що дужче первісна операція визначалася
спекуляцією на піднесення чи спад товарових цін. Однак, очевидно, що
з розвитком продуктивної сили праці, а тому і з поширенням маштабу продукції
1) ринки ширшають та віддаляються від місця продукції, 2) тому кредити мусять
ставати більш довготерміновими, і отже 3) спекулятивний елемент мусить чимраз
дужче опановувати операції. Продукція у великому масштабі та для віддалених ринків
кидає ввесь продукт до рук торговлі; але неможливо, щоб капітал нації збільшився
удвоє, так що торговля сама про себе була б в стані закупити на власний
капітал увесь національний продукт та знову його продати. Отже, кредит
тут неминучий; кредит, що обсягом своїм зростає зі зростом обсягу вартости
продукції і щодо часу стає більш довготерміновим, чим дужче зростає віддаленість
ринків. Тут відбувається взаємний вплив. Розвиток процесу продукції
поширює кредит, а кредит призводить до поширу промислових і торговельних
операцій.

Якщо розглядати цей кредит окремо від банкірського кредиту, то очевидно,
що він зростає разом з обсягом самого промислового капіталу. Позичковий капітал
і капітал промисловий є тут тотожні; позичені капітали є товарові капітали,
призначені або для остаточного індивідуального спожитку, або для заміщення
сталих елементів продуктивного капіталу. Отже, те, що тут здається
позиченим капіталом, є завжди капітал, що, перебуває в певній фазі процесу репродукції,
але переходить однак через купівлю й продаж з рук до рук, тимчасом
коли еквівалент за нього покупець\footnote*{
В тексті стоїть «dem Käufen», тобто «покупцеві». Очевидна помилка. Пр. Ред.
} платить тільки пізніше, в умовний реченець.
Напр., бавовна переходить за вексель до рук прядуна, пряжа переходить за
вексель до рук фабриканта ситцю, ситець переходить за вексель до рук купця,
а з його рук за вексель до рук експортера, з рук експортера за вексель до рук
купця в Індії, що продає його та купує за одержані гроші індиґо і т. ін. Протягом
цього переходу з рук до рук бавовна перетворюється на ситець, а ситець,
кінець-кінцем, транспортують до Індії й вимінюють на індиґо, що на кораблях
відправляється до Европи й там знову ввіходить у репродукцію. Різні фази процесу
репродукції відбуваються тут за посередництвом кредиту, так що прядун