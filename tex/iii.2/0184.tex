але ціну продукції в 3 1/2 ф. стерл. реґулювала б не найгірша земля А, а краща
земля В. Звичайно, при цьому припускається, що нова земля якости А такого
ж зручного положення, як оброблювана до цього часу, є неприступна, і що довелося
би зробити другу витрату капіталу на вже оброблюваній дільниці А, але
з більшою ціною продукції, або довелось би притягнути до обробітку ще гіршу
землю А1. Коли в наслідок послідовних витрат капіталу починає діяти днференційна
рента II, то може статися, що межі підвищуваної ціни продукції реґулюватимуться
кращою землею і гірша земля, база диференційної ренти І, тоді
теж може давати ренту. Таким чином, при наявності самої лише диференційної
ренти всі оброблювані землі почали б тоді давати ренту. Ми мали б тоді такі
дві таблиці, в яких під ціною продукції розуміється суму авансованого капіталу
плюс 20%  зиску, отже на кожні 2 1/2 ф. стерл. капіталу по 1/2 ф. стерл.
зиску, разом 3 ф. стерл. (див. табл. І).

Таке становище
речей перед новою
витратою капіталу в
3 1/2 ф стерл. на В,
що дає тільки 1 квартер.
Після цієї витрати
капіталу справа
стоїть так: (див.
табл. II).

[Це знов не зовсім
вірно обчислено.
Орендареві В продукція
цих 4 1/2 квартерпів
коштує, по-перше,
9 1/2 ф. стерл.
ціни продукції і, по-друге,
41/2 ф. стерл.
ренти, разом 14 ф
стерл.; пересічно за
квартер 31/9 ф. стерл.
Ця пересічна ціна
всієї його продукції
стає через це за реґуляційну
ринкову
ціну. Тому рента на
А становила б 1/9 ф.
стерл. замість 1/2 ф. стерл., а рента на В лишалася б, як і давніш, 4 1/2 ф.
стерл.: 4 1/2 квартерн по 3 1/2 ф. стерл. = 14 ф. стерл., звідси вирахувати
9 1/2 ф. стерл. ціни продукції, лишається надзиск в 41/2 ф. стерл. Бачимо: не
зважаючи на змінені числа, приклад показує, як з допомогою диференційноі
ренти II краща земля, що вже дає ренту, може стати за регуляційну щодо ціни,
і через це вся земля, також і та, що до того часу не давала ренти, може перетворитися
в рентодайну. — Ф. Е.].

Збіжжева рента мусить підвищитись, скоро підвищується реґуляційна
ціна продукції збіжжя, отже, скоро підвищується ціна продукції квартера збіжжя
на реґуляційній землі, або реґуляційна витрата капіталу на одному з родів
землі. Це все одно, як коли б усі роди землі стали менш плодючі і продукували
б, наприклад, на кожні 21/2 ф. стерл. нових витрат капіталу лише по 5/7
квартера замість 1 квартера. Весь надмір збіжжя, що його вони продукують

Рід землі    Акри    Ціна продукції    Продукт в кварт. Продажна  ціна    Грошовий  здобуток   
Збіжжева рента    Грошова  рента
        ф. стер. ф. стер. ф. стер. ф. стер. ф. стер
А 1 3 1 3 3 0 0
В                    1    6    3 1/2    3    101/2    11/2    41/2
С                    1    6    51/2    3    16 1/2    31/2    101/2
D                    1    6    71/2    3    221/2    51/2    161/2
Разом          4    21    171/2        52 1/2    101/2    311/2

Рід землі    Акри    Ціна продукції    Продукт в кварт. Продажна  ціна    Грошовий  здобуток   
Збіжжева рента    Грошова  рента
        ф. стер. ф. стер. ф. стер. ф. стер. ф. стер
А                    1    3             1            31/2       31/2        1/7            1/2
В                    1    9 1/2       41/2    31/2       153/4        111/14    61/4
С                    1    6             51/2    31/2        191/2        311/14    131/4
D                    1    6             71/2    31/2        261/2        511/14     201/4
Разом           4    241/2    181/2           643/4    111/2            401/4
