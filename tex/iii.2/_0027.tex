\index{iii2}{0027}  %% посилання на сторінку оригінального видання
1. Перетворення грошей на позичковий капітал.

Ми вже бачили, що може статися скупчення, надмір, позичкового капіталу,
що лише остільки є у зв’язку з продуктивним нагромадженням, оскільки воно
перебуває до останнього у зворотному відношенні. Де буває в двох фазах
промислового циклу, а саме, поперше, тоді, коли промисловий капітал, в обох
своїх формах продуктивного та товарового капіталу, меншає, отже, на початку
циклу по кризі; та, подруге, тоді, коли справи починають кращати,
але комерційний кредит ще мало удається до банкового кредиту. У першому
випадку грошовий капітал, що його раніше уживалося в продукції та торговлі,
видається незайнятим позичковим капіталом; у другому випадку він
видається капіталом, що його уживають чимраз більше, але за дуже низький
рівень проценту, бо тепер промисловий та торговельний капіталіст накидають
свої умови грошовому капіталістові. Надмір позичкового капіталу у першому
випадку є вираз застою промислового капіталу, а в другому — відносної
незалежности комерційного кредиту від банкового, незалежности, що спирається
на невпинний поворот капіталів, короткі реченці кредиту, та на переважне
орудування власним капіталом. Спекулянти, що важать на чужий кредитовий
капітал, ще не рушили в похід; люди, що орудують власним капіталом,
ще дуже далекі від чистих кредитових операцій. В першій фазі надмір позичкового
капіталу є вираз прямої протилежности дійсного нагромадження. В другій
фазі він збігається з поновним поширом процесу репродукції, відбувається поряд
нього, але не є його причина. Надмір позичкового капіталу вже меншає, він є
вже тільки відносний супроти попиту. В обох випадках поширові дійсного процесу
нагромадження сприяє те, що' низький процент, збігаючись у першому
випадку з низькими цінами, а в другому — з повільно чимраз більшими цінами,
збільшує ту частину зиску, що перетворюється на підприємецький бариш. Ще
більше відбувається це тоді, коли процент підноситься до свого пересічного рівня
підчас найвищої точки розцвіту, коли процент, хоч і зростає, але не пропорціонально
до зиску.

З другого боку, ми бачили, що нагромадження позичкового капіталу може
відбуватися без усякого дійсного нагромадження, за допомогою лише технічних
засобів, як от пошир та концентрація банкової справи, заощадження резерву
для циркуляції або й запасного фонду належних приватним особам платіжних засобів,
що тим способом раз-у-раз перетворюються на короткий час у позичковий
капітал. Хоч цей позичковий капітал — його тому звуть теж текучим капіталом
(floating capital) — завжди лише на короткий час зберігає форму позичкового капіталу
(і так само лише на короткі періоди має використовуватися для дисконтування),
проте він раз-у-раз припливає й відпливає. Коли один його забирає, то другий його
приносить. Таким чином маса позичкового грошового капіталу (ми говоримо тут
взагалі не про позики на роки, а тільки про короткотермінові позики під векселі
та заклади) в дійсності зростає цілком незалежно від дійсного нагромадження.

В. С. 1857. Питання 501. «Що ви розумієте під floating capital?» [Пан
Weguelin, управитель Англійського банку:] «Це — капітал, що його уживають
для грошових позик на короткий час... (502) банкноти Англійського банку...
провінціяльних банків та сума наявних у країні грошей. — [Питання:] За тими
відомостями, що є в комісії, здається, що, коли ви під floating capital розумієте
активну циркуляцію» [а саме банкнот Англійського банку], «то в цій активній
циркуляції не відбувається якогось дуже значного коливання? [Однак дуже велика
є ріжниця в тому, хто авансує цю активну циркуляцію, грошовий позикодавець,
чи сам репродуктивний капіталіст. — Відповідь Weguelin’a:] До floating capital я
залічую резерви банкірів, в яких буває значне коливання». Це значить, отже, що
значне коливання буває в тій частині вкладів, що її не уживали банкіри для нових
\parbreak{}  %% абзац продовжується на наступній сторінці
