за часів Рікардо знецінення паперових грошей протягом XVIII та на початку
XIX віку, а великі бурі на світовому ринку, що в них розраджується колізія
усіх елементів буржуазного процесу продукції; бурі, що їхнє походження та усунення
шукають в межах найповерховнішої та найабстракнішої сфери цього процесу,
в сфері грошової циркуляції. Власне теоретична передумова, що з неї виходить
ця школа економічних ворожбитів години сходить в дійсності ні на що інше, а
тільки на догму, що Рікардо відкрив закони суто-металевої циркуляції. Їм лишилося
тільки підбити під ці закони кредитову циркуляцію та циркуляцію банкнот.

«Найзагальнішим та найнаочнішим явищем торговельних криз є раптовий
загальний спад товарових цін, що настає по довшому загальному піднесенні їх.
Загальний спад товарових цін можна визначити як піднесення відносної вартости
грошей проти всіх товарів, а загальне піднесення цін, навпаки, як спад
відносної вартости грошей. В обох способах вислову явище тільки констатується,
а не пояснюється... Всіляка фразеологія лишає завдання так само
незмінним, як це зробив би переклад його з німецької мови англійською. Тому
грошова теорія Рікардо з’явилася дуже до речі, бо вона тавтології надає подобу
причинного відношення. Від чого постає періодичний загальний спад цін? Від
періодичного піднесення відносної вартости грошей. Навпаки, від чого постає
періодичне загальне піднесення товарових цін? Від періодичного спаду відносної
вартости грошей. Можна було б так само слушно сказати, що періодичне піднесення
та спад цін постає від їхнього періодичного піднесення та спаду...
Коли визнати перетвір тавтології на причинне відношення, то все інше легко
пояснюється само собою. Піднесення товарових цін постає від спаду вартости
грошей. А спад вартости грошей, як ми дізнаємося від Рікардо, постає від переповненої
циркуляції, тобто від того, що маса грошей, що є в циркуляції, підноситься
понад певний рівень, що його визначає їхня власна іманентна вартість
та іманентні вартості товарів. А так само й навпаки, загальний спад товарових цін
постає від піднесення вартости грошей понад їхню іманентну вартість в наслідок
неповної циркуляції. Отже, ціни підносяться та спадають періодично, бо періодично
в циркуляції буває занадто багато або занадто мало грошей. І хоч би й було
доведено, що піднесення цін збігається зі зменшенням грошової циркуляції, а
спад цін зі збільшенням тієї циркуляції, проте можна б запевняти, що в наслідок
деякого — хоч статистично цього ніяк не можна довести — зменшення або збільшення
маси товарів, що обертаються, кількість грошей в циркуляції, хоч і не
абсолютно, але відносно збільшилась або зменшилась. Ми бачили отже, що за
Рікардо ці загальні коливання цін мусять відбуватися й за суто-металевої циркуляції,
але тут вони вирівнюються таким чергуванням, що, напр., неповна
циркуляція зумовлює спад товарових цін, вивіз товарів за кордон, але цей вивіз
зумовлює довіз золота в країну, а цей доплив грошей знову зумовлює піднесення
товарових цін. Навпаки буває при переповненій циркуляції, коли товари імпортують,
а золото експортують. А що не зважаючи на ці загальні коливанпя цін,
що постають з самої природи Рікардової металевої циркуляції, їхня інтенсивна
та потужна форма, форма криз, належить до періодів розвинутої кредитової системи,
то й цілком очевидно, що видання банкнот регулюється не достеменно за законами
металевої циркуляції. Металева циркуляція має свій цілющий засіб в імпорті
та експорті благородних металів, що, негайно ввіходячи в циркуляції як монета,
своїм допливом та відпливом знижують або підносять товарові ціни. Той самий
вплив на товарові ціни мусять тепер штучно справляти банки, наслідуючи закони
металевої циркуляції. Коли гроші припливають з-за кордону, то це доказ того,
що циркуляція неповна, вартість грошей занадто висока, а товарові ціни занадто
низькі, отже, в циркуляцію доводиться кинути банкнот пропорціонально кількості
новоімпортованого золота. Навпаки, їх доводиться забирати з циркуляції відповідно
до того, як золото відпливає з країни. Інакше кажучи, видання банкнот мусить
