ІІри нормальному стані продукції тільки частина новодолученої праці
вживається на продукцію і тому на покриття сталого капіталу: саме якраз та
частина, що покриває сталий капітал, витрачений у продукції засобів споживання,
речових елементів доходу. Це вирівнюється тим, що ця стала частина,
кляси II не коштує новодолучуваної праці. Але сталий капітал, який (коли розглядати
сукупний процес репродукції, що в ньому, отже, включено і те вирівнювання
кляс І і II), не є продукт новодолученої праці, хоч цей продукт неможливо
було б випродукувати без нього, — цей сталий капітал, розглядуваний з речового
боку, підлягає підчас процесу репродукції, випадковостям і небезпекам,
які можуть його зменшити. (Але далі, коли розглядати його щодо вартости, то
він також може знецінитися в наслідок зміни у продуктивній силі праці;
проте, це стосується лише до поодиноких капіталістів). Відповідно до цього частина
зиску, отже, додаткової вартости, а тому і додаткового продукту, що в ньому
(коли розглядати його з погляду вартости) репрезентується лише новодолучена
праця, служить страховим фондом. При цьому суть справи ані трохи не змінюється
від того, чи порядкує цим страховим фондом страхове товариство як
окреме підприємство, чи ні. Це є однісінька частина доходу, що не споживається
як такий, і не служить неодмінно фондом акумуляції. Чи служить вона фактично
фондом акумуляції, чи лише покриває прогріхи репродукції, це залежить від
випадку. Це також однісінька частина додаткової вартости і додаткового продукту,
отже, додаткової праці, що крім частини, яка служить для акумуляції, отже,
для поширення процесу репродукції, мусить існувати і далі по знищенні
капіталістичного способу продукції. Звичайно, це має своєю передумовою, що
частина, регулярно споживувана безпосередніми продуцентами, не лишиться обмеженою на своєму
теперішньому мінімальному рівні. За винятком додаткової
праці на тих, хто через свій вік ще не може або вже не може брати участи у продукції,
відпаде всяка праця на утримання тих, хто не працює. Коли ми уявимо
собі суспільство при його виниканні, то побачимо, що тут немає ще випродукованих
засобів продукції, отже, немає сталого капіталу, що його вартість увіходить
у продукт, і при репродукції в тому самому маштабі мусить покриватися
in natura з продукту в розмірі, визначуваному його вартістю. Але природа
безпосередньо дає тут засоби існування, їх не доводиться продукувати. Тому
вона залишає також дикунові, що йому доводиться задовольняти лише малі потреби,
час, — не на те, щоб використати ще не сущі в наявності засоби продукції для
нової продукції, а на те, щоб, крім праці, якої коштує привласнення наявних
у природі засобів існування, витрачати працю на перетворення інших продуктів
природи на засоби продукції, лук, кам’яний ніж, човен і т. ін. Процес цей,

куплені по ринковій ціні, і теж мають в собі заробітні плати, ренти на капітали, земельні ренти і
підприємницькі зиски... спостереження це доводить лише неможливість розкласти потрібну ціну на її
простіші елементи» (ib., примітка). — У своїх Considérations sur la nature du revenu national (Paris
1824)
Шторх y своїй полеміці s Сеєм, правда, розуміє все безглуздя, що до нього призводить помилкова
аналіза товарової вартости, що розкладає її без остачі тільки на доходи, і правильно висловлюється
про все безглуздя цих висновків — з погляду не поодиноких капіталістів, а нації, — але сам він не
робить і кроку вперед в аналізі prix nécessaire (потрібної ціни), відносно якої він, замість
відсувати розв’язання
завдання до безконечности, заявляє в своєму «Cours» що її неможливо розкласти на її дійсні елементи.
«Ясно, що вартість річного продукту поділяється почасти на капітали, почасти на зиски, і що кожна з
цих частин вартости річного продукту регулярно купує продукти, потрібні нації так для збереження її
капіталу, як і для відновлення її споживного фонду (р. 134—135)... Чи зможе вона (селянська родина,
що працює самостійно) жити у своїх клунях і стайнях, живитись тільки насінням
і травою, одягатися з своєї робочої худоби, витрачати свої хліборобські зваряддя? Згідно з
твердженими п. Сея, слід було б відповісти позитивно на всі ці питання (135—136)... Коли визнати, що
дохід нації дорівнює її гуртовому продуктові, тобто, що з нього не доводиться вираховувати капітали,
то доведеться також визнати, що вона може непродуктивно витратити всю вартість свого річного
продукту,
не роблячи найменшої шкоди своєму майбутньому доходові. (147) Продукти, що складають капітал нації,
не підлягають споживанню», (р. 150)
