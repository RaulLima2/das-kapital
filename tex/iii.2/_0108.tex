\parcont{}  %% абзац починається на попередній сторінці
\index{iii2}{0108}  %% посилання на сторінку оригінального видання
в руках великих землевласників. Увесь Вестенд Лондону, на північ та на південь
від Temple Ваг, належить майже виключно приблизно півтузіневі великих
землевласників, які винаймають його за величезну орендну плату, і якщо
контракти ще подекуди не закінчились, то реченці хутко надходять їм одному по
одному. Те саме має силу більш або менш і для кожного міста в королівстві. Але
навіть на цьому ця жадоблива система виключности та монополії ще не спиняється. Майже всі докові
споруди наших приморських міст в наслідок такого
самого процесу узурпації перебувають у руках великих землевласників-левіятанів» (1. с., р. 93). В
таких обставинах ясно, що хоч перепис 1861 року для
Англії та Велзу при загальній цифрі людности в 20.066.224 визначає число
домовласників в 36.032, проте, відношення числа власників до числа будинків
та цифри людности набуло б цілком іншого вигляду, коли б великих власників
показати окремо від дрібних.

Цей приклад власности на будівлі важливий тим, 1) що він ясно показує
ріжницю між власне земельною рентою і процентом на долучений до землі
основний капітал, процентом, що може становити додаток до земельної ренти.
Поки триває строк орендного договору, процент на будівлі, а так само і на
капітал, вкладений у хліборобстві орендарем у землю, дістається промисловому капіталістові, —
будівельному спекулянтові або орендареві — і сам по собі
не має нічого спільного з земельною рентою, яку щороку в певні терміни доводиться виплачувати за
користування землею; 2) що він показує, як разом з
землею, кінець-кінцем, дістається землевласникові долучений до неї чужий капітал, і що процент з
нього збільшує ренту.

Деякі письменники, почасти як оборонці земельної власности проти нападів
буржуазних економістів, почасти, намагаючись перетворити капіталістичну систему продукції з системи
суперечностей в систему «гармонії», як, наприклад,
Кері, намагалися виставити, ніби земельна рента, специфічний економічний
вираз земельної власности, ідентична з процентом. Цим була б загладжена суперечність між земельними
власниками і капіталістами. Зворотну методу вживано при початку капіталістичної продукції. Тоді для
звичайного уявлення земельна власність була ще примітивною і поважною формою приватної власности,
тимчасом
як процент на капітал ганьблено як лихварство. Тому Dudley North, Locke
та інші змальовували процент на капітал як форму аналогічну земельній
ренті, — цілком так само як Тюрґо виводив виправдання проценту з існування
земельної ренти. — Згадані новітні письменники — цілком залишаючи осторонь той факт, що земельна
рента може існувати й існує в чистому вигляді,
без усякого додатку проценту на долучений до землі капітал, — забувають, що
земельний власник одержує таким чином не тільки процент на чужий капітал,
який йому нічого не коштував, але, крім того, одержує ще даром і чужий
капітал. Виправдання земельної власности, як і всяких інших форм власности,
відповідних певному способові продукції, є в тому, що самий спосіб продукції,
а, отже, і відносини продукції та обміну, що випливають з нього, являють собою
історичну минущу конечність. Правда, як ми побачимо опісля, земельна власність
відрізняється від інших видів власности тим, що на певному рівні розвитку
навіть з погляду капіталістичного способу продукції, вона виявляється зайвою
і шкідливою.

Земельна рента може сплутуватися з процентом ще в одній формі, і таким чином її специфічний характер
може лишитися незрозумілим. Земельна
рента визначається в певній сумі грошей, що її земельний власник щорічно здобуває від здачі на
оренду певної дільниці землі. Ми бачили, як відбувається
капіталізація всякого певного грошового доходу, тобто, яким чином
цей грошовий дохід можна розглядати як процент на уявлюваний капітал.
Коли, наприклад, пересічний розмір проценту 5\%, то й річна земельна рента
\parbreak{}  %% абзац продовжується на наступній сторінці
