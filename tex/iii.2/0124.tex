тість, а сила води не має вартости, користання парою все ж давало б рішучі
переваги, недосяжні при використанні сили води, і коли б ці переваги більше
ніж компенсували силу води, то сила води не мала б застосування і не могла б
породити жодного надзиску, а, отже, і ренти.

Третє: Сила природи не є джерело надзиску, а лише його природна
база, бо це є природна база виключно підвищеної продуктивної сили праці. Так
взагалі споживна вартість є носій мінової вартости, а не причина її. Коли б
ту саму споживну вартість можна було створювати без праці, вона б не мала
жодної мінової вартости, але як і давніш, мала б свою природну корисність
як споживна вартість. Але, з другого боку, без споживної вартости, отже,
без такого природного носія праці, річ не має жодної мінової вартости. Коли б
різні вартості не вирівнювались у ціни продукції і різні індивідуальні ціни
продукції не вирівнювались би в загальну ціну продукції, яка реґулює ринок,
то звичайне підвищення продуктивної сили праці в наслідок використання водоспаду,
лише знизило б ціну товарів, продукованих з допомогою водоспаду, але
не підвищило б тієї частини зиску, що міститься в цих товарах; так само, як,
з другого боку, ця підвищена продуктивна сила праці взагалі не перетворювалась
би на додаткову вартість, коли б капітал продуктивну силу вживаної ним
праці, природну і суспільну, не привлащував би як свою власну.

Четверте: Земельна власність на водоспад сама по собі не має ніякого
чинення до створення цієї частини додаткової вартости (зиску), а тому і взагалі
ціни товару, який продукується з допомогою водоспаду. Цей надзиск існував
би і тоді коли б не існувало земельної власности, коли б, наприклад,
земля, до якої належить водоспад, використовувалась фабрикантом, як безгосподарна
земля. Отже, земельна власність не створює тієї частини вартости,
яка перетворюється в надзиск, а лише дає земельному власникові, власникові
водоспаду, можливість перекласти цей надзиск з кишені фабриканта у свою
власну. Земельна власність є причина не створення цього надзиску, а його
перетворення у форму земельної ренти, отже, привласнення цієї частини зиску,
зглядно ціни товару, власником землі або водоспаду.

П’яте: Ясно, що ціна водоспаду, отже, ціна, яку одержав би земельний
власник, коли б він продав його третій особі, або самому фабрикантові,
спочатку не входить у ціну продукції товарів, хоч входить в індивідуальні
витрати продукції даного фабриканта; бо рента виникає тут з ціни продукції
товарів того самого роду, продукованих з допомогою парових машин,
з ціни продукції, що реґулюється незалежно від водоспаду. Але, далі, ця ціна
водоспаду взагалі є іраціональний вираз, що за ним ховається реальне економічне
відношення. Водоспад, як земля взагалі, як усі сили природи, не має
жодної вартости, бо в ньому не зрічевлено жодної праці, а тому не має він
жодної ціни, яка нормально є не що інше, як виражена в грошах вартість.
Де немає вартости, там ео ipso *) нічого виражати в грошах. Ця ціна є не
що інше, як капіталізована рента. Земельна власність дає власникові можливість
захоплювати ріжницю між індивідуальним зиском і пересічним зиском,
захоплюваний в такий спосіб зиск, що відновляється щорічно, може бути капіталізований
і тоді виступає як ціна самої сили природи. Коли надзиск, що
його дає фабрикантові використання водоспаду, становить 10 ф. ст. на рік, а
пересічний процент 5%, то ці 10 ф. ст. на рік становлять проценти з капіталу в
200 ф. ст. і ця капіталізація річних 10 ф. ст., що водоспад дає змогу власникові
його захоплювати їх у фабриканта, виступає тоді, яв капітальна вартість самого
водоспаду. Те, що водоспад не має вартости і що ціна його є звичайний
відбиток захоплюваного надзиску, капіталістично обчисленого, це одразу

*) Ео ipso — тим самим. Пр. Ред.
