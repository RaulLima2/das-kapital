\parcont{}  %% абзац починається на попередній сторінці
\index{iii2}{0213}  %% посилання на сторінку оригінального видання
інтереси кляси капіталістів і збагачення взагалі за конечну мету держави і прокламують
буржуазне суспільство протилежно старій надземній державі. Але разом
з цим виявляється свідомість того, що розвиток інтересів капіталу і кляси капіталістів,
капіталістичної продукції, зробився за базу національної сили і національної
переваги в сучасному суспільстві.

Далі, у фізіократів справедливе те, що на ділі вся продукція додаткової вартости,
отже, і ввесь розвиток капіталу, розглядуваний з боку природної бази,
ґрунтується на продуктивності хліборобської праці. Коли б люди взагалі не могли
продукувати протягом одного робочого дня більше засобів існування, отже, в вузькому
розумінні, більше хліборобських продуктів, ніж потрібно кожному робітникові
для його власної репродукції, — коли б денної витрати всієї його робочої сили
було досить лише для того, щоб випродукувати засоби існування, потрібні для
його особистого споживання, то взагалі не могло б бути мови ані про додатковий
продукт, ані про додаткову вартість. Продуктивність хліборобської праці, що
перебільшує індивідуальну потребу робітника, є база всякого суспільства, і насамперед
база капіталістичної продукції, яка дедалі більшу частину суспільства
відриває від продукції безпосередніх засобів існування і перетворює її, за висловом
Стюарта, в free heads\footnote*{
Free heads — дослівно вільні голови, тобто вільні робочі руки. \emph{Прим. Ред.}
}, дає можливість користатися нею в інших сферах.

Але, що сказати про тих новіших письменників-економістів, котрі як
Daire, Passy та інші, на схилі життя всієї клясичної економії, навіть на її смертельній
постелі, повторюють найпервісніші уявлення про природні умови додаткової
праці, і, отже, додаткової вартости взагалі, і гадають, ніби вони цим дають
щось нове і переконливе про земельну ренту після того як цю земельну ренту
вже давно описано як осібну форму і специфічну частину додаткової вартости?
Саме для вульгарної економії характеристичне таке: те, що на певнім пережитім ступені
розвитку було нове, оригінальне, глибоке і слушне, вона повторює в
такий час, коли воно є тривіяльне, відстале і фалшиве. Вона визнає таким
чином, що в неї не має навіть передчуття про проблеми, які цікавили клясичну
економію. Вона сплутує їх з питаннями, що могли ставитись лише на нижчому
ступені розвитку буржуазного суспільства. Так само стоїть справа з її безнастанним
та самозадоволеним пережовуванням фізіократичних засад про вільну
торгівлю. Ці засади давно втратили всякий теоретичний інтерес, хоч би як
практично вони цікавили ту або іншу державу.

У власне натуральному господарстві, де хліборобський продукт зовсім не
вступає в процес циркуляці, або вступає в нього лише дуже незначна частина
цього продукту, і навіть лише порівняно незначна частка тієї частини продукту,
яка становить дохід земельного власника, — як наприклад, в багатьох
староримських лятифундіях, в віллах Карла Великого, а також (дивись Vincard,
Histoire du travail) в більшій чи меншій мірі протягом усього середньовіччя, —
продукт і додатковий продукт великих маєтків зовсім не складається тільки
з продуктів хліборобської праці. Він охоплює також і продукти промислової
праці. Домашня реміснича і мануфактурна праця, як допоміжна продукція
при хліборобстві, що становить базу, є умова того способу продукції, на якому
ґрунтується це натуральне господарство так у давній і середньовічній Европі, якще
до нашого часу — і в індійській громаді, де її традиційна організація ще не
зруйнована. Капіталістичний спосіб продукції цілком знищує це сполучення:
процес, який у великому маштабі можна вивчити особливо на прикладі Англії
за останню третину XVIII століття. Голови, що виросли у більш чи менш напівфевдальних
суспільствах, Гереншванд, наприклад, ще в кінці XVIII століття
вбачають у цьому відокремленні хліборобства від мануфактури одчайдушно сміливий
суспільний експеримент, незрозуміло ризикований спосіб існування. І навіть
\parbreak{}  %% абзац продовжується на наступній сторінці
