і тоді можна було б перевести закордонні платежі (by which the foreign payment
would he accomplished). — 906. Штучне обмеження повноважень банку, заведене
актом 1844 року, замість давньої та природної межі його повноважень, що визначалася
дійсною сумою його металевого запасу, породжує штучні труднощі в
справах, а разом з тим і такий вплив на товарові ціни, який без цього акту
був би цілком непотрібен. — 968. За чинности акту 1844 року в звичайних
умовах не можна зменшити металевий запас банку значно нижче від 9 1/2 міл.
Це справило б вплив на ціни та на кредит, що мусило б призвести до такої
зміни в закордонних вексельних курсах, що довіз золота зріс би, збільшивши
тим суму золота в емісійному відділі. — 996. При сучасному обмеженні ви»
[банк] «не порядкуєте рухом срібла, що потрібно в ті часи, коли потребують
срібла, щоб впливати на закордонний курс. — 999. Яку мету мав припис, який
обмежує срібний запас банку на 1/5 його металевого запасу? — На це питання
я не можу відповісти».

Мета була — удорожити гроші; цілком та сама, що її малося на оці —
незалежно від currency-теорії, — відокремлюючи відділи та зобов’язуючи шотландські
та ірландські банки тримати в резерві золото в тому випадку, коли
вони видаватимуть банкноти понад певну норму. Оттак постала децентралізація
національного металевого скарбу, що зменшила його здатність виправляти несприятливі
вексельні курси. На піднесення рівня проценту спрямовано всі дальші
постанови, а саме: що Англійський банк не сміє видавати банкноти понад 14 міл.,
крім тих банкнот, що матимуть на покриття золотий запас; що банковий відділ має
управлятися як звичайний банк, знижуючи рівень проценту підчас надміру
грошей та підносячи його підчас скрути; обмеження срібного запасу, цього
головного способу виправляти вексельний курс у стосунках з континентом та
Азією; приписи щодо шотландських та ірландських банків, що ніколи не потребують
золота для експорту, а тепер мусять тримати його з приводу сутоілюзорного
розміну своїх банкнот. Факт той, що акт 1844 року вперше породив
в 1857 році штурм на шотландські банки по золото. Нове банкове
законодавство не робить теж ніякої ріжниці між відпливом золота закордон та
відпливом його з банку всередину країни, хоч їхні впливи, звичайно, цілком
відмінні. Відси повсякчасні різкі коливання в ринковій нормі проценту. Щодо
срібла, Palmer двічі, № 992 та № 994, каже, що банк може купувати срібло
за банкноти тільки тоді, коли вексельний курс для Англії сприятливий, отже,
коли срібло є зайве; бо, «1093. Однісінька мета, що задля неї можна тримати
значну частину металевого скарбу в сріблі, є та, щоб полегшувати закордонні
платежі протягом того часу, коли вексельні курси несприятливі для Англії. —
1008. Срібло — товар, що для цієї мети» [платіж закордон] «є... найзручніший
товар, бо воно є гроші для решти цілого світу. Лише Сполучені Штати брали
останніми часами виключно золото».

На його думку, підчас скрути Англійському банкові не треба було підвищувати
процент понад старий рівень, понад 5%, доки несприятливі вексельні
курси не почнуть відтягати золото закордон. Коли б не було акту 1844 року,
то банк міг би при цьому дисконтувати легко всі першорядні векселі (first class
bills), що їх подавалось йому до дисконту. [1018—20]. Але при акті 1844 року,
й при тому стані, що в ньому опинився банк у жовтні 1847 року, «не було
такого рівня проценту, що його він не міг би вимагати від кредитоздатних фірм
та що його ті фірми не платили б охоче, аби тільки переводити далі свої платежі».
І цей високий рівень проценту був саме метою того акту.

«1029. Я мушу добре відрізняти вплив рівня проценту на закордонний
попит» [на благородний метал] «та підвищення того рівня з метою загальмувати
штурм на банк підчас браку кредиту всередині країни. — 1023. Перед актом
1844 року, коли курси були сприятливі для Англії, а в країні панувало зане-
