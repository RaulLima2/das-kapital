\parcont{}  %% абзац починається на попередній сторінці
\index{iii2}{0153}  %% посилання на сторінку оригінального видання
капітал. Це включає, за незмінних ріжниць між родами землі, зріст надпродукту,
пропорційний зростові вкладеного капіталу. Отже, випадок цей виключає всяку
додаткову витрату на землю А, яка вплинула б на диференційну ренту. На цій землі
норма надзиску = 0; отже, вона лишається = 0, бо ми припустили, що продуктивна
сила додаткового капіталу, а тому і норма надзиску лишаються сталими.

Але реґуляційна ціна продукції може за цих умов лише знизитися, бо замість
ціни продукції з А реґуляційною стає ціна продукції ближчої якістю землі
В або взагалі з будь-якої землі, кращої, ніж А; отже, коли б ціна продукції
на землі С зробилась регуляційною, то капітал був би вилучений з А,
або навіть з А і В, і таким чином всі землі, гірші, ніж С, випали б з конкуренції
земель, на яких сіють пшеницю. Умова, потрібна для цього за даних
припущень, є в тому, щоб надпродукт з додаткових капіталовкладень задовольняв
потребам, і щоб тому продукція на гіршій землі А тощо зробилася
зайвою для поновлення подання.

Отже, візьмімо, наприклад, таблицю II, але змінимо її так, щоб замість20
квартерів, потребу задовольняли 18 квартерів. Земля А відпала б; D, а
разом з нею ціна продукції в 30 шил. за кв. стала б реґуляційною. Диференційна
рента набуває тоді такої форми:

Таблиця IV

Рід землі    Акри    Капітал    Зиск    Ціна продукції
Продукт в квартер. Продажна ціна    Здобуток    Рента    Норма надзиску
                                В збіжжі    В грошах
     ф. ст. ф. ст. ф. ст. ф. ст. ф. ст. ф. ст. Кварт. ф. ст.
B    1    5    1    6    4    1 1/2    6    0    0    0
C    1    5    1    6    6    1 1/2    9    2    3    60\%
D    1    5    1    6    8    1 1/2    12    4    6    120\%
 Разом    3    15    3    18    18        27    6    9

Отже, вся рента порівняно з таблицею II знизилась би з 36 ф. стерл.
до 9, а в збіжжі з 12 кварт, до 6; вся продукція знизилася б лише на 2
квартери, з 20 до 18. Норма надзиску, обчислена у відношенні до капіталу,
знизилася б наполовину, з 180 до 90\%\footnote*{
До 60\%, тобто знизилася б втроє, бо в таблиці II вона = 36/20 × 100 = 180\%, а в тaблиці
IV вона = 9/15 × 100 = 60\%. Прим. Ред.
}. Отже, пониженню ціни продукції
тут відповідає зменшення збіжжевої і грошової ренти.

Порівняно з таблицею І, відбувається лише зменшення грошової ренти;
збіжжева рента в обох випадках дорівнює 6 квартерам; але тільки в одному
випадку вона = 18 ф. стерл., а в другому = 9 ф. стерл. Для земель С і D
збіжжева рента проти таблиці І лишилась та сама\footnote*{
Те, що сказано тут, правильне лише для землі С, але неправильне для землі D, бо в табл. І земля
D дає 3 кв. ренти, а в табл. IV земля D дає 4 кв. ренти. Те, що тут сказано, було б правильне, коли
взяти загальну ренту з земель B, C і D. Прим. Ред.
}. В дійсності, в наслідок
того, що додаткова продукція, досягнена з допомогою додаткового капіталу рівної
продуктивности, витиснула з ринку продукт А і разом з тим усунула землю А
з числа конкурентних аґентів продукції — в наслідок цього в дійсності створилася
нова диференційна рента І, в якій краща земля В від грає ту саму ролю,
яку давніш відігравала гірша земля А. В наслідок цього, з одного боку, відпадає
рента з В; з другого боку, згідно з припущенням, вкладення додаткового
\parbreak{}  %% абзац продовжується на наступній сторінці
