талу. Але, щоправда, є перешкода, що виникає в наслідок законів зросту капітальної
вартости, в наслідок тих меж, що в них вартість капіталу може зростати
як капітал. Багатість грошового капіталу, як такого, не означає неминуче ані
надмірної продукції, ані бодай лише браку сфер для вжитку капітала.

Нагромадження позичкового капіталу полягає просто в тому, що гроші
осідають, як гроші призначені до позичання. Цей процес дуже відмінний від
дійсного перетворення на капітал; це — тільки нагромадження грошей в такій
формі, в якій вони можуть перетворюватись на капітал. Але це нагромадження,
як уже доведено, може виражати моменти, дуже відмінні від дійсного нагромадження.
При постійному поширі дійсного нагромадження це поширене нагромадження
грошового капіталу може почасти бути його результатом, почасти —
результатом моментів, що, відбуваючись одночасно з поширом дійсного нагромадження,
проте цілком відмінні від нього, а почасти, насамкінець, навіть результатом
спину дійсного нагромадження. Вже тому, що нагромадження позичкового
капіталу незвичайно зростає з причини таких моментів, які від дійсного
нагромадження незалежні, а проте відбуваються поряд нього, вже тому мусить
в певні фази циклу раз-у-раз поставати багатість грошового капіталу і ця
багатість мусить розвиватися з розвитком кредиту. Отже, разом з цією багатістю
мусить одночасно розвиватися необхідність поширювати процес продукції поза його
капіталістичні межі: звідси надмірна торговля, надмірна продукція, надмірний
кредит. Одночасно мусить це відбуватися завжди в формах, що викликають
реакцію.

Щодо нагромадження грошового капіталу з земельної ренти, заробітної плати,
і т. ін., то зайво на цьому тут спинятися. Слід підкреслити лише те, що справа
дійсного заощадження та поздержливости (серед тих, що збирають скарби), оскільки
вона дає елементи нагромадження, з поступом капіталістичної продукції, через поділ
праці, припадає тим, що одержують мінімум таких елементів та ще досить часто
гублять своє заощаджене, як от робітники при банкрутствах банків. З одного
боку, промисловий капіталіст не сам «заощаджує» свій капітал, але порядкує чужими
заощадженнями пропорціонально величині власного капіталу; з другого боку,
грошовий капіталіст робить з чужих заощаджень свій капітал, а кредит, що його
дають капіталісти репродукції одні одним, і кредит, що їм дає публіка, повертає
він у джерело свого особистого збагачування. Так знищується до краю остання
ілюзія капіталістичної системи, ніби капітал породжується власною працею та ощадженням.
Не тільки зиск є присвоювання чужої праці, але й капітал, що ним
пускається в рух та визискується чужу працю, складається з чужої власности,
яку грошовий капіталіст дає до розпорядку промисловому капіталістові та за
це своєю чергою визискує його.

Ще треба дещо зауважити про кредитовий капітал.

Як часто та сама грошова монета може фігурувати як позичковий капітал,
це цілком залежить, як уже розвинуто вище, від того:

1) як часто вона реалізує товарові вартості в продажі та платежі, отже,
переносить капітал, і далі від того, як часто вона реалізує дохід. Тому, як
часто вона переходить до інших рук як реалізована вартість — чи то капіталу
або доходу, це залежить, очевидно, від обсягу та маси дійсних оборотів;

2) це залежить від економії на платежах та від розвитку й організації
кредитової справи.

3) Насамкінець, від зв’язку кредиту між собою та швидкости руху кредитів,
так що, коли в одному пункті вона осідає як вклад, в другому негайно
виходить вона знову, як позика.

Навіть, коли припустити за форму, що в ній існує позичковий капітал, лише
форму дійсних грошей, золота або срібла, товарів, що їхня речовина править за
мірило вартостей, то велика частина цього грошового капіталу завжди неминуче
