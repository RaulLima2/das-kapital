Пан Neave, управитель Англійського банку, дає таке свідчення перед
банковою Комісією 1858 року: «№ 947. (Питання): Хоч і яких заходів ви
раз-у-раз уживаєте, проте сума банкнот в руках публіки, як ви кажете,
лишається однакова; тобто приблизно 20 міл. ф. ст.? — За звичайних часів,
здається, потреби публіки вимагають приблизно 20 міл. В певні часи, що періодично
повторюються протягом року, сума та підноситься на 1 або 1 1/2 міл.
Коли публіка потребує більше, вона може, як я сказав, завжди одержати
потрібну кількість в Англійському банку. — 948. Ви сказали, що підчас паніки
публіка не дозволяла вам зменшувати суму банкнот; чи не буде ваша ласка
обґрунтувати цю вашу думку? — Під час паніки публіка, так мені здається, має
повну змогу добувати собі банкноти; і, природно, поки банк має зобов’язання,
публіка може, на основі цього зобов’язання, брати з банку банкноти. — 949.
Отже, здається, що повсякчас потрібно банкнот Англійського банку приблизно
на 20 міл. ф. ст.? — 20 міл. ф. ст. банкнотами в руках публіки; це число
зміняється. Буває 18 1/2,19, 20 міл. і т. ін.; але пересічно ви можете сказати —
19—20 мільйонів».

Свідчення Тома Тука перед комісією лордів про Commercial Distress*)
(С. D. 1848/57); № 3094: «Банк не має сили своєвільно збільшувати суму
банкнот в руках публіки; він має силу зменшувати суму банкнот в руках
публіки; але лише за допомогою дуже ґвалтовної операції».

І. C. Wright, банкір в Нотігемі протягом останніх 30 років, з’ясувавши
докладно неможливість того, щоб провінціяльні банки могли коли-небудь зберігати
в циркуляції більше банкнот, ніж того потребує та хоче публіка, каже про банкноти
Англійського банку (С. D. 1848/57) № 2844: «Я не знаю ніяких меж»
(щодо видання банкнот) «для Англійського банку, але всякий надмір циркуляції
переходитиме у вклади й так прийматиме іншу форму».

Те саме має силу для Шотляндії, де циркулюють майже самі лише паперові
гроші, бо там, як і в Ірляндії, дозволено й однофунтівки, а також
тому що, «the Scotch hate gold**)». Кеннеді, директор одного шотляндського банку,
заявляє, що банки не спромоглися б навіть зменшити циркуляції своїх банкнот,
та дотримується він «тієї думки, що, доки операції всередині країни для свого
здійснення вимагають банкнот або золота, доти банкіри мусять постачати стільки
засобів циркуляції, скільки того потребують ці операції, — чи на вимогу своїх
вкладників, чи то якось інакше... Шотляндські банки можуть обмежити свої
операції, але вони не можуть контролювати видання своїх банкнот» (ib. № 3446 —
48). Так само висловлюється Андерсон, директор Union Bank of Scotland, ib.
№ 3578: «Чи заважає система взаємного обміну банкнотами» [між шотляндськими
банками]» надмірному виданню банкнот з боку якогось поодинокого банку? —
Так; але ми маємо більш дійсний засіб, ніж обмін банкнотами» [який в дійсності
не має з цим нічого до діла, але, що правда, забезпечує здатність банкнот кожного
банку обертатись по всій Шотляндії], «а саме — і це є загальний звичай
в Шотляндії — мати рахунок в банку; кожен, хто має сякі-такі гроші, має
й рахунок в якомусь банку й щодня складає до банку гроші, що їх він сам
безпосередньо не потребує, так що наприкінці кожного операційного дня всі
гроші є в банках опріч тих, що їх кожен має в кишені».

Так само і в Ірляндії; див. свідчення управителя Ірляндського банку,
Mac Donnall’я, та директора провінціяльного банку Ірляндії, Murray’я, перед
тією самою комісією.

Так само як циркуляція банкнот не залежить від волі Англійського банку,
вона не залежить і від стану того золотого скарбу в коморах банку, що забез-

*) Commercial Distress — торговельна криза. Прим. Ред.

**) Шотляндець ненавидить золото. Прим. Ред.
