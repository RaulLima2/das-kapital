згідно з припущенням, реґулює загальну ринкову ціну, чи входить вона таким самим чином, як податок в
ціну товару, оподаткованого ним, тобто як елемент, незалежний від вартости цього товару.

       Це аж ніяк не випливає доконечно, а робили такий висновок лише тому,
що до цього часу не розуміли різниці між вартістю товарів і їхньою ціною
продукції. Ми бачили, що ціна продукції певного товару зовсім не тотожня
з його вартістю, хоч ціни продукції товарів, розглядувані в цілому, регулюються
тільки їхньою загальною вартістю, і хоч рух цін продукції різних сортів
товарів, припускаючи всі інші умови незмінними, визначається виключно
рухом їхніх вартостей. Ми показали, що ціна продукції певного товару може
бути вища або нижча його вартости і лише як виняток збігається з його вартістю.
Тому, той факт, що продукти землі продаються вище їхньої ціни продукції,
зовсім ще не доводить, що вони також продаються вище їхної вартости; як той факт, що промислові
продукти пересічно продаються по їхній ціні продукції зовсім не доводить, що вони продаються по
їхній вартості. Можливо, що хліборобські продукти продаються вище від їхньої ціни продукції і нижче
від їхньої вартости, як з другого боку багато промислових продуктів дають ціну продукції тільки
тому, що вони продаються вище від їхньої вартости.

      Відношення ціни продукції певного товару до його вартости визначається
виключно тим відношенням, в якому змінна частина капіталу, що випродукував товар, стоїть до його
сталої частини, або органічним складом капіталу, яким випродуковано товар. Коли склад капіталу у
певній сфері продукції нижчий, ніж склад пересічного суспільного капіталу, тобто коли відношення
його змінної складової частини, витраченої на заробітну плату, до його сталої складової частини,
витраченої на речові умови праці, вище, ніж в суспільному пересічному капіталі, то вартість його
продукту мусить стояти вище за його ціну продукції. Тобто такий капітал тому, що він вживає відносно
більше живоїпраці, продукує при рівній експлуатації праці більше додаткової вартости, отже, більше
зиску, ніж рівновелика відповідна частина пересічного суспільного капіталу. Тому вартість його
продукту буде вища за ціну його продукції, бо ця ціна продукції рівна покриттю капіталу плюс
пересічний зиск, а пересічний зиск нижчий, ніж зиск створений у цьому товарі. Додаткова вартість,
створена пересічним суспільним капіталом, менша за додаткову вартість, створену капіталом цього
низького складу. Зворотне буває тоді, коли капітал, вкладений у певну сферу продукції, є вищого
складу, ніж пересічний суспільний капітал. Вартість випродукованих ним товарів нижча за їхню ціну
продукції, що становить загальне явище для продуктів найрозвиненіших галузей промисловості.
        Коли капітал у певній сфері продукції має нижчий склад, ніж пересічний
суспільний капітал, то це насамперед є лише інший вираз того, що продуктивна
сила суспільної праці в цій окремій сфері продукції нижча від пересічного
рівня; бо досягнений ступінь продуктивної сили набуває собі виразу
у відносній перевазі сталої частини капіталу над змінною, або в постійному
зменшенні тієї складової частини даного капіталу, яку витрачається на заробітну
плату. Коли навпаки, капітал у певній сфері продукції має вищий склад,
то це є вираз такого розвитку продуктивної сили, що перевищує пересічний рівень.
        А втім, не говорячи про власне художні роботи, що їх розгляд по суті
справи не стосується до нашої теми, само собою зрозуміло, що різні сфери про дукції потребують, за
їхніми технічними особливостями, різного відношення між сталим і змінним капіталом, і що жива праця
мусить займати в одних сферах більше місця, в інших менше. Наприклад, у видобувній промисловості,
яку слід пильно відрізняти від хліборобства, сировий матеріял як елемент сталого капіталу цілком
відпадає, та й допоміжний матеріял лише де-не-де відіграє
