фонду страхування і резервного фонду, закону конкуренції і т. ін. і практично
доводять йому, що зиск не є просто категорія розподілу продукту, призначеного
для особистого споживання. Далі, ввесь капіталістичний процес продукції реґулюється
цінами продукту. Але реґуляційні ціни продукції в свою чергу реґулюється
процесом вирівнювання норми зиску і відповідним їй розподілом капіталу
між різними сферами суспільної продукції. Отже, зиск з’являється тут, як головний
чинник, не розподілу продукту, а самої його продукції, чинником розподілу
капіталів і самої праці між різними сферами продукції. Розпадання зиску на
підприємницький бариш і процент виступає як поділ того самого доходу. Але воно
виникає насамперед з розвитку капіталу як вартости, що сама з себе зростає, створює
додаткову вартість, — з розвитку цієї певної суспільної форми панівного
процесу продукції. Воно розвиває з себе кредит і кредитові заклади, і разом з
тим дану форму продукції. У формі проценту і т. ін. позірні форми розподілу
входять до складу ціни як визначальні продукційні моменти.

Щодо земельної ренти, то могло б здатися, що вона є просто розподільча
форма, бо земельна власність, як така, не виконує в самому процесі продукції
жодної, принаймні, жодної нормальної функції. Але та обставина, що 1) рента
обмежується надміром над пересічним зиском, 2) що земельний власник з керівника
і владаря процесу продукції і всього життєвого суспільного процесу зводиться
до ролі звичайного здавача землі в оренду, земельного лихваря, звичайного
одержувача ренти, — ця обставина є специфічний історичний витвір капіталістичного
способу продукції. Те явище, що земля набула форми земельної
власности, є історична передумова цього способу продукції. Та обставина, що
земельна власність набуває форм, які допускають капіталістичний спосіб провадження
сільського господарства, є витвір специфічного характеру цього способу
продукції. Дохід земельного власника і при інших формах суспільства
можна було б назвати рентою. Але він був би істотно відмінний від ренти,
якою вона є при цьому способі продукції.

Отже, так звані розподільчі відносини, виникають з історично певних,
специфічно суспільних форм процесу продукції і тих відносин, в які вступають
між собою люди в процесі репродукції свого людського життя, — виникають з цих
форм процесу продукції і відносин і їм відповідають. Історичний характер цих
відносин розподілу є історичний характер продукційних відносин, і визначають вони
тільки один бік останніх. Капіталістичний розподіл відрізняється від тих форм
розподілу, що постають з інших способів продукції, і кожна форма розподілу зникає
разом з певного формою продукції, що з неї вона постає і якій вона відповідає.

Погляд, що згідно з ним, як історично дані розглядається лише розподільчі
відносини, а не продукційні відносини, є, з одного боку, лише погляд щойно пробудженої,
але ще не визволеної критики буржуазної економії. Але, з другого боку,
він ґрунтується на сплутуванні і ототожнюванні суспільного процесу продукції з
простим процесом праці, що його мусила б чинити і протиприродно ізольована
людина, поза всякою допомогою суспільства. Оскільки процес праці є тільки процес
між людиною і природою, його прості елементи спільні всім формам суспільного
розвитку. Але кожна певна історична форма цього процесу розвиває далі його матеріяльні
основи і його суспільні форми. Досягнувши певного ступеня достиглости,
дана історична форма усувається і поступається місцем вищій формі. Що момент
такої кризи наступив, виявляється, скоро лише суперечність і протилежність між
розподільчими відносинами, а тому й історично певного формою відповідних їм відносин
продукції, з одного боку, і продуктивними силами, продуктивною здібністю
і розвитком її чинників, з другого боку, набуває широти й глибочини. Тоді вибухає
конфлікт між матеріяльним розвитком продукції та її суспільною формою 57).

57) Дивись працю про Competition and Cooperation (1832?)
