виявиться, що в наслідок збільшення маси ренти підвищився і її рівень. Той
самий акр, що давав 2 ф. стерл. ренти, дає тепер 4 ф. стерл.\footnote{
Одна з заслуг Родбертуса, що до його важливої праці про ренту ми вернемося в книзі IV,
є в тому, що він розвинув цей пункт. Але, поперше, він помиляється, припускаючи, ніби для капіталу
зріст зиску завжди виявляється як і зріст капіталу, так що при збільшенні маси зиску відношення
лишається
те саме. А проте, це невірно, бо коли склад капіталу змінюється, норма зиску, не зважаючи на
незмінну експлуатацію праці, може підвищитись саме тому, що відносна вартість сталої частини
капіталу
проти змінної знизилася. — Подруге, він помиляється, трактуючи це відношення грошової ренти до
кількісно певної дільниці землі, наприклад, до одного акра, як щось таке, що взагалі припускає
тисячна
економія в її дослідженнях про підвищення або пониження ренти. Це знов невірно. Вона постійно
розглядає норму ренти у відношенні до продукту, — оскільки вона розглядає ренту в її натуральній
формі, — і у відношенні до авансованого капіталу, — оскільки вона розглядає ренту як грошову ренту,
—
бo це в дійсності є раціональні вирази.
}.

Відношення певної частини додаткової вартости, грошової ренти, — бо гроші
є самостійний вираз вартости, — до землі само по собі є безглузде й іраціональне;
бо це не співмірні величини, що тут виміряються одна одною, певна
споживна вартість, дільниця землі на стільки-от квадратових футів з одного
боку, і вартість, точніше, додаткова вартість — з другого. В дійсності це не виражає
нічого іншого, а тільки те, що в даних відносинах власність на стільки-от квадратових
футів землі дає земельному власникові можливість уловлювати певну кількість
неоплаченої праці, реалізованої капіталом, який риється на цих квадратових футах,
як свиня у картоплі (в рукопису тут стоїть в дужках, але закреслене: Лібіх). Але
prima fасiе цей вираз є те саме, як коли б ми здумали говорити про відношення
п’ятифунтової банкноти до діяметра землі. Однак, до посередництва тих іраціональних
форм, в яких виступають і на практиці резюмуються певні економічні
відносини, практичним носіям цих відносин у їхньому житті-бутті немає
жодного діла; а що вони привикли рухатися в цих посередницьких відносинах,
то їхній розум ані трохи не спотикається на них. Цілковита суперечність для них
не має рішуче нічого таємничого. У формах проявлення, відчужених від внутрішнього
зв’язку і безглуздих, коли їх узяти самих по собі, вони почувають себе
так само вдома, як риба у воді. Тут справедливе те, що Геґель сказав про
відомі математичні формули: те, що звичайний людський розум вважає за
і раціональне, є раціональне, а раціональне для нього є сама і раціональність.

Отже, коли розглядати справу у відношенні до самої площі землі, то підвищення
маси ренти виражається цілком так само, як підвищення норми ренти;
а звідси труднощі, що постають, коли умови, які пояснювали б один випадок,
відсутні в іншому випадку.

Але ціна землі може підвищитись навіть тоді, коли ціна продукту землі
зменшується.

В цьому випадку в наслідок дальшого диференціювання може збільшитися
диференційна рента, а тому й ціна кращих земель. Або ж коли цього немає, то
при збільшеній продуктивній силі праці ціна хліборобського продукту може понизитись,
але так, що це буде більш, ніж урівноважено збільшенням продукції. Припустімо,
що квартер коштував 60 шил. Коли на тім самім акрі при тому самому
капіталі будуть випродуковані 2 квартери замість одного, і квартер понизиться
до 40 шил, то 2 квартери дадуть 80 шил, так що вартість продукту того самого
капіталу на тому самому акрі підвищиться на одну третину, хоч ціна акра
понизилась на одну третину. Як це можливо без того, щоб продукт продавався
вище його ціни продукції або вартости, було показано при дослідженні диференційної
ренти. В дійсності це можливо тільки в два способи. Або гіршу
землю вилучається з конкуренції, але ціна кращої землі зростає, коли
диференційна рента зростає, отже, коли загальне поліпшення діє нерівномірно
на різні роди землі. Або ж на найгіршій землі та сама ціна продукції
(і та сама вартість, коли виплачується абсолютну ренту) в наслідок під-