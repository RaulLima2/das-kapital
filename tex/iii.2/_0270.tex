\parcont{}  %% абзац починається на попередній сторінці
\index{iii2}{0270}  %% посилання на сторінку оригінального видання
вартостетворчих чинників, що реґулюються різними законами та випливають
з різних джерел.

\emph{Почетверте}. Для окремого капіталіста цілком байдуже, продаються чи
ні його товари по їхніх вартостях; отже саме визначення вартости для нього
цілком байдуже. Це визначення становить очевидно щось таке, що відбувається
за його спиною, під впливом незалежних від нього відносин, бо не вартості
товарів, а відмінні від них ціни продукції створюють реґуляційні пересічні
ціни в кожній сфері продукції. Визначення вартости як таке, цікавить і є визначальне
для окремого капіталіста і капіталу в кожній окремій сфері продукції лише
остільки, оскільки зменшена або збільшена кількість праці, що її потрібно з підвищенням
або пониженням продуктивности праці для продукції товарів, дає йому можливість
в одному випадку, при даних ринкових цінах, одержати надзиск, а в другому
випадку примушує його підвищити ціни на товари, бо тепер на кожну одиницю
продукту або на кожен окремий товар припадає більше заробітної плати, більше
сталого капіталу, а тому і більше проценту. Визначення вартости цікавить
його лише остільки, оскільки воно підвищує або понижує для нього самого
витрати продукції товару, отже, остільки, оскільки воно ставить його в виняткове
становище.

Навпаки, заробітна плата, процент і рента здаються йому межами, що
реґулюють не тільки ту ціну, при якій він може реалізувати частину зиску,
підприємницький бариш, що припадає йому як капіталістові, що функціонує,
але такими, що реґулюють взагалі ту ціну, за яку він мусить продавати товари,
щоб міг безупинно тривати процес репродукції. Для нього цілком байдуже, чи
реалізує він, чи ні при продажу вартість і додаткову вартість, що міститься в
товарі, аби тільки він здобував з продажної ціни товару звичайний або більший
ніж звичайний підприємницький бариш, понад витрати продукції, індивідуально
дані для нього величиною заробітної плати, проценту й ренти. Отже, коли лишити
осторонь сталу частину капіталу, то заробітна плата, процент і рента здаються
йому обмежувальними, а тому витворними, визначальними елементами товарової
ціни. Коли йому пощастить, наприклад, понизити заробітну плату нижче вартости
робочої сили, отже, нижче її нормальної висоти, одержати капітал за понижений
процент і виплачувати орендну плату в розмірах, що не досягають
нормального рівня ренти, то для нього цілком байдуже, чи продає він продукт
нижче від його вартости, чи навіть нижче від загальної ціни продукції,
так що частину вміщеної в товарі додаткової праці він віддає даром. Це
має силу і щодо сталої частини капіталу. Коли, наприклад, промисловець
може купити сировий матеріял нижче за його ціну продукції, то це ґарантує
його від втрати навіть в тому випадку, коли він і собі продає цей матеріял
в готовому товарі нижче за ціни продукції. Його підприємницький бариш
може лишатись незмінним і навіть зрости, скоро тільки лишається незмінним
або зростає надмір товарової ціни над тими її елементами, що мусять
бути оплачені, покриті еквівалентом. Але крім вартости засобів продукції, що
входять в продукцію його товарів, як кількісно дані елементи ціни, якраз
заробітна плата, процент і рента є такі обмежувальні і реґуляційні елементи
ціни, що входять в продукцію. Отже, вони здаються йому елементами, що
визначають ціну товарів. З цього погляду здається, що підприємницький бариш
визначається або надміром ринкової ціни, яка залежить від випадкових
відносин конкуренції, над іманентною вартістю товарів, визначуваною вищезгадаченими
елементами ціни; абож, оскільки сам підприємницький бариш справляє
визначальний вплив на ринкові ціни, він сам своєю чергою видається залежним
від конкуренції між продавцями і покупцями.

Так у конкуренції між окремими капіталістами, як і в конкуренції на світовому
ринку, дані і наперед припущені величини заробітної плати, проценту
