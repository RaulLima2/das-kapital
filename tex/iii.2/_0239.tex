\parcont{}  %% абзац починається на попередній сторінці
\index{iii2}{0239}  %% посилання на сторінку оригінального видання
відчужености й самостійности проти праці, отже, та перетворена форма умов
праці, що в ній випродуковані засоби продукції пертворюються на капітал, а
земля на монополізовану землю, на земельну власність, — ця належна певному
історичному періодові форма збігається, отже, з буттям і функцією випродукованих
засобів продукції і землі в процесі продукції взагалі. Ці засоби продукції
є капіталом сами по собі, з природи; капітал є не що інше, як просто «економічний
термін» для цих засобів продукції; і оттак земля є сама по собі, з
природи, землею, монополізованою певного кількістю земельних власників. Як
у капіталі і в капіталістові, — який на ділі є не що інше, як персоніфікований
капітал, — продукти стають самостійною силою проти продуцентів, так і в земельному
власникові персоніфікується земля, яка теж стає дибки, і як самостійна
сила вимагає своєї частини у випродукованому з її допомогою продукті;
так що не земля одержує належну їй частину продукту для відновлення
і підвищення продуктивности, а замість неї земельний власник одержує частину
цього продукту на прогулювання і марнотратство. Ясно, що капітал має
своєю предумовою працю як найману працю. Але так само ясно, що коли виходити
з праці як найманої праці, так що тотожність праці взагалі з найманою
працею уявляється самоочевидною, то тоді капітал і монополізована земля
в свою чергу мусять уявлятись природною формою умов праці проти праці
взагалі. Бути капіталом, — це уявляється тепер природною формою засобів праці,
а тому й як їх суто-речовий характер, що виникає з їхньої функції
в процесі праці взагалі. Таким чином, капітал і випродукований засіб продукції
стають тотожніми виразами. Так само земля і монополізована приватною власністю
земля стають тотожніми виразами. Тому за джерело зиску стають засоби
праці як такі, як капітал з природи, так само як за джерело ренти стає земля
як така.

Працю як таку, в її простій визначеності доцільної продуктивної діяльности,
ставиться в відношення до засобів продукції, взятим не в їхній суспільній
визначеності форми, а в їхній речевій субстанції як матеріялу і засобів праці,
що відрізняються між собою теж лише речево, як споживні вартості: земля —
як невипродукований засіб праці, інші — як випродуковані засоби праці. Отже,
коли праця збігається з найманою працею, то та певна суспільна форма, в
якій умови праці протистоять тепер праці, також збігається з їхнім речевим
буттям. Тоді засоби праці як такі є капітал, і земля як така є земельна
власність. Тоді формальне усамостійнення цих умов праці, проти праці і та
особлива форма цього усамостійнення, яку вони мають проти найманої праці, видається
властивістю невідійманною від них як від речей, як від матеріяльних умов
продукції, видається властивістю, що неодмінно належить та іманентно зрослася
з ними як з елементами продукції. Їхній визначуваний певною історичною
епохою певний соціяльний характер в капіталістичному процесі продукції видається
їхнім речовим характером, природно і так би мовити споконвіку природженим
їм, як елементам процесу продукції. Тому відповідна участь, яку земля
як первісна сфера діяльности праці, як царство природних сил, як наявний
арсенал усіх речей праці, і та друга відповідна участь, яку випродуковані засоби
продукції (знаряддя, сирові матеріяли тощо) беруть у процесі продукції
взагалі — мусять тоді здаватись участю, що знаходить собі вираз у відповідних
частинах, які в формі зиску (проценту) і ренти дістаються їм як капіталові і
земельній власності, тобто їхнім соціальним представникам, як для робітника
мусить здаватись, що та участь, яку його праця бере в процесі продукції,
виражається в заробітній платі. Таким чином, здається, що рента, зиск, заробітна
плата породжуються тією ролею, яку земля, випродуковані засоби продукції й
праця відіграють у простому процесі праці, навіть і тоді, коли б ми розглядали
цей процес праці як процес просто між людиною й природою і абстрагувались
\parbreak{}  %% абзац продовжується на наступній сторінці
