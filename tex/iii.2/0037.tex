Зі зростом матеріяльного багатства зростає кляса грошових капіталістів;
з одного боку, більшає число й багатство капіталістів, що відходять від справ, —
рантьє; з другого боку, складаються сприятливі обставини для розвитку кредитової
системи, а разом з цим більшає число банкірів, грошових позикодавців, фінансистів
і т. ін. — З розвитком вільного грошового капіталу збільшується кількість
процентодайних паперів, державних паперів, акцій тощо, як це ми вже раніше
розвинули. Але одночасно з цим зростає попит на вільний грошовий капітал, при
чому Jobbers (маклери), що спекулюють тими паперами, відіграють головну ролю на
грошовому ринку. Коли б усі купівлі та продажі цих паперів означали лише
дійсне приміщення капіталу, то мали б рацію сказати, що вони не можуть
впливати на попит позичкового капіталу, бо, коли А продає свій папір, він
забирає саме стільки грошей, скільки B приміщує в той папір. Тимчасом навіть
тоді, коли папір хоч і існує, але немає того капіталу (принаймні, як грошового
капіталу), що його той папір первісно представляв, — навіть тоді він, цей папір,
завжди породжує pro tanto попит на такий грошовий капітал. Але в усякому
разі це — той грошовий капітал, що ним порядкував спочатку В, а тепер
порядкує А.

В. А. 1857. № 4886: «Чи на вашу думку слушно зазначено причини,
що визначають норму дисконту, коли я кажу, що її регулюється масою капіталу
на ринку, уживаного для дисконту торговельних векселів, у відміну від
інших родів цінних паперів? — [Chapman:] Ні; я тримаюсь тієї думки, що на
рівень проценту впливають усі ті цінні папери, які легко перетворюються
на гроші (all convertible securities of a current character); було б неслушно
обмежувати це питання лише на дисконті векселів; бо, коли є великий попит
на гроші під [заставу] консолів або навіть посвідок державної скарбниці — як
це нещодавно дуже часто траплялося — та ще й за процент, куди вищий за
торговельний процент, то було б абсурдом казати, що це не зачіпає нашого торговельного
світу; це зачіпає його дуже й дуже. — 4890. Коли на ринку є добрі
та ходові цінні папери, що їх банкіри визнають за такі, і коли власники
хочуть узяти в позику гроші під ці папери, то, звичайно, це матиме свій вплив на
торговельні векселі; я не можу, напр., сподіватися, що якась особа дасть мені
свої гроші під торговельний вексель за 5%, коли вона того самого часу може
визичити їх за 6% під консолі і т. ін.; таким самим способом це впливатиме й на
нас; ніхто не може від мене вимагати, щоб я дисконтував його векселі за 5 1/2%,
коли я маю змогу визичити свої гроші за 6% — 4892. Про людей, що на 2000
ф. ст., або на 5000 ф. ст., або ж на 10.000 ф. ст. купують цінні папери, вважаючи
їх за добре приміщення капіталу, ми не кажемо, що вони значно впливають
на грошовий ринок. Коли ви питаєте мене про рівень проценту під [заставу]
консолів, то я кажу про людей, що роблять операції на сотні тисяч, про так
званих Jobbers’iв, що підписують або купують на ринку громадські позики на
великі суми, а потім мусять тримати ці папери, поки матимуть змогу збути їх
з зиском; ці люди мусять позичати для цього гроші».

З розвитком кредитової справи утворюються великі концентровані грошові
ринки, як от Лондон, що одночасно є головний центр торговлі цими паперами.
Банкіри дають банді цих торговців до розпорядку маси грошового капіталу публіки,
і так зростає це кодло цих грачів. «На фондовій біржі гроші звичайно

чено далі, — то йому платять не його власними грішми, а грішми, що їх склав хтось інший. Коли він
платить В борг чеком на свого банкіра, і В складає цей чек вкладом у свого банкіра, а банкір того
вкладника А має теж чек на банкіра вкладника В, так що обидва банкіри тільки вимінюють ці чеки,
то гроші, складені А, двічі виконали грошову функцію; поперше, в руках того, хто одержав гроші,
складені
А; по-друге, в руках самого А. В другій функції це є вирівнюванння боргових вимог (боргова вимога А
до
свого банкіра та боргова вимога останнього до банкіра В) без посередництва грошей. Тут вклад діє
двічі як гроші, а саме одного разу як дійсні гроші, а потім як вимога на гроші. Просто вимоги на
гроші можуть заступати місце грошей лише через вирівнювання боргових вимог.
