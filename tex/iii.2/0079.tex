час та при тій нормі дисконту було вигідно відправляти золото з Петербургу до
Лондону та визичати його там за 8%, аж до реченця тримісячних векселів, видаваних
під те продане золото. — 2573. При всяких операціях з золотом, треба
брати на увагу чимало пунктів; треба зважати на вексельний курс та на рівень
проценту, що за нього можна примістити гроші аж до реченця [виданого під
золото] векселя».

Вексельний курс з Азією

Далі наведені пункти важливі, з одного боку, тому, що вони виявляють,
як Англії — коли вона має несприятливий для себе вексельний курс з Азією —
доводиться надолужувати на інших країнах, що їхній імпорт з Азії оплачується
за посередництвом Англії. А з другого боку, тому, що пан Вілсон знову робить
тут дурну спробу ідентифікувати вплив вивозу благородних металів на вексельний
курс з впливом вивозу капіталу взагалі на цей курс; при чому в обох
випадках справа йде про експорт не як платіжного або купівного засобу, а для
приміщення капіталу. Передусім само собою зрозуміла річ, що хоч посилатимуться
до Індії для приміщення в її залізницях стільки і стільки мільйонів ф. ст. благородним
металом, хоч залізними шинами — і одне і друге становлять лише
різну форму перенесення тієї самої капітальної суми з однієї країни до іншої;
і то саме таке перенесення, що не увіходить в обрахунок звичайних торговельних
операцій, перенесення, що при ньому країна, яка експортує, не сподівається
ніякого іншого зворотного припливу, а тільки пізнішого річного доходу з виручки
цих залізниць. Якщо цей експорт відбувається у формі благородного металу,
то — саме тому, що це благородний метал і як такий безпосередньо позичковий
грошовий капітал та база всієї грошової системи — він необов’язково в
усяких обставинах, але тільки в раніше розвинутих безпосередньо впливатиме
на грошовий ринок, отже, й на рівень проценту країни, що той благородний
метал експортує. Так само безпосередньо впливатиме він на вексельний курс.
Благородний метал відправляють саме лише тому й лише остільки, що й оскільки
векселів, напр., на Індію, що їх постачають на лондонському грошовому ринку,
не вистачає на те, щоб робити такі екстра-рімеси. Отже, постає такий попит
на векселі на Індію, що переважає подання їх, і таким чином в даний
момент курс обертається проти Англії не тому, що вона заборгувалась Індії, а
тому, що має слати до Індії екстра-ординарні суми. Коли така відправа благородного
металу до Індії триває протягом довгого часу, то вона мусить впливати
в тому напрямі, що індійський попит на англійські товари більшатиме, бо ця
відправа посередньо підносить споживчу здатність Індії до європейських товарів.
А коли, навпаки, капітал відправляють у формі шин і т. ін., то це зовсім не
може мати жодного впливу на вексельний курс, бо Індія не повинна нічого за
ті шини платити. З тієї самої причини, можливо, що це не матиме жодного
впливу на грошовий ринок. Вілсон силкується довести наявність такого впливу
тим способом, що, мовляв, такі екстра-витрати зумовлюють надзвичайний попит
на грошовий кредит і тому впливатимуть на рівень проценту. Це може траплятися;
але запевняти, що це мусить відбуватися в усяких обставинах, цілком
помилково. Хоч і куди відправлятимуть ті шини, хоч і на яку колію їх кластимуть,
чи то на англійській, чи на індійській землі, однаково вони становлять
не що інше, як тільки певний пошир англійської продукції в певній сфері. Запевняти,
що пошир продукції, навіть у дуже широких межах, не може відбутись
без піднесення рівня проценту, — це дурниця. Грошові позики можуть зростати,
тобто можуть більшати суми тих операцій, що в них увіходять кредитові
операції; але ці останні операції можуть збільшуватися при даному незмінному
рівні проценту. Це дійсно було в Англії 40-ми роками підчас залізничої манії.
