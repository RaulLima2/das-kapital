Рівень проценту не підвищився. І очевидно, що — оскільки братимемо на увагу
дійсний капітал, тобто тут товари — вплив на грошовий ринок є той самий,
все одно, чи ці товари призначено для закордону, чи для внутрішнього споживання.
Ріжниця була б тільки тоді, коли б приміщення англійського капіталу
закордоном впливали на комерційний експорт Англії, обмежуючи його, — експорт,
що його доводиться оплачувати, що, отже, тягне за собою зворотний приплив
капіталу, — або тільки тоді, коли б ці приміщення капіталу взагалі були вже
симптомом надмірного напруження кредиту та початку спекулятивних операцій.

Далі питає Вілсон, а відповідає Newmarch.

«1786. Раніше ви казали про попит на срібло для Східньої Азії, що, на
вашу думку, вексельні курси з Індією сприятливі для Англії, дарма що до Східньої
Азії невпинно відправлялось чималі металеві скарби; маєте підстави до цього? —
Звичайно... Я гадаю, що дійсна вартість вивозу Сполученого Королівства до
Індії становила в 1851 році 7.420.000 ф. ст.; до цього треба додати суму
векселів India House, тобто суму тих фондів, що їх витягає Ост-індська компанія
з Індії для покриття своїх власних видатків. Ці тратти становили в тім році
3.200.000 ф. ст.; так що цілий вивіз Сполученого Королівства до Індії становив
10.620.000 ф. ст. В 1855 році дійсна вартість товарового експорту піднеслася
до 10.350.000 ф. ст.; тратти India House становили 3.700.000 ф. ст.;
отже, цілий вивіз 14.050.000 ф. ст. Для 1851 року, думається мені, в нас немає
ніякого засобу визначити дійсну вартість довозу товарів з Індії до Англії; але
для 1854 та 1855 років ми можемо це зробити. В 1855 році дійсна вартість
цілого довозу товарів з Індії до Англії становила 12.670.000 ф. ст., і сума ця —
коли її порівняти з отими 14.050.000 ф. ст., лишає сприятливий для Англії
балянс в безпосередній торговлі між обома країнами на 1.380.000 ф. ст.».

На це Вілсон зауважує, що на вексельні курси впливає й посередня торговля.
Напр., вивіз з Індії до Австралії й Північної Америки покривається траттами
на Лондон і тому він впливає на вексельний курс цілком так само, як
коли б товари йшли безпосередньо з Індії до Англії. Далі, коли взяти Індію та
Китай разом, то балянс буде несприятливий для Англії, бо Китай раз-у-раз
має робити значні платежі Індії за опій, а Англія — Китаєві, й, таким чином, цим
кружним шляхом суми з Англії йдуть до Індії (1787, 88.).

1789. Тепер Вілсон запитує, чи не буде вплив на вексельний курс той
самий, всеодно, чи капітал «йтиме закордон у формі залізничих шин та локомотивів,
чи в формі металевих грошей». На це Newmarch відповів цілком слушно:
12 міл. ф. ст., останніми роками відправлені до Індії для будування залізниць,
послужили для купівлі річної ренти, що її Індія має платити Англії у певні
реченці. «Якщо мати на увазі безпосередній вплив на ринок благородного металу,
то приміщення тих 12 міл. ф. ст. може чинити такий вплив лише остільки,
оскільки доводилося відправляти метал для дійсного приміщення його у формі
грошей».

1797. [Weguelin питає:] «Коли не відбувається жодного зворотного припливу
за це залізо (шини), то як можна казати, що це впливає на вексельний курс? —
Я не думаю, що частина витрати, відправлена закордон у формі товарів, впливала
на стан вексельного курсу... на стан курсу між двома країнами, можна
сказати, впливає виключно тільки кількість зобов’язань або векселів, що їх
подають в одній країні, проти тієї кількости, що її подають у другій країні; така
є раціональна теорія вексельного курсу. Щождо відправи тих 12 мільйонів, то
їх передусім підписано тут; коли б ця операція була такого роду, що всі ці
12 міл. було б складено металевими грішми в Калькуті, Бомбеї та Мадрасі...
то цей раптовий попит надзвичайно вплинув би на ціну срібла й на вексельний
курс, цілком так само, як коли б Ост-індська компанія оповістила завтра, що
вона свої тратти збільшить від 3 до 12 міл. Але половину цих 12 міл. витра-
