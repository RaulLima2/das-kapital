провадить сільське господарство, тільки як осібне поле експлуатації капіталу, як
приміщення свого капіталу в осібній сфері продукції. Цей фармер-капіталіст у
певні реченці, напр., щороку, платить земельному власникові, власникові визискуваної
ним землі, певну, контрактом усталену грошову суму (цілком так, як позикоємець
грошового капіталу платить певний процент) за дозвіл уживати свій капітал
на цьому осібному полі продукції. Ця грошова сума зветься земельною рентою,
все одно, чи платиться її від орної землі, будівельної ділянки, копалень, рибальства,
лісів і т. ін. Її платиться протягом всього часу, що на нього земельний
власник за контрактом визичив, винайняв землю орендареві. Отже, земельна
рента становить тут ту економічну форму, що в ній земельна власність економічно
реалізується, даючи вартість. Далі, ми маємо тут усі три кляси — найманого
робітника, промислового капіталіста, земельного власника, — що всі разом
та одна проти однієї являють кістяк новітнього суспільства.

Капітал може бути зафіксований в землі, долучений до неї, почасти
більше тимчасово, як от при поліпшеннях хемічної натури, удобреннях і т. ін.,
почасти більше постійно, як от при дренажі, зрошувальних спорудах, нівелюваннях,
господарчих будівлях і т. ін. В іншому місці я назвав капітал, що
отак долучається до землі, la terre-capital\footnote{
Misère de la Philosophie р. 165. Там я розрізняю terre-matière і terre-capital. «Досить лише
примістити до ділянок землі, вже перетворених на засоби продукції, нові суми капіталу, — і ми
збільшуємо la terre-capital, ані трохи не збільшуючи la terre-matière, тобто простору землі... La
terre-capital так само не є вічний, як і всякий інший капітал. t. La terre-capital e основний
капітал, але основний капітал зужитковується так само, яві оборотні капітали».
}. Він належить до категорії основного
капіталу. Процент за долучений до землі капітал та за поліпшення, що їх вона
тим способом одержує, як знаряддя продукції, може становити частину тієї
ренти, що її платить фармер земельному власникові\footnote{
Я кажу «може», бо в певних обставинах цей процент регулюється законом земельної ренти, а тому як
от при конкуренції нових земель, що мають велику природну родючість, може зникнути.
}, однак ця частина не
являє собою власне земельної ренти, що її платиться за користування землею
як такою, однаково, чи перебуває та земля в природному стані, — чи її культивується.
Коли б ми систематично — що до нашого плану не належить — розглядали
земельну власність, то треба було б докладно з’ясувати цю частину
доходу земельного власника. Тут досить буде сказати кілька слів про це.
Капіталовкладання більш тимчасового характеру, що їх викликають звичайні
процеси продукції в хліборобстві, всі без винятку переводить фармер. Ці вкладання,
як і простий обробіток землі взагалі, коли його проводять до певної
міри раціонально, отже коли він не сходить до брутального виснажування
ґрунту, як от, напр., в колишніх американських рабовласників, — проти чого
однак панове земельні власники забезпечують себе в контракті, — ці вкладання
поліпшують ґрунт\footnote{
Дивись James Anderson і Сагеу.
}, збільшують кількість продуктів землі та перетворюють
землю з простої матерії на землю — капітал. Оброблене поле більш варте, ніж
необроблене тієї самої природної якости. І основні капітали, що долучені до
землі на довший час, зужитковуються протягом довгого часу, витрачає, здебільшого,
фармер, а в деяких сферах іноді тільки сам фармер. Коли ж усталений в
договорі час оренди мине — і це одна з причин, чому з розвитком капіталістичного
способу продукції земельний власник силкується по змозі дужче скоротити
час оренди, — то пороблені в землі поліпшення як приналежність невідійманна
від субстанції, від землі, припадають як власність власникові тієї землі.
Роблячи новий орендний контракт, земельний власник додає до власне земельної
ренти процент на капітал, долучений до землі; однаково, чи винаймає він землю
тому фармерові, що ті поліпшення поробив, — чи якомусь іншому фармерові.
Таким чином його рента бубнявіє; або, коли він хоче продати землю — ми далі