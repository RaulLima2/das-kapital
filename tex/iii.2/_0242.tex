\parcont{}  %% абзац починається на попередній сторінці
\index{iii2}{0242}  %% посилання на сторінку оригінального видання
виключних умов, як здається, зумовлює лише відхили від пересічного зиску,
а не самий зиск. Розпад зисків на підприємницький бариш і процент (не кажучи
вже про втручання комерційного зиску і зиску від торгівлі грішми, що, як
здається, ґрунтуються на циркуляції і цілком виникають з неї, а не з процесу
самої продукції) завершує усамостійнення форми додаткової вартости, скостеніння
її форми проти її субстанції, її істоти. Одна частина зиску, протилежно до
другої, цілком відривається від капіталістичного відношення як такого, і має
такий вигляд, ніби вона виникає не з функції експлуатації найманої праці,
а з найманої праці самого капіталіста. У протилежність цьому процент має
такий вигляд, ніби він виникає, незалежно так від найманої праці робітника,
як і від власної праці капіталіста, а з капіталу як свого власного, незалежного
джерела. Коли первісно, на поверхні циркуляції, капітал здавався капіталом-фетишем,
вартістю, що породжує вартість, то тепер він знову виступає в вигляді
капіталу, що дає процент як у своїй найдальшій від дійсних відносин і
найсвоєріднішій формі. А тому форма: «капітал-процент» як третій член до
«земля-рента» і «праця-заробітна плата», геть послідовніший, ніж «капітал-зиск»,
бо в зиску все ще зберігається спогад про його походження, який у проценті
не тільки згладжений, але й фіксований у формі, протилежній цьому
походженню.

Нарешті, поряд з капіталом, як самостійне джерело додаткової вартости,
виступає земельна власність, яка обмежує пересічний зиск і передає частину
додаткової вартости такій клясі, що сама ані працює, ані безпосередньо визискує
робітників, і що не може, як процентодайний капітал, вдаватися в морально-напутливі
втішні міркування, наприклад, про риск і жертви при позиченні
капіталу. А що тут частина додаткової вартости, як здається, зв’язана безпосередньо
не з суспільними відносинами, а з природним елементом — землею, то цим
завершується форма відчуження і скостеніння різних частин додаткової вартости
однієї проти однієї, внутрішній зв’язок остаточно розривається, і її джерела
цілком закриваються саме усамостійненням відносин продукції одних проти одних,
що їх тепер зв’язується з різними речевими елементами процесу продукції.

В формулі: капітал-зиск, або, ще краще, капітал-процент, земля-земельна
рента, праця-заробітна плата, в цій економічній триєдиності як зв’язку
складових частин вартости і багатства взагалі з його джерелами, завершується
містифікація капіталістичного способу продукції, зрічевлення суспільних відносин,
безпосереднє зрощення речевих відносин продукції з їхньою історично-суспільною
визначеністю: заворожений, перекручений і на голову поставлений
світ, в якому monsieur le Capital та madame la Terre\footnote*{
Monsieur, le Capital et madame la Terre — (франц.) пан Капітал та пані Земля. \emph{Пр. Ред.}
} як соціяльні характери
і одночасно безпосередньо як просто речі зчиняють свій шабаш. Велика заслуга
клясичної економії є в тому, що вона зруйнувала цю фалшиву видимість
та ілюзію, це усамостійнення і скостеніння різних суспільних елементів
багатства одного проти одного, цю персоніфікацію речей і зрічевлення
відносин продукції, цю релігію повсякденного життя, — зруйнувала тим, що вона
звела процент до частини зиску і ренту до надміру над пересічним зиском, так
що обидві сполучаються в додатковій вартості; тим, що вона подала процес
циркуляції як просту метаморфозу форм і, нарешті, в безпосередньому процесі
продукції звела вартість і додаткову вартість товарів до праці. Проте, навіть
найкращі з її представників, — та інакше воно й бути не може за буржуазного
погляду, — в більшій чи меншій мірі лишаються захоплені тим світом позірної
видимости, що його вони критично розв’язали, і тому всі вони в більшій або меншій
мірі впадають у непослідовність, половинчатість і нерозв’язні суперечності.
Не менш природно, з другого боку, що дійсні аґенти продукції почувають себе
\parbreak{}  %% абзац продовжується на наступній сторінці
