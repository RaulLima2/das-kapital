буде вирівнювати або загострювати ріжниці, диференційна рента з кращих земель,
а разом з тим і загальна сума ренти знизиться або підвищиться, як це було
вже в випадку з диференційною рентою І. В решті, це залежить від величини земельної
площі й капіталу, вилучених разом з А, і від відносного розміру авансованого
капіталу, потрібного за висхідної продуктивности для того, щоб дати
додаткову кількість продукту для покриття попиту.

Єдиний пункт, на дослідженні якого тут варто спинитися, і який взагалі
вертає нас до дослідження того, як цей диференційний зиск перетворюється
на диференційну ренту, є такий:

У першому випадку, коли ціна продукції лишається та сама, додатковий
капітал, вкладений в землю А, не справляє впливу на диференційну ренту, як
таку, бо земля А, як і давніш, не дає ренти, ціна її продукту лишається та
сама, і продовжує реґулювати ринок.

У другому випадку, варіянт І, коли ціна продукції за незмінної норми продуктивности
понижується, земля А неодмінно відпадає, і ще в більшій мірі це
відбувається у варіянті II (низхідна ціна продукції за низхідної норми продуктивности),
бо в противному разі додатковий капітал, вкладений у землю А,
мусив би підвищити ціну продукції. Але тут, у варіянті III другого випадку,
коли ціна продукції понижується, бо продуктивність додаткового капіталу підвищується,
цей додатковий капітал за певних умов може бути вкладений так
в землю А, як і в землі кращої якости.

Припустімо, що додатковий капітал в 2 1/2 ф. стерл., вкладений в землю
А, продукує 1 1/5 кварт. замість 1 квартера.

Таблиця VI.
(таблиця надіслана окремо)

Цю таблицю слід порівняти, крім основної таблиці І, і з таблицею II, в якій
подвоєне вкладення капіталу сполучається з сталою продутивністю, пропорційною
капіталовкладенню.

Згідно з припущенням, регуляційна ціна продукції понижується. Коли б
вона залишалася сталою, 3 ф. стерл., то найгірша земля А, що давніш, при
капіталовкладенні лише в 2 1/2 ф. стерл., не давала ренти, тепер почала б давати
ренту, хоч ніякої нової найгіршої землі не було б притягнено до оброблення;
це сталося б саме в наслідок того, що продуктивність на ній збільшилася б, але
лише для частини капіталу, а не для первісно вкладеного капіталу. Перші 3 ф.
стерл. ціни продукції дають 1 квартер; другі — 1 1/5 квартера; але ввесь продукт в
2 1/5 квартери продається тепер по його пересічній ціні. А що норма продуктивности
зростає з додатковим капіталовкладенням, то це включає й поліпшення.

*) Тут пересічну норму надзиску обчислено не до всього вкладеного капіталу, а тільки до капіталу,
вкладеного в рентодайні дільниці В, С і D. Прим. Ред.
