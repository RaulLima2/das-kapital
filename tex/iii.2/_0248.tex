\parcont{}  %% абзац починається на попередній сторінці
\index{iii2}{0248}  %% посилання на сторінку оригінального видання
лишаючи осторонь акумуляцію, то сума вартости заробітної плати, зиску й ренти
кляси I мусить дорівнювати вартості сталої частини капіталу кляси II. Інакше,
або у кляси II не буде можливости покрити свій сталий капітал, або у кляси І —
можливости перетворити свій дохід з неспоживної на споживну форму.

Отже, вартість річного товарового продукту, цілком так само як і вартість
товарового продукту окремого капіталовкладення, так само як і вартість кожного
поодинокого товару, розпадається на дві складові частини вартости: на одну А, що
покриває вартість авансованого сталого капіталу, і другу В, що виступає в формі
доходу, як заробітна плата, зиск і рента. Остання складова частина вартости, В,
становить протилежність першої, А, оскільки ця остання, А, за інших рівних
обставин, 1) ніколи не набуває форми доходу, 2) постійно повертається в формі
капіталу, а саме в формі сталого капіталу. Проте, друга складова частина, В,
в свою чергу, в собі самій має протилежність. Зиск і рента мають те спільне з заробітною
платою, що всі вони становлять три форми доходу. Не зважаючи на це,
вони істотно відрізняються між себе тим, що в зиску й ренті репрезентується
додаткова вартість, отже, неоплачена праця, а в заробітній платі — оплачена
праця. Частина вартости продукту, що репрезентує витрачену заробітну плату,
яка, отже, покриває заробітну плату, і при нашому припущенні, що згідно з ним
репродукція відбувається в тому самому маштабі і в тих самих умовах, знову
перетворюється на заробітну плату, ця частина припливає назад насамперед
як змінний капітал, як складова частина капіталу, який знову повинен бути
авансований на репродукцію. Ця складова частина функціонує подвійно. Вона
існує спочатку в формі капіталу і обмінюється як такий на робочу силу. В руках
робітника вона перетворюється на дохід, що його робітник здобуває з продажу
своєї робочої сили, обмінюється як дохід на засоби існування і споживається.
Цей подвійний процес виявляється за посередництвом грошевої циркуляції. Змінний
капітал авансується в формі грошей, виплачується у вигляді заробітної плати.
Це — його перша функція як капіталу. Він обмінюється на робочу силу і перетворюється
в вияв цієї робочої сили, в працю. Це — процес у його відношенні до
капіталіста. Але, подруге, на ці гроші робітники купують частину свого товарового
продукту, вимірювану цими грішми, і споживають її як дохід. Коли ми
абстрагуємось від грошевої циркуляції, то частина продукту робітника перебуває
в руках капіталіста у формі наявного капіталу. Цю частину він авансує як
капітал, дає її робітникові за нову робочу силу, тимчасом як робітник споживає
її як дохід безпосередньо, або за посередництвом обміну її на інші товари.
Отже, частина вартости продукту, призначена при репродукції до того, щоб перетворитись
на заробітну плату, на дохід для робітників, спочатку припливає
назад в руки капіталіста у формі капіталу, точніш, змінного капіталу. Те, що
вона припливає назад у цій формі, є істотна умова постійно відновлюваної
репродукції праці як найманої праці, засобів продукції як капіталу, і самого
процесу продукції як капіталістичного процесу продукції.

Коли ми не хочемо створювати собі марних труднощів, то ми мусимо відрізняти
гуртовий здобуток і чистий здобуток від гуртового доходу і чистого
доходу.

Гуртовий здобуток або гуртовий продукт це — ввесь репродукований продукт.
За вирахуванням застосованої, але не спожитої частини основного капіталу,
вартість гуртового здобутку або брутто-продукту дорівнює вартості
авансованого та зужиткованого у продукції капіталу, сталого і змінного, плюс
додаткова вартість, яка розпадається на зиск і ренту. Або, коли розглядати
продукт не поодинокого капіталу, а сукупного суспільного капіталу, то гуртовий
здобуток дорівнює речовим елементам, що становлять сталий і змінний
капітали, плюс речеві елементи додаткового продукту, в яких репрезентовані
зиск і рента.
