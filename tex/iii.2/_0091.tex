\parcont{}  %% абзац починається на попередній сторінці
\index{iii2}{0091}  %% посилання на сторінку оригінального видання
Lectures on Pol. Econ. London 1851, p. 44). Він недобачає того, що тут маємо
ріжнпцю двох суспільних способів продукції та відповідних кожному з них
суспільних ладів, і що протилежністю між багатством та убозтвом того питання
не розв’язати. На ділі ж лихвар, що висисає убогого дрібного продуцента, йде
попліч того лихваря, що висисає багатих великих земельних власників. Коли
лихварство римських патриціїв цілком зруйнувало римських плебеїв, дрібних
селян, то прийшов кінець цій формі визиску, і місце дрібно-селянського господарства
заступило господарство суто-невільницьке.

В формі проценту лихвар може поглинути тут увесь надмір (що пізніше
виявляється як зиск і земельна рента) понад найпотрібніші засоби існування
продуцента (що пізніше-становить суму його заробітної плати), й тому надзвизвичайно
безглуздо порівнювати висоту \emph{цього} проценту там, де лихвар за винятком
того, що припадає державі, присвоює собі \emph{всю} додаткову вартість, з висотою
новітнього рівня проценту, коли процент, принаймні, нормальний процент, є лише
частина цієї додаткової вартости. При цьому забувають, що найманий робітник
продукує й віддає капіталістові, який уживає його, зиск, процент і земельну
ренту, коротко, цілу додаткову вартість. Кері робить таке безглузде порівняння,
щоб тим показати, оскільки вигідний для робітників розвиток капіталу та спад
рівня проценту, що відбувається поряд того розвитку. Далі, коли лихвар, не
задовольняючись видушуванням додаткової праці з своєї жертви, поволі набуває
собі титул власности на сами умови її праці, землю, хату і т. ін., та постійно
її отак експропріює, то, заначаючи це, знову забувають, що ця цілковита експропріація
умов праці в робітника є не результат, до якого прагне капіталістичний
спосіб продукції, а готова передумова, що з неї той спосіб виходить. Наймит-невільник
так само, як і дійсний невільник з причини свого стану ніяк не може
зробитися невільником позикодавця, принаймні, не може, як продуцент; він може
стати ним хіба лише як споживач. Лихварський капітал в тій формі, що в ній
він дійсно присвоює собі всю додаткову працю безпосередніх продуцентів, не
змінюючи способу продукції; що в ній власність — або володіння — продуцентів
на умови праці та відповідна тій формі розрізнена невелика продукція становлять
головну його передумову; що в ній, отже, капітал не підбиває праці собі безпосередньо,
а тому й не протистоїть їй як промисловий капітал, — цей лихварський
капітал зубожує цей спосіб продукції, паралізує продуктивні сили
замість розвивати їх, та одночасно увічнює ці мізерні відносини, що серед них
суспільна продуктивність праці не розвивається коштом самої праці, як то є за
капіталістичної продукції.

Отож, лихварство, з одного боку, впливає на античне й февдальне багатство
та на античну й февдальну власність, підтинаючи та руйнуючи їх. З другого
боку, воно підтинає та руйнує дрібно-селянську та дрібно-буржуазну
продукцію, коротко — усі ті форми, що в них продуцент ще є власник своїх
засобів продукції. За розвинутого капіталістичного способу продукції робітник
не є власник своїх умов продукції, поля, що він його обробляє, сировини, що
її він перероблює і т. ін. Алє цьому відчуженню умов продукції від продуцента
тут відповідає дійсний переворот в самому способі продукції. Роз’єднані
робітники об’єднуються у великій робітні до хоч і різподіленої, але взаємно
переплетеної діяльности; знаряддя праці стає машиною. Сам спосіб продукції не
дозволяє вже цього роздрібнення знарядь продукції, сполученого з дрібною власністю,
а так само й ізолювання самого робітника. За капіталістичного способу
продукції лихвар вже не в стані відокремити умови продукції від продуцента, бо
вони вже відокремлені від нього.

Лихварство централізує грошове майно там, де засоби продукції роздрібнені.
Воно не зміняє способу продукції, але міцно присисається до нього, як
паразит, цілком виснажаючи його. Воно висисає його, позбавлює його нервів та
\parbreak{}  %% абзац продовжується на наступній сторінці
