проценту і т. ін., щоб забезпечити умови цього перетворення. Це може бути
доведено до більш-менш загостреного становища помилковим законодавством,
заснованим на помилкових теоріях грошей, та накинутими нації інтересами
торговців грішми, Оверстонів і компанії. Але основу цих кредитових грошей
дано основою самого способу продукції. Знецінення кредитових грошей
(зовсім не кажучи вже про втрату ними властивостей грошей, втрату, однак, лише
уявлювану) розхитало б усі існуючі відносини. Тому вартість товарів віддається
в жертву на те, щоб забезпечити фантастичне та самостійне існування цієї вартости
в грошах. Як грошова вартість вона забезпечена взагалі тільки доти, доки
гроші забезпечені. Отже задля кількох мільйонів грошей доводиться віддати в жертву
багато мільйонів товарів. Це є річ неминуча за капіталістичної продукції та становить
одну з її прикрас. За попередніх способів продукції цього не трапляється, бо
при тій вузькій базі, що на ній вони рухаються, ані кредит, ані кредитові гроші не
розвиваються. Поки суспільний характер праці виявляється як грошова форма
буття товарів, а тому — як річ поза дійсною продукцією, доти неминучі грошові
кризи, незалежно від дійсних криз або як загострення їх. З другого
боку, очевидно, що, поки не розхитано кредит певного банку, цей банк в таких
випадках, збільшуючи число кредитових грошей, зменшує паніку, а, зменшуючи
кількість цих грошей, збільшує ту паніку. Вся історія новітньої промисловости
свідчить, що, коли б продукція в країні була організована, то металь був би
потрібен в дійсності тільки для вирівнювання розрахунків по міжнародній торговлі,
скоро рівновагу її за даної хвилини порушено. Що всередині країни вже тепер
не потрібно металевих грошей, показує припинення платежів готівкою так званими
національними банками, припинення, до якого як до одинокого порятунку удаються
в усіх крайніх випадках.

Було б смішно сказати про двох індивідів, що вони у відносинах між собою
мають один проти одного платіжний балянс. Якщо вони обоє навзаєм є
винуватець і кредитор, то очевидно, що, коли їхні вимоги не компенсуються,
лише один мусить бути щодо решти винуватцем другого. Щодо нації цього
ніяк не буває. І те, що цього не буває, всі економісти визнають в тій тезі,
що платіжний балянс може бути сприятливий або несприятливий для певної
нації, дарма що її торговельний балянс, кінець-кінцем, мусить вирівнятися.
Платіжний балянс відрізняється від торговельного балянсу тим, що він є той
торговельний балянс, який треба в певний час вирівняти. Отож кризи діють так,
що вони стискують ріжницю між платіжним балянсом та балянсом торговельним
в певний короткий час; і ті певні обставини, що розвиваються в нації, де є
криза, і де тому й настає тепер і реченець платежа, — ці обставини приносять уже
самі по собі таке скорочення часу, що протягом його має відбутись вирівнювання
платежів. Поперше, відправа благородних металів, потім розпродаж комісійних
товарів по низьких цінах; експортування товарів, аби тільки їх збути, або ж
добути під них грошову позику всередині країни; піднесення рівня проценту,
відмова в кредиті, спад курсу цінних паперів, розпродаж чужих цінних паперів,
притягування чужоземного капіталу до приміщення в цих знецінених цінних
паперах, насамкінець, банкрутство, що ліквідує масу вимог. При цьому ще часто
металь відправляють до тієї країни, де вибухла криза, бо векселі на ту країну
непевні, отже, платіж в металі є найпевніший. До цього долучається ще та
обставина, що проти Азії всі капіталістичні нації є, здебільша, одночасно, безпосередньо
чи посередньо, її винуватці. Скоро ці різні обставини справлять свій повний
вплив на другу причетну націю, то одразу й у неї починається експорт золота або
срібла, коротко — надходить реченець платежів, і ті самі явища повторюються.

При комерційному кредиті процент, як ріжниця між ціною в кредит та
ціною за готівку, ввіходить в ціну товарів лише остільки, оскільки векселі
мають реченець, довший за звичайний. В іншому разі того не буває. І це поясню-
