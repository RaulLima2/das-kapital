англійська робітнича кляса платить 100\% льомбардам, цим нащадкам тих Monts-de-piete21).
Так само не говоримо ми про кредитові фантазії якогось доктора
Hugh Chainperleyne або John Briscoe, що в останнє десятиріччя 17 віку силкувалися
емансипувати англійську аристократію від лихварства за допомогою
земельного банку з паперовими грішми які базувалися на земельній власності22).

Кредитові асоціації, що утворилися в 12 та 14 віці в Венеції та Генуї,
постали з потреби морської торговлі та заснованої на ній гуртової торговлі
емансипуватися від панування старомодного лихварства та монополізаторів
торговлі грішми. Якщо справжні банки, засновані в цих міських республіках,
були одночасно установи громадського кредиту, звідки держава одержувала позики
під податки, що вона мала одержати, то не слід забувати, що купці, які утворювали
ті асоціяції, сами були тими першими людьми в тих державах та були
зацікавлені в тому, щоб емансипувати від лихварства уряд, так само як і самих
себе23), а разом з цим таким способом більше та певніше підбити собі державу.
Тому, коли малося заснувати Англійський банк, торі теж заперечували: «Банки,
то — республіканські установи. Банки, що процвітають, є в Венеції, Генуї,
Амстердамі та Гамбурзі. А чи чув хто колись про французький або еспанський
банк».

Амстердамський банк (1609 р.), так само як і Гамбурзький банк (1619 р.)
не визначає жодної епохи в розвитку новітньої кредитової системи. Це був
чисто депозитний банк. Бони, видавані тим банком, в дійсності були лише
досвідки про одержання складеного до банку карбованого й некарбованого благородного
металу та циркулювали лише з передатним написом осіб, що їх одержували.
Однак, в Голяндії разом з торговлею та мануфактурою розвинулися
комерційний кредит та торговля грішми, а капітал, що дає процент, в процесі
самого розвитку було підпорядковано промисловому й торговельному капіталові.
Це виявлялося вже в низькому рівні проценту. Але в 17 віці Голяндію вважалось
за зразкову країну економічного розвитку, як от тепер Англія. Монополія
старомодного лихварства, що базувалося на убозтві, знищилася там сама собою.

Протягом усього 18 віку чути крики — і законодавство діє в цьому ж напрямку,
— які покликаючися на Голяндію, вимагають силоміць знизити рівень
проценту, щоб підпорядкувати капітал який дає процент торговельному та
промисловому капіталові, а не навпаки. Головний виразник цього є сер Josiah

21) «Процент ломбардів стає таким надмірним через часті застави та викупи протягом того
-самого місяця та заставу однієї речі, щоб забрати з ломбарду другу та одержати при цьому невеличку
ріжницю грішми. В Лондоні є 240 дозволених установ, що визичають гроші під заставу, а в провінції
є їх приблизно 1450. Капітал, що Його ті установи уживають, становить приблизно 1 міл. Він
обертається,
принаймні, тричі на рік, даючи кожного разу пересічно 33 1/2\%: так що нижчі кляси
Англії платять приблизно 100\% річно за часову позику одного мільйона, не рахуючи при цьому втрати
заставлених речей з причини невикупу їх у певний реченець». (I. I. Tuckeett, A History of the Past
aud Present State of the Labouring Population. London 1846. I., p. 114).

22) Навіть в назвах своїх праць вони подавали, як головну мету, «загальне благо земельних
власників, велике піднесення вартости землеволодіння, звільнення шляхти та gentry і т. ін. від
податків,
збільшення їхнього річного доходу і т. ін.». Від цього, мовляв, втратили б тільки лихварі, ці
найгірші
вороги нації, що зробили більше шкоди шляхті та ylomanry, ніж це міг би зробити напад французького
війська.

23) «Напр., ще Карл II Англійський мав платити величезні лихварські проценти та промінне
«золотарям» (попередникам банкірів), 20—30\%. Така вигідна справа чимраз більше спонукала «золотарів»
давати королеві позики, наперед антиципувати суми, що мали надходити від податків, забирати
як забезпечення всяке асиґнування парляментом грошей, скоро те асиґнування було зроблено, а також
наввипередки купувати та брати у заставу bills, , orders і tallies, так що у дійсності всі державні
доходи
йшли через їхні руки». (Iohn Francis, History of the Bank of England. 1848. London I. p 31).
«Заснування
банку пропонувалося кілька разів уже й раніше. Насамкінець, воно стало потрібним». (I. с., р. 38).
«Банк потрібен був вже бодай лише для того, щоб уряд, що його соки висисали лихварі, мав змогу
одержувати гроші за людський рівень проценту під забезпечення парляментськими асиґновками».
I. c., p. 59, 60).
