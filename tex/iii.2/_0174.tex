\parcont{}  %% абзац починається на попередній сторінці
\index{iii2}{0174}  %% посилання на сторінку оригінального видання
Отож з цих таблиць випливає таке.

Насамперед, що в ряду рент відношення точно такі самі, як у ряду ріжниць
родючости, виходячи з реґуляційної землі, що не дає ренти, як нулевого
пункту. Рента визначається не абсолютним здобутком, а лише ріжницями здобутку.
Чи дають різного роду землі 1, 2, 3, 4, 5 бушелів, чи 11, 12, 13, 14,
15 бушелів здобутку з акра, ренти в обох випадках становлять ряд: 0, 1, 2,
З, 4 бушелі, або відповідний цьому грошовий здобуток.

Але куди важливіший наслідок є у відношенні до загальної суми, ренти
при повторній витраті капіталу на тій самій землі.

В п’ятьох випадках з досліджених тринадцятьох подвоюється разом з витратою
капіталу і загальна сума ренти; замість 10 × 12 шил. вона стає
10 × 24 шил. = 240 шил. Випадки ці такі:

Випадок І, стала ціна, варіянт І: незмінне підвищення продукції (таблиця
XII).

Випадок II, низхідна ціна, варіянт III: ростуче підвищення продукції
(таблиця XVIII).

Випадок III, висхідна ціна, перша видозміна, коли земля А залишається
реґуляційною у всіх трьох варіянтах (таблиці XIX, XX, XXI).

У чотирьох випадках рента підвищується більш, ніж удвоє, а саме:

Випадок І, варіянт III: стала ціна, але ростуче підвищення продукції
(таблиця XV). Сума ренти підвищується до 330 шил.

Випадок III, друга видозміна, коли земля А дає ренту в усіх трьох варіянтах
(таблиця XXII, рента = 15 × 30 = 450 шил.; таблиця ХХІІІ, рента = 5 ×
20 + 10 × 28 = 380 шил.; таблиця XXIV, рента = 5 × 15 + 15 × 33 3/4 =
581 1/4 шил.).

В одному випадку вона підвищується, але не до подвійної суми проти
ренти, одержуваної при першій витраті капіталу.

Випадок І, стала ціна, варіянт II: низхідна продуктивність другої витрати
за умов, коли В не стає землею, що зовсім не дає ренти (таблиця XIV, рента
= 4 × 6 + 6 × 21 = 150 шил.).

Нарешті, тільки в трьох випадках загальна рента при другій витраті
капіталу, для всіх родів землі разом, лишається в тому самому становищі, як при
першій витраті (таблиця XI); це ті випадки, коли земля А перестає брати участь
у конкуренції, а земля В стає реґуляційною, і тому землею, що не дає ренти.
Отже, рента відпадає не тільки з В, але вона зменшується в кожному наступному
члені ряду рент; цим зумовлюється наслідок. Випадки ці такі:

Випадок І, варіянт II, коли умови такі, що земля А випадає (таблиця XIII).
Сума рент дорівнює 6 × 20, отже 10 × 12 = 120, як у таблиці XI.

Випадок II, варіянт І і II. Тут, згідно з припущеннями, неодмінно випадає
земля А (таблиці XVI і XVII), і сума ренти є знову 6 × 20 = 10 × 12 =
120 шил.

Таким чином це значить: в переважній більшості всіх можливих випадків
в наслідок збільшеного приміщення капіталу в землю рента підвищується як
з акра землі, що дає ренту, так і в її загальній сумі. Лише в трьох випадках
з досліджених тринадцятьох загальна сума її лишається без зміни. Це ті випадки,
коли найгіршої якости земля, що до того часу не давала ренти й відігравала
ролю реґуляційної землі, перестає брати участь у конкуренції, і земля безпосередньо
краща якістю стає на її місце, отже, перестає давати ренту. Але і в
цих випадках ренти з кращих земель підвищуються проти тих рент, які завдячують
своїм походженням першій витраті капіталу; коли рента з С понижується
з 24 до 20, то для D і Е вона підвищується з 36 і 48 до 40 і 60 шил.

Пониження загальної суми рент нижче від того рівня, що вона мала за
першої витрати капіталу (таблиця XI) було б можливе лише тоді, коли б, крім
\parbreak{}  %% абзац продовжується на наступній сторінці
