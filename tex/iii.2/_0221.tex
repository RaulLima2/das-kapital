\parcont{}  %% абзац починається на попередній сторінці
\index{iii2}{0221}  %% посилання на сторінку оригінального видання
Як мало може здійснитися це перетворення без певного розвитку суспільної
продуктивної сили праці, свідчать різні невдалі спроби цього перетворення,
зроблені за римських імператорів, і повернення до натуральної ренти після
того як пробували перетворити на грошову ренту принаймні усю ту частину
цієї ренти, яка існувала у вигляді державного податку. Про таку саму трудність
переходу свідчить, наприклад, перед революцією у Франції з’єднання і
фалшування грошової ренти рештками її давніших форм.

Грошова рента, як перетворена форма ренти продуктами і, протилежно до
неї, є остання форма і одночасно форма розпаду того роду земельної ренти,
який ми розглядали до цього часу, саме земельної ренти як нормальної форми
додаткової вартости і неоплаченоі додаткової праці, яку доводиться віддавати
власникові умов продукції. У своїй чистій формі ця рента, як і відробітна
рента і рента продуктами, не являє якогось надміру над зиском. За своїм
поняттям вона поглинає зиск. Оскільки зиск фактично виникає поряд з рентою
як особлива частина надмірної праці, грошова рента, як і рента в її давнішніх
формах, все ще лишається нормальною межею цього ембріонального зиску,
який може розвиватись лише в міру того, як розвивається можливість експлуатації
чи то власної надмірної, чи то чужої праці, яка лишається по виконанні
додаткової праці, втіленої в грошовій ренті. Коли дійсно виникає зиск поряд
з цією рентою, то тоді не зиск є межа ренти, а навпаки, рента є межа зиску.
Але, як уже сказано, грошова рента є одночасно форма розпаду розглядуваної
до цього часу земельної ренти, що prima facie збігається з додатковою вартістю
і додатковою працею, земельної ренти як нормальної і панівної форми додаткової
вартости.

У своєму дальшому розвитку грошова рента мусить призвести, — лишаючи
осторонь всі проміжні форми, як, наприклад, форму дрібноселянських
орендарів, — або до перетворення землі на вільну селянську власність або до
форми капіталістичного способу продукції, до ренти, виплачуваної капіталістичним
орендарем.

При грошовій ренті традиційні звичаєві-правні відносини між підлеглим
продуцентом, що посідає і обробляє частину землі, і між земельним
власником доконечно перетворюються на договірні, визначувані точними нормами
позитивного закону, чисті грошові відносини. Тому обробник-посідач по
суті справи стає звичайним орендарем. Це перетворення за інших сприятливих
загальних відносин продукції, з одного боку, використовувалось для того, щоб поступово
експропріювати старих селян-посідачів, і на їхнє місце посадити капіталістичного
орендаря; з другого боку, воно призводить до того, що колишній
посідач викуповує своє зобов’язання виплачувати ренту і перетворюється на незалежного
селянина з цілковитою власністю на оброблювану ним землю. Далі
перетворення натуральної ренти на грошову ренту не тільки доконечно супроводиться,
але навіть антиципується створенням кляси поденників, що нічого
не мають і наймаються за гроші. Тому в період виникнення цієї нової кляси,
коли вона виступає ще лише спорадично, у зобов’язаних до виплати ренти
селян, які перебувають в кращому становищі, доконечно розвивається звичай
визискувати сільських найманих робітників власним коштом, — цілком так само,
як за доби февдалізму заможніші підлеглі селяни і собі держали підлеглих.
Таким чином, для них поступово розвивається можливість зібрати певне майно
і самим перетворитись на майбутніх капіталістів. Так навіть серед старих посідачів
землі, що працюють сами, виникає розсадник капіталістичних орендарів,
що їх розвиток зумовлюється загальним розвитком капіталістичної продукції
поза сільським господарством, і які розцвітають з особливою швидкістю, коли
на допомогу їм приходять, як в XVI столітті в Англії, особливо сприятливі
обставини, подібно до тодішнього проґресивного знецінення грошей, що
\parbreak{}  %% абзац продовжується на наступній сторінці
