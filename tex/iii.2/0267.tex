обставин, то може постати відповідне номінальне підвищення цін цих товарів. Це
підвищення відносної вартости одного сорту товарів проти інших, що для них
заробітна плата лишається незмінна, є тоді лише реакція проти місцевого порушення
рівномірности в розподілі додаткової вартости між різними сферами продукції,
засіб вирівнювання окремих норм зиску на загальну норму. Той «досвід», який
тут здобувається, свідчить знову про те, що ціна визначається заробітною платою.
Отже, досвід в обох випадках виявляє, що заробітна плата визначає товарові
ціни. Чого не показує досвід — це прихованої причини цієї залежности.
Далі: пересічна ціна праці, тобто вартість робочої сили, визначається ціною
продукції потрібних засобів існування. Коли вона підвищується або понижується
то підвищується або понижується й перша. Досвід виявляє тут знов таки лише
існування залежности між заробітною платою і ціною товарів; але причина
може набути вигляду наслідку, наслідок — вигляду причини, як це буває, наприклад,
при русі ринкових цін, коли підвищенню заробітної плати понад пересічний
рівень відповідає зв’язане з періодом розквіту підвищення ринкових цін
понад ціни продукції, а наступному потім падінню заробітної плати нижче пересічного
рівня відповідає падіння ринкових цін нижче від цін продукції. Коли
лишити осторонь коливальний рух ринкових цін, то залежності цін продукції
від вартости товарів мусив би prima facie відповідати той виявлений досвідом
факт, що при підвищенні заробітної плати понижується норма зиску і навпаки.
Але ми бачили, що норма зиску може визначатися змінами в вартості сталого
капіталу незалежно від змін заробітної плати; так що заробітна плата і
норма зиску можуть рухатися не в протилежному, а в тому самому напрямку,
можуть одночасно підвищуватись або понижуватися. Коли б норма додаткової
вартости безпосередньо збігалася з нормою зиску, то це було б неможливе.
Так само, коли заробітна плата підвищується, в наслідок зросту цін засобів
існування, норма зиску може лишитись незмінна або навіть підвищитися, в наслідок
більшої інтенсивности праці, або подовження робочого дня. Всі ці досвіди
потверджують ілюзію, спричинену самостійною перекрученою формою складових
частин вартости, — ілюзію, ніби сама заробітна плата, або заробітна плата разом
з зиском визначають вартість товарів. Коли взагалі така ілюзія постає
щодо заробітної плати, коли, отже, здається, що ціна праці збігається з вартістю,
породженою працею, то зрозуміла річ, що така ілюзія постає також щодо зиску
й ренти. Їхні ціни, тобто їхні грошові вирази, мусять тоді регулюватись
незалежно від праці і породженої нею вартости.

Потретє: Припустімо, що вартості товарів, або їхні ціни продукції, що
лише здаються незалежними від вартостей товарів, — в зовнішньому виявленні
безпосередньо й завжди збігаються з ринковими цінами товарів, замість, навпаки,
пробиватися лише як реґуляційні пересічні ціни через постійне вирівнювання
безупинних коливань ринкових цін. Припустімо далі, що репродукція відбувається
завжди в тих самих незмінних відношеннях, отже, що продуктивність праці в усіх
елементах капіталу лишається стала. Припустімо, нарешті, що частина вартости
товарового продукту, яка в кожній сфері продукції створюється через долучення
до вартости засобів продукції нової кількости праці, отже, нововипродукованої
вартости, — що ця частина розпадається в завжди незмінюваній пропорції на заробітну
плату, зиск і ренту, так що дійсно виплачена заробітна плата, фактично
реалізованний зиск, та фактична рента завжди безпосередньо збігаються перша
з вартістю робочої сили, другий — з тією частиною сукупної додаткової вартости,
що, згідно з пересічною нормою зиску, припадає на кожну частину сукупного
капіталу, який самостійно функціонує, і, нарешті, третя — з тими межами, що в
них normaliter вміщена земельна рента на даній базі. Припустімо, одним словом,
що розподіл вартости суспільного продукту і регулювання цін продукції відбувається
на капіталістичній базі, але з усуненням конкуренції.
