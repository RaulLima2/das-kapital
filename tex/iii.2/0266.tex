номінальну додачу певних відсотків до заробітної плати, то ціна підвищувалася
б і падала б разом з заробітною платою. Заробітна плата спочатку прирівнюється
тут до вартости товару і потім знову відокремлюється від неї.
В дійсності такі теорії спричинюються несвідомими манівцями до того висновку,
що вартість товару визначається кількістю вміщеної в нього праці, вартість заробітної
плати ціною потрібних засобів існування, а надмір вартости над заробітною
платою становить зиск і ренту.

Розпадання вартости товарів за вирахуванням вартости зужиткованих на
їхню продукцію засобів продукції; розпадання цієї даної маси вартости, визначеної
кількістю праці, зрічевленої в товаровому продукті, на три складові частини,
які набувають потім у формі заробітної плати, зиску й земельної ренти
самостійного вигляду незалежних одна від однієї форм доходу, — це розпадання
виступає в перекрученому вигляді, на приступній для всіх поверхні капіталістичної
продукції, отже, і в уявленні аґентів захоплених цією продукцією.

Хай сукупна вартість будь-якого товару = 300, що з них 200 є вартість
зужиткованих на його продукцію засобів продукції або елементів сталого
капіталу. Отже, лишається 100, що є сума нової вартости, долученої до цього
товару в процесі його продукції. Ця нова вартість в 100 вичерпує собою все.
що може піддаватись поділові на ці три форми доходу. Коли припустимо, що
заробітна плата = x, зиск = y, земельна рента = z, то в нашому випадку
сума x + y + z завжди = 100. Але в уявленні промисловців, купців і банкірів,
як і в уявленні вульґарних економістів, це набуває цілком іншого вигляду. Для
них не вартість товару за вирахуванням вартости зужиткованих на його продукцію
засобів продукції становить дану величину = 100, що розпадається потім
на x, y і z. Навпаки, для них ціна товару складається просто з вартости заробітної
плати, вартости зиску і вартости ренти, що їхні величини визначаються
незалежно від вартости товару і одна від однієї, так що x, y, z є величини
дані й визначені кожна самостійно, і лише з суми цих величин, яка може бути
менша й більша за 100, складається величина вартости самого товару, як результат
підсумовання цих складових частин його вартости. Таке quid pro quo
є неминуче:

Поперше, тому що складові частини вартости товару протистоять одна
одній як самостійні доходи, що як такі, стосуються до трьох, цілком відмінних
один від одного, аґентів продукції, праці, капіталу і землі, в наслідок
чого здається, що вони виникають з цих останніх. Власність на робочу силу, на
капітал і землю є причина того, що ці різні складові частини вартости товарів
припадають цим відповідним власникам і тому перетворюються для них на доходи.
Але вартість не виникає з перетворення на дохід, вона мусить бути вже в наявності,
раніш ніж вона матиме змогу перетворитись на дохід, набути цієї
форми. Протилежна ілюзія мусить дедалі більш зміцнюватись тому, що визначення
відносних величин цих трьох частин одна проти однієї відбувається за
різними законами, що їхній зв’язок з самою вартістю товарів і обмеження їх
нею зовсім не виявляється на поверхні явищ.

Подруге. Ми бачили, що загальне підвищення або пониження заробітної
плати, спричиняючи при інших рівних обставинах рух загальної норми
зиску у протилежному напрямку, змінює ціни продукції різних товарів, підвищує
одні з них, понижує інші, залежно від пересічного складу капіталу у відповідних
сферах продукції. Отже, тут, принаймні, в деяких галузях продукції,
досвід виявляє, що пересічна ціна товару підвищується тому, що підвищується
заробітна плата і понижується тому, що вона понижується. Але чого не виявляє
«досвід» — це прихованого регулювання цих змін вартістю товарів, незалежною
від заробітної плати. Коли, навпаки, підвищення заробітної плати є місцеве, коли
воно відбувається лише в окремих галузях продукції, в наслідок своєрідних
