\index{iii2}{0163}  %% посилання на сторінку оригінального видання
Таблиця VІa.

Земля
Акри
Капітал Ф. ст.
Зиск Ф. ст.
Продукт з акра в квартерах
Продажна ціна Ф. ст.
Здобуток Ф. ст.
Рента
Збіжж. Кварт.
Грош. Ф. ст.

А    1    2 1/2 + 2 1/2 = 5    1    1 + 3 = 4               1 1/2      6          0*)          0
В    1    2 1/2 + 2 1/2 = 5    1    2 + 2 1/2 = 4 1/2  1 1/2     6 3/4    1/2         3/4
С    1    2 1/2 + 2 1/2 = 5    1    3 + 5 = 8               1 1/2     12         4                6
D    1    2 1/2 + 2 1/2 = 5    1    4 + 12 = 16           1 1/2    24         12            18
Разом    4    20                             32 1/2                      16 1/2
  24 3/4

Нарешті, грошова рента підвищилася б, коли б у кращі земельні дільниці,
при тому самому відносному підвищенні родючости, вкладено було більше
додаткового капіталу, ніж у землю А, або коли б додаткові вкладання капіталу в кращі
земельні дільниці впливали, підвищуючи норму продуктивности. В обох випадках
ріжниці зростали б.

Грошова рента понижується, коли поліпшенння, що сталося в наслідок
додаткової витрати капіталу, зменшує всі ріжниці, або частину їх, впливаючи
більше на А, ніж на В і С. Вона понижується то більше, що незначніше
підвищення продуктивности кращих земельних дільниць. Від відносної неоднаковости
впливу залежить, чи підвищиться збіжжева рента, чи понизиться або
залишиться без зміни.

Грошова рента підвищується, а також і збіжжева рента, або тоді, коли за
незмінної відносної ріжниці в додатковій родючості різних земель більше вкладається
додаткового капіталу в землю, що дає ренту, ніж у землю А, що не дає
ренти, і більше у землю, що дає вищу, ніж у землю, що дає нижчу ренту;
абож тоді, коли родючість, при однаковому додатковому капіталі, більше зростає
на кращій і найкращій землі, ніж на землі А, причому грошова і збіжжева
рента підвищується саме у такому відношенні, в якому це збільшення родючости
на вищих розрядах землі вище, ніж на нижчих.

Але за всяких обставин рента відносно підвищується, коли підвищена продуктивність
є наслідок додаткової витрати капіталу, а не просто наслідок
збільшеної родючости за незмінної витрати капіталу. Це є абсолютний погляд,
який показує, що тут, як і в усіх давніших випадках, рента і збільшена рента з акра
(подібно до того, як при диференційній ренті І висота пересічної ренти на всю
оброблювану площу) є наслідок збільшеної витрати капіталу на землю, при
чому байдуже, чи функціонує ця витрата з сталою нормою продуктивности за
сталих або понижених цін, чи з низхідною нормою продуктивности за сталих або
за понижених цін, чи з висхідною нормою продуктивности за понижених цін.
Бо наше припущення: стала ціна за сталої, низхідної або висхідної норми продуктивности додаткового
капіталу, і низхідна ціна, за сталої, низхідної і висхідної
норми продуктивности, зводиться ось до чого: стала норма продуктивности додаткового капіталу при
сталій або низхідній ціні, низхідна норма продуктивности
при сталій або низхідній ціні, висхідна норма продуктивности за сталої

*) В німецькому тексті тут очевидно помилково стоїть «6». Прим. Ред.
\parbreak{}  %% абзац продовжується на наступній сторінці
