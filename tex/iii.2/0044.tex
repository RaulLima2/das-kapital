одного bill-broker’a, як от, напр., фірми Overend, Gurney s С°, була одним з головних
засобів місцево поширити маштаб цього вирівнювання. За допомогою
цього економізування підвищується чинність засобів циркуляції, оскільки стає
потрібною менша кількість цих засобів до простого вирівнювання балянсу.
З другого боку, швидкість руху грошей, що обертаються як засобі циркуляції, залежить
(і це теж дає економію) цілком від потоку купівель та продажів або теж
від зчеплювання платежів, оскільки вони відбуваються одні по одних у грошах.
Але кредит обслуговує циркуляцію й тим підвищує її швидкість. Поодинока монета
може, напр., здійснити лише п’ять оборотів та — як простий засіб циркуляції
без посередництва кредиту — довше лишається в кожних поодиноких руках,
коли А, її первісний державець, купує в В, В в С, С в D, D в Е, Е в F, отже,
коли її перехід з рук до рук відбувається лише за посередництвом дійсних купівель
та продаж. Коли ж В гроші, одержані від А як платіж, складає у свого
банкіра, а цей останній витрачає їх на дисконт векселя С, С купує в D, D складає
гроші у свого банкіра, а останній позичає їх Е, що купує в F, то сама
швидкість руху грошей як простого засобу циркуляції (купівного засобу) обумовлюється декількома
кредитовими операціями: В складає гроші у свого банкіра,
а останній дисконтує вексель для С, D складає гроші у свого банкіра, а останній
дисконтує вексель для Е; отже за посередництвом чотирьох кредитових операцій.
Без цих кредитових операцій та сама монета протягом даного часу не виконала
б п’яти купівель однієї по одній. Що дана монета зміняла руки — як вклад та
в дисконті — без посередництва дійсних купівель та продажів, то і прискорила
вона тут свій перехід з рук в руки в черзі дійсних операцій купівлі-продаж.

Вище виявилося, що одна й та сама банкнота може являти вклади в
різних банкірів. Так само може вона являти різні вклади в того самого банкіра.
Банкір дисконтує вексель В тією банкнотою, що її склав в нього А, В платить
нею С, С складає ту саму банкноту в того самого банкіра, що її видав.

Вже при розгляді простої грошової циркуляції (Книга І, розд. III, 2) показано,
що маса грошей, які дійсно є в циркуляції, якщо швидкість циркуляції
та економію на платежах припущено за дані, визначається цінами товарів та
масою операцій. Той самий закон має чинність і в циркуляції банкнот.

В таблиці на стор. 45 подано для кожного року пересічну річну суму банкнот
Англійського банку, оскільки вони були в руках публіки, а саме суми банкнот
в 5 та 10 ф. ст., банкнот від 20 до 100 ф. ст. та суми банкнот вищих купюр від
200 до 1000 ф. ст.; і так само подано кожну з цих рубрик у відсотках до
всієї суми банкнот, що перебували в циркуляції. Суми подано в тисячах, чому
три останні знаки закреслено.

(В. А. 1858, р. I, II). Отже, ціла сума банкнот, що були в циркуляції, протягом
від 1844 до 1857 року позитивно зменшилася, хоч торговельний оборот,
оскільки його виявив вивіз та довіз, більше як подвоївся. Дрібніші банкноти,
в 5 та 10 ф. ст., як показує таблиця, збільшилися сумою від 9.263.000 ф. ст.
в 1844 р. до 10.659.000 ф. ст. в 1857 р. І це одночасно з саме тоді значним
збільшенням циркуляції золота. Зате зменшення банкнот вищої вартости (від
200 до 1000 ф. ст.) від 5.865.000 ф. ст. в 1852 році до 3.241.000 ф. ст. в
1857 році. Отже, зменшення більше, ніж на 21/2 міл. ф. ст. Це пояснюється
так: «8 червня 1854 року приватні банкіри Лондону притягли акційні банки
до участи в організації Clearing House, і незабаром по тому було заведено
остаточний розрахунок (clearing) в Англійському банку. Щоденні вирівнювання
переводяться переписуванням на рахунках, що їх різні банки мають в Англійському
банку. Через заведення цієї системи банкноти вищої вартости, що ними
банки користувалися раніше для вирівнювання своїх взаємних рахунків, стали
зайвими (В. А. 1858, р. V).
