дукції, та цілком не розуміють того, що землю як і капітал визичають тільки
капіталістам. Звичайно, замість грошей можна визичати засоби продукції in natura,
як от машини, промислові будинки і т. ін. Але тоді вони становлять певну
грошову суму, і те, що, крім проценту, платиться певну частку за зужиткування,
випливає з споживчої вартости, з особливої натуральної форми цих елементів
капіталу. Справу вирішує тут знову те, чи визичають їх безпосереднім продуцентам,
що має собі за передумову відсутність капіталістичного способу продукції,
принаймні в тій сфері, де трапляється це; або чи визичають їх промисловим
капіталістам, передумова, що може бути саме на базі капіталістичного
способу продукції. Ще більш недоречно та іраціонально притягувати сюди визичання
домів і т. ін. для індивідуального споживання. Що робітничу клясу
обшахровують і в цій формі, та й ще страшенно обшахровують, це — відомий
факт; але те саме робить і дрібний крамар, що постачає тій клясі життьові
засоби. Це другоступневий визиск, який іде поряд первісного, що відбувається
безпосередньо в самому процесі продукції. Ріжниця між продажем та визичанням
є тут цілком байдужа й формальна річ, яка, як уже показано, тільки при
певному нерозумінні дійсного зв’язку здається істотною.

Лихварство, як і торговля, визискують даний спосіб продукції, не утворюючи
його та ставлячись до нього зовнішнім способом. Лихварство силкується
просто зберегти його, щоб мати змогу знову й знову визискувати його, воно консервативне
й робить той спосіб тільки злиденнішим. Що менше елементи продукції
ввіходять до процесу продукції, як товари, та виходять з нього, як товари,
то більше добування їх з грошей видається осібним актом. Що незначніша
є роля, що її відіграє циркуляція в суспільній репродукції, то більше розцвітає
лихварство.

Те, що грошове майно розвивається, як осібне майно, означає відносно
лихварського капіталу, що він усі свої вимоги має в формі грошових вимог.
Він розвивається в країні то більше, що більше продукція у своїй масі обмежується
на натуральних відбутках і т. ін., отже, на продукції споживчої вартости.

Оскільки лихварство здійснює дві речі: поперше, взагалі утворює поряд
купецтва самостійне грошове майно, подруге, присвоює собі умови праці, тобто
руйнує власників колишніх умов праці, — остільки є воно могутня підойма до
утворення передумов для промислового капіталу.

Процент у середні віки.

«В середні віки людність була суто хліборобська. А серед такої людности,
як і за февдального режиму можуть бути лише невеликі торговельні зв’язки, а
тому й лише невеликий зиск. Тому закони про лихварство в середні віки були
цілком правні. Сюди долучається те, що у хліборобській країні рідко хто доходить
такого стану, щоб позичати гроші, хіба тільки тоді, коли він опинився
серед убозства та злиднів... Генріх VII обмежує процент на 10%, Яків І —
на 8, Карл II — на 6. Анна — на 5%... Тоді позикодавці були, як що й неправні,
то все ж фактичні монополісти, і тому треба було обмежити їх, як і інших монополістів...
В наші часи норма зиску регулює норму проценту, а тоді норма проценту
реґулювала норму зиску. Коли грошовий позикодавець накидав купцеві тягар високої
норми проценту, купець мусив долучати вищу норму зиску до своїх товарів.
Тому велику суму грошей бралося з кешень покупців, щоб перекласти їх до кешень
грошових позикодавців». (Gilbart, and History Princ. of Banking, p. 164, 165).

«Мені кажуть, що тепер беруть на кожну ляйпцізьку марку 10 ґульденів
на рік, що становить 30 на 100; декотрі ще додають сюди наєнбурзьку марку,
