\parcont{}  %% абзац починається на попередній сторінці
\index{iii2}{0142}  %% посилання на сторінку оригінального видання
продуцентам, перетвореним в найманих робітників. Оскільки капіталістичний
спосіб продукції виступає тут з своїми характеристичними особливостями,
відбувається це спочатку поперше і особливо в галузі вівчарства і скотарства;
але далі це виявляється не в концентрації капіталу на відносно невеликій
площі землі, а в продукції в більшому маштабі, що дає заощадження на
утриманні коней і інших витратах продукції; але в дійсності тут немає вживання
більшого капіталу на тій самій землі. Далі, в природних законах хліборобства
є те, що за певної висоти культури і відповішого їй виснаження
ґрунту, капітал, що його тут розуміється також, як уже вироблені засоби
продукції, стає вирішальним елементом хліборобства. Поки оброблювана
земля становить відносно невелику площу проти необробленої, і сила землі ще
не виснажена, (а таке становище маємо при перевазі скотарства і м’ясної їжі
за періоду, що передує періодові переваги власне хліборобства і рослинної їжі)
народжуваний новий спосіб продукції протистоїть селянській продукції саме розміром
земельної площі, що обробляється коштом одного капіталіста, отже, знов таки
екстенсивним вживанням капіталу на земельній площі більшого простору. Таким
чином, від самого початку слід установити, що диференційна рента І є тією
історичною основою, яка править за вихідний пункт. З другого боку, рух диференційної
ренти II для кожного даного моменту починається лише в такій царині,
яка сама знову-таки становить мозаїкову основу для диференційної ренти І.

\emph{Подруге}. При диференційній ренті в формі II то ріжниці родючости приєднуються
ріжниці в розподілі капіталу (і кредитоспроможности) між орендарями.
У власне мануфактурі для кожної галузі продукції швидко створюється особливий
мінімум розміру підприємства і відповідно до цього мінімум капіталу,
без якого не можна успішно провадити окремі підприємства. Так само в кожній
галузі продукції створюється нормальний пересічний розмір капіталу, який
перевищує цей мінімум і яким мусять порядкувати і порядкують більшість продуцентів.
Капітал більший понад цей розмір може дати надзиск; менший — не
дає і пересічного зиску. Капіталістичний спосіб продукції лише повільно і нерівномірно
охоплює сільське господарство, як це можна спостерігати в Англії,
клясичній країні капіталістичного способу продукції в хліборобстві. Коли немає
вільного довозу збіжжя, або його дія лише обмежена, бо розмір довозу обмежений,
то ринкову ціну визначають продуценти, що працюють на гіршій землі,
отже, в умовах продукції несприятливіших, ніж пересічні. Більша частина загальної
маси капіталу, який вживається в хліборобстві і яким воно взагалі
порядкує, перебуває в їхніх руках.

Вірно, що селянин, наприклад, витрачає багато праці на свою маленьку
парцелю. Але праці ізольованої і позбавленої об’єктивних, так суспільних, як і
матеріяльних умов продуктивности, — праці від цих умов відлученої.

Ця обставина призводить до того, що справжні капіталістичні орендарі
мають змогу привласнювати собі частину надзиску; це відпало б, принаймні,
оскільки справа торкається цього пункту, тоді, коли капіталістичний спосіб продукції
був би так само рівномірно розвинений у сільському господарстві, як у
мануфактурі.

Розгляньмо насамперед створення надзиску при диференційній ренті II, не
торкаючись умов, за яких може відбутися перетворення цього надзиску в земельну
ренту.

Тоді ясно, що диференційна рента II є лише інший вираз диференційної
ренти І, а по суті збігається з нею. Різна родючість різних земель впливає при
створенні диференційної ренти. І лише остільки, оскільки вона призводить до
того, що вкладений в землю капітал дає неоднакові наслідки, неоднакову кількість
продуктів на капітали однакової величини, або пропорційно до їхньої відносної
величини. Чи постає ця нерівність для різних капіталів, вкладуваних
\parbreak{}  %% абзац продовжується на наступній сторінці
