тивних сил праці, які набувають, проте, проти робітника вигляду самостійних
сил капіталу і перебувають тому в безпосередній суперечності з власним його,
робітника, розвитком. Продукція ради вартости і додаткової вартости включає,
як показали наші дальші досліди, безупинно діющу тенденцію, що намагається
звести робочий час, потрібний для продукції товару, тобто вартість останнього,
до сущої в кожний даний момент суспільної пересічної. Прагнення звести витрати
продукції до їхнього мінімуму стає найсильнішим знаряддям підвищення
суспільної продуктивної сили праці, яка проте, лише тут видається безупинним
підвищенням продуктивної сили капіталу.

Той авторитет, що його набуває капіталіст як персоніфікація капіталу в
безпосередньому процесі продукції, та суспільна функція, яку він має як керівник
і владар продукції, посутньо відмінні від авторитету, що виростає на базі
продукції рабської, крепацької тощо.

Тимчасом як на базі капіталістичної продукції масі безпосередніх продуцентів
протистоїть суспільний характер їхньої продукції, в формі суворого реґуляційного
авторитету й цілком розвиненої ієрархії суспільного механізму їхнього
трудового процесу, — при чому, одначе, цього авторитету його носії набувають
лише як персоніфікація умов праці в протилежність самій праці, а не як політичні
або теократичні владарі, як це було за давніших форм продукції — серед
самих носіїв цього авторитету, серед самих капіталістів, які протистоять один
одному лише як товаропосідачі, панує цілковита анархія, що в її рямцях суспільний
зв’язок продукції здійснюється лише як могутній закон природи наперекір
індивідуальній сваволі.

Тільки в наслідок того, що праця в формі найманої праці і засоби продукції
в формі капіталу дані як передумова — отже, тільки в наслідок цієї специфічно
суспільної форми цих двох істотних чинників продукції — частина вартости
(продукту) виступає як додаткова вартість і ця додаткова вартість як
зиск (рента), як бариш капіталіста, як додаткове, що є в його розпорядженні,
належне йому, багатство. Але тільки тому, що вона виступає таким чином яв
його зиск, додаткові засоби продукції, що призначені для поширення репродукції
і становлять частину зиску капіталіста, виступають як додатковий капітал,
а поширення процесу репродукції взагалі виступає як процес капіталістичної
акумуляції.

Хоч наймана праця є форма праці, що має вирішне значіння для
форми всього процесу і для специфічного характеру самої продукції, проте,
не найманою працею визначається вартість. При визначенні вартости справа
йде про суспільний робочий час взагалі, про кількість праці, що нею взагалі
може порядкувати суспільство, і поглинення якої в різних пропорціях різними
продуктами визначає, так би мовити, їхню питому суспільну вагу. Та певна форма,
що в ній суспільний робочий час визначально здійснюється у вартості товарів,
перебуває звичайно в зв’язку з найманою працею як формою праці, і капіталом
як відповідною формою засобів продукції остільки, оскільки лише на цій базі
товарова продукція стає загальною формою продукції.

Розгляньмо, проте, так звані розподільчі відносини сами по собі. Заробітна
плата має своєю передумовою найману працю, зиск — капітал. Отже, ці певні
форми розподілу мають своєю передумовою певний суспільний характер умов
продукції і певні суспільні відносини діячів продукції. Отже, певні розподільчі
відносини є лише вираз історично певних продукційних відносин.

А тепер візьмімо зиск. Ця певна форма додаткової вартости є передумова
того, що створення нових засобів продукції відбувається в формі капіталістичної
продукції; отже, це є відношення, що панує над репродукцією, хоч окремому
капіталістові і здається, що він міг би власне проїсти ввесь свій зиск як дохід. Він
наражається, проте, при цьому на межі, що постають перед ним уже у формі
