печу є розмінність цих банкнот. «18 вересня 1846 року циркуляція банкнот
Англійського банку становила 20.900.000 ф. ст., а його металевий скарб —
16.273.000 ф. ст.; 5 квітня 1847 р. циркуляція — 20.815.000 ф. ст., а металевий
скарб — 10.246.000 ф. ст. Отже, не зважаючи на експорт 6 мільйонів ф. ст. благородного
металю, не настало зменшення циркуляції». (I. G. Kinnear, The Crisis
and the Currency, Ld. 1847, p 5). Однак, само собою зрозуміла річ, що це мав силу
тільки в тих умовах, що тепер панують в Англії, та й то лише остільки, оскільки
законодавство не визначить якогось іншого відношення між виданням банкнот
та металевим скарбом.

Отже, тільки потреби самої лише торговлі (des Geschäfts selbst) мають
вплив на кількість грошей — банкнот та золота — в циркуляції. Тут, насамперед,
треба звернути увагу на періодичні коливання, що-повторюються кожного року,
хоч і який був би загальний стан справ, так що протягом останніх 20 років
«одного певного місяця циркуляція є висока, другого — низька, а третього певного
місяця доходиться середньої точки». (Newmarch, Б. А. 1857, № 1650).

Напр., у серпні місяці кожного року кілька мільйонів, здебільша золотом,
переходять з Англійського банку у внутрішню циркуляцію на оплату видатків
в зв’язку з жнивами; що тут головна справа у виплаті заробітної плати, то
в Англії в цій справі менше вживають банкнот. До кінця року ці гроші знову
припливають до банку. В Шотландії замість золотих соверенів є майже самі лише
банкноти однофунтівки; тому тут у відповідному випадку поширюється циркуляція
банкнот, і то саме двічі на рік, у травні та листопаді, від 3 до 4 мільйонів;
по 14 днях починається вже зворотний приплив, а за місяць він майже закінчується
(Anderson, 1., с; № 3595—3600).

Циркуляція банкнот Англійського банку щочверть року зазнає ще й
тимчасових коливань, бо виплачується щочверть року «дивіденди», тобто проценти
на державні борги, через що спочатку банкноти вилучаються з циркуляції,
а по тому знову їх викидається проміж публіку, але вони дуже скоро припливають
назад. Weguelin (В. А. 1857, № 38) подає суму викликаного цим коливання
циркуляції банкнот в 2 1/2 мільйони. Навпаки, пан Chapman з відомої фірми
Overend Gurney and C° обчислює суму порушення на грошовому ринку, викликану
тим явищем, далеко вище. «Коли ви податками заберете з циркуляції 6 або
7 мільйонів, щоб ними виплатити дивіденди, то мусить же бути хтось, хто дав
би цю суму до розпорядку на проміжний час». (В. А. 1857, № 5196).

Далеко значніші та тримаються довший час ті коливання суми засобів
циркуляції, що відповідають різним фазам промислового циклу. Про це послухаймо
іншого Associé\footnote*{
Associé — спільник. Пр. Ред.
} тієї фірми шановного квакера Samuel Gurney’я (С. D.
1848/57, № 2645): «Наприкінці жовтня (1847 р.) в руках публіки було банкнот
на 20.800.000 ф. ст. Тоді на грошовому ринку було дуже тяжко одержувати
банкноти. Це постало з загальної опаски, що в наслідок обмеження банковим
актом 1844 не можна буде добувати банкноти. Тепер [березень 1848 року] сума
банкнот в руках публіки становить... 17.700.000 ф. ст., але що тепер немає
ніякої комерційної паніки, то і є ця сума далеко більша за ту, що потрібна.
В Лондоні немає жодного банкіра або торговця грішми, що не мав би банкнот
більше, ніж він може їх ужити. — 2650. Сума банкнот... опріч тих, що є на
схові в Англійському банку, являє собою цілком недостатній покажчик активного
стану циркуляції, коли одночасно теж не взяти на увагу... стану торговельного
світу та кредиту. — 2651. Почуття, що тепер при сучасній сумі циркуляції в
руках публіки є надмір банкнот, постає в значній мірі з нашого сучасного становища,
з великого застою в справах. За високих цін та жвавих справ ця
кількість, 17.700.000 ф. ст., викликала б у нас почуття недостачі».