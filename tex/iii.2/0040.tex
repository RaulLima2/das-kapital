може постати й спекулятивне утворення запасів, почасти, щоб використати сприятливий
момент в інтересах продукції, почасти, сподіваючись пізнішого піднесення
цін. В цьому разі попит на позичковий капітал міг би зростати, а підвищений
рівень проценту свідчив би про те, що капітал витрачається на утворення
надмірних запасів елементів продуктивного капіталу. Тут ми розглядаємо
тільки попит на позичковий капітал, оскільки на нього впливає попит та
подання товарового капіталу. Вже раніше з’ясовано, як мінливий стан процесу
репродукції впливає в поодиноких фазах промислового циклу на подання позичкового
капіталу. Тривіяльну тезу, що ринкову норму проценту визначається
поданням та попитом (позичкового) капіталу, Оверстон хитро-мудро кидає до
однієї купи з тим своїм власним припущенням, що за ним позичковий капітал
є тотожній з капіталом взагалі й намагається тим способом перетворити лихваря
в однісінького капіталіста, а його капітал — в однісінький капітал.

Підчас пригнічення попит на позичковий капітал є попит на платіжні засоби
та й більш нічого; ніяк не попит на гроші, як купівний засіб. При цьому
рівень проценту може дуже високо піднятись, однаково, чи маємо ми реального
капіталу — продуктивного та товарового капіталу — понад міру, чи обмаль.
Попит на платіжні засоби є просто попит на перетворність у гроші, оскільки
купці та продуценти можуть подати добрі ґарантії; оскільки ж цього не буває,
оскільки, отже, авансування платіжних засобів дає їм не тільки грошову
форму, а ще й той еквівалент, що його їм бракує, до платежа, однаково в якій
формі той еквівалент має бути, — остільки той попит є попит на грошовий капітал.
Це — той пункт, де обидві течії поширеної вульґарної теорії, розглядаючи
кризи, мають рацію і одночасно не мають рації. Ті, хто каже, що існує лише недостача
платіжних засобів, або мають на оці тільки державців ґарантій bona
fide\footnote{
Bona fide — лат. вислів — значить: з добрим довір’ям; тут стосується до ґарантій і означає
«надійні», «певні» ґарантії. Пр. Ред.
}, або ж вони такі дурні, що гадають начебто банк має обов’язок та силу
папірцем перетворювати усіх збанкрутованих спекулянтів у виплатоздатних солідних
капіталістів. Ті, хто каже, що є лише недостача капіталу, або просто
крутять словами, бо саме за таких часів у наслідок надмірного довозу та надпродукції
є маси капіталу неперетворного в гроші, або вони говорять тільки
про тих лицарів кредиту, що в дійсності опинилися в таких умовах, коли вони,
не одержуючи вже більше чужого капіталу для своїх операцій, вимагають
отже, щоб банк не тільки поміг їм виплатити втрачений капітал, а ще й дав
їм змогу провадити далі спекулятивні операції.

Основою капіталістичної продукції є те, що гроші, як самостійна форма вартости,
протистоять товарові, або що мінова вартість мусить одержати в грошах
самостійну форму, а це можливо лише тоді, коли певний товар стає тим матеріялом,
що в його вартості виміряється всі інші товари, що цей товар саме тому
стає загальним товаром, товаром par excellence протилежно до всіх інших товарів.
Це мусить виявлятись у двох напрямах, особливо в капіталістично розвитих
націй, які до значної міри заміняють гроші, з одного боку, кредитовими операціями,
з другого боку — кредитовими грішми. Підчас пригнічення, коли кредит
скорочується або зовсім припиняється, гроші раптом стають абсолютно проти
всіх товарів як єдиний платіжний засіб та справжня форма буття вартости.
Відси загальне знецінювання товарів, трудність, навіть неможливість перетворення
їх у гроші, тобто в їхню власну суто-фантастичну форму. Але, подруге: самі
кредитові гроші лише остільки є гроші, оскільки вони заступають абсолютно
дійсні гроші на суму їхньої номінальної вартости. З відпливом золота можливість
перетворення їх у гроші стає проблематичною, тобто стає проблематичною
їхня тотожність з дійсним золотом. Відси примусові заходи, підвищення рівня