\parcont{}  %% абзац починається на попередній сторінці
\index{iii2}{0062}  %% посилання на сторінку оригінального видання
реґулюватись імпортом та експортом благородних металів або вексельним курсом.
Помилкове припущення Рікардо, ніби золото є тільки монета, що отже все імпортоване
золото збільшує кількість грошей в циркуляції, а тому й підносить ціни,
а все експортоване золото зменшує кількість монет, а тому й знижує ціни —
це теоретичне припущення стає тут практичною спробою лишати стільки
монети в циркуляції, скільки є в наявності золота в кожний даний
момент. Лорд Оверстон (банкір Jones Loyd), полковник Torrens, Normam, Clay,
Arbuthnot та деякі інші письменники, відомі в Англії під назвою школи «Currency
Principle», не тільки проповідували цю доктрину, але й за допомогою банкових
актів сера Піля з 1844 та 1845 р. р. зробили її основою англійського та шотляндського
банкового законодавства. Її ганебне теоретичне й практичне фіяско після
експериментів найбільшого національного маштабу ми зможемо розглянути лише
в науці про кредит» (І. с. р. 165—168*).

Критику цієї школи подали Томас Тук, Джемс Вілсон (в Economist’i 1844—47 р. р.)
та Джон Фулартон. Але як хибно також і вони розуміли природу золота
та як неясно було їм відношення між грішми та капіталом, це ми бачили не
раз, а саме в розділі XXVIII цієї книги. Тут подамо ще деякі матеріяли з дебатів
комісії нижчої палати в 1857 році про банкові акти Піля. (В. С. 1857) — Ф. Е.].

I. G. Hubbard, колишній управитель Англійського банку, свідчить: «2400. —
Вивіз золота... абсолютно не впливає на товарові ціни. Навпаки, він дуже
значно впливає на ціни цінних паперів, бо в міру того, як змінюється рівень
проценту, неминуче справляється незвичайний вплив на вартість товарів, що втілюють
цей процент». Він подає про роки 1834—43 та 1844—56 дві таблиці,
які доводять, що рух цін п’ятнадцятьох найзначніших торговельних речей був
цілком незалежний від відпливу та припливу золота й від рівня проценту. Але
зате ці таблиці доводять щільний зв’язок між відпливом та припливом золота,
що дійсно «є представник нашого капіталу, який шукає приміщення», і рівнем
проценту. — «В 1847 році дуже велику суму американських цінних паперів переслали
назад до Америки, так само й російські цінні папери — до Росії, а інші
континентальні папери — до тих країн, звідки ми імпортували збіжжя».

15 головних товарів, покладених в основу далі поданих таблиць Hubbard’a,
є такі: бавовна, бавовняна пряжа, бавовняні тканини, вовна, сукно,
льон, полотно, індиґо, чавун, бляха, мідь, волове сало, цукор, кава, шовк.

І. 1834—1843

Час        Металева готівка банку ф. ст. Ринкова  норма  дисконту  в\%     З 15 головних товарів
Піднеслися ціною      Впали     Без змін
1834, 1    березня   9104000    2 3/4 ———
1835, 1    »    6274000    3 3/4    7    7    1
1836, 1    »    7918000    3 1/4    11    3    1
1837, 1    »  4079000    5    5    9    1
1838, 1    »   10471000    2 3/4    4    11 —
1839, 1    вересня   2684000    6    8    5    2
1840, 1    червня  4571000    4 3/4    5    9    1
1840, 1    грудня 3642000    5 3/4    7    6    2
1841, 1    »    4873000    5    3    12 —
1842, 1    »    10603000    2 1/2    2    13 —
1843, 1    червня  11566000    2 1/4    1    14 —

*) Україн. видання, 1926. ДВУ ст. 192—194. Прим. Ред.
