Коли звести заробітну плату до її загальної основи, тобто до тієї частини
продукту власної праці, що входить в особисте споживання робітника: коли
звільнити цю частину від капіталістичних обмежень і поширити споживання до
такого розміру, який, з одного боку, допускається наявною продуктивною силою
суспільства (тобто суспільною продуктивною силою його власної праці як дійсно
суспільної) і якого, з другого боку, потребує цілковитий розвиток індивідуальности; коли звести далі
додаткову працю, й додатковий продукт до тих розмірів,
які при даних суспільних умовах продукції потрібні, з одного боку, для створення
страхового і резервного фонду, з другого боку, для безупинного поширення
репродукції в тій мірі, що визначається суспільною потребою; коли включити
нарешті, в № 1, в потрібну працю, і в № 2, в додаткову працю, ту кількість
праці, що її мусять завжди виконувати працездатні члени суспільства на ще
непрацездатних або вже непрацездатних членів суспільства; отже, коли таким
чином усунути всі специфічно капіталістичні риси так у заробітній платі, як і в
додатковій вартості, так у потрібній, як і в додатковій праці, — тоді перед
нами залишаться вже не ці форми, а лише їхні основи, спільні всім суспільним
способам продукції.

Проте, треба сказати, що таке підведення було властиве і колишнім панівним
способам продукції, наприклад, февдальному. Продукційні відносини, що
цілком не відповідали йому, стояли цілком поза ним, підводились під февдальні
відносини, наприклад в, Англії tenures in common socage\footnote*{
Володіння на основі панщини. Прим. Ред.
} (протилежно до tenures
оn knight’s service)\footnote*{
Володіння на основі рабськоі праці. Прим. Ред.
}, які мали в собі виключно грошові зобов’язання
лише з назви були февдальними.

Розділ п’ятдесят перший.

Розподільчі відносини й продукційні відносини.

Вартість, новостворювана щорічно нововитрачуваною працею, — отже, і та
частина річного продукту, що в ній визначається ця вартість і яка може бути
вилучена, виділена з сукупного продукту, — розпадається, отже на три частини,
що набувають трьох різних форм доходу, форм, які виражають одну частину
цієї вартости, як належну або взагалі припалу посідачеві робочої
сили, другу — посідачеві капіталу, третю — посідачеві земельної власности.
Отже, це є відносини або форми розподілу, бо вони визначають ті відносини,
що в них сукупна новоспродукована вартість розподіляється між посідачами
різних чинників продукції.

Згідно з звичайним поглядом ці відносини розподілу виступають як природні
відносини, відносини, що виникають з природи всякої суспільної продукції, з
законів людської продукції взагалі. Хоч і немає можливости заперечувати, що
докапіталістичні суспільства виявляють інші способи розподілу, проте, ці останні
тлумачаться, як нерозвинені, недосконалі й замасковані, що не досягли свого
найчистішого виразу і своєї найвищої форми, своєрідно забарвлені різностаті
цих природних розподільчих відносин.

В такому уявленні правильне лише одно: коли дано суспільну продукцію,
хоч би якого роду (наприклад, коли дано суспільну продукцію природно вирослої
індійської громади або більш штучно розвиненого перуанського комунізму),
то завжди можна відрізнити ту частину праці, що її продукт безпосередньо
особисто споживається продуцентами та їхніми    родинами, — лишаючи
осторонь працю, що припадає продуктивному споживанню, — від тієї