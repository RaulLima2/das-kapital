Таблиця Xа

Рід землі
Акри
Капітал Ф. ст.
Зиск Ф. ст.
Ціна продукції Ф. ст.
Продукт в квартерах
Продажна ціна Ф. ст.
Здобуток Ф. ст.
Рента Кварт. Ф. ст.
Підвищення

а  1                5            1  6                  1 1/8   5 1/3   6            0          0    
      0
A  1  2 1/2 + 2 1/2     1  6  1 +  1/4 =1 1/4   5 1/3   6 2/3     1/8       2/3       2/3
B  1  2 1/2 + 2 1/2     1  6  2 + 1/2 = 2 1/2   5 1/3   13 1/3   1 3/8    7 1/3    2/3 + 6 2/3
C  1  2 1/2 + 2 1/2     1  6  3 + 3/4 = 3 3/4   5 1/3   20          2 5/8   14         2/3 + 2 × 6
2/3
D  1  2 1/2 + 2 1/2     1  6  4 + 1 = 5             5 1/3   26 2/3   3 7/8   20 2/3   2/3 + 3 × 6
2/3
                                        30             13 5/8                72 2/3   8          42
2/3

Приєднанням землі а породжується нову диференційну ренту І; на цій
новій основі розвивається потім диференційна рента II теж у зміненому вигляді.
Земля а має в кожній з трьох вищенаведених таблиць ріжну родючість; ряд
відповідно висхідних ступенів родючости починається лише з А. Відповідно до
цього розміщується і ряд висхідних рент. Рента з найгіршої рентодайної землі,
що раніш ренти не давала, становить постійну величину, яка просто приєднується
до всіх вищих рент; лише за вирахуванням цієї сталої величини ясно виступає
при порівнянні вищих рент ряд ріжниць і його паралелізм з рядом, що
визначає родючість різних земель. У всіх таблицях різні ступені родючости, починаючи
з А до D, стосуються один до одного, як 1: 2 : 3 : 4, і відповідно до
цього ренти стосуються одна до однієї:

в VIIa, як 1 : 1 + 7 : 1 + 2 × 7 : 1 + 3 × 7,
в VIIIa, як 1 1/5 : 1 1/5 + 7 1/5 : 1  1/5 + 2 × 7 1/5 : 1 1/5 + 3 × 7 1/5,
в Xa, як   2/3 : 2/3 + 6 2/3 : 2/3 + 2 × 6 2/3 : 2/3 + 3 × 6 2/3.

Коротко: коли рента з А = n, а рента з землі безпосередньо вищої родючости
= n + m, то ряд буде такий: n: n + m: n + 2m : n + З m і т. д. — Ф. Е.]

[А що вищенаведений третій випадок в рукопису не був опрацьований —
там є лише його заголовок, — то завдання редактора було по змозі доповнити
це, як зроблено вище. Але йому лишається ще зробити загальні висновки, що
випливають з усього попереднього дослідження диференційної ренти II в її трьох
головних випадках і дев’ятьох похідних випадках. Але для цієї мети наведені
в рукопису випадки придаються лише дуже мало. По-перше, в них порівнюються
дільниці землі, що з них здобутки для площ однакової величини стосуються
як 1: 2 : 3 : 4; отже, беруться ріжниці, що вже від самого початку дуже перебільшені,
і які в дальшому розвитку зроблених на цій основі припущень і обчислень
призводять до цілком насильницьких числових відношень. Але подруге,
