криз — той прегарний теоретичний дуалізм. Поки освічена економія розглядає
«капітал» ex professo 1), вона з найбільшою зневагою позирає на золото й срібло,
як на таку форму капіталу, що в дійсності є найбайдужіша та найнепотрібніша.
Скоро вона починає розглядати банкову справу, все це зміняється, і золото та
срібло стають капіталом par excellence, що для його збереження треба жертвувати
всяку іншу форму капіталу й праці. Чим же відрізняються золото й
срібло від інших форм багатства? Не величиною вартості, бо ця остання визначається
кількістю зречевленої в них праці. А лише як самостійні втілення, як вияви
суспільного характеру багатства. [Багатство суспільства існує лише як багатство
поодиноких людей, що є його приватні власники. Воно зберігає свій суспільний
характер лише тому, що ці поодинокі люди для задоволення своїх потреб
обмінюють одні на одні споживчі вартості, відмінні якістю. За капіталістичної
продукції вони можуть це робити лише за посредництвом грошей. Таким
чином лише за посредництвом грошей багатство поодинокої людини здійснюється як
суспільне багатство: в грошах, в цій речі, втілено суспільну природу цього багатства.
— Ф. Е.]. Отже, це його суспільне буття виявляється, як щось потустороннє,
як річ, як предмет, як товар поряд та поза межами дійсних елементів суспільного
багатства. Поки продукція розвивається плавко, це забувають. Кредит, як теж
суспільна форма багатства, витискує гроші та узурпує їхнє місце. Саме довір’я
до суспільного характеру продукції призводить до того, що грошова форма продуктів
виявляється як дещо лише минуще та ідеальне, як просте уявлення. Але
скоро кредит розхитано — а ця фаза постає неминучо в циклі сучасної промисловости,
— то тоді усе реальне багатство повинно дійсно та раптом перетворитись
на гроші, на золото й срібло, — божевільне домагання, що, однак, неминуче виростає
з самої системи. І усе золото й срібло, що має задовольнити ці величезні
домагання, становить кілька мільйонів у льохах банку\footnote{
«Ви цілком погоджуєтеся з тим, що для зміни попиту на золото немає жодного шляху, крім
підвищення
рівня проценту? — Chapman [спільник великої bill-broker’івської фірми Overend Gurney and Co]:
Це — мій погляд. Коли кількість нашого золота спадає до певної міри, то ми найкраще зробимо, якщо
одразу задзвонимо на сполох та скажемо: ми занепадаємо, і хто відправляє золото закордон, мусить
робити це на свій риск». — В. А. 1857, Evid. № 5057.
}. Отже, у впливах,
зумовлених цим відпливом золота яскраво виступає та обставина, що продукцію,
як суспільну продукцію, в дійсності не підбито суспільному контролеві, —
виступає яскраво у тій формі, що суспільна форма багатства існує як річ поза
ним. В дійсності це явище є спільне капіталістичній системі і попереднім системам
продукції, оскільки вони спирались на товарову торговлю та приватний
обмін. Але лише в капіталістичній системі воно виявляється найяскравіше та
в найдивніший формі абсурдної суперечности та безглуздя, бо: 1) в капіталістичній
системі найповніше усунено продукцію для безпосередньої споживчої
вартости, продукцію для власного спожитку продуцентів, отже, багатство існує
лише як суспільний процес, що виявляється, як сплетіння продукції й циркуляції;
2) бо з розвитком кредитової системи капіталістична продукція постійно
намагається усунути цю металеву межу, цю одночасно речову та фантастичну
межу багатства й його руху, але раз-у-раз розбиває собі голову об цю межу.

Підчас кризи постає домагання, щоб усі векселі, цінні папери, товари
одразу одночасно перетворювались на банкові гроші, а ці банкові гроші знову
на золото.

II. Вексельний курс

Як відомо, за барометр міжнароднього руху грошових металів є вексельний
курс. Коли Англія має платити Німеччині більше, ніж Німеччина

1) Робити щось ex professo значить робити, як людина цілком обізнана з своїм предметом. В даному
разі, точніше — з погляду власних засад. Пр. Ред.