\parcont{}  %% абзац починається на попередній сторінці
\index{iii2}{0273}  %% посилання на сторінку оригінального видання
частини праці, яка завжди становить додаткову працю, що її продукт завжди
служить задоволенню загальних суспільних потреб, хоч би як розподілялось
цей додатковий продукт, і хоч би хто функціонував як представник цих суспільних
потреб. Отже, тотожність різних способів розподілу зводиться до того,
що вони тотожні, коли ми абстрагуємось від їхніх ріжниць і специфічних форм
і фіксуємо увагу тільки на їхніх спільних рисах, протилежно їхнім ріжницям.

Розум розвиненіший, критичніший, визнає\footnote{
J. Stuart Mill: Some Unsettled Questions in Pol. Econ. London 1844.
}, правда, історично розвинений
характер розподільчих відносин, але з то більшою упертістю тримається за незмінний,
що виникає з людської природи, а тому й незалежний від усякого історичного
розвитку, характер самих відносин продукції.

Навпаки, наукова аналіза капіталістичного способу продукції показує, що
він становить спосіб продукції особливого роду, спосіб продукції, що має специфічно
історичну визначеність; що він, як і всякий інший певний спосіб продукції,
має своєю передумовою даний ступінь суспільних продуктивних сил і
форм їхнього розвитку, як свою історичну умову, — умову, що сама є історично
наслідок і продукт попереднього процесу, і що з неї як своєї даної основи
походить новий спосіб продукції; що відповідні до цього специфічного, історично
визначеного способу продукції продукційні, відносини, — відносини, в які стають
люди у своєму суспільному життєвому процесі в створенні свого суспільного
життя, — мають також специфічний, історичний та минущий характер; що, нарешті,
відносини розподілу, істотно тотожні з цими продукційними відносинами,
являють собою лише зворотний бік їх, так що ті і ті мають той самий
історично минущий характер.

При розгляді розподільчих відносин виходять насамперед з того позірного
факту, що річний продукт розподіляється як заробітна плата, зиск і земельна
рента. Але формульований таким чином цей факт є невірний. Продукт поділяється,
з одного боку, на капітал, з другого боку, на доходи. Один з цих доходів,
заробітна плата, сам завжди набуває лише форми доходу, доходу робітника, по
тому, як він раніш протистояв тому самому робітникові у формі капіталу.
Протиставлення випродукованих умов праці і продуктів праці взагалі, як капіталу
безпосереднім продуцентам, зрозуміло, вже включає певний суспільний характер
речових умов праці проти робітників, а тим самим і певне відношення, що
в нього вони вступають у самій продукції до посідачів умов праці і один
до одного. Перетворення цих умов праці на капітал своєю чергою включає
експропріацію землі, належної безпосереднім продуцентам, отже, певну форму
земельної власности.

Коли б одна частина продукту не перетворювалась на капітал, то друга
не могла б набути форми заробітної плати, зиску або ренти.

З другого боку, коли капіталістичний спосіб продукції має за передумову цей
певний суспільний характер умов продукції, то він одночасно безупинно репродукує
його. Він не тільки продукує матеріяльні продукти, але й безупинно репродукує
і ті продукційні відносини, що в них ці продукти продукуються,
репродукує, отже, і відповідні розподільчі відносини.

Можна звичайно сказати, що сам капітал (і земельна власність, яку
він включає в себе як свою протилежність) вже має за передумову певний
розподіл: експропріяцію в робітника умов його праці, концентрацію
цих умов в руках меншости індивідуумів, виключну власність на землю
інших індивідуумів, коротко кажучи, всі ті відносини, що їх викладено в розділі
про первісну акумуляцію (книга І, розділ XXIV). Але цей розподіл цілком
відмінний від того, що розуміють під розподільчими відносинами, коли приписують
їм, протилежно до продукційних відносин, історичний характер. При
\parbreak{}  %% абзац продовжується на наступній сторінці
