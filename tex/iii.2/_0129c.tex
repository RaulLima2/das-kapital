\parcont{}  %% абзац починається на попередній сторінці
\index{iii2}{0129}  %% посилання на сторінку оригінального видання
що перша земля А поліпшилась в наслідок постійного раціонального обробітку,
або що вона при незмінності витрат стала продуктивніше оброблятися,
наприклад, в наслідок заведення конюшини тощо, так що її продукт, за незмінної
величини авансованого капіталу, збільшився до 11/3 кварт. Припустімо
далі, що землі В, С, В, як і давніш, дають ту саму кількість продукту, але що
почали оброблятися нові землі А' пересічної між А і В родючости, далі В', В", що
містяться своєю родючістю між В і С; в цьому випадку постали б такі явища:

Перше: ціна продукції квартера пшениці, або її реґуляційна ринкова
ціна, зменшилась би з 60 до 45 шил., або па 25\%.

Друге: відбувся б одночасний перехід від родючішої землі до менш
родючої, і від менш родючої землі до родючішої. Земля А' родючіша, ніж А, але
менш родюча, ніж В, С, D, що оброблялись до цього часу; а В', В" родючіші, ніж
А, А' і В, але менш родючі, ніж С і D. Отже, перехід від однієї землі до другої
відбувався б у всіх напрямках; відбувся б перехід не до абсолютно
менш родючої землі проти А тощо, а до відносно менш родючої, порівняно
з землями С і D, які до цього часу були найродючіші; з другого боку, перехід
відбувався б не до абсолютно родючішої землі, а до відносно родючішої проти
земель А, — або А і В, — які до цього часу були найменш родючі.

Третє: Рента з В знизилася б; а також рента з С і D; але загальна
сума ренти, визначена в збіжжі, піднеслась би з 6 до 72/3 кв.; маса землі, що
обробляється і дає ренту, збільшилася б, а також збільшилася б і маса продукту
з 10 до 17 квар. Зиск, хоч він і лишився без перемін для А, визначений у
збіжжі, підвищився б; але можливо, що навіть норма зиску підвищилася б, бо
підвищилася б відносна додаткова вартість. В цьому випадку в наслідок здешевлення
засобів
існування
зменшилася б
заробітна плата,
отже, витрата
на змінний
капітал,
отже, і загальні
витрати. Вся
сума ренти,
визначена в
грошах, знизилась
би з
360до345шил.

Подаємо
нову послідовність
переходу
(див. табл. II).

Нарешті,
коли б, як і

давніш, оброблялись тільки землі А, В, С, D, але продуктивність їхня зросла б
остільки, що земля А замість 1 квартера давала б 2, В замість 2 квартерів — 4,
С замість 3 квартерів — 7 і D замість 4 квартерів — 10, отже, коли б ті самі
причини по-різному вплинули б на різні землі, то вся продукція підвищилася
б з 10 до 23 квартерів. Припустімо, що попит в наслідок приросту
людности і пониження ціни поглинув би ці 23 квартери, в такому разі ми
мали б такий результат (див. табл. III на ст. 130).

Числові відношення тут, як і в попередніх таблицях, довільні, але припущення
цілком раціональні.

Таблиця II

Рід  землі    Продукт        Витрата  капіталу    Зиск    Рента    Ціна про-дукції квартера
    Квар-тери    Шилінґи        Квар-тери    Шилінґи    Квар-тери    Шилінґи
А    11/3    60    50    2/9           10 — —           45 шил.
А'    12/3    75    50    5/9           25    1/3          15    36»
B    2         90    50    8/9           40    2/3           30    30»
В'    21/2    105    50    12/9    55    1           45    252/7»
В”    22/3    120    50    15/9    70    11/3    60    221/2»
C     3       135    50    28/9    85    12/3    75    20»
D    4     180    50              130    22/3    120    15» 1
Разом 17                        72/3    345
