В. Коли гірша (позначувана літерою а) земля стає землею, яка реґулює
ціну і через те А починає давати ренту. Де не виключає можливости незмінюваної
продуктивности другої витрати для всіх варіянтів.

Варіянт 1: Незмінювана продуктивність другої витрати капіталу.

Таблиця XXII.

Рід  землі
Ціна продукції. Шил.
Продукт. Бушелі
Продажна  ціна. Шил.
Здобуток. Шилінґи
Рента. Шил.
Підвищення ренти
а    120                                               16    7 1/2    120          0    0
А    60 + 60 = 120    10 + 10 = 20    71/2    150         30    30
В    60 + 60 = 120      12 + 12 = 24    71/2    180         60    2 × 30
C    60 + 60 = 120    14 + 14 = 28    71/2    210         90    3 × 30
D    60 + 60 = 120    16 + 16 = 32    71/2    240        120    4 × 30
Е    60 + 60 = 120    18 + 18 = 36    71/2    270        150    5 × 30
                    450    15 × 30

Варіянт 2: Низхідна продуктивність другої витрати капіталу.
  Таблиця XXIII.

Рід  землі
Ціна продукції. Шил.
Продукт. Бушелі
Продажна  ціна. Шил.
Здобуток. Шилінґи
Рента. Шил.
Підвищення ренти
а                       120                        15            8    120            0    0
А 60 + 60 = 120 10 + 7 1/2 = 17 1/2 8 140 20 20
В 60 + 60 = 120 12 + 9 = 21 8 168 48 20 + 28
С 60 + 60 = 120 14 + 101/2 = 241/2    8    196            76    20 + 2 × 28
D 60 + 60 = 120 16 + 12 = 28 8 224 104 20 + 3 × 28
Е 60 + 60 = 120 18 + 131/2 = 311/2    8    252            132    20 + 4 × 28
                    380    20 × 5 + 10 × 28

Варіянт 3: Висхідна продуктивність другої витрати капіталу.

 Таблиця XXIV.

Рід  землі
Ціна продукції. Шил.
Продукт. Бушелі
Продажна  ціна. Шил.
Здобуток. Шилінґи
Рента. Шил.
Підвищення ренти
а                       120                           16              71/2    120            0      
          0
А    60 + 60 = 120    10 + 121/2 = 221/2     71/2    1683/4    483/4    15 + 333/4
В    60 + 60 = 120    12 + 15 = 27                      71/2    2021/2    821/2    15 + 2 × 333/4
С    60 + 60 = 120    14 + 171/2 = 311/2      71/2    2361/4    1161/4    15 + 3 × 333/4
D    60 + 60 = 120    16 + 20 = 36                      71/2    270            150            15 + 4
× 333/4
Е    60 + 60 = 120    18 + 221/2 = 401/2      71/2    3033/4    1833/4    15 + 5 × 333/4
                    5811/4    5 × 15 + 15 × 333/4
