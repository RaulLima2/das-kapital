стерл.) становить 240\% замість 180\%. Увесь продукт збільшився з 10 до
36 квартерів.

Порівняно з Іb, де загальне число оброблених акрів, застосований капітал
і ріжниці між обробленими родами землі лишились ті самі, але розподіл
їх інший, продукт становить 36 квартерів замість 26 кварт., пересічна рента
з акра становить 6 ф. стерл. замість 3 ½ і норма ренти щодо всього авансового
капіталу тієї самої величини — 240\% замість 140\%.

Хоч як би ми стали розглядати різні становища, подані в таблицях Іа,
Іb, Іс, чи як становища, що одночасно існують одно біля одного в різних країнах,
чи як послідовні становища в тій самій країні, — однаково, виявиться таке:
за сталої ціни збіжжя, сталої тому, що продукт з найгіршої землі, яка не дає
ренти, лишається той самий; за незмінної різниці у родючости різних розрядів
оброблюваної землі; при відносно однаковій кількості продукту, а, отже,
при однаковій витраті капіталу на відповідно однакові частини (акри) земельної
площі, оброблюваної в кожному розряді; при сталому в наслідок цього
відношенні між рентами з акра кожного роду землі і при однаковій нормі ренти
на капітал, вкладений у кожну дільницю землі того самого роду — виявиться,
поперше, що сума ренти завжди зростає, разом з поширенням оброблюваної
площі, а тому із збільшенням витрати капіталу, за винятком того випадку, коли
весь приріст припав би на землю, що не дає ренти. Подруге, так пересічна
рента на акр (загальна сума ренти, поділена на все число оброблюваних
акрів), як і пересічна норма ренти (загальна сума ренти, поділена на ввесь
витрачений капітал) можуть значно варіювати, і хоч обидві в одному напрямку,
але в різних пропорціях у відношенні одна до однієї. Якщо не брати
на увагу того випадку, коли приріст відбувається лише на землі кг
яка не дає ренти, то виявляється, що пересічна рента на акр і пересічна,
норма ренти на капітал, вкладений у хліборобство, залежить від того, які пропорційні
частини всієї оброблюваної землі становлять землі різних розрядів;
або, що схопіть на те саме, від розподілу всього застосованого капіталу між
землями різної родючости. Чи багато, чи мало землі обробляється і тому (за
винятком того випадку, коли приріст припадає лише на А) чи більша, чи
менша є загальна сума ренти, пересічна рента на акр або пересічна рента на
застосований капітал лишається та сама, доки відношення різних родів оброблюваної
землі до всієї її площі лишається те саме. Дарма, що з поширенням
культури і збільшенням застосованого капіталу відбувається підвищення і
навіть значне підвищення загальної суми ренти, пересічна рента на акр і
пересічна норма ренти на капітал понижується, коли поширення земельних
дільниць, що не дають ренти, або дають лише незначну диференційну ренту,
зростає швидше, ніж поширення кращих земельних дільниць, що дають більшу
ренту. Навпаки, пересічна рента на акр і пересічна норма ренти на капітал підвищується в міру того,
як кращі землі починають становити відносно більшу
частину всієї площі, і тому на них припадає відносно більше застосованого
капіталу.

Таким чином, коли розглядати пересічну ренту на акр або гектар усієї
оброблюваної землі, як це звичайно робиться в статистичних працях при порівнянні
різних країн за тієї самої доби, або різних діб у тій самій країні, та
виявляється, що пересічна висота ренти на акр, а тому і загальна сума ренти
в певних пропорціях (хоч зовсім не в тих самих, а в швидше ростучих) відповідає
не відносній, а абсолютній родючості хліборобства в країні, тобто відповідає,
масі продуктів, одержуваній пересічно з однакової земельної площі. Бо що
більшу частину із загальної площі становлять кращі землі, то більша маса
продуктів, одержувана з земельної площі однакової величини за однакової величини
застосованого капіталу, і то більша пересічна рента на акр. Зворотне в зворот-
