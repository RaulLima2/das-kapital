\parcont{}  %% абзац починається на попередній сторінці
\index{iii2}{0179}  %% посилання на сторінку оригінального видання
частку всього витраченого капіталу становить цей менш продуктивний капітал,
а з другого боку, відповідно до зменшення його продуктивности. Пересічна
ціна продукту цього менш продуктивного капіталу все ще була б нижча
від регуляційної ціни, і тому все ще лишався б надзиск, який міг би перетворитися
на ренту.

Припустімо тепер, що пересічна ціна квартера з землі В збігається з загальною
ціною продукції, в наслідок чотирьох послідовних витрат капіталу
(2  1/2, 2  1/2, 5 і 5 ф. ст.) з низхідною продуктивністю:

Капітал Зиск Здобуток в кварт. Ціна продукції Продажна ціна Здобуток Додаток до ренти
 Фунт. стерл. Фунт. стерл.  За квартер разом Фунт. стерл. Фунт. стерл. Кварт. Фунт. стерл.
    Фунт. стерл. Фунт. стерл.
1)
2)
3)
4) 2 ½
2 ½
5
5 ½
½
1
1 2
1 ½
1 ½
1
 1 ½
2
4
6 3
3
6
6 3
3
3
3 6
4 ½
4 ½
3

 1
½
-½
-1
 3
1½
-1½
-3
 15 3 6  18  18 0 0

Тут орендар продає кожен квартер по його індивідуальній ціні продукції,
і тому все число квартерів продає він по їхній пересічній ціні продукції квартера,
яка збігається з регуляційною ціною в 3 ф. стерл. Він одержує тому на свій
капітал в 15 ф. стерл., як і давніш, 20\% зиску = З ф. стерл. Але рента зникла.
Куди ж дівся надмір при цьому вирівнянні індивідуальних цін продукції кожного
квартера з загальною ціною продукції?

Надзиск з перших 2  1/2 ф. стерл. був 3 ф. стерл.; з других 2  1/2 ф. стерл.
він був 1 1/2 ф. стерл.; разом надзиск на  1/3 авансованого капіталу, тобто на
5 ф. стерл. = 4  1/2 ф. стерл. = 90\%.

При витраті капіталу 3) 5 ф. стерл. не тільки не дають надзиску, але
продукт їхній в 1 1/2 квартера, проданий по загальній ціні продукції, дає мінус в
1  1/2  ф. стерл. Нарешті, при витраті капіталу 4) теж в 5 ф. стерл., продукт
їхній в 1 кв., проданий по загальній ціні продукції, дає мінус в 3 ф. стерл. Отже,
обидві витрати капіталу, взяті разом, дають мінус в 4  1/2 ф. стерл., рівний надзискові
в 4  1/2  ф. стерл., який постав від витрат капіталу 1) і 2).

Надзиск і мінус-зиск урівноважуються. Тому рента зникає. Але в дійсності
це можливе тому, що елементи додаткової вартости, які раніш становили
надзиск або ренту, входять тепер в створення пересічного зиску. Орендар одержує
цей пересічний зиск в розмірі 3 ф. стерл. на 15 ф. стерл., або в розмірі
20\% коштом ренти.

Вирівняння індивідуальної пересічної ціни продукції з землі В з загальною
ціною продукції А, яка регулює ринкову ціну, має за передумову, що ріжниця,
на яку індивідуальна ціна продукту перших витрат капіталу нижча,
ніж регуляційна ціна, дедалі більше зрівноважується і нарешті знищується
ріжницею, на яку продукт пізніших витрат капіталу починає перебільшувати
регуляційну ціну. Те, що являє собою надзиск, поки продукт перших витрат
капіталу продається сам по собі, в такий спосіб поступово стає частиною їхньої
пересічної ціни продукції, і разом з тим входить в створення пересічного зиску,
аж поки, нарешті, не буде зовсім поглинуте цим пересічним зиском.

Коли б замість вкладати в землю В 15 ф. стерл. капіталу, в неї було вкладено
лише 5 ф. стерл. і додаткові 2  1/2 квартери останньої таблиці були випродуковані
\parbreak{}  %% абзац продовжується на наступній сторінці
