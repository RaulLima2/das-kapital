залежно від специфічної родючости землі кожного типу, пропорційно величині
додаткового капіталу. В XXXIX розділі ми виходили з такої таблиці І:

Таблиця І

Рід землі    Акри Капітал    Ціна продукц. Продукт в кварт.
Продажна  ціна    Здобуток    Рента    Норма  надзиску
        1 ф. ст. ф. ст. ф. ст. ф. ст. ф. ст. Кварт. ф. ст.
А....    1    2 1/2    1/2    3    1    3    3    0    0    0
В....    1    2 1/2    1/2    3    2    3    6    1    3    120\%*)
C....    1    2 1/2    1/2    3    3    3    9    2    6    240\%
D....    1    2 1/2    1/2    3    4    3    12    3    9    360\%
Разом. .4    10               12    10        30    6    18

Тепер ця таблиця перетворюється на:

Таблиця II

Рід  землі    Акри  Капітал    Зиск    Ціна продукції Продукт в кварт.
Продажна  ціна    Здобуток    Рента    Норма  надзиску
        ф. ст. ф. ст. ф. ст. ф. ст. ф. ст. Кварт. ф. ет.
А. . .    1    2 1/2 + 2 1/2 = 5    1    6    2    3    6    0    0    0
В. . .    1    2 1/2 + 2 1/2 = 5    1    6    4    3    12    2    6    120\%
С. . .    1    2 1/2 + 2 1/2 = 5    1    6    6    3    18    4    12    240\%
D. . .    1    2 1/2 + 2 1/2 = 5    1    6    8    3    24    6    18    360\%
Разом.    4          20            20        20            60    12    36

Тут немає потреби в тому, щоб капітал вкладати у кожний з типів землі
в подвоєному розмірі, як це є в таблиці. Закон лишається той самий, скоро
тільки на якийсь один або декілька родів землі, що дають ренту, вжито додатковий капітал, хоч би в
якому розмірі. Треба лише, щоб продукція на землях
кожного роду збільшувалася в тому самому відношенні, в якому збільшується
капітал. Рента підвищується тут виключно в наслідок збільшення вкладеного
в землю капіталу і відповідно до цього збільшення капіталу. Це збільшення
продукту і ренти, в наслідок збільшення вкладеного капіталу і пропорційно
йому, є,  щодо кількости продукту і ренти, цілком таке саме, як у тому випадку,
коли оброблювана площа рівних за якістю дільниць землі, що дають ренту,
збільшилася б, оброблючись з такою самою витратою капіталу, з якою давніш оброблялись земельні
дільниці тієї самої якости. В випадку, поданому в таблиці II, наприклад, наслідок був би той самий,
коли б додатковий капітал,
в 2 1/2 ф. стерл. на акр було вкладено в другі акри земель B, C і D.

*) В німецькому тексті тут стоїть «12\%, 24\%, 36\%». Очевидна помилка. Прим. Ред.
