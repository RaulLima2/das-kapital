ному випадку. Тому здається, що рента визначається не відношенням диференційної
родючости, а абсолютною родючістю, і що таким чином закон диференційної
ренти знищується. Тому деякі явища заперечуються, або їх намагаються
пояснити несущими різницями пересічних цін хліба і ріжницями диференційної
родючости оброблюваних дільниць землі, тимчасом, як ці явища ґрунтуються
просто на тому, що відношення загальної суми ренти так до всієї площі оброблюваної
землі, як і до всього капіталу, вкладеного в землю, за однакової родючости
землі, що не дає ренти, а тому і за однакових цін продукції і за
однакової ріжниці між землями різних родів визначаються не тільки рентою
на акр, або нормою ренти на капітал, але в такій же мірі відношенням числа
акрів кожного роду до загального числа оброблюваних акрів; або, що сходить
на те саме, розподілом усього застосованого капіталу між різними родами землі.
До цього часу на цю обставину, дивовижно, зовсім не звертали уваги. В усякому
разі виявляється, і це є важливе для нашого дальшого досліду, що відносна
висота пересічної ренти на акр і пересічна норма ренти, або відношення
загальної суми ренти до всього вкладеного в землю капіталу, може збільшуватися
або зменшуватися просто в наслідок екстенсивного поширення культури,
за незмінних цін, незмінної ріжниці в родючості оброблюваних дільниць землі
і незмінної ренти з акра, або норми ренти на капітал, вкладений в акр
кожного розряду землі, що дійсно дає ренту, або на ввесь капітал, що дійсно
дає ренту.

Треба зробити ще такі доповнення щодо тієї форми диференційної ренти,
яка досліджена в нас під рубрикою І, і що почасти мають також значіння і
для диференційної ренти II.

Перше: Ми бачили, як пересічна рента з акра або пересічна норма
ренти на капітал може підвищитись з поширенням культури, за сталих цін і
незмінної ріжниці в родючості оброблюваних земельних дільниць. Скоро вся
земля в будь-якій країні буде привласнена, вкладення капіталу в землю, культура
і людність досягнуть певної висоти — обставини, наявність яких доводиться
припускати, скоро капіталістичний спосіб продукції став панівним і упідлеглив
собі і хліборобство, — ціна необроблюваної землі різної якости (просто припускаючи
існування диференційної ренти) визначається ціною оброблюваних дільниць
землі однакової якости і рівноцінного положення. Ціна є така сама — за вирахуванням
витрат на обробіток, що приєднується до неї — хоч ця земля і не
дає ренти. Ціна землі, звичайно, є не що інше, як капіталізована рента. Але
і в ціні оброблених земельних дільниць оплачуються лише майбутні ренти, наприклад,
одним заходом виплачується наперед ренти за 20 років, коли міродайний
розмір проценту є 5%. Коли продається земля, вона продається як така, що дає
ренту, і перспективний характер ренти (яку розглядається тут як витвір землі, чим
вона є тільки з видимости) призводить до того, що необроблювана земля не
відрізняється від оброблюваної. Ціна необроблюваних дільниць землі, як і рента
з них, — а ціна становить лише зосереджену формулу ренти — має суто ілюзорний
характер, поки ці дільниці не будуть дійсно використані. Але вона, таким
чином, визначається a priori і реалізується, скоро знаходяться покупці. Тому,
коли дійсна пересічна рента в певній країні визначається дійсною пересічною
річною сумою ренти і відношенням цієї останньої до всієї оброблюваної площі,
то ціна необроблюваної частини земельної площі визначається ціною оброблюваної
і є тому лише відбиток вкладення капіталу та його наслідків на оброблюваних
земельних дільницях. А що за винятком найгіршої землі, землі усіх родів
дають ренту (а ця рента, як ми побачимо в рубриці II, зростає з масою капіталу
а з відповідною до цієї маси інтенсивністю культури), то і створюється таким чином
