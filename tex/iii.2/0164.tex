і низхідної ціни. Хоч в усіх цих випадках рента може залишитися без зміни
і може понизитись, вона понизилася б значніше, коли б додаткове вживання
капіталу, за інших незмінних обставин не зумовлювало збільшення родючости. Додаткове вкладення
капіталу тоді завжди є за причину відносної висоти ренти,
хоча б абсолютно вона й понизилась.

Розділ сорок третій.

Дифереційна рента II. — Третій випадок:
висхідна ціна продукції.

[Підвищення ціни продукції має за свою передумову, що продуктивність землі
найгіршої якости, що не дає ренти, зменшується. Ціна продукції, взята нами за
реґуляційну, може піднестися вище від 3 ф. ст. за кв., лише тоді, коли 2 1/2 ф. ст.,
витрачені на А, продукуватимуть менш за 1 квартер, або 5 ф. ст. менш за
2 квартери, або коли б довелося обробляти землю ще гіршої якости, ніж А.

За незмінної або навіть висхідної продуктивности другого вкладення капіталу
це було б можливе лише тоді, коли б продуктивність першого вкладення в 2 1/2 ф. с.
зменшилась. Цей випадок трапляється досить часто. Наприклад, коли виснажений
при поверховій оранці зверхній шар ґрунту дає при старій системі обробітку
дедалі менші врожаї, то витягнений на поверхню з допомогою глибшої
оранки нижній шар за раціонального обробітку починає давати вищі
урожаї, ніж давніш. Але цей сцеціяльний випадок, точно кажучи, сюди не
стосується. Пониження продуктивности першої витрати капіталу в 2 1/2 ф. ст. зумовлює для кращих
земель, навіть коли там припустити аналогічні відношення,
пониження диференційної ренти І; проте тут ми розглядаємо лише диференційну
ренту II. Але тому, що даний спеціяльний випадок не може статися, коли не
припускається існування диференційної ренти II і тому, що він в дійсності
становить відбитий вплив модифікації диференційної ренти І на диференційну
ренту II, то ми наведемо приклад цього випадку.

Таблиця VII

Рід землі
Акри
Капіталовкладення Ф. ст.
Зиск Ф. ст.
Ціна продукц. Ф. ст.
Продукт в квартерах
Продажна ціна Ф. ст.
Здобуток Ф. ст.
Рента
Збіжж. Кварт.
Грош. Ф. ст.
Норма ренти

А    1    2  1/2 + 2 1/2     1    6      1/2 + 1 1/4 = 1 3/4     3 3/7      6                0      
        0           0
в    1    2  1/2 + 2 1/2     1     6      1 + 2 1/2 = 3 1/2        3 3/7     12               1 3/4 
     6        120\%
С    1    2  1/2 + 2 1/2    1     6      1 1/2 + 3 3/4 = 5 1/4  3 3/7     18               3 1/2    
 12        240\%
D    1    2  1/2 + 2 1/2    1    6       2 + 5 = 7                       З 3/7      24             5
1/4      18        360\%
                      20                                              17 1/2                        
60             10 1/2      36      240\%*)

Грошова рента, як і грошовий здобуток, лишаються ті самі, що і в таблиці II.
Підвищена реґуляційна ціна продукції точнісінько покриває те, що втрачено
на кількості продукту; а що ця ціна продукції і кількість продукту змінюються
в зворотному відношенні, то само собою зрозуміло, що здобуток їх лишається
той самий.

*) Тут, як і далі в таблицях VIII, IX, і X пересічну норму ренти обчислено не до всього
вкладеного капіталу, а тільки до капіталу, вкладеного в рентодайні дільниці. Пр. Ред.
