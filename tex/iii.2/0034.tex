Отже, нагромадження позичкового грошового капіталу виявляє почасти
тільки той факт, що всякі гроші, на котрі промисловий капітал перетворюється
в процесі свого кругообороту, набирають форму не тих грошей, що їх
авансують капіталісти репродукції, а тих грошей, що їх вони собі позивають;
так що в дійсності авансування грошей, що мусить відбуватися в процесі
репродукції, видається авансуванням позичених грошей. В дійсності на
основі комерційного кредиту один визичає іншому гроші, що їх той потребує
в процесі репродукції. Але це набирає тепер таку форму, що банкір, якому
частина капіталістів репродукції позичає гроші, визичає їх другій частині
тих капіталістів, при чому цей банкір в такому разі видається благ подателем;
одночасно це набирає таку форму, що порядкування цим капіталом потрапляє
цілком до рук банкіра, як посередника.

Тепер треба навести ще деякі осібні форми нагромадження грошового
капіталу. Капітал стає вільним, напр., з причини спаду цін на елементи продукції,
на сирові матеріяли і т. ін. Якщо промисловець не може безпосередньо поширити
свій процес репродукції, то частину його грошового капіталу виштовхується
як зайву з кругообороту, і перетворюється на позичковий грошовий капітал.
Але, подруге, капітал звільняється у грошовій формі саме в купця, скоро
настають перерви в торговлі. Якщо купець зробив ряд справ і в наслідок таких
перерв може почати новий ряд їх тільки згодом, то реалізовані гроші становлять
для нього тільки скарб, надлишковий капітал. Однак одночасно вони безпосередньо
є нагромадження позичкового грошового капіталу. В першому випадку нагромадження
грошового капіталу означає повторювання процесу репродукції серед
сприятливіших умов, дійсне звільнення частини раніше зв’язаного капіталу,
отже, змогу поширювати процес репродукції тими самими грошовими засобами.
Навпаки, в другому випадку воно означає просту перерву в потоці операцій.
Але в обох випадках звільнений грошовий капітал перетворюється на позичковий
грошовий капітал, представляє нагромадження цього капіталу, впливає рівномірно
на грошовий ринок та на рівень проценту, дарма що в першому випадку
звільнення грошового капіталу означає сприятливі умови процесу дійсного нагромадження,
а в другому — гальмування його. Насамкінець, нагромадженню грошового
капіталу сприяють ті люди, що забезпечують собі певний дохід та усуваються від
репродукції. Що більше добувається зиску протягом промислового циклу, то більше
таких людей. Тут нагромадження позичкового грошового капіталу означає, з одного
боку, дійсне нагромадження (за його відносним обсягом); з другого боку,
лише обсяг перетвору промислових капіталістів на простих грошових капіталістів.

Щождо другої частини зиску, не призначеної для спожитку в формі доходу,
то вона перетворюється на грошовий капітал тільки тоді, коли її не можна
ужити безпосередньо до поширу підприємства в тій сфері продукції, де ту частину
зиску добувається. Це може поставати з двох причин. Або тому, що ця сфера
насичена капіталом. Або тому, що нагромадження для того, щоб мати змогу
функціонувати як капітал, мусить перше досягти певного обсягу, відповідно
до кількісних пропорцій приміщення нового капіталу в цьому певному підприємстві.
Отже, нагромадження передусім перетворюється на позичковий грошовий капітал
і придається до поширу продукції в інших сферах. Коли припустити, що всі
інші обставини однакові, то маса зиску, призначеного до зворотного перетворення
в капітал, залежатиме від маси добутого зиску, отже й від поширу самого процесу
репродукції. Коли ж це нове нагромадження наражається в своєму ужитку
на труднощі, брак сфер приміщення, коли, отже, відбувається переповнення ділянок
продукції позичковим капіталом та надмірне постачання його, то ця багатість
(Plethora) позичкового грошового капіталу доводить тільки обмеженість капіталістичної
продукції. Ажіотаж в кредитовій справі, що по тому приходить, доводить,
що немає жодної позитивної перешкоди для ужитку цього надмірного капі-
