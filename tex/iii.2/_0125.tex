\parcont{}  %% абзац починається на попередній сторінці
\index{iii2}{0125}  %% посилання на сторінку оригінального видання
виявляється в тому, що ціна в 200 ф. ст. становить лише продукт надзиску
в 10 ф. ст. на 20 років, тимчасом як цей самий водоспад в інших однакових
умовах дає власникові можливість щорічно захоплювати ці 10 ф. ст. протягом
невизначеного часу, 30, 100, х років, і тимчасом як з другого боку, коли нова
метода продукції, що її не можна застосувати до водяної сили, знизить витрати
продукції товарів, вироблюваних з допомогою парової машини з 100 до 90 ф. ст.,
то зникне надзиск, а разом з ним і рента, а разом з нею і ціна водоспаду.

Встановивши таким чином загальне поняття диференційної ренти, ми переходимо
тепер до розгляду її власне у хліборобстві. Що буде сказано про нього,
взагалі стосується і до гірництва.

Розділ тридцять дев’ятий,

Перша форма диференційної ренти (диференційна
Рента і).

Рікардо цілком має слушність в таких засадах:

«Рента» [тобто диференційна рента; він гадає, що взагалі не існує якоїсь
іншої ренти, крім диференційної] «завжди становить ріжницю між продуктом,
одержаним в наслідок рівновеликих витрат капіталу і праці» (Principles р. 59).
«На однакових величиною земельних дільницях» треба було б йому додати,
оскільки справа йде про земельну ренту, а не про надзиск взагалі.

Іншими словами: надзиск, коли він створюються нормально, а не в наслідок
випадкових обставин в процесі циркуляції, завжди продукується як ріжниця
між продуктом двох однакових кількостей капіталу і праці, і цей надзиск перетворюється
на земельну ренту, коли дві однакові кількості капіталу і праці з
неоднаковими наслідками зайняті на однакових величиною земельних дільницях.
Проте, немає безумовної доконечности в тому, щоб цей надзиск виникав з
неоднакових наслідків однакових кількостей зайнятого капіталу. В різних
підприємствах можуть бути зайняті і різної величини капітали; здебільша справа
навіть так і стоїть; але рівні пропорційні частини, отже, наприклад, 100 ф. ст. кожного
капіталу, дають неоднакові наслідки; тобто норма зиску різна. Це — загальна
передумова існування надзиску в усякій сфері приміщення капіталу взагалі.
Друга — є перетворення цього надзиску на форму земельної ренти (взагалі ренти,
як форми відмінної від зиску); в усякому разі треба дослідити, коли, як і в
яких обставинах відбувається це перетворення.

Далі Рікардо, висловлюючи таку засаду, має слушність, лише оскільки вона
обмежується диференційною рентою:

«Усе, що зменшує ріжницю в продукті, одержаному з тієї самої або з
нової землі, має тенденцію зменшити ренту; а все, що збільшує цю ріжницю,
неодмінно призводить до протилежного ефекту і має тенденцію її збільшити»
(р. 74).

До числа цих причин належать не тільки загальні (родючість і положення),
але й 1) розподіл податків залежно від того, чи впливають вони
рівномірно, чи ні; останнє завжди буває, коли вони не централізовані, як наприклад,
в Англії, і коли податок береться з землі, а не з ренти; 2) ріжниці,
що походить з неоднакового розвитку хліборобства в різних частинах країни,
а ця галузь промисловости, за своїм традиційним характером, важче нівелюється,
ніж мануфактура і 3) нерівномірний розподіл капіталу між орендарями.
А що завойовання хліборобства капіталістичним способом продукції, перетворення
селянина з самостійного господаря в найманого робітника є в дійсності останнім
завойованням цього способу продукції взагалі, то ці ріжниці тут більші,
ніж в будь-якій іншій галузі промисловости.
