час у позичковий капітал, — де збільшення так само мало свідчить про зростання
продуктивного капіталу, як і зріст вкладів в акційних банках Лондону, скоро
ці останні почали платити проценти на вклади. Поки розмір продукції лишається
той самий, це збільшення викликав тільки багатість позичкового грошового
капіталу проти капіталу продуктивного. Відси низький рівень проценту.

Коли процес репродукції знову досягає стану розцвіту, що йде попереду надмірного
напруження, то комерційний кредит досягає дуже великого поширу, що в такі
часи дійсно знову має «здорову» базу легкого зворотного припливу капіталів та поширу
продукції. При такому стані справ рівень проценту все ще низький, хоч він і
підноситься понад свій мінімум. Це — справді єдиний момент, коли можна сказати,
що низький рівень проценту, а тому й відносна багатість позичкового капіталу
збігаються з дійсним поширом промислового капіталу. Легкість та реґулярність
зворотного припливу капіталів, сполучені з поширом комерційного кредиту,
забезпечують подання позичкового капіталу, дарма що попит збільшився, та
заважають зростові рівня проценту. З другого боку, тільки тепер з’являються
помітно лицарі, що працюють без запасного капіталу або взагалі без капіталу
й тому провадять свої операції цілком на основі грошового кредиту. Сюди
долучається тепер ще й значний пошир основного капіталу в усіх формах та
масове відкриття нових широко організованих підприємств. Тепер процент підноситься
до своєї пересічної висоти. Свого максимуму він досягає знову тоді, коли вибухає
нова криза, раптом спиняється кредит, припиняються платежі, паралізується
процес репродукції та, за згаданими раніше винятками, поряд майже абсолютного
браку позичкового капіталу настає надмір бездіяльного промислового капіталу.

Отже, рух позичкового капіталу, оскільки він виявляється в рівні проценту,
відбувається взагалі в напрямку, протилежному рухові промислового капіталу.
Фаза, коли низький рівень проценту, що однак є вищий за мінімум, збігається
з «поліпшенням» та з чим раз більшим — по кризі — довір’ям, і особливо фаза,
коли він досягає своєї пересічної висоти, середини, однаково віддаленої від його
мінімуму та максимуму, — тільки ці два моменти виявляють збіг багатости позичкового
капіталу зі значним поширом промислового капіталу. Але на початку
промислового циклу низький рівень проценту збігається зі зменшенням, а наприкінці
циклу високий рівень проценту з надміром промислового капіталу. Низький
рівень проценту, що супроводить «поліпшення», виявляє те, що комерційний
кредит тільки в незначній мірі потребує банкового кредиту, бо він ще стоїть
на своїх власних ногах.

Цей промисловий цикл має ту особливість, що після того, як уже дано перший
поштовх, той самий кругооборот мусить періодично репродукуватися 8). В стані
підупаду продукція знижується нижче від того щабля, що його вона досягла
за попереднього циклу, і що для нього тепер покладено технічну базу. В періоді
розцвіту — середньому періоді — продукція розвивається на цій базі далі. В період
надмірної продукції та спекуляції продукція напружує продуктивні сили до найвищої
точки, аж поза капіталістичні межі продукційного процесу.

8) [Як я вже зауважував в іншому місці, від часу останньої великої загальної кризи тут настала
зміна. Гостра форма періодичного процесу з її дотеперішнім десятирічним циклом, здається, відступила
місце більш хронічному, довгочасному чергуванню, що поширюється на різні індустріяльні країни в
різні
часи, чергуванню порівняно короткого та млявого поліпшення справ, та порівняно, довгого
нерозв'язного
пригнічення. Однак може й таке бути, що тут маємо ми лише збільшення часу тривання циклу. За
дитинства
світової торговлі, в роки 1815—47, можна виявити кризи, що повторювались приблизно через
кожні п’ять років; в роки 1847—67 цикл є виразно десятирічний; чи не перебуваємо ми в періоді
зародження нового світового краху нечуваної сили? Дещо, здається, вказує на це. Від часу останньої
загальної кризи 1867 року настали великі зміни. Колосальний пошир засобів комунікації — океанські
пароплави,
залізниці, електричні телеграфи, Суецький канал — уперше дійсно утворив світовий ринок. Побіч
Англії, що раніш монополізувала промисловість, постав ряд промислових країн — конкурентів; для
приміщення надмірного европейського капіталу одкрилися по всіх частинах світу безмежно більші та
