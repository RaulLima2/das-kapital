\parcont{}  %% абзац починається на попередній сторінці
\index{iii2}{0228}  %% посилання на сторінку оригінального видання
оборотного капіталу\footnote{
Пан д-р H. Maron (Extensiv oder Intensiv?) [точніше ця брошура не позначена] виходить з
фалшивого припущення тих, з ким він бореться. Він припускає, що капітал, витрачуваний на купівлю
землі є Anlagekapital, і потім починає сперечатися про відповідні поняття Anlagekapital і
Betriebskapital, тобто основний капітал і оборотний капітал. Його цілком школярські уявлення про
капітал взагалі, між іншим в наслідок стану німецької «науки про народнє господарство» простимі для
не економіста, — заховують від нього, що цей капітал не є ані Anlagekapital (основний) ані
Betriebskapital
(оборотний капітал) точнісінько так само, як капітал, що його хтось приміщує на біржі на купівлю
акцій або державних фондів, і що для нього особисто становить вкладання капіталу, в дійсності не
«вкладається» ні в яку галузь продукції.
}; навпаки, він тільки дає покупцеві титул на одержання
щорічної ренти, але до створення цієї ренти абсолютно не має жодного чинення.
Адже покупець землі виплачує капітал саме тому, хто продає землю, і продавець
за це відмовляється від своєї власности на землю. Отже, як капітал покупця
цей капітал уже не існує; у покупця його вже немає; отже, він не належить
до того капіталу, який тим або іншим способом покупець міг би вкласти
в саму землю. Дорого чи дешево покупець купив землю, чи одержав її даром,
це нічого не змінює в капіталі, що його вкладає орендар в господарювання, і
це нічого не змінює в ренті, але змінює лише те, чи здаватиметься йому рента
процентом чи ні, зглядно високим чи низьким процентом.

Візьмімо, наприклад, рабовласницьке господарство. Ціна, виплачена тут
за раба, є не що інше, як антиципована і капіталізована додаткова вартість,
або зиск, що має бути вичавлений з нього. Але капітал, виплачений при купівлі
раба, не належить до того капіталу, з допомогою якого з раба добувається
зиск, додаткову працю. Навпаки. Це є капітал, що його рабовласник позбувся,
вираховання з того капіталу, що ним він порядкує в дійсній продукції. Він
перестав існувати для рабовласника, цілком так само, як капітал, витрачений
на купівлю землі, перестав існувати для хліборобства. Це найкраще доводиться
тим, що він знову почне існувати для рабовласника або для власника землі
лише тоді, коли він знову продасть раба або землю. Але тоді в такому самому
становищі опиниться покупець. Та обставина, що він купив раба, не дає ще
йому без дальших околичностей можливости визискувати раба. Цю можливість
дасть йому лише дальший капітал, що його він приміщує в саме рабовласницьке
господарство.

Той самий капітал не існує двічі — одного разу в руках продавця і другого
разу в руках покупця землі. З рук покупця він переходить в руки продавця,
і цим справа закінчується. У покупця тепер немає капіталу, але в нього
натомість є дільниця землі. Та обставина, що ренту, одержувану від дійсного
приміщення капіталу в цю дільницю землі, новий земельний власник вважає
тепер за процент на капітал, який він не вклав у землю, а витратив на придбання
землі, нічого не змінює в економічній природі чинника-землі, — так само
як та обставина, що хтось заплатив 1000 фун. стерлінґів за трипроцентові консолі,
не має ніякого чинення до того капіталу, з доходу якого виплачуються проценти
на державну позику.

В дійсності гроші, витрачені на купівлю землі, цілком так само як і
гроші, витрачені на купівлю державних паперів, є капітал лише в собі, як
всяка сума вартости на базі капіталістичного способу продукції є капітал в собі,
потенціальний капітал. За землю виплачується, як і за державні фонди, як і
за інші куповані товари, певну суму грошей. Вона є капітал в собі, бо вона
може бути перетворена на капітал. Від того способу, в який продавець споживає
одержані ним гроші, залежить, чи перетворяться вони дійсно на капітал, чи ні.
Для покупця вони вже ніколи не можуть функціонувати як капітал, — подібно
до всяких інших грошей, які він остаточно витратив. В його розрахунках вони
фігурують для нього як капітал, що дає процент, бо дохід, одержуваний як
\parbreak{}  %% абзац продовжується на наступній сторінці
