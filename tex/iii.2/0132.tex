умова її виникнення е в тому, що підвищення абсолютної родючості всієї земельної площі не знищує
тієї неоднаковості, але збільшує її, або залишає без зміни, або ж лише зменшує.

Від початку і до половини ХТІІІ століття панувало в Англії, не зважаючи на пониження ціни золота і
срібла, безперервне падіння цін збіжжя одночасно (коли розглядати весь період) з ростом ренти,
загальної суми ренти, розміру оброблюваної земельної площі, хліборобської продукції і людности. Де
відповідає таблиці І, комбінованій з таблицею II, у висхідному напрямку, але так, що гірша земля А
або поліпшується або виключається з числа земель, оброблюваних під збіжжя; це, звичайно, не значить,
що вона не буде використана для інших сільськогосподарських або промислових цілей.

Від початку XIX століття (треба точніше подати час) до 1815 року безперервне підвищення цін збіжжя
одночасно з постійним зростом ренти, загальної суми ренти, розміру оброблюваної земельної площі,
хліборобської продукції і людності. Це відповідає таблиці І у низхідному напрямку. (Тут слід навести
цитату щодо обробітку гірших земель за того часу).

За доби Петті і Давенанта скарги сільської людности і земельних власників на поліпшення і поширення
обробітку; пониження ренти на кращих землях, підвищення загальної суми ренти в наслідок поширення
площі землі, що дає ренту.

(До цих трьох пунктів навести потім дальші цитати; також щодо ріжниці у родючості різних частин
обробленої землі в країні).

Щодо диференційної ренти слід взагалі зауважити, що ринкова вартість завжди перевищує загальну ціну
продукції даної маси продуктів. Для прикладу візьмімо таблицю І. 10 кватерів всього продукту
продаються за 600 шил., бо ринкова ціна визначається ціною продукції на А, яка становить 60 шил. за
квартер. Але дійсна ціна продукції є:

Дійсна ціна продукції 10 квартерів є 240 шил.; вони продаються за 600, тобто на 250°/0 дорожче.
Дійсна пересічна ціна 1 квартера є 24 шил.: ринкова ціна — 60 шил., тобто теж на 250°/0 дорожча.

Тут маємо визначення за посередництвом ринкової вартости в тому її вигляді, як вона на базі
капіталістичного способу продукції пробивається за посередництвом конкуренції; ця остання породжує
фалшиву соціальну вартість. Це постає з закону ринкової вартости, якому підпорядковані продукти
хліборобства.
Визначення ринкової вартости продуктів, отже, і хліборобських продуктів, є суспільний акт, хоч і акт
суспільно несвідомий і ненавмисний, акт, що неминуче ґрунтується на міновій вартості продукту, а не
на землі і ріжницях її родючості. Коли уявити собі, що капіталістична форма суспільства знищена і
суспільство організоване як свідома і плянова асоціація, то ці 10 квартерів являтимуть собою
кількість самостійного робочого часу, рівну тому, що міститься в 240 шил. Отже, суспільство не
купувало б цього хліборобського продукту за таку кількість робочого часу, яка в 2 1/2, раза більша
за робочий час, який дійсно міститься в цьому продукті. Тим самим відпала б база класу власників
землі. Це впливало б цілком так само, як здешевлення продукту на таку суму в наслідок чужоземного
довозу. Тому, оскільки справедливо було б сказати, що — в умовах збереження сучасного способу
продукції, але припускаючи, що диференційна рента діставатиметься державі — ціни земельних
продуктів, за інших незміних умов, залишились би тими самими, так само помилково було б

А 1 кварт. = 60 шил.
В 2 кварт. = 60 шил.
С 3 кварт. = 60 шил.
1) 4 кварт. = 60 шил.

1 кварт. 60 шил.
1 кварт. ЗО шил.
1 кварт. 20 шил.
1 кварт. 15 шил.

10 кварт. — 240 шил.; пересічно 1 кварт. = 24 шил.
