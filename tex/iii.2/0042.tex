ється тим, що кожен однією рукою бере кредит, а другою його дає. [Це не погоджується
з моїм досвідом. Ф. Е.]. Але оскільки в цій формі сюди ввіходить
дисконто (disconto), його реґулюється не цим комерційним кредитом, а грошовим
ринком.

Коли б попит та подання грошового капіталу, що визначають рівень
проценту, були тотожні з попитом та поданням дійсного капіталу, як твердить
Оверстон, то процент мусів би одночасно бути низьким та високим, відповідно до
того, чи розглядають різні товари, чи той самий, товар на різних стадіях (сировина,
півфабрикат, готовий продукт). В 1844 році рівень проценту Англійського банку
коливався між 4% (від січня до вересня) та 2 1/2—3% від листопада до
кінця року. В 1845 році він становив 2 1/2, 2 3/4, 3% від січня до жовтня, між
З та 5% протягом останній місяців. Пересічна ціна чистої орлеанської бавовни
була в 1844 році 6 1/4 пенсів, а в 1845 році 4 7/8 пенси. З березня 1844 року
запас бавовни в Ліверпулі становить 627042 паки, а 3 березня 1845 року
773 800 паків. Якщо робити висновки з низької ціни бавовни, то рівень проценту
в 1845 році мусів бути низький, що і було в дійсності протягом більшої
частини цього часу. Коли ж робити висновки за ціною пряжі, то рівень проценту
мусів би бути високий, бо ціни були відносно, а зиски абсолютно високі.
З бавовни по 4 пенси за фунт можна було в 1845 році при 4 пенсах видатків
на прядіння випрясти пряжу (№ 40 добрий secunda mule twist), що отже коштувала
б прядунові 8 пенсів, а продати її прядун міг в вересні та жовтні 1845 року
по 10 1/2 або 11 1/2 пенсів за фунт (Дивись нижче посвідчення Wylie).

Всю цю справу можна розв’язати таким способом:

Попит та подання позичкового капіталу були б тотожні з попитом та поданням
капіталу взагалі (хоч ця остання фраза — абсурд; для промисловця або
купця товар є форма його капіталу, а проте він ніколи не вимагає капіталу як
такого, а завжди лише цей спеціальний товар як такий, він купує та оплачує
його як товар, збіжжя або бавовну, незалежно він тієї ролі, що її товар має
набути в кругообороті його капіталу), коли б не було грошових позикодавців,
а замість них капіталісти-позикодавці володіли б машинами, сировиною і т. ін. та
визичали б або віддавали їх в найми, так як от тепер наймають доми промисловим
капіталістам, що самі є власники частини цих речей. В таких умовах
подання позичкового капіталу було б тотожнє з поданням елементів продукції
для промислових капіталістів, — товарів для купців. Але очевидно, що тоді розподіл
зиску між позикодавцем і позикоємцем передусім цілком залежав би від
того відношення, у якому цей капітал визичено та у якому він є власність
того, хто його вживає.

За паном Weguelin’oм (В. А. 1857) рівень проценту визначається «масою
незайнятого капіталу» (252); є «лише покажчик маси незайнятого капіталу, що
шукає приміщення» (271); пізніше він називає цей незайнятий капітал «floating
capital» (485), а під останнім він розуміє «банкноти Англійського банку й інші
засоби циркуляції в країні; напр., банкноти провінціяльних банків та наявну в
країні монету... я розумію під floating capital також і запаси банків» (502, 503),
а пізніше він додає сюди й зливки золота (503). Так той самий Weguelin каже, що
Англійський банк має великий вплив на рівень проценту в той час, «коли ми» (Англійський
банк) «фактично маємо більшу частину незайнятого капіталу у своїх
руках» (1198), тимчасом коли за вищенаведеними виказами пана Оверстона Англійський
банк — «не місце для капіталу». Далі Weguelin каже: «На мою думку
норма дисконту реґулюється кількістю незайнятого капіталу країні. Кількість незайнятого
капіталу представлена запасом Англійського банку, що фактично є металевий
запас. Отже, коли металевий скарб меншає, то це зменшує кількість незайнятого
капіталу в країні, і таким чином підносить вартість наявної решти капіталу»
(1258). Дж. Стюарт Міл каже (1102): «Для того щоб зберегти виплатоздатність
