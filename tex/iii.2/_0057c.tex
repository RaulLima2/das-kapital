\parcont{}  %% абзац починається на попередній сторінці
\index{iii2}{0057}  %% посилання на сторінку оригінального видання
зовсім перестає функціонувати; нікому не потрібна обіцянка платежа, бо кожен
хоче мати платежі лише готівкою; тільки банкнота зберігає, принаймні до цього
часу в Англії, свою здатність до циркуляції, бо за Англійським банком стоїть
нація з цілим своїм багатством.

Ми бачили, що навіть пан Chapman, який в 1857 році й сам був магнатом
на грошовому ринку, гірко скаржиться на те, що в Лондоні є кілька великих
грошевих капіталістів, досить дужих до того, щоб в певний момент викликати
нелад на цілому грошовому ринку та в такий найганебніший спосіб драти шкуру
з дрібніших торговців грішми. Отже, є кілька таких великих акул, що мають
змогу значно загострити скруту, продавши на 1—2 мільйони консолів та забравши
тим способом з ринку на цю суму банкнот (а одночасно вільного позичкового
капіталу). Для того, щоб таким способом перетворити скруту на паніку,
вистачило б спільного чину трьох великих банків.

Найбільша сила капіталу в Лондоні є, природно, Англійський банк, але
з причини свого становища напівдержавної установи він не має змоги виявляти
своє панування таким брутальним способом. Проте й він знає досить засобів
та шляхів — особливо від часу банкового акту 1844 року, — як йому виходити
сухеньким з води.

Англійський банк має капіталу 14.553.000 ф. ст. та, крім того, є в нього
до розпорядку щось коло 3 мільйонів ф. ст. «решти», тобто нерозподілених зисків,
а так само й усі гроші, що, приходячи до урядової скарбниці від податків
і т. ін., мусять складатись до Англійського банку, поки їх почнуть уживати.
Коли додати до цього ще суму інших грошових вкладів (за звичайних часів
коло 30 міл. ф. ст.) та банкнот, виданих без покриття, то ми визнаємо цінування
Newmarch’a ще за доволі помірне, коли він (В. А. 1857, № 1889)
каже: «я переконався, що загальну суму фондів, яка раз-у-раз є в операціях
на (лондонському) грошовому ринку, можна визначити приблизно на 120 міл.
ф. ст.; і з цих 120 мільйонів Англійський банк порядкує дуже значною частиною,
щось 15—20\%».

Оскільки банк видає банкноти, які не покриті металевим запасом, що є в його
коморах, він утворює знаки вартости, що являють для нього не тільки засіб циркуляції,
але й додатковий — хоч і фіктивний — капітал на номінальну суму цих непокритих
банкнот. І цей додатковий капітал дає йому додатковий зиск — В. А. 1857.
Wilson питає Newmarch’a: «1563. Циркуляція власних банкнот якогось банку,
тобто та сума їх, що пересічно лишається в руках публіки, являє додаток до
чинного капіталу того банку, чи не так? — Цілком певно. — 1564. Отже, весь
зиск, що його той банк добуває з цієї циркуляції, становить зиск, що походить
від кредиту, а не від того капіталу, що його дійсно має банк? — Цілком певно».

Природно, те саме має силу й щодо приватних банків, що видають банкноти.
У своїх відповідях №№ 1866—78 Newmarch розглядає дві третини всіх виданих
приватними банками банкнот (для останньої третини ці банки мусять тримати
металевий запас) як «утворення відповідної суми капіталу», бо на цю суму заощаджується
металевих грошей. Зиск банкіра з цієї причини може бути і не більший
за зиск інших капіталістів. Лишається той факт, що банкір добуває зиск від цього
національного заощадження металевих грошей. Те, що національне заощадження
металевих грошей становить приватний зиск, ані трохи не бентежить буржуазних
економістів, бо зиск є взагалі присвоювання національної праці. Чи є щось
іще безглуздіше, ніж, прим., Англійський банк 1797—1817 р. р., що до його
банкнот було довір’я лише дякуючи державі, та який змушував потім державу,
тобто публіку, оплачувати в формі процентів за державні позики надану йому
державою силу перетворювати ті банкноти з паперу на гроші й потім визичати
їх державі?
