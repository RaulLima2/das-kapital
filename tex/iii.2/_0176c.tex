\parcont{}  %% абзац починається на попередній сторінці
\index{iii2}{0176}  %% посилання на сторінку оригінального видання
2) Диференційна рента І, що походить з ріжниці в родючості різних земельних
дільниць.

3) Диференційна рента II, що походить з послідовної витрати капіталу
на тій самій землі. Диференційна рента II підлягає дослідженню:

a) при сталій,

b) при низхідній,

c) при висхідній ціні продукції.

І далі:

d) перетворення надзиску в ренту.

4) Вплив цієї ренти на норму зиску.

B. Абсолютна рента.

C. Ціна землі.

D. Кінцеві зауваження про земельну ренту.

\pfbreak

Як загальний наслідок розгляду диференційної ренти, виходить таке:

\emph{Перше:} Створення надзиску може відбуватися різними шляхами. З одного
боку, на базі диференційної ренти І, тобто на базі витрати всього
хліборобського капіталу на земельній площі, що складається з земель різної
родючости. Далі, як диференційна рента II, на базі різної диференційної продуктивности
послідовних витрат капіталу на тій самій землі, тобто на базі
більшої продуктивности, визначеної, наприклад, у квартерах пшениці, ніж та,
що постає при тій самій витраті капіталу на найгіршій землі, що не дає ренти,
але реґулює ціну продукції. Але хоч би як виникали ці надзиски, перетворення
їх у ренту, отже, їх перехід від орендаря до землевласника, завжди припускав
як попередню умову, що різні дійсні індивідуальні ціни продукції (тобто незалежно
від загальної ціни продукції, що реґулює ринок) окремих продуктів, окремих
послідовних витрат капіталу попередньо вирівнюються в індивідуальну
пересічну ціну продукції. Надмір загальної регуляційної ціни продукції продукту
з акра над цією його індивідуальною пересічною ціною продукції становить
і визначає величину ренти на акр. При диференційній ренті І диференційні
наслідки розпізнаються сами по собі, бо вони постають на різних дільницях
землі, що лежать одна поза однією і одна біля однієї, — за такої витрати капіталу
на акр, яку береться за нормальну, і при відповідному до цієї витрати
нормальному обробленні. При диференційній ренті II їх спершу треба зробити
розрізнюваними; справді, вони мусять бути перетворені зворотно у диференційну
ренту І, а це можна зробити лише зазначеним способом. Візьмімо, наприклад,
таблицю III, ст. 149.

%% TODO: add link

Земля В дає в наслідок першої витрати капіталу в 2\sfrac{1}{2} ф. стерл. 2 квартери
з акра, а в наслідок другої витрати, однакової розміром, — 1\sfrac{1}{2} квартери; разом
3\sfrac{1}{2} квартери з того самого акра. За цими 3\sfrac{1}{2} квартерами, що виросли
на тій самій землі, не можна побачити, яка частина з них є продуктом витрати
капіталу І і яка витрати капіталу II. Вони в дійсності становлять продукт
усього капіталу в 5 ф. стерл; і дійсний факт є лише в тому, що капітал
в 2\sfrac{1}{2} ф. стерл. дав 2 квартери, а капітал в 5 ф. стерл. — не 4, а 3\sfrac{1}{2} квартери.
Справа ані трохи не змінилася б, якби ці 5 ф. стерл. дали 4 квартери, так що
продукти обох витрат капіталу були б однакові, або навіть 5 квартерів, так
що друга витрата капіталу дала б надмір в 1 квартер. Ціна продукції перших
двох квартерів дорівнює 1\sfrac{1}{2} ф. стерл. за квартер, ціна продукції других 1\sfrac{1}{2} квартерів є 2 ф.
стерл. за квартер. Ці 3\sfrac{1}{2} квартери разом коштують тому 6 ф. стерл.
Це є індивідуальна ціна продукції всього продукту, а пересічно вона становить
1 ф. стерл. 14\sfrac{2}{7} шил. за квартер, округло, скажімо, 1\sfrac{3}{4} ф. стерл. За загальної ціни
продукції в 3 ф. стерл, що визначається землею А, це дає надзиск в 1\sfrac{1}{4} ф. стерл.
на квартер і, отже, для 3\sfrac{1}{2} квартерів разом. — 4\sfrac{3}{8} ф. стерл. За пересічної ціни
\parbreak{}  %% абзац продовжується на наступній сторінці
