\parcont{}  %% абзац починається на попередній сторінці
\index{iii2}{0231}  %% посилання на сторінку оригінального видання
і раціональному обробленню, збереженню і поліпшенню самого ґрунту, розвиваються
і в тому і в тому разі, тільки в різних формах, і в суперечках про ці специфічні
форми лиха забувається його останню причину.

Дрібна земельна власність має за свою передумову, що рішуче переважна
більшість людности є сільська, і що за панівну форму праці є не суспільна, а ізольована
праця; що, отже, в таких умовах виключається можливість багатства й розвитку
репродукції, так само його матеріяльних, як і інтелектуальних умов, — а тому
і умов раціональної культури. З другого боку, велика земельна власність скорочує
хліборобську людність до постійно знижуваного мінімуму, і протиставляє
їй дедалі більш ростучу, концентровану в великих містах промислову людність;
цим самим вона породжує умови, що спричинюють непоправну шкоду в процесі
суспільного обміну речовин, диктованого природними законами життя, в наслідок
чого сила ґрунту марнотратиться, а торговля виводить це марнотратство далеко
за межі власної країни (Лібіх).

Коли дрібна земельна власність створює клясу варварів, яка наполовину
стоїть поза суспільством, яка сполучає в собі всю брутальність первісних суспільних
форм з усіма муками і всіма злиднями цивілізованих країн, то велика
земельна власність підтинає робочу силу в тій останній галузі, куди ховається
її примітивна енергія, і в якій вона зберігається як резервний фонд для відродження
життєвої сили націй, — в самому селі. Велика промисловість і промислово
проваджене велике хліборобство діють поруч. Коли первісно вони відрізняються
тим, що перша виснажає і руйнує більше робочу силу, отже, природну
силу людини, тимчасом як останнє більше безпосередньо природну силу землі,
то пізніше, з перебігом розвитку вони подають руку одне одному, так що промислова
система і в селі виснажає робітника, а промисловість і торгівля з свого
боку створюють для хліборобства засоби виснаження ґрунту.
