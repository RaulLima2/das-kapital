\parcont{}  %% абзац починається на попередній сторінці
\index{iii2}{0218}  %% посилання на сторінку оригінального видання
панщиною або крепаків. Тимчасом ясно, що при тому примітивному і нерозвиненому
стані, на якому ґрунтується це суспільне продукційне відношення і
відповідний йому спосіб продукції, традиція мусить відігравати переважну ролю.
Далі ясно, що тут, як і всюди, переважна частина суспільства заінтересована
в тому, щоб усвячувати суще, як закон, і ті його межі, які дано звичаєм і
традицією, фіксувати як законні. Проте, лишаючи все інше осторонь, це стається
само собою, скоро постійна репродукція бази сущого стану, відношення,
що лежить в його основі, набуває з перебігом часу уреґульованої і упорядкованої
форми; і ця уреґульованість і цей порядок сами є доконечний момент всякого
способу продукції, коли він має набути суспільної сталости і незалежности від
звичайного випадку або сваволі. Уреґульованість і порядок є саме форма суспільного
зміцнення даного способу продукції, і тому його відносної емансипації
від просто сваволі і звичайного випадку. Він досягає цієї форми при застійному
стані так процесу продукції, як і відповідних до нього суспільних відносин
через просту повторну репродукцію їх самих. Коли ця форма проіснувала протягом
певного часу, вона зміцнюється, як звичай і традиція, і нарешті усвячується
як виразний закон. А що форма цієї додаткової праці, панщинна праця,
ґрунтується на нерозвиненості всіх суспільних продуктивних сил праці, на
примітивності самого способу праці, то і мусить вона природно віднімати у
безпосереднього продуцента незрівняно меншу відповідну частину всієї праці,
ніж за розвинених способів продукції й особливо за капіталістичної продукції.
Припустімо, наприклад, що панщинна праця на земельного власника первісно становила
два дні на тиждень. Ці два дні панщинної праці на тиждень таким чином
усталились, вони є стала величина, законно уреґульована звичаєвим або писаним
правом. Але продуктивність решти днів тижня, що ними може порядкувати
сам безпосередній продуцент, є величина змінна, яка мусить розвиватися в
процесі його досвіду, — цілком так само, як нові потреби, з якими він знайомиться,
цілком так само як поширення ринку для його продукту, ростуча забезпеченість
порядкування для самого себе цією частиною своєї робочої сили,
підганятиме його до підвищеного напруження робочої сили, при чому не слід
забувати, що вживання цієї робочої сили зовсім не обмежується хліборобством,
але охоплює й сільську домашню промисловість. Тут дана можливість певного
економічного розвитку, зрозуміла річ, залежно від більш або менш сприятливих
обставин, від природженого расового характеру тощо.

III. Рента продуктами.

Перетворення відробітної ренти на ренту продуктами, економічною мовою
висловлюючись, нічого не змінює в суті земельної ренти. Суть земельної ренти при
таких умовах, які ми розглядаємо тут, в тому, що земельна рента є однісінька
панівна і нормальна форма додаткової вартости, або додаткової праці; а це в
свою чергу виражається в тому, що вона становить однісіньку додаткову працю
або однісінький додатковий продукт, що його безпосередній продуцент, який
посідав умови праці, що потрібні для його власної репродукції, повинен
дати власникові такої умови праці, яка в цьому стані охоплює все, тобто
власникові землі; і що з другого боку тільки земля і протистоїть йому, як
умова праці, що перебуває в чужій власності, відокремлена проти нього і
персоніфікована у земельному власникові. Коли рента продуктами становить
панівну і найрозвиненішу форму земельної ренти, вона все ж постійно в більшій
або меншій мірі супроводиться рештками попередньої форми, тобто ренти, що
її виплачується безпосередньо працею, отже, панщинною працею, і це однаково,
чи є земельним власником приватна особа чи держава. Рента продуктами має
своєю передумовою вищий культурний рівень безпосереднього продуцента, отже
\parbreak{}  %% абзац продовжується на наступній сторінці
