маса нехліборобських товаропродуцентів і нехліборобської товарової продукції.
Але тому що це відбувається без його участи, то і видається як якась його
специфічна особливість, те, що маса вартости, маса додаткової вартости і перетворення
частини цієї додаткової вартости на земельну ренту залежить від
суспільного процесу продукції, від розвитку товарової продукції' взагалі. Тому
Dove наприклад, хоче звідси вивести ренту. Він говорить, що рента залежить
не від маси хліборобського продукту, а від його вартости; а ця вартість залежить
від маси і продуктивности нехліборобської людности. Але ж і для всякого іншого
продукту справедливо, що він як товар розвивається поруч з тим, як розвивається
почасти маса, почасти різноманітність ряду інших товарів, що становлять у відношенні
до нього еквіваленти. Це вже було показано при загальному викладі
вартости. З одного боку, здібність до обміну певного продукту взагалі залежить
від різноманітности товарів, що існують крім нього. З другого боку від цього ж
залежить особливо та кількість, в якій саме цей продукт можна виробити як товар.

Жоден продуцент, — ні промисловий, ні хліборобський, — розглядуваний
ізольовано, не продукує вартости або товару. Його продукт стає вартістю і товаром
лише в певних суспільних відносинах. Поперше, — оскільки він є вираз суспільної
праці, отже, оскільки власний робочий час даного продуцента є частина суспільного
робочого часу взагалі; подруге, в грошовому характері продукту і в його загальній
здібності до обміну, визначуваній ціною, цей суспільний характер праці продуцента
виступає як наданий (aufgepräger) його продуктові суспільний характер.

Отже, коли пояснення ренти заміняють, з одного боку, поясненням додаткової
вартости або, при ще обмеженішому розумінні, поясненням додаткового
продукту взагалі, то тут, з другого боку, роблять ту помилку, що характер, властивий
усім продуктам як товарам і вартостям, приписують виключно хліборобським
продуктам. Пояснення це стає ще більш вульґарним, коли від загального визначення
вартости переходять до реалізації певної товарової вартости. Усякий товар
може реалізувати свою вартість лише в процесі циркуляції, а чи реалізує
він її та в якій мірі реалізує, це кожного разу залежить від умов ринку.

Отже, своєрідна особливість земельної ренти є не в тому, що хліборобські
продукти розвиваються в вартості і як вартості, тобто не в тому, що вони як
товари протистоять іншим товарам, і нехліборобські продукти протистоять їм як
товари, або що вони розвиваються як особливі вирази суспільної праці. Своєрідна
особливість є в тому, що разом з умовами, в яких хліборобські продукти
розвиваються як вартості (товари), і разом з умовами реалізації їхніх вартостей
розвивається і сила земельної власности привлащувати собі чим раз більшу
частину цих створюваних без її участи вартостей, в тому, що чим раз більша
частина додаткової вартости перетворюється на земельну ренту.

розділ тридцять восьмий.

Диференційна рента. загальні уваги.

Аналізуючи земельну ренту, ми спочатку виходитимемо з припущення, що
продукти, з яких виплачується таку ренту, що в них частина додаткової вартости,
а тому і частина всієї ціни зводиться до ренти — для нашої цілі досить мати на
увазі хліборобські продукти або також продукти копалень, — отже, що продукти
ґрунту або копалень, як усі інші товари продаються по цінах їхньої продукції.
Тобто їхні продажні ціни дорівнюють елементам витрат їхньої продукції (вартості
спожитого сталого і змінного капіталу) плюс зиск, визначений загальною
нормою зиску, обчислений на весь авансований капітал, спожитий і неспожитий.
Отже, ми припускаємо, що пересічні продажні ціни цих продуктів дорівнюють їхнім
