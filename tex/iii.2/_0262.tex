\parcont{}  %% абзац починається на попередній сторінці
\index{iii2}{0262}  %% посилання на сторінку оригінального видання
праці змінюється дійсна вартість товарів, а також ті випадки, коли процес продукції
порушується якимись природними або суспільними лихами, то ми будемо
здивовані, поперше, відносною вузькістю меж відхилів і, подруге тією реґулярністю,
з якою ці відхили зрівноважуються. Ми знайдемо тут те саме панування
реґуляційних пересічних, що його Кетле довів для суспільних явищ. Коли вирівнювання
товарових вартостей на ціни продукції не наражається на жодні
перешкоди, то рента зводиться до диференційної ренти, тобто обмежується вирівнянням
тих надзисків, що їх реґуляційна ціна продукції дала б певній
частині капіталістів, і що їх тепер привласнює земельний власник. Отже, тут рента
має свою певну вартісну межу в тих відхилах індивідуальних норм зиску, які
спричиняються реґулюванням цін продукції загальною нормою зиску. Коли земельна
власність створює перешкоди вирівнюванню товарових вартостей на ціни
продукції і таким чином привласнює собі абсолютну ренту, то ця остання обмежена
надміром вартости хліборобських продуктів над їхньою ціною продукції,
отже, надміром вміщеної в них додаткової вартости проти зиску, який припадає
окремим капіталам, відповідно до загальної норми зиску. Ця ріжниця становить
тоді межу ренти, яка тепер, як, і давніш, становить лише певну частину
даної і вміщеної в товарах додаткової вартости.

Нарешті, коли б вирівнювання додаткової вартости на пересічний зиск
наразилось у різних сферах продукції на перешкоду у вигляді штучних або
природних монополій, і зокрема у вигляді монополії земельної власности,
так що зробилася б можливою монопольна ціна, що перевищує ціну продукції
і вартість товарів, на які поширюється дія монополії, — то й тоді не були б
знищені межі, визначувані вартістю товарів. Монопольна ціна певних товарів
лише перенесла б частину зиску продуцентів інших товарів на товари з монопольною
ціною. Посередньо виникло б місцеве порушення в розподілі додаткової
вартости між різними сферами продукції, але воно залишило б незмінною
межу самої цієї додаткової вартости. Коли товар з монопольною
ціною входить в число речей потрібного споживання робітника, то це підвищує
заробітну плату, і тим самим понижується додаткова вартість, якщо
тільки робітникові по-давнішому виплачують усю вартість його робочої сили.
Це може також спричинити падіння заробітної плати нижче від вартости
робочої сили, але лише в тому випадку, коли заробітна плата перевищує
межі її фізичного мінімуму. В цьому випадку монопольна ціна виплачується
через вирахування з реальної заробітної плати (тобто з суми споживних вартостей,
одержуваних робітником в наслідок даної кількості праці) і з зиску
інших капіталістів. Таким чином межі, в яких монопольна ціна може порушити
нормальне реґулювання товарових цін, піддаються твердому визначенню і точному
облікові.

Отже, подібно до того, як поділ новодолученої до товарів вартости і взагалі
вартости товарів, яка розкладається на доходи, знаходить свої об’єктивні
і реґуляційні межі у співвідношенні між потрібною і додатковою працею, заробітною
платою і додатковою вартістю, так само і поділ самої додаткової вартости
на зиск і земельну ренту знаходить свої межі в законах, що реґулюють
вирівнювання норми зиску. При розпаді на процент і підприємницький
бариш, межу їх, обох разом узятих, становить сам пересічний зиск, Він
дає тут певну суму вартости, що в її межах цей поділ має відбутись, і тільки
й може відбутись. Та певна пропорція, що в ній відбувається поділ, має
тут випадковий характер, тобто визначається виключно умовами конкуренції.
Тимчасом як в інших випадках збіг попиту і подання знищує відхили ринкових
цін від їхніх реґуляційних пересічних цін, тобто знищує вплив конкуренції,
тут вона є єдино визначальна. Але чому? Тому, що тут той
самий чинник продукції, капітал, повинен розподілити належну йому додаткову
\parbreak{}  %% абзац продовжується на наступній сторінці
