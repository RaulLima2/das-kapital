\parcont{}  %% абзац починається на попередній сторінці
\index{iii2}{0191}  %% посилання на сторінку оригінального видання
земельного власника зовсім не підстава для того, щоб даром передати свою землю
до розпорядження орендареві і, виявши філантропічне ставлення до цього в
справах приятеля, запровадити crédit gratuit\footnote*{
Безплатний кредит. Прим. Ред.
}. Таке припущення має в собі
абстрагування від земельної власности, знищення земельної власности, що її
існування саме і ставить межу для приміщення капіталу і вільного використовування
його на землі, — межу, яка зовсім не відпадає від самого міркування
орендаря, що стан збіжжевих цін дозволив би йому здобути з свого
капіталу з допомогою експлуатації землі роду А звичайний зиск, коли б йому
не довелося виплачувати ренти, тобто коли б він міг на практиці ставитись до
земельної власности так, наче б її не існувало. Але монополію земельної власности,
земельну власність як межу капіталу припускається диференційною
рентою, бо без цього надзиск не перетворився б па земельну ренту, і не дістався
б земельному власникові замість орендареві. І земельна власність як межа,
продовжує існувати і там, де рента як диференційна рента відпадає, тобто на
землі А. Якщо ми розглянемо випадки, коли в країні капіталістичної продукції
капітал може вкладатися в землю без виплати ренти, то ми знайдемо, що всі
вони включають хоч і не юридичне, то фактичне знищення власности на землю,
знищення, яке, проте, може статися лише за цілком певних і своєю природою
випадкових обставин.

Перше. Коли земельний власник сам є капіталіст, або капіталіст сам є
земельний власник. Коли ринкова ціна піднеслась так високо, що на тому,
що є тепер землею роду А, можна здобути ціну продукції, тобто покриття капіталу
плюс пересічний зиск, то він може в цьому випадку сам господарювати
на своїй дільниці землі. Але чому? Тому, що у відношенні до нього земельна
власність не створює будь-якої межі для приміщення його капіталу.
Він може обробляти землю як простий елемент природи і тому він може керуватися
виключно міркуваннями про використання свого капіталу, капіталістичними
міркуваннями. Такі випадки трапляються на практиці, але тільки як винятки.
Так само як капіталістичне оброблення землі має за передумову роз’єднення капіталу,
що функціонує, і земельної власности, цілком так само воно виключає як
загальне правило провадження господарства самим земельним власником. Одразу
видно, що таке провадження господарства самим земельним власником є цілком
випадкове. Коли збільшений попит на збіжжя потребує оброблення більшої кількости
землі А, ніж її є у власників, які сами провадять господарство, коли, отже,
частина її мусить бути віддана в оренду для того, щоб вона могла взагалі оброблятися,
тоді зараз же відпадає ця гіпотетичність погляду на межу, яку земельна
власність створює для приміщення капіталу. Постає недоладне противенство,
коли виходять з відповідного капіталістичному способові продукції відокремлення
між землею і капіталом, орендарем і земельним власником, а потім,
навпаки, припускають, що господарство провадять, як загальне правило, самі
земельні власники до такого обсягу і повсюди, де капітал, коли б незалежно
від нього не існувало жодної земельної власности, не здобував бн з оброблення
землі жодної ренти. (Див. у А. Світа місце про ренту з копалень, цитоване
значно далі). Це знищення земельної власности є випадкове. Воно може статись
або не статись.

Друге. В складі орендованих земель можуть бути такі окремі дільниці
землі, що за даного ріння ринкових цін не дають ренти, отже, на ділі здаються
даром, але земельний власник не вважає їх за такі, бо він бачить загальну суму
ренти з землі, віддай її в оренду, а не осібні ренти з окремих складових дільниць
його землі. В цьому випадку для орендаря, — оскільки справа йде про
нерентодайні орендовані дільниці, — земельна власність як межа приміщення
\parbreak{}  %% абзац продовжується на наступній сторінці
