\parcont{}  %% абзац починається на попередній сторінці
\index{iii2}{0020}  %% посилання на сторінку оригінального видання
заплатити, — в цей час маса капіталу, не того капіталу, що є вільний та шукає
приміщення, а того, що загальмувався у своєму процесі репродукції є саме тоді
найбільша, коли й недостача кредиту є найбільша (а тому норма дисконту є найвища
на банкірський кредит). Капітал, уже приміщений, в дійсності в такі часи
масами є недіяльний, бо процес репродукції припинився. Фабрики стоять, сирові
матеріяли нагромаджуються, готові фабрикати в формі товарів переповнюють
ринок. Отже, немає нічого більш помилкового, як приписувати такий стан бракові
продуктивного капіталу. Саме в такі часи є надмір продуктивного капіталу,
почасти порівняно з нормальним, але за даної хвилини зменшеним маштабом
репродукції, почасти ж порівняно з підупалим спожиткуванням.

Уявім собі, що ціле суспільство складається лише з промислових капіталістів
та найманих робітників. Далі, лишім без уваги ті зміни цін, що заважають
великим частинам сукупного капіталу замінятися, як того потребують їхні
пересічні взаємовідносини, зміни, що — при тому загальному зв’язку цілого
процесу репродукції, що його особливо розвинув кредит, — мусять раз-у-раз
викликати тимчасові загальні припинення. Лишім без уваги також фіктивні
підприємства та спекулятивні операції, що їм сприяє кредитова справа. Тоді
кризу можна було б пояснити тільки диспропорціональністю продукції по різних
ділянках та диспропорціональністю, яка може статись між споживанням самих
капіталістів та їхнім нагромадженням. Але за даного стану речей повернення
капіталів, приміщених в продукції, залежить, здебільша, від спожиткової здібности
непродуктивних кляс; тимчасом коли спожиткова здібність робітників є
обмежена почасти законами заробітної плати, почасти ж тим, що робітників
уживають лише доти, доки їх можна уживати з зиском для кляси капіталістів.
Останньою причиною всіх дійсних криз завжди лишаються злидні та
обмеженість споживання мас проти намагання капіталістичної продукції розвивати
продуктивні сили так, наче їхню межу являє лише абсолютна спожиткова
здібність суспільства.

Про дійсну недостачу продуктивного капіталу, принаймні у капіталістично
розвинутих націй, можна говорити тільки при загальних неврожаях, чи то
головних харчових засобів, чи найголовніших промислових сирових матеріялів.

Але до цього комерційного кредиту долучається власно грошовий кредит. Позичання
промисловців та купців одних в одних переплітається з позичанням їм
грошей банкірами та грошовими позикодавцями. При дисконтуванні векселя позика
є лише номінальна. Фабрикант продає свій продукт за вексель та дисконтує
цей вексель в якогось billbroker’a. В дійсності останній позичає першому
лише кредит свого банкіра, який визичає йому знову лише грошовий капітал
своїх вкладників, які складаються з тих самих промисловців і купців та ще й
з робітників (через ощадні банки), одержувачів земельної ренти й інших непродуктивних
кляс. Так кожен індивідуальний фабрикант або купець оминає як
неминучу потребу мати солідний запасний капітал, як і залежність від дійсних
припливів капіталу назад. Але, з другого боку, цілий процес — почасти через просто
шахрайські векселі, а почасти через такі торговельні підприємства, що за мету
собі мають лише фабрикацію векселів, — стає таким складним, що зовнішня подоба
дуже солідного підприємства та жвавих зворотних припливів капіталу може
спокійно існувати й далі, вже після того, як у дійсності ці припливи уже давно
відбувалися лише почасти коштом обшахрованих грошових позикодавців, а почасти
коштом обшахрованих продуцентів. Тому саме безпосередньо перед крахом
підприємство майже завжди здається прибільшено міцнішим. Найкращий
доказ тому дають, напр., Reports on Bank Akts з 1857 та 1858 років, де всі
директори банків, купці, коротко — всі запрошені експерти, на чолі з лордом
Оверстоном, ґратулювали одні одних з приводу розквіту й доброго стану справ, —
саме місяць перед тим, як в серпні 1857 року вибухла криза. І дивна річ, Тук
\parbreak{}  %% абзац продовжується на наступній сторінці
