Отже, за таких припущень, коли б вартість товарів була і уявлялась величиною
сталою, коли б частина вартости товарового продукту, що розкладається
на доходи, лишалася сталою величиною і саме такою уявлялася, коли б, нарешті,
ця дана і стала частина вартости розпадалася на заробітну плату, зиск і ренту
завжди в незмінній пропорції, — навіть за таких припущень дійсний рух неминуче
мусив би виступати в перекрученому вигляді: не як розпад наперед даної
величини вартости на три частини, що набувають форми незалежних один від
одного доходів, але навпаки, як створення цієї вартости із суми незалежних,
самостійно визначуваних кожен сам по собі, складових її елементів: заробітної
плати, зиску й земельної ренти. Ця ілюзія виникла б неминуче, бо в дійсному
русі окремих капіталів і їхніх товарових продуктів не вартість товару виступає
як передумова її розпаду на складові частини, а навпаки, складові частини, на які
вони розпадаються, виступають як передумова вартости товарів. Насамперед, для
окремого капіталіста, як ми бачили, витрати продукції товару виступають
як дана величина, що як така, завжди визначається в дійсній ціні продукції.
Але витрати продукції дорівнюють вартості сталого капіталу, авансованих засобів
продукції, плюс вартість робочої сили, яка проте, в очах аґентів продукції набуває
іраціональної форми ціни праці, так що заробітна плата одночасно виступає
як дохід робітника. Пересічна ціна праці є величина дана, бо вартість робочої
сили, як і всякого іншого товару, визначається робочим часом, потрібним
для її репродукції. Щождо тієї частини вартости товарів, яка становить заробітну
плату, то вона виникає не з того, що вона набуває цієї форми заробітної плати,
не з того, що капіталіст авансує робітникові його частину в його власному
продукті в зовнішній формі заробітної плати, а в наслідок того, що робітник
створює еквівалент, відповідний до його заробітної плати, тобто протягом певної
частини своєї щоденної або річної праці створює вартість, яка міститься в ціні
його робочої сили. Але заробітну плату встановлюється контрактом раніш, ніж
випродуковано відповідний їй еквівалент вартости. Як елемент ціни, що його
величина дана раніш, ніж випродуковано товар і товарову вартість, як складова
частина витрат продукції, заробітна плата виступає тому не як частина,
що в самостійній формі відривається від усієї вартости товару, а навпаки, як
величина дана, що наперед визначає всю вартість товару, тобто, як чинник,
що створює ціну або вартість. Ролю аналогічну тій, що її заробітна плата
відіграє у витратах продукції товару, пересічний зиск відіграє в ціні продукції
товару, бо ціна продукції дорівнює витратам продукції плюс пересічний
зиск на авансований капітал. Цей пересічний зиск практично входить в
уявлення і розрахунки самого капіталіста, як регуляційний елемент, — і не
тільки в тому розумінні, що він регулює перенесення капіталу з однієї сфери
приміщення в другу, але й також при всяких купівлях і контрактах, що охоплюють
процес репродукції за довший період. Але оскільки це так, остільки пересічний
зиск є наперед дана величина, що дійсно не залежить від вартости і додаткової
вартости, створюваної в кожній певній галузі продукції, а тому тим паче —
кожним окремим капіталом, приміщеним у межах кожної такої галузі. Пересічний
зиск у своєму зовнішньому вияві видається не наслідком розпаду вартости,
а радше величиною, незалежною від вартости товарового продукту, наперед даною
в процесі продукції товарів і до того ж визначальною для пересічної ціни
товарів, тобто чинником, що створює вартість. При цьому додаткова вартість
в наслідок розпадання її різних частин, на цілком незалежні одна від однієї
форми, виступає в ще конкретнішій формі як передумова створення вартости
товарів. Частина пересічного зиску, в формі проценту, протистоїть — самостійно
як елемент, що є передумова продукції товарів і їхньої вартости, — капіталістові,
що функціонує. Хоч би як коливалась величина проценту, за кожного
даного моменту і для кожного даного капіталіста вона є величина дана, що
