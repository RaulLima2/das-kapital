Розвиток лихварського капіталу прилучається до розвитку купецького капіталу
та зокрема до розвитку грошево-торговельного капіталу. В стародавньому
Римі, починаючи від останніх часів республіки, коли мануфактура стояла далеко
нижче від пересічного античного розвитку, купецький капітал, грошево-торговельний
капітал та лихварський капітал розвинувся — в межах античних форм —
до найвищої точки.

Ми бачили, як разом з грішми неминуче утворюються скарби. Однак професійний
творець скарбів набуває ваги тільки тоді, коли він перетворюється на
лихваря.

Купець бере в позику гроші на те, щоб добувати від грошей зиск, щоб уживати
їх, як капітал, тобто, щоб витрачати. Отже і в попередніх формах проти
нього стоїть грошовий позикодавець, цілком так само, як і проти сучасного капіталіста.
Це специфічне відношення відчували ft католицькі університети. «Університети
Алькаля, Саля. манки, Інґольштадта, Фрайбурґа в Брай. іґау, Майнца, Кельна та
Тріра, один по одному визнали законність процентів від торговельних позик.
Перші п’ять цих визнань складено до архівів консулату міста Ліону та вндруковано
в додатку до «Traite de l’usure et des intérêts, Lyon, Bruyset-Ponthus».
(M. Angier, Le crédit public etc. Paris 1842, p. 206 і. В усіх тих формах, де невільницьке
господарство (не патріархальне, а в тій формі, що в ній воно існувало
за пізніших грецьких та римських часів) існує, як засіб до збагачення,
де. отже, гроші є засіб до присвоювання чужої праці за допомогою купівлі невільників,
землі і т. ін., — в усіх таких формах гроші починають давати проценти
саме тому, що їх можна приміщувати таким способом, можна уживати
як капітал.

Однак ті характеристичні форми, що в них існує лихварський капітал за
часів передкапіталістичного способу продукції, бувають двоякого роду. Я кажу
характеристичні форми. Ці самі форми повторюються на базі капіталістичної
продукції, але лише як підпорядковані форми. Тут вони — вже не ті форми, які
визначають характер капіталу, що дає процент. Ці дві форми такі: ІІотрше,
лихварство за допомогою визичання грошей марнотратним вельможам, головно
земельним власникам; подрцге, лихварство за посередництвом визичання грошей
дрібним продуцентам, що володіють своїми власними умовами праці, тим продуцентам,
що до них належить ремесник, а особливо селянин, бо взагалі за передкапіталістичяих
обставин — оскільки вони допускають існування дрібних самостійних
поодиноких продуцентів — селянська кляса мусить становити велику більшість
тих продуцентів.

Обидві ці форми, як руйнування багатих земельних власників лихварством,
так само і визискування ним дрібних продуцентів приводять до утворення та
концентрації великих грошових капіталів. Але, до якої міри цей процес нищить
старий спосіб продукції, як це було в новітній Европі, та чи ставить він на
його місце капіталістичний спосіб продукції — це цілком залежить від щабля
історичного розвитку та від даних тим розвитком обставин.

Лихварський капітал, як характеристична форма капіталу, що дає процент,
відповідає пануванню дрібної продукції селян та дрібних ремесників-майстрів,
що працюють самостійно. Там, де робітникові — як от за розвинутого капіталістичного
способу продукції — протистоять умови праці та продукт праці, як капітал,
там робітник, як продуцент, не має потреби позичати грошей. Коли він позичає, то
робиться це, як от в ломбарді, для особистих потреб. Навпаки, там, де робітник
є власник — дійсний або номінальний — умов своєї праці та свого продукту, він
як продуцент перебуває у певному відношенні до капіталу грошового позикодавця,
що протистоїть йому, як лихварський капітал. Newman висловлює це
цілком вульгарно, кажучи, що банкіра поважають, а лихваря ненавидять та зневажають,
бо перший визичає багатим, а другий — убогим. (І. W. Newmann
