\parcont{}  %% абзац починається на попередній сторінці
\index{iii2}{0013}  %% посилання на сторінку оригінального видання
само можуть вони раз за разом знову правити за засіб для повернення тих
позик». (Book II, chap. IV.)

Що та сама монета може здійснювати декілька купівель, відповідно до швидкости
своєї циркуляції, то й може вона так само здійснювати декілька позик, бо купівлі
переносять її з одних рук до інших, а позика становить тільки перенесення з одних
рук до інших, яке відбувається без посередництва купівлі. Для кожного з продавців
гроші є перетворена форма його товару; в наші дні, коли кожна вартість набирає
вигляду капітальної вартости, гроші представляють у різних позиках послідовний
ряд різних капіталів, а це є лише інший вислів тієї попередньої тези, що гроші
можуть послідовно реалізувати різні товарові вартості. Одночасно правлять
вони за засіб циркуляції для того, щоб переносити речові капітали з одних рук до
інших. При позиках гроші переходять з рук до рук не як засіб циркуляції. Поки
вони лишаються в руках позикодавця, вони є в його руках не засіб циркуляції, а
форма існування вартости його капіталу. І в цій формі він передає гроші, позичаючи
їх третій особі. Коли б А позичив гроші В, а В позичив їх С без
посередництва продажів, то ті самі гроші становили б не три капітали, а тільки
один, тільки одну капітальну вартість. Скільки капіталів в дійсності представляють
гроші, це залежить від того, як часто вони функціонують, як форма
вартости різних товарових капіталів.

Те саме, що А. Сміт каже про позики взагалі, має силу й для вкладів,
бо вони є лише особлива назва для тих позик, що їх публіка дає банкірам.
Ті самі гроші можуть бути знаряддям для будь-якого числа вкладів.

«Безперечно правильно, що ті 1000 ф. ст., які хтось сьогодні склав в А,
завтра знову витратиться, й вони являтимуть вклад у В. Другого дня, повернуті
від В, можуть вони являти вклад у С і так далі без кінця. Тому ті самі
1000 ф. ст. у грошах можуть в наслідок ряду передач помножитися до такої
суми вкладів, що її абсолютно не сила визначити. Отже можливо, що \sfrac{9}{10} усіх
вкладів у Сполученому Королівстві існують лише в бухгальтерських записах
по книгах банкірів, що своєю чергою мають провадити між собою розрахунки...
Напр., в Шотляндії, де кількість грошей в циркуляції ніколи не була більшою
за 3 мільйони ф. ст., сума вкладів проте становила 27 мільйонів. І коли б не
було загального штурму на банки по депозити, то ті самі 1000 ф. ст. могли б
перебігаючи свій шлях назад, з тією самою легкістю вирівняти знову таку суму,
що її так само не сила визначити. А що ті самі 1000 ф. ст., що ними сьогодні
хтось оплатив борг крамареві, завтра можуть вирівняти борг останнього купцеві,
день потому — борг купця банкові і т. д. без краю, то і можуть ті сам,
1000 ф. ст. мандрувати з рук до рук і від банку до банку та вирівнювати
всяку суму вкладів, яку тільки можна уявити». (The Currency Question Reviewed,
p. 162, 163).

А що все в цій кредитовій системі подвоюється та потроюється, перетворюючись
у просту химеру, то й має це силу й для «запасних фондів», що в них
нарешті сподіваються знайти дещо солідне.

Послухаймо знову пана Моріса, управителя Англійського банку: «Резерви
приватних банків перебувають в руках Англійського банку, як вклади. Перший
вплив відпливу золота, здається, зачіпає тільки Англійський банк; але той
відплив так само впливатиме й на резерви інших банків, бо він є відплив
частини тих резервів інших банків, що їх вони мають в нашому банку. Так
само впливатиме він і на резерви всіх провінціяльних банків» (Commercial
Distress 1847—48). Отже, кінець-кінцем, запасні фонди у дійсності сходять на
запасний фонд Англійського банку\footnote{
[Як дуже розвинулося таке становище з того часу, показують наведені далі офіційні відомості,
взяті з Daily New s з дня 15 грудня 1892 р., про банкові резерви п’ятнадцятьох найбільших
лондонських
банків у листопаді 1892 р:

  \begin{tabularx}{\textwidth}{l r r r}

  Назва банку & \makecell{Пасив \\ ф. ст.} & \makecell{Запаси готівкою \\ ф. ст.} & У відсотках \\

City \dotfill{}                                    & 9317629  & 746551  & 8,01 \\
Capital and Counties \dotfill{}                   & 11392744 & 1307483 & 11,47 \\
Imperial \dotfill{}                               & 3987400  & 447157  & 11,21 \\
Lloyds \dotfill{}                                 & 23800937 & 2966806 & 12,46 \\
Lon. and Westminster \dotfill{}                   & 24671559 & 3818885 & 15,50 \\
London and S. Western \dotfill{}                  & 5570268  & 812353  & 13,58 \\
London joint Stock \dotfill{}                     & 12127993 & 1288977 & 10,62 \\
London and Midland \dotfill{}                     & 8814499  & 1127280 & 12,79 \\
London and County \dotfill{}                      & 37111035 & 3600374 & 9,70 \\
National \dotfill{}                               & 11163829 & 1426225 & 12,77 \\
National Provincial \dotfill{}                     & 41907384 & 4614780 & 11,01 \\
Parrs and the Alliance \dotfill{}                 & 12794489 & 1532707 & 11,93 \\
Prescott and C° \dotfill{}                        & 4041058  & 538517  & 13,07 \\
Union of Loudon \dotfill{}                       & 15502618 & 2300084 & 14,84 \\
Williams, Deacon and Manchester and C°\dotfill{} & 10452381 & 1317628 & 12,60 \\
\cmidrule(lr){1-4}
Разом                                  & 232655823 & 27845807 & 11,97

  \end{tabularx}

З цих майже 28 мільйонів запасів, принаймні 25 мільйонів покладено до Англійського банку, а
щонайбільше 3 мільйони готівкою є в касах самих 15 банків. Однак запас готівкою в банковому відділі
Англійського банку в тому самому листопаді 1892 р. ніколи не становив повних 16 мільйонів! — Ф. Е.].

}. Але й цей запасний фонд має знову двоїсте
\index{iii2}{0014}  %% посилання на сторінку оригінального видання
існування. Запасний фонд banking department (банкового відділу) є рівний
надмірові банкнот, що їх банк має право видати понад ті банкноти, що є в
циркуляції. Усталений законом максимум банкнот до емісії є = 14 мільйонам
(для цього не треба жодного металевого резерву; це є приблизна сума боргу
держави банкові) плюс сума запасу благородних металів банку. Отже, коли цей запас
є = 14 мільйонам ф. ст., то банк може емітувати банкнот на 28 мільйонів ф. ст.,
а коли 20 мільйонів з тієї суми є в циркуляції, то запасний фонд банкового відділу
є = 8 мільйонам. Ці 8 мільйонів банкнот становлять тоді законний банкірський
капітал, що ним має порядкувати банк, а одночасно й запасний фонд для його
вкладів. Отже, коли настане відплив золота, що зменшить металевий запас на
6 мільйонів — а в наслідок цього треба знищити стільки ж банкнот, — то запас
банкового відділу спаде від 8 до 2 мільйонів. З одного боку, банк мав би дуже
підвищити свій рівень проценту; з другого боку, ті банки, що складали в нього
свої гроші, та інші вкладники побачили б, що запасний фонд банку, яким забезпечено
їхні вклади в нього, дуже зменшився. В 1857 році чотири найбільші
акційні банки Лондону загрожували, що, коли англійський банк не доб’ється «урядового листа», щоб
припинити чинність банкового акту 1844 року,\footnote{Припинення чинности банкового акту 1844 року позволяє банкові видавати довільне число банкнот, не
зважаючи на покриття їх тим золотим скарбом, що є в його руках; отже, дозволяє утворювати довільну
кількість паперового фіктивного грошового капіталу й тим способом давати позики банкам та векселевим
маклерам, а через них і торговлі.} то вони
заберуть свої вклади, від чого банковий відділ став би банкротом. Оттак банковий
відділ може, як от 1847 року, збанкрутувати, тимчасом коли в issue departement
(емісійному відділі) лежить багато мільйонів (напр., в 1847 р. 8 мільйонів),
як ґарантія розмінности банкнот, що перебувають в циркуляції. Але це
останнє є знову ілюзорне.

«Велика частина вкладів, що на них самі банкіри не мають безпосереднього
попиту, йде до рук billbrokers (буквально: векселевих маклерів, а справді —
напівбанкірів), що дають банкірові як забезпечення за свою в нього позику
торговельні векселі, вже дисконтовані ними для різних осіб у Лондоні та в
провінції. Billbroker відповідає перед банкіром за повернення цих money at
call (гроші, що їх мають на вимогу негайно повернути); і ці операції мають
такий величезний обсяг, що пан Neave, сучасний управитель банку [Англійського]
каже у своєму свідченні: «Нам відомо, що один broker мав 5 мільйонів,

\parbreak{}  %% абзац продовжується на наступній сторінці
