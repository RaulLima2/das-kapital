\parcont{}  %% абзац починається на попередній сторінці
\index{iii2}{0088}  %% посилання на сторінку оригінального видання
Це збільшило суму векселів, виданих Англією, тимчасом, з другого боку, за
високого рівня проценту' трудність добувати гроші була така велика, що попит
на ці векселі впав, хоча їхня сума зросла. З тієї самої причини сталося так,
що замовлення на чужоземні товари анулювали, і англійські капітали, приміщенні
в закордонних цінних паперах, зреалізували, а гроші відправили для
приміщення в Англію. Напр., читаємо ми в Rio de Janeiro Prices Current з 10
травня: «Вексельний курс» [на Англію] «зазнав нового спаду, викликаного,
насамперед, тисненням на римесовий ринок виторгу від значних продажів [бразилійських]
державних фондів за англійський рахунок». Англійський капітал, приміщений
закордоном в різних цінних паперах, коли рівень проценту був дуже
низький, забрали таким чином назад, коли рівень проценту піднісся.

\subsubsection{Торговельний балянс Англії}

Сама лише Індія має платити щось 5 міл. данини за «добрий уряд»,
проценти й дивіденди на британський капітал і т. ін., при чому тут ще зовсім
не зараховано ті суми, що їх відправляють щороку для приміщення в Англії
почасти урядовці, як заощадження з їхнього утримання, почасти купці, як частину
їхнього зиску. З тих самих причин кожній англійській колонії доводиться
раз-у-раз робити великі перекази грошей. Більшість банків в Австралії, Вест-Індії,
Канаді засновано на британський капітал, дивіденди доводиться виплачувати
в Англії. Так само Англія має багато чужоземних державних паперів, європейських,
північно- та південно-американських, що від них має вона одержувати
проценти. Далі, до цього долучається ще причетність Англії до закордонних
залізниць, каналів, копалень і т. ін. з відповідними дивідендами. Римеси від
усіх цих джерел робляться виключно майже тільки в продуктах, далеко понад
суму англійського вивозу. З другого боку, те, що з Англії йде закордон державцям
англійських цінних паперів та на споживання англійців закордоном, становить
порівняно з вищезгаданим нікчемно малу суму.

Це питання, оскільки воно торкається торговельного балянсу та вексельних
курсів, є «в кожен даний момент питання часу. Звичайно... Англія дає
довготермінові кредити на свій вивіз, оплачуючи довіз готівкою. В певні моменти
ця ріжниця звичаїв має значний вплив на курси. В ті часи, коли наш
вивіз більшає дуже значно, як-от в 1850 р., приміщення британського капіталу
мусять невпинно зростати... отже, в 1850 році можна було б зробити римеси
за товари, експортовані в 1840 році. Але коли вивіз 1850 року перевищував
вивіз 1849 р. на 6 міл., то результатом цього на практиці мусило бути те, що
поза межі країни відправили більше грошей, порівняно з тією сумою, що вернулася
назад того самого року; і таким способом викликано вплив на курси
та на рівень проценту. Навпаки, скоро наші справи опиняться в стані кризи,
а наш вивіз дуже зменшиться, рімеси за більший експорт попередніх років, рімеси,
що їм надійшов реченець, починають дуже значно перевищувати вартість нашого
довозу; відповідно до цього курси обертаються на нашу користь, капітал швидко
нагромаджується в країні й рівень проценту спадає». (Economist, 11 січня 1851).

Закордонний вексельний курс може змінятися:

1) в наслідок платіжного балянсу даної хвилини, хоч і якими причинами він
визначається суто-торговими, приміщенням капіталу закордоном або ж державними
видатками, підчас воєн і т. ін., скоро тільки при цьому платежі закордон
робиться готівкою.

2) в наслідок знецінення грошей в країні, однаково, чи металевих, чи
паперових грошей. В цьому випадку зміна курса суто-номінальна. Коли І ф. ст.
репрезентує лише половину тих грошей, що він репрезентував раніше, його
звичайно будуть рахувати за 12 \sfrac{1}{2} фр., замість 25 фр.
