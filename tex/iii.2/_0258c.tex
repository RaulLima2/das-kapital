\parcont{}  %% абзац починається на попередній сторінці
\index{iii2}{0258}  %% посилання на сторінку оригінального видання
в 400. Тому створена в зазначений спосіб і визначувана кількістю зрічевленою
в ній праці товарова вартість в 250 становить межу тієї суми дивіденду,
яку робітник, капіталіст і земельний власник можуть здобути з цієї вартости
в формі доходу, в формі заробітної плати, зиску й ренти.

Хай капітал з тим самим органічним складом, тобто з тим самим відношенням
між ужитою живою робочою силою і пущеним в рух сталим капіталом,
мусить платити 150 фунтів стерл., замість 100 за ту саму робочу силу,
що пускає в рух сталий капітал в 400; хай далі додаткова вартість поділяється
знов таки, в новому відношенні, на зиск і ренту. А що припущено,
що змінний капітал в 150 фунтів стерл. пускає в рух ту саму масу праці,
яка раніш пускалась у рух капіталом в 100, то новоспродукована вартість, як
і давніш, дорівнювала б 250, і вартість сукупного продукту, як і давніш, дорівнювала
б 650, але ми мали тоді: $400c + 150v + 100m$; і ці $100m$ розпадаються,
скажімо, на 45 зиску плюс 55 ренти. Пропорція, що в ній сукупна новостворена
вартість розподіляється тепер між заробітною платою, зиском і рентою,
була б цілком відмінна від колишньої; цілком інша була б також величина
сукупного авансованого капіталу, хоч він пускає в рух ту саму сукупну масу
праці. Заробітна плата становила б 27\sfrac{3}{11}\%, зиск — 8\sfrac{2}{11}\%, рента — 10\% на
авансований капітал; отже, сукупна додаткова вартість трохи більша за 18\%.

В наслідок підвищення заробітної плати змінилася б величина доплаченої
частини сукупної праці, отже, і величина додаткової вартости. При десятигодинному
робочому дні робітникові довелося б тепер 6 годин працювати на
себе і лише 4 години на капіталіста. Відношення між зиском і рентою також
змінилося б; зменшена додаткова вартість ділилася б у новій пропорції між
капіталістом і земельним власником. Нарешті, в наслідок того, що вартість
сталого капіталу лишилась незмінна, а вартість авансованого змінного капіталу
зросла, понижена додаткова вартість виражається в ще пониженішій гуртовій
нормі зиску, під якою ми розуміємо тут відношення сукупної додаткової вартости
до всього авансованого капіталу.

Зміна вартости заробітної плати, норми зиску і норми ренти, хоч би яка
була дія законів, що реґулюють взаємовідношення цих частин, могла б рухатися
лише в межах, визначуваних величиною новостворенної товарової вартости
в 250. Виняток міг би бути лише тоді, коли б рента ґрунтувалась на
монопольній ціні. Це не відмінімо б закону, а лише ускладнило б дослідження.
Коли б ми в цьому випадку почали розглядати тільки самий продукт, то зміна
виявилася б лише в розподілі додаткової вартости; коли ж ми стали б розглядати
відносну вартість цього продукту проти інших товарів, то ми знайшли б
лише ту відмінність, що частина додаткової вартости останніх переноситься на
цей специфічний товар.

Отже, підсумуймо:
\begin{table}[h]

\footnotesize

\begin{tabular}{с c c c}
Вартість продукту & \makecell{Новостворена \\ вартість}  & \makecell{Норма додатко- \\вої вартости}  &  \makecell{Норма гурто- \\вого зиску} \\
\midrule
Перший випадок: 400c + 100v + 150m = 650   &    250  & 150\phantom{\sfrac{2}{3}}\%     &   30\phantom{\sfrac{2}{11}}\% \\
Другий випадок: 400c + 150v + 100m =  650  &    250  & \phantom{0}66\sfrac{2}{3}\%\footnotemarkZ{}
  &  18\sfrac{2}{11}\% \\
\end{tabular}
\end{table}
\footnotetextZ{У німецькому тексті тут помилково стоїть «66\sfrac{1}{3}\%». \emph{Пр. ред}}

Перш за все додаткова вартість понижується на третину своєї колишньої
величини, з 150 до 100. Норма зиску понижується трохи більш, ніж на одну
третину, з 30\% до 18\%, бо зменшена додаткова вартість обчислюється тепер
на вирослий авансований сукупний капітал. Але вона понижується далеко не
\parbreak{}  %% абзац продовжується на наступній сторінці
