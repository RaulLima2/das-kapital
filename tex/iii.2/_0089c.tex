\index{iii2}{0089}  %% посилання на сторінку оригінального видання
3) Коли мовиться про курс між країнами, що з них одна як «гроші»
уживає срібло, а друга золото, то вексельний курс залежить від коливань відносної
вартости цих двох металів, бо ці коливання, очевидно, змінюють паритет
між обома металами. За приклад цього були курси в 1850 році; вони були
несприятливі для Англії, дарма, що її експорт незвичайно зріс; проте відпливу
золота не відбувалось Де було наслідком раптового піднесення вартости срібла
проти вартости золота (Дивись Economist, ЗО листопада 1857 р.).

Паритет вексельного курсу за 1 ф. ст. є: на Париж 25 фр. 20 сант.; на
Ґамбурґ 13 банкових марок 10 1/2  шил.; на Амстердам 11 флор. 97 центів.

В міру того як вексельний курс на Париж підноситься понад 25.20, він стає
сприятливіший 'для англійських довжників Франції, або для покупців французьких
товарів. В обох випадках потрібно менше фунтів стерлінґів, щоб досягти
своєї мети.

По віддаленіших країнах, де не легко добувати благородний металг
а векселів обмаль та не вистачає для платежів, що мають робитися Англії,
природний наслідок підвищення курсу є піднесення цін на ті продукти, що їх.
звичайно, відправляють до Англії, бо на ці продукти постає тепер більший попит
для відправи їх до Англії замість векселів; це часто трапляється в Індії.

Несприятливий вексельний курс і навіть відплив золота може відбуватись
тоді, коли в Англії панує дуже великий надмір грошей, низький рівень проценту,
висока ціна на цінні папери.

Протягом 1848 року Англія одержала багато срібла з Індії, бо добрих
векселів було обмаль, а середньої якости векселі неохоче брали в наслідок
кризи 1847 року та великого підупаду кредиту в операціях з Індією. Усе це
срібло, ледве прибувши, незабаром знаходило собі шлях до континенту, де революція
зумовила утворення скарбів по всіх закутках. В 1850 році те саме
срібло, здебільша, помандрувало назад до Індії, бо стан вексельного курсу тепер
робив це вигідним.

Монетарна система є в своїй основі католицька, кредитова система — в своїй
основі протестантська. «The Scotch hate gold»\footnote*{
Шотландець ненавидить золото. Прим. Ред.
}. В паперових грошах грошове
буття товарів є лише суспільне буття. Де — віра, що робить спасеним. Віра
у грошову вартість, як іманентний дух товарів, віра в спосіб продукції та в його
небом усталений лад, віра в поодиноких агентів продукції, як просту персоніфікацію
капіталу, що його вартість зростає сама собою. Але, як протестантизм
не емансипувався від основ католицтва, так само й кредитова система не емансипувалася
від бази монетарної системи.

Розділ тридцять шостий

Передкапіталістичні відносини

Капітал, що дає процент або, як ми можемо означити його в його стародавній
формі, лихварський капітал, разом зі своїм братом-близнюком, купецьким
капіталом, належить до передпотопних форм капіталу, що йдуть поперед задовго
до капіталістичного способу продукції та знаходяться в різноманітних
суспільно-економічних формаціях.

Для існування лихварського капіталу не треба нічого, крім того, щоб,
принаймні, частина продуктів перетворювалась на товари, й одночасно з товаровою
торговлею гроші розвинулися в своїх різних функціях.
