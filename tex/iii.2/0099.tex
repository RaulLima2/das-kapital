земельної власности), кредит як такий не має вже ніякого сенсу, що однак
зрозуміли й сами сен-сімоністи. З другого боку, поки капіталістичний спосіб
продукції існує далі, існує далі як одна з його форм і капітал, що дає процент,
і дійсно являє базу його кредитової системи. Тільки той самий сенсаційний письменник,
Прудон, що, лишаючи товарову продукцію, хотів знищити гроші\footnote{
Karl Marx, Philosophie de la Misère, Bruxelles et Paris. 1847 — Karl Marx, Kritik der Polit.
Ökonomie p. 64.
},
був здатен вимріяти таке чудерство, як crédit gratuit\footnote*{
Безплатного кредиту. Пр. Ред.
}, цю ніби реалізацію побожного
бажання дрібно буржуазного погляду.

В «Religion Saint-simonienne, Economie et Politique», на стор. 45 сказано:
«В суспільстві, де одні мають знаряддя промисловости, не маючи здібности або
охоти уживати їх, а інші вправні промислові люди не мають жодного знаряддя
праці, -там кредит має собі за мету перенести яко мога найлегшим способом
ці знаряддя з рук перших їхніх власників до рук тих інших, що тямлять їх
уживати. Зауважмо, що за цим визначенням кредит є наслідок того способу, що
ним ту власність усталилось». Отже, кредит відпадає разом з оцим усталенням
власности. Далі на стор. 98, сказано: Сучасні банки «дивляться на себе, як на
установи, призначені йти за тим рухом, що його породили підприємства, які діють
поза їх межами, а що сами вони імпульсу до цього руху не повинні давати; інакше
кажучи, банки виконують ролю капіталістів проти тих travailleurs, що їм вони позичають
капітали». В тій думці, що сами банки мають взяте на себе провід та відзначатися
«кількістю та корисністю підприємств, що ними вони порядкують, і кількістю
тих робіт, що переводиться за їх приводом» (р. 101), — в ній маємо заховану
ідею crédit mobilier. Так само Charles Peequeur вимагає, щоб банки (те, що сен-сімонисти
звуть Système général des banques) «порядкували продукцією». Взагалі Pecquer
є в основі сен-сімоніст, хоч і багато радикальніший. Він хоче, щоб «кредитова
установа... управляла цілим рухом національної продукції». — «Спробуйте
утворити національну кредитову установу, що позичала б засоби незаможному,
який має талант та заслуги, не зв’язуючи тих довжників між собою примусовою
взаємною солідарністю в продукції та споживанні, а навпаки позичала б
ті засоби так, щоб довжники сами визначали акти свого обміну та продукції.
Дим шляхом ви досягнете лише того, чого вже тепер досягають приватні банки, —
анархії, диспропорції між продукцією та спожитком, раптової руїни одних та
раптового збагачення інших; так що ваша установа ніколи не піде далі від того,
щоб утворити для одних певну суму добробуту, рівну сумі лиха, що припадає
іншим... найманим робітникам, яких ви підтримуєте позиками, ви тільки дасте
засіб до тієї самої взаємної конкуренції, яку тепер одні одним роблять їхні капіталістичні
хазяїни». (Ch. Pecqueur, Théorie Nouvelle d’Econoiuie Soc. et Fol. Paris
1842 p. 434.).

Ми бачили, що купецький капітал та капітал, що дає процент, є найдавніші
форми капіталу. Однак, з самої природи справи випливає те, що капітал, який дає
процент, видається в народній уяві як форма капіталу par excellence. В купецькому
капіталі маємо посередницьку діяльність, хоч і як її тлумачити, як шахрайство, чи
як працю, чи якось інак. Навпаки, в процентодайному капіталі виявляється у чистій
формі характер капіталу, що сам себе репродукує, вартість, що сама собою зростає,
продукція додаткової вартости, як певна таємна якість. Відси ж постає й те, що
навіть частина політикоекономів, особливо в країнах, де, як от у Франції,
промисловий капітал ще не цілком розвинувся, твердо вважають капітал, який
дає процент, за основну форму капіталу, розуміючи, напр. земельну ренту
лише як його іншу форму, бо і тут панує форма визичання. З цієї причини
цілком не розуміють внутрішньої диференціяції капіталістичного способу про-